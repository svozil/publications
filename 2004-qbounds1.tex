\documentclass[pra,preprint,showpacs,showkeys,amsfonts]{revtex4}
%\documentclass[pra,showpacs,showkeys,amsfonts]{revtex4}
\usepackage{graphicx}
\RequirePackage{times}
\RequirePackage{courier}
\RequirePackage{mathptm}
\begin{document}



\title{The min-max principle generalizes Cirel'son's bound}
\author{Stefan Filipp}
\email{sfilipp@ati.ac.at}
\affiliation{Atominstitut der {\"{O}}sterreichischen Universit{\"{a}}ten,
Stadionallee 2, A-1020 Vienna, Austria}
\author{Karl Svozil}
\email{svozil@tuwien.ac.at}
\homepage{http://tph.tuwien.ac.at/~svozil}
\affiliation{Institut f\"ur Theoretische Physik, University of Technology Vienna,
Wiedner Hauptstra\ss e 8-10/136, A-1040 Vienna, Austria}


\begin{abstract}
Bounds on the norm of quantum operators associated with classical Bell-type inequalities
can be derived from their maximal eigenvalues.
This quantative method enables detailed
predictions of the maximal violations of Bell-type inequalities.
\end{abstract}

\pacs{03.67.-a,03.65.Ta}
\keywords{tests of quantum mechanics, correlation polytopes, probability theory}

\maketitle

The violations of Bell-type inequalities represent
a cornerstone of our present understanding of quantum probability theory
\cite{peres}.
Thereby, the usual procedure is as follows:
(i)
First, the (in)equalities bounding the classical probabilities and expectations are
derived systematically; e.g., by enumerating all conceivable classical possibilities
and their associated two-valued measures.
These form the extremal points which span
the classical correlation polytopes
\cite{cirelson:80,cirelson,froissart-81,pitowsky-86,pitowsky,pitowsky-89a,Pit-91,Pit-94,2000-poly,collins-gisin-2003,sliwa-2003};
the faces of which are expressed by Bell-type inequalities
which characterize the bounds of the classical probabilities and expectations;
in Boole's term \cite{Boole,Boole-62}, the ``conditions of possible experience.''
(Generating functions are another method to find bounds on classical expectations \cite{werner-wolf-2001,schachner-2003}.)
The Bell-type inequalities contain sums of (joint) probabilities and expectations.
(ii)
In a second step, the classical probabilities and expectations in
the Bell-type inequalities are substituted by quantum quantum probabilities and expectations.
The resulting operators violate the classical bounds.
Until recently, little was known about the fine structe of the violations.
Cirel'son published an absolute bound for the violation for a particular Bell-type inequality,
the Clauser-Horne-Shimony-Holt (CHSH) inequality \cite{cirelson:80,cirelson:87,cirelson,khalfin-97}
Cabello has published a violation of the CHSH inequality beyond the quantum mechanical
bound by applying selection schemes to particles in a GHZ-state
\cite{cabello-02a,cabello-02b}.
Recently, detailed numerical  \cite{filipp-svo-04-qpoly}
and analytical studies \cite{cabello-2003a} stimulated
experiments \cite{bovino-2003} to test the quantum bounds of certain Bell-type inequalities.

In what follows,
a general method to compute quantum bounds on Bell-type inequalities systematically will be reviewed.
It makes use of the {\em min-max principle} for self-adjoint transformations
(Ref.~\cite{halmos-vs}, Sec.~90) and (Ref.~\cite{reed-sim4}, Sec.~75)
stating that the operator norm is bounded by the minimal and maximal eigenvalues.
These ideas are not entirely new and have been mentioned previously
\cite{werner-wolf-2001,filipp-svo-04-qpoly,cabello-2003a},
yet to our knowledge no systematic investigation has been undertaken yet.
It should also be kept in mind that this method {\it a priori}
cannot produce quantum polytopes \cite{pit:range-2001,filipp-svo-04-qpoly},
only the quantum correspondents of classical polytopes.
Indeed, as will be demonstrated explicitly, the resulting geometric forms will not be convex.
This, however,
does not diminish the relevance of these quantum predictions
to experiments testing the quantum violations
of classical Bell-type inequalities.

As a starting point note that,
since $(A+B)^\dagger =A^\dagger +B^\dagger = A+B= (A+B)$ for arbitrary self-adjoint transformations $A,B$,
the sum of self-adjoint transformations is again self-adjoint.
That is, all self-adjoint transformations entering the quantum correspondent of any Bell-type inequality
is again a self-adjoint transformation.
Note that the sum does not preserve eigenvectors and eigenvalues;
i.e., $A+B$ can have different eigenvectors and eigenvalues than $A$ and $B$ taken separately.
This property is to be expected, since $A$ and $B$ need not necessarily commute;
i.e., $[A,B]\neq 0$.
The norm of the self-adoint transformation resulting from summing the quantum counterparts
of all the classical terms contributing to a particular Bell inequality obeys the min-max principle.
Thus determining the maximal violations of classical Bell inequalities amounts to
solving an eigenvalue problem.
The associated eigenstates are the multi-partite states which yield a maximum violation
of the classical bounds under the given experimental (parameter) setup.

Let us demonstrate the method with a few examples.
The simplest nontrivial case is two particles measured along a {\em single}
(but not necessarily identical) direction on either side.
The vertices are $(p_1,p_2,p_{12}=p_1p_2)$ for $p_1, p_2 \in \{0,1\}$ and thus
$(0,  0,  0)$,
$(0,  1,  0)$,
$(1,  0,  0)$,
$(1,  1,  1)$;
the corresponding face (Bell-type) inequalities of the polytope spanned by the four vertices
are given by
$p_{12} \le p_2$,
$0\le p_{12}\le 1$, and
\begin{equation}
p_1+p_2-p_{2}\le 1.
\label{2004-qbounds-e1}
\end{equation}
In the last equation, the classical probabilities have to be substited by the quantum ones;
i.e.,
\begin{equation}
\begin{array}{lll}
p_1 &\rightarrow& q_1 (\theta ) =
{1\over 2}\left[{\Bbb I}_2 + {\bf \sigma}( \theta )\right] \otimes  {\Bbb I}_2,
\\
p_2 &\rightarrow& q_2 (\theta ) =
{\Bbb I}_2 \otimes {1\over 2}\left[{\Bbb I}_2 + {\bf \sigma}( \theta )\right],
\\
p_{12}&\rightarrow& q_{12} (\theta ,\theta ') =
{1\over 2}\left[{\Bbb I}_2 + {\bf \sigma}( \theta )\right]
\otimes
{1\over 2}\left[{\Bbb I}_2 + {\bf \sigma}( \theta ')\right],
\end{array}
\label{2004-qbounds-e2}
\end{equation}
with
$
{\bf \sigma}( \theta )=  \left(\matrix{ \cos \theta & \sin \theta  \cr \sin\theta & -\cos \theta }\right)
$,
where $\theta $ is the relative measurement angle in the $x$--$z$-plane, and the two particles propagate along the $y$-axis.
Then, the self-adjoint transformation corresponding to the classical Bell-type inequality
can be defined by
\begin{equation}
O_{11}(0,\theta) = q_1(0)+q_2(\theta)-q_{12} (0, \theta ) =
\left(\matrix{
   1 & 0 & 0 & 0 \cr 0 & 1 & 0 & 0 \cr 0 & 0 & {\cos^2
     \frac{\theta }{2}} & \frac{\sin \theta }
   {2} \cr 0 & 0 & \frac{\sin \theta }{2} & {\sin^2  \frac{\theta }{2}} \cr  }
\right).
\end{equation}
The eigenvalues of $O_{11}$ are $0$ and $1$, irrespective of $\theta$.
The min-max principle thus predicts a maximal bound of $O_{11}$
which does not exceed the classical bound (\ref{2004-qbounds-e1}).

2-2 CHSH  CH

3-3 Sliwa/Gisin

4-4 Sliwa/Gisin


Finally, let us relate to the experimental testability of the quantum hull predictions.
In principle, as the self-adjoint operators $O_{ii}$ are known,
their extreme eigenvalues and the associated eigenstates are operationalizable.
Multiport interferometry offers a direct implementation
\cite{rzbb,zukowski-97,svozil-2004-analog}.

Implementations by entangled particles would have to use the coherent superposition of states
to the appropriate eigenvectors corresponding to the associated extreme eigenvalues.
We propose to generalize the Cabello {\it ansatz}  [Ref.~\cite{cabello-2003a}, Eqs.~(14-15)]
for the production of a suitable wavefunction from a two-particle (two-dimensional) singlet state
by
\begin{equation}
\vert \varphi (\alpha_1,\beta_1,\sigma_1,\phi_1;\alpha_2,\beta_2,\sigma_2,\phi_2) \rangle
=
U(\alpha_1,\beta_1,\sigma_1,\phi_1)\otimes U(\alpha_2,\beta_2,\sigma_2,\phi_2)
\vert \psi^-\rangle
,
\label{2004-qbounds-ca1}
\end{equation}
where $U(\alpha ,\beta ,\sigma ,\phi )$
stands for the four-parameter unitary transformation [e.g., Ref.~\cite{murnaghan}, Chapter 2]
\begin{equation}
U(\alpha ,\beta ,\sigma ,\phi )
=
\left(
\matrix{
e^{i\alpha }&0\cr
0& e^{i\beta }
}
\right)
\left(
\matrix{
\cos \phi &- e^{-i\sigma } \sin \phi \cr
e^{i\sigma } \sin \phi & \cos \phi
}
\right)
\label{2004-qbounds-ca2}
\end{equation}


In summary, we have shown how to construct the exact quantum bounds of Bell-type inequalities
by solving the eigenvalue problem of the associated self-adjoint transformation.
Some problems which we shall briefly mention remain unsolved.
First, we may conjecture that the
exact quantum correlation hull is directly derivable by extending the classical Bell-type
inequalities in the same way presented above;
i.e., by substituting the quantum probabilities for the classical ones.
This is by no means trivial, as the sections of the quantum hull need not
necessarily be derivable by mere classical extensions.
A second open question is related to the geometric structures
arising from quantum expectation values.
These need not necessarily be convex.
Again, the question of direct extensibility remains open for the hull of quantum expectations
from the classical ones.

This research has been supported by the Austrian Science Foundation (FWF), Project Nr. F1513.


\bibliography{svozil}
\bibliographystyle{apsrev}
\end{document}


########### 1-1
---------------------- 1-1.ext
* 1-1
*  $a1  $b1   $a1*$b1
*
V-representation
begin
   4  4  real
1  0  0  0
1  0  1  0
1  1  0  0
1  1  1  1
end
hull

---------------------- 1-1.ine
* cdd+: Double Description Method in C++:Version 0.76a1 (June 8, 1999)
* Copyright (C) 1999, Komei Fukuda, fukuda@ifor.math.ethz.ch
* Compiled for Floating-Point Arithmetic
*Input File:1-1.ext(4x4)
*HyperplaneOrder: LexMin
*Degeneracy preknowledge for computation: None (possible degeneracy)
*Hull computation is chosen.
*Zero tolerance = 1e-06
*Computation starts     at Fri Feb 13 09:44:17 2004
*            terminates at Fri Feb 13 09:44:17 2004
*Total processor time = 0 seconds
*                     = 0h 0m 0s
*Since hull computation is chosen, the output is a minimal inequality system
*FINAL RESULT:
*Number of Facets = 4
H-representation
begin
4  4  real
 1 -1 -1 1
 0 1 0 -1
 0 0 1 -1
 0 0 0 1
end

---------------------- 1-1.m

<< Algebra`ReIm`

TensorProduct[a_, b_] :=    Table[(*a, b are nxn and mxm - martices*) a[[Ceiling[s/Length[b]], Ceiling[t/Length[b]]]]*b[[s - Floor[(s - 1)/Length[b]]*Length[b],t - Floor[(t - 1)/Length[b]]*Length[b]]], {s, 1,Length[a]*Length[b]}, {t, 1, Length[a]*Length[b]}];

vecsig[r_, tt_, p_ ]:= r * { {Cos[tt], Sin[tt] Exp[-I p]}, {Sin[tt] Exp[I p], -Cos[tt]}}

SingleProbH1[x_] :=   TensorProduct[1/2(IdentityMatrix[2] + vecsig[1, x, 0]), IdentityMatrix[2]]

SingleProbH2[x_] :=   TensorProduct[IdentityMatrix[2], 1/2(IdentityMatrix[2] + vecsig[1, x, 0])]

JointProb[x_, y_] :=  TensorProduct[1/2(IdentityMatrix[2] + vecsig[1, x, 0]), 1/2(IdentityMatrix[2] + vecsig[1, y, 0])]

O11[a_, b_] :=   -JointProb[a, b]  + SingleProbH1[a] + SingleProbH2[b]

O11[0, \[Theta]] // FullSimplify // MatrixForm

EVO11[\[Theta]_] := Eigenvalues[O11[0, \[Theta]]]

EVO11[1] // FullSimplify

Plot[{EVO11[\[Theta]][[1]],EVO11[\[Theta]][[2]],EVO11[\[Theta]][[3]],EVO11[\[Theta]][[4]]}, {\[Theta], 0, \[Pi]}, PlotStyle -> {Hue[0.1], Hue[0.3], Hue[0.5], Hue[0.7]}]





###########

<< Algebra`ReIm`

TensorProduct[a_, b_] :=    Table[(*a, b are nxn and mxm - martices*) a[[Ceiling[s/Length[b]], Ceiling[t/Length[b]]]]*b[[s - Floor[(s - 1)/Length[b]]*Length[b],t - Floor[(t - 1)/Length[b]]*Length[b]]], {s, 1,Length[a]*Length[b]}, {t, 1, Length[a]*Length[b]}];

vecsig[r_, tt_, p_ ]:= r * { {Cos[tt], Sin[tt] Exp[-I p]}, {Sin[tt] Exp[I p], -Cos[tt]}}

SingleProbH1[x_] :=   TensorProduct[1/2(IdentityMatrix[2] + vecsig[1, x, 0]), IdentityMatrix[2]]

SingleProbH2[x_] :=   TensorProduct[IdentityMatrix[2], 1/2(IdentityMatrix[2] + vecsig[1, x, 0])]

JointProb[x_, y_] :=  TensorProduct[1/2(IdentityMatrix[2] + vecsig[1, x, 0]), 1/2(IdentityMatrix[2] + vecsig[1, y, 0])]

CoGi[a_, b_, c_, d_, e_, f_] :=   JointProb[a, d] + JointProb[a, e] + JointProb[a, f] + JointProb[b, d] + JointProb[b, e] - JointProb[b, f] + JointProb[c, d] - JointProb[c, e] - SingleProbH1[a] - 2*SingleProbH2[d] - SingleProbH2[e]

CoGi[0, \[Theta], 2*\[Theta], 0,\[Theta], 2*\[Theta]] // FullSimplify // MatrixForm

EVCoGi[\[Theta]_] := Eigenvalues[CoGi[0, \[Theta], 2*\[Theta], 0, \[Theta], 2*\[Theta]]]

Plot[{EVCoGi[\[Theta]][[1]],EVCoGi[\[Theta]][[2]],EVCoGi[\[Theta]][[3]],EVCoGi[\[Theta]][[4]]}, {\[Theta], 0, \[Pi]}, PlotStyle -> {Hue[0.1], Hue[0.3], Hue[0.5], Hue[0.7]}]

Eigenvalues[CoGi[0, \[Theta], 2*\[Theta], 0, \[Theta], 2*\[Theta]]]  // FullSimplify // MatrixForm


a=
{
{a11,a12,a13},
{a21,a22,a23},
{a31,a32,a33}
};

b={b1,b2,b3};

Solve[{
a.{0,0,0}=={-1,-1,+1},
a.{0,1,0}=={-1,+1,-1},
a.{1,0,0}=={+1,-1,-1},
a.{1,1,1}=={1,1,1}},{a11,a12,a13,a21,a22,a23,a31,a32,a33,b1,b2,b3}]

a=
{
{a11,a12,a13},
{a21,a22,a23},
{a31,a32,a33}
};

b={b1,b2,b3};

Solve[{
a.{0,0,0}=={0,0,1},
a.{0,1,0}=={0,1,0},
a.{1,0,0}=={1,0,0},
a.{1,1,1}=={1,1,1}},{a11,a12,a13,a21,a22,a23,a31,a32,a33,b1,b2,b3}]



#####################################################################


Sliwa hull
Mime-Version: 1.0
Content-Type: text/plain; charset="us-ascii"; format=flowed
X-Virus-Scanned: by amavisd-milter (http://amavis.org/)
X-Spam-Status: LOW ; -30
X-Spam-Level: -
X-Spam-TU-Processing-Host: mri1
X-UIDL: a&i!!S"/!!UX"#!HO%#!
Status: RO

<< Algebra`ReIm`

TensorProduct[a_, b_] :=    Table[(*a, b are nxn and mxm - martices*) a[[Ceiling[s/Length[b]], Ceiling[t/Length[b]]]]*b[[s - Floor[(s - 1)/Length[b]]*Length[b],t - Floor[(t - 1)/Length[b]]*Length[b]]], {s, 1,Length[a]*Length[b]}, {t, 1, Length[a]*Length[b]}];

vecsig[r_, t_, p_ ]:= r * { {Cos[t], Sin[t] Exp[-I p]}, {Sin[t] Exp[I p], -Cos[t]}}

CorrellOp[x_, y_] := TensorProduct[vecsig[1, x, 0], vecsig[1, y, 0]];
SingleOpH1[x_] := TensorProduct[vecsig[1, x, 0], IdentityMatrix[2]];
SingleOpH2[x_] := TensorProduct[IdentityMatrix[2], vecsig[1, x, 0]];

PITOpSingle[a_, b_, c_, d_, e_, f_] := CorrellOp[a, d] + CorrellOp[a, e] + CorrellOp[b, d] + CorrellOp[b, e] +      CorrellOp[c, d] - CorrellOp[c, e] + CorrellOp[a, f] - CorrellOp[b, f] +      SingleOpH1[a] + SingleOpH1[b] + SingleOpH2[d] + SingleOpH2[e];

PITOpSingle[0, x, 2*x, 0, x, 2*x] // FullSimplify // MatrixForm

Eigenvalues[PITOpSingle[0, x, 2*x, 0, x, 2*x]] // FullSimplify

eeeSingle[x_] := Eigenvalues[PITOpSingle[0, x, 2*x, 0, x, 2*x]]

Plot[{eeeSingle[x][[1]], eeeSingle[x][[2]], eeeSingle[x][[3]], eeeSingle[x][[4]]}, {x, 0, Pi}, PlotStyle -> {Hue[0.1], Hue[0.3], Hue[0.5], Hue[0.7]}]



