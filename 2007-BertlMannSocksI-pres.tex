%\documentclass[pra,showpacs,showkeys,amsfonts,amsmath,twocolumn]{revtex4}
\documentclass[amsmath,blue,handout,table]{beamer}
%\documentclass[pra,showpacs,showkeys,amsfonts]{revtex4}
\usepackage[T1]{fontenc}
\usepackage{beamerthemeshadow}
%\usepackage[dark]{beamerthemesidebar}
%\usepackage[headheight=24pt,footheight=12pt]{beamerthemesplit}
%\usepackage{beamerthemesplit}
%\usepackage[bar]{beamerthemetree}
\usepackage{graphicx}
\usepackage{pgf}
%\usepackage[usenames]{color}
%\newcommand{\Red}{\color{Red}}  %(VERY-Approx.PANTONE-RED)
%\newcommand{\Green}{\color{Green}}  %(VERY-Approx.PANTONE-GREEN)

%\RequirePackage[german]{babel}
%\selectlanguage{german}
%\RequirePackage[isolatin]{inputenc}

\pgfdeclareimage[height=0.5cm]{logo}{tu-logo}
\logo{\pgfuseimage{logo}}
\beamertemplatetriangleitem
\begin{document}

\title{\bf \textcolor{yellow}{Unknowables in Physics}}
\subtitle{\textcolor{yellow!60}{http://tph.tuwien.ac.at/$\sim$svozil/publ/2007-BertlMannSocksI-pres.pdf\\
http://arxiv.org/abs/physics/0701163}}
\author{Karl Svozil}
\institute{Institut f\"ur Theoretische Physik, University of Technology Vienna, \\
Wiedner Hauptstra\ss e 8-10/136, A-1040 Vienna, Austria\\
svozil@tuwien.ac.at
%{\tiny Disclaimer: Die hier vertretenen Meinungen des Autors verstehen sich als Diskussionsbeitr�ge und decken sich nicht notwendigerweise mit den Positionen der Technischen Universit�t Wien oder deren Vertreter.}
}
\date{SS 2007}
\maketitle

\frame{\tableofcontents}


%%%%%%%%%%%%%%%%%%%%%%%%%%%%%%%%%%%%%%%%%%%%%%%%%%%%%%%%%%%%%%%%%%%%%%%%%%%%%%%%%%%%%%%%%%%%%%%%%%%%%%%%%
%%%%%%%%%%%%%%%%%%%%%%%%%%%%%%%%%%%%%%%%%%%%%%%%%%%%%%%%%%%%%%%%%%%%%%%%%%%%%%%%%%%%%%%%%%%%%%%%%%%%%%%%%
%%%%%%%%%%%%%%%%%%%%%%%%%%%%%%%%%%%%%%%%%%%%%%%%%%%%%%%%%%%%%%%%%%%%%%%%%%%%%%%%%%%%%%%%%%%%%%%%%%%%%%%%%
%%%%%%%%%%%%%%%%%%%%%%%%%%%%%%%%%%%%%%%%%%%%%%%%%%%%%%%%%%%%%%%%%%%%%%%%%%%%%%%%%%%%%%%%%%%%%%%%%%%%%%%%%


\frame{
\includegraphics<1>[height=7cm]{bertlmann.jpg}
\includegraphics<2>[height=7cm]{BertlmannSMALL.jpg}
}


\section{G\"odel-type unknowables}

\subsection{Results in Formal Logic \& Computer Science}

\frame{
\frametitle{History}
\begin{itemize}
\item<+->
G\"odel incompleteness (independence) 1931
\item<+->
Tarski "Der Wahrheitsbegriff in den Sprachen der deduktiven Disziplinen" 1932
\item<+->
Turing's recursive unsolvability of the Halting problem
1936-1937
\end{itemize}
G\"odel states,
 \begin{quote}
 {\em
 ``$\ldots$ that a complete epistemological description
 of a language $A$ cannot be given in the same language $A$, because
 the concept of truth of sentences of $A$ cannot be defined in $A$. It
 is this theorem which is the true reason for the existence of
 undecidable propositions in the formal systems containing arithmetic.''}
 \end{quote}
}



\frame{
\frametitle{Turing uncomputability and the halting problem}
{\scriptsize
\begin{itemize}
\item<+->
Consider a universal computer $U$ and  an arbitrary algorithm
$B(X)$ whose input is a string of symbols $X$.  Assume that there exists
a ``halting algorithm'' ${\tt HALT}$ which is able to decide whether $B$
terminates on $X$ or not.
The domain of ${\tt HALT}$  is the set of legal programs.
The range of ${\tt HALT}$ are classical bits.
\item<+->
Using ${\tt HALT}(B(X))$ we shall construct another deterministic
computing agent $A$, which has as input any effective program $B$ and
which proceeds as follows:  Upon reading the program $B$ as input, $A$
makes a copy of it.  This can be readily achieved, since the program $B$
is presented to $A$ in some encoded form
$\ulcorner B\urcorner $,
i.e., as a string of
symbols.  In the next step, the agent uses the code
$\ulcorner B\urcorner $
 as input
string for $B$ itself; i.e., $A$ forms  $B(\ulcorner B\urcorner )$,
henceforth denoted by
$B(B)$.  The agent now hands $B(B)$ over to its subroutine ${\tt HALT}$.
Then, $A$ proceeds as follows:  if ${\tt HALT}(B(B))$ decides that
$B(B)$ halts, then the agent $A$ does not halt; this can for instance be
realized by an infinite {\tt DO}-loop; if ${\tt HALT}(B(B))$ decides
that $B(B)$ does {\em not} halt, then $A$ halts.
\item<+->
The agent $A$ will now be confronted with the following paradoxical
task:  take the own code as input and proceed to determine whether or not it halts.
Then, whenever $A(A)$
halts, ${\tt HALT}(A(A))$, by the definition of $A$, would force $A(A)$ not to halt.
Conversely,
whenever $A(A)$ does not halt, then ${\tt HALT}(A(A))$ would steer
$A(A)$ into the halting mode.  In both cases one arrives at a complete
contradiction.  Classically, this contradiction can only be consistently
avoided by assuming the nonexistence of $A$ and, since the only
nontrivial feature of $A$ is the use of the peculiar halting algorithm
${\tt HALT}$, the impossibility of it.
\end{itemize}
}
}



\frame[shrink=2]{
\frametitle{Undecidability of the rule inference (induction) problem}

Induction in physics is the inference of general rules
dominating and generating physical behaviors from these behaviors.
For any deterministic system strong enough to support
universal computation, the general induction problem
is provable unsolvable.
Induction is thereby reduced to the unsolvability of
the rule inference problem,

Informally, the algorithmic idea of the proof is to take any sufficiently powerful
rule or method of induction and, in using it, define some
functional behavior which is not identified by it.
This amounts
to constructing an algorithm which
(passively!)
 ``fakes'' the ``guesser'' by simulating some particular function $\varphi $
until the guesser
pretends to guess this function correctly.
In a second,  diagonalization step, the ``faking'' algorithm then switches to a different
 function $\varphi^\ast  \neq \varphi $, such that the guesser's guesses become incorrect.



}

\subsection{Reduction}
\frame{
\frametitle{Physical meaning by reduction}
Reduction means that physical undecidability is linked or reduced to logical undecidability.
A typical example for such a reduction is the embedding of a Turing machine or any type of computer capable of
universal computation into a physical system.
As a consequence, the physical system inherits
any type of unsolvability derivable for universal computers, such as the
unsolvability of the halting problem:
since the computer is part of the physical system, so are its behavioral patterns.
\begin{itemize}
\item<+->
Undecidability of the general forecasting problem
\item<+->
Undecidability of the general induction problem
\item<+->
Busy beaver function as the maximal recurrence time of finite systems
\item<+->
Specker's theorems in recursive analysis, Pour-El\& Richards
\end{itemize}}

\frame[shrink=2]{
\frametitle{Busy Beaver Number}

The busy beaver function
addresses the following
question: given a finite system;
i.e., a system whose algorithmic description is of finite length.
What is the biggest number producible by such a system before halting?

Let $\Sigma (n)$ denote the busy beaver function of $n$.
 Originally, T. Rado
 asked how
 many $1$'s a Turing machine with $n$ possible states and an empty
 input tape
 could print on that tape before halting.

 The first values of the Turing busy beaver function $\Sigma _T(x)$
 are finite and are known:                       \\
 $\Sigma _T(1)=1$,                               \\
 $\Sigma _T(2)= 4$,                              \\
  $\Sigma _T(3)=6$,                              \\
 $\Sigma _T(4)= 13$,                             \\
 $\Sigma _T(5) \ge 1915$,                        \\
 $\Sigma_T(7)\ge 22961$,                         \\
 $\Sigma_T(8)\ge 3\cdot (7\cdot 3^{92}-1)/2$.    \\


}



%%%%%%%%%%%%%%%%%%%%%%%%%%%%%%%%%%%%%%%%%%%%%%%%%%%%%%%%%%%%%%%%%%%%%%%%%%%%%%%%%%%%%%%%%%%%%%%%%%%%%%%%%

\section{Deterministic chaos from continuum models}



\subsection{$N$--body problem}
\frame[shrink=2]{
\frametitle{Laplace 1825: Classical omniscience}



{\em
Present events are connected with preceding ones
by a tie based upon the evident principle that a thing
cannot occur without a cause which produces it. This
axiom, known by the name of the principle of sufficient
reason, extends even to actions which are considered
indifferent $\ldots$


We ought then to regard the present state of the
universe as the effect of its anterior state and as the
cause of the one which is to follow. Given for one
instant an intelligence which could comprehend all the
forces by which nature is animated and the respective
situation of the beings who compose it an intelligence
sufficiently vast to submit these data to analysis it
would embrace in the same formula the movements of
the greatest bodies of the universe and those of the
lightest atom; for it, nothing would be uncertain and
the future, as the past, would be present to its eyes.
}

}



\subsection{Convergence of series solution}
\frame{
\frametitle{Poincar�e 1914: Concerns about the convergence of solution of the  $n$-body problem}

{\em
If we would know the laws of Nature and the state of the Universe precisely
for a certain time,
we would be able to predict with certainty
the state of the Universe for any later time.
But
[[~$\ldots$~]]
it can be the case that small differences in the initial values
produce great differences in the later phenomena;
a small error in the former may result in a large error in the latter.
The prediction becomes impossible and we have a ``random phenomenon.''}
}

\frame{
\frametitle{Convergence of solution the $n$-body problem}
\begin{itemize}
\item<+->
The $3$--body problem was already solved in 1912 by Sundman.
\item<+->   The $n$--body problem has been solved by Wang in 1981.
\item<+->The solutions are given in terms of power series.
\item<+->
Suppose we are able to construct a universal computer based on the $n$--body problem.
This can, for instance, be achieved by ballistic computation, such as the
``Billiard Ball'' model of computation
which effectively ``embeds'' a universal computer into a system of $n$--bodies.
Then, by reduction, it follows that certain predictions are impossible.

What are the consequences of this reduction for the convergence of the series solutions?
It can be expected that not only do the series converge ``very slowly,''
like in deterministic chaotic systems,
but that in general there does not exist any computable radius of convergence
for the series solutions of particular observables.

\end{itemize}
}


\subsection{``Deterministic Chaos''}
\frame[shrink=2]{
\frametitle{Lyapunov exponent}

 The {\em Lyapunov exponent} $\lambda$
\index{Lyapunov exponent}
 can be introduced as
 a measure
 \label{lyap}
 of the separation of
 two distinct initial values.
 Consider a discrete
 time evolution of the form $x_{n+1}=f(x_n)$ and an uncertainty
 interval $(x_0,x_0+\epsilon )$ of measure $\epsilon$ which,
 after $n$ iterations, becomes
 $(f^{(n)}(x_0),f^{(n)}(x_0+\epsilon
 ))$, which is of measure $\epsilon \exp \{n\lambda (x_0)\}$.
 $f^{(n)}$ stands for
 the $n$-fold iteration of $f$.
 The {\em Lyapunov
 exponent}
 $\lambda (x_0)$
 is defined for $\epsilon \rightarrow 0$ and $n\rightarrow \infty$ as
 \begin{eqnarray}
 \lambda (x_0)&=&\lim_{n\rightarrow \infty} \lim_{\epsilon \rightarrow
 0}   {1\over n}\log \left\vert {f^{(n)}(x_0+\epsilon
 )-f^{(n)}(x_0) \over \epsilon }\right\vert \nonumber \\
            & &\quad = \lim_{n\rightarrow \infty}{1\over n} \log
 \left\vert \left. {d\over dx}   f^{(n)}(x)\right\vert_{x=x_0}
\right\vert   \nonumber     \\
            & &\quad \quad = \lim_{n\rightarrow \infty}{1\over n} \log
 \left\vert \left. \prod_{i=0}^{n-1}{f'(x)}
\right\vert_{x=x_i}
\right\vert
\nonumber  \\
            & &\quad \quad \quad = \lim_{n\rightarrow \infty}{1\over n}
  \sum_{i=0}^{n-1}\log \left\vert \left. {f'(x)}
\right\vert_{x=x_i}
  \right\vert
\quad ,
 \label{ly-ex}
 \end{eqnarray}
 where the chain rule
 $$\left.{d\over d x}f^{(n)}(x)\right\vert_{x=x_0}=
 \left.{d\over d x}f(f(\cdots f(x)\cdots ))\right\vert_{x=x_0}=
 f'(x_{n-1})f'(x_{n-2})\cdots f'(x_1)f'(x_0)$$
and $\vert a_1 a_2 a_3 \cdots a_n\vert =
 \vert a_1\vert \vert  a_2\vert  \vert a_3\vert  \cdots \vert a_n\vert $
for $a_i\in {\Bbb R}$, $i=1,\ldots n$,
 has been used.
}

%%%%%%%%%%%%%%%%%%%%%%%%%%%%%%%%%%%%%%%%%%%%%%%%%%%%%%%%%%%%%%%%%%%%%%%%%%%%%%%%%%%%%%%%%%%%%%%%%%%%%%%%%

\frame{

A criterion for {\em ``deterministic chaos''}
 is a ``suitable'' evolution function capable of ``unfolding'' the
 information of a random real associated with the ``true'' but unknown
 initial value $x_0$. I.e.,
 either the uncertainty $\delta x_0$ of the
 initial value  or a corresponding variation of the initial value
 increases with time.

For positive Lyaponov exponent, an initial uncertainty increases
exponentially with
 time.




}


\frame{
For continuous maps $G(x)=dx/dt$, one obtains a change of uncertainty
 $\delta x$ by \begin{equation}
 \label{lya}
{d(\delta x)\over dt} \approx {\partial G(x)\over \partial x}
\delta x  \quad .
 \end{equation}

With
 $\lambda (x_0)
 = \partial
G/\partial x\mid_{x_0}$, then
for constant
$\lambda $ the
solution  can be obtained by integration:  $\delta
x(t-t_0)=\delta x_0\exp [\lambda (t-t_0)]$.

 For $\lambda > 0$, a {\sl linear} increase in the precision
 of the initial value $\delta x_0$ renders merely a {\sl logarithmic}
 increase in the accuracy of the prediction.

}


%%%%%%%%%%%%%%%%%%%%%%%%%%%%%%%%%%%%%%%%%%%%%%%%%%%%%%%%%%%%%%%%%%%%%%%%%%%%%%%%%%%%%%%%%%%%%%%%%%%%%%%%%


\frame[shrink=2]{
\frametitle{Continuum urn}

With probability 1, a real initial value ``taken from the continuum urn'' is uncomputable.
Deterministic chaos ``unfolds'' the initial value.

In the measure theoretic sense, ``almost all'' reals are
 uncomputable. This can be demonstrated by the following argument:
 Let $M=\{ r_i\}$ be an  infinite point set (i.e., $M$ is a
 set of
 points $r_i$) which is denumerable and which is the subset of a dense
 set. Then, for instance, every $r_i\in M$ can be enclosed in the
 interval \begin{equation}
 I(i,\delta)= [r_i-2^{-i-1}\delta
 ,
 r_i+2^{-i-1}\delta]\quad ,
 \end{equation}
 where $\delta $ may be arbitrary small (we choose $\delta$ to be
 small enough that all intervals are disjoint).
 Since $M$ is denumerable, the measure $\mu$ of these intervals can
 be summed up, yielding
  \begin{equation}
 \sum_i \mu( I(i,\delta))= \delta \sum_{i=1}^\infty 2^{-i}=\delta \quad
 . \end{equation}
 From $\delta \rightarrow 0$ follows $\mu (M)=0$.


}

%%%%%%%%%%%%%%%%%%%%%%%%%%%%%%%%%%%%%%%%%%%%%%%%%%%%%%%%%%%%%%%%%%%%%%%%%%%%%%%%%%%%%%%%%%%%%%%%%%%%%%%%%
%%%%%%%%%%%%%%%%%%%%%%%%%%%%%%%%%%%%%%%%%%%%%%%%%%%%%%%%%%%%%%%%%%%%%%%%%%%%%%%%%%%%%%%%%%%%%%%%%%%%%%%%%
%%%%%%%%%%%%%%%%%%%%%%%%%%%%%%%%%%%%%%%%%%%%%%%%%%%%%%%%%%%%%%%%%%%%%%%%%%%%%%%%%%%%%%%%%%%%%%%%%%%%%%%%%
%%%%%%%%%%%%%%%%%%%%%%%%%%%%%%%%%%%%%%%%%%%%%%%%%%%%%%%%%%%%%%%%%%%%%%%%%%%%%%%%%%%%%%%%%%%%%%%%%%%%%%%%%



\frame[shrink=1.2]{
\frametitle{$x_{n+1}=f(x_n)=\alpha x_n(1-x_n)$}


\includegraphics<1>[height=9cm]{ch-big.jpg}

}


\frame{

for $\alpha \in [0,a_1] $, there exists one stable fixed point
 $x^\ast_1
=f(x^\ast_1 )$ and a system converges and remains at $x^\ast_1$;
}
\frame{
\frametitle{Periodic regime}
for $\alpha \in (a_1, a_\infty )$ there is, depending on the parameter
 $\alpha$, a hierarchy
 of fixed points and associated periodic trajectories. By varying
 $\alpha$ one notices a succession of
fixed point instabilities accompanied by bifurcations at $a_N$: if an
$N$'th
order fixed point $x^\ast_N$ is defined by its recurrence after $N$
computing steps (and not before), that is after $N$ iterations of $f$,
$$x_N^\ast = f^{(N)}(x_N^\ast )=\underbrace{f(f(\cdots f}_{N\; {\rm
times}} (x_N^\ast )\cdots ))\qquad ,$$
then $x_N^\ast $ characterises a periodic evolution in state space
(for the logistic equation, $a_1\approx 3.00$ and $a_\infty \approx
3.57$). for large $N$, the following scaling laws are
universal: for adjacent fixed points,
$$F_1 =\lim_{N\rightarrow \infty}{a_N-a_{N-1}\over
 a_{N+1}-a_N}=4.6692\cdots ,\quad
F_2 =\lim_{N\rightarrow \infty}{x_N^\ast-x_{N-1}^\ast \over
 x_{N+1}^\ast -x_N^\ast } = 2.5029\cdots
\qquad ;$$

}


\frame{
\frametitle{``Chaotic'' regime}

for $\alpha \in [a_\infty ,4)$ aperiodicity sets in, followed by a
fine-structure which is not explained here. In this regime the Lyapunov
exponent is mostly positive, the unfolding of the algorithmic
 information of the initial value.


}


\frame{
\frametitle{``Chaotic'' regime}
for $\alpha =4$ and after the variable transformation $x_n=\sin^2(\pi
X_n)$ one obtains a map $f:X_n\rightarrow X_{n+1}=2X_n$ (mod 1),
where (mod 1) means that one has to drop the integer part of $2X_n$.
By assuming a starting value $X_0$, the formal solution to $n$
iterations is $f^{(n)}(X_0)=X_n=2^nX_0$ (mod 1).
$f$ is easily computable:
if $X_0$ is in binary representation, $f^{(n)}$ is just $n$ times a left
 shift
of the digits of $X_0$, followed by a left truncation before the decimal
point.
Now assume $X_0\in (0,1)$ is Martin-L\"of/Solovay/Chaitin random.
Then the computable function $f^{(n)}(X_0)$ yields
 a Mar\-tin-L\"of\-/So\-lo\-vay/\-Chai\-tin  random evolution.
It should be stressed again that in {\em ``deterministic chaos,''} the
 evolution function $f$ itself is
computable / recursive, $X_0$ is random, and $f$ ``unfolds'' the
 ``information''
contained in $X_0$ in time.
(For $\alpha \in (4, \infty ) $ the evolution for most  points $x_0\in
(0,1)$ diverges.)

}





%%%%%%%%%%%%%%%%%%%%%%%%%%%%%%%%%%%%%%%%%%%%%%%%%%%%%%%%%%%%%%%%%%%%%%%%%%%%%%%%%%%%%%%%%%%%%%%%%%%%%%%%%
%%%%%%%%%%%%%%%%%%%%%%%%%%%%%%%%%%%%%%%%%%%%%%%%%%%%%%%%%%%%%%%%%%%%%%%%%%%%%%%%%%%%%%%%%%%%%%%%%%%%%%%%%
%%%%%%%%%%%%%%%%%%%%%%%%%%%%%%%%%%%%%%%%%%%%%%%%%%%%%%%%%%%%%%%%%%%%%%%%%%%%%%%%%%%%%%%%%%%%%%%%%%%%%%%%%
%%%%%%%%%%%%%%%%%%%%%%%%%%%%%%%%%%%%%%%%%%%%%%%%%%%%%%%%%%%%%%%%%%%%%%%%%%%%%%%%%%%%%%%%%%%%%%%%%%%%%%%%%

\section{Quantum unknowables}


\subsection{Random events}
\frame{
\frametitle{Random events: quantum coin tosses}

Commercial interface cards such as Quantis (http://www.idquantique.com) perform at a rate of 4 to 16 Mbit/s.
}

\subsection{Complementarity}
\frame[shrink=1.01]{
\frametitle{(Computational) Complementarity and generalized urn model}

Edward F. Moore,
{Gedanken-Experiments on Sequential Machines},
in {Automata Studies},
ed. by {C. E. Shannon and J. McCarthy},
{Princeton  University Press},
 {Princeton},
(1956)
\includegraphics{SubwayAutomaton.pdf}
}


\subsection{Value indefiniteness versus omniscience}
\frame[shrink=2]{
\frametitle{Proof of the Kochen-Specker theorem by Cabello et al.\\ in four-dimensional real vector space.}

\begin{center}
%TeXCAD Picture [4.pic]. Options:
%\grade{\on}
%\emlines{\off}
%\epic{\off}
%\beziermacro{\on}
%\reduce{\on}
%\snapping{\off}
%\quality{8.00}
%\graddiff{0.01}
%\snapasp{1}
%\zoom{5.6569}
\unitlength 1mm % = 2.85pt
\linethickness{0.8pt}
\ifx\plotpoint\undefined\newsavebox{\plotpoint}\fi % GNUPLOT compatibility
\begin{picture}(134.09,125.99)(0,0)
%\emline(86.39,101.96)(111.39,58.46)
\multiput(86.39,101.96)(.067385445,-.117250674){371}{\line(0,-1){.117250674}}
%\end
%\emline(86.39,14.96)(111.39,58.46)
\multiput(86.39,14.96)(.067385445,.117250674){371}{\line(0,1){.117250674}}
%\end
%\emline(36.47,101.96)(11.47,58.46)
\multiput(36.47,101.96)(-.067385445,-.117250674){371}{\line(0,-1){.117250674}}
%\end
%\emline(36.47,14.96)(11.47,58.46)
\multiput(36.47,14.96)(-.067385445,.117250674){371}{\line(0,1){.117250674}}
%\end
\put(86.39,101.71){\line(-1,0){50}}
\put(86.39,15.21){\line(-1,0){50}}
\put(86.28,101.76){\circle{2.97}}
\put(86.28,15.16){\circle{2.97}}
\put(93.53,89.21){\circle{2.97}}
\put(93.53,27.71){\circle{2.97}}
\put(29.24,89.21){\circle{2.97}}
\put(29.24,27.71){\circle{2.97}}
\put(102.37,73.47){\circle{2.97}}
\put(102.37,43.44){\circle{2.97}}
\put(20.4,73.47){\circle{2.97}}
\put(20.4,43.44){\circle{2.97}}
\put(111.21,58.45){\circle{2.97}}
\put(11.56,58.45){\circle{2.97}}
\put(36.34,101.76){\circle{2.97}}
\put(36.34,15.16){\circle{2.97}}
\put(52.99,101.76){\circle{2.97}}
\put(52.99,15.16){\circle{2.97}}
\put(69.68,101.76){\circle{2.97}}
\put(69.68,15.16){\circle{2.97}}
\qbezier(29.2,27.73)(23.55,-5.86)(52.99,15.24)
\qbezier(93.57,27.73)(99.22,-5.86)(69.78,15.24)
\qbezier(29.2,27.88)(36.93,75)(69.63,101.91)
\qbezier(93.57,27.88)(85.84,75)(53.13,101.91)
\qbezier(52.69,15.24)(87.47,40.96)(93.72,89.27)
\qbezier(70.08,15.24)(35.3,40.96)(29.05,89.27)
\qbezier(93.72,89.27)(98.4,125.99)(69.49,102.06)
\qbezier(29.05,89.27)(24.37,125.99)(53.28,102.06)
\qbezier(20.15,73.72)(-11.67,58.52)(20.15,43.31)
\qbezier(20.33,73.72)(61.34,93.16)(102.36,73.72)
\qbezier(102.36,73.72)(134.09,58.52)(102.53,43.31)
\qbezier(102.53,43.31)(60.99,23.43)(20.15,43.49)
\put(30.41,114.02){\makebox(0,0)[cc]{$(0,1,-1,0)$}}
\put(30.41,2.65){\makebox(0,0)[cc]{$(0,0,1,-1)$}}
\put(52.68,114.38){\makebox(0,0)[cc]{$(1,0,0,1)$}}
\put(52.68,2.3){\makebox(0,0)[cc]{$(1,-1,0,0)$}}
\put(91.93,114.2){\makebox(0,0)[cc]{$(-1,1,1,1)$}}
\put(91.93,2.48){\makebox(0,0)[cc]{$(1,1,1,1)$}}
\put(69.65,114.38){\makebox(0,0)[cc]{$(1,1,1,-1)$}}
\put(73.65,2.3){\makebox(0,0)[cc]{$(1,1,-1,-1)$}}
\put(103.24,94.22){\makebox(0,0)[cc]{$(1,1,-1,1)$}}
\put(19.45,94.22){\makebox(0,0)[cc]{$(0,1,1,0)$}}
\put(106.24,22.45){\makebox(0,0)[cc]{$(1,-1,1,-1)$}}
\put(19.45,22.45){\makebox(0,0)[cc]{$(0,0,1,1)$}}
\put(110.13,77.96){\makebox(0,0)[cc]{$(1,0,1,0)$}}
\put(12.55,77.96){\makebox(0,0)[cc]{$(0,0,0,1)$}}
\put(110.13,38.72){\makebox(0,0)[cc]{$(1,0,-1,0)$}}
\put(12.55,38.72){\makebox(0,0)[cc]{$(0,1,0,0)$}}
\put(120.92,57.98){\makebox(0,0)[l]{$(1,1,0,-1)$}}
\put(1.77,57.98){\makebox(0,0)[rc]{$(1,0,0,0)$}}
\end{picture}
\end{center}
}
\frame{
\frametitle{Kochen-Specker theorem cntd.}
Nine interconnected contexts (or four-pods) are represented by smooth, unbroken curves.
The graph contains possible quantum observables represented by 18 points, which are explicitly enumerated.
It cannot be colored by the two colors red (associated with truth)
and green (associated with falsity) such that every context contains exactly one red and three green points.
For, by construction, on the one hand, every red point occurs in exactly two contexts (four-pods), and hence
there is an even number of red points in a table containing the points of the contexts as columns
and the enumeration of contexts as rows.
On the other hand, there are nine contexts involved; thus by the rules it follows that there
is an odd number of red points in this table (exactly one per context).
Thus, our assumption about the colorability and therefore about
possible consistent truth assignments for this finite set of quantum observables
leads to a complete contradiction.
}



%%%%%%%%%%%%%%%%%%%%%%%%%%%%%%%%%%%%%%%%%%%%%%%%%%%%%%%%%%%%%%%%%%%%%%%%%%%%%%%%%%%%%%%%%%%%%%%%%%%%%%%%%
%%%%%%%%%%%%%%%%%%%%%%%%%%%%%%%%%%%%%%%%%%%%%%%%%%%%%%%%%%%%%%%%%%%%%%%%%%%%%%%%%%%%%%%%%%%%%%%%%%%%%%%%%
%%%%%%%%%%%%%%%%%%%%%%%%%%%%%%%%%%%%%%%%%%%%%%%%%%%%%%%%%%%%%%%%%%%%%%%%%%%%%%%%%%%%%%%%%%%%%%%%%%%%%%%%%
%%%%%%%%%%%%%%%%%%%%%%%%%%%%%%%%%%%%%%%%%%%%%%%%%%%%%%%%%%%%%%%%%%%%%%%%%%%%%%%%%%%%%%%%%%%%%%%%%%%%%%%%%






\end{document}


\frame{
\frametitle{ }



}

\frame{
\frametitle{ }



}

\frame{
\frametitle{ }



}

\frame{
\frametitle{ }



}

\frame{
\frametitle{ }



}

\frame{
\frametitle{ }



}

\frame{
\frametitle{ }



}



