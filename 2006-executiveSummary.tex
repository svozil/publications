\documentclass[pra,amssymb,preprint,12pt]{revtex4}
%\documentclass[amsfonts,a4,12pt]{article}
\RequirePackage[german]{babel}
\RequirePackage{times}
\RequirePackage{courier}
\RequirePackage{mathptm}
%\RequirePackage{bookman}
%\RequirePackage{helvetic}
%\RequirePackage{times}
\selectlanguage{german}
\RequirePackage[isolatin]{inputenc}
\begin{document}
\pagestyle{empty}
Alles was ich hier schreibe, erscheint vermutlich in ein paar Jahren als pathetischer klischeehafter, platit�denhafter Schwachsinn.

Worum geht es im Leben wirklich? Was ist das Leitmotiv?
In mir ist ein Ur-Erstaunen: warum alles ist und nicht Nicht-Ist.
Wie es organisiert ist, ist eine sekund�re Frage, mit der sich die Naturwissenschaften befassen.

Je tiefer wir zu blicken w�hnen, umso leerer wird die Welt.
Sichere Fundamente zerflie�en in Luft, in d�nne Luft,
und so wie  wesenlose Luftgesichte, so wird diese gro�e Erdkugel selbst,
und alles was sie in sich fasst, zerschmelzen, und nicht die mindeste Spur zur�cklassen.
Wir sind das  Zeug, woraus Tr�ume gemacht werden, und unser kleines Leben umfasst  vielleicht ein Schlaf.

Mein Leitmotiv ist die interstellare Raumfahrt: durchs Raue zu den Sternen. Als theoretischer Physiker befinde ich mich gewisserma�en noch in der Konzeptionsphase derselben --- unerf�llt und gl�cklos, aber zumindest bem�ht.
Das Schicksal, in unausweichliche historische Abl�ufe eingebettet zu sein, welche die entsprechenden Schritte nicht zulassen, schmerzt.
Und selbst wenn einem das ungeheure Gl�ck zufiele, ein Tor zu den Sternen zu �ffnen ---
man w�re doch nur auf einer der unz�hligen Welten, denen sich die Sterne �ffnen, ge�ffnet haben und �ffnen werden.

Meine Sehnsucht geh�rt dennoch den Griff zu den Sternen, sowie der allumfassenden romantischen Liebe.
Sehr viel habe ich auch hier gelitten, und leide noch immer.

Mein sch�nstes Geburtstagegeschenk ist die Entdeckung der Marx'schen Herbstsymphonie; voll von Sehnsucht und Leidenschaft.
Gibt es Erl�sung? Ja, vielleicht zumindest zeitweise. Bang ist mein Herz und darauf hoffend, endlich daheim anzukommen, um  im Schlaf vergessenes Gl�ck und Jugend neu zu lernen.

Die Beliebigkeit unserer sozialen, politischen Existenz und ihr existentielles Hineingeworfensein, sowie die ungleichen Rollenverteilungen sind schier unfassbar.
Da ist die Rolle des Milliard�rs,
die Rolle des armen Schluckers, die Rolle des zensurierenden Lehrers ---
so viele Schicksale, so viele Z�ge und Welten in ein und derselben Zeit!

Manches habe ich erreicht, das Meiste ist Fragment und wird es wohl auch bleiben.
Und immer wieder Fragen �ber Fragen ...

\end{document}
\end
