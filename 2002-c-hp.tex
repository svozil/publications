\documentclass{epl}

\title{Time independence of Boole-Bell conditions of possible classical experience}
\author{K. Svozil\inst{1}}
\institute{
  \inst{1} Institut f\"ur Theoretische Physik, University of Technology Vienna,
Wiedner Hauptstra\ss e 8-10/136, A-1040 Vienna, Austria
}
\pacs{03.65.Ud}{Entanglement and quantum nonlocality}

\begin{document}

\maketitle

\begin{abstract}
The theorem of Bell is a variant of Boole's legendary
consistency ``conditions of possible experience.''
Such an interpretation appears to be immune to arguments involving time dependencies
put forward recently by Hess and Philipp [Europhysics Letters {\bf 57}(6), 775-781, 2002],
although experiments need not be.
\end{abstract}

\section{Boole's ``conditions of possible experience'' and Pitowsky correlation polytopes}

In the middle of the 19th century the English mathematician George Boole
formulated a theory of ``conditions of possible experience'' (COPE)
\cite{Boole,Boole-62,Hailperin,pitowsky,Pit-94}.
These conditions subsume the consistency requirements
satisfied by relative frequencies or probabilities of
classical events.
They are expressed by certain equations or inequalities.
Here, the term ``classical'' refers to the fact that events
can be joined and united by the usual rules of Boolean algebra.

More recently, similar equations for a particular setup relevant in the
quantum mechanical context have been discussed by Bell, Clauser\&Horne and others
\cite{bell-87,cl-horne,chsh,clauser}.
Pitowsky has given a geometrical interpretation
of COPE in terms of correlation polytopes
\cite{pitowsky-89a,pitowsky,Pit-91,Pit-94}.
Thereby, the rows of the truth tables of events and their joints are interpreted
as vectors in a real linear vector space.
A correlation polytope is defined by taking all such vectors and interpreting them as
the extreme points of the polytope.

The Minkowski-Weyl representation theorem (e.g., \cite[p. 29]{ziegler})
states that  compact convex sets
are ``spanned'' by their extreme points; and furthermore that the
representation of this polytope by the inequalities corresponding to the planes of their
faces is an equivalent one.
Stated differently, every convex polytope has a dual description:
either as the convex hull of its vertices (V-representation),
or as the intersection of a finite number of half-spaces,
each one given by a linear inequality (H-representation).
The problem to obtain all inequalities from the vertices of a convex
polytope is known as the  hull problem.
It is computationally hard  \cite{Pit-91} but recursively enumerable.
One solution strategy is the Double Description Method \cite{MRTT53}.

The physical interpretation of the inequalities representing the boundaries of the
Pitowsky correlation polytope is  this:
Any face of the polytope has an ``inside'' and an ``outside,''
and corresponds to a Boole-Bell type inequality.
It can be viewed as a sort of demarcation line,
a maximal border, between the classically allowed probabilities
and the ones (outside of the polytope) which are inconsistent
with a classical description
of events as a Boolean algebra and Kolmogorivian probability ansatz.
Quantum probabilities exceed the borders of the classical correlation
polytopes: they "lie outside."
But also quantum probabilities are subject to certain constraints
\cite{cirelson:80,cirelson:87,cirelson,pit:range-2001}
and do not violate
the inequalities maximally.

As an example, we shall derive the Clauser-Horne (CH)
and the Clauser, Horne, Shimony, and Holt (CHSH) inequalities,
which must be obeyed by the probability distribution of two particles, whether separated or not.
Consider the four events $A_{1},A_{2},B_{1},B_{2}$
such as {\em ``measurement of
the electron spin in the $x_1$-direction on particle $A$ yields the value `up'.''}
 $A_{1},A_{2}$
correspond to
measurements on the first particle,
and $B_{1},B_{2}$
correspond to
measurements on the second particle,
Consider further
certain joint events
$A_1B_1$,
$A_1B_2$,
$A_2B_1$,
$A_2B_2$, interpreted as $A_1B_1\equiv${\em ``measurement of
the electron spin
in the $x_1$-direction on particle $A$
and
in the $x_1$-direction on particle $B$
yields the value `up'.''}
(Here ``spin'' denotes an arbitrary dichotomic observable which,
for the sake of contradiction with quantum mechanics,
is assumed to be a classical property.)
In order to derive the
CH inequalities, we list the $2^4$ extreme cases where the probability of
the
elementary events $A_{1},A_{2},B_{1},B_{2}$ are set to be either zero or
one in the truth table~\ref{t-tt-2000-poly},
where \ $t(A_{i}),\ t(B_{j})$ $\in \{0,1\}$.
Assume that each one of the sixteen
rows in the truth table is
a vector in an eight dimensional real vector space.
\begin{table}
\caption{Truth table corresponding to the CH inequalities.
$t(A_{i}),\ t(B_{j})$ $\in \{0,1\}$ \label{t-tt-2000-poly}}
\begin{tabular}{cccccccc}
\hline\hline
$A_{1}$ & $A_{2}$ & $B_{1}$ & $B_{2}$&
$A_{1}B_{1}$& $A_{1}B_{2}$ &$A_{2}B_{1}$ & $A_{2}B_{2}$ \\
\hline
$t(A_{1})$ & $t(A_{2})$ & $t(B_{1})$ & $t(B_{2})$&
$t(A_{1})t(B_{1})$& $t(A_{1})t(B_{2})$ & $t(A_{2})t(B_{1})$ & $t(A_{2})t(B_{2})$ \\
\hline\hline
\end{tabular}
\end{table}
Denote by ${\cal C}$ the convex
hull of the sixteen
vectors taken as vertices. ${\cal C}$\ is a correlation polytope. Now, let
$P$\ be any classical probability
distribution on the Boolean algebra generated by the events $
A_{1},A_{2},B_{1},B_{2} $.
The vector $
p=(P(A_{1}),\ P(A_{2}),\ P(B_{1}),\ P(B_{2}),\ P(A_{1}B_{1}),\
P(A_{1}B_{2}),\ P(A_{2}B_{1}),\ P(A_{2}B_{2}))$
is an element of ${\cal C}$ \cite[pp. 28-29]{pitowsky}.
That is, all classical probability distributions can be represented as
elements of one or more faces of the classical correlation polytope ${\cal C}$.
The faces corresponding to the inequalities are obtained by solving the hull problem; i.e.,
\begin{eqnarray}
0&\leq & P(A_{i}B_{j})\leq P(A_{i}),\ P(B_{j}) \qquad  i=1,\ 2,\qquad  j=1,\ 2,
\label{e-2002-c-hp2a} \\
1&\geq & P(A_{i}), P(B_{j}),P(A_{i}B_{j})\qquad i=1,\ 2,\qquad j=1,\ 2,
\label{e-2002-c-hp2b} \\
-1&\leq& P(A_{1}B_{1})+P(A_{1}B_{2})+P(A_{2}B_{2})-P(A_{2}B_{1})-P(A_{1})-P(B_{2}) \leq 0,
\label{e-2002-c-hp2c} \\
-1&\leq&  P(A_{2}B_{1})+P(A_{2}B_{2})+P(A_{1}B_{2})-P(A_{1}B_{1})-P(A_{2})-P(B_{2})\leq 0,
\label{e-2002-c-hp2d} \\
-1&\leq&  P(A_{1}B_{2})+P(A_{1}B_{1})+P(A_{2}B_{1})-P(A_{2}B_{2})-P(A_{1})-P(B_{1})\leq 0,
\label{e-2002-c-hp2e} \\
-1&\leq&  P(A_{2}B_{2})+P(A_{2}B_{1})+P(A_{1}B_{1})-P(A_{1}B_{2})-P(A_{2})-P(B_{1})\leq 0.
\label{e-2002-c-hp2f}
\end{eqnarray}
The CH inequality is among
(\ref{e-2002-c-hp2c})-(\ref{e-2002-c-hp2f}).
The CHSH
inequality is obtained by inserting the estimates of (\ref{e-2002-c-hp2b})
into
(\ref{e-2002-c-hp2c})-(\ref{e-2002-c-hp2f}).
Much more general examples generating a complete set of Boole-Bell type
inequalities for the Greenberger-Horne-Zeilinger (GHZ) setup and for
the case of two particles and three measurement types per particle
have been discussed by Pitowsky and Svozil in \cite{2000-poly}.

A derivation of Boole-Bell  inequalities
via the associated Pitowsky correlation polytope
is very elegant and convenient.
Moreover, it is free from auxiliary observables
which could be called ``hidden parameters.''
Those hidden classical parameters might or might not be present,
implicitly or implicitly,
but they have no influence on the syntax of the proof, nor on its semantic.


\section{Time dependence of Boole-Bell inequalities}

Boole's COPE and hence also all Boole-Bell type inequalities
have no time dependence whatsoever.
They just state consistency requirements for classical probabilities
and correlations of a multiparticle system.
Thereby, the particles could, but need not be spatially separated.
Indeed, for the rest of this discussion, we shall assume that they are
not spatially separated.
In such a case, attempts to reproduce
quantum-like behaviours by considering certain time dependencies and correlations
originating from it must fail:
Since  the two particles
need not be spatially separated
and can be measured simultaneously in a particular reference frame,
there needs not be a time difference between measurements
in all other relativistic space-time frames.
If there is no time difference,
there cannot be any time dependence and no correlations originating from them.

The considerations do not make any use of
quantum nonlocality,
because no spatial distance is involved in the
proposed physical setup at all.
Rather, it is based on consistency requirements.
These consistency requirements must be satisfied by classical particle properties.
Whether or not the single particles forming the whole system are spatially or temporally separated
does not matter at all.
Indeed, as long as the dimension of the associated Hilbert space
is larger than two, just a single particle suffices
for a violation of classical probabilities:
The Kochen-Specker theorem
\cite{specker-60,kochen1,peres,mermin-93,CalHerSvo,svozil-ql}
and related theorems
\cite{ZirlSchl-65,Alda,Alda2,kamber64,kamber65}
were motivated by a theorem of Gleason~\cite{Gleason}; for an experimental
test of certain related quantum features such as contextuality, see, e.g.,
\cite{redhead,cabello-98,Mi-Wein-Zu-2000,Si-Zu-Wein-Ze-2000}.
proves the nonexistence of two-valued measures or ``elements of physical reality''
(from which every classical probabilities is constructed)
already for single particles and associated Hilbert spaces of dimension larger than two.
That is, a single three-state particle suffices to prove the inconsistency
of the quantum mechanical probabilities with
classical noncontextual hidden parameter models.


Admittedly, bringing the entangled particle pair close together is not
the most natural thing to do, in particular in the spirit of
the  Einstein, Podolsky, and  Rosen
(EPR) paper \cite{epr},
in which spatial separateness is a criterion to assure the impossibility of
conspiratorial communication between the parts of the whole.
The experimental procedures testing a violation of the classical
COPE have so far designed to be spatially extended.
Spatial separateness has not been considered a flaw
of the setup but rather an essential feature,
although the very early arrangements
had problems with the intrinsic time correlations \cite{zeilinger-86}.
Just on the opposite, experiments in the past always have attempted to assure
spatial separateness between the entangled particles of an EPR-pair
(e.g., the Innsbruck experiment \cite{zeilinger-epr-99}).
The same could be said for experimental tests
of the GHZ case involving
three particles and a complete, and not merely statistical, contradiction.
Thus one way to perceive the argument of
Hess and Philipp \cite{Hess&Philipp2002}
(whose ideas have been earlier challenged
\cite{Gill-Weihs-Z-Z} and wo responded to this criticism in
\cite{Hess&Philipp2002a})
is the suggestion to look at quantum entanglement
between particles which are either not spatially separated at all,
or with varying distances.

\acknowledgments
I am indebted Marek Zukowski for bringing to my attention the paper by K. Hess and W. Philipp.


%\bibliography{svozil}
%\bibliographystyle{apsrev}
%\bibliographystyle{unsrt}
%\bibliographystyle{plain}

\begin{thebibliography}{10}

\bibitem{Boole}
George Boole.
\newblock {\em An investigation of the laws of thought}.
\newblock Dover edition, New York, 1958.

\bibitem{Boole-62}
George Boole.
\newblock On the theory of probabilities.
\newblock {\em Philosophical Transactions of the Royal Society of London},
  152:225--252, 1862.

\bibitem{Hailperin}
Theodore Hailperin.
\newblock {\em Boole's logic and probability (Studies in logic and the
  foundations of mathematics ; 85)}.
\newblock North-Holland, Amsterdam, 1976.

\bibitem{pitowsky}
Itamar Pitowsky.
\newblock {\em Quantum Probability---Quantum Logic}.
\newblock Springer, Berlin, 1989.

\bibitem{Pit-94}
Itamar Pitowsky.
\newblock {G}eorge {B}oole's `conditions od possible experience' and the
  quantum puzzle.
\newblock {\em Brit. J. Phil. Sci.}, 45:95--125, 1994.

\bibitem{bell-87}
John~S. Bell.
\newblock {\em Speakable and Unspeakable in Quantum Mechanics}.
\newblock Cambridge University Press, Cambridge, 1987.

\bibitem{cl-horne}
J.~F. Clauser and M.~A. Horne.
\newblock {\em Physical Review}, D10:526, 1974.

\bibitem{chsh}
J.~F. Clauser, M.~A. Horne, A.~Shimony, and R.~A. Holt.
\newblock {\em Physical Review Letters}, 23:880--884, 1969.

\bibitem{clauser}
J.~F. Clauser and A.~Shimony.
\newblock {B}ell's theorem: experimental tests and implications.
\newblock {\em Rep. Prog. Phys.}, 41:1881--1926, 1978.

\bibitem{pitowsky-89a}
Itamar Pitowsky.
\newblock From {G}eorge {B}oole to {J}ohn {B}ell: The origin of {B}ell's
  inequality.
\newblock In M.~Kafatos, editor, {\em {B}ell's Theorem, Quantum Theory and the
  Conceptions of the Universe}. Kluwer, Dordrecht, 1989.

\bibitem{Pit-91}
Itamar Pitowsky.
\newblock Correlation polytopes their geometry and complexity.
\newblock {\em Mathematical Programming}, 50:395--414, 1991.

\bibitem{ziegler}
G{\"{u}}nter~M. Ziegler.
\newblock {\em Lectures on Polytopes}.
\newblock Springer, New York, 1994.

\bibitem{MRTT53}
T.S. Motzkin, H.~Raiffa, G.L. Thompson, and R.M. Thrall.
\newblock The double description method.
\newblock In H.W. Kuhn and A.W.Tucker, editors, {\em Contributions to theory of
  games, Vol. 2}. Princeton University Press, New Jersey, Princeton, 1953.

\bibitem{cirelson:80}
B.~S. Cirel'son.
\newblock Quantum generalizations of {B}ell's inequality.
\newblock {\em Letters in Mathematical Physics}, 4:93--100, 1980.

\bibitem{cirelson:87}
B.~S. Tsirel'son.
\newblock Qantum analogues of the {B}ell inequalities. {T}he case of two
  spatially separated domains.
\newblock {\em Journal of Soviet Mathematics}, 36(4):557--570, 1987.

\bibitem{cirelson}
B.~S. {Cirel'son (=Tsirelson)}.
\newblock Some results and problems on quantum {B}ell-type inequalities.
\newblock {\em Hadronic Journal Supplement}, 8:329--345, 1993.

\bibitem{pit:range-2001}
Itamar Pitowsky.
\newblock Range theorems for quantum probability and entanglement.
\newblock In {\em Quantum Theory: Reconsideration of Foundations, Proceeding of
  the 2001 Vaxjo Conference}. World Scientific, Singapore.
\newblock forthcoming.

\bibitem{2000-poly}
Itamar Pitowsky and Karl Svozil.
\newblock New optimal tests of quantum nonlocality.
\newblock {\em Physical Review}, A64:014102, 2001.

\bibitem{specker-60}
Ernst Specker.
\newblock {D}ie {L}ogik nicht gleichzeitig entscheidbarer {A}ussagen.
\newblock {\em Dialectica}, 14:175--182, 1960.
\newblock Reprinted in \cite[pp. 175--182]{specker-ges}; English translation:
  {\it The logic of propositions which are not simultaneously decidable},
  reprinted in \cite[pp. 135-140]{hooker}.

\bibitem{kochen1}
Simon Kochen and Ernst~P. Specker.
\newblock The problem of hidden variables in quantum mechanics.
\newblock {\em Journal of Mathematics and Mechanics}, 17(1):59--87, 1967.
\newblock Reprinted in \cite[pp. 235--263]{specker-ges}.

\bibitem{peres}
Asher Peres.
\newblock {\em Quantum Theory: Concepts and Methods}.
\newblock Kluwer Academic Publishers, Dordrecht, 1993.

\bibitem{mermin-93}
N.~D. Mermin.
\newblock Hidden variables and the two theorems of {J}ohn {B}ell.
\newblock {\em Reviews of Modern Physics}, 65:803--815, 1993.

\bibitem{CalHerSvo}
Cristian Calude, Peter Hertling, and Karl Svozil.
\newblock Embedding quantum universes into classical ones.
\newblock {\em Foundations of Physics}, 29(3):349--379, 1999.

\bibitem{svozil-ql}
Karl Svozil.
\newblock {\em Quantum Logic}.
\newblock Springer, Singapore, 1998.

\bibitem{ZirlSchl-65}
Neal Zierler and Michael Schlessinger.
\newblock Boolean embeddings of orthomodular sets and quantum logic.
\newblock {\em Duke Mathematical Journal}, 32:251--262, 1965.

\bibitem{Alda}
V.~Alda.
\newblock On\/ {\rm 0-1} measures for projectors i.
\newblock {\em Aplik. mate.}, 25:373--374, 1980.

\bibitem{Alda2}
V.~Alda.
\newblock On\/ {\rm 0-1} measures for projectors ii.
\newblock {\em Aplik. mate.}, 26:57--58, 1981.

\bibitem{kamber64}
Franz Kamber.
\newblock Die {S}truktur des {A}ussagenkalk{\"{u}}ls in einer physikalischen
  {T}heorie.
\newblock {\em Nachr. Akad. Wiss. G{\"{o}}ttingen}, 10:103--124, 1964.

\bibitem{kamber65}
Franz Kamber.
\newblock Zweiwertige {W}ahrscheinlichkeitsfunktionen auf
  orthokomplement{\"{a}}ren {V}erb{\"{a}}nden.
\newblock {\em Mathematische Annalen}, 158:158--196, 1965.

\bibitem{Gleason}
Andrew~M. Gleason.
\newblock Measures on the closed subspaces of a {H}ilbert space.
\newblock {\em Journal of Mathematics and Mechanics}, 6:885--893, 1957.

\bibitem{redhead}
Michael Redhead.
\newblock {\em Incompleteness, Nonlocality, and Realism: A Prolegomenon to the
  Philosophy of Quantum Mechanics}.
\newblock Clarendon Press, Oxford, 1990.

\bibitem{cabello-98}
Ad{\'{a}}n Cabello.
\newblock Proposed experimental tests of the {B}ell-{K}ochen-{S}pecker theorem.
\newblock {\em Physical Review Letters}, 80(9):1797--1799, 2002.

\bibitem{Mi-Wein-Zu-2000}
Markus Michler, Harald Weinfurter, and Marek Zukowski.
\newblock Experiments towards falsification of noncontextual hidden variable
  theories.
\newblock {\em Physical Review Letters}, 84(24):5457--5461, 2000.
\newblock eprint quant-ph/0009061.

\bibitem{Si-Zu-Wein-Ze-2000}
Markus Michler, Harald Weinfurter, and Marek Zukowski.
\newblock Feasible ``{K}ochen-{S}pecker'' experiment with single particles.
\newblock {\em Physical Review Letters}, 85(9):1783--1786, 2000.
\newblock eprint quant-ph/0009074.

\bibitem{epr}
Albert Einstein, Boris Podolsky, and Nathan Rosen.
\newblock Can quantum-mechanical description of physical reality be considered
  complete?
\newblock {\em Physical Review}, 47:777--780, 1935.
\newblock Reprinted in \cite[pages. 138-141]{wheeler-Zurek:83}.

\bibitem{zeilinger-86}
Anton Zeilinger.
\newblock Testing {B}ell's inequalities with periodic switching.
\newblock {\em Phys.Lett. A}, 118:1--2, 1986.

\bibitem{zeilinger-epr-99}
G.~Weihs, T.~Jennewein, C.~Simon, H.~Weinfurter, and A.~Zeilinger.
\newblock {B}ell experiment under strict {E}instein locality conditions.
\newblock In D.~Greenberger, editor, {\em Epistemological and Experimental
  Perspectives on Quantum Physics}, pages 267--269, Dordrecht, 1999. Kluwer.
\newblock Vienna Circle Institute yearbook ; 7 = 1999.

\bibitem{Hess&Philipp2002}
K.~Hess and W.~Philipp.
\newblock Exclusion of time in the theorem of {B}ell.
\newblock {\em Europhysics Letters}, 57(6):775--781, 2002.

\bibitem{Gill-Weihs-Z-Z}
R.D. Gill, G.~Weihs, A.~Zeilinger, and M.~Zukowski.
\newblock Comment on ``{E}xclusion of time in the theorem of {B}ell'' by {K}.
  {H}ess and {W}. {P}hilipp.
\newblock eprint quant-ph/0204169, 2002.

\bibitem{Hess&Philipp2002a}
K.~Hess and W.~Philipp.
\newblock Logical inconsistencies in proofs of the theorem of bell, 2002.
\newblock eprint quant-ph/0206046.

\bibitem{specker-ges}
Ernst Specker.
\newblock {\em Selecta}.
\newblock Birkh{\"{a}}user Verlag, Basel, 1990.

\bibitem{hooker}
Clifford~Alan Hooker.
\newblock {\em The logico-algebraic approach to quantum mechanics}.
\newblock D. Reidel Pub. Co., Dordrecht; Boston, [1975]-1979.

\bibitem{wheeler-Zurek:83}
John~Archibald Wheeler and Wojciech~Hubert Zurek.
\newblock {\em Quantum Theory and Measurement}.
\newblock Princeton University Press, Princeton, 1983.

\end{thebibliography}


\end{document}

%See fig.~\ref{f.1}, table~\ref{t.1} and eq.~(\ref{e.1}).
%See also~\cite{b.a,b.b}.
%\begin{equation}
%\label{e.1}
%0\neq1
%\end{equation}
%
%\begin{figure}
%\caption{Figure caption.}
%\label{f.1}
%\end{figure}
%
%\begin{table}
%\caption{Table caption.}
%\label{t.1}
%\begin{center}
%\begin{tabular}{lcr}
%first  & table & row\\
%second & table & row
%\end{tabular}
%\end{center}
%\end{table}
%
%%\acknowledgments
%%Paper text.
%%
%%\begin{thebibliography}{0}
%
%\bibitem{b.a}
% \Name{Author F., Author S. \and Author T.}
% \REVIEW{Some Rev. A}{69}{1969}{9691}.
%
%\bibitem{b.b}
% \Name{Author F. \and Author S.}
% \Book{Some Book of Interest}
% \Editor{A. Editor}
% \Vol{9}
% \Publ{Publishing house, City}
% \Year{1939}
% \Page{666}.
%
%\bibitem{b.c}
% \Editor{Editor A.}
% \Book{Some Book of Interest}
% \Vol{9}
% \Publ{Publishing house, City}
% \Year{1939}
% \Page{666}.
%
%\end{thebibliography}
