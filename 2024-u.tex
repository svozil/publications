\newif\ifws
%\wstrue
\ifws

\documentclass{article}

\usepackage{graphicx}        % standard LaTeX graphics tool
\usepackage[dvipsnames]{xcolor}

\usepackage{hyperref}
\hypersetup{
    colorlinks,
    linkcolor={blue!80!black},
    citecolor={red!75!black},
    urlcolor={blue!80!black}
}

% Damit die Verwendung der deutschen Sprache nicht ganz so umst\"andlich wird,
% sollte man die folgenden Pakete einbinden:


%German
%\usepackage[latin1]{inputenc}% erm\"oglich die direkte Eingabe der Umlaute
%\usepackage[T1]{fontenc} % das Trennen der Umlaute
%\usepackage{ngerman} % hiermit werden deutsche Bezeichnungen genutzt und
                     % die W\"orter werden anhand der neue Rechtschreibung
                     % automatisch getrennt.
%\title{Ridding Unitarity Through Nested Wigner's Friends}
\title{From Unitarity to Irreversibility: The Role of Infinite Tensor Products and Nested Wigner's Friends}
%Nesting of Wigner's friends gets rid of unitarity  at the end
%"Breaking Unitarity Through Nested Wigner's Friend Scenarios'
%"Unitarity Violation in Extended Wigner's Friend Experiments'
%"Nested Wigner's Friend Paradigms and the Demise of Unitarity'
%"Exploring Unitarity Breakdown via Nested Wigner's Friend Models'
%"Challenging Unitarity with Nested Wigner's Friend Scenarios'

\author{Karl Svozil \\
        Institute for Theoretical Physics,
TU Wien,  \\
Wiedner Hauptstrasse 8-10/136,
1040 Vienna,  Austria
        }

\date{\today}
% Hinweis: \title{um was auch immer es geht}, \author{wer es auch immer
% geschrieben hat} und  \date{wann auch immer das war} k\"onnen vor
% oder nach dem  Kommando \begin{document} stehen
% Aber der \maketitle Befehl mu\ss{} nach dem \begin{document} Kommando stehen!
\begin{document}

\maketitle


\begin{abstract}
\end{abstract}


\else
\PassOptionsToPackage{dvipsnames}{xcolor}
\documentclass[
reprint,
%   preprint,
 % twocolumn,
 %superscriptaddress,
 %groupedaddress,
 %unsortedaddress,
 %runinaddress,
 %frontmatterverbose,
 showpacs,
 showkeys,
 preprintnumbers,
 %nofootinbib,
 %nobibnotes,
 %bibnotes,
 amsmath,amssymb,
 aps,
 % prl,
  pra,
 % prb,
 % rmp,
 %prstab,
 %prstper,
  longbibliography,
 floatfix,
 %lengthcheck,
 ]{revtex4-2}

%\usepackage{cdmtcs-pdf}

\usepackage{mathptmx}% http://ctan.org/pkg/mathptmx



\usepackage{amssymb,amsthm,amsmath}

\usepackage{mathbbol}

\usepackage{tikz}
\usetikzlibrary{calc,math}
\usepackage {pgfplots}
\pgfplotsset {compat=1.8}
\usepackage{graphicx}% Include figure files
%\usepackage{url}

\usepackage{xcolor}

\usepackage{hyperref}
\hypersetup{
    colorlinks,
    linkcolor={blue},
    citecolor={red!75!black},
    urlcolor={blue}
}

\usepackage{soul}
\usepackage{tcolorbox}
\tcbuselibrary{breakable} % Load the breakable library

\begin{document}


\title{From Unitarity to Irreversibility: The Role of Infinite Tensor Products and Nested Wigner's Friends}

\author{Karl Svozil}
\email{svozil@tuwien.ac.at}
\homepage{http://tph.tuwien.ac.at/~svozil}

\affiliation{Institute for Theoretical Physics,
TU Wien,
Wiedner Hauptstrasse 8-10/136,
1040 Vienna,  Austria}



\date{\today}

\begin{abstract}
The transition from  unitary, reversible von Neumann-Everett quantum processes to non-unitary, irreversible processes and measurements is explored through infinite tensor products interpreted as nested, chained, or iterated Wigner's friend scenarios. Infinite tensor products can disrupt unitary equivalence through sectorization and factorization, drawing parallels to concepts from real analysis, recursive mathematics, and statistical physics.
\end{abstract}

%\pacs{03.65.Aa, 03.65.Ta, 03.65.Ud, 03.67.-a}
\keywords{quantum measurement,
quantum decoherence,
infinite tensor products,
nested Wigner's friend,
quantum decoherence,
von Neumann algebras,
irreversibility,
quantum entanglement}
%\preprint{CDMTCS preprint nr. x}

\maketitle

\newpage
\fi



\section{Introduction}

Unitary equivalence can be surpassed through infinite means.
This observation is consistent with findings in number theory and analysis, where finite operations on rational numbers cannot yield results beyond the rational domain.
However, when infinite methods and techniques are employed---such as Cantor's diagonalization~\cite{cantor-set-engl,Yanofsky-BSL:9051621} or the construction of Specker sequences~\cite{specker57,specker-ges,kreisel}, exemplified by Chaitin's halting probability~\cite{chaitin3,calude-dinneen06}---it
becomes possible to conceptualize irrational, incomputable, and random (algorithmically incompressible) numbers.

The unitary group, which formalizes quantum state evolution (excluding irreversible measurements and processes such as tracings), is, like all groups, inherently `hermetic' by definition. In particular, its closedness under unitary transformations reflects a fundamental property of group theory, connected to mere permutations or one-to-one transformations of the identity element. Consequently, it seems that irreversibility cannot emerge from purely unitary evolution.
To explore this further, let us revisit the historical arguments that gave rise to the conundrum posed by von Neumann and others.


%\section{Interactions from measurement create entanglement}

In the von Neumann scheme for ideal quantum measurement~\cite{landau-1927,wigner-1963},
the `object' is prepared in an initial state $\vert \psi \rangle$.
With respect to a `mismatching' context (relative to that preparation)---or,
 equivalently, orthonormal basis or maximal operator~\cite[Satz~8]{v-neumann-31} (\cite[Theorem~1, \S~84]{halmos-vs})---$\vert \psi \rangle$ is in a coherent superposition (linear combination) $\vert \psi \rangle = \sum_{i=1}^n a_i \vert \psi_i \rangle$ of (basis) elements $\vert \psi_i\rangle$ of that context.

The `measurement ancilla'---along with synonymous terms such as `provision', `component', or `arrangement'---should be represented by another state,
denoted as $\vert \varphi \rangle$.
This state, in relation to a suitable basis ${\vert \varphi_1 \rangle, \ldots, \vert \varphi_n \rangle}$,
can also be expressed as a coherent superposition: $\vert \varphi \rangle = \sum_{j=1}^n b_j \vert \varphi_j \rangle$.
When an interaction occurs between the `object' and the `measurement ancilla', the combined state $\vert \Psi \rangle$
\begin{equation}
\vert \Psi \rangle = \sum_{i,j=1}^n c_{ij} \vert \psi_i \rangle \otimes \vert \varphi_j \rangle =
\sum_{i,j=1}^n c_{ij} \vert \psi_i \varphi_j \rangle
\label{2024-u-vNsiqm}
\end{equation}
becomes a non-factorizable tensor product, meaning that the coefficients $c_{ij}$
cannot be written as products $a_i b_j$.

From now on, when referring to the `object' and the `measurement ancilla', apostrophes will be omitted. Since in entangled systems individuality is traded for relationality between individual components~\cite{zeil-99}, any conceptualization of a Heisenberg cut between these entangled constituents is a classical notion that may be maintained for all practical purposes (FAPP~\cite{bell-a}) but, strictly speaking, is not applicable.


\section{Infinite tensor products}

By recursively applying the von Neumann scheme for ideal quantum measurement~(\ref{2024-u-vNsiqm}), we can construct increasingly larger product spaces as more factors are added. To form a Hilbert space, we take the closure of this space under a suitable norm derived from the inner product. This process can be understood as taking the `double dual', or more specifically, the dual of the vector space of all bilinear forms on the vector spaces participating in the product~\cite{halmos-vs}.

%\begin{equation}
%\vert \Psi \rangle  =
%\lim_{k \rightarrow \infty}
%\vert \Psi_k \rangle
%=
%\lim_{k \rightarrow \infty}
%\sum_{i_1,\ldots ,i_k=1}^n c_{i_1\cdots  i_k} \vert \psi_{i_1} \cdots  \psi_{i_k} \rangle
%.
%\label{2024-u-vNsiqmlimit}
%\end{equation}




\subsection{Elementary Tensors as Products of Basis Vectors}

Given a countable collection of Hilbert spaces
\(
\Big\{  \mathcal{H}_n \Big| \; 1 \le k_n \le d_n \in \mathbb{N} \Big\}
\),
let
\(
\Big\{ \vert k_n \rangle  \Big| \; 1 \le k_n \le d_n \in \mathbb{N} \Big\}
\)
be an orthonormal basis for each \(\mathcal{H}_n\),
where the dimension \(d_n\) of \(\mathcal{H}_n\) could be a finite positive integer or countably infinite.

An elementary tensor product \(\bigotimes_{n=1}^\infty \vert k_n \rangle \) is then given by
\[ \bigotimes_{n=1}^\infty \vert k_n \rangle
= \vert k_1 \rangle \otimes \vert k_2 \rangle \otimes \vert k_3 \rangle \otimes \cdots
= \vert k_1 k_2 k_3 \cdots \rangle
\]
where \(\vert k_n \rangle\) is the $k_n$th basis vector from \(\mathcal{H}_n\).

%To keep things simple, one of von Neumann's major achievements---the discovery of
%equivalence classes~\cite[Theorem~I, p.~24]{vonNeumann1939},
%referred to as sectors by Van Den Bossche and Grangier\cite[Section~2.3]{van-den-bossche-2023-a}---will not be reviewed here.
%For a discussion on the relationship between sectors and factors, briefly covered later in
%Subsections~\ref{2024-u-factors} and
%Subsections~\ref{2024-u-sectors},
%see also Sorce~\cite[Section~5]{sorce-2023}.

\subsection{Tensor Product Space}

Let $I$ be a countable (enumerable) infinite index set, identified with the set of all natural numbers, $\mathbb{N}$.
The labelling $n$ represents the $n$th subfactor of the tensor product.
For the sake of simplicity, from now on, we will consider the set of all natural numbers as our index set.
So, whenever we write, say, $\bigotimes_{n=1}^\infty | k_n \rangle $ we really mean $\bigotimes_{n \in I} | n \rangle$.

To form the tensor product space \(\bigotimes_{n=1}^\infty \mathcal{H}_n\)
\begin{itemize}

\item[(i)]
we start with elementary tensors \(\bigotimes_{n=1}^\infty \vert k_n \rangle \),
where $n$ labels the $n$th subfactor of the tensor product,
and $k_n$ represents the $k_n$th basis vector in \(\mathcal{H}_n\).

\item[(ii)]
We define the inner product on elementary tensors by the product of the individual inner products~\cite[Definition~II.5., p.~63]{Guenin1969}:
\[ \left\langle \bigotimes_{n=1}^\infty \vert k_n\rangle \middle| \bigotimes_{n=1}^\infty \vert j_n\rangle  \right\rangle
= \begin{cases}
    \prod_{n=1}^\infty \langle k_n \vert j_n \rangle_{\mathcal{H}_n} &  \text{converging,}\\
    0 & \text{otherwise.}
\end{cases}
\]
Nonvanishing inner products will later, in Subsection~\ref{2024-u-sectors}, serve as a criterion for vectors to belong to the same sector.

\item[(iii)]
We then consider finite linear combinations of these elementary tensors:
   \[ \sum_{i} c_i \left( \bigotimes_{n=1}^\infty \vert {k_n^{(i)}} \rangle \right), \]
   where \(c_i\) are complex coefficients and \(\vert k_n^{(i)} \rangle\)
are basis vectors in the elementary tensor product labeled by a countable (enumerable) index~$i$ (see discussion later).

\item[(iv)]
We finally obtain the Hilbert space \(\bigotimes_{n=1}^\infty \mathcal{H}_n\) by taking the completion of the space of finite linear combinations of elementary tensors with respect to the norm induced by the inner product, ensuring that the space is complete and satisfies the properties of a Hilbert space.

\end{itemize}

By defining elementary tensors as products of basis vectors from the bases of the factors,
a concrete and manageable set of elementary tensors is obtained, spanning the tensor product space.
This approach, derived from finite tensor products~\cite[Theorem~1, \S~24,25]{halmos-vs},
simplifies both the definition and the computation of the inner product, ensuring that the space \(\bigotimes_{n=1}^\infty \mathcal{H}_n\) has a well-defined Hilbert space structure.

However, this construction does not directly address the uncountable infinity of elementary products. To illustrate this,
we can draw an analogy with the representation of real numbers as expansions in an $n$-ary system, where they are encoded using a (finite) set of basis elements.
Just as Cantor's diagonal argument shows that the reals cannot be enumerated by any countable set of indices,
so too can we not enumerate the uncountable infinity of elementary products in the infinite tensor product.
In von Neumann's own words \textit{``generalisations of the direct product lead to higher
set-theoretical powers (G. Cantor's ``Alephs'')''}~\cite[S~4, p.~4]{vonNeumann1939}.
Following Von Neumann's `incomplete infinite direct products'~\cite[Chapter~4]{vonNeumann1939},
Thirring and Wehrl  define the infinite tensor product in terms of equivalence classes~\cite[\S~2]{thirring-1967}
(see also Thirring~\cite[Definition~II.4., p.~63]{Guenin1969}) discussed later in the context of sectorization.
One could even go so far as to suspect that many of the upcoming issues related to continuity originate from this fact.



\subsection{Violation of unitary equivalence}

In finite dimensions, unitarity is a property of a single operator, characterized by its ability to preserve the inner product and possessing an inverse equal to its conjugate transpose.
On the other hand, unitary equivalence is a relation between two operators or orthonormal bases, signifying that one can be transformed into the other via a unitary transformation.
Fundamentally, unitarity captures the properties of an individual operator, whereas unitary equivalence captures the relationship between two operators or orthonormal bases.

Infinite tensor products pose significant challenges to maintaining unitary equivalence, primarily due to difficulties in defining a consistent inner product,
achieving proper normalization, preserving the required topological structure of the Hilbert space, and managing unbounded operators.
These challenges make it problematic to uphold the fundamental principles of quantum mechanics,
including the ability to execute arbitrary unitary transformations within the Hilbert space.
Notably, certain dynamical processes, such as the interaction between sectors
(as explored in the infinite limit in subsection~\ref{2024-u-sectors}),
become severely restricted when confined to finite resources.


\subsubsection{Inner product and orthogonality}



The inner product is a crucial concept in quantum mechanics, ensuring that probability amplitudes are well-defined and that the evolution is unitary.
However, in an infinite tensor product space, defining an inner product that adheres to the properties of a Hilbert space poses significant challenges.
Key among these challenges are issues with convergence, which are closely tied to the orthogonality of states.

As long as the tensor product is finite, the inner product is well-behaved.
When we extend to an infinite tensor product, say
\( | \Psi \rangle = \bigotimes_{i=1}^{\infty}| \psi_i \rangle \) and \( |  \Phi \rangle = \bigotimes_{i=1}^{\infty} | \phi_i \rangle \),
the inner product would apparently be
\(
\langle \Psi \vert \Phi \rangle = \prod_{i=1}^{\infty} \langle \psi_i \vert \phi_i \rangle
\).

The central issue is whether this infinite product converges to a non-zero value or not.
Suppose, for the sake of demonstration, that each \( \langle \psi_i \vert \phi_i \rangle \)
is very slightly less than 1. As a consequence, the infinite product can converge to zero, and thus those
vectors which are only `slightly apart' appear orthogonal.
Formally, suppose that
\(
\langle \psi_i \vert \phi_i \rangle = 1 - \epsilon_i  = \delta_i
\),
or
\(
  \epsilon_i = 1 - \langle \psi_i \vert \phi_i \rangle
\),
where
\(
0 < \epsilon_i \ll 1
\).
For a large number of factors, the infinite product behaves approximately as
\begin{equation}
\langle \Psi \vert \Phi \rangle = \prod_{i=1}^{\infty} (1 - \epsilon_i) \approx \exp\left(-\sum_{i=1}^{\infty} \epsilon_i\right)
.
\label{2024-u-dwsp}
\end{equation}
If the series \( \sum_{i=1}^{\infty} \epsilon_i \) diverges (even if slowly), this product will converge to zero, that is,
\(
\prod_{i=1}^{\infty} (1 - \epsilon_i) \rightarrow 0.
\)

% ### Implications for Inner Products


Furthermore, the inner product would also become zero for states $\vert \Psi \rangle$ and $\vert \Phi \rangle$
that differ in only a single or a finite number of the infinitely many subfactor components, where $\langle \psi_i \vert \phi_i \rangle = 0$, with all the rest being identical.
Additionally, there may be issues related to the phases, as explored by von Neumann~\cite{vonNeumann1939}
and by Van Den Bossche and Grangier~\cite{van-den-bossche-2023-a}.


This highlights the challenges infinite tensor products face in preserving a consistent inner product structure.
In numerous instances, the inner product may become undefined or yield counterintuitive outcomes,
contravening the anticipated properties of a Hilbert space and subsequently impacting unitary equivalence.


\subsubsection{Norm}

Issues with inner products in turn translate into problems with normalization, as
the polarization identity
expresses the inner product of two vectors in terms of the norm of their differences; that is,
\(
\langle  \Psi \vert  \Phi \rangle
=
\frac{1}{4}\left[
\|   \Psi +  \Phi  \|^2
-
\|   \Psi -  \Phi  \|^2
+ i
\left(
 \|   \Psi - i \Phi  \|^2
-
\|   \Psi + i \Phi  \|^2
\right)
\right]
\).
Thus, for $\langle  \Psi \vert  \Phi \rangle =0$,
$\|   \Psi +  \Phi  \|^2 = \|   \Psi -  \Phi  \|^2$,
and
$\|   \Psi - i \Phi  \|^2
-
\|   \Psi + i \Phi  \|^2  $.
This is true for finite tensor products but not necessarily for infinite ones
if, as before,
vectors \(  \vert \Psi \rangle \) and \(  \vert \Phi \rangle \)
represent physically distinct states located `close to each other',
such that the subfactors $\langle \psi_i \vert \phi_i \rangle = 1 - \epsilon_i$  where
\(
0 < \epsilon_i \ll 1
\).

As before this applies also to states $\vert \Psi \rangle$ and $\vert \Phi \rangle$
differing in only a single one or a finite number of infinitely many subfactor components
where $\langle \psi_i \vert \phi_i \rangle = 0$, with all others remaining identical.


%Topological Considerations:

%The Hilbert space structure relies on the completeness and the norm topology. For infinite tensor products, the resulting space might not be complete or might not respect the norm topology. This affects the ability to define unitary operators that preserve the inner product structure, leading to potential violations of unitarity.

\subsubsection{Bounded Operators}


Let us consider an example involving an infinite tensor product of projection operators to illustrate issues
with bounded operators on infinite tensor products.

Consider the Hilbert space \(\mathcal{H} = \mathbb{C}^2\) (2-dimensional complex space).
Let \(E\) be the rank-one projection operator onto the subspace spanned by the vector
\(\vert \uparrow \rangle = \begin{pmatrix}1 , 0\end{pmatrix}^\intercal\), with
\(
E = \text{diag}\begin{pmatrix} 1 , 0 \end{pmatrix}
\).
Now, consider the operator \(F = \bigotimes_{n=1}^{\infty} E\), which is the infinite tensor product of \(E\) with itself.

Initially, it may seem that \(E\) being a projection operator with \(\|E\| = 1\), \(F\) would be a well-defined bounded operator with \(\|F\| = 1\). However, this is not the case. To understand why,
let us examine the action of \(F\) on specific vectors.

Let us represent a general vector in the infinite tensor product space as \(\vert \psi \rangle = \bigotimes_{n=1}^{\infty} \vert \psi_n \rangle\), where \(\vert \psi_n \rangle\) are vectors in \(\mathcal{H}_n\).
For simplicity, assume each \(\vert \psi_n \rangle\) is a normalized vector in \(\mathbb{C}^2\).

When \(F\) is applied to \(\vert \psi \rangle\), we get
\(
F \vert \psi \rangle = \bigotimes_{n=1}^{\infty} E \vert \psi_n \rangle
\).

Since \(E\) projects onto \(\begin{pmatrix} 1 , 0 \end{pmatrix}^\intercal\), the resulting vector will be non-zero only if each \(\vert \psi_n \rangle\) has a component along \(\begin{pmatrix} 1 , 0 \end{pmatrix}^\intercal\). In an infinite product, the probability of each \(\vert \psi_n \rangle\) having a non-zero component along \(\begin{pmatrix} 1 , 0 \end{pmatrix}^\intercal\) diminishes rapidly, effectively leading to the result that \(F \vert \psi \rangle = 0\) for almost all \(\vert \psi \rangle\).


For instance, consider the vector \(\vert \psi \rangle = \vert \uparrow \rangle \otimes \vert \uparrow \rangle \otimes \cdots\), where
\(F \vert \psi \rangle = \vert \psi \rangle\) and \(\| F \vert \psi \rangle \| = \|\vert \psi \rangle\| = 1\).
On the other hand, for any vector
containing a component orthogonal to \(\begin{pmatrix} 1 , 0 \end{pmatrix}^\intercal\) in at least one factor
$
\vert \downarrow \rangle = \begin{pmatrix}  0,1  \end{pmatrix}^\intercal
$,
\(F\) maps it to the zero vector. For example, for the vector
\(
\vert \varphi \rangle = \vert \uparrow \rangle \otimes \vert \uparrow \rangle \otimes \cdots \otimes
\vert \downarrow \rangle \otimes \vert \uparrow \rangle \otimes \cdots,
\)
we have \(F \vert \varphi \rangle = 0\).

This demonstrates that the infinite tensor product \(F = \bigotimes_{n=1}^\infty E\) does not behave as a well-defined bounded operator in the infinite tensor product space. Although \(F\) leaves certain vectors unchanged---those entirely within the span of \(\begin{pmatrix} 1 , 0 \end{pmatrix}^\intercal\)---it maps any vector with even a single orthogonal component to zero.
This behavior leads to some counterintuitive physical properties because \(F\) is extremely sensitive to changes in its input: changing even one factor from \(\begin{pmatrix}1 , 0\end{pmatrix}^\intercal\) to any other vector results in mapping the vector to zero.

Furthermore, the behavior of \(F\) is consistent with finite tensor products of \(E\). In both finite and infinite cases, the result is a rank-one projection. However, the key difference is that in the infinite case, this leads to a projection onto a one-dimensional subspace of an infinite-dimensional space, which has some unique properties.



\subsection{Sectorization}
\label{2024-u-sectors}

%#### 2. **Choice of Reference Vectors (Normalization)**


Von Neumann's concept of `incomplete infinite direct products'~\cite[Chapter~4]{vonNeumann1939}, as reflected in the notion of superselection sectors in algebraic quantum field theory~\cite{haag-1964,doplicher-1969,doplicher-1971,doplicher-1974,haag-1996}, provides a solution to the problem that a single (or finitely many) subfactor(s) could nullify the inner product by `grouping' vectors that differ from each other in only finitely many subfactors, or are otherwise `close to' each other.
These groupings are mutually orthogonal and can be demonstrated to be equivalence classes referred to as sectors.
This implies that vectors from different sectors differ in an infinite number of subfactors and are orthogonal in the sense that their scalar product is zero.
In von Neumann's own words,
\textit{``What happens could be described in the quantum-mechanical
terminology as a `splitting up' of [[the full tensor product]] into `non-intercombining
systems of states', corresponding to the `incomplete' direct products''~\cite[\S~6, p.~4]{vonNeumann1939}}


Formally, within each sector are only those infinite tensor products that are located `close to' each other,
such that their deviations from each other are `small'. Two vectors $ \vert \Psi \rangle $ and $ \vert \Phi \rangle $
are in the same sector if they are equivalent, denoted by $ \vert \Psi \rangle \sim \vert \Phi \rangle $,
when all but finitely many subfactors are either equal or unitary equivalent and to or
`close to' one another (only a unitary transformation apart); that is,
with the notation from Eq.~(\ref{2024-u-dwsp}) we require~\cite[Eq.~(9)]{van-den-bossche-2023-a}
\begin{equation}
\sum_{i=1}^{\infty} \left( 1 - \langle \psi_i \vert \phi_i \rangle \right)
=\sum_{i=1}^{\infty} \epsilon_i
\le
\sum_{i=1}^{\infty} \left| 1 - \langle \psi_i \vert \phi_i \rangle \right|
< \infty
.
\end{equation}
(With $0< \epsilon_i \ll 1$ as above this would be an equality.)
This condition ensures that the product
$ \prod_{i=1}^{\infty} \langle \psi_i \vert \phi_i \rangle $
converges to a non-zero value within the same sector.
The inner product of infinite tensor products
belonging to different sectors vanishes.


For finite tensor products resulting in finite-dimensional Hilbert spaces,
sectorization has no meaningful relevance: Since, in finite dimensions, all orthonormal bases are unitarily equivalent.

%%%%%%%%%%%%%%%%%%%%%%%%%%%%%%%%%%%%%%%%%%%%%%%%%%%%%%%%%%%%%%%


Therefore, for infinite tensor products, instead of directly dealing with the entire infinite tensor product space,
one should consider regions or sectors within it. These sectors are equivalence classes of vectors that differ only by a finite number of components
in the tensor product, or are otherwise close to (unitary equivalent) each other. These sectors represent different `global' or `macroscopic'
configurations of the system~\cite{hepp-1972}.


States in different sectors cannot be coherently superposed by finite (unitary) means.
One may say that, with respect to these finite unitary means,  `coherence is lost'.
Hepp even went so far as to state that \textit{``leads to macroscopically  different `pointer positions' and to a rigorous
'reduction of the wave packet'~''}~\cite{hepp-1972}.
Let us demonstrate this with an example.
Consider an infinite array of spin-$\frac{1}{2}$ particles,
where each particle has a Hilbert space $\mathcal{H} = \mathbb{C}^2$,
spanned by the states $\vert \uparrow\rangle$ (spin up) and $\vert \downarrow\rangle$ (spin down).
The entire system is then described by the infinite tensor product of these spaces
\(
\mathcal{H}_{\text{total}} = \bigotimes_{i=1}^{\infty} \mathbb{C}^2_i
\).

In such a setup, a  {sector} can be defined by specifying the `macroscopic' behavior of the system, such as the average magnetization:

\begin{itemize}
    \item  {Sector A (All Spins Up)}: Consider the state where every spin is up; that is,
    \(
    \vert \psi_{\text{up}}\rangle = \vert \uparrow\rangle \otimes \vert \uparrow\rangle \otimes \vert \uparrow\rangle \otimes \cdots
    \).
    This state belongs to a sector where all spins are aligned up.

    \item  {Sector B (All Spins Down)}: Similarly, consider the state
    \(
    \vert \psi_{\text{down}}\rangle = \vert \downarrow\rangle \otimes \vert \downarrow\rangle \otimes \vert \downarrow\rangle \otimes \cdots
    \)
    where every spin is down.
    This state belongs to a different sector where all spins are aligned down.

    \item  {Sector C (Mixed Alignment)}: Now consider a state where half the spins are up and half are down, such as
    \(
    \vert \psi_{\text{mixed}}\rangle = \vert \uparrow\rangle \otimes \vert \downarrow\rangle \otimes \vert \uparrow\rangle \otimes \vert \downarrow\rangle \otimes \cdots
    \).
    This state belongs to yet another sector, where the system exhibits a different macroscopic behavior.
\end{itemize}

To illustrate the challenges encountered, let us attempt to superpose states from different sectors.

\begin{itemize}
    \item  {Within the Same Sector}:
    Superpositions of states within the same sector are possible. For example, superpositions of states that differ by a finite number of spins can be physically meaningful, such as
        \(
        \vert \psi\rangle = \alpha \vert \uparrow\rangle \otimes \vert \uparrow\rangle \otimes \vert \uparrow\rangle \otimes \cdots + \beta \vert \uparrow\rangle \otimes \vert \downarrow\rangle \otimes \vert \uparrow\rangle \otimes \cdots
        \)
        Both states essentially belong to the same `all spins up' sector with minor fluctuations.

    \item  {Across Different Sectors}:
     Attempting to superpose states from different sectors, such as:
        \(
        \vert \phi\rangle = \alpha \vert \psi_{\text{up}}\rangle + \beta \vert \psi_{\text{down}}\rangle,
        \)
        results in a superposition that is  {not physically meaningful}.
States from different sectors (like all spins up versus all spins down) represent distinct macroscopic configurations,
and there is no way to coherently combine them in an infinite system using finite means.

\end{itemize}

Let us now address the question of why coherence---the ability to linearly superpose states from different sectors---is lost
in an infinite tensor product space.
It is essential to note that different sectors are orthogonal: States from different sectors (like all up versus all down)
become orthogonal in the limit of an infinite number of particles.
This orthogonality is a reflection of the fact that they represent fundamentally distinct physical configurations.
Furthermore, there exists no observable capable of coherently mixing states from different sectors,
implying that any attempt to superpose them would not result in interference effects.
In the limit case of infinite tensor product states, the system effectively `forgets' any phase relationship between these states,
leading to a loss of coherence.



In the infinite tensor product of spin-$\frac{1}{2}$ systems,
sectors correspond to different macroscopic configurations of spins---for instance, all up, all down, and mixed.
States from different sectors cannot be coherently superposed
because they are orthogonal and no (finite) transformation connects them---they tend
to `crystallize' or `decohere' into different macroscopic domains or realms---leading to a loss of coherence. This example illustrates how sectors naturally arise in infinite tensor products and
why superpositions across sectors are not physically meaningful.
With this kind of sectorization, or transition into different sectors, global unitarity with respect to finite unitary means is lost.



\subsection{Factorization}
\label{2024-u-factors}



Von Neumann algebras, also known as \( W^*\)-algebras,
are operator algebras that are classified into types I, II, and III,
introduced by von Neumann and Murray~\cite{Murray1936}.
These algebras are closed under addition, operator and scalar multiplication, and contain the identity.
The `star' symbol $^*$ indicates closure under adjoint transformations.
They are also closed in the weak operator topology with respect to operator sequences
converging towards a limit, ensuring that they are complete under this topology.

A von Neumann algebra \( \mathcal{M} \) is called a \textit{factor} if its center \( \mathcal{Z}(\mathcal{M}) \)---the set of all operators in \( \mathcal{M} \) that commute with every operator in \( \mathcal{M} \)---consists only of scalar multiples of the identity operator.
Factors are indecomposable in the sense that they cannot be decomposed into a direct sum of two non-trivial von Neumann algebras.
Furthermore, any von Neumann algebra can be written as a direct sum of its factors~\cite{Penington2022Dec}.

%\subsubsection*{Types of Von Neumann Factors}

Von Neumann factors are classified into three types: I, II, and III. This classification is based on the structure of projections in the algebra and the trace properties.

%\subsubsection*{Type I Factors}

Type I factors are those that are isomorphic to all bounded operators on a Hilbert space:
\begin{itemize}

\item {Type I\(_n\):} The factor is isomorphic to \( M_n(\mathbb{C}) \), the algebra of \( n \times n \) matrices over the complex numbers. These factors correspond to finite-dimensional Hilbert spaces.


\item {Type I\(_\infty\):} The factor is isomorphic to \( \mathcal{B}(\mathcal{H}) \),
the algebra of all bounded operators on an infinite-dimensional separable Hilbert space \( \mathcal{H} \).
These factors act on Hilbert spaces with countably infinite dimension.

\end{itemize}

It is reasonable to identify (orthogonal) projections of type I$_n$
factors with elements of orthonormal bases, or equivalently, with contexts, blocks in quantum logic, or maximal operators.
This identification is supported by~\cite[Satz~8]{v-neumann-31} (see also~\cite[\S~82]{halmos-vs}).
Type I\(_n\) factors  are the only ones in finite dimensional Hilbert space.
They have minimal orthogonal projections (self-adjoint and idempotent) that correspond
to one-dimensional subspaces of the Hilbert space,
as well as convex combinations thereof (projecting into higher-dimensional subspaces).
Indeed, any sequence of mutually orthogonal (orthogonal) projections $|\psi_i\rangle$,
combined with any sequence of probabilities $p_i \in [0,1]$ satisfying $\sum_{i=1}^{k} p_i = 1$ where $k \leq n$,
forms a density operator $\rho = \sum_{i=1}^{k} p_i |\psi_i\rangle\langle\psi_i|$.

In terms of quantum mechanical states, this amounts to both pure and mixed states~\cite{sorce-2023}.
Note that, in the context of a finite-dimensional Hilbert space, the trace of a
$k$-dimensional projection in an $n$-dimensional space (where $k\le n$) is simply the positive integer $k$.

In terms of entanglement, type I\(_n\) factors can (but need not) represent a finite number of entangled particles.
Type I\(_\infty\) factors are infinite-dimensional.
%These projections are all comparable.
%In order to understand the term `compare' it is necessary to introduce the notion of a partial isometry,
%which is an element $v$ of a von Neumann algebra which yield two projection operators:
%an initial projection $p= v^* v$  ($*$ indicates the adjoint) onto the initial subspace $\text{range}(v^*  v)$,
%and a final projection $q = vv^*$ onto the final subspace $\text{range}(v  v^*)=\text{range}(v)$
%(note that $(v  v^*) x=$.

%\subsubsection*{Type II Factors}

Type II factors are characterized by their occurrence in infinite-dimensional Hilbert spaces.
In contrast to type I factors, they are considered `diffuse', meaning they lack minimal projections,
which are projections onto one-dimensional subspaces or convex combinations thereof~\cite{houdayer-aittIIf}.
This property is often linked to mixed states in the context of quantum mechanical states~\cite{sorce-2023}.

\begin{itemize}

\item {Hyperfinite type II\(_1\) factor:}
In contrast to type I factors, the entanglement in a hyperfinite type II\(_1\) factor is more `diffuse', making it impossible to identify individual entangled states.
For example, the factor might comprise an infinite number of qubit pairs, with all but a finite number of pairs in a maximally entangled state~\cite{Penington2022Dec}.

\item {Type II\(_\infty\) factor:} This factor is simply the tensor product of a type II\(_1\) factor and a type I\(_\infty\) factor.

\end{itemize}

Despite being diffuse, the trace of a projection in a type II factor is still faithful, normal, and semi-finite.
A faithful trace is one that does not vanish on any non-zero positive element of the von Neumann algebra.
A normal trace respects the convergence of operators (in the weak topology), ensuring that the trace of a limit of operators equals the limit of their traces.
A semi-finite trace is one such that for any positive element, there is a non-zero `sub-element' on which the trace is finite.
For type II\(_1\) factors, the trace assigns a value in the continuous interval \([0,1]\), where 0 corresponds to the zero projection and 1 corresponds to the identity projection.
This trace function behaves like a measure of `dimension' but is not tied to integer dimensions as in finite-dimensional spaces.
For type II\(_\infty\) factors, the trace can take values in \([0,\infty]\).

%\begin{itemize}
%\item {Type II\(_1\)}: These factors possess a finite trace, which is a linear functional \( \tau: \mathcal{M} \rightarrow \mathbb{C} \) satisfying \( \tau(ab) = \tau(ba) \) for all \( a, b \in \mathcal{M} \) and \( \tau(p) \geq 0 \) for projections \( p \), with \( \tau(\mathcal{I}) = 1 \). In type II\(_1\) factors, all projections have the same "size" in terms of the trace, and the algebra supports a unique normalized trace.
%
%\item {Type II\(_\infty\)}: These factors have an infinite trace, meaning that the trace functional exists but is not finite for all projections. A type II\(_\infty\) factor can be thought of as an amplification of a type II\(_1\) factor.
%\end{itemize}
%
%\subsubsection*{Type III Factors}


Type III factors are characterized by the absence of faithful normal semi-finite traces.
In the context of quantum mechanical states, this implies that states on Type III factors
cannot be represented by density operators in the conventional sense.
The distinction between pure and mixed states becomes more intricate,
as all normal states on a Type III factor are, in some sense, `mixed.'
Nevertheless, notions of pure states (as extreme points of the state space) and mixed states still exist,
but they exhibit different behaviors compared to those in Type I or II factors.
In terms of entanglement, we can expect `infinite entanglement' but also `infinite fluctuations' in this entanglement~\cite{Penington2022Dec}.

%Type III factors have a much more complicated structure and are related to phenomena like the dynamics of quantum systems, particularly in the context of infinite systems and local algebras in quantum field theory.

%\subsubsection*{Examples from Spin-\(\frac{1}{2}\) Theory}

The origin of the term `factor' may come from a tensor product factorization:
Suppose
$\mathcal{H} = \mathcal{H}_1 \otimes \mathcal{H}_2$.
Then
$F_1 = \mathcal{B}(\mathcal{H}_1) \otimes \mathbb{1}$
and
$F_2 = \mathbb{1} \otimes \mathcal{B}(\mathcal{H}_2)$
are factors of
$\mathcal{B}(\mathcal{H})$~\cite[Exercise~3.3.8]{Jones-vNA}.

%In the context of quantum mechanics, particularly in the study of spin-\(\frac{1}{2}\) particles,
%von Neumann algebras and their classification into types I, II, and III can be illustrated through the
%structure of the algebra of observables and the states of the system.



%%\subsubsection*{Type I Factors}
%
%%\textbf{Example: Spin-\(\frac{1}{2}\) System}
%
%For a single spin-\(\frac{1}{2}\) particle, the relevant Hilbert space is \( \mathcal{H} = \mathbb{C}^2 \). The algebra of observables is the algebra of all \(2 \times 2\) matrices acting on this space, which is denoted by \( \mathcal{B}(\mathbb{C}^2) \).
%
%%\begin{itemize}
%%\item {Type I\(_2\) Factor:}
%The algebra \( \mathcal{B}(\mathbb{C}^2) \) is a type I\(_2\) factor. It contains minimal projections
%corresponding to measurements of spin along a given axis;
%for instance,
%$\vert \uparrow \rangle \langle \uparrow \vert = \text{diag}\begin{pmatrix} 1,0 \end{pmatrix}$
%and
%$\vert \downarrow \rangle \langle \downarrow \vert = \text{diag}\begin{pmatrix} 0,1  \end{pmatrix}$
%for measurements of spin along the $z$ axis.
%
%For an example of a type I\(_4\) factor consider two qubits, or two spin-\(\frac{1}{2}\) particles.
%The corresponding  Hilbert space is the tensor product of the individual Hilbert spaces of each particle.
%Each particle's Hilbert space is \(\mathcal{H} = \mathbb{C}^2\), so the total Hilbert space is:
%\(
%\mathcal{H}_{\text{total}} = \mathcal{H}_1 \otimes \mathcal{H}_2 = \mathbb{C}^2 \otimes \mathbb{C}^2 = \mathbb{C}^4
%\).
%
%The algebra of all bounded operators on this Hilbert space is denoted by \(\mathcal{B}(\mathbb{C}^4)\).
%This algebra is a type I\(_4\) factor,
%which consists of all \(4 \times 4\) complex matrices acting on the Hilbert space \(\mathbb{C}^4\).
%
%%\subsection{Type I\(_4\) Factor in the Cartesian Basis}
%
%In the {Cartesian basis}, the standard basis states for the two-particle system are
%\(
%\{ \vert  \uparrow \uparrow \rangle, \vert  \uparrow \downarrow \rangle,
%\vert  \downarrow \uparrow \rangle, \vert  \downarrow \downarrow \rangle \}
%\),
%where \(\vert  \uparrow \rangle\) and \(\vert  \downarrow \rangle\) represent the spin-up and spin-down states, respectively.
%
%Any operator \(A\) in \(\mathcal{B}(\mathbb{C}^4)\) with components
%\(A_{ij}\)
%can be represented as a \(4 \times 4\) matrix in this basis,
%where each \(A_{ij}\) is a complex number corresponding to the matrix elements \(\langle i | A | j \rangle\)
%for some orthonormal, in this case Cartesian, basis states.
%
%Alternatively we may choose the {Bell basis}
%%$\vert  \Phi^+ \rangle = \frac{1}{\sqrt{2}} \left( \vert  \uparrow \uparrow \rangle + \vert  \downarrow \downarrow \rangle \right)$,
%%$\vert  \Phi^- \rangle = \frac{1}{\sqrt{2}} \left( \vert  \uparrow \uparrow \rangle - \vert  \downarrow \downarrow \rangle \right)$,
%%$\vert  \Psi^+ \rangle = \frac{1}{\sqrt{2}} \left( \vert  \uparrow \downarrow \rangle + \vert  \downarrow \uparrow \rangle \right)$, and
%%$\vert  \Psi^- \rangle = \frac{1}{\sqrt{2}} \left( \vert  \uparrow \downarrow \rangle - \vert  \downarrow \uparrow \rangle \right)$,
% whose elements are interpreted as maximally entangled quantum states.
%The Bell basis provides a different representation of the same Type I\(_4\) factor,
%emphasizing the entanglement structure of the two-particle system,
%but it does not imply a direct sum decomposition of the algebra itself.
%
%
%
%
%%A type I\(_4\) factor  \(\mathcal{B}(\mathbb{C}^4)\)
%%does not naturally decompose into a direct sum of smaller factors because it is already a simple algebra
%%with no non-trivial two-sided ideals.
%%The notion of a direct sum is more applicable when dealing with algebras that can be decomposed into orthogonal components,
%%but this does not apply to a Type I factor of this form.
%
%
%
%
%
%
%
%
%It would seem not entirely unreasonable to identify (orthogonal) projections of type I$_n$ factors with elements of orthonormal bases, or equivalently, with contexts, blocks in quantum logic, or maximal operators. This identification is supported by~\cite[Satz~8]{v-neumann-31} (see also~\cite[\S~82]{halmos-vs}).
%
%%\item {
%For an example of a type I$ _\infty$ factor consider an infinite collection of non-interacting spin-\(\frac{1}{2}\) particles. The Hilbert space for this system is the infinite tensor product
%  \(
%  \mathcal{H} = \bigotimes_{n=1}^\infty \mathbb{C}^2.
%  \)
%  The algebra of bounded operators on this Hilbert space, \( \mathcal{B}(\mathcal{H}) \), is a type I\(_\infty\) factor. It has minimal projections corresponding to measurements that act on individual spins.
%%\end{itemize}
%
%%\subsubsection*{Type II Factors}
%
%%\textbf{Example: Spin Chains with Symmetry}
%
%%Consider a spin chain where each site has a spin-\(\frac{1}{2}\) particle, and the system has a global symmetry, such as translational symmetry or a global \( \text{SU}(2) \) spin symmetry. The algebra of observables might be the weak closure of the algebra generated by local observables (finite products of Pauli matrices acting on finite subsets of the chain).
%
%%\begin{itemize}
%%\item {Type II\(_1\) Factor:} If the system has certain symmetry properties and the algebra is invariant under averaging procedures (like in the case of a KMS state at finite temperature), the resulting algebra may have a unique trace, leading to a type II\(_1\) factor. In practice, type II\(_1\) factors are more abstract and arise in cases where you can define a finite trace over the algebra without having minimal projections, which might occur in infinite, symmetric spin systems.
%%
%%\item {Type II\(_\infty\) Factor:} If we consider an infinite spin chain with a certain kind of infinite-volume limit where the trace becomes infinite, the corresponding von Neumann algebra might be a type II\(_\infty\) factor. For instance, a translation-invariant infinite spin chain might exhibit such a structure.
%%\end{itemize}
%%
%%\subsubsection*{Type III Factors}
%%
%%
%%Type III factors are typically encountered in the study of quantum field theory and statistical mechanics, especially in systems at zero temperature or in the thermodynamic limit.
%%
%%For an example of a type III factor consider an infinite spin-\(\frac{1}{2}\) chain at zero temperature. The algebra of local observables in the thermodynamic limit can be a type III factor. In this context, there is no normalizable trace because the system is infinitely large, and the states describing the system are often highly entangled. Such algebras have no non-trivial central elements and no trace, making them type III.
%%
%%\subsubsection*{Summary}
%
%%In the spin-\(\frac{1}{2}\) theory:
%%\begin{itemize}
%%    \item \textbf{Type I factors} are associated with finite systems (like a single spin or a finite number of spins) or infinite systems where individual spins are treated separately.
%%
%%    \item \textbf{Type II factors} may arise in infinite systems with symmetries that allow for a notion of trace without minimal projections, particularly in statistical mechanics or thermodynamic limits.
%%
%%    \item \textbf{Type III factors} are related to infinite systems at zero temperature, where the traditional notion of a trace fails, and the algebraic structure reflects the complexity and entanglement of the system.
%%\end{itemize}
%

Unitary equivalence preserves the type of the algebra and does not change it.
As a consequence, there are no unitary operators, permutations, or any other operations
within the framework of von Neumann algebras that can convert, say, a type I factor into a type II or III factor.

To facilitate transitions between factor types, more powerful tools than unitary operations are needed.
One such tool (among others) is the use of inductive limits, which enable the construction of large and complex algebras from simpler, smaller ones~\cite{houdayer-2009}.
To transition from type I factors to type II factors, sequences or non-trace-preserving embeddings that 'diffuse' the trace structure are required.



\section{Nested Wigner's friends as infinite tensor products}

Nesting or chaining refers to the repeated and iterated application of the von Neumann type measurement-by-entanglement,
as formalized by Equation~(\ref{2024-u-vNsiqm}), as expressed to first order by Wigner~\cite{wigner:mb}.
As a consequence, we end up with large and potentially infinite tensor products.
It is tempting to ascribe this measurement conceptualization to von Neumann~\cite{Taub:1961:JNCc,vonNeumannCompendium}.

Grangier and Van Den Bossche have recently proposed that the apparent loss of coherence
in such situations is attributable to sectorization
and the consequent loss of unitary equivalence within finite systems~\cite{Grangier-2020,van-den-bossche-2023-a,van-den-bossche-2023-b,van-den-bossche-2023-c},
as previously discussed in subsection~\ref{2024-u-sectors}.
According to their proposal, sectorization is a physical process in infinite algebras: Separable
sectors correspond to `classical outcomes' and `macroscopic states' of pointers~\cite{hepp-1972,bub-1988,bub-2015}.
Within a sector, the context and the resulting measurement have fixed macroscopic values,
but a finite (sub)system within each sector can still undergo unitary evolution as long as the state remains within its sector.
Any new measurement results in a shift to a different sector,
thus establishing a new context and corresponding outcome.
In the extreme case, a sector can be associated with a single element of a context.
Contexts and their associated sectors are `reshuffled' or `scrambled' when new incompatible measurements are performed.

The connection between sectors and factors remains an open question.
A fundamental difference is that factors pertain to algebras of operators---specifically,
(generalized) density operators if a trace exists---while sectors pertain to (unitarily equivalent) elements or subspaces of Hilbert space.

For type I factors, some of these operators can be interpreted as pure
(that is, `minimal' one-dimensional orthogonal projection operators) or mixed states.
Indeed, intuition from finite tensor products suggests that, through spectral decomposition of the operators in a factor,
the respective orthogonal projections in type I$_n$ factors correspond to elements of contexts
(associated with maximal operators, see~\cite[Satz~8]{v-neumann-31}).
This implies that they refer to pure states spanned by vectors in the respective Hilbert space.

In finite dimensions, sectors do not carry much significance, as all vectors are unitarily equivalent.
However, for infinite tensor products, sectorizations `form naturally' through the unitary equivalence of vectors
(or their span, and the associated `minimal' one-dimensional orthogonal projection operators)
and can be associated with macroscopic quantities.

The transition between different sectors does not correspond to any unitary transformation, as these sectors are not unitarily equivalent.
This results in an apparent loss of coherence, as different sectors cannot be in coherent superposition.
Measurements with mismatched pre- and post-selection are linked to distinct sectorizations of the Hilbert space.
In the context of infinite tensor products, such `context translations' cannot be achieved through unitary operations.

Factorization offers another potential mechanism beyond unitary evolution:
The infinite limit of nested Wigner's friends, facilitated by entanglement, could enable transitions between distinct factor types.
Consequently, both factorization and sectorization, in the infinite limit, might contribute to a loss of unitary equivalence and decoherence.


Furthermore,
as previously discussed, any nesting construction is highly susceptible to
alterations in the focus of observations by Wigner's friends---specifically, with regard to changes in the sequence of entangled basis vectors.
This sensitivity arises not only from potential state changes within a (Type I) factor,
but also from the mismatches and entanglements that occur between infinite sequences of nested
von Neumann measurements,
which can lead to transitions into distinct sectors and factors.
Consequently, even the slightest mismatch and change in nested observables cumulatively leads to a complete loss of information
about the initial state (preparation).

More explicitly, as has been pointed out earlier in the context of difficulties in defining the inner product, any slight mismatch between (successive) friends' measurements `builds up' into a total loss of coherence.
This results in a vanishing inner product \(\langle \Psi \vert  \Psi' \rangle\) which converges to zero, indicating that the product states
\(\vert \Psi\rangle\) and \(\vert \Psi'\rangle\)
are orthogonal even if each single mismatch characterized by \(\langle \psi_i \vert  \psi_i' \rangle\) is very close to 1.
This type of `decoherence' is gradual and smooth in the sense that there is no abrupt discontinuous
transition---indicating a well-defined, localizable Heisenberg cut at some scale---but a gradual, continuous loss
of information about the initial state: Let $0 \ll \vert  \langle \psi_i \vert  \psi_i' \rangle \vert  = \delta_i < 1$ be this match per friends $i$ and $i'$, then
\begin{equation}
\vert  \langle \Psi \vert  \Psi' \rangle \vert
= \prod_{i=1}^{\infty} \vert  \langle \psi_i \vert  \psi_i' \rangle \vert
= \prod_{i=1}^{\infty} \delta_i
=0.
\end{equation}
One could also interpret $\epsilon_i$ in  $\delta_i= 1-\epsilon_i$ as a (measure of) stochastic `input' per Wigner's friend $i$ that contributes to a context translation~\cite{svozil-2003-garda,svozil-2013-omelette} but introduces additional input from the Wigner's friend (environment). This is particularly true if Wigner's friends attempt to `measure'
a state in which the quantum system is not prepared~\cite{zeil-99}.

This model diverges from the reduction model of Hepp~\cite{hepp-1972,bub-2015} and the recent papers by
Grangier and Van Den Bossche~\cite{Grangier-2020,van-den-bossche-2023-a,van-den-bossche-2023-b,van-den-bossche-2023-c}
in that it proposes a sequence of mismatch measurements by Wigner's friends that ultimately transcends sectors or even factors,
and does not depend on sectorization, that is, the creation of sectors interpretable as macroscopic `pointers'.

Bell's argument~\cite{Bell-1975} against transfinite recursion remains valid for an infinite number of Wigner's friends.
However, his later FAPP argument~\cite{bell-a}---that, although any Heisenberg cut is relative, it exists for all practical purposes
and experimental capacities---can be maintained.
I concur that any finite number of Wigner's friends does not lead to a violation of unitary equivalence,
and thus state reduction or decoherence.

%\begin{tcolorbox}[colback=yellow, coltext=black, boxrule=0pt, sharp corners, breakable, left=0pt, right=0pt, boxsep=0pt]

Grangier and Van Den Bossche circumvent Bell's argument ontologically by positing that a dual quantum and classical description is necessary to understand quantum mechanics~\cite{van-den-bossche-2023-c}.
They argue that the mathematical formalism should provide a consistent description, rather than a complete (isomorphic) representation of reality.
In this context (see also the quote by Hertz mentioned later),
the use of mathematical infinities becomes a valid tool for description.


Another approach to addressing Bell's argument involves the concept of infinity processes, as discussed by Weyl~\cite[pp.41,42]{weyl:49}
in the context of Zeno's paradoxes (of infinite divisibility). If we consider the continuum
(or at least the infinite divisibility of space and time), \textit{``if the segment of length 1 really consists of infinitely
many subsegments of lengths $1/2, 1/4,1/8, \ldots$, as of `chopped-off'
wholes, then it is incompatible with the character of the infinite as the
`incompletable' that Achilles should have been able to traverse them all.''}
This implies that even for classical motion in a continuum to be possible, we require transfinite capacities.
This concept can be applied to the infinite nesting of Wigner's friends by considering this nesting or chaining
as the effective oneness which we experience; resulting in irreversible measurements in the transfinite limit.

%\end{tcolorbox}

%In the limit of infinitely many Wigner's friends, and relative to this transfinite capacity, we have addressed the formal conundrum of how irreversible,
%non-unitary measurements---von Neumann's~\cite[Chapter~5]{vonNeumann2018Feb} (and Everett's~\cite{everett})
%process 1 type---can originate from uniform, even one-to-one,
%reversible unitary evolution---von Neumann's process 2 type---essentially through an infinite recursion of the latter.

\section{Historical Analogues}

This section explores several related but distinct concepts that have been investigated in various areas of physics.

\subsection{Infinity and Transfinite Capacities}

This concept is similar to the convergence of sequences of rational numbers to an irrational number in the real numbers.
For instance, consider the continued fraction or the binomial series expansions
\(
\sqrt{2} = (1 + 1)^{1/2} = \sum_{n=0}^{\infty} \binom{1/2}{n} \cdot 1^n = 1 + \frac{1}{2} \cdot 1 - \frac{1}{8} \cdot 1^2 + \frac{1}{16} \cdot 1^3 - \cdots
\)
of \(\sqrt{2}\), truncated at various points.

Another analogue is from recursive analysis: Specker sequences of computable numbers converge to an uncomputable limit~\cite{specker49,specker-ges,kreisel,simonsen-2005}. One example is Chaitin's constant, the halting probability of prefix-free program codes on a universal computer~\cite{rtx100200236p,calude-dinneen06}, whose rate of convergence is tied to the halting time, and therefore, `grows faster' than any computable function.

Many of these metamathematical results are based on Cantor's diagonal argument~\cite{book:486992}, which demonstrates that, `in the limit, enumerable sets become non-enumerable continua'.

\subsection{Statistical physics}

Loschmidt's \textit{Umkehreinwand}~\cite{darrigol-2021} poses a challenge to the concept of irreversible processes at the macroscopic level, given the time-reversibility of microphysical laws. Loschmidt argued that if the microscopic laws are reversible, then any macroscopic process should also be reversible if we could precisely reverse the velocities of all particles in a system. This appears to contradict our everyday experience of irreversible processes and the postulate of the increase of entropy.

The canonical response to the \textit{Umkehreinwand} may seem evasive: while technically correct, due to statistical-probabilistic considerations, the \textit{Umkehreinwand} is means-relative~\cite{Myrvold2011237} and therefore only FAPP~\cite{bell-a} invalid. This is exemplified in Maxwell's pragmatic approach, avoiding detailed inquiries about individual molecules that would complicate the argument~\cite{Maxwell-1879,garber}: \textit{``avoiding all personal inquiries [[about individual molecules]] which would only get me into trouble.''}

One example of `irreversibility-in-the-limit' is the computation of $\sqrt{2}$, as mentioned in the aforementioned two examples:
the continued fraction expansion yields
$1,\frac{3}{2},\frac{7}{5},\frac{17}{12},\frac{41}{29},\frac{99}{70},\frac{239}{169},\frac{577}{408},\frac{1393}{985},\frac{3363}{2378},\ldots , \sqrt{2}$,
whereas the binomial series expansion yields
$1,\frac{3}{2},\frac{11}{8},\frac{23}{16},\frac{179}{128},\frac{365}{256},\frac{1439}{1024},\frac{2911}{2048},\frac{46147}{32768},\frac{93009}{65536}, \ldots, \sqrt{2}$.
Suppose that we delete all common terms from the two series. Then we end up with two series that are different, yet their limit is the same.
(Alternatively, take just the binomial series and rescale its summands by adding the term $1/n$ to each summand.)
Once the limit is reached, and no memory is maintained, it is impossible to determine which of the two series the result originates from~\cite{Svozil-2023-axioms12010072}.


\section{Summary}

We have presented both formal and pragmatic (FAPP) arguments for converting unitary, reversible von Neumann-Everett type 2 processes into non-unitary, irreversible type 1 processes. This conversion utilizes infinite tensor products, which, unlike in finite-dimensional Hilbert spaces, are not bound by unitarity. While objections may arise regarding the operational correspondence of infinite mathematical processes~\cite{bridgman}, a more practical approach involves considering finite subsequences or prefixes of these constructions.

These prefixes can be viewed as nested, iterated, or chained Wigner's friends, each encountering growing challenges in retrieving the original information from a quantum state, particularly when there is a discrepancy between state preparation and measurement. This phenomenon is reminiscent of environmental monitoring, resulting in quantum decoherence~\cite{schlosshauer-2005,schlosshauer-2019}, where environmental interactions lead to a loss of quantum coherence.

Parallels can also be drawn to noise introduction in micro-state amplification~\cite{Glauber-cat-86}, which illustrates the quantum no-cloning theorem and the disruption of quantum states through interactions or measurements.

Schr\"odinger's `jellification' argument~\cite{schroedinger-interpretation} emphasizes the possibility of
unobserved quantum states `spreading' as coherent superpositions without being fixed by irreversible measurement.
Nesting Wigner's friends provides a solution characterized by three key aspects:
(i) The incorporation of environmental information, unrelated to the original state, leads to FAPP uncontrollable
(but not irreducible~\cite{zeil-05_nature_ofQuantum}) systematic stochasticity,
successfully converting the original (preselected) state into the measured (postselected) state.
(ii) Sectorization, which involves the effective orthogonalization and partitioning of the Hilbert space
into macroscopic regions corresponding to measurement outcomes, illustrates the practical difficulties of maintaining coherence when scaling up
the system to macroscopic dimensions~\cite{Grangier-2020,van-den-bossche-2023-a,van-den-bossche-2023-b,van-den-bossche-2023-c}.
(iii) Factorization, occurring according to the depth and modes of entanglement among the Wigner's friends,
additionally contributes to the emergence of classical-like behavior.

The processes of sectorization, which leads to the emergence of macroscopic observables, and factorization, which involves entanglement and the perception of isolated measurement outcomes,
both contribute to the loss of coherence and the formation of non-unitarily equivalent states in the infinite tensor product limit.
Given the pivotal role of entanglement in both nested Wigner's friends and factorization,
it is not totally unreasonable to conjecture a connection between these two phenomena.


Resolutions of the quantum measurement problem and the \textit{Umkehreinwand} in statistical physics through means relativity entail significant epistemological commitments. Previous attempts to simulate measurement processes using von Neumann algebras, such as those by Hepp~\cite{hepp-1972}, have faced criticism for relying on transfinite concepts without operational validation~\cite{Bell-1975,bub-2015}. Nonetheless, these findings could be reconciled by adopting the perspective that ``(FAPP) Infinity (FAPP) Does It.''

Classical analysis, recursive function theory, and von Neumann algebras offer potential ontological frameworks or 'escape routes'
from uniform reversibility and unitarity. However, their viability depends on the acceptance of
infinite limits as meaningful physical concepts~\cite{wigner}.
Modern resolutions of Zeno's and the Eleatics' paradoxes suggest that
without infinite limits and transfinite capacities, there is no motion in a continuum.

Another straightforward pragmatic approach could be considered:
Since this paper employs infinite processes to dispel unitary equivalence,
one could avoid the infinite limit by transcribing the discussion into the framework of `for all practical purposes' (FAPP).
For operationalists who prefer to avoid the use of strict limits,
the term `limit' can be substituted with `FAPP unboundedness' or `too-large-to-handle,'
and the symbol $\infty$ can be replaced with $\infty_\text{\tiny FAPP}$.


Yet, we must remain cognizant that both FAPP and transfinite irreversibility remain mathematical constructs,
Hertz's `images of our imagination'~\cite{hertz-94e} which ultimately
are justified by their practical usefulness and correspondence with phenomenology.
We should therefore exercise caution, not conflating the practical utility of our models with absolute certainty about physical reality.


Quantum erasure arguments~\cite{PhysRevA.25.2208,greenberger2,kim-2000,Ma22012013}
and the Humpty-Dumpty problem~\cite{engrt-sg-I,engrt-sg-II,Englert2013} further illustrate the challenges of reversibility and state reconstruction.
In a similar manner to classical statistical arguments, a macrostate corresponds to numerous microstates, making reversal attempts futile and mirroring the quantum context.

In my opinion, we cannot accept classical irreversibility without accepting irreversible quantum measurement; conversely,
FAPP insistence on classical and quantum reversibility expresses the same resistance towards transfinite, possibly nonconstructive means.
Pointedly stated, the central question in this comparative aspect becomes:
What is a viable position towards Loschmidt's \textit{Umkehreinwand},
and how does this stance translate to quantum measurement?
Whatever answer one might feel comfortable with regarding classical irreversibility,
one may apply its analog to quantum measurement irreversibility.
In both domains, the conceptualization~\cite{anderson:73} of macroscopic observables through grouping and sectorization,
as well as the entanglement-driven factorization in the infinite limit, challenge our notions of reversibility.


Ultimately, the parallel between classical and quantum irreversibility underscores a fundamental unity in physics,
while also highlighting the epistemological challenges and commitments we face in bridging the gap between our
finite experimental capabilities and the infinite limits of our theoretical constructs.

\begin{acknowledgments}
I would like to express my gratitude to Philippe Grangier and Mathias Van Den Bossche for their patient explanations of certain concepts in earlier drafts. Any remaining errors are, however, solely due to my own limitations.

I had the opportunity to meet Eugene Wigner personally during the 1st International Seminar on Nuclear War, held in Erice, Italy, from 14 to 19 August 1981.
However, I did not discuss his ``remarks on the mind-body question''~\cite{wigner:mb} with him.

The author declares no conflict of interest.

The AI assistants ChatGPT from OpenAI, mistral chat from mistral.ai, and Claude from Anthropic (partly through VSC enhanced Continue and Cursor) were used for formulating parts of the argument,
symbolic transformations into \LaTeX, and grammar and syntax checks.
\end{acknowledgments}


\bibliography{svozil}
\ifws

\bibliographystyle{spmpsci}

\else
 %\bibliographystyle{apsrev}

\fi

\end{document}

%%%%%%%%%%%%%%%%%%%%%%%%%%%%%%%%%%%%%%%%%%%%%%%%%%%%%%%%%%%%%%%

(* Function to compute the first n approximations (convergents) of the continued fraction expansion of Sqrt[2] *)
ContinuedFractionApproximationsSqrt2[n_Integer] := Module[
  {a, m = 0, d = 1, a0, h0 = 1, h1, k0 = 0, k1 = 1, h, k, approximations},

  a0 = Floor[Sqrt[2]];
  h1 = a0;
  approximations = {h1/k1}; (* The first approximation *)

  For[i = 1, i < n, i++,
    m = d*a0 - m;
    d = (2 - m^2)/d;
    a = Floor[(Sqrt[2] + m)/d];

    (* Update numerators and denominators for the next convergent *)
    h = a*h1 + h0;
    k = a*k1 + k0;

    (* Store the current convergent *)
    AppendTo[approximations, h/k];

    (* Update previous terms for the next iteration *)
    h0 = h1; h1 = h;
    k0 = k1; k1 = k;
    a0 = a;
  ];

  approximations
]

(* Example usage: compute the first 10 approximations *)
ContinuedFractionApproximationsSqrt2[5]

%%%%%%%%%%%%%%%%%%%%%%%%%%%%%%%%%%%%%%%%%%%%%%%%%%%%%%%%%%%%%%%

(* Function to compute the first n approximations of the binomial series expansion of Sqrt[2] *)
BinomialSeriesSqrt2[n_Integer] := Module[
  {approximation, term},

  approximation = 0;

  For[k = 0, k < n, k++,
    term = Binomial[1/2, k] * 1^k;
    approximation += term;
  ];

  approximation
]

(* Example usage: compute the first 10 approximations *)
Table[BinomialSeriesSqrt2[i], {i, 1, 5}]
