\documentclass[%
 %reprint,
  twocolumn,
 %superscriptaddress,
 %groupedaddress,
 %unsortedaddress,
 %runinaddress,
 %frontmatterverbose,
 % preprint,
 showpacs,
 showkeys,
 preprintnumbers,
 %nofootinbib,
 %nobibnotes,
 %bibnotes,
 amsmath,amssymb,
 aps,
 % prl,
  pra,
 % prb,
 % rmp,
 %prstab,
 %prstper,
  longbibliography,
 %floatfix,
 %lengthcheck,%
 ]{revtex4-1}

%\usepackage{cdmtcs-pdf}

\usepackage{mathptmx}% http://ctan.org/pkg/mathptmx

\usepackage{amssymb,amsthm,amsmath}

\usepackage{tikz}
\usepackage[breaklinks=true,colorlinks=true,anchorcolor=blue,citecolor=blue,filecolor=blue,menucolor=blue,pagecolor=blue,urlcolor=blue,linkcolor=blue]{hyperref}
\usepackage{graphicx}% Include figure files
\usepackage{url}

\usepackage{xcolor}

\usepackage[american,greek]{babel}

%%%%%%%%%%%%%%%%%%%%%%%%%%%%%%%%%%%%%%%%%%%%%%%%%%%
\newcommand{\cris}[1]{\textcolor{red}{#1 (Cris)}}
\newcommand{\karl}[1]{\textcolor{blue}{#1  (Karl)}}

\begin{document}
\selectlanguage{american}

\title{A quantum physical glance at Molinist theology}

\author{Cristian S. Calude}
\email{cristian@cs.auckland.ac.nz}
\homepage{http://www.cs.auckland.ac.nz/~cristian}
\affiliation{Department of Computer Science, University of Auckland,
Private Bag 92019, Auckland, New Zealand}


\author{Frederick William Kroon}
\email{f.kroon@auckland.ac.nz}
\homepage{http://www.arts.auckland.ac.nz/people/fkro002}
\affiliation{Department of Philosophy, University of Auckland,
Private Bag 92019, Auckland, New Zealand}


\author{Karl Svozil}
\email{svozil@tuwien.ac.at}
\homepage{http://tph.tuwien.ac.at/~svozil}

\affiliation{Institute for Theoretical Physics,
Vienna  University of Technology,
Wiedner Hauptstrasse 8-10/136,
1040 Vienna,  Austria}

\affiliation{Department of Computer Science, University of Auckland,
Private Bag 92019, Auckland, New Zealand}

\date{\today}

\begin{abstract}
The typology of Molinistic knowledge is compared and contrasted with quantum counterfactuals. From this perspective, possible reinterpretations of the Molinistic middle knowledge are discussed. Finally, we suggest that the whole discussion can be re-visited using an ``inconsistency-tolerant''  logic instead of the more rigid classical one.
%Quantum mechanics is reviewed from the perspective of Molinism, and {\it vice versa}. By comparing quantum counterfactuals to Molinistic middle knowledge, revisions to the latter are suggested which cope with the quantum nature of the physical universe.
\end{abstract}

%\pacs{03.65.Aa, 03.65.Ta, 03.65.Ud, 03.67.-a}
\keywords{Molinism, counterfactuals, many-worlds interpretation, Kochen-Specker theorem, quantum value indefiniteness, quantum indeterminism, quantum randomness}
%\preprint{CDMTCS preprint nr. x}

\maketitle


\section{Introduction}%Molinistic theology}
\label{2016-moninism-sec1}

Although  dogmas  (not only theologic ones)  have sometimes famously crippled the pursuit of natural sciences,
theology and physics should not {\it a priori} be perceived as ``natural adversaries:'' after all, both are dedicated to the
pursuit of truth~\cite{Sagan-Contact},
and both have a proven capacity to stimulate and enlighten one another.

Scholastic concepts~\cite{Solana-Historia}
of ``Infuturabilien'' (translated ``future contingencies'') for example, have been motivation and guidance for an early
announcement~\cite{specker-60} of the strong quantum physics no-go (limiting) result known as
%what is now known as
the Kochen-Specker theorem~\cite{kochen1}.
Indeed, many issues plaguing modern physics have their roots in problems which have concerned philosophers and theologians for centuries.
It might %be prudent
pay off to see what solutions, or at least attempts towards an understanding of these conundrums, have been suggested by past scholars.



One of the most fascinating metaphysical claims~\cite{zeil-05_nature_ofQuantum,Bradley-2016} (or rather inclinations~\cite{born-26-1}) is that,
although many macro-physical %aspects
phenomena are fairly predictable and foreseeable,
they emerge from an ubiquitous microphysical phenomenology  which  appears to be irreducibly indeterministic.
For the sake of an example, the (spontaneous or stimulated) emission of an individual photon contributing to, say,
the ambient electromagnetic field occurs at indeterminate time (although the emission process is subject to statistical and other constraints).
Pointedly stated, our entire surrounding physical environment
appears to be grounded in what ancient Greek mythology and cosmology called \textgreek{q'aoc}.


Let us end with a {\it caveat:} science is reflexive and will never deliver absolute truth.
As it was proved long time ago, even in mathematics there is no consistent absolute notion of ``truth.''  Indeed,
the ability to deal with truth of
any sufficiently  strong formal system is severely limited by  the
 unprovability of its own
consistency (G\"odel's second incompleteness theorem~\cite{godel1})
and    the formal undefinability of ``truth'' (Tarski's undefinability theorem~\cite{tarski:36}).
The only way out is to  focus on processes pursuing relative, contextual forms of ``truth''.


\section{Brief Molinistic theology}

In this note  {\em providence} -- ¬ó from Latin  {\it providentia}  ``foresight, prudence,'' from
{\it pro-}  ``ahead'' and {\it videre}  ``to see'' -- is an attribute of God that means ``knowledge of the future''
or omniscience~\cite{sep-providence-divine}. An analogue physical term is ``prediction''~\cite{specker-60,acs-2015-info6040773}.

Molinism~\cite{book:502563,book:1310279} %, in particular,
was created as an attempt~\cite{Flint-2009} to reconcile divine providence with individuals (creatures) free will.
Intuitively, appropriating both appears to be a zero-sum game:
every increase of individual creature liberty seems to curb divine providence, and {\it vice versa}.
In particular, an omnipotent, omniscient Demiurge God (or, more secularly, an agent henceforth abbreviated by G) seems to exclude free will,
which in turn appears to challenge morality as a guiding principle for creatures,
as well as the benign character of G.

Molinism attempts to solve this conundrum by assuming the ``existence'' of a particular kind of
{\em middle knowledge} which bears some resemblance to physical
counterfactuals~\cite[p.~240]{penrose-oliver-70}:
{\em ``that is, things that might have happened, although they did not in fact happen.''}

It holds that G's middle knowledge constitutes  every conceivable counterfactual --
obtained by ``parsing'' through all conceivable universes obtained by pursuing all possible choices or alternative outcomes.
Afterwords, G proceeds with choices -- that is, with G's choice for a particular kind of factual, and with the individual decisions of G's creatures.
Thereby, Molinism allows for individual creature free will by assuming that it represents contingent truth which is not in any way under G's direct control.
And yet G has complete middle knowledge of all such contingent decisions, and can freely choose what to create and what not to create
-- and once created, through middle knowledge, he knows exactly, and has foreknowledge about, what will happen.

Molinism further holds that middle knowledge is  in-between (thus the term ``middle'') two other types of knowledge:
On the one hand there is {\em natural knowledge} of necessary truth such as $2+2=4$ in some system of mathematics.
This type of knowledge is supposed to be a sort of immutable framework consisting of unconditional, non-contingent propositions which are held true.
Whether or not such absolute truth can be defined is questionable, at least from the point of view of metamathematics. Yet one might say that
G could just define a system, such as Peano arithmetic, to be universally valid.

And on the other hand,  {\em free knowledge} is the post-volitional knowledge about G's (or a creature) free will and choices exercised.

Middle knowledge (of G) thus represents true counterfactuals; in particular also  of  creature  freedom, and are contingent propositions not under G's control.
G may choose to create conditions such that the respective contingent propositions ``become true.''

For the sake of an example of middle knowledge, suppose G knows that in a bar Bob will freely decide to order red wine (and not a beer or a lemonade).
G may freely decide not to create Bob; but once created, Bob will order that red wine in the bar -- and G knows Bob's free decision in advance (before creating Bob).
Of course, G may create another character Charlie who orders beer in that same bar, or he may not create the bar environment.

One can conceptualize the situation by four phases~\cite[p.~43]{book:502563}:
(i) The first phase is characterized by
natural knowledge.
(ii) The second phase is about middle knowledge; about all possible worlds and counterfactuals.
(iii) The third phase is the creative act of will, a kind of {\it creatio continua.}
(iv) In the final, fourth post-volitional phase, free knowledge has been obtained by the actual choices made, and free will enacted.

We will not further go into subtleties~\cite{book:502563,book:1310279,book:421072}, criticism~\cite{10.2307/20009657,hasker-12},
and consequences (such as, for instance, the question of whether individuals could be responsible for sins which they have not committed but would have committed
if the context were different)
but rather proceed to a discussion on how to represent and perceive middle knowledge if the universe is   quantized.



\section{Quantum counterfactuals}

As expressed by Specker~\cite{kochen1}, scholastic philosophy pondered the question whether G's omniscience may include counterfactuals; that is,
events which would have occurred if something had happened, which however did not happen.
In the physical context we could, for instance,
ask if an ensemble of mutually complementary (maximal) observables (or contexts) can (consistently) co-exist. Each one of these maximal observables could in principle have been measured,
but due to complementarity, only one can actually be measured; the others would remain counterfactuals~\cite{vaidman:2009}.
As is often the case, this answer depends on the  assumptions made, and thus remains not absolute but means relative.
Indeed, even the mere notion that there somehow exists a multitude of possible observables which are dormant -- to quote Hamlet, an {\it ``undiscover'd country, from whose bourn no traveller returns''} --
seems highly questionable, as under reasonable %side
assumptions, the resulting system of counterfactuals is inconsistent~\cite{kochen1,peres222,pitowsky:218,2015-AnalyticKS}.

The quantum-inspired concept most resembling the Molinist middle knowledge is the {\em many-worlds}
interpretation of quantum mechanics, also called {\em splitting worlds theory}~\cite{everett-thesis}.
This hypothesis is based on a particular reading~\cite{Barrett-2011,everett-collw}
of Everett's `relative state' formulation of quantum theory~\cite{everett,sep-qm-everett}.
%          http://plato.stanford.edu/archives/win2008/entries/qm-everett/#3

In the splitting worlds dialect of Everett's relative state formulation,
every time a quantized system is in a coherent superposition of outcomes with respect to a particular measurement
%and this measurement is ``actually performed''
(notwithstanding issues resulting from ``undoing'' measurements, such as in quantum erasure experiments discussed later),
the universe splits into mutually distinct and totally separated ``branches'' or ``worlds.''
Unless they interfere, these worlds continue to have their own independent existence.

As Podolsky put it, the Everettian relative state formulation suggests that {\em ``somehow or other we
have here the parallel times or parallel worlds that science
fiction likes to talk about so much. Every time a decision is
made, the observer proceeds along one particular time while the
other possibilities still exist and have physical reality.~$\ldots$
 It looks like we would have a non-denumerable infinity
of worlds.''}
To this Everett responded by saying, {\em ``Yes.''}
Hence Podolsky continues by saying, {\em ``Each proceeding with its own set of choices
that have been made''}~\cite[pp.~89,90]{podolsky-Xavier-conference}.


It seems that Everett even went a step further by considering the multiplicity of coherent superpositions in terms of
{\em arbitrary bases}.
If taken to the extreme, this would mean that {\em all,} that is, a continuity of conceivable bases, need to be taken into account at all times:
from dimension two onwards, orthogonal bases of vector spaces can be characterized by continuous parameters~\cite{Schwinger.60,murnaghan}.
This yields an enormous multiplicity of relative states as compared to the splitting worlds dialect, which
just considers a single  basis at the time of measurement --
a situation referred to by Barrett as the {\em many-many-worlds} interpretation~\cite[p.~289]{Barrett-2011},
suggesting that it was ``Everett's understanding of
branches that they might be individuated with respect to any basis whatsoever.''



Just as Molinism was created to cope with the problem of the seemingly contradictory assumptions of
divine providence and foreknowledge with individual free creaturely
will, so Everett's relative state argument tried
to cope with what he called {\em the question of the consistency}~\cite[p.~73]{everett-1956}
between two quantum processes, an issue which later became known as ``Wigner's friend''~\cite{wigner:mb}:
how to reconcile,
on the one hand, the continuous unitary quantum evolution process (which essentially amounts to a one-to-one transformation)
of the wave function in between measurements,
with the irreversible measurement process associated with a discontinuous collapse of the quantum state on the other hand?
Because if the former quantum state evolution is ubiquitous and universally valid,
there cannot be any irreversible measurement:
indeed, by a {\em nesting} argument perceived by Everett as an {\em extremely hypothetical drama}~\cite[pp.~74-75]{everett-1956}  it is not at all clear why the
{\em combined} system  comprising `the measurement apparatus merged  with the object observed by the latter'
should not evolve according to some continuous unitary quantum evolution process;
thereby obliterating any alleged discontinuous collapse of the quantum state from the former measurement.
More formally, no
one-to-one unitary transformation of the state  can give rise to some many-to-one process
which ``singles out'' some eigenstate from a nontrivial coherent superposition of many such eigenstates,
thereby loosing information and spoiling reversibility.

[As in the second law of thermodynamics~\cite{Myrvold2011237}, the apparent irreversibility appears to be
means relative to the operational capacities
and valid {\em for all practical purposes} (henceforth abbreviated by fapp).]



Unfortunately, this approach might result in the sort of troubles so vividly expressed by Schr\"odinger's cat ``paradox''
and ``quantum jellification:'' without irreversible measurements the universe would soon decay into a huge
superposition of classically distinct states,
all co-existing simultaneously and interfering with one another.
Schr\"odinger's concern~\cite[p.~19]{schroedinger-interpretation} was that, without measurements,
nature cannot be {``prevented from rapid jellification''}, and that, say, within
{``a quarter of an hour, we should find our surroundings rapidly
turning into a quagmire, or sort of a featureless jelly or plasma, all contours
becoming blurred, we ourselves probably becoming jelly fish.''}



Everett's answer to these concerns might have been that,
since there is {``a homomorphism between its model and the world as experienced,''}
individual embedded observers experience {\it empirical faithfulness}~\cite[Sect.~4]{sep-qm-everett}.
In particular, metaphorically,
any one of the zillion individual ones of {`us'} embedded in Schr\"odinger's quantum quagmire
subjectively perceives history and nature uniquely and solidly nonjellified.


%This seemingly contradicts our experience that the world appears fairly consistent, and not ``jellified.''
%Pointedly stated, the splitting worlds interpretation settles the quantum jellification issue by assuming that every possible
%outcome to an experiment actually occurs in a respective world, thereby placing observers in all these splitting worlds.



One might argue that the splitting worlds interpretation is ontologically different from middle knowledge as it actually assumes the existence of all these parallel worlds.
But what difference does it make to intrinsic, embedded creatures between G imagining ``going down'' counterfactual evolutionary paths, and the universe  evolving that same paths?
Surely there is a metaphysical difference, but
%we doubt that
is there is any operational way to decide between those options? The relevant foundational limiting results in mathematics and physics seem to point to a negative answer.
Also, from a splitting world point of view, there is no necessity, and indeed no room for Molinist free knowledge, as all middle knowledge is realized in the many worlds spanned by G's choices.


\section{Beam splitter as Rosetta Stone}

Suppose G's situation is reduced to an elementary act of volition: a particle passing a 50:50 beam splitter~\cite{green-horn-zei},
often referred to a ``quantum coin toss''~\cite{svozil-qct}.
Quantum mechanics formalizes a 50:50 beam splitter by a 2-dimensional unitary Hadamard matrix
which mixes the state of any particle in any input port into a coherent superposition of states in the two output ports.
Unless measured, any such state (in the output port) is a counterfactual.

Suppose, as our central assumption, that
\begin{itemize}
\item[(i)]
G's natural knowledge %would
involves, among other things, the state evolution of the beam splitter formalized by the Hadamard matrix.
\item[(ii)]
G's consideration of the two distinct states in the two distinct output ports, or even the entire formalized superposition of such states,
%could
can be identified with, and constitutes G's middle knowledge regarding the particle passing the beam splitter.
\item[(iii)]
G's free knowledge is about its  choice (among the two equally likely output ports) of which way the particle takes and is subsequently detected by a fapp irreversible measurement.
\end{itemize}

This choice in (iii) is in no way contingent on any immanent entity, but solely results from G's volition.
For an intrinsic, embedded observer bound by operational means, G's choice %represents
reveals as unpredictable~\cite{acs-2015-info6040773} {\it creatio continua} (one would be tempted to say {\it ex nihilo} but this is reserved in theology to the original act of creation of the universe).



To repeat the identification, Molinism supposes that G contemplates that a particle, upon hitting a beam splitter, passes, and is detected thereafter in all conceivable output ports.
Molinism calls this counterfactual knowledge about all possible outcomes the middle knowledge.
Free knowledge is the information about the actual path in which the particle has been detected after a fapp irreversible measurement has taken place.

In the absence of direct intervention,
G's volition, his freedom of choice, is mediated by  the presence of gaps~\cite[Sect.~II,~12]{frank,franke} in the laws of nature.
Thereby a {\em gap} stands for the {\em incompleteness} of the laws of nature,
which allow for the occurrence of events without natural (immanent, intrinsic) cause~\cite[Sect.~II,~9]{frank,franke}: {\em ``Under
certain circumstances they do not say what definitely has to happen
but allow for several possibilities; which of these possibilities comes
about depends on that higher power which therefore can intervene
without violating laws of nature.''}
In this conception, G constantly creates the world by his voluntary micro-choices.
This  assumption restricts severely the ``universality'  of ``law of nature''~\cite{lawless}.




The beam splitter example could be taken one step further: the serial composition of two beam splitters
--
the output ports of the first beam splitter are the input ports of the second beam splitter
(augmented with an optional phase shifter in one of the beam paths)
--
renders a Mach-Zehnder interferometer which essentially exhibits the capacity of
counterfactuals to interfere. This construction is very much in the spirit of Everett,
who never considered a total separation of the relative states
formed~\cite{sep-qm-everett}.
By analogy, the possibility of branches of middle knowledge to interfere in certain special circumstances should also be acknowledged and discussed by Molinism.




\section{Summary and outlook}

Molinists have so far never struggled with the more (or less) subtle issues regarding quantum counterfactuals.
In particular, in order to cope with Kochen-Specker type theorems~\cite{kochen1,peres222,pitowsky:218,2015-AnalyticKS}
some middle knowledge needs to be either self-contradictory (relative to apparently reasonable side assumption such as noncontextuality),
or contextual.

Also, measurements can be ``undone'' and
``erased''~\cite{PhysRevD.22.879,PhysRevA.25.2208,greenberger2,Nature351,Zajonc-91,PhysRevA.45.7729,PhysRevLett.73.1223,PhysRevLett.75.3783,hkwz},
and, this time quoting Prospero, {\em ``like this insubstantial pageant faded,
leave not a rack behind.''}
Thereby middle knowledge which, %once has become
for a while was free knowledge becomes middle knowledge again.
Indeed, free knowledge can be permuted back and forth into middle knowledge in an unlimited way (of course there are fapp restrictions).
In Molinistic terms, such experiments convert free knowledge back to middle knowledge by reversing G's volition.
Therefore one wonders if there should be any difference between middle and free knowledge.

This exchange between free and middle knowledge reflects the quantum mechanical debate whether  a measurement ``truly'' exist.
As already pointed out by Everett~\cite{everett}
the assumption of universal ubiquitous validity of the quantum state evolution (which is unitary and essentially a state permutation)
rules out any irreversibility, because the latter would require a many-to-one mapping, whereas the former remains strictly one-to-one.

We should also bear in mind that, as long we do not recognize intentionality, G's {\em creatio continua} is just another word for the ancient Greek notion of \textgreek{q'aoc}.


Let us finally point out that, if one is willing to ease the tight restrictions imposed upon creatures and divine volition by a zero-sum game,
%there appears to be
a simple ``solution'' may be considered: that creatures are, indeed, expressions or aspects of G, and therefore any of their volitions
%by them
is a volition of G
(we do not wish to enter into the debate on evil at this point).
Statements like {A}l {H}osayn-ibn-{M}ansour al-{H}allaj's~\cite{Massignon-1922} ``I am God'' sometimes have been considered incomprehensible or even heresy.
And yet the Vedic {\it ``Tat tvam asi''} (Engl. translation ``this is you'') lies at the heart of Schr\"odinger's world view~\cite[p.~22]{book:1170675}.


The mathematical limitations of the standard approach to the notion of ``truth'' quoted in Section~\ref{2016-moninism-sec1}
arise from a rigid ``inconsistency-intolerant'' position, which is perfectly justified in metamathematical studies because
in most classical logics the {\em principle of explosion} holds:  {\em ex contradictione sequitur quodlibet}, that is,
``from a contradiction, anything follows.'' However,  both
theological and quantum studies may benefit from a more flexible,
 `inconsistency-tolerant'' view.
This means to work with {\em paraconsistent logics}~\cite{sep-logic-paraconsistent} --
weaker, more permissive systems that can be used to reason with inconsistent information in a controlled,
discriminating, non-trivial way -- instead of the more rigid classical logics.


The ``gigantic'' multitude of counterfactual worlds should not
concern us too much -- after all, our recent estimates
yield trillions of galaxies~\cite{0004-637X-830-2-83}
in the observable universe, each carrying  billions to hundred trillions of stars.


\medskip


\begin{acknowledgments}
This work was supported in part by Marie Curie FP7-PEOPLE-2010-IRSES Grant RANPHYS (CC \& KS),
and by the John Templeton Foundation's {\em  Randomness and Providence: an Abrahamic Inquiry Project} (KS). We thank Jeff Barrett, Kelly Clark, Jeffrey Koperski and Drago\c{s} Vaida for useful comments and suggestions.

\end{acknowledgments}

\bibliography{svozil}


\end{document}
