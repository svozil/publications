\documentclass[11pt]{article}
\usepackage{geometry,url}                % See geometry.pdf to learn the layout options. There are lots.
\geometry{letterpaper}                   % ... or a4paper or a5paper or ...
%\geometry{landscape}                % Activate for for rotated page geometry
%\usepackage[parfill]{parskip}    % Activate to begin paragraphs with an empty line rather than an indent
\usepackage{graphicx}
\usepackage{amssymb}
\usepackage{epstopdf}
\DeclareGraphicsRule{.tif}{png}{.png}{`convert #1 `dirname #1`/`basename #1 .tif`.png}

\title{Exercises in De-Quantisation}
\author{Cristian S. Calude and Karl Svozil}
%\date{}                                           % Activate to display a given date or no date

\begin{document}
\maketitle
\section{Simulation of quantum algorithms based on superpositions}
%\subsection{}
Consider a Boolean function $f: \{0,1\} \rightarrow \{0,1\}$
and suppose that we have a black box to compute it.
Deutsch's problem asks to test
whether $f$ is {\it constant} (that is, $f(0) = f(1)$) or {\it balanced} ($f(0) \not= f(1)$) allowing  {\it only one query} on the black box computing $f$.

In a famous paper published in 1985,  Deutsch \cite{D85} obtained a ``quantum''  {\it partial affirmative answer}.  In 1998, a complete, probability-one
solution was found by Cleve, Ekert, Macchiavello, and Mosca \cite{CEMM}. In \cite{Cris}
it was shown that the quantum solution can be {\it de-quantised} to a deterministic simpler solution which is as efficient as the quantum one.

Th core technique used  by the quantum solutions is to {\it coherently} embed the classical black box into a quantum black box (the quantum black box produces the same outputs as the classical black box when the inputs are the pure Qbits $|0\rangle, |1\rangle$), then perform a special computation with the quantum black box on a {\it superposition} of carefully chosen quantum states (this computation has
no classical meaning for the original black box), and finally {\it measure} the output produced. The analysis proposed in \cite{Cris} showed that
the same quantum technique,
embedding plus computation on a ``superposition'', leads to a  classical solution which is as efficient as the quantum one. More, the quantum solution is {\it probabilistic}, while the classical solution is {\it deterministic}.\\


{\bf Question 1. }
How does the classical solution compare with the quantum one in terms of physical resources?  A simple analogical scheme  can implement the classical solution with two
registers each using a real number as in the quantum case when we need just two Qbits. However, a more realistic analysis should involve the complexity of the black box, the complexity of the implementation of the embedding, as well as the complexity of the query performed.\\


{\bf Question 2. } The simulation of superposition doesn't scale with the
idea below. Show how to obtain a similar solution  for a fixed $n$, but not uniformly (in each case a different function is used).
%it takes exponentially long to read in the general state
%with $n$ basis vectors, since you have $n$ choose $k$ monomials  %of degree $k$.
Of course, uniformly the solution discussed in this note is not scalable, because  $n$ Qbits can represent $2^n$
states at the same time, which outgrows any linear function of $n$ (see \cite{DJ92}).\\


{\bf Question 3. } Find other quantum algorithms based on the ``superposition'' technique only and try to construct classical algorithms as efficient as the quantum ones.

\section{Simulation of quantum algorithms based on entanglement}

Proposed issues:
\begin{itemize}
\item
Short description of quantum entanglement.
\item
Example of the simplest quantum algorithm based on entanglement. Quantum Fourier Transform?

\item An example of dequatisation.
\item Proposed questions.


\end{itemize}


\begin{thebibliography}{99}
\bibitem{Cris} C. S. Calude. De-quantising  the  solution of Deutsch's
problem, {\em International Journal of Quantum Information} 5, 4(2007), 1--7.

\bibitem{CEMM} R. Cleve, A. Ekert, C. Macchiavello,  M. Mosca. Quantum algorithms revisited, {\em   Proceedings of the Royal Society of London}  Series  A454 (1998), 339--354.
\bibitem{D85} D. Deutsch.  Quantum theory, the Church-Turing principle, and the universal quantum computer, {\em Proceedings of the Royal Society of London}  Series  A400 (1985),  97--117.
\bibitem{DJ92} D. Deutsch and R. Jozsa.  Rapid solutions of problems by  quantum
computation,  {\em Proceedings of the Royal Society of London}  Series A439 (1992),  553.
\bibitem{Gruska} J. Gruska. {\it Quantum Computing}, McGraw-Hill, London, 1999.
\bibitem{Mermin} D. Mermin. {\em Quantum Computation
Lecture Notes and Homework Assignments}, Chapter 2,
Cornell University, Spring 2006, \url{http://people.ccmr.cornell.edu/~mermin/qcomp/chap2.pdf}, accessed on 7 October 2006.
\bibitem{NC} M. A. Nielsen, I. L. Chuang. {\em Quantum Computation and Quantum Information}, Cambridge University Press, Cambridge, 2001.
\end{thebibliography}

\end{document}

