\PassOptionsToPackage{usenames,dvipsnames}{xcolor}
%\documentclass[amsmath,table,sans,amsfonts, handout]{beamer}
\documentclass[amsmath,table,sans,amsfonts,hyperref={colorlinks,citecolor=blue,linkcolor=blue,urlcolor=purple}]{beamer}
\usepackage[T1]{fontenc}
%%\usepackage{beamerthemeshadow}
%%\usepackage[headheight=1pt,footheight=10pt]{beamerthemeboxes}
%%\addfootboxtemplate{\color{structure!80}}{\color{white}\tiny \hfill Karl Svozil (TU Vienna)\hfill}
%%\addfootboxtemplate{\color{structure!65}}{\color{white}\tiny \hfill mur.sat \hfill}
%%\addfootboxtemplate{\color{structure!50}}{\color{white}\tiny \hfill Graz, 2010-12-11\hfill}
%\usepackage[dark]{beamerthemesidebar}
%\usepackage[headheight=24pt,footheight=12pt]{beamerthemesplit}
%\usepackage{beamerthemesplit}
%\usepackage[bar]{beamerthemetree}
\usepackage{graphicx}
\usepackage{pgf}
%\usepackage{eepic}
%\newcommand{\Red}{\color{Red}}  %(VERY-Approx.PANTONE-RED)
%\newcommand{\Green}{\color{Green}}  %(VERY-Approx.PANTONE-GREEN)

\definecolor{applegreen}{rgb}{0.55, 0.71, 0.0}

\usepackage{fourier-orns}  %fancy symbols https://mirror.easyname.at/ctan/fonts/fourier-GUT/doc/fourier-orns-doc.pdf

%\usepackage{musixtex}

\newcommand{\Abschnitt}[1]{{\section #1}}

%%%%%%%%%%%%%%%%%%%%%%%%%%%%%
\usepackage{iftex}
\ifxetex
\usepackage{fontspec}% Schriftumschaltung mit den nativen XeTeX-Anweisungen
                     % vornehmen. Voreinstellung: Latin Modern
%\usepackage[ngerman]{babel}% Sprachumschaltung: Deutsch nach neuer Rechtschreibung



%
% XeLaTeX
%
\XeTeXinputencoding cp1252
\usepackage{fontspec}
%%\setmainfont{Times New Roman}
\setmainfont{Garamond}
\setsansfont{Garamond}
%\setmainfont{EB Garamond}
%\setsansfont{EB Garamond}
%
\else
\usepackage[latin1]{inputenc}
\usepackage[T1]{fontenc}
\fi
%%%%%%%%%%%%%%%%%%%%%%%%%%%%%

%\RequirePackage[german]{babel}
%\selectlanguage{german}
%\RequirePackage[isolatin]{inputenc}

%\pgfdeclareimage[height=0.5cm]{logo}{tu-logo}
%\logo{\pgfuseimage{logo}}
\beamertemplatetriangleitem
%\beamertemplateballitem

\beamerboxesdeclarecolorscheme{alert}{red}{red!15!averagebackgroundcolor}
%\begin{beamerboxesrounded}[scheme=alert,shadow=true]{}
%\end{beamerboxesrounded}

%\beamersetaveragebackground{yellow!10}

%\beamertemplatecircleminiframe

\newtheorem{question}{Question}
\newtheorem{conjecture}[question]{Principle}
\newtheorem{challenge}[question]{Challenge}
\usepackage{tikz}
\newcommand{\bra}[1]{\left< #1 \right|}
\newcommand{\ket}[1]{\left| #1 \right>}

\newcommand{\iprod}[2]{\langle #1 | #2 \rangle}
\newcommand{\mprod}[3]{\langle #1 | #2 | #3 \rangle}
\newcommand{\oprod}[2]{| #1 \rangle\langle #2 |}

\newcommand{\proj}[3]{\begin{smallmatrix} #1 & #2 & #3 \end{smallmatrix}}
\newcommand{\projbf}[3]{\begin{smallmatrix} \mathbf{#1} & \mathbf{#2} & \mathbf{#3} \end{smallmatrix}}

\sloppy
\parskip .7em %vskip between paragraphs

\newcommand{\seq}[1]{\mathbf{#1}}
\newcommand{\floor}[1]{\left\lfloor #1 \right\rfloor}
\newcommand{\ceil}[1]{\left\lceil #1 \right\rceil}
\newcommand{\m}[1]{\widetilde{#1}}
%\newcommand{\p}[1]{\scriptsize\textcolor{black}{$[#1]$}}

\usepackage[most]{tcolorbox}
\begin{document}

\title{\textcolor{black}{\bf Emergent Relativity: Transcribing Hilbert into Configuration Space}}
\subtitle{
\small \url{http://tph.tuwien.ac.at/~svozil/publ/2023-DG-pres.pdf}
 \\
\small based on \href{https://arxiv.org/abs/2308.09715}{arXiv:2308.09715}
}
\author{\textcolor{black}{Karl Svozil}}
\institute{\textcolor{black}{\small Institute for Theoretical Physics, TU Wien}\\
\textcolor{black}{\small karl.svozil@tuwien.ac.at}
%{\tiny Disclaimer: Die hier vertretenen Meinungen des Autors verstehen sich als Diskussionsbeitr�ge und decken sich nicht notwendigerweise mit den Positionen der Technischen Universit�t Wien oder deren Vertreter.}
}
\date{{\color{purple}\small {\color{black}presented on September 28, 2023, at} \\
From GHZ to Tic Tac Toe:
A symposium to celebrate Danny
Greenberger's 90th birthday, \\
Vienna, Austria, September 27--29, 2023}}
\maketitle


% \frame{
% \frametitle{Contents}
% \tableofcontents
% }

\section{Category formation -- the `bigger' picture}

\begin{frame}
 \frametitle{What I have learned from Danny Greenberger}

Danny inspired and encouraged me (after experiencing some mild PTSD after my studies in Vienna and Heidelberg) to
\begin{itemize}
\item[$\bullet$]
$\ldots$ perceive physical categories as historical and temporal. Eg, from my memory at the `Solar Eclipse Workshop, August 8 - 11, 1999':
{\it ``{\color{purple}In four hundred years from now, everything we know in physics today will appear laughable}.''}

\item[$\bullet$]
learn from magicians to create {\bf \it \color{purple} `mind-boggling'} physical situations:
I was once invited to stay in Suzy's and Danny's New York home, and I lodged in the library with a HUGE stack of books on
{\color{purple}magic}.

\item[$\bullet$]
%$\ldots$ think lightweightedly about the world, and, in particular, about physics! Take nothing as granted!
$\ldots$ to, in Dyson's words, {\color{purple}`pursue the unfashionable'}.

\end{itemize}


\end{frame}

\begin{frame}
 \frametitle{Ontologic versus epistemic Magic?}

Here is a metaphysical question: {\color{purple}Is the quantum magic ontologic or epistemic?}

Or, stated differently: {\color{purple} Are our current limitations and ignorance about fundamental aspects of nature reducible,
or do they represent an irreducible 'hard impenetrable core'?}

Or, in terms of theology and Philip Frank: {\color{purple} Are we experiencing creatio continua -- gaps in the laws of Nature -- on a massive scale?}

{\color{blue} \footnotesize Beware: there is only a small gap between practical magic � la Husimi and the hocus-pocus and abra-kadabra of charlatans!
As with all the arts, such as music, painting, architecture, or theater, the range of bearable aesthetic complexity lies within a relatively narrow bracket, bounded by monotony and chaos.
Physicists are like Odysseus, navigating between Scylla and Charybdis, between neurosis and psychosis,
between Nietzsche's Apollonian \& Dionysian traits.
}


\end{frame}

\begin{frame} [shrink=8]
\frametitle{Some rants on `transcribing quantum Hilbert into classical configuration space'}

This title is just `too good to be true' currently. So I shall rather present some rants or thoughts on the subject.

I share with Pascal the feeling that `I have made this longer' than necessary  because I lack both ability and time to make it shorter.

I shall talk about:
\begin{itemize}
\item[$\bullet$]
System science, SI conventions and Alexandrov's Theorem

\item[$\bullet$]
The conundrum of persistance of relational properties without local value definiteness

\item[$\bullet$]
Bell inequalities for arbitrary space-time separations

\item[$\bullet$]
After seeing the title of Adan's talk I hereby  advertize a recent paper ``Generalized Greenberger-Horne-Zeilinger (GHZ) Arguments
from Quantum Logical Analysis'' \href{https://doi.org/10.1007/s10701-021-00515-z}{DOI: 10.1007/s10701-021-00515-z};
which contains a quantum logical analysis, as well as generalized GHZ games.
for Danny, in particular: ``Stranger-than-Quantum GHZ Games''.

\end{itemize}

\end{frame}

\begin{frame}[shrink=8]
 \frametitle{Intrinsic perception and the complementarity-versus-contextuality game}

In 1978 Tommaso Toffoli published `The Role of the Observer in Uniform Systems'  \url{https://doi.org/10.1007/978-1-4757-0555-3_29} :

{\it ``$\ldots$~ certain problems that arise when one tries to model a given system as an
{\color{purple}object embedded in a uniform medium}.''}

Are we such objects embedding in, say, a computational universe, such as Zuse's cellular `{\color{purple}Rechnender Raum}'?
Very few of us realize that such computational universes are subject to `{\color{purple}computational complementarity}' akin to Wright's `geberalized urn models'
(cf. partition logics).

Or is the stuff we and our experiments are made of, according to the Church of the Large(r) Hilbert Space, some {\color{purple}vectors} in some tensor product space, according to von Neumann's quantization-by-Hilbert-space?
Whereas such supposedly  {\color{purple}indecomposable vectors  akin to `entangled states' may be value definite}, a {\color{purple}classical re-interpretation forced upon them cannot be value definite}
-- cf Kochen \& Specker's Theorem~0: `separability by truth assignments as demarcation criterion'.

\end{frame}




\begin{frame}%[shrink=8]
 \frametitle{Category formation and Hertz'es `images or symbols of external objects'}

 {\color{black}
Evolutionary biology and psychology `pushed' or `stipulated' a particular `worldview'
or formation of categories of perception  in terms of what we might call `objects' or `mental images'
of our thinking.

Cf Hertz: {\it ``We form for ourselves {\color{purple}images} or symbols of external objects;
and the form which we give them is such that the necessary
consequents of the images in thought are always the images of
the necessary consequents in nature of the things pictured. In
order that this requirement may be satisfied, there must be a
certain {\color{purple}conformity between nature and our thought}.''}

Those objects have no stringent ontological status, although we might believe so.
In particular, space-time frames are such `images'.
We need to be careful in conceptualizing them in physical terms,
and thereby {\color{purple}separate the `chaff from the wheat'}: the conventions and assumptions from the physical content.
}

\end{frame}

\section{Challenges to space-time formation for `non-localized' contextuality}



\begin{frame}[shrink=10]
 \frametitle{Alexandrov's Theorem of Incidence Geometry}
\href{https://www.mathnet.ru/eng/rm/v5/i3/p187}{1949 Communication} by Alexandrov:

{\it ``Lorentz transformations can be defined
as mutually single-valued transformations
of four-dimensional space on itself, translating any [[light]] cone
$
(x_1-x_1^0)^2
+
(x_2-x_2^0)^2
+
(x_3-x_3^0)^2
-
(x_4-x_4^0)^2
=0$
into a cone of the same kind.''}

So, if constancy of light (and uniqueness) is assumed---as is \href{https://doi.org/10.1038/303373a0}
{currently the case within the SI system}---then the Lorentz transformation follow.

As \href{https://doi.org/10.1038/312010b0}{Peres dryly remarked}:
{\it ``Now, this has become a
definition and the corresponding empirical
fact is that the length of a solid body
depends neither on its spatial orientation,
nor on the inertial frame where that body is
at rest.''}

The physical content is in the form invariance of the equations of motion, eg, the Maxwell equations.

(Alexandrov's result remained unnoticed and thus has been re-derived multiple times;
cf Lester's review \href{https://doi.org/10.1016/B978-044488355-1/50018-9}{DOI: 10.1016/B978-044488355-1/50018-9}.)





\end{frame}


\begin{frame}[shrink=10]
 \frametitle{Challenges to space-time formation for `non-localized' contextuality}

 {\color{black}
`Non-localized' contextuality exploiting entanglement---ie non-factorizable quantum states---present a potential problem to classical space-time constructions,
eg via Einstein synchronization in conjunction with randomness of the outcomes:

\begin{itemize}
\item[$\bullet$]
The (pure) quantum state of the entangled constituents essentially
(due to non-local unitary transformation of localized product states) re-encodes or scrambles those state to be {\color{purple} relational}
and (unlike classical states) {\color{purple} devoid of indidual definite `localized' properties or local shares}---only
the relational properties among its constituents are value definite.

\item[$\bullet$]
If any single random outcome, under strict Einstein locality conditions,
is somehow only subjected to or influenced by the `local' environment of the particle constituent's
environment, and at the same time (see earlier) the entire state lacks any individual local property or local share of the constituent particles whatsoever---then~$\ldots$

\item[$\bullet$]
{\color{purple} How come relationality is maintained and strictly observed?}

\end{itemize}
}
\end{frame}


\frame{
 \frametitle{Challenges to space-time formation for `non-localized' contextuality cntd.}

{\color{black}

This kind of  of `relationality without individual definiteness', without any kind of local shares, between space-time regions
(defined via standard Poincar\'e-Einstein radar coordinates) that
spacially separated under strict Einstein locality conditions,
seems to be the gist of the conundrum.

Recall that classical states, as for instance pointed out by Peres in `unperformed experiments have no result',
perform relationally by {\color{purple} possessing correllated and value definite local shares},
which are revealed by measurements, and, therefore, may even result in (perfect) relational correllations.

(Even cosine-type correlations are insufficient; but communication of contexts are: cf. \url{https://arxiv.org/abs/2209.09590})
}
}



\begin{frame}%[shrink=8]
 \frametitle{Challenges to space-time formation for `non-localized' contextuality cntd.}

 {\color{black}
In order to reasonably categorize space-time (frames) we need to take
quantum mechanics as primary, and develop and operationalize space-time frames entirely by quantum means.

This is different from Newtonian (versus Leibnitzian) and possibly Kantian conceptualizations of space-time
`theatre' as `intuited a priori'.

As a consequence we need to observe what, in quantum terms, may be considered {\color{purple} separate}, and what not.

I postulate that constituents in entangled states {\color{purple} as well as measurement outcomes on such states cannot be considered separate}.

}

\end{frame}
\begin{frame}%[shrink=8]
 \frametitle{Challenges to space-time formation for `non-localized' contextuality cntd.}

 {\color{black}

Therefore, at least in the entangled observables, the constituents are not spatially separated at all: in other words,
for such `affected' observables, spatial distances shrink to zero.

This does not necessarily mean that with respect to other observables of these constituents, the separation is zero.

In this view {\color{purple} spatial separation is means relative}, and thus {\color{purple}  space-time frames are means relative} with respect to
the quantum shares involved.

}

\end{frame}

\begin{frame}%[shrink=8]
 \frametitle{`Generalized' Boole-Bell type inequalities for generalized space-time locatedness}

 We aim to encompass time-like entanglement.  This type of entanglement can---in the customary space-time frames that
we assume to be ad hoc creations of certain nonentangled elements, such as light rays of classical optics,
in the standard Poincar\'e-Einstein `radar' synchronization protocols---be generated through processes such as delayed-choice entanglement swapping.
Formally, achieving this involves reordering the product $\vert \Psi^-_{12} \Psi^-_{34} \rangle$,
expressed in terms of the four individual product states
$\vert \Psi^+_{14}   \Psi^+_{23}  \rangle$,
$\vert \Psi^-_{14}   \Psi^-_{23}  \rangle$,
$\vert \Phi^+_{14}   \Phi^+_{23}  \rangle$, and
$\vert \Phi^-_{14}   \Phi^-_{23}  \rangle$
of the Bell bases of the `outer' (14) and `inner' (23)
particles.
Bell state measurements of the latter, `inner' particles yield a rescrambling of the `outer'
correlations. Hence, postselecting the `inner' pair (23) results in the desired `outer'
Bell states (14), respectively.

\end{frame}

\begin{frame} [shrink=10]
 \frametitle{Mixed Boole-Bell type inequalities for generalized space-time locatedness}

 In more detail, in terms of the  Bell basis,
\begin{equation}
\begin{split}
\vert \Psi_{12}^- \Psi_{34}^- \rangle &=   \frac{1}{2} \left(\vert \Psi_{14}^+ \Psi_{23}^+  \rangle - \vert \Psi_{14}^- \Psi_{23}^-  \rangle     - \vert \Phi_{14}^+ \Phi_{23}^+  \rangle + \vert \Phi_{14}^- \Phi_{23}^- \rangle \right), \\
\vert \Psi_{12}^+ \Psi_{34}^+ \rangle &=   \frac{1}{2} \left(\vert \Psi_{14}^+ \Psi_{23}^+  \rangle - \vert \Psi_{14}^- \Psi_{23}^-  \rangle     + \vert \Phi_{14}^+ \Phi_{23}^+  \rangle - \vert \Phi_{14}^- \Phi_{23}^- \rangle \right), \\
\vert \Phi_{12}^- \Phi_{34}^- \rangle &=   \frac{1}{2} \left(-\vert \Psi_{14}^+ \Psi_{23}^+  \rangle - \vert \Psi_{14}^- \Psi_{23}^-  \rangle   + \vert \Phi_{14}^+ \Phi_{23}^+  \rangle + \vert \Phi_{14}^- \Phi_{23}^- \rangle \right), \\
\vert \Phi_{12}^+ \Phi_{34}^+ \rangle &=   \frac{1}{2} \left(\vert \Psi_{14}^+ \Psi_{23}^+  \rangle + \vert \Psi_{14}^- \Psi_{23}^-  \rangle   + \vert \Phi_{14}^+ \Phi_{23}^+  \rangle + \vert \Phi_{14}^- \Phi_{23}^- \rangle \right).
\end{split}
\label{2023-st-MagicBellBasisstates1factoring}
\end{equation}
In the `magic' Bell basis where $\vert \Psi^- \rangle $ and $\vert \Phi^+ \rangle $
are multiplied by the imaginary unit  $i$, the relative phases change accordingly.

Delay lines serve as essential components for temporal entanglement.
These delay lines could, in principle, also lead to mixed temporal-spatial quantum correlations.

\end{frame}



\section{Suggestions for a new protocol of clock synchronization}
\frame{
 \frametitle{Suggestions for a new protocol of clock synchronization}

 {\color{black}
For entangled shares I therefore suggest to abandon Einstein clock synchronization by exchanges of (light) signals.

I suggest to employ a Bennett-Brassard-Eckert-type protocol utilizing random outcomes of entangled multi-partite states
as a time standard.
Thereby, local entangled time is successively made precise and generated by the correlated outcomes of entangled states.

As a result, relativity theory is `relativized' further by into a multitude of means relative `patches' of space-time (frames).

}

}

\frame{

\centerline{\Large {\color{purple} Thank you for your attention!}}

\begin{center}
\includegraphics[width=0.5\textwidth]{Scilla-and-Charybdis1.jpeg}
\\
{\color{purple} Non est ad astra mollis e terris via\\
-- per aspera ad astra}
\\
\color{orange}
$\widetilde{\qquad \qquad }$
$\widetilde{\qquad \qquad}$
$\widetilde{\qquad \qquad }$
\end{center}
 }
 \end{document}


















\section{ }

\frame{
 \frametitle{ }

\begin{itemize}
\item[$\bullet$] {
%\color{purple}
}
\pause
\item[$\bullet$] {
%\color{purple}
}
\end{itemize}
}

\section{ }

\frame{
 \frametitle{ }

\begin{itemize}
\item[$\bullet$] {
%\color{purple}
}
\pause
\item[$\bullet$] {
%\color{purple}
}
\end{itemize}
}

\section{ }

\frame{
 \frametitle{ }

\begin{itemize}
\item[$\bullet$] {
%\color{purple}
}
\pause
\item[$\bullet$] {
%\color{purple}
}
\end{itemize}
}

\section{ }

\frame{
 \frametitle{ }

\begin{itemize}
\item[$\bullet$] {
%\color{purple}
}
\pause
\item[$\bullet$] {
%\color{purple}
}
\end{itemize}
}

