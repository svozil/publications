%%tth:\begin{html}<LINK REL=STYLESHEET HREF="/~svozil/ssh.css">\end{html}
\documentclass[12pt]{article}
\RequirePackage{times}
\RequirePackage{amsfonts}
\RequirePackage{courier}
\RequirePackage{mathptm}
\RequirePackage{showkeys}
\begin{document}
\title{Solve et coagula}
\author{Karl Svozil\\
 {\small Institut f\"ur Theoretische Physik, University of Technology Vienna }     \\
  {\small Wiedner Hauptstra\ss e 8-10/136,}
  {\small A-1040 Vienna, Austria   }            \\
  {\small e-mail: svozil@tuwien.ac.at}\\
  {\small and}\\
Rom\`an Zapatrin\\
 {\small Quantum Information Group, Fondazione I.S.I. }     \\
  {\small Villa Gualino, Viale Settimio Severo 65,}
  {\small I-10133 Torino, Italy   }            \\
  {\small e-mail: zapatrin@isiosf.isi.it}
}
\maketitle
\begin{abstract}
We exploit the fact that a mixed state of a multipartite quantum
system can be represented by ensembles of pure states in
essentially different ways with respect to the localization of the
entanglement among parties. In more detail, this means the
following. Suppose we have two ensembles E1 and E2 which give rise
to the same quantum state. Then it may occur that two particles may
be entangled when thay are in E1 and not entangled when they are in
E2.
All ensembles representing a mixed multipartite state can be
described by a collection of partitions of the set of all
particles. We start with a formal analogy between this description
and the algebraic structure of the set of propsitions of a quantum
system. It turns out that for the case of separable states this can
be also simulated by Generalized Urn Models which are classical
ones.
Our goal is to provide an algebraic description in which the notion
of a particle - a component of a multipartite state - is secondary,
that is, derived from the statistics of the observations upon a
state.
Provisionally, this can be used for engineering of entangled states
based on `large' quantum systems in such a way that their
multipartite structure is nothing but a simulation, however,
undistinguishable by a given set of available measurements.
\end{abstract}
\section{Separability of multipartite mixed states}
In what follows, we shall pursue a radical inductive approch to
multipartite systems by starting from the operationally obtainable
properties of the system and attempting to recover its structure.
Thereby, we cannot a priori claim that there is something
fundamental to the notion of, say, located particles. We are just
dealing with propositions about physical systems.
Entanglement is a crucial resource for quantum information
processing and quantum communication. As it turns out, while it may
be easy to produce nonentangled states, it is difficult to
fabricate and maintain entangled ones. Recently, multipartite
entanglement has been classified by the use of partitions.
For instance, given a particular state and two subsystems of the
system which is in that state; the fact of whether or not these
subsystems are entangled depends on a particular ensemble which
represent this state. This phenomenon even starts with three qubit
systems (see  the 2-qubit biseparable state in
\cite{2000-duer-cirac} as a concrete example).
Add Horodeckies.....
Consider a composite ${N}$-partite system which is in a state
described by a density matrix ${\rho}$. Then, ${\rho}$ is called
{\em decomposable} if it can be represented as a tensor product of
density matrices of subsystems; i.e., $$ {\rho}_1
\otimes\ldots\otimes {\rho}_{N}. $$ A state ${\rho}$ is called
{\em separable} if its density matrix is a convex combination of
decomposable ones; i.e., \begin{equation}\label{e03s} {\rho} = \sum
p_\alpha {\rho}_1^\alpha \otimes\ldots\otimes {\rho}_{N}^\alpha   ,
\end{equation} with $p_\alpha\ge0$ and $\sum{}p_\alpha=1$.
The states which are not separable are called {\em entangled}. It
worth mentioning that for classical systems any state is separable,
therefore the notion of entangled states makes  sense only for
quantum systems.
Let us briefly discuss how such a separable state might be
prepared. The  composite system consists of what we  shall call
`particles' located at ${N}$ sites, or ${N}$ loci ({\tt Karl, do
you like the term `loci'?}. At each site $i$ an observer who can
prepare a system in any of the states $\rho^\alpha_i$,
($\alpha=1,\ldots,M$). There is also a `supervisor' out there who
tosses ageneralised coin with $M$ faces with probabilities
$p_\alpha$. When an outcome $\alpha$ occurs, the supervisor
communicates the result (the value $\alpha$ or the
which state to prepare) to all local observers usinng classical
communication means.
We would like to dwell on one issue in this scheme which is always
treated as default. Namely, there exists a classical observable,
call it $Q$ whose values are coordinates - the locations of
appropriate particles constituting the composite system.
After that we can refine the proposed operationalistic  scheme. The
supervisor sends the instructions to prepare the state
$\rho^\alpha_i$ to the location associated withthevalue $i$ of the
observable $Q$.
{\tt This I would like to delete.
BEGIN DELETE
}
Alternatively, we might associate the supervisor role to one of the
local observers. In that way, if this observer measures a
particular state on his particle or group of particles, the results
are communicated to all the other observers to the effect that
their local systems are selected or not.
{\tt END DELETE}
\section{Generalization to parties}
Let us weaken the condition (\ref{e03s}) for states of a composite
system to be decomposable and separable. Consider a partition of
the whole set of ${N}$ subsystems onto $k$ subsets, associate with
each subset the appropriate Hilbert space ${\mathfrak F}_i$. Every
factor space ${\mathfrak F}_i$ corresponds to the tensor product of
one or more Hilbert spaces of initial subsystems which are
associated with the same partition element. In this way, the state
space ${\mathfrak H} ={\mathfrak H}_1 \otimes\cdots\otimes
{\mathfrak H}_{N}$
of the composite system can be represented as a tensor product of factor spaces
\begin{equation}
{\mathfrak H}
=
{\mathfrak F}_1
\otimes\cdots\otimes
{\mathfrak F}_k .
\end{equation}
This condition of separability now becomes {\em relative} with respect to a
partition ${P}$ of the set of all subsystems;
that is, with respect to the decomposition of ${\mathfrak H}$ into single particle Hilbert spaces.
Given a partition ${P}=\{{P}_1,\ldots,{P}_M\}$  and
a  density matrix ${\rho}$.
Again ${\rho}$ is
called {\em ${P}$-decomposable} whenever it can be  represented
as a  tensor product
$
{\rho}
=
{\rho}_{{P}_1}
\otimes\cdots\otimes
{\rho}_{{P}_M}$;
and {\em ${P}$-separable} if it is a convex
combination of ${P}$-decomposable states; i.e.,
\begin{equation}\label{e04}
{\rho}
=
\sum
p_\alpha
{\rho}_{{P}_1}^\alpha
\otimes\ldots\otimes
{\rho}_{{P}_{N}}^\alpha
\end{equation}
In other words, (\ref{e04}) means that we can prepare
${\rho}$ as an ensemble of mixed states ``located at sites''
${P}_1,\ldots,{P}_M$.
Given an arbitrary mixed state ${\rho}$, for each partition ${P}$ of the set  of subsystems if
${\rho}$,
we  may  ask whether or not ${\rho}$ is
${P}$-separable.
As a result we  obtain the set
${\Pi}({\rho})$ of partitions  with respect to which
${\rho}$ is separable (\ref{e04}):
\begin{equation}\label{e05}
{P}\in{\Pi}({\rho})
\quad\Leftrightarrow\quad
{\rho}
\quad
\mbox{is ${P}$-separable}
\end{equation}
If a state $\rho$ is separable with respect to a certain partition,
then it remains also separable with respect to any coarser
partition. Hence, in order to specify ${\Pi}({\rho})$ we need to
provide only maximal partitions ${P}$ such that ${\rho}$ is
${P}$-separable.
How this affects the operationalistic preparation scheme proposed
above? It occurs that in order to prepare the same state different,
generically overlapping loci should be used. At first sight that
means that instead of one observable  $Q$  mentioned  above we now
have several observables $Q_\Pi$ each associated with a maximal
partition $\Pi$ with respect to which the state $\rho$ in question
is separable.
Our first observation is just formal one. Recall that when we  are
dealing with a quantum system we may have several no-commutibg
observables. This similarity can be extended. On one hand,
different partiotions $\Pi, \Pi'$ may have  common elements (loci)
and  on the  other hand, different quantum observables though
non-commuting, may have  common eigenvectors.
When we are dealing with equivalence classes of quantum observables
with respect to mutual commutativity, the standard formalism to
capture it is to introduce a  quantum logic - the collection of
properties of the system in question in our experimental setup. Let
us briefly recall the basic definitions.
{\tt HERE GOES A BRIEF INTRO TO QL}
Our key suggestion is thefollowing. In orderto capture the  variety
of ensembles representing a given multipartite state, we make the
beehaviour of the supervisor quantum!. That means that preparing a
mixed state includes  one more wavefunction collapse---that leading
to a  supervisor's choice.
{\tt 2 remarks. 1. I am afraid this becomes more and more `fremd'
for Phys Rev. 2. I have a feeling which I can not yet express that
this is directly connected with your relative nits. Now I am yet
unable to express why.
}
What is the result of this pasing of Boolean algebras viewed as a logica;
i.e., as a lattice of propositions?
It is not in the state space of the composite system;
maybe with the choice of the particular particular partition
a sort of ``collapse�� takes place, which is connected to the act of
preparation or measurement.
\section{Single particles as special particles, definition of single particles}
An example EPR: initially it is a single particle, then it becomes a pair of particles.
At all times, $\rho$ is the same, the only difference being the representation as a partition.
(1 element -> 2 elements.)
We just define the operationally obtainable ``finest�� elements of all partitions as ``particles.��
From a technical point of view, this mathematical problem is similar to
the reconstruction of a finite automaton by its input/output behavior, and to the
construction of the respective irreducible???? automaton \cite{conway}.
Define the smallest (finest) combined elements of partitions
to be called "particles" with respect to the given set of states.
\section{Reproducing the statistic of separable states by GUMs etc.}
\section{Summary}
Novelty: whereas in previous papers (except the paper by Zanardi),
the notion of ``particle'' is treated as something given,
here the particle entities are reconstructed by the observed properties
corresponding to the available set of states.
We also provide a constructive method of how to obtain these particles.
If the set of available states changes,
then the very definition of particle may change.
Example: different energy ``resolutions'' by result in different particles.
Refer to Anderson \cite{anderson:73}.
\bibliography{svozil}
%\bibliographystyle{apsrev}
\bibliographystyle{unsrt}
\end{document}
Now let us weaken the condition for states of a composite system to
be product and separable. Namely, instead of requiring \eqref{e03}
for the tensor product
\[
H
=
H_1
\otimes\cdots\otimes
H_N
\]
\noindent we make this condition \underline{relative} with respect
to a partition $P$ of the set of subsystems of the `big' system,
that is, with respect to the decomposition \eqref{e02}.
\medskip
Given a partition $P=\{P_1,\ldots,\P_M\}$ of the set of all
particles which connstitute the composite system and a a density
matrix $\rho$ in the state space of the system, $\rho$ is called
{\em $P$-product} whenever it can be represented as a tensor
product
$$
\rho = \rho_{P_1} \otimes\cdots\otimes \rho_{P_M}
$$
and {\em $P$-separable} if it is a convex
combination of $P$-product states:
\begin{equation}\label{e04}
\rh0 = \sum_{\alpha=1}^A p_\alpha \rrh_{P_1}^\alpha
\otimes\ldots\otimes
\rho_{P_N}^\alpha
\end{equation}
In other words, \eqref{e04} means that we can prepare
$\rho$ as an ensemble of mixed states located at sites
$P_1,\ldots,\P_M$.
\medskip
Given a state $\rho$, we  may now ask for each partition $P$ of
the set of subsystems if $\rho$ is
$P$-separable or not. As a result we  obtain the set
$\Pi(\rho)$ of partitions with respect to which
$\rho$ is separable \eqref{e04}:
\begin{equation}\label{e05}
P\in\Pi(\rho)
\quad\Leftrightarrow\quad
\rho
\quad
\mbox{is $P$-separable}
\end{equation}
\noindent For a given state $\rho$ the set $\Pi(\rho)$ has the
following property \cite{durcirac}:
\begin{equation}\label{e05n}
\forall \:\rho
\quad
\left\lbrace
\begin{array}{l}
P\in\Pi(\rrh)
\cr
P'\preceq\P
\end{array}
\right.
\quad\Leftrightarrow\quad
P'\in\Pi(\rrh)
\end{equation}
\paragraph{Corollary.} To specify $\pttr(\rrh)$ we need to provide
only maximal (with respect to the order \eqref{e01}, that is,
finest) partitions $\pttt$ such that $\rrh$ is $\pttt$-separable.
