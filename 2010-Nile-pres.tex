%\documentclass[pra,showpacs,showkeys,amsfonts,amsmath,twocolumn,handou]{revtex4}
\documentclass[amsmath,red,table,sans,handout]{beamer}
%\documentclass[pra,showpacs,showkeys,amsfonts]{revtex4}
%\documentclass[pra,showpacs,showkeys,amsfonts]{revtex4}
\usepackage[T1]{fontenc}
%%\usepackage{beamerthemeshadow}
\usepackage[headheight=1pt,footheight=10pt]{beamerthemeboxes}
\addfootboxtemplate{\color{structure!80}}{\color{white}\tiny \hfill Karl Svozil (TU Vienna)\hfill}
\addfootboxtemplate{\color{structure!65}}{\color{white}\tiny \hfill Chocolate cryptography\hfill}
\addfootboxtemplate{\color{structure!50}}{\color{white}\tiny \hfill PC10, Sept 3, 2010, Aswan, Nile, Egypt\hfill}
%\usepackage[dark]{beamerthemesidebar}
%\usepackage[headheight=24pt,footheight=12pt]{beamerthemesplit}
%\usepackage{beamerthemesplit}
%\usepackage[bar]{beamerthemetree}
\usepackage{graphicx}
\usepackage{pgf}
%\usepackage{eepic}
%\usepackage[x11names]{xcolor}
%\newcommand{\Red}{\color{Red}}  %(VERY-Approx.PANTONE-RED)
%\newcommand{\Green}{\color{Green}}  %(VERY-Approx.PANTONE-GREEN)



%\RequirePackage[german]{babel}
%\selectlanguage{german}
%\RequirePackage[isolatin]{inputenc}

\pgfdeclareimage[height=0.5cm]{logo}{tu-logo}
\logo{\pgfuseimage{logo}}
\beamertemplatetriangleitem
%\beamertemplateballitem

\beamerboxesdeclarecolorscheme{alert}{red}{red!15!averagebackgroundcolor}
\beamerboxesdeclarecolorscheme{alert2}{orange}{orange!15!averagebackgroundcolor}
%\begin{beamerboxesrounded}[scheme=alert,shadow=true]{}
%\end{beamerboxesrounded}

%\beamersetaveragebackground{green!10}

%\beamertemplatecircleminiframe

%\usepackage{feynmf}             %Package for feynman diagrams.

\begin{document}


\title{\bf \textcolor{red}{Chocolate cryptography}}
%\subtitle{Naturwissenschaftlich-Humanisticher Tag am BG 19\\Weltbild und Wissenschaft\\http://tph.tuwien.ac.at/\~{}svozil/publ/2005-BG18-pres.pdf}
\subtitle{\textcolor{orange!60}{\small http://tph.tuwien.ac.at/$\sim$svozil/publ/2010-Nile-pres.pdf}
}
\author{Karl Svozil}
\institute{Institut f\"ur Theoretische Physik, Vienna University of Technology, \\
Wiedner Hauptstra\ss e 8-10/136, A-1040 Vienna, Austria\\
svozil@tuwien.ac.at
%{\tiny Disclaimer: Die hier vertretenen Meinungen des Autors verstehen sich als Diskussionsbeitr�ge und decken sich nicht notwendigerweise mit den Positionen der Technischen Universit�t Wien oder deren Vertreter.}
}
\date{Sept 3, 2010, Aswan, Nile, Egypt}
\maketitle



%\frame{
%\frametitle{Contents}
%\tableofcontents
%}


%\section{}
\frame{
\frametitle{``Exectutive Summary''}
{
\begin{itemize}

\item<1->
Some quantum cryptographic protocols can be implemented with specially prepared chocolate balls; others protected by value indefiniteness cannot.

\item<1->
This latter feature, which follows from Bell- and Kochen-Specker type arguments, is only present in systems with three or more mutually exclusive outcomes.

\item<1->
Conversely, there exist chocolate ball configurations utilizable for cryptography which cannot be realized by quantum systems.

\end{itemize}
}
}

\frame{
\frametitle{Generalized Urn Models~[R. Wright, 1990, DOI:10.1007/BF01889696] }
{
\begin{itemize}

\item<1->
Generalized urn models
utilize
black balls painted
with two or more symbols in two or more colors.

\item<1->
Suppose, for instance,  just two symbols ``$0$'' and ``$1$''
in just two colors, say,  ``pink'' and ``light blue'',
resulting in four types of conceivable balls:
\begin{center}
\unitlength 0.7mm \begin{picture}(8,8) \put(4,2){\circle*{12}} \put(4,2){\makebox(0,0)[cc]{${\color{red!55} {\bf 0}}{\color{blue!55} {\bf 0}}$}} \end{picture},
\unitlength 0.7mm \begin{picture}(8,8) \put(4,2){\circle*{12}} \put(4,2){\makebox(0,0)[cc]{${\color{red!55} {\bf 0}}{\color{blue!55} {\bf 1}}$}} \end{picture},
\unitlength 0.7mm \begin{picture}(8,8) \put(4,2){\circle*{12}} \put(4,2){\makebox(0,0)[cc]{${\color{red!55} {\bf 1}}{\color{blue!55} {\bf 0}}$}} \end{picture},
\unitlength 0.7mm \begin{picture}(8,8) \put(4,2){\circle*{12}} \put(4,2){\makebox(0,0)[cc]{${\color{red!55} {\bf 1}}{\color{blue!55} {\bf 1}}$}} \end{picture}
\end{center}
--- many copies of which are randomly
distributed in an urn.

\item<1->
Suppose the experimenter looks at them with one of two differently colored eyeglasses,
each one ideally  matching the colors of only one of the symbols,
such that only light in this wave length passes through.
Thereby, the choice of the color decides which one of the two
``complementary'' observables associated with ``pink'' and ``light blue'' is measured.
Propositions refer to the possible ball types drawn from the urn, given the information printed in the chosen color.

\end{itemize}
}
}

\frame{
\frametitle{Utensils required for BB84-type chocolate cryptography by generalized urn models}
{
\begin{center}
\includegraphics<1>[height=6.5cm]{2005-ln1e-utensils}
\end{center}
}
}

\frame{
\frametitle{``Mimicking'' the Ekert protocol with a classical ``singlet'' state fails}
{
\begin{center}
\unitlength 0.7mm \begin{picture}(8,8) \put(4,2){\circle*{12}} \put(4,2){\makebox(0,0)[cc]{${\color{red!55} {\bf 0}}{\color{blue!55} {\bf 0}}$}} \end{picture}---\unitlength 0.7mm \begin{picture}(8,8) \put(4,2){\circle*{12}} \put(4,2){\makebox(0,0)[cc]{${\color{red!55} {\bf 1}}{\color{blue!55} {\bf 1}}$}} \end{picture}
\\
\unitlength 0.7mm \begin{picture}(8,8) \put(4,2){\circle*{12}} \put(4,2){\makebox(0,0)[cc]{${\color{red!55} {\bf 0}}{\color{blue!55} {\bf 1}}$}} \end{picture}---\unitlength 0.7mm \begin{picture}(8,8) \put(4,2){\circle*{12}} \put(4,2){\makebox(0,0)[cc]{${\color{red!55} {\bf 1}}{\color{blue!55} {\bf 0}}$}} \end{picture}
\\
\unitlength 0.7mm \begin{picture}(8,8) \put(4,2){\circle*{12}} \put(4,2){\makebox(0,0)[cc]{${\color{red!55} {\bf 1}}{\color{blue!55} {\bf 0}}$}} \end{picture}---\unitlength 0.7mm \begin{picture}(8,8) \put(4,2){\circle*{12}} \put(4,2){\makebox(0,0)[cc]{${\color{red!55} {\bf 0}}{\color{blue!55} {\bf 1}}$}} \end{picture}
\\
\unitlength 0.7mm \begin{picture}(8,8) \put(4,2){\circle*{12}} \put(4,2){\makebox(0,0)[cc]{${\color{red!55} {\bf 1}}{\color{blue!55} {\bf 1}}$}} \end{picture}---\unitlength 0.7mm \begin{picture}(8,8) \put(4,2){\circle*{12}} \put(4,2){\makebox(0,0)[cc]{${\color{red!55} {\bf 0}}{\color{blue!55} {\bf 0}}$}} \end{picture}
\\
$\;$\\
\end{center}
Fails miserably! Why?
Because Bell-type configurations utilize quantum value indefinitenes whereas generalized urns are classical and value definite.
}
}


\frame{
\frametitle{Nonprotection by value indefiniteness}
{
\begin{itemize}
\item<1->
Chocolates are never protected by value indefiniteness; they are just protected by complementarity.
\item<1->
More quantitatively: use all two-valued states on the logic to construct ball types.
\item<1->
Ekert protocol is able to differentiate between quantum and chocolate cryptography, BB84 is not.
\end{itemize}
}
}

\frame{
\frametitle{Greechie diagrams of realizable chocolate ball configurations for chocolate-crypto}
{
\begin{center}
\begin{tabular}{cccccc}
%TeXCAD Picture [1.pic]. Options:
%\grade{\on}
%\emlines{\off}
%\epic{\off}
%\beziermacro{\on}
%\reduce{\on}
%\snapping{\off}
%\pvinsert{% Your \input, \def, etc. here}
%\quality{8.000}
%\graddiff{0.005}
%\snapasp{1}
%\zoom{4.0000}
\unitlength 0.3mm % = 2.85pt
%\linethickness{0.8pt}
\ifx\plotpoint\undefined\newsavebox{\plotpoint}\fi % GNUPLOT compatibility
\begin{picture}(132.5,122)(0,0)
\put(20,20){\color{blue!55}\line(1,0){110}}
%\emline(20,20)(75,110)
\multiput(20,20)(.03372164316,.05518087063){1631}{\color{yellow}\line(0,1){.05518087063}}
%\end
%\emline(75,110)(130,20)
\multiput(75,110)(.03372164316,-.05518087063){1631}{\color{red!55}\line(0,-1){.05518087063}}
%\end
\put(20,20){\color{blue!55}\circle*{9}}
\put(20,20){\color{yellow}\circle*{5.5}}
\put(20,20){\color{yellow}\circle*{1.5}}
\put(56.25,20){\color{blue!55}\circle*{5.5}}
\put(56.25,20){\color{blue!55}\circle*{1.5}}
\put(92.5,20){\color{blue!55}\circle*{5.5}}
\put(92.5,20){\color{blue!55}\circle*{1.5}}
\put(129.75,20){\color{red!55}\circle*{9}}
\put(129.75,20){\color{blue!55}\circle*{5.5}}
\put(129.75,20){\color{blue!55}\circle*{1.5}}
\put(56.25,79.75){\color{yellow}\circle*{5.5}}
\put(56.25,79.75){\color{yellow}\circle*{1.5}}
\put(38.75,51.25){\color{yellow}\circle*{5.5}}
\put(38.75,51.25){\color{yellow}\circle*{1.5}}
\put(74.75,109.75){\color{yellow}\circle*{9}}
\put(74.75,109.75){\color{red!55}\circle*{5.5}}
\put(74.75,109.75){\color{red!55}\circle*{1.5}}
\put(93.75,79.75){\color{red!55}\circle*{5.5}}
\put(93.75,79.75){\color{red!55}\circle*{1.5}}
\put(111.25,51.25){\color{red!55}\circle*{5.5}}
\put(111.25,51.25){\color{red!55}\circle*{1.5}}
\end{picture}
&
\qquad
&
%TeXCAD Picture [1.pic]. Options:
%\grade{\on}
%\emlines{\off}
%\epic{\off}
%\beziermacro{\on}
%\reduce{\on}
%\snapping{\off}
%\pvinsert{% Your \input, \def, etc. here}
%\quality{8.000}
%\graddiff{0.005}
%\snapasp{1}
%\zoom{4.0000}
\unitlength 0.3mm % = 2.85pt
%\linethickness{0.8pt}
\ifx\plotpoint\undefined\newsavebox{\plotpoint}\fi % GNUPLOT compatibility
\begin{picture}(132.5,122)(0,0)
\put(20,20){\color{blue!55}\line(1,0){73}}
%\emline(20,20)(75,110)
%\multiput(20,20)(.03372164316,.05518087063){1631}{\color{yellow}\line(0,1){.05518087063}}
\multiput(20,20)(.03372164316,.05518087063){1080}{\color{red!55}\line(0,1){.05518087063}}
%\end
\put(20,20){\color{blue!55}\circle*{9}}
\put(20,20){\color{red!55}\circle*{5.5}}
\put(20,20){\color{red!55}\circle*{1.5}}
\put(56.25,20){\color{blue!55}\circle*{5.5}}
\put(56.25,20){\color{blue!55}\circle*{1.5}}
\put(92.5,20){\color{blue!55}\circle*{5.5}}
\put(92.5,20){\color{blue!55}\circle*{1.5}}
\put(56.25,79.75){\color{red!55}\circle*{5.5}}
\put(56.25,79.75){\color{red!55}\circle*{1.5}}
\put(38.75,51.25){\color{red!55}\circle*{5.5}}
\put(38.75,51.25){\color{red!55}\circle*{1.5}}
\end{picture}
\end{tabular}
\end{center}
The 14 ball types corresponding to the 14 different two-valued measures are as follows:
\unitlength 0.7mm \begin{picture}(12,12)\put(6,2){\circle*{12}} \put(6,2){\makebox(0,0)[cc]{${\color{red!55} {\bf 0}}{\color{blue!55} {\bf 0}}{\color{yellow} {\bf 0}}$}}\end{picture},
\unitlength 0.7mm \begin{picture}(12,12)\put(6,2){\circle*{12}} \put(6,2){\makebox(0,0)[cc]{${\color{red!55} {\bf 0}}{\color{blue!55} {\bf 1}}{\color{yellow} {\bf 0}}$}}\end{picture},
\unitlength 0.7mm \begin{picture}(12,12)\put(6,2){\circle*{12}} \put(6,2){\makebox(0,0)[cc]{${\color{red!55} {\bf 1}}{\color{blue!55} {\bf 2}}{\color{yellow} {\bf 1}}$}}\end{picture},
\unitlength 0.7mm \begin{picture}(12,12)\put(6,2){\circle*{12}} \put(6,2){\makebox(0,0)[cc]{${\color{red!55} {\bf 1}}{\color{blue!55} {\bf 0}}{\color{yellow} {\bf 2}}$}}\end{picture},
\unitlength 0.7mm \begin{picture}(12,12)\put(6,2){\circle*{12}} \put(6,2){\makebox(0,0)[cc]{${\color{red!55} {\bf 1}}{\color{blue!55} {\bf 0}}{\color{yellow} {\bf 3}}$}}\end{picture},
\unitlength 0.7mm \begin{picture}(12,12)\put(6,2){\circle*{12}} \put(6,2){\makebox(0,0)[cc]{${\color{red!55} {\bf 1}}{\color{blue!55} {\bf 1}}{\color{yellow} {\bf 2}}$}}\end{picture},
\unitlength 0.7mm \begin{picture}(12,12)\put(6,2){\circle*{12}} \put(6,2){\makebox(0,0)[cc]{${\color{red!55} {\bf 1}}{\color{blue!55} {\bf 1}}{\color{yellow} {\bf 3}}$}}\end{picture},
\unitlength 0.7mm \begin{picture}(12,12)\put(6,2){\circle*{12}} \put(6,2){\makebox(0,0)[cc]{${\color{red!55} {\bf 2}}{\color{blue!55} {\bf 2}}{\color{yellow} {\bf 1}}$}}\end{picture},
\unitlength 0.7mm \begin{picture}(12,12)\put(6,2){\circle*{12}} \put(6,2){\makebox(0,0)[cc]{${\color{red!55} {\bf 2}}{\color{blue!55} {\bf 0}}{\color{yellow} {\bf 2}}$}}\end{picture},
\unitlength 0.7mm \begin{picture}(12,12)\put(6,2){\circle*{12}} \put(6,2){\makebox(0,0)[cc]{${\color{red!55} {\bf 2}}{\color{blue!55} {\bf 0}}{\color{yellow} {\bf 3}}$}}\end{picture},
\unitlength 0.7mm \begin{picture}(12,12)\put(6,2){\circle*{12}} \put(6,2){\makebox(0,0)[cc]{${\color{red!55} {\bf 2}}{\color{blue!55} {\bf 1}}{\color{yellow} {\bf 2}}$}}\end{picture},
\unitlength 0.7mm \begin{picture}(12,12)\put(6,2){\circle*{12}} \put(6,2){\makebox(0,0)[cc]{${\color{red!55} {\bf 2}}{\color{blue!55} {\bf 1}}{\color{yellow} {\bf 3}}$}}\end{picture},
\unitlength 0.7mm \begin{picture}(12,12)\put(6,2){\circle*{12}} \put(6,2){\makebox(0,0)[cc]{${\color{red!55} {\bf 3}}{\color{blue!55} {\bf 3}}{\color{yellow} {\bf 2}}$}}\end{picture}, and
\unitlength 0.7mm \begin{picture}(12,12)\put(6,2){\circle*{12}} \put(6,2){\makebox(0,0)[cc]{${\color{red!55} {\bf 3}}{\color{blue!55} {\bf 3}}{\color{yellow} {\bf 3}}$}}\end{picture}.
}
}

\frame{
\frametitle{Weirder-than- quantum noncommutative chocolate cryptography which cannot be realized quantum mechanically}
{
Propositional structure allowing (value definite)
chocolate ball realizations with three atoms per context or block which does not allow a quantum analogue.

\begin{center}
%TeXCAD Picture [1.pic]. Options:
%\grade{\on}
%\emlines{\off}
%\epic{\off}
%\beziermacro{\on}
%\reduce{\on}
%\snapping{\off}
%\pvinsert{% Your \input, \def, etc. here}
%\quality{8.000}
%\graddiff{0.005}
%\snapasp{1}
%\zoom{4.0000}
\unitlength .15mm % = 2.85pt
%\linethickness{0.8pt}
\ifx\plotpoint\undefined\newsavebox{\plotpoint}\fi % GNUPLOT compatibility
\begin{picture}(132.5,122)(0,0)
\put(20,20){\color{blue!55}\line(1,0){110}}
%\emline(20,20)(75,110)
\multiput(20,20)(.03372164316,.05518087063){1631}{\color{yellow}\line(0,1){.05518087063}}
%\end
%\emline(75,110)(130,20)
\multiput(75,110)(.03372164316,-.05518087063){1631}{\color{red!55}\line(0,-1){.05518087063}}
%\end
\put(20,20){\color{blue!55}\circle*{9}}
\put(20,20){\color{yellow}\circle*{5.5}}
\put(20,20){\color{yellow}\circle*{1.5}}
\put(74.375,20){\color{blue!55}\circle*{5.5}}
\put(74.375,20){\color{blue!55}\circle*{1.5}}
\put(129.75,20){\color{red!55}\circle*{9}}
\put(129.75,20){\color{blue!55}\circle*{5.5}}
\put(129.75,20){\color{blue!55}\circle*{1.5}}
\put(47.5,65.5){\color{yellow}\circle*{5.5}}
\put(47.5,65.5){\color{yellow}\circle*{1.5}}
\put(74.75,109.75){\color{yellow}\circle*{9}}
\put(74.75,109.75){\color{red!55}\circle*{5.5}}
\put(74.75,109.75){\color{red!55}\circle*{1.5}}
\put(102.5,65.5){\color{red!55}\circle*{5.5}}
\put(102.5,65.5){\color{red!55}\circle*{1.5}}
\put(15,5){\makebox(0,0)[cc]{\tiny $\{1\}$}}
\put(74.375,5){\makebox(0,0)[cc]{\tiny $\{3,4\}$}}
\put(138,5){\makebox(0,0)[cc]{\tiny $\{2\}$}}
\put(74.75,122){\makebox(0,0)[cc]{\tiny $\{3\}$}}
\put(112,65.8){\makebox(0,0)[lc]{\tiny $\{1,4\}$}}
\put(34,65.8){\makebox(0,0)[rc]{\tiny $\{2,4\}$}}
\end{picture}
\end{center}

The associated partition logic is given by
$$
\begin{array}{c}
\{
\{
\{1
\},
\{2
\},
\{  3,4
\}
\}, \\
\{
\{1,4
\},
\{ 2
\},
\{ 3
\}
\}, \\
\{
\{ 1
\},
\{ 2,4
\},
\{ 3
\}
\}
\}.
\end{array}
$$


}
}

\frame{
\frametitle{ }
{
Four two-valued states:
\begin{center}
\begin{tabular}{cccccccccc}
%TeXCAD Picture [1.pic]. Options:
%\grade{\on}
%\emlines{\off}
%\epic{\off}
%\beziermacro{\on}
%\reduce{\on}
%\snapping{\off}
%\pvinsert{% Your \input, \def, etc. here}
%\quality{8.000}
%\graddiff{0.005}
%\snapasp{1}
%\zoom{4.0000}
\unitlength .15mm % = .569pt
%\linethickness{0.8pt}
\ifx\plotpoint\undefined\newsavebox{\plotpoint}\fi % GNUPLOT compatibility
\begin{picture}(130,110)(0,0)
\put(20,20){\line(1,0){110}}
%\emline(20,20)(75,110)
\multiput(20,20)(.1198257081,.1960784314){459}{\line(0,1){.1960784314}}
%\end
%\emline(75,110)(130,20)
\multiput(75,110)(.1198257081,-.1960784314){459}{\line(0,-1){.1960784314}}
%\end
\put(20,20){\circle*{8}}
\put(15,5){\makebox(0,0)[cc]{$1$}}
%1
\put(102,66.25){\circle*{8}}
\put(111,73){\makebox(0,0)[lc]{$4$}}
%5
\put(74,52){\makebox(0,0)[cc]{\Large \bf 1}}
\put(74,52){\circle{40}}
\end{picture}
&
%TeXCAD Picture [1.pic]. Options:
%\grade{\on}
%\emlines{\off}
%\epic{\off}
%\beziermacro{\on}
%\reduce{\on}
%\snapping{\off}
%\pvinsert{% Your \input, \def, etc. here}
%\quality{8.000}
%\graddiff{0.005}
%\snapasp{1}
%\zoom{4.0000}
\unitlength .15mm % = .569pt
%\linethickness{0.8pt}
\ifx\plotpoint\undefined\newsavebox{\plotpoint}\fi % GNUPLOT compatibility
\begin{picture}(138.75,110)(0,0)
\put(20,20){\line(1,0){110}}
%\emline(20,20)(75,110)
\multiput(20,20)(.1198257081,.1960784314){459}{\line(0,1){.1960784314}}
%\end
%\emline(75,110)(130,20)
\multiput(75,110)(.1198257081,-.1960784314){459}{\line(0,-1){.1960784314}}
%\end
\put(130,20){\circle*{8}}
\put(138.75,5){\makebox(0,0)[cc]{$3$}}
%1
\put(48.25,66.25){\circle*{8}}
\put(30,73){\makebox(0,0)[lc]{$6$}}
%5
\put(74,52){\makebox(0,0)[cc]{\Large \bf 2}}
\put(74,52){\circle{40}}
\end{picture}
&
%TeXCAD Picture [1.pic]. Options:
%\grade{\on}
%\emlines{\off}
%\epic{\off}
%\beziermacro{\on}
%\reduce{\on}
%\snapping{\off}
%\pvinsert{% Your \input, \def, etc. here}
%\quality{8.000}
%\graddiff{0.005}
%\snapasp{1}
%\zoom{4.0000}
\unitlength .15mm % = .569pt
%\linethickness{0.8pt}
\ifx\plotpoint\undefined\newsavebox{\plotpoint}\fi % GNUPLOT compatibility
\begin{picture}(130,120.25)(0,0)
\put(20,20){\line(1,0){110}}
%\emline(20,20)(75,110)
\multiput(20,20)(.1198257081,.1960784314){459}{\line(0,1){.1960784314}}
%\end
%\emline(75,110)(130,20)
\multiput(75,110)(.1198257081,-.1960784314){459}{\line(0,-1){.1960784314}}
%\end
\put(74.25,20){\circle*{8}}
\put(74.25,6){\makebox(0,0)[cc]{$2$}}
%1
\put(75.25,109.75){\circle*{8}}
\put(75.75,123.25){\makebox(0,0)[cc]{$5$}}
%5
\put(74,52){\makebox(0,0)[cc]{\Large \bf 3}}
\put(74,52){\circle{40}}
\end{picture}
&
%TeXCAD Picture [1.pic]. Options:
%\grade{\on}
%\emlines{\off}
%\epic{\off}
%\beziermacro{\on}
%\reduce{\on}
%\snapping{\off}
%\pvinsert{% Your \input, \def, etc. here}
%\quality{8.000}
%\graddiff{0.005}
%\snapasp{1}
%\zoom{4.0000}
\unitlength .15mm % = .569pt
%\linethickness{0.8pt}
\ifx\plotpoint\undefined\newsavebox{\plotpoint}\fi % GNUPLOT compatibility
\begin{picture}(130,110)(0,0)
\put(20,20){\line(1,0){110}}
%\emline(20,20)(75,110)
\multiput(20,20)(.1198257081,.1960784314){459}{\line(0,1){.1960784314}}
%\end
%\emline(75,110)(130,20)
\multiput(75,110)(.1198257081,-.1960784314){459}{\line(0,-1){.1960784314}}
%\end
\put(74.25,20){\circle*{8}}
\put(73,6){\makebox(0,0)[cc]{$2$}}
%1
\put(47.75,66){\circle*{8}}
\put(101.25,66){\circle*{8}}
\put(30,73){\makebox(0,0)[lc]{$6$}}
\put(110.25,73.5){\makebox(0,0)[lc]{$4$}}
%5
\put(74,52){\makebox(0,0)[cc]{\Large \bf 4}}
\put(74,52){\circle{40}}
\end{picture}
\end{tabular}
\end{center}

$\ldots$~corresponding to four ball types:

\begin{center}
\unitlength 0.7mm \begin{picture}(12,12)\put(6,2){\circle*{12}} \put(6,2){\makebox(0,0)[cc]{${\color{red!55} {\bf 0}}{\color{blue!55} {\bf 1}}{\color{yellow} {\bf 2}}$}}\end{picture},
\unitlength 0.7mm \begin{picture}(12,12)\put(6,2){\circle*{12}} \put(6,2){\makebox(0,0)[cc]{${\color{red!55} {\bf 2}}{\color{blue!55} {\bf 0}}{\color{yellow} {\bf 1}}$}}\end{picture},
\unitlength 0.7mm \begin{picture}(12,12)\put(6,2){\circle*{12}} \put(6,2){\makebox(0,0)[cc]{${\color{red!55} {\bf 1}}{\color{blue!55} {\bf 2}}{\color{yellow} {\bf 0}}$}}\end{picture}, and
\unitlength 0.7mm \begin{picture}(12,12)\put(6,2){\circle*{12}} \put(6,2){\makebox(0,0)[cc]{${\color{red!55} {\bf 1}}{\color{blue!55} {\bf 1}}{\color{yellow} {\bf 1}}$}}\end{picture}.
\end{center}

}
}


\section{Summary}
\frame{
\frametitle{Summary}
{
\begin{itemize}

\item<1->
Some quantum cryptographic protocols can be implemented with specially prepared chocolate balls; others protected by value indefiniteness cannot.

\item<1->
This latter feature, which follows from Bell- and Kochen-Specker type arguments, is only present in systems with three or more mutually exclusive outcomes.

\item<1->
Conversely, there exist chocolate ball configurations utilizable for cryptography which cannot be realized by quantum systems.

\end{itemize}
}
}

\frame{

\centerline{\Large Thank you for your attention!}

\begin{center}
$\widetilde{\qquad \qquad }$
$\widetilde{\qquad \qquad}$
$\widetilde{\qquad \qquad }$
\end{center}
 }

\end{document}
