\def\revtex{1}

\ifx\revtex\undefined

%  LaTeX support: latex@mdpi.com
%  For support, please attach all files needed for compiling as well as the log file, and specify your operating system, LaTeX version, and LaTeX editor.

%=================================================================
\documentclass[entropy,article,submit,oneauthor,pdftex]{Definitions/mdpi}

% For posting an early version of this manuscript as a preprint, you may use "preprints" as the journal and change "submit" to "accept". The document class line would be, e.g., \documentclass[preprints,article,accept,moreauthors,pdftex]{mdpi}. This is especially recommended for submission to arXiv, where line numbers should be removed before posting. For preprints.org, the editorial staff will make this change immediately prior to posting.

%--------------------
% Class Options:
%--------------------
%----------
% journal
%----------
% Choose between the following MDPI journals:
% acoustics, actuators, addictions, admsci, adolescents, aerospace, agriculture, agriengineering, agronomy, ai, algorithms, allergies, analytica, animals, antibiotics, antibodies, antioxidants, appliedchem, applmech, applmicrobiol, applnano, applsci, arts, asi, atmosphere, atoms, audiolres, automation, axioms, batteries, bdcc, behavsci, beverages, biochem, bioengineering, biologics, biology, biomechanics, biomedicines, biomedinformatics, biomimetics, biomolecules, biophysica, biosensors, biotech, birds, bloods, brainsci, buildings, businesses, cancers, carbon, cardiogenetics, catalysts, cells, ceramics, challenges, chemengineering, chemistry, chemosensors, chemproc, children, civileng, cleantechnol, climate, clinpract, clockssleep, cmd, coatings, colloids, compounds, computation, computers, condensedmatter, conservation, constrmater, cosmetics, crops, cryptography, crystals, curroncol, cyber, dairy, data, dentistry, dermato, dermatopathology, designs, diabetology, diagnostics, digital, disabilities, diseases, diversity, dna, drones, dynamics, earth, ebj, ecologies, econometrics, economies, education, ejihpe, electricity, electrochem, electronicmat, electronics, encyclopedia, endocrines, energies, eng, engproc, entropy, environments, environsciproc, epidemiologia, epigenomes, fermentation, fibers, fire, fishes, fluids, foods, forecasting, forensicsci, forests, fractalfract, fuels, futureinternet, futuretransp, futurepharmacol, futurephys, galaxies, games, gases, gastroent, gastrointestdisord, gels, genealogy, genes, geographies, geohazards, geomatics, geosciences, geotechnics, geriatrics, hazardousmatters, healthcare, hearts, hemato, heritage, highthroughput, histories, horticulturae, humanities, hydrogen, hydrology, hygiene, idr, ijerph, ijfs, ijgi, ijms, ijns, ijtm, ijtpp, immuno, informatics, information, infrastructures, inorganics, insects, instruments, inventions, iot, j, jcdd, jcm, jcp, jcs, jdb, jfb, jfmk, jimaging, jintelligence, jlpea, jmmp, jmp, jmse, jne, jnt, jof, joitmc, jor, journalmedia, jox, jpm, jrfm, jsan, jtaer, jzbg, kidney, land, languages, laws, life, liquids, literature, livers, logistics, lubricants, machines, macromol, magnetism, magnetochemistry, make, marinedrugs, materials, materproc, mathematics, mca, measurements, medicina, medicines, medsci, membranes, metabolites, metals, metrology, micro, microarrays, microbiolres, micromachines, microorganisms, minerals, mining, modelling, molbank, molecules, mps, mti, nanoenergyadv, nanomanufacturing, nanomaterials, ncrna, network, neuroglia, neurolint, neurosci, nitrogen, notspecified, nri, nursrep, nutrients, obesities, oceans, ohbm, onco, oncopathology, optics, oral, organics, osteology, oxygen, parasites, parasitologia, particles, pathogens, pathophysiology, pediatrrep, pharmaceuticals, pharmaceutics, pharmacy, philosophies, photochem, photonics, physchem, physics, physiolsci, plants, plasma, pollutants, polymers, polysaccharides, proceedings, processes, prosthesis, proteomes, psych, psychiatryint, publications, quantumrep, quaternary, qubs, radiation, reactions, recycling, regeneration, religions, remotesensing, reports, reprodmed, resources, risks, robotics, safety, sci, scipharm, sensors, separations, sexes, signals, sinusitis, smartcities, sna, societies, socsci, soilsystems, solids, sports, standards, stats, stresses, surfaces, surgeries, suschem, sustainability, symmetry, systems, taxonomy, technologies, telecom, textiles, thermo, tourismhosp, toxics, toxins, transplantology, traumas, tropicalmed, universe, urbansci, uro, vaccines, vehicles, vetsci, vibration, viruses, vision, water, wevj, women, world

%---------
% article
%---------
% The default type of manuscript is "article", but can be replaced by:
% abstract, addendum, article, book, bookreview, briefreport, casereport, comment, commentary, communication, conferenceproceedings, correction, conferencereport, entry, expressionofconcern, extendedabstract, datadescriptor, editorial, essay, erratum, hypothesis, interestingimage, obituary, opinion, projectreport, reply, retraction, review, perspective, protocol, shortnote, studyprotocol, systematicreview, supfile, technicalnote, viewpoint, guidelines, registeredreport, tutorial
% supfile = supplementary materials

%----------
% submit
%----------
% The class option "submit" will be changed to "accept" by the Editorial Office when the paper is accepted. This will only make changes to the frontpage (e.g., the logo of the journal will get visible), the headings, and the copyright information. Also, line numbering will be removed. Journal info and pagination for accepted papers will also be assigned by the Editorial Office.

%------------------
% moreauthors
%------------------
% If there is only one author the class option oneauthor should be used. Otherwise use the class option moreauthors.

%---------
% pdftex
%---------
% The option pdftex is for use with pdfLaTeX. If eps figures are used, remove the option pdftex and use LaTeX and dvi2pdf.

%=================================================================
% MDPI internal commands
\firstpage{1}
\makeatletter
\setcounter{page}{\@firstpage}
\makeatother
\pubvolume{1}
\issuenum{1}
\articlenumber{0}
\pubyear{2021}
\copyrightyear{2021}
%\externaleditor{Academic Editor: Firstname Lastname} % For journal Automation, please change Academic Editor to "Communicated by"
\datereceived{}
\dateaccepted{}
\datepublished{}
\hreflink{https://doi.org/} % If needed use \linebreak
%------------------------------------------------------------------
% The following line should be uncommented if the LaTeX file is uploaded to arXiv.org
%\pdfoutput=1

%%%% If original paper need add "Retraction", please release the following command!!%%%%%%
%\retractiondate{Date Month Year} % For example,  13 October 2020
%\retractionnoticeyear{Year}
%\retractionnoticevolume{0}
%\retractionnoticeidnumber{0000}
%\retractionnoticedoi{10.3390/xxx}

%=================================================================
% Add packages and commands here. The following packages are loaded in our class file: fontenc, inputenc, calc, indentfirst, fancyhdr, graphicx, epstopdf, lastpage, ifthen, lineno, float, amsmath, setspace, enumitem, mathpazo, booktabs, titlesec, etoolbox, tabto, xcolor, soul, multirow, microtype, tikz, totcount, changepage, paracol, attrib, upgreek, cleveref, amsthm, hyphenat, natbib, hyperref, footmisc, url, geometry, newfloat, caption

%=================================================================
%% Please use the following mathematics environments: Theorem, Lemma, Corollary, Proposition, Characterization, Property, Problem, Example, ExamplesandDefinitions, Hypothesis, Remark, Definition, Notation, Assumption
%% For proofs, please use the proof environment (the amsthm package is loaded by the MDPI class).

%=================================================================
% Full title of the paper (Capitalized)
\Title{Varieties of contextuality}

% MDPI internal command: Title for citation in the left column
\TitleCitation{Varieties of contextuality emphasizing (non)embeddability}

% Author Orchid ID: enter ID or remove command
\newcommand{\orcidauthorA}{0000-0001-6554-2802} % Add \orcidA{} behind the author's name
%\newcommand{\orcidauthorB}{0000-0000-0000-000X} % Add \orcidB{} behind the author's name

% Authors, for the paper (add full first names)
\Author{Karl Svozil $^{1}$\orcidA{}}

% MDPI internal command: Authors, for metadata in PDF
\AuthorNames{Karl Svozil}

% MDPI internal command: Authors, for citation in the left column
\AuthorCitation{Svozil, K.}
% If this is a Chicago style journal: Lastname, Firstname, Firstname Lastname, and Firstname Lastname.

% Affiliations / Addresses (Add [1] after \address if there is only one affiliation.)
\address[1]{%
$^{1}$ \quad Institute for Theoretical Physics, TU Wien, Wiedner Hauptstrasse 8-10/136, 1040 Vienna,  Austria; svozil@tuwien.ac.at; \url{http://tph.tuwien.ac.at/~svozil}}

% Contact information of the corresponding author
\corres{Correspondence: svozil@tuwien.ac.at}

% Current address and/or shared authorship
%\firstnote{Current address: Affiliation 3}
%\secondnote{These authors contributed equally to this work.}
% The commands \thirdnote{} till \eighthnote{} are available for further notes

%\simplesumm{} % Simple summary

%\conference{} % An extended version of a conference paper

% Abstract (Do not insert blank lines, i.e. \\)
\abstract{}

% Keywords
\keyword{quantum randomness; Gleason theorem; Kochen-Specker theorem; Born rule; object construction; emergent space-time; quantum entanglement}

% The fields PACS, MSC, and JEL may be left empty or commented out if not applicable
\PACS{03.65.Ca, 02.50.-r, 02.10.-v, 03.65.Aa, 03.67.Ac, 03.65.Ud}
%\MSC{}
%\JEL{}

%%%%%%%%%%%%%%%%%%%%%%%%%%%%%%%%%%%%%%%%%%
% Only for the journal Diversity
%\LSID{\url{http://}}

%%%%%%%%%%%%%%%%%%%%%%%%%%%%%%%%%%%%%%%%%%
% Only for the journal Applied Sciences:
%\featuredapplication{Authors are encouraged to provide a concise description of the specific application or a potential application of the work. This section is not mandatory.}
%%%%%%%%%%%%%%%%%%%%%%%%%%%%%%%%%%%%%%%%%%

%%%%%%%%%%%%%%%%%%%%%%%%%%%%%%%%%%%%%%%%%%
% Only for the journal Data:
%\dataset{DOI number or link to the deposited data set in cases where the data set is published or set to be published separately. If the data set is submitted and will be published as a supplement to this paper in the journal Data, this field will be filled by the editors of the journal. In this case, please make sure to submit the data set as a supplement when entering your manuscript into our manuscript editorial system.}

%\datasetlicense{license under which the data set is made available (CC0, CC-BY, CC-BY-SA, CC-BY-NC, etc.)}

%%%%%%%%%%%%%%%%%%%%%%%%%%%%%%%%%%%%%%%%%%
% Only for the journal Toxins
%\keycontribution{The breakthroughs or highlights of the manuscript. Authors can write one or two sentences to describe the most important part of the paper.}

%%%%%%%%%%%%%%%%%%%%%%%%%%%%%%%%%%%%%%%%%%
% Only for the journal Encyclopedia
%\encyclopediadef{Instead of the abstract}
%\entrylink{The Link to this entry published on the encyclopedia platform.}
%%%%%%%%%%%%%%%%%%%%%%%%%%%%%%%%%%%%%%%%%%

\begin{document}
%%%%%%%%%%%%%%%%%%%%%%%%%%%%%%%%%%%%%%%%%%
%\setcounter{section}{-1} %% Remove this when starting to work on the template.


\else
\documentclass[%
 %reprint,
 %twocolumn,
 %superscriptaddress,
 %groupedaddress,
 %unsortedaddress,
 %runinaddress,
 %frontmatterverbose,
  preprint,
 %showpacs,
 %showkeys,
 %preprintnumbers,
  nofootinbib,
 %nobibnotes,
 %bibnotes,
 amsmath,amssymb,
 aps,
 % prl,
 pra,
 % prb,
 % rmp,
 %prstab,
 %prstper,
  longbibliography,
 %floatfix,
 %lengthcheck,%
 ]{revtex4-1}

%\usepackage{cdmtcs-pdf}




\usepackage[dvipsnames]{xcolor}

\usepackage{mathptmx}% http://ctan.org/pkg/mathptmx

\usepackage{amssymb,amsthm,amsmath,bm}

\usepackage{tikz}
\usetikzlibrary{calc,decorations.pathreplacing,decorations.markings,positioning,shapes,snakes}
%\usetikzlibrary{calc,decorations.pathreplacing,decorations.markings,positioning,shapes,snakes,external}
%\tikzexternalize

\usepackage[breaklinks=true,colorlinks=true,anchorcolor=blue,citecolor=blue,filecolor=blue,menucolor=blue,pagecolor=blue,urlcolor=blue,linkcolor=blue]{hyperref}
\usepackage{graphicx}% Include figure files
\usepackage{url}

%%%%%%%%%%%%%%%%%%%%%%%%%%%%%
\usepackage{iftex}
\ifxetex
%
% XeLaTeX
%
\usepackage{fontspec}
%%\setmainfont{Times New Roman}
%\setmainfont{Garamond}
\setmainfont{Garamond Libre}
\setsansfont{Garamond Libre}
%
\fi
%%%%%%%%%%%%%%%%%%%%%%%%%%%%%


\begin{document}

\title{QRAND memo}
\author{}


%\author{Karl Svozil}
%\email{svozil@tuwien.ac.at}
%\homepage{http://tph.tuwien.ac.at/~svozil}

%\affiliation{Institute for Theoretical Physics,
%TU Wien,
%Wiedner Hauptstrasse 8-10/136,
%1040 Vienna,  Austria}


\date{\today}

\begin{abstract}
This draft contains a very brief description of activities of QRAND ---{\bf Q}antum {\bf R}esearch {\bf AN}d {\bf D}evelopment---in the domain of quantum computation and information, with emphasis on, but not limited to,
\begin{itemize}
\item[(i)]  industry-standard production and distribution of      random n-ary information;
\item[(ii)]  improvement and upgrade---e.g. normalization---of existing n-ary sequences relative to pre-defined criteria;
\item[(iii)] evaluation of quantum cryptanalytic attacks on cryptocurrencies by (relative to transaction processing) ``fast'' computation of the private key from the public key of digital signatures;
\item[(iv)] search for and development of quantum cryptographic protocols for  digital signatures which are save with respect to quantum cryptanalytic attacks;
\item[(v)] search for and development of (quantum) cryptographic protocols which are novel and asymmetric (e.g., based on hypergraph theory);
\item[(vi)] search for and development of quantum algorithms utilizing quantum parallelism via ``spread-process-fold'' strategies;
\item[(vii)] search for and develop capacity to generalized continuum (analog) computation.
\end{itemize}

\end{abstract}

%\keywords{Quantum contextuality, Gleason theorem, Kochen-Specker theorem, Born rule, quantum logic, probability distributions}
%\pacs{03.65.Ca, 02.50.-r, 02.10.-v, 03.65.Aa, 03.67.Ac, 03.65.Ud}

\maketitle

\fi

\newpage

\section{Services and Products}

QRAND intends to realize several main services and products in the following areas:
\begin{itemize}
\item[(i)]  quantum randomness;
\item[(ii)]  quantum cryptography;
\item[(iii)] quantum computation.
\item[(iv)] continuum (quantum) computation.
\end{itemize}


\newpage

\subsection{Randomness}
\begin{itemize}

\item[(i)]
certification and due diligence of existing random number generators and random sequences;

\item[(ii)]
improvement and upgrade---e.g. normalization, see later (iv)---of existing random sequences, relative to pre-defined goals and tasks;

\item[(iii)]
production of $n$-ary ($n > 1$ but finite) random sequences; e.g., by recording detector clicks from quantum systems, which are in some coherent superposition (aka linear combination in Hilbert space) and subject to quantum features such as complementarity, nonlocality, and contextuality;

\item[(iv)]
un-biasing by normalization of $n$-ary sequences to eliminate bias from such sequences because of unavoidable imperfections (e.g., thermal drift, misalignments) by modern advanced coding techniques.


\end{itemize}

\newpage

\subsection{Cryptanalysis}

Attempts in quantum cryptanalysis include, but are not limited to:

\begin{itemize}

\item[(i)] the evaluation of quantum cryptanalytic attacks on cryptocurrencies by (relative to transaction processing) ``fast'' computation of the private key from the public key of digital signatures;
\item[(ii)] the search for and development of quantum cryptographic protocols for digital signatures which are save to quantum cryptanalytic attacks.
\item[(iii)] the search for and development of (quantum) cryptographic protocols which are novel and asymmetric (e.g., based on hypergraph theory);
\item[(iv)] certification and due diligence of existing realizations of quantum communication protocols;
\item[(v)] search into improved quantum communication protocols.

\end{itemize}

\newpage

\subsection{Computation}


Quite trivially, any existing computer is a quantum computer because its microphysical layer is quantized.
Increased miniaturization, as indicated by the end of Moore's Law~\cite{end-of-moores-law1,end-of-moores-law2}, enforces processor designs and manufacturers to cope with quantization.
Besides, there are potential quantum advantages,

\begin{itemize}

\item[(i)] such as parallelization through coherent superpositions ``inclusions'' of classical exclusive states;
\item[(ii)] entanglement through that is relational encoding of multi-partite states, accompanied by individual value indefiniteness of the states of its constituents;
\item[(iii)] quantum contextuality in its various forms~\cite{Dzhafarov-2017,Abramsky2018,Grangier_2002,Auffeves-Grangier-2018,Auffves2020,Grangier-2020,cabello2021contextuality};
\item[(iv)] quantum nonlocality~\cite{epr,Howard1985171} associated with multipartite states.


\end{itemize}

Quantum computation can be seen as a trifold ``spread-process-fold'' strategy to obtain relevant information by encoded into incomplete knowledge states,
very much like Robert Musil's ``bridge metaphor'' of complex numbers:
at the beginning and the end there is solid information, in-between there are quantum states ``bridging an abyss''.



\newpage

\subsection{Continuum computation}

So far, the von Neumann type digital computer used today is based on Turing's ``paper-and-pencil operations'' performed by some algorithmic agent.
Yet, according to physics the universe is not disrete but continuous in many ways, shape and forms.
The structural richness of continuous computation has hardly been utilized so far.

To give just a hint: in the finite interval $\left[1,2\right]$:
\begin{itemize}

\item[(i)] there are an uncountable number of reals, as compared to only two integers;
\item[(ii)] even if compared to all rational numbers, or even if compared to all Turing computable numbers,
there are an undenumerable infinity ``more'' numbers in the real continuum interval $\left[1,2\right]$;
\item[(iii)] quantitatively, that is, in measure theoretic terms, compared to all numbers in the real continuum interval $\left[1,2\right]$,
all computable (let alone rational or integer numbers) numbers are of zero measure;
\item[(iv)] with probability one, any number ``drawn'' (via the axiom of choice) from the ``continuum urn'' is
Martin-L\"of/Kolmogorov/Chaitin~\cite{martin-lof} random.
\item[(v)] There exist powerful theorems transcending the Turing computable domain relative to the application of continua~\cite{specker57,kreisel}.


\end{itemize}

Harvesting the full structural capacity of physical continua may yield entirely novel ways and capacities of computation;
potentially beyond the Turing limit that is derived relative to computing machinery based on ``paper-and-pencil operations''.

\newpage


\section{Main Protagonists, Knowhow and Skills}

QRAND exploits quantum mechanical properties, based on contemporary physical conceptions of nature.

The main protagonists of this startup will be two academic seed centres ``across the globe'': sone in the EU (Vienna) and sone in New Zealand (Auckland);
with the founders Prof. Karl Svozil, TU Wien, Vienna, Austria, and Prof. Cristian Calude, University of Auckland, Auckland, NZ.
The QRAND Organization is based on academic cooperation; with previous EU/NZ backed joint grants (by the same name) on the subject.

Svozil has a long track of publications on the generation of quantum random sequences.
These include very early suggestions of ``quantum coin toss'' experiments~\cite{svozil-qct},
very early suggestions to utilize quantum contextuality for the production of random sequences~\cite{svozil-2009-howto},
as well as a recent monography on the subject~\cite{svozil-2016-pu-book}.

Calude has a long track of records on mathematical and computer science aspects of randomness,
including the formal definition and certification of randomness~\cite{calude:02,calude-dinneen05}, as well as in ``un-biasing''
imperfect sequencing through normalization algorithms~\cite{DBLP:conf/dlt/Calude93,10.1007/978-3-642-21341-0_10,Calude2017,Abbott_2019}.

Both researchers have collaborated in various forms and multiple papers on the certification~\cite{PhysRevA.82.022102}
and proposals for the physical production of quantum random numbers, certified by quantum mechanical features such as contextuality and
value indefiniteness~\cite{2012-incomput-proofsCJ,PhysRevA.89.032109,2015-AnalyticKS}.

For the production of quantum randomness hardware ``chips''  various groups around the globe
have acquired knowledge and know-how that can be exploited royalty-free.


\newpage

\bibliography{svozil}

\end{document}
