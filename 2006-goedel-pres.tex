%\documentclass[pra,showpacs,showkeys,amsfonts,amsmath,twocolumn]{revtex4}
\documentclass[amsmath,blue,handout,table]{beamer}
%\documentclass[pra,showpacs,showkeys,amsfonts]{revtex4}
\usepackage[T1]{fontenc}
\usepackage{beamerthemeshadow}
%\usepackage{beamerthemetree}
%\usepackage[dark]{beamerthemesidebar}
%\usepackage[headheight=24pt,footheight=12pt]{beamerthemesplit}
%\usepackage{beamerthemesplit}
%\usepackage[bar]{beamerthemetree}
\usepackage{graphicx}
\usepackage{pgf}
%\usepackage[usenames]{color}
%\newcommand{\Red}{\color{Red}}  %(VERY-Approx.PANTONE-RED)
%\newcommand{\Green}{\color{Green}}  %(VERY-Approx.PANTONE-GREEN)

%\RequirePackage[german]{babel}
%\selectlanguage{german}
%\RequirePackage[isolatin]{inputenc}

\pgfdeclareimage[height=0.5cm]{logo}{tu-logo}
\logo{\pgfuseimage{logo}}
\beamertemplatetriangleitem
\begin{document}

\title{\bf \textcolor{yellow}{Unknowability in physics}}
\subtitle{\textcolor{yellow!60}{``Randomness \& Undecidability in Physics'' (World Scientific, 1993)
http://tph.tuwien.ac.at/$\sim$svozil/publ/2006-goedel-pres.pdf}}
\author{Karl Svozil}
\institute{Institut f\"ur Theoretische Physik, University of Technology Vienna, \\
Wiedner Hauptstra\ss e 8-10/136, A-1040 Vienna, Austria\\
svozil@tuwien.ac.at
%{\tiny Disclaimer: Die hier vertretenen Meinungen des Autors verstehen sich als Diskussionsbeitr�ge und decken sich nicht notwendigerweise mit den Positionen der Technischen Universit�t Wien oder deren Vertreter.}
}
\date{April. 28, 2006}
\maketitle

\frame[shrink=1.3]{\tableofcontents}


%%%%%%%%%%%%%%%%%%%%%%%%%%%%%%%%%%%%%%%%%%%%%%%%%%%%%%%%%%%%%%%%%%%%%%%%%%%%%%%%%%%%%%%%%%%%%%%%%%%%%%%%%
%%%%%%%%%%%%%%%%%%%%%%%%%%%%%%%%%%%%%%%%%%%%%%%%%%%%%%%%%%%%%%%%%%%%%%%%%%%%%%%%%%%%%%%%%%%%%%%%%%%%%%%%%
%%%%%%%%%%%%%%%%%%%%%%%%%%%%%%%%%%%%%%%%%%%%%%%%%%%%%%%%%%%%%%%%%%%%%%%%%%%%%%%%%%%%%%%%%%%%%%%%%%%%%%%%%
%%%%%%%%%%%%%%%%%%%%%%%%%%%%%%%%%%%%%%%%%%%%%%%%%%%%%%%%%%%%%%%%%%%%%%%%%%%%%%%%%%%%%%%%%%%%%%%%%%%%%%%%%

\section{G\"odelian incompleteness theorems in physics}
\frame{ }

\subsection{G\"odel did not believe in applicability to quantum mechanics}
\frame{ }


\subsection{Due to the embedment of a universal
computer into a physical system, and due to the reduction to the halting problem,
there exist provable undecidable physical propositions.}
\frame{ }


\subsection{Due to the recursive insolvability of the Rule Inference Problem,
the induction problem is provable unsolvable}
\frame{ }


\subsection{Additional issues: intrinsic versus extrinsic description; where exactly to place the cut between object and observer}
\frame{ }


\section{Randomness in classical physics}
\frame{ }


\subsection{``Almost all'' elements of the continuum are not only noncomputable, but random; i.e., algorithmically incompressible}
\frame{ }

\subsection{Classical chaos due to ``unfolding'' of the initial value drawn from the ``continuum urn''}
\frame{ }


\section{``Quantum'' unknowability}
\frame{ }


\subsection{Quantum coin tosses}
\frame{ }


\subsection{Complementarity}
\frame{ }


\subsection{Value indefiniteness: there is no global consistent truth assignment consistent with even a finite number of local ones}
\frame{ }


\end{document}


