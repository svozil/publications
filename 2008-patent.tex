\documentstyle[amsfonts,a4,12pt]{article}
\RequirePackage[german]{babel}
\RequirePackage{times}
\RequirePackage{courier}
\RequirePackage{mathptm}
%\RequirePackage{bookman}
%\RequirePackage{helvetic}
%\RequirePackage{times}
\selectlanguage{german}
\RequirePackage[isolatin]{inputenc}
\begin{document}
\pagestyle{empty}

There exist several measures of fractality,
the most popular being the ``Hausdorff measure'' or ``Hausdorff dimension (exponent)'' \cite{shaw,ott81,schuster1,svozil-93}
promoted by  B. B. Mandelbrot \cite{mandelbrot-77,mandelbrot-83,falconer1,falconer2}.
The Hurst exponent is one such measure discussed for describing and quantifying fractal geometry,
but it is in no way a prevalent  or indispensible  one.
It is only descriptive but we do not use or need it in any way to construct the geometric patterns or behaviors.



\bibliography{svozil}
%\bibliographystyle{apalike}
%\bibliographystyle{named}
\bibliographystyle{osa}

\begin{center}
$\widetilde{\qquad \qquad }$
$\widetilde{\qquad \qquad}$
$\widetilde{\qquad \qquad }$
\end{center}

\end{document}
\end
