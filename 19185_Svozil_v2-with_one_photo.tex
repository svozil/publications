%\documentclass[prb,amssymb,showpacs,showkeys,twocolumn]{revtex4}
\documentclass[prb,amssymb,preprint]{revtex4}

\usepackage{graphicx}

\usepackage[usenames]{color}
\newcommand{\Red}{\color{Red}}
\newcommand{\Green}{\color{Green}}
%\newcommand{\Red}{red}
%\newcommand{\Green}{green}
\def\ddarrow{{\swarrow\mkern-18mu\nearrow}}
\def\cddarrow{{\searrow\mkern-18mu\nwarrow}}

\begin{document}

\title{Staging quantum cryptography with chocolate balls}

\thanks{The author reserves the copyright for all public performances.
Performance lincenses are granted for educational institutions
and other not-for-profit performances for free;
these institutions are kindly asked to send a small note about the
performance to the author.}

\author{Karl Svozil}
\email{svozil@tuwien.ac.at}
\homepage{http://tph.tuwien.ac.at/~svozil}
\affiliation{Institut f\"ur Theoretische Physik, University of Technology
Vienna, Wiedner Hauptstra\ss e 8-10/136, A-1040 Vienna, Austria}

\begin{abstract}
Moderated by a director, laypeople and students assume the
role of quanta and enact a quantum cryptographic protocol. The performance
is based on a generalized urn model capable of reproducing complementarity
even for classical chocolate balls.
\end{abstract}

%\pacs{03.67.Dd,01.20.+x,01.75.+m,01.40.-d}
%\keywords{quantum cryptography, role-playing game, RPG, quanta}


\maketitle

\begin{quote}
\begin{flushright}
{%\footnotesize
Dedicated to Antonin Artaud, \\
author of {\it Le th{\'{e}}{\^{a}}tre et son double} \cite{Arthaud}.
 }
\end{flushright}
\end{quote}

\section{Background}

Quantum cryptography is a relatively recent and very active field of
research. Its main characteristic is the use of individual particles for
encrypted information transmission. Its objective is to encrypt messages or
to create and enlarge a set of secret equal random numbers between two
spatially separated agents by means of elementary particles, such as single
photons, that are transmitted through a quantum channel.

The history of quantum cryptography dates back to around 1970 to the
manuscript by Wiesner\cite{wiesner} and a protocol
by Bennett \& Brassard\cite{benn-82,benn-84,ekert91,benn-92,gisin-qc-rmp}
in 1984, henceforth called BB84. Since
then, experimental prototyping has advanced rapidly.
Without going into too
much detail and just to name a few examples, the work ranges from the
first experiments carried out at the IBM Yorktown Heights Laboratory
by Bennett and co-workers in
1989,\cite{benn-92}
to signal transmissions across Lake Geneva in 1993,\cite{gisin-qc-rmp} and
the network in the Boston Area which has been sponsored by DARPA since
2003.\cite{ell-co-05} In a much publicized, spectacular demonstration, a
quantum cryptographic aided bank transfer took place via optical fibers
installed in the sewers of Vienna.\cite{pflmubpskwjz}

Quantum cryptography forms an important link between quantum theory and
experimental technology and possibly even industrial applications. The
public is much interested in quantum physics and quantum cryptography, but
the protocols used are rarely made available to the layperson or student in
any detail. For an outsider these subjects seem to be shrouded in a kind
of ``mystic veil" and are very difficult to understand.


In what follows,
a play will be proposed which closely follows quantum cryptographic protocols.
It involves a moderator, actors and possibly spectators
and requires a couple of properly colored chocolate balls
(the chocolate is not essential but pleasant).
The coloring of the chocolate balls follows
a simple but effective generalized urn
model introduced by Wright\cite{wright,wright:pent,svozil-2001-eua} to mimic
complementarity. A generalized urn model is
characterized by an ensemble of balls with black background color.
Printed on these balls are color symbols from a symbolic alphabet.
A particular ball type is associated with a unique combination of
mono-colored (no mixture of wavelength) symbols printed on the black ball
background. Every ball contains just one single symbol per color.

Assume further some mono-spectral filters or eyeglasses that are
perfect and that totally absorb light of all other colors
but a particular one.
In this way, every color can be associated with a particular eyeglass and
vice versa.

When one
looks
at a particular ball through such an eyeglass, the only operationally
recognizable symbol will be the one with the particular color that is
transmitted through the eyeglass. All other colors are absorbed, and the
symbol on the ball will appear black and therefore cannot be
differentiated from the black background. Hence the ball appears to carry a
different message or symbol depending on the eyeglass through which it is
viewed. We will present an explicit example featuring complementarity, which
is similar in many ways to quantum complementarity.

The difference between the chocolate
 (black) balls and quanta is the
possibility of viewing all the different symbols on the chocolate balls by
taking off the eyeglasses. Quantum mechanics does not provide us with such a
possibility. On the contrary, there are strong arguments suggesting that
the assumption of a simultaneous physical existence\cite{epr} of such
complementary observables yields a complete
contradiction,\cite{kochen1,ghz,ghsz} a result that has recently been
experimentally confirmed.\cite{panbdwz}

The differences between the quasi-classical and the quantum cases should be
made explicit. The protocols used with the quasi-classical
chocolate ball model appear very similar to the quantum mechanical ones.
However, we should not conclude that in the quantum domain
there merely exist some constraints on the measurements (such as the
mono-colored glasses) that prevent us from perceiving the ``real''
picture. In the quantum domain, the simultaneous imprinting of different
symbols in different colors is impossible in general. This impossibility
can be a very good starting point for a better understanding of quantum
systems and their differences with classical systems.

\section{Principles of conduct}

To make quantum cryptography a real-life experience, we have turned the
quantum world into a drama in which actors and a moderator present
a quantum cryptographic protocol on stage. The audience is
actively involved in the presentation. If
possible, the event should be moderated by a well-known comedian or by a
physics teacher.


The entire play
is analogous to a surreal
experiment: single quanta are not completely
predictable. Their behavior is determined by random events, and marked by
the ``noise'' that would accompany the public presentation of the quantum
cryptographic protocols. Therefore, the interference of individual
participants is even encouraged and not a deficiency of the performance.

Throughout the performance, everybody should have fun, relax, and try to
feel and act like an elementary particle in the spirit of the meditative
Zen koan.
The participants might try to feel like
Schr\"{o}dinger's cat\cite{schrodinger} or like a particle simultaneously
passing through two spatially separated slits.We might
contemplate how conscious minds could experience a coherent quantum
superposition between two states of consciousness. However, this kind of
sophistication is neither necessary nor important for dramatizing quantum
cryptographic protocols.

Our entire empirical knowledge of the world is based on the occurrence of
elementary (binary) events, such as the reactions caused by quanta in
particle detectors yielding a click. Therefore, the following simple
syntactic rules should not be dismissed as
mere recipes, for even quantum mechanics can be
considered to be a
sophisticated set of laws with a possibly superfluous\cite{fuchs-peres}
semantic superstructure.

\section{Instructions for staging the performance}

Our objective is to generate a secret sequence of random numbers known only
by two agents, Alice and Bob. To do so, the following
utensils will be required:

\begin{enumerate}

\item Two sets each of fully saturated eyeglasses in red and green
(complementary colors).

\item An urn or bucket.

\item A large number of foil-wrapped chocolate balls (called
Mozartkugeln in Austria) or similar balls, each with a black background
imprinted with one red and one green symbol (either 0 or 1) to be placed
inside the urn.
The symbols can, for instance be prepared as color stickers shaped in different geometries
(e.g., circles and stars or squares).
There are four types of
balls, which are listed in Table~\ref{2005-nl1-t1}. There are an equal
number of each type in the urn.

\item Small red and green flags, two of each.

\item Two blackboards and chalk (or two secret notebooks).

\item Two coins.

\end{enumerate}

The following actors are involved:

\begin{enumerate}

\item A moderator who makes comments and ensures that the participants more
or less adhere to the protocol as described below. The moderator has many
liberties and may even choose to stage cryptographic attacks.

\item Alice and Bob, two spatially separated parties.

\item Ideally, but not necessarily, some actors who know the protocol and
introduce new visitors to the roles of Alice, Bob and the quanta.

\item A large number of people assuming the roles of the quanta. They are in
charge of transmitting the chocolates and may eat them in the course of
events or afterward.

\end{enumerate}

In the performance chocolates marked with the symbols 0 and 1 in red
correspond to horizontally
($\leftrightarrow$) and vertically ($\updownarrow$) polarized photons in
quantum optics, respectively. Chocolates marked with the symbols
0 and 1 in green correspond to left and right circularly polarized
photons, or alternatively to linearly polarized photons with polarization
directions ($\ddarrow$) and ($\cddarrow$) rotated by 45$^\circ$
from the horizontal and the vertical, respectively.

In the basic quantum key distribution protocol mimicked here, Alice sends
Bob a random sequence of polarized photons in four states belonging to two
different conjugate bases. (These four states correspond to the four
different kinds of balls; the two bases correspond to
the two complementary colors.) In the second phase, Bob chooses randomly and
independently of Alice which one of the two different conjugate bases he
wishes to use to perform his photon measurements. (This choice
will correspond to Bob's choice of colored eyeglass.) Bob then performs the
measurements and records both the results (corresponding to the symbols read
through the colored eyeglass) as well as his measurement basis
(corresponding to the color of the eyeglass) for each photon. Bob
then announces publicly the bases (colors) chosen, but not the outcome of
his measurements. Alice compares these bases (colors) to the ones she used
in sending out the photons (balls). They then publicly agree to use the
matching bases (colors), thereby discarding all events in which their bases
(colors) are different, or in which Bob has not received any photon at all.
In a final step, they form a (random) sequence by taking the succession of
all of these outcomes,
coded in a binary alphabet.

In more detail, the protocol is as follows:

\begin{enumerate}

\item Alice flips a coin to chose one of the two pairs of glasses: heads
for the green glasses, tails for the red ones. She puts them on and randomly
draws a chocolate from the urn. She can only read the symbol in the color
of her glasses.  She
writes the symbol she can read and the color used, either on the blackboard
or in her notebook. Should she take off her glasses or look at
the symbols with the other pair, the player carrying the chocolate ball
is required to eat it
at once.

\item After writing down the symbol, Alice hands the chocolate to the
quantum, who carries it to the recipient Bob. During this process,
the chocolate could become lost and never reach its destination (those with
a sweet tooth might for example not be able to wait and eat their chocolate
immediately).

\item Before Bob can take the chocolate and look at it, he needs to
flip a coin to choose a pair of glasses. He puts on the glasses and takes a
look at the chocolate ball he has just received. He, too, will only be able
to read one of the symbols, because the other one is imprinted in the
complementary color and appears black to him. Then he makes a note of the
symbol he has read and the color used. As before, should he take off his
glasses or look at the symbols with the other glasses, the quantum is
required to eat the chocolate at once.

\item After the legal transmission has taken place, the quantum may eat the
chocolate ball just transfered from Alice to Bob, or give it away.

\item Now Bob uses one of the two flags (red or green) to tell Alice whether
he has received anything at all and what color his glasses are. He does not
communicate the symbol itself.

\item At the same time, Alice uses one of her flags to inform Bob of the
color of her glasses. She also does not tell Bob the symbol she identified.

\item
Alice and Bob only registers the symbol on a blackboard or on a note
if they both received the corresponding chocolate, and if the
color of their glasses (that is, their flags) matched. Otherwise, they
dismiss the entry.

\item The entire process (1--7) is repeated several times.
\end{enumerate}

As a result, Alice and Bob obtain an identical random sequence of the
symbols 0 and 1 representing identical outcomes.
This random
sequence can be interpreted as a ``random key'' that could be used in a
cryptographic application. A more amusing application is to let Alice
communicate to Bob secretly whether (1) or not (0) she would consider giving
him her cell phone number. For this task only a single bit of the
sequence they have created is required. Alice forms the sum $s= i\oplus j= i+j
\,{\rm mod}\, 2$ of her decision $i$ and the secret bit $j$ and cries it out loudly
over to Bob. Bob can decode Alice's message to plain text by simply forming
the sum $s\oplus t=i$ of Alice's encrypted message $s$ and the secret bit
$t=j$ shared with Alice, for $j\oplus t=j\oplus j=0$. This task is a
romantic and easily communicable way of employing one-time pads generated by
quantum cryptography.

Alice and Bob compare some of the symbols directly to make sure that
there has been no attack by an eavesdropper. Indeed, if the eavesdropper Eve
is bound to one color and cannot perceive both symbols imprinted on the balls
simultaneously, then she will sometimes choose the wrong color, which does
not match Alice's and Bob's. If Eve digests the chocolate after observing
it (and does not merely retransmit it), she can only guess the symbol in the
other color with a 50:50 chance. Thus the new ball she has to send to Bob
will carry the wrong symbol in half the cases when her color does not
match Alice's and Bob's color. Thus, if Alice and Bob compare some of their
symbols, they could realize that Eve is listening.

\section{Alternative protocols}

There exist numerous possible variants of the dramatization of the BB84
protocol. A great simplification would be the total abandonment of the black
background of the chocolate balls as well as the colored eyeglasses. In
this case, both Alice and Bob simply decide by themselves which colored
eyeglasses to take and record the symbols in the color chosen.

In the following, we will present yet another BB84-type protocol
within the context of the translation principle.\cite{svozil-2003-garda}
First, Alice (the sender) defines one of two possible contexts or
colors; in this case either red or green.
Then, the receiver Bob chooses another color, which is independent of Alice's
choice. If the two colors do not match, a color (or context\cite{svozil-2003-garda})
translation
is carried out by flipping a coin. In this case, there is
no correlation between the two symbols.

In this protocol, we use sets of two chocolate figures shaped like 0 and 1,
and uniformly colored in red and green, as shown in
Table~\ref{2005-nl1-t1a}. An equal amount of each type of figure is placed
inside an urn. No colored glasses are necessary to carry out this protocol.

The protocol is as follows:

\begin{enumerate}

\item First, Alice randomly draws one figure from the urn and makes a
note of its value (0 or 1) and its color. Then she gives the figure to one
of the spectators carrying the figures.

\item
The quantum carries the figure to Bob.

\item
Bob flips a coin and chooses one of two colors.

\begin{enumerate}

\item If the color corresponds to that of the figure chosen by Alice and
presented by the quantum ,
the symbol of the figure counts and Bob makes a note of the symbol and its
color.

\item If it does not correspond, Bob takes the result of the coin he has
just flipped and assigns heads to 0 and tails to 1.
This results in a randomization of the outcome, just as in quantum mechanics.
If he
wishes, he may flip the coin
again and use the new result
instead; just to emphasize the distinction between the random choice related to the
type of measurement and  the random outcome.
\end{enumerate}
In any case Bob writes down the resulting symbol and the color.

\item The remaining steps corresponds to the previous protocol.

\end{enumerate}

With this protocol Alice and Bob, by keeping the symbols in the matching
colors, arrive at two identical random sequences on their sides.
If chocolate balls are not readily available,
the advantage of this procedure over the protocol involving black chocolate balls
with red and green symbols imprinted on them
is that here only
two arbitrary but different geometric shapes of chocolate pieces in two different colors are required.

\section{Further dramaturgical aspects, attacks and realization}

It is possible to scramble the protocol in its simplest form and thus the
encryption by drawing two or more chocolate balls, with or without identical
symbols on them, from the urn at once; or by breaking the time order of
events.
This would correspond to technological problems related to
implementations of quantum cryptography.

It is allowed to eavesdrop on the encrypted
messages. Forthe first protocol, every potential eavesdroper needs
to wear colored glasses. Note that no one (not even
the quanta) may take additional chocolates or chocolate figures from the
urn, which are identical to the one originally drawn by Alice.
In a sense, this rule implements the no-cloning theorem, which states that
it is not possible to copy an arbitrary quantum if it is in a coherent
superposition of the two classical states.

The most promising eavesdropping strategy is the man-in-the-middle
attack, which is often used in mobile phone networks.
The attacker manages to impersonate Bob when communicating with Alice and
vice versa. What basically happens is that two different quantum
cryptographic protocols are connected in series, or carried out
independently from each other. Quantum cryptography is not immune to this
kind of attack.

The first performance of the quantum drama we have sketched took place
in Vienna at the University of Technology as a part of
``Lange Nacht der Forschung'' (long night of science). Experience showed
that a considerable fraction of the audience obtained some understanding of
the protocol; in particular the players acting as Alice and Bob. The photograph in Figure \ref{2005-ln1e-pics} depicts a player trying to carry
a chocolate ball across a walkway.
Most
of the audience got the feeling that quantum cryptography is not so
cryptic after all.

For students of physics the most important questions
are those related to the differences and similarities between chocolate balls
and quanta.
It should be stated quite clearly from the beginning that,
although the quasi-classical protocols resemble the quantum cryptographic
ones, there are fundamental differences
with regard to the quantum physical properties and observables:
it is not just sufficient to assume that quantum properties are hidden
by operational inaccessibility, such as colored eyeglasses
blocking the recognition of symbols painted in the complementary color.
In quantum physics the Bell-type, Kochen-Specker,
and Greenberger-Horne-Zeilinger theorems\cite{mermin-93}
lead to the conclusion that certain observables do not
have a defined value prior to their measurement.
The quasi-classical analogies discussed here serve
as a good introduction to the quantum cryptographic protocols,
and are also a good motivation and starting point
for considerations of quantum complementarity and value-indefiniteness.

\begin{acknowledgements}
The idea for this paper was born over a coffee conversation with G\"unther
Krenn. The first public performance was sponsored by Lange Nacht der
Forschung,
\url{<www.langenachtderforschung.at>}.
The chocoate balls
were donated by Manner \url{www.manner.com>}, and the black foil covering
the balls was donated by Constantia Packaging
\url{<www.constantia-packaging.com>}.
Thanks go to Karin Peter and the public relation office of the TU Vienna for
providing the infrastructure, to the Impro Theater for the stage
performance, and to Martin Puntigam for moderating part of the performances.
A second performance took place at Auckland University's
Centre for Discrete Mathematics and Theoretical Computer Science.
\end{acknowledgements}

\begin{thebibliography}{10}
\newcommand{\enquote}[1]{``#1''}
\expandafter\ifx\csname url\endcsname\relax
\def\url#1{\texttt{#1}}\fi
\expandafter\ifx\csname urlprefix\endcsname\relax\def\urlprefix{URL }\fi
\providecommand{\eprint}[2][]{\url{#2}}

\bibitem{Arthaud}
A. Artaud, \emph{Le th{\'{e}}{\^{a}}tre et Don Double} (Gallimard, Paris,
1938).

\bibitem{wiesner} S. Wiesner, ``Conjugate coding,'' Sigact News
\textbf{15}, 78--88 (1983).
Manuscript written {\it circa} 1970.\cite{benn-92}

\bibitem{benn-82} C. H. Bennett, G. Brassard, S. Breidbart, and S. Wiesner,
``Quantum
cryptography, or unforgable subway tokens,'' in \emph{Advances in
Cryptography: Proceedings of Crypto '82} (Plenum Press, New York,
1982), pp. 78--82.

\bibitem{benn-84}
C. H. Bennett and G. Brassard, ``Quantum cryptography: Public key
distribution and coin tossing,'' in \emph{Proceedings of the IEEE
International Conference on Computers, Systems, and Signal Processing,
Bangalore, India} (IEEE Computer Society Press, 1984), pp. 175--179.

\bibitem{ekert91}
A. Ekert, ``Quantum cryptography based on Bell's theorem,'' Phys.
Rev. Lett. \textbf{67}, 661--663 (1991).

\bibitem{benn-92}
C. H. Bennett, F. Bessette, G. Brassard, L. Salvail, and J. Smolin,
``Experimental quantum cryptography,'' J. Cryptology
\textbf{5}, 3--28 (1992).

\bibitem{gisin-qc-rmp}
N. Gisin, G. Ribordy, W. Tittel, and H. Zbinden, ``Quantum
cryptography,'' Rev. Mod.Phys. \textbf{74}, 145--195 (2002).

\bibitem{ell-co-05}
C. Elliott, A. Colvin, D. Pearson, O. Pikalo, J. Schlafer, and H. Yeh,
``Current status of the DARPA quantum network,''
quant-ph/0503058.

\bibitem{pflmubpskwjz}
A. Poppe, A. Fedrizzi, T. Loruenser, O. Maurhardt, R. Ursin, H. R. Boehm,
M. Peev, M. Suda, C. Kurtsiefer, H. Weinfurter, T. Jennewein, and
A. Zeilinger, ``Practical quantum key distribution with
polarization-entangled photons,'' Optics Express \textbf{12}, 3865--3871
(2004) or quant-ph/0404115.

\bibitem{wright}
R. Wright, ``Generalized urn models,'' Found. Phys.
\textbf{20}, 881--903 (1990).

\bibitem{wright:pent}
R. Wright, ``The state of the pentagon. A nonclassical example,'' in
\emph{Mathematical Foundations of Quantum Theory}, edited by A. R.
Marlow (Academic Press, New York, 1978), pp.
255--274.

\bibitem{svozil-2001-eua}
K. Svozil, ``Logical equivalence between generalized urn models and
finite automata,'' Int. J. Theor. Phys. \textbf{44},
745--754 (2005) or quant-ph/0209136.

\bibitem{epr}
A. Einstein, B. Podolsky, and N. Rosen, ``Can quantum-mechanical
description of physical reality be considered complete?,'' Phys. Rev.
\textbf{47}, 777--780 (1935).

\bibitem{kochen1}
S. Kochen and E. P. Specker, ``The problem of hidden variables in
quantum mechanics,'' J. Math. Mechanics \textbf{17} (1),
59--87 (1967). Reprinted in Ref.~\onlinecite{specker-ges}, pp. 235--263.

\bibitem{ghz}
D. M. Greenberger, M. A. Horne, and A. Zeilinger, ``Going beyond Bell's
theorem,'' in \emph{Bell's Theorem, Quantum Theory, and Conceptions of the
{U}niverse}, edited by M. Kafatos (Kluwer Academic
Publishers,
Dordrecht, 1989), pp. 73--76.

\bibitem{ghsz}
D. M. Greenberger, M. A. Horne, A. Shimony, and A. Zeilinger,
``Bell's
theorem without inequalities,'' Am. J. Phys. \textbf{58},
1131--1143 (1990).

\bibitem{panbdwz}
J.-W. Pan, D. Bouwmeester, M. Daniell, H. Weinfurter, and A. Zeilinger,
``Experimental test of quantum nonlocality in three-photon
Greenberger-Horne-Zeilinger entanglement,'' Nature \textbf{403},
515--519 (2000).



\bibitem{schrodinger}
E. Schr{\"{o}}dinger, ``Die gegenw{\"{a}}rtige {S}ituation in der
{Q}uantenmechanik,'' Naturwissenschaften \textbf{23}, 807--812, 823--828,
844--849 (1935). English translation in Refs.~\onlinecite{trimmer} and
\onlinecite{wheeler-Zurek:83}, pp.
152--167,
\urlprefix\url{<http://wwwthep.physik.uni-mainz.de/~matschul/rot/schroedinger.pdf>}.

\bibitem{fuchs-peres}
C. A. Fuchs and A. Peres, ``Quantum theory needs no interpretation,''
Phys. Today \textbf{53} (4), 70--71 (2000). Further discussions of the
article can be found in Phys.
Today {\bf 53} (8), 11--14 (2000).

\bibitem{svozil-2003-garda}
K. Svozil, ``Quantum information via state partitions and the context
translation principle,'' J. Mod. Optics \textbf{51}, 811--819
(2004), or quant-ph/0308110.

\bibitem{mermin-93}
N. D. Mermin, ``Hidden variables and the two theorems of {J}ohn
{B}ell,''
Rev. Mod. Phys. \textbf{65}, 803--815 (1993).

\bibitem{specker-ges}
E. Specker, \emph{Selecta} (Birkh{\"{a}}user Verlag, Basel, 1990).

\bibitem{trimmer} J. D. Trimmer, ``The present situation in quantum
mechanics: A translation of {S}chr{\"{o}}dinger's `cat paradox','' Proc. Am.
Phil. Soc.
\textbf{124}, 323--338 (1980). Reprinted in
Ref.~\onlinecite{wheeler-Zurek:83}, pp.
152--167.

\bibitem{wheeler-Zurek:83}
J. A. Wheeler and W. H. Zurek, \emph{Quantum Theory and Measurement}
(Princeton
University Press, Princeton, 1983).

\end{thebibliography}

\newpage
\section*{Tables}

\begin{table}[h]
\begin{tabular}{ccc}
\hline\hline
ball type &{\Red{red}}&{\Green{green}}\\
\hline
1&{\Red{0}} & {\Green{0}}\\
2&{\Red {0}}&{\Green {1}}\\
3&{\Red {1}}&{\Green {0}}\\
4&{\Red {1}}&{\Green {1}}\\
\hline\hline
\end{tabular}
\caption{Schema for imprinting of the chocolate balls.
\label{2005-nl1-t1}}
\end{table}

\begin{table}
\begin{tabular}{ccc}
\hline\hline
ball type&{\Red red}&{\Green green}\\
\hline
1& --- &{\Green 0}\\
2& --- &{\Green 1}\\
3&{\Red 0}& --- \\
4&{\Red 1}& --- \\
\hline\hline
\end{tabular}
\caption{Color and geometry of the four chocolate
figures.
\label{2005-nl1-t1a}}
\end{table}


\newpage
\section*{Figure}




\begin{figure}[h]
\begin{center}
\includegraphics[width=15cm]{Svozil6}
\end{center}
\caption{A player carrying a chocolate ball across the walkway. Some  ``agents'' try to steal the chocolate ball.}
\label{2005-ln1e-pics}
\end{figure}

\end{document}
