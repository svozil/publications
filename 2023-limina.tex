\newif\ifws
%\wstrue
\ifws

\documentclass{article}

\usepackage{graphicx}        % standard LaTeX graphics tool
\usepackage{xcolor}

\usepackage{hyperref}
\hypersetup{
    colorlinks,
    linkcolor={blue!80!black},
    citecolor={red!75!black},
    urlcolor={blue!80!black}
}

% Damit die Verwendung der deutschen Sprache nicht ganz so umst\"andlich wird,
% sollte man die folgenden Pakete einbinden:


%German
%\usepackage[latin1]{inputenc}% erm\"oglich die direkte Eingabe der Umlaute
%\usepackage[T1]{fontenc} % das Trennen der Umlaute
%\usepackage{ngerman} % hiermit werden deutsche Bezeichnungen genutzt und
                     % die W\"orter werden anhand der neue Rechtschreibung
                     % automatisch getrennt.
\title{The pitfalls of explaining too much: scientific overreach and explanation traps}
\author{Karl Svozil \\
        Institute for Theoretical Physics,
Vienna  University of Technology,  \\
Wiedner Hauptstrasse 8-10/136,
1040 Vienna,  Austria
        }

\date{\today}
% Hinweis: \title{um was auch immer es geht}, \author{wer es auch immer
% geschrieben hat} und  \date{wann auch immer das war} k\"onnen vor
% oder nach dem  Kommando \begin{document} stehen
% Aber der \maketitle Befehl mu\ss{} nach dem \begin{document} Kommando stehen!
\begin{document}

\maketitle


\begin{abstract}
This paper builds upon Clarke's famous quote that advanced technology appears as magic, and examines the dangers of over-extending current scientific concepts. We argue that too much explanation at the current level of category formation can hinder understanding and propose a methodological framework to avoid this pitfall.
\end{abstract}


\else
\documentclass[%
 %reprint,
  twocolumn,
 %superscriptaddress,
 %groupedaddress,
 %unsortedaddress,
 %runinaddress,
 %frontmatterverbose,
 % preprint,
 showpacs,
 showkeys,
 preprintnumbers,
 %nofootinbib,
 %nobibnotes,
 %bibnotes,
 amsmath,amssymb,
 aps,
 % prl,
  pra,
 % prb,
 % rmp,
 %prstab,
 %prstper,
  longbibliography,
 floatfix,
 %lengthcheck,%
 ]{revtex4-1}

%\usepackage{cdmtcs-pdf}

\usepackage{mathptmx}% http://ctan.org/pkg/mathptmx

\usepackage{amssymb,amsthm,amsmath}

\usepackage{tikz}
\usepackage{graphicx}% Include figure files
%\usepackage{url}

\usepackage{xcolor}

\usepackage{hyperref}
\hypersetup{
    colorlinks,
    linkcolor={blue},
    citecolor={red!75!black},
    urlcolor={blue}
}


\begin{document}


\title{The pitfalls of explaining too much: scientific overreach and explanation traps}


\author{Karl Svozil}
\email{svozil@tuwien.ac.at}
\homepage{http://tph.tuwien.ac.at/~svozil}

\affiliation{Institute for Theoretical Physics, Vienna  University of Technology, Wiedner Hauptstrasse 8-10/136, 1040 Vienna,  Austria}



\date{\today}

\begin{abstract}
This paper builds upon Clarke's famous quote that advanced technology appears as magic, and examines the dangers of over-extending current scientific concepts. We argue that too much explanation at the current level of category formation can hinder understanding and propose a methodological framework to avoid this pitfall.
\end{abstract}

%\pacs{03.65.Aa, 03.65.Ta, 03.65.Ud, 03.67.-a}
\keywords{UAP, UFO, data volume trap, explanation trap, anti-gravity}
%\preprint{CDMTCS preprint nr. x}

\maketitle

\fi

\section{Magic turns science}


Arthur Charles Clarke's Third Law, which states that ``any sufficiently advanced technology is indistinguishable from magic,'' has become a well-known concept~\cite{Clarke2000Jan}. Essentially, as technology advances beyond a certain point in relation to a fixed base level, it becomes increasingly challenging for those who are only familiar with the base level to distinguish the advanced technology from magic.
This statement has been used frequently in popular culture and has become a popular meme.

Clarke's Third Law is a matter of epistemology rather than ontology, as it focuses on the ability of the recipient and observer to understand and interpret such phenomena, rather than on the inherent nature of the phenomena themselves.
Additionally, this principle suggests that, by analogy, any supposedly magical performance or phenomenon may be indicative of our own lack of understanding rather than the intervention of divine or demonic forces in the course of phenomena.
This principle can also be employed as a framework for processing highly advanced technological artifacts.

This paper seeks to enrich Clarke's Third Law by placing it within the context of the ongoing debate in the history and philosophy of science and technology. We will then introduce a quantified version of this principle and apply it to the domain of Unidentified Anomalous Phenomena (UAP).

\section{Succession of scientific revolutions}

Thomas Samuel Kuhn~\cite{kuhn} and Imre Lakatos~\cite{lakatosch,lakatos_1978}
are two philosophers of science who presented different viewpoints on the progression of scientific knowledge that were both targeted against Popper's
demarcation criterion of falsifiability.
Kuhn believed that science as a socio-psychological process is not a linear and cumulative process, but rather a series of sudden, disruptive shifts or revolutions.
He argued that science is guided by a dominant paradigm, which includes a set of assumptions, methods, and problems that direct typical scientific research. Normal science is characterized by puzzle-solving activities that aim to expand and refine the paradigm. However, anomalies or phenomena may arise that cannot be explained by the paradigm. When these anomalies accumulate and challenge the paradigm's validity, a crisis ensues that may result in a scientific revolution. A revolution occurs when a new paradigm replaces the old one and redefines the standards and criteria of scientific practice. Kuhn argued that paradigms are incommensurable, implying that they cannot be objectively compared or evaluated since they have distinct conceptual frameworks and values.

Lakatos, on the other hand, believed that scientific progress is not only about replacing old theories with new ones but also about improving them by adding auxiliary assumptions.
He proposed a methodology called ``research programs'' which are sets of theories and methods that aim to explain and resolve problems in science.
Research programs are characterized on the one hand
by their ``hard core'' which consists of fundamental assumptions that are not open to revision,
and on the other hand their ``protective belt'' which includes auxiliary hypotheses that can be modified or abandoned if the research program fails to predict or explain
the penomenology, or fails to solve upcoming problems.
Lakatos argued that research programs can be compared and evaluated based on their ability to solve problems and generate novel predictions.


Some examples of scientific revolutions include the Copernican revolution in astronomy, the Darwinian revolution in biology, and the Einsteinian revolution in physics. These revolutions involved a radical shift in the way scientists thought about the world and led to new paradigms that redefined the standards and criteria of scientific practice.


Some criticisms of Kuhn's view include that it is too relativistic and does not provide a clear criterion for distinguishing between scientific and non-scientific theories. Some criticisms of Lakatos' view include that it does not account for the role of social and historical factors in shaping scientific knowledge.





Both Kuhn and Lakatos, and to some extent Paul Feyerabend~\cite{feyerabend,fey-philpapers1,fey-philpapers2}), share a common perspective that:
\begin{enumerate}
\item During prolonged periods, there exist beliefs, hard-core assumptions, and practices that form a dominant scientific program.
\item Any dominant scientific program comprises core semantical concepts that are expressed through theoretical, syntactic formalizations.
\item Eventually, a dominant scientific program will be overturned by another scientific program.
\item Paradigms are incommensurable~\cite{sep-incommensurability}, meaning that the semantical concepts of competing or successive scientific programs are un(cor)related. However, their theoretical and syntactic formalizations might coincide to some extent.
\end{enumerate}



Hence, it is crucial to consider the temporal aspect of scientific progress, which is often taken for granted in historical contexts. This is particularly important when civilizations with vastly different backgrounds, scientific and technological capabilities confront each other.

\section{Spread of scientific and technologic advancements}

One important consequence of the rapid pace of scientific research is that it can lead to incommensurability and incomprehensibility
between civilizations with different levels of technological development.
When individuals or groups are engaged in a particular research program,
they may not be able to understand or reconstruct phenomena and technology associated with a program more than one step ahead.
This means that if they encounter a technology that is based on a scientific revolution that they have not experienced yet, they will likely fail to comprehend its principles and mechanisms. As a result, attempting to understand advanced technology from a civilization that is more than one scientific revolution ahead may prove to be futile.

Consequently, the spread of scientific and technological advancements may become insurmountable from the point of view of less advanced civilizations.
Such technology would appear incomprehensible, mysterious, or even magical~\cite{Clarke2000Jan} to them.

The above statement is only accurate in the context of a closed system,
where civilizations have no means of accessing outside sources of information
about advanced technology.
In such a closed system, civilizations would not have access to some ``alien Prometheus''
who could provide them with a detailed explanation of the principles and mechanisms underlying advanced technology, such as the ``crashed phenotype.''

However, if civilizations have access to outside intelligence resources, the situation may be different. In this case, they could potentially leverage the knowledge and expertise of more advanced civilizations to better understand advanced technology.
This may involve collaboration, knowledge-acquisision, or even reverse engineering of existing technology.

While knowledge-sharing and collaboration between civilizations could potentially help bridge the gap in scientific development, the motivation for technology transfer is a crucial factor to consider. It is essential to question what motivates more advanced civilizations to share their knowledge and technology with less advanced ones.

Some potential motivations could include economic gain, political influence, or even altruistic motives such as a desire to improve the well-being of all humanity. In some cases, technology transfer could also be motivated by a desire to build strategic partnerships and alliances.

However, it is also important to acknowledge that technology transfer can have negative consequences. It may lead to the loss of intellectual property and market advantages for the more advanced civilization. It could also result in the destabilization of social and economic systems in the less advanced civilization, or the potential for the technology to be misused or weaponized.

We need to ask ourselves: why should ``These Others'' communicate with ``Us?''
To quote a passage from Charles Fort's ``The Book of the Damned''~\cite{FortBotD},
``Would we, if we could, educate and sophisticate pigs, geese, cattle?
Would it be wise to establish diplomatic relation with the hen that now functions, satisfied with mere sense of achievement by way of compensation?
I think we're property.''

\section{The delusion of  expert competence}


It is possible that relying solely on contemporary experts, such as theoretical physicists or rocket scientists, may not always be the most effective approach in understanding complex phenomena. These experts may be biased and invested in their respective fields, leading them to emphasize their current thinking and potentially overlook alternative perspectives or hypotheses.

In the context of Reich's Segmental Armouring Theory, experts may carry and apply their current expertise like a vendor's tray, surrounded by an impenetrable armor of alleged wisdom. This can make it difficult for them to consider alternative viewpoints or to recognize the limitations of their own understanding.

When it comes to evaluating very advanced technology or progressive research programs, the thinking and belief systems of contemporary experts may be inappropriate or even distracting. In some cases, their understanding may be outrightly wrong, leading to missed opportunities and wasted costs. To illustrate this point, consider asking a shaman medicine man of Borneo to explain a WWII airplane flying overhead. The shaman's belief system and understanding of the world may not align with modern physics and technology, leading to an inaccurate or incomplete explanation.



\section{Overcoming the explanation trap by suspended attention}

The ``mind projection fallacy'' or ``explanation trap'' is a cognitive error that occurs
when individuals assume that their perceptions of the world reflect objective reality.
This fallacy arises when someone projects their own mental constructs or subjective experiences onto the external world,
as if these constructs were objective features of the world. This can lead to confusion and misunderstandings in various fields of science and philosophy.

The term ``mind projection fallacy'' was first coined by Edwin Thompson Jaynes~\cite{jaynes-90}, a physicist and Bayesian philosopher,
who argued that this fallacy is particularly prevalent in the study of probability and statistics.
Jaynes pointed out that individuals often assume that randomness and disorder are objective features of the world, rather than subjective judgments based on their own lack of knowledge.

However, the mind projection fallacy is not limited to probability and statistics; it can also occur in other areas, such as philosophy, psychology, and neuroscience. For example, individuals might assume that their own subjective experiences of consciousness or free will are universal features of human experience, without considering the possibility that these experiences are influenced by cultural, historical, or individual factors.

To avoid the mind projection fallacy, I propose adopting an analytical approach called ``evenly-suspended attention''~\cite{Freud-itp,Freud-itpe}.
This involves observing phenomena without projecting one's own mental constructs or biases onto them. By suspending judgment and preconceived ideas,
individuals can allow their observations to settle in and reduce the influence of their own imagination or ignorance on their perception of reality.

Practicing evenly-suspended attention, which involves being fully attentive to whatever may occur instead of projecting expectations onto a situation, can be challenging, as it demands a high degree of mindfulness and self-awareness. Nevertheless, it can be an invaluable tool for enhancing critical thinking and avoiding cognitive biases. By recognizing the limitations of their own knowledge and perceptions, receptiveness can cultivate receptiveness to new possibilities and gain a more nuanced understanding of the world.

\section{Could we hack us into their database?}

The idea of accessing and penetrating an alien computer system, specifically the information storage and handling facility in a recovered craft, may seem like a viable option for obtaining information about their technology, means of propulsion, and energy source. However, given the vast technological differences between our civilizations, attempting to hack into their systems may prove fruitless.

To identify potential storage facilities in a recovered flying saucer, we would first need to determine its hardware.
It is likely that any such storage would be highly advanced and thus have a complex material layer with a high information density stored per unit of material. This would make any memory device appear irregular and randomly aligned, similar to white noise.
However, this paradoxical feature of highly organized, dense coding also makes it difficult to decipher.

Therefore, we would need to look for areas of the craft that appear structurally irregular and chaotic to locate potential storage devices.
We could also use advanced scanning techniques to analyze the material at an atomic level and identify any areas of high information density.
Overall, a multi-faceted approach combining structure analysis of the material with algorithmic pattern analysis would be necessary
to successfully identify and cryptanalyze storage facilities in a recovered flying saucer.

As a side note I would like to point out that a conventional search for extraterrestrial intelligence (SETI) may not be effective
in detecting advanced civilizations that have moved beyond using narrowband radio signals.
As technology progresses and broadcasts become digitized,
the emitted signals tend to approach white noise, making them much more difficult to analyze and decipher.
Therefore, alternative methods, such as the identification and analysis of recovered alien technology,
 may be necessary to obtain information about advanced civilizations.


\section{Similarities and differences of processing UAP and NAZI technology in the USA}


During World War II, Nazi Germany pursued advanced technology, which included the development of jet engines and rocketry. The US government was quick to study and reverse engineer these technologies after the war ended, with the help of Operation Paperclip~\cite{Jacobsen2014}. This program allowed the US to recruit and employ German scientists who had worked on these technologies, in order to gain an advantage in the Cold War against the Soviet Union.

Given this history of studying and reverse engineering foreign technologies, it is not totally unreasonable to speculate that, if crashed ``flying saucers'' were recovered, then  the US government may have attempted to ``process'' such (presumably alien) artifacts similarly to how they studied Nazi and other Earth-bound technologies. If such artifacts did exist, the US government would likely view them as a potential source of technological advantage over other countries.

The distinction between retrieved ``alien'' crafts and other foreign technologies could be twofold: first, as per Clarke's third law, the technology operating such a craft would likely appear ``magical'' and therefore beyond comprehension. Second, there may be no alien Prometheus'' equivalent to Wernher von Braun who could provide an explanation of the technology to humans.


%Acknowledgements: This paper was post-processed by OpenAI's ChatGTP-3.5, as well as Microsoft Bing (GTP-4).
\bibliography{svozil,ufo}
\ifws

\bibliographystyle{spmpsci}

\else
 \bibliographystyle{apsrev}

\fi

\end{document}
