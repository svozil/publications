\documentclass[prl,preprint,amsfonts,showpacs,showkeys]{revtex4}
%\documentclass[pra,showpacs,showkeys,amsfonts]{revtex4}
\usepackage{graphicx}
%\documentstyle[amsfonts]{article}
\RequirePackage{times}
%\RequirePackage{courier}
\RequirePackage{mathptm}
%\renewcommand{\baselinestretch}{1.3}

\begin{document}
%\sloppy



\title{Microphysical analogues of flyby anomalies}


\author{Karl Svozil}
\email{svozil@tuwien.ac.at}
\homepage{http://tph.tuwien.ac.at/~svozil}
\affiliation{Institute for Theoretical Physics, Vienna University of Technology,
Wiedner Hauptstra\ss e 8-10/136, A-1040 Vienna, Austria}


\begin{abstract}
We discuss Doppler shift and interferometric measurements in analogy to the recently reported macroscopic flyby anomalies.
\end{abstract}

\pacs{04.80.Cc,03.65.Nk}
\keywords{flyby anomaly, dynamical anomaly, interference}




\maketitle

In what follows we propose microscopic analogues of macroscopic flyby configurations for which an unexpected frequency increase
in the post encounter radio Doppler data have been reported \cite{anderson:091102,anderson:newast,Dittus}.
We first start with a straightforward discussion of flyby configurations and later concentrate on interferometric setups.
In a first approach, consider a beam of neutral particles, such as neutrons or photons, being deflected by a heavy rotating (magnetic) object.
The flyby should be along a path parallel to the rotation axis.
%This configuration is depicted in Fig.~\ref{2008-flyby-f-simple_flyby}.
%\begin{figure}
%\centering
%%TeXCAD Picture [1.pic]. Options:
%%\grade{\on}
%%\emlines{\off}
%%\epic{\off}
%%\beziermacro{\on}
%%\reduce{\on}
%%\snapping{\off}
%%\pvinsert{% Your \input, \def, etc. here}
%%\quality{8.000}
%%\graddiff{0.005}
%%\snapasp{1}
%%\zoom{26.9087}
%\unitlength 2mm % = 2.845pt
%\linethickness{0.5pt}
%\ifx\plotpoint\undefined\newsavebox{\plotpoint}\fi % GNUPLOT compatibility
%\begin{picture}(17,40)(0,0)
%%\end
%\put(10,20){\circle{14}}
%%\dashline{1}(10,10)(10,30)
%\put(9.93,9.93){\line(0,1){.9524}}
%\put(9.93,11.834){\line(0,1){.9524}}
%\put(9.93,13.739){\line(0,1){.9524}}
%\put(9.93,15.644){\line(0,1){.9524}}
%\put(9.93,17.549){\line(0,1){.9524}}
%\put(9.93,19.454){\line(0,1){.9524}}
%\put(9.93,21.358){\line(0,1){.9524}}
%\put(9.93,23.263){\line(0,1){.9524}}
%\put(9.93,25.168){\line(0,1){.9524}}
%\put(9.93,27.073){\line(0,1){.9524}}
%\put(9.93,28.977){\line(0,1){.9524}}
%%\end
%\qbezier(7.172,28.727)(6.281,27.723)(9.997,27.612)
%\qbezier(12.821,28.727)(13.713,27.723)(9.997,27.612)
%\qbezier(9.997,29.507)(7.916,29.526)(7.172,28.727)
%\qbezier(12.338,29.173)(11.948,29.451)(9.997,29.507)
%%\vector(11,29.47)(12.264,29.21)
%\put(12.264,29.21){\vector(4,-1){.07}}\multiput(11,29.47)(.157941,-.032517){8}{\line(1,0){.157941}}
%%\end
%%\circle*(9.997,20.031){13.919}
%\put(5.0195,15.0535){\rule{9.9545\unitlength}{9.9545\unitlength}}
%\multiput(7.2773,24.8955)(0,-11.3506){2}{\rule{5.4389\unitlength}{1.6211\unitlength}}
%\multiput(8.5828,26.4041)(0,-13.2552){2}{\rule{2.8279\unitlength}{.5085\unitlength}}
%\multiput(9.2584,26.8001)(0,-13.7515){2}{\rule{1.4767\unitlength}{.2127\unitlength}}
%\multiput(10.6227,26.8001)(-1.7033,0){2}{\multiput(0,0)(0,-13.7097){2}{\rule{.4515\unitlength}{.1709\unitlength}}}
%\multiput(11.2982,26.4041)(-3.3779,0){2}{\multiput(0,0)(0,-13.0893){2}{\rule{.775\unitlength}{.3425\unitlength}}}
%\multiput(11.2982,26.6342)(-3.0487,0){2}{\multiput(0,0)(0,-13.4105){2}{\rule{.4457\unitlength}{.2036\unitlength}}}
%\multiput(11.9607,26.4041)(-4.3647,0){2}{\multiput(0,0)(0,-12.9821){2}{\rule{.4368\unitlength}{.2354\unitlength}}}
%\multiput(12.2851,26.4041)(-4.8492,0){2}{\multiput(0,0)(0,-12.9226){2}{\rule{.2726\unitlength}{.1759\unitlength}}}
%\multiput(12.6038,24.8955)(-6.5296,0){2}{\multiput(0,0)(0,-10.7075){2}{\rule{1.3156\unitlength}{.978\unitlength}}}
%\multiput(12.6038,25.761)(-5.9438,0){2}{\multiput(0,0)(0,-11.9241){2}{\rule{.7299\unitlength}{.4636\unitlength}}}
%\multiput(12.6038,26.1121)(-5.6387,0){2}{\multiput(0,0)(0,-12.4288){2}{\rule{.4247\unitlength}{.266\unitlength}}}
%\multiput(12.6038,26.2656)(-5.4834,0){2}{\multiput(0,0)(0,-12.6535){2}{\rule{.2695\unitlength}{.1836\unitlength}}}
%\multiput(12.916,26.1121)(-6.1045,0){2}{\multiput(0,0)(0,-12.3539){2}{\rule{.266\unitlength}{.1911\unitlength}}}
%\multiput(13.2212,25.761)(-6.8584,0){2}{\multiput(0,0)(0,-11.7557){2}{\rule{.4097\unitlength}{.2952\unitlength}}}
%\multiput(13.2212,25.9437)(-6.7109,0){2}{\multiput(0,0)(0,-12.0245){2}{\rule{.2621\unitlength}{.1985\unitlength}}}
%\multiput(13.5184,25.761)(-7.3011,0){2}{\multiput(0,0)(0,-11.6661){2}{\rule{.2579\unitlength}{.2056\unitlength}}}
%\multiput(13.8069,24.8955)(-8.2814,0){2}{\multiput(0,0)(0,-10.3007){2}{\rule{.661\unitlength}{.5711\unitlength}}}
%\multiput(13.8069,25.3541)(-8.0121,0){2}{\multiput(0,0)(0,-10.9695){2}{\rule{.3917\unitlength}{.3226\unitlength}}}
%\multiput(13.8069,25.5643)(-7.8737,0){2}{\multiput(0,0)(0,-11.2797){2}{\rule{.2533\unitlength}{.2125\unitlength}}}
%\multiput(14.0862,25.3541)(-8.4273,0){2}{\multiput(0,0)(0,-10.866){2}{\rule{.2484\unitlength}{.2192\unitlength}}}
%\multiput(14.3555,24.8955)(-9.0886,0){2}{\multiput(0,0)(0,-10.0775){2}{\rule{.3711\unitlength}{.348\unitlength}}}
%\multiput(14.3555,25.131)(-8.9606,0){2}{\multiput(0,0)(0,-10.4262){2}{\rule{.2432\unitlength}{.2256\unitlength}}}
%\multiput(14.4862,25.131)(-9.1556,0){2}{\multiput(0,0)(0,-10.37){2}{\rule{.1768\unitlength}{.1694\unitlength}}}
%\multiput(14.6142,24.8955)(-9.4724,0){2}{\multiput(0,0)(0,-9.9613){2}{\rule{.2376\unitlength}{.2318\unitlength}}}
%\multiput(14.6142,25.0148)(-9.4102,0){2}{\multiput(0,0)(0,-10.1391){2}{\rule{.1754\unitlength}{.171\unitlength}}}
%\multiput(14.7393,24.8955)(-9.659,0){2}{\multiput(0,0)(0,-9.902){2}{\rule{.174\unitlength}{.1725\unitlength}}}
%\multiput(14.8615,17.3112)(-11.3506,0){2}{\rule{1.6211\unitlength}{5.4389\unitlength}}
%\multiput(14.8615,22.6377)(-10.7075,0){2}{\multiput(0,0)(0,-6.5296){2}{\rule{.978\unitlength}{1.3156\unitlength}}}
%\multiput(14.8615,23.8409)(-10.3007,0){2}{\multiput(0,0)(0,-8.2814){2}{\rule{.5711\unitlength}{.661\unitlength}}}
%\multiput(14.8615,24.3894)(-10.0775,0){2}{\multiput(0,0)(0,-9.0886){2}{\rule{.348\unitlength}{.3711\unitlength}}}
%\multiput(14.8615,24.6481)(-9.9613,0){2}{\multiput(0,0)(0,-9.4724){2}{\rule{.2318\unitlength}{.2376\unitlength}}}
%\multiput(14.8615,24.7732)(-9.902,0){2}{\multiput(0,0)(0,-9.659){2}{\rule{.1725\unitlength}{.174\unitlength}}}
%\multiput(14.9808,24.6481)(-10.1391,0){2}{\multiput(0,0)(0,-9.4102){2}{\rule{.171\unitlength}{.1754\unitlength}}}
%\multiput(15.0971,24.3894)(-10.4262,0){2}{\multiput(0,0)(0,-8.9606){2}{\rule{.2256\unitlength}{.2432\unitlength}}}
%\multiput(15.0971,24.5201)(-10.37,0){2}{\multiput(0,0)(0,-9.1556){2}{\rule{.1694\unitlength}{.1768\unitlength}}}
%\multiput(15.3202,23.8409)(-10.9695,0){2}{\multiput(0,0)(0,-8.0121){2}{\rule{.3226\unitlength}{.3917\unitlength}}}
%\multiput(15.3202,24.1201)(-10.866,0){2}{\multiput(0,0)(0,-8.4273){2}{\rule{.2192\unitlength}{.2484\unitlength}}}
%\multiput(15.5303,23.8409)(-11.2797,0){2}{\multiput(0,0)(0,-7.8737){2}{\rule{.2125\unitlength}{.2533\unitlength}}}
%\multiput(15.727,22.6377)(-11.9241,0){2}{\multiput(0,0)(0,-5.9438){2}{\rule{.4636\unitlength}{.7299\unitlength}}}
%\multiput(15.727,23.2551)(-11.7557,0){2}{\multiput(0,0)(0,-6.8584){2}{\rule{.2952\unitlength}{.4097\unitlength}}}
%\multiput(15.727,23.5523)(-11.6661,0){2}{\multiput(0,0)(0,-7.3011){2}{\rule{.2056\unitlength}{.2579\unitlength}}}
%\multiput(15.9098,23.2551)(-12.0245,0){2}{\multiput(0,0)(0,-6.7109){2}{\rule{.1985\unitlength}{.2621\unitlength}}}
%\multiput(16.0781,22.6377)(-12.4288,0){2}{\multiput(0,0)(0,-5.6387){2}{\rule{.266\unitlength}{.4247\unitlength}}}
%\multiput(16.0781,22.95)(-12.3539,0){2}{\multiput(0,0)(0,-6.1045){2}{\rule{.1911\unitlength}{.266\unitlength}}}
%\multiput(16.2317,22.6377)(-12.6535,0){2}{\multiput(0,0)(0,-5.4834){2}{\rule{.1836\unitlength}{.2695\unitlength}}}
%\multiput(16.3701,18.6168)(-13.2552,0){2}{\rule{.5085\unitlength}{2.8279\unitlength}}
%\multiput(16.3701,21.3322)(-13.0893,0){2}{\multiput(0,0)(0,-3.3779){2}{\rule{.3425\unitlength}{.775\unitlength}}}
%\multiput(16.3701,21.9947)(-12.9821,0){2}{\multiput(0,0)(0,-4.3647){2}{\rule{.2354\unitlength}{.4368\unitlength}}}
%\multiput(16.3701,22.319)(-12.9226,0){2}{\multiput(0,0)(0,-4.8492){2}{\rule{.1759\unitlength}{.2726\unitlength}}}
%\multiput(16.6002,21.3322)(-13.4105,0){2}{\multiput(0,0)(0,-3.0487){2}{\rule{.2036\unitlength}{.4457\unitlength}}}
%\multiput(16.7662,19.2923)(-13.7515,0){2}{\rule{.2127\unitlength}{1.4767\unitlength}}
%\multiput(16.7662,20.6566)(-13.7097,0){2}{\multiput(0,0)(0,-1.7033){2}{\rule{.1709\unitlength}{.4515\unitlength}}}
%\put(16.956,20.031){\line(0,1){.3976}}
%\put(16.945,20.428){\line(0,1){.3963}}
%\put(16.911,20.825){\line(0,1){.3937}}
%\put(16.854,21.218){\line(0,1){.3898}}
%\put(16.775,21.608){\line(0,1){.3847}}
%\multiput(16.674,21.993)(-.03075,.09457){4}{\line(0,1){.09457}}
%\multiput(16.551,22.371)(-.028882,.074123){5}{\line(0,1){.074123}}
%\multiput(16.406,22.742)(-.03307,.072352){5}{\line(0,1){.072352}}
%\multiput(16.241,23.103)(-.030958,.05862){6}{\line(0,1){.05862}}
%\multiput(16.055,23.455)(-.029363,.048648){7}{\line(0,1){.048648}}
%\multiput(15.85,23.796)(-.032094,.046891){7}{\line(0,1){.046891}}
%\multiput(15.625,24.124)(-.03038,.039358){8}{\line(0,1){.039358}}
%\multiput(15.382,24.439)(-.032579,.037558){8}{\line(0,1){.037558}}
%\multiput(15.121,24.739)(-.030819,.031676){9}{\line(0,1){.031676}}
%\multiput(14.844,25.024)(-.036651,.033597){8}{\line(-1,0){.036651}}
%\multiput(14.551,25.293)(-.038511,.031448){8}{\line(-1,0){.038511}}
%\multiput(14.243,25.545)(-.045994,.033367){7}{\line(-1,0){.045994}}
%\multiput(13.921,25.778)(-.047825,.030685){7}{\line(-1,0){.047825}}
%\multiput(13.586,25.993)(-.05775,.032553){6}{\line(-1,0){.05775}}
%\multiput(13.24,26.188)(-.059515,.029201){6}{\line(-1,0){.059515}}
%\multiput(12.882,26.364)(-.073304,.030904){5}{\line(-1,0){.073304}}
%\multiput(12.516,26.518)(-.09369,.03333){4}{\line(-1,0){.09369}}
%\put(12.141,26.651){\line(-1,0){.3818}}
%\put(11.759,26.763){\line(-1,0){.3875}}
%\put(11.372,26.853){\line(-1,0){.392}}
%\put(10.98,26.92){\line(-1,0){.3952}}
%\put(10.585,26.965){\line(-1,0){.3971}}
%\put(10.188,26.987){\line(-1,0){.3978}}
%\put(9.79,26.987){\line(-1,0){.3971}}
%\put(9.393,26.964){\line(-1,0){.3951}}
%\put(8.998,26.918){\line(-1,0){.3918}}
%\put(8.606,26.85){\line(-1,0){.3873}}
%\put(8.218,26.759){\line(-1,0){.3815}}
%\multiput(7.837,26.646)(-.09361,-.03355){4}{\line(-1,0){.09361}}
%\multiput(7.463,26.512)(-.073232,-.031074){5}{\line(-1,0){.073232}}
%\multiput(7.096,26.357)(-.059447,-.029339){6}{\line(-1,0){.059447}}
%\multiput(6.74,26.181)(-.057674,-.032687){6}{\line(-1,0){.057674}}
%\multiput(6.394,25.985)(-.047753,-.030796){7}{\line(-1,0){.047753}}
%\multiput(6.059,25.769)(-.045916,-.033474){7}{\line(-1,0){.045916}}
%\multiput(5.738,25.535)(-.038437,-.031537){8}{\line(-1,0){.038437}}
%\multiput(5.43,25.283)(-.036573,-.033682){8}{\line(-1,0){.036573}}
%\multiput(5.138,25.013)(-.030746,-.031748){9}{\line(0,-1){.031748}}
%\multiput(4.861,24.727)(-.032492,-.037634){8}{\line(0,-1){.037634}}
%\multiput(4.601,24.426)(-.030289,-.039429){8}{\line(0,-1){.039429}}
%\multiput(4.359,24.111)(-.031985,-.046966){7}{\line(0,-1){.046966}}
%\multiput(4.135,23.782)(-.029249,-.048716){7}{\line(0,-1){.048716}}
%\multiput(3.93,23.441)(-.030821,-.058692){6}{\line(0,-1){.058692}}
%\multiput(3.745,23.089)(-.032902,-.072429){5}{\line(0,-1){.072429}}
%\multiput(3.581,22.727)(-.02871,-.07419){5}{\line(0,-1){.07419}}
%\multiput(3.437,22.356)(-.03053,-.09464){4}{\line(0,-1){.09464}}
%\put(3.315,21.977){\line(0,-1){.3849}}
%\put(3.215,21.592){\line(0,-1){.39}}
%\put(3.137,21.202){\line(0,-1){.3938}}
%\put(3.081,20.809){\line(0,-1){.3964}}
%\put(3.048,20.412){\line(0,-1){1.1914}}
%\put(3.085,19.221){\line(0,-1){.3936}}
%\put(3.142,18.827){\line(0,-1){.3896}}
%\put(3.222,18.438){\line(0,-1){.3844}}
%\multiput(3.324,18.053)(.03097,-.09449){4}{\line(0,-1){.09449}}
%\multiput(3.448,17.675)(.029054,-.074056){5}{\line(0,-1){.074056}}
%\multiput(3.593,17.305)(.033238,-.072275){5}{\line(0,-1){.072275}}
%\multiput(3.76,16.943)(.031094,-.058548){6}{\line(0,-1){.058548}}
%\multiput(3.946,16.592)(.029475,-.04858){7}{\line(0,-1){.04858}}
%\multiput(4.153,16.252)(.032203,-.046816){7}{\line(0,-1){.046816}}
%\multiput(4.378,15.924)(.030472,-.039288){8}{\line(0,-1){.039288}}
%\multiput(4.622,15.61)(.032666,-.037483){8}{\line(0,-1){.037483}}
%\multiput(4.883,15.31)(.030893,-.031605){9}{\line(0,-1){.031605}}
%\multiput(5.161,15.026)(.036729,-.033511){8}{\line(1,0){.036729}}
%\multiput(5.455,14.758)(.038584,-.031358){8}{\line(1,0){.038584}}
%\multiput(5.764,14.507)(.046071,-.03326){7}{\line(1,0){.046071}}
%\multiput(6.086,14.274)(.047896,-.030574){7}{\line(1,0){.047896}}
%\multiput(6.421,14.06)(.057825,-.032419){6}{\line(1,0){.057825}}
%\multiput(6.768,13.865)(.059583,-.029062){6}{\line(1,0){.059583}}
%\multiput(7.126,13.691)(.073375,-.030733){5}{\line(1,0){.073375}}
%\multiput(7.493,13.537)(.09376,-.03311){4}{\line(1,0){.09376}}
%\put(7.868,13.405){\line(1,0){.382}}
%\put(8.25,13.294){\line(1,0){.3877}}
%\put(8.637,13.205){\line(1,0){.3922}}
%\put(9.03,13.139){\line(1,0){.3953}}
%\put(9.425,13.095){\line(1,0){.3972}}
%\put(9.822,13.074){\line(1,0){.3978}}
%\put(10.22,13.075){\line(1,0){.397}}
%\put(10.617,13.099){\line(1,0){.395}}
%\put(11.012,13.146){\line(1,0){.3917}}
%\put(11.404,13.215){\line(1,0){.3871}}
%\put(11.791,13.307){\line(1,0){.3812}}
%\multiput(12.172,13.42)(.074825,.027013){5}{\line(1,0){.074825}}
%\multiput(12.546,13.555)(.073159,.031244){5}{\line(1,0){.073159}}
%\multiput(12.912,13.711)(.059379,.029477){6}{\line(1,0){.059379}}
%\multiput(13.268,13.888)(.057598,.032821){6}{\line(1,0){.057598}}
%\multiput(13.614,14.085)(.047682,.030907){7}{\line(1,0){.047682}}
%\multiput(13.947,14.301)(.045838,.033581){7}{\line(1,0){.045838}}
%\multiput(14.268,14.536)(.038364,.031627){8}{\line(1,0){.038364}}
%\multiput(14.575,14.789)(.03244,.030015){9}{\line(1,0){.03244}}
%\multiput(14.867,15.06)(.030672,.031819){9}{\line(0,1){.031819}}
%\multiput(15.143,15.346)(.032404,.037709){8}{\line(0,1){.037709}}
%\multiput(15.402,15.648)(.030197,.039499){8}{\line(0,1){.039499}}
%\multiput(15.644,15.964)(.031876,.04704){7}{\line(0,1){.04704}}
%\multiput(15.867,16.293)(.029136,.048784){7}{\line(0,1){.048784}}
%\multiput(16.071,16.634)(.030685,.058764){6}{\line(0,1){.058764}}
%\multiput(16.255,16.987)(.032733,.072505){5}{\line(0,1){.072505}}
%\multiput(16.419,17.35)(.028537,.074257){5}{\line(0,1){.074257}}
%\multiput(16.562,17.721)(.03031,.09471){4}{\line(0,1){.09471}}
%\put(16.683,18.1){\line(0,1){.3851}}
%\put(16.782,18.485){\line(0,1){.3902}}
%\put(16.859,18.875){\line(0,1){.394}}
%\put(16.914,19.269){\line(0,1){.7618}}
%%\end
%\thicklines
%%\vector[middle](1,0)(1,40)
%\put(1,20){\vector(0,1){.07}}\put(1,0){\line(0,1){40}}
%\end{picture}
%\caption{A particle beam flies by a heavy rotating (magnetic) object.
%The flyby should be along a path parallel to the rotation axis.}
%\label{2008-flyby-f-simple_flyby}
%\end{figure}

For the prediction of the frequency shift, we take formula~(2) of Ref.~\cite{anderson:091102}:
$
\Delta \omega = 2K\omega (\cos \delta_i -\cos \delta_o)
$,
with $\delta_i$ and $\delta_o$ as the declinations of the incoming and outgoing osculating asymptotic velocity vectors,
and $K= 2\Omega R /c$, where $\Omega$ is the angular rotational velocity and $R$ the mean radius of the heavy object. $c$ stands for the speed of light in vacuum.
In our case, as $\delta_i = -\delta_o = 1$, we obtain as an estimate for the frequency shift
$
\Delta \omega = {8\Omega R\over c}\;\omega
$.
One straightforward way to realize heavy (magnetic) rotating objects are stacks of data disks which operate at about 7200 revolutions per minute, corresponding to 120~Hz or 754~rad/s.
This is about seven orders of magnitude greater than Earth's angular rotation velocity.
Typical disks have a radius of about 0.05~m, which is about eight orders of magnitude smaller than Earth's mean radius.
Thus, in this case,
$
K= 2.513\times 10^{-7}
$
is only about a factor of ten smaller than for Earth.

Another possibility to measure possible frequency shifts of microscopic particles and light is by comparing the phases acquired along two different paths.
Consider, for the sake of demonstration, a Mach-Zehnder device depicted in Fig.~\ref{2008-flyby-f-Mach-Zehnder},
in which heavy rotating (magnetic) mass is placed along one beam path, whereas the other beam path is left unperturbed.
\begin{figure}
\centering
%TeXCAD Picture [1.pic]. Options:
%\grade{\off}
%\emlines{\off}
%\epic{\off}
%\beziermacro{\off}
%\reduce{\on}
%\snapping{\off}
%\quality{0.200}
%\graddiff{0.010}
%\snapasp{1}
%\zoom{11.3137}
\unitlength 1.2mm % = 1.992pt
\linethickness{0.4pt}
\ifx\plotpoint\undefined\newsavebox{\plotpoint}\fi % GNUPLOT compatibility
\begin{picture}(83.67,61)(0,0)
\put(62.67,40){\line(0,-1){25}}
\put(10,55){\makebox(0,0)[cc]{$L$}}
\put(10,55){\circle{10}}
\put(15,55){\line(1,0){40}}
\put(44.67,55){\line(1,0){18}}
\put(25,55){\line(0,-1){25}}
\put(25,30){\line(1,0){13}}
\put(62.67,55){\line(0,-1){25}}
\put(62.67,30){\line(-1,0){35}}
\put(62.67,30){\line(1,0){13}}
\put(62.67,30){\line(0,-1){13}}
%\emline(58.67,59)(66.67,51)
\multiput(58.67,59)(.048192771,-.048192771){166}{\line(0,-1){.048192771}}
%\end
\put(21,35){\line(4,-5){8}}
\put(80.17,30){\oval(7,8)[r]}
\put(83.67,36){\makebox(0,0)[cc]{$D_1$}}
\put(28,42){\makebox(0,0)[cc]{$c$}}
\put(25,61){\makebox(0,0)[cc]{$S_1$}}
\put(56.67,38){\makebox(0,0)[cc]{$S_2$}}
%\emline(24,56)(26,54)
\multiput(24,56)(.04761905,-.04761905){42}{\line(0,-1){.04761905}}
%\end
%\emline(23,57)(21,59)
\multiput(23,57)(-.04761905,.04761905){42}{\line(0,1){.04761905}}
%\end
%\emline(27,53)(29,51)
\multiput(27,53)(.04761905,-.04761905){42}{\line(0,-1){.04761905}}
%\end
%\emline(61.67,31)(63.67,29)
\multiput(61.67,31)(.04761905,-.04761905){42}{\line(0,-1){.04761905}}
%\end
%\emline(60.67,32)(58.67,34)
\multiput(60.67,32)(-.04761905,.04761905){42}{\line(0,1){.04761905}}
%\end
%\emline(64.67,28)(66.67,26)
\multiput(64.67,28)(.04761905,-.04761905){42}{\line(0,-1){.04761905}}
%\end
\put(18,51){\makebox(0,0)[cc]{$a$}}
\put(74.327,43){\makebox(0,0)[cc]{$\Delta \omega$}}
\put(64.327,43){\makebox(0,0)[cc]{$O$}}
\put(70.67,33){\makebox(0,0)[cc]{$d$}}
\put(62.67,13.33){\oval(8.67,8)[b]}
\put(70,9){\makebox(0,0)[cc]{$D_2$}}
\put(65.33,20.33){\makebox(0,0)[cc]{$e$}}
\put(43,61){\makebox(0,0)[cc]{$b$}}
\put(62,61){\makebox(0,0)[cc]{$M$}}
\put(27.33,21){\makebox(0,0)[cc]{$M$}}
\put(68.324,43.31){\circle*{5.127}}
\put(68.236,39.598){\line(0,1){8.662}}
%\qbezier(65.332,48.084)(64.441,47.08)(68.157,46.969)
\put(65.332,48.084){\line(0,-1){.0484}}
\put(65.291,48.036){\line(0,-1){.0474}}
\put(65.256,47.988){\line(0,-1){.0463}}
\put(65.226,47.942){\line(0,-1){.0453}}
\put(65.202,47.897){\line(0,-1){.0442}}
\put(65.183,47.852){\line(0,-1){.2796}}
\put(65.205,47.573){\line(0,-1){.0357}}
\put(65.23,47.537){\line(0,-1){.0346}}
\put(65.26,47.503){\line(1,0){.036}}
\put(65.296,47.469){\line(1,0){.0415}}
\put(65.338,47.436){\line(1,0){.047}}
\put(65.385,47.405){\line(1,0){.0525}}
\put(65.437,47.375){\line(1,0){.0579}}
\put(65.495,47.345){\line(1,0){.0634}}
\put(65.559,47.317){\line(1,0){.0689}}
\put(65.627,47.29){\line(1,0){.0744}}
\put(65.702,47.264){\line(1,0){.0799}}
\put(65.782,47.239){\line(1,0){.0853}}
\put(65.867,47.215){\line(1,0){.0908}}
\put(65.958,47.192){\line(1,0){.0963}}
\put(66.054,47.17){\line(1,0){.1018}}
\put(66.156,47.149){\line(1,0){.1073}}
\put(66.263,47.129){\line(1,0){.1128}}
\put(66.376,47.111){\line(1,0){.1182}}
\put(66.494,47.093){\line(1,0){.1237}}
\put(66.618,47.076){\line(1,0){.1292}}
\put(66.747,47.061){\line(1,0){.1347}}
\put(66.882,47.046){\line(1,0){.1402}}
\put(67.022,47.033){\line(1,0){.1456}}
\put(67.168,47.021){\line(1,0){.3077}}
\put(67.475,46.999){\line(1,0){.3296}}
\put(67.805,46.982){\line(1,0){.3516}}
%\end
%\qbezier(70.981,48.084)(71.873,47.08)(68.157,46.969)
\put(70.981,48.084){\line(0,-1){.0484}}
\put(71.021,48.036){\line(0,-1){.0474}}
\put(71.057,47.988){\line(0,-1){.0463}}
\put(71.086,47.942){\line(0,-1){.0453}}
\put(71.111,47.897){\line(0,-1){.0442}}
\put(71.13,47.852){\line(0,-1){.2796}}
\put(71.108,47.573){\line(0,-1){.0357}}
\put(71.083,47.537){\line(0,-1){.0346}}
\put(71.052,47.503){\line(-1,0){.036}}
\put(71.016,47.469){\line(-1,0){.0415}}
\put(70.975,47.436){\line(-1,0){.0469}}
\put(70.928,47.405){\line(-1,0){.0524}}
\put(70.876,47.375){\line(-1,0){.0579}}
\put(70.818,47.345){\line(-1,0){.0634}}
\put(70.754,47.317){\line(-1,0){.0689}}
\put(70.685,47.29){\line(-1,0){.0744}}
\put(70.611,47.264){\line(-1,0){.0798}}
\put(70.531,47.239){\line(-1,0){.0853}}
\put(70.446,47.215){\line(-1,0){.0908}}
\put(70.355,47.192){\line(-1,0){.0963}}
\put(70.259,47.17){\line(-1,0){.1018}}
\put(70.157,47.149){\line(-1,0){.1073}}
\put(70.05,47.129){\line(-1,0){.1127}}
\put(69.937,47.111){\line(-1,0){.1182}}
\put(69.819,47.093){\line(-1,0){.1237}}
\put(69.695,47.076){\line(-1,0){.1292}}
\put(69.566,47.061){\line(-1,0){.1347}}
\put(69.431,47.046){\line(-1,0){.1401}}
\put(69.291,47.033){\line(-1,0){.1456}}
\put(69.146,47.021){\line(-1,0){.3077}}
\put(68.838,46.999){\line(-1,0){.3296}}
\put(68.508,46.982){\line(-1,0){.3516}}
%\end
%\qbezier(68.157,48.864)(66.076,48.883)(65.332,48.084)
\put(68.157,48.864){\line(-1,0){.4624}}
\put(67.694,48.858){\line(-1,0){.1462}}
\put(67.548,48.851){\line(-1,0){.1423}}
\put(67.406,48.841){\line(-1,0){.1383}}
\put(67.267,48.829){\line(-1,0){.1344}}
\put(67.133,48.815){\line(-1,0){.1304}}
\put(67.003,48.798){\line(-1,0){.1265}}
\put(66.876,48.779){\line(-1,0){.1225}}
\put(66.754,48.758){\line(-1,0){.1185}}
\put(66.635,48.734){\line(-1,0){.1146}}
\put(66.52,48.707){\line(-1,0){.1106}}
\put(66.41,48.679){\line(-1,0){.1067}}
\put(66.303,48.647){\line(-1,0){.1027}}
\put(66.2,48.614){\line(-1,0){.0988}}
\put(66.102,48.578){\line(-1,0){.0948}}
\put(66.007,48.539){\line(-1,0){.0909}}
\put(65.916,48.498){\line(-1,0){.0869}}
\put(65.829,48.455){\line(-1,0){.0829}}
\put(65.746,48.409){\line(-1,0){.079}}
\put(65.667,48.361){\line(-1,0){.075}}
\put(65.592,48.311){\line(-1,0){.0711}}
\put(65.521,48.258){\line(-1,0){.0671}}
\put(65.454,48.202){\line(-1,0){.0632}}
\put(65.391,48.144){\line(0,-1){.0603}}
%\end
%\qbezier(70.498,48.53)(70.108,48.808)(68.157,48.864)
\put(70.498,48.53){\line(-1,0){.0429}}
\put(70.455,48.557){\line(-1,0){.0507}}
\put(70.404,48.583){\line(-1,0){.0585}}
\put(70.345,48.608){\line(-1,0){.0663}}
\put(70.279,48.632){\line(-1,0){.0741}}
\put(70.205,48.655){\line(-1,0){.0819}}
\put(70.123,48.677){\line(-1,0){.0897}}
\put(70.033,48.697){\line(-1,0){.0975}}
\put(69.936,48.717){\line(-1,0){.1053}}
\put(69.83,48.735){\line(-1,0){.1131}}
\put(69.717,48.753){\line(-1,0){.121}}
\put(69.596,48.769){\line(-1,0){.1288}}
\put(69.468,48.784){\line(-1,0){.1366}}
\put(69.331,48.798){\line(-1,0){.1444}}
\put(69.187,48.81){\line(-1,0){.1522}}
\put(69.035,48.822){\line(-1,0){.3278}}
\put(68.707,48.842){\line(-1,0){.359}}
\put(68.348,48.858){\line(-1,0){.1912}}
%\end
%\vector(69.16,48.827)(70.424,48.567)
\put(70.424,48.567){\vector(4,-1){.1}}\multiput(69.16,48.827)(.210667,-.043333){6}{\line(1,0){.210667}}
%\end
\end{picture}
\caption{Mach-Zehnder interferometer, in which one beam path is parallel to the rotation axis of a heavy rotating (magnetic) object $O$.}
\label{2008-flyby-f-Mach-Zehnder}
\end{figure}
If all components are lossless, then the state transitions can be written as  \cite{green-horn-zei,svozil-2004-analog}
\begin{equation}
\begin{array}{rcl}
S_1:\; a  &\rightarrow& ( b  +i c)/\sqrt{2}\quad , \\
O:\; b  &\rightarrow&  b e^{i \Delta \omega }\quad ,\\
S_2:\; b  &\rightarrow& ( e  + id )/\sqrt{2}\quad ,\\
S_2:\; c  &\rightarrow& ( d  + ie )/\sqrt{2}\quad .
\end{array}
\end{equation}
The resulting transition is
\begin{equation}
  a  \rightarrow \psi =i\left( {e^{i\Delta \omega} +1\over
2}\right)
d  +
\left( {e^{i\Delta \omega} -1\over 2}\right)
e  \quad .
\label{e:mz}
\end{equation}
The phase shift $\Delta \omega$ induces nonzero detection probabilities for $D_2$:
\begin{equation}
P_{D_2}(\Delta \omega )= 1-P_{D_1}(\Delta \omega )=\vert ( e, \psi ) \vert^2=\sin^2({\Delta \omega \over 2})=\sin^2({{4\Omega R\over c}\;\omega }).
\end{equation}
Taking the estimate for the hard disk configuration as before, the frequency shifts are of the order of $5 \times 10^{-7}$ times the frequency of light or of the associated matter waves.


In summary, we have suggested to test the flyby of microscopic particles along heavy (magnetic) rotating objects whose rotation axis are parallel to the beam path.
Another intereferometric test would involve Michelson-Morley type interferometers \cite{cahill-2008}.
Because of their sensitivity,  interferometric setups appear to be particularly promising for terrestrial tests of flyby anomalies.

%\bibliography{svozil}
%\bibliographystyle{osa}
%\bibliographystyle{apsrev}
%\bibliographystyle{unsrt}
%\bibliographystyle{plain}


\begin{thebibliography}{1}
\newcommand{\enquote}[1]{``#1''}
\expandafter\ifx\csname url\endcsname\relax
  \def\url#1{{#1}}\fi
\expandafter\ifx\csname urlprefix\endcsname\relax\def\urlprefix{}\fi

\bibitem{anderson:091102}
J.~D. Anderson, J.~K. Campbell, J.~E. Ekelund, J.~Ellis, and J.~F. Jordan,
  \enquote{Anomalous Orbital-Energy Changes Observed during Spacecraft Flybys
  of Earth,} Physical Review Letters {\bf 100}, 091\,102 (2008).
\newline http://dx.doi.org/10.1103/PhysRevLett.100.091102

\bibitem{anderson:newast}
J.~D. Anderson, J.~K. Campbell, and M.~M. Nieto, \enquote{The energy transfer
  process in planetary flybys,} New Astronomy {\bf 12}, 383--397 (2007).
\newline http://dx.doi.org/10.1016/j.newast.2006.11.004

\bibitem{Dittus}
H.~Dittus, C.~L{\"{a}}mmerzahl, and S.~G. Turyshev, {\em Lasers, Clocks and
  Drag-Free Control Exploration of Relativistic Gravity in Space\/},
  Astrophysics and Space Science Library, Vol. 349 (Springer, Heidelberg,
  2006).

\bibitem{green-horn-zei}
D.~M. Greenberger, M.~A. Horne, and A.~Zeilinger, \enquote{Multiparticle
  interferometry and the superposition principle,} Physics Today {\bf 46},
  22--29 (1993).

\bibitem{svozil-2004-analog}
K.~Svozil, \enquote{Noncontextuality in multipartite entanglement,} J. Phys. A:
  Math. Gen. {\bf 38}, 5781--5798 (2005).
\newline http://dx.doi.org/10.1088/0305-4470/38/25/013

\bibitem{cahill-2008}
R.~T. Cahill, \enquote{Resolving Spacecraft Earth-Flyby Anomalies with Measured
  Light Speed Anisotropy,}  (2008).
\newline http://arxiv.org/abs/0804.0039

\end{thebibliography}

\end{document}
