\documentclass[pra,amsfonts,showpacs,preprint,showkeys]{revtex4}
%\documentclass[pra,showpacs,showkeys,amsfonts]{revtex4}
\usepackage[T1]{fontenc}
\usepackage{graphicx}

%\documentclass[12pt,a4paper]{article}
%\usepackage[english, USenglish]{babel}
\usepackage{amsmath}
%\linespread{1.3}
%\usepackage{amsfonts}
%\usepackage[dvips]{graphicx}
%\usepackage{hyperref}
%\usepackage{a4wide}
%\usepackage{rotating}
\usepackage{longtable}
%\usepackage{lscape}
\begin{document}
\title{All the singlet states}

\author{Peter Kasperkowitz}
\author{Maria Schimpf}
\author{Karl Svozil}
\affiliation{Institute of Theoretical Physics, Vienna
  University of Technology, Wiedner Hauptstra\ss e 8-10/136, A-1040
    Vienna, Austria}
\email{svozil@tuwien.ac.at}


\begin{abstract}
Singlet states of multiple particles have the property that they
are form invariant with respect to overall changes of measurement
direction. We present a group theoretic method to construct all
$N$-particle singlet states by recursion and iteration.
\end{abstract}

\pacs{03.67.-a,02.20.-a,03.65.Ca }

\keywords{Quantum information, singlet states, group theory}

\maketitle

%\section{Introduction}

Singlet states are among the most useful states in quantum
mechanics; yet their explicit structure---although well understood
in general terms in group theory---has up to now neither been
enumerated nor investigated beyond a few instances for
spin-${1\over 2}$ and spin-$1$ particles. Recent theoretical and
experimental studies in multipartite production (e.g.,
Ref.~\cite{egbkzw}) elicit that a more systematic way to generate
the complete set of arbitrary $N$-particle singlet states seems to
be desirable.

In the present study we pursue an algorithmic generation strategy,
and tabulate some of the first singlet states. The recursive
method employed is based on the triangle relations and
Clebsch-Gordan coefficients (e.g., Ch.~13, Sec.~27 of
Ref.~\cite{messiah-62}). With this approach a complete table of
all angular momentum states is created. The singlet states stem
from the various pathways towards the $j= m=0$ states. The
procedure can best be illustrated in a triangular diagram where
the states in ascending order of $J$ are drawn against the number
of particles. In such a diagram, the ``lowest'' states correspond
to singlets.



%\subsection{Description of the algorithm for obtaining spin-${1\over 2}$ and spin-$1$ $N$-particle singlet states}

There always exist ``zigzag'' singlet states, which are
the product of $r$ two-particle singlet states stemming from the
rising and lowering of consecutive states. The situation is
depicted in Fig.~\ref{2005-singlet-f1-zigzag}. For $J=1$ and
$N=3r$ there exist ``zigzag'' singlet states, which are the
product of $r$ three-particle singlet states. For singlet states
with $N=2r+3t$ ($r,t$ integer) there exist singlet states being
the product of $r$
two-particle singlet states and $t$ three-particle singlet states.
\begin{figure}
\begin{center}
%TexCad Options
%\grade{\off}
%\emlines{\off}
%\beziermacro{\on}
%\reduce{\on}
%\snapping{\off}
%\quality{2.00}
%\graddiff{0.01}
%\snapasp{1}
%\zoom{0.80}
\unitlength 0.40mm \linethickness{0.4pt}
\begin{picture}(305.33,150.00)
\put(15.00,5.00){\makebox(0,0)[cc]{$0$}}
\put(45.00,5.00){\makebox(0,0)[cc]{$1$}}
\put(75.00,5.00){\makebox(0,0)[cc]{$2$}}
\put(105.00,5.00){\makebox(0,0)[cc]{$3$}}
\put(135.00,5.00){\makebox(0,0)[cc]{$4$}}
\put(5.00,15.00){\makebox(0,0)[cc]{${0}$}}
\put(5.00,45.00){\makebox(0,0)[cc]{${l}$}}
\put(5.00,60.00){\makebox(0,0)[cc]{$J$}}
\put(10.00,10.00){\line(0,1){40.00}}
\put(10.00,10.00){\line(1,0){130.00}}
%\put(15.00,15.00){\line(1,1){125.00}}
%\put(15.00,15.00){\circle*{4.00}}
\put(45.00,45.00){\circle*{4.00}}
\put(75.00,15.00){\circle*{4.00}}
\put(105.00,45.00){\circle*{4.00}}
\put(135.00,15.00){\circle*{4.00}}
%\put(15.00,15.00){\vector(1,1){28.00}}
\put(45.00,45.00){\vector(1,-1){28.00}}
\put(75.00,15.00){\vector(1,1){28.00}}
\put(105.00,45.00){\vector(1,-1){28.00}}
\put(170.33,5.00){\makebox(0,0)[cc]{$N-4$}}
\put(200.33,5.00){\makebox(0,0)[cc]{$N-3$}}
\put(230.33,5.00){\makebox(0,0)[cc]{$N-2$}}
\put(260.33,5.00){\makebox(0,0)[cc]{$N-1$}}
\put(290.33,5.00){\makebox(0,0)[cc]{$N$}}
\put(305.33,5.00){\makebox(0,0)[cc]{$N$}}
\put(165.33,10.00){\line(1,0){130.00}}
\put(170.33,15.00){\circle*{4.00}}
\put(200.33,45.00){\circle*{4.00}}
\put(230.33,15.00){\circle*{4.00}}
\put(260.33,45.00){\circle*{4.00}}
\put(290.33,15.00){\circle*{4.00}}
\put(170.33,15.00){\vector(1,1){28.00}}
\put(200.33,45.00){\vector(1,-1){28.00}}
\put(230.33,15.00){\vector(1,1){28.00}}
\put(260.33,45.00){\vector(1,-1){28.00}}
\put(290.33,15.00){\circle{8.00}}
\put(152.08,10.00){\makebox(0,0)[cc]{$\cdots$}}
\end{picture}
\end{center}
\caption{ Construction of the ``zigzag'' singlet state of $N$
particles which effectively is a product state of $\frac{N}{2}$
spin-$l$ particle states. \label{2005-singlet-f1-zigzag}}
\end{figure}


Here we present a method to construct all states for a given
number of particles. They are the basis to construct non-trivial,
e.g., non-zigzag  singlet states, which are not just
products of singlet states of a smaller number of particles.


We start by considering the spin state of a single spin-${1\over
2}$ particle. A second spin-${1\over 2}$ particle is added by
combining two angular momenta $\frac{1}{2}$ to all possible
angular momenta $j_{12}=0,1$. Next a third particle is introduced
by coupling a third angular momentum $\frac{1}{2}$ to all
previously derived states. Following the triangular equation, the
resulting $j$-values for each $j_{12}$ are: $|j_{12}-j_3|\leq
j\leq j_{12}+j_3$.

In order to obtain all $N$-particle singlet states, we
successively produce all states (not only singlets) of
$\frac{N}{2}$ particles. From this point on, only certain states
are necessary for the further procedure. For $\frac{N}{2}\leq
h\leq N$ particles we only need angular momentum states with
$0\leq j\leq\frac{N-h}{2}$.

Angular momentum states will be written as
$|h,j,m,i\rangle$, where $h$ denotes the particle number, $j$ the
angular momentum, $m$ the magnetic quantum number, $i$ the number
of state. The Clebsch-Gordan coefficient is denoted $\langle
j_1,j_2,m_1,m_2|j,m\rangle$. $f[j+1,h-1]$ denotes the number of
states at $h$ particles and angular momentum $\frac{j}{2}$.

Explicit procedure: In order to obtain the states $|h,j\rangle $
we first consider the states of $h-1$ particles and angular
momentum $j+{1\over2}$. To produce the concrete state
$|h,j,m,i\rangle$ we multiply the Clebsch-Gordan coefficient
$\langle j+{1\over2},m-{1\over2},{1\over2},{1\over2}|j,m\rangle$
with the product state
$|h-1,j+{1\over2},m-{1\over2},i\rangle\otimes|1,{1\over2},{1\over2},1\rangle$.
We take the state $|h-1,j+{1\over2},m+{1\over2},i\rangle$, build
the product state
$|h-1,j+{1\over2},m+{1\over2},i\rangle\otimes|1,{1\over2},-{1\over2},1\rangle$
and multiply it with the Clebsch-Gordan coefficient $\langle
j+{1\over2},m+{1\over2},{1\over2},-{1\over2}|j,m\rangle$. Adding
the two results we obtain the state $|h,j,m,i\rangle$. We do this
for $m= -j,j$ and $i= 1, f[(2j+1)+1,h-1]$. If $j$ is bigger than
zero, we look at the states
$|h-1,j-{1\over2},m-{1\over2},i\rangle$ and
$|h-1,j-{1\over2},m+{1\over2},i\rangle$ and obtain the
$|h,j,m,i\rangle$ particle state as the sum of $\langle
j-{1\over2},m-{1\over2},{1\over2},{1\over2}|j,m\rangle|h-1,j-{1\over2},m-{1\over2},i\rangle\otimes|1,{1\over2},{1\over2},1\rangle$
and $\langle
j-{1\over2},m+{1\over2},{1\over2},-{1\over2}|j,m\rangle|h-1,j-{1\over2},m+{1\over2},i\rangle\otimes|1,{1\over2},-{1\over2},1\rangle$.
This procedure is done for $m= -j,j$ and $i=
f[(2j+1)+1,h-1]+1,f[(2j+1)+1,h-1]+f[(2j+1)-1,h-1]$.

%\begin{figure}
%\begin{align}
%&|h-1,j+{1\over2},m-{1\over2},i\rangle\rightarrow \langle
%j+{1\over2},m-{1\over2},{1\over2},{1\over2}|j,m\rangle|h-1,j+{1\over2},m-{1\over2}\rangle\otimes|{1\over2},{1\over2}\rangle\\
%&|h-1,j+{1\over2},m+{1\over2},i\rangle\rightarrow \langle
%j+{1\over2},m+{1\over2},{1\over2},-{1\over2}|j,m\rangle|h-1,j+{1\over2},m+{1\over2}\rangle\otimes|{1\over2},-{1\over2}\rangle
%\end{align}
%Joining eq.1 and eq.2 $\rightarrow |h,j,m,i\rangle $ for $m= -j,j$
%and $i= 1, f[(2j+1)+1,h-1]$\\\\
%If $j>0$
%\begin{align}
%&|h-1,j-{1\over2},m-{1\over2},i\rangle\rightarrow \langle
%j-{1\over2},m-{1\over2},{1\over2},{1\over2}|j,m\rangle|h-1,j-{1\over2},m-{1\over2}\rangle\otimes|{1\over2},{1\over2}\rangle\\
%&|h-1,j-{1\over2},m+{1\over2},i\rangle\rightarrow \langle
%j-{1\over2},m+{1\over2},{1\over2},-{1\over2}|j,m\rangle|h-1,j-{1\over2},m+{1\over2}\rangle\otimes|{1\over2},-{1\over2}\rangle
%\end{align}
%Joining eq.3 and eq.4 $\rightarrow |h,j,m,i\rangle $ for $m= -j,j$
%and $i= f[(2j+1)+1,h-1]+1,f[(2j+1)+1,h-1]+f[(2j+1)-1,h-1]$
%\caption{Sketch of the construction of the $|h,j,m\rangle$ product
%states for spin-${1\over 2}$ particles. $h$ denotes the particle
%number. f[j+1,h] denotes the number of states at $h$ particles and
%angular momentum $j$.}
%\end{figure}

%PETER, KOENNTEST DU HIER BITTE NOCH EINE KURZE BESCHREIBUNG DER METHODE GEBEN?



A concrete example is drawn in Fig.~\ref{2005-singlet-f12-e1}.
\begin{figure}
\begin{center}
\begin{tabular}{ccc}
%TexCad Options
%\grade{\off}
%\emlines{\off}
%\beziermacro{\on}
%\reduce{\on}
%\snapping{\off}
%\quality{2.00}
%\graddiff{0.01}
%\snapasp{1}
%\zoom{1.00}
\unitlength 0.40mm
\linethickness{0.4pt}
\begin{picture}(150.00,150.00)
\put(15.00,5.00){\makebox(0,0)[cc]{$0$}}
\put(45.00,5.00){\makebox(0,0)[cc]{$1$}}
\put(75.00,5.00){\makebox(0,0)[cc]{$2$}}
\put(105.0,5.00){\makebox(0,0)[cc]{$3$}}
\put(135.00,5.00){\makebox(0,0)[cc]{$4$}}
\put(150.00,5.00){\makebox(0,0)[cc]{$N$}}
\put(5.00,45.00){\makebox(0,0)[cc]{${1\over 2}$}}
\put(5.00,75.00){\makebox(0,0)[cc]{$1$}}
\put(5.00,105.00){\makebox(0,0)[cc]{${3\over 2}$}}
\put(5.00,135.00){\makebox(0,0)[cc]{$2$}}
\put(5.00,150.00){\makebox(0,0)[cc]{$J$}}
\put(10.00,10.00){\line(0,1){130.00}}
\put(10.00,10.00){\line(1,0){130.00}}
%\put(15.00,15.00){\line(1,1){125.00}}
\put(45.00,45.00){\circle*{4.00}}
\put(75.00,15.00){\circle*{4.00}}
\put(105.00,45.00){\circle*{4.00}}
\put(135.00,15.00){\circle*{4.00}}
\put(45.00,45.00){\vector(1,-1){28.00}}
\put(75.00,15.00){\vector(1,1){28.00}}
\put(105.00,45.00){\vector(1,-1){28.00}}
\put(75.00,75.00){\circle{4.00}}
%\put(105.00,75.00){\circle{4.00}}
\put(135.00,75.00){\circle{4.00}}
\put(105.00,105.00){\circle{4.00}}
%\put(135.00,105.00){\circle{4.00}}
\put(135.00,135.00){\circle{4.00}}
%\put(75.00,45.00){\circle{4.00}}
%\put(135.00,45.00){\circle{4.00}}
\put(135.00,15.00){\circle{8.00}}
\end{picture}
&
$\qquad$
&
%TexCad Options
%\grade{\off}
%\emlines{\off}
%\beziermacro{\on}
%\reduce{\on}
%\snapping{\off}
%\quality{2.00}
%\graddiff{0.01}
%\snapasp{1}
%\zoom{1.00}
\unitlength 0.40mm
\linethickness{0.4pt}
\begin{picture}(150.00,150.00)
\put(15.00,5.00){\makebox(0,0)[cc]{$0$}}
\put(45.00,5.00){\makebox(0,0)[cc]{$1$}}
\put(75.00,5.00){\makebox(0,0)[cc]{$2$}}
\put(105.0,5.00){\makebox(0,0)[cc]{$3$}}
\put(135.00,5.00){\makebox(0,0)[cc]{$4$}}
\put(150.00,5.00){\makebox(0,0)[cc]{$N$}}
\put(5.00,45.00){\makebox(0,0)[cc]{${1\over 2}$}}
\put(5.00,75.00){\makebox(0,0)[cc]{$1$}}
\put(5.00,105.00){\makebox(0,0)[cc]{${3\over 2}$}}
\put(5.00,135.00){\makebox(0,0)[cc]{$2$}}
\put(5.00,150.00){\makebox(0,0)[cc]{$J$}}
\put(10.00,10.00){\line(0,1){130.00}}
\put(10.00,10.00){\line(1,0){130.00}}
%\put(15.00,15.00){\line(1,1){125.00}}
%\put(15.00,15.00){\circle*{4.00}}
\put(45.00,45.00){\circle*{4.00}} \put(75.00,15.00){\circle{4.00}}
\put(105.00,45.00){\circle*{4.00}}
\put(135.00,15.00){\circle*{4.00}}
%\put(15.00,15.00){\vector(1,1){28.00}}
\put(75.00,75.00){\vector(1,-1){28.00}}
\put(45.00,45.00){\vector(1,1){28.00}}
\put(105.00,45.00){\vector(1,-1){28.00}}
\put(75.00,75.00){\circle*{4.00}}
%\put(105.00,75.00){\circle{4.00}}
\put(135.00,75.00){\circle{4.00}}
\put(105.00,105.00){\circle{4.00}}
%\put(135.00,105.00){\circle{4.00}}
\put(135.00,135.00){\circle{4.00}}
%\put(75.00,45.00){\circle{4.00}}
%\put(135.00,45.00){\circle{4.00}}
\put(135.00,15.00){\circle{8.00}}
\end{picture}
\\
a)&&b)
\end{tabular}
\end{center}
\caption{ Construction of both singlet states of four
spin-${1\over 2}$ particles. Concentric circles indicate the
target states. \label{2005-singlet-f12-e1}}
\end{figure}
It contains the pathways leading to the construction of both
singlet states of four spin-${1\over 2}$ particles. Another
example is the construction of the single three spin-1 particle
singlet state drawn in
 Fig.~\ref{2005-singlet-f1-e2}.
\begin{figure}
\begin{center}
%TexCad Options
%\grade{\off}
%\emlines{\off}
%\beziermacro{\on}
%\reduce{\on}
%\snapping{\off}
%\quality{2.00}
%\graddiff{0.01}
%\snapasp{1}
%\zoom{1.00}
\unitlength 0.40mm
\linethickness{0.4pt}
\begin{picture}(120.00,120.00)
\put(15.00,5.00){\makebox(0,0)[cc]{$0$}}
\put(45.00,5.00){\makebox(0,0)[cc]{$1$}}
\put(75.00,5.00){\makebox(0,0)[cc]{$2$}}
\put(105.00,5.00){\makebox(0,0)[cc]{$3$}}
\put(120.00,5.00){\makebox(0,0)[cc]{$N$}}
\put(5.00,45.00){\makebox(0,0)[cc]{${1}$}}
\put(5.00,75.00){\makebox(0,0)[cc]{$2$}}
\put(5.00,105.00){\makebox(0,0)[cc]{${3}$}}
\put(5.00,120.00){\makebox(0,0)[cc]{$J$}}
\put(10.00,10.00){\line(0,1){100.00}}
\put(10.00,10.00){\line(1,0){100.00}}
%\put(15.00,15.00){\line(1,1){125.00}}
%\put(15.00,15.00){\circle*{4.00}}
%\put(45.00,15.00){\circle{4.00}}
\put(45.00,45.00){\circle*{4.00}}
\put(75.00,15.00){\circle{4.00}}
\put(105.00,15.00){\circle*{4.00}}
\put(105.00,15.00){\circle{8.00}}
%\put(15.00,15.00){\vector(1,1){28.00}}
\put(75.00,44.67){\vector(1,-1){28.00}}
\put(75.00,75.00){\circle{4.00}}
\put(105.00,75.00){\circle{4.00}}
\put(105.00,105.00){\circle{4.00}}
\put(75.00,45.00){\circle*{4.00}}
\put(105.00,45.00){\circle{4.00}}
%\put(45.00,15.00){\vector(1,1){28.00}}
\put(45.00,45.00){\vector(1,0){27.00}}
\end{picture}
\end{center}
\caption{ Construction of the singlet state of three spin-$1$
particles. \label{2005-singlet-f1-e2}}
\end{figure}



Next the singlet states of up to 6 spin-${1\over 2}$ and 4
spin-${1}$ particle are explicitly enumerated in
Tables \ref{2005-singlet-t12} and \ref{2005-singlet-t1}.

%\begin{table}
%\begin{tabular}{ccccc}
%\hline\hline
%k & \# & \\
%\hline\hline
%\hline\hline
%2& 1 &$-\frac{1}{2} \vert -+ \rangle +\frac{1}{2} \vert +- \rangle $\\
%\hline
%4& 1 &
%${\frac{1}{\sqrt{3}} \vert --++ \rangle }+{-\frac{1}{2\sqrt{3}} \vert -+-+ \rangle }
%+{-\frac{1}{2\sqrt{3}} \vert +--+ \rangle }$\\
%& 2 &
%$+{-\frac{1}{2\sqrt{3}} \vert -++- \rangle }
%+{-\frac{1}{2\sqrt{3}} \vert +-+- \rangle }+{\frac{1}{{\sqrt{3}}} \vert ++-- \rangle }$\\
%& 3 &
%${{\frac{1}{2} \vert -+-+ \rangle }+{-\frac{1}{2} \vert +--+ \rangle }
%+{-\frac{1}{2} \vert -++- \rangle }+{\frac{1}{2} \vert +-+- \rangle }}$

%\hline\hline
%\end{tabular}
%\caption{First singlet states of k spin-${1\over 2}$ quanta.
%\label{2005-singlet-t12}}
%\end{table}


%\begin{table}
%\begin{tabular}{ccccc}
%\hline\hline
%k & \# & \\


%\hline\hline
%\end{tabular}
%\caption{First singlet states of k  spin-${1}$ quanta.
%\label{2005-singlet-t1}}
%\end{table}
\clearpage
\begin{table}
\begin{tabular}{ccc}
\hline\hline
N & \# & \\
\hline\hline
2&1&$\frac{1}{{\sqrt{2}}}\big(|+,-\rangle-|-,+\rangle\big);$\\\hline
4&1&$-\frac{1}{2\sqrt{3}} \big(|-,+,-,+\rangle
+|-,+,+,-\rangle+|+,-,-,+\rangle +|+,-,+,-\rangle \big)+$\\&&$
+\frac{1}{\sqrt{3}}\big(|-,-,+,+\rangle +|+,+,-,-\rangle\big)
;$\\


4&2&$\big(-\frac{1}{{\sqrt{2}}}|-,+\rangle
+\frac{1}{{\sqrt{2}}}|+,-\rangle\big)^2
% -\frac{1}{2}\big(|-1,+,+,-1\rangle
%+|1,-1,-1,1\rangle \big) +\frac{1}{2}\big(|-1,1,-1,1\rangle+
%|1,-1,1,-1\rangle\big)
;$\\\hline


6&1& $-\frac{1}{2}|-,-,-,+,+,+\rangle +-\frac{1}{6}\big(
|-,+,+,-,-,+\rangle + |-,+,+,-,+,-\rangle+$\\&&$
+|-,+,+,+,-,-\rangle + |+,-,+,-,-,+\rangle +|+,-,+,-,+,-\rangle
+$\\&&$+ |+,-,+,+,-,-\rangle +|+,+,-,-,-,+\rangle +
|+,+,-,-,+,-\rangle +$\\&&$+|+,+,-,+,-,-\rangle\big) +
\frac{1}{6}\big(|-,-,+,-,+,+\rangle +|-,-,+,+,-,+\rangle +$\\&&$+
|-,-,+,+,+,-\rangle +|-,+,-,-,+,+\rangle + |-,+,-,+,-,+\rangle
+$\\&&$+|-,+,-,+,+,-\rangle + |+,-,-,-,+,+\rangle
+|+,-,-,+,-,+\rangle +$\\&&$+
|+,-,-,+,+,-\rangle\big) +\frac{1}{2}|+,+,+,-,-,-\rangle;$\\

6&2&$ -\frac{{\sqrt{2}}}{3}|-,-,+,-,+,+\rangle +-\frac{1}{3
{\sqrt{2}}}\big(|-,+,+,+,-,-\rangle +|+,-,+,+,-,-\rangle+$\\&&$
+|+,+,-,-,-,+\rangle+
 |+,+,-,-,+,-\rangle\big) +-\frac{1}{6 {\sqrt{2}}}\big(|-,+,-,+,-,+\rangle+$\\&&$ +|-,+,-,+,+,-\rangle
 +
 |+,-,-,+,-,+\rangle +|+,-,-,+,+,-\rangle\big)+$\\&&$ +\frac{1}{6 {\sqrt{2}}}\big(|-,+,+,-,-,+\rangle +
 |-,+,+,-,+,-\rangle +|+,-,+,-,-,+\rangle+$\\&&$ +|+,-,+,-,+,-\rangle\big)+
 \frac{1}{3 {\sqrt{2}}}\big(|-,-,+,+,-,+\rangle +|-,-,+,+,+,-\rangle+$\\&&$ +|-,+,-,-,+,+\rangle +
|+,-,-,-,+,+\rangle\big)+\frac{{\sqrt{2}}}{3}|+,+,-,+,-,-\rangle;$\\

6&3&$ -\frac{1}{{\sqrt{6}}}\big(|-,+,-,-,+,+\rangle
+|-,+,+,+,-,-\rangle\big)+-\frac{1}{2
{\sqrt{6}}}\big(|+,-,-,+,-,+\rangle +$\\&&$+ |+,-,-,+,+,-\rangle+
 |+,-,+,-,-,+\rangle +|+,-,+,-,+,-\rangle\big)+$\\&&$ +\frac{1}{2 {\sqrt{6}}}\big(|-,+,-,+,-,+\rangle +
 |-,+,-,+,+,-\rangle +|-,+,+,-,-,+\rangle
 +$\\&&$+|-,+,+,-,+,-\rangle\big)+
 \frac{1}{{\sqrt{6}}}\big(|+,-,-,-,+,+\rangle +|+,-,+,+,-,-\rangle\big)
 ;$\\

6&4&$ -\frac{1}{{\sqrt{6}}}\big(|-,-,+,+,-,+\rangle +
|+,+,-,-,-,+\rangle\big) +-\frac{1}{2
{\sqrt{6}}}\big(|-,+,-,+,+,-\rangle+$\\&&$ +|-,+,+,-,+,-\rangle +
|+,-,-,+,+,-\rangle +|+,-,+,-,+,-\rangle\big)+$\\&&$ +\frac{1}{2
{\sqrt{6}}}\big(|-,+,-,+,-,+\rangle +
 |-,+,+,-,-,+\rangle +|+,-,-,+,-,+\rangle+$\\&&$ +|+,-,+,-,-,+\rangle\big) +
 \frac{1}{{\sqrt{6}}}\big(|-,-,+,+,+,-\rangle
 +|+,+,-,-,+,-\rangle\big);$\\

6&5&$\big(-\frac{1}{{\sqrt{2}}}|-,+\rangle
+\frac{1}{{\sqrt{2}}}|+,-\rangle\big)^3.$\\\hline\hline
%& -\frac{1}{2
%{\sqrt{2}}}\big(|-1,1,-1,1,-1,1\rangle +
% |-1,1,1,-1,1,-1\rangle +
% |1,-1,-1,1,1,-1\rangle +$\\&&$+
% |1,-1,1,-1,-1,1\rangle\big)+
 %\frac{1}{2 {\sqrt{2}}}\big(|-1,1,-1,1,1,-1\rangle +
 %|-1,1,1,-1,-1,1\rangle +$\\&&$+
 %|1,-1,-1,1,-1,1\rangle +
 %|1,-1,1,-1,1,-1\rangle\big)

\end{tabular}
\caption{First singlet states of $N$ spin-${1\over 2}$ particles.
\label{2005-singlet-t12}}
\end{table}
\clearpage

\begin{longtable}{ccccc}
%\begin{tabular}{ccccc}
\hline\hline N & \# & \\\hline\hline
2&1&$\frac{1}{{\sqrt{3}}}\big(-|0,0\rangle+|-1,1\rangle+|1,-1\rangle\big);$\\\hline


3&1&$
-\frac{1}{{\sqrt{6}}}\big(|-1,0,1\rangle+|0,1,-1\rangle+|1,-1,0\rangle\big)+$\\&&$
+\frac{1}{{\sqrt{6}}}\big(|-1,1,0\rangle+|0,-1,1\rangle+|1,0,-1\rangle\big);$\\\hline

4&1&$ -\frac{1}{2 {\sqrt{5}}}\big(|-1,0,0,1\rangle+
|-1,0,1,0\rangle+|0,-1,0,1\rangle+
|0,-1,1,0\rangle+$\\&&$+|0,1,-1,0\rangle+
|0,1,0,-1\rangle+|1,0,-1,0\rangle+
|1,0,0,-1\rangle\big)+$\\&&$+\frac{1}{6
{\sqrt{5}}}\big(|-1,1,-1,1\rangle+
|-1,1,1,-1\rangle+|1,-1,-1,1\rangle+
|1,-1,1,-1\rangle\big)+$\\&&$+\frac{1}{3
{\sqrt{5}}}\big(|-1,1,0,0\rangle+
|0,0,-1,1\rangle+|0,0,1,-1\rangle+
|1,-1,0,0\rangle\big)+$\\&&$+\frac{2}{3
{\sqrt{5}}}|0,0,0,0\rangle+
\frac{1}{{\sqrt{5}}}\big(|-1,-1,1,1\rangle+|1,1,-1,-1\rangle\big);$\\

4&2&$-\frac{1}{2
{\sqrt{3}}}\big(|-1,0,1,0\rangle+|-1,1,-1,1\rangle+
|0,-1,0,1\rangle+|0,1,0,-1\rangle+$\\&&$+
|1,-1,1,-1\rangle+|1,0,-1,0\rangle\big)+\frac{1}{2
{\sqrt{3}}}\big(|-1,0,0,1\rangle+|-1,1,1,-1\rangle+$\\&&$+
|0,-1,1,0\rangle+|0,1,-1,0\rangle+
|1,-1,-1,1\rangle+|1,0,0,-1\rangle\big);$\\
4&3&$\big(\frac{1}{{\sqrt{3}}}\big(-|0,0\rangle+|-1,1\rangle+|1,-1\rangle\big)\big)^2
%&-\frac{1}{3}(|-1,1,0,0\rangle+|0,0,-1,1\rangle+|0,0,1,-1\rangle+
%|1,-1,0,0\rangle\big)+$\\&&$+\frac{1}{3}\big(|-1,1,-1,1\rangle+|-1,1,1,-1\rangle+
%|0,0,0,0\rangle+|1,-1,-1,1\rangle+|1,-1,1,-1\rangle\big)
;$\\\hline 5&1&$-{\sqrt{\frac{2}{15}}}|-1,-1,0,1,1\rangle+
-\frac{1}{{\sqrt{30}}}\big(|-1,0,1,0,0\rangle+|0,-1,1,0,0\rangle+$\\&&$+
|0,0,-1,0,1\rangle+|0,0,-1,1,0\rangle+
|0,1,1,-1,-1\rangle+$\\&&$+|1,0,1,-1,-1\rangle+
|1,1,-1,-1,0\rangle+|1,1,-1,0,-1\rangle\big)+$\\&&$+ -\frac{1}{2
{\sqrt{30}}}\big(|-1,0,1,-1,1\rangle+|-1,0,1,1,-1\rangle+
|-1,1,-1,0,1\rangle+$\\&&$+|-1,1,-1,1,0\rangle+
|0,-1,1,-1,1\rangle+|0,-1,1,1,-1\rangle+$\\&&$+
|0,1,0,-1,0\rangle+|0,1,0,0,-1\rangle+
|1,-1,-1,0,1\rangle+$\\&&$+|1,-1,-1,1,0\rangle+
|1,0,0,-1,0\rangle+|1,0,0,0,-1\rangle\big)+$\\&&$+ \frac{1}{2
{\sqrt{30}}}\big(|-1,0,0,0,1\rangle+|-1,0,0,1,0\rangle+
|-1,1,1,-1,0\rangle+$\\&&$+|-1,1,1,0,-1\rangle+
|0,-1,0,0,1\rangle+|0,-1,0,1,0\rangle+$\\&&$+
|0,1,-1,-1,1\rangle+|0,1,-1,1,-1\rangle+
|1,-1,1,-1,0\rangle+$\\&&$+|1,-1,1,0,-1\rangle+
|1,0,-1,-1,1\rangle+|1,0,-1,1,-1\rangle\big)+$\\&&$+
\frac{1}{{\sqrt{30}}}\big(|-1,-1,1,0,1\rangle+|-1,-1,1,1,0\rangle+
|-1,0,-1,1,1\rangle+$\\&&$+|0,-1,-1,1,1\rangle+
|0,0,1,-1,0\rangle+|0,0,1,0,-1\rangle+$\\&&$+
|0,1,-1,0,0\rangle+|1,0,-1,0,0\rangle\big)+{\sqrt{\frac{2}{15}}}|1,1,0,-1,-1\rangle;$\\\hline

6&7&$-\frac{1}{{\sqrt{15}}}\big(|-1,-1,0,1,1,0\rangle+|1,1,0,-1,-1,0\rangle)+
-\frac{1}{2 {\sqrt{15}}}(|-1,-1,1,0,0,1\rangle+$\\&&$+
|-1,-1,1,1,-1,1\rangle+ |-1,0,-1,1,0,1\rangle+
|-1,0,1,0,1,-1\rangle+$\\&&$+ |0,-1,-1,1,0,1\rangle+
|0,-1,1,0,1,-1\rangle+ |0,0,-1,0,1,0\rangle+$\\&&$+
|0,0,-1,1,1,-1\rangle+ |0,0,1,-1,-1,1\rangle+
|0,0,1,0,-1,0\rangle+$\\&&$+ |0,1,-1,0,-1,1\rangle+
|0,1,1,-1,0,-1\rangle+ |1,0,-1,0,-1,1\rangle+$\\&&$+
|1,0,1,-1,0,-1\rangle+ |1,1,-1,-1,1,-1\rangle+
|1,1,-1,0,0,-1\rangle\big)+$\\&&$+ -\frac{1}{4
{\sqrt{15}}}\big(|-1,0,0,0,0,1\rangle+ |-1,0,0,1,-1,1\rangle+
|-1,0,1,-1,1,0\rangle+$\\&&$+ |-1,0,1,1,0,-1\rangle+
|-1,1,-1,0,1,0\rangle+ |-1,1,-1,1,1,-1\rangle+$\\&&$+
|-1,1,1,-1,-1,1\rangle+ |-1,1,1,0,-1,0\rangle+
|0,-1,0,0,0,1\rangle+$\\&&$+ |0,-1,0,1,-1,1\rangle+
|0,-1,1,-1,1,0\rangle+ |0,-1,1,1,0,-1\rangle+$\\&&$+
|0,1,-1,-1,0,1\rangle+ |0,1,-1,1,-1,0\rangle+
|0,1,0,-1,1,-1\rangle+$\\&&$+ |0,1,0,0,0,-1\rangle+
|1,-1,-1,0,1,0\rangle+ |1,-1,-1,1,1,-1\rangle+$\\&&$+
|1,-1,1,-1,-1,1\rangle+ |1,-1,1,0,-1,0\rangle+
|1,0,-1,-1,0,1\rangle+$\\&&$+ |1,0,-1,1,-1,0\rangle+
|1,0,0,-1,1,-1\rangle+ |1,0,0,0,0,-1\rangle\big)+$\\&&$+
\frac{1}{4
{\sqrt{15}}}\big(|-1,0,0,0,1,0\rangle+|-1,0,0,1,1,-1\rangle+
|-1,0,1,-1,0,1\rangle+$\\&&$+ |-1,0,1,1,-1,0\rangle+
|-1,1,-1,0,0,1\rangle+ |-1,1,-1,1,-1,1\rangle+$\\&&$+
|-1,1,1,-1,1,-1\rangle+ |-1,1,1,0,0,-1\rangle+
|0,-1,0,0,1,0\rangle+$\\&&$+|0,-1,0,1,1,-1\rangle+
|0,-1,1,-1,0,1\rangle+ |0,-1,1,1,-1,0\rangle+$\\&&$+
|0,1,-1,-1,1,0\rangle+ |0,1,-1,1,0,-1\rangle+
|0,1,0,-1,-1,1\rangle+$\\&&$+
|0,1,0,0,-1,0\rangle+|1,-1,-1,0,0,1\rangle+
|1,-1,-1,1,-1,1\rangle+$\\&&$+ |1,-1,1,-1,1,-1\rangle+
|1,-1,1,0,0,-1\rangle+ |1,0,-1,-1,1,0\rangle+$\\&&$+
|1,0,-1,1,0,-1\rangle+ |1,0,0,-1,-1,1\rangle+
|1,0,0,0,-1,0\rangle\big)+$\\&&$+\frac{1}{2
{\sqrt{15}}}\big(|-1,-1,1,0,1,0\rangle+ |-1,-1,1,1,1,-1\rangle+
|-1,0,-1,1,1,0\rangle+$\\&&$+ |-1,0,1,0,-1,1\rangle+
|0,-1,-1,1,1,0\rangle+ |0,-1,1,0,-1,1\rangle+$\\&&$+
|0,0,-1,0,0,1\rangle+|0,0,-1,1,-1,1\rangle+
|0,0,1,-1,1,-1\rangle+$\\&&$+
|0,0,1,0,0,-1\rangle+|0,1,-1,0,1,-1\rangle+
|0,1,1,-1,-1,0\rangle+$\\&&$+ |1,0,-1,0,1,-1\rangle+
|1,0,1,-1,-1,0\rangle+ |1,1,-1,-1,-1,1\rangle+$\\&&$+
|1,1,-1,0,-1,0\rangle\big)+
\frac{1}{{\sqrt{15}}}\big(|-1,-1,0,1,0,1\rangle+|1,1,0,-1,0,-1\rangle\big).$\\
\hline\hline\\
%\end{tabular}
\caption{First singlet states of $N$ spin-${1}$ particles.
\label{2005-singlet-t1}}
\end{longtable}

In summary, we have present a detailed, algorithmic description of
how to obtain all singlet states of spin-${1\over 2}$ and spin-$1$ particles.
The method can applied analogously for the construction of $N$-partite singlet states from
particles with
higher dimensional spin.


\bibliography{svozil}
\bibliographystyle{osa}

%%%%%%%%%%%%%%%%%%%%%%%%%%%%%%%%%%%%%%%%%%%%%%%%%%%%%%%%%%%%%%%%%%%%%%%%%%
\end{document}

\section{Symmetries}
One might wonder which symmetries can be found. For $N=2r$
particles, there always exist ``zigzag'' singlet states, which are
the product of $r$ two-particle singlet states stemming from the
rising and lowering of consecutive states. The situation is
depicted in Fig.~\ref{2005-singlet-f1-zigzag}. For $J=1$ and
$N=3r$ there exist ``zigzag'' singlet states, which are the
product of $r$ three-particle singlet states. For singlet states
with $N=2r+3t$ ($r,t$ integer) there exist singlet states being
the product of $r$
two-particle singlet states and $t$ three-particle singlet states.\\
With our approach the singlet states are orthogonal to each other.
Since the formula beyond holds, states stemming from different
$j_1$ values are orthogonal to each other. Hence, also the singlet
states derived from them are orthogonal. By iteration it follows
that even singlet states stemming from the same $j_1$ are
orthogonal.\\We construct the full basis for each singlet space,
which has the appropriate dimension.
\begin{align}
\langle(j_1'j_2')jm|(j_1j_2)jm\rangle =&\sum_{m_1'+m_2'=m,
m_1+m_2=m}\langle(j_1'j_2')jm|j_1'm_1'j_2'm_2'\rangle\times\nonumber\\
&\langle j_1'm_1'j_2'm_2'|j_1m_1j_2m_2\rangle\langle
j_1m_1j_2m_2|(j_1j_2)jm\rangle\\=&\delta_{j_1j_1'}\delta_{j_2j_2'}\delta_{m_1m_1'}\delta_{m_2m_2'}.\nonumber
%\sqrt{2j_3+1}(-1)^{j_1-j_2-m_3}\left(\begin{array}{ccc}
%j_1&j_2&j_3\\m_1&m_2&m_3\end{array}\right),
\end{align}

\begin{figure}
\begin{center}
%TexCad Options
%\grade{\off}
%\emlines{\off}
%\beziermacro{\on}
%\reduce{\on}
%\snapping{\off}
%\quality{2.00}
%\graddiff{0.01}
%\snapasp{1}
%\zoom{0.80}
\unitlength 0.40mm \linethickness{0.4pt}
\begin{picture}(305.33,150.00)
\put(15.00,5.00){\makebox(0,0)[cc]{$0$}}
\put(45.00,5.00){\makebox(0,0)[cc]{$1$}}
\put(75.00,5.00){\makebox(0,0)[cc]{$2$}}
\put(105.00,5.00){\makebox(0,0)[cc]{$3$}}
\put(135.00,5.00){\makebox(0,0)[cc]{$4$}}
\put(5.00,15.00){\makebox(0,0)[cc]{${0}$}}
\put(5.00,45.00){\makebox(0,0)[cc]{${l}$}}
\put(5.00,60.00){\makebox(0,0)[cc]{$J$}}
\put(10.00,10.00){\line(0,1){40.00}}
\put(10.00,10.00){\line(1,0){130.00}}
%\put(15.00,15.00){\line(1,1){125.00}}
%\put(15.00,15.00){\circle*{4.00}}
\put(45.00,45.00){\circle*{4.00}}
\put(75.00,15.00){\circle*{4.00}}
\put(105.00,45.00){\circle*{4.00}}
\put(135.00,15.00){\circle*{4.00}}
%\put(15.00,15.00){\vector(1,1){28.00}}
\put(45.00,45.00){\vector(1,-1){28.00}}
\put(75.00,15.00){\vector(1,1){28.00}}
\put(105.00,45.00){\vector(1,-1){28.00}}
\put(170.33,5.00){\makebox(0,0)[cc]{$N-4$}}
\put(200.33,5.00){\makebox(0,0)[cc]{$N-3$}}
\put(230.33,5.00){\makebox(0,0)[cc]{$N-2$}}
\put(260.33,5.00){\makebox(0,0)[cc]{$N-1$}}
\put(290.33,5.00){\makebox(0,0)[cc]{$N$}}
\put(305.33,5.00){\makebox(0,0)[cc]{$N$}}
\put(165.33,10.00){\line(1,0){130.00}}
\put(170.33,15.00){\circle*{4.00}}
\put(200.33,45.00){\circle*{4.00}}
\put(230.33,15.00){\circle*{4.00}}
\put(260.33,45.00){\circle*{4.00}}
\put(290.33,15.00){\circle*{4.00}}
\put(170.33,15.00){\vector(1,1){28.00}}
\put(200.33,45.00){\vector(1,-1){28.00}}
\put(230.33,15.00){\vector(1,1){28.00}}
\put(260.33,45.00){\vector(1,-1){28.00}}
\put(290.33,15.00){\circle{8.00}}
\put(152.08,10.00){\makebox(0,0)[cc]{$\cdots$}}
\end{picture}
\end{center}
\caption{ Construction of the ``zigzag'' singlet state of $N$
particles which effectively is a product state of $\frac{N}{2}$
spin-$l$ particle states. \label{2005-singlet-f1-zigzag}}
\end{figure}

\subsection{Symmetries of the states with respect to changing
all magnetic quantum numbers into their negative}

%Further symmetries are revealed by considering the symmetry
%relations of Wigner's $3j$-symbol.
%\subsection{\bf Symmetry relations of Wigner's 3j-symbol}
%Instead of the Clebsch-Gordan coefficients, Wigner introduced a symbolic representation
%of the coupling of two spins which shows enhanced symmetry.
%These symbols are defined by
For the Clebsch-Gordan coefficients the following formula holds
\begin{align}\langle j_1,-m_1,j_2,-m_2|j,-m\rangle
=(-1)^{j_1+j_2-j}\langle j_1m_1j_2m_2|jm\rangle.\end{align} Let us
consider the $J=1$ case first. The symmetry described above
implies for coupling $j$ to $j+1$:
\begin{align}\langle j,-m-1,1,1|j+1,-m\rangle
= &(-1)^{0}\langle j,m+1,1,-1|j+1,m\rangle\nonumber\\
\langle j,-m,1,0|j+1,-m\rangle = &(-1)^{0}\langle
j,m,1,0|j+1,m\rangle,\end{align} i.e. the Clebsch-Gordan
Coefficients are the same.
%\langle j,m+1,1,-1|j+1,m\rangle = &(-1)^{0}\langle
%j,-m-1,1,1|j+1,-m\rangle\nonumber.\end{align} As the overall sign
%is plus, we see that all these Clebsch-Gordan coefficients stay
%the same under a change of all the magnetic quantum numbers into
%their negatives.
For coupling $j$ to $j$,
\begin{align}\langle j,-m-1,1,1|j,-m\rangle
= &(-1)^{1}\langle j,m+1,1,-1|j,m\rangle\nonumber\\
\langle j,-m,1,0|j,-m\rangle = &(-1)^{1}\langle
j,m,1,0|j,m\rangle,\end{align} i.e. all Clebsch-Gordan
Coefficients change their sign.
%\\\langle j,m+1,1,-1|j,m\rangle = &(-1)^{1}\langle
%j,-m-1,1,1|j,-m\rangle\nonumber.\end{align} Thus, all these
%Clebsch-Gordan coefficients change their signs if all the magnetic
%quantum numbers are changed into their negatives.
Similarly for coupling $j+1$ to $j$,
\begin{align}\langle j+1,m,1,1|j,m+1\rangle
= &(-1)^{2}\langle j+1,-m,1,-1|j,-m-1\rangle\nonumber\\
\langle j+1,m,1,0|j,m\rangle = &(-1)^{2}\langle
j+1,-m,1,0|j,-m\rangle,\end{align} i.e. they all stay the same.
%\\\langle j+1,-m,1,-1|j,-m-1\rangle = &(-1)^{2}\langle
%j+1,m,1,1|j,m+1\rangle\nonumber.\end{align} Thus, all these
%Clebsch-Gordan coefficients stay the same under a change of all
%the magnetic quantum numbers into their negatives.\\\\

%Thus we follow for coupling $j$ to $j+1$ and $j+1$ to $j$ that if
%we negate all magnetic quantum numbers all Clebsch-Gordan
%coefficients stay the same. For coupling $j$ to $j$ all the
%Clebsch-Gordan change their signs.\\\\
Using these symmetries we conclude that the symmetry behavior
stays the same if one goes from the angular momentum subspace
$|N,J\rangle$ to the angular momentum subspace $|N+1,J+1\rangle$.
The symmetry behavior doesn't changes for coupling $|N,J+1\rangle$
to $|N+1,J\rangle$.
%If one goes from the angular
%momentum subspace $|N,J\rangle$ to the angular momentum subspace
%$|N+1,J\rangle$ it changes from even to odd and from odd to even,
%respectively.
Coupling $|N,J\rangle$ to $|N+1,J\rangle$ it changes from even to
odd and from odd to even.
%respectively.
The situation is depicted in
Fig.~\ref{2005-singlet-f1-ta-takohalf}.


In particular, $N$-particle singlet states with $N$ even are even,
whereas $N$-particle singlet states with $N$ odd are odd.


%\subsubsection{Application of the symmetries to the $J=\frac{1}{2}$ case}%
Now we turn to the $J=\frac{1}{2}$ case:


For coupling $j$ to $j+\frac{1}{2}$ the resulting Clebsch-Gordan
coefficients are
\begin{align}
\langle
j,-m-\frac{1}{2},\frac{1}{2},\frac{1}{2}|j+\frac{1}{2},-m\rangle =
&(-1)^{0}\langle
j,m+\frac{1}{2},\frac{1}{2},-\frac{1}{2}|j+\frac{1}{2},m\rangle\nonumber \\
\langle
j,m+\frac{1}{2},\frac{1}{2},-\frac{1}{2}|j+\frac{1}{2},m\rangle =
&(-1)^{0}\langle
j,-m-\frac{1}{2},\frac{1}{2},\frac{1}{2}|j+\frac{1}{2},-m\rangle.
\end{align}
%We conclude that all these Clebsch-Gordan coefficients stay the
%same under a change of all magnetic quantum numbers into their
%negatives.
If we negate all the magnetic quantum numbers all these
Clebsch-Gordan coefficients stay the same. For coupling
$j+\frac{1}{2}$ to $j$,
\begin{align}
\langle
j+\frac{1}{2},m,\frac{1}{2},\frac{1}{2}|j,m+\frac{1}{2}\rangle =
&(-1)^{1}\langle
j+\frac{1}{2},-m,\frac{1}{2},-\frac{1}{2}|j,-m-\frac{1}{2}\rangle\nonumber\\
\langle
j+\frac{1}{2},-m,\frac{1}{2},-\frac{1}{2}|j,-m-\frac{1}{2}\rangle
= &(-1)^{1}\langle
j+\frac{1}{2},m,\frac{1}{2},\frac{1}{2}|j,m+\frac{1}{2}\rangle.
\end{align}
%Thus, all these Clebsch-Gordan coefficients change their signs if
%all the magnetic quantum numbers are changed into their
%negatives.\\\\
Here all the Clebsch-Gordan coefficients change their signs.\\\\
We conclude that the symmetry behavior stays the same if one goes
from the angular momentum subspace $|N,J\rangle$ to the angular
momentum subspace $|N+1,J+{1\over 2}\rangle$. Going from the
subspace $|N,J\rangle$ to the subspace $|N+1,J-{1\over2}\rangle$
the symmetry behavior changes from even to odd and from odd to
even, respectively. The situation is depicted in
Fig.~\ref{2005-singlet-f1-ta-tako}. In particular, singlet states
where $N$ is $k\cdot 2\cdot 2$ (k is an integer) are even and ones
where $N$ is $k\cdot 2\cdot (2+1)$ are odd.

%?????????????

%KEIN ENGLISCHER SATZ!!!!!!!!!!!!!




\begin{figure}
\begin{center}
\unitlength 0.40mm \linethickness{0.4pt}
\begin{picture}(350.00,200.00)
\put(15.00,5.00){\makebox(0,0)[cc]{$0$}}
\put(45.00,5.00){\makebox(0,0)[cc]{$1$}}
\put(75.00,5.00){\makebox(0,0)[cc]{$2$}}
\put(105.0,5.00){\makebox(0,0)[cc]{$3$}}
\put(135.00,5.00){\makebox(0,0)[cc]{$4$}}
\put(165.00,5.00){\makebox(0,0)[cc]{$5$}}
\put(195.00,5.00){\makebox(0,0)[cc]{$6$}}
\put(225.0,5.00){\makebox(0,0)[cc]{$7$}}
\put(255.00,5.00){\makebox(0,0)[cc]{$8$}}
\put(285.0,5.00){\makebox(0,0)[cc]{$9$}}
\put(315.00,5.00){\makebox(0,0)[cc]{$10$}}
\put(330.00,5.00){\makebox(0,0)[cc]{$N$}}
\put(5.00,15.00){\makebox(0,0)[cc]{$0$}}
\put(5.00,45.00){\makebox(0,0)[cc]{${1\over 2}$}}
\put(5.00,75.00){\makebox(0,0)[cc]{$1$}}
\put(5.00,105.00){\makebox(0,0)[cc]{${3\over 2}$}}
\put(5.00,135.00){\makebox(0,0)[cc]{$2$}}
\put(5.00,165.00){\makebox(0,0)[cc]{${5\over 2}$}}
\put(5.00,180.00){\makebox(0,0)[cc]{$J$}}
\put(10.00,10.00){\line(0,1){160.00}}
\put(10.00,10.00){\line(1,0){310.00}}
%\put(15.00,15.00){\line(1,1){125.00}}
\put(75.00,15.00){\circle*{4.00}}
\put(135.00,15.00){\circle*{4.00}}
\put(135.00,15.00){\circle{8.00}}
\put(195.00,15.00){\circle*{4.00}}
\put(255.00,15.00){\circle*{4.00}}
\put(255.00,15.00){\circle{8.00}}
\put(315.00,15.00){\circle*{4.00}}
\put(45.00,45.00){\circle*{4.00}} \put(45.00,45.00){\circle{8.00}}
\put(105.00,45.00){\circle*{4.00}}
\put(165.00,45.00){\circle*{4.00}}
\put(165.00,45.00){\circle{8.00}}
\put(225.00,45.00){\circle*{4.00}}
\put(285.00,45.00){\circle*{4.00}}
\put(285.00,45.00){\circle{8.00}}
\put(75.00,75.00){\circle*{4.00}}
 \put(75.00,75.00){\circle{8.00}}
\put(135.00,75.00){\circle*{4.00}}
\put(195.00,75.00){\circle*{4.00}}
\put(195.00,75.00){\circle{8.00}}
\put(255.00,75.00){\circle*{4.00}}
\put(105.00,105.00){\circle*{4.00}}
\put(105.00,105.00){\circle{8.00}}
\put(165.00,105.00){\circle*{4.00}}
\put(225.00,105.00){\circle*{4.00}}
\put(225.00,105.00){\circle{8.00}}
\put(135.00,135.00){\circle*{4.00}}
\put(135.00,135.00){\circle{8.00}}
\put(195.00,135.00){\circle*{4.00}}
\put(165.00,165.00){\circle*{4.00}}
\put(165.00,165.00){\circle{8.00}}


\put(75.00,15.00){\vector(1,1){28.00}}
\put(135.00,15.00){\vector(1,1){28.00}}
\put(195.00,15.00){\vector(1,1){28.00}}
\put(255.00,15.00){\vector(1,1){28.00}}
\put(45.00,45.00){\vector(1,-1){28.00}}
\put(45.00,45.00){\vector(1,1){28.00}}
\put(105.00,45.00){\vector(1,-1){28.00}}
\put(105.00,45.00){\vector(1,1){28.00}}
\put(165.00,45.00){\vector(1,-1){28.00}}
\put(165.00,45.00){\vector(1,1){28.00}}
\put(225.00,45.00){\vector(1,-1){28.00}}
\put(225.00,45.00){\vector(1,1){28.00}}
\put(285.00,45.00){\vector(1,-1){28.00}}
\put(75.00,75.00){\vector(1,-1){28.00}}
\put(75.00,75.00){\vector(1,1){28.00}}
\put(135.00,75.00){\vector(1,-1){28.00}}
\put(135.00,75.00){\vector(1,1){28.00}}
\put(195.00,75.00){\vector(1,-1){28.00}}
\put(195.00,75.00){\vector(1,1){28.00}}
\put(255.00,75.00){\vector(1,-1){28.00}}
\put(105.00,105.00){\vector(1,-1){28.00}}
\put(105.00,105.00){\vector(1,1){28.00}}
\put(165.00,105.00){\vector(1,-1){28.00}}
\put(165.00,105.00){\vector(1,1){28.00}}
\put(225.00,105.00){\vector(1,-1){28.00}}
\put(135.00,135.00){\vector(1,-1){28.00}}
\put(135.00,135.00){\vector(1,1){28.00}}
\put(195.00,135.00){\vector(1,-1){28.00}}
\put(165.00,165.00){\vector(1,-1){28.00}}


\put(80,15){\makebox(10,0){\it $1$}}
 \put(140,15){\makebox(10,0){\bf $2$}}
 \put(200,14){\makebox(10,0){\it $5$}}
\put(260,15){\makebox(10,0){\bf $14$}}
 \put(320,15){\makebox(10,0){\it $42$}}
 \put(50,45){\makebox(10,0){\bf $1.2$}}
\put(110,45){\makebox(10,0){\it $2.2$}}
 \put(170,45){\makebox(10,0){\bf $5.2$}}
 \put(230,45){\makebox(10,0){\it $14.2$}}
\put(290,45){\makebox(10,0){\bf $42.2$}}
 \put(80,75){\makebox(10,0){\bf $1.3$}}
 \put(140,75){\makebox(10,0){\it $3.3$}}
\put(200,75){\makebox(10,0){\bf $9.3$}}
 \put(260,75){\makebox(10,0){\it $28.3$}}
 \put(110,105){\makebox(10,0){\bf $1.4$}}
\put(170,105){\makebox(10,0){\it $4.4$}}
 \put(230,105){\makebox(10,0){\bf $14.4$}}
 \put(140,135){\makebox(10,0){\bf $1.5$}}
\put(200,135){\makebox(10,0){\it $5.5$}}
\put(170,165){\makebox(10,0){\bf $1.6$}}
%\put(225.00,145.00){\circle*{4.00}}
%\put(225.00,145.00){\circle{8.00}}
%\put(225.00,130.00){\circle*{4.00}}
%\put(230,145){\makebox(0,0)[l]{Number of symmetric subspaces}}
%\put(230,130){\makebox(0,0)[l]{Number of antisymmetric subspaces}}



\end{picture}
\end{center}
\caption{Symmetries of spin-${1\over 2}$ particles.
\label{2005-singlet-f1-ta-tako} Even subspaces are denoted by
concentric circles, odd subspaces are denoted by filled circles.
The numbers denote the dimension of the subspaces. The first
number stands for the number of states $|h,j\rangle$ and the
second stands for the $2j+1$ projections. Arrows represent the way
of coupling.}
\end{figure}



\begin{figure}
\begin{center}
\unitlength 0.40mm \linethickness{0.4pt}
\begin{picture}(290.00,150.00)
\put(15.00,5.00){\makebox(0,0)[cc]{$0$}}
\put(45.00,5.00){\makebox(0,0)[cc]{$1$}}
\put(75.00,5.00){\makebox(0,0)[cc]{$2$}}
\put(105.0,5.00){\makebox(0,0)[cc]{$3$}}
\put(135.00,5.00){\makebox(0,0)[cc]{$4$}}
\put(165.00,5.00){\makebox(0,0)[cc]{$5$}}
\put(195.00,5.00){\makebox(0,0)[cc]{$6$}}
\put(225.00,5.00){\makebox(0,0)[cc]{$7$}}
\put(255.0,5.00){\makebox(0,0)[cc]{$8$}}
\put(285.00,5.00){\makebox(0,0)[cc]{$9$}}
\put(315.00,5.00){\makebox(0,0)[cc]{$10$}}
\put(330.00,5.00){\makebox(0,0)[cc]{$N$}}
\put(5.00,15.00){\makebox(0,0)[cc]{$0$}}
\put(5.00,45.00){\makebox(0,0)[cc]{$1$}}
\put(5.00,75.00){\makebox(0,0)[cc]{$2$}}
\put(5.00,105.00){\makebox(0,0)[cc]{$3$}}
\put(5.00,135.00){\makebox(0,0)[cc]{$4$}}
\put(5.00,165.00){\makebox(0,0)[cc]{$5$}}
\put(5.00,180.00){\makebox(0,0)[cc]{$J$}}
\put(10.00,10.00){\line(0,1){160.00}}
\put(10.00,10.00){\line(1,0){310.00}}
%\put(15.00,15.00){\line(1,1){125.00}}
%\put(15.00,15.00){\circle*{4.00}}
\put(75.00,15.00){\circle*{4.00}} \put(75.00,15.00){\circle{8.00}}
\put(105.00,15.00){\circle*{4.00}}
 \put(135.00,15.00){\circle*{4.00}}
\put(135.00,15.00){\circle{8.00}}
 \put(165.00,15.00){\circle*{4.00}}
\put(195.00,15.00){\circle*{4.00}}
 \put(195.00,15.00){\circle{8.00}}
 \put(225.00,15.00){\circle*{4.00}}
  \put(255.00,15.00){\circle*{4.00}}
\put(255.00,15.00){\circle{8.00}}
 \put(285.00,15.00){\circle*{4.00}}
\put(315.00,15.00){\circle*{4.00}}
 \put(315.00,15.00){\circle{8.00}}
\put(45.00,45.00){\circle*{4.00}}
 \put(45.00,45.00){\circle{8.00}}
 \put(75.00,45.00){\circle*{4.00}}
\put(105.00,45.00){\circle*{4.00}}
\put(105.00,45.00){\circle{8.00}}
 \put(135.00,45.00){\circle*{4.00}}
\put(165.00,45.00){\circle*{4.00}}
 \put(165.00,45.00){\circle{8.00}}
 \put(195.00,45.00){\circle*{4.00}}
\put(225.00,45.00){\circle{8.00}}
 \put(225.00,45.00){\circle*{4.00}}
\put(255.00,45.00){\circle*{4.00}}
 \put(285.00,45.00){\circle*{4.00}}
  \put(285.00,45.00){\circle{8.00}}
\put(75.00,75.00){\circle*{4.00}} \put(75.00,75.00){\circle{8.00}}
\put(105.00,75.00){\circle*{4.00}}
 \put(135.00,75.00){\circle*{4.00}}
\put(135.00,75.00){\circle{8.00}}
 \put(165.00,75.00){\circle*{4.00}}
\put(195.00,75.00){\circle*{4.00}}
 \put(195.00,75.00){\circle{8.00}}
 \put(225.00,75.00){\circle*{4.00}}
\put(255.00,75.00){\circle*{4.00}}
 \put(255.00,75.00){\circle{8.00}}
\put(105.00,105.00){\circle*{4.00}}
\put(105.00,105.00){\circle{8.00}}
 \put(135.00,105.00){\circle*{4.00}}
\put(165.00,105.00){\circle*{4.00}}
 \put(165.00,105.00){\circle{8.00}}
  \put(195.00,105.00){\circle*{4.00}}
\put(225.00,105.00){\circle{8.00}}
 \put(225.00,105.00){\circle*{4.00}}
  \put(135.00,135.00){\circle*{4.00}}
\put(135.00,135.00){\circle{8.00}}
 \put(165.00,135.00){\circle*{4.00}}
\put(195.00,135.00){\circle*{4.00}}
 \put(195.00,135.00){\circle{8.00}}
\put(165.00,165.00){\circle*{4.00}}
 \put(165.00,165.00){\circle{8.00}}


\put(75,15){\vector(1,1){28}}
\put(105,15){\vector(1,1){28}}\put(135,15){\vector(1,1){28}}
\put(165,15){\vector(1,1){28}}\put(195,15){\vector(1,1){28}}
\put(225,15){\vector(1,1){28}}\put(255,15){\vector(1,1){28}}
 \put(45,45){\vector(1,1){28}}
\put(45,45){\vector(1,0){28}} \put(45,45){\vector(1,-1){28}}
\put(75,45){\vector(1,1){28}} \put(75,45){\vector(1,0){28}}
\put(75,45){\vector(1,-1){28}} \put(105,45){\vector(1,1){28}}
\put(105,45){\vector(1,0){28}}
\put(105,45){\vector(1,-1){28}}\put(135,45){\vector(1,1){28}}
\put(135,45){\vector(1,0){28}}
\put(135,45){\vector(1,-1){28}}\put(165,45){\vector(1,1){28}}
\put(165,45){\vector(1,0){28}} \put(165,45){\vector(1,-1){28}}
\put(195,45){\vector(1,1){28}} \put(195,45){\vector(1,0){28}}
\put(195,45){\vector(1,-1){28}} \put(225,45){\vector(1,1){28}}
\put(225,45){\vector(1,0){28}} \put(225,45){\vector(1,-1){28}}
\put(255,45){\vector(1,0){28}} \put(255,45){\vector(1,-1){28}}
\put(285,45){\vector(1,-1){28}}
 \put(75,75){\vector(1,1){28}}
\put(75,75){\vector(1,0){28}} \put(75,75){\vector(1,-1){28}}
\put(105,75){\vector(1,1){28}} \put(105,75){\vector(1,0){28}}
\put(105,75){\vector(1,-1){28}}\put(135,75){\vector(1,1){28}}
\put(135,75){\vector(1,0){28}}
\put(135,75){\vector(1,-1){28}}\put(165,75){\vector(1,1){28}}
\put(165,75){\vector(1,0){28}} \put(165,75){\vector(1,-1){28}}
\put(195,75){\vector(1,1){28}} \put(195,75){\vector(1,0){28}}
\put(195,75){\vector(1,-1){28}} \put(225,75){\vector(1,0){28}}
\put(225,75){\vector(1,-1){28}} \put(225,75){\vector(1,-1){28}}
\put(255,75){\vector(1,-1){28}} \put(105,105){\vector(1,1){28}}
\put(105,105){\vector(1,0){28}}
\put(105,105){\vector(1,-1){28}}\put(135,105){\vector(1,1){28}}
\put(135,105){\vector(1,0){28}}
\put(135,105){\vector(1,-1){28}}\put(165,105){\vector(1,1){28}}
\put(165,105){\vector(1,0){28}} \put(165,105){\vector(1,-1){28}}
\put(195,105){\vector(1,0){28}} \put(195,105){\vector(1,-1){28}}
\put(195,105){\vector(1,-1){28}} \put(225,105){\vector(1,-1){28}}
\put(135,135){\vector(1,1){28}} \put(135,135){\vector(1,0){28}}
\put(135,135){\vector(1,-1){28}} \put(165,135){\vector(1,0){28}}
\put(165,135){\vector(1,-1){28}}
\put(195,135){\vector(1,-1){28}}\put(165,165){\vector(1,-1){28}}



\put(80,15){\makebox(10,5){\bf $1$}}
\put(110,15){\makebox(10,5){\it $1$}}
\put(140,15){\makebox(10,5){\bf $3$}}
\put(170,15){\makebox(10,5){\it $6$}}
\put(200,15){\makebox(10,5){\bf $15$}}
\put(230,15){\makebox(10,5){\it $36$}}
\put(263,15){\makebox(10,5){\bf $91$}}
\put(290,15){\makebox(10,5){\it $232$}}
\put(323,15){\makebox(10,5){\bf $603$}}
\put(50,45){\makebox(10,7){\bf $1.3$}}
\put(80,45){\makebox(10,7){\it $1.3$}}
\put(110,45){\makebox(10,7){\bf $3.3$}}
\put(140,45){\makebox(10,7){\it $6.3$}}
\put(170,45){\makebox(10,7){\bf $15.3$}}
\put(200,45){\makebox(10,7){\it $36.3$}}
\put(230,45){\makebox(10,7){\bf $91.3$}}
\put(260,45){\makebox(10,7){\it $232.3$}}
\put(293,45){\makebox(10,7){\bf $603.3$}}
\put(80,75){\makebox(10,7){\bf $1.5$}}
\put(110,75){\makebox(10,7){\it $2.5$}}
\put(140,75){\makebox(10,7){\bf $6.5$}}
\put(170,75){\makebox(10,7){\it $15.5$}}
\put(200,75){\makebox(10,7){\bf $40.5$}}
\put(230,75){\makebox(10,7){\it $105.5$}}
\put(263,75){\makebox(10,7){\bf $280.5$}}
\put(110,105){\makebox(10,7){\bf $1.7$}}
\put(140,105){\makebox(10,7){\it $3.7$}}
\put(170,105){\makebox(10,7){\bf $10.7$}}
\put(200,105){\makebox(10,7){\it $29.7$}}
\put(230,105){\makebox(10,7){\bf $84.7$}}
\put(140,135){\makebox(10,7){\bf $1.9$}}
\put(170,135){\makebox(10,7){\it $4.9$}}
\put(200,135){\makebox(10,7){\bf $15.9$}}
\put(170,165){\makebox(10,7){\bf $1.11$}}
%\put(225.00,145.00){\circle*{4.00}}
%\put(225.00,145.00){\circle{8.00}}
%\put(225.00,130.00){\circle*{4.00}}
%\put(230,145){\makebox(0,0)[l]{ Number of symmetric subspaces}}
%\put(230,130){\makebox(0,0)[l]{ Number of antisymmetric subspaces}}


%\put(15.00,15.00){\vector(1,1){28.00}}


\end{picture}
\end{center}
\caption{Symmetries of spin-$1$ particle states.
\label{2005-singlet-f1-ta-takohalf} Even subspaces are denoted by
concentric circles, odd subspaces are denoted by filled circles.
The numbers denote the dimension of the subspaces. The first
number stands for the number of states $|h,j\rangle$ and the
second stands for the $2j+1$ projections. Arrows represent the way
of coupling. }
\end{figure}

\subsection{Symmetries of the states obtained from consideration of
the symmetric group}We want to assign the appropriate
representation to irreducible singlet spaces. Therefore we
consider the symmetric group. In every product state of every
$N$-particle state we permute the $N$ magnetic quantum numbers,
more explicitly, we apply $(N-1)$ transpositions, since every
permutation of $N$ particles can be written as the product of
(N-1) transpositions. We analyse $(N-1)$ transpositions of the
form $(j,j+1)$, the transposition of $j$ and $j+1$, which generate
the whole symmetric group and in particular all the $N\cdot
(N-1)/2$ transpositions, since $(j,k+1)=(k,k+1)(j,k)(k,k+1).$
%~\cite{fripertinger}.
Hence we look at the class $(21^{N-2})$. Each irreducible
representation can be labelled by an ordered partition of integers
which corresponds to a specific Young diagram.
\subsubsection{Application of the symmetries to the $J={1\over 2}$ case}
``The space spanned by the vectors of total spins $(SM)$ formed by
$N$ identical spins ${1\over 2}$ is associated with an irreducible
representation of $S_N$, the representation whose Young diagram
corresponds to the partition $[{1\over 2}N+S,{1\over 2}N-S]$ of
the integer $N$.'' (App.~D, Sec.~14 of Ref.~\cite{messiah-62}) It
is apparent that the Young tableaux for the irreducible components
of the representation of $S_N$ have at most two lines. For $N>2$,
any state contains at least two individual spins in the same
state, supposing that it contains the factor $u_+^{(i)}u_+^{(j)}$,
i.e. $m_i, m_j ={1\over 2}$. Since $A=A{1\over 2}(1-(i,j))$, $A$
is the antisymmetrizer, and ${1\over
2}(1-(ij))u_+^{(i)}u_+^{(j)}=0$, necessarily $A|jm\rangle=0$.\\\\

Using the theorem mentioned above the Young diagrams of the
irreducible spaces of the N-particle singlet states correspond to
the partitions $[{1\over 2}N,{1\over 2}N]$. Hence the two-particle
singlet state is an antisymmetric one dimensional space. The four-
and six-particle singlet spaces form a two and a five dimensional
irreducible space whose Young diagrams are of the form $[2,2]$ and
$[3,3]$. Using the formula for the dimension of an irreducible
representation having the partition $[\lambda]$
\begin{align}f^\lambda = n!\frac{\Pi_{i<j\leq
k}(\lambda_i-\lambda_j+j-i)}{\Pi^k_{i=1}(\lambda_i+k-i)!}\label{dim}\end{align}
(e.g., Ref.~\cite{wybourne}) we can check the dimension.
\subsubsection{Application of the symmetries to the $J=1$ case}
Here, the Young tableaux for the irreducible components of the
representation of $S_N$ have at most three lines. For $N>3$, any
state contains at least two individual spins in the same state,
hence we can argue as above. We calculate the matrix
representations of the
 (N-1) transpositions.\\Next we use Schur's Lemma: If a matrix $S$ commutes
with all the matrices of an irreducible representation $G$ of a
group, then it is a multiple of the unit matrix (e.g., App.~D,
Sec.~8 of Ref.~\cite{messiah-62}) to check if the representation
is irreducible.\\ Furthermore, we can calculate the characteristic
of each
element. The %spur or %
trace of the matrix representing the element $S_i$ which belongs
to the irreducible representation $\Gamma^{(j)}$ is called the
characteristic of $S_i$ in $\Gamma^{(j)}$ and is denoted by
$\chi^{(j)}(S_i)$. The set of characteristics of all elements $S$
of the group as represented in $\Gamma^{(j)}$ is called a group
character $\chi^{(j)}$. All elements of the same class $\rho$ have
the same characteristic $\chi^{(j)}_\rho$ ~\cite{wybourne}.\\ We
use the tables of characters to compare our results and to find
the appropriate partition. We check our results by calculating the
characteristic of one element per class. Moreover we consider the
outer product of irreducible representations. Considering tables
(e.g., List of tables, B1 of Ref.~\cite{wybourne}).\\ The
two-particle singlet state is a one dimensional symmetric space.
The three-particle singlet state is an antisymmetric one
dimensional space. Hence their Young diagrams are of the form
$[2]$ and $[1^3]$, respectively. Again we can check the dimension
using eq.~\ref{dim}. For four particles the three dimensional
space (see Table~\ref{2005-singlet-t1}) of singlets is reducible
and the class $(21^2)$ has in this representation characteristic
1. The character of a reducible representation $\Gamma^{(j)}$ is
the sum of the characters of the irreducible parts of
$\Gamma^{(j)}$.
%(e.g., Satz 22 of Ref.~\cite{dolhaine}).
Hence we build the class sum, the sum of all the matrix
representations of the transpositions. Calculating the eigenvalues
and the eigenvectors yields the result that the space decomposes
into a one and a two dimensional subspace. The one dimensional
state is symmetric $[4]$. The two dimensional irreducible
representation has characteristic 0, which yields, using the table
of characters, the partition [22]. We consider the outer product
of two irreducible representations whose Young diagrams correspond
to the partition [2], ``the product of two two-particle singlet
states'':
\begin{align} [2]\times[2]=[4]+[31]+[22].\end{align}
This also yields the wanted partitions. But there are further
restrictions, e.g. $m_1=m_3$. Hence only some of the partitions
are realized. For five particles the six dimensional space is
irreducible and has the partition $[31^2]$. This we derive from
the table of characters.
%(e.g.,Ref.~\cite{SYMMETRICA}).
Looking at the outer product of [111]x[2], ``the product of the
three- and the two-particle singlet state'', yields
\begin{align} [111]\times[2]=[31^2]+[21^3].\end{align}
Here the first partition has the required dimension. The 15
dimensional space of singlets for six particles is reducible. In
this representation the class $(21^4)$ has characteristic three.
The space decomposes into a one, a five and a nine dimensional
irreducible subspace. Using the outer product yields for
[2]x[2]x[2], ``the two-particle singlet state to the cube'',
\begin{align} [4]\times[2]=[42]+[51]+[6]\qquad\mbox{and}\qquad
[22]\times[2]=[42]+[321]+[222],\label{dim-6.1}
\end{align}
and for ``the product of two three-particle singlet states''
\begin{align} [111]\times[111]=[111111]+[21111]+[2211]+[222]\label{dim-6.2}.\end{align}
The nine dimensional representation has characteristic three which
yields, by using the eq.~\ref{dim} and the table of characters,
that the wanted partition is [42]. For the five dimensional
irreducible representation we find characteristic -1, which leads
to the partition $[2^3]$. Crosschecking with eq.~\ref{dim-6.1} and
eq.~\ref{dim-6.2} we see that, as required, these partitions are
contained in the outer products. Finally, the one dimensional
state has characteristic one which leads to the partition [6],
hence it is a
symmetric state.\\\\
Using the outer product, neglecting tableaux with more then three
lines, we give the partitions for up to ten particles.\\

\begin{tabular}{rl}
N &partition\nonumber \\
 7& [331],[511]\nonumber\\
 8& [8],[62],[44],[422] \\
 9& [711],[531],[333]\nonumber\\
10& [10],[82],[64],[442],[622].\nonumber
\end{tabular}\\

Conjecture: Following this we proclaim that for $N$ even one
obtains all possible diagrams from the diagram [N] by splitting
the partition [2] as often
as possible and adding them again to all possible tableaux with at most three lines.\\
For $N$ odd one obtains all diagrams by splitting [2] as often as
possible from the tableaux [N-2,11] and adding them again to the
diagram to all possible regular tableaux with at most three lines.
\begin{longtable}{llllll}
%\begin{tabular}{ccccc}
\hline\hline N  & \\\hline\hline
6&$-\frac{3\sqrt{3}}{7}|0,0,0,0,0,0\rangle +\frac{3
\sqrt{3}}{35}\big(|-1,0,0,0,0,1\rangle +|-1,0,0,0,1,0\rangle
+|-1,0,0,1,0,0\rangle +  $\\&$ |-1,0,1,0,0,0\rangle
+|-1,1,0,0,0,0\rangle +|0,-1,0,0,0,1\rangle +|0,-1,0,0,1,0 \rangle
+ $\\&$ |0,-1,0,1,0,0\rangle +|0,-1,1,0,0,0\rangle
+|0,0,-1,0,0,1\rangle +|0,0,-1,0,1,0\rangle +  $\\&$
|0,0,-1,1,0,0\rangle +|0,0,0,-1,0,1\rangle +|0,0,0,-1,1,0\rangle
+|0,0,0,0,-1,1\rangle +  $\\&$ |0,0,0,0,1,-1\rangle
+|0,0,0,1,-1,0\rangle +|0,0,0,1,0,-1\rangle +|0,0,1,-1,0,0\rangle
+  $\\&$ |0,0,1,0,-1,0\rangle +|0,0,1,0,0,-1\rangle
+|0,1,-1,0,0,0\rangle +|0,1,0,-1,0,0\rangle +  $\\&$
|0,1,0,0,-1,0\rangle +|0,1,0,0,0,-1\rangle +|1,-1,0,0,0,0\rangle
+|1,0,-1,0,0,0\rangle +  $\\&$ |1,0,0,-1,0,0\rangle
+|1,0,0,0,-1,0\rangle +|1,0,0,0,0,-1\rangle\big) +
-\frac{2\sqrt{3}}{35}\big(|-1,-1,0,0,1,1\rangle +  $\\&$
|-1,-1,0,1,0,1\rangle +|-1,-1,0,1,1,0\rangle
+|-1,-1,1,0,0,1\rangle +|-1,-1,1,0,1,0\rangle +  $\\&$
|-1,-1,1,1,0,0\rangle +|-1,0,-1,0,1,1\rangle
+|-1,0,-1,1,0,1\rangle +|-1,0,-1,1,1,0\rangle +  $\\&$
|-1,0,0,-1,1,1\rangle +|-1,0,0,1,-1,1\rangle
+|-1,0,0,1,1,-1\rangle +|-1,0,1,-1,0,1\rangle +  $\\&$
|-1,0,1,-1,1,0\rangle +|-1,0,1,0,-1,1\rangle
+|-1,0,1,0,1,-1\rangle +|-1,0,1,1,-1,0\rangle +  $\\&$
|-1,0,1,1,0,-1\rangle +|-1,1,-1,0,0,1\rangle
+|-1,1,-1,0,1,0\rangle +|-1,1,-1,1,0,0\rangle +  $\\&$
|-1,1,0,-1,0,1\rangle +|-1,1,0,-1,1,0\rangle
+|-1,1,0,0,-1,1\rangle +|-1,1,0,0,1,-1\rangle +  $\\&$
|-1,1,0,1,-1,0\rangle +|-1,1,0,1,0,-1\rangle
+|-1,1,1,-1,0,0\rangle +|-1,1,1,0,-1,0\rangle +  $\\&$
|-1,1,1,0,0,-1\rangle +|0,-1,-1,0,1,1\rangle
+|0,-1,-1,1,0,1\rangle +|0,-1,-1,1,1,0\rangle +  $\\&$
|0,-1,0,-1,1,1\rangle +|0,-1,0,1,-1,1\rangle
+|0,-1,0,1,1,-1\rangle +|0,-1,1,-1,0,1\rangle +  $\\&$
|0,-1,1,-1,1,0\rangle +|0,-1,1,0,-1,1\rangle
+|0,-1,1,0,1,-1\rangle +|0,-1,1,1,-1,0\rangle +  $\\&$
|0,-1,1,1,0,-1\rangle +|0,0,-1,-1,1,1\rangle
+|0,0,-1,1,-1,1\rangle +|0,0,-1,1,1,-1\rangle +  $\\&$
|0,0,1,-1,-1,1\rangle +|0,0,1,-1,1,-1\rangle
+|0,0,1,1,-1,-1\rangle +|0,1,-1,-1,0,1\rangle +  $\\&$
|0,1,-1,-1,1,0\rangle +|0,1,-1,0,-1,1\rangle
+|0,1,-1,0,1,-1\rangle +|0,1,-1,1,-1,0\rangle +  $\\&$
|0,1,-1,1,0,-1\rangle +|0,1,0,-1,-1,1\rangle
+|0,1,0,-1,1,-1\rangle +|0,1,0,1,-1,-1\rangle +  $\\&$
|0,1,1,-1,-1,0\rangle +|0,1,1,-1,0,-1\rangle
+|0,1,1,0,-1,-1\rangle +|1,-1,-1,0,0,1\rangle +  $\\&$
|1,-1,-1,0,1,0\rangle +|1,-1,-1,1,0,0\rangle
+|1,-1,0,-1,0,1\rangle +|1,-1,0,-1,1,0\rangle +  $\\&$
|1,-1,0,0,-1,1\rangle +|1,-1,0,0,1,-1\rangle
+|1,-1,0,1,-1,0\rangle +|1,-1,0,1,0,-1\rangle +  $\\&$
|1,-1,1,-1,0,0\rangle +|1,-1,1,0,-1,0\rangle
+|1,-1,1,0,0,-1\rangle +|1,0,-1,-1,0,1\rangle +  $\\&$
|1,0,-1,-1,1,0\rangle +|1,0,-1,0,-1,1\rangle
+|1,0,-1,0,1,-1\rangle +|1,0,-1,1,-1,0\rangle +  $\\&$
|1,0,-1,1,0,-1\rangle +|1,0,0,-1,-1,1\rangle
+|1,0,0,-1,1,-1\rangle +|1,0,0,1,-1,-1\rangle +  $\\&$
|1,0,1,-1,-1,0\rangle +|1,0,1,-1,0,-1\rangle
+|1,0,1,0,-1,-1\rangle +|1,1,-1,-1,0,0\rangle +  $\\&$
|1,1,-1,0,-1,0\rangle +|1,1,-1,0,0,-1\rangle
+|1,1,0,-1,-1,0\rangle +|1,1,0,-1,0,-1\rangle +  $\\&$
|1,1,0,0,-1,-1\rangle\big)
+\frac{6\sqrt{3}}{35}\big(|-1,-1,-1,1,1,1\rangle
+|-1,-1,1,-1,1,1\rangle +|-1,-1,1,1,-1,1\rangle +  $\\&$
|-1,-1,1,1,1,-1\rangle +|-1,1,-1,-1,1,1\rangle
+|-1,1,-1,1,-1,1\rangle +|-1,1,-1,1,1,-1\rangle +  $\\&$
|-1,1,1,-1,-1,1\rangle +|-1,1,1,-1,1,-1\rangle
+|-1,1,1,1,-1,-1\rangle +|1,-1,-1,-1,1,1\rangle +  $\\&$
|1,-1,-1,1,-1,1\rangle +|1,-1,-1,1,1,-1\rangle
+|1,-1,1,-1,-1,1\rangle +|1,-1,1,-1,1,-1\rangle +  $\\&$
|1,-1,1,1,-1,-1\rangle +|1,1,-1,-1,-1,1\rangle
+|1,1,-1,-1,1,-1\rangle +|1,1,-1,1,-1,-1\rangle +  $\\&$
|1,1,1,-1,-1,-1\big)$\\
\hline\hline\\
%\end{tabular}
\caption{The one dimensional irreducible six-particle singlet
state. \label{EV1}}
\end{longtable}
\clearpage





\bibliography{svozil}
\bibliographystyle{osa}
%\bibliographystyle{apsrev}
%\bibliographystyle{unsrt}
%\bibliographystyle{plain}
%\bibliographystyle{aipprocl}


\end{document}


\begin{thebibliography}{99}
\bibitem{chaichian} M. Chaichian and R. Hagedorn, {\it Symmetries in Quantum Mechanics},
(Institute of Physics Publishing, Bristol and Philadelphia, 1998).
\bibitem{wybourne} B. Wybourne, {\it Symmetry Principles and
Atomic Spectroscopy}, (Wiley Interscience, USA, 1970).
%\bibitem{SYMMETRICA} SYMMETRICA, URL
%http://www.mathe2.uni-bayreuth.de/axel/symneu_engl.html.
%\bibitem{fripertinger} A. Betten, H. Fripertinger, A. Kerber, {\it Algebraic Combinatorics
%Via Finite Group Actions}, (2005), URL
%http://www.mathe2.uni-bayreuth.de/frib/html/book/hyl00.html.
%\bibitem{dolhaine} H. Dolhaine, {\it manuscripts in mathematical and in m_match
%computer chemistry}, (Univ. of Bayreuth, Bayreuth, 2000), URL
%http://www.mathe2.uni-bayreuth.de/match/online/dolhaine.pdf.
\bibitem{gaylord} R. Gaylord, S. Kamin, P.Wellin, {\it Introduction to Programming
with Mathematica}, (Springer-Verlag, New York, 1993).\\
\bibitem{messiah} A. Messiah, {\it Quantenmechanik 2}, (Walter de Gruyter, Berlin.
New York, 1990).
\bibitem{cohen-t} C. Cohen-Tannoudji, B. Diu, F.
Lalo\"e, {\it Quantum Mechanics, Volume two}, (Hermann, Paris,
1977).
\bibitem{breuer}H. Breuer, {\it Entanglement in $SO(3)$-invariant bipartite quantum
systems},\href{http://arxiv.org/abs/quant-ph/0503079v1}{hep-th/0503079}.
\bibitem{schliemann} J. Schliemann, {\it Entanglement in
SU(2)-invariant quantum spin systems}, Physical Review A {\bf68},
012309 (2003), URL http://dx.doi.org/10.1103/PhysRevA.68.012309.
\end{thebibliography}



\subsection*{\bf How to calculate the states}

I calculate all singlet states for a specific number of particles.
I look at bosonic particles with spin( angular momentum) j=1 and
consequently magnetic quantum number m = -1, 0, 1. There is no
singlet state for one particle. For N = 2, 3, 4, 5, 6, 7, 8, 9, 10
particles, the number of singlet states is 1, 1, 3, 6, 15, 36, 91,
232, 603 respectively. First I calculate the states for different
angular momentum, in the coupled basis of two particles. To couple
a third particle I take the coupled state of two particles as
one-particle state and couple a third particle. (Skizze: NUMBER OF
STATES: nr. of particles, angular momentum) In the numbers in the
graphic, I didn't include the magnetic quantum number m, so each
nrumber has to be multiplied by 2j+1 to get the nr. of states for
each particlenumbers n and angular momentum j. (I will stick to
this in the following.) To see which n-1 particle states
contribute to which n particle state we have a look at the
triangular equation $|j1-j2|\leq j \leq j1+j2$. In your case j2 is
always 1, so we could write it like this $|j_{(n-1)}-1|\leq j \leq
j_{(n-1)}+1$. For j = 0 it follows $|j_{(n-1)}-1|\leq 0 \leq
j_{(n-1)}+1$, so $j_{(n-1)}$ has to be 1 to fulfill the equation.
We see that to the states at particlenumber n and angular momentum
j, for $j>0$ states from n-1 particles and angular momentum j-1,
-j and j+1 contribute. Where as to the singlet states at n
particles only the spin 1 states at n-1 particles contribute.\\

\subsection*{\bf How the mathematica-program works}
In my mathematica package there are four functions
anzahl[N], viereck[N]%DREIECK?
, drehimpuls1[N], drehimpuls2[N]. In anzahl I calculate the
nrumber of states for particle nrumber n and angular momentum j.
To do that I have to sum the numbers of states N-1 and angular
momentum j-1, j, j+1. I get as many singlet states as I have spin
1 states at N-1. To get all singlet states for N particles, I have
to know all spin states $0\leq j\leq k$ for k up to Floor(N/2)
particles. Then I need all spin states $0\leq j\leq N-k$ for
Floor(N/2)+1$\leq k\leq N$ particles.

\subsubsection*{\bf Description of the functions}
anzahl\\\\
This function calculates the number of states for angular momentum
$0\leq j\leq$ Floor(N/2) and particle number $2\leq k\leq N$. I
define a matrix with Floor(N/2)+2 lines,( I need one more line for
the algorithm) and N columns. In a Do-loop I set all components to
zero. In the next Do-loop I calculate the number of states for
$k\leq$ Floor(N/2) particles and in the following one, the nrumber
of states for the rest Floor(N/2)+1 to N particles. f[j+1,h]
signifies the number of states
at h particles and angular momentum j.\\\\
viereck\\\\
In viereck I build a structure for the states. I define a matrix L
of dimension Floor(N/2)+1 x N, who's elements are submatrices with
2j+1 lines, for the number of magnetic momentums, and the number
of states for j and N is the number of columns.
So to the state $|$n,j,m,number of state$>$ is referred to as g[j+1,n][m+j+1,f(j+1,n)], which is a list.\\\\
drehimpuls1\\\\
This function calculates all the states until Ceiling(N/2). a[x],
b[x], c[x] are functions which append 1, 0, -1 respectively to the
end of a list. The Outermost Do-loop loops over values of j from 0
to Floor(N/2) for each particle number h, h from 2 to
Ceiling(N/2). In the next one we loop over k = 1, 2j+1 for each i,
i = 1 to f[(j+1)+1,h-1]. We first look at the state
$|h-1,j+1,m1,1>$. Now we check if the intersection of m1, the Rang
of magnetic momentums of this state, and m-1 is not empty and if
the Clebsch-Gordan coefficient $<$j+1 m-1 1 1$|$j m$>$ is not
equal to zero, not do consider states which are zero, in that
cause we go on, otherwise ga is an empty list. We multiply the
state by the Clesch-Gordan coefficient $<$j+1 m-1 1 1$|$j m$>$ and
take the first argument of this list, which is the new
Clebsch-Gordan coefficient. Then we take the rest of the state
$|h-1,j+1,m1,1>$, everything expect for the Clebsch-Gordan
coefficient and append 1 to it. Now we make into one list with
MapThread[List,....] and Map[Flatten,....]. The next step is to
calculate gb. Now we check if m1 and m intersect. The state which
contributes now is $|h-1,j+1,m,i>$. It is multiplied by the
Clebsch-Gordan coefficient $<$j+1 m 1 0$|$j m$>$ and now we append
0. To calculate gc we have to check if m1 and m+1 intersect. We
have to take into account the Clebsch-Gordan coefficient $<$j+1
m+1 1 -1$|$j m$>$ and append -1. Finally we join ga, gb and gc and
get the states g[j+1,h][m+j+1,i] for all different magnetic
momentum quantum numbers -j to j and for i from 1 to
f[(j+1)+1,h-1]. If j is bigger than zero we look at the state
$|h-1,j,m1,i>$. We have the same three Do-loops as before: First
we check the intersection of m1 and m-1 ..... . We get the states
g[j+1,h][m+j+1,i] for all different magnetic momentum quantum
numbers -j to j and for i from f[(j+1)+1,h-1]+1 to
f[(j+1)+1,h-1]+f[j+1,h-1].\\
To get the rest of the g[j+1,h] states, if $j>0$, we take into consideration $|h-1,j-1,m1,i>$ and proceed as before.\\\\
drehimpuls2\\\\
In drehimpuls2 two we calculate the states from Ceiling(N/2)+1 to
N particles. The Outermost Do-loop is looping over j = 0, N-h for
each value of h = Ceiling(N/2)+1, N. It drehimpuls2 proceeds like
drehimpuls1. So in the end we get all our singlet states for N
particles.






































\end{document}
