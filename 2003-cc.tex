%%tth:\begin{html}<LINK REL=STYLESHEET HREF="http://tph.tuwien.ac.at/~svozil/ssh.css">\end{html}
%\documentclass[prl,preprint,showpacs,showkeys,amsfonts]{revtex4}
%\usepackage{graphicx}
\documentstyle[]{article}
%\RequirePackage{times}
%\RequirePackage{courier}
%\RequirePackage{mathptm}
\renewcommand{\baselinestretch}{1.3}
\begin{document}





\title{Some motivation for the Cantor expanson of random reals in
physics}
\author{}
\date{ }
\maketitle

\begin{abstract}
Just a few comments \& examples
of the usefulness of Cantor expansion in physical systems.
\end{abstract}

%http://www.chemguide.co.uk/basicorg/isomerism/polarised.html
%http://www.exploratorium.edu/snacks/polarized_mosaic.html
%\pacs{03.67.-a,03.67.Hk,03.65.Ta}
%\keywords{quantum information theory,quantum measurement theory}


One of the first examples of the usefulness
of Cantor expansions in physical systems
that comes into one's mind
is a ball in gravitational fall
impinging onto a  board of nails with different numbers
$f: n \rightarrow  b_n$ of nails
at different horizontal levels
(here, $n$ stands for the $n$'th horizontal level,
and $b_n$ stands for the basis corresponding to the position $n$).
A real number $R.r_1r_2\cdots r_n \cdots $ in the Cantor expansion
can be constructed as follows.
Let us assume that the layers are ``sufficiently far apart''
(and that there are periodic boundary conditions realizable by
elastic mirrors).
Then, depending on which one of the $n$ openings the ball takes,
one identifies the associated number
(counted from $0$ to $b_n-1$) with
the $n$'th position  $r_n \in \{0,\ldots ,b_n-1\}$ after the comma.


Let us next consider a quantum correspondent to the  board of nails
harnessing irreducible complementarity and the randomness  in the outcome of measurements on single  particles.
Take a quantized system with at least two  complementary
observables
$\hat{A},\hat{B}$,
each one associated with $N$ different outcomes $a_i,b_j$, $i,j \in
\{0,\ldots ,N-1\}$, respectively.
Notice that, in principle, $N$ could be a large (but finite) number.
Suppose further that $\hat{A},\hat{B}$ are ``maximally'' entangled
in the sense that measurement of $\hat{A}$ totally randomizes the outcome
of $\hat{B}$ and {\it vice versa}
(this should not be confused with optimal mutually unbiased measurements \cite{WooFie}).
A real number $R.r_1r_2\cdots r_n \cdots $ in the Cantor expansion
can be constructed from successive  measurements of $\hat{A}$
and $\hat{B}$ as follows.
Since all bases $b_n$ used for the Cantor
expansion are assumed to be bounded, choose $N$ to be
the Least Common Multiple of all bases $b_n$.
Then partition the $N$ outcomes into even partitions,
one per different base,
containing as many elements as are required for
associating different elements of the $n$'th partition
with  numbers from $\{0,\ldots ,b_n-1\}$.
Then, by measuring $\hat{A},\hat{B},\hat{A},\hat{B},\hat{A},\hat{B},\ldots$
successively, the $n$'th position  $r_n \in \{0,\ldots ,b_n-1\}$
can be identified
with the number associated with the element of the partition which
contains the measurement outcome.

As an example, consider a Cantor expansion of a number in the bases
 2, 6, and 9. As the  Least Common Multiple is 18, we choose two
observables with 18 different outcomes; e.g., angular momentum
components in two perpendicular directions of a particle of total
angular
momentum  ${9\over 2} \hbar$ with outcomes in  (units are in $\hbar$)
$\{ -{9\over 2},-4,-{7\over 2},\ldots, +{7\over 2},+4,+{9\over 2}\}$.
Associate with the outcomes the set
$\{0,1,2,\ldots ,17\}$,
and form the even partitions
\begin{eqnarray}
&\{\{0,1,2,3,4,5,6,7,8\},\{9,10,11,12,13,14,15,16,17\}\},&\nonumber  \\
&\{\{0,1,2\},\{3,4,5\},\{6,7,8\},\{9,10,11\},\{12,13,14\},\{15,16,17\}\},&\nonumber  \\
&\{\{0,1\},\{2,3\},\{4,5\},\{6,7\},\{8,9\},\{10,11\},\{12,13\},\{14,15\},\{16,17\}\},&\nonumber
\end{eqnarray}
(or any partition obtained by permutating the elements of $\{0,1,2,\ldots ,17\}$)
associated with the bases  2, 6, and 9, respectively.
Then, upon successive measurements of angular momentum
components in the two perpendicular directions yields
outcomes  in the bases  2, 6, and 9.

As the above quantum example may appear ``cooked up,'' since  the
coding is based on a uniform radix $N$ expansion, one
might consider successive measurements of the location and the velocity of
a single particle. In such a case, the value $r_n$ is obtained by associating with it
the click in a particular detector (or a range thereof) associated with
spatial or momentum measurements. Any such arrangements are not very
different in principle, since every measurement of a quantized system
corresponds to registering a discrete event associated with a detector
click \cite{sum-3}.

%The genuine strength of the Cantor expansion unfolds when we consider varying choices and varying interactions on different scale levels.

\bibliography{svozil}
%\bibliographystyle{apsrev}
\bibliographystyle{unsrt}

\end{document}

