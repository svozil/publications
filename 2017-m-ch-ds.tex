\chapter{Divergent series}
\index{divergent series}

\label{2011-m-ch-ds}
In this final chapter we will consider {\em divergent series}, which sometimes occur
in physical situations;
for instance in celestial mechanics or in quantum field theory
\cite[-30mm]{Boyd99thedevil,PhysRev.85.631,PhysRevD.57.1144,PhysRevD.62.076001}.
According to Abel\cite{Hardy:1949,Boyd99thedevil}
appear to be the ``invention of the devil,'' and one may wonder with him~\cite{rousseau-2004} why, {\em ``for the most part,
it is true that the results are correct, which is very strange.''}
There appears to be another, complementary, view on diverging series,
a view that has been expressed by Berry as follows\cite{berry-92}:
{\em ``$\ldots$ an asymptotic series $\ldots$ is a compact encoding of a function,
and its divergence should be regarded not as a deficiency but as a source of information about the function.''
}

\section{Convergence and divergence}
Let us first define {\em convergence} in the context of series.
\index{convergence}
A series
\begin{equation}
\sum_{j=0}^\infty a_j =a_0+a_1+a_2+\cdots
\end{equation}
is said to converge to the {\em sum}
$s$, if the {\em partial sum}
\begin{equation}
s_n=  \sum_{j=0}^n a_j =a_0+a_1+a_2+\cdots + a_n
\end{equation}
tends to a finite limit $s$ when $n\rightarrow \infty$;
otherwise it is said to be divergent.
\index{divergence}

% https://www.coursera.org/learn/advanced-calculus/lecture/koIrf/does-the-series-1-1-2-1-3-converge-or-diverge
A widely known diverging series is the
{\em harmonic series}
\index{harmonic series}
\begin{equation}
s=  \sum_{j=1}^\infty \frac{1}{j} = 1+\frac{1}{2}+\frac{1}{3}+\frac{1}{4}+\cdots .
\end{equation}
A medieval proof by Oresme (cf. p.~92 of Ref.~\cite[10mm]{Edwards-1979})
uses approximations: Oresme points out that increasing numbers of summands in the series can be rearranged to yield numbers bigger than, say, $\frac{1}{2}$;
more explicitly,
$\frac{1}{3}+\frac{1}{4}> \frac{1}{4}+\frac{1}{4}  =\frac{1}{2}$,
$\frac{1}{5}+ \cdots +\frac{1}{8} > 4 \frac{1}{8}  =\frac{1}{2}$,
$\frac{1}{9}+ \cdots +\frac{1}{16} > 8 \frac{1}{16}  =\frac{1}{2}$,
and so on, such that the entire series must grow larger as $\frac{n}{2}$.
As $n$ approaches infinity, the series is unbounded, and thus diverges.

More generally, the Riemann zeta function (sometimes also referred to as the Euler-Riemann zeta function),
\index{zeta function}
\index{Riemann zeta function}
\index{Euler-Riemann zeta function}
defined for $\Re t > 1$ by
\begin{equation}
\zeta (t) \stackrel{{\rm def}}{=}
\sum_{n=1}^\infty \frac{1}{n^t}
=
\prod_{p\;{\rm prime}} \left(\sum_{n=1}^\infty p^{-nt}\right)
=
\prod_{p\;{\rm prime}} \frac{1}{1-\frac{1}{p^t}}
,
\label{2016-m-ds-zeta}
\end{equation}
can be  continued analytically to all complex values $t\neq 1$, so that, for $t=-1$,
$\zeta (-1) = \sum_{n=1}^\infty n = 1+2+3+4+5 + \cdots $.
It will be re-defined later in Eq.~(\ref{2016-m-ch-ds-natnumser}).


One of the most prominent divergent series is the Grandi's series, sometimes also referred to as
\index{Grandi's series}
\index{Leibniz's series}
Leibniz series~\cite{leibnitz-1860,moore-1938,Hardy:1949,everest-2003}
\begin{equation}
s = \sum_{j=0}^\infty (-1)^j=1-1+1-1+1-\cdots ,
\label{2009-fiftyfifty-1s}
\end{equation}
whose summands may be -- inconsistently
-- ``rearranged,''
yielding
\begin{equation*}
\begin{split}
\textrm{ either }
1-1+1-1+1-1+\cdots = (1-1)+(1-1)+(1-1)-\cdots =0\\
\textrm{ or }
1-1+1-1+1-1+\cdots = 1+(-1+1)+ (-1+1) +\cdots =1.
\end{split}
\end{equation*}
One could tentatively associate the arithmetical average $1/2$ to represent ``the sum of Grandi's series.''

Another tentative approach  would be to first regularize this nonconverging expression by introducing a ``small entity''
$\varepsilon$ with $0 < \varepsilon <1$, such that $\vert \varepsilon - 1 \vert  <1$, which allows to formally sum up the geometric series
\begin{equation*}
s_\varepsilon \stackrel{\text{def}}{=} \sum_{j=0}^\infty (\varepsilon -1)^j=\frac{1}{1-(\varepsilon -1)}   =\frac{1}{2- \varepsilon  };
\end{equation*}
and then take the limit $s \stackrel{\text{def}}{=}  \lim_{\varepsilon \rightarrow 0^+} s_\varepsilon
= \lim_{\varepsilon \rightarrow 0^+}  1/(2- \varepsilon  ) =   1/2$.


By {\em Riemann's rearrangement theorem},
\index{Riemann rearrangement theorem}
convergent series  which do not absolutely converge
(i.e., $\sum_{j=0}^n a_j$ converges but $\sum_{j=0}^n \left| a_j \right|$ diverges)
can converge to any arbitrary (even infinite) value
by permuting (rearranging) the (ratio of) positive and negative terms
(the series of which must both be divergent).

\section{Abel summation}
\index{Abel sum}

The Grandi series is a particular ``disallowed'' case $q=-1$ of a {\em geometric series}
\index{geometric series}
\begin{equation}
s(q) = \sum_{j=0}^\infty q^j=1+q+q^2+q^3+ \cdots  =1+q s
\label{2009-fiftyfifty-1sgs}
\end{equation}
which, since $s(q)=1+q s(q)$, for $\vert q\vert <1$, converges  to
\begin{equation}
s(q)= \sum_{j=0}^\infty q^j= \frac{1}{1-q}.
\label{2012-fiftyfifty-1sgscont}
\end{equation}
One way to sum the Grandi series is by ``continuing''
Eq. (\ref{2012-fiftyfifty-1sgscont})
for arbitrary $q\neq 1$, thereby defining the
{\em Abel sum}
\index{Abel sum}
\begin{equation}
s=\sum_{j=0}^\infty(-1)^j  = 1-1+1-1+1-1+\cdots \stackrel{{\rm A}}{=} \frac{1}{1-(-1)} = \frac{1}{2}.
\label{2012-fiftyfifty-1sgscont1}
\end{equation}


Another divergent series,
which can be obtained by formally expanding  the square of the Abel sum of the
Grandi series
$
s^2 \stackrel{{\rm A}}{=} [1-(-x)]^{-2} = (1+x)^{-2}
$ for  $x=1$
around $x_0=0$ and inserting\cite{Kline-83}   $x-x_0=x=1$
\begin{equation}
\begin{split}
s^2   \stackrel{{\rm A}}{=} (1+x)^{-2}  |_{x =1}=
\sum_{j=0}^\infty \frac{1}{n!}\left.\left[\frac{d^n}{d{x_0}^n}  (1+x_0)^{-2}\right]   (1-x_0)^n\right|_{x_0=0}
\\
=
\sum_{j=0}^\infty (-1)^{j} (j+1) =
\sum_{j=0}^\infty (-1)^{j+1} j =
1-2+3-4+5-\cdots
.
\label{2009-fiftyfifty-1s1}
\end{split}
\end{equation}
In the same sense as the Grandi series, this yields the Abel sum
\begin{equation}
s^2 =
\left(\sum_{j=0}^\infty (-1)^j\right)
\left(\sum_{k=0}^\infty (-1)^k\right)
\stackrel{{\rm A}}{=} \left(\frac{1}{2}\right)^2=\frac{1}{4}.
\end{equation}



Once this identification is established, all of Abel's hell breaks loose:
One could, for instance, ``compute the finite sum of all natural numbers''
(a sum even mentioned on page~22 in a book on String Theory~\cite{polchinski_1998}) {\it via}
\begin{equation}
S  =
\sum_{j=0}^\infty j = 1+2+3+4+5 + \cdots \stackrel{{\rm A}}{=} - \frac{1}{12}
\label{2016-m-ch-ds-natnumser}
\end{equation}
by sorting out
\begin{equation}
\begin{split}
S - \frac{1}{4}\stackrel{{\rm A}}{=}
S-s^2  = 1+2+3+4+5 + \cdots -
(1-2+3-4+5-\cdots )   \\
= 4 + 8+ 12 +  \cdots = 4 S
,
\label{2016-m-ch-ds-natnumser2}
\end{split}
\end{equation}
so that  $3 S
\stackrel{{\rm A}}{=}  - \frac{1}{4}$,
and, finally, $S
\stackrel{{\rm A}}{=}  - \frac{1}{12}$.

Note that the sequence of the partial sums $s^2_n=\sum_{j=0}^n (-1)^{j+1} j $
of $s^2$
yield every integer once; that is,
$s^2_0 =0$,
$s^2_1 =0+1=1$,
$s^2_2 =0+1-2=-1$,
$s^2_3 =0+1-2+3=2$,
$s^2_4 =0+1-2+3-4=-2$,
$\ldots$,
$s^2_n =-\frac{n}{2}$ for even $n$,
and
$s^2_n =-\frac{n+1}{2}$ for odd $n$.
It thus establishes a strict one-to-one mapping
$s^2: {\Bbb N} \mapsto {\Bbb Z}$
of the natural numbers onto the integers.







\section{Borel's resummation method -- ``The Master forbids it''}
\index{Borel sum}

In what follows we shall review\cite[-50mm]{rousseau-2004}
a resummation method
\sidenote[][-30mm]{For more resummation techniques, please see Chapter 16 of
\cite{Kleinert-Schulte}
%http://users.physik.fu-berlin.de/~kleinert/b8/psfiles/16.pdf
}
invented by Borel \cite{Borel1899}
\marginnote{{\em ``The idea that a function could be determined by a divergent asymptotic series was a foreign one to the nineteenth century mind.
Borel, then an unknown young man, discovered that his summation method gave the ``right'' answer for many classical divergent series.
He decided to make a pilgrimage to Stockholm to see Mittag-Leffler, who was the recognized lord of complex analysis.
Mittag-Leffler listened politely to what Borel had to say and then,
 placing his hand upon the complete works by Weierstrass, his teacher, he said in Latin,
``The Master forbids it.''
A tale of Mark Kac,''} quoted (on page 38) by \cite{reed-sim4}}
to obtain the exact convergent solution
(\ref{2011-m-ch-dseefasol})
of the differential equation  (\ref{2011-m-ch-dsee})
from the divergent series solution (\ref{2011-m-ch-dseess}).
First we can rewrite a suitable infinite series by an integral representation,
thereby using the integral representation of the factorial (\ref{2017-m-ch-sf-edgamma})
% and (\ref{2011-m-ch-dsee15})
as follows:
\begin{equation}
\begin{split}
\sum_{j=0}^\infty
a_j
=
\sum_{j=0}^\infty
a_j  \frac{j!}{j!}
=
\sum_{j=0}^\infty
  \frac{a_j}{j!}  j!
\\
=
\sum_{j=0}^\infty
  \frac{a_j}{j!}  \int_0^\infty t^j e^{-t} dt
\stackrel{{\rm B}}{=}
\int_0^\infty \left(\sum_{j=0}^\infty   \frac{a_j t^j}{j!}\right)   e^{-t} dt
.
\end{split}
\label{2012-m-ch-dsborel}
\end{equation}



A series  $\sum_{j=0}^\infty   a_j $
is {\em Borel summable}
\index{Borel summable}
if
$\sum_{j=0}^\infty   \frac{a_j t^j}{j!}$ has a non-zero radius of convergence,
if it can be extended along the positive real axis, and if the integral
(\ref{2012-m-ch-dsborel}) is convergent.
This integral is called the
{\em Borel sum}
\index{Borel sum}
of the series.
It can be obtained by taking $a_j$, computing the sum $\sigma (t) = \sum_{j=0}^\infty   \frac{a_j t^j}{j!}$,
and integrating $\sigma (t)$ along the positive real axis with a ``weight factor'' $e^{-t}$.


More generally, suppose
\begin{equation}
S(z)= z \sum_{j=0}^\infty
a_j  z^j  = \sum_{j=0}^\infty
a_j  z^{j+1}
\label{2018-m-ch-ds-fps}
\end{equation}
is some formal power series.
Then  its {\em Borel transformation}
\index{Borel transformation}
is defined by
\begin{equation}
\begin{split}
\sum_{j=0}^\infty
a_j z^{j+1}
=
\sum_{j=0}^\infty
a_j z^{j+1}  \frac{j!}{j!}
=
\sum_{j=0}^\infty
  \frac{a_j z^{j+1} }{j!}  \underbrace{j!}_{ \int_0^\infty t^j e^{-t}   dt }  \\
=
\sum_{j=0}^\infty
  \frac{a_jz^{j} }{j!}  \int_0^\infty t^j e^{-t} z dt
\stackrel{{\rm B}}{=}
\int_0^\infty \left(\sum_{j=0}^\infty   \frac{a_j (z t)^j}{j!}\right)  e^{-t} z dt \\
[\textrm{variable substitution }  y= z t, \; t = \frac{y}{z}  , \; dy = z \,dt, \; dt = \frac{dy}{z}]\\
\stackrel{{\rm B}}{=}
\int_0^\infty \left(\sum_{j=0}^\infty   \frac{a_j y^j}{j!}\right)  e^{-\frac{y}{z}}   dy  =
\int_0^\infty {\cal B}S (y)  e^{-\frac{y}{z}}   dy
.
\end{split}
\label{2012-m-ch-dsboreltrafo}
\end{equation}

Often, this is written with $z=1/t$, such that the  Borel transformation
\index{Borel transformation}
is defined by
\begin{equation}
\begin{split}
\sum_{j=0}^\infty
a_j t^{-(j+1)}
\stackrel{{\rm B}}{=}
\int_0^\infty {\cal B}S (y)  e^{- yt }   dy
.
\end{split}
\label{2012-m-ch-dsboreltrafo2}
\end{equation}

The {\em Borel transform}
\index{Borel transform}
of   $S(z)=   \sum_{j=0}^\infty
a_j  z^{j+1} =  \sum_{j=0}^\infty
a_j  t^{-(j+1)}$
is thereby defined as
\begin{equation}
{\cal B}S (y)
=
  \sum_{j=0}^\infty   \frac{a_j y^j}{j!}
.
\label{2012-m-ch-dsboreltransform}
\end{equation}


%\subsection{Some example of Borel sums}

{
\color{blue}
\bexample

In the following a few examples will be given.

\begin{itemize}
\item[(i)]
The Borel sum
of the Grandi series (\ref{2009-fiftyfifty-1s})
\index{Grandi series}
is equal to its Abel sum:
\begin{equation}
\begin{split}
s= \sum_{j=0}^\infty (-1)^j
\stackrel{{\rm B}}{=}
\int_0^\infty \left(\sum_{j=0}^\infty   \frac{(-1)^j t^j}{j!}\right)   e^{-t} dt  \\
=
\int_0^\infty \underbrace{\left(\sum_{j=0}^\infty   \frac{(- t)^j}{j!}\right)}_{e^{-t}}   e^{-t} dt
=
\int_0^\infty    e^{-2t} dt  \\
[\textrm{variable substitution } 2t = \zeta, dt = \frac{1}{2} d \zeta ]\\
=
\frac{1}{2}
\int_0^\infty    e^{-\zeta } d\zeta      \\
=
\frac{1}{2}
\left.     \left(-e^{-\zeta }\right) \right|_{\zeta=0}^\infty
=
\frac{1}{2} \left(- \underbrace{e^{-\infty}}_{=0} + \underbrace{e^{-0}}_{=1}\right) = \frac{1}{2}
.
\end{split}
\end{equation}


\item[(ii)]
A similar calculation for $s^2$ defined in Eq.
(\ref{2009-fiftyfifty-1s1})
yields
\begin{equation}
\begin{split}
s^2= \sum_{j=0}^\infty (-1)^{j+1} j = (-1) \sum_{j=1}^\infty (-1)^j j
\stackrel{{\rm B}}{=}
-\int_0^\infty \left(\sum_{j=1}^\infty   \frac{(-1)^j j t^j}{j!}\right)   e^{-t} dt  \\
=
-\int_0^\infty \left(\sum_{j=1}^\infty   \frac{(-t)^j}{(j-1)!}\right)   e^{-t} dt
=
-\int_0^\infty \left(\sum_{j=0}^\infty   \frac{(-t)^{j+1}}{j!}\right)   e^{-t} dt  \\
=
-\int_0^\infty (-t) \underbrace{\left(\sum_{j=0}^\infty   \frac{(-t)^j}{j!}\right)}_{e^{-t}}   e^{-t} dt
=
-\int_0^\infty  (-t)  e^{-2t} dt  \\
[\textrm{variable substitution } 2t = \zeta, dt = \frac{1}{2} d \zeta ]\\
=
\frac{1}{4}
\int_0^\infty  \zeta  e^{-\zeta } d\zeta
=
\frac{1}{4}
\Gamma (2)
=
\frac{1}{4} 1! = \frac{1}{4}
,
\end{split}
\end{equation}
which is again equal to the Abel sum.
\index{Abel sum}

\item[(iii)]
The Borel transform
of a geometric series (\ref{2009-fiftyfifty-1s})
\index{geometric series}
$g(z)=a z \sum_{j=0}^\infty    z^{j}=a \sum_{j=0}^\infty    z^{j+1}$ with constant $a$ and $0> z > 1$
is
\begin{equation}
{\cal B}g (y)
=
a \sum_{j=0}^\infty   \frac{y^j}{j!} = a e^y.
\end{equation}
The Borel transformation~(\ref{2012-m-ch-dsboreltrafo}) of this geometric series  is
\begin{equation}
\begin{split}
g(z) \stackrel{{\rm B}}{=} \int_0^\infty {\cal B}g (y)   e^{-\frac{y}{z}}   dy
=
\int_0^\infty a e^y  e^{-\frac{y}{z}}   dy
=
a \int_0^\infty e^{-\frac{y(1-z)}{z}}   dy  \\
\left[\textrm{variable substitution } x =  -y\frac{1-z}{z}, dy = -\frac{z}{1-z} dx \right]\\
=
\frac{-a z}{1-z} \int_0^{-\infty} e^{x}   dx
=
\frac{a z}{1-z} \int_{-\infty}^0 e^{x}   dx
=
a \frac{z}{1-z} (\underbrace{e^0}_{1}-\underbrace{e^{-\infty}}_{0})    =  \frac{a z}{1-z}.
\end{split}
\end{equation}

Likewise, the Borel transformation~(\ref{2012-m-ch-dsboreltrafo2})
of the geometric series
$g(t^{-1})=a   \sum_{j=0}^\infty    t^{-(j+1)}$ with constant $a$ and $t > 1$
is
\begin{equation}
\begin{split}
g(t^{-1}) \stackrel{{\rm B}}{=} \int_0^\infty {\cal B}g (y)   e^{-yt}   dy
=
\int_0^\infty a e^y  e^{-yt}   dy
=
a \int_0^\infty e^{- y(t-1) }   dy  \\
\left[\textrm{variable substitution } x =  - y(t-1) , dy = -\frac{1}{t-1} dx \right]\\
=
\frac{-a}{t-1} \int_0^{-\infty} e^{x}   dx
=
\frac{a }{t-1} \int_{-\infty}^0 e^{x}   dx
=
a \frac{1}{t-1} (\underbrace{e^0}_{1}-\underbrace{e^{-\infty}}_{0})    =  \frac{a }{t-1}.
\end{split}
\end{equation}


\end{itemize}
\eexample
}


\section{Divergent series as solutions to differential equations}
%\index{Euler differential equation}

In what follows we demonstrate that divergent series may make sense, in the way Abel
wondered.
That is, we shall show that the first partial sums of divergent series
may yield ``good'' approximations; or, if summed properly, even the exact result.

Moreover, more terms contributing to the
sum  might worsen the approximation rather an make it better -- a situation totally different
from convergent series, where more terms always result in better approximations.

%Already Euler in 1760 observed~\cite[\S~19,~p.~220]{Euler60} that
Already Euler in 1760 observed~\cite{Euler60,Costin-2009} that
the series
\begin{equation}
s(x)= x - x^2+2x^3-6x^4 + \ldots= -\sum_{j=0}^\infty   (-x)^{j+1} j!  = \sum_{j=0}^\infty (-1)^j  x^{j+1}j!,
\label{2011-m-ch-dseess}
\end{equation}
when differentiated, satisfies
\begin{equation}
\frac{d}{dx}s(x)= \frac{x-s(x)}{x^2},
\end{equation}
and thus is a solution of the  differential equation
%consider the {\em Euler differential equation}
%\index{Euler differential equation}
\begin{equation}
\begin{split}
\left(x^2 \frac{d}{dx} +1\right) s(x) = {x},\;\text{ or }\;
\left(\frac{d}{dx} +\frac{1}{x^2}\right) s(x) = \frac{1}{x}.
\end{split}
\label{2011-m-ch-dsee}
\end{equation}

% ListPlot[  Table[Sum[(-1)^j Factorial[j] 1^(j + 1), {j, 1, n}], {n, 1, 15}],  Joined -> True]

Without prior knowledge of $s(x)$ in~(\ref{2011-m-ch-dseess}) an
immediate way to solve~(\ref{2011-m-ch-dsee}) is a quasi {\em ad hoc} series {\it Ansatz}
\begin{equation}
 s(x)= \sum_{j=0}^\infty a_jx^j,
\end{equation}
which, when inserted into~(\ref{2011-m-ch-dsee}), yields
\begin{equation}
\begin{split}
\left(x^2 \frac{d}{dx} +1\right) s(x) =
\left(x^2 \frac{d}{dx} +1\right) \sum_{j=0}^\infty a_jx^j = {x} \\
x^2 \sum_{j=0}^\infty a_j j x^{j-1} + \sum_{j=0}^\infty a_jx^j =
\sum_{j=0}^\infty a_j j x^{j+1} + \sum_{j=0}^\infty a_jx^j = {x} \\
  \left[\text{index substitution in first sum } i= j+1,\; j=i-1 \text{; then } i \rightarrow j\right]  \\
\sum_{i=1}^\infty a_{i-1} (i-1) x^i + \sum_{j=0}^\infty a_j x^j =
a_0 + \sum_{j=1}^\infty \left(a_{j-1} (j-1) +  a_j x\right) x^j = x \\
a_0 + a_1 x + \sum_{j=2}^\infty \left(a_{j-1} (j-1) +  a_j x\right) x^j = x
.
\end{split}
\label{2018-m-ch-adhocss}
\end{equation}
Since polynomials of different degrees are linear independent,
a comparison of coefficients appearing on the left hand side of~(\ref{2018-m-ch-adhocss}) with $x$
yields
\begin{equation}
\begin{split}
a_0 =0,
\;
a_1 = 1,
\\
a_j =  - a_{j-1} (j-1)  = - (-1)^j (j-1)!= (-1)^{j-1} (j-1)! \text{ for } j \ge 2
,
\end{split}
\label{2018-m-ch-adhocss1}
\end{equation}
which renders the divergent sum~(\ref{2011-m-ch-dseess}) enumerated by Euler:
\begin{equation}
\begin{split}
0+ x +\sum_{j=2}^\infty (-1)^{j-1} (j-1)! x^j \\
=[j\rightarrow j+1] = x +\sum_{j=1}^\infty (-1)^{j} j! x^{j+1}
=  \sum_{j=0}^\infty (-1)^{j} j! x^{j+1}
.
\end{split}
\label{2018-m-ch-adhocss12}
\end{equation}

The ordinary differential equation~(\ref{2011-m-ch-dsee})
is not of the Fuchsian type, but it can be solved in two ways:
on the one hand, by an exact solution, and on the other hand,
by a proper summation of  the divergent series~(\ref{2011-m-ch-dseess}).
This latter divergent series can be approximated by computing it up to some order,
and the approximation subsequently compared to the exact solution (by taking
the difference).



In what follows the Borel summation will be used to formally sum up
the divergent series~(\ref{2011-m-ch-dseess}) enumerated by Euler.
A comparison
between~(\ref{2018-m-ch-ds-fps})
and~(\ref{2011-m-ch-dseess})
renders the coefficients
\begin{equation}
a_j =  (-1)^j \, j!,
\end{equation}
which can be used to compute the Borel transform~(\ref{2012-m-ch-dsboreltransform}) of Euler's divergent
series~(\ref{2011-m-ch-dseess})
\begin{equation}
{\cal B}S (y)
=
  \sum_{j=0}^\infty   \frac{a_j y^j}{j!}
  =   \sum_{j=0}^\infty   \frac{(-1)^j \, j!\, y^j}{j!}
  =   \sum_{j=0}^\infty    (-y)^j  = \frac{1}{1+y}
.
\label{2018-m-ch-dsboreltransform-euler}
\end{equation}
resulting in the Borel transformation~(\ref{2012-m-ch-dsboreltrafo} ) of Euler's divergent
series~(\ref{2011-m-ch-dseess})
\begin{equation}
\begin{split}
s(x) = \sum_{j=0}^\infty
a_j z^{j+1}
\stackrel{{\rm B}}{=}
\int_0^\infty {\cal B}S (y)  e^{-\frac{y}{x}}   dy
= \int_0^\infty \frac{ e^{-\frac{y}{x}} }{1+y}    dy \\
\left[\textrm{variable substitution } t =  \frac{y}{x} , dy =  z dt \right]\\
= \int_0^\infty \frac{ x e^{-t }}{1+xt}    dt
.
\end{split}
\label{2018-m-ch-dsboreltrafo2-Euler}
\end{equation}
Notice~\cite{rousseau-2004} that the Borel transform~(\ref{2018-m-ch-dsboreltransform-euler})
``rescales'' or
``pushes'' the divergence of the series (\ref{2011-m-ch-dseess}) with zero radius of convergence
towards a ``disk'' or interval with finite radius of convergence and a singularity at $y=-1$.



An exact solution of~(\ref{2011-m-ch-dsee})
can also be found directly by {\em quadrature;}
that is, by explicit integration (see, for instance, Chapter one of Ref. \cite{birkhoff-Rota-48}).
It is not immediately obvious how to utilize direct integration in this case; the trick
is to make the following {\it Ansatz}:
\begin{equation}
s(x) = y (x) \exp \left( - \int \frac{ d x}{x^2} \right) = y(x)\exp \left[-\left(-\frac{1}{x}+C\right)\right]= k y(x)e^\frac{1}{x},
\label{201m-m-ch-dsans}
\end{equation}
with constant $k=e^{-C}$,
so that the ordinary differential equation~(\ref{2011-m-ch-dsee}) transforms into
\begin{equation}
\begin{split}
\left(x^2 \frac{d}{dx} +1\right)   s(x) =
\left(x^2 \frac{d}{dx} +1\right) y (x) \exp \left( - \int \frac{ d x}{x^2} \right) = x, \\
x^2 \frac{d  }{dx}\left[ y  \exp \left( - \int \frac{ d x}{x^2} \right)\right] + y  \exp \left( - \int \frac{ d x}{x^2} \right) = x, \\
x^2  \exp \left( - \int \frac{ d x}{x^2} \right)\frac{d y }{d x } + x^2 y \left(- \frac{1}{x^2}\right) \exp \left( - \int \frac{ d x}{x^2} \right) + y   \exp \left( - \int \frac{ d x}{x^2} \right) = x, \\
x^2  \exp \left( - \int \frac{ d x}{x^2} \right)\frac{d y }{d x }  = x, \\
  \exp \left( - \int \frac{ d x}{x^2} \right)\frac{d y }{d x } = \frac{1}{x}, \\
\frac{d y }{d x }  = \frac{ \exp \left( \int \frac{ d x}{x^2} \right)}{x}, \\
 y(x)  = \int \frac{1}{x} e^{ \int_x \frac{ d t}{t^2} } {d x }.
\end{split}
\end{equation}
More precisely, insertion into~(\ref{201m-m-ch-dsans}) yields, for some $a\neq 0$,
\begin{equation}
\begin{split}
s(x) =   e^{ - \int_a^x \frac{ dt}{t^2}}  y(x)=
-  e^{ - \int_a^x \frac{ dt}{t^2}} \int_0^x e^{ \int_a^t \frac{ ds}{s^2}} \left(-\frac{ 1}{t}\right) dt
\\
=
  e^{ - \left. \left(-\frac{1}{t}\right )\right|_a^x} \int_0^x e^{  \left. -\frac{1}{s}\right|_a^t} \left(\frac{ 1}{t}\right) dt
\\
=
  e^{ \frac{1}{x} -  \frac{1}{a} } \int_0^x e^{  -\frac{1}{t} +  \frac{1}{a} } \left(\frac{ 1}{t}\right) dt
\\
=
  e^{ \frac{1}{x}}\underbrace{e^{ -  \frac{1}{a} + \frac{1}{a}}}_{=e^0=1} \int_0^x e^{ -\frac{1}{t} } \left(\frac{ 1}{t}\right) dt
\\
%=
%  e^{ \frac{1}{x}}e^{ -  \frac{1}{a} } \int_0^x e^{ -\frac{1}{t} }e^{ \frac{1}{a} } \left(\frac{ 1}{t}\right) dt
%\\
=
e^{   \frac{1}{x}} \int_0^x  \frac{ e^{ - \frac{1}{t}}}{t} dt
\\
=
 \int_0^x  \frac{ e^{ \frac{1}{x} -\frac{1}{t}}}{t} dt.
\end{split}
\label{2011-m-ch-dseeeesola}
\end{equation}
With a change of the integration variable
\begin{equation}
\begin{split}
\frac{ z }{x} = \frac{1}{t}-\frac{1}{x}
, \; \textrm{ and thus }  \;
 z  = \frac{x}{t}-1
, \;  \textrm{ and } \;
t =  \frac{x}{1+  z }
, \\
\frac{dt}{d z  } =  -\frac{x}{(1+  z )^2}  , \;
\textrm{ and thus }  \;
dt =  -\frac{x}{(1+  z )^2} d z , \;    \\
\textrm{ and thus }  \;
\frac{dt}{t } =   \frac{-\frac{x}{(1+  z )^2}}{\frac{x}{1+  z }} d z
=   -\frac{ d z }{1+  z },
\end{split}
\label{2011-m-ch-dseeans}
\end{equation}
the integral (\ref{2011-m-ch-dseeeesola}) can be rewritten as
\begin{equation}
\begin{split}
y(x)= \int_{\infty}^0
\left(-\frac{e^{-\frac{ z }{x}}}{1+ z }\right) d z
= \int_0^\infty
\frac{e^{-\frac{ z }{x}}}{1+ z } d z  .
\end{split}
\label{2011-m-ch-dseefasol}
\end{equation}
%It is proportional to the {\em Stieltjes Integral} \index{Stieltjes Integral}~\cite{Bender-Orszag,Boyd99thedevil}
%\begin{equation} S(x)= \int_0^\infty \frac{e^{- z }}{1+x z } d z  . \label{2012-m-ch-ds-si} \end{equation}

Note that whereas the series solution diverges for all nonzero $x$,
the solution by quadrature~(\ref{2011-m-ch-dseefasol}) and by the Borel summation~(\ref{2018-m-ch-dsboreltrafo2-Euler})
converges and is well defined for all $x\ge 0$.

Let us now estimate the absolute difference between $y_{k}(x)$
which represents the partial sum ``$y_s(x)$ truncated
after the $k$th term'' and $y(x)$; that is, let us consider
\begin{equation}
\begin{split}
\left\vert y(x) - y_{k}(x) \right\vert=
\left\vert \int_0^\infty
\frac{e^{-\frac{ z }{x}}}{1+ z } d z
-
\sum_{j=0}^k (-1)^j j! x^{j+1} \right\vert .
\end{split}
\label{2011-m-ch-dseeanest}
\end{equation}

For any $x\ge 0$ this difference can be estimated~\cite{rousseau-2004} by  a bound from above
\begin{equation}
\left| R_k(x)\right|
\stackrel{{\tiny \textrm{ def }}}{=}
\vert y(x) - y_{k}(x) \vert
\le
k! x^{k+1},
\label{2011-m-ch-dseeest}
\end{equation}
that is, this difference between the exact solution $y(x)$ and the diverging partial series
$y_{k}(x)$ is smaller than the first neglected term; and all subsequent ones.
This could we seen as an example of what Boyd~\cite{Boyd99thedevil} refers to as
{\em Carrier�s Rule}~\cite{Boyd99thedevil}:
{\em ``Divergent series converge faster than convergent
series (because they don't have to converge); their
leading term is often a very good approximation.''}




{\color{OliveGreen}
\bproof

For a proof, observe that,
since
a partial {\em geometric series}
\index{geometric series}
is the sum of all the numbers in a geometric progression up to a certain power;
that is,
\begin{equation}
\sum_{k=0}^n r^k =   1+r+r^2+ \cdots +r^k+ \cdots +r^n .
\label{2011-m-ch-dsee124567}
\end{equation}
By multiplying both sides with $1-r$,
the sum (\ref{2011-m-ch-dsee124567}) can be rewritten as
\begin{equation}
\begin{split}
(1-r) \sum_{k=0}^n r^k=
(1-r) (1+ r+r^2+ \cdots +r^k+ \cdots +r^n)\\
=1+ r+r^2+ \cdots +r^k+ \cdots +r^n -
r(1+r+r^2+ \cdots +r^k+ \cdots +r^n +r^{n}) \\
=1+ r+r^2+ \cdots +r^k+ \cdots +r^n -
(r+r^2+ \cdots +r^k+ \cdots +r^n +r^{n+1}) \\
= 1-r^{n+1}
,
\end{split}
\end{equation}
and, since the middle terms all cancel out,
\begin{equation}
\sum_{k=0}^n r^k =  \frac{1-r^{n+1}}{1-r},
\;
\textrm{ or }
\;
\sum_{k=0}^{n-1} r^k =  \frac{1-r^{n}}{1-r}  =  \frac{1}{1-r} - \frac{r^{n}}{1-r}
.
\label{2011-m-ch-dsee12}
\end{equation}
Thus, for $r=-\zeta$, it is true that
\begin{equation}
\begin{split}
\frac{1}{1+\zeta}=\sum_{k=0}^{n-1} (-1)^k \zeta^k + (-1)^n \frac{\zeta^n}{1+\zeta}.
\end{split}
\label{2011-m-ch-dsee13}
\end{equation}
Thus
\begin{equation}
\begin{split}
f(x) =\int_0^\infty \frac{e^{-\frac{\zeta}{x}}}{1+\zeta}d\zeta \\
\qquad =
\int_0^\infty  e^{-\frac{\zeta}{x}}\left[
\sum_{k=0}^{n-1} (-1)^k \zeta^k + (-1)^n \frac{\zeta^n}{1+\zeta}
\right]
d\zeta \\
\qquad =
\sum_{k=0}^{n-1}(-1)^k\int_0^\infty  \zeta^k  e^{-\frac{\zeta}{x}}   d\zeta
 +
(-1)^n \int_0^\infty \frac{\zeta^ne^{-\frac{\zeta}{x}}}{1+\zeta}
d\zeta  .
\end{split}
\label{2011-m-ch-dsee14}
\end{equation}
Since [cf Eq.~(\ref{2017-m-ch-sf-edgamma})]
\begin{equation}
k!= \Gamma(k+1)=    \int_0^\infty z^k e^{-z} dz,
\label{2011-m-ch-dsee15}
\end{equation}
one obtains
\begin{equation}
\begin{split}
\int_0^\infty  \zeta^k  e^{-\frac{\zeta}{x}}   d\zeta     \\
\qquad \textrm{[substitution:  }z=\frac{\zeta}{x}, d \zeta =x dz  \textrm{ ]  } \\
\qquad = \int_0^\infty x^{k+1} z^k  e^{-z}   dz
\\
\qquad =  x^{k+1} k! ,
\end{split}
\label{2011-m-ch-dsee16}
\end{equation}
and hence
\begin{equation}
\begin{split}
f(x)  =
\sum_{k=0}^{n-1}(-1)^k\int_0^\infty  \zeta^k  e^{-\frac{\zeta}{x}}   d\zeta
 +
(-1)^n \int_0^\infty \frac{\zeta^ne^{-\frac{\zeta}{x}}}{1+\zeta}
d\zeta  \\
\qquad =
\sum_{k=0}^{n-1}  (-1)^k x^{k+1} k!
 +
\int_0^\infty (-1)^n \frac{\zeta^ne^{-\frac{\zeta}{x}}}{1+\zeta}
d\zeta  \\
\qquad =
f_n(x)  +R_n(x),
\end{split}
\label{2011-m-ch-dsee17}
\end{equation}
where $f_n(x)$ represents the partial sum of the power series, and $R_n(x)$ stands for the remainder,
the difference between $f(x)$ and $f_n(x)$.
The absolute of the remainder can be estimated by
\begin{equation}
\begin{split}
\left| R_n(x)\right|
=
\int_0^\infty  \frac{\zeta^n e^{-\frac{\zeta}{x}}}{1+\zeta} d\zeta
\le
\int_0^\infty  \zeta^n e^{-\frac{\zeta}{x}} d\zeta
 = n! x^{n+1}.
\end{split}
\label{2011-m-ch-dsee18}
\end{equation}
\eproof
}




\begin{center}
{\color{olive}   \Huge
%\decofourright
 %\decofourright
%\decofourleft
%\aldine X \decoone c
 \floweroneright
% \aldineleft ]
% \decosix
%\leafleft
% \aldineright  w  \decothreeleft f \leafNE
% \aldinesmall Z \decothreeright h \leafright
% \decofourleft a \decotwo d \starredbullet
%\decofourright
% \floweroneleft
}
\end{center}
