%%%%%%%%%%%%%%%%%%%%% chapter.tex %%%%%%%%%%%%%%%%%%%%%%%%%%%%%%%%%
%
% sample chapter
%
% Use this file as a template for your own input.
%
%%%%%%%%%%%%%%%%%%%%%%%% Springer-Verlag %%%%%%%%%%%%%%%%%%%%%%%%%%
%\motto{Use the template \emph{chapter.tex} to style the various elements of your chapter content.}
\chapter{Some flight characteristics}
\label{2023-UFO-part-Perception-flight-characteristics} % Always give a unique label
% use \chaptermark{}
% to alter or adjust the chapter heading in the running head


\abstract*{This chapter provides an overview of the perceived flight characteristics of Unidentified Flying Objects (UFOs) or Unidentified Aerial Phenomena (UAPs) based on reports from various sources. These characteristics include the ability to rapidly change direction and speed, a total lack of noise or humming sounds, strange shapes and structures, and different types of motion, such as wobble and zig-zag. The authors highlight that these flight characteristics are not consistent with any known conventional aircraft or natural phenomenon, which adds to the mystery of what these objects could be. The wide range of shapes and structures reported is also a major puzzle in understanding the nature of these objects.}


\abstract{This chapter provides an overview of the perceived flight characteristics of Unidentified Flying Objects (UFOs) or Unidentified Aerial Phenomena (UAPs) based on reports from various sources. These characteristics include the ability to rapidly change direction and speed, a total lack of noise or humming sounds, strange shapes and structures, and different types of motion, such as wobble and zig-zag. The authors highlight that these flight characteristics are not consistent with any known conventional aircraft or natural phenomenon, which adds to the mystery of what these objects could be. The wide range of shapes and structures reported is also a major puzzle in understanding the nature of these objects.}



\section{Explanation trap through conceptual and theoretical overreach}
\label{2023-UFO-part-Perception-flight-characteristics-trpo}
\index{overreach}
\index{explanation trap}

As we explore possible explanations for the flight characteristics of UFOs, it is important to keep in mind a significant issue that could affect our understanding. It is likely that we will not be able to fully comprehend some aspects of their movement due to the limitations and inadequacies of our current understanding and methods of analysis.
Thereby, we may suffer from an explanation overreach.

Forcing an explanation in such a situation of explanation overreach may lead to the ``explanation trap,''
where inaccurate models and beliefs persist and are not acknowledged, despite the lack of understanding.
This topic is discussed further in Appendix~\ref{2023-UFO-Appendix-Explanation-Trap}.

\section{Rapid change of direction and speed}
\label{2023-UFO-part-Perception-flight-characteristics-rcds}

One of the most striking characteristics of UFO/UAP sightings is their ability
to rapidly change direction and speed. This is often described as sudden accelerations, sudden stops,
or sudden changes in direction.

A recent proposal for amendment to the appropriations for fiscal year 2024 for military activities
of the Department of Defense
prominently highlights this apparent absence of inertia~\cite{PropAment2024AFY}:

\begin{svgraybox}
Unidentified anomalous phenomena are differentiated from both attributed and temporarily non-attributed objects by one or more of the following observables:
\begin{enumerate}
\item[(i)] Instantaneous acceleration absent apparent inertia.

\item[(ii)] Hypersonic velocity absent a thermal signature and sonic shockwave.

\item[(iii)] Transmedium (such as space-to-ground and air-to-undersea) travel.

\item[(iv)] Positive lift contrary to known  aerodynamic principles.

\item[(v)] Multispectral signature control.

\item[(vi)] Physical or invasive biological effects to close observers and the environment.
\end{enumerate}
\end{svgraybox}


Kevin H. Knuth, Robert M. Powell, and Peter Reali estimated minimum accelerations of the objects during the observed maneuvers, ranging from 100 g to thousands of g's, with no visible air disturbance, no sonic booms, and no evidence of heat production~\cite{Knuth-e21100939,NimitzSCURep2019Mar}. An earlier investigation by Paul Richard Hill came to similar conclusions~\cite[pp.~48,49]{Hill2014Jun}.
More recent estimates by Daniel Coumbe appear to confirm these calculations~\cite{Coumbe2022Oct}.

This flight behavior is not consistent with any known conventional aircraft or natural phenomenon,
which adds to the mystery of what these objects could be.
For instance,  friction of a ``fast mover'' with the surrounding air is expected to generate a bright
optical fireball and a sonic boom~\cite{Loeb2022Oct,Loeb2023Mar}.

One should add that these calculations depend on certain assumptions, such as the solidity and ``nuts-and-bolts''
character of these craft-type phenomena.
For example, a holographic projection
or laser created plasma UFOs~\cite{Hambling2020May,Mizokami2020May} may not be limited to inertial motion:
such ``objects'' have little or no mass/inertia.

Another factor to consider is the (mostly unknown) distance from the observer.


\section{Total lack of noise or humming sounds}
\label{2023-UFO-part-Perception-flight-characteristics-lnhs}

A common feature of UFO/UAP sightings is their apparent lack of noise. While some objects are reported to produce a low humming sound, the majority are said to be completely silent, even when moving at high speed. This lack of noise is a significant deviation from the usual behavior of conventional aircraft, which generate a significant amount of noise due to their engines and air movements.

The lack of noise associated with UFO/UAP sightings suggests that these objects may not be powered by conventional means, or that they have advanced technology that eliminates noise.

\section{No sonic booms}

According to some (alleged government) reports~\cite{Daniels2021Mar,Pappas2021Mar},
there is evidence of UFOs breaking the sound barrier without a sonic boom.
(This might also be achievable with current means and special designs~\cite{Figliozzi2020Nov}.)


\section{Shapes}
\label{2023-UFO-part-Perception-flight-characteristics-sh}

In addition to their peculiar flight behavior, UFO/UAPs are frequently reported to possess particular shapes and structures. Some are characterized as circular or spherical objects, triangular, oval/elliptical/egg/avocado-shaped, cylindrical/cigar-shaped, and saucer-shaped, while others exhibit irregular shapes or geometric patterns~\cite[p.~12]{Hill2014Jun}.

The 1968 RAND report~\cite[p.~24]{Kocher-RAND-1968Jan} quotes a collection of 575 NICAP cases \cite{Hall1964}, with the following percentages:
disc-shaped 26{\%},
round/ball-shaped  17{\%},
oval/elliptical-shaped 13{\%},
cylindrical-shaped 8.3{\%},
triangular-shaped 2{\%},
and other (just radar, light sources, or not categorized) 33.7{\%}.
It is interesting to compare these ratios from the 1960s with later findings.
The follow-up report by Hall~\cite[p.~446]{Hall2001Jan} categorized the shapes of 225 UFO sightings as follows:
disc-shaped 44{\%},
oval/elliptical-shaped 13{\%},
round/ball-shaped  12{\%},
hemisphere-shaped 6{\%},
cigar/cylindrical-shaped   5{\%},
cone/spindle-shaped 5{\%},
light sources 4{\%},
delta/triangular-shaped 4{\%}, and
others 7{\%}.





Some are reported to emit beams of light or be luminous, while others are reported to be completely dark. The broad range of reported shapes and structures poses a significant puzzle in comprehending the nature of these objects. It is challenging to attribute these reports to a single source or explanation.



\section{Motion types}
\label{2023-UFO-part-Perception-flight-characteristics-mt}


\subsection{Wobble}

Oscillation is a common type of motion, and NICAP~\cite[p.~26]{Kocher-RAND-1968Jan} classifies it into three categories: ``wobble on axis'' (also known as fluttering, flipping, or tipping), pendulum motion during slow ascent, hovering, and descent (also known as ``falling leaf motion''), and sometimes a side-to-side oscillation that is observed as the UFO moves horizontally. These movements are typically observed in disc-shaped UFOs, although comparable behaviors have been seen in other shapes.

\subsection{Zig-zag}

Another motion category can be characterized by violent and unpredictable movements~\cite[p.~26]{Kocher-RAND-1968Jan}. This motion appears to lack inertia and has no clear explanation according to current physical theory. Witnesses describe these movements using terms such as ``bobbing,'' ``erratic,'' ``jerky,'' ``zig-zag,'' and ``shot away,'' and they involve high angular accelerations and velocities.

\section{Underwater unidentified submerged objects}

There have been reports~\cite{Sanderson-invisibeRes,DolanDisclosure2023Mar} of unidentified submerged objects (USOs) with remarkable characteristics. USOs allegedly emerge from or dive into lakes and the sea, producing sonar signatures and nocturnal underwater lights. For an older bibliography, see Chapters 26 and 27 of George M. Eberhart's review~\cite{Eberhart-I-1986Jan}.

A dramatic example of an unconfirmed episode~\cite{Brodler1952} is retold in Section~\ref{2023-UFO-part-History-chapter-post-1945-pre-1953-sswl}. It is similar to many maritime reports, such as in a conversation mentioned by Charles Fort~\cite[Chapter~21]{FortBotD}.

The phenomenon of geometrical, phosphorescent displays on the surface of tropical seas, such as moving parallel bands of light or luminous pinwheels, is often explained as the result of bioluminescent discharges from microorganisms. The mystery lies in what causes creatures to emit so much light in such specific patterns. These displays have been associated with USOs in some UFO literature~\cite[Chapter27]{Eberhart-I-1986Jan}. For another unconfirmed anecdote, see also the alleged downed UFO/USO after a high altitude nuclear explosion presented in Section~\ref{2023-UFO-part-Perception-crash-retreivals-htdau}. A better understanding of nocturnal light USOs may come from studying marine luminescence.

Let me mention a dramatic unconfirmed USO episode retold by Ivan T. Sanderson~\cite[Chapter~1]{Sanderson-invisibeRes} in his book ``Invisible Residents.'' It allegedly took place during Operation Deep Freeze in the late 1950s. Dr. Rubens J. Villela, a Brazilian scientist working with the U.S. Navy's Operation Deep Freeze in Antarctica witnessed a massive object suddenly emerge from the sea, shoot up through more than ten meters of ice, and soar into the sky like a silvery bullet. The incident occurred in Admiralty Bay, facing the South Atlantic Ocean. Enormous blocks of ice were thrown high into the air, cascading down around the hole created in the thick ice sheet. The water was rolling, and steam was coming from both the hole and the descending ice.


\section{Evasive not aggressive}

Often but not always UFOs tend to be evasive when approached. They seek to make distance by adopting very high speeds.
