%%tth:\begin{html}<LINK REL=STYLESHEET HREF="/~svozil/ssh.css">\end{html}
\documentstyle[a4]{article}
\RequirePackage{graphicx}
%\documentstyle[amsfonts]{article}
%\RequirePackage{times}
%\RequirePackage{courier}
%\RequirePackage{mathptm}
%\renewcommand{\baselinestretch}{1.3}
\begin{document}

%\def\frak{\cal }
%\def\Bbb{\bf }
\sloppy



\title{How to acknowledge hypercomputation?}
\author{Authors\\
 {\small Institute}}
\date{ }
\maketitle

%\begin{flushright}
%{\scriptsize http://tph.tuwien.ac.at/$\widetilde{\;\;}\,$svozil/publ/2000-cesena.$\{$htm,ps,tex$\}$}
%\end{flushright}

\begin{abstract}
Besides attempts to break the Karp-Cook Thesis by various speedups
there are efforts to conceptualize hypercomputation,
mostly in the framework of some advanced physical theory such as relativity theory
and quantum mechanics.
Rather than discussing these scenarios in detail,
this paper discusses the feasability of operationalizable
verifications and tractable verifications of claims
that certain agents or oracles transcend Turing computability and
recursive function theory.
\end{abstract}


\section{On black boxes which are hypercomputers}

Already in 1958, Davis \cite[p. 11]{davis-58}
sets the stage of the following discussion by pointing out
 {\em `` $\ldots$ how can we ever exclude the possibility of our
 presented,
 some day (perhaps by some extraterrestrial visitors), with a (perhaps
 extremely complex) device or ``oracle'' that ``computes'' a
 non computable function?''
 }
While this may have been a remote, amusing issue in the days written,
the advancement of physical theory in the past decades
has made necessary
a careful evaluation of the possibilities and options for
verification and falsification of certain claims that a concrete physical system
``computes''   a  non computable function.


\subsection{Conceptualization of black box}

Device or agent or oracle which one knows nothing about,
has no rational understanding (in the traditional algorithmic sense) of the intrinsic working.

\subsection{Conceptualization of interface}

Input/output interface facilitating sybolic (eg. binary) exchange
with the black box.


\section{Tests}

\subsection{Tractability of the verification}

Tractable means verifiability by ``low'' (polynomial?)
degree of resources (time, memory space).

\subsection{Translation of solved problems into hard ones}

Example: prime factorization, RSA


\subsection{NP-complete cases}

Conjecture: by operational means it is not possible to go beyond
tests of hyper-NP-completeness.

\subsection{Harder cases with tractable verification}

Do there exist (decision) problems which are harder
than the known NP-complete cases,
possible having no recursively enumerable solution and proof methods,
whose results nevertheless are tractable verifiable?

\subsection{Is provability necessary, what does one gain?}

\subsection{Can two or more hard problems interfere and be combined to form a tractable test?}


\bibliography{svozil}
\bibliographystyle{unsrt}
\end{document}
