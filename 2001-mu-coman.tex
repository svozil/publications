
%%tth:\begin{html}<LINK REL=STYLESHEET HREF="/~svozil/ssh.css">\end{html}
\documentclass[prl,preprint,showpacs,showkeys,amsfonts]{revtex4}
\usepackage{graphicx}
%\documentstyle[amsfonts]{article}
 \RequirePackage{times}
%\RequirePackage{courier}
\RequirePackage{mathptm}
%\renewcommand{\baselinestretch}{1.3}
\begin{document}

%\def\mathfrak{\cal }
%\def\Bbb{\bf }
%\sloppy




\title{Complex analysis\\
Handout ``Methoden der Theoretischen Physik-\"Ubungen''}
\author{Karl Svozil}
 \email{svozil@tuwien.ac.at}
\homepage{http://tph.tuwien.ac.at/~svozil}
\affiliation{Institut f\"ur Theoretische Physik, University of Technology Vienna,
Wiedner Hauptstra\ss e 8-10/136, A-1040 Vienna, Austria}

\begin{abstract}
Some results of complex analysis are reviewed.
\end{abstract}


\pacs{45.10.Na,02.70}
\keywords{Linear vector spaces}

\maketitle


 Falls nicht anders erw\"ahnt,  wird vorausgesetzt, da\ss $\,$ die
Riemann'sche Fl\"ache der
 Funktion $f$ einfach zusammenh\"angend ist.

 \subsection{Definition Differenzierbarkeit, Analytizit\"at}
 Gegeben Funktion $f(z)$ auf Gebiet $G\subset {\rm Domain}(f)$. $f$
 hei\ss t in $z_0$ {\em differenziebar}, falls der Differentialquotient
 $${df\over dz}\vert_{z_0}=f'(z)\vert_{z_0} ={\partial f\over \partial
 x}\vert_{z_0} ={1\over i}{\partial f\over \partial y}\vert_{z_0}$$
 existiert.

 Falls $f$ auf dem gesamten Gebiet $G$ differenzierbar ist, hei\ss t
 die Funktion {\em analytisch / regul\"ar / holomorph}.




 \subsection{Cauchy--Riemann'sche Differentialgleichungen}
 Die Funktion $f(z)=u(z)+iv(z)$ (wobei $u$ \& $v$ reelwertig sind) ist
 {\em analytisch / regul\"ar / holomorph} dann und nur dann, wenn
 ($a_b=\partial a/\partial b$)
 $$u_x=v_y, \qquad u_y=-v_x\quad .$$



  \subsection{Definition analytical function}
 $f$ analytisch auf $G$ $\Longrightarrow$
 alle Ableitungen von $f$ existieren.


 $f=u+iv$ analytisch auf $G$ $\Longrightarrow$
 $$
 \left({\partial^2\over \partial x^2}
 + {\partial^2\over \partial y^2}\right)u=0\quad ,
 \left({\partial^2\over \partial x^2}
 + {\partial^2\over \partial y^2}\right)v=0\quad .
 $$


 $f=u+iv$ analytisch auf $G$ $\Longrightarrow$
 Linien $u=const$ \& Linien $v=const$ sind normal aufeinander.



 \subsection{Cauchy'scher Integralsatz}
 $f$ analytisch auf $G$ und dessen Rand $\partial G$
 $\Longrightarrow$
 $$\oint_{\partial G}f(z)dz=0\quad .$$


 $\oint_{C\subset \partial G}f(z)dz
$ ist unabh\"angig vom Verlauf,
 nur abh\"angig von Anfangs-- und Endpunkt.

 Beweis: Subtrahiere zwei Integrale voneinander, welche
 durch zwei verschiedene Wege $C_1$ und $C_2$ mit
 denselben Anfangs-- und Endpunkten und demselben Kern definiert sind,
 dann drehe die Integrationsrichtum von einem Weg um. Der resultierende
 geschlossene Umlauf ergibt (nach dem Cauchy'shen Integralsatz) 0.


 \subsection{Cauchy'sche Integralformel}
 $f$ analytisch auf $G$ und dessen Rand $\partial G$
 $\Longrightarrow$
 $$f(z_0)={1\over 2\pi i}\oint_{\partial G}{f(z)\over z-z_0}dz\quad
 .$$

 \subsection{Allgemeine Cauchy'sche Integralformel}
 $f$ analytisch auf $G$ und dessen Rand $\partial G$
 $\Longrightarrow$
 $$f^{(n)}(z_0)={n!\over 2\pi i}\oint_{\partial G}{f(z)\over
 (z-z_0)^{n+1}}dz\quad
 .$$

{\em Anwendung:} sei $g(z)$ eine Funktion mit Pol n--ter Ordnung in
 $z_0$, d. h.
 $$g(z)= {f(z)\over (z-z_0)^n}$$
 mit analytischem $f(z)$. Dann ist das Integral
 $$\oint_{\partial G}g(z)dz={2\pi i\over (n-1)!}f^{(n-1)}(z_0)\quad .$$



 \subsection{Laurentreihenentwicklung}
 Jede in einem Kreisring $R_1< \vert z-z_0\vert <R_2$ analytische
 Funktion $f$ ist dort eindeutig in die {\em Laurentreihe}
 $$f(z)=\sum_{k=-\infty}^\infty (z-z_0)^k a_k$$
 zu entwickeln. Die Koeffizienten $a_k$ sind gegeben durch
 ($C$ im Kreisring)
 $$a_k={1\over 2\pi i}\oint_C (\chi -z_0)^{-k-1}f(\chi ) d\chi \quad.$$
 Der Koeffizient $a_{-1}={1\over 2\pi i}\oint_Cf(\chi )d\chi $ hei\ss t
{\em Residuum}
 ${\rm Res}$.

{\em Anwendung:} sei wieder $g(z)$ eine Funktion mit Pol n--ter Ordnung
 in $z_0$, d.h.,
 $g(z)= {h(z)/ (z-z_0)^n}$
 mit analytischem $h(z)$. Dann bricht die Laurentreihe ab
 f\"ur $k\le -(n+1)$. Beweis: aus Cauchy'schem Integralsatz folgt
 $$a_k ={1\over 2\pi i}\oint_c(\chi -z_0)^{-k-n-1}h(\chi )d\chi =0$$
 f\"ur $-k-n-1\ge 0$.


 \subsection{Residuensatz}
 Sei $f$ analytisch in $G$ mit Ausnahme endlich vieler Stellen $z_i$
 und habe dort keine Verzweigungspunkte. Dann gilt
 $$\oint_{\partial G} f(z)dz=2\pi i \sum_{z_i} {\rm Res}f(z_i)\quad .$$




 Wenn Riemann'sche Fl\"ache der
 Funktion $f$ {\em nicht} einfach zusammenh\"angend ist.


 \subsection{Definition Verzweigungspunkt}
 Ein Punkt $z_0$ der Funktion $f(z)$ hei\ss t {\em Verzweigungspunkt,}
 falls es geschlossene Kurven um $z_0$ gibt, deren Bild eine offene
 Kurve ist.


 \subsection{Definition Riemann'sche Fl\"ache}
 Sei $f(z)$ eine mehrdeutige Funktion.
 Die verschiedenen $z$--Ebenen, auf denen $f(z)$ jeweils eindeutig
 definiert ist, bilden zusammen mit ihren Zusammenhangseigenschaften
 [definiert durch Verszeigungsschnitt(e) und Verzweigungspunkt(e)]
 die zu $f(z)$ geh\"orige {\em Riemann'sche Fl\"ache.}
 Die ben\"otigten Ebenen hei\ss en {\em Riemann'sche Bl\"atter.}


 \subsection{Definition Verzweigungspunkt $n$--ter Ordnung}
 Ein Punkt $z_0$ der Funktion $f(z)$ hei\ss t {\em Verzweigungspunkt
 $n$--ter Ordnung,} falls durch ihn und dem dazugeh\"origen
 Verzweigungsschnitt(en) $n+1$ Riemann'sche Bl\"atter verkn\"upft sind.


 \end{document}
