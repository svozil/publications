%%tth:\begin{html}<LINK REL=STYLESHEET HREF="http://tph.tuwien.ac.at/~svozil/ssh.css">\end{html}
\documentclass[prl,preprint,showpacs,showkeys,amsfonts]{revtex4}
\usepackage{graphicx}
%\documentstyle[]{article}
\RequirePackage{times}
%\RequirePackage{courier}
\RequirePackage{mathptm}
%\renewcommand{\baselinestretch}{1.3}
\begin{document}





\title{Compact quantum codes}
\author{Karl Svozil}
 \email{svozil@tuwien.ac.at}
\homepage{http://tph.tuwien.ac.at/~svozil}
\affiliation{Institut f\"ur Theoretische Physik, University of Technology Vienna,
Wiedner Hauptstra\ss e 8-10/136, A-1040 Vienna, Austria}

\begin{abstract}
We consider the quantum observables corresponding to {\em eutactic stars}.
These systems of vectors are the lower dimensional ``shadow'' image,
the orthogonal view, of $n$-dimensional orthonormal frames.
Because the quantum states are  higher dimensional,
quantum observables corresponding to eutactic stars
are comeasurable, although they do not correspond to any orthonormal basis.
This makes it possible to encode an arbitrary finite amount of information
into an effectively twodimensional quantized system.
\end{abstract}


\pacs{03.67.-a,03.67.Hk,03.65.Ta}
\keywords{quantum information theory,quantum measurement theory}


\maketitle


Almost-orthogonal systems of vectors

\begin{equation}
\{ \{ {\sqrt{\frac{2}{3}}},0,\frac{1}{{\sqrt{3}}}\} ,
  \{ - \frac{1}{{\sqrt{6}}}  ,
   \frac{1}{{\sqrt{2}}},\frac{1}{{\sqrt{3}}}\} ,
  \{ - \frac{1}{{\sqrt{6}}}  ,
   - \frac{1}{{\sqrt{2}}}  ,
   \frac{1}{{\sqrt{3}}}\} \}
\end{equation}

\begin{eqnarray}
E_1&=&\left(
\matrix{ \frac{2}{3} & 0 & \frac{{\sqrt{2}}}
   {3} \cr 0 & 0 & 0 \cr \frac{{\sqrt{2}}}{3} & 0 &
    \frac{1}{3} \cr  }
\right), \\
\qquad
E_2&=&\left(
\matrix{ \frac{1}{6} & \frac{-1}{2\,{\sqrt{3}}} &
    \frac{-1}{3\,{\sqrt{2}}} \cr \frac{-1}
   {2\,{\sqrt{3}}} & \frac{1}{2} & \frac{1}
   {{\sqrt{6}}} \cr \frac{-1}{3\,{\sqrt{2}}} & \frac{1}
   {{\sqrt{6}}} & \frac{1}{3} \cr  }
\right),   \\
\qquad
E_3&=& \left(
\matrix{ \frac{1}{6} & \frac{1}{2\,{\sqrt{3}}} &
    \frac{-1}{3\,{\sqrt{2}}} \cr \frac{1}
   {2\,{\sqrt{3}}} & \frac{1}{2} & - \frac{1}
     {{\sqrt{6}}}   \cr \frac{-1}
   {3\,{\sqrt{2}}} & - \frac{1}{{\sqrt{6}}}
      & \frac{1}{3} \cr  }
\right)
\end{eqnarray}

\begin{eqnarray}
E_1'&=& \frac{2}{3} \left(
\matrix{ 1 & 0 & 0 \cr 0 & 0 & 0 \cr 0 & 0 & 0 \cr  }
\right), \\
\qquad
E_2'&=&{1\over 2}\left(
\matrix{ \frac{1}{3} & \frac{-1}{{\sqrt{3}}} & 0 \cr \frac{-1}{{\sqrt{3}}} & 1 & 0 \cr 0 & 0 & 0 \cr  }
\right),   \\
\qquad
E_3'&=& {1\over 2}\left(
\matrix{ \frac{1}{3} & \frac{1}{{\sqrt{3}}} & 0 \cr \frac{1}{{\sqrt{3}}} & 1 & 0 \cr 0 & 0 & 0 \cr  }
\right)
\end{eqnarray}
\bibliography{svozil}
\bibliographystyle{apsrev}

\end{document}

v1[t_] := { Sin[t]*Cos[0] , Sin[t]*Sin[0] , Cos[t] };
v2[t_] := { Sin[t]*Cos[2Pi/3] , Sin[t]*Sin[2Pi/3] , Cos[t] };
v3[t_] := { Sin[t]*Cos[4Pi/3] , Sin[t]*Sin[4Pi/3] , Cos[t] };


Solve[ {v1[t].v2[t]==0 , v1[t].v3[t]==0 , v3[t].v2[t]==0 } ,t ]

{v1[ArcCos[1/3]], v2[ArcCos[1/3]], v3[ArcCos[1/3]]}

TP1[a_, b_] := Table[(*a, b are n and m - dim vectors*)
    a[[s]]*b[[t]], {s, 1, Length[a]}, {t, 1, Length[b]}]

E1tex=TeXForm[MatrixForm[TP1[v1[ArcCos[1/Sqrt[3]]],v1[ArcCos[1/Sqrt[3]]]]]];
E2tex=TeXForm[MatrixForm[TP1[v2[ArcCos[1/Sqrt[3]]],v2[ArcCos[1/Sqrt[3]]]]]];
E3tex=TeXForm[MatrixForm[TP1[v3[ArcCos[1/Sqrt[3]]],v3[ArcCos[1/Sqrt[3]]]]]];
E1=TP1[v1[ArcCos[1/Sqrt[3]]],v1[ArcCos[1/Sqrt[3]]]];
E2=TP1[v2[ArcCos[1/Sqrt[3]]],v2[ArcCos[1/Sqrt[3]]]];
E3=TP1[v3[ArcCos[1/Sqrt[3]]],v3[ArcCos[1/Sqrt[3]]]];


Pxy = {{1, 0, 0}, {0, 1, 0}, {0, 0, 0}};
Pz = {{0, 0, 0}, {0, 0, 0}, {0, 0, 1}};
E1xy = Pxy.E1.Pxy;
E2xy = Pxy.E2.Pxy;
E3xy = Pxy.E3.Pxy;


p1={1,0,0};
p2={-1/2,Sqrt[3]/2,0};
p3={-1/2,-Sqrt[3]/2,0};
F1=TP1[p1,p1];
F2=TP1[p2,p2];
F3=TP1[p3,p3];


Show[Graphics3D[{
      Line[{{0, 0, 0}, v1[ArcCos[1/3]]}], Line[{{0, 0, 0}, v2[ArcCos[1/3]]}],
      Line[{{0, 0, 0}, v3[ArcCos[1/3]]}],
      Line[{{0, 0, 0}, {1, 0, 0}}],
      Line[{{0, 0, 0}, {-1/2, Sqrt[3]/2, 0}}],
      Line[{{0, 0, 0}, {-1/2, -Sqrt[3]/2, 0}}]
      }], AspectRatio -> 1, PlotRange -> {-1, 1},
  ViewPoint -> {-1.245, -1.032, 2.972}]
