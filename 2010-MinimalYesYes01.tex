%\documentclass[twocolumn,aps,pra,showpacs]{revtex4}
\documentclass[twocolumn,aps,pra,showpacs]{revtex4-1}
%\documentclass[preprint,aps,pra,showpacs]{revtex4-1}

\usepackage{graphicx}
\usepackage{amsfonts}
\usepackage{amsmath}
\usepackage{amssymb}

\def\endproof{\vrule height6pt width6pt depth0pt}

%%%%%%%%%%%%%%%%%%%%%%%%%%%%%%%%%%%%%%%%%%%%%%%%%%%%%%%%%%%%%%%%%%%

\begin{document}

%%%%%%%%%%%%%%%%%%%%%%%%%%%%%%%%%%%%%%%%%%%%%%%%%%%%%%%%%%%%%%%%%%%

\title{Minimum yes-implies-yes sets}

%%%%%%%%%%%%%%%%%%%%%%%%%%%%%%%%%%%%%%%%%%%%%%%%%%%%%%%%%%%%%%%%%%%

\author{Ad\'an Cabello}
 \email{adan@us.es}
 \affiliation{Departamento de F\'{\i}sica
 Aplicada II, Universidad de Sevilla, E-41012 Sevilla, Spain}

\author{Jos\'{e} R. Portillo}
 \email{josera@us.es}
 \affiliation{Departamento de Matem\'{a}tica Aplicada I,
 Universidad de Sevilla, E-41012 Sevilla, Spain}

\author{Karl Svozil}
 \email{svozil@tuwien.ac.at}
 \affiliation{Institute for Theoretical Physics, University of Technology Vienna,
 Wiedner Hauptstrasse 8-10/136, 1040 Vienna, Austria}

%%%%%%%%%%%%%%%%%%%%%%%%%%%%%%%%%%%%%%%%%%%%%%%%%%%%%%%%%%%%%%%%%%%

\date{\today}

%First version: September 2 2010 (Aswan).

%%%%%%%%%%%%%%%%%%%%%%%%%%%%%%%%%%%%%%%%%%%%%%%%%%%%%%%%%%%%%%%%%%%

\begin{abstract}
Any set of quantum propositions such that, due to exclusiveness
and completeness, if proposition $A$ is true, then a
nonexclusive proposition $B$ must be true in any theory which
assign coexistent answers to all the propositions, provides a
proof of quantum contextuality. Indeed, by concatenating these
sets, a proof valid for any initial state can be constructed.
Sets with this property are called {\em yes-implies-yes sets}
(YIYS). Examples of them are known since 1965 for systems
described by Hilbert spaces of any dimension $d>2$. Here we
solve the problem of which are the YIYS with a minimum number
of propositions for any $d>2$. We prove that, in $d=3$, there
is only one and has $n=10$ propositions (including $A$ and
$B$). In $d=4$,\ldots
\end{abstract}

%%%%%%%%%%%%%%%%%%%%%%%%%%%%%%%%%%%%%%%%%%%%%%%%%%%%%%%%%%%%%%%%%%%

\pacs{03.65.Ta}

%Foundations of quantum mechanics; measurement theory

\maketitle

%%%%%%%%%%%%%%%%%%%%%%%%%%%%%%%%%%%%%%%%%%%%%%%%%%%%%%%%%%%%%%%%%%%

\section{Introduction}

%%%%%%%%%%%%%%%%%%%%%%%%%%%%%%%%%%%%%%%%%%%%%%%%%%%%%%%%%%%%%%%%%%%

{\em Quantum contextuality (of results)} \cite{Specker60, KS65,
Bell66, KS67},
or, by another wording {\em quantum value indefiniteness}~\cite{peres222},\marginpar{{\bf KS}}
is the term used to designate the impossibility
of reproducing some predictions of quantum mechanics by means
of any theory which assigns coexistent noncontextual answers to
all propositions (i.e., which assigns the same answer with
independence of which other sets of compatible propositions are
considered simultaneously).

Two propositions are {\em exclusive} when both cannot be
simultaneously true. A set of mutually exclusive compatible
propositions is {\em complete} when one of the propositions is
true and the others are false. A {\em context} is a set
(complete or not) of mutually exclusive compatible
propositions.

A {\em yes-implies-yes set} (YIYS) is a set $S$ of quantum
propositions (i.e., rank-one projection operators) in which,
due to the exclusiveness and completeness of some of the
elements of $S$, if proposition $A \in S$ is true, then a
proposition $B \in S$ must be true in any theory which assigns\marginpar{{\bf KS}}
coexistent answers to all the propositions, even tough $A$ and
$B$ are nonexclusive, and {\it vice versa}.\marginpar{{\bf KS}}
Explicit examples will be presented
below. A YIYS is said {\em critical} when it does not contain a
subset which is a YIYS.

A YIYS serves to prove quantum contextuality, since, according
to quantum mechanics, in the quantum state ``$A$ true'' there
is a nonzero probability of obtaining $B$ false (see, e.g.,
\cite{Stairs83, Clifton93, CG95}). These kind of proofs can be
converted into experimental tests of whether nature admit
theories satisfying the assumption of noncontextuality (of
results) \cite{CT10}. On the other hand, a proof of quantum
contextuality valid for {\em any} initial state (i.e., a
state-independent proof) can be constructed, by concatenating
several YIYS \cite{KS67, CG96}. State-independent proofs have
been recently converted into state-independent experimental
tests \cite{Cabello08, BBCP09, KZGKGCBR09, ARBC09, Cabello10}.

There are other reasons which make YIYS interesting.\marginpar{{\bf KS}}
A collection of quantum observables containing a YIYS  may still allow
two-valued states which can be associated with pre-defined truth values.
Nevertheless,
the {\em set of all such  two-valued states} (if they exist) does not allow the separation
of all propositions in YIYS by these truth values alone.
As a consequence of this {\em nonseparating} set of two-valued states,
a YIYS cannot be faithfully (i.e., in a homeomorphic, structure preserving manner)
embedded into a Boolean algebra, and thus has no ``classical interpretation''~\cite{pulmannova-91,svozil-ql,CalHerSvo,svozil-2006-omni}.
Thus, unlike the Kochen-Specker case, where there appears to be no context independent truth assignment,
a YIYS still allows for such ``definite values,''
but they are ``too scarce'' to render a fully classical interpretation preserving all the logical operations and implications among the
potentially observable propositions.

A YIYS can be represented by a {\em graph} in which vertices
represent propositions, edges connect exclusive propositions,
and $d$ mutually connected vertices represent (complete)
contexts.

YIYS are known for any physical system described by a Hilbert
space of dimension $d>2$ \cite{KS65, Bell66, KS67}. In $d=3$,
Bell found one with $n=13$ propositions \cite{Bell66}, and
Kochen and Specker found a simpler YIYS with $n=10$ \cite{KS65,
KS67}, which is illustrated in Fig. \ref{KS10}. Both sets
belong to a broader family with $n=10+3 m$, with $m=0,1,\ldots$
\cite{CG95}. For $d>2$, YIYS with $n=7+d$ are easy to construct
from the set of Fig. \ref{KS10} by adding the vector with all
components zero but the one corresponding to the new dimension
\cite{CG96}. In this paper we address the problem of which are
the {\em simplest} YIYS, i.e., those with a minimum number of
propositions for any $d>2$.

%%%%%%%%%%%%%%%%%%%%%%%%%%%%%%%%%%%%%%%%%%%%%%%%%%%%%%%%%%%%%%%%%%%%
% Fig. 1
%%%%%%%%%%%%%%%%%%%%%%%%%%%%%%%%%%%%%%%%%%%%%%%%%%%%%%%%%%%%%%%%%%%%

\begin{figure}
%\vspace{3.4cm}
\centerline{\includegraphics[width=8.0cm]{KS10block.eps}}
\caption{\label{KS10}
The simplest known YIYS in $d=3$. Vertices represent propositions,
edges represent that the connected propositions cannot be both true (exclusiveness).
Triangles represents complete sets (i.e., one and only one of the propositions must be yes).
If $A$ is yes then $B$ is yes. This set is realizable in $S^2$ by taking, for instance,
$\langle A|=(1,1,1)/\sqrt{3}$, $\langle v_1|=(1,-1,0)/\sqrt{2}$, $\langle v_2|=(0,1,-1)/\sqrt{2}$, $\langle v_3|=(1,0,-1)/\sqrt{2}$, $\langle v_4|=(0,0,1)$,
$\langle v_5|=(1,0,0)$, $\langle v_6|=(1,1,0)/\sqrt{2}$, $\langle v_7|=(0,1,1)/\sqrt{2}$, $\langle v_8|=(1,-1,1)/\sqrt{4}$, and $\langle B|=(1,2,1)/\sqrt{6}$.
The proposition $v_i$ is defined as $|v_i\rangle \langle v_i|$.
To obtain a YIYS set in $d=4$ is enough to add $\langle v_9|=(0,0,0,1)$, and similarly to obtain YIYS in higher dimensions.}
\end{figure}

%%%%%%%%%%%%%%%%%%%%%%%%%%%%%%%%%%%%%%%%%%%%%%%%%%%%%%%%%%%%%%%%%%%%

\section{Method to obtain YIYS with minimum number of propositions}

%%%%%%%%%%%%%%%%%%%%%%%%%%%%%%%%%%%%%%%%%%%%%%%%%%%%%%%%%%%%%%%%%%%

A graph is said to be {\em nonrealizable} in dimension~$d$ if
it represents a set of rays which is not realizable in
$S^{d-1}$.

%%%%%%%%%%%%%%%%%%%%%%%%%%%%%%%%%%%%%%%%%%%%%%%%%%%%%%%%%%%%%%%%%%%%

{\em Lemma 1: }The simplest nonrealizable graph in $d=1$
consists of two vertices. The simplest nonrealizable graph in
$d=2$ has three vertices with one of them connected to the
other two. From these two nonrealizable graphs one can
recursively construct nonrealizable graphs in any dimension~$d$
by starting from the nonrealizable graph in dimension~$d-2$ and
adding to it two vertices connected with all vertices of the
nonrealizable graph in~$d-2$.

%%%%%%%%%%%%%%%%%%%%%%%%%%%%%%%%%%%%%%%%%%%%%%%%%%%%%%%%%%%%%%%%%%%%%

{\em Proof:} One cannot have two different rays in~$S^0$
because there is only one ray in~$S^0$. In~$S^1$, if a ray is
orthogonal to a second ray, and the second is orthogonal to a
third, then the first and third rays should be the same. If we
add two dimensions and two different rays, both orthogonal to
those of a previous set~$I$, then these two new rays span a
two-dimensional subspace orthogonal to~$I$. Therefore, if~$I$
was nonrealizable, then the resulting set is also
nonrealizable.\hfill\endproof

%%%%%%%%%%%%%%%%%%%%%%%%%%%%%%%%%%%%%%%%%%%%%%%%%%%%%%%%%%%%%%%%%%%%%

{\em Lemma 2: }Every $n$-vertex graph corresponding to a
critical YIYS in dimension $d$ contains a $n+1-d$-vertex graph
corresponding to a yes-implies-no set.

{\em Proof: }Obvious {\bf [EXPAND THIS]}.\hfill\endproof

{\em Lemma 3: }Every vertex of a graph corresponding to a
yes-implies-no set must be connected to, at least, two vertices
(i.e., the graph must have minimal valence two).

{\em Proof: }Obvious {\bf [EXPAND THIS]}.\hfill\endproof

{\em Lemma 4: }The graph of a critical yes-implies-no set must
be biconnected (i.e., when removing any vertex the resulting
graph must be still connected).

{\em Proof: }Otherwise the yes-implies-no set is not
critical.\hfill\endproof

{\em Lemma 5: }The graph of any 8-vertex yes-implies-no set in
$d=3$ must contain at least two triangles.

{\em Proof: }Otherwise the graph can be colored as follows: $A$
true and everything else false.\hfill\endproof

Therefore, to obtain the minimal YIYS in $d=3$.

{\em Step 1: }We generate {\em all} nonisomorphic $n$-vertex
biconnected graphs of minimal valence two, not containing
cycles on length four, and containing at least two triangles.
This can be efficiently done using the computer program {\em
nauty} \cite{McKay07}. We obtain that there are $2$ graphs for
$n=7$, and $8$ graphs for $n=8$. All of them are illustrated in
Fig. \ref{nauty3}.

%%%%%%%%%%%%%%%%%%%%%%%%%%%%%%%%%%%%%%%%%%%%%%%%%%%%%%%%%%%%%%%%%%%%
% Fig. 2
%%%%%%%%%%%%%%%%%%%%%%%%%%%%%%%%%%%%%%%%%%%%%%%%%%%%%%%%%%%%%%%%%%%%

\begin{figure}
%\vspace{3.4cm}
\centerline{\includegraphics[width=7.8cm]{nauty3.eps}}
\caption{\label{nauty3}
All nonisomorphic $7$-vertex (first row) and $8$-vertex (the rest)
biconnected graphs of minimal valence two, not containing
cycles on length four, and containing at least two triangles.}
\end{figure}

%%%%%%%%%%%%%%%%%%%%%%%%%%%%%%%%%%%%%%%%%%%%%%%%%%%%%%%%%%%%%%%%%%%%

{\em Step 2: }For every graph obtained after step 1, consider
all possible pairs of vertices $(v_i,v_j)$. If, for one
$(v_i,v_j)$, the graph does not admit a noncontextual
assignment when $v_i=1$ and $v_j=1$ (i.e., when both are true),
then the graph is a yes-implies-no set in which $A=v_i$ and
$B=v_j$. The test of whether or not a graph admits a
noncontextual assignment can be done using a simple computer
program (e.g., \cite{Peres93}).

After step 2, we find that only the last graph in Fig.
\ref{nauty3} corresponds to a yes-implies-no set. This graph
was first introduced by Kochen and Specker \cite{KS65} and is a
subgraph of the graph in Fig. \ref{KS10}. This proves that, in
$d=3$, there is no yes-implies-no set with less than or equal
number of propositions than the one introduced by Kochen and
Specker in 1965, which means that there is no YIYS with less
than or equal number of propositions than the one in Fig.
\ref{KS10} used in \cite{KS67}.

%%%%%%%%%%%%%%%%%%%%%%%%%%%%%%%%%%%%%%%%%%%%%%%%%%%%%%%%%%%%%%%%%%%

\section{Higher dimensions}

%%%%%%%%%%%%%%%%%%%%%%%%%%%%%%%%%%%%%%%%%%%%%%%%%%%%%%%%%%%%%%%%%%%

Similarly for $d=4,5,\ldots$

{\bf [JOSERRA]: Use {\em nauty} to calculate all nonisomorphic
$d+4$-vertex and $d+5$-vertex biconnected graphs of minimal
valence two, not containing nonrealizable subgraphs (see Lemma
1), and containing at least two disjoint $d$-vertex mutually
connected subgraphs. Fisrt do it for $d=4$.}

In $d=4$, by inspection, I have found no $8$-proposition, but
{\em several} $9$-proposition yes-implies-no sets. All of them
contain the following $9$-vertex subgraph (which constitutes by
itself a yes-implies-no set): $A1$, $A4$, $A5$, 12, 13, 14, 23,
24, $2B$, 34, 36, 45, 46, 47, 56, 57, 67, and $7B$. To my
knowledge, this graph was fist introduced in \cite{CEG96} and
first explicitly presented as a graph in \cite{Cabello96}.

Does it happen that there is a basic minimal yes-implies-no set
in any $d$ and all the other minimal sets are obtained by
adding edges?

%%%%%%%%%%%%%%%%%%%%%%%%%%%%%%%%%%%%%%%%%%%%%%%%%%%%%%%%%%%%%%%%%%%

\section{Conclusions and open problems}

%%%%%%%%%%%%%%%%%%%%%%%%%%%%%%%%%%%%%%%%%%%%%%%%%%%%%%%%%%%%%%%%%%%

Peres conjectured that the simplest possible state-independent
proof is the one in \cite{CEG96}, which requires 18
propositions in $d=4$ \cite{Peres03}. The method we have
developed in this article, and the obtained minimal YIYS, can
be helpful to prove Peres' conjecture \cite{CPP10}. Other
related open problem which can benefit from these results is
which is the the simplest state-independent proof in $d=3$.

%%%%%%%%%%%%%%%%%%%%%%%%%%%%%%%%%%%%%%%%%%%%%%%%%%%%%%%%%%%%%%%%%%%

\begin{acknowledgments}
A.C. acknowledges support from the Spanish MCI Project No.\
FIS2008-05596.
\end{acknowledgments}

%%%%%%%%%%%%%%%%%%%%%%%%%%%%%%%%%%%%%%%%%%%%%%%%%%%%%%%%%%%%%%%%%%%

\begin{thebibliography}{99}

%%%%%%%%%%%%%%%%%%%%%%%%%%%%%%%%%%%%%%%%%%%%%%%%%%%%%%%%%%%%%%%%%%%

\bibitem{Specker60}
 E. P. Specker,
 %``Die Logik nicht gleichzeitig entscheidbarer Aussagen'',
 Dialectica {\bf 14}, 239 (1960).

\bibitem{KS65}
 S. Kochen and E. P. Specker,
 %``Logical structures arising in quantum theory'',
 in {\em Symposium on the Theory of Models},
 edited by J. W. Addison, L. Henkin, and A. Tarski
 (North-Holland, Amsterdam, Holland, 1965), p.~177.

\bibitem{Bell66}
 J. S. Bell,
 %``On the problem of hidden variables in quantum mechanics'',
 Rev. Mod. Phys. {\bf 38}, 447 (1966).

\bibitem{KS67}
 S. Kochen and E. P. Specker,
 %``The problem of hidden variables in quantum mechanics'',
 J. Math. Mech. {\bf 17}, 59 (1967).

%%%%%%%%%%%%%%%%%%%%%%%%%%%%%%%%%%%%%%%%%%%%%%%%%%%%%%%%%%%%%%%%%%%

\bibitem{Stairs83}
 A. Stairs,
 %``Quantum logic, realism, and value definiteness'',
 Phil. Sci. {\bf 50}, 578 (1983).

\bibitem{Clifton93}
 R. Clifton,
 %``Getting contextual and nonlocal elements-of-reality the easy way'',
 Am. J. Phys. {\bf 61}, 443 (1993);
 H. Bechmann Johansen,
 %`Comment on ``Getting contextual and nonlocal
 %elements-of-reality the easy way'', by R. Clifton',
 Am. J. Phys. {\bf 62}, 471 (1994);
 P. E. Vermaas,
 %`Comment on ``Getting contextual and nonlocal
 %elements-of-reality the easy way'', by Rob Clifton [{\em Am. J.
 %Phys.} {\bf 61}, 443-447 (1993)]',
 Am. J. Phys. {\bf 62}, 658 (1994).

\bibitem{CG95}
 A. Cabello and G. Garc\'{\i}a-Alcaine,
 %``A hidden-variables versus quantum mechanics experiment'',
 J. Phys.~A {\bf 28}, 3719 (1995).

\bibitem{CT10}
 A. Cabello and M. Terra Cunha,
 %``Proposal of a two-qutrit contextuality test
 %free of the finite precision and compatibility loopholes'',
 \eprint{arXiv:1009.2330}.

%%%%%%%%%%%%%%%%%%%%%%%%%%%%%%%%%%%%%%%%%%%%%%%%%%%%%%%%%%%%%%%%%%%

\bibitem{CG96}
 A. Cabello and G. Garc\'{\i}a-Alcaine,
 %``Bell-Kochen-Specker theorem for any finite dimensions $n \geq 3$'',
 J. Phys.~A {\bf 29}, 1025 (1996).

%%%%%%%%%%%%%%%%%%%%%%%%%%%%%%%%%%%%%%%%%%%%%%%%%%%%%%%%%%%%%%%%%%%

\bibitem{Cabello08}
 A. Cabello,
 %``Experimentally testable state-independent quantum contextuality'',
 Phys. Rev. Lett. {\bf 101}, 210401 (2008).

\bibitem{BBCP09}
 P. Badzi{\c a}g, I. Bengtsson, A. Cabello, and I. Pitowsky,
 %``Universality of state-independent violation of inequalities for non-contextual
 %theories'',
 Phys. Rev. Lett. {\bf 103}, 050401 (2009).

\bibitem{KZGKGCBR09}
 %G. Kirchmair {\em et al.},
 G. Kirchmair, F. Z\"ahringer, R. Gerritsma, M. Kleinmann,
 O. G{\"u}hne, A. Cabello, R. Blatt, and C. F. Roos,
 %``State-independent experimental test of quantum contextuality'',
 Nature (London) {\bf 460}, 494 (2009).

\bibitem{ARBC09}
 E. Amselem, M. R{\aa }dmark, M. Bourennane, and A. Cabello,
 %``State-independent quantum contextuality with single photons'',
 Phys. Rev. Lett. {\bf 103}, 160405 (2009).

\bibitem{Cabello10}
 A. Cabello,
 %``Proposal for revealing quantum nonlocality via local
 %contextuality'',
 Phys. Rev. Lett. {\bf 104}, 220401 (2010).

%%%%%%%%%%%%%%%%%%%%%%%%%%%%%%%%%%%%%%%%%%%%%%%%%%%%%%%%%%%%%%%%%%%%%

 \bibitem{McKay07}
 B. D. McKay,
 {\em {\tt nauty} User's Guide (Version 2.4)}
 (Department of Computer Science, Australian National
 University, Canberra, Australia, 2007).

\bibitem{Peres93}
 A. Peres,
 {\em Quantum Theory: Concepts and Methods}
 (Kluwer, Dordrecht, 1993), p.~209.

%%%%%%%%%%%%%%%%%%%%%%%%%%%%%%%%%%%%%%%%%%%%%%%%%%%%%%%%%%%%%%%%%%%%%

\bibitem{CEG96}
 A. Cabello, J. M. Estebaranz and G. Garc\'{\i}a-Alcaine,
 %``Bell-Kochen-Specker theorem: A proof with 18 vectors'',
 Phys. Lett. A {\bf 212}, 183 (1996), Eqs. (20)--(26).

\bibitem{Cabello96}
 A. Cabello,
 {\em Pruebas Algebraicas de Imposibilidad de Variables Ocultas
 en Mac\'anica Cu\'antica}, Ph. D. Thesis, Universidad
 Complutense de Madrid, 1996, p. 201. It can be dowlowded from
 www.adancabello.com.

%%%%%%%%%%%%%%%%%%%%%%%%%%%%%%%%%%%%%%%%%%%%%%%%%%%%%%%%%%%%%%%%%%%%%

\bibitem{Peres03}
 A. Peres,
%``What's wrong with these observables?'',
 Found. Phys. {\bf 33}, 1543 (2003).

\bibitem{CPP10}
 A. Cabello, J. R. Portillo, and G. Potel
 (unpublished).

%%%%%%%%%%%%%%%%%%%%%%%%%%%%%%%%%%%%%%%%%%%%%%%%%%%%%%%%%%%%%%%%%%%%%

\end{thebibliography}

\end{document}

%%%%%%%%%%%%%%%%%%%%%%%%%%%%%%%%%%%%%%%%%%%%%%%%%%%%%%%%%%%%%%%%%%%
