\documentclass[reprint,aps,pra,superscriptaddress,longbibliography]{revtex4-2}

\usepackage[T1]{fontenc}
\usepackage{graphicx}
\usepackage{amsmath,amssymb}
\usepackage{siunitx}
\usepackage{hyperref}
\usepackage{orcidlink}

\begin{document}

\title{Role of Mechanical Impedance in Wind-Driven Vehicles That Outrun the Wind}

\author{Karl Svozil\,\orcidlink{0000-0001-6554-2802}}
\affiliation{Institute for Theoretical Physics, TU Wien, Wiedner Hauptstrasse 8-10/136, A-1040 Vienna, Austria}

\date{\today}

\begin{abstract}
Propeller-driven ``sailcars'' have repeatedly demonstrated ground speeds well above the free-stream wind speed, both downwind and upwind. Although the governing physics is straightforward, many explanations lack a pedagogical bridge between formal analysis and intuition. We build such a bridge through three complementary mechanical ``images of nature'' in the sense of Hertz: a lever, a gearbox, and Sandu Popescu's sliding-boat thought experiment. All three highlight the \emph{necessity of mechanical impedance}---a reaction body of (almost) zero velocity---to enable directional energy transfer from the wind.
\end{abstract}

\maketitle

Wind-driven craft that outrun the very air that propels them
have intrigued engineers since at least the fourteenth century~\cite{mcdonald2021}.
The twenty-first-century renaissance of the topic culminated in the ``Blackbird''
land yacht, officially clocked at \num{2.8} times true wind speed
downwind~\cite{nalsa2010}, as well as \num{2.01} upwind~\cite{blackbird2011-casterttw}. Vigorous on-line debate and even a
\$10,000 wager among physicists followed~\cite{veritasium2021,VeritasiumWager2021}, showing
that the phenomenon still feels counter-intuitive.

The key is embarrassingly simple: Two media, such as solid ground and air, but also water and air~\cite{Blackford1978}, moving at different velocities constitute an energy source that can be tapped once \emph{mechanical impedance}---a reaction force path to the ground or water---is provided.
This paper endeavors to provide such a framework, not through rigorous mathematical derivation, but through a series of mechanical analogies. Our goal is to create what Heinrich Hertz termed ``images of nature'' (\textit{innere Scheinbilder})---conceptual models that, while not being the reality itself, allow us to grasp the essential mechanics at play \cite{hertz-94e}. As Hertz suggested in the introduction to his \textit{Principles of Mechanics}, the utility of such images lies in their ability to allow our thoughts to mirror the necessary consequences of natural phenomena.

We will begin by introducing the simplest mechanical amplifier of motion: the lever. We will explore how a lever, appropriately configured, can translate a slower input speed into a faster output speed, both in the direction of the initial force and against it. We will then translate this lever analogy into the language of control theory by conceptualizing gearboxes that achieve positive and negative transmission ratios.

Following this, we will discuss a related conceptual puzzle mentioned by Professor Sandu Popescu, which we term the ``Popescu glide,'' to further illustrate the critical role of a fixed reference frame. Finally, we will synthesize these analogies and connect them directly to the operation of high-performance land and sea yachts, arguing that the ``grounding'' or ``impedance'' offered by the earth or water is the non-negotiable element that makes this ``impossible'' feat possible.

\section{The Lever Analogy for Wind-Driven Acceleration}

The first image or metaphor we employ is that of a lever, a fundamental simple machine that amplifies force or speed through a pivot point. This analogy underscores how wind impedance---the resistance provided by a grounded pivot---can be leveraged to achieve velocities exceeding the wind speed. Critically, this model idealizes the system as non-elastic and frictionless, which is a simplification; in reality, elastic deformations and friction would introduce losses, potentially invalidating unbounded speed claims.

\subsection{With the Wind}

Consider a configuration depicted in Fig.~\ref{fig:lever-downupwind}(a):
\begin{enumerate}
    \item  A non-elastic, very long lever or rod.
    \item A pivot point at one (bottom, left) end of this rod, grounded to provide impedance against motion---assume the pivot is bolted to an immovable base.
    \item A sail attached in the middle or some lower portion of the rod.
    \item An observation deck with a sensor at the other (top, right) end.
    \item Initially, the lever stands upright, vertically, and perpendicular to the wind direction (from left to right).
\end{enumerate}

\begin{figure*}[htbp]
\centering
\begin{tabular}{ccc}
\includegraphics[width=0.48\textwidth]{2025-wind-lww.jpg}
&$\qquad$&
\includegraphics[width=0.48\textwidth]{2025-wind-law.jpg}
\\
(a)&&(b)
\end{tabular}
\caption{Schematic of the lever setup (a) for downwind acceleration. The pivot is at the left end, sail in the middle, and observation deck at the right end. Wind blows from left to right, tilting the lever. (b) For upwind acceleration the pivot is in the middle, sail at the left end, and observation deck at the right end. Wind blows from left to right, tilting the lever to move the deck leftward.}
\label{fig:lever-downupwind}
\end{figure*}

When wind blows from left to right at speed $v_w$, it exerts force on the sail, tilting the lever around the pivot. The sail, at distance $d_s$ from the pivot, moves at speed $v_s = \alpha v_w$, where $0< \alpha < 1$ is a fraction accounting for inefficiencies (e.g., sail drag coefficient).

The lever length is $L$, with the deck at $L$. Angular velocity $\omega = v_s / d_s = \alpha v_w / d_s$. Deck speed $v_d = \omega L = \alpha (L / d_s) v_w$. For $L \gg d_s$, if $L / d_s > 1/\alpha$, then $v_d > v_w$. Thus, a sufficiently long lever accelerates the deck beyond wind speed.

There is a tradeoff: Wind force $F_w \propto v_w^2$ (quadratic drag law~\cite{Anderson2024}) yields deck force $F_d = F_w (d_s / L)$, decreasing with $L$. This highlights a limitation---while speed amplifies, force diminishes, akin to impedance mismatch in circuits, where high velocity comes at low power transfer. In practice, this could limit sustained motion if external loads are applied.

\subsection{Against the Wind}

Now consider a modified setup for upwind motion, as shown in Fig.~\ref{fig:lever-downupwind}(b). The pivot is somewhere in-between the ends, at distance $d_p$ from the left end. The sail is at the left end, and the observation deck at the right end.

Wind pushes the sail rightward (downwind), so the sail velocity $v_s = \beta v_w$, again with $0 <\beta <1$. This tilts the lever such that the deck moves leftward, against the wind. Angular velocity $\omega = v_s / d_p = \beta v_w / d_p$. Deck velocity $v_d = -\omega (L - d_p) = -(\beta v_w / d_p) (L - d_p)$.

The relative velocity is negative; for $L - d_p \gg d_p$, if $(L - d_p)/d_p > 1/\beta$, $|v_d| > v_w$.

However, once the deck begins to slide \emph{against} the free-stream
wind, its own ``apparent-wind'' speed increases because deck velocity
and wind velocity \emph{add} rather than subtract.  Define the
dimensionless speed ratio
$\eta \equiv \frac{|v_d|}{v_w}$,
so that the apparent wind at the deck is
$v_{\text{rel},d} = v_w + |v_d|= v_w\,(1+\eta)$.
Applying the quadratic drag law~\cite{Anderson2024} to this relative flow gives the aerodynamic load on the deck
(or on any small ``platform sail'' mounted there):
\begin{align}
F_\text{drag}
= \gamma k v_{\text{rel},d}^2
= \gamma k [v_w\,(1+\eta)]^2
. \notag
\end{align}
Here $k$ is the standard dynamic--pressure coefficient, and
$0<\gamma\le 1$ accounts for the deck's smaller reference area or lower
drag coefficient compared with the main sail. Choosing $\gamma=1$
therefore yields a conservative (largest-drag) estimate.   Because
$F_{\text{drag}}\propto(1+\eta)^{2}$, the deck drag rises rapidly once
$\eta \gtrsim 1$, thereby setting a practical upper bound on the attainable upwind speed.

The force generated by the original sail on the other side of the pivot,
 transmitted to the deck side via the lever's mechanical advantage,
is $F_\text{gen} = F_w \cdot (d_p / (L - d_p)) = F_w / r$, where $F_w = k [v_w (1 - \beta)]^2$
is the force on the original sail and $r = (L - d_p)/d_p$ is the leverage ratio.
As defined earlier the speed ratio $\eta = |v_d| / v_w = \beta r$, so $r = \eta / \beta$.

The moment when the platform stops moving against the wind occurs when $F_\text{drag} = F_\text{gen}$, thereby encoding a force balance, implying zero net torque and no further acceleration. Substituting yields $\gamma (1 + \eta)^2 = (1 - \beta)^2 / r$. With $r = \eta / \beta$, this becomes $\gamma (1 + \eta)^2 = (1 - \beta)^2 \cdot (\beta / \eta)$.
This transcendental equation can be solved numerically for any $0 < \beta < 1$ and $\gamma \leq 1$:
For example, with $\gamma=1$ (symmetric), $\beta=0.1$ gives $\eta \approx 0.35$, $\beta=0.5$
gives $\eta \approx 0.42$, and $\beta=0.9$ gives $\eta \approx 0.12$ (all $<1$, so $|v_d| < v_w$).
The non-monotonic trend arises because while larger $\beta$ (across different setups) improves kinematic amplification
(smaller $r$ needed for given $\eta$), it also reduces the sail's relative wind
(the effective wind speed and direction experienced by the sail) $(1 - \beta)$,
thereby weakening $F_w$---the weakening dominates for large $\beta$. For $\gamma=0.5$ (weaker deck drag),
$\eta$ increases (e.g., $\beta=0.5$ gives $\eta \approx 0.6 <1$); for $\gamma=0.1$, $\eta \approx 1.3 >1$ ($|v_d| > v_w$).
A general approximation is $\eta \approx \sqrt[3]{\beta (1 - \beta)^2 / \gamma}$,
showing that lower $\gamma$ or optimized $\beta$ can enable $\eta >1$.
Note that this drag-limited estimate often caps speeds below $v_w$ (especially for $\gamma \approx 1$),
and seemingly contradicting the ideal kinematic case where $|v_d| > v_w$ is possible for large $r$.
This highlights the model's limitation---drag bounds the effective $r$ achievable (via $r = \eta / \beta$),
preventing arbitrary amplification unless $\gamma \ll 1$.
As a caveat, the above estimate is heuristic, assuming steady-state conditions---in real upwind sailing,
apparent wind limits direct upwind speed to less than $v_w$, though angled tacking allows effective progress.
This metaphor thus risks overestimating upwind potential without considering aerodynamic stalls.

It is essential to emphasize that the pivot point and its ``grounding'' provide the necessary impedance: Without it, no impedance against the wind would take place, and the lever would not function as such but would be a uniform rod subjected to wind, translating without tilting or amplification. This grounding is analogous to electrical impedance in circuits, where mismatch prevents power transfer---a critical insight often overlooked in simplistic models.

In the lever-gear analogy the \emph{pivot---or, in the following gearbox picture, the
casing---}corresponds to the cart's \emph{wheel axle} (the ground contact),
which supplies the reaction force that provides the essential mechanical
impedance. The \emph{short lever arm} maps onto the \emph{inner region of the
propeller blades}, where the local velocity is low but the aerodynamic thrust
is high. This is the ``turbine side'' of the cart. By contrast, the
\emph{long lever arm} is analogous to the \emph{vehicle's wheel-side components},
which moves fast while sustaining only a small opposing force. Finally, the
\emph{mechanical advantage} of the lever or the gear ratio of the gearbox is
implemented physically by the combination of \emph{blade size and pitch and wheel
radius}; together they set the conversion ratio between blade rotation and
ground translation.

\section{Gearboxes, Impedance, and Control Theory}

The translation of the lever metaphor into the language of control theory~\cite{bechhoefer-2021} and mechanical engineering is not merely pedagogical:
It reveals the deep necessity of impedance (or ``grounding'') for directional energy transfer. Here,
we analyze two canonical gear train configurations, drawing explicit analogies to electrical networks and control-theoretic concepts such as the inerter~\cite{Smith2002}.
Thereby it is important to keep in mind that there are three separate velocity sign-changing operations:
\begin{enumerate}
    \item gear-gear meshing (every external mesh flips rotation);
    \item gear-rack contact (top versus bottom of the pitch circle flips the rack's linear direction for a given sense of rotation),
    \item any inserted idler adds an extra gear-gear mesh.
\end{enumerate}

\subsection{Gearbox with Positive Transmission}

\begin{figure*}[htbp]
    \centering
    \begin{tabular}{ccc}
        \includegraphics[width=0.48\textwidth]{2025-windgww.jpg}
        &$\qquad$&
        \includegraphics[width=0.48\textwidth]{2025-windgaw.jpg}
        \\
        (a)&&(b)
    \end{tabular}
    \caption{Schematic for gearboxes. (a) Positive transmission for downwind motion (same direction, velocity amplified). (b) Negative transmission for upwind motion (opposite direction, velocity reversed and amplified). Both require grounding (impedance) for effective operation. Replace with actual diagrams if desired.}
    \label{fig:gearboxes}
\end{figure*}

Consider a mechanical system composed of the following elements as depicted in Fig.~\ref{fig:gearboxes}(a):

\begin{enumerate}
    \item A lower rack (Rack 1), free to move horizontally, with position $x_1(t)$.
    \item A gear of radius $r_1$ in contact with Rack 1, rotating with angular velocity $\omega_1$.
    \item A pinion of radius $r_2$ meshed with the gear (on a parallel axis), rotating with angular velocity $\omega_2$ in the opposite direction.
    \item A large gear of radius $r_3$ rigidly attached concentrically to the pinion (mounted coaxially with the pinion and thereby sharing the same axis), rotating with angular velocity $\omega_3 = \omega_2$.
    \item An upper rack (Rack 2), in contact with the large gear, with position $x_2(t)$.
\end{enumerate}

Assume all gears are ideal (no slip, no backlash, no friction), and the gear train is contained in a rigid frame (the ``gearbox'') that is itself fixed to the ground (that is, infinite impedance).

Rack 1 drives \emph{under} Gear 1; the large compound gear drives
Rack 2 on its \emph{top} side. Let counter-clockwise rotations be
positive. Then the kinematic relations are as follows:
\begin{align}
    v_1 &= r_1 \omega_1 ,
           \label{eq:pos_v1}\\
    r_1 \omega_1 &= -r_2 \omega_2 ,
           \label{eq:pos_mesh}\\
    \omega_3 &= \omega_2 ,
           \label{eq:pos_rigid}\\
    v_2 &= -r_3 \omega_3 .
           \label{eq:pos_v2}
\end{align}
The two minus signs---one from the gear-gear mesh in Eq.~\eqref{eq:pos_mesh} and one from the top-side rack contact in Eq.~\eqref{eq:pos_v2}---cancel. Combining these equations shows that Rack 2 moves in the \emph{same} direction as Rack 1, amplified by the factor $r_3/r_2$:
\begin{equation}
   v_2 = \frac{r_3}{r_2}\,v_1 \qquad \text{(positive transmission)}.
\end{equation}
Therefore, if $r_3 > r_2$, the upper rack moves faster than the lower rack, and in the same direction.

Let $F_1$ and $F_2$ be the forces applied to Rack 1 and Rack 2, respectively. Neglecting losses, power is conserved, $F_1 v_1 = F_2 v_2$, hence
\begin{equation}
   F_2 = F_1 \frac{v_1}{v_2} = F_1 \frac{r_2}{r_3},
\end{equation}
which is the expected inverse of the velocity ratio: The force magnitude is reduced by the same factor as the velocity is increased, as expected from mechanical advantage. The signs are consistent with power flow, where input power equals output power in magnitude.

This analysis shows that the gearbox acts as a mechanical transformer, analogous to a lever or an electrical transformer. However, the entire mechanism only works if the gearbox is fixed to the ground (that is, the frame provides infinite impedance). If the frame is free to move, the entire assembly will simply translate, and no relative motion between racks is possible. This is akin to an ungrounded open electrical circuit, where no effective current can flow due to the absence of a closed path and reference potential.

\subsection{Gearbox with Negative Transmission}

Now, modify the previous setup by inserting a single idler between Gear\,1 and the pinion as depicted in Fig.~\ref{fig:gearboxes}(b). The idler reverses the direction of motion.

Assume again that all gears are ideal and that the gearbox is fixed to the ground. Let the idler have radius $r_\mathrm{i}$ and keep the same sign conventions as before. The kinematic relations are as follows:
\begin{align}
    v_1 &= r_1 \omega_1 ,
           \label{eq:neg_v1}\\
    r_1 \omega_1 &= -r_\mathrm{i} \omega_\mathrm{i} ,
           \label{eq:neg_mesh1}\\
    r_\mathrm{i} \omega_\mathrm{i} &= -r_2 \omega_2 ,
           \label{eq:neg_mesh2}\\
    \omega_3 &= \omega_2 ,
           \label{eq:neg_rigid}\\
    v_2 &= -r_3 \omega_3 .
           \label{eq:neg_v2}
\end{align}
The three minus signs---one from each external gear mesh in Eqs.~\eqref{eq:neg_mesh1} and \eqref{eq:neg_mesh2}, and one from the top-side rack contact in Eq.~\eqref{eq:neg_v2}---leave a single minus overall. Rack 2 moves in the \emph{opposite} direction to Rack 1 with a magnitude amplified by:
\begin{equation}
   v_2 = -\frac{r_3}{r_2}\,v_1 \qquad \text{(negative transmission).}
\end{equation}
With ideal gears (no loss), the input and output powers are again equal, $F_1 v_1 = F_2 v_2$, which implies:
\begin{equation}
     F_2 = F_1 \frac{v_1}{v_2} = - F_1 \frac{r_2}{r_3},
\end{equation}
confirming that the force drops by the inverse of the velocity ratio and points opposite to Rack 1's force, as required by energy conservation.

As before, if the gearbox is not fixed to the ground, the entire assembly will simply move as a whole, and no relative motion between racks will occur. The necessity of impedance (grounding) is thus not a mere technicality, but a fundamental requirement for directional energy transfer.

\subsection{Control-Theoretic Perspective: The Inerter and Beyond}

The inerter, introduced by Smith~\cite{Smith2002}, is a two-terminal mechanical device in which the force is proportional to the relative acceleration of the terminals:
\begin{equation}
    F = b (\ddot{x}_1 - \ddot{x}_2),
\end{equation}
where $b$ is the inertance. The inerter generalizes the concept of mass (which is a one-terminal device) to two-terminal networks, enabling the synthesis of mechanical circuits analogous to electrical ones.

While it may be mathematically tempting to consider devices with negative inertance, it is crucial to distinguish between passive elements and active systems. A true passive component with negative inertance is forbidden by the second law of thermodynamics. Such a device, if it existed, could theoretically amplify thermal fluctuations to generate organized work or lead to runaway instability from the smallest perturbation, representing a perpetual motion machine of the second kind~\cite{saha-2021}.

However, the \emph{behavior} of negative inertance can be emulated using an \emph{active} system. Such a system would typically consist of sensors to measure the relative acceleration, a controller, and an actuator (e.g., a linear motor) that injects energy from an external power source to generate the required counter-intuitive force. This emulation does not violate any physical laws, as the energy required to produce the ``anti-inertial'' force is explicitly supplied by the power source, and the overall system is not passive. The concept therefore remains a valuable tool for thought experiments and for understanding the theoretical limits and possibilities of mechanical control.

For instance, imagine an active system emulating an ``anti-inerter'' with negative inertance $-b$ ($b > 0$), coupled to a system with stored energy, such as a flywheel. Suppose an input force $F_{\text{in}}$ acts on one terminal, inducing an acceleration $\ddot{x}_1$. The active system would sense this and apply a force with its actuator:
\begin{equation}
    F = -b (\ddot{x}_1 - \ddot{x}_2),
\end{equation}
potentially drawing from an external power source (perhaps replenished by the flywheel's stored energy) to produce an output acceleration $\ddot{x}_2 > \ddot{x}_1$. In a wind-driven context, this heuristically models how an active control system could, in principle, achieve arbitrary speed gains. However, without proper impedance (grounding), the underlying physical system would still be subject to the conservation laws that prevent net motion from internal forces alone. The active emulation of negative mass or inertia remains a staple of theoretical control, highlighting the boundary between passive physical limitations and active engineering possibilities.

\section{The Popescu Glide}

Consider the scenario described by Popescu~\cite{popescu_talk_2024} and depicted in Fig.~\ref{fig:popescu-glide} (from memory):

\begin{enumerate}
    \item Ann is stationary, either in a fixed boat or on a dock, serving as a reference point.
    \item Bob is in a boat (or a suspended platform capable of nearly frictionless horizontal motion) of mass $M$, with his own mass being $m$.
    \item Bob's boat is initially at rest, with Bob positioned a distance $d$ from Ann.
    \item Bob attempts to approach Ann by walking from the stern to the bow of his boat, covering a distance $l$ toward her direction.
\end{enumerate}

\begin{figure}[htbp]
    \centering
    \includegraphics[width=0.8\columnwidth]{2025-wind-f-AliceBobPopescu.jpg}
    \caption{Schematic illustration of the Popescu glide thought experiment. Ann is stationary, while Bob attempts to approach by moving inside his frictionless boat. The boat recoils oppositely, preventing net progress without external impedance (e.g., an anchor).}
    \label{fig:popescu-glide}
\end{figure}

\subsection{Center-of-Mass Analysis}

Let $x_B$ be the position of Bob relative to the boat, and $x$ the position of the boat relative to the water (assumed frictionless). The total center of mass (com) of the system (Bob + boat) is:
\begin{equation}
    x_\text{com} = \frac{m (x + x_B) + M x}{m + M}.
\end{equation}
Since there are no external horizontal forces (assuming no water resistance), the com remains fixed:
\begin{equation}
    \frac{d x_\text{com}}{dt} = 0.
\end{equation}
Suppose Bob moves from $x_B = 0$ (stern) to $x_B = l$ (bow). The change in the boat's position $x$ is determined by setting the initial and final com positions equal. Let the boat's initial position be $x_0$ and final position be $x_f$:
\begin{align}
    x_\text{com, initial} &= \frac{m (x_0 + 0) + M x_0}{m + M} = x_0, \\
    x_\text{com, final} &= \frac{m (x_f + l) + M x_f}{m + M} = x_f + \frac{m}{m + M} l.
\end{align}
Setting $x_\text{com, initial} = x_\text{com, final}$:
\begin{equation}
    x_0 = x_f + \frac{m}{m + M} l \implies x_f = x_0 - \frac{m}{m + M} l.
\end{equation}
Thus, as Bob moves forward by $l$, the boat moves backward by $\frac{m}{m + M} l$. The net displacement of Bob relative to the water is:
\begin{equation}
    \Delta x_\text{Bob} = (x_f + l) - x_0 = l - \frac{m}{m + M} l = \frac{M}{m + M} l.
\end{equation}
For finite masses, this net advance is always less than $l$. Bob never reaches Ann unless $M \to \infty$ (that is, the boat is infinitely massive relative to Bob), which requires coupling to an external body of much larger mass. In that limit, recoil vanishes, and $\Delta x_\text{Bob} \to l$. Conversely, if $M \to 0$ (a massless boat, which is also unphysical), the net displacement $\Delta x_\text{Bob} \to 0$, as the boat recoils fully, resulting in no progress toward Ann. Even with repeated steps, each yielding a fractional advance, Bob can only asymptotically ``Zeno-like'' approach Ann but never fully close an arbitrary distance $d$ in finite steps without external impedance, as the fixed center of mass limits total progress.

\subsection{Implications and the Need for Impedance}

This analysis demonstrates that, in the absence of external resistance (impedance), internal motion cannot produce net translation of the system. To achieve net motion toward Ann, Bob's boat must be anchored (e.g., by dropping an anchor or by water resistance). This is a direct mechanical analog of the need for grounding in the lever and gearbox examples.

The Popescu glide is a vivid illustration of ``M\"unchhausen bootstrapping'': Without external impedance, internal rearrangements cannot produce (center of mass) net motion. This is a fundamental consequence of the conservation of momentum and the absence of external forces. In the context of wind-driven vehicles, the ``grounding'' is provided by friction with the ground (for land yachts) or water resistance (for sea yachts).

\section{Summary and Broader Implications}

We have explored three distinct conceptual ``images''---the lever, the gearbox, and the Popescu glide---to build an intuitive understanding of how wind-powered vehicles can travel faster than the wind. Each analogy highlights the same critical principle from a different angle.

The lever shows how a fixed pivot point allows for the amplification of velocity (traded for force). The gearbox formalizes this into a system of transmission, again emphasizing that the casing must be held firm. The Popescu glide offers a narrative illustration of why movement relative to an external medium is indispensable for propulsion.

These analogies map directly onto the real-world performance of land and sea yachts. The wheels of the Blackbird land yacht or the keel/centerboard of a sailboat provide the crucial ``grounding'' or ``impedance.'' They anchor the vehicle to a reference frame (the Earth) that is stationary relative to the moving medium (the air). The wind, interacting with a sail or a propeller, acts as one input to the system. The ground, interacting with the wheels or keel, acts as the other.

It is the velocity difference between the air and the ground that the vehicle taps into. The vehicle's machinery---be it a simple sail or a complex geared propeller---is nothing more than a sophisticated lever or gearbox. It translates the low-speed, high-force input of the wind into a high-speed, low-force output (or vice versa), but it can only do so by bracing against the immovable impedance of the ground. Without this grounding, there is no leverage, no transmission, and no ``impossible'' speed. Impedance is, indeed, what you need.

\begin{acknowledgments}
The author thanks Sandu Popescu for a communication regarding his inspiring boat analogy and acknowledges Open Access support from TU Wien.

This text was partially created and revised with the following Large Language Model chatbots: Grok4-0709, Gemini 2.5 Pro, o3-2025-04-16. Prompts provided by the author.
\end{acknowledgments}

\bibliography{svozil}

\end{document}




Please produce a Revtex manuscript file (no drawings, even if I say so, but placeholder text) --- and feel free to add your thoughts and criticism to a scientific article:

* entitled  ``Impedance is what you need''

* Karl Svozil, TUWIEN, ...

* Abstract: short abstract

** Introduction section:

pointing out (with references) that the Counterintuitive Performance of Land and Sea Yachts is known since at least the 14th century \cite{McDonald2021},
and that recent debates have been erupted after the demo runs of the blackbird.

Whats lacking is an heuristic yet intuitive understanding of the phenomenon to go wind-driven ``arbitrary fast'' in both directions---with the wind and against its direction.
... In the sense of Hertz's ``images of nature'', as developed in the introoduction of mechanics.

Give a brief summary of the following sections here.

** Section introducing levers:

The first image or metaphor is that of a lever:

*** subsection with the wind:

Describe and draw
+ a non-elastic ``(very long) lever'' or rod,
+ with the pivot point at one (left) end, and attached to it
+ a sail in the middle or some lower portion of the rod,
+ and an observation deck witth a  sensor on the other end of the rod.

Initially its stannds upright, vertically, in the direction perpendicular to the wind.

If the wind blows, the lever is tilted around the pivot point at, say, a fraction of the wind speed.

If the lever is long enought (calculate) this fractional wind speed at the sail will be accelerated to arbitrary but higher speeds toward the top of the lever; in particular, the observation deck with the sensor on the other end of the lever can move  faster than the wind.

Of course, there is a tradeoff between the wind forces on the sail and the force the observation deck on topis pushed---calculate.


*** subsection against the wind:

Describe and draw the same, but now:

+ the pivot is somewhere in-between the two ernds of the rod
+ the sail is now on the opposite (left) end of the rod, opposite to the observation desk.

Compute the relative (negative relative) velocities in this case.

At some point, as the observation deck in this going-against-the--wind regime, experiences a  wind that is negative relative  to the original wind, this might cause a limit to the motion of the observation desk at the top of the lever (calculate or estimmate, if possible).

Emphasize that the pivot point and its ``grounding'' is essential: without it, no ``resistance'' or rather ``impedance'' against the wind would take klace, an therefore  the lever would be no lever, but just a rod subjected uniformly to the wind; no tilting would take place.



*** Section translating this into a gear box for control theory:

Mention the idea of the inerter  in IEEE TRANSACTIONS ON AUTOMATIC CONTROL, VOL. 47, NO. 10, OCTOBER 2002
Synthesis of Mechanical Networks: The Inerter
Malcolm C. Smith, Fellow, IEEE, add DOI and bib ref)
Mention that it would be funny to realize an ``anti-inverter'' with ``negative capacitance/resistance'' thus of ``negative inertia and mass''.

** subsection describing a gear box with positive transmission:

+ one horizontal rack capable of forward motion (indicated by an arrow)

+ one gear wheel, running on top of this rack, and connected to it, indicating the rotation direction through a round arrow

+ one pinion wheel, running on top of this gear wheel, and connected to it, indicating the (opposite) rotation direction through a round arrow

+ concentric to this pinion, and fixed connected to it (but in a way that it does not block the previous gear wheel, a very large gear wheel, sunning at the same angular velocity as the pinion, as indicated by a rotation arrow

+ one horizontal rack on top of this very large gear wheel, capable of forward motion (indicated by an arrow)

+ also draw the axes (centers of rotations)

+ all of this gadgetry should be contained in a gear box with two output terminals: one on the left for the first rack, and another one for the second rack on the opposite side of the box

Compute the relative transmission by the relations of radii from the gear, the pinion and the very large gear.


** subsection describing a gear box with negtive transmission:

Do the same, but,

+ on top of the very large gear, take an additional gear, and

+place the second rack on top of this gear.

+ all of this gadgetry should be contained in a gear box with two output terminals: one on the left for the first rack, and another one for the second rack on the left (same) side of the box

Compute the relative transmission by the relations of radii from the gear, the pinion and the very large gear.

Note that, without the gear box somehow ``fixated'' (eg through impedance, and without mass/inertia) to the ground, this gear box would not take translate motion.

*** section for the Popescu glide

Mention (bibtex entry) that Professor Sandru popescu has mentioned the following gonfiguration during a talk on Sep. 24, 2024:

+ imagine a boat, with Ann sitting in that boat (alternatively a dock, and Ann standing on it)

+ a (second) boat, not far away, with Bob in it, trying to get closer to Ann, by moving inside the boat (which is swimming in the water) toward Ann.

Analyse this quantitatively wrt Bob&boat's center-of-gravity: Bob's boat will glide in the opposite direction of Bob, so that Bob can never reach Ann.

What is needed here is a fixation of Bob's boat, sich as an ancor

*** in a final section,

+ recapitulate all configurations, and

+ compare this with the discussion of  Land and Sea Yachts going with the wind faster than the wind, or in opposite directions: in all instances, one needs a ``grounding'' to stabilize and ``anchor'' the movements (through eg water restistance or ground resistance of wheels). Sails are equivalent to wind turbines etc.


! Note that  the bib file is ``svozil.bib''
