\documentstyle[12pt]{article}
\begin{document}
 \title{Stronger-than-quantum correlations}
\author{G. Krenn\\
 {\small Atominstitut der \"Osterreichischen Universit\"aten}  \\
  {\small Sch\"uttelstra\ss e 115}    \\
  {\small A-1020 Vienna, Austria   }            \\
  {\small krenn@ati.ac.at}   \\
{\small and}\\
K. Svozil\\
 {\small Institut f\"ur Theoretische Physik}  \\
  {\small Technische Universit\"at Wien   }     \\
  {\small Wiedner Hauptstra\ss e 8-10/136}    \\
  {\small A-1040 Vienna, Austria   }            \\
  {\small svozil@tph.tuwien.ac.at}}
\date{}
\maketitle

\begin{abstract}
After an elementary derivation of Bell's inequality, classical,
quantum mechanical and stronger-than-quantum correlation functions
for 2-particle-systems are discussed.
In particular hypothetical stronger-than-quantum
correlation functions are investigated which give rise to an extreme
violation of Bell's inequality.
Referring to a specific quantum system it is shown that such an extreme
violation would contradict basic laws of physics.
\end{abstract}

\section{Introduction}
In 1964 Bell formulated a condition for the possibility of local
hidden variable models \cite{bell} known as Bell\rq s inequality.
The fact that quantum mechanics violates Bell\rq s inequality has caused
a great variety of experimental as well as theoretical investigations.
In this paper we will focus on the consequences of a violation of
Bell\rq s inequality stronger than permitted by quantum mechanics.

Consider
 Bell\rq s inequality in the form of the
Clauser-Horne-Shimony-Holt (CHSH) inequality \cite{chsh,clauser}
\begin{equation}
-2 \le E(\alpha ,\beta )+
E(\alpha ',\beta )+
E(\alpha ,\beta ') -
E(\alpha ',\beta ')\le 2 \quad .
\label{Bell1}
\end{equation}
$E$ is the quantum mechanical correlation function for two particle
correlations which will be explained later in detail.
For the moment it is only
necessary to know that $E$ may have values in the range of $-1$ to $+1$.
In general the function $E$ could be such that the four terms in
(\ref{Bell1}) can take on values completely independent of each other.
In such a case the maximum violation of the inequality is $4$ and occurs
for
\[
E(\alpha ,\beta )=E(\alpha ',\beta )=E(\alpha ,\beta ')=
-E(\alpha ',\beta ')=1.
\]

Now it is a well known fact that in quantum mechanics the maximum violation
of Bell\rq s inequality is $2\sqrt{2}$ \cite{cirelson:80,cirelson}.
This implies that $E$ is restricted to such functions
which prevent a stronger violation than $2\sqrt{2}$.
Because the maximum possible violation 4 is not realized in quantum
mechanics several questions arise.
Is the limit of $2\sqrt{2}$ forced by probability theory or by physics?
Is a violation larger than $2\sqrt{2}$ consistent with the
foundations of quantum mechanics, e.g. the randomness of elementary
processes? Would a stronger violation of Bell\rq s inequality destroy the
peaceful coexistence of quantum mechanics and relativity theory and enable
faster-than-light communication? Related questions have been raised
before by several authors
\cite{shimony2,shimony3,pop-rohr,rohrlich-97,grunhaus-96,popescu-97}.

Although we are not able to answer these questions in general, we will
discuss a system in which stronger-than-quantum correlations would lead to
inconsistencies with fundamental laws of physics.
For this purpose we start with a detailed discussion of classical,
quantum mechanical and stronger-than-quantum correlations.

\section{Derivation of Bell\rq s inequality}
\noindent
Let\rq s consider two correlated spin-1/2 particles or equivalent
systems like correlated polarized photons. On each one of the two particles
measurements with two possible outcomes ($+1$ and $-1$)
are performed in space-like separated regions.
On the first particle a measurement of the dichotomic
(two-valued) observable
$R_\alpha$ with the possible results $r_\alpha \in \{-1,1\}$
(e.g. the spin along a direction $\vec{\alpha}$ which is defined by the
angle $\alpha$ within the plan perpendicular to the momentum of the particle)
is made by observer A. Likewise,
the dichotomic observable $R_\beta$ with $r_\beta \in \{-1,1\}$
is measured on the second particle by experimenter B.
Then for $N$ such particle pairs a correlation function can be defined by
\begin{equation}
E(\alpha ,\beta )=\langle R_\alpha R_\beta \rangle =
\lim_{N \rightarrow \infty}{1\over N}
\sum_{i=1}^N  r_{\alpha,i} r_{\beta,i} =
\lim_{N \rightarrow \infty}
\frac{N - 2 n(\alpha,\beta)}{N}
\quad ,
\label{e-1}
\end{equation}
where $n(\alpha,\beta)$ is the number of instances in which different
results in the
measurements of $R_\alpha$ and $R_\beta$ are obtained and
$r_{\alpha,i}$ and $r_{\beta,i}$ are the results of the measurements
on the $i-{\rm th}$ particle pair. This function is $+1$ if all $N$
results of observers A and B are equal
($n(\alpha,\beta)=0$ ;
$r_{\alpha,i} r_{\beta,i} = -1-1$ or
$r_{\alpha,i} r_{\beta,i} = +1+1 , \; i=1 \dots N$) and
$-1$ if all $N$ results have different sign
($n(\alpha,\beta)=N$ ;
$r_{\alpha,i} r_{\beta,i} = -1 +1$ or
$r_{\alpha,i} r_{\beta,i} = +1 -1 , \; i=1 \dots N$).
In general this function takes on values in the range between $-1$ and $+1$.

The assumption of local hidden variables implies the existence of a
hidden classical arena.
The reader may think of a mechanism determining
the results of all measurements
observer A (B) may perform for each individual pair of
correlated particles.
In the following we consider the measurements $R_\alpha$, $R_{\alpha '}$
of observer A and $R_\beta$, $R_{\beta '}$  of observer B.
With the assumption of local hidden variables the results of both
measurements, $R_\alpha$, $R_{\alpha '}$ and $R_\beta$, $R_{\beta '}$,
respectively, are defined simultaneously for each individual pair of
correlated particles.
Consider a series of $N$ such particle pairs.
For each pair the values of
$r_\alpha$ ,
$r_{\alpha '}$
($r_\beta$ ,
$r_{\beta '}$) are determined.
Writing down these values for all $N$ particle pairs, we get four lists as
shown in Fig.\ref{f-1}. For our considerations arbitrary lists of
results can be chosen.
We will demonstrate that any results which can be listed in
such a way have to fulfill a simple condition which is equivalent to
Bell\rq s inequality. This condition imposes a restriction on the
correlation of the results and therefore on the correlation function $E$.

\begin{figure}
\begin{center}
%TexCad Options
%\grade{\off}
%\emlines{\off}
%\beziermacro{\off}
%\reduce{\on}
%\snapping{\off}
%\quality{4.00}
%\graddiff{0.01}
%\snapasp{1}
%\zoom{1.00}
\unitlength 1.00mm
\linethickness{0.4pt}
\begin{picture}(136.00,90.00)
\put(20.00,10.00){\framebox(80.00,10.00)[lc]{$\,-+-+-++++-++++-+\;
\cdots \; +$}}
\put(20.00,30.00){\framebox(80.00,10.00)[lc]{$\,++-++++-+-++-+-+\;
\cdots \; -$}}
\put(20.00,50.00){\framebox(80.00,10.00)[lc]{$\,-+-++-+-+--+-+--\;
\cdots \; -$}}
\put(10.00,15.00){\makebox(0,0)[cc]{$\alpha '$}}
\put(10.00,35.00){\makebox(0,0)[cc]{$\beta $}}
\put(10.00,55.00){\makebox(0,0)[cc]{$\alpha $}}
\put(10.00,75.00){\makebox(0,0)[cc]{$\beta '$}}
%\emline(105.00,12.00)(130.00,12.00)
\put(105.00,12.00){\line(1,0){25.00}}
%\end
%\emline(130.00,12.00)(130.00,78.00)
\put(130.00,12.00){\line(0,1){66.00}}
%\end
%\emline(105.00,38.00)(108.00,38.00)
\put(105.00,38.00){\line(1,0){3.00}}
%\end
%\emline(108.00,38.00)(108.00,52.00)
\put(108.00,38.00){\line(0,1){14.00}}
%\end
%\emline(105.00,18.00)(108.00,18.00)
\put(105.00,18.00){\line(1,0){3.00}}
%\end
%\emline(108.00,18.00)(108.00,32.00)
\put(108.00,18.00){\line(0,1){14.00}}
%\end
%\emline(105.00,58.00)(108.00,58.00)
\put(105.00,58.00){\line(1,0){3.00}}
%\end
%\emline(108.00,58.00)(108.00,72.00)
\put(108.00,58.00){\line(0,1){14.00}}
%\end
\put(113.00,25.00){\makebox(0,0)[lc]{$n(\alpha ',\beta )$}}
\put(113.00,45.00){\makebox(0,0)[lc]{$n(\alpha ,\beta )$}}
\put(113.00,65.00){\makebox(0,0)[lc]{$n(\alpha ,\beta ')$}}
\put(136.00,45.00){\makebox(0,0)[lc]{$n(\alpha ',\beta ')$}}
\put(20.00,90.00){\makebox(0,0)[lc]{Particle pair}}
\put(20.00,85.00){\makebox(0,0)[lc]{$\;\,1\;\,2\;\,3\;\:4\;\:5\;\:6\;\%%@
:7
\cdots$}}
\put(100.00,85.00){\makebox(0,0)[rc]{$\cdots \;\, N$}}
\put(20.00,70.00){\framebox(80.00,10.00)[cc]{}}
\put(20.00,75.00){\makebox(0,0)[lc]{$\,-+--+++-+--+-+--\; \cdots \; +$}}
\put(130.00,78.00){\vector(-1,0){25.00}}
\put(108.00,72.00){\vector(-1,0){3.00}}
\put(108.00,52.00){\vector(-1,0){3.00}}
\put(108.00,32.00){\vector(-1,0){3.00}}
\end{picture}
\end{center}
\caption{
For $N$ pairs of correlated particles the results of measurements which may
be performed by observer A ($R_\alpha$, $R_{\alpha '}$) and B
($R_\beta$, $R_{\beta '}$) are shown
(``$+$'' stands for $+1$,
``$-$'' stands for $-1$).
As expressed by Eq.(\ref{e-1}) the correlation
function $E(\alpha,\beta)$ is given by the number of different results in
lists $\alpha$ and $\beta$ $n(\alpha,\beta)$.
In such a way the correlation of the
results in lists $\alpha '$ and $\beta '$ is defined by
$n(\alpha ',\beta ')$ (``outer path'').
At the same time a limit on the number $n(\alpha ',\beta ')$ is imposed
by the values of $n(\alpha,\beta)$, $n(\alpha ',\beta)$ and
$n(\alpha,\beta ')$
($E(\alpha,\beta)$, $E(\alpha ',\beta)$ and $E(\alpha,\beta ')$)
(``inner path'').
Only in case of local realistic results the value of $n(\alpha ',\beta ')$
is within this limit.
Then the results of {\em all four} measurements can be defined
simultaneously in agreement with $E(\alpha,\beta)$ and consequently written
down as shown in this picture.
\label{f-1}}
\end{figure}

To find out the restriction for the correlation function $E(\alpha,\beta)$,
we determine
the number of different signs (results) in the four pairs of lists
$(\alpha ',\beta )$, $(\alpha ,\beta )$, $(\alpha ,\beta ')$ and
$(\alpha ',\beta ')$.
As expressed by Eq.(\ref{e-1})
for $N$ particle pairs the correlation function $E(\alpha,\beta)$ is
given by the number of cases $n(\alpha, \beta)$ in which different results
are obtained in the
measurements of $R_{\alpha}$ and $R_{\beta}$.
Having determined the four values
$n(\alpha ',\beta )$,
$n(\alpha ,\beta )$,
$n(\alpha ,\beta ')$ and
$n(\alpha ',\beta ')$
(cf. Fig. \ref{f-1}), we make a simple observation \cite{krenn}.

A limit on the number $n(\alpha ',\beta ')$
(``outer path'' in Fig. \ref{f-1})
and thus on the correlation function $E(\alpha ',\beta ')$
(cf.Eq.(\ref{e-1}))
is imposed by the
values of $n(\alpha ',\beta )$,
$n(\alpha ,\beta )$ and $n(\alpha ,\beta ')$.
Along the ``inner path''
$\alpha '
\rightarrow
\beta
\rightarrow
\alpha
\rightarrow
\beta  '$
from list $\alpha '$ to list $\beta'$ in Fig. \ref{f-1} we have to
change
$n(\alpha ',\beta )$ signs in the first step to get list $\beta$,
$n(\alpha ,\beta )$ signs in the second step to get list $\alpha$,
and $n(\alpha ,\beta ')$ signs in the last step to obtain list
$\beta '$.
At the end of this procedure the number of different signs in lists
$\alpha '$ and $\beta '$ $n(\alpha ',\beta ')$ can be no greater than
$n(\alpha ',\beta )+
n(\alpha ,\beta )+
n(\alpha ,\beta ')$\footnote{Without loss of generality we have assumed that
$n(\alpha ',\beta )+
n(\alpha ,\beta )+
n(\alpha ,\beta ') \le N$}.
This can be expressed by the inequality
\begin{equation}
n(\alpha ',\beta )+
n(\alpha ,\beta )+
n(\alpha ,\beta ') \ge
n(\alpha ',\beta ')\quad .
\label{e-2}
\end{equation}

The probability $P^{\neq} (\alpha , \beta )$ for different signs (results)
in measurements of $R_{\alpha}$ and $R_{\beta}$ on
$N$ particle pairs can be approximated by the relative frequency
$n(\alpha ,\beta)/N$.
Analogously, the probability for equal signs $P^=(\alpha , \beta )$
is approximately given by $1 - n(\alpha ,\beta)/N$.
By definition (\ref{e-1}), the correlation function can be written as
\begin{equation}
E(\alpha ,\beta )=
P^=(\alpha , \beta )-P^{\neq } (\alpha ,\beta )=2P^=(\alpha ,\beta
)-1 \quad.
\label{e-1a}
\end{equation}
Using these identities, Eq. (\ref{e-2}) can easily be rewritten into
the CHSH inequality \cite{chsh} form
\begin{equation}
E(\alpha ,\beta )+
E(\alpha ',\beta )+
E(\alpha ,\beta ') -
E(\alpha ',\beta ')\le 2\quad .
\label{e-3}
\end{equation}
The bound from below
\begin{equation}
E(\alpha ,\beta )+
E(\alpha ',\beta )+
E(\alpha ,\beta ') -
E(\alpha ',\beta ')\ge -2
\label{e-31}
\end{equation}
can be derived by a similar argument, considering
the number of equal signs (results) $u(\alpha ,\beta )= N-n(\alpha,\beta )$
instead of the number of different signs (results).
$u(\alpha ,\beta)$ satisfies the same inequality
(\ref{e-2}) as $n(\alpha ,\beta )$. Bell's inequality in the form of
Eq. (\ref{Bell1}) is given by the combination of (\ref{e-3}) and
(\ref{e-31}).

We have seen that the value of the correlation function
$E(\alpha ',\beta ')$ is related to the values of
$E(\alpha ,\beta )$,
$E(\alpha ',\beta )$ and
$E(\alpha ,\beta ')$.
Only results which can be represented as shown in Fig. \ref{f-1}
and thus are defined simultaneously and locally for all four
possible experiments $R_{\alpha},R_{\alpha '}$ and $R_{\beta},R_{\beta '}$
(as by local realistic models)
fulfill this relation and therefore also Bell's inequality.

Now let us consider a system whose correlations are such
that the {\em maximum} number of sign
changes along the ``inner path'' ($n(\alpha ',\beta )+
n(\alpha ,\beta )+
n(\alpha ,\beta ')$)
is {\em smaller} than the number of sign changes
along the ``outer path'' ($n(\alpha ',\beta ')$).
Then {\em not a single} set of lists
$(\alpha, \beta, \alpha ', \beta ')$ exists,
which satisfies all the correlations as defined by
$n(\alpha ',\beta)$, $n(\alpha ,\beta )$, $n(\alpha ,\beta ')$ and
$n(\alpha ',\beta ')$ ($E(\alpha ',\beta )$, $E(\alpha ,\beta )$,
$E(\alpha ,\beta ')$ and $E(\alpha ',\beta ')$) simultaneously.
A certain fraction of the results in lists
$\alpha '$ and $\beta '$ would {\em always} be {\em inconsistent}
with the values of the correlation functions.
For the maximum violation of Bell\rq s inequality
permitted by quantum mechanics -- $2\sqrt{2}$ -- this fraction is
$(\sqrt{2}-1)100 \approx 40\%$.
For stronger-than-quantum correlations this fraction reaches $100\%$
in the limit of a violation of Bell\rq s inequality with
the maximum value 4 (cf. section 4).

\section{Classical and quantum mechanical correlations}

Bell\rq s inequality is a condition which must be fulfilled by local
realistic, i.e. classical correlation functions. Quantum mechanical
correlation functions violate
Bell\rq s inequality by a maximum value of $2\sqrt{2}$:
\[
\mid E_{qm}(\alpha ',\beta )+
E_{qm}(\alpha ,\beta )+
E_{qm}(\alpha ,\beta ') -
E_{qm}(\alpha ',\beta ')\mid \le 2\sqrt{2}  \quad .
\]
In the following we will give an example for a classical as well as a
quantum mechanical correlation function.

First of all we consider pairs of correlated classical particles with total
angular momentum zero. $\vec{j_1}$ and $\vec{j_2}$ are the classical angular
momenta of particle 1 and 2, respectively. Then, by measuring the angular
momentum of particle 1 (2) along a direction $\vec{\alpha}$
($\vec{\beta}$) defined by the angle $\alpha$ ($\beta$) within the plane
perpendicular to the momentum of the particles
the classical observable $R_\alpha ={\rm sgn} (\vec{\alpha} \cdot
\vec{j_1})$
($R_\beta ={\rm sgn} (\vec{\beta} \cdot \vec{j_2})$) can be defined.
It can be shown \cite[Eq. 10]{bell}
(see also \cite{peres222,peres}) that for such observables the classical
correlation function is
given by
\begin{equation}
E_{c}(\alpha,\beta)=E_{c}(\theta)=
\frac{2 \theta}{\pi} -1  \quad ,
\label{e-cef}
\end{equation}
where $\theta$ is the relative angle $\vert \alpha - \beta \vert$.
By comparing this function with Eq. (\ref{e-1a}) we find that
\begin{equation}
P^=(\theta )= \frac{\theta}{\pi} \quad .
\nonumber
\end{equation}
This corresponds to the expectation that the probability for
equal results in measurements of $R_\alpha$ and $R_\beta$
($P^=(\theta)$) is proportional to the relative angle
$\theta$. By inserting (\ref{e-cef}) into (\ref{e-3}) one can easily see
that Bell\rq s inequality is not violated, which also implies that
condition (\ref{e-2}) is fulfilled.

To derive a quantum mechanical correlation function we now consider
two particles of spin $j$ in a singlet state. Then the correlation
function is given by (cf. appendix and ref. \cite{gisin-peres})
\begin{equation}
C(\theta )=
- {j(j+1)\over 3} \, \cos \theta \quad .
\end{equation}
Again $\theta$ is the relative angle $\vert \alpha - \beta \vert$ of two
angles within the plane perpendicular to the momentum of the particles.
To be comparable to the classical correlation function,
the quantum correlation function must be normalized
such that $E_{qm}(\pi )=-E_{qm}(0)=1$
($E_{qm}(\theta ) = {3/[j(j+1)]} C(\theta )$).
Thus for two correlated spin-$\frac{1}{2}$ particles in a singlet state the
quantum mechanical correlation function is given by
\begin{equation}
E_{qm}(\alpha,\beta)=-\vec{\alpha} \cdot \vec{\beta} %%@
=E_{qm}(\theta) =
-\cos \theta \quad ,
\label{e-qm}
\end{equation}
where the vectors $\vec{\alpha}$ and $\vec{\beta}$ are defined by the angles
$\alpha$ and $\beta$ within the plane perpendicular to the momentum of the
particles.

$E_{c}(\theta)$ and $E_{qm}(\theta)$ are drawn in Fig.\ref{f:99}.
One can see
that for almost all angles $\theta$,
the quantum mechanical correlations are {\em stronger} than the
classical ones. Therefore $E_{qm}$ violates Bell\rq s inequality but the
violation does not exceed $2\sqrt{2}$ as one can proof by inserting
(\ref{e-qm}) into (\ref{e-3}). Results described by a
quantum mechanical correlation function $E_{qm}$ can in general not be
represented consistently by local realistic models.
As demonstrated for the angles $\alpha$, $\alpha '$ and
$\beta$, $\beta '$ the results of the measurements $R_{\alpha '}$ and
$R_{\beta '}$ can not be defined in such a way as to correspond to
$E_{qm}(\alpha ',\beta ')$  as well as to $E_{qm}(\alpha,\beta)$,
$E_{qm}(\alpha ',\beta)$ and $E_{qm}(\alpha,\beta ')$.


\section{Stronger-than-quantum correlations}

We now turn our attention to ---merely hypothetical---
``extremely nonclassical correlations''
and assume a
{\em stronger-than-quantum} correlation function of the form
\begin{equation}
E_s(\alpha,\beta) = E_s(\theta) = {\rm sgn} (2\theta / \pi -1) =
{\rm sgn} (E_c(\theta ))
\quad ,
\label{e:7}
\end{equation}
where $E_c(\theta)$ is the {\em classical} correlation function
(\ref{e-cef}).
$E_s(\theta )$, along with $E_c(\theta )$ and $E_{qm}(\theta )$,
is drawn in Fig. \ref{f:99}. One can clearly see that $E_s(\theta)$ takes
the tendency of the quantum correlation function to exceed classical
correlations to an extreme.
This is also expressed by the fact that,
since for $x = 2\theta / \pi -1$ and $0 \leq \theta \leq \pi $
\begin{eqnarray}
{\rm sgn} (x)
&=&\left\{ \begin{array}{rl}
-1 \; &{\rm  for}\; x<0 \\
 0 \; &{\rm  for}\; x=0 \\
+1 \; &{\rm  for}\; 0< x
   \end{array} \right. \nonumber \\
&=&{4\over \pi }\sum_{n=0}^\infty {\sin [
(2n+1)x]\over
(2n+1)}
\nonumber
\\
&=&{4\over \pi }\sum_{n=0}^\infty (-1)^n{\cos [
(2n+1)(x-\pi /2)]\over
(2n+1)}\quad ,
\label{e:10}
\end{eqnarray}
the quantum mechanical correlation function can be attributed to the first
summation term in Eq. (\ref{e:10}). By considering also
terms of higher order in expansion (\ref{e:10}) we get correlations which
are stronger than the quantum correlations. Then Bell\rq s inequality is
violated by a larger value than $2\sqrt{2}$.

The extreme correlation expressed by $E_s(\theta)$ implies that for angles
$\alpha , \beta$ with $\pi /2 \leq \vert\alpha - \beta\vert \leq \pi$,
the results of observers A and B are
perfectly correlated ($E_s(\theta) = 1 ,\;
r_{\alpha,i}\,r_{\beta,i} = +\,+ {\rm or} -\,- ,\;
i=1 \dots N$),
whereas they are perfectly
anticorrelated ($ E_s(\theta) = -1 ,\;
r_{\alpha,i}\,r_{\beta,i} = +\,- {\rm or} -\,+ ,\;
i=1 \dots N$) for
angles $\alpha , \beta$ with $0 \leq \mid\alpha - \beta\mid \leq \pi /2$.
This cannot be accommodated by any classical theory under the assumption of
local realism, nor can we think of any quantum correlation satisfying it.

\begin{figure}
\begin{center}
%TexCad Options
%\grade{\off}
%\emlines{\off}
%\beziermacro{\off}
%\reduce{\on}
%\snapping{\off}
%\quality{4.00}
%\graddiff{0.01}
%\snapasp{1}
%\zoom{1.00}
\unitlength 1.00mm
\linethickness{0.4pt}
\begin{picture}(102.00,102.00)
%\emline(10.00,10.00)(10.00,100.00)
\put(10.00,10.00){\line(0,1){90.00}}
%\end
\put(10.00,55.00){\line(1,0){45.00}}
\put(55.00,55.00){\line(1,0){45.00}}
\put(10.00,10.00){\line(1,1){90.00}}
\put(100.00,100.00){\line(-1,0){45.00}}
\put(10.00,10.00){\line(1,0){45.00}}
%\bezier{284}(10.00,10.00)(30.00,10.00)(55.00,55.00)
\put(10.00,10.00){\line(1,0){1.41}}
\multiput(11.41,10.06)(0.71,0.08){2}{\line(1,0){0.71}}
\multiput(12.84,10.22)(0.48,0.09){3}{\line(1,0){0.48}}
\multiput(14.28,10.50)(0.36,0.10){4}{\line(1,0){0.36}}
\multiput(15.73,10.89)(0.29,0.10){5}{\line(1,0){0.29}}
\multiput(17.20,11.39)(0.25,0.10){6}{\line(1,0){0.25}}
\multiput(18.67,12.01)(0.21,0.10){7}{\line(1,0){0.21}}
\multiput(20.16,12.73)(0.21,0.12){7}{\line(1,0){0.21}}
\multiput(21.66,13.57)(0.19,0.12){8}{\line(1,0){0.19}}
\multiput(23.18,14.52)(0.17,0.12){9}{\line(1,0){0.17}}
\multiput(24.70,15.58)(0.15,0.12){10}{\line(1,0){0.15}}
\multiput(26.24,16.75)(0.14,0.12){11}{\line(1,0){0.14}}
\multiput(27.79,18.03)(0.13,0.12){12}{\line(1,0){0.13}}
\multiput(29.36,19.43)(0.12,0.12){13}{\line(1,0){0.12}}
\multiput(30.93,20.94)(0.11,0.12){14}{\line(0,1){0.12}}
\multiput(32.52,22.55)(0.11,0.12){14}{\line(0,1){0.12}}
\multiput(34.12,24.28)(0.12,0.13){14}{\line(0,1){0.13}}
\multiput(35.74,26.12)(0.12,0.14){14}{\line(0,1){0.14}}
\multiput(37.36,28.08)(0.12,0.15){14}{\line(0,1){0.15}}
\multiput(39.00,30.14)(0.12,0.16){14}{\line(0,1){0.16}}
\multiput(40.65,32.32)(0.12,0.16){14}{\line(0,1){0.16}}
\multiput(42.31,34.60)(0.12,0.17){14}{\line(0,1){0.17}}
\multiput(43.99,37.00)(0.11,0.17){15}{\line(0,1){0.17}}
\multiput(45.67,39.51)(0.11,0.17){15}{\line(0,1){0.17}}
\multiput(47.37,42.14)(0.11,0.18){15}{\line(0,1){0.18}}
\multiput(49.09,44.87)(0.11,0.19){15}{\line(0,1){0.19}}
\multiput(50.81,47.72)(0.12,0.20){15}{\line(0,1){0.20}}
\multiput(52.55,50.67)(0.12,0.21){21}{\line(0,1){0.21}}
%\end
%\bezier{284}(55.00,55.00)(80.00,100.00)(100.00,100.00)
\multiput(55.00,55.00)(0.12,0.21){15}{\line(0,1){0.21}}
\multiput(56.75,58.11)(0.12,0.20){15}{\line(0,1){0.20}}
\multiput(58.50,61.11)(0.12,0.19){15}{\line(0,1){0.19}}
\multiput(60.23,64.00)(0.11,0.19){15}{\line(0,1){0.19}}
\multiput(61.94,66.78)(0.11,0.18){15}{\line(0,1){0.18}}
\multiput(63.65,69.45)(0.11,0.17){15}{\line(0,1){0.17}}
\multiput(65.34,72.01)(0.12,0.17){14}{\line(0,1){0.17}}
\multiput(67.02,74.45)(0.12,0.17){14}{\line(0,1){0.17}}
\multiput(68.69,76.78)(0.12,0.16){14}{\line(0,1){0.16}}
\multiput(70.34,79.00)(0.12,0.15){14}{\line(0,1){0.15}}
\multiput(71.99,81.11)(0.12,0.14){14}{\line(0,1){0.14}}
\multiput(73.62,83.11)(0.12,0.13){14}{\line(0,1){0.13}}
\multiput(75.23,84.99)(0.11,0.13){14}{\line(0,1){0.13}}
\multiput(76.84,86.77)(0.11,0.12){14}{\line(0,1){0.12}}
\multiput(78.43,88.43)(0.12,0.12){13}{\line(1,0){0.12}}
\multiput(80.01,89.98)(0.13,0.12){12}{\line(1,0){0.13}}
\multiput(81.58,91.42)(0.13,0.11){12}{\line(1,0){0.13}}
\multiput(83.14,92.75)(0.14,0.11){11}{\line(1,0){0.14}}
\multiput(84.68,93.97)(0.15,0.11){10}{\line(1,0){0.15}}
\multiput(86.21,95.07)(0.17,0.11){9}{\line(1,0){0.17}}
\multiput(87.73,96.06)(0.19,0.11){8}{\line(1,0){0.19}}
\multiput(89.24,96.94)(0.21,0.11){7}{\line(1,0){0.21}}
\multiput(90.73,97.71)(0.25,0.11){6}{\line(1,0){0.25}}
\multiput(92.21,98.37)(0.29,0.11){5}{\line(1,0){0.29}}
\multiput(93.68,98.92)(0.36,0.11){4}{\line(1,0){0.36}}
\multiput(95.14,99.36)(0.48,0.11){3}{\line(1,0){0.48}}
\multiput(96.58,99.68)(0.72,0.11){2}{\line(1,0){0.72}}
\put(98.02,99.89){\line(1,0){1.98}}
%\end
\put(5.00,100.00){\makebox(0,0)[cc]{$+1$}}
\put(5.00,10.00){\makebox(0,0)[cc]{$-1$}}
\put(5.00,55.00){\makebox(0,0)[cc]{$0$}}
\put(58.00,50.00){\makebox(0,0)[cc]{$\pi /2$}}
\put(100.00,50.00){\makebox(0,0)[cc]{$\pi$}}
\put(102.00,59.00){\makebox(0,0)[cc]{$\theta$}}
\put(14.00,102.00){\makebox(0,0)[cc]{$E$}}
\put(30.00,38.00){\makebox(0,0)[cc]{$E_c(\theta )$}}
\put(46.00,28.00){\makebox(0,0)[cc]{$E_{qm}(\theta )$}}
\put(35.00,13.00){\makebox(0,0)[cc]{$E_s(\theta )$}}
\put(55.00,55.00){\circle*{2.00}}
\end{picture}
\end{center}
\caption{$E_c(\theta )$, $E_{qm}(\theta )$ and $E_s(\theta )$.
\label{f:99}}
\end{figure}

The hypothetical correlation function $E_s(\theta )$ gives rise to a
maximum violation of Bell's inequality, since for the four angles
$\alpha = \pi $,
$\alpha '= 6 \pi / 8$,
$\beta = \pi / 8$ and
$\beta '= 3   \pi / 8$
\[
E_s(\alpha , \beta )+ E_s(\alpha ', \beta ) +E_s(\alpha ,\beta ')-
E_s(\alpha ',\beta ') = 4.
\]
A violation of Bell's inequality by the maximum value of $4$
has also been studied by Popescu and Rohrlich \cite{pop-rohr} and, for a
classical system, by Aerts  \cite{aerts:82}.
As already mentioned in the introduction it has been shown that
the maximum violation of Bell's inequality permitted by quantum mechanics
is $2\sqrt{2}$ \cite{cirelson:80,cirelson}.

For the angles $\alpha = \pi $,
$\alpha '= 6 \pi / 8$,
$\beta = \pi / 8$ and
$\beta '= 3   \pi / 8$
we now try to write down results which are correlated as defined by
$E_s(\alpha,\beta)$
in the same way as shown in Fig.\ref{f-1} .
Because
$n(\alpha ',\beta )=
n(\alpha ,\beta )=
n(\alpha ,\beta ')=0$ ($E_s(\alpha , \beta ) = E_s(\alpha ', \beta ) =
E_s(\alpha ,\beta ') = 1$)
the results in lists $\alpha '$ and $\beta '$ have to
be identical.
This demand is satisfied by the list $\beta_{\rm in} '$ in Fig.\ref{f-8}.
At the same time these results have to be sign-reversed
because
$n(\alpha ',\beta ')=N$ ($E_s(\alpha ',\beta ')=-1$),
which is expressed by the list $\beta_{\rm out} '$.

In contrast to the classical case (Fig.\ref{f-1}) it\rq s now no longer
possible to find four lists of results which satisfy the correlations as
described by $E_s(\alpha , \beta )$ (\ref{e:7}). Therefore two different
lists $\beta '$ ($\beta_{\rm in} '$, $\beta_{\rm out} '$) are shown in
Fig. \ref{f-8}. Of course the fraction of different results in these two
lists may vary depending on the function $E$.
A comparison of the correlation functions discussed in this paper
($E_c$, $E_{qm}$ and $E_s$) is given in table \ref{tab-1}.
For $E_s$ the fraction of different results in lists
$\beta_{\rm in} '$
and $\beta_{\rm out} '$ is $100 \%$ (cf. Fig.\ref{f-8}).
For classical correlation functions ($E_c$) this fraction is
$0 \%$ ($\beta_{\rm in} ' = \beta_{\rm out} ' = \beta '$)
and for quantum mechanical correlation functions ($E_{qm}$) it
is smaller than $(\sqrt{2}-1) 100 \approx 41.42 \%$.
Whereas $E_{qm}$ contradicts local-realistic models only
on a statistical level,
$E_s$ leads to a complete contradiction.
This means that out of all $N$ particle pairs there is {\em not a single}
one to which a consistent quadruple of outcomes ($r_\alpha$,$r_{\alpha '}$,
$r_\beta$ and $r_{\beta '}$)
can be assigned. Consequently a violation of Bell\rq s inequality by
the maximum value of 4 would be a
two-particle analogue to the GHZ argument \cite{ghz}.

\begin{figure}
\begin{center}
%TexCad Options
%\grade{\off}
%\emlines{\off}
%\beziermacro{\off}
%\reduce{\on}
%\snapping{\off}
%\quality{2.00}
%\graddiff{0.01}
%\snapasp{1}
%\zoom{1.00}
\unitlength 1.00mm
\linethickness{0.4pt}
\begin{picture}(125.00,90.00)
\put(10.00,90.00){\makebox(0,0)[lc]{Particle pair}}
\put(10.00,85.00){\makebox(0,0)[lc]{$\;\,1\;\,2\;\,3\;\:4\;\:5\;\:6\;\%%@
:7
\cdots$}}
\put(90.00,85.00){\makebox(0,0)[rc]{$\cdots \;\, N$}}
\put(10.00,10.00){\framebox(80.00,10.00)[lc]{$\,++-++++-+-++-++- \;
\cdots -$}}
\put(10.00,30.00){\framebox(80.00,10.00)[lc]{$\,++-++++-+-++-++- \;
\cdots -$}}
\put(10.00,50.00){\framebox(80.00,10.00)[lc]{$\,++-++++-+-++-++- \;
\cdots -$}}
\put(0.00,15.00){\makebox(0,0)[cc]{$\alpha '$}}
\put(0.00,35.00){\makebox(0,0)[cc]{$\beta $}}
\put(0.00,55.00){\makebox(0,0)[cc]{$\alpha $}}
\put(0.00,75.00){\makebox(0,0)[cc]{$\beta '$}}
%\emline(95.00,12.00)(120.00,12.00)
\put(95.00,12.00){\line(1,0){25.00}}
%\end
%\emline(120.00,12.00)(120.00,78.00)
\put(120.00,12.00){\line(0,1){66.00}}
%\end
%\emline(95.00,38.00)(98.00,38.00)
\put(95.00,38.00){\line(1,0){3.00}}
%\end
%\emline(98.00,38.00)(98.00,52.00)
\put(98.00,38.00){\line(0,1){14.00}}
%\end
%\emline(95.00,18.00)(98.00,18.00)
\put(95.00,18.00){\line(1,0){3.00}}
%\end
%\emline(98.00,18.00)(98.00,32.00)
\put(98.00,18.00){\line(0,1){14.00}}
%\end
%\emline(95.00,58.00)(98.00,58.00)
\put(95.00,58.00){\line(1,0){3.00}}
%\end
%\emline(98.00,58.00)(98.00,72.00)
\put(98.00,58.00){\line(0,1){14.00}}
%\end
\put(99.00,25.00){\makebox(0,0)[lc]{$n(\alpha ',\beta )=0$}}
\put(99.00,45.00){\makebox(0,0)[lc]{$n(\alpha ,\beta )=0$}}
\put(99.00,65.00){\makebox(0,0)[lc]{$n(\alpha ,\beta ')=0$}}
\put(125.00,44.67){\makebox(0,0)[lc]{$n(\alpha ',\beta ')=N$}}
\put(10.00,70.00){\framebox(80.00,10.00)[cc]{}}
\put(10.00,77.00){\makebox(0,0)[lc]{$\,--+----+-+--+--+ \; \cdots +$}}
\put(10.00,73.00){\makebox(0,0)[lc]{$\,++-++++-+-++-++- \; \cdots -$}}
\put(3.00,78.00){\makebox(0,0)[lc]{$\beta_{\rm out} '$}}
\put(3.00,72.00){\makebox(0,0)[lc]{$\beta_{\rm in} '$}}
\put(120.00,78.00){\vector(-1,0){25.00}}
\put(98.00,72.00){\vector(-1,0){3.00}}
\put(98.00,52.00){\vector(-1,0){3.00}}
\put(98.00,32.00){\vector(-1,0){3.00}}
\end{picture}
\end{center}
\caption{
For the angles $\alpha = \pi $, $\alpha '= 6 \pi / 8$,
$\beta = \pi / 8$ and
$\beta '= 3   \pi / 8$ results are shown which are correlated in a way
defined by $E_s(\alpha,\beta)$ (\ref{e:7}). Again, ``$+$'' stands for $+1$
and ``$-$'' for $-1$.
The correlation of the results in lists $\alpha '$ and
$\beta '$ as defined by the ``inner path''
($n(\alpha ',\beta ) = n(\alpha ,\beta ) =
n(\alpha ,\beta ')=0 $,
$E_s(\alpha , \beta ) = E_s(\alpha ', \beta ) =
E_s(\alpha ,\beta ') = 1$, i.e.
no sign changes) is completely inconsistent
with the correlation of the same results as defined by the
``outer path'' ($n(\alpha ',\beta ') = N$,
$E_s(\alpha ',\beta ')=-1$, i.e. $N$ sign changes).
Therefore the two lists $\beta_{\rm in} '$ and $\beta_{\rm out} '$ are
completely sign-reversed.
For correlation functions $E$ which violate Bell\rq s inequality
the fraction of different results in the two lists $\beta_{\rm in} '$ and
$\beta_{\rm out} '$ is given by the extent of the violation and reaches
$100 \%$ for the hypothetical correlation function $E_s$ as shown in this
figure. Such an extreme correlation would be a two-particle analogue to
the GHZ-argument.
}
\label{f-8}
\end{figure}


\begin {table}[ht]
\begin {center}
\begin {tabular}{|c|c|c|c|}
\hline\hline
& c & qm & s \\
\hline
\hline
$P^=(\theta ) =2P^{++}(\theta )=2P^{--} (\theta )$&  $\theta /
\pi$ &$\sin^2(\theta /
2)$&$H(2\theta / \pi -1)$\\
$P^{\neq} (\theta )=2P^{+-}(\theta )=2P^{-+} (\theta )$&   $1-\theta /
\pi$ &$\cos^2(\theta /
2)$&$H(1-2\theta / \pi)$\\
\hline
\hline
$E(\theta ) = P^=(\theta ) -P^{\neq} (\theta )  $& $2\theta / \pi -1$
&$-\cos({\theta })$&${\mbox sgn}(2\theta / \pi -1 )$\\
\hline
\hline
\end {tabular}
\end {center}
\caption{Table of classical (c), quantum mechanical (qm) and
stronger-than-quantum (s) probabilities and correlation functions. H is the
Heaviside-function.
\label {tab-1}}
\end {table}

\section{Discussion}
We have seen that in case of an extreme violation of Bell's inequality
with the value 4 the results
of observers A and B are either perfectly correlated ($E_s(\theta)=1$)
(\ref{e:7}) or perfectly anticorrelated ($E_s(\theta)= -1$),
depending on the
relative angle $\theta = \vert \alpha - \beta \vert$.
If the angle $\beta$ is fixed, observer A may ``switch'' between perfect
correlation and perfect anticorrelation by changing the angle $\alpha$
adequately. One might think that in such a way superluminal signals can
be sent from observer A to observer B.

It becomes clear that this is
impossible if one takes into account that the
outcomes of the single measurements on either side cannot be controlled
and occur at random.
Experimenter A recording the outcomes for particle 1 of subsequent
particle
pairs would for instance measure a random sequence
$++-+-- \cdots$,
whereas, depending on the relative angle $\theta$,
observer B, recording the outcomes for the second particle of the
respective
pairs, would measure
either the sequence
$=++-+-- \cdots$
(for $\theta > \pi/2$),
or the sequence
$=--+-++ \cdots$
(for $\theta < \pi/2$).
Since for both experimenters the sequences of outcomes
appear totally uncontrollable and at random
it is impossible to infer the value of $\theta$ on the basis of one of
those sequences alone. This expresses the impossibility of
faster-than-light communication due to the
outcome independence.
Thus, as long as one assumes
unpredictability and/or randomness of the single outcomes
(cf. \cite{pop-rohr}),
the stronger-than-quantum correlation function $E_s$ saturates
the Roy-Singh inequalities \cite{roy-singh}.
See references \cite{khalfin-85,rastall-85,summers-87} for other works
which find
maximal violation of the CHSH inequality consistent with relativity.

Whereas for 2-particle systems there seems to be no reason why
stronger-than-quantum  correlations should be inconsistent either with the
foundations of physics or with probability theory, stronger-than-quantum
correlations lead to inconsistencies in 3-particle systems, as will be
demonstrated in the following.

Consider an entangled system of 3 spin-$\frac{1}{2}$ particles.
Spin measurements are performed on the particles in space-like separated
regions by three observers. The state is such that each observer gets the
result $+1$ or $-1$ with equal probability independent of the direction
along which the spin is measured and of course also independent of the measurements %%@
performed on the other particles.
We divide the data of observers 1 and 2 into two subensembles
by a simple rule:
If observer 3 gets the result $-1$ $(+1)$, the corresponding results of
observers 1 and 2 are put into subensemble $-$ $(+)$.
In such a way two subensembles of the results of observers 1
and 2 are defined by the results of observer 3.
We can now investigate the two-particle correlations in each subensemble
applying Bell's inequality.
Although in quantum mechanics the maximum violation of Bell's inequality is $2\sqrt{2}$ %%@
\cite{cirelson:80,cirelson} we may discuss hypothetical situations in which stronger %%@
correlations occur. Especially we are interested in the question, if the results within the %%@
subensembles can in principle (i.e. without leading to inconsistencies, either with special %%@
relativity or probability theory) be correlated in such a way that Bell's inequality %%@
(\ref{Bell1}) is violated by the maximum value of 4.

Consider the results of observers 1 and 2 before the separation into subensembles took place. %%@
We assume that observers 1 and 2 have performed spin measurements yielding
a value of the correlation function (\ref{e-1}) different from zero, meaning that less than %%@
half of the results are equal, i.e. have equal sign ($++$, $--$), or that less than half of the %%@
results are different, i.e. have 
different sign ($+-$, $-+$). It will turn out, that this 
assumption is sufficient to obtain an example for a case in which a violation of Bell's 
inequality by the maximum value of 4 would be inconsistent with special relativity.

The results of observers 1 and 2 are now separated into subensembles following the %%@
procedure described above. Because by assumption half of the results of observer 3 are $-%%@
$(+) this separation cannot result in two subensembles with
the absolute value of the correlation function being 1 in both %%@
subensembles.\footnote{The absolute value of the correlation function is 1 only if {\em %%@
all} \rm results within an ensemble have different or equal sign. In the considered 
case this is impossible because half of the results of observers 1 and 2 are put into 
subensemble $-$ (+), but only less than half of all results of observers 1 and 2 are
equal (different).} However for a maximum violation of Bell's inequality (by the value of 4) it %%@
is a necessary condition that the values of the correlation functions in (\ref{Bell1}) are 1. 
As soon as only one correlation function in (\ref{Bell1}) has an absolute value smaller than %%@
1 a maximum violation is no longer possible. Therefore in our special case the results within %%@
at least one of the subensembles (either subensemble + or $-$)
can never be correlated in a way leading to a maximum violation of Bell's inequality. The %%@
reason which strictly excludes this possibility is the fact, that for observer 3 the probability to %%@
measure + or 
$-$ must not depend on the kind of measurements performed by observers 1 and 2, since %%@
such a dependence would enable faster-than-light communication. 

It now becomes clear that we have two conflicting assumptions in our consideration, which %%@
prohibit the selection of subensembles appropriate for a maximum violation of Bell's %%@
inequality. On the one hand the assumption, that less than half of the results of observers 1 %%@
and 2 are equal (different) and on the other hand the assumption, that observer 3 gets the %%@
result + and $-$ with equal probability, independent of the correlations (measurements) %%@
measured (performed) by observers 1 and 2. Because the second assumption expresses the %%@
impossibility of superluminal signalling we may conclude, that in the special case denoted by %%@
the first assumption a correlation within the subensembles leading to a violation of Bell's %%@
inequality by the maximum value of 4 would be inconsistent with special relativity.


\subsubsection*{Acknowledgments}
One author (K.S.) acknowledges the help of Professor Rainer Dirl
with the quantum mechanics of spin; both authors acknowledge many
discussions with Professor Anton Zeilinger.
This paper has been partly supported by the Austrian Fonds zur
F\"orderung der Wissenschaftlichen Forschung, project nr. S65-11,
Schwerpunkt Quantenoptik.
\newpage

\subsubsection*{Appendix: Quantum expectation value of two particles of
spin $j$ in a singlet state}
%$\hat{J}^{A,B}$ are the quantum mechanical angular momentum
%operators.
\begin{eqnarray}
C(\theta )&=&
\langle J= 0 ,M= 0\mid \alpha \cdot \hat{J}^A \otimes \beta \cdot
\hat{J}^B\mid
J=0,M=0\rangle
\nonumber
 \\
&=&
 \sum_{m,m'}
\langle  00 \mid jm,j-m \rangle
\langle  jm',j-m'\mid 00 \rangle \times \nonumber
\\
&&
\qquad
\qquad
\qquad
\times
^A\langle jm\mid  ^B\langle j-m\mid
\alpha \cdot \hat{J}^A \otimes \beta \cdot \hat{J}^B
\mid jm'\rangle ^A \mid j-m'\rangle ^B  \nonumber
\\
&=&
 \sum_{m,m'}
\langle  00 \mid jm,j-m \rangle
\langle  jm',j-m'\mid 00 \rangle       \times \nonumber  \\
&&
\qquad
\qquad
\qquad   \times
\langle jm\mid
\alpha \cdot \hat{J}^{A}
\mid jm'\rangle
 \langle j-m\mid
 \beta \cdot \hat{J}^B
 \mid j-m'\rangle                             \nonumber
\\
&=&
 \sum_{m,m'}
{(-1)^{j-m}(-1)^{j-m'}\over 2j+1}
\langle jm\mid
 \hat{J}^A_z
\mid jm'\rangle
\langle j-m\mid
 \beta \cdot \hat{J}^B
 \mid j-m'\rangle \nonumber
\\
&=&
 \sum_{m,m'}
{(-1)^{j-m}(-1)^{j-m'}\over 2j+1}
m \delta _{m m'}
\langle j-m\mid
 \beta \cdot \hat{J}^B
 \mid j-m'\rangle \nonumber
\\
&=&  \sum_m
m{(-1)^{2j-2m}\over 2j+1}
 \langle j-m\mid
 \beta \cdot \hat{J}^B
 \mid j-m\rangle  \nonumber
\\
&=& {1\over 2j+1}  \sum_m
-m^2 \beta_z      \nonumber
\\
&=& -{1 \over 2j+1}\cos \theta  \sum_{m=-j}^j
m^2
\qquad {\rm for} \; 0\le \theta \le \pi
 \nonumber
\\
&=&- {j(j+1)\over 3} \, \cos \theta
\qquad {\rm for} \; 0\le \theta \le \pi
\quad .
 \nonumber
\end{eqnarray}

\clearpage


%\bibliography{svozil}
%%\bibliographystyle{plain}
%\bibliographystyle{unsrt}

\begin{thebibliography}{10}

\bibitem{bell}
John~S. Bell.
\newblock On the {E}instein {P}odolsky {R}osen paradox.
\newblock {\em Physics}, 1:195--200, 1964.
\newblock Reprinted in \cite[pp. 403-408]{wheeler-Zurek:83}.

\bibitem{chsh}
J.~F. Clauser, M.~A. Horne, A.~Shimony, and R.~A. Holt.
\newblock {\em Physical Review Letters}, 23:880--884, 1969.

\bibitem{clauser}
J.~F. Clauser and A.~Shimony.
\newblock {B}ell s theorem: experimental tests and implications.
\newblock {\em Rep. Prog. Phys.}, 41:1881--1926, 1978.

\bibitem{cirelson:80}
B.~S. Cirel'son.
\newblock Quantum generalizations of {B}ell s inequality.
\newblock {\em Letters in Mathematical Physics}, 4:93--100, 1980.

\bibitem{cirelson}
B.~S. {Cirel'son (=Tsirelson)}.
\newblock Some results and problems on quantum {B}ell-type inequalities.
\newblock {\em Hadronic Journal Supplement}, 8:329--345, 1993.

\bibitem{shimony2}
Abner Shimony.
\newblock Controllable and uncontrollable non-locality.
\newblock In S.~Kamefuchi {\it et al.}, editor, {\em Proceedings of the
  International Symposium on the Foundations of Quantum Mechanics}, pages
  225--230, Tokyo, 1984. Physical Society of Japan.
\newblock See also J. Jarrett, {\sl Bell's Theorem, Quantum Mechanics and Local
  Realism}, Ph. D. thesis, Univ. of Chicago, 1983; {\sl Nous}, {\bf 18}, 569
  (1984).

\bibitem{shimony3}
Abner Shimony.
\newblock Events and processes in the quantum world.
\newblock In R.~Penrose and C.~I. Isham, editors, {\em Quantum Concepts in
  Space and Time}, pages 182--203. Clarendon Press, Oxford, 1986.

\bibitem{pop-rohr}
S.~Popescu and D.~Rohrlich.
\newblock Quantum nonlocality as an axiom.
\newblock {\em Foundations of Physics}, 24(3):379--358, 1994.

\bibitem{rohrlich-97}
D.~Rohrlich and S.~Popescu.
\newblock In A.~Mann and M.~Revzen, editors, {\em The Dilemma of Einstein,
  Podolsky and Rosen 60 Years Later, Annals of the Israel Physical Society,
  Vol. 12}. Adam Hilger, Bristol.
\newblock in print.

\bibitem{grunhaus-96}
J.~Grunhaus, S.~Popescu, and D.~Rohrlich.
\newblock Jamming non-local quantum correlations.
\newblock {\em Physical Review}, A53:3781, 1996.

\bibitem{popescu-97}
S.~Popescu and D.~Rohrlich.
\newblock In R.S. Cohen, M.A. Horne, and J.~Stachel, editors, {\em Quantum
  Potentiality, Entanglement, and Passion-at-a-Distance: Essays for Abner
  Shimony}. Kluwer Academic publishers, Dordrecht.
\newblock in print.

\bibitem{krenn}
G{\"{u}}nther Krenn.
\newblock The probabilistic origin of {B}ell's inequality.
\newblock In D.~Han, Y.~S. Kim, N.~H. Rubin, Y.~Shih, and W.~W. Zachary,
  editors, {\em Proceedings of the Third International Workshop on Squeezed
  States and Uncertainty Relations, Maryland, August 10-13, 1993, NASA
  Conference publication Nr. 3270}, pages 603--608, Greenbelt, Maryland 20771,
  August 1993. NASA.

\bibitem{peres222}
Asher Peres.
\newblock Unperformed experiments have no results.
\newblock {\em American Journal of Physics}, 46:745--747, 1978.

\bibitem{peres}
Asher Peres.
\newblock {\em Quantum Theory: Concepts and Methods}.
\newblock Kluwer Academic Publishers, Dordrecht, 1993.

\bibitem{gisin-peres}
N.~Gisin and A.~Peres.
\newblock Maximal violation of {B}ell s inequality for arbitrarily large spin.
\newblock {\em Physics Letters}, A162:15--17, 1992.

\bibitem{aerts:82}
D.~Aerts.
\newblock Example of a macroscopic classical situation that violates {B}ell
  inequalities.
\newblock {\em Lettere al Nuovo Cimento}, 34(4):107--111, 1982.
\newblock The suggested analogy to two entangled spin--${1\over 2}$ particles
  is challenged by the fact that the proposed expectation functions are {\em
  not} invariant with respect to temporal order.

\bibitem{ghz}
Daniel~M. Greenberger, Mike~A. Horne, and Anton Zeilinger.
\newblock Going beyond bell's theorem.
\newblock In M.~Kafatos, editor, {\em Bell's Theorem, Quantum Theory, and
  Conceptions of the {U}niverse}, pages 73--76. Kluwer Academic Publishers,
  Dordrecht, 1989.
\newblock See also \cite{ghsz} and \cite{mermin}.

\bibitem{roy-singh}
S.~M. Roy and V.~Singh.
\newblock Hidden variable theories without non-local signalling and their
  experimental tests.
\newblock {\em Physics Letters}, A139(9):437--441, 1989.

\bibitem{khalfin-85}
L.~Khalfin and B.~Tsirelson.
\newblock In P.~Lahti, editor, {\em Symposium on the Foundations of Modern
  Physics '85}, page 441. World Scientific, Singapore, 1985.

\bibitem{rastall-85}
P.~Rastall.
\newblock {\em Foundations of Physics}, 15:963, 1995.

\bibitem{summers-87}
S.~Summers and R.~Werner.
\newblock {\em Journal of Mathematical Physics}, 28:2440, 1987.

\bibitem{ghsz}
Daniel~M. Greenberger, Mike~A. Horne, A.~Shimony, and Anton Zeilinger.
\newblock {B}ell s theorem without inequalities.
\newblock {\em American Journal of Physics}, 58:1131--1143, 1990.

\bibitem{mermin1}
N.~D. Mermin.
\newblock {\em American Journal of Physics}, 58:731, 1990.

\bibitem{krenn1}
G.~Krenn and A.Zeilinger.
\newblock Entangled entanglement.
\newblock {\em Phys.Rev. A}, 54(3):1793-1797, 1996.


\bibitem{wheeler-Zurek:83}
John~Archibald Wheeler and Wojciech~Hubert Zurek.
\newblock {\em Quantum Theory and Measurement}.
\newblock Princeton University Press, Princeton, 1983.

\bibitem{mermin}
N.~D. Mermin.
\newblock What's wrong with these elements of reality?
\newblock {\em Physics Today}, 43(6):9--10, June 1990.

\end{thebibliography}

\end{document}
