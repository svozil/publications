%%tth:\begin{html}<LINK REL=STYLESHEET HREF="http://tph.tuwien.ac.at/~svozil/ssh.css">\end{html}
%\documentclass[prl,preprint,showpacs,showkeys,amsfonts]{revtex4}
%\usepackage{graphicx}
\documentstyle[]{article}
%\RequirePackage{times}
%\RequirePackage{courier}
%\RequirePackage{mathptm}
\renewcommand{\baselinestretch}{1.3}
\begin{document}





\title{Problem: Coding of (interaction) function in Cantor base}
\author{}
\date{ }
\maketitle

\begin{abstract}
The goal is to represent a physical (nonlinear) problem by reals in Cantor representation
such that the physical system appears linear.
Linear physical systems are much easier to handle \& solve.
\end{abstract}



Usually it takes an exponential amount of knowledge of the initial value to forecast
a nonlinear ``chaotic'' system with positive Lyapunov exponent in linear growing time.
Stated differently, the possibility to forecast diminishes logarithmically with
a linear increase in precision of the initial values.

Maybe it is possible to hide this exponential increase into the base representation
of a real initial value such that there is a linear correspondence between forecast and
knowledge of initial value.

This may be difficult, because the arithmetic should then proceed not by the usual rules
but by different ones dictated by the Cantor representation.


%TexCad Options
%\grade{\off}
%\emlines{\off}
%\beziermacro{\on}
%\reduce{\on}
%\snapping{\off}
%\quality{2.00}
%\graddiff{0.01}
%\snapasp{1}
%\zoom{1.00}
\unitlength 1mm
\linethickness{0.4pt}
\begin{picture}(90.00,44.55)
\put(0.00,0.00){\line(1,0){29.67}}
\put(29.67,0.00){\line(1,2){15.33}}
\put(45.00,30.67){\line(1,-2){15.33}}
\put(60.33,0.00){\line(1,0){29.67}}
\end{picture}

%TexCad Options
%\grade{\off}
%\emlines{\off}
%\beziermacro{\on}
%\reduce{\on}
%\snapping{\off}
%\quality{2.00}
%\graddiff{0.01}
%\snapasp{1}
%\zoom{1.00}
\unitlength 0.3mm
\linethickness{0.4pt}
\begin{picture}(90.00,44.55)
\put(0.00,0.00){\line(1,0){29.67}}
\put(29.67,0.00){\line(0,1){30.00}}
\put(29.67,30.00){\line(1,0){30.33}}
\put(60.00,30.00){\line(0,-1){30.00}}
\put(60.00,0.00){\line(1,0){30.00}}
\end{picture}
%TexCad Options
%\grade{\off}
%\emlines{\off}
%\beziermacro{\on}
%\reduce{\on}
%\snapping{\off}
%\quality{2.00}
%\graddiff{0.01}
%\snapasp{1}
%\zoom{1.00}
\unitlength 0.09mm
\linethickness{0.4pt}
\begin{picture}(90.00,44.55)
\put(0.00,0.00){\line(1,0){29.67}}
\put(30.00,0.00){\line(-2,5){11.47}}
\put(18.53,28.67){\line(5,3){26.47}}
\put(90.00,0.00){\line(-1,0){29.67}}
\put(60.00,0.00){\line(2,5){11.47}}
\put(71.47,28.67){\line(-5,3){26.47}}
\end{picture}

Another, even more speculative vision, is the coding of exponential (or arbitrary nonlinear)
interaction by reals in Cantor representation, such that again the interaction appears to
be linear.



\bibliography{svozil}
%\bibliographystyle{apsrev}
\bibliographystyle{unsrt}

\end{document}

