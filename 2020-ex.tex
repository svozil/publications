\documentclass[%
  reprint,
  twocolumn,
 %superscriptaddress,
 %groupedaddress,
 %unsortedaddress,
 %runinaddress,
 %frontmatterverbose,
 % preprint,
 showpacs,
 showkeys,
 preprintnumbers,
 %nofootinbib,
 %nobibnotes,
 %bibnotes,
 amsmath,amssymb,
 aps,
 % prl,
 pra,
 % prb,
 % rmp,
 %prstab,
 %prstper,
  longbibliography,
 %floatfix,
 %lengthcheck,%
 ]{revtex4-1}

%\usepackage{cdmtcs-pdf}

\usepackage[dvipsnames]{xcolor}

\usepackage{mathptmx}% http://ctan.org/pkg/mathptmx

\usepackage{amssymb,amsthm,amsmath,bm}

\usepackage{tikz}
\usetikzlibrary{calc,decorations.pathreplacing,decorations.markings,positioning,shapes}

\usepackage[breaklinks=true,colorlinks=true,anchorcolor=blue,citecolor=blue,filecolor=blue,menucolor=blue,pagecolor=blue,urlcolor=blue,linkcolor=blue]{hyperref}
\usepackage{graphicx}% Include figure files
\usepackage{url}


\begin{document}

\title{Quantum violation of the Suppes-Zanotti inequalities and ``contextuality''}


\author{Karl Svozil}
\email{svozil@tuwien.ac.at}
\homepage{http://tph.tuwien.ac.at/~svozil}

\affiliation{Institute for Theoretical Physics,
TU Wien,
\\
Wiedner Hauptstrasse 8-10/136,
1040 Vienna,  Austria}


\date{\today}

\begin{abstract}
The Suppes-Zanotti inequalities involving the joint expectations of just three binary quantum observables are (re-)derived by the hull computation of the respective correlation polytope.
A min-max calculation reveals its maximal quantum violations correspond to a generalized Tsirelson bound.
Notions of ``contextuality'' motivated by such violations are critically reviewed.
\end{abstract}

\keywords{Suppes-Zanotti inequalities, Greenberger-Horne-Zeilinger argument, Kochen-Specker theorem, Born rule, min-max calculation, convex polytope, contextuality}
\pacs{03.65.Ca, 02.50.-r, 02.10.-v, 03.65.Aa, 03.67.Ac, 03.65.Ud}

\maketitle

\section{Two-partite vector-based expectations not satisfying classical bounds}

Classical bounds on probabilities and expectations can be expected to be ``violated'' by or ``being different'' from quantum probabilities and expectations because the latter are based on multi-dimensional vectorial entities
whereas the former are based on scalars in (sub)sets of power sets.
Exactly how these violations are operationalized and measured has developed from an intuitive, heuristic search in the early days~\cite{bell,chsh,wigner-70}
into a systematic method~\cite{froissart-81,Garg1984,pitowsky-86,cirelson,svozil-2017-b}.


The most elementary expression of the classical versus quantum difference quoted earlier
is the two-partite correlation function of two dichotomic observables $X,Y\in \{-1,+1\}$.
It is empirically collected from a series of $N$ measurements of $X$ and $Y$ and defined by
$\langle X,Y \rangle_s  \approx  \frac{1}{N}\sum_{i=1}^N   X_i Y_i$, where the index $i$ refers to the $i$th measurement,
and $s$ refers to a specific (unaltered) state on which these repetitive measurements are performed.
It is assumed that, if $N$ increases, the limit exists and is monotonically approached --
that is, for ``large enough'' $N$, $\langle X,Y \rangle_s$ is a ``good approximation''~\cite{Uffink2011-UFFSPS}.

\subsection{Classical predictions on ``singlet-type'' states}

It is not too difficult to model a classical two-partite state $q$ which shows ``singlet-like'' characteristics
-- an example would be the angular momentum in a particular spacial direction
of two fragments of a bomb which originally had no angular momentum in any direction~\cite{peres}.
An argument involving equidistribution of angular momenta of the fragments reveals a linear classical correlation on such a state; that is,
$\langle X,Y \rangle_c  = \textsf{\textbf{E}}(X,Y) = \textsf{\textbf{E}}(\theta) = -1+2 \theta    /\pi$,
where the angle $0 \le \theta \le \pi$ characterizes the ``spatial separation'' of the directions of these observables $X$ and $Y$.


\subsection{General classical predictions}

To construct a generic classical situation, a generalized urn model~\cite{wright} is introduced
which can also be phrased in terms of finite-state identification problems of automata allowing
complementarity~\cite{e-f-moore}.
Its formalization is in terms of set-theoretic partitions~\cite{svozil-2001-eua} and power sets.

In terms of generalized urn models, we consider urns filled with black balls painted with three different colors,
one color per observable $X$, $Y$, and $Z$.
Since each observable may have two different outcomes we can, for instance, label these outcomes by ``$+$'' and ``$-$'',
printed on these balls in the respective colors.
There are eight such ball types. As the urn is filled with an arbitrary distribution of ball types, it can only be ascertained that
they occur with probabilities $0\le \lambda_{\pm \pm \pm} \le 1$, were the indices refer to the respective
symbols in the colors associated with our three observables.
Since in such a scheme the ball types are mutually exclusive and their enumeration is complete (i.e., exhaustive),
we can suppose that $\sum_{i,j,k \in \{+,-\}}\lambda_{ijk}=1$, and
the joint expectations add up accordingly; e.g.,
$\textsf{\textbf{E}}(X,Y)
= \sum_{k \in \{+,-\}}\left[ \lambda_{++,k} + \lambda_{--,k} - \left( \lambda_{+-,k} + \lambda_{-+,k}\right)\right]$.




\subsection{Quantum predictions on a singlet state}

The quantum predictions of a single observable in an arbitrary direction
characterized by the spherical coordinates $0\le \theta \le \pi$ and $0\le \varphi < 2\pi$
is derived from the Pauli spin matrices $\sigma_x$,  $\sigma_y$ and $\sigma_z$
forming the spin operator
${\bm{\sigma}} (\theta , \varphi ) = \sigma_x \sin\theta \cos\varphi  + \sigma_y \sin\theta \sin\varphi  + \sigma_z \cos\theta$
and the single particle projection operator
$\textsf{\textbf{S}}_\pm (\theta , \varphi) =\frac{1}{2} \left[ \mathbb{I}_2  \pm \bm{\sigma} (\theta , \varphi ) \right]$
for the states ``$-$'' and ``$+$'', respectively.
The respective two-partite projection operators are
$\textsf{\textbf{S}}_{\pm_1\pm_2} (\theta_1 , \varphi_1, \theta_2 , \varphi_2) = \textsf{\textbf{S}}_{\pm_1} (\theta_1 , \varphi_1) \otimes  \textsf{\textbf{S}}_{\pm_2} (\theta_2 , \varphi_2)$.
Finally, the operator associated with the two-partite expectations is
$\textsf{\textbf{F}}(X,Y) = \textsf{\textbf{F}}(\theta_1 , \varphi_1, \theta_2 , \varphi_2) =
\textsf{\textbf{S}}_{++}+
\textsf{\textbf{S}}_{--}-
\left(
\textsf{\textbf{S}}_{+-}+
\textsf{\textbf{S}}_{-+}
\right)={\bm{\sigma}} (\theta_1 , \varphi_1 ) \otimes {\bm{\sigma}} (\theta_2 , \varphi_2 )$.

Suppose we are interested in the correlation function for a singlet state in the Bell basis
$q = \vert \Psi_- \rangle \langle \Psi_-\vert $
with
$\vert \Psi_- \rangle = \frac{1}{\sqrt{2}}\begin{pmatrix}0,1,-1,0\end{pmatrix}^\intercal$,
then the quantum prediction yields
$\langle X,Y \rangle_q  = \textsf{\textbf{F}}(\theta_1,\varphi_1,\theta_2,\varphi_2) =
 - \left[\cos\theta_1 \cos\theta_2 + \cos(\varphi_1 - \varphi_2) \sin\theta_1 \sin\theta_2 \right]$.
For $\varphi_1=\varphi_2$ this reduces to the well-known cosine form
$\langle X,Y \rangle_q  =\textsf{\textbf{F}}(\theta_1,0,\theta_2,0) =
\textsf{\textbf{F}}(\theta_1-\theta_2) = - \cos (\theta_1 - \theta_2)$,
that is, the two-partite correlation for dichotomic observables $X,Y=\pm 1$ of the two-partite singlet state
is proportional to the Euclidean scalar product between the vectors associated with $X$ and $Y$.

The maximal quantum-to-classical violations
\begin{equation}
\max_{\theta \in \{0, \pi\}} \left|\textsf{\textbf{E}}(\theta ) -\textsf{\textbf{F}}(\theta )\right|
=
\sqrt{1-\left(\frac{2}{\pi}\right)^2}-\frac{2}{\pi }\cos^{-1}\frac{2}{\pi }\approx 0.2
\end{equation}
resulting from less,
as well as more,
equal occurrences of the joint observables $++$/$--$ and $+-$/$-+$,
occur at angles $(d/d\theta )\left[ \textsf{\textbf{E}}(\theta ) -\textsf{\textbf{F}}(\theta ) \right] = 0$,
that is, at
\begin{equation}
\begin{aligned}
\theta &=\sin ^{-1} \frac{2}{\pi}\text{ as well as}\\
\theta &= \pi - \sin ^{-1} \frac{2}{\pi} \text{, respectively.}
\end{aligned}
\end{equation}


% N[ArcSin[2/Pi]]; Plot[{-Cos[t], -1 + 2 t/Pi}, {t, 0, Pi}]; D [Cos[t] - 1 + 2 t/Pi, t]; Plot[2/Pi - Sin[t], {t, 0, Pi}]; TeXForm  [TrigExpand[-1 + (2 ArcSin [2/Pi])/Pi + Cos[ArcSin[2/Pi]]]]

\subsection{Quantum predictions on more general pure states}

By a min-max calculation~\cite{filipp-svo-04-qpoly-prl} it is not too difficult to compute those quantum states which,
given arbitrary angles between the two observables $X$ and $Y$, yield the minimal and maximal correlations:
all that is needed is the eigensystem  of $\textsf{\textbf{F}}(\theta_1 , \varphi_1, \theta_2 , \varphi_2)$.
Rather than enumerating this eigensystem in full generality the special case $\theta_1=\theta$ and $\theta_2 = \varphi_1 = \varphi_2 =0$ is posted,
resulting in the (decomposable) vectors (modulo normalization)
\begin{equation}
\begin{aligned}
\vert \psi_\text{1,min}\rangle &= \begin{pmatrix}
0,  {\cos \theta +1},0,{\sin \theta}
\end{pmatrix}^\intercal \text{ as well as}\\
\vert \psi_\text{2,min}\rangle &= \begin{pmatrix}
 {\cos \theta -1},0 ,{\sin \theta},0
\end{pmatrix}^\intercal
\end{aligned}
\end{equation}
for the minimal expectation $\langle X,Y \rangle  = -1$;
and
\begin{equation}
\begin{aligned}
\vert \psi_\text{1,max}\rangle &= \begin{pmatrix}
0,  {\cos \theta -1},0,{\sin \theta}
\end{pmatrix}^\intercal\text{ as well as}\\
\vert \psi_\text{2,max}\rangle &= \begin{pmatrix}
 {\cos \theta +1},0 ,{\sin \theta},0
\end{pmatrix}^\intercal
\end{aligned}
\end{equation}
for the maximal expectation $\langle X,Y \rangle  = 1$,
respectively.


\section{The case of three observables}
\subsection{Classical bounds}

One might as well stop here, contemplate the elementary difference between two forms of probabilities based on scalars and power sets in the classical case,
and on vectors and the vector space spanned by them in the quantum case,
and leave it at that.
However, this is not what happened historically: Bell and others tried to find criteria for non-compliance with classical behavior involving more than just two observables.
In particular, Suppes and Zanotti~\cite{Suppes-81,Brody-1989,Khrennikov2020} presented  special cases of what Boole called ``conditions of possible experience''~\cite{Boole-62,Pit-94}
involving just three dichotomic observables $X,Y,Z\in \{-1,+1\}$.

The original method of deriving these bounds is rather involved.
But with today's convex polytope techniques~\cite{froissart-81,pitowsky,cirelson} it is not too difficult to derive those inequalities:
(i) form all possible combinations of joint occurrences by multiplying the respective dichotomic observables --
in this case
$\textsf{\textbf{E}}(X,Y) = XY$,
$\textsf{\textbf{E}}(X,Z) = XZ$,
$\textsf{\textbf{E}}(Y,Z) = YZ$;
(ii) form the 3-tuples (that is, the finite ordered list or sequence) of all three numbers for particular instances of $X,Y,Z \in \{-1,+1\}$
$ \begin{pmatrix}
\textsf{\textbf{E}}(X,Y) , \textsf{\textbf{E}} (X,Z) , \textsf{\textbf{E}} (Y,Z)
\end{pmatrix}
=
\begin{pmatrix} XY , XZ , YZ \end{pmatrix} $,
(iii) pretend these 3-tuples are coordinates (with respect to the Cartesian three-dimensional standard basis)
of vertices of a convex polytope,
and
(iv) according to the Minkowski-Weyl ``main'' representation theorem~\cite{ziegler,Schrijver,Fukuda-techrep}
represent this polytope as its facets obtained by the hull computation~\cite{Fukuda-techrep,Pit-91}.
These facet (in)equalities represent Boole-Bell type ``conditions of possible (classical) experience''.

With three dichotomic observables, such procedures result in eight three-dimensional row vectors.
Four of them are linearly independent.
They are interpreted as the vertices of a correlation polytope.
The row vectors, stacked on top of one another, form a
 $4 \times 3$ Travis~\cite{travis-mt-62} matrix~\cite{greechie-66-PhD}
\begin{equation}
T_{ij}=\begin{pmatrix}
   +1  & +1 &  +1  \\
   +1  & -1 &  -1 \\
   -1  & +1 &  -1 \\
   -1  & -1 &  +1
\end{pmatrix} .
\end{equation}
The hull computation
(eg, by  {\tt pycddlib}~\cite{pycddlib}, a Python wrapper of Fukuda's {\tt cddlib} algorithm~\cite{cdd-pck}
implementing  the Double Description Method~\cite{MRTT53})
yields the four Suppes-Zanotti-Brodi inequalities~\cite{Suppes-81,Brody-1989}
\begin{equation}
\begin{split}
- 1  \le   \textsf{\textbf{E}}(X,Y) + \textsf{\textbf{E}}(X,Z) + \textsf{\textbf{E}}(Y,Z)   ,   \\
- 1  \le   - \textsf{\textbf{E}}(X,Y) - \textsf{\textbf{E}}(X,Z) + \textsf{\textbf{E}}(Y,Z) ,   \\
- 1  \le   \textsf{\textbf{E}}(X,Y) - \textsf{\textbf{E}}(X,Z) - \textsf{\textbf{E}}(Y,Z)   ,   \\
- 1  \le   -\textsf{\textbf{E}}(X,Y) + \textsf{\textbf{E}}(X,Z) - \textsf{\textbf{E}}(Y,Z)
.
\end{split}
\label{2020-ex-SZI}
\end{equation}

\subsection{Quantum bounds by min-max calculation}

The min-max calculations~\cite{filipp-svo-04-qpoly-prl}
of the associated operators
$
\textsf{\textbf{F}}(X,Y) \pm \textsf{\textbf{F}}(X,Z) \pm \textsf{\textbf{F}}(Y,Z)
$
with the quantum expectation $\textsf{\textbf{F}}$ as defined earlier
amounts to summing up the separate terms and determining the eigensystem of these new observables.
It yields quantum bounds allowing ranges bounded by
\begin{equation}
-3 <  \textsf{\textbf{F}}(X,Y) \pm \textsf{\textbf{F}}(X,Z) \pm \textsf{\textbf{F}}(Y,Z)   < 3
\label{2020-ex-SZIq}
\end{equation}
which violate the classical ones~(\ref{2020-ex-SZI})
by almost the greatest algebraically possible amount.

For the sake of more concrete realizations, we shall set all azimuthal angles to zero and take equidistant polar angles
such that the directions of $X$, $Y$, and $Z$ in configuration space are $0$, $\theta$, and $2 \theta$, respectively.
Then the min-max computation associated with
$\textsf{\textbf{F}}(0,\theta ) + \textsf{\textbf{F}}(0,2 \theta ) + \textsf{\textbf{F}}(\theta , 2 \theta )$
exhibits two eigenvalues
\begin{equation}
\mu_1= -(5 + 4 \cos\theta)^{1/2} \le  -(1 + 2 \cos \theta )=\mu_2
\end{equation}
which, in a certain domain of $\theta$, violate the first inequality in~(\ref{2020-ex-SZI}).
The associated pure states are proportional to
% TeXForm[FullSimplify[   TrigExpand[    FullSimplify[      TrigReduce[       Eigensystem[SZ1[0, 0, \[Theta], 0, 2 \[Theta], 0]][[2,         3]]]]]*2 (1 + Cos[\[Theta]])]]
\begin{equation}
\begin{aligned}
& \vert {\bf x}_1 \rangle =
\begin{pmatrix}
a ,
b  ,
 -b  ,
a
\end{pmatrix}^\intercal
\text{, where}\\
&\quad a  = 2 (\cos \theta +1)\sin \theta \text{ and } \\
&\quad b =  2 \cos \theta +\cos (2 \theta )+\sqrt{5+4 \cos \theta} \text{, as well as }\\
% TeXForm[FullSimplify[   TrigExpand[    FullSimplify[      TrigReduce[       Eigensystem[SZ1[0, 0, \[Theta], 0, 2 \[Theta], 0]][[2,         1]]]]]*2 (1 + Cos[\[Theta]])*Sin[\[Theta]]]]
&\vert {\bf x}_2 \rangle =        \begin{pmatrix}
-\sin  \theta  ,\cos  \theta  ,\cos  \theta  ,\sin  \theta
\end{pmatrix}^\intercal
\text{, respectively.}
\end{aligned}
\end{equation}
Note that for $\theta \rightarrow 0$ these two states converge to indecomposable vectors proportional to the  Bell basis states
$
\begin{pmatrix}
0 , 1  , -1 , 0
\end{pmatrix}^\intercal
$
as well as
$
\begin{pmatrix}
0  ,1  ,1  ,0
\end{pmatrix}^\intercal
$.
Indeed,   for  $\theta \rightarrow 0$,
the two other eigenstates rendering the two eigenvalues
$(5 + 4 \cos\theta )^{1/2} ,
1 + 2 \cos \theta  \rightarrow 3$, converge to the remaining states in the Bell basis.

\subsection{Composition of higher-order distribution by lower-order ones}

For some ``practical'' application recall
Specker's story about~\cite{specker-60} {\em ``a wise man from Ninive $\ldots$ who was $\ldots$ concerned almost exclusively about his daughter''}
and an oracle potential suitors had to cope with:
{\em ``The suitors were led in front of a table on which three boxes were positioned in a row, and they were ordered to indicate which of the boxes contained a gem and which were empty.
And now no matter how many times they tried, it seemed to be impossible to solve the task.
After their predictions, each of the suitors was ordered to open two boxes which they had indicated to be both empty or both not empty:
it turned out each time that one contained a gem and the other did not, and,
to be precise, sometimes the gem was in the first, sometimes in the second of the boxes that were opened.
But how can it be possible that from three boxes neither two can be indicated as empty, nor as not empty?''}

A similar scheme was mentioned by Garg and Mermin~\cite{Garg1984}:
{\em ``if we have three dichotomic
variables each of which assumes either the value 1 or -1 with equal
probability and all the pair distributions vanish unless the members of the
pair have different values~$\ldots$~.''}

These scenarios mention three observables and strict anti-correlations between pairs of observable outcomes, such that
$ \textsf{\textbf{E}}(X,Y) =\textsf{\textbf{E}}(X,Z) = \textsf{\textbf{E}}(Y,Z)=-1 $.
As can be readily checked by the (maximal) violation of the first Suppes-Zanotti-Brodi inequalities~(\ref{2020-ex-SZI}) no classical global probability distribution allows this.
But quantum mechanics can ``almost'' provide a realization as it yields ``almost perfect'' anti-correlations
at ``almost vanishing'' angles $0< \theta \ll 1$.
The ``reason'' for this is threefold: (i) the quantum expectation function, as mentioned earlier, is
$\langle X,Y \rangle_q  =\textsf{\textbf{F}}(\theta_1,0,\theta_2,0) = - \cos (\theta_1 - \theta_2)$;
(ii) the three expectation functions are complementary and therefore cannot be measured simultaneously -- they have no simultaneous value definiteness; and
(iii) the quantum resources exploit a four-dimensional Hilbert space with probabilities based on vectors rather than scalars.

It might be worth noting that Greenberger, Horne, and Zeilinger proposed another, adaptive, protocol involving expectations of order three and
going beyond stochastic quantum violations of classical predictions~\cite{ghz,ghsz,mermin}
which could be rewritten as a game ``people play''~\cite{PhysRevLett.82.1345,panbdwz,bacon-ghzgames-2006}
in which particular quantum states allow certain players always to win whereas this is not guaranteed classically~\cite{svozil-2020-ghz}.





\section{The case of four and more observables}

For completeness, we just mention that the addition of an additional variable yields the well-known Clauser-Horne-Shimony-Holt inequalities~\cite{chsh}.
A polytope derivation can be found in Refs.~\cite{froissart-81,pitowsky,cirelson}.
Its quantum bound $-2\sqrt{2} \le \textsf{\textbf{F}}(W,Y) + \textsf{\textbf{F}}(W,Z) + \textsf{\textbf{F}}(X,Y)  - \textsf{\textbf{F}}(X,Z) \le
2\sqrt{2}$ derived by Cirel'son (aka Tsirelson)~\cite{cirelson:80} can be straightforwardly obtained from a min-max calculation~\cite{filipp-svo-04-qpoly-prl}
of its eigensystem. The quantum states rendering this bound can be represented by the vectors proportional to
$\begin{pmatrix} -1, 1, 1, 1\end{pmatrix}^\intercal$
and $\begin{pmatrix} -1, -1, -1, 1\end{pmatrix}^\intercal$, respectively.

The polytope method can be straightforwardly scaled to derive Boolean ``bounds of classical experience''
for over four observables~\cite{2000-poly,sliwa-2003,collins-gisin-2003}.
Their respective quantum violations can again be derived by a min-max calculation~\cite{filipp-svo-04-qpoly-prl}.


\section{``Contextuality'' in context}

Let me add a cautionary remark on the widely held opinion that violations of classical Boolean criteria
such as the Suppes-Zanotti-Brodi inequalities
suggest or even imply ``contextuality''.
Presently the term ``contextual''~\cite{Dzhafarov-2017,Abramsky2018,Grangier_2002,Khrennikov2017,Jaeger2019,Jaeger2020,Auffeves-Grangier-2018,Auffves2020,Grangier-2020,cabello2021contextuality,svozil-2021-context}
is often heuristically used as
{\em ``violation of some inequality
that is derived by assuming classical probability distributions''}~\cite{cabello:210401,cabello2020converting}.
There are a variety of notions~\cite{svozil:040102}
and accompanying measures~\cite{svozil-2011-enough,Abramsky-2017,Khrennikov2017,KujalaDzhafarov-2019} for the term ``contextuality''.

This ``modern'' quantitative use of the word can be contrasted with Bohr's synthetic suggestion of a
{\em conditionality of phenomena} by~\cite{bohr-1949,Khrennikov2017,Jaeger2019}
{\em ``the impossibility of any sharp separation between the behavior of atomic
objects and the interaction with the measuring instruments which serve to define the conditions
under which the phenomena appear.''}
A related proposition from the realist Bell contends that~\cite{bell-66}
{\em ``the result of an observation may reasonably depend $\ldots$ on the complete disposition of the apparatus.''}

In this line of thought an experimental outcome---or, in another wording,
a phenomenon that should be considered as~\cite{bohr:39caus,Bohr-CW7-1996-cpip,sep-qt-uncertainty}
{\em ``the comprehension of the effects observed under given experimental conditions''}---is
composed of contributions from both the measured object as well as from the measurement apparatus.
Therefore the entire experimental configuration---effectively the experimental context---needs to be taken into account.
As not all experimental contexts can be expected to be physically realizable simultaneously,
not all observables can be expected to be jointly measurable.
In essence this view suggests that contextuality reduces to what Bohr considers to be complementarity~\cite[B1--B3]{Khrennikov2017};
and also to Heisenberg's related Principle of Indeterminacy or Uncertainty Principle---contextuality from indeterminacy~\cite{Jaeger2019,Jaeger2020,Ozawa2003}.
(See also Glauber's concrete quantum amplifier model~\cite{glauber,glauber-collected-cat,Glauber-cat-86},
as well as the ``{H}umpty-{D}umpty'' model of spin measurements~\cite{engrt-sg-I,engrt-sg-II}.)

However, this does not imply -- and it may be even misleading to believe --
that these conceivable ``results of an observation'' (aka outcome/event) are ``dormant'' properties of the object (alone) which become ``visible/actuated''
by some ``complete disposition of the apparatus'' (aka context).
More precisely, there need not be any functional
(in the sense of uniqueness) dependency of the outcome that originates from inherent information, causes or factors residing in or determined by the observed system;
no value definite intrinsic property of the object alone.
One could understand Bohr and Bell also by their insistence that the value definite properties (characterizing its physical state) of the object
become ``amalgamated'' with (properties of) the measurement apparatus,
so that an observation signals the combined information both of the object as well as of the measurement apparatus.

If one prepares a quantized system to be in a pure state formalized by a vector,
then it is perfectly value definite for observable properties corresponding to that same preparation (context).
But if there is a mismatch between preparation and measurement, the latter environment distorts value definiteness by an ``inflow''
of information from ``outside of'' the object.
Consequently, it makes no sense to speak of any such measurement result as ``being an element of physical reality'' associated with the observed system alone --
one has to add the (open) environment which ``translates'' the preparation into the measurement,
thereby introducing (external with respect to the object) noise~\cite{Glauber-cat-86,svozil-2003-garda}.

\section{What propositions support which probabilities?}

For comparing probabilities and expectations on propositional structures
I maintain that in all such considerations two issues need to be distinguished as separate criteria:
\begin{itemize}
\item[(i)]
Given some particular type of propositional structure (aka logics); which variety of probability distribution(s) is(are) supported by
this propositional structure?
\item[(ii)]
Given two or more such varieties of probability distributions, exactly what types of probability distributions should be compared with one another?
Is this not a question that needs to be settled for the particular type of systems dealt with?
\end{itemize}

I am unaware of any systematic way of answering the first question (i).
One approach, motivated by
Gleason-type theorems~\cite{Busch-2003,caves-fuchs-2004,Granstrom-mt,wright-Victoria,Wright_2019,Wright2019},
is in terms of  is Cauchy-type functional equations.

For instance, the same propositional structure may,
on the one hand, support a classical hidden variable theory
based on scalars as well as on (subsets of) a single Boolean algebra,
while
on the other hand, accommodate a quantum interpretation based on multi-dimensional vector space entities~\cite{svozil-2018-b}.
Take, for example, the Specker bug/cat's cradle~\cite{kochen2,Pitowsky2003395,pitowsky-06},
or the house/pentagon/pentagram~\cite{greechie-1974,wright:pent,Klyachko-2008} logics:
both have a classical interpretation in terms of partitions of the sets of two-valued measures~\cite{svozil-2001-eua}
as well as a faithful orthogonal representation~\cite{lovasz-79,lovasz-89} as vectors.

But there are also structures that do not allow any global classical probability distribution
yet support a vector coordinatization (aka faithful orthogonal representation).
Examples are the Specker bug combo denoted by $\Gamma_3$ by Kochen and Specker~\cite{kochen1}
that has a nonseparable set of two-valued states.
In the extreme case there exists no classical truth assignment (relative to admissibility; ie, exclusivity and completeness):
take, for example, $\Gamma_2$~\cite{kochen1}, or the logics introduced in Refs.~\cite{cabello-96,2015-AnalyticKS}.
One ``demarcation criterion'' is the separability of the observables by two-valued states, as expressed
in Kochen and Specker's Theorem~0~\cite{kochen1}.

Conversely, there exists a plethora of propositional structures~\cite{svozil-2018-b}
that allow a partition logic interpretation, and therefore
global classical probability distributions; and yet they do not support any
faithful orthogonal representation,
and therefore no quantization and no quantum probabilities.
The simplest such example are three observables which, when depicted in a hypergraph~\cite{greechie:71,kalmbach-83,Bretto-MR3077516},
form a cyclical triangular structure.

It might not be too unreasonably to state that quantum ``contextuality'' needs only to show up if the observables
satisfy Kochen and Specker's demarcation criterion by forming some propositional structure that has no classical realization
and no joint probability distribution~\cite{svozil-2021-context}.
Before that one is talking about ``complementary'' configurations, which also allow global classical probability distributions
-- albeit with different probabilistic predictions yielding violations of Boole's ``conditions of possible (classical) experience''.

\section{Contextuality as object constructions}

As has been mentioned earlier, most investigations into ``contextuality'' concentrate on the second criterion (ii)
and compare discords between classical versus quantum probabilistic predictions.
Thereby a presumption is an insistence that one is only willing to accept classical Boolean propositional structures representable by (power) sets as ontological entities.

This presumption is meshed with what Bell claimed to be true: that {\it ``everything has definite properties''}~\cite{Bertlmann2020}.
That is, there is a common belief in ``Omni-definiteness'', that any outcome of some measurement reflects an ``inner property'' or ``element of physical reality''~\cite{epr}
of the ``object'' one is
pretending to ``measure''. No doubts are raised about the construction of this ``object'' which may involve important signal contributions from the measurement apparatus.
Pointedly stated: the very notion of  ``physical object''~\cite{Yanofsky-object}
-- rather than an ``image of our mind'' in the sense of Hertz~\cite{hertz-94,hertz-94e} --
may be a naive conception that is inappropriate for situations in which one is dealing with
certain types of complementary ``observables''  and, in particular,
that have no simultaneous value definiteness~\cite{pitowsky:218,2015-AnalyticKS}.
If, for instance, one would also be willing to contemplate vectors as fundamental ontological entities, then value definiteness ensues as pure states,
and arguments based on the scarcity or even absence of classical ``non-contextual'' truth assignments decay into thin air.




\begin{acknowledgments}
This research was funded in whole, or in part, by the Austrian Science Fund (FWF), Project No. I 4579-N.
For the purpose of open access, the author has applied a CC BY public copyright licence to any Author Accepted Manuscript version arising from this submission.

The author declares no conflict of interest.

This paper was stimulated by a discussion on ``objectification'' and a respective draft paper~\cite{Yanofsky-object} of
Noson Yanofsky, as well as by a
question raised by Andrei Khrennikov in an email message relating to his recent paper~\cite{Khrennikov2020}.
\end{acknowledgments}


%\bibliography{svozil}


%merlin.mbs apsrev4-1.bst 2010-07-25 4.21a (PWD, AO, DPC) hacked
%Control: key (0)
%Control: author (0) dotless jnrlst
%Control: editor formatted (1) identically to author
%Control: production of article title (0) allowed
%Control: page (1) range
%Control: year (0) verbatim
%Control: production of eprint (0) enabled
\begin{thebibliography}{98}%
\makeatletter
\providecommand \@ifxundefined [1]{%
 \@ifx{#1\undefined}
}%
\providecommand \@ifnum [1]{%
 \ifnum #1\expandafter \@firstoftwo
 \else \expandafter \@secondoftwo
 \fi
}%
\providecommand \@ifx [1]{%
 \ifx #1\expandafter \@firstoftwo
 \else \expandafter \@secondoftwo
 \fi
}%
\providecommand \natexlab [1]{#1}%
\providecommand \enquote  [1]{``#1''}%
\providecommand \bibnamefont  [1]{#1}%
\providecommand \bibfnamefont [1]{#1}%
\providecommand \citenamefont [1]{#1}%
\providecommand \href@noop [0]{\@secondoftwo}%
\providecommand \href [0]{\begingroup \@sanitize@url \@href}%
\providecommand \@href[1]{\@@startlink{#1}\@@href}%
\providecommand \@@href[1]{\endgroup#1\@@endlink}%
\providecommand \@sanitize@url [0]{\catcode `\\12\catcode `\$12\catcode
  `\&12\catcode `\#12\catcode `\^12\catcode `\_12\catcode `\%12\relax}%
\providecommand \@@startlink[1]{}%
\providecommand \@@endlink[0]{}%
\providecommand \url  [0]{\begingroup\@sanitize@url \@url }%
\providecommand \@url [1]{\endgroup\@href {#1}{\urlprefix }}%
\providecommand \urlprefix  [0]{URL }%
\providecommand \Eprint [0]{\href }%
\providecommand \doibase [0]{http://dx.doi.org/}%
\providecommand \selectlanguage [0]{\@gobble}%
\providecommand \bibinfo  [0]{\@secondoftwo}%
\providecommand \bibfield  [0]{\@secondoftwo}%
\providecommand \translation [1]{[#1]}%
\providecommand \BibitemOpen [0]{}%
\providecommand \bibitemStop [0]{}%
\providecommand \bibitemNoStop [0]{.\EOS\space}%
\providecommand \EOS [0]{\spacefactor3000\relax}%
\providecommand \BibitemShut  [1]{\csname bibitem#1\endcsname}%
\let\auto@bib@innerbib\@empty
%</preamble>
\bibitem [{\citenamefont {Bell}(1964)}]{bell}%
  \BibitemOpen
  \bibfield  {author} {\bibinfo {author} {\bibfnamefont {John~Stuard}\
  \bibnamefont {Bell}},\ }\bibfield  {title} {\enquote {\bibinfo {title} {On
  the {E}instein {P}odolsky {R}osen paradox},}\ }\href {\doibase
  10.1103/PhysicsPhysiqueFizika.1.195} {\bibfield  {journal} {\bibinfo
  {journal} {Physics Physique Fizika}\ }\textbf {\bibinfo {volume} {1}},\
  \bibinfo {pages} {195--200} (\bibinfo {year} {1964})}\BibitemShut {NoStop}%
\bibitem [{\citenamefont {Clauser}\ \emph {et~al.}(1969)\citenamefont
  {Clauser}, \citenamefont {Horne}, \citenamefont {Shimony},\ and\
  \citenamefont {Holt}}]{chsh}%
  \BibitemOpen
  \bibfield  {author} {\bibinfo {author} {\bibfnamefont {John~F.}\ \bibnamefont
  {Clauser}}, \bibinfo {author} {\bibfnamefont {Michael~A.}\ \bibnamefont
  {Horne}}, \bibinfo {author} {\bibfnamefont {Abner}\ \bibnamefont {Shimony}},
  \ and\ \bibinfo {author} {\bibfnamefont {Richard~A.}\ \bibnamefont {Holt}},\
  }\bibfield  {title} {\enquote {\bibinfo {title} {Proposed experiment to test
  local hidden-variable theories},}\ }\href {\doibase
  10.1103/PhysRevLett.23.880} {\bibfield  {journal} {\bibinfo  {journal}
  {Physical Review Letters}\ }\textbf {\bibinfo {volume} {23}},\ \bibinfo
  {pages} {880--884} (\bibinfo {year} {1969})}\BibitemShut {NoStop}%
\bibitem [{\citenamefont {Wigner}(1970)}]{wigner-70}%
  \BibitemOpen
  \bibfield  {author} {\bibinfo {author} {\bibfnamefont {Eugene~P.}\
  \bibnamefont {Wigner}},\ }\bibfield  {title} {\enquote {\bibinfo {title} {On
  hidden variables and quantum mechanical probabilities},}\ }\href {\doibase
  10.1119/1.1976526} {\bibfield  {journal} {\bibinfo  {journal} {American
  Journal of Physics}\ }\textbf {\bibinfo {volume} {38}},\ \bibinfo {pages}
  {1005--1009} (\bibinfo {year} {1970})}\BibitemShut {NoStop}%
\bibitem [{\citenamefont {Froissart}(1981)}]{froissart-81}%
  \BibitemOpen
  \bibfield  {author} {\bibinfo {author} {\bibfnamefont {M.}~\bibnamefont
  {Froissart}},\ }\bibfield  {title} {\enquote {\bibinfo {title} {Constructive
  generalization of {B}ell's inequalities},}\ }\href {\doibase
  10.1007/BF02903286} {\bibfield  {journal} {\bibinfo  {journal} {Il Nuovo
  Cimento B (11, 1971-1996)}\ }\textbf {\bibinfo {volume} {64}},\ \bibinfo
  {pages} {241--251} (\bibinfo {year} {1981})}\BibitemShut {NoStop}%
\bibitem [{\citenamefont {Garg}\ and\ \citenamefont {Mermin}(1984)}]{Garg1984}%
  \BibitemOpen
  \bibfield  {author} {\bibinfo {author} {\bibfnamefont {Anupam}\ \bibnamefont
  {Garg}}\ and\ \bibinfo {author} {\bibfnamefont {David~N.}\ \bibnamefont
  {Mermin}},\ }\bibfield  {title} {\enquote {\bibinfo {title} {{F}arkas's lemma
  and the nature of reality: Statistical implications of quantum
  correlations},}\ }\href {\doibase 10.1007/BF00741645} {\bibfield  {journal}
  {\bibinfo  {journal} {Foundations of Physics}\ }\textbf {\bibinfo {volume}
  {14}},\ \bibinfo {pages} {1--39} (\bibinfo {year} {1984})}\BibitemShut
  {NoStop}%
\bibitem [{\citenamefont {Pitowsky}(1986)}]{pitowsky-86}%
  \BibitemOpen
  \bibfield  {author} {\bibinfo {author} {\bibfnamefont {Itamar}\ \bibnamefont
  {Pitowsky}},\ }\bibfield  {title} {\enquote {\bibinfo {title} {The range of
  quantum probability},}\ }\href {\doibase 10.1063/1.527066} {\bibfield
  {journal} {\bibinfo  {journal} {Journal of Mathematical Physics}\ }\textbf
  {\bibinfo {volume} {27}},\ \bibinfo {pages} {1556--1565} (\bibinfo {year}
  {1986})}\BibitemShut {NoStop}%
\bibitem [{\citenamefont {Tsirelson}(1993)}]{cirelson}%
  \BibitemOpen
  \bibfield  {author} {\bibinfo {author} {\bibfnamefont {Boris~S.}\
  \bibnamefont {Tsirelson}},\ }\bibfield  {title} {\enquote {\bibinfo {title}
  {Some results and problems on quantum {B}ell-type inequalities},}\ }\href
  {http://www.tau.ac.il/~tsirel/download/hadron.pdf} {\bibfield  {journal}
  {\bibinfo  {journal} {Hadronic Journal Supplement}\ }\textbf {\bibinfo
  {volume} {8}},\ \bibinfo {pages} {329--345} (\bibinfo {year}
  {1993})}\BibitemShut {NoStop}%
\bibitem [{\citenamefont {Svozil}(2020{\natexlab{a}})}]{svozil-2017-b}%
  \BibitemOpen
  \bibfield  {author} {\bibinfo {author} {\bibfnamefont {Karl}\ \bibnamefont
  {Svozil}},\ }\bibfield  {title} {\enquote {\bibinfo {title} {What is so
  special about quantum clicks?}}\ }\href {\doibase 10.3390/e22060602}
  {\bibfield  {journal} {\bibinfo  {journal} {Entropy}\ }\textbf {\bibinfo
  {volume} {22}},\ \bibinfo {pages} {602} (\bibinfo {year}
  {2020}{\natexlab{a}})},\ \Eprint {http://arxiv.org/abs/arXiv:1707.08915}
  {arXiv:1707.08915} \BibitemShut {NoStop}%
\bibitem [{\citenamefont {Uffink}(2011)}]{Uffink2011-UFFSPS}%
  \BibitemOpen
  \bibfield  {author} {\bibinfo {author} {\bibfnamefont {Jos}\ \bibnamefont
  {Uffink}},\ }\bibfield  {title} {\enquote {\bibinfo {title} {Subjective
  probability and statistical physics},}\ }in\ \href {\doibase
  10.1093/acprof:oso/9780199577439.003.0002} {\emph {\bibinfo {booktitle}
  {Probabilities in Physics}}},\ \bibinfo {editor} {edited by\ \bibinfo
  {editor} {\bibfnamefont {Claus}\ \bibnamefont {Beisbart}}\ and\ \bibinfo
  {editor} {\bibfnamefont {Stephan}\ \bibnamefont {Hartmann}}}\ (\bibinfo
  {publisher} {Oxford University Press},\ \bibinfo {address} {Oxford, UK},\
  \bibinfo {year} {2011})\ pp.\ \bibinfo {pages} {25--49}\BibitemShut {NoStop}%
\bibitem [{\citenamefont {Peres}(1993)}]{peres}%
  \BibitemOpen
  \bibfield  {author} {\bibinfo {author} {\bibfnamefont {Asher}\ \bibnamefont
  {Peres}},\ }\href@noop {} {\emph {\bibinfo {title} {Quantum Theory: Concepts
  and Methods}}}\ (\bibinfo  {publisher} {Kluwer Academic Publishers},\
  \bibinfo {address} {Dordrecht},\ \bibinfo {year} {1993})\BibitemShut
  {NoStop}%
\bibitem [{\citenamefont {Wright}(1990)}]{wright}%
  \BibitemOpen
  \bibfield  {author} {\bibinfo {author} {\bibfnamefont {Ron}\ \bibnamefont
  {Wright}},\ }\bibfield  {title} {\enquote {\bibinfo {title} {Generalized urn
  models},}\ }\href {\doibase 10.1007/BF01889696} {\bibfield  {journal}
  {\bibinfo  {journal} {Foundations of Physics}\ }\textbf {\bibinfo {volume}
  {20}},\ \bibinfo {pages} {881--903} (\bibinfo {year} {1990})}\BibitemShut
  {NoStop}%
\bibitem [{\citenamefont {Moore}(1956)}]{e-f-moore}%
  \BibitemOpen
  \bibfield  {author} {\bibinfo {author} {\bibfnamefont {Edward~F.}\
  \bibnamefont {Moore}},\ }\bibfield  {title} {\enquote {\bibinfo {title}
  {Gedanken-experiments on sequential machines},}\ }in\ \href {\doibase
  10.1515/9781400882618-006} {\emph {\bibinfo {booktitle} {Automata Studies.
  {(AM-34)}}}},\ \bibinfo {editor} {edited by\ \bibinfo {editor} {\bibfnamefont
  {C.~E.}\ \bibnamefont {Shannon}}\ and\ \bibinfo {editor} {\bibfnamefont
  {J.}~\bibnamefont {McCarthy}}}\ (\bibinfo  {publisher} {Princeton University
  Press},\ \bibinfo {address} {Princeton, NJ},\ \bibinfo {year} {1956})\ pp.\
  \bibinfo {pages} {129--153}\BibitemShut {NoStop}%
\bibitem [{\citenamefont {Svozil}(2005)}]{svozil-2001-eua}%
  \BibitemOpen
  \bibfield  {author} {\bibinfo {author} {\bibfnamefont {Karl}\ \bibnamefont
  {Svozil}},\ }\bibfield  {title} {\enquote {\bibinfo {title} {Logical
  equivalence between generalized urn models and finite automata},}\ }\href
  {\doibase 10.1007/s10773-005-7052-0} {\bibfield  {journal} {\bibinfo
  {journal} {International Journal of Theoretical Physics}\ }\textbf {\bibinfo
  {volume} {44}},\ \bibinfo {pages} {745--754} (\bibinfo {year} {2005})},\
  \Eprint {http://arxiv.org/abs/arXiv:quant-ph/0209136}
  {arXiv:quant-ph/0209136} \BibitemShut {NoStop}%
\bibitem [{\citenamefont {Filipp}\ and\ \citenamefont
  {Svozil}(2004)}]{filipp-svo-04-qpoly-prl}%
  \BibitemOpen
  \bibfield  {author} {\bibinfo {author} {\bibfnamefont {Stefan}\ \bibnamefont
  {Filipp}}\ and\ \bibinfo {author} {\bibfnamefont {Karl}\ \bibnamefont
  {Svozil}},\ }\bibfield  {title} {\enquote {\bibinfo {title} {Generalizing
  {T}sirelson's bound on {B}ell inequalities using a min-max principle},}\
  }\href {\doibase 10.1103/PhysRevLett.93.130407} {\bibfield  {journal}
  {\bibinfo  {journal} {Physical Review Letters}\ }\textbf {\bibinfo {volume}
  {93}},\ \bibinfo {pages} {130407} (\bibinfo {year} {2004})},\ \Eprint
  {http://arxiv.org/abs/arXiv:quant-ph/0403175} {arXiv:quant-ph/0403175}
  \BibitemShut {NoStop}%
\bibitem [{\citenamefont {Suppes}\ and\ \citenamefont
  {Zanotti}(1981)}]{Suppes-81}%
  \BibitemOpen
  \bibfield  {author} {\bibinfo {author} {\bibfnamefont {Patrick}\ \bibnamefont
  {Suppes}}\ and\ \bibinfo {author} {\bibfnamefont {Mario}\ \bibnamefont
  {Zanotti}},\ }\bibfield  {title} {\enquote {\bibinfo {title} {When are
  probabilistic explanations possible?}}\ }\href {\doibase 10.1007/BF01063886}
  {\bibfield  {journal} {\bibinfo  {journal} {Synthese}\ }\textbf {\bibinfo
  {volume} {48}},\ \bibinfo {pages} {191--199} (\bibinfo {year}
  {1981})}\BibitemShut {NoStop}%
\bibitem [{\citenamefont {Brody}(1989)}]{Brody-1989}%
  \BibitemOpen
  \bibfield  {author} {\bibinfo {author} {\bibfnamefont {T.~A.}\ \bibnamefont
  {Brody}},\ }\bibfield  {title} {\enquote {\bibinfo {title} {The
  {S}uppes-{Z}anotti theorem and the {B}ell inequalities},}\ }\href
  {https://rmf.smf.mx/ojs/rmf/article/view/2042/2010} {\bibfield  {journal}
  {\bibinfo  {journal} {Revista Mexicana de F\'{\i}sica}\ }\textbf {\bibinfo
  {volume} {35}},\ \bibinfo {pages} {170--187} (\bibinfo {year}
  {1989})}\BibitemShut {NoStop}%
\bibitem [{\citenamefont {Khrennikov}(2020)}]{Khrennikov2020}%
  \BibitemOpen
  \bibfield  {author} {\bibinfo {author} {\bibfnamefont {Andrei}\ \bibnamefont
  {Khrennikov}},\ }\bibfield  {title} {\enquote {\bibinfo {title} {Can there be
  given any meaning to contextuality without incompatibility?}}\ }\href
  {\doibase 10.1007/s10773-020-04666-z} {\bibfield  {journal} {\bibinfo
  {journal} {International Journal of Theoretical Physics}\ } (\bibinfo {year}
  {2020}),\ 10.1007/s10773-020-04666-z}\BibitemShut {NoStop}%
\bibitem [{\citenamefont {Boole}(1862)}]{Boole-62}%
  \BibitemOpen
  \bibfield  {author} {\bibinfo {author} {\bibfnamefont {George}\ \bibnamefont
  {Boole}},\ }\bibfield  {title} {\enquote {\bibinfo {title} {On the theory of
  probabilities},}\ }\href {\doibase 10.1098/rstl.1862.0015} {\bibfield
  {journal} {\bibinfo  {journal} {Philosophical Transactions of the Royal
  Society of London}\ }\textbf {\bibinfo {volume} {152}},\ \bibinfo {pages}
  {225--252} (\bibinfo {year} {1862})}\BibitemShut {NoStop}%
\bibitem [{\citenamefont {Pitowsky}(1994)}]{Pit-94}%
  \BibitemOpen
  \bibfield  {author} {\bibinfo {author} {\bibfnamefont {Itamar}\ \bibnamefont
  {Pitowsky}},\ }\bibfield  {title} {\enquote {\bibinfo {title} {{G}eorge
  {B}oole's `conditions of possible experience' and the quantum puzzle},}\
  }\href {\doibase 10.1093/bjps/45.1.95} {\bibfield  {journal} {\bibinfo
  {journal} {The British Journal for the Philosophy of Science}\ }\textbf
  {\bibinfo {volume} {45}},\ \bibinfo {pages} {95--125} (\bibinfo {year}
  {1994})}\BibitemShut {NoStop}%
\bibitem [{\citenamefont {Pitowsky}(1989)}]{pitowsky}%
  \BibitemOpen
  \bibfield  {author} {\bibinfo {author} {\bibfnamefont {Itamar}\ \bibnamefont
  {Pitowsky}},\ }\href {\doibase 10.1007/BFb0021186} {\emph {\bibinfo {title}
  {Quantum Probability --- Quantum Logic}}},\ \bibinfo {series} {Lecture Notes
  in Physics}, Vol.\ \bibinfo {volume} {321}\ (\bibinfo  {publisher}
  {Springer-Verlag},\ \bibinfo {address} {Berlin, Heidelberg},\ \bibinfo {year}
  {1989})\BibitemShut {NoStop}%
\bibitem [{\citenamefont {Ziegler}(1994)}]{ziegler}%
  \BibitemOpen
  \bibfield  {author} {\bibinfo {author} {\bibfnamefont {G{\"{u}}nter~M.}\
  \bibnamefont {Ziegler}},\ }\href@noop {} {\emph {\bibinfo {title} {Lectures
  on Polytopes}}}\ (\bibinfo  {publisher} {Springer},\ \bibinfo {address} {New
  York},\ \bibinfo {year} {1994})\BibitemShut {NoStop}%
\bibitem [{\citenamefont {Schrijver}(1998)}]{Schrijver}%
  \BibitemOpen
  \bibfield  {author} {\bibinfo {author} {\bibfnamefont {Alexander}\
  \bibnamefont {Schrijver}},\ }\href
  {http://eu.wiley.com/WileyCDA/WileyTitle/productCd-0471982326.html} {\emph
  {\bibinfo {title} {Theory of Linear and Integer Programming}}},\ Wiley Series
  in Discrete Mathematics \& Optimization\ (\bibinfo  {publisher} {John Wiley
  \& Sons},\ \bibinfo {address} {New York, Toronto, London},\ \bibinfo {year}
  {1998})\BibitemShut {NoStop}%
\bibitem [{\citenamefont {Fukuda}(2014)}]{Fukuda-techrep}%
  \BibitemOpen
  \bibfield  {author} {\bibinfo {author} {\bibfnamefont {Komei}\ \bibnamefont
  {Fukuda}},\ }\href
  {ftp://ftp.math.ethz.ch/users/fukudak/reports/polyfaq040618.pdf} {\enquote
  {\bibinfo {title} {Frequently asked questions in polyhedral computation},}\ }
  (\bibinfo {year} {2014}),\ \bibinfo {note} {accessed on July 29th,
  2017}\BibitemShut {NoStop}%
\bibitem [{\citenamefont {Pitowsky}(1991)}]{Pit-91}%
  \BibitemOpen
  \bibfield  {author} {\bibinfo {author} {\bibfnamefont {Itamar}\ \bibnamefont
  {Pitowsky}},\ }\bibfield  {title} {\enquote {\bibinfo {title} {Correlation
  polytopes their geometry and complexity},}\ }\href {\doibase
  10.1007/BF01594946} {\bibfield  {journal} {\bibinfo  {journal} {Mathematical
  Programming}\ }\textbf {\bibinfo {volume} {50}},\ \bibinfo {pages} {395--414}
  (\bibinfo {year} {1991})}\BibitemShut {NoStop}%
\bibitem [{\citenamefont {Travis}(1962)}]{travis-mt-62}%
  \BibitemOpen
  \bibfield  {author} {\bibinfo {author} {\bibfnamefont {Raymond~David}\
  \bibnamefont {Travis}},\ }\emph {\bibinfo {title} {The Logic of a Physical
  Theory}},\ \href@noop {} {Master's thesis},\ \bibinfo  {school} {Wayne State
  University}, \bibinfo {address} {Detroit, Michigan, USA} (\bibinfo {year}
  {1962}),\ \bibinfo {note} {{M}aster's {T}hesis under the supervision of David
  J. Foulis}\BibitemShut {NoStop}%
\bibitem [{\citenamefont {Greechie}(1996)}]{greechie-66-PhD}%
  \BibitemOpen
  \bibfield  {author} {\bibinfo {author} {\bibfnamefont {Richard~Joseph}\
  \bibnamefont {Greechie}},\ }\emph {\bibinfo {title} {Orthomodular
  Lattices}},\ \href {https://ufdc.ufl.edu/UF00097858/00001/pdf} {Ph.D.
  thesis},\ \bibinfo  {school} {University of Florida}, \bibinfo {address}
  {Florida, USA} (\bibinfo {year} {1996})\BibitemShut {NoStop}%
\bibitem [{\citenamefont {Troffaes}(2020)}]{pycddlib}%
  \BibitemOpen
  \bibfield  {author} {\bibinfo {author} {\bibfnamefont {Matthias}\
  \bibnamefont {Troffaes}},\ }\href {https://pypi.org/project/pycddlib/}
  {\enquote {\bibinfo {title} {{\tt pycddlib} is a {P}ython wrapper for {K}omei
  {F}ukuda's {\tt cddlib}},}\ } (\bibinfo {year} {2020}),\ \bibinfo {note}
  {accessed on December 22th, 2020}\BibitemShut {NoStop}%
\bibitem [{\citenamefont {Fukuda}(2000,2017)}]{cdd-pck}%
  \BibitemOpen
  \bibfield  {author} {\bibinfo {author} {\bibfnamefont {Komei}\ \bibnamefont
  {Fukuda}},\ }\href {http://www.inf.ethz.ch/personal/fukudak/cdd\_home/}
  {\enquote {\bibinfo {title} {{cdd} and {cddplus} homepage, {cddlib package
  cddlib-094h}},}\ } (\bibinfo {year} {2000,2017}),\ \bibinfo {note} {accessed
  on July 1st, 2017}\BibitemShut {NoStop}%
\bibitem [{\citenamefont {Motzkin}\ \emph {et~al.}(1953)\citenamefont
  {Motzkin}, \citenamefont {Raiffa}, \citenamefont {Thompson},\ and\
  \citenamefont {Thrall}}]{MRTT53}%
  \BibitemOpen
  \bibfield  {author} {\bibinfo {author} {\bibfnamefont {T.S.}\ \bibnamefont
  {Motzkin}}, \bibinfo {author} {\bibfnamefont {H.}~\bibnamefont {Raiffa}},
  \bibinfo {author} {\bibfnamefont {G.L.}\ \bibnamefont {Thompson}}, \ and\
  \bibinfo {author} {\bibfnamefont {R.M.}\ \bibnamefont {Thrall}},\ }\bibfield
  {title} {\enquote {\bibinfo {title} {The double description method},}\ }in\
  \href@noop {} {\emph {\bibinfo {booktitle} {Contributions to theory of games,
  Vol. 2}}},\ \bibinfo {editor} {edited by\ \bibinfo {editor} {\bibfnamefont
  {H.W.}\ \bibnamefont {Kuhn}}\ and\ \bibinfo {editor} {\bibnamefont
  {A.W.Tucker}}}\ (\bibinfo  {publisher} {Princeton University Press, New
  Jersey},\ \bibinfo {address} {Princeton, NJ},\ \bibinfo {year}
  {1953})\BibitemShut {NoStop}%
\bibitem [{\citenamefont {Specker}(1960)}]{specker-60}%
  \BibitemOpen
  \bibfield  {author} {\bibinfo {author} {\bibfnamefont {Ernst}\ \bibnamefont
  {Specker}},\ }\bibfield  {title} {\enquote {\bibinfo {title} {{D}ie {L}ogik
  nicht gleichzeitig entscheidbarer {A}ussagen},}\ }\href {\doibase
  10.1111/j.1746-8361.1960.tb00422.x} {\bibfield  {journal} {\bibinfo
  {journal} {Dialectica}\ }\textbf {\bibinfo {volume} {14}},\ \bibinfo {pages}
  {239--246} (\bibinfo {year} {1960})},\ \bibinfo {note} {english traslation at
  {https://arxiv.org/abs/1103.4537}},\ \Eprint
  {http://arxiv.org/abs/arXiv:1103.4537} {arXiv:1103.4537} \BibitemShut
  {NoStop}%
\bibitem [{\citenamefont {Greenberger}\ \emph {et~al.}(1989)\citenamefont
  {Greenberger}, \citenamefont {Horne},\ and\ \citenamefont {Zeilinger}}]{ghz}%
  \BibitemOpen
  \bibfield  {author} {\bibinfo {author} {\bibfnamefont {Daniel~M.}\
  \bibnamefont {Greenberger}}, \bibinfo {author} {\bibfnamefont {Mike~A.}\
  \bibnamefont {Horne}}, \ and\ \bibinfo {author} {\bibfnamefont {Anton}\
  \bibnamefont {Zeilinger}},\ }\bibfield  {title} {\enquote {\bibinfo {title}
  {Going beyond {B}ell's theorem},}\ }in\ \href {\doibase
  10.1007/978-94-017-0849-4\_10} {\emph {\bibinfo {booktitle} {Bell's Theorem,
  Quantum Theory, and Conceptions of the {U}niverse}}},\ \bibinfo {series}
  {Fundamental Theories of Physics}, Vol.~\bibinfo {volume} {37},\ \bibinfo
  {editor} {edited by\ \bibinfo {editor} {\bibfnamefont {Menas}\ \bibnamefont
  {Kafatos}}}\ (\bibinfo  {publisher} {Kluwer Academic Publishers, Springer
  Netherlands},\ \bibinfo {address} {Dordrecht},\ \bibinfo {year} {1989})\ pp.\
  \bibinfo {pages} {69--72},\ \Eprint
  {http://arxiv.org/abs/http://arxiv.org/abs/0712.0921}
  {http://arxiv.org/abs/0712.0921} \BibitemShut {NoStop}%
\bibitem [{\citenamefont {Greenberger}\ \emph {et~al.}(1990)\citenamefont
  {Greenberger}, \citenamefont {Horne}, \citenamefont {Shimony},\ and\
  \citenamefont {Zeilinger}}]{ghsz}%
  \BibitemOpen
  \bibfield  {author} {\bibinfo {author} {\bibfnamefont {Daniel~M.}\
  \bibnamefont {Greenberger}}, \bibinfo {author} {\bibfnamefont {Mike~A.}\
  \bibnamefont {Horne}}, \bibinfo {author} {\bibfnamefont {A.}~\bibnamefont
  {Shimony}}, \ and\ \bibinfo {author} {\bibfnamefont {Anton}\ \bibnamefont
  {Zeilinger}},\ }\bibfield  {title} {\enquote {\bibinfo {title} {{B}ell's
  theorem without inequalities},}\ }\href {\doibase 10.1119/1.16243} {\bibfield
   {journal} {\bibinfo  {journal} {American Journal of Physics}\ }\textbf
  {\bibinfo {volume} {58}},\ \bibinfo {pages} {1131--1143} (\bibinfo {year}
  {1990})}\BibitemShut {NoStop}%
\bibitem [{\citenamefont {Mermin}(1990)}]{mermin}%
  \BibitemOpen
  \bibfield  {author} {\bibinfo {author} {\bibfnamefont {David~N.}\
  \bibnamefont {Mermin}},\ }\bibfield  {title} {\enquote {\bibinfo {title}
  {What's wrong with these elements of reality?}}\ }\href {\doibase
  10.1063/1.2810588} {\bibfield  {journal} {\bibinfo  {journal} {Physics
  Today}\ }\textbf {\bibinfo {volume} {43}},\ \bibinfo {pages} {9--10}
  (\bibinfo {year} {1990})}\BibitemShut {NoStop}%
\bibitem [{\citenamefont {Bouwmeester}\ \emph {et~al.}(1999)\citenamefont
  {Bouwmeester}, \citenamefont {Pan}, \citenamefont {Daniell}, \citenamefont
  {Weinfurter},\ and\ \citenamefont {Zeilinger}}]{PhysRevLett.82.1345}%
  \BibitemOpen
  \bibfield  {author} {\bibinfo {author} {\bibfnamefont {Dik}\ \bibnamefont
  {Bouwmeester}}, \bibinfo {author} {\bibfnamefont {Jian-Wei}\ \bibnamefont
  {Pan}}, \bibinfo {author} {\bibfnamefont {Matthew}\ \bibnamefont {Daniell}},
  \bibinfo {author} {\bibfnamefont {Harald}\ \bibnamefont {Weinfurter}}, \ and\
  \bibinfo {author} {\bibfnamefont {Anton}\ \bibnamefont {Zeilinger}},\
  }\bibfield  {title} {\enquote {\bibinfo {title} {Observation of three-photon
  greenberger-horne-zeilinger entanglement},}\ }\href {\doibase
  10.1103/PhysRevLett.82.1345} {\bibfield  {journal} {\bibinfo  {journal}
  {Physical Review Letters}\ }\textbf {\bibinfo {volume} {82}},\ \bibinfo
  {pages} {1345--1349} (\bibinfo {year} {1999})}\BibitemShut {NoStop}%
\bibitem [{\citenamefont {Pan}\ \emph {et~al.}(2000)\citenamefont {Pan},
  \citenamefont {Bouwmeester}, \citenamefont {Daniell}, \citenamefont
  {Weinfurter},\ and\ \citenamefont {Zeilinger}}]{panbdwz}%
  \BibitemOpen
  \bibfield  {author} {\bibinfo {author} {\bibfnamefont {Jian-Wei}\
  \bibnamefont {Pan}}, \bibinfo {author} {\bibfnamefont {D.}~\bibnamefont
  {Bouwmeester}}, \bibinfo {author} {\bibfnamefont {M.}~\bibnamefont
  {Daniell}}, \bibinfo {author} {\bibfnamefont {H.}~\bibnamefont {Weinfurter}},
  \ and\ \bibinfo {author} {\bibfnamefont {Anton}\ \bibnamefont {Zeilinger}},\
  }\bibfield  {title} {\enquote {\bibinfo {title} {Experimental test of quantum
  nonlocality in three-photon {G}reenberger-{H}orne-{Z}eilinger
  entanglement},}\ }\href {\doibase 10.1038/35000514} {\bibfield  {journal}
  {\bibinfo  {journal} {Nature}\ }\textbf {\bibinfo {volume} {403}},\ \bibinfo
  {pages} {515--519} (\bibinfo {year} {2000})}\BibitemShut {NoStop}%
\bibitem [{\citenamefont {Bacon}(2006)}]{bacon-ghzgames-2006}%
  \BibitemOpen
  \bibfield  {author} {\bibinfo {author} {\bibfnamefont {Dave}\ \bibnamefont
  {Bacon}},\ }\href
  {https://courses.cs.washington.edu/courses/cse599d/06wi/lecturenotes14.pdf}
  {\enquote {\bibinfo {title} {The {GHZ}game},}\ } (\bibinfo {year} {2006}),\
  \bibinfo {note} {section {I} of {CSE 599d} -- Quantum Computing Quantum
  Entanglement and {B}ell's Theorem, lecture notes accessed on January 9th,
  2021}\BibitemShut {NoStop}%
\bibitem [{\citenamefont {Svozil}(2020{\natexlab{b}})}]{svozil-2020-ghz}%
  \BibitemOpen
  \bibfield  {author} {\bibinfo {author} {\bibfnamefont {Karl}\ \bibnamefont
  {Svozil}},\ }\href {https://arxiv.org/abs/2006.14623} {\enquote {\bibinfo
  {title} {Revisiting the {G}reenberger-{H}orne-{Z}eilinger argument in terms
  of its logical structure, orthogonality, and probabilities},}\ } (\bibinfo
  {year} {2020}{\natexlab{b}}),\ \Eprint
  {http://arxiv.org/abs/arXiv:2006.14623} {arXiv:2006.14623} \BibitemShut
  {NoStop}%
\bibitem [{\citenamefont {{Cirel'son (=Tsirel'son)}}(1980)}]{cirelson:80}%
  \BibitemOpen
  \bibfield  {author} {\bibinfo {author} {\bibfnamefont {Boris~S.}\
  \bibnamefont {{Cirel'son (=Tsirel'son)}}},\ }\bibfield  {title} {\enquote
  {\bibinfo {title} {Quantum generalizations of {B}ell's inequality},}\ }\href
  {\doibase 10.1007/BF00417500} {\bibfield  {journal} {\bibinfo  {journal}
  {Letters in Mathematical Physics}\ }\textbf {\bibinfo {volume} {4}},\
  \bibinfo {pages} {93--100} (\bibinfo {year} {1980})}\BibitemShut {NoStop}%
\bibitem [{\citenamefont {Pitowsky}\ and\ \citenamefont
  {Svozil}(2001)}]{2000-poly}%
  \BibitemOpen
  \bibfield  {author} {\bibinfo {author} {\bibfnamefont {Itamar}\ \bibnamefont
  {Pitowsky}}\ and\ \bibinfo {author} {\bibfnamefont {Karl}\ \bibnamefont
  {Svozil}},\ }\bibfield  {title} {\enquote {\bibinfo {title} {New optimal
  tests of quantum nonlocality},}\ }\href {\doibase 10.1103/PhysRevA.64.014102}
  {\bibfield  {journal} {\bibinfo  {journal} {Physical Review A}\ }\textbf
  {\bibinfo {volume} {64}},\ \bibinfo {pages} {014102} (\bibinfo {year}
  {2001})},\ \Eprint {http://arxiv.org/abs/arXiv:quant-ph/0011060}
  {arXiv:quant-ph/0011060} \BibitemShut {NoStop}%
\bibitem [{\citenamefont {Sliwa}(2003)}]{sliwa-2003}%
  \BibitemOpen
  \bibfield  {author} {\bibinfo {author} {\bibfnamefont {Cezary}\ \bibnamefont
  {Sliwa}},\ }\bibfield  {title} {\enquote {\bibinfo {title} {Symmetries of the
  {B}ell correlation inequalities},}\ }\href {\doibase
  10.1016/S0375-9601(03)01115-0} {\bibfield  {journal} {\bibinfo  {journal}
  {Physics Letters A}\ }\textbf {\bibinfo {volume} {317}},\ \bibinfo {pages}
  {165--168} (\bibinfo {year} {2003})},\ \Eprint
  {http://arxiv.org/abs/arXiv:quant-ph/0305190} {arXiv:quant-ph/0305190}
  \BibitemShut {NoStop}%
\bibitem [{\citenamefont {Colins}\ and\ \citenamefont
  {Gisin}(2004)}]{collins-gisin-2003}%
  \BibitemOpen
  \bibfield  {author} {\bibinfo {author} {\bibfnamefont {Daniel}\ \bibnamefont
  {Colins}}\ and\ \bibinfo {author} {\bibfnamefont {Nicolas}\ \bibnamefont
  {Gisin}},\ }\bibfield  {title} {\enquote {\bibinfo {title} {A relevant two
  qbit {B}ell inequality inequivalent to the {CHSH} inequality},}\ }\href
  {\doibase 10.1088/0305-4470/37/5/021} {\bibfield  {journal} {\bibinfo
  {journal} {Journal of Physics A: Math. Gen.}\ }\textbf {\bibinfo {volume}
  {37}},\ \bibinfo {pages} {1775--1787} (\bibinfo {year} {2004})},\ \Eprint
  {http://arxiv.org/abs/arXiv:quant-ph/0306129} {arXiv:quant-ph/0306129}
  \BibitemShut {NoStop}%
\bibitem [{\citenamefont {Dzhafarov}\ \emph {et~al.}(2017)\citenamefont
  {Dzhafarov}, \citenamefont {Cervantes},\ and\ \citenamefont
  {Kujala}}]{Dzhafarov-2017}%
  \BibitemOpen
  \bibfield  {author} {\bibinfo {author} {\bibfnamefont {Ehtibar~N.}\
  \bibnamefont {Dzhafarov}}, \bibinfo {author} {\bibfnamefont {Victor~H.}\
  \bibnamefont {Cervantes}}, \ and\ \bibinfo {author} {\bibfnamefont
  {Janne~V.}\ \bibnamefont {Kujala}},\ }\bibfield  {title} {\enquote {\bibinfo
  {title} {Contextuality in canonical systems of random variables},}\ }\href
  {\doibase 10.1098/rsta.2016.0389} {\bibfield  {journal} {\bibinfo  {journal}
  {Philosophical Transactions of the Royal Society A: Mathematical, Physical
  and Engineering Sciences}\ }\textbf {\bibinfo {volume} {375}},\ \bibinfo
  {pages} {20160389} (\bibinfo {year} {2017})},\ \Eprint
  {http://arxiv.org/abs/arXiv:1703.01252} {arXiv:1703.01252} \BibitemShut
  {NoStop}%
\bibitem [{\citenamefont {Abramsky}(2018)}]{Abramsky2018}%
  \BibitemOpen
  \bibfield  {author} {\bibinfo {author} {\bibfnamefont {Samson}\ \bibnamefont
  {Abramsky}},\ }\bibfield  {title} {\enquote {\bibinfo {title} {Contextuality:
  At the borders of paradox},}\ }in\ \href {\doibase
  10.1093/oso/9780198748991.003.0011} {\emph {\bibinfo {booktitle} {Categories
  for the Working Philosopher}}},\ \bibinfo {editor} {edited by\ \bibinfo
  {editor} {\bibfnamefont {Elaine}\ \bibnamefont {Landry}}}\ (\bibinfo
  {publisher} {Oxford University Press},\ \bibinfo {address} {Oxford, UK},\
  \bibinfo {year} {2018})\ pp.\ \bibinfo {pages} {262--285},\ \Eprint
  {http://arxiv.org/abs/arXiv:2011.04899} {arXiv:2011.04899} \BibitemShut
  {NoStop}%
\bibitem [{\citenamefont {Grangier}(2002)}]{Grangier_2002}%
  \BibitemOpen
  \bibfield  {author} {\bibinfo {author} {\bibfnamefont {Philippe}\
  \bibnamefont {Grangier}},\ }\bibfield  {title} {\enquote {\bibinfo {title}
  {Contextual objectivity: a realistic interpretation of quantum mechanics},}\
  }\href {\doibase 10.1088/0143-0807/23/3/312} {\bibfield  {journal} {\bibinfo
  {journal} {European Journal of Physics}\ }\textbf {\bibinfo {volume} {23}},\
  \bibinfo {pages} {331--337} (\bibinfo {year} {2002})},\ \Eprint
  {http://arxiv.org/abs/arXiv:quant-ph/0012122} {arXiv:quant-ph/0012122}
  \BibitemShut {NoStop}%
\bibitem [{\citenamefont {Khrennikov}(2017)}]{Khrennikov2017}%
  \BibitemOpen
  \bibfield  {author} {\bibinfo {author} {\bibfnamefont {Andrei}\ \bibnamefont
  {Khrennikov}},\ }\bibfield  {title} {\enquote {\bibinfo {title} {{B}ohr
  against {B}ell: complementarity versus nonlocality},}\ }\href {\doibase
  10.1515/phys-2017-0086} {\bibfield  {journal} {\bibinfo  {journal} {Open
  Physics}\ }\textbf {\bibinfo {volume} {15}},\ \bibinfo {pages} {734--738}
  (\bibinfo {year} {2017})}\BibitemShut {NoStop}%
\bibitem [{\citenamefont {Jaeger}(2019)}]{Jaeger2019}%
  \BibitemOpen
  \bibfield  {author} {\bibinfo {author} {\bibfnamefont {Gregg}\ \bibnamefont
  {Jaeger}},\ }\bibfield  {title} {\enquote {\bibinfo {title} {Quantum
  contextuality in the copenhagen approach},}\ }\href {\doibase
  10.1098/rsta.2019.0025} {\bibfield  {journal} {\bibinfo  {journal}
  {Philosophical Transactions of the Royal Society A: Mathematical, Physical
  and Engineering Sciences}\ }\textbf {\bibinfo {volume} {377}},\ \bibinfo
  {pages} {20190025} (\bibinfo {year} {2019})}\BibitemShut {NoStop}%
\bibitem [{\citenamefont {Jaeger}(2020)}]{Jaeger2020}%
  \BibitemOpen
  \bibfield  {author} {\bibinfo {author} {\bibfnamefont {Gregg}\ \bibnamefont
  {Jaeger}},\ }\bibfield  {title} {\enquote {\bibinfo {title} {Quantum
  contextuality and indeterminacy},}\ }\href {\doibase 10.3390/e22080867}
  {\bibfield  {journal} {\bibinfo  {journal} {Entropy}\ }\textbf {\bibinfo
  {volume} {22}},\ \bibinfo {pages} {867} (\bibinfo {year} {2020})}\BibitemShut
  {NoStop}%
\bibitem [{\citenamefont {Auff\'eves}\ and\ \citenamefont
  {Grangier}(2018)}]{Auffeves-Grangier-2018}%
  \BibitemOpen
  \bibfield  {author} {\bibinfo {author} {\bibfnamefont {Alexia}\ \bibnamefont
  {Auff\'eves}}\ and\ \bibinfo {author} {\bibfnamefont {Philippe}\ \bibnamefont
  {Grangier}},\ }\bibfield  {title} {\enquote {\bibinfo {title}
  {Extracontextuality and extravalence in quantum mechanics},}\ }\href
  {\doibase 10.1098/rsta.2017.0311} {\bibfield  {journal} {\bibinfo  {journal}
  {Philosophical Transactions of the Royal Society {A}: Mathematical, Physical
  and Engineering Sciences}\ }\textbf {\bibinfo {volume} {376}},\ \bibinfo
  {pages} {20170311} (\bibinfo {year} {2018})},\ \Eprint
  {http://arxiv.org/abs/arXiv:1801.01398} {arXiv:1801.01398} \BibitemShut
  {NoStop}%
\bibitem [{\citenamefont {Auff\`eves}\ and\ \citenamefont
  {Grangier}(2020)}]{Auffves2020}%
  \BibitemOpen
  \bibfield  {author} {\bibinfo {author} {\bibfnamefont {Alexia}\ \bibnamefont
  {Auff\`eves}}\ and\ \bibinfo {author} {\bibfnamefont {Philippe}\ \bibnamefont
  {Grangier}},\ }\bibfield  {title} {\enquote {\bibinfo {title} {Deriving
  born's rule from an inference to the best explanation},}\ }\href {\doibase
  10.1007/s10701-020-00326-8} {\bibfield  {journal} {\bibinfo  {journal}
  {Foundations of Physics}\ }\textbf {\bibinfo {volume} {50}},\ \bibinfo
  {pages} {1781--1793} (\bibinfo {year} {2020})},\ \Eprint
  {http://arxiv.org/abs/arXiv:1910.13738} {arXiv:1910.13738} \BibitemShut
  {NoStop}%
\bibitem [{\citenamefont {Grangier}(2020)}]{Grangier-2020}%
  \BibitemOpen
  \bibfield  {author} {\bibinfo {author} {\bibfnamefont {Philippe}\
  \bibnamefont {Grangier}},\ }\href {https://arxiv.org/abs/2003.03121}
  {\enquote {\bibinfo {title} {Completing the quantum formalism in a
  contextually objective framework},}\ } (\bibinfo {year} {2020}),\ \bibinfo
  {note} {preprint arXiv:2003.03121},\ \Eprint
  {http://arxiv.org/abs/arXiv:2003.03121} {arXiv:2003.03121} \BibitemShut
  {NoStop}%
\bibitem [{\citenamefont {Budroni}\ \emph {et~al.}(2021)\citenamefont
  {Budroni}, \citenamefont {Cabello}, \citenamefont {G\"uhne}, \citenamefont
  {Kleinmann},\ and\ \citenamefont {\r{A}ke
  Larsson}}]{cabello2021contextuality}%
  \BibitemOpen
  \bibfield  {author} {\bibinfo {author} {\bibfnamefont {Costantino}\
  \bibnamefont {Budroni}}, \bibinfo {author} {\bibfnamefont {Ad\'an}\
  \bibnamefont {Cabello}}, \bibinfo {author} {\bibfnamefont {Otfried}\
  \bibnamefont {G\"uhne}}, \bibinfo {author} {\bibfnamefont {Matthias}\
  \bibnamefont {Kleinmann}}, \ and\ \bibinfo {author} {\bibfnamefont {Jan}\
  \bibnamefont {\r{A}ke Larsson}},\ }\href {https://arxiv.org/abs/2102.13036}
  {\enquote {\bibinfo {title} {Quantum contextuality},}\ } (\bibinfo {year}
  {2021}),\ \Eprint {http://arxiv.org/abs/2102.13036} {arXiv:2102.13036
  [quant-ph]} \BibitemShut {NoStop}%
\bibitem [{\citenamefont {Svozil}(2021)}]{svozil-2021-context}%
  \BibitemOpen
  \bibfield  {author} {\bibinfo {author} {\bibfnamefont {Karl}\ \bibnamefont
  {Svozil}},\ }\href {https://arxiv.org/abs/2103.06110} {\enquote {\bibinfo
  {title} {Varieties of contextuality emphasizing (non)embeddability},}\ }
  (\bibinfo {year} {2021}),\ \Eprint {http://arxiv.org/abs/arXiv:2103.06110}
  {arXiv:2103.06110} \BibitemShut {NoStop}%
\bibitem [{\citenamefont {Cabello}(2008)}]{cabello:210401}%
  \BibitemOpen
  \bibfield  {author} {\bibinfo {author} {\bibfnamefont {Ad\'an}\ \bibnamefont
  {Cabello}},\ }\bibfield  {title} {\enquote {\bibinfo {title} {Experimentally
  testable state-independent quantum contextuality},}\ }\href {\doibase
  10.1103/PhysRevLett.101.210401} {\bibfield  {journal} {\bibinfo  {journal}
  {Physical Review Letters}\ }\textbf {\bibinfo {volume} {101}},\ \bibinfo
  {eid} {210401} (\bibinfo {year} {2008})},\ \Eprint
  {http://arxiv.org/abs/arXiv:0808.2456} {arXiv:0808.2456} \BibitemShut
  {NoStop}%
\bibitem [{\citenamefont {Cabello}(2020)}]{cabello2020converting}%
  \BibitemOpen
  \bibfield  {author} {\bibinfo {author} {\bibfnamefont {Ad\'an}\ \bibnamefont
  {Cabello}},\ }\href {https://arxiv.org/abs/2011.13790} {\enquote {\bibinfo
  {title} {Converting contextuality into nonlocality},}\ } (\bibinfo {year}
  {2020}),\ \Eprint {http://arxiv.org/abs/2011.13790} {arXiv:2011.13790
  [quant-ph]} \BibitemShut {NoStop}%
\bibitem [{\citenamefont {Svozil}(2009)}]{svozil:040102}%
  \BibitemOpen
  \bibfield  {author} {\bibinfo {author} {\bibfnamefont {Karl}\ \bibnamefont
  {Svozil}},\ }\bibfield  {title} {\enquote {\bibinfo {title} {Proposed direct
  test of a certain type of noncontextuality in quantum mechanics},}\ }\href
  {\doibase 10.1103/PhysRevA.80.040102} {\bibfield  {journal} {\bibinfo
  {journal} {Physical Review A}\ }\textbf {\bibinfo {volume} {80}},\ \bibinfo
  {eid} {040102} (\bibinfo {year} {2009})}\BibitemShut {NoStop}%
\bibitem [{\citenamefont {Svozil}(2012)}]{svozil-2011-enough}%
  \BibitemOpen
  \bibfield  {author} {\bibinfo {author} {\bibfnamefont {Karl}\ \bibnamefont
  {Svozil}},\ }\bibfield  {title} {\enquote {\bibinfo {title} {How much
  contextuality?}}\ }\href {\doibase 10.1007/s11047-012-9318-9} {\bibfield
  {journal} {\bibinfo  {journal} {Natural Computing}\ }\textbf {\bibinfo
  {volume} {11}},\ \bibinfo {pages} {261--265} (\bibinfo {year} {2012})},\
  \Eprint {http://arxiv.org/abs/arXiv:1103.3980} {arXiv:1103.3980} \BibitemShut
  {NoStop}%
\bibitem [{\citenamefont {Abramsky}\ \emph {et~al.}(2017)\citenamefont
  {Abramsky}, \citenamefont {Barbosa},\ and\ \citenamefont
  {Mansfield}}]{Abramsky-2017}%
  \BibitemOpen
  \bibfield  {author} {\bibinfo {author} {\bibfnamefont {Samson}\ \bibnamefont
  {Abramsky}}, \bibinfo {author} {\bibfnamefont {Rui~Soares}\ \bibnamefont
  {Barbosa}}, \ and\ \bibinfo {author} {\bibfnamefont {Shane}\ \bibnamefont
  {Mansfield}},\ }\bibfield  {title} {\enquote {\bibinfo {title} {Contextual
  fraction as a measure of contextuality},}\ }\href {\doibase
  10.1103/PhysRevLett.119.050504} {\bibfield  {journal} {\bibinfo  {journal}
  {Phys. Rev. Lett.}\ }\textbf {\bibinfo {volume} {119}},\ \bibinfo {pages}
  {050504} (\bibinfo {year} {2017})}\BibitemShut {NoStop}%
\bibitem [{\citenamefont {Kujala}\ and\ \citenamefont
  {Dzhafarov}(2019)}]{KujalaDzhafarov-2019}%
  \BibitemOpen
  \bibfield  {author} {\bibinfo {author} {\bibfnamefont {Janne~V.}\
  \bibnamefont {Kujala}}\ and\ \bibinfo {author} {\bibfnamefont {Ehtibar~N.}\
  \bibnamefont {Dzhafarov}},\ }\bibfield  {title} {\enquote {\bibinfo {title}
  {Measures of contextuality and non-contextuality},}\ }\href {\doibase
  10.1098/rsta.2019.0149} {\bibfield  {journal} {\bibinfo  {journal}
  {Philosophical Transactions of the Royal Society A. Mathematical, Physical
  and Engineering Sciences}\ }\textbf {\bibinfo {volume} {377}},\ \bibinfo
  {pages} {20190149, 16} (\bibinfo {year} {2019})}\BibitemShut {NoStop}%
\bibitem [{\citenamefont {Bohr}(1949)}]{bohr-1949}%
  \BibitemOpen
  \bibfield  {author} {\bibinfo {author} {\bibfnamefont {Niels}\ \bibnamefont
  {Bohr}},\ }\bibfield  {title} {\enquote {\bibinfo {title} {Discussion with
  {E}instein on epistemological problems in atomic physics},}\ }in\ \href
  {\doibase 10.1016/S1876-0503(08)70379-7} {\emph {\bibinfo {booktitle}
  {{A}lbert {E}instein: Philosopher-Scientist}}},\ \bibinfo {editor} {edited
  by\ \bibinfo {editor} {\bibfnamefont {P.~A.}\ \bibnamefont {Schilpp}}}\
  (\bibinfo  {publisher} {The Library of Living Philosophers},\ \bibinfo
  {address} {Evanston, Ill.},\ \bibinfo {year} {1949})\ pp.\ \bibinfo {pages}
  {200--241}\BibitemShut {NoStop}%
\bibitem [{\citenamefont {Bell}(1966)}]{bell-66}%
  \BibitemOpen
  \bibfield  {author} {\bibinfo {author} {\bibfnamefont {John~Stuard}\
  \bibnamefont {Bell}},\ }\bibfield  {title} {\enquote {\bibinfo {title} {On
  the problem of hidden variables in quantum mechanics},}\ }\href {\doibase
  10.1103/RevModPhys.38.447} {\bibfield  {journal} {\bibinfo  {journal}
  {Reviews of Modern Physics}\ }\textbf {\bibinfo {volume} {38}},\ \bibinfo
  {pages} {447--452} (\bibinfo {year} {1966})}\BibitemShut {NoStop}%
\bibitem [{\citenamefont {Bohr}(1939)}]{bohr:39caus}%
  \BibitemOpen
  \bibfield  {author} {\bibinfo {author} {\bibfnamefont {Niels}\ \bibnamefont
  {Bohr}},\ }\bibfield  {title} {\enquote {\bibinfo {title} {The causality
  problem in atomic physics},}\ }in\ \href@noop {} {\emph {\bibinfo {booktitle}
  {New Theories in Physics}}}\ (\bibinfo  {publisher} {International Institute
  of Intellectual Co-operation},\ \bibinfo {address} {Paris},\ \bibinfo {year}
  {1939})\ pp.\ \bibinfo {pages} {11--30},\ \bibinfo {note} {conference
  organized in collaboration with the International Union of Physics and the
  Polish Intellectual Co-operation Committee, Warsaw, May 30th--June 3rd 1938,
  reprinted in~\cite{Bohr-CW7-1996-cpip}}\BibitemShut {NoStop}%
\bibitem [{\citenamefont {Bohr}(1996)}]{Bohr-CW7-1996-cpip}%
  \BibitemOpen
  \bibfield  {author} {\bibinfo {author} {\bibfnamefont {Niels}\ \bibnamefont
  {Bohr}},\ }\bibfield  {title} {\enquote {\bibinfo {title} {The causality
  problem in atomic physics},}\ }in\ \href {\doibase
  10.1016/s1876-0503(08)70376-1} {\emph {\bibinfo {booktitle} {Niels Bohr
  Collected Works}}},\ Vol.~\bibinfo {volume} {7}\ (\bibinfo  {publisher}
  {Elsevier},\ \bibinfo {year} {1996})\ pp.\ \bibinfo {pages}
  {299--322}\BibitemShut {NoStop}%
\bibitem [{\citenamefont {Hilgevoord}\ and\ \citenamefont
  {Uffink}(2016)}]{sep-qt-uncertainty}%
  \BibitemOpen
  \bibfield  {author} {\bibinfo {author} {\bibfnamefont {Jan}\ \bibnamefont
  {Hilgevoord}}\ and\ \bibinfo {author} {\bibfnamefont {Jos}\ \bibnamefont
  {Uffink}},\ }\bibfield  {title} {\enquote {\bibinfo {title} {The uncertainty
  principle},}\ }in\ \href
  {https://plato.stanford.edu/archives/win2016/entries/qt-uncertainty/} {\emph
  {\bibinfo {booktitle} {The {S}tanford Encyclopedia of Philosophy}}},\
  \bibinfo {editor} {edited by\ \bibinfo {editor} {\bibfnamefont {Edward~N.}\
  \bibnamefont {Zalta}}}\ (\bibinfo  {publisher} {Metaphysics Research Lab,
  Stanford University},\ \bibinfo {year} {2016})\ \bibinfo {edition} {winter
  2016}\ ed.\BibitemShut {Stop}%
\bibitem [{\citenamefont {Ozawa}(2003)}]{Ozawa2003}%
  \BibitemOpen
  \bibfield  {author} {\bibinfo {author} {\bibfnamefont {Masanao}\ \bibnamefont
  {Ozawa}},\ }\bibfield  {title} {\enquote {\bibinfo {title} {Universally valid
  reformulation of the {H}eisenberg uncertainty principle on noise and
  disturbance in measurement},}\ }\href {\doibase 10.1103/physreva.67.042105}
  {\bibfield  {journal} {\bibinfo  {journal} {Physical Review A}\ }\textbf
  {\bibinfo {volume} {67}} (\bibinfo {year} {2003}),\
  10.1103/physreva.67.042105}\BibitemShut {NoStop}%
\bibitem [{\citenamefont {Glauber}(1986{\natexlab{a}})}]{glauber}%
  \BibitemOpen
  \bibfield  {author} {\bibinfo {author} {\bibfnamefont {Roy~J.}\ \bibnamefont
  {Glauber}},\ }\bibfield  {title} {\enquote {\bibinfo {title} {Amplifiers,
  attenuators and the quantum theory of measurement},}\ }in\ \href@noop {}
  {\emph {\bibinfo {booktitle} {Frontiers in Quantum Optics}}},\ \bibinfo
  {editor} {edited by\ \bibinfo {editor} {\bibfnamefont {E.~R.}\ \bibnamefont
  {Pikes}}\ and\ \bibinfo {editor} {\bibfnamefont {S.}~\bibnamefont {Sarkar}}}\
  (\bibinfo  {publisher} {Adam Hilger},\ \bibinfo {address} {Bristol},\
  \bibinfo {year} {1986})\BibitemShut {NoStop}%
\bibitem [{\citenamefont {Glauber}(2007)}]{glauber-collected-cat}%
  \BibitemOpen
  \bibfield  {author} {\bibinfo {author} {\bibfnamefont {Roy~J.}\ \bibnamefont
  {Glauber}},\ }\enquote {\bibinfo {title} {Amplifiers, attenuators and
  {S}chr\"odingers cat},}\ in\ \href {\doibase 10.1002/9783527610075.ch14}
  {\emph {\bibinfo {booktitle} {Quantum Theory of Optical Coherence}}}\
  (\bibinfo  {publisher} {Wiley-VCH Verlag GmbH \& Co. KGaA},\ \bibinfo {year}
  {2007})\ pp.\ \bibinfo {pages} {537--576}\BibitemShut {NoStop}%
\bibitem [{\citenamefont {Glauber}(1986{\natexlab{b}})}]{Glauber-cat-86}%
  \BibitemOpen
  \bibfield  {author} {\bibinfo {author} {\bibfnamefont {Roy~J.}\ \bibnamefont
  {Glauber}},\ }\bibfield  {title} {\enquote {\bibinfo {title} {Amplifiers,
  attenuators, and schr\"odinger's cat},}\ }\href {\doibase
  10.1111/j.1749-6632.1986.tb12437.x} {\bibfield  {journal} {\bibinfo
  {journal} {Annals of the New York Academy of Sciences}\ }\textbf {\bibinfo
  {volume} {480}},\ \bibinfo {pages} {336--372} (\bibinfo {year}
  {1986}{\natexlab{b}})}\BibitemShut {NoStop}%
\bibitem [{\citenamefont {Englert}\ \emph {et~al.}(1988)\citenamefont
  {Englert}, \citenamefont {Schwinger},\ and\ \citenamefont
  {Scully}}]{engrt-sg-I}%
  \BibitemOpen
  \bibfield  {author} {\bibinfo {author} {\bibfnamefont {Berthold-Georg}\
  \bibnamefont {Englert}}, \bibinfo {author} {\bibfnamefont {Julian}\
  \bibnamefont {Schwinger}}, \ and\ \bibinfo {author} {\bibfnamefont
  {Marlan~O.}\ \bibnamefont {Scully}},\ }\bibfield  {title} {\enquote {\bibinfo
  {title} {Is spin coherence like {H}umpty-{D}umpty? {I}. {S}implified
  treatment},}\ }\href {\doibase 10.1007/BF01909939} {\bibfield  {journal}
  {\bibinfo  {journal} {Foundations of Physics}\ }\textbf {\bibinfo {volume}
  {18}},\ \bibinfo {pages} {1045--1056} (\bibinfo {year} {1988})}\BibitemShut
  {NoStop}%
\bibitem [{\citenamefont {Schwinger}\ \emph {et~al.}(1988)\citenamefont
  {Schwinger}, \citenamefont {Scully},\ and\ \citenamefont
  {Englert}}]{engrt-sg-II}%
  \BibitemOpen
  \bibfield  {author} {\bibinfo {author} {\bibfnamefont {Julian}\ \bibnamefont
  {Schwinger}}, \bibinfo {author} {\bibfnamefont {Marlan~O.}\ \bibnamefont
  {Scully}}, \ and\ \bibinfo {author} {\bibfnamefont {Berthold-Georg}\
  \bibnamefont {Englert}},\ }\bibfield  {title} {\enquote {\bibinfo {title} {Is
  spin coherence like {H}umpty-{D}umpty? {II}. {G}eneral theory},}\ }\href
  {\doibase 10.1007/BF01384847} {\bibfield  {journal} {\bibinfo  {journal}
  {Zeitschrift f{\"u}r Physik D: Atoms, Molecules and Clusters}\ }\textbf
  {\bibinfo {volume} {10}},\ \bibinfo {pages} {135--144} (\bibinfo {year}
  {1988})}\BibitemShut {NoStop}%
\bibitem [{\citenamefont {Svozil}(2004)}]{svozil-2003-garda}%
  \BibitemOpen
  \bibfield  {author} {\bibinfo {author} {\bibfnamefont {Karl}\ \bibnamefont
  {Svozil}},\ }\bibfield  {title} {\enquote {\bibinfo {title} {Quantum
  information via state partitions and the context translation principle},}\
  }\href {\doibase 10.1080/09500340410001664179} {\bibfield  {journal}
  {\bibinfo  {journal} {Journal of Modern Optics}\ }\textbf {\bibinfo {volume}
  {51}},\ \bibinfo {pages} {811--819} (\bibinfo {year} {2004})},\ \Eprint
  {http://arxiv.org/abs/arXiv:quant-ph/0308110} {arXiv:quant-ph/0308110}
  \BibitemShut {NoStop}%
\bibitem [{\citenamefont {Busch}(2003)}]{Busch-2003}%
  \BibitemOpen
  \bibfield  {author} {\bibinfo {author} {\bibfnamefont {Paul}\ \bibnamefont
  {Busch}},\ }\bibfield  {title} {\enquote {\bibinfo {title} {Quantum states
  and generalized observables: a simple proof of {G}leason's theorem},}\ }\href
  {\doibase 10.1103/PhysRevLett.91.120403} {\bibfield  {journal} {\bibinfo
  {journal} {Physical Review Letters}\ }\textbf {\bibinfo {volume} {91}},\
  \bibinfo {pages} {120403, 4} (\bibinfo {year} {2003})}\BibitemShut {NoStop}%
\bibitem [{\citenamefont {Caves}\ \emph {et~al.}(2004)\citenamefont {Caves},
  \citenamefont {Fuchs}, \citenamefont {Manne},\ and\ \citenamefont
  {Renes}}]{caves-fuchs-2004}%
  \BibitemOpen
  \bibfield  {author} {\bibinfo {author} {\bibfnamefont {Carlton~M.}\
  \bibnamefont {Caves}}, \bibinfo {author} {\bibfnamefont {Christopher~A.}\
  \bibnamefont {Fuchs}}, \bibinfo {author} {\bibfnamefont {Kiran~K.}\
  \bibnamefont {Manne}}, \ and\ \bibinfo {author} {\bibfnamefont {Joseph~M.}\
  \bibnamefont {Renes}},\ }\bibfield  {title} {\enquote {\bibinfo {title}
  {Gleason-type derivations of the quantum probability rule for generalized
  measurements},}\ }\href {\doibase 10.1023/B:FOOP.0000019581.00318.a5}
  {\bibfield  {journal} {\bibinfo  {journal} {Foundations of Physics. An
  International Journal Devoted to the Conceptual Bases and Fundamental
  Theories of Modern Physics}\ }\textbf {\bibinfo {volume} {34}},\ \bibinfo
  {pages} {193--209} (\bibinfo {year} {2004})}\BibitemShut {NoStop}%
\bibitem [{\citenamefont {Granstr\"om}(2006)}]{Granstrom-mt}%
  \BibitemOpen
  \bibfield  {author} {\bibinfo {author} {\bibfnamefont {Helena}\ \bibnamefont
  {Granstr\"om}},\ }\emph {\bibinfo {title} {{G}leason's theorem}},\ \href
  {http://3dhouse.se/ingemar/exjobb/helena-master.pdf} {Master's thesis},\
  \bibinfo  {school} {Stockholm University} (\bibinfo {year}
  {2006})\BibitemShut {NoStop}%
\bibitem [{\citenamefont {Wright}(2019)}]{wright-Victoria}%
  \BibitemOpen
  \bibfield  {author} {\bibinfo {author} {\bibfnamefont {Victoria~J}\
  \bibnamefont {Wright}},\ }\emph {\bibinfo {title} {{G}leason-type theorems
  and general probabilistic theories}},\ \href
  {http://etheses.whiterose.ac.uk/25354/} {Ph.D. thesis},\ \bibinfo  {school}
  {University of York} (\bibinfo {year} {2019})\BibitemShut {NoStop}%
\bibitem [{\citenamefont {Wright}\ and\ \citenamefont
  {Weigert}(2019{\natexlab{a}})}]{Wright_2019}%
  \BibitemOpen
  \bibfield  {author} {\bibinfo {author} {\bibfnamefont {Victoria~J}\
  \bibnamefont {Wright}}\ and\ \bibinfo {author} {\bibfnamefont {Stefan}\
  \bibnamefont {Weigert}},\ }\bibfield  {title} {\enquote {\bibinfo {title} {A
  gleason-type theorem for qubits based on mixtures of projective
  measurements},}\ }\href {\doibase 10.1088/1751-8121/aaf93d} {\bibfield
  {journal} {\bibinfo  {journal} {Journal of Physics A: Mathematical and
  Theoretical}\ }\textbf {\bibinfo {volume} {52}},\ \bibinfo {pages} {055301}
  (\bibinfo {year} {2019}{\natexlab{a}})}\BibitemShut {NoStop}%
\bibitem [{\citenamefont {Wright}\ and\ \citenamefont
  {Weigert}(2019{\natexlab{b}})}]{Wright2019}%
  \BibitemOpen
  \bibfield  {author} {\bibinfo {author} {\bibfnamefont {Victoria~J.}\
  \bibnamefont {Wright}}\ and\ \bibinfo {author} {\bibfnamefont {Stefan}\
  \bibnamefont {Weigert}},\ }\bibfield  {title} {\enquote {\bibinfo {title}
  {{G}leason-type theorems from {C}auchy's {F}unctional {E}quation},}\ }\href
  {\doibase 10.1007/s10701-019-00275-x} {\bibfield  {journal} {\bibinfo
  {journal} {Foundations of Physics}\ }\textbf {\bibinfo {volume} {49}},\
  \bibinfo {pages} {594--606} (\bibinfo {year}
  {2019}{\natexlab{b}})}\BibitemShut {NoStop}%
\bibitem [{\citenamefont {Svozil}(2020{\natexlab{c}})}]{svozil-2018-b}%
  \BibitemOpen
  \bibfield  {author} {\bibinfo {author} {\bibfnamefont {Karl}\ \bibnamefont
  {Svozil}},\ }\bibfield  {title} {\enquote {\bibinfo {title} {Faithful
  orthogonal representations of graphs from partition logics},}\ }\href
  {\doibase 10.1007/s00500-019-04425-1} {\bibfield  {journal} {\bibinfo
  {journal} {Soft Computing}\ }\textbf {\bibinfo {volume} {24}},\ \bibinfo
  {pages} {10239--10245} (\bibinfo {year} {2020}{\natexlab{c}})},\ \Eprint
  {http://arxiv.org/abs/arXiv:1810.10423} {arXiv:1810.10423} \BibitemShut
  {NoStop}%
\bibitem [{\citenamefont {Kochen}\ and\ \citenamefont
  {Specker}(1965)}]{kochen2}%
  \BibitemOpen
  \bibfield  {author} {\bibinfo {author} {\bibfnamefont {Simon}\ \bibnamefont
  {Kochen}}\ and\ \bibinfo {author} {\bibfnamefont {Ernst~P.}\ \bibnamefont
  {Specker}},\ }\bibfield  {title} {\enquote {\bibinfo {title} {Logical
  structures arising in quantum theory},}\ }in\ \href
  {https://www.elsevier.com/books/the-theory-of-models/addison/978-0-7204-2233-7}
  {\emph {\bibinfo {booktitle} {The Theory of Models, {P}roceedings of the 1963
  International Symposium at {B}erkeley}}}\ (\bibinfo  {publisher} {North
  Holland},\ \bibinfo {address} {Amsterdam, New York, Oxford},\ \bibinfo {year}
  {1965})\ pp.\ \bibinfo {pages} {177--189},\ \bibinfo {note} {reprinted in
  Ref.~\cite[pp. 209-221]{specker-ges}}\BibitemShut {NoStop}%
\bibitem [{\citenamefont {Pitowsky}(2003)}]{Pitowsky2003395}%
  \BibitemOpen
  \bibfield  {author} {\bibinfo {author} {\bibfnamefont {Itamar}\ \bibnamefont
  {Pitowsky}},\ }\bibfield  {title} {\enquote {\bibinfo {title} {Betting on the
  outcomes of measurements: a bayesian theory of quantum probability},}\ }\href
  {\doibase 10.1016/S1355-2198(03)00035-2} {\bibfield  {journal} {\bibinfo
  {journal} {Studies in History and Philosophy of Science Part B: Studies in
  History and Philosophy of Modern Physics}\ }\textbf {\bibinfo {volume}
  {34}},\ \bibinfo {pages} {395--414} (\bibinfo {year} {2003})},\ \bibinfo
  {note} {quantum Information and Computation},\ \Eprint
  {http://arxiv.org/abs/arXiv:quant-ph/0208121} {arXiv:quant-ph/0208121}
  \BibitemShut {NoStop}%
\bibitem [{\citenamefont {Pitowsky}(2006)}]{pitowsky-06}%
  \BibitemOpen
  \bibfield  {author} {\bibinfo {author} {\bibfnamefont {Itamar}\ \bibnamefont
  {Pitowsky}},\ }\bibfield  {title} {\enquote {\bibinfo {title} {Quantum
  mechanics as a theory of probability},}\ }in\ \href {\doibase
  10.1007/1-4020-4876-9_10} {\emph {\bibinfo {booktitle} {Physical Theory and
  its Interpretation}}},\ \bibinfo {series} {The Western Ontario Series in
  Philosophy of Science}, Vol.~\bibinfo {volume} {72},\ \bibinfo {editor}
  {edited by\ \bibinfo {editor} {\bibfnamefont {William}\ \bibnamefont
  {Demopoulos}}\ and\ \bibinfo {editor} {\bibfnamefont {Itamar}\ \bibnamefont
  {Pitowsky}}}\ (\bibinfo  {publisher} {Springer Netherlands},\ \bibinfo {year}
  {2006})\ pp.\ \bibinfo {pages} {213--240},\ \Eprint
  {http://arxiv.org/abs/arXiv:quant-ph/0510095} {arXiv:quant-ph/0510095}
  \BibitemShut {NoStop}%
\bibitem [{\citenamefont {Gerelle}\ \emph {et~al.}(1974)\citenamefont
  {Gerelle}, \citenamefont {Greechie},\ and\ \citenamefont
  {.Miller}}]{greechie-1974}%
  \BibitemOpen
  \bibfield  {author} {\bibinfo {author} {\bibfnamefont {E.~R.}\ \bibnamefont
  {Gerelle}}, \bibinfo {author} {\bibfnamefont {Richard~Joseph}\ \bibnamefont
  {Greechie}}, \ and\ \bibinfo {author} {\bibfnamefont {F.~R}\ \bibnamefont
  {.Miller}},\ }\bibfield  {title} {\enquote {\bibinfo {title} {Weights on
  spaces},}\ }in\ \href {\doibase 10.1007/978-94-010-2274-3} {\emph {\bibinfo
  {booktitle} {Physical Reality and Mathematical Description}}},\ \bibinfo
  {editor} {edited by\ \bibinfo {editor} {\bibfnamefont {Charles~P.}\
  \bibnamefont {Enz}}\ and\ \bibinfo {editor} {\bibfnamefont {Jagdish}\
  \bibnamefont {Mehra}}}\ (\bibinfo  {publisher} {D. Reidel Publishing Company,
  Springer Netherlands},\ \bibinfo {address} {Dordrecht, Holland},\ \bibinfo
  {year} {1974})\ pp.\ \bibinfo {pages} {167--192}\BibitemShut {NoStop}%
\bibitem [{\citenamefont {Wright}(1978)}]{wright:pent}%
  \BibitemOpen
  \bibfield  {author} {\bibinfo {author} {\bibfnamefont {Ron}\ \bibnamefont
  {Wright}},\ }\bibfield  {title} {\enquote {\bibinfo {title} {The state of the
  pentagon. {A} nonclassical example},}\ }in\ \href
  {https://www.elsevier.com/books/mathematical-foundations-of-quantum-theory/marlow/978-0-12-473250-6}
  {\emph {\bibinfo {booktitle} {Mathematical Foundations of Quantum Theory}}},\
  \bibinfo {editor} {edited by\ \bibinfo {editor} {\bibfnamefont {A.~R.}\
  \bibnamefont {Marlow}}}\ (\bibinfo  {publisher} {Academic Press},\ \bibinfo
  {address} {New York},\ \bibinfo {year} {1978})\ pp.\ \bibinfo {pages}
  {255--274}\BibitemShut {NoStop}%
\bibitem [{\citenamefont {Klyachko}\ \emph {et~al.}(2008)\citenamefont
  {Klyachko}, \citenamefont {Can}, \citenamefont
  {Binicio\ifmmode~\breve{g}\else \u{g}\fi{}lu},\ and\ \citenamefont
  {Shumovsky}}]{Klyachko-2008}%
  \BibitemOpen
  \bibfield  {author} {\bibinfo {author} {\bibfnamefont {Alexander~A.}\
  \bibnamefont {Klyachko}}, \bibinfo {author} {\bibfnamefont {M.~Ali}\
  \bibnamefont {Can}}, \bibinfo {author} {\bibfnamefont {Sinem}\ \bibnamefont
  {Binicio\ifmmode~\breve{g}\else \u{g}\fi{}lu}}, \ and\ \bibinfo {author}
  {\bibfnamefont {Alexander~S.}\ \bibnamefont {Shumovsky}},\ }\bibfield
  {title} {\enquote {\bibinfo {title} {Simple test for hidden variables in
  spin-1 systems},}\ }\href {\doibase 10.1103/PhysRevLett.101.020403}
  {\bibfield  {journal} {\bibinfo  {journal} {Physical Review Letters}\
  }\textbf {\bibinfo {volume} {101}},\ \bibinfo {pages} {020403} (\bibinfo
  {year} {2008})},\ \Eprint {http://arxiv.org/abs/arXiv:0706.0126}
  {arXiv:0706.0126} \BibitemShut {NoStop}%
\bibitem [{\citenamefont {Lov\'asz}(1979)}]{lovasz-79}%
  \BibitemOpen
  \bibfield  {author} {\bibinfo {author} {\bibfnamefont {L\'aszl\'o}\
  \bibnamefont {Lov\'asz}},\ }\bibfield  {title} {\enquote {\bibinfo {title}
  {On the {S}hannon capacity of a graph},}\ }\href {\doibase
  10.1109/TIT.1979.1055985} {\bibfield  {journal} {\bibinfo  {journal} {IEEE
  Transactions on Information Theory}\ }\textbf {\bibinfo {volume} {25}},\
  \bibinfo {pages} {1--7} (\bibinfo {year} {1979})}\BibitemShut {NoStop}%
\bibitem [{\citenamefont {Lov\'asz}\ \emph {et~al.}(1989)\citenamefont
  {Lov\'asz}, \citenamefont {Saks},\ and\ \citenamefont
  {Schrijver}}]{lovasz-89}%
  \BibitemOpen
  \bibfield  {author} {\bibinfo {author} {\bibfnamefont {L\'aszl\'o}\
  \bibnamefont {Lov\'asz}}, \bibinfo {author} {\bibfnamefont {M.}~\bibnamefont
  {Saks}}, \ and\ \bibinfo {author} {\bibfnamefont {Alexander}\ \bibnamefont
  {Schrijver}},\ }\bibfield  {title} {\enquote {\bibinfo {title} {Orthogonal
  representations and connectivity of graphs},}\ }\href {\doibase
  10.1016/0024-3795(89)90475-8} {\bibfield  {journal} {\bibinfo  {journal}
  {Linear Algebra and its Applications}\ }\textbf {\bibinfo {volume}
  {114-115}},\ \bibinfo {pages} {439--454} (\bibinfo {year} {1989})},\ \bibinfo
  {note} {special Issue Dedicated to Alan J. Hoffman}\BibitemShut {NoStop}%
\bibitem [{\citenamefont {Kochen}\ and\ \citenamefont
  {Specker}(1967)}]{kochen1}%
  \BibitemOpen
  \bibfield  {author} {\bibinfo {author} {\bibfnamefont {Simon}\ \bibnamefont
  {Kochen}}\ and\ \bibinfo {author} {\bibfnamefont {Ernst~P.}\ \bibnamefont
  {Specker}},\ }\bibfield  {title} {\enquote {\bibinfo {title} {The problem of
  hidden variables in quantum mechanics},}\ }\href {\doibase
  10.1512/iumj.1968.17.17004} {\bibfield  {journal} {\bibinfo  {journal}
  {Journal of Mathematics and Mechanics (now Indiana University Mathematics
  Journal)}\ }\textbf {\bibinfo {volume} {17}},\ \bibinfo {pages} {59--87}
  (\bibinfo {year} {1967})}\BibitemShut {NoStop}%
\bibitem [{\citenamefont {Cabello}\ \emph {et~al.}(1996)\citenamefont
  {Cabello}, \citenamefont {Estebaranz},\ and\ \citenamefont
  {Garc{\'{i}}a-Alcaine}}]{cabello-96}%
  \BibitemOpen
  \bibfield  {author} {\bibinfo {author} {\bibfnamefont {Ad{\'{a}}n}\
  \bibnamefont {Cabello}}, \bibinfo {author} {\bibfnamefont {Jos{\'{e}}~M.}\
  \bibnamefont {Estebaranz}}, \ and\ \bibinfo {author} {\bibfnamefont
  {G.}~\bibnamefont {Garc{\'{i}}a-Alcaine}},\ }\bibfield  {title} {\enquote
  {\bibinfo {title} {{B}ell-{K}ochen-{S}pecker theorem: {A} proof with 18
  vectors},}\ }\href {\doibase 10.1016/0375-9601(96)00134-X} {\bibfield
  {journal} {\bibinfo  {journal} {Physics Letters A}\ }\textbf {\bibinfo
  {volume} {212}},\ \bibinfo {pages} {183--187} (\bibinfo {year} {1996})},\
  \Eprint {http://arxiv.org/abs/arXiv:quant-ph/9706009}
  {arXiv:quant-ph/9706009} \BibitemShut {NoStop}%
\bibitem [{\citenamefont {Abbott}\ \emph {et~al.}(2015)\citenamefont {Abbott},
  \citenamefont {Calude},\ and\ \citenamefont {Svozil}}]{2015-AnalyticKS}%
  \BibitemOpen
  \bibfield  {author} {\bibinfo {author} {\bibfnamefont {Alastair~A.}\
  \bibnamefont {Abbott}}, \bibinfo {author} {\bibfnamefont {Cristian~S.}\
  \bibnamefont {Calude}}, \ and\ \bibinfo {author} {\bibfnamefont {Karl}\
  \bibnamefont {Svozil}},\ }\bibfield  {title} {\enquote {\bibinfo {title} {A
  variant of the {K}ochen-{S}pecker theorem localising value indefiniteness},}\
  }\href {\doibase 10.1063/1.4931658} {\bibfield  {journal} {\bibinfo
  {journal} {Journal of Mathematical Physics}\ }\textbf {\bibinfo {volume}
  {56}},\ \bibinfo {eid} {102201} (\bibinfo {year} {2015})},\ \Eprint
  {http://arxiv.org/abs/arXiv:1503.01985} {arXiv:1503.01985} \BibitemShut
  {NoStop}%
\bibitem [{\citenamefont {Greechie}(1971)}]{greechie:71}%
  \BibitemOpen
  \bibfield  {author} {\bibinfo {author} {\bibfnamefont {Richard~Joseph}\
  \bibnamefont {Greechie}},\ }\bibfield  {title} {\enquote {\bibinfo {title}
  {Orthomodular lattices admitting no states},}\ }\href {\doibase
  10.1016/0097-3165(71)90015-X} {\bibfield  {journal} {\bibinfo  {journal}
  {Journal of Combinatorial Theory. {S}eries {A}}\ }\textbf {\bibinfo {volume}
  {10}},\ \bibinfo {pages} {119--132} (\bibinfo {year} {1971})}\BibitemShut
  {NoStop}%
\bibitem [{\citenamefont {Kalmbach}(1983)}]{kalmbach-83}%
  \BibitemOpen
  \bibfield  {author} {\bibinfo {author} {\bibfnamefont {Gudrun}\ \bibnamefont
  {Kalmbach}},\ }\href@noop {} {\emph {\bibinfo {title} {Orthomodular
  Lattices}}},\ \bibinfo {series} {London Mathematical Society Monographs},
  Vol.~\bibinfo {volume} {18}\ (\bibinfo  {publisher} {Academic Press},\
  \bibinfo {address} {London and New York},\ \bibinfo {year}
  {1983})\BibitemShut {NoStop}%
\bibitem [{\citenamefont {Bretto}(2013)}]{Bretto-MR3077516}%
  \BibitemOpen
  \bibfield  {author} {\bibinfo {author} {\bibfnamefont {Alain}\ \bibnamefont
  {Bretto}},\ }\href {\doibase 10.1007/978-3-319-00080-0} {\emph {\bibinfo
  {title} {Hypergraph theory}}},\ Mathematical Engineering\ (\bibinfo
  {publisher} {Springer},\ \bibinfo {address} {Cham, Heidelberg, New York,
  Dordrecht, London},\ \bibinfo {year} {2013})\ pp.\ \bibinfo {pages}
  {xiv+119}\BibitemShut {NoStop}%
\bibitem [{\citenamefont {Bertlmann}(2020)}]{Bertlmann2020}%
  \BibitemOpen
  \bibfield  {author} {\bibinfo {author} {\bibfnamefont {Reinhold~A.}\
  \bibnamefont {Bertlmann}},\ }\bibfield  {title} {\enquote {\bibinfo {title}
  {Real or not real that is the question~$\ldots$},}\ }\href {\doibase
  10.1140/epjh/e2020-10022-x} {\bibfield  {journal} {\bibinfo  {journal} {The
  European Physical Journal H}\ }\textbf {\bibinfo {volume} {45}},\ \bibinfo
  {pages} {205--236} (\bibinfo {year} {2020})},\ \Eprint
  {http://arxiv.org/abs/arXiv:2005.08719} {arXiv:2005.08719} \BibitemShut
  {NoStop}%
\bibitem [{\citenamefont {Einstein}\ \emph {et~al.}(1935)\citenamefont
  {Einstein}, \citenamefont {Podolsky},\ and\ \citenamefont {Rosen}}]{epr}%
  \BibitemOpen
  \bibfield  {author} {\bibinfo {author} {\bibfnamefont {Albert}\ \bibnamefont
  {Einstein}}, \bibinfo {author} {\bibfnamefont {Boris}\ \bibnamefont
  {Podolsky}}, \ and\ \bibinfo {author} {\bibfnamefont {Nathan}\ \bibnamefont
  {Rosen}},\ }\bibfield  {title} {\enquote {\bibinfo {title} {Can
  quantum-mechanical description of physical reality be considered complete?}}\
  }\href {\doibase 10.1103/PhysRev.47.777} {\bibfield  {journal} {\bibinfo
  {journal} {Physical Review}\ }\textbf {\bibinfo {volume} {47}},\ \bibinfo
  {pages} {777--780} (\bibinfo {year} {1935})}\BibitemShut {NoStop}%
\bibitem [{\citenamefont {Yanofsky}(2019)}]{Yanofsky-object}%
  \BibitemOpen
  \bibfield  {author} {\bibinfo {author} {\bibfnamefont {Noson~S.}\
  \bibnamefont {Yanofsky}},\ }\href
  {http://www.sci.brooklyn.cuny.edu/~noson/Mind%20and%20Physics.pdf} {\enquote
  {\bibinfo {title} {The mind and the limitations of physics},}\ } (\bibinfo
  {year} {2019}),\ \bibinfo {note} {preprint, , accessed on January 14,
  2021}\BibitemShut {NoStop}%
\bibitem [{\citenamefont {Hertz}(1894)}]{hertz-94}%
  \BibitemOpen
  \bibfield  {author} {\bibinfo {author} {\bibfnamefont {Heinrich}\
  \bibnamefont {Hertz}},\ }\href
  {https://archive.org/details/dieprinzipiende00hertgoog} {\emph {\bibinfo
  {title} {{P}rinzipien der {M}echanik}}}\ (\bibinfo  {publisher} {Johann
  Ambrosius Barth (Arthur Meiner)},\ \bibinfo {address} {Leipzig},\ \bibinfo
  {year} {1894})\ \bibinfo {note} {mit einem Vorewort von H. von
  Helmholtz}\BibitemShut {NoStop}%
\bibitem [{\citenamefont {Hertz}(1899)}]{hertz-94e}%
  \BibitemOpen
  \bibfield  {author} {\bibinfo {author} {\bibfnamefont {Heinrich}\
  \bibnamefont {Hertz}},\ }\href
  {https://archive.org/details/principlesofmech00hertuoft} {\emph {\bibinfo
  {title} {The principles of mechanics presented in a new form}}}\ (\bibinfo
  {publisher} {MacMillan and Co., Ltd.},\ \bibinfo {address} {London and New
  York},\ \bibinfo {year} {1899})\ \bibinfo {note} {with a foreword by H. von
  Helmholtz, translated by D. E. Jones and J. T. Walley}\BibitemShut {NoStop}%
\bibitem [{\citenamefont {Pitowsky}(1998)}]{pitowsky:218}%
  \BibitemOpen
  \bibfield  {author} {\bibinfo {author} {\bibfnamefont {Itamar}\ \bibnamefont
  {Pitowsky}},\ }\bibfield  {title} {\enquote {\bibinfo {title} {Infinite and
  finite {G}leason's theorems and the logic of indeterminacy},}\ }\href
  {\doibase 10.1063/1.532334} {\bibfield  {journal} {\bibinfo  {journal}
  {Journal of Mathematical Physics}\ }\textbf {\bibinfo {volume} {39}},\
  \bibinfo {pages} {218--228} (\bibinfo {year} {1998})}\BibitemShut {NoStop}%
\bibitem [{\citenamefont {Specker}(1990)}]{specker-ges}%
  \BibitemOpen
  \bibfield  {author} {\bibinfo {author} {\bibfnamefont {Ernst}\ \bibnamefont
  {Specker}},\ }\href {\doibase 10.1007/978-3-0348-9259-9} {\emph {\bibinfo
  {title} {Selecta}}}\ (\bibinfo  {publisher} {Birkh{\"{a}}user Verlag},\
  \bibinfo {address} {Basel},\ \bibinfo {year} {1990})\BibitemShut {NoStop}%
\end{thebibliography}%

\end{document}

here is the cddlib / https://pypi.org/project/pycddlib / python code for anybody wishing to repeat my calculation:

perl

###########################################################

# Created 2014 by Karl Svozil
# Adapted 04-2014 by Karl Svozil for CHSH etc
# Adapted 09-2015 by Karl Svozil for Cabello's KS config etc

# get name of input source file; entries are ";"-delimited (CSV)
$sourcefilename = "three-variables-aka-triangle";

# get date
use Time::Piece;

my $today = Time::Piece->new->strftime('%Y-%m-%d');



#print ">".$today."-".$sourcefilename."-anonymous.txt";


open (OUTA, ">".$today."-".$sourcefilename.".ext");

for ($count01 = 1; $count01 >= -1; $count01-- ) {
for ($count02 = 1; $count02 >= -1; $count02-- ) {
for ($count03 = 1; $count03 >= -1; $count03-- ) {


if ( $count01 *
     $count02 *
     $count03
!= 0)
{
print OUTA sprintf ("%5.0f",$count01*$count02);
print OUTA sprintf ("%5.0f",$count01*$count03);
print OUTA sprintf ("%5.0f",$count02*$count03);

print OUTA "\n";
}

}
}
}
close OUTA;
exit;

Results in:

    1    1    1
    1   -1   -1
   -1    1   -1
   -1   -1    1
   -1   -1    1
   -1    1   -1
    1   -1   -1
    1    1    1

#################################################################


python

on >>> say:


#################################################################

import cdd

mat = cdd.Matrix([
[ 1,  1 ,   1 ,   1 ],
[ 1,  1 ,  -1 ,  -1 ],
[ 1, -1 ,   1 ,  -1 ],
[ 1, -1 ,  -1 ,   1 ],
[ 1, -1 ,  -1 ,   1 ],
[ 1, -1 ,   1 ,  -1 ],
[ 1,  1 ,  -1 ,  -1 ],
[ 1,  1 ,   1 ,   1 ] ], number_type='fraction')

mat.rep_type = cdd.RepType.GENERATOR
poly = cdd.Polyhedron(mat)
print(poly)

ine = poly.get_inequalities()
print(ine)



f = open('0.0','w')
print(ine, file=f)

f.close()


Results in:

V-representation
begin
14  10   real
1 1 0 0 0 1 0 0 0 0
1 1 0 0 0 0 1 0 0 0
1 0 1 0 0 0 0 1 0 0
1 0 1 0 0 1 0 0 1 0
1 0 1 0 0 1 0 0 0 1
1 0 1 0 0 0 1 0 1 0
1 0 1 0 0 0 1 0 0 1
1 0 0 1 0 0 0 1 0 0
1 0 0 1 0 1 0 0 1 0
1 0 0 1 0 1 0 0 0 1
1 0 0 1 0 0 1 0 1 0
1 0 0 1 0 0 1 0 0 1
1 0 0 0 1 0 0 0 1 0
1 0 0 0 1 0 0 0 0 1
end

H-representation
begin
 4 4 rational
 1 -1 -1 1
 1 1 -1 -1
 1 -1 1 -1
 1 1 1 1
end

square.png


~~~~~~~~~~~~~~~~~~~~~~~~~~~~~~~~~~~~~~~~~~~~~~~


(*Definition of "my" Tensor Product*)
(*a,b are nxn and mxm-matrices*)

MyTensorProduct[a_, b_] :=
  Table[
   a[[Ceiling[s/Length[b]], Ceiling[t/Length[b]]]]*
    b[[s - Floor[(s - 1)/Length[b]]*Length[b],
      t - Floor[(t - 1)/Length[b]]*Length[b]]], {s, 1,
    Length[a]*Length[b]}, {t, 1, Length[a]*Length[b]}];


(*Definition of the Tensor Product between two real valued vectors*)

TensorProductVec[x_, y_] :=
  Flatten[Table[
    x[[i]] y[[j]], {i, 1, Length[x]}, {j, 1, Length[y]}]];


(*Definition of the Dyadic Product*)

DyadicProductVec[x_] :=
  Table[x[[i]] Conjugate[x[[j]]], {i, 1, Length[x]}, {j, 1,
    Length[x]}];

(*Definition of the sigma matrices*)


vecsig[r_, tt_, p_] :=
 r*{{Cos[tt], Sin[tt] Exp[-I p]}, {Sin[tt] Exp[I p], -Cos[tt]}}

(*Definition of some vectors*)

BellBasis = (1/Sqrt[2]) {{1, 0, 0, 1}, {0, 1, 1, 0}, {0, 1, -1,
     0}, {1, 0, 0, -1}};

Basis = {{1, 0, 0, 0}, {0, 1, 0, 0}, {0, 0, 1, 0}, {0, 0, 0, 1}};



(*~~~~~~~~~~~~~~~~~~~~~~~~~  2  PARTICLES ~~~~~~~~~~~~~~~~~~~~~~~~~~~~~~~~~~~~~~~*)

(*~~~~~~~~~~~~~~~~~~~~~~~~~  2  State System ~~~~~~~~~~~~~~~~~~~~~~~~~~~~~~~~~~~~~~~

% ~~~~~~~~~~~~~~~   2 x 2
% ~~~~~~~~~~~~~~~   2 x 2
% ~~~~~~~~~~~~~~~   2 x 2
% ~~~~~~~~~~~~~~~   2 x 2
% ~~~~~~~~~~~~~~~   2 x 2
% ~~~~~~~~~~~~~~~   2 x 2

*)


(*Definition of operators*)

(* Definition of one-particle operator *)

M2X =   {{0, 1}, {1, 0}};
M2Y =   {{0, -I}, {I, 0}};
M2Z =   {{1, 0}, {0, -1}};

(* Definition of spin operator

S2[t_, p_] := FullSimplify[M2X *Sin[t] Cos[p] + M2Y *Sin[t] Sin[p] + M2Z *Cos[t], {Element[t, Reals],
  Element[p, Reals]}];

vecsig[1, ttt, ppp] === S2[ttt, ppp]

*)

(*Definition of single projection operators for occurrence of spin up or down *)

SingleParticleProjector[t_, p_, pm_] :=   (1/2) (IdentityMatrix[2] + pm*vecsig[1, t, p])


(*Definition of two-particle joint projection operator for occurrence of spin up and down*)

JointProjector2[t1_, p1_, t2_,  p2_, pm1_, pm2_] :=  MyTensorProduct[ SingleParticleProjector[t1, p1, pm1],  SingleParticleProjector[t2, p2, pm2] ]


(*Definition of two-particle joint expectation operator for occurrence of spin up and down*)

JointExpectation[t1_, p1_, t2_, p2_] :=  FullSimplify[JointProjector2[t1, p1, t2, p2, 1, 1] + JointProjector2[t1, p1, t2, p2, -1, -1] -(JointProjector2[t1, p1, t2, p2, 1, -1] + JointProjector2[t1, p1, t2, p2, -1, +1])];

FullSimplify[JointExpectation[tt1, pp1, tt2, pp2]] ===  FullSimplify[MyTensorProduct[ vecsig[1, tt1, pp1]  ,  vecsig[1, tt2, pp2] ]]

(* two-partite correlations: quantum versus classical


Plot[{-1 + 2 t/Pi , -Cos[t]}, {t, 0, Pi}]

diff[t_] := -1 + 2 t/Pi + Cos[t]

Plot[diff[t], {t, 0, Pi}]

Solve[D[diff[t], t] == 0]

{{t -> ConditionalExpression[\[Pi] - ArcSin[2/\[Pi]] +
     2 \[Pi] ConditionalExpression[1, \[Placeholder]],
    ConditionalExpression[1, \[Placeholder]] \[Element]
     Integers]}, {t ->
   ConditionalExpression[
    ArcSin[2/\[Pi]] +
     2 \[Pi] ConditionalExpression[1, \[Placeholder]],
    ConditionalExpression[1, \[Placeholder]] \[Element] Integers]}}

FullSimplify[ diff[ Pi - ArcSin[2/Pi] ]]

FullSimplify[ diff[  ArcSin[2/Pi] ]]

(Sqrt[-4 + \[Pi]^2] - 2 ArcCos[2/\[Pi]])/\[Pi]

 *)

(*

FullSimplify[Eigensystem[JointExpectation[ttt, 0, 0, 0]]]

{{-1, -1, 1, 1},
  {{0, Cot[ttt] + Csc[ttt], 0, 1},
   {Cot[ttt] - Csc[ttt], 0, 1, 0},
   {0, Cot[ttt] - Csc[ttt], 0, 1},
   {Cot[ttt] + Csc[ttt], 0, 1, 0}}}

*)


(* check: Tsirelson bound for the maximal quantum violation of the CHSH inequality | E12 + E13 + E23 - E24 | \le 2 *)

\[Theta] = Pi/4;

sangle1 = {0,0};
sangle2 = {0,2\[Theta]};
sangle3 = {0,\[Theta]};
sangle4 = {0,3\[Theta]};

TS = FullSimplify[ JointExpectation[sangle1[[2]],sangle1[[1]],sangle3[[2]],sangle3[[1]]]  +  JointExpectation[sangle2[[2]],sangle2[[1]],sangle3[[2]],sangle3[[1]]]  + JointExpectation[sangle1[[2]],sangle1[[1]],sangle4[[2]],sangle4[[1]]] -  JointExpectation[sangle2[[2]],sangle2[[1]],sangle4[[2]],sangle4[[1]]]   ];
Eigensystem[TS]

(*

{{-2 Sqrt[2], 2 Sqrt[2], 0, 0}, {{-1, 1, 1, 1}, {-1, -1, -1, 1}, {1, 0, 0, 1}, {0, -1, 1, 0}}}

FullSimplify[
 Tr[(JointExpectation[sangle1[[2]], sangle1[[1]], sangle3[[2]],
       sangle3[[1]]] +
      JointExpectation[sangle2[[2]], sangle2[[1]], sangle3[[2]],
       sangle3[[1]]] +
      JointExpectation[sangle1[[2]], sangle1[[1]], sangle4[[2]],
       sangle4[[1]]] -
      JointExpectation[sangle2[[2]], sangle2[[1]], sangle4[[2]],
       sangle4[[1]]]).{{1, -1, -1, -1}, {-1, 1, 1, 1}, {-1, 1, 1,
      1}, {-1, 1, 1, 1}} (1/4)]]

-2 Sqrt[2]

*)


TrigReduce[Tr[JointExpectation[t1,0,t2,0].{{0,0,0,0},{0,1,-1,0},{0,-1,1,0},{0,0,0,0}}(1/2)]]

(* -Cos[t1 - t2] *)

FullSimplify[ TrigReduce[  Tr[JointExpectation[t1, p1, t2, p2].{{0, 0, 0, 0}, {0, 1, -1, 0}, {0, -1, 1, 0}, {0, 0, 0, 0}} (1/2)]]]

(* Cos[t1] Cos[t2] - Cos[p1 - p2] Sin[t1] Sin[t2] *)

FullSimplify[TrigReduce[Eigensystem[JointExpectation[tt, 0, 0, 0]]]]


(*
{{-1, -1, 1,
  1}, {{0, Cot[tt] + Csc[tt], 0, 1}, {Cot[tt] - Csc[tt], 0, 1, 0}, {0,
    Cot[tt] - Csc[tt], 0, 1}, {Cot[tt] + Csc[tt], 0, 1, 0}}}
*)


FullSimplify[TrigReduce[Eigensystem[JointExpectation[Pi/4, 0, 0, 0]]]]

(*
{{-1, -1, 1,
  1}, {{0, 1 + Sqrt[2], 0, 1}, {1 - Sqrt[2], 0, 1, 0}, {0,
   1 - Sqrt[2], 0, 1}, {1 + Sqrt[2], 0, 1, 0}}}
*)



(* Suppes-Zanotti operator from

@article {Suppes-81,
    AUTHOR = {Suppes, Patrick and Zanotti, Mario},
     TITLE = {When are probabilistic explanations possible?},
   JOURNAL = {Synthese},
  FJOURNAL = {Synthese. An International Journal for Epistemology,
              Methodology and Philosophy of Science},
    VOLUME = {48},
      YEAR = {1981},
    NUMBER = {2},
     PAGES = {191-199},
      ISSN = {0039-7857},
   MRCLASS = {60A05 (03B48 03G12 81B10)},
  MRNUMBER = {625996},
MRREVIEWER = {Dennis J. Packard},
       DOI = {10.1007/BF01063886},
       URL = {https://doi.org/10.1007/BF01063886},
}

Suppes-Zanotti classical predictions

- 1  \le     E_12 + E_13 + E_23     ... which is Eq. (10) of the S&Z 81
- 1  \le    -E_12 - E_13 + E_23
- 1  \le     E_12 - E_13 - E_23
- 1  \le    -E_12 + E_13 - E_23

*)


SZ1[t1_,p1_,t2_,p2_,t3_,p3_] := FullSimplify[   JointExpectation[t1, p1, t2, p2]  +  JointExpectation[t1, p1, t3, p3]  + JointExpectation[t2, p2, t3, p3]   ];
SZ2[t1_,p1_,t2_,p2_,t3_,p3_] := FullSimplify[ - JointExpectation[t1, p1, t2, p2]  -  JointExpectation[t1, p1, t3, p3]  + JointExpectation[t2, p2, t3, p3]   ];
SZ3[t1_,p1_,t2_,p2_,t3_,p3_] := FullSimplify[   JointExpectation[t1, p1, t2, p2]  -  JointExpectation[t1, p1, t3, p3]  - JointExpectation[t2, p2, t3, p3]   ];
SZ4[t1_,p1_,t2_,p2_,t3_,p3_] := FullSimplify[ - JointExpectation[t1, p1, t2, p2]  +  JointExpectation[t1, p1, t3, p3]  - JointExpectation[t2, p2, t3, p3]   ];

(* Spherical coordinates, azimuthal angle = 0  *)


FullSimplify[Eigensystem[SZ1[0,0,\[Theta],0,2 \[Theta],0]]]

FullSimplify[Eigensystem[SZ2[0,0,\[Theta],0,2 \[Theta],0]]]

FullSimplify[Eigensystem[SZ3[0,0,\[Theta],0,2 \[Theta],0]]]

FullSimplify[Eigensystem[SZ4[0,0,\[Theta],0,2 \[Theta],0]]]


\[Theta] = Pi/3;

FullSimplify[Eigensystem[SZ1[0,0,\[Theta],0,2 \[Theta],0]]]

FullSimplify[Eigensystem[SZ2[0,0,\[Theta],0,2 \[Theta],0]]]

FullSimplify[Eigensystem[SZ3[0,0,\[Theta],0,2 \[Theta],0]]]

FullSimplify[Eigensystem[SZ4[0,0,\[Theta],0,2 \[Theta],0]]]


\[Theta] = 0;

FullSimplify[Eigensystem[SZ1[0,0,\[Theta],0,2 \[Theta],0]]]

FullSimplify[Eigensystem[SZ2[0,0,\[Theta],0,2 \[Theta],0]]]

FullSimplify[Eigensystem[SZ3[0,0,\[Theta],0,2 \[Theta],0]]]

FullSimplify[Eigensystem[SZ4[0,0,\[Theta],0,2 \[Theta],0]]]

(*plot*)

Plot[Union[  Append[Eigenvalues[SZ1[0, 0, \[Theta], 0, 2 \[Theta], 0]],  -1]], {\[Theta], 0, Pi}]

Plot[Union[  Append[Eigenvalues[SZ2[0, 0, \[Theta], 0, 2 \[Theta], 0]],  -1]], {\[Theta], 0, Pi}]

Plot[Union[  Append[Eigenvalues[SZ3[0, 0, \[Theta], 0, 2 \[Theta], 0]],  -1]], {\[Theta], 0, Pi}]

Plot[Union[  Append[Eigenvalues[SZ4[0, 0, \[Theta], 0, 2 \[Theta], 0]],  -1]], {\[Theta], 0, Pi}]


(* check single terms entering SZi with one "good" vector state (1/Sqrt[2]) {-Sin[tt], Cos[tt], Cos[tt], Sin[tt]} *)


FullSimplify[  Tr[DyadicProductVec[(1/Sqrt[2]) {-Sin[tt], Cos[tt], Cos[tt], Sin[tt]}].JointExpectation[0, 0, tt, 0]]]

FullSimplify[  Tr[DyadicProductVec[(1/Sqrt[2]) {-Sin[tt], Cos[tt], Cos[tt], Sin[tt]}].JointExpectation[tt, 0, 2*tt, 0]]]

FullSimplify[  Tr[DyadicProductVec[(1/Sqrt[2]) {-Sin[tt], Cos[tt], Cos[tt], Sin[tt]}].JointExpectation[0, 0, 2*tt, 0]]]



(* with singlet Bell state *)

\[Theta] =.;

FullSimplify[ TrigReduce[  Eigensystem[SZ1[0, 0, \[Theta], 0, 2 \[Theta], 0]]]]

(*
{{-1 - 2 Cos[\[Theta]],
  1 + 2 Cos[\[Theta]], -Sqrt[5 + 4 Cos[\[Theta]]], Sqrt[
  5 + 4 Cos[\[Theta]]]}, {{-1, Cot[\[Theta]], Cot[\[Theta]],
   1}, {-1, -Tan[\[Theta]], -Tan[\[Theta]],
   1}, {1, -(1/
     4) (2 Cos[\[Theta]] + Sqrt[5 + 4 Cos[\[Theta]]] +
      Cos[2 \[Theta]]) Csc[\[Theta]] Sec[\[Theta]/
     2]^2, ((2 Cos[\[Theta]] + Sqrt[5 + 4 Cos[\[Theta]]] +
      Cos[2 \[Theta]]) Csc[\[Theta]])/(2 + 2 Cos[\[Theta]]),
   1}, {1, ((-2 Cos[\[Theta]] + Sqrt[5 + 4 Cos[\[Theta]]] -
      Cos[2 \[Theta]]) Csc[\[Theta]])/(
   2 + 2 Cos[\[Theta]]), ((2 Cos[\[Theta]] - Sqrt[
      5 + 4 Cos[\[Theta]]] + Cos[2 \[Theta]]) Csc[\[Theta]])/(
   2 (1 + Cos[\[Theta]])), 1}}}
*)

FullSimplify[
 TrigReduce[Eigenvalues[SZ1[0, 0, \[Theta], 0, 2 \[Theta], 0]]]]

Plot[Union[
  Append[{-1 - 2 Cos[\[Theta]], -Sqrt[
     5 + 4 Cos[\[Theta]]]}, -1]], {\[Theta], 0, Pi/5}]
(*
{-1 - 2 Cos[\[Theta]],
 1 + 2 Cos[\[Theta]], -Sqrt[5 + 4 Cos[\[Theta]]], Sqrt[
 5 + 4 Cos[\[Theta]]]}
*)


FullSimplify[ TrigReduce[  Tr[SZ1[0, 0, \[Theta], 0, 2 \[Theta], 0].{{0, 0, 0, 0}, {0, 1, -1, 0}, {0, -1, 1, 0}, {0, 0, 0, 0}} (1/2)]]]

(* -2 Cos[\[Theta]] - Cos[2 \[Theta]] *)

(* GHZ

import cdd

mat = cdd.Matrix([
[ 1,  1 ,   1 ,   1 , 1 ],
[ 1,  -1 ,  -1 ,  -1 , -1 ] ], number_type='fraction')

mat.rep_type = cdd.RepType.GENERATOR
poly = cdd.Polyhedron(mat)
print(poly)

ine = poly.get_inequalities()
print(ine)



f = open('0.0','w')
print(ine, file=f)

f.close()

*)

FullSimplify[ JointExpectation[]  +  JointExpectation[sangle2[[2]],sangle2[[1]],sangle3[[2]],sangle3[[1]]]  + JointExpectation[sangle1[[2]],sangle1[[1]],sangle4[[2]],sangle4[[1]]] -  JointExpectation[sangle2[[2]],sangle2[[1]],sangle4[[2]],sangle4[[1]]]   ];
Eigensystem[TS]



(* GHZ game *)

zp1 = (xp1 + xm1); zm1 = (xp1 - xm1);
zp2 = (xp2 + xm2); zm2 = (xp2 - xm2);
zp3 = (xp3 + xm3); zm3 = (xp3 - xm3);
FullSimplify[zp1*zp2*zp3 + zm1*zm2*zm3]


zp1 = (xp1 + xm1); zm1 = (xp1 - xm1);
zp2 = (yp2 + ym2); zm2 = (-I)(yp2 - ym2);
zp3 = (yp3 + ym3); zm3 = (-I)(yp3 - ym3);
FullSimplify[zp1*zp2*zp3 + zm1*zm2*zm3]


zp1 = (yp1 + ym1); zm1 = (-I)(yp1 - ym1);
zp2 = (xp2 + xm2); zm2 = (xp2 - xm2);
zp3 = (yp3 + ym3); zm3 = (-I)(yp3 - ym3);
FullSimplify[zp1*zp2*zp3 + zm1*zm2*zm3]

zp1 = (yp1 + ym1); zm1 = (-I)(yp1 - ym1);
zp2 = (yp2 + ym2); zm2 = (-I)(yp2 - ym2);
zp3 = (xp3 + xm3); zm3 = (xp3 - xm3);
FullSimplify[zp1*zp2*zp3 + zm1*zm2*zm3]


~~~~~~~~~~~~~~~~~~~~~~~~~~~~~~~~~~~~~~~~~~~~~~~~~~~~~~~~~~~~~~~~~

a = Table[1, {i, 37}];

(* a[[ a ]] *) a[[ 36 ]] = {1,0,0};
(* a[[ b ]] *) a[[ 37 ]] = {Sqrt[2],1,1};
a[[ 1 ]] = {0,1,1};
a[[ 2 ]] = {0,1,-1};
a[[ 3 ]] = {Sqrt[2],-1,-1};
a[[ 4 ]] = {0,0,1};
a[[ 5 ]] = {0,1,0};
a[[ 6 ]] = {Sqrt[2],1,-3};
a[[ 7 ]] = {1,-Sqrt[2],0};
a[[ 8 ]] = {Sqrt[2],-3,1};
a[[ 9 ]] = {1,0,-Sqrt[2]};
a[[ 10 ]] = {Sqrt[2],1,0};
a[[ 11 ]] = {Sqrt[2],0,1};
a[[ 12 ]] = {Sqrt[2],-2,-3};
a[[ 13 ]] = {1,-Sqrt[2],Sqrt[2]};
a[[ 14 ]] = {Sqrt[2],-3,-2};
a[[ 15 ]] = {1,Sqrt[2],-Sqrt[2]};
a[[ 16 ]] = {Sqrt[8],1,-1};
a[[ 17 ]] = {Sqrt[8],-1,1};
a[[ 18 ]] = {Sqrt[2],-7,-3};
a[[ 19 ]] = {Sqrt[2],-1,3};
a[[ 20 ]] = {Sqrt[2],-3,-7};
a[[ 21 ]] = {Sqrt[2],3,-1};
a[[ 22 ]] = {1,Sqrt[2],0};
a[[ 23 ]] = {1,0,Sqrt[2]};
a[[ 24 ]] = {Sqrt[2],-1,-3};
a[[ 25 ]] = {Sqrt[2],-1,1};
a[[ 26 ]] = {Sqrt[2],-3,-1};
a[[ 27 ]] = {Sqrt[2],1,-1};
a[[ 28 ]] = {Sqrt[2],-1,0};
a[[ 29 ]] = {Sqrt[2],0,-1};
a[[ 30 ]] = {Sqrt[2],2,3};
a[[ 31 ]] = {Sqrt[2],3,2};
a[[ 32 ]] = {Sqrt[2],3,7};
a[[ 33 ]] = {Sqrt[2],7,3};
a[[ 34 ]] = {Sqrt[2],1,3};
a[[ 35 ]] = {Sqrt[2],3,1};

Length[a]==Length[Union[a]]


b=Table[FullSimplify[Normalize[ a[[i]] ]] ,{i,37}]


rot=RotationTransform[{{1,0,0},(1/2){Sqrt[2],1,1}}]

m = TransformationMatrix[rot]

FullSimplify[(1/2)*m[[1 ;; 3, 1 ;; 3]].{Sqrt[2], 1, 1}]

FullSimplify[m[[1 ;; 3, 1 ;; 3]]]

c=Table[ FullSimplify[Normalize[ m[[1 ;; 3, 1 ;; 3]].a[[i]] ]]  ,{i,37}]

Intersection[b, c]


##########################################################
Dear Editors,

I would like to thank both Referees for their suggestions, which I have fully followed in the revised version of the manuscript.

In accord with the suggestions of the first Referee, I quoted the suggested reference; and I also added a paragraph (please see below) discussing it.

In accord with the suggestions of the second Referee I have quoted all references suggested, and discussed the "contextuality from indeterminacy" approach in the new paragraph in Section 4 mentioned earlier:

"In this line of thought an experimental outcome---or, in another wording,
a phenomenon that should be considered as~cite{bohr:39caus,Bohr-CW7-1996-cpip,sep-qt-uncertainty}
{\em ``the comprehension of the effects observed under given experimental conditions''}---is
composed of contributions from both the measured object as well as from the measurement apparatus.
Therefore the entire experimental configuration---effectively the experimental context---needs to be taken into account.
As not all experimental contexts can be expected to be physically realizable simultaneously,
not all observables can be expected to be jointly measurably.
In essence this view suggests that contextuality reduces to what Bohr considers to be complementarity~\cite[B1--B3]{Khrennikov2017},
and also in Heisenberg's related Principle of Indeterminacy or Uncertainty Principle---contextuality from indeterminacy~\cite@article{Jaeger2019,Jaeger2020,Ozawa2003}.
(See also Glauber's concrete quantum amplifier model~\cite{glauber,glauber-collected-cat,Glauber-cat-86},
as well as the ``{H}umpty-{D}umpty'' model of spin measurements~\cite{engrt-sg-I,engrt-sg-II}.)"

I hope that, with these revisions, the manuscript can be published.

Thank you for all your care, attention & efforts!

Yours Sincerely,

Karl Svozil

