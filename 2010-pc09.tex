
\documentclass[%
 %reprint,
 %superscriptaddress,
 %groupedaddress,
 %unsortedaddress,
 %runinaddress,
 %frontmatterverbose,
 preprint,
 showpacs,
 showkeys,
 preprintnumbers,
 %nofootinbib,
 %nobibnotes,
 %bibnotes,
  amsmath,amssymb,
  aps,
 % prl,
 pra,
 %prb,
 % rmp,
 %prstab,
 %prstper,
  longbibliography,
  floatfix,
  %lengthcheck,%
 ]{revtex4-1}

\usepackage[breaklinks=true,colorlinks=true,anchorcolor=blue,citecolor=blue,filecolor=blue,menucolor=blue,pagecolor=blue,urlcolor=blue,linkcolor=blue]{hyperref}

%\documentclass[pra,amsfonts,showpacs,showkeys,preprint,nofootinbib,numerical]{revtex4-1}
\sloppy
\usepackage{graphicx}% Include figure files
\usepackage{epstopdf}
\usepackage{eepic}
\usepackage{xcolor}
\usepackage{braket}
\usepackage{amsmath}
\usepackage{amsthm}

\usepackage{bm}

\RequirePackage{times}
\RequirePackage{mathptm}




%\usepackage{cdmtcs}
\begin{document}


%\cdmtcsauthor{Karl Svozil}
%\cdmtcstitle{Proposed direct test of quantum contextuality}
%\cdmtcsaffiliation{Vienna University of Technology}
%\cdmtcstrnumber{348}
%\cdmtcsdate{February 2009}
%\colourcoverpage

\title{Quantum value indefiniteness}


\author{Karl Svozil}
\email{svozil@tuwien.ac.at}
\homepage{http://tph.tuwien.ac.at/~svozil}
\affiliation{Institut f\"ur Theoretische Physik, Vienna University of Technology,  \\  Wiedner Hauptstra\ss e 8-10/136, A-1040 Vienna, Austria}

\begin{abstract}
The indeterministic outcome of a measurement of an individual quantum is certified by the impossibility of the simultaneous, unique, definite, deterministic pre-existence of all conceivable observables from physical conditions of that quantum alone.
\end{abstract}



 \pacs{03.65.Ta,03.65.Ud}
 \keywords{Quantum value indefiniteness, quantum contextuality, quantum oracle, quantum random number generator}

\maketitle



\section{Introduction}

One of the most astounding consequences of the assumption of the validity of the quantum formalism in terms
of Hilbert spaces \cite{v-neumann-49} is the apparent impossibility of its classical interpretation.
More precisely, a classical interpretation of a quantum logical structure~\cite{birkhoff-36}
is either identified with a Boolean algebra, or at least with a homomorphic embedding
(structurally preserving all quantum logical relations and operations) into some Boolean algebra~\cite{CalHerSvo}.
Quantum logics are obtained by identifying (unit) vectors
(associated with the one-dimensional subspaces corresponding to the linear spans of the vectors,
and with the corresponding one dimensional projectors) with  elementary yes-no propositions.
The logical {\it and, or,} and {\it not} operations are identified with the set theoretic intersection,
with the linear span of two subspaces, and
with forming the orthogonal subspace, respectively.
Suppose further that orthogonality among subspaces indicates mutual exclusive propositions or
experimental outcomes.

Then, in at least three-dimensional Hilbert (sub)spaces, there does not exist a (classical) truth assignment
on (finite sets of) elementary yes-no propositions which would
\begin{description}
\item[(Rule~1---``countable  additivity:'')]  ascribe truth to
exactly one observable outcome among each set of maximal commeasurable mutually exclusive outcomes,
and falsity to the others,
such that
\item[(Rule~2---``noncontextuality:'')]
for ``overlapping'' link observables belonging to more than one commeasurable set of observables,
henceforth called {\em context,}
the truth value remains the same, independent of the particular commeasured
observables~\cite{specker-60,kochen1,ZirlSchl-65,Alda,Alda2,kamber64,kamber65,peres-91,mermin-93,svozil-tkadlec,cabello-96,cabello:210401}.
\end{description}
Proofs (e.g., \cite{kochen1}) could be finitistic and by contradiction (i.e., {\it via reductio ad absurdum}),
so there should not be any metamathematical issues about their applicability in physics.
Countable additivity (Rule~1) is the basis of
a theorem~\cite{Gleason,pitowsky:218,rich-bridge,r:dvur-93} by Gleason which derives the Born rule
$\langle \textsf{\textbf{A}} \rangle = {\rm Tr}\left({\bm \rho} \textsf{\textbf{A}}\right)$,
where  $\langle \textsf{\textbf{A}} \rangle$ and  ${\bm \rho}$
stand for the expectation value of an observable $\textsf{\textbf{A}}$ and for the quantum state, respectively.


Yet, there are metaphysical issues related to the impossibility of a classical interpretation of the
quantum formalism; in particular the explicit and indispensible use of {\em counterfactuals}
in the argument~\cite{svozil-2006-omni}.
Remarkably, this has been already emphasized in the first announcement of the formal result~\cite{specker-60}.
Counterfactuals are ``observables'' which {\em could} have been measured {\em if}
the experimenter {\em would have} chosen a different, i.e., complementary, measurement setup,
but {\em actually chose another} (complementary) one.
Hence, from the point of view of the quantum formalism,
any proof of the impossibility of a classical interpretation of quantum mechanics
uses complementary observables, which cannot possibly be simultaneously measured.
Pointedly stated, from a strictly operational point of view, due to quantum complementarity~\cite[p.~7]{pauli:58},
the entities occurring in the proofs cannot physically coexist.

So, it may not be totally unjustified to ask
why one should bother about nonoperational quantities and their consequences at all?
There may be two affirmative apologies for the use of counterfactuals:
First, although these observables could not be measured simultaneously,
they are perfectly reasonable physical observables if the experimenter chooses to measure them.
Secondly, through a measurement setup involving two correlated  particles,
two complementary observables can be measured counterfactually~\cite{epr}
on two space-like separated~\cite{wjswz-98}
but entangled~\cite{schrodinger,CambridgeJournals:1737068,CambridgeJournals:2027212} particles.
Because of constraints on the uniqueness of the arguments, this ``indirect measurement'' cannot be
extended to more than two counterfactual observables~\cite{svozil-2006-uniquenessprinciple}.

Quantum ``value (in)definiteness,''  sometimes also termed ``counterfactual (in)definiteness''~\cite{MuBae-90},
refers to the (im)possibility of the simultaneous existence of definite outcomes of conceivable measurements
under certain assumptions [e.g. noncontextuality; see Rule~2 above] --- that is, unperformed measurements can(not)
have definite results~\cite{peres222}.
``(In)determinacy'' often (but not always) refers to the absence (presence) of causal laws
---
in the sense of the principle of sufficient reason stating that every phenomenon has its explanation and cause
---
governing a physical behavior.
Thus ``value (in)definiteness'' relates to a static property, whereas ``(in)determinacy'' is often used for temporal evolutions.
Sometimes, quantum  value indefiniteness is considered as one of the expressions of quantum indeterminacy;
another expression of quantum indeterminacy is, for instance, associated with
the (radioactive) decay of some
excited states~\cite{Kragh-1997AHESradioact,Kragh-2009_RePoss5}.

In what follows we shall review some explicit physical consequences of the impossibility to interpret the quantum formalism classically.
We shall also review consequences for the construction of quantum mechanical devices
capable of generating particular
indeterministic outcomes~\cite{svozil-qct,rarity-94,zeilinger:qct,stefanov-2000,0256-307X-21-10-027,wang:056107,fiorentino:032334,svozil-2009-howto,Kwon:09,10.1038/nature09008},
which have been already discussed in an article~\cite{2008-cal-svo} by Calude and the author.





Any particular maximal set of (mutually exclusive) observables will be called {\em context}~\cite{svozil-2008-ql}.
It constitutes a ``maximal collection of co-measurable observables,'' or, stated differently,
a ``classical mini-universe'' located within the continuity of complementary quantum propositions.
The spectral theorem suggests that a context can be formalized by a single  ``maximal'' self-adjoint operator, such that
there exist ``maximal'' sets of mutually compatible, co-measurable, mutually exclusive orthogonal projectors
which appear in its spectral decomposition
(e.g., \cite[Sec.~II.10, p. 90, English translation p.~173]{v-neumann-49},
\cite[\S~2]{kochen1}, \cite[pp.~227,228]{neumark-54}, and \cite[\S~84]{halmos-vs}).

\section{Contextual interpretation}

In a ``desperate'' attempt to save realism~\cite{stace},
Bell~\cite{bohr-1949,bell-66,hey-red,redhead}
proposed to abandon the noncontextuality assumption Rule~2 that the truth or falsity of an individual outcome of a measurement of some observable
is independent of what other (mutually exclusive) observables are measured ``alongside''
of it.
In Bell's own words~\cite[Sec.~5]{bell-66},
the ``danger'' in the implicit assumption is this\footnote{
Bell cites Bohr's remark~\cite{bohr-1949} about
{\em ``the impossibility of any sharp separation
between the behavior of atomic objects and the interaction with the measuring instruments which serve to define
the conditions under which the phenomena appear.''}}:
\begin{quote}
{\em
``It was tacitly assumed that measurement of an observable must yield the same value independently of
what other measurements may be made simultaneously.
$\ldots$
The result of an observation may reasonably depend
not only on the state of the system  $\ldots$
but also on the complete disposition  of the apparatus.''}
\end{quote}
This ``contextual interpretation'' of quantum mechanics  will be henceforth called {\em contextuality.}

Notice that contextuality
does not suggest that any {\em statistical} property is context dependent;
this would be ruled out by the Born rule, which is context independent.
Instead, the contextual interpretation claims that the {\em individual outcome} --- Bell's ``result of an observation'' ---
depends on the context.
This is somewhat similar to the parameter independence but outcome dependence of correlated quantum events~\cite{shimony2}.

The exact formalization or causes of this type of ``contextual outcome dependence'' remains an open question.
Individual quantum events are generally {\em conventionalized} to happen acausally and indeterministically~\cite{born-26-1,born-26-2};
according to the prevalent quantum canon~\cite{zeil-05_nature_ofQuantum},
{\em ``$\ldots$ for the individual event in quantum physics, not only do we not know the cause, there is no cause.''}
In this belief system, indeterminism can be  trivially certified by the convention of the ``random outcome''
of individual quantum events,
a view which is further
``backed'' by our inability to ``come up'' with a causal model, and by the statistical analysis~\cite{PhysRevA.82.022102} of the assumption
of stochasticity and randomness of strings generated {\em via} the context mismatch between preparation and measurement.
Nevertheless, one should always keep in mind that this kind of indeterminism may be epistemic and not ontic.
Furthermore, due to the ambiguities of a formal definition,
and by reduction to the halting problem~\cite{rogers1,davis,Barwise-handbook-logic,enderton72,odi:89,Boolos-07},
the incomputablity, and even more so randomness, of arbitrary (finite) sequences remains provably unprovable~\cite{calude:02}.


\subsection{Violation of probabilistic bounds}

For the sake of getting a more intuitive understanding of quantum contextuality, a few examples of its consequences will be discussed next.
As any violations of Boole-Bell type elements of physical reality indicate the impossibility of its classical interpretation
by probabilistic constraints~\cite{pitowsky-89a,Pit-94,pitowsky-86,pitowsky},
every violation of Boole-Bell type inequalities can be re-interpreted as (experimental)
``proof of contextuality''~\cite{hasegawa:230401,Bartosik-09,PhysRevLett.103.160405,kirch-09}.
Indeed, as expressed by \cite{cabello:210401},
{\em ``Because of the lack of spacelike separation between one
observer's choice and the other observer's outcome, the
immense majority of the experimental violations of Bell
inequalities does not prove quantum nonlocality, but just
quantum contextuality.''}
Alas, while certainly most (with the exception of, e.g., \cite{wjswz-98})  experimental violations of Bell
inequalities do not prove quantum nonlocality,
these statistical violations are no direct proof of contextuality in general.
Nevertheless, they may indicate counterfactual indefiniteness~\cite{MuBae-90}.

Note that in a geometric framework~\cite{froissart-81,cirelson:80,cirelson,pitowsky-89a,Pit-94,pitowsky-86,pitowsky,2000-poly},
Boole-Bell type inequalities are just the {\em facet inequalities} of a classical probability (correlation) polytope
obtained by (i) forming all probabilities and joint probabilities of independent events,
(ii) taking all two-valued measures (interpretable as truth assignments) associated with this structure,
(iii) for each of the probabilities and joint probabilities forming a vector whose components are the (encoded truth) values
(either ``$0$'' or ``$1$'') of the two-valued measures
(hence, the dimensionality of the problem is equal to the number of entries corresponding to  probabilities and joint probabilities);
every such vector is a vertex of the {\em correlation polytope},
(iv) applying the
Minkoswki-Weyl representation theorem (e.g., \cite[p.29]{ziegler}),  stating that
every convex polytope has a dual (equivalent) description  as the intersection of a finite number of half-spaces.
Such facets are given by linear inequalities,
which are obtained
from the set of vertices
by solving the (computationally hard \cite{pit:90})
{\em hull problem}.
The inequalities coincide with Boole's {\em ``conditions of possible experience,''} and with Bell type inequalities.

Any  ``proof'' of contextuality based on Boole-Bell type inequalities necessarily involves the {\em statistical} behavior of
many counterfactual quantities  contained in  Boole-Bell type inequalities.
These quantities cannot be obtained simultaneously,
but merely one after another in different experimental configuration runs
involving ``lots of particles.''
Due to the statistical nature of the argument and its implicit
improvable assumption that contextuality --- that is, the abandonment of Rule~2 --- is the only possible cause for
the violations of the classical probabilistic bounds, these ``proofs'' lack the {\em sufficiency} of the formal argument.

\subsection{Tables of counterfactual ``outcomes''}

Previously,  tables of hypothetical and counterfactual experimental outcomes
have been used to argue against the noncontextual classical interpretation
of the quantum probabilities~\cite{peres222,MuBae-90,krenn1}.
In what follows tables of contextual outcomes violating Rule~2 will be enumerated which could be compatible
with quantum probabilities.
These tables may serve as a demonstration of the kind of behavior which is required by (hypothetical and counterfactual)
individual events capable of rendering the desired violations of Boole-Bell type violations of bounds on classical probabilities.

Let ``$\textsf{\textbf{X}} \{  \textsf{\textbf{Y}} \}  $'' stand for ``observable $\textsf{\textbf{X}}$
measured alongside observable (or context) $\textsf{\textbf{Y}}$.''
Consider the hypothetical counterfactual outcomes enumerated in Table~\ref{2010-pc09-t1}
for simultaneous quantum observables associated with the  Clauser-Horne-Shimony-Holt inequality
\begin{equation}
\vert
\textsf{\textbf{A}}_1 \{  \textsf{\textbf{B}}_1 \} \textsf{\textbf{B}}_1 \{  \textsf{\textbf{A}}_1 \}  +
\textsf{\textbf{A}}_1 \{  \textsf{\textbf{B}}_2 \} \textsf{\textbf{B}}_2 \{  \textsf{\textbf{A}}_1 \}  +
\textsf{\textbf{A}}_2 \{  \textsf{\textbf{B}}_1 \} \textsf{\textbf{B}}_1 \{  \textsf{\textbf{A}}_2 \}  -
\textsf{\textbf{A}}_2 \{  \textsf{\textbf{B}}_2 \} \textsf{\textbf{B}}_2 \{  \textsf{\textbf{A}}_2 \}
\vert \le 2.
\end{equation}
They are contextual, as for some cases $\textsf{\textbf{X}} \{  \textsf{\textbf{Y}}_1 \}  \neq  \textsf{\textbf{X}} \{  \textsf{\textbf{Y}}_2 \}  $, as indicated in the enumeration.
(Note that noncontextuality would imply the independence of $\textsf{\textbf{X}}$ from $\textsf{\textbf{Y}}$; i.e.,
$\textsf{\textbf{X}} \{  \textsf{\textbf{Y}}_1 \}  =  \textsf{\textbf{X}} \{  \textsf{\textbf{Y}}_2 \}  =\textsf{\textbf{X}}$.)
\begin{table}
\begin{center}
\begin{tabular}{l ccccccccccccccccccccccccccccccccccc}
\hline\hline
%$\;$\\
$\textsf{\textbf{A}}_1 \{  \textsf{\textbf{B}}_1 \}  $ & & $+$       &  $\cdot$   &   $\cdot$ &    $\cdot$ &  $\cdot$   &   $\cdot$    & $\cdot$   &  $\cdot$   &   $\cdot$ &    $\cdot$ &  $+$ &   $\cdot$ &     & $\cdots$  \\
$\textsf{\textbf{A}}_1 \{  \textsf{\textbf{B}}_2 \}  $ & & $-$       &  $\cdot$   &   $\cdot$ &    $\cdot$ &  $\cdot$   &   $\cdot$    & $\cdot$   &  $\cdot$   &   $\cdot$ &    $\cdot$ &  $-$ &   $\cdot$ &     & $\cdots$  \\
$\textsf{\textbf{A}}_2 \{  \textsf{\textbf{B}}_1 \}  $ & & $\cdot$   &  $-$       &   $\cdot$ &    $\cdot$ &  $\cdot$   &   $\cdot$    & $+$       &  $\cdot$   &   $\cdot$ &    $\cdot$ &  $-$ &   $\cdot$ &     & $\cdots$  \\
$\textsf{\textbf{A}}_2 \{  \textsf{\textbf{B}}_2 \}  $ & & $\cdot$   &  $+$       &   $\cdot$ &    $\cdot$ &  $\cdot$   &   $\cdot$    & $-$       &  $\cdot$   &   $\cdot$ &    $\cdot$ &  $+$ &   $\cdot$ &     & $\cdots$  \\
$\textsf{\textbf{B}}_1 \{  \textsf{\textbf{A}}_1 \}  $ & & $\cdot$   &  $\cdot$   &   $\cdot$ &    $\cdot$ &  $-$       &   $\cdot$    & $\cdot$   &  $\cdot$   &   $\cdot$ &    $\cdot$ &  $-$ &   $+$ &         & $\cdots$  \\
$\textsf{\textbf{B}}_1 \{  \textsf{\textbf{A}}_2 \}  $ & & $\cdot$   &  $\cdot$   &   $\cdot$ &    $\cdot$ &  $+$       &   $\cdot$    & $\cdot$   &  $\cdot$   &   $\cdot$ &    $\cdot$ &  $+$ &   $-$ &         & $\cdots$  \\
$\textsf{\textbf{B}}_2 \{  \textsf{\textbf{A}}_1 \}  $ & & $\cdot$   &  $\cdot$   &   $\cdot$ &    $\cdot$ &  $+$       &   $\cdot$    & $\cdot$   &  $-$       &   $\cdot$ &    $\cdot$ &  $-$ &   $\cdot$ &     & $\cdots$  \\
$\textsf{\textbf{B}}_2 \{  \textsf{\textbf{A}}_2 \}  $ & & $\cdot$   &  $\cdot$   &   $\cdot$ &    $\cdot$ &  $-$       &   $\cdot$    & $\cdot$   &  $+$       &   $\cdot$ &    $\cdot$ &  $+$ &   $\cdot$ &     & $\cdots$  \\
%$\;$\\
\hline\hline
\end{tabular}
\end{center}
\caption{Hypothetical counterfactual contextual outcomes of an experiment capable of violating the Boole-Bell type inequalities
involving binary outcomes (denoted by ``$-$, $+$'') of two observables (subscripts ``1, 2'') on two particles (denoted by ``$\textsf{\textbf{A}}$, $\textsf{\textbf{B}}$'').
The expression ``$\textsf{\textbf{X}} \{  \textsf{\textbf{Y}} \}  $''
stands for ``observable $\textsf{\textbf{X}}$ measured alongside observable $\textsf{\textbf{Y}}$.''
Time progresses from left to right;
rows contain the individual conceivable, potential measurement values of the eight observables
$\textsf{\textbf{A}}_1 \{  \textsf{\textbf{B}}_1 \}  $,
$\textsf{\textbf{A}}_1 \{  \textsf{\textbf{B}}_2 \}  $,
$\textsf{\textbf{A}}_2 \{  \textsf{\textbf{B}}_1 \}  $,
$\textsf{\textbf{A}}_2 \{  \textsf{\textbf{B}}_2 \}  $,
$\textsf{\textbf{B}}_1 \{  \textsf{\textbf{A}}_1 \}  $,
$\textsf{\textbf{B}}_1 \{  \textsf{\textbf{A}}_2 \}  $,
$\textsf{\textbf{B}}_2 \{  \textsf{\textbf{A}}_1 \}  $,  and
$\textsf{\textbf{B}}_2 \{  \textsf{\textbf{A}}_2 \}  $
which ``simultaneously co-exist.''
Dots indicate any value in $\{-,+\}$.
\label{2010-pc09-t1} }
\end{table}


The difference between ``truth tables''
associated with configurations
for the statistical arguments against value indefiniteness involving Boole-Bell type inequalities
on the one hand,
and for direct proofs (e.g. by the Kochen-Specker theorem)
on the other hand,
is that the former tables need not always contain contextual assignments
---
although it can be expected that the violations of noncontextuality should
increase with increasing deviations from the classical Boole-Bell bounds on joint probabilities\footnote{
Note that for stronger-than-quantum correlations~\cite{pop-rohr,svozil-krenn} rendering a maximal
violation of the Clauser-Horne-Shimony-Holt inequality by
$
\textsf{\textbf{A}}_1 \{  \textsf{\textbf{B}}_1 \} \textsf{\textbf{B}}_1 \{  \textsf{\textbf{A}}_1 \}  +
\textsf{\textbf{A}}_1 \{  \textsf{\textbf{B}}_2 \} \textsf{\textbf{B}}_2 \{  \textsf{\textbf{A}}_1 \}  +
\textsf{\textbf{A}}_2 \{  \textsf{\textbf{B}}_1 \} \textsf{\textbf{B}}_1 \{  \textsf{\textbf{A}}_2 \}  -
\textsf{\textbf{A}}_2 \{  \textsf{\textbf{B}}_2 \} \textsf{\textbf{B}}_2 \{  \textsf{\textbf{A}}_2 \}  = \pm 4
$,
if
$\textsf{\textbf{A}}_1 \{  \textsf{\textbf{B}}_2 \} = \textsf{\textbf{A}}_2 \{  \textsf{\textbf{B}}_2 \}$,
then
$\textsf{\textbf{B}}_2 \{  \textsf{\textbf{A}}_1 \} = -\textsf{\textbf{B}}_2 \{  \textsf{\textbf{A}}_2 \}$,
and
if
$\textsf{\textbf{B}}_1 \{  \textsf{\textbf{A}}_2 \} = \textsf{\textbf{B}}_2 \{  \textsf{\textbf{A}}_2 \}$,
then
$\textsf{\textbf{A}}_2 \{  \textsf{\textbf{B}}_1 \} = -\textsf{\textbf{A}}_2 \{  \textsf{\textbf{B}}_2 \}$.}
---
whereas the latter tables require some violation(s) of noncontextuality at every single column.
For example, in the compact 18-vector configuration allowing a Kochen-Specker proof
introduced in~\cite{cabello-96,cabello-99} and depicted in Fig.~\ref{2007-miracles-ksc},
one is forced to violate the noncontextuality assumption Rule~2 for at least one link observable.
This can be readily demonstrated by considering all 36 entries per column in Table~\ref{2010-pc09-t2},
Whether one violation of the noncontextuality Rule~2 is enough for consistency (i.e., the necessary extent of the violation of contextuality)
with the quantum probabilities remains unknown.
\begin{figure}
\begin{center}
%TeXCAD Picture [1.pic]. Options:
%\grade{\on}
%\emlines{\off}
%\epic{\off}
%\beziermacro{\on}
%\reduce{\on}
%\snapping{\off}
%\quality{8.000}
%\graddiff{0.010}
%\snapasp{1}
%\zoom{5.6569}
\unitlength .6mm % = 1.423pt
\allinethickness{2pt} %\thicklines %\linethickness{0.8pt}
\ifx\plotpoint\undefined\newsavebox{\plotpoint}\fi % GNUPLOT compatibility
\begin{picture}(134.09,125.99)(0,0)

%\emline(86.39,101.96)(111.39,58.46)
\multiput(86.39,101.96)(.119617225,-.208133971){209}{{\color{green}\line(0,-1){0.208133971}}}
%\end
%\emline(86.39,14.96)(111.39,58.46)
\multiput(86.39,14.96)(.119617225,.208133971){209}{{\color{red}\line(0,1){0.208133971}}}
%\end
%\emline(36.47,101.96)(11.47,58.46)
\multiput(36.47,101.96)(-.119617225,-.208133971){209}{{\color{yellow}\line(0,-1){0.208133971}}}
%\end
%\emline(36.47,14.96)(11.47,58.46)
\multiput(36.47,14.96)(-.119617225,.208133971){209}{{\color{magenta}\line(0,1){0.208133971}}}
%\end
\color{blue}\put(86.39,15.21){\color{blue}\line(-1,0){50}}
\put(86.39,101.71){\color{violet}\line(-1,0){50}}
%
\put(36.34,15.16){\color{magenta}\circle{6}}
\put(36.34,15.16){\color{blue}\circle{4}}
\put(52.99,15.16){\color{blue}\circle{4}}
\put(52.99,15.16){\color{cyan}\circle{6}}
\put(69.68,15.16){\color{blue}\circle{4}}
\put(69.68,15.16){\color{orange}\circle{6}}
\put(86.28,15.16){\color{blue}\circle{4}}
\put(86.28,15.16){\color{red}\circle{6}}
%
\put(93.53,27.71){\color{red}\circle{4}}
\put(93.53,27.71){\color{orange}\circle{6}}
\put(102.37,43.44){\color{red}\circle{4}}
\put(102.37,43.44){\color{olive}\circle{6}}
\put(111.21,58.45){\color{red}\circle{4}}
\color{green}\put(111.21,58.45){\circle{6}}
%
\put(102.37,73.47){\color{green}\circle{4}}
\put(102.37,73.47){\color{olive}\circle{6}}
\put(93.53,89.21){\color{green}\circle{4}}
\put(93.53,89.21){\color{cyan}\circle{6}}
\put(86.28,101.76){\color{green}\circle{4}}
\put(86.28,101.76){\color{violet}\circle{6}}
%
\put(69.68,101.76){\color{violet}\circle{4}}
\put(69.68,101.76){\color{cyan}\circle{6}}
\put(52.99,101.76){\color{violet}\circle{4}}
\put(52.99,101.76){\color{orange}\circle{6}}
\put(36.34,101.76){\color{violet}\circle{4}}
\put(36.34,101.76){\color{yellow}\circle{6}}
%
\put(29.24,89.21){\color{yellow}\circle{4}}
\put(29.24,89.21){\color{orange}\circle{6}}
\put(20.4,73.47){\color{yellow}\circle{4}}
\put(20.4,73.47){\color{olive}\circle{6}}
\put(11.56,58.45){\color{yellow}\circle{4}}
\put(11.56,58.45){\color{magenta}\circle{6}}

\put(20.4,43.44){\color{magenta}\circle{4}}
\put(20.4,43.44){\color{olive}\circle{6}}
\put(29.24,27.71){\color{magenta}\circle{4}}
\put(29.24,27.71){\color{cyan}\circle{6}}

\color{cyan}
\qbezier(29.2,27.73)(23.55,-5.86)(52.99,15.24)
\qbezier(29.2,27.88)(36.93,75)(69.63,101.91)
\qbezier(52.69,15.24)(87.47,40.96)(93.72,89.27)
\qbezier(93.72,89.27)(98.4,125.99)(69.49,102.06)
\color{orange}
\qbezier(93.57,27.73)(99.22,-5.86)(69.78,15.24)
\qbezier(93.57,27.88)(85.84,75)(53.13,101.91)
\qbezier(70.08,15.24)(35.3,40.96)(29.05,89.27)
\qbezier(29.05,89.27)(24.37,125.99)(53.28,102.06)
\color{olive}
\qbezier(20.15,73.72)(-11.67,58.52)(20.15,43.31)
\qbezier(20.33,73.72)(61.34,93.16)(102.36,73.72)
\qbezier(102.36,73.72)(134.09,58.52)(102.53,43.31)
\qbezier(102.53,43.31)(60.99,23.43)(20.15,43.49)
{\color{black}
\put(30.41,114.02){\makebox(0,0)[cc]{$\textsf{\textbf{M}}$}}
\put(30.41,2.65){\makebox(0,0)[cc]  {$\textsf{\textbf{A}}$}}
\put(52.68,114.38){\makebox(0,0)[cc]{$\textsf{\textbf{L}}$}}
\put(52.68,2.3){\makebox(0,0)[cc]   {$\textsf{\textbf{B}}$}}
\put(91.93,114.2){\makebox(0,0)[cc] {$\textsf{\textbf{J}}$}}
\put(91.93,2.48){\makebox(0,0)[cc]  {$\textsf{\textbf{D}}$}}
\put(69.65,114.38){\makebox(0,0)[cc]{$\textsf{\textbf{K}}$}}
\put(73.65,2.3){\makebox(0,0)[cc]   {$\textsf{\textbf{C}}$}}
\put(103.24,94.22){\makebox(0,0)[cc]{$\textsf{\textbf{I}}$}}
\put(17.45,94.22){\makebox(0,0)[cc] {$\textsf{\textbf{N}}$}}
\put(106.24,22.45){\makebox(0,0)[cc]{$\textsf{\textbf{E}}$}}
\put(17.45,22.45){\makebox(0,0)[cc] {$\textsf{\textbf{R}}$}}
\put(115.13,77.96){\makebox(0,0)[cc]{$\textsf{\textbf{H}}$}}
\put(8.55,77.96){\makebox(0,0)[cc]  {$\textsf{\textbf{O}}$}}
\put(115.13,38.72){\makebox(0,0)[cc]{$\textsf{\textbf{F}}$}}
\put(10.55,38.72){\makebox(0,0)[cc] {$\textsf{\textbf{Q}}$}}
\put(120.92,57.98){\makebox(0,0)[l] {$\textsf{\textbf{G}}$}}
\put(1.77,57.98){\makebox(0,0)[rc]  {$\textsf{\textbf{P}}$}}
}
\put(61.341,9.192){\color{blue}\makebox(0,0)[cc]    {$\textsf{\textbf{a}}$}}
\put(102.883,35.355){\color{red}\makebox(0,0)[cc]   {$\textsf{\textbf{b}}$}}
\put(102.53,84.322){\color{green}\makebox(0,0)[cc]  {$\textsf{\textbf{c}}$}}
\put(60.457,108.01){\color{violet}\makebox(0,0)[cc] {$\textsf{\textbf{d}}$}}
\put(18.031,84.145){\color{yellow}\makebox(0,0)[cc] {$\textsf{\textbf{e}}$}}
\put(18.561,33.057){\color{magenta}\makebox(0,0)[cc]{$\textsf{\textbf{f}}$}}
\put(61.341,39.774){\color{olive}\makebox(0,0)[cc]  {$\textsf{\textbf{g}}$}}
\put(72.124,67.882){\color{orange}\makebox(0,0)[cc] {$\textsf{\textbf{h}}$}}
\put(48.79,67.705){\color{cyan}\makebox(0,0)[cc]    {$\textsf{\textbf{i}}$}}
\end{picture}
\end{center}
\caption{(Color online) Greechie diagram of a finite subset of the continuum of blocks or contexts embeddable in
four-dimensional real Hilbert space without a two-valued probability measure~\protect\cite{cabello-96,cabello-99}  .
The proof of the Kochen-Specker theorem  uses  nine tightly interconnected contexts
$\color{blue}    \textsf{\textbf{a}}=\{\textsf{\textbf{A}},\textsf{\textbf{B}},\textsf{\textbf{C}},\textsf{\textbf{D}}\}$,
$\color{red}     \textsf{\textbf{b}}=\{\textsf{\textbf{D}},\textsf{\textbf{E}},\textsf{\textbf{F}},\textsf{\textbf{G}}\}$,
$\color{green}   \textsf{\textbf{c}}=\{\textsf{\textbf{G}},\textsf{\textbf{H}},\textsf{\textbf{I}},\textsf{\textbf{J}}\}$,
$\color{violet}  \textsf{\textbf{d}}=\{\textsf{\textbf{J}},\textsf{\textbf{K}},\textsf{\textbf{L}},\textsf{\textbf{M}}\}$,
$\color{yellow}  \textsf{\textbf{e}}=\{\textsf{\textbf{M}},\textsf{\textbf{N}},\textsf{\textbf{O}},\textsf{\textbf{P}}\}$,
$\color{magenta} \textsf{\textbf{f}}=\{\textsf{\textbf{P}},\textsf{\textbf{Q}},\textsf{\textbf{R}},\textsf{\textbf{A}}\}$,
$\color{orange}  \textsf{\textbf{g}}=\{\textsf{\textbf{B}},\textsf{\textbf{I}},\textsf{\textbf{K}},\textsf{\textbf{R}}\}$,
$\color{olive}   \textsf{\textbf{h}}=\{\textsf{\textbf{C}},\textsf{\textbf{E}},\textsf{\textbf{L}},\textsf{\textbf{N}}\}$,
$\color{cyan}    \textsf{\textbf{i}}=\{\textsf{\textbf{F}},\textsf{\textbf{H}},\textsf{\textbf{O}},\textsf{\textbf{Q}}\}$
consisting of the 18 projectors associated with the one dimensional subspaces spanned by
$ \textsf{\textbf{A}}=(0,0,1,-1)    $,
$ \textsf{\textbf{B}}=(1,-1,0,0)    $,
$ \textsf{\textbf{C}}=(1,1,-1,-1)   $,
$ \textsf{\textbf{D}}=(1,1,1,1)     $,
$ \textsf{\textbf{E}}=(1,-1,1,-1)  $,
$ \textsf{\textbf{F}}=(1,0,-1,0)   $,
$ \textsf{\textbf{G}}=(0,1,0,-1)   $,
$ \textsf{\textbf{H}}=(1,0,1,0)    $,
$ \textsf{\textbf{I}}=(1,1,-1,1)   $,
$ \textsf{\textbf{J}}=(-1,1,1,1)    $,
$ \textsf{\textbf{K}}=(1,1,1,-1)    $,
$ \textsf{\textbf{L}}=(1,0,0,1)     $,
$ \textsf{\textbf{M}}=(0,1,-1,0)    $,
$ \textsf{\textbf{N}}=(0,1,1,0)    $,
$ \textsf{\textbf{O}}=(0,0,0,1)    $,
$ \textsf{\textbf{P}}=(1,0,0,0)    $,
$ \textsf{\textbf{Q}}=(0,1,0,0)    $,
$ \textsf{\textbf{R}}=(0,0,1,1)    $.
%
Greechie diagram representing atoms by points, and  contexts by maximal smooth, unbroken curves.
%
Every observable proposition occurs in exactly two contexts.
Thus, in an enumeration of the four observable propositions of each of the nine contexts,
there appears to be an {\em even} number of true propositions.
Yet, as there is an odd number of contexts,
there should be an {\em odd} number (actually nine) of true propositions.   \label{2007-miracles-ksc} }
\end{figure}




\begin{table}
\begin{center}
\begin{tabular}{l ccccccccccccccccccccccccccccccccccc}
\hline\hline
%$\;$\\
\color{blue}$\textsf{\textbf{A}} \{   \textsf{\textbf{a}}  \} $   & & $1$   &  $0$   &   $0$ &    $0$ &  $0$   &   $1$    & $0$   &  $1$   &  $0$ &    $0$ &  $0$   &   $1$  &   $1$         & $\cdots$  \\
\color{magenta} $\textsf{\textbf{A}} \{  \textsf{\textbf{f}} \}  $ & & $0$   &  $1$   &   $0$ &    $0$ &  $0$   &   $1$    & $0$   &  $1$   &   $0$ &    $0$ &  $0$ &   $1$   &   $1$         & $\cdots$  \\
\color{blue}$\textsf{\textbf{B}} \{ \textsf{\textbf{a}}  \}  $    & & $0$   &  $0$   &   $1$ &    $0$ &  $0$   &   $0$    & $1$   &  $0$   &  $1$ &    $0$ &  $0$   &   $0$  &   $0$         & $\cdots$  \\
\color{cyan} $\textsf{\textbf{B}} \{  \textsf{\textbf{i}} \}  $ & &$\cdot$&  $0$       &   $\cdot$ &    $0$     &  $0    $   &   $\cdot$    & $\cdot$       &  $0$       &   $\cdot$ &    $\cdot$ &  $0$     &   $0    $ & $\cdot$& $\cdots$  \\
\color{blue}$\textsf{\textbf{C}} \{ \textsf{\textbf{a}} \} $    & & $0$   &  $1$   &   $0$ &    $0$ &  $1$   &   $0$    & $0$   &  $0$   &  $0$ &    $0$ &  $1$   &   $0$  &   $0$         & $\cdots$  \\
\color{olive} $\textsf{\textbf{C}} \{  \textsf{\textbf{h}} \}  $ & & $\cdot$   &  $\cdot$   &   $\cdot$ &    $\cdot$ &  $\cdot$       &   $\cdot$    & $\cdot$   &  $\cdot$   &   $\cdot$ &    $\cdot$ &  $\cdot$ &   $\cdot$ &  $\cdot$& $\cdots$  \\
\color{blue}$\textsf{\textbf{D}} \{ \textsf{\textbf{a}} \} $    & & $0$   &  $0$   &   $0$ &    $1$ &  $0$   &   $0$    & $0$   &  $0$   &  $0$ &    $1$ &  $0$   &   $0$  &   $0$         & $\cdots$ \\
\color{red} $\textsf{\textbf{D}} \{  \textsf{\textbf{b}} \}  $ & & $\cdot$   &  $\cdot$   &   $\cdot$ &    $\cdot$ &  $\cdot$       &   $\cdot$    & $\cdot$   &  $\cdot$       &   $\cdot$ &    $\cdot$ &  $\cdot$ &   $\cdot$ & $\cdot$ & $\cdots$  \\
$\cdots$  & & $\cdot$   &  $\cdot$   &   $\cdot$ &    $\cdot$ &  $\cdot$       &   $\cdot$    & $\cdot$   &  $\cdot$       &   $\cdot$ &    $\cdot$ &  $\cdot$ &   $\cdot$ &   $\cdot$   & $\cdots$  \\
\color{yellow} $\textsf{\textbf{P}} \{  \textsf{\textbf{e}} \}  $ & & $\cdot$       &  $\cdot$   &   $\cdot$ &    $\cdot$ &  $\cdot$   &   $\cdot$    & $\cdot$   &  $\cdot$   &   $\cdot$ &    $\cdot$ &  $\cdot$ &   $\cdot$ &  $\cdot$& $\cdots$  \\
\color{magenta} $\textsf{\textbf{P}} \{  \textsf{\textbf{f}} \}  $ & & $0$   &  $0$   &   $0$ &    $1$ &  $0$   &   $0$    & $1$   &  $0$   &   $0$ &    $1$ &  $0$ &   $0$   &   $0$         & $\cdots$  \\
\color{orange} $\textsf{\textbf{Q}} \{  \textsf{\textbf{g}} \}  $ & & $\cdot$   &  $\cdot$       &   $\cdot$ &    $\cdot$ &  $\cdot$   &   $\cdot$    & $\cdot$       &  $\cdot$   &   $\cdot$ &    $\cdot$ &  $\cdot$ &   $\cdot$ & $\cdot$& $\cdots$  \\
\color{magenta} $\textsf{\textbf{Q}} \{  \textsf{\textbf{f}} \}  $ & & $1$   &  $0$   &   $0$ &    $0$ &  $0$   &   $0$    & $0$   &  $0$   &   $1$ &    $0$ &  $0$ &   $0$   &   $0$         & $\cdots$  \\
\color{cyan} $\textsf{\textbf{R}} \{  \textsf{\textbf{i}} \}  $ & & $0$   &  $\cdot$   &   $0$     &    $\cdot$ &  $\cdot$   &   $0$        & $0$           &  $\cdot$   &   $0$     &    $0$     &  $\cdot$ &   $\cdot$ &  $0$&    $\cdots$  \\
\color{magenta} $\textsf{\textbf{R}} \{  \textsf{\textbf{f}} \}  $ & & $0$   &  $0$   &   $1$ &    $0$ &  $1$   &   $0$    & $0$   &  $0$   &   $0$ &    $0$ &  $1$ &   $0$   &   $0$         & $\cdots$  \\
%$\;$\\
\hline\hline
\end{tabular}
\end{center}
\caption{(Color online) Hypothetical counterfactual contextual outcomes of  experiments associated
with a compact proof of the Kochen-Specker theorem~\protect\cite{cabello-96,cabello-99}
involving binary outcomes  ``$0$'' or $1$'' of 18 observables,  adding up to one within each of the nine contexts
denoted by ``$\textsf{\textbf{a}}$, $\ldots$, $\textsf{\textbf{i}}$''.
The expression ``$\textsf{\textbf{X}} \{  \textsf{\textbf{y}} \}  $''
stands for ``observable $\textsf{\textbf{X}}$ measured alongside the context $\textsf{\textbf{y}}$.''
Time progresses from left to right;
rows contain the individual conceivable, potential measurement values of the  observables
$\textsf{\textbf{A}} \{  \textsf{\textbf{a}} \}  , \ldots  ,\textsf{\textbf{R}} \{  \textsf{\textbf{i}} \}  $
which ``simultaneously co-exist.''
Dots indicate any value in $\{0,1\}$
subject to at least one violation of the noncontextuality assumption,
that is, $\textsf{\textbf{X}}(\textsf{\textbf{y}}) \neq \textsf{\textbf{X}}(\textsf{\textbf{y}}')$
for $\textsf{\textbf{y}} \neq \textsf{\textbf{y}}'$.
\label{2010-pc09-t2} }
\end{table}

If such signatures of contextuality exist cannot be decided experimentally, as direct observations
are operationally blocked by quantum complementarity. Thus this type of contextuality remains metaphysical.

\subsection{Indirect simultaneous tests}

\begin{figure}
\begin{center}
\begin{tabular}{ccccc}
%TeXCAD Picture [1.pic]. Options:
%\grade{\on}
%\emlines{\off}
%\epic{\off}
%\beziermacro{\on}
%\reduce{\on}
%\snapping{\off}
%\pvinsert{% Your \input, \def, etc. here}
%\quality{8.000}
%\graddiff{0.005}
%\snapasp{1}
%\zoom{4.0000}
\unitlength 0.3mm % = 2.85pt
\allinethickness{2pt}%\linethickness{0.8pt}
\ifx\plotpoint\undefined\newsavebox{\plotpoint}\fi % GNUPLOT compatibility
\begin{picture}(132.5,122)(0,0)
\put(20,20){\color{green}\line(1,0){73}}
%\emline(20,20)(75,110)
%\multiput(20,20)(.03372164316,.05518087063){1631}{\color{blue}\line(0,1){0.05518087063}}
\multiput(20,20)(.03372164316,.05518087063){1080}{\color{red}\line(0,1){0.05518087063}}
%\end
\put(20,20){\color{red}\circle{5.5}}
\put(20,20){\color{red}\circle{1.5}}
\put(20,20){\color{green}\circle{9}}
\put(56.25,20){\color{green}\circle{5.5}}
\put(56.25,20){\color{green}\circle{1.5}}
\put(92.5,20){\color{green}\circle{5.5}}
\put(92.5,20){\color{green}\circle{1.5}}
\put(56.25,79.75){\color{red}\circle{5.5}}
\put(56.25,79.75){\color{red}\circle{1.5}}
\put(38.75,51.25){\color{red}\circle{5.5}}
\put(38.75,51.25){\color{red}\circle{1.5}}
\put(15,5){\makebox(0,0)[cc]{$\textsf{\textbf{A}}$}}
\put(56.25,5){\makebox(0,0)[cc]{{\color{green}$\textsf{\textbf{B}}$}}}
\put(92.5,5){\makebox(0,0)[cc]{{\color{green}$\textsf{\textbf{C}}$}}}
\put(28,52.25){\makebox(0,0)[rc]{{\color{red}$\textsf{\textbf{D}}$}}}
\put(42.5,80.25){\makebox(0,0)[rc]{{\color{red}$\textsf{\textbf{E}}$}}}
\end{picture}
&
%TeXCAD Picture [1.pic]. Options:
%\grade{\on}
%\emlines{\off}
%\epic{\off}
%\beziermacro{\on}
%\reduce{\on}
%\snapping{\off}
%\pvinsert{% Your \input, \def, etc. here}
%\quality{8.000}
%\graddiff{0.005}
%\snapasp{1}
%\zoom{4.0000}
\unitlength 0.2mm % = 2.85pt
\allinethickness{2pt}%\linethickness{0.8pt}
\ifx\plotpoint\undefined\newsavebox{\plotpoint}\fi % GNUPLOT compatibility
\begin{picture}(132.5,122)(0,0)
\put(20,20){\color{green}\line(1,0){110}}
%\emline(20,20)(75,110)
\multiput(20,20)(.03372164316,.05518087063){1631}{\color{blue}\line(0,1){0.05518087063}}
%\end
%\emline(75,110)(130,20)
\multiput(75,110)(.03372164316,-.05518087063){1631}{\color{red}\line(0,-1){0.05518087063}}
%\end
\put(20,20){\color{blue}\circle{5.5}}
\put(20,20){\color{blue}\circle{1.5}}
\put(20,20){\color{green}\circle{9}}
\put(56.25,20){\color{green}\circle{5.5}}
\put(56.25,20){\color{green}\circle{1.5}}
\put(92.5,20){\color{green}\circle{5.5}}
\put(92.5,20){\color{green}\circle{1.5}}
\put(129.75,20){\color{green}\circle{5.5}}
\put(129.75,20){\color{green}\circle{1.5}}
\put(129.75,20){\color{red}\circle{9}}
\put(56.25,79.75){\color{blue}\circle{5.5}}
\put(56.25,79.75){\color{blue}\circle{1.5}}
\put(38.75,51.25){\color{blue}\circle{1.5}}
\put(38.75,51.25){\color{blue}\circle{5.5}}
\put(74.75,109.75){\color{red}\circle{5.5}}
\put(74.75,109.75){\color{red}\circle{1.5}}
\put(74.75,109.75){\color{blue}\circle{9}}
\put(93.75,79.75){\color{red}\circle{5.5}}
\put(93.75,79.75){\color{red}\circle{1.5}}
\put(111.25,51.25){\color{red}\circle{5.5}}
\put(111.25,51.25){\color{red}\circle{1.5}}
\put(15,5){\makebox(0,0)[cc]{$\textsf{\textbf{A}}$}}
\put(56.25,5){\makebox(0,0)[cc]{{\color{green}$\textsf{\textbf{B}}$}}}
\put(92.5,5){\makebox(0,0)[cc]{{\color{green}$\textsf{\textbf{C}}$}}}
\put(138,5){\makebox(0,0)[cc]{$\textsf{\textbf{D}}$}}
\put(125,52.25){\makebox(0,0)[lc]{{\color{red}$\textsf{\textbf{E}}$}}}
\put(108,80.25){\makebox(0,0)[lc]{{\color{red}$\textsf{\textbf{F}}$}}}
\put(74.75,128){\makebox(0,0)[cc]{$\textsf{\textbf{G}}$}}
\put(42.5,80.25){\makebox(0,0)[rc]{{\color{blue}$\textsf{\textbf{H}}$}}}
\put(25.5,52.25){\makebox(0,0)[rc]{{\color{blue}$\textsf{\textbf{I}}$}}}
\end{picture}
&
$\quad \quad$&
%TexCad Options
%\grade{\off}
%\emlines{\off}
%\beziermacro{\on}
%\reduce{\on}
%\snapping{\off}
%\quality{2.00}
%\graddiff{0.01}
%\snapasp{1}
%\zoom{1.00}
\unitlength 0.30mm
\allinethickness{2pt}
%\thicklines %\linethickness{0.4pt}
\begin{picture}(103.67,92.00)
\put(0.00,80.33){{\color{red}\line(2,-1){40.00}}}
{{\color{red} \bezier{104}(40.00,60.33)(50.33,55.00)(50.00,40.33)}}
\put(50.00,40.33){{\color{red}\line(0,-1){40.00}}}
\put(102.00,80.33){{\color{green}\line(-2,-1){40.00}}}
{{\color{green} \bezier{104}(62.00,60.33)(51.67,55.00)(52.00,40.33)}}
\put(52.00,40.33){{\color{green}\line(0,-1){40.00}}}
\put(0.00,80.33){{\color{red}\circle{3.33}}}
\put(40.00,60.33){{\color{red}\circle{3.33}}}
\put(51.00,40.33){{\color{red}\circle{3.33}}}
\put(0.00,80.33){{\color{red}\circle{1}}}
\put(40.00,60.33){{\color{red}\circle{1}}}
\put(51.00,40.33){{\color{red}\circle{1}}}
\put(51.00,40.33){{\color{green}\circle{5}}}
\put(51.00,0.33){{\color{red}\circle{3.33}}}
\put(51.00,0.33){{\color{red}\circle{1}}}
\put(51.00,0.33){{\color{green}\circle{5}}}
\put(102.00,80.33){{\color{green}\circle{3.33}}}
\put(62.00,60.33){{\color{green}\circle{3.33}}}
\put(102.00,80.33){{\color{green}\circle{1}}}
\put(62.00,60.33){{\color{green}\circle{1}}}
\put(45.00,0.33){\makebox(0,0)[rc]{$\textsf{\textbf{A}}$}}
\put(45.00,40.33){\makebox(0,0)[rc]{$\textsf{\textbf{B}}$}}
\put(40.00,71){{\color{red}\makebox(0,0)[cc]{$\textsf{\textbf{C}}$}}}
\put(0.00,90){{\color{red}\makebox(0,0)[cc]{$\textsf{\textbf{D}}$}}}
\put(62.00,71){{\color{green}\makebox(0,0)[cc]{$\textsf{\textbf{E}}$}}}
\put(102.00,90.00){{\color{green}\makebox(0,0)[cc]{$\textsf{\textbf{F}}$}}}
\end{picture}
\\
$\;$\\
a)& b)&& c) \\
\end{tabular}
\end{center}
\caption{(Color online) Diagrammatical representation of interlinked contexts by Greechie (orthogonality) diagrams
(points stand for individual basis vectors, and entire contexts are drawn as smooth curves):
a) two tripods with a common leg;
b) three interconnected fourpods (this configuration with tripods would be
irrepresentable in three-dimensional vector space~\protect\cite{kalmbach-83,pulmannova-91});
b) two contexts in four dimensions interconnected by two link observables.
}
\label{2010-pc09-f1}
\end{figure}

There exist ``explosion views'' of counterfactual configurations involving singlet or other correlated states of two
three- and more state particles which, due to the counterfactual uniqueness properties~\cite{svozil-2006-uniquenessprinciple},
are capable of indirectly testing the quantum contextuality
assumption~\cite{svozil:040102} by a simultaneous measurement of two complementary contexts~\cite{epr}.
For the sake of  explicit demonstration, consider Fig.~\ref{2010-pc09-f1} depicting
three orthogonality (Greechie) diagrams of such configurations of observables.
Every diagram is representable in three- or four-dimensional vector space.

For the configuration depicted in Fig.~\ref{2010-pc09-f1}a),
contextuality predicts that there exist experimental outcomes with
$
\textsf{\textbf{A}} \{  \textsf{\textbf{B}},\textsf{\textbf{C}} \}
\neq
\textsf{\textbf{A}} \{  \textsf{\textbf{D}},\textsf{\textbf{E}} \}
$.
As detailed quantum mechanical calculations~\cite{svozil:040102} show,  this is not predicted by quantum mechanics.

For the configuration depicted in Fig.~\ref{2010-pc09-f1}b),
contextuality predicts that there exist experimental outcomes with
$
\textsf{\textbf{A}} \{  \textsf{\textbf{B}},\textsf{\textbf{C}},\textsf{\textbf{D}} \}
\neq
\textsf{\textbf{A}} \{  \textsf{\textbf{G}},\textsf{\textbf{H}},\textsf{\textbf{I}} \}
$,
as well as
$
\textsf{\textbf{A}} \{  \textsf{\textbf{G}},\textsf{\textbf{H}},\textsf{\textbf{I}} \}
=
\textsf{\textbf{D}} \{  \textsf{\textbf{E}},\textsf{\textbf{F}},\textsf{\textbf{G}} \}
=1
$, and their cyclic permutations.

For the configuration depicted in Fig.~\ref{2010-pc09-f1}c),
contextuality predicts that there exist experimental outcomes with
$
\textsf{\textbf{A}} \{  \textsf{\textbf{B}},\textsf{\textbf{C}},\textsf{\textbf{D}} \}
\neq
\textsf{\textbf{A}} \{  \textsf{\textbf{B}},\textsf{\textbf{E}},\textsf{\textbf{F}} \}
$,
as well as
$
\textsf{\textbf{B}} \{  \textsf{\textbf{A}},\textsf{\textbf{C}},\textsf{\textbf{D}} \}
\neq
\textsf{\textbf{B}} \{  \textsf{\textbf{A}},\textsf{\textbf{E}},\textsf{\textbf{F}} \}
$.
Again, this is not predicted quantum mechanically~\cite{svozil:040102}.

Experiment will clarify and decide the contradiction
between the predictions by the contextuality assumption and quantum mechanics,
but it is not too unreasonable to suspect that the quantum predictions will prevail.
As a consequence, and subject to experimental falsification,
any {\it ad hoc} ``ontic'' contextuality assumption might turn out to be physically unfounded.

One may argue that quantum contextuality only ``appears'' if measurement configurations are encountered which do not allow a
set of two-valued states. The same might be said  for measurement configurations allowing only a ``meager'' set of two-valued states which
cannot be used for the construction of any homomorphic (i.e. preseving relations and operations among quantum propositions) embedding
into a classical (Boolean) algebra.
Alas, configurations of observables such as the one depicted  in Fig.~\ref{2010-pc09-f1}a)
are just subconfigurations of proofs of the Kochen-Specker theorem~\cite{kochen1}, in particular their
$\Gamma_2$ and $\Gamma_3$; so it would be difficult to imagine
why Fig.~\ref{2010-pc09-f1}a) feature context independence because of the experimenter takes into account only {\em two} contexts,
whereas context dependence is encountered when the experimenter has in mind, say, the {\em entire} structure of
all the 117 Kochen-Specker contexts contained in $\Gamma_3$.

\section{Context translation principle}

In view of the inapplicability of the quantum contextuality assumption and the fact that,
although quantized systems can only be prepared in a certain single context\footnote{
We would even go so far to speculate that
the ignorance of state preparation resulting in mixed states is an epistemic, not ontologic, one.
Thus all quantum states are ``ontologically'' pure.}
quantized systems yield measurement results when measured ``along'' different, nonmatching context,
one may speculate that the measurement apparatus must be capable of ``translating''
between the preparation context and the measurement context~\cite{svozil-2003-garda}.
Variation of the capabilities of the measurement apparatus to translate nonmatching quantum contexts
with its physical condition  yields possibilities to detect this mechanism.

In this scenario, stochasticity is introduced {\it via} the context translation process;
albeit not necessarily an irreversible,
irreducible one, as the unitary quantum state evolution (in-between measurements)
is  deterministic, reversible and one-to-one~\cite{everett}.
Nevertheless, one may further speculate
that, at least for finite experimental time series and for finite algorithmic tests,
any such quasi-deterministic form of stochasticity will result in very similar statistical behaviors
as is predicted for acausality.

Context translation might present an ``epistemic'' contextuality,
since the ``complete disposition  of the measurement apparatus'' (see Bell~\cite[Sec.~5]{bell-66})
may enter in the translation function $\tau$ formalizing the ``state reduction''
\begin{equation}
{\bm \rho}  \longrightarrow \tau_{\textsf{{D}}\left(\textsf{\textbf{X}},\textsf{\textbf{Y}}\right)}  ({\bm \rho}) \in S_\textsf{\textbf{X}},
\label{2010-pc09-e1}
\end{equation}
where
${\bm \rho}$ stands for the quantum state,
$S_\textsf{\textbf{X}}$ for the spectrum of the operator $\textsf{\textbf{X}}$,
$\textsf{{D}}$ for the ``disposition of the apparatus,''
$\textsf{\textbf{X}}$ for the observable
and $\textsf{\textbf{Y}}$ for the context.

In general, even in the absence of some concrete ``translation mechanism,''
$\tau$ is subject to some probabilistic constraints, such as Malus' law~\cite{zeil-bruk-99}.
In order to be able to account for the nonlocal quantum correlation functions even at space-like separations~\cite{wjswz-98}
$\tau$ should also be nonlocal.
Ideally, if preparation and measurement context match, and if
${\bm \rho}$ is in some eigenstate $\textsf{\textbf{E}}_i$ of $\textsf{\textbf{X}}$ with an associated eigenvalue $x_i$,
then  Eq.~(\ref{2010-pc09-e1}) reduces to its context and apparatus independent form
$\textsf{\textbf{E}}_i  \longrightarrow \tau_{\textsf{{D}}\left(\textsf{\textbf{X}},\textsf{\textbf{Y}}\right)}
(\textsf{\textbf{E}}_i) =  \tau_{\textsf{\textbf{X}}}  (\textsf{\textbf{E}}_i) =x_i$ for all $\textsf{\textbf{E}}_i$
in the spectral sum
$\textsf{\textbf{X}} = \sum_i x_i \textsf{\textbf{E}}_i$.
This reduction postulate appears to be the reason for an absence of contextuality in the ``explosion view'' type configurations
discussed above.

For all the other cases, the measurement apparatus will introduce a stochastic element
which, in this scenario, is the reason for the quantum indeterminism of individual events.
Of course, the {\em degree of stochasticity} will depend on the context mismatch,
and on the ``disposition of the apparatus.''
But again, as for the {\em ad hoc} ``ontic'' type of contextuality discussed above,
in no way can the measurement outcome of an individual particle be completely determined
by a pre-existing element of physical reality~\cite{epr} of that particle alone.
In this sense, as only observables associated with one context have a definite value and all other observables have none,
one is  lead to a quasi-classical ``effective value indefiniteness,''
giving rise to a natural classical theory not requiring value definiteness.

\section{Summary}

We have discussed the ``current state of affairs'' with regard to the interpretation of quantum value indefiniteness,
and the limited  operationalizability of its interpretation in terms of {\em context dependence (contextuality)} of observables.
Of course, due to complementarity, quantum counterfactuals are not directly simultaneously measurable;
and thus --- despite the prevalence of counterfactuals in quantum information, communication and computation theory ---
anyone considering their physical existence is,
to paraphrase von Neumann's words~\cite{von-neumann1}, at least empirically, {\em ``in a state of  sin.''}

In any case, the absence of classical interpretations of the quantum formalism,
and in particular the strongest expression of it
---
the absence of any global truth function for quantum systems of three or more mutually exclusive outcomes
---
presents the possibility to render a quantum random number generator
by preparing a quantum state in a particular context and measuring it in another.
As has been pointed out already by Calude and the author~\cite{2008-cal-svo}, the resulting measurement outcomes
are ``quantum certified'' (i.e., true with respect to the validity of quantum mechanics) and do not correspond to any
pre-existing physical observable of the ``isolated'' individual system before the measurement process.
Exactly how this kind of quantum oracle for randomness operates remains open.
One may hold that, somehow, due to the lack of determinacy, this type of randomness emerges ``out of nowhere'' and essentially is
irreducible~\cite{zeil-99,zeil-05_nature_ofQuantum}.
One may also put forward the idea that, at least when complementarity is involved, quantum randomness is rendered by
a quasi-classical context translation which maps an incompatible preparation context into some outcome,
thereby introducing stochasticity.
In any case, for all practical purposes, the resulting oracles for randomness, when subjected to tests~\cite{PhysRevA.82.022102},
might be ``hardly differentiable'' from each other even asymptotically.

\section*{Acknowledgements}
The author would like to thank two anonymous Referees for very valuable observations and suggestions.



%\bibliography{svozil}


%merlin.mbs apsrev4-1.bst 2010-07-25 4.21a (PWD, AO, DPC) hacked
%Control: key (0)
%Control: author (0) dotless jnrlst
%Control: editor formatted (1) identically to author
%Control: production of article title (0) allowed
%Control: page (1) range
%Control: year (0) verbatim
%Control: production of eprint (0) enabled
\begin{thebibliography}{99}%
\makeatletter
\providecommand \@ifxundefined [1]{%
 \@ifx{#1\undefined}
}%
\providecommand \@ifnum [1]{%
 \ifnum #1\expandafter \@firstoftwo
 \else \expandafter \@secondoftwo
 \fi
}%
\providecommand \@ifx [1]{%
 \ifx #1\expandafter \@firstoftwo
 \else \expandafter \@secondoftwo
 \fi
}%
\providecommand \natexlab [1]{#1}%
\providecommand \enquote  [1]{``#1''}%
\providecommand \bibnamefont  [1]{#1}%
\providecommand \bibfnamefont [1]{#1}%
\providecommand \citenamefont [1]{#1}%
\providecommand \href@noop [0]{\@secondoftwo}%
\providecommand \href [0]{\begingroup \@sanitize@url \@href}%
\providecommand \@href[1]{\@@startlink{#1}\@@href}%
\providecommand \@@href[1]{\endgroup#1\@@endlink}%
\providecommand \@sanitize@url [0]{\catcode `\\12\catcode `\$12\catcode
  `\&12\catcode `\#12\catcode `\^12\catcode `\_12\catcode `\%12\relax}%
\providecommand \@@startlink[1]{}%
\providecommand \@@endlink[0]{}%
\providecommand \url  [0]{\begingroup\@sanitize@url \@url }%
\providecommand \@url [1]{\endgroup\@href {#1}{\urlprefix }}%
\providecommand \urlprefix  [0]{URL }%
\providecommand \Eprint [0]{\href }%
\providecommand \doibase [0]{http://dx.doi.org/}%
\providecommand \selectlanguage [0]{\@gobble}%
\providecommand \bibinfo  [0]{\@secondoftwo}%
\providecommand \bibfield  [0]{\@secondoftwo}%
\providecommand \translation [1]{[#1]}%
\providecommand \BibitemOpen [0]{}%
\providecommand \bibitemStop [0]{}%
\providecommand \bibitemNoStop [0]{.\EOS\space}%
\providecommand \EOS [0]{\spacefactor3000\relax}%
\providecommand \BibitemShut  [1]{\csname bibitem#1\endcsname}%
\let\auto@bib@innerbib\@empty
%</preamble>
\bibitem [{\citenamefont {{von Neumann}}(1932)}]{v-neumann-49}%
  \BibitemOpen
  \bibfield  {author} {\bibinfo {author} {\bibfnamefont {John}\ \bibnamefont
  {{von Neumann}}},\ }\href@noop {} {\emph {\bibinfo {title} {{M}athematische
  {G}rundlagen der {Q}uantenmechanik}}}\ (\bibinfo  {publisher} {Springer},\
  \bibinfo {address} {Berlin},\ \bibinfo {year} {1932})\ \bibinfo {note}
  {{E}nglish translation in Ref.~\cite{v-neumann-55}}\BibitemShut {NoStop}%
\bibitem [{\citenamefont {Birkhoff}\ and\ \citenamefont {{von
  Neumann}}(1936)}]{birkhoff-36}%
  \BibitemOpen
  \bibfield  {author} {\bibinfo {author} {\bibfnamefont {Garrett}\ \bibnamefont
  {Birkhoff}}\ and\ \bibinfo {author} {\bibfnamefont {John}\ \bibnamefont {{von
  Neumann}}},\ }\bibfield  {title} {\enquote {\bibinfo {title} {The logic of
  quantum mechanics},}\ }\href {\doibase 10.2307/1968621} {\bibfield  {journal}
  {\bibinfo  {journal} {Annals of Mathematics}\ }\textbf {\bibinfo {volume}
  {37}},\ \bibinfo {pages} {823--843} (\bibinfo {year} {1936})}\BibitemShut
  {NoStop}%
\bibitem [{\citenamefont {Calude}\ \emph {et~al.}(1999)\citenamefont {Calude},
  \citenamefont {Hertling},\ and\ \citenamefont {Svozil}}]{CalHerSvo}%
  \BibitemOpen
  \bibfield  {author} {\bibinfo {author} {\bibfnamefont {Cristian}\
  \bibnamefont {Calude}}, \bibinfo {author} {\bibfnamefont {Peter}\
  \bibnamefont {Hertling}}, \ and\ \bibinfo {author} {\bibfnamefont {Karl}\
  \bibnamefont {Svozil}},\ }\bibfield  {title} {\enquote {\bibinfo {title}
  {Embedding quantum universes into classical ones},}\ }\href {\doibase
  10.1023/A:1018862730956} {\bibfield  {journal} {\bibinfo  {journal}
  {Foundations of Physics}\ }\textbf {\bibinfo {volume} {29}},\ \bibinfo
  {pages} {349--379} (\bibinfo {year} {1999})}\BibitemShut {NoStop}%
\bibitem [{\citenamefont {Specker}(1960)}]{specker-60}%
  \BibitemOpen
  \bibfield  {author} {\bibinfo {author} {\bibfnamefont {Ernst}\ \bibnamefont
  {Specker}},\ }\bibfield  {title} {\enquote {\bibinfo {title} {{D}ie {L}ogik
  nicht gleichzeitig entscheidbarer {A}ussagen},}\ }\href {\doibase
  10.1111/j.1746-8361.1960.tb00422.x} {\bibfield  {journal} {\bibinfo
  {journal} {Dialectica}\ }\textbf {\bibinfo {volume} {14}},\ \bibinfo {pages}
  {239--246} (\bibinfo {year} {1960})},\ \bibinfo {note} {reprinted in
  Ref.~\cite[pp. 175--182]{specker-ges}; {E}nglish translation: {\it The logic
  of propositions which are not simultaneously decidable}, Reprinted in
  Ref.~\cite[pp. 135-140]{hooker}}\BibitemShut {NoStop}%
\bibitem [{\citenamefont {Kochen}\ and\ \citenamefont
  {Specker}(1967)}]{kochen1}%
  \BibitemOpen
  \bibfield  {author} {\bibinfo {author} {\bibfnamefont {Simon}\ \bibnamefont
  {Kochen}}\ and\ \bibinfo {author} {\bibfnamefont {Ernst~P.}\ \bibnamefont
  {Specker}},\ }\bibfield  {title} {\enquote {\bibinfo {title} {The problem of
  hidden variables in quantum mechanics},}\ }\href {\doibase
  10.1512/iumj.1968.17.17004} {\bibfield  {journal} {\bibinfo  {journal}
  {Journal of Mathematics and Mechanics (now Indiana University Mathematics
  Journal)}\ }\textbf {\bibinfo {volume} {17}},\ \bibinfo {pages} {59--87}
  (\bibinfo {year} {1967})},\ \bibinfo {note} {reprinted in Ref.~\cite[pp.
  235--263]{specker-ges}}\BibitemShut {NoStop}%
\bibitem [{\citenamefont {Zierler}\ and\ \citenamefont
  {Schlessinger}(1965)}]{ZirlSchl-65}%
  \BibitemOpen
  \bibfield  {author} {\bibinfo {author} {\bibfnamefont {Neal}\ \bibnamefont
  {Zierler}}\ and\ \bibinfo {author} {\bibfnamefont {Michael}\ \bibnamefont
  {Schlessinger}},\ }\bibfield  {title} {\enquote {\bibinfo {title} {Boolean
  embeddings of orthomodular sets and quantum logic},}\ }\href@noop {}
  {\bibfield  {journal} {\bibinfo  {journal} {Duke Mathematical Journal}\
  }\textbf {\bibinfo {volume} {32}},\ \bibinfo {pages} {251--262} (\bibinfo
  {year} {1965})}\BibitemShut {NoStop}%
\bibitem [{\citenamefont {Alda}(1980)}]{Alda}%
  \BibitemOpen
  \bibfield  {author} {\bibinfo {author} {\bibfnamefont {V\'aclav}\
  \bibnamefont {Alda}},\ }\bibfield  {title} {\enquote {\bibinfo {title} {On\/
  {\rm 0-1} measures for projectors {I}},}\ }\href {http://dml.cz/dmlcz/103871}
  {\bibfield  {journal} {\bibinfo  {journal} {Aplikace matematiky (Applications
  of Mathematics)}\ }\textbf {\bibinfo {volume} {25}},\ \bibinfo {pages}
  {373--374} (\bibinfo {year} {1980})}\BibitemShut {NoStop}%
\bibitem [{\citenamefont {Alda}(1981)}]{Alda2}%
  \BibitemOpen
  \bibfield  {author} {\bibinfo {author} {\bibfnamefont {V\'aclav}\
  \bibnamefont {Alda}},\ }\bibfield  {title} {\enquote {\bibinfo {title} {On\/
  {\rm 0-1} measures for projectors {II}},}\ }\href
  {http://dml.cz/dmlcz/103894} {\bibfield  {journal} {\bibinfo  {journal}
  {Aplikace matematiky (Applications of Mathematics)}\ }\textbf {\bibinfo
  {volume} {26}},\ \bibinfo {pages} {57--58} (\bibinfo {year}
  {1981})}\BibitemShut {NoStop}%
\bibitem [{\citenamefont {Kamber}(1964)}]{kamber64}%
  \BibitemOpen
  \bibfield  {author} {\bibinfo {author} {\bibfnamefont {Franz}\ \bibnamefont
  {Kamber}},\ }\bibfield  {title} {\enquote {\bibinfo {title} {Die {S}truktur
  des {A}ussagenkalk{\"{u}}ls in einer physikalischen {T}heorie},}\ }\href@noop
  {} {\bibfield  {journal} {\bibinfo  {journal} {{N}achrichten der {A}kademie
  der {W}issenschaften in {G}{\"{o}}ttingen, {M}athematisch-{P}hysikalische
  {K}lasse}\ }\textbf {\bibinfo {volume} {10}},\ \bibinfo {pages} {103--124}
  (\bibinfo {year} {1964})}\BibitemShut {NoStop}%
\bibitem [{\citenamefont {Kamber}(1965)}]{kamber65}%
  \BibitemOpen
  \bibfield  {author} {\bibinfo {author} {\bibfnamefont {Franz}\ \bibnamefont
  {Kamber}},\ }\bibfield  {title} {\enquote {\bibinfo {title} {Zweiwertige
  {W}ahrscheinlichkeitsfunktionen auf orthokomplement{\"{a}}ren
  {V}erb{\"{a}}nden},}\ }\href {\doibase 10.1007/BF01359975} {\bibfield
  {journal} {\bibinfo  {journal} {Mathematische Annalen}\ }\textbf {\bibinfo
  {volume} {158}},\ \bibinfo {pages} {158--196} (\bibinfo {year}
  {1965})}\BibitemShut {NoStop}%
\bibitem [{\citenamefont {Peres}(1991)}]{peres-91}%
  \BibitemOpen
  \bibfield  {author} {\bibinfo {author} {\bibfnamefont {Asher}\ \bibnamefont
  {Peres}},\ }\bibfield  {title} {\enquote {\bibinfo {title} {Two simple proofs
  of the {K}ochen-{S}pecker theorem},}\ }\href {\doibase
  10.1088/0305-4470/24/4/003} {\bibfield  {journal} {\bibinfo  {journal}
  {Journal of Physics A: Mathematical and General}\ }\textbf {\bibinfo {volume}
  {24}},\ \bibinfo {pages} {L175--L178} (\bibinfo {year} {1991})},\ \bibinfo
  {note} {reprinted in Ref.~\cite[pp. 186-200]{peres}}\BibitemShut {NoStop}%
\bibitem [{\citenamefont {Mermin}(1993)}]{mermin-93}%
  \BibitemOpen
  \bibfield  {author} {\bibinfo {author} {\bibfnamefont {N.~D.}\ \bibnamefont
  {Mermin}},\ }\bibfield  {title} {\enquote {\bibinfo {title} {Hidden variables
  and the two theorems of {J}ohn {B}ell},}\ }\href {\doibase
  10.1103/RevModPhys.65.803} {\bibfield  {journal} {\bibinfo  {journal}
  {Reviews of Modern Physics}\ }\textbf {\bibinfo {volume} {65}},\ \bibinfo
  {pages} {803--815} (\bibinfo {year} {1993})}\BibitemShut {NoStop}%
\bibitem [{\citenamefont {Svozil}\ and\ \citenamefont
  {Tkadlec}(1996)}]{svozil-tkadlec}%
  \BibitemOpen
  \bibfield  {author} {\bibinfo {author} {\bibfnamefont {Karl}\ \bibnamefont
  {Svozil}}\ and\ \bibinfo {author} {\bibfnamefont {Josef}\ \bibnamefont
  {Tkadlec}},\ }\bibfield  {title} {\enquote {\bibinfo {title} {Greechie
  diagrams, nonexistence of measures in quantum logics and {K}ochen--{S}pecker
  type constructions},}\ }\href {\doibase 10.1063/1.531710} {\bibfield
  {journal} {\bibinfo  {journal} {Journal of Mathematical Physics}\ }\textbf
  {\bibinfo {volume} {37}},\ \bibinfo {pages} {5380--5401} (\bibinfo {year}
  {1996})}\BibitemShut {NoStop}%
\bibitem [{\citenamefont {Cabello}\ \emph {et~al.}(1996)\citenamefont
  {Cabello}, \citenamefont {Estebaranz},\ and\ \citenamefont
  {Garc{\'{i}}a-Alcaine}}]{cabello-96}%
  \BibitemOpen
  \bibfield  {author} {\bibinfo {author} {\bibfnamefont {Ad{\'{a}}n}\
  \bibnamefont {Cabello}}, \bibinfo {author} {\bibfnamefont {Jos{\'{e}}~M.}\
  \bibnamefont {Estebaranz}}, \ and\ \bibinfo {author} {\bibfnamefont
  {G.}~\bibnamefont {Garc{\'{i}}a-Alcaine}},\ }\bibfield  {title} {\enquote
  {\bibinfo {title} {{B}ell-{K}ochen-{S}pecker theorem: A proof with 18
  vectors},}\ }\href {\doibase 10.1016/0375-9601(96)00134-X} {\bibfield
  {journal} {\bibinfo  {journal} {Physics Letters A}\ }\textbf {\bibinfo
  {volume} {212}},\ \bibinfo {pages} {183--187} (\bibinfo {year}
  {1996})}\BibitemShut {NoStop}%
\bibitem [{\citenamefont {Cabello}(2008)}]{cabello:210401}%
  \BibitemOpen
  \bibfield  {author} {\bibinfo {author} {\bibfnamefont {Adan}\ \bibnamefont
  {Cabello}},\ }\bibfield  {title} {\enquote {\bibinfo {title} {Experimentally
  testable state-independent quantum contextuality},}\ }\href {\doibase
  10.1103/PhysRevLett.101.210401} {\bibfield  {journal} {\bibinfo  {journal}
  {Physical Review Letters}\ }\textbf {\bibinfo {volume} {101}},\ \bibinfo
  {eid} {210401} (\bibinfo {year} {2008})}\BibitemShut {NoStop}%
\bibitem [{\citenamefont {Gleason}(1957)}]{Gleason}%
  \BibitemOpen
  \bibfield  {author} {\bibinfo {author} {\bibfnamefont {Andrew~M.}\
  \bibnamefont {Gleason}},\ }\bibfield  {title} {\enquote {\bibinfo {title}
  {Measures on the closed subspaces of a {H}ilbert space},}\ }\href {\doibase
  10.1512/iumj.1957.6.56050"} {\bibfield  {journal} {\bibinfo  {journal}
  {Journal of Mathematics and Mechanics (now Indiana University Mathematics
  Journal)}\ }\textbf {\bibinfo {volume} {6}},\ \bibinfo {pages} {885--893}
  (\bibinfo {year} {1957})}\BibitemShut {NoStop}%
\bibitem [{\citenamefont {Pitowsky}(1998)}]{pitowsky:218}%
  \BibitemOpen
  \bibfield  {author} {\bibinfo {author} {\bibfnamefont {Itamar}\ \bibnamefont
  {Pitowsky}},\ }\bibfield  {title} {\enquote {\bibinfo {title} {Infinite and
  finite {G}leason's theorems and the logic of indeterminacy},}\ }\href
  {\doibase 10.1063/1.532334} {\bibfield  {journal} {\bibinfo  {journal}
  {Journal of Mathematical Physics}\ }\textbf {\bibinfo {volume} {39}},\
  \bibinfo {pages} {218--228} (\bibinfo {year} {1998})}\BibitemShut {NoStop}%
\bibitem [{\citenamefont {Richman}\ and\ \citenamefont
  {Bridges}(1999)}]{rich-bridge}%
  \BibitemOpen
  \bibfield  {author} {\bibinfo {author} {\bibfnamefont {Fred}\ \bibnamefont
  {Richman}}\ and\ \bibinfo {author} {\bibfnamefont {Douglas}\ \bibnamefont
  {Bridges}},\ }\bibfield  {title} {\enquote {\bibinfo {title} {A constructive
  proof of {G}leason's theorem},}\ }\href {\doibase 10.1006/jfan.1998.3372}
  {\bibfield  {journal} {\bibinfo  {journal} {Journal of Functional Analysis}\
  }\textbf {\bibinfo {volume} {162}},\ \bibinfo {pages} {287--312} (\bibinfo
  {year} {1999})}\BibitemShut {NoStop}%
\bibitem [{\citenamefont {Dvure{\v{c}}enskij}(1993)}]{r:dvur-93}%
  \BibitemOpen
  \bibfield  {author} {\bibinfo {author} {\bibfnamefont {Anatolij}\
  \bibnamefont {Dvure{\v{c}}enskij}},\ }\href@noop {} {\emph {\bibinfo {title}
  {{G}leason's Theorem and Its Applications}}}\ (\bibinfo  {publisher} {Kluwer
  Academic Publishers},\ \bibinfo {address} {Dordrecht},\ \bibinfo {year}
  {1993})\BibitemShut {NoStop}%
\bibitem [{\citenamefont {Svozil}(2009{\natexlab{a}})}]{svozil-2006-omni}%
  \BibitemOpen
  \bibfield  {author} {\bibinfo {author} {\bibfnamefont {Karl}\ \bibnamefont
  {Svozil}},\ }\bibfield  {title} {\enquote {\bibinfo {title} {Quantum
  scholasticism: On quantum contexts, counterfactuals, and the absurdities of
  quantum omniscience},}\ }\href {\doibase 10.1016/j.ins.2008.06.012}
  {\bibfield  {journal} {\bibinfo  {journal} {Information Sciences}\ }\textbf
  {\bibinfo {volume} {179}},\ \bibinfo {pages} {535--541} (\bibinfo {year}
  {2009}{\natexlab{a}})}\BibitemShut {NoStop}%
\bibitem [{\citenamefont {Pauli}(1958)}]{pauli:58}%
  \BibitemOpen
  \bibfield  {author} {\bibinfo {author} {\bibfnamefont {Wolfgang}\
  \bibnamefont {Pauli}},\ }\bibfield  {title} {\enquote {\bibinfo {title}
  {{D}ie allgemeinen {P}rinzipien der {W}ellenmechanik},}\ }in\ \href@noop {}
  {\emph {\bibinfo {booktitle} {{H}andbuch der {P}hysik. {B}and {V}, {T}eil 1.
  {P}rinzipien der {Q}uantentheorie {I}}}},\ \bibinfo {editor} {edited by\
  \bibinfo {editor} {\bibfnamefont {S.}~\bibnamefont {Fl{\"{u}}gge}}}\
  (\bibinfo  {publisher} {Springer},\ \bibinfo {address} {Berlin,
  G{\"{o}}ttingen and Heidelberg},\ \bibinfo {year} {1958})\ pp.\ \bibinfo
  {pages} {1--168}\BibitemShut {NoStop}%
\bibitem [{\citenamefont {Einstein}\ \emph {et~al.}(1935)\citenamefont
  {Einstein}, \citenamefont {Podolsky},\ and\ \citenamefont {Rosen}}]{epr}%
  \BibitemOpen
  \bibfield  {author} {\bibinfo {author} {\bibfnamefont {Albert}\ \bibnamefont
  {Einstein}}, \bibinfo {author} {\bibfnamefont {Boris}\ \bibnamefont
  {Podolsky}}, \ and\ \bibinfo {author} {\bibfnamefont {Nathan}\ \bibnamefont
  {Rosen}},\ }\bibfield  {title} {\enquote {\bibinfo {title} {Can
  quantum-mechanical description of physical reality be considered complete?}}\
  }\href {\doibase 10.1103/PhysRev.47.777} {\bibfield  {journal} {\bibinfo
  {journal} {Physical Review}\ }\textbf {\bibinfo {volume} {47}},\ \bibinfo
  {pages} {777--780} (\bibinfo {year} {1935})}\BibitemShut {NoStop}%
\bibitem [{\citenamefont {Weihs}\ \emph {et~al.}(1998)\citenamefont {Weihs},
  \citenamefont {Jennewein}, \citenamefont {Simon}, \citenamefont
  {Weinfurter},\ and\ \citenamefont {Zeilinger}}]{wjswz-98}%
  \BibitemOpen
  \bibfield  {author} {\bibinfo {author} {\bibfnamefont {Gregor}\ \bibnamefont
  {Weihs}}, \bibinfo {author} {\bibfnamefont {Thomas}\ \bibnamefont
  {Jennewein}}, \bibinfo {author} {\bibfnamefont {Christoph}\ \bibnamefont
  {Simon}}, \bibinfo {author} {\bibfnamefont {Harald}\ \bibnamefont
  {Weinfurter}}, \ and\ \bibinfo {author} {\bibfnamefont {Anton}\ \bibnamefont
  {Zeilinger}},\ }\bibfield  {title} {\enquote {\bibinfo {title} {Violation of
  {B}ell's inequality under strict {E}instein locality conditions},}\ }\href
  {\doibase 10.1103/PhysRevLett.81.5039} {\bibfield  {journal} {\bibinfo
  {journal} {Physical Review Letters}\ }\textbf {\bibinfo {volume} {81}},\
  \bibinfo {pages} {5039--5043} (\bibinfo {year} {1998})}\BibitemShut {NoStop}%
\bibitem [{\citenamefont
  {Schr{\"{o}}dinger}(1935{\natexlab{a}})}]{schrodinger}%
  \BibitemOpen
  \bibfield  {author} {\bibinfo {author} {\bibfnamefont {Erwin}\ \bibnamefont
  {Schr{\"{o}}dinger}},\ }\bibfield  {title} {\enquote {\bibinfo {title} {Die
  gegenw{\"{a}}rtige {S}ituation in der {Q}uantenmechanik},}\ }\href {\doibase
  10.1007/BF01491891, 10.1007/BF01491914, 10.1007/BF01491987} {\bibfield
  {journal} {\bibinfo  {journal} {Naturwissenschaften}\ }\textbf {\bibinfo
  {volume} {23}},\ \bibinfo {pages} {807--812, 823--828, 844--849} (\bibinfo
  {year} {1935}{\natexlab{a}})},\ \bibinfo {note} {{E}nglish translation in
  Ref.~\cite{trimmer} and in Ref.~\cite[pp.
  152-167]{wheeler-Zurek:83}}\BibitemShut {NoStop}%
\bibitem [{\citenamefont
  {Schr{\"{o}}dinger}(1935{\natexlab{b}})}]{CambridgeJournals:1737068}%
  \BibitemOpen
  \bibfield  {author} {\bibinfo {author} {\bibfnamefont {Erwin}\ \bibnamefont
  {Schr{\"{o}}dinger}},\ }\bibfield  {title} {\enquote {\bibinfo {title}
  {Discussion of probability relations between separated systems},}\ }\href
  {\doibase 10.1017/S0305004100013554} {\bibfield  {journal} {\bibinfo
  {journal} {Mathematical Proceedings of the Cambridge Philosophical Society}\
  }\textbf {\bibinfo {volume} {31}},\ \bibinfo {pages} {555--563} (\bibinfo
  {year} {1935}{\natexlab{b}})}\BibitemShut {NoStop}%
\bibitem [{\citenamefont
  {Schr{\"{o}}dinger}(1936)}]{CambridgeJournals:2027212}%
  \BibitemOpen
  \bibfield  {author} {\bibinfo {author} {\bibfnamefont {Erwin}\ \bibnamefont
  {Schr{\"{o}}dinger}},\ }\bibfield  {title} {\enquote {\bibinfo {title}
  {Probability relations between separated systems},}\ }\href {\doibase
  10.1017/S0305004100019137} {\bibfield  {journal} {\bibinfo  {journal}
  {Mathematical Proceedings of the Cambridge Philosophical Society}\ }\textbf
  {\bibinfo {volume} {32}},\ \bibinfo {pages} {446--452} (\bibinfo {year}
  {1936})}\BibitemShut {NoStop}%
\bibitem [{\citenamefont {Svozil}(2006)}]{svozil-2006-uniquenessprinciple}%
  \BibitemOpen
  \bibfield  {author} {\bibinfo {author} {\bibfnamefont {Karl}\ \bibnamefont
  {Svozil}},\ }\bibfield  {title} {\enquote {\bibinfo {title} {Are simultaneous
  {B}ell measurements possible?}}\ }\href {\doibase 10.1088/1367-2630/8/3/039}
  {\bibfield  {journal} {\bibinfo  {journal} {New Journal of Physics}\ }\textbf
  {\bibinfo {volume} {8}},\ \bibinfo {pages} {39, 1--8} (\bibinfo {year}
  {2006})},\ \Eprint {http://arxiv.org/abs/quant-ph/0401113} {quant-ph/0401113}
  \BibitemShut {NoStop}%
\bibitem [{\citenamefont {Murata}(1990)}]{MuBae-90}%
  \BibitemOpen
  \bibfield  {author} {\bibinfo {author} {\bibfnamefont {T.}~\bibnamefont
  {Murata}},\ }\bibfield  {title} {\enquote {\bibinfo {title} {Quantum
  nonlocality without counterfactual definiteness?}}\ }\href {\doibase
  10.1007/BF00769704} {\bibfield  {journal} {\bibinfo  {journal} {Foundations
  of Physics Letters}\ }\textbf {\bibinfo {volume} {3}},\ \bibinfo {pages}
  {325--342} (\bibinfo {year} {1990})}\BibitemShut {NoStop}%
\bibitem [{\citenamefont {Peres}(1978)}]{peres222}%
  \BibitemOpen
  \bibfield  {author} {\bibinfo {author} {\bibfnamefont {Asher}\ \bibnamefont
  {Peres}},\ }\bibfield  {title} {\enquote {\bibinfo {title} {Unperformed
  experiments have no results},}\ }\href {\doibase 10.1119/1.11393} {\bibfield
  {journal} {\bibinfo  {journal} {American Journal of Physics}\ }\textbf
  {\bibinfo {volume} {46}},\ \bibinfo {pages} {745--747} (\bibinfo {year}
  {1978})}\BibitemShut {NoStop}%
\bibitem [{\citenamefont {Kragh}(1997)}]{Kragh-1997AHESradioact}%
  \BibitemOpen
  \bibfield  {author} {\bibinfo {author} {\bibfnamefont {Helge}\ \bibnamefont
  {Kragh}},\ }\bibfield  {title} {\enquote {\bibinfo {title} {The origin of
  radioactivity: from solvable problem to unsolved non-problem},}\ }\href
  {\doibase 10.1007/BF00374597} {\bibfield  {journal} {\bibinfo  {journal}
  {Archive for History of Exact Sciences}\ }\textbf {\bibinfo {volume} {50}},\
  \bibinfo {pages} {331--358} (\bibinfo {year} {1997})}\BibitemShut {NoStop}%
\bibitem [{\citenamefont {Kragh}(2009)}]{Kragh-2009_RePoss5}%
  \BibitemOpen
  \bibfield  {author} {\bibinfo {author} {\bibfnamefont {Helge}\ \bibnamefont
  {Kragh}},\ }\href@noop {} {\enquote {\bibinfo {title} {Subatomic determinism
  and causal models of radioactive decay, 1903-1923},}\ } (\bibinfo {year}
  {2009}),\ \bibinfo {note} {rePoSS: Research Publications on Science Studies
  5. Department of Science Studies, University of Aarhus}\BibitemShut {NoStop}%
\bibitem [{\citenamefont {Svozil}(1990)}]{svozil-qct}%
  \BibitemOpen
  \bibfield  {author} {\bibinfo {author} {\bibfnamefont {Karl}\ \bibnamefont
  {Svozil}},\ }\bibfield  {title} {\enquote {\bibinfo {title} {The quantum coin
  toss---testing microphysical undecidability},}\ }\href {\doibase
  10.1016/0375-9601(90)90408-G} {\bibfield  {journal} {\bibinfo  {journal}
  {Physics Letters A}\ }\textbf {\bibinfo {volume} {143}},\ \bibinfo {pages}
  {433--437} (\bibinfo {year} {1990})}\BibitemShut {NoStop}%
\bibitem [{\citenamefont {Rarity}\ \emph {et~al.}(1994)\citenamefont {Rarity},
  \citenamefont {Owens},\ and\ \citenamefont {Tapster}}]{rarity-94}%
  \BibitemOpen
  \bibfield  {author} {\bibinfo {author} {\bibfnamefont {J.~G.}\ \bibnamefont
  {Rarity}}, \bibinfo {author} {\bibfnamefont {M.~P.~C.}\ \bibnamefont
  {Owens}}, \ and\ \bibinfo {author} {\bibfnamefont {P.~R.}\ \bibnamefont
  {Tapster}},\ }\bibfield  {title} {\enquote {\bibinfo {title} {Quantum
  random-number generation and key sharing},}\ }\href {\doibase
  10.1080/09500349414552281} {\bibfield  {journal} {\bibinfo  {journal}
  {Journal of Modern Optics}\ }\textbf {\bibinfo {volume} {41}},\ \bibinfo
  {pages} {2435--2444} (\bibinfo {year} {1994})}\BibitemShut {NoStop}%
\bibitem [{\citenamefont {Jennewein}\ \emph {et~al.}(2000)\citenamefont
  {Jennewein}, \citenamefont {Achleitner}, \citenamefont {Weihs}, \citenamefont
  {Weinfurter},\ and\ \citenamefont {Zeilinger}}]{zeilinger:qct}%
  \BibitemOpen
  \bibfield  {author} {\bibinfo {author} {\bibfnamefont {Thomas}\ \bibnamefont
  {Jennewein}}, \bibinfo {author} {\bibfnamefont {Ulrich}\ \bibnamefont
  {Achleitner}}, \bibinfo {author} {\bibfnamefont {Gregor}\ \bibnamefont
  {Weihs}}, \bibinfo {author} {\bibfnamefont {Harald}\ \bibnamefont
  {Weinfurter}}, \ and\ \bibinfo {author} {\bibfnamefont {Anton}\ \bibnamefont
  {Zeilinger}},\ }\bibfield  {title} {\enquote {\bibinfo {title} {A fast and
  compact quantum random number generator},}\ }\href {\doibase
  10.1063/1.1150518} {\bibfield  {journal} {\bibinfo  {journal} {Review of
  Scientific Instruments}\ }\textbf {\bibinfo {volume} {71}},\ \bibinfo {pages}
  {1675--1680} (\bibinfo {year} {2000})},\ \Eprint
  {http://arxiv.org/abs/quant-ph/9912118} {quant-ph/9912118} \BibitemShut
  {NoStop}%
\bibitem [{\citenamefont {Stefanov}\ \emph {et~al.}(2000)\citenamefont
  {Stefanov}, \citenamefont {Gisin}, \citenamefont {Guinnard}, \citenamefont
  {Guinnard},\ and\ \citenamefont {Zbinden}}]{stefanov-2000}%
  \BibitemOpen
  \bibfield  {author} {\bibinfo {author} {\bibfnamefont {Andr{\'{e}}}\
  \bibnamefont {Stefanov}}, \bibinfo {author} {\bibfnamefont {Nicolas}\
  \bibnamefont {Gisin}}, \bibinfo {author} {\bibfnamefont {Olivier}\
  \bibnamefont {Guinnard}}, \bibinfo {author} {\bibfnamefont {Laurent}\
  \bibnamefont {Guinnard}}, \ and\ \bibinfo {author} {\bibfnamefont {Hugo}\
  \bibnamefont {Zbinden}},\ }\bibfield  {title} {\enquote {\bibinfo {title}
  {Optical quantum random number generator},}\ }\href {\doibase
  10.1080/095003400147908} {\bibfield  {journal} {\bibinfo  {journal} {Journal
  of Modern Optics}\ }\textbf {\bibinfo {volume} {47}},\ \bibinfo {pages}
  {595--598} (\bibinfo {year} {2000})}\BibitemShut {NoStop}%
\bibitem [{\citenamefont {Hai-Qiang}\ \emph {et~al.}(2004)\citenamefont
  {Hai-Qiang}, \citenamefont {Su-Mei}, \citenamefont {Da}, \citenamefont
  {Jun-Tao}, \citenamefont {Ling-Ling}, \citenamefont {Yan-Xue},\ and\
  \citenamefont {Ling-An}}]{0256-307X-21-10-027}%
  \BibitemOpen
  \bibfield  {author} {\bibinfo {author} {\bibfnamefont {Ma}~\bibnamefont
  {Hai-Qiang}}, \bibinfo {author} {\bibfnamefont {Wang}\ \bibnamefont
  {Su-Mei}}, \bibinfo {author} {\bibfnamefont {Zhang}\ \bibnamefont {Da}},
  \bibinfo {author} {\bibfnamefont {Chang}\ \bibnamefont {Jun-Tao}}, \bibinfo
  {author} {\bibfnamefont {Ji}~\bibnamefont {Ling-Ling}}, \bibinfo {author}
  {\bibfnamefont {Hou}\ \bibnamefont {Yan-Xue}}, \ and\ \bibinfo {author}
  {\bibfnamefont {Wu}~\bibnamefont {Ling-An}},\ }\bibfield  {title} {\enquote
  {\bibinfo {title} {A random number generator based on quantum entangled
  photon pairs},}\ }\href {\doibase 10.1088/0256-307X/21/10/027} {\bibfield
  {journal} {\bibinfo  {journal} {Chinese Physics Letters}\ }\textbf {\bibinfo
  {volume} {21}},\ \bibinfo {pages} {1961--1964} (\bibinfo {year}
  {2004})}\BibitemShut {NoStop}%
\bibitem [{\citenamefont {Wang}\ \emph {et~al.}(2006)\citenamefont {Wang},
  \citenamefont {Long},\ and\ \citenamefont {Li}}]{wang:056107}%
  \BibitemOpen
  \bibfield  {author} {\bibinfo {author} {\bibfnamefont {P.~X.}\ \bibnamefont
  {Wang}}, \bibinfo {author} {\bibfnamefont {G.~L.}\ \bibnamefont {Long}}, \
  and\ \bibinfo {author} {\bibfnamefont {Y.~S.}\ \bibnamefont {Li}},\
  }\bibfield  {title} {\enquote {\bibinfo {title} {Scheme for a quantum random
  number generator},}\ }\href {\doibase 10.1063/1.2338830} {\bibfield
  {journal} {\bibinfo  {journal} {Journal of Applied Physics}\ }\textbf
  {\bibinfo {volume} {100}},\ \bibinfo {eid} {056107} (\bibinfo {year}
  {2006})}\BibitemShut {NoStop}%
\bibitem [{\citenamefont {Fiorentino}\ \emph {et~al.}(2007)\citenamefont
  {Fiorentino}, \citenamefont {Santori}, \citenamefont {Spillane},
  \citenamefont {Beausoleil},\ and\ \citenamefont {Munro}}]{fiorentino:032334}%
  \BibitemOpen
  \bibfield  {author} {\bibinfo {author} {\bibfnamefont {M.}~\bibnamefont
  {Fiorentino}}, \bibinfo {author} {\bibfnamefont {C.}~\bibnamefont {Santori}},
  \bibinfo {author} {\bibfnamefont {S.~M.}\ \bibnamefont {Spillane}}, \bibinfo
  {author} {\bibfnamefont {R.~G.}\ \bibnamefont {Beausoleil}}, \ and\ \bibinfo
  {author} {\bibfnamefont {W.~J.}\ \bibnamefont {Munro}},\ }\bibfield  {title}
  {\enquote {\bibinfo {title} {Secure self-calibrating quantum random-bit
  generator},}\ }\href {\doibase 10.1103/PhysRevA.75.032334} {\bibfield
  {journal} {\bibinfo  {journal} {Physical Review A}\ }\textbf {\bibinfo
  {volume} {75}},\ \bibinfo {eid} {032334} (\bibinfo {year}
  {2007})}\BibitemShut {NoStop}%
\bibitem [{\citenamefont {Svozil}(2009{\natexlab{b}})}]{svozil-2009-howto}%
  \BibitemOpen
  \bibfield  {author} {\bibinfo {author} {\bibfnamefont {Karl}\ \bibnamefont
  {Svozil}},\ }\bibfield  {title} {\enquote {\bibinfo {title} {Three criteria
  for quantum random-number generators based on beam splitters},}\ }\href
  {\doibase 10.1103/PhysRevA.79.054306} {\bibfield  {journal} {\bibinfo
  {journal} {Physical Review A}\ }\textbf {\bibinfo {volume} {79}},\ \bibinfo
  {eid} {054306} (\bibinfo {year} {2009}{\natexlab{b}})},\ \Eprint
  {http://arxiv.org/abs/arXiv:quant-ph/0903.2744} {arXiv:quant-ph/0903.2744}
  \BibitemShut {NoStop}%
\bibitem [{\citenamefont {Kwon}\ \emph {et~al.}(2009)\citenamefont {Kwon},
  \citenamefont {Cho},\ and\ \citenamefont {Kim}}]{Kwon:09}%
  \BibitemOpen
  \bibfield  {author} {\bibinfo {author} {\bibfnamefont {Osung}\ \bibnamefont
  {Kwon}}, \bibinfo {author} {\bibfnamefont {Young-Wook}\ \bibnamefont {Cho}},
  \ and\ \bibinfo {author} {\bibfnamefont {Yoon-Ho}\ \bibnamefont {Kim}},\
  }\bibfield  {title} {\enquote {\bibinfo {title} {Quantum random number
  generator using photon-number path entanglement},}\ }\href {\doibase
  10.1364/AO.48.001774} {\bibfield  {journal} {\bibinfo  {journal} {Applied
  Optics}\ }\textbf {\bibinfo {volume} {48}},\ \bibinfo {pages} {1774--1778}
  (\bibinfo {year} {2009})}\BibitemShut {NoStop}%
\bibitem [{\citenamefont {Pironio}\ \emph {et~al.}(2010)\citenamefont
  {Pironio}, \citenamefont {Ac{\'i}n}, \citenamefont {Massar}, \citenamefont
  {{Boyer de la Giroday}}, \citenamefont {Matsukevich}, \citenamefont {Maunz},
  \citenamefont {Olmschenk}, \citenamefont {Hayes}, \citenamefont {Luo},
  \citenamefont {Manning},\ and\ \citenamefont {Monroe}}]{10.1038/nature09008}%
  \BibitemOpen
  \bibfield  {author} {\bibinfo {author} {\bibfnamefont {S.}~\bibnamefont
  {Pironio}}, \bibinfo {author} {\bibfnamefont {A.}~\bibnamefont {Ac{\'i}n}},
  \bibinfo {author} {\bibfnamefont {S.}~\bibnamefont {Massar}}, \bibinfo
  {author} {\bibfnamefont {A.}~\bibnamefont {{Boyer de la Giroday}}}, \bibinfo
  {author} {\bibfnamefont {D.~N.}\ \bibnamefont {Matsukevich}}, \bibinfo
  {author} {\bibfnamefont {P.}~\bibnamefont {Maunz}}, \bibinfo {author}
  {\bibfnamefont {S.}~\bibnamefont {Olmschenk}}, \bibinfo {author}
  {\bibfnamefont {D.}~\bibnamefont {Hayes}}, \bibinfo {author} {\bibfnamefont
  {L.}~\bibnamefont {Luo}}, \bibinfo {author} {\bibfnamefont {T.~A.}\
  \bibnamefont {Manning}}, \ and\ \bibinfo {author} {\bibfnamefont
  {C.}~\bibnamefont {Monroe}},\ }\bibfield  {title} {\enquote {\bibinfo {title}
  {Random numbers certified by {B}ell's theorem},}\ }\href {\doibase
  10.1038/nature09008} {\bibfield  {journal} {\bibinfo  {journal} {Nature}\
  }\textbf {\bibinfo {volume} {464}},\ \bibinfo {pages} {1021--1024} (\bibinfo
  {year} {2010})}\BibitemShut {NoStop}%
\bibitem [{\citenamefont {Calude}\ and\ \citenamefont
  {Svozil}(2008)}]{2008-cal-svo}%
  \BibitemOpen
  \bibfield  {author} {\bibinfo {author} {\bibfnamefont {Cristian~S.}\
  \bibnamefont {Calude}}\ and\ \bibinfo {author} {\bibfnamefont {Karl}\
  \bibnamefont {Svozil}},\ }\bibfield  {title} {\enquote {\bibinfo {title}
  {Quantum randomness and value indefiniteness},}\ }\href {\doibase
  10.1166/asl.2008.016} {\bibfield  {journal} {\bibinfo  {journal} {Advanced
  Science Letters}\ }\textbf {\bibinfo {volume} {1}},\ \bibinfo {pages}
  {165--168} (\bibinfo {year} {2008})},\ \Eprint
  {http://arxiv.org/abs/arXiv:quant-ph/0611029} {arXiv:quant-ph/0611029}
  \BibitemShut {NoStop}%
\bibitem [{\citenamefont {Svozil}(2009{\natexlab{c}})}]{svozil-2008-ql}%
  \BibitemOpen
  \bibfield  {author} {\bibinfo {author} {\bibfnamefont {Karl}\ \bibnamefont
  {Svozil}},\ }\bibfield  {title} {\enquote {\bibinfo {title} {Contexts in
  quantum, classical and partition logic},}\ }in\ \href
  {http://arxiv.org/abs/quant-ph/0609209} {\emph {\bibinfo {booktitle}
  {Handbook of Quantum Logic and Quantum Structures}}},\ \bibinfo {editor}
  {edited by\ \bibinfo {editor} {\bibfnamefont {Kurt}\ \bibnamefont
  {Engesser}}, \bibinfo {editor} {\bibfnamefont {Dov~M.}\ \bibnamefont
  {Gabbay}}, \ and\ \bibinfo {editor} {\bibfnamefont {Daniel}\ \bibnamefont
  {Lehmann}}}\ (\bibinfo  {publisher} {Elsevier},\ \bibinfo {address}
  {Amsterdam},\ \bibinfo {year} {2009})\ pp.\ \bibinfo {pages} {551--586},\
  \Eprint {http://arxiv.org/abs/arXiv:quant-ph/0609209}
  {arXiv:quant-ph/0609209} \BibitemShut {NoStop}%
\bibitem [{\citenamefont {Neumark}(1954)}]{neumark-54}%
  \BibitemOpen
  \bibfield  {author} {\bibinfo {author} {\bibfnamefont {M.~A.}\ \bibnamefont
  {Neumark}},\ }\bibfield  {title} {\enquote {\bibinfo {title} {Principles of
  quantum theory},}\ }in\ \href@noop {} {\emph {\bibinfo {booktitle}
  {{S}owjetische {A}rbeiten zur {F}unktionalanalysis. {B}eiheft zur
  {S}owjetwissenschaft}}},\ Vol.~\bibinfo {volume} {44},\ \bibinfo {editor}
  {edited by\ \bibinfo {editor} {\bibfnamefont {Klaus}\ \bibnamefont
  {Matthes}}}\ (\bibinfo  {publisher} {Gesellschaft f{\"{u}}r
  Deutsch-Sowjetische Freundschaft},\ \bibinfo {address} {Berlin},\ \bibinfo
  {year} {1954})\ pp.\ \bibinfo {pages} {195--273}\BibitemShut {NoStop}%
\bibitem [{\citenamefont {Halmos}(1974)}]{halmos-vs}%
  \BibitemOpen
  \bibfield  {author} {\bibinfo {author} {\bibfnamefont {Paul~R..}\
  \bibnamefont {Halmos}},\ }\href@noop {} {\emph {\bibinfo {title}
  {Finite-dimensional vector spaces}}}\ (\bibinfo  {publisher} {Springer},\
  \bibinfo {address} {New York, Heidelberg, Berlin},\ \bibinfo {year}
  {1974})\BibitemShut {NoStop}%
\bibitem [{\citenamefont {Stace}(1934)}]{stace}%
  \BibitemOpen
  \bibfield  {author} {\bibinfo {author} {\bibfnamefont {Walter~Terence}\
  \bibnamefont {Stace}},\ }\bibfield  {title} {\enquote {\bibinfo {title} {The
  refutation of realism},}\ }\href {http://www.jstor.org/stable/2250077}
  {\bibfield  {journal} {\bibinfo  {journal} {Mind}\ }\textbf {\bibinfo
  {volume} {43}},\ \bibinfo {pages} {145--155} (\bibinfo {year} {1934})},\
  \bibinfo {note} {reprinted in \cite[pp.~364-372]{stace1}}\BibitemShut
  {NoStop}%
\bibitem [{\citenamefont {Bohr}(1949)}]{bohr-1949}%
  \BibitemOpen
  \bibfield  {author} {\bibinfo {author} {\bibfnamefont {Niels}\ \bibnamefont
  {Bohr}},\ }\bibfield  {title} {\enquote {\bibinfo {title} {Discussion with
  {E}instein on epistemological problems in atomic physics},}\ }in\ \href
  {\doibase 10.1016/S1876-0503(08)70379-7} {\emph {\bibinfo {booktitle}
  {{A}lbert {E}instein: Philosopher-Scientist}}},\ \bibinfo {editor} {edited
  by\ \bibinfo {editor} {\bibfnamefont {P.~A.}\ \bibnamefont {Schilpp}}}\
  (\bibinfo  {publisher} {The Library of Living Philosophers},\ \bibinfo
  {address} {Evanston, Ill.},\ \bibinfo {year} {1949})\ pp.\ \bibinfo {pages}
  {200--241}\BibitemShut {NoStop}%
\bibitem [{\citenamefont {Bell}(1966)}]{bell-66}%
  \BibitemOpen
  \bibfield  {author} {\bibinfo {author} {\bibfnamefont {John~S.}\ \bibnamefont
  {Bell}},\ }\bibfield  {title} {\enquote {\bibinfo {title} {On the problem of
  hidden variables in quantum mechanics},}\ }\href {\doibase
  10.1103/RevModPhys.38.447} {\bibfield  {journal} {\bibinfo  {journal}
  {Reviews of Modern Physics}\ }\textbf {\bibinfo {volume} {38}},\ \bibinfo
  {pages} {447--452} (\bibinfo {year} {1966})},\ \bibinfo {note} {reprinted in
  Ref.~\cite[pp. 1-13]{bell-87}}\BibitemShut {NoStop}%
\bibitem [{\citenamefont {Heywood}\ and\ \citenamefont
  {Redhead}(1983)}]{hey-red}%
  \BibitemOpen
  \bibfield  {author} {\bibinfo {author} {\bibfnamefont {Peter}\ \bibnamefont
  {Heywood}}\ and\ \bibinfo {author} {\bibfnamefont {Michael L.~G.}\
  \bibnamefont {Redhead}},\ }\bibfield  {title} {\enquote {\bibinfo {title}
  {Nonlocality and the {K}ochen-{S}pecker paradox},}\ }\href {\doibase
  10.1007/BF00729511} {\bibfield  {journal} {\bibinfo  {journal} {Foundations
  of Physics}\ }\textbf {\bibinfo {volume} {13}},\ \bibinfo {pages} {481--499}
  (\bibinfo {year} {1983})}\BibitemShut {NoStop}%
\bibitem [{\citenamefont {Redhead}(1990)}]{redhead}%
  \BibitemOpen
  \bibfield  {author} {\bibinfo {author} {\bibfnamefont {Michael}\ \bibnamefont
  {Redhead}},\ }\href@noop {} {\emph {\bibinfo {title} {Incompleteness,
  Nonlocality, and Realism: A Prolegomenon to the Philosophy of Quantum
  Mechanics}}}\ (\bibinfo  {publisher} {Clarendon Press},\ \bibinfo {address}
  {Oxford},\ \bibinfo {year} {1990})\BibitemShut {NoStop}%
\bibitem [{Note1()}]{Note1}%
  \BibitemOpen
  \bibinfo {note} {Bell cites Bohr's remark~\cite {bohr-1949} about {\protect
  \em ``the impossibility of any sharp separation between the behavior of
  atomic objects and the interaction with the measuring instruments which serve
  to define the conditions under which the phenomena appear.''}}\BibitemShut
  {NoStop}%
\bibitem [{\citenamefont {Shimony}(1984)}]{shimony2}%
  \BibitemOpen
  \bibfield  {author} {\bibinfo {author} {\bibfnamefont {Abner}\ \bibnamefont
  {Shimony}},\ }\bibfield  {title} {\enquote {\bibinfo {title} {Controllable
  and uncontrollable non-locality},}\ }in\ \href@noop {} {\emph {\bibinfo
  {booktitle} {Proceedings of the International Symposium... Proceedings of the
  International Symposium Foundations of Quantum Mechanics in the Light of New
  Technology}}},\ \bibinfo {editor} {edited by\ \bibinfo {editor}
  {\bibfnamefont {Susumu}\ \bibnamefont {Kamefuchi}}\ and\ \bibinfo {editor}
  {\bibfnamefont {Nihon~Butsuri}\ \bibnamefont {Gakkai}}}\ (\bibinfo
  {publisher} {Physical Society of Japan},\ \bibinfo {address} {Tokyo},\
  \bibinfo {year} {1984})\ pp.\ \bibinfo {pages} {225--230},\ \bibinfo {note}
  {see also J. Jarrett, {\sl Bell's Theorem, Quantum Mechanics and Local
  Realism}, Ph. D. thesis, Univ. of Chicago, 1983; {\sl Nous}, {\bf 18}, 569
  (1984)}\BibitemShut {NoStop}%
\bibitem [{\citenamefont {Born}(1926{\natexlab{a}})}]{born-26-1}%
  \BibitemOpen
  \bibfield  {author} {\bibinfo {author} {\bibfnamefont {Max}\ \bibnamefont
  {Born}},\ }\bibfield  {title} {\enquote {\bibinfo {title} {Zur
  {Q}uantenmechanik der {S}to{\ss}vorg{\"{a}}nge},}\ }\href {\doibase
  10.1007/BF01397477} {\bibfield  {journal} {\bibinfo  {journal} {Zeitschrift
  f{\"{u}}r Physik}\ }\textbf {\bibinfo {volume} {37}},\ \bibinfo {pages}
  {863--867} (\bibinfo {year} {1926}{\natexlab{a}})}\BibitemShut {NoStop}%
\bibitem [{\citenamefont {Born}(1926{\natexlab{b}})}]{born-26-2}%
  \BibitemOpen
  \bibfield  {author} {\bibinfo {author} {\bibfnamefont {Max}\ \bibnamefont
  {Born}},\ }\bibfield  {title} {\enquote {\bibinfo {title} {{Q}uantenmechanik
  der {S}to{\ss}vorg{\"{a}}nge},}\ }\href {\doibase 10.1007/BF01397184}
  {\bibfield  {journal} {\bibinfo  {journal} {Zeitschrift f{\"{u}}r Physik}\
  }\textbf {\bibinfo {volume} {38}},\ \bibinfo {pages} {803--827} (\bibinfo
  {year} {1926}{\natexlab{b}})}\BibitemShut {NoStop}%
\bibitem [{\citenamefont {Zeilinger}(2005)}]{zeil-05_nature_ofQuantum}%
  \BibitemOpen
  \bibfield  {author} {\bibinfo {author} {\bibfnamefont {Anton}\ \bibnamefont
  {Zeilinger}},\ }\bibfield  {title} {\enquote {\bibinfo {title} {The message
  of the quantum},}\ }\href {\doibase 10.1038/438743a} {\bibfield  {journal}
  {\bibinfo  {journal} {Nature}\ }\textbf {\bibinfo {volume} {438}},\ \bibinfo
  {pages} {743} (\bibinfo {year} {2005})}\BibitemShut {NoStop}%
\bibitem [{\citenamefont {Calude}\ \emph {et~al.}(2010)\citenamefont {Calude},
  \citenamefont {Dinneen}, \citenamefont {Dumitrescu},\ and\ \citenamefont
  {Svozil}}]{PhysRevA.82.022102}%
  \BibitemOpen
  \bibfield  {author} {\bibinfo {author} {\bibfnamefont {Cristian~S.}\
  \bibnamefont {Calude}}, \bibinfo {author} {\bibfnamefont {Michael~J.}\
  \bibnamefont {Dinneen}}, \bibinfo {author} {\bibfnamefont {Monica}\
  \bibnamefont {Dumitrescu}}, \ and\ \bibinfo {author} {\bibfnamefont {Karl}\
  \bibnamefont {Svozil}},\ }\bibfield  {title} {\enquote {\bibinfo {title}
  {Experimental evidence of quantum randomness incomputability},}\ }\href
  {\doibase 10.1103/PhysRevA.82.022102} {\bibfield  {journal} {\bibinfo
  {journal} {Phys. Rev. A}\ }\textbf {\bibinfo {volume} {82}},\ \bibinfo
  {pages} {022102} (\bibinfo {year} {2010})}\BibitemShut {NoStop}%
\bibitem [{\citenamefont {{Rogers, Jr.}}(1967)}]{rogers1}%
  \BibitemOpen
  \bibfield  {author} {\bibinfo {author} {\bibfnamefont {Hartley}\ \bibnamefont
  {{Rogers, Jr.}}},\ }\href@noop {} {\emph {\bibinfo {title} {Theory of
  Recursive Functions and Effective Computability}}}\ (\bibinfo  {publisher}
  {MacGraw-Hill},\ \bibinfo {address} {New York},\ \bibinfo {year}
  {1967})\BibitemShut {NoStop}%
\bibitem [{\citenamefont {Davis}(1965)}]{davis}%
  \BibitemOpen
  \bibfield  {author} {\bibinfo {author} {\bibfnamefont {Martin}\ \bibnamefont
  {Davis}},\ }\href@noop {} {\emph {\bibinfo {title} {The Undecidable. Basic
  Papers on Undecidable, Unsolvable Problems and Computable Functions}}}\
  (\bibinfo  {publisher} {Raven Press},\ \bibinfo {address} {Hewlett, N.Y.},\
  \bibinfo {year} {1965})\BibitemShut {NoStop}%
\bibitem [{\citenamefont {Barwise}(1978)}]{Barwise-handbook-logic}%
  \BibitemOpen
  \bibfield  {author} {\bibinfo {author} {\bibfnamefont {Jon}\ \bibnamefont
  {Barwise}},\ }\href@noop {} {\emph {\bibinfo {title} {Handbook of
  Mathematical Logic}}}\ (\bibinfo  {publisher} {North-Holland},\ \bibinfo
  {address} {Amsterdam},\ \bibinfo {year} {1978})\BibitemShut {NoStop}%
\bibitem [{\citenamefont {Enderton}(2001)}]{enderton72}%
  \BibitemOpen
  \bibfield  {author} {\bibinfo {author} {\bibfnamefont {H.}~\bibnamefont
  {Enderton}},\ }\href@noop {} {\emph {\bibinfo {title} {{A Mathematical
  Introduction to Logic}}}},\ \bibinfo {edition} {2nd}\ ed.\ (\bibinfo
  {publisher} {{Academic Press}},\ \bibinfo {address} {San Diego},\ \bibinfo
  {year} {2001})\BibitemShut {NoStop}%
\bibitem [{\citenamefont {Odifreddi}(1989)}]{odi:89}%
  \BibitemOpen
  \bibfield  {author} {\bibinfo {author} {\bibfnamefont {Piergiorgio}\
  \bibnamefont {Odifreddi}},\ }\href@noop {} {\emph {\bibinfo {title}
  {Classical Recursion Theory, Vol. 1}}}\ (\bibinfo  {publisher}
  {North-Holland},\ \bibinfo {address} {Amsterdam},\ \bibinfo {year}
  {1989})\BibitemShut {NoStop}%
\bibitem [{\citenamefont {Boolos}\ \emph {et~al.}(2007)\citenamefont {Boolos},
  \citenamefont {Burgess},\ and\ \citenamefont {Jeffrey}}]{Boolos-07}%
  \BibitemOpen
  \bibfield  {author} {\bibinfo {author} {\bibfnamefont {George~S.}\
  \bibnamefont {Boolos}}, \bibinfo {author} {\bibfnamefont {John~P.}\
  \bibnamefont {Burgess}}, \ and\ \bibinfo {author} {\bibfnamefont
  {Richard~C.}\ \bibnamefont {Jeffrey}},\ }\href@noop {} {\emph {\bibinfo
  {title} {Computability and Logic}}},\ \bibinfo {edition} {5th}\ ed.\
  (\bibinfo  {publisher} {Cambridge University Press},\ \bibinfo {address}
  {Cambridge},\ \bibinfo {year} {2007})\BibitemShut {NoStop}%
\bibitem [{\citenamefont {Calude}(2002)}]{calude:02}%
  \BibitemOpen
  \bibfield  {author} {\bibinfo {author} {\bibfnamefont {Cristian}\
  \bibnamefont {Calude}},\ }\href@noop {} {\emph {\bibinfo {title} {Information
  and Randomness---An Algorithmic Perspective}}},\ \bibinfo {edition} {2nd}\
  ed.\ (\bibinfo  {publisher} {Springer},\ \bibinfo {address} {Berlin},\
  \bibinfo {year} {2002})\BibitemShut {NoStop}%
\bibitem [{\citenamefont {Pitowsky}(1989{\natexlab{a}})}]{pitowsky-89a}%
  \BibitemOpen
  \bibfield  {author} {\bibinfo {author} {\bibfnamefont {Itamar}\ \bibnamefont
  {Pitowsky}},\ }\bibfield  {title} {\enquote {\bibinfo {title} {From {G}eorge
  {B}oole to {J}ohn {B}ell: The origin of {B}ell's inequality},}\ }in\
  \href@noop {} {\emph {\bibinfo {booktitle} {{B}ell's Theorem, Quantum Theory
  and the Conceptions of the Universe}}},\ \bibinfo {editor} {edited by\
  \bibinfo {editor} {\bibfnamefont {M.}~\bibnamefont {Kafatos}}}\ (\bibinfo
  {publisher} {Kluwer},\ \bibinfo {address} {Dordrecht},\ \bibinfo {year}
  {1989})\ pp.\ \bibinfo {pages} {37--49}\BibitemShut {NoStop}%
\bibitem [{\citenamefont {Pitowsky}(1994)}]{Pit-94}%
  \BibitemOpen
  \bibfield  {author} {\bibinfo {author} {\bibfnamefont {Itamar}\ \bibnamefont
  {Pitowsky}},\ }\bibfield  {title} {\enquote {\bibinfo {title} {{G}eorge
  {B}oole's `conditions of possible experience' and the quantum puzzle},}\
  }\href {\doibase 10.1093/bjps/45.1.95} {\bibfield  {journal} {\bibinfo
  {journal} {The British Journal for the Philosophy of Science}\ }\textbf
  {\bibinfo {volume} {45}},\ \bibinfo {pages} {95--125} (\bibinfo {year}
  {1994})}\BibitemShut {NoStop}%
\bibitem [{\citenamefont {Pitowsky}(1986)}]{pitowsky-86}%
  \BibitemOpen
  \bibfield  {author} {\bibinfo {author} {\bibfnamefont {Itamar}\ \bibnamefont
  {Pitowsky}},\ }\bibfield  {title} {\enquote {\bibinfo {title} {The range of
  quantum probabilities},}\ }\href@noop {} {\bibfield  {journal} {\bibinfo
  {journal} {Journal of Mathematical Physics}\ }\textbf {\bibinfo {volume}
  {27}},\ \bibinfo {pages} {1556--1565} (\bibinfo {year} {1986})}\BibitemShut
  {NoStop}%
\bibitem [{\citenamefont {Pitowsky}(1989{\natexlab{b}})}]{pitowsky}%
  \BibitemOpen
  \bibfield  {author} {\bibinfo {author} {\bibfnamefont {Itamar}\ \bibnamefont
  {Pitowsky}},\ }\href@noop {} {\emph {\bibinfo {title} {Quantum
  Probability---Quantum Logic}}}\ (\bibinfo  {publisher} {Springer},\ \bibinfo
  {address} {Berlin},\ \bibinfo {year} {1989})\BibitemShut {NoStop}%
\bibitem [{\citenamefont {Hasegawa}\ \emph {et~al.}(2006)\citenamefont
  {Hasegawa}, \citenamefont {Loidl}, \citenamefont {Badurek}, \citenamefont
  {Baron},\ and\ \citenamefont {Rauch}}]{hasegawa:230401}%
  \BibitemOpen
  \bibfield  {author} {\bibinfo {author} {\bibfnamefont {Yuji}\ \bibnamefont
  {Hasegawa}}, \bibinfo {author} {\bibfnamefont {Rudolf}\ \bibnamefont
  {Loidl}}, \bibinfo {author} {\bibfnamefont {Gerald}\ \bibnamefont {Badurek}},
  \bibinfo {author} {\bibfnamefont {Matthias}\ \bibnamefont {Baron}}, \ and\
  \bibinfo {author} {\bibfnamefont {Helmut}\ \bibnamefont {Rauch}},\ }\bibfield
   {title} {\enquote {\bibinfo {title} {Quantum contextuality in a
  single-neutron optical experiment},}\ }\href {\doibase
  10.1103/PhysRevLett.97.230401} {\bibfield  {journal} {\bibinfo  {journal}
  {Physical Review Letters}\ }\textbf {\bibinfo {volume} {97}},\ \bibinfo {eid}
  {230401} (\bibinfo {year} {2006})}\BibitemShut {NoStop}%
\bibitem [{\citenamefont {Bartosik}\ \emph {et~al.}(2009)\citenamefont
  {Bartosik}, \citenamefont {Klepp}, \citenamefont {Schmitzer}, \citenamefont
  {Sponar}, \citenamefont {Cabello}, \citenamefont {Rauch},\ and\ \citenamefont
  {Hasegawa}}]{Bartosik-09}%
  \BibitemOpen
  \bibfield  {author} {\bibinfo {author} {\bibfnamefont {H.}~\bibnamefont
  {Bartosik}}, \bibinfo {author} {\bibfnamefont {J.}~\bibnamefont {Klepp}},
  \bibinfo {author} {\bibfnamefont {C.}~\bibnamefont {Schmitzer}}, \bibinfo
  {author} {\bibfnamefont {S.}~\bibnamefont {Sponar}}, \bibinfo {author}
  {\bibfnamefont {A.}~\bibnamefont {Cabello}}, \bibinfo {author} {\bibfnamefont
  {H.}~\bibnamefont {Rauch}}, \ and\ \bibinfo {author} {\bibfnamefont
  {Y.}~\bibnamefont {Hasegawa}},\ }\bibfield  {title} {\enquote {\bibinfo
  {title} {Experimental test of quantum contextuality in neutron
  interferometry},}\ }\href {\doibase 10.1103/PhysRevLett.103.040403}
  {\bibfield  {journal} {\bibinfo  {journal} {Physical Review Letters}\
  }\textbf {\bibinfo {volume} {103}},\ \bibinfo {pages} {040403} (\bibinfo
  {year} {2009})},\ \Eprint {http://arxiv.org/abs/arXiv:0904.4576}
  {arXiv:0904.4576} \BibitemShut {NoStop}%
\bibitem [{\citenamefont {Amselem}\ \emph {et~al.}(2009)\citenamefont
  {Amselem}, \citenamefont {R\aa{}dmark}, \citenamefont {Bourennane},\ and\
  \citenamefont {Cabello}}]{PhysRevLett.103.160405}%
  \BibitemOpen
  \bibfield  {author} {\bibinfo {author} {\bibfnamefont {Elias}\ \bibnamefont
  {Amselem}}, \bibinfo {author} {\bibfnamefont {Magnus}\ \bibnamefont
  {R\aa{}dmark}}, \bibinfo {author} {\bibfnamefont {Mohamed}\ \bibnamefont
  {Bourennane}}, \ and\ \bibinfo {author} {\bibfnamefont {Ad\'an}\ \bibnamefont
  {Cabello}},\ }\bibfield  {title} {\enquote {\bibinfo {title}
  {State-independent quantum contextuality with single photons},}\ }\href
  {\doibase 10.1103/PhysRevLett.103.160405} {\bibfield  {journal} {\bibinfo
  {journal} {Physical Review Letters}\ }\textbf {\bibinfo {volume} {103}},\
  \bibinfo {pages} {160405} (\bibinfo {year} {2009})}\BibitemShut {NoStop}%
\bibitem [{\citenamefont {Kirchmair}\ \emph {et~al.}(2009)\citenamefont
  {Kirchmair}, \citenamefont {Z{\"{a}}hringer}, \citenamefont {Gerritsma},
  \citenamefont {Kleinmann}, \citenamefont {G{\"{u}}hne}, \citenamefont
  {Cabello}, \citenamefont {Blatt},\ and\ \citenamefont {Roos}}]{kirch-09}%
  \BibitemOpen
  \bibfield  {author} {\bibinfo {author} {\bibfnamefont {G.}~\bibnamefont
  {Kirchmair}}, \bibinfo {author} {\bibfnamefont {F.}~\bibnamefont
  {Z{\"{a}}hringer}}, \bibinfo {author} {\bibfnamefont {R.}~\bibnamefont
  {Gerritsma}}, \bibinfo {author} {\bibfnamefont {M.}~\bibnamefont
  {Kleinmann}}, \bibinfo {author} {\bibfnamefont {O.}~\bibnamefont
  {G{\"{u}}hne}}, \bibinfo {author} {\bibfnamefont {A.}~\bibnamefont
  {Cabello}}, \bibinfo {author} {\bibfnamefont {R.}~\bibnamefont {Blatt}}, \
  and\ \bibinfo {author} {\bibfnamefont {C.~F.}\ \bibnamefont {Roos}},\
  }\bibfield  {title} {\enquote {\bibinfo {title} {State-independent
  experimental test of quantum contextuality},}\ }\href {\doibase
  10.1038/nature08172} {\bibfield  {journal} {\bibinfo  {journal} {Nature}\
  }\textbf {\bibinfo {volume} {460}},\ \bibinfo {pages} {494--497} (\bibinfo
  {year} {2009})},\ \Eprint {http://arxiv.org/abs/arXiv:0904.1655}
  {arXiv:0904.1655} \BibitemShut {NoStop}%
\bibitem [{\citenamefont {Froissart}(1981)}]{froissart-81}%
  \BibitemOpen
  \bibfield  {author} {\bibinfo {author} {\bibfnamefont {M.}~\bibnamefont
  {Froissart}},\ }\bibfield  {title} {\enquote {\bibinfo {title} {Constructive
  generalization of {B}ell's inequalities},}\ }\href@noop {} {\bibfield
  {journal} {\bibinfo  {journal} {Nuovo Cimento B}\ }\textbf {\bibinfo {volume}
  {64}},\ \bibinfo {pages} {241--251} (\bibinfo {year} {1981})}\BibitemShut
  {NoStop}%
\bibitem [{\citenamefont {{Cirel'son (=Tsirel'son)}}(1980)}]{cirelson:80}%
  \BibitemOpen
  \bibfield  {author} {\bibinfo {author} {\bibfnamefont {Boris~S.}\
  \bibnamefont {{Cirel'son (=Tsirel'son)}}},\ }\bibfield  {title} {\enquote
  {\bibinfo {title} {Quantum generalizations of {B}ell's inequality},}\
  }\href@noop {} {\bibfield  {journal} {\bibinfo  {journal} {Letters in
  Mathematical Physics}\ }\textbf {\bibinfo {volume} {4}},\ \bibinfo {pages}
  {93--100} (\bibinfo {year} {1980})}\BibitemShut {NoStop}%
\bibitem [{\citenamefont {{Cirel'son (=Tsirel'son)}}(1993)}]{cirelson}%
  \BibitemOpen
  \bibfield  {author} {\bibinfo {author} {\bibfnamefont {Boris~S.}\
  \bibnamefont {{Cirel'son (=Tsirel'son)}}},\ }\bibfield  {title} {\enquote
  {\bibinfo {title} {Some results and problems on quantum {B}ell-type
  inequalities},}\ }\href@noop {} {\bibfield  {journal} {\bibinfo  {journal}
  {Hadronic Journal Supplement}\ }\textbf {\bibinfo {volume} {8}},\ \bibinfo
  {pages} {329--345} (\bibinfo {year} {1993})}\BibitemShut {NoStop}%
\bibitem [{\citenamefont {Pitowsky}\ and\ \citenamefont
  {Svozil}(2001)}]{2000-poly}%
  \BibitemOpen
  \bibfield  {author} {\bibinfo {author} {\bibfnamefont {Itamar}\ \bibnamefont
  {Pitowsky}}\ and\ \bibinfo {author} {\bibfnamefont {Karl}\ \bibnamefont
  {Svozil}},\ }\bibfield  {title} {\enquote {\bibinfo {title} {New optimal
  tests of quantum nonlocality},}\ }\href {\doibase 10.1103/PhysRevA.64.014102}
  {\bibfield  {journal} {\bibinfo  {journal} {Physical Review A}\ }\textbf
  {\bibinfo {volume} {64}},\ \bibinfo {pages} {014102} (\bibinfo {year}
  {2001})},\ \Eprint {http://arxiv.org/abs/quant-ph/0011060} {quant-ph/0011060}
  \BibitemShut {NoStop}%
\bibitem [{\citenamefont {Ziegler}(1994)}]{ziegler}%
  \BibitemOpen
  \bibfield  {author} {\bibinfo {author} {\bibfnamefont {G{\"{u}}nter~M.}\
  \bibnamefont {Ziegler}},\ }\href@noop {} {\emph {\bibinfo {title} {Lectures
  on Polytopes}}}\ (\bibinfo  {publisher} {Springer},\ \bibinfo {address} {New
  York},\ \bibinfo {year} {1994})\BibitemShut {NoStop}%
\bibitem [{\citenamefont {Pitowsky}(1990)}]{pit:90}%
  \BibitemOpen
  \bibfield  {author} {\bibinfo {author} {\bibfnamefont {Itamar}\ \bibnamefont
  {Pitowsky}},\ }\bibfield  {title} {\enquote {\bibinfo {title} {The physical
  {C}hurch-{T}uring thesis and physical computational complexity},}\
  }\href@noop {} {\bibfield  {journal} {\bibinfo  {journal} {Iyyun}\ }\textbf
  {\bibinfo {volume} {39}},\ \bibinfo {pages} {81--99} (\bibinfo {year}
  {1990})}\BibitemShut {NoStop}%
\bibitem [{\citenamefont {Krenn}\ and\ \citenamefont
  {Zeilinger}(1996)}]{krenn1}%
  \BibitemOpen
  \bibfield  {author} {\bibinfo {author} {\bibfnamefont {G{\"{u}}nther}\
  \bibnamefont {Krenn}}\ and\ \bibinfo {author} {\bibfnamefont {Anton}\
  \bibnamefont {Zeilinger}},\ }\bibfield  {title} {\enquote {\bibinfo {title}
  {Entangled entanglement},}\ }\href {\doibase 10.1103/PhysRevA.54.1793}
  {\bibfield  {journal} {\bibinfo  {journal} {Physical Review A}\ }\textbf
  {\bibinfo {volume} {54}},\ \bibinfo {pages} {1793--1797} (\bibinfo {year}
  {1996})}\BibitemShut {NoStop}%
\bibitem [{Note2()}]{Note2}%
  \BibitemOpen
  \bibinfo {note} {Note that for stronger-than-quantum correlations~\cite
  {pop-rohr,svozil-krenn} rendering a maximal violation of the
  Clauser-Horne-Shimony-Holt inequality by $ \protect \textsf {\protect \textbf
  {A}}_1 \protect \{ \protect \textsf {\protect \textbf {B}}_1 \protect \}
  \protect \textsf {\protect \textbf {B}}_1 \protect \{ \protect \textsf
  {\protect \textbf {A}}_1 \protect \} + \protect \textsf {\protect \textbf
  {A}}_1 \protect \{ \protect \textsf {\protect \textbf {B}}_2 \protect \}
  \protect \textsf {\protect \textbf {B}}_2 \protect \{ \protect \textsf
  {\protect \textbf {A}}_1 \protect \} + \protect \textsf {\protect \textbf
  {A}}_2 \protect \{ \protect \textsf {\protect \textbf {B}}_1 \protect \}
  \protect \textsf {\protect \textbf {B}}_1 \protect \{ \protect \textsf
  {\protect \textbf {A}}_2 \protect \} - \protect \textsf {\protect \textbf
  {A}}_2 \protect \{ \protect \textsf {\protect \textbf {B}}_2 \protect \}
  \protect \textsf {\protect \textbf {B}}_2 \protect \{ \protect \textsf
  {\protect \textbf {A}}_2 \protect \} = \pm 4 $, if $\protect \textsf
  {\protect \textbf {A}}_1 \protect \{ \protect \textsf {\protect \textbf
  {B}}_2 \protect \} = \protect \textsf {\protect \textbf {A}}_2 \protect \{
  \protect \textsf {\protect \textbf {B}}_2 \protect \}$, then $\protect
  \textsf {\protect \textbf {B}}_2 \protect \{ \protect \textsf {\protect
  \textbf {A}}_1 \protect \} = -\protect \textsf {\protect \textbf {B}}_2
  \protect \{ \protect \textsf {\protect \textbf {A}}_2 \protect \}$, and if
  $\protect \textsf {\protect \textbf {B}}_1 \protect \{ \protect \textsf
  {\protect \textbf {A}}_2 \protect \} = \protect \textsf {\protect \textbf
  {B}}_2 \protect \{ \protect \textsf {\protect \textbf {A}}_2 \protect \}$,
  then $\protect \textsf {\protect \textbf {A}}_2 \protect \{ \protect \textsf
  {\protect \textbf {B}}_1 \protect \} = -\protect \textsf {\protect \textbf
  {A}}_2 \protect \{ \protect \textsf {\protect \textbf {B}}_2 \protect
  \}$.}\BibitemShut {Stop}%
\bibitem [{\citenamefont {Cabello}(2000)}]{cabello-99}%
  \BibitemOpen
  \bibfield  {author} {\bibinfo {author} {\bibfnamefont {Ad{\'{a}}n}\
  \bibnamefont {Cabello}},\ }\bibfield  {title} {\enquote {\bibinfo {title}
  {{K}ochen-{S}pecker theorem and experimental test on hidden variables},}\
  }\href {\doibase 10.1142/S0217751X00002020} {\bibfield  {journal} {\bibinfo
  {journal} {International Journal of Modern Physics}\ }\textbf {\bibinfo
  {volume} {A 15}},\ \bibinfo {pages} {2813--2820} (\bibinfo {year} {2000})},\
  \Eprint {http://arxiv.org/abs/quant-ph/9911022} {quant-ph/9911022}
  \BibitemShut {NoStop}%
\bibitem [{\citenamefont {Kalmbach}(1983)}]{kalmbach-83}%
  \BibitemOpen
  \bibfield  {author} {\bibinfo {author} {\bibfnamefont {Gudrun}\ \bibnamefont
  {Kalmbach}},\ }\href@noop {} {\emph {\bibinfo {title} {Orthomodular
  Lattices}}}\ (\bibinfo  {publisher} {Academic Press},\ \bibinfo {address}
  {New York},\ \bibinfo {year} {1983})\BibitemShut {NoStop}%
\bibitem [{\citenamefont {Pt{\'{a}}k}\ and\ \citenamefont
  {Pulmannov{\'{a}}}(1991)}]{pulmannova-91}%
  \BibitemOpen
  \bibfield  {author} {\bibinfo {author} {\bibfnamefont {Pavel}\ \bibnamefont
  {Pt{\'{a}}k}}\ and\ \bibinfo {author} {\bibfnamefont {Sylvia}\ \bibnamefont
  {Pulmannov{\'{a}}}},\ }\href@noop {} {\emph {\bibinfo {title} {Orthomodular
  Structures as Quantum Logics}}}\ (\bibinfo  {publisher} {Kluwer Academic
  Publishers},\ \bibinfo {address} {Dordrecht},\ \bibinfo {year}
  {1991})\BibitemShut {NoStop}%
\bibitem [{\citenamefont {Svozil}(2009{\natexlab{d}})}]{svozil:040102}%
  \BibitemOpen
  \bibfield  {author} {\bibinfo {author} {\bibfnamefont {Karl}\ \bibnamefont
  {Svozil}},\ }\bibfield  {title} {\enquote {\bibinfo {title} {Proposed direct
  test of a certain type of noncontextuality in quantum mechanics},}\ }\href
  {\doibase 10.1103/PhysRevA.80.040102} {\bibfield  {journal} {\bibinfo
  {journal} {Physical Review A}\ }\textbf {\bibinfo {volume} {80}},\ \bibinfo
  {eid} {040102} (\bibinfo {year} {2009}{\natexlab{d}})}\BibitemShut {NoStop}%
\bibitem [{Note3()}]{Note3}%
  \BibitemOpen
  \bibinfo {note} {We would even go so far to speculate that the ignorance of
  state preparation resulting in mixed states is an epistemic, not ontologic,
  one. Thus all quantum states are ``ontologically'' pure.}\BibitemShut {Stop}%
\bibitem [{\citenamefont {Svozil}(2004)}]{svozil-2003-garda}%
  \BibitemOpen
  \bibfield  {author} {\bibinfo {author} {\bibfnamefont {Karl}\ \bibnamefont
  {Svozil}},\ }\bibfield  {title} {\enquote {\bibinfo {title} {Quantum
  information via state partitions and the context translation principle},}\
  }\href {\doibase 10.1080/09500340410001664179} {\bibfield  {journal}
  {\bibinfo  {journal} {Journal of Modern Optics}\ }\textbf {\bibinfo {volume}
  {51}},\ \bibinfo {pages} {811--819} (\bibinfo {year} {2004})},\ \Eprint
  {http://arxiv.org/abs/quant-ph/0308110} {quant-ph/0308110} \BibitemShut
  {NoStop}%
\bibitem [{\citenamefont {Everett}(1957)}]{everett}%
  \BibitemOpen
  \bibfield  {author} {\bibinfo {author} {\bibfnamefont {Hugh}\ \bibnamefont
  {Everett}},\ }\bibfield  {title} {\enquote {\bibinfo {title} {`relative
  state' formulation of quantum mechanics},}\ }\href {\doibase
  10.1103/RevModPhys.29.454} {\bibfield  {journal} {\bibinfo  {journal}
  {Reviews of Modern Physics}\ }\textbf {\bibinfo {volume} {29}},\ \bibinfo
  {pages} {454--462} (\bibinfo {year} {1957})},\ \bibinfo {note} {reprinted in
  Ref.~\cite[pp. 315-323]{wheeler-Zurek:83}}\BibitemShut {NoStop}%
\bibitem [{\citenamefont {Brukner}\ and\ \citenamefont
  {Zeilinger}(1999)}]{zeil-bruk-99}%
  \BibitemOpen
  \bibfield  {author} {\bibinfo {author} {\bibfnamefont {{\v{C}}aslav}\
  \bibnamefont {Brukner}}\ and\ \bibinfo {author} {\bibfnamefont {Anton}\
  \bibnamefont {Zeilinger}},\ }\bibfield  {title} {\enquote {\bibinfo {title}
  {Malus' law and quantum information},}\ }\href
  {http://www.univie.ac.at/qfp/publications3/pdffiles/1999-08.pdf} {\bibfield
  {journal} {\bibinfo  {journal} {Acta Physica Slovaca}\ }\textbf {\bibinfo
  {volume} {49}},\ \bibinfo {pages} {647--652} (\bibinfo {year}
  {1999})}\BibitemShut {NoStop}%
\bibitem [{\citenamefont {{von Neumann}}(1951)}]{von-neumann1}%
  \BibitemOpen
  \bibfield  {author} {\bibinfo {author} {\bibfnamefont {John}\ \bibnamefont
  {{von Neumann}}},\ }\bibfield  {title} {\enquote {\bibinfo {title} {Various
  techniques used in connection with random digits},}\ }\href@noop {}
  {\bibfield  {journal} {\bibinfo  {journal} {National Bureau of Standards
  Applied Math Series}\ }\textbf {\bibinfo {volume} {12}},\ \bibinfo {pages}
  {36--38} (\bibinfo {year} {1951})},\ \bibinfo {note} {reprinted in {\sl John
  {von Neumann}, Collected Works, (Vol. V)}, A. H. Traub, editor, MacMillan,
  New York, 1963, p. 768--770.}\BibitemShut {Stop}%
\bibitem [{\citenamefont {Zeilinger}(1999)}]{zeil-99}%
  \BibitemOpen
  \bibfield  {author} {\bibinfo {author} {\bibfnamefont {Anton}\ \bibnamefont
  {Zeilinger}},\ }\bibfield  {title} {\enquote {\bibinfo {title} {A
  foundational principle for quantum mechanics},}\ }\href {\doibase
  10.1023/A:1018820410908} {\bibfield  {journal} {\bibinfo  {journal}
  {Foundations of Physics}\ }\textbf {\bibinfo {volume} {29}},\ \bibinfo
  {pages} {631--643} (\bibinfo {year} {1999})}\BibitemShut {NoStop}%
\bibitem [{\citenamefont {{von Neumann}}(1955)}]{v-neumann-55}%
  \BibitemOpen
  \bibfield  {author} {\bibinfo {author} {\bibfnamefont {John}\ \bibnamefont
  {{von Neumann}}},\ }\href@noop {} {\emph {\bibinfo {title} {Mathematical
  Foundations of Quantum Mechanics}}}\ (\bibinfo  {publisher} {Princeton
  University Press},\ \bibinfo {address} {Princeton, NJ},\ \bibinfo {year}
  {1955})\BibitemShut {NoStop}%
\bibitem [{\citenamefont {Specker}(1990)}]{specker-ges}%
  \BibitemOpen
  \bibfield  {author} {\bibinfo {author} {\bibfnamefont {Ernst}\ \bibnamefont
  {Specker}},\ }\href@noop {} {\emph {\bibinfo {title} {Selecta}}}\ (\bibinfo
  {publisher} {Birkh{\"{a}}user Verlag},\ \bibinfo {address} {Basel},\ \bibinfo
  {year} {1990})\BibitemShut {NoStop}%
\bibitem [{\citenamefont {Hooker}(1975)}]{hooker}%
  \BibitemOpen
  \bibfield  {author} {\bibinfo {author} {\bibfnamefont {Clifford~Alan}\
  \bibnamefont {Hooker}},\ }\href@noop {} {\emph {\bibinfo {title} {The
  Logico-Algebraic Approach to Quantum Mechanics. {V}olume {I}: Historical
  Evolution}}}\ (\bibinfo  {publisher} {Reidel},\ \bibinfo {address}
  {Dordrecht},\ \bibinfo {year} {1975})\BibitemShut {NoStop}%
\bibitem [{\citenamefont {Peres}(1993)}]{peres}%
  \BibitemOpen
  \bibfield  {author} {\bibinfo {author} {\bibfnamefont {Asher}\ \bibnamefont
  {Peres}},\ }\href@noop {} {\emph {\bibinfo {title} {Quantum Theory: Concepts
  and Methods}}}\ (\bibinfo  {publisher} {Kluwer Academic Publishers},\
  \bibinfo {address} {Dordrecht},\ \bibinfo {year} {1993})\BibitemShut
  {NoStop}%
\bibitem [{\citenamefont {Trimmer}(1980)}]{trimmer}%
  \BibitemOpen
  \bibfield  {author} {\bibinfo {author} {\bibfnamefont {J.~D.}\ \bibnamefont
  {Trimmer}},\ }\bibfield  {title} {\enquote {\bibinfo {title} {The present
  situation in quantum mechanics: a translation of {S}chr{\"{o}}dinger's ``cat
  paradox''},}\ }\href {http://www.tu-harburg.de/rzt/rzt/it/QM/cat.html}
  {\bibfield  {journal} {\bibinfo  {journal} {Proceedings of the American
  Philosophical Society}\ }\textbf {\bibinfo {volume} {124}},\ \bibinfo {pages}
  {323--338} (\bibinfo {year} {1980})},\ \bibinfo {note} {reprinted in
  Ref.~\cite[pp. 152-167]{wheeler-Zurek:83}}\BibitemShut {NoStop}%
\bibitem [{\citenamefont {Wheeler}\ and\ \citenamefont
  {Zurek}(1983)}]{wheeler-Zurek:83}%
  \BibitemOpen
  \bibfield  {author} {\bibinfo {author} {\bibfnamefont {John~Archibald}\
  \bibnamefont {Wheeler}}\ and\ \bibinfo {author} {\bibfnamefont
  {Wojciech~Hubert}\ \bibnamefont {Zurek}},\ }\href@noop {} {\emph {\bibinfo
  {title} {Quantum Theory and Measurement}}}\ (\bibinfo  {publisher} {Princeton
  University Press},\ \bibinfo {address} {Princeton, NJ},\ \bibinfo {year}
  {1983})\BibitemShut {NoStop}%
\bibitem [{\citenamefont {Stace}(1949)}]{stace1}%
  \BibitemOpen
  \bibfield  {author} {\bibinfo {author} {\bibfnamefont {Walter~Terence}\
  \bibnamefont {Stace}},\ }\bibfield  {title} {\enquote {\bibinfo {title} {The
  refutation of realism},}\ }in\ \href@noop {} {\emph {\bibinfo {booktitle}
  {Readings in Philosophical Analysis}}},\ \bibinfo {editor} {edited by\
  \bibinfo {editor} {\bibfnamefont {Herbert}\ \bibnamefont {Feigl}}\ and\
  \bibinfo {editor} {\bibfnamefont {Wilfrid}\ \bibnamefont {Sellars}}}\
  (\bibinfo  {publisher} {Appleton-Century-Crofts},\ \bibinfo {address} {New
  York},\ \bibinfo {year} {1949})\ pp.\ \bibinfo {pages} {364--372},\ \bibinfo
  {note} {previously published in {\em Mind} {\bf 53}, 349-353
  (1934)}\BibitemShut {NoStop}%
\bibitem [{\citenamefont {Bell}(1987)}]{bell-87}%
  \BibitemOpen
  \bibfield  {author} {\bibinfo {author} {\bibfnamefont {John~S.}\ \bibnamefont
  {Bell}},\ }\href@noop {} {\emph {\bibinfo {title} {Speakable and Unspeakable
  in Quantum Mechanics}}}\ (\bibinfo  {publisher} {Cambridge University
  Press},\ \bibinfo {address} {Cambridge},\ \bibinfo {year} {1987})\BibitemShut
  {NoStop}%
\bibitem [{\citenamefont {Popescu}\ and\ \citenamefont
  {Rohrlich}(1994)}]{pop-rohr}%
  \BibitemOpen
  \bibfield  {author} {\bibinfo {author} {\bibfnamefont {S.}~\bibnamefont
  {Popescu}}\ and\ \bibinfo {author} {\bibfnamefont {D.}~\bibnamefont
  {Rohrlich}},\ }\bibfield  {title} {\enquote {\bibinfo {title} {Quantum
  nonlocality as an axiom},}\ }\href {\doibase 10.1007/BF02058098} {\bibfield
  {journal} {\bibinfo  {journal} {Foundations of Physics}\ }\textbf {\bibinfo
  {volume} {24}},\ \bibinfo {pages} {379--358} (\bibinfo {year}
  {1994})}\BibitemShut {NoStop}%
\bibitem [{\citenamefont {Krenn}\ and\ \citenamefont
  {Svozil}(1998)}]{svozil-krenn}%
  \BibitemOpen
  \bibfield  {author} {\bibinfo {author} {\bibfnamefont {G{\"{u}}nther}\
  \bibnamefont {Krenn}}\ and\ \bibinfo {author} {\bibfnamefont {Karl}\
  \bibnamefont {Svozil}},\ }\bibfield  {title} {\enquote {\bibinfo {title}
  {Stronger-than-quantum correlations},}\ }\href {\doibase
  10.1023/A:1018821314465} {\bibfield  {journal} {\bibinfo  {journal}
  {Foundations of Physics}\ }\textbf {\bibinfo {volume} {28}},\ \bibinfo
  {pages} {971--984} (\bibinfo {year} {1998})}\BibitemShut {NoStop}%
\end{thebibliography}%

\end{document}



*) According to the first Reviewer's suggestion, I have added a sentence expressing

"Alas, while certainly most (but not all~\cite{wjswz-98})  experimental violations of Bell
inequalities do not prove ``quantum nonlocality,''
these statistical violations are no direct proof of contextuality in general."


*) According to the first Reviewer's suggestion, I have deleted the following sentence (I agree that this might cause misinterpretations and is not directly related to the argument):

"The situation somewhat resembles the conventionalization
of the constancy of the speed of light by the SI system of units~\cite{peres-84}."


*) According to the first Reviewer's suggestion, at the end of the section on the Context Translation Principle, I have added a sentence expressing

"In this sense, as only observables associated with one context have a definite value and all other observables have none,
one is  lead to a quasi-classical ``effective value indefiniteness,''
giving rise to a natural classical theory not requiring value definiteness."


*) According to the first Reviewer's suggestion, I have deleted the following sentence from the abstract (I agree that this is not dealt with in detail in the main text):

"We discuss possible interpretations and consequences for quantum oracles."

*) According to the first Reviewer's suggestion, I have deleted the quotations in the sentence ("the experimenter takes into account"), since they were misleading in giving the impression that I had quoted somebody whereas I just wanted to express this fact. Instead, I have emphasized "two" & "entire."

*) According to the first Reviewer's suggestion, I have changed the wording of a sentence in the conclusion to

"anyone considering their physical existence is, to paraphrase von Neumann's words~\cite{von-neumann1}, empirically {\em ``in a state of  sin.''} "

*) According to the first Reviewer's suggestion, I have corrected the typo in "Tabel."

*) According to the first Reviewer's suggestion, I have changed the wording on (the previous) p.7 to:

"One may argue that quantum contextuality only ``appears'' if measurement configurations are encountered which do not allow a
set of two-valued states. The same might be said  for measurement configurations allowing only a ``meager'' set of two-valued states which
cannot be used for the construction of any faithful (i.e. preserving relations and operations among quantum propositions) embedding."

This, I hope, clarifies what I wanted to say.

*) According to the second Reviewer's suggestion (which corresponds also to a criticism of the first Reviewer; cf. above), I have deleted the following sentence from the abstract (I agree that this is not dealt with in detail in the main text):

"We discuss possible interpretations and consequences for quantum oracles."

Furthermore, I have added two sentences about some of these consequences in the conclusion, as they have some bearing on the aims and scope of Natural Computing," as pointed out by the second Reviewer.

*) According to the second Reviewer's suggestion, I have added a sentence explaining the difference between "value indefiniteness" and "indeterminacy" in the introduction:

"Quantum ``value (in)definiteness,''  sometimes also termed ``counterfactual (in)definiteness''~\cite{MuBae-90},
refers to the (im)possibility of the simultaneous existence of definite outcomes of conceivable measurements
under certain assumptions (e.g. noncontextuality; cf. below) --- that is, unperformed measurements can(not)
have definite results~\cite{peres222}.
``(In)determinacy'' often (but not always) refers to the absence (presence) of causal laws
 ---
in the sense of the principle of sufficient reason stating that every phenomenon has its explanation and cause
---
governing a physical behavior.
This, in a sense, ``value (in)definiteness'' relates to a static property, whereas ``indeterminacy'' is often used for temporal evolutions.
Sometimes, quantum  value indefiniteness is considered as one of the expressions of quantum indeterminacy;
another expression of quantum value indefiniteness for instance being associated with the (radioactive) decay of some
excited states~\cite{Kragh-1997AHESradioact,Kragh-2009_RePoss5}."

*) According to the second Reviewer's suggestion, I have corrected the misspelling of "homomorphic."

*) According to the second Reviewer's suggestion (see also the advice of the first Reviewer concerning the last sentence of this paragraph) concerning the last paragraph just before A on page 4, I have eliminated its last sentence (cf. above), and included the rest of the text in the previous paragraph.

*) Due to the second Reviewer's observation that only a single table had been given, I have added one more example in the form of a table representing a Kochen-Specker configuration.

*) According to the second Reviewer's suggestion on p8, under C, "capable to indirectly testing" has been changed to "capable of indirectly testing."

*) According to the second Reviewer's observation on p8, under III, one reference has been delayed to a later paragraph, thereby effectively leaving a single reference instead of two "nearby" references.

*) According to the second Reviewer's observation on p8, the wording has been changed to "albeit not necessarily an irreversible ..."

*) According to the second Reviewer's observation on p9, under IV, the wording has been changed to "with regard to ..."

*) According to the second Reviewer's suggestion, the citation style has been adapted to Natural Computing. Also, footnotes have been separated from the references.

*) According to the second Reviewer's suggestion, the very relevant reference to <W. Muynck and W. Baere, "Quantum nonlocality without counterfactual definiteness?," Foundations of Physics Letters 3, no. 4 (1990): 325-342.> has been included.

I am very grateful for the very valuable suggestions of the Reviewers and have added an acknowledgement to their contributions at the end of the manuscript.

I have also revised several text passages to improve the overall comprehensibility.

Please do not hesitate to contact me if you have any further requests, comments, criticism and suggestions.

With many thanks for all your efforts, and with kind regards,

Karl Svozil


~~~~~~~~~~~~~~~~~~~~~~~~~~~~~~~~~~~~~~~~~

Dear Editors,

Again, the suggestions and comments of the two Referees have been extremely helpful.

In what follows, I have listed the revisions in more detail:

*) According to the first Reviewer's suggestion, on pages 2 and 10, I have changed both "cf's" in the text to "see."

*) According to the first Reviewer's suggestion, on several pages the citation style has been adjusted to fit with the journal guidelines.

*) According to the first Reviewer's suggestion, on page 4, in the last paragraph before section 2.1, I have changed "provable improvable" to "provably unprovable."

*) According to the second Reviewer's suggestion, on page 3, I have added the missing opening quotation marks.

*) According to the second Reviewer's suggestion, on page 5, the space between the superscript for footnote 2 and the word "probabilities" has been omitted; the same on page 9.

*) According to the second Reviewer's suggestion, the names of all journals are given in full in the bibliography now, including the former Aplik. mate. (Alda) and J. Math. Phys. (Pitowksi).

*) According to the second Reviewer's suggestion, in the reference to the 1998 paper by Pitowsky, the "G" in "Gleason" has been capitalized.

*) According to the second Reviewer's suggestion, in the reference to the 1994 paper by Pitowsky, the "od" has been changed to "of."

*) According to the second Reviewer's suggestion, in the reference to the 2002 book by Calude "2 edition" has been changed to "second edition."

Please do not hesitate to contact me if you have any further requests, comments, criticism and suggestions.

With many thanks for all your efforts, and with kind regards,

Karl Svozil
