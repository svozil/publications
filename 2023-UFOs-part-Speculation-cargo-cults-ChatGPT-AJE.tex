%%%%%%%%%%%%%%%%%%%%% chapter.tex %%%%%%%%%%%%%%%%%%%%%%%%%%%%%%%%%
%
% sample chapter
%
% Use this file as a template for your own input.
%
%%%%%%%%%%%%%%%%%%%%%%%% Springer-Verlag %%%%%%%%%%%%%%%%%%%%%%%%%%
%\motto{Use the template \emph{chapter.tex} to style the various elements of your chapter content.}
\chapter{Are UFOs cargo?}
\label{2023-UFO-part-Speculation-cargo-cults} % Always give a unique label
% use \chaptermark{}
% to alter or adjust the chapter heading in the running head


\abstract*{Cargo cults developed in traditional societies in the Pacific region in response to the introduction of Western technology and material goods. These cults involve the belief that spirits or deities will bring valuable cargo, such as manufactured goods, food, and resources, to believers. The cargo is often incomprehensible to these societies and is considered ``magic.'' By analogy, it is possible to assume that our society is in a similar situation to these cargo cults relative to alien technology and capacities. This situation is characterized by conceptual, theoretical, and technological overreach, and the affected culture cannot properly comprehend either the context of their situation or the phenomena encountered. As a result, these phenomena may be absorbed into cults and religions.}

\abstract{Cargo cults developed in traditional societies in the Pacific region in response to the introduction of Western technology and material goods. These cults involve the belief that spirits or deities will bring valuable cargo, such as manufactured goods, food, and resources, to believers. The cargo is often incomprehensible to these societies and is considered ``magic.'' By analogy, it is possible to assume that our society is in a similar situation to these cargo cults relative to alien technology and capacities. This situation is characterized by conceptual, theoretical, and technological overreach, and the affected culture cannot properly comprehend either the context of their situation or the phenomena encountered. As a result, these phenomena may be absorbed into cults and religions.}


One ontic way of interpreting UFOs is in terms of the divine. This may have happened with Nossa Senhora de F\'atima (Our Lady of F\'atima)
in 1917 or with Chris Bledsoe's apparition of The Lady.

A secular way of connecting the UFO phenomenon to religion and God(s) is an epistemic one in terms of cargo cult.
Thus, such experiences are transformed into esoteric images.
There is a long line of authors arguing along these sober, epistemic lines, including
Charles Fort~\cite{FortBotD},\index{Fort, Charles}
Brinsley Le Poer Trench~\cite{lePoerTrench1961Jan},\index{Le Poer Trench, Brinsley}
Carl Sagan~\cite[Section~5, pp.~495,496]{Sagan_1963},\index{Sagan, Carl}
William Bramley~\cite{Bramley1993Mar},\index{Bramley, William}
Erich von D\"anicken~\cite{vonDaeniken2019Jun,vonDaenikenCC},\index{D\"anicken, Erich von}
Tom De{L}onge\index{DeLonge, Tom} and Peter  Levenda~\cite{DeLongeLevenda-Gods,DeLongeLevenda-Men}, \index{Levenda, Peter}
and, to some extent, although they seem to remain undecided,
Jeffrey J. Kripal~\cite{Kripal2011Nov,Kripal2021Jun} and \index{Kripal, Jeffrey J.}
Diana Walsh Pasulka~\cite{Pasulka2019Feb}. \index{Pasulka, Diana Walsh}




\section{Cargo cults}
\label{2023-UFO-part-Speculation-cargo-cults-gen} % Always give a unique label

Erich von D\"aniken has described~\cite{vonDaeniken2019Jun,vonDaenikenCC} and, like others~\cite{MondoCane1962},
visualized~\cite{vonDaeniken1970}
the reaction of the local tribes when the US established a base camp in Hollandia, New Guinea, in 1945.
The area was bustling with nonstop plane landings and departures, bringing supplies for the Pacific war theatre.
The local Papuan bush dwellers observed these activities, which must have appeared ``strange'' to them.
American soldiers distributed small gifts, such as chocolate, gum, old shoes, and empty bottles, which the natives referred to as ``cargo.''
The natives eventually ventured to the edge of the runway to watch the large silver planes take off and wished they could fly directly to their tribe.
This led to the creation of a ``ghost airport'' on the island of Wewak with imitated runways and wooden and straw airplanes.
The behavior of cult members often included imitating the routines and fashion of American soldiers.
This involved conducting parade drills with wooden or salvaged rifles and wearing wooden headphones while sitting in
simulated ``control towers,'' mock ``radio stations,'' and ``isolators''
made of rolled-up leaves, and wooden and iron wristwatches.
Imitated steel helmets made from turtle shells also appeared.
The natives would signal landings by waving flags on the runway and lighting signal fires and torches.
to illuminate runways and lighthouses.
Additionally, they constructed life-sized airplane replicas made of straw and cleared military-style landing strips
in the jungle in the hopes of attracting more aircraft.

Cargo cults~\cite{Worsley2009may,Lindstrom_2019} are a type of religious movement that has emerged in some traditional societies in the Pacific region.
These cults often develop in response to the introduction of Western technology and material goods, and they typically involve the belief that spirits or deities will bring valuable cargo
(such as manufactured goods, food, and other resources) to believers.

Relative to the technological status of those societies, this kind of cargo is incomprehensible and ``magic''~\cite{Clarke2000Jan}.
Sometimes the intended original purpose might be comprehensible, but even in those cases, individuals or groups in those societies
cannot comprehend the physical, chemical, and biological principles on which this cargo is based, nor can they, by mere imitation, reconstruct or reproduce such devices.

Imagine, for example, a truck or jeep delivered to a Melanesian tribe.
Even if some members of these societies were trained to drive these vehicles, they would not understand why they move,
and they would be unable to build one---they would, for instance, lack the metallurgy to build a motor.



\section{Contemporary cargo cults reflecting alien technologies}
\label{2023-UFO-part-Speculation-cargo-cults-ldl} % Always give a unique label

It is not entirely implausible to assume that, by analogy, our societies are in a situation similar to that of cargo cults concerning alien technology and capabilities. This situation is characterized by a conceptual, theoretical, and technological overreach \index{overreach} described in Section~\ref{2023-UFO-part-Perception-flight-characteristics-to}. The affected culture suffers from an explanation trap\index{explanation trap}, as it is unable to understand the phenomenon as it presents itself.

Charles Fort in ``The Book of the Damned''~\cite{FortBotD}, Brinsley Le Poer Trench in ``The Sky People''~\cite{lePoerTrench1961Jan},
and William Bramley in ``The Gods of Eden''~\cite{Bramley1993Mar} discuss ancient civilizations absorbing
extraterrestrial visitations in terms of religions and cults. Erich von D\"aniken quotes the two Soviet scientists,
Alexander Petrowitsch Kasanzew\index{Kasanzew, Alexander Petrowitsch} and Vyacheslav Saizew,
for the conjecture that some (if not a large portion) of our myths and religions are the result of cargo cults based on the
observation of extraterrestrial alien visitations~\cite{vonDaeniken1970}.

Recently, Peter Levenda and Tom DeLonge suggested very similar hypotheses in a series of books devoted to cargo cults and ``the phenomenon''~\cite{DeLongeLevenda-Gods,DeLongeLevenda-Men}. Allegedly, they have been guided by approximately ten~\cite{Iandoli2023Feb} ``advisers'' who are high-ranking military and industrial leaders ``in the know.''

Levenda and DeLonge postulate that some aspects of many religions that developed over the ages were reflections of what people saw ``in the heavens'' and how they coped with these phenomena. This process continues. Both religions and sciences are part of a larger, ongoing cargo cult in which society is dedicated to achieving two goals: travel to the stars and immortality. Levenda and DeLonge also point out that, in July 1952, in the press conference after the Washington DC flyovers, the US Air Force attempted to frame those events in religious terms~\cite[Footnote~100]{DeLongeLevenda-Men}.
