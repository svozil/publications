\newif\ifws
%\wstrue
\ifws

\documentclass{article}

\usepackage{graphicx}        % standard LaTeX graphics tool
\usepackage[dvipsnames]{xcolor}

\usepackage{hyperref}
\hypersetup{
    colorlinks,
    linkcolor={blue!80!black},
    citecolor={red!75!black},
    urlcolor={blue!80!black}
}

% Damit die Verwendung der deutschen Sprache nicht ganz so umst\"andlich wird,
% sollte man die folgenden Pakete einbinden:


%German
%\usepackage[latin1]{inputenc}% erm\"oglich die direkte Eingabe der Umlaute
%\usepackage[T1]{fontenc} % das Trennen der Umlaute
%\usepackage{ngerman} % hiermit werden deutsche Bezeichnungen genutzt und
                     % die W\"orter werden anhand der neue Rechtschreibung
                     % automatisch getrennt.
\title{The wondrous Penrose-Escher-M\"obius graph}
\author{Mirko Navara \\
        Faculty of Electrical Engineering
Czech Technical University in Prague,
Technick\'a 2,
CZ-166~27 Prague 6,
Czech Republic
        }
\author{Karl Svozil \\
        Institute for Theoretical Physics,
TU Wien,  \\
Wiedner Hauptstrasse 8-10/136,
1040 Vienna,  Austria
        }

\date{\today}
% Hinweis: \title{um was auch immer es geht}, \author{wer es auch immer
% geschrieben hat} und  \date{wann auch immer das war} k\"onnen vor
% oder nach dem  Kommando \begin{document} stehen
% Aber der \maketitle Befehl mu\ss{} nach dem \begin{document} Kommando stehen!
\begin{document}

\maketitle


\begin{abstract}
nn
\end{abstract}


\else
\PassOptionsToPackage{dvipsnames}{xcolor}
\documentclass[
 %reprint,
 % twocolumn,
 %superscriptaddress,
 %groupedaddress,
 %unsortedaddress,
 %runinaddress,
 %frontmatterverbose,
  preprint,
 showpacs,
 showkeys,
 preprintnumbers,
 %nofootinbib,
 %nobibnotes,
 %bibnotes,
 amsmath,amssymb,
 aps,
 % prl,
  pra,
 % prb,
 % rmp,
 %prstab,
 %prstper,
  longbibliography,
 floatfix,
 %lengthcheck,
 ]{revtex4-2}

%\usepackage{cdmtcs-pdf}

\usepackage{mathptmx}% http://ctan.org/pkg/mathptmx

\usepackage{amssymb,amsthm,amsmath}

\usepackage{tikz}
\usetikzlibrary{calc,math}
\usepackage {pgfplots}
\pgfplotsset {compat=1.8}
\usepackage{graphicx}% Include figure files
%\usepackage{url}

\usepackage{xcolor}

\usepackage{hyperref}
\hypersetup{
    colorlinks,
    linkcolor={blue},
    citecolor={red!75!black},
    urlcolor={blue}
}


\begin{document}


\title{Wondrous M\"obius-Penrose-Escher type periodic diagrams and their quantum doubles}

\author{Mirko Navara}
\email{navara@fel.cvut.cz}
\homepage{https://cmp.felk.cvut.cz/~navara}

\affiliation{Faculty of Electrical Engineering,
Czech Technical University in Prague,
Technick\'a 2,
CZ-166~27 Prague 6,
Czech Republic}

\author{Karl Svozil}
\email{svozil@tuwien.ac.at}
\homepage{http://tph.tuwien.ac.at/~svozil}

\affiliation{Institute for Theoretical Physics,
TU Wien,
Wiedner Hauptstrasse 8-10/136,
1040 Vienna,  Austria}



\date{\today}

\begin{abstract}
\end{abstract}

%\pacs{03.65.Aa, 03.65.Ta, 03.65.Ud, 03.67.-a}
\keywords{contextuality, two-valued states, quantum states}
%\preprint{CDMTCS preprint nr. x}

\maketitle

\fi



\section{Introduction}
In the following we shall discuss a quantum analogue, in terms of orthogonality hypergraphs, of
what Lionel Sharples Penrose and Roger Penrose,
based on paradoxical drawings of Maurits Cornelis Escher~\cite{Escher1954},
termed `impossible object'~\cite{PENROSE_1958}.
Historically, the related M\"obius strip has found many artistic representations in antiquity~\cite{cartwright-2016}.




\section{Faithful orthogonal coordinatatization of the M\"obius-Escher hypergraph}

What will henceforth called the M\"obius-Escher hypergraph was introduced
in a previous publication~\cite[Fig.~3, Equ.~(5)]{2023-navara-svozil}

\begin{equation}
\alpha
=
2 \cot ^{-1}\left(\sqrt{\frac{11}{9}+\frac{1}{81}
   \sqrt[3]{2262816-69984 \sqrt{69}}+\frac{2}{9} 2^{2/3}
   \sqrt[3]{97+3 \sqrt{69}}}\right)
\end{equation}



\begin{figure}
\begin{center}
\resizebox{.46\textwidth}{!}{\begin{tikzpicture}  [scale=1] %,label distance=1pt
\tikzstyle{every path}=[line width=2pt]

\tikzmath{\a = 1; \b = 2; \c = 0;}


\newdimen\ms
\ms=0.05cm


\tikzstyle{c3}=[circle,inner sep={\ms/8},minimum size=6*\ms]
\tikzstyle{c2}=[circle,inner sep={\ms/8},minimum size=4*\ms]
\tikzstyle{c1}=[circle,inner sep={\ms/8},minimum size=0.8*\ms]

\coordinate (1) at ({2*( \a +  \a + \b ) + 2*\c},{( \a +  \a + \b ) + 2*\c});
\coordinate (3) at ({( \a +  \a + \b )},0);
\coordinate (2) at ($(1)!0.5!(3)$);
\coordinate (5) at ({0-2*\c},{( \a +  \a + \b )+2*\c});
\coordinate (4) at ($(3)!0.5!(5)$);
\coordinate (13) at ({2*( \a +  \a + \b )-\a+\c},{( \a +  \a + \b )+\c});
\coordinate (15) at ({( \a +  \a + \b )},\a);
\coordinate (14) at ($(13)!0.5!(15)$);
\coordinate (17) at ({\a-\c},{( \a +  \a + \b )+\c});
\coordinate (16) at ($(15)!0.5!(17)$);
\coordinate (7) at ({2*( \a +  \a + \b )-2*\a},{( \a +  \a + \b )});
\coordinate (9) at ({( \a +  \a + \b )},{2*\a});
\coordinate (8) at ($(7)!0.5!(9)$);
\coordinate (11) at ({2*\a},{( \a +  \a + \b )});
\coordinate (10) at ($(9)!0.5!(11)$);

%\coordinate (6) at ($(5)!0.915!(7)$);
 \coordinate (12) at ($(11)!0.5!(13)$);
 \coordinate (18) at ($(17)!0.5!(1)$);

\draw [red] (1) -- (3);
\draw [blue] (3) -- (5);

\draw [green] (13) -- (15);
\draw [orange] (15) -- (17);

\draw [magenta] (7) -- (9);
\draw [cyan] (9) -- (11);


\draw [olive] (4) -- (10);
\draw [teal] (8) -- (2);


\draw [gray] (5) .. controls (\a,{1.6*( \a +  \a + \b )}) and ({ 2*( \a +  \a + \b ) - 2} ,{1.6*( \a +  \a + \b )}) .. (7)
                coordinate [pos=0.3] (6);
%              node[pos=0.3, c3,fill=gray, label={above left: $6$}] {};

\draw [brown] (17) .. controls ({ ( \a +  \a + \b ) - 2},{1.5*( \a +  \a + \b )}) and ({  ( \a +  \a + \b ) + 2} ,{1.5*( \a +  \a + \b )}) .. (1)
                coordinate [pos=0.8] (18);
%              node[pos=0.3, c3,fill=gray, label={above left: $6$}] {};

\draw [violet] (11) .. controls ({ ( \a +  \a + \b ) - 2},{1.3*( \a +  \a + \b )}) and ({ ( \a +  \a + \b ) + 2} ,{1.3*( \a +  \a + \b )}) .. (13)
                coordinate [pos=0.2] (12);
%              node[pos=0.3, c3,fill=gray, label={above left: $6$}] {};


\draw (1) coordinate[c3,fill=red,label={below right: $1$}];
\draw (2) coordinate[c3,fill=red,label={below right: $2$}];
\draw (3) coordinate[c3,fill=red,label={below: $3$}];
\draw (4) coordinate[c3,fill=blue,label={below: $4$}];
\draw (5) coordinate[c3,fill=blue,label={below: $5$}];
\draw (6) coordinate[c3,fill=gray,label={above left: $6$}];
\draw (7) coordinate[c3,fill=magenta,label={below: $7$}];
\draw (8) coordinate[c3,fill=magenta,label={below: $8$}];
\draw (9) coordinate[c3,fill=cyan,label={below: $9$}];
\draw (10) coordinate[c3,fill=cyan,label={below: $10$}];
\draw (11) coordinate[c3,fill=cyan,label={below: $11$}];
\draw (12) coordinate[c3,fill=violet,label={below: $12$}];
\draw (13) coordinate[c3,fill=green,label={below: $13$}];
\draw (14) coordinate[c3,fill=green,label={below: $14$}];
\draw (15) coordinate[c3,fill=orange,label={below: $15$}];
\draw (16) coordinate[c3,fill=orange,label={below: $16$}];
\draw (17) coordinate[c3,fill=orange,label={below: $17$}];
\draw (18) coordinate[c3,fill=brown,label={above: $18$}];


\draw (2) coordinate[c2,fill=teal];
\draw (8) coordinate[c2,fill=teal];
\draw (14) coordinate[c2,fill=teal];

\draw (4) coordinate[c2,fill=olive];
\draw (16) coordinate[c2,fill=olive];
\draw (10) coordinate[c2,fill=olive];

\draw (9) coordinate[c2,fill=magenta];
\draw (15) coordinate[c2,fill=green];
\draw (3) coordinate[c2,fill=blue];

\draw (5) coordinate[c2,fill=gray];
\draw (17) coordinate[c2,fill=brown];
\draw (11) coordinate[c2,fill=violet];

\draw (7) coordinate[c2,fill=gray];
\draw (1) coordinate[c2,fill=brown];
\draw (13) coordinate[c2,fill=violet];


\end{tikzpicture}}
\end{center}
\caption{\label{2023-e-f1}
M\"obius-Penrose-Escher type periodic diagram.
}
\end{figure}
\begin{figure}
\begin{center}
\resizebox{.96\textwidth}{!}{\begin{tikzpicture}  [scale=1] %,label distance=1pt
\tikzstyle{every path}=[line width=2pt]

\tikzmath{\a = 1; \b = 2; \c = 0;}


\newdimen\ms
\ms=0.05cm


\tikzstyle{c3}=[circle,inner sep={\ms/8},minimum size=6*\ms]
\tikzstyle{c2}=[circle,inner sep={\ms/8},minimum size=4*\ms]
\tikzstyle{c1}=[circle,inner sep={\ms/8},minimum size=0.8*\ms]

\coordinate (1) at ({2*( \a +  \a + \b ) + 2*\c},{( \a +  \a + \b ) + 2*\c});
\coordinate (3) at ({( \a +  \a + \b )},0);
\coordinate (2) at ($(1)!0.5!(3)$);
\coordinate (5) at ({0-2*\c},{( \a +  \a + \b )+2*\c});
\coordinate (4) at ($(3)!0.5!(5)$);
\coordinate (13) at ({2*( \a +  \a + \b )-\a+\c},{( \a +  \a + \b )+\c});
\coordinate (15) at ({( \a +  \a + \b )},\a);
\coordinate (14) at ($(13)!0.5!(15)$);
\coordinate (17) at ({\a-\c},{( \a +  \a + \b )+\c});
\coordinate (16) at ($(15)!0.5!(17)$);
\coordinate (7) at ({2*( \a +  \a + \b )-2*\a},{( \a +  \a + \b )});
\coordinate (9) at ({( \a +  \a + \b )},{2*\a});
\coordinate (8) at ($(7)!0.5!(9)$);
\coordinate (11) at ({2*\a},{( \a +  \a + \b )});
\coordinate (10) at ($(9)!0.5!(11)$);

%\coordinate (6) at ($(5)!0.915!(7)$);
 \coordinate (12) at ($(11)!0.5!(13)$);
 \coordinate (18) at ($(17)!0.5!(1)$);

\draw [red] (1) -- (3);
\draw [blue] (3) -- (5);

\draw [green] (13) -- (15);
\draw [orange] (15) -- (17);

\draw [magenta] (7) -- (9);
\draw [cyan] (9) -- (11);


\draw [olive] (4) -- (10);
\draw [teal] (8) -- (2);


\draw [gray] (5) .. controls (\a,{1.6*( \a +  \a + \b )}) and ({ 2*( \a +  \a + \b ) - 2} ,{1.6*( \a +  \a + \b )}) .. (7)
                coordinate [pos=0.3] (6);
%              node[pos=0.3, c3,fill=gray, label={above left: $6$}] {};

\draw [brown] (17) .. controls ({ ( \a +  \a + \b ) - 2},{1.5*( \a +  \a + \b )}) and ({  ( \a +  \a + \b ) + 2} ,{1.5*( \a +  \a + \b )}) .. (1)
                coordinate [pos=0.8] (18);
%              node[pos=0.3, c3,fill=gray, label={above left: $6$}] {};

\draw [violet] (11) .. controls ({ ( \a +  \a + \b ) - 2},{1.3*( \a +  \a + \b )}) and ({ ( \a +  \a + \b ) + 2} ,{1.3*( \a +  \a + \b )}) .. (13)
                coordinate [pos=0.2] (12);
%              node[pos=0.3, c3,fill=gray, label={above left: $6$}] {};


\draw (1) coordinate[c3,fill=red,label={below right:  $\{1,2,3\}$               }];
\draw (2) coordinate[c3,fill=red,label={below right:  $\{4,5,6,7\}$             }];
\draw (3) coordinate[c3,fill=red,label={below:        $\{8,9,10,11,12\}$        }];
\draw (4) coordinate[c3,fill=blue,label={below left:       $\{1,4,5,6\}$             }];
\draw (5) coordinate[c3,fill=blue,label={below left:       $\{2,3,7\}$               }];
\draw (6) coordinate[c3,fill=gray,label={above left: \tiny $\{1,4,5,8,9,10\}$        }];
\draw (7) coordinate[c3,fill=magenta,label={below left:  \tiny  $\{6,11,12\}$             }];
\draw (8) coordinate[c3,fill=magenta,label={above left:  \tiny  $\{1,2,8,9\}$             }];
\draw (9) coordinate[c3,fill=cyan,label={below:    \tiny   $\{3,4,5,7,10\}$          }];
\draw (10) coordinate[c3,fill=cyan,label={above right:   \tiny   $\{2,8,9,11\}$            }];
\draw (11) coordinate[c3,fill=cyan,label={below right :   \tiny   $\{1,6,12\}$              }];
\draw (12) coordinate[c3,fill=violet,label={below right:     $\{2,3,4,8,10,11\}$       }];
\draw (13) coordinate[c3,fill=green,label={below:  \tiny   $\{5,7,9\}$               }];
\draw (14) coordinate[c3,fill=green,label={below:   \tiny  $\{3,10,11,12\}$          }];
\draw (15) coordinate[c3,fill=orange,label={below:  \tiny  $\{1,2,4,6,8\}$           }];
\draw (16) coordinate[c3,fill=orange,label={below:  \tiny  $\{3,7,10,12\}$           }];
\draw (17) coordinate[c3,fill=orange,label={below:  \tiny  $\{5,9,11\}$              }];
\draw (18) coordinate[c3,fill=brown,label={above right:     $\{4,6,7,8,10,12\}$       }];



\draw (2) coordinate[c2,fill=teal];
\draw (8) coordinate[c2,fill=teal];
\draw (14) coordinate[c2,fill=teal];

\draw (4) coordinate[c2,fill=olive];
\draw (16) coordinate[c2,fill=olive];
\draw (10) coordinate[c2,fill=olive];

\draw (9) coordinate[c2,fill=magenta];
\draw (15) coordinate[c2,fill=green];
\draw (3) coordinate[c2,fill=blue];

\draw (5) coordinate[c2,fill=gray];
\draw (17) coordinate[c2,fill=brown];
\draw (11) coordinate[c2,fill=violet];

\draw (7) coordinate[c2,fill=gray];
\draw (1) coordinate[c2,fill=brown];
\draw (13) coordinate[c2,fill=violet];


\end{tikzpicture}}
\end{center}
\caption{\label{2023-e-f1}
M\"obius-Penrose-Escher type periodic diagram.
}
\end{figure}

\begin{figure}
\begin{center}
\resizebox{.45\textwidth}{!}{\begin{tikzpicture}  [scale=1] %,label distance=1pt
\tikzstyle{every path}=[line width=2pt]

\tikzmath{\a = 1; \b = 2; \c = 0;}


\newdimen\ms
\ms=0.05cm


\tikzstyle{c3}=[circle,inner sep={\ms/8},minimum size=6*\ms]
\tikzstyle{c2}=[circle,inner sep={\ms/8},minimum size=4*\ms]
\tikzstyle{c1}=[circle,inner sep={\ms/8},minimum size=0.8*\ms]

\coordinate (1) at ({2*( \a +  \a + \b ) + 2*\c},{( \a +  \a + \b ) + 2*\c});
\coordinate (3) at ({( \a +  \a + \b )},0);
\coordinate (2) at ($(1)!0.5!(3)$);
\coordinate (5) at ({0-2*\c},{( \a +  \a + \b )+2*\c});
\coordinate (4) at ($(3)!0.5!(5)$);
\coordinate (13) at ({2*( \a +  \a + \b )-\a+\c},{( \a +  \a + \b )+\c});
\coordinate (15) at ({( \a +  \a + \b )},\a);
\coordinate (14) at ($(13)!0.5!(15)$);
\coordinate (17) at ({\a-\c},{( \a +  \a + \b )+\c});
\coordinate (16) at ($(15)!0.5!(17)$);
\coordinate (7) at ({2*( \a +  \a + \b )-2*\a},{( \a +  \a + \b )});
\coordinate (9) at ({( \a +  \a + \b )},{2*\a});
\coordinate (8) at ($(7)!0.5!(9)$);
\coordinate (11) at ({2*\a},{( \a +  \a + \b )});
\coordinate (10) at ($(9)!0.5!(11)$);

%\coordinate (6) at ($(5)!0.915!(7)$);
 \coordinate (12) at ($(11)!0.5!(13)$);
 \coordinate (18) at ($(17)!0.5!(1)$);

\draw [red] (1) -- (3);
\draw [blue] (3) -- (5);

\draw [green] (13) -- (15);
\draw [orange] (15) -- (17);

\draw [magenta] (7) -- (9);
\draw [cyan] (9) -- (11);



\draw [gray] (5) .. controls (\a,{1.6*( \a +  \a + \b )}) and ({ 2*( \a +  \a + \b ) - 2} ,{1.6*( \a +  \a + \b )}) .. (7)
                coordinate [pos=0.3] (6);
%              node[pos=0.3, c3,fill=gray, label={above left: $6$}] {};

\draw [brown] (17) .. controls ({ ( \a +  \a + \b ) - 2},{1.5*( \a +  \a + \b )}) and ({  ( \a +  \a + \b ) + 2} ,{1.5*( \a +  \a + \b )}) .. (1)
                coordinate [pos=0.223] (18);
%              node[pos=0.3, c3,fill=gray, label={above left: $6$}] {};

\draw [violet] (11) .. controls ({ ( \a +  \a + \b ) - 2},{1.3*( \a +  \a + \b )}) and ({ ( \a +  \a + \b ) + 2} ,{1.3*( \a +  \a + \b )}) .. (13)
                coordinate [pos=0.2] (12);
%              node[pos=0.3, c3,fill=gray, label={above left: $6$}] {};

\draw [teal] (6) -- (12);

\draw (1) coordinate[c3,fill=red,label={below right: $1$}];
\draw (2) coordinate[c3,fill=red,label={below right: $2$}];
\draw (3) coordinate[c3,fill=red,label={below: $3$}];
\draw (4) coordinate[c3,fill=blue,label={below: $4$}];
\draw (5) coordinate[c3,fill=blue,label={below: $5$}];
\draw (6) coordinate[c3,fill=gray,label={above left: $6$}];
\draw (7) coordinate[c3,fill=magenta,label={below: $7$}];
\draw (8) coordinate[c3,fill=magenta,label={below: $8$}];
\draw (9) coordinate[c3,fill=cyan,label={below: $9$}];
\draw (10) coordinate[c3,fill=cyan,label={below: $10$}];
\draw (11) coordinate[c3,fill=cyan,label={below: $11$}];
\draw (12) coordinate[c3,fill=violet,label={below: $12$}];
\draw (13) coordinate[c3,fill=green,label={below: $13$}];
\draw (14) coordinate[c3,fill=green,label={below: $14$}];
\draw (15) coordinate[c3,fill=orange,label={below: $15$}];
\draw (16) coordinate[c3,fill=orange,label={below: $16$}];
\draw (17) coordinate[c3,fill=orange,label={below: $17$}];
\draw (18) coordinate[c3,fill=brown,label={left: $18$}];



\draw (9) coordinate[c2,fill=magenta];
\draw (15) coordinate[c2,fill=green];
\draw (3) coordinate[c2,fill=blue];

\draw (5) coordinate[c2,fill=gray];
\draw (17) coordinate[c2,fill=brown];
\draw (11) coordinate[c2,fill=violet];

\draw (7) coordinate[c2,fill=gray];
\draw (1) coordinate[c2,fill=brown];
\draw (13) coordinate[c2,fill=violet];

\draw (12) coordinate[c2,fill=teal];
\draw (18) coordinate[c2,fill=teal];
\draw (6) coordinate[c2,fill=teal];


\end{tikzpicture}}
\end{center}
\caption{\label{2023-e-f1}
M\"obius-Penrose-Escher type periodic diagram.
}
\end{figure}


\begin{figure}
\begin{center}
\resizebox{.46\textwidth}{!}{\begin{tikzpicture}  [scale=1] %,label distance=1pt
\tikzstyle{every path}=[line width=2pt]

\tikzmath{\a = 1; \b = 2; \c = 0;}


\newdimen\ms
\ms=0.05cm


\tikzstyle{c3}=[circle,inner sep={\ms/8},minimum size=6*\ms]
\tikzstyle{c2}=[circle,inner sep={\ms/8},minimum size=4*\ms]
\tikzstyle{c1}=[circle,inner sep={\ms/8},minimum size=0.8*\ms]

\coordinate (1) at ({2*( \a +  \a + \b ) + 2*\c},{( \a +  \a + \b ) + 2*\c});
\coordinate (3) at ({( \a +  \a + \b )},0);
\coordinate (2) at ($(1)!0.5!(3)$);
\coordinate (5) at ({0-2*\c},{( \a +  \a + \b )+2*\c});
\coordinate (4) at ($(3)!0.5!(5)$);
\coordinate (13) at ({2*( \a +  \a + \b )-\a+\c},{( \a +  \a + \b )+\c});
\coordinate (15) at ({( \a +  \a + \b )},\a);
\coordinate (14) at ($(13)!0.5!(15)$);
\coordinate (17) at ({\a-\c},{( \a +  \a + \b )+\c});
\coordinate (16) at ($(15)!0.5!(17)$);
\coordinate (7) at ({2*( \a +  \a + \b )-2*\a},{( \a +  \a + \b )});
\coordinate (9) at ({( \a +  \a + \b )},{2*\a});
\coordinate (8) at ($(7)!0.5!(9)$);
\coordinate (11) at ({2*\a},{( \a +  \a + \b )});
\coordinate (10) at ($(9)!0.5!(11)$);

%\coordinate (6) at ($(5)!0.915!(7)$);
 \coordinate (12) at ($(11)!0.5!(13)$);
 \coordinate (18) at ($(17)!0.5!(1)$);

\draw [red] (1) -- (3);
\draw [blue] (3) -- (5);

\draw [green] (13) -- (15);
\draw [orange] (15) -- (17);

\draw [magenta] (7) -- (9);
\draw [cyan] (9) -- (11);



\draw [gray] (5) .. controls (\a,{1.6*( \a +  \a + \b )}) and ({ 2*( \a +  \a + \b ) - 2} ,{1.6*( \a +  \a + \b )}) .. (7)
                coordinate [pos=0.3] (6);
%              node[pos=0.3, c3,fill=gray, label={above left: $6$}] {};

\draw [brown] (17) .. controls ({ ( \a +  \a + \b ) - 2},{1.5*( \a +  \a + \b )}) and ({  ( \a +  \a + \b ) + 2} ,{1.5*( \a +  \a + \b )}) .. (1)
                coordinate [pos=0.8] (18);
%              node[pos=0.3, c3,fill=gray, label={above left: $6$}] {};

\draw [violet] (11) .. controls ({ ( \a +  \a + \b ) - 2},{1.3*( \a +  \a + \b )}) and ({ ( \a +  \a + \b ) + 2} ,{1.3*( \a +  \a + \b )}) .. (13)
                coordinate [pos=0.2] (12);
%              node[pos=0.3, c3,fill=gray, label={above left: $6$}] {};


\draw (1) coordinate[c3,fill=red,label={below right: $1$}];
\draw (2) coordinate[c3,fill=red,label={below right: $2$}];
\draw (3) coordinate[c3,fill=red,label={below: $3$}];
\draw (4) coordinate[c3,fill=blue,label={below: $4$}];
\draw (5) coordinate[c3,fill=blue,label={below: $5$}];
\draw (6) coordinate[c3,fill=gray,label={above left: $6$}];
\draw (7) coordinate[c3,fill=magenta,label={below: $7$}];
\draw (8) coordinate[c3,fill=magenta,label={below: $8$}];
\draw (9) coordinate[c3,fill=cyan,label={below: $9$}];
\draw (10) coordinate[c3,fill=cyan,label={below: $10$}];
\draw (11) coordinate[c3,fill=cyan,label={below: $11$}];
\draw (12) coordinate[c3,fill=violet,label={below: $12$}];
\draw (13) coordinate[c3,fill=green,label={below: $13$}];
\draw (14) coordinate[c3,fill=green,label={below: $14$}];
\draw (15) coordinate[c3,fill=orange,label={below: $15$}];
\draw (16) coordinate[c3,fill=orange,label={below: $16$}];
\draw (17) coordinate[c3,fill=orange,label={below: $17$}];
\draw (18) coordinate[c3,fill=brown,label={above: $18$}];



\draw (9) coordinate[c2,fill=magenta];
\draw (15) coordinate[c2,fill=green];
\draw (3) coordinate[c2,fill=blue];

\draw (5) coordinate[c2,fill=gray];
\draw (17) coordinate[c2,fill=brown];
\draw (11) coordinate[c2,fill=violet];

\draw (7) coordinate[c2,fill=gray];
\draw (1) coordinate[c2,fill=brown];
\draw (13) coordinate[c2,fill=violet];


\end{tikzpicture}}
\end{center}
\caption{\label{2023-e-f1}
M\"obius-Penrose-Escher type periodic diagram.
}
\end{figure}


\begin{figure}
\begin{center}
\resizebox{.46\textwidth}{!}{\begin{tikzpicture}  [scale=1] %,label distance=1pt
\tikzstyle{every path}=[line width=2pt]

\tikzmath{\a = 1; \b = 2; \c = 0;}


\newdimen\ms
\ms=0.05cm


\tikzstyle{c3}=[circle,inner sep={\ms/8},minimum size=6*\ms]
\tikzstyle{c2}=[circle,inner sep={\ms/8},minimum size=4*\ms]
\tikzstyle{c1}=[circle,inner sep={\ms/8},minimum size=0.8*\ms]

\coordinate (1) at ({2*( \a +  \a + \b ) + 2*\c},{( \a +  \a + \b ) + 2*\c});
\coordinate (3) at ({( \a +  \a + \b )},0);
\coordinate (2) at ($(1)!0.5!(3)$);
\coordinate (5) at ({0-2*\c},{( \a +  \a + \b )+2*\c});
\coordinate (4) at ($(3)!0.5!(5)$);
\coordinate (13) at ({2*( \a +  \a + \b )-\a+\c},{( \a +  \a + \b )+\c});
\coordinate (15) at ({( \a +  \a + \b )},\a);
\coordinate (14) at ($(13)!0.5!(15)$);
\coordinate (17) at ({\a-\c},{( \a +  \a + \b )+\c});
\coordinate (16) at ($(15)!0.5!(17)$);
\coordinate (7) at ({2*( \a +  \a + \b )-2*\a},{( \a +  \a + \b )});
\coordinate (9) at ({( \a +  \a + \b )},{2*\a});
\coordinate (8) at ($(7)!0.5!(9)$);
\coordinate (11) at ({2*\a},{( \a +  \a + \b )});
\coordinate (10) at ($(9)!0.5!(11)$);

%\coordinate (6) at ($(5)!0.915!(7)$);
 \coordinate (12) at ($(11)!0.5!(13)$);
 \coordinate (18) at ($(17)!0.5!(1)$);

\draw [red] (1) -- (3);
\draw [blue] (3) -- (5);

\draw [magenta] (7) -- (9);
\draw [cyan] (9) -- (11);




\draw [gray] (5) .. controls (\a,{1.6*( \a +  \a + \b )}) and ({ 2*( \a +  \a + \b ) - 2} ,{1.6*( \a +  \a + \b )}) .. (7)
                coordinate [pos=0.3] (6);
%              node[pos=0.3, c3,fill=gray, label={above left: $6$}] {};

\draw [violet] (11) .. controls ({ ( \a +  \a + \b ) - 2},{1.6*( \a +  \a + \b )}) and ({ ( \a +  \a + \b ) + 2} ,{1.6*( \a +  \a + \b )}) .. (1)
                coordinate [pos=0.7] (12);
%              node[pos=0.3, c3,fill=gray, label={above left: $6$}] {};


\draw (1) coordinate[c3,fill=red,label={below right: $1$}];
\draw (2) coordinate[c3,fill=red,label={below right: $2$}];
\draw (3) coordinate[c3,fill=red,label={below: $3$}];
\draw (4) coordinate[c3,fill=blue,label={below: $4$}];
\draw (5) coordinate[c3,fill=blue,label={below: $5$}];
\draw (6) coordinate[c3,fill=gray,label={above left: $6$}];
\draw (7) coordinate[c3,fill=magenta,label={below: $7$}];
\draw (8) coordinate[c3,fill=magenta,label={below: $8$}];
\draw (9) coordinate[c3,fill=cyan,label={below: $9$}];
\draw (10) coordinate[c3,fill=cyan,label={below: $10$}];
\draw (11) coordinate[c3,fill=cyan,label={below: $11$}];
\draw (12) coordinate[c3,fill=violet,label={above: $12$}];



\draw (9) coordinate[c2,fill=magenta];
\draw (3) coordinate[c2,fill=blue];

\draw (5) coordinate[c2,fill=gray];
\draw (11) coordinate[c2,fill=violet];

\draw (7) coordinate[c2,fill=gray];
\draw (13) coordinate[c2,fill=violet];


\end{tikzpicture}}
\end{center}
\caption{\label{2023-e-f1}
Periodic diagram which is equivalent to a hexagon.
}
\end{figure}


\begin{acknowledgments}
We are grateful to Josef Tkadlec for providing a {\em Pascal} program that computes and analyses the set of two-valued states of collections of contexts.
We are also grateful to  Norman D. Megill and Mladen Pavi{\v{c}}i{\'{c}} for providing a {\em C++} program that heuristically computes the faithful orthogonal representations of hypergraphs written in MMP format, given possible vector components.


This research was funded in whole, or in part, by the Austrian Science Fund (FWF), Project No. I 4579-N,
and by the Czech Science Foundation grant 20-09869L.

The authors declare no conflict of interest.
\end{acknowledgments}


\bibliography{svozil}
\ifws

\bibliographystyle{spmpsci}

\else
 \bibliographystyle{apsrev}

\fi

\end{document}





1={2,4,-5},
2={1,2,2},
3={2,-1,0},
4={0,0,1},
5={1,2,0},
6={2, -1, 5},
7={-2,1,1},
8={0,1,-1},
9={1,1,1},
A={1,-1,0},
B={1,1,-2},
C={-45, 15, -15},
D={6, 21, 3},
E={4,-1,-1},
F={-1,1,-5},
G={1,1,0},
H={-10,10,4},
I={11,7,10}

FORUnnormalized={
{2,4,-5},
{1,2,2},
{2,-1,0},
{0,0,1},
{1,2,0},
{2, -1, 5},
{-2,1,1},
{0,1,-1},
{1,1,1},
{1,-1,0},
{1,1,-2},
{-45, 15, -15},
{6, 21, 3},
{4,-1,-1},
{-1,1,-5},
{1,1,0},
{-10,10,4},
{11,7,10}
};
FOR = Table[ Normalize[ FORUnnormalized[[i]] ],{i,1,Length[FORUnnormalized]}]

Length[FOR]==Length[Union[FOR]]

(* should be mutually orthogonal within contexts/blocks *)

checkorthogonality[a_,b_,c_] := {a.b,a.c,b.c};

FullSimplify[checkorthogonality[FOR[[  1 ]] , FOR[[2  ]] , FOR[[ 3 ]]]]
FullSimplify[checkorthogonality[FOR[[  3 ]] , FOR[[ 4 ]] , FOR[[ 5 ]]]]
FullSimplify[checkorthogonality[FOR[[ 5  ]] , FOR[[ 6 ]] , FOR[[ 7 ]]]]
FullSimplify[checkorthogonality[FOR[[ 7  ]] , FOR[[ 8 ]] , FOR[[ 9 ]]]]
FullSimplify[checkorthogonality[FOR[[ 9  ]] , FOR[[ 10 ]] , FOR[[ 11 ]]]]
FullSimplify[checkorthogonality[FOR[[ 11  ]] , FOR[[ 12 ]] , FOR[[ 13  ]]]]
FullSimplify[checkorthogonality[FOR[[ 13  ]] , FOR[[ 14 ]] , FOR[[ 15 ]]]]
FullSimplify[checkorthogonality[FOR[[ 15  ]] , FOR[[ 16 ]] , FOR[[ 17 ]]]]
FullSimplify[checkorthogonality[FOR[[ 17  ]] , FOR[[ 18 ]] , FOR[[ 1 ]]]]
FullSimplify[checkorthogonality[FOR[[ 4  ]] , FOR[[ 16 ]] , FOR[[ 10 ]]]]
FullSimplify[checkorthogonality[FOR[[ 2  ]] , FOR[[ 14 ]] , FOR[[ 8 ]]]]











%%%%%%%%%%%%%%%%%%%%%%%%%%%%%%%%%%%%%%%%%%%%%%%%%%%%%%%%%%%%%%%%%%%%%%%%%%%%%%%%%%%%%%%%%%%%%%%%%%%%%%%%%%%%
%%%%%%%%%%%%%%%%%%%%%%%%%%%%%%%%%%%%%%%%%%%%%%%%%%%%%%%%%%%%%%%%%%%%%%%%%%%%%%%%%%%%%%%%%%%%%%%%%%%%%%%%%%%%
%%%%%%%%%%%%%%%%%%%%%%%%%%%%%%%%%%%%%%%%%%%%%%%%%%%%%%%%%%%%%%%%%%%%%%%%%%%%%%%%%%%%%%%%%%%%%%%%%%%%%%%%%%%%
%%%%%%%%%%%%%%%%%%%%%%%%%%%%%%%%%%%%%%%%%%%%%%%%%%%%%%%%%%%%%%%%%%%%%%%%%%%%%%%%%%%%%%%%%%%%%%%%%%%%%%%%%%%%
%%%%%%%%%%%%%%%%%%%%%%%%%%%%%%%%%%%%%%%%%%%%%%%%%%%%%%%%%%%%%%%%%%%%%%%%%%%%%%%%%%%%%%%%%%%%%%%%%%%%%%%%%%%%
%%%%%%%%%%%%%%%%%%%%%%%%%%%%%%%%%%%%%%%%%%%%%%%%%%%%%%%%%%%%%%%%%%%%%%%%%%%%%%%%%%%%%%%%%%%%%%%%%%%%%%%%%%%%
%%%%%%%%%%%%%%%%%%%%%%%%%%%%%%%%%%%%%%%%%%%%%%%%%%%%%%%%%%%%%%%%%%%%%%%%%%%%%%%%%%%%%%%%%%%%%%%%%%%%%%%%%%%%
%%%%%%%%%%%%%%%%%%%%%%%%%%%%%%%%%%%%%%%%%%%%%%%%%%%%%%%%%%%%%%%%%%%%%%%%%%%%%%%%%%%%%%%%%%%%%%%%%%%%%%%%%%%%
%%%%%%%%%%%%%%%%%%%%%%%%%%%%%%%%%%%%%%%%%%%%%%%%%%%%%%%%%%%%%%%%%%%%%%%%%%%%%%%%%%%%%%%%%%%%%%%%%%%%%%%%%%%%
%%%%%%%%%%%%%%%%%%%%%%%%%%%%%%%%%%%%%%%%%%%%%%%%%%%%%%%%%%%%%%%%%%%%%%%%%%%%%%%%%%%%%%%%%%%%%%%%%%%%%%%%%%%%
%%%%%%%%%%%%%%%%%%%%%%%%%%%%%%%%%%%%%%%%%%%%%%%%%%%%%%%%%%%%%%%%%%%%%%%%%%%%%%%%%%%%%%%%%%%%%%%%%%%%%%%%%%%%
%%%%%%%%%%%%%%%%%%%%%%%%%%%%%%%%%%%%%%%%%%%%%%%%%%%%%%%%%%%%%%%%%%%%%%%%%%%%%%%%%%%%%%%%%%%%%%%%%%%%%%%%%%%%
%%%%%%%%%%%%%%%%%%%%%%%%%%%%%%%%%%%%%%%%%%%%%%%%%%%%%%%%%%%%%%%%%%%%%%%%%%%%%%%%%%%%%%%%%%%%%%%%%%%%%%%%%%%%
%%%%%%%%%%%%%%%%%%%%%%%%%%%%%%%%%%%%%%%%%%%%%%%%%%%%%%%%%%%%%%%%%%%%%%%%%%%%%%%%%%%%%%%%%%%%%%%%%%%%%%%%%%%%
%%%%%%%%%%%%%%%%%%%%%%%%%%%%%%%%%%%%%%%%%%%%%%%%%%%%%%%%%%%%%%%%%%%%%%%%%%%%%%%%%%%%%%%%%%%%%%%%%%%%%%%%%%%%
%%%%%%%%%%%%%%%%%%%%%%%%%%%%%%%%%%%%%%%%%%%%%%%%%%%%%%%%%%%%%%%%%%%%%%%%%%%%%%%%%%%%%%%%%%%%%%%%%%%%%%%%%%%%
%%%%%%%%%%%%%%%%%%%%%%%%%%%%%%%%%%%%%%%%%%%%%%%%%%%%%%%%%%%%%%%%%%%%%%%%%%%%%%%%%%%%%%%%%%%%%%%%%%%%%%%%%%%%
%%%%%%%%%%%%%%%%%%%%%%%%%%%%%%%%%%%%%%%%%%%%%%%%%%%%%%%%%%%%%%%%%%%%%%%%%%%%%%%%%%%%%%%%%%%%%%%%%%%%%%%%%%%%
%%%%%%%%%%%%%%%%%%%%%%%%%%%%%%%%%%%%%%%%%%%%%%%%%%%%%%%%%%%%%%%%%%%%%%%%%%%%%%%%%%%%%%%%%%%%%%%%%%%%%%%%%%%%
%%%%%%%%%%%%%%%%%%%%%%%%%%%%%%%%%%%%%%%%%%%%%%%%%%%%%%%%%%%%%%%%%%%%%%%%%%%%%%%%%%%%%%%%%%%%%%%%%%%%%%%%%%%%
%%%%%%%%%%%%%%%%%%%%%%%%%%%%%%%%%%%%%%%%%%%%%%%%%%%%%%%%%%%%%%%%%%%%%%%%%%%%%%%%%%%%%%%%%%%%%%%%%%%%%%%%%%%%
%%%%%%%%%%%%%%%%%%%%%%%%%%%%%%%%%%%%%%%%%%%%%%%%%%%%%%%%%%%%%%%%%%%%%%%%%%%%%%%%%%%%%%%%%%%%%%%%%%%%%%%%%%%%
%%%%%%%%%%%%%%%%%%%%%%%%%%%%%%%%%%%%%%%%%%%%%%%%%%%%%%%%%%%%%%%%%%%%%%%%%%%%%%%%%%%%%%%%%%%%%%%%%%%%%%%%%%%%
%%%%%%%%%%%%%%%%%%%%%%%%%%%%%%%%%%%%%%%%%%%%%%%%%%%%%%%%%%%%%%%%%%%%%%%%%%%%%%%%%%%%%%%%%%%%%%%%%%%%%%%%%%%%
%%%%%%%%%%%%%%%%%%%%%%%%%%%%%%%%%%%%%%%%%%%%%%%%%%%%%%%%%%%%%%%%%%%%%%%%%%%%%%%%%%%%%%%%%%%%%%%%%%%%%%%%%%%%
%%%%%%%%%%%%%%%%%%%%%%%%%%%%%%%%%%%%%%%%%%%%%%%%%%%%%%%%%%%%%%%%%%%%%%%%%%%%%%%%%%%%%%%%%%%%%%%%%%%%%%%%%%%%
%%%%%%%%%%%%%%%%%%%%%%%%%%%%%%%%%%%%%%%%%%%%%%%%%%%%%%%%%%%%%%%%%%%%%%%%%%%%%%%%%%%%%%%%%%%%%%%%%%%%%%%%%%%%
%%%%%%%%%%%%%%%%%%%%%%%%%%%%%%%%%%%%%%%%%%%%%%%%%%%%%%%%%%%%%%%%%%%%%%%%%%%%%%%%%%%%%%%%%%%%%%%%%%%%%%%%%%%%
%%%%%%%%%%%%%%%%%%%%%%%%%%%%%%%%%%%%%%%%%%%%%%%%%%%%%%%%%%%%%%%%%%%%%%%%%%%%%%%%%%%%%%%%%%%%%%%%%%%%%%%%%%%%
%%%%%%%%%%%%%%%%%%%%%%%%%%%%%%%%%%%%%%%%%%%%%%%%%%%%%%%%%%%%%%%%%%%%%%%%%%%%%%%%%%%%%%%%%%%%%%%%%%%%%%%%%%%%
%%%%%%%%%%%%%%%%%%%%%%%%%%%%%%%%%%%%%%%%%%%%%%%%%%%%%%%%%%%%%%%%%%%%%%%%%%%%%%%%%%%%%%%%%%%%%%%%%%%%%%%%%%%%
%%%%%%%%%%%%%%%%%%%%%%%%%%%%%%%%%%%%%%%%%%%%%%%%%%%%%%%%%%%%%%%%%%%%%%%%%%%%%%%%%%%%%%%%%%%%%%%%%%%%%%%%%%%%
%%%%%%%%%%%%%%%%%%%%%%%%%%%%%%%%%%%%%%%%%%%%%%%%%%%%%%%%%%%%%%%%%%%%%%%%%%%%%%%%%%%%%%%%%%%%%%%%%%%%%%%%%%%%
%%%%%%%%%%%%%%%%%%%%%%%%%%%%%%%%%%%%%%%%%%%%%%%%%%%%%%%%%%%%%%%%%%%%%%%%%%%%%%%%%%%%%%%%%%%%%%%%%%%%%%%%%%%%
%%%%%%%%%%%%%%%%%%%%%%%%%%%%%%%%%%%%%%%%%%%%%%%%%%%%%%%%%%%%%%%%%%%%%%%%%%%%%%%%%%%%%%%%%%%%%%%%%%%%%%%%%%%%
%%%%%%%%%%%%%%%%%%%%%%%%%%%%%%%%%%%%%%%%%%%%%%%%%%%%%%%%%%%%%%%%%%%%%%%%%%%%%%%%%%%%%%%%%%%%%%%%%%%%%%%%%%%%


(*


Mirko Navara degenerate left side
18 atoms
11 blocks
 0 proper subsets of blocks
 3   1  2  3
 3   3  4  5
 3   5  6  7
 3   7  8  9
 3   9 10 11
 3  11 12 13
 3  13 14 15
 3  15 16 17
 3  17 18  1
 3   4 10 16
 3   2 14  8
12 2-valued evaluations of atoms:
1 0 0 1 0 1 0 1 0 0 1 0 0 0 1 0 0 0
1 0 0 0 1 0 0 1 0 1 0 1 0 0 1 0 0 0
1 0 0 0 1 0 0 0 1 0 0 1 0 1 0 1 0 0
0 1 0 1 0 1 0 0 1 0 0 1 0 0 1 0 0 1
0 1 0 1 0 1 0 0 1 0 0 0 1 0 0 0 1 0
0 1 0 1 0 0 1 0 0 0 1 0 0 0 1 0 0 1
0 1 0 0 1 0 0 0 1 0 0 0 1 0 0 1 0 1
0 0 1 0 0 1 0 1 0 1 0 1 0 0 1 0 0 1
0 0 1 0 0 1 0 1 0 1 0 0 1 0 0 0 1 0
0 0 1 0 0 1 0 0 1 0 0 1 0 1 0 1 0 1
0 0 1 0 0 0 1 0 0 1 0 1 0 1 0 0 1 0
0 0 1 0 0 0 1 0 0 0 1 0 0 1 0 1 0 1

set of 2-valued evaluations of atoms:
nonempty: yes
unital: yes
separating atoms: yes
separating: yes
1s on nonorthogonal atoms (=OD if no noncomplete block): no for atoms
 1/7 1/13 2/10 4/14 5/11 5/17 7/13 8/16 11/17
order determining: no for sets of atoms (ordered elements?)
 2/9+11
 7/2+3
 13/2+3
 1/5+6
 1/11+12
 10/1+3
 5/9+10
 5/15+16
 4/13+15
 14/3+5
 11/3+4
 17/3+4
 7/11+12
 13/5+6
 8/15+17
 16/7+9
 11/15+16
 17/9+10


*)

states = {
{1, 0, 0, 1, 0, 1, 0, 1, 0, 0, 1, 0, 0, 0, 1, 0, 0, 0},
{1, 0, 0, 0, 1, 0, 0, 1, 0, 1, 0, 1, 0, 0, 1, 0, 0, 0},
{1, 0, 0, 0, 1, 0, 0, 0, 1, 0, 0, 1, 0, 1, 0, 1, 0, 0},
{0, 1, 0, 1, 0, 1, 0, 0, 1, 0, 0, 1, 0, 0, 1, 0, 0, 1},
{0, 1, 0, 1, 0, 1, 0, 0, 1, 0, 0, 0, 1, 0, 0, 0, 1, 0},
{0, 1, 0, 1, 0, 0, 1, 0, 0, 0, 1, 0, 0, 0, 1, 0, 0, 1},
{0, 1, 0, 0, 1, 0, 0, 0, 1, 0, 0, 0, 1, 0, 0, 1, 0, 1},
{0, 0, 1, 0, 0, 1, 0, 1, 0, 1, 0, 1, 0, 0, 1, 0, 0, 1},
{0, 0, 1, 0, 0, 1, 0, 1, 0, 1, 0, 0, 1, 0, 0, 0, 1, 0},
{0, 0, 1, 0, 0, 1, 0, 0, 1, 0, 0, 1, 0, 1, 0, 1, 0, 1},
{0, 0, 1, 0, 0, 0, 1, 0, 0, 1, 0, 1, 0, 1, 0, 0, 1, 0},
{0, 0, 1, 0, 0, 0, 1, 0, 0, 0, 1, 0, 0, 1, 0, 1, 0, 1}};

If[Length[states]==12,Print["true"],Print["false"]];



stateindex= Table[{},{i,1,18}];

Do[ Do[ If[ states[[j,i]]==1, AppendTo[ stateindex[[i]],j]   ] ,{j,1,12}] ,{i,1,18}];

(*
Print[MatrixForm[stateindex]]


Union[ stateindex[[5]],stateindex[[11]],stateindex[[17]] ] == Union[ stateindex[[1]],stateindex[[13]],stateindex[[7]] ]

stateindex[[5]]
stateindex[[11]]
stateindex[[17]]
stateindex[[1]]
stateindex[[13]]
stateindex[[7]]
*)


stateindex[[2]]
stateindex[[3]]
stateindex[[4]]
stateindex[[10]]
stateindex[[9]]
stateindex[[8]]

stateindex[[2]]
stateindex[[3]]
stateindex[[4]]
stateindex[[16]]
stateindex[[15]]
stateindex[[14]]



%%%%%%%%%%%%%%%%%%%%%%%%%%%%%%%%%%%%%%%%%%%%%%%%%%%%%%%%%%%%%%%%%%%%%%%%%%%%
%%%%%%%%%%%%%%%%%%%%%%%%%%%%%%%%%%%%%%%%%%%%%%%%%%%%%%%%%%%%%%%%%%%%%%%%%%%%


(*
alpha = -Pi/3;
*)

alpha = ToRadicals[
   FullSimplify[
    ToRadicals[2 ArcTan[ Root[-27 + 99 #^2 + 87 #^4 + 25 #^6& , 2, 0]]]]];

v2={Sqrt[2/3], 0, Sqrt[1/3]};
v14= {-Sqrt[1/6], Sqrt[1/2], Sqrt[1/3]};
v26= {-Sqrt[1/6], -Sqrt[1/2], Sqrt[1/3]};

v12=RotationMatrix[ alpha , {0, 0, 1}] . v2;
v24=RotationMatrix[ alpha , {0, 0, 1}] . v14;
v36=RotationMatrix[ alpha , {0, 0, 1}] . v26;

v1  = Normalize[Cross[v2,v12]];
v13 = Normalize[Cross[v14,v24]];
v25 = Normalize[Cross[v26,v36]];

v3  = FullSimplify[ Normalize[Cross[v1,v2]    ] ]   ;
v15 = FullSimplify[ Normalize[Cross[v13,v14]  ] ]   ;
v27 = FullSimplify[ Normalize[Cross[v25,v26]  ] ]   ;
v11 = FullSimplify[ Normalize[Cross[v1,v12]   ] ]    ;
v23 = FullSimplify[ Normalize[Cross[v13,v24]  ] ]    ;
v35 = FullSimplify[ Normalize[Cross[v25,v36]  ] ]    ;

v16 = FullSimplify[ Normalize[Cross[v35,v3 ]  ] ]    ;
v17 = FullSimplify[ Normalize[Cross[v23,v27]  ] ]    ;
v18 = FullSimplify[ Normalize[Cross[v11,v15]  ] ]    ;



checkorthogonality[a_,b_,c_] := {a.b,a.c,b.c};

FullSimplify[checkorthogonality[ v2  ,  v14   ,  v26 ]]
FullSimplify[checkorthogonality[ v12  ,  v24  ,  v36 ]]
FullSimplify[checkorthogonality[ v1   ,  v2  ,  v3 ]]
FullSimplify[checkorthogonality[ v1   ,  v12  ,  v11 ]]
FullSimplify[checkorthogonality[ v13  ,  v14  ,  v15 ]]
FullSimplify[checkorthogonality[ v13   ,  v24  ,  v23  ]]
FullSimplify[checkorthogonality[ v25   ,  v26  ,  v27 ]]
FullSimplify[checkorthogonality[ v25   ,  v35 ,  v36 ]]
FullSimplify[checkorthogonality[ v35   ,  v3  ,  v16 ]]
FullSimplify[checkorthogonality[ v23   ,  v27  ,  v17 ]]
FullSimplify[checkorthogonality[ v11   ,  v15  ,  v18 ]]

N[ToRadicals[
  FullSimplify[ToRadicals[checkorthogonality[v16, v17, v18]]]]]

ToRadicals[FullSimplify[ToRadicals[v2  ]]]
ToRadicals[FullSimplify[ToRadicals[v14 ]]]
ToRadicals[FullSimplify[ToRadicals[v26 ]]]

ToRadicals[FullSimplify[ToRadicals[v12 ]]]
ToRadicals[FullSimplify[ToRadicals[v24 ]]]
ToRadicals[FullSimplify[ToRadicals[v36 ]]]

ToRadicals[FullSimplify[ToRadicals[v1  ]]]
ToRadicals[FullSimplify[ToRadicals[v13 ]]]
ToRadicals[FullSimplify[ToRadicals[v25 ]]]

ToRadicals[FullSimplify[ToRadicals[v3  ]]]
ToRadicals[FullSimplify[ToRadicals[v15 ]]]
ToRadicals[FullSimplify[ToRadicals[v27 ]]]
ToRadicals[FullSimplify[ToRadicals[v11 ]]]
ToRadicals[FullSimplify[ToRadicals[v23 ]]]
ToRadicals[FullSimplify[ToRadicals[v35 ]]]

ToRadicals[FullSimplify[ToRadicals[v16 ]]]
ToRadicals[FullSimplify[ToRadicals[v17 ]]]
ToRadicals[FullSimplify[ToRadicals[v18 ]]]


 (*Definition of `my' Tensor Product*)(*a,b are nxn and mxm-matrices*)

MyTensorProduct[a_, b_] :=
  Table[a[[Ceiling[s/Length[b]], Ceiling[t/Length[b]]]]*
    b[[s - Floor[(s - 1)/Length[b]]*Length[b],
      t - Floor[(t - 1)/Length[b]]*Length[b]]], {s, 1,
    Length[a]*Length[b]}, {t, 1, Length[a]*Length[b]}];

(*Definition of the Dyadic Product*)

DyadicProductVec[x_] :=
  Table[x[[i]]  Conjugate[x[[j]]], {i, 1, Length[x]}, {j, 1,
    Length[x]}];

(*Commutator*)

Commutator[a_, b_] := a . b - b . a;


triade = ToRadicals[FullSimplify[ToRadicals[DyadicProductVec[v3]+DyadicProductVec[v15]+DyadicProductVec[v27] ]]]

Eigensystem[triade]



triadeconnecting =
 ToRadicals[
  FullSimplify[
   ToRadicals[
    DyadicProductVec[v16] + DyadicProductVec[v17] +
     DyadicProductVec[v18]]]]














Navara Escher bug  extended
18 atoms
11 blocks
 0 proper subsets of blocks
 3   1  2  3
 3   3  4  5
 3   5  6  7
 3   7  8  9
 3   9 10 11
 3  11 12 13
 3  13 14 15
 3  15 16 17
 3  17 18  1
 3   4 16 10
 3   2  8 14
12 2-valued evaluations of atoms:
1 0 0 1 0 1 0 1 0 0 1 0 0 0 1 0 0 0
1 0 0 0 1 0 0 1 0 1 0 1 0 0 1 0 0 0
1 0 0 0 1 0 0 0 1 0 0 1 0 1 0 1 0 0
0 1 0 1 0 1 0 0 1 0 0 1 0 0 1 0 0 1
0 1 0 1 0 1 0 0 1 0 0 0 1 0 0 0 1 0
0 1 0 1 0 0 1 0 0 0 1 0 0 0 1 0 0 1
0 1 0 0 1 0 0 0 1 0 0 0 1 0 0 1 0 1
0 0 1 0 0 1 0 1 0 1 0 1 0 0 1 0 0 1
0 0 1 0 0 1 0 1 0 1 0 0 1 0 0 0 1 0
0 0 1 0 0 1 0 0 1 0 0 1 0 1 0 1 0 1
0 0 1 0 0 0 1 0 0 1 0 1 0 1 0 0 1 0
0 0 1 0 0 0 1 0 0 0 1 0 0 1 0 1 0 1

set of 2-valued evaluations of atoms:
nonempty: yes
unital: yes
separating atoms: yes
separating: yes
1s on nonorthogonal atoms (=OD if no noncomplete block): no for atoms
 1/7 1/13 2/10 4/14 5/11 5/17 7/13 8/16 11/17
order determining: no for sets of atoms (ordered elements?)
 2/9+11
 7/2+3
 13/2+3
 1/5+6
 1/11+12
 10/1+3
 5/9+10
 5/15+16
 4/13+15
 14/3+5
 11/3+4
 17/3+4
 7/11+12
 13/5+6
 8/15+17
 16/7+9
 11/15+16
 17/9+10

