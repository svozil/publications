%%tth:\begin{html}<LINK REL=STYLESHEET HREF="/~svozil/ssh.css">\end{html}
\documentclass[prl,preprint,showpacs,showkeys]{revtex4}
\usepackage{graphicx}
%\documentstyle[amsfonts]{article}
%\RequirePackage{times}
%\RequirePackage{courier}
%\RequirePackage{mathptm}
%\renewcommand{\baselinestretch}{1.3}
\begin{document}

%\def\frak{\cal }
%\def\Bbb{\bf }
%\sloppy



\title{States on the pentagon as a test of Hilbert space}
\author{Itamar Pitowsky}
\email{itamarp@vms.huji.ac.il}
\affiliation{Department of Philosophy, The Hebrew University,
Mount Scopus, Jerusalem 91905, Israel}
\author{Karl Svozil}
\email{svozil@tuwien.ac.at}
\homepage{http://tph.tuwien.ac.at/~svozil}
\affiliation{Institut f\"ur Theoretische Physik, University of Technology Vienna,
Wiedner Hauptstra\ss e 8-10/136, A-1040 Vienna, Austria}


\begin{abstract}
By Gleason's theorem,
quantum probabilities are strongly tied to
Hilbert space.
An experiment is proposed to test these quantum probabilities
on specific quantum structures which allow more general propability measures.
If such measures were indeed found, then the only consistent alternative
would be either the abandonment of Hilbert space, or of the most reasonable requirements
of probability theory, such as subadditivity.
\end{abstract}

\pacs{03.65.Ud,03.65.Ta}
\keywords{correlation polytopes, probability theory}

\maketitle

%\section{Boole-Bell inequalities}



\bibliography{svozil}
\bibliographystyle{apsrev}
%\bibliographystyle{unsrt}
%\bibliographystyle{plain}


\end{document}
