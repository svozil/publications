%\documentclass[pra,showpacs,showkeys,amsfonts,amsmath,twocolumn]{revtex4}
\documentclass[amsmath,red]{beamer}
%\documentclass[pra,showpacs,showkeys,amsfonts]{revtex4}

%\usepackage{beamerthemeshadow}
%\usepackage[dark]{beamerthemesidebar}
%\usepackage[headheight=24pt,footheight=12pt]{beamerthemesplit}
\usepackage{beamerthemesplit}
%\usepackage[bar]{beamerthemetree}
\usepackage{graphicx}
\usepackage{pgf}

\pgfdeclareimage[height=0.6cm]{logo}{tu-logo}
\logo{\pgfuseimage{logo}}

\title{\bf On Counterfactuals and Contextuality}
\subtitle{http://www.arxiv.org/abs/quant-ph/0406014}
\author{Karl Svozil}
\institute{Institut f\"ur Theoretische Physik, University of Technology Vienna, \\
Wiedner Hauptstra\ss e 8-10/136, A-1040 Vienna, Austria}
\begin{document}
\maketitle

%\frame{\tableofcontents}



\section{Counterfactuals}
\subsection{Debate in philosophy \& theology}

\frame[shrink=2]{
\frametitle{Realism-Idealism in philosophy \& theology}

\begin{itemize}
\item<+-> Realism: Some entities sometimes exist without being experienced by any finite mind.

\item<+-> Idealism:
$\ldots$
we have not the faintest reason for believing in the existence of
unexperienced entities
$\ldots$
[[Realism]] has been adopted
$\ldots$
solely because it simplifies our view of the universe.
\\
(W.T. Stace, Mind {\bf 53}, 1934 \& ``Readings in Philosophical
Analysis'', ed. by Feigl \& Sellars).
\end{itemize}
 }



\frame[shrink=2]{
\frametitle{Scholasticism in quantum physics}

%\begin{columns}
%\begin{column}{5cm}
%\pgfdeclareimage[height=2cm]{Specker}{specker}
%\pgfuseimage{Specker}
%\end{column}
%\begin{column}{9cm}
\begin{itemize}
\item<+->
Specker related the discussion on the foundations of quantum mechanics to scholastic philosophy;
in particular to scholastic speculations  about the existence of ``infuturabilities'' or
``counterfactuals.''

\item<+->
Question: Does
the omniscience (comprehensive knowledge) of God extend to events which
would have occurred if something  had happened which did not
happen?

\item<+->
Question: If so, can all events be pasted together to form a consistent whole?
\end{itemize}
%\end{column}
%\end{columns}
 }


\subsection{New features of quantum mechanics}
\frame[shrink=2]{
\frametitle{New features of quantum mechanics}

\begin{itemize}
\item<+-> Complementarity (nondistributive propositional structure not necessarily implies total abandonment of nonclassicality; e.g., automaton logic)

\item<+-> Value indefiniteness:
``not enough'' two-valued states to allow a faithful embedding into Boolean algebras.
E.g., nonseparable or unital set of two-valued states; even nonexistence of two-valued states on certain propositional structures (``Kochen-Specker theorem'')

\item<+-> Question:
Does the set of two-valued measures limit the class of probability measures (e.g., less two-valued measures, more states)? Or is the converse true? Or none of these statements?
\end{itemize}
 }


\section{Uniqueness property and explosion views}
\frame[shrink=2]{
\frametitle{Uniqueness property and explosion views}

\begin{itemize}
\item<+->
A multiquantum state satisfies the {\em uniqueness property} if
knowledge of a property of one quantum entails the certainty
that, if this property
were measured on the other quantum (or quanta) as well, the outcome of the measurement would be
a unique function of the outcome of the measurement performed.

\item<+->
One may pretend to obtain knowledge of
noncommeasurable observables referring to a single quantum
by measurement of one observable per quantum in a multiquantum state satisfying the uniqueness property at a time, and by subsequent counterfactual inference.

\item<+->
This kind of setups will be referred to as of the {\em ``explosion type.''}
\end{itemize}
}

\frame[shrink=2]{
\frametitle{Explosion views (contd)}

\begin{itemize}
\item<+->
Advantage: No terms present which refer to different detector parameter setups at different times.
(This makes standard CHSH \& GHZ measurements vulnerable to critique by Hess \& Philipp through time delayed consecutive measurements of relevant terms.)

\item<+->
Possible measurement of contextuality.

\item<+->
Difficulty: more than one particle is involved; so explosion views rely on the existence of counterfactual elements of physical reality, as suggested by EPR.
\end{itemize}
}

\section{Contextuality}
\frame[shrink=2]{
\frametitle{Contextuality envisioned by Bohr \& Bell}

\begin{itemize}
\item<+->
Bohr~\cite{bohr-1949}:
{\em ``the impossibility of any sharp separation
between the behavior of atomic objects and the interaction with the measuring instruments which serve to define
the conditions under which the phenomena appear.''}

\item<+->
Bell (Ref.~\cite{bell-66}, Sec.~5): the {\em ``$\ldots$
result of an observation may reasonably depend
not only on the state of the system  $\ldots$
but also on the complete disposition  of the apparatus.''}

\item<+->
That is, the outcome of the measurement of an observable  $A$
might depend on which other observables
from systems of maximal observables
are measured alongside with $A$.
\end{itemize}
}


\subsection{Context and link observables}
\frame[shrink=2]{
\frametitle{Definition of context}

\begin{itemize}
\item<+->
A {\em context} is a single (nondegenerate) ``maximal'' self-adjoint operator  ${ C}$.
It has a spectral
decomposition into some complete set of orthogonal projectors ${ E}_i$.
That is, ${ C}=\sum_{i=1}^n e_i { E}_i$
with mutually different $e_i$ and some orthonormal basis $\{{ E}_i{ H} \mid i=1,\ldots n\}$ of
$n$-dimensional Hilbert space ${ H}$.

\item<+->
In the finite subalgebras considered, an observable belonging to two or more contexts is called {\em link observable}.

\item<+->
In $n$ dimensions, contexts can be viewed as $n$-pods spanned by the $n$ orthogonal projectors
${ E}_1,
{ E}_2, \cdots,{ E}_n$.

\item<+->
In quantum logic, contexts are often referred to as {\em subalgebras} or {\em blocks.}
\end{itemize}
}

\frame[shrink=2]{
\frametitle{Classicality within contexts}
\begin{itemize}
\item<+->
Classical Boole-Kolmogorovian probability theory is based on the axiom
that the probability of the occurrence
of pairwise disjoint events $E_i$, $i=1,2, \ldots$
is the sum of those probabilities, and that the probability is bounded by zero and one;
i.e.,
$0\le  P(E_1,E_2,\ldots )= P(E_1)+P(E_2)+\cdots \le 1$.

\item<+->
Quantum mechanically, these ``classical'' properties persist within a given context.

\item<+->
Can these classical parts be pasted together to form a consistent whole?
\end{itemize}
}




\subsection{States}
\frame[shrink=2]{
\frametitle{Singlet state of two spin one quanta}

For singlet states of two quanta of three dimensions,  uniqueness holds:
$$
\vert \Psi_2 \rangle
= {1\over \sqrt{3}}(
\vert + -\rangle
+
\vert - +\rangle
-
\vert 0 0\rangle
).$$
}

\frame[shrink=2]{
\frametitle{Singlet state of three spin one quanta}

The only singlet state is does not satisfy the uniqueness property:
$$
\vert \Psi_3 \rangle
= {1\over \sqrt{6}}(
\vert - + 0\rangle
-
\vert - 0 +\rangle
+
\vert + 0 - \rangle
-
\vert + - 0\rangle
+
\vert 0 - + \rangle
-
\vert 0 + - \rangle
).
\label{2004-qnc-e1}
$$

Suppose the outcome of a spin measurement on the first quantum is ``$-$.''
$\vert \Psi_3 \rangle$ reduces to
$${1\over \sqrt{2}}(
\vert - + 0\rangle
-
\vert - 0 +\rangle
).$$

Two possibilities ``$0$''
and ``$+$'' remain for the state of every one of the other quanta.
This ambiguity in the counterfactual argument results in nonuniqueness.
}




\frame[shrink=2]{
\frametitle{GHZM state of three spin one half quanta}
\begin{itemize}
\item<+->
For measuring entangled particles at different contexts, the uniqueness property
must hold for {\em every} such context.
This is not true for the Greenberger, Horne and Zeilinger three spin one half quanta state
in the form proposed by Mermin
$$
\vert \Psi_{GHZM} \rangle
= {1\over \sqrt{2}}(
\vert z+ z+ z+\rangle
+
\vert z- z- z-\rangle
).$$
Here $z\pm$ stands for the outcome $\pm$ of a spin measurement  measured along the $z$-axis.

\item<+->
$\vert \Psi_{GHZM}\rangle $
satisfies the uniqueness property only along a single direction, the $z$-axis, of spin state measurements;
otherwise $\vert \Psi_{GHZM}\rangle $ contains eight summands.
\end{itemize}
}

\frame[shrink=2]{
\frametitle{Open issues}
\begin{itemize}
\item<+->
Do there exist nontrivial states  which
satisfy the uniqueness property in ``sufficiently many'' spin state measurement directions
to make them useful for counterfactual reasoning?

\item<+-> Protection schemes:
The lack of uniqueness  may be the reason why inconsistencies such as the ones derived in the
Kochen-Specker type proofs cannot be operationalized by ``explosion views.''
\end{itemize}
}

\subsection{Experimental realization}
\frame[shrink=2]{
\frametitle{Two contexts in three dimensions}
Arrangement
of five observables $A,B,C,D,K$ with two contexts
$\{A,B,C\}$
and
$\{D,K,A\}$ interconnected at link observable $A$.
This propositional structure can be represented in three dimensional Hilbert space
by two tripods with a single common leg.

\begin{figure}
\begin{center}
\begin{tabular}{ccccc}
%TexCad Options
%\grade{\off}
%\emlines{\off}
%\beziermacro{\on}
%\reduce{\on}
%\snapping{\off}
%\quality{2.00}
%\graddiff{0.01}
%\snapasp{1}
%\zoom{1.00}
\unitlength 0.60mm
\linethickness{0.4pt}
\begin{picture}(40.00,49.67)
%\put(60.33,15.00){\circle{0.00}}
%\put(60.33,15.00){\circle{2.00}}
%\put(45.33,10.00){\circle{2.00}}
%\put(30.33,5.00){\circle{2.00}}
%\put(15.33,10.00){\circle{2.00}}
%\put(0.33,16.00){\circle{0.00}}
%\put(0.33,15.00){\circle{2.00}}
\put(15.00,45.00){\line(0,-1){30.00}}
\put(15.00,15.00){\line(-1,-1){15.00}}
\put(15.00,15.00){\line(1,0){25.00}}
\put(15.00,15.00){\line(3,-4){11.00}}
\put(15.00,15.00){\line(5,3){16.67}}
\put(3.33,-1.67){\makebox(0,0)[cc]{$B$}}
\put(30.00,0.00){\makebox(0,0)[cc]{$D$}}
\put(40.00,11.33){\makebox(0,0)[cc]{$C$}}
\put(35.00,23.67){\makebox(0,0)[cc]{$K$}}
\put(19.33,49.67){\makebox(0,0)[cc]{$A$}}
%\bezvec{60}(7.67,6.67)(14.33,2.67)(20.00,6.67)
\put(20.00,6.67){\vector(2,1){0.2}}
\bezier{60}(7.67,6.67)(14.33,2.67)(20.00,6.67)
%\end
%\bezvec{36}(29.67,16.00)(32.33,19.67)(30.00,23.00)
\put(30.00,23.00){\vector(-1,2){0.2}}
\bezier{36}(29.67,16.00)(32.33,19.67)(30.00,23.00)
%\end
\put(13.33,1.00){\makebox(0,0)[cc]{$\varphi$}}
\put(40.00,18.67){\makebox(0,0)[rc]{$\varphi$}}
\end{picture}
&&
%TexCad Options
%\grade{\off}
%\emlines{\off}
%\beziermacro{\on}
%\reduce{\on}
%\snapping{\off}
%\quality{2.00}
%\graddiff{0.01}
%\snapasp{1}
%\zoom{1.00}
\unitlength 0.60mm
\linethickness{0.4pt}
\begin{picture}(61.33,36.00)
%\emline(0.33,35.00)(30.33,25.00)
\multiput(0.33,35.00)(0.36,-0.12){84}{\line(1,0){0.36}}
%\end
%\emline(30.33,25.00)(60.33,35.00)
\multiput(30.33,25.00)(0.36,0.12){84}{\line(1,0){0.36}}
%\end
%\put(60.33,15.00){\circle{0.00}}
%\put(60.33,15.00){\circle{2.00}}
%\put(45.33,10.00){\circle{2.00}}
%\put(30.33,5.00){\circle{2.00}}
%\put(15.33,10.00){\circle{2.00}}
%\put(0.33,16.00){\circle{0.00}}
%\put(0.33,15.00){\circle{2.00}}
\put(30.33,25.00){\circle{2.00}}
\put(45.33,30.00){\circle{2.00}}
\put(60.33,35.00){\circle{2.00}}
\put(0.33,35.00){\circle{2.00}}
\put(15.33,30.00){\circle{2.00}}
\put(60.33,31.00){\makebox(0,0)[cc]{$K$}}
\put(45.33,26.00){\makebox(0,0)[cc]{$D$}}
\put(30.33,30.00){\makebox(0,0)[cc]{$A$}}
\put(15.33,26.00){\makebox(0,0)[cc]{$C$}}
\put(0.33,31.00){\makebox(0,0)[cc]{$B$}}
\bezier{24}(0.00,20.00)(0.00,17.33)(3.00,17.33)
\bezier{28}(3.00,17.33)(10.00,17.00)(10.00,17.00)
\bezier{32}(10.00,17.00)(15.00,16.00)(15.00,13.33)
\bezier{24}(30.00,20.00)(30.00,17.33)(27.00,17.33)
\bezier{28}(27.00,17.33)(20.00,17.00)(20.00,17.00)
\bezier{32}(20.00,17.00)(15.00,16.00)(15.00,13.33)
%\put(15.00,10.33){\makebox(0,0)[cc]{context}}
\put(15.00,5.33){\makebox(0,0)[cc]{$\{B,C,A\}$}}
\bezier{24}(60.00,20.00)(60.00,17.33)(57.00,17.33)
\bezier{28}(57.00,17.33)(50.00,17.00)(50.00,17.00)
\bezier{32}(50.00,17.00)(45.00,16.00)(45.00,13.33)
\bezier{24}(30.00,20.00)(30.00,17.33)(33.00,17.33)
\bezier{28}(33.00,17.33)(40.00,17.00)(40.00,17.00)
\bezier{32}(40.00,17.00)(45.00,16.00)(45.00,13.33)
%\put(45.00,10.33){\makebox(0,0)[cc]{context}}
\put(45.00,5.33){\makebox(0,0)[cc]{$\{A,D,K\}$}}
\end{picture}
&&
%TexCad Options
%\grade{\off}
%\emlines{\off}
%\beziermacro{\on}
%\reduce{\on}
%\snapping{\off}
%\quality{2.00}
%\graddiff{0.01}
%\snapasp{1}
%\zoom{1.00}
\unitlength 0.60mm
\linethickness{0.4pt}
\begin{picture}(51.37,10.00)
\put(0.00,25.00){\circle{2.75}}
\put(30.00,25.00){\circle{2.75}}
\put(1.33,25.00){\line(1,0){27.33}}
\put(15.00,30.00){\makebox(0,0)[cc]{$A$}}
%\put(30.00,20.00){\makebox(0,0)[cc]{context}}
\put(30.00,15.00){\makebox(0,0)[cc]{$\{A,D,K\}$}}
%\put(0.00,20.00){\makebox(0,0)[cc]{context}}
\put(0.00,15.00){\makebox(0,0)[cc]{$\{B,C,A\}$}}
\end{picture}
\end{tabular}
\end{center}
\end{figure}
}

\frame[shrink=2]{
\frametitle{Two contexts in three dimensions (contd)}

The operators  $B,C,A$ and $D,K,A$ can be identified with the projectors corresponding
to the two bases
\begin{equation*}
\begin{array}{lcl}
B_{B-C-A}&=&
\{
(1,0,0)^T,
(0,1,0)^T,
(0,0,1)^T
\}
,
\\
B_{D-K-A}&=&
\{
(\cos \varphi , \sin \varphi ,0)^T,
(-\sin \varphi ,\cos \varphi , 0)^T,
(0,0,1)^T
\},
\end{array}
\end{equation*}
The associated contexts can be represented by the sum of the dyadic products of these vectors:
for $e_i\neq e_j \in {\Bbb R}$ and  $e_i'\neq e_j' \in {\Bbb R}$ for $i\neq j$,
\begin{equation*}
\begin{array}{lcl}
C_{B-C-A}&=&
\left(
\begin{array}{ccc}
e_1&0&0\\
0&e_2&0\\
0&0&e_3\\
\end{array}
\right)
,
\\
C_{D-K-A}
&=&
\left(
\begin{array}{ccc}
e_1' \cos^2 \varphi + e_2'\sin^2 \varphi&(e_1'-e_2')\sin \varphi \cos \varphi &0\\
(e_1'-e_2')\sin \varphi \cos \varphi &e_2' \cos^2 \varphi + e_1'\sin^2 \varphi &0\\
0&0&e_3'\\
\end{array}
\right).
\end{array}
\label{e-vaxjo2}
\end{equation*}
}



\frame[shrink=2]{
\frametitle{Experimental realization}
In the indirect, counterfactual sense, contextuality becomes measurable.
From an experimental point of view, this amounts to performing two tasks:
\begin{itemize}
\item<+->
First, in the preparation stage,
a singlet state of two spin one quanta must be realized.
This has become feasible recently by
engineering entangled
states in any arbitrary dimensional Hilbert space.
\item<+->
Second, in the analyzing stage,
the context structure  $\{B,C,A\}$ and $\{D,K,A\}$ interlinked at $A$ must be realized.
\end{itemize}
}



\frame[shrink=2]{
\frametitle{Three contexts in three dimensions}
Consider
$\{A,B,C\}$,
$\{A,D,K\}$
and
$\{K,L,M\}$ interconnected at $A$ and $K$.

The ``explosion view'' type of setup
encounters the problem of nonuniqueness:
for the three quantum singlet state
$
\vert \Psi_3 \rangle
$
the uniqueness property does not hold.


\begin{center}
$\qquad \qquad$
%TexCad Options
%\grade{\off}
%\emlines{\off}
%\beziermacro{\on}
%\reduce{\on}
%\snapping{\off}
%\quality{2.00}
%\graddiff{0.01}
%\snapasp{1}
%\zoom{1.00}
\unitlength 0.70mm
\linethickness{0.4pt}
\begin{picture}(91.34,36.00)
%\emline(0.33,35.00)(30.33,25.00)
\multiput(0.33,35.00)(0.36,-0.12){84}{\line(1,0){0.36}}
%\end
%\put(60.33,15.00){\circle{0.00}}
%\put(60.33,15.00){\circle{2.00}}
%\put(45.33,10.00){\circle{2.00}}
%\put(30.33,5.00){\circle{2.00}}
%\put(15.33,10.00){\circle{2.00}}
%\put(0.33,16.00){\circle{0.00}}
%\put(0.33,15.00){\circle{2.00}}
\put(30.33,25.00){\circle{2.00}}
\put(45.33,25.00){\circle{2.00}}
\put(0.33,35.00){\circle{2.00}}
\put(15.33,30.00){\circle{2.00}}
%\put(45.33,21.00){\makebox(0,0)[cc]{$x_2$}}
%\put(30.33,33.00){\makebox(0,0)[cc]{$x_1=x_1''$}}
%\put(15.33,26.00){\makebox(0,0)[cc]{$x_2''$}}
%\put(0.33,31.00){\makebox(0,0)[cc]{$x_3''$}}
\bezier{24}(0.00,20.00)(0.00,17.33)(3.00,17.33)
\bezier{28}(3.00,17.33)(10.00,17.00)(10.00,17.00)
\bezier{32}(10.00,17.00)(15.00,16.00)(15.00,13.33)
\bezier{24}(30.00,20.00)(30.00,17.33)(27.00,17.33)
\bezier{28}(27.00,17.33)(20.00,17.00)(20.00,17.00)
\bezier{32}(20.00,17.00)(15.00,16.00)(15.00,13.33)
%\put(15.00,10.33){\makebox(0,0)[cc]{context}}
\put(15.00,5.33){\makebox(0,0)[cc]{$\{B,C,A\}$}}
\bezier{24}(60.67,20.00)(60.67,17.33)(57.67,17.33)
%\bezier{24}(60.00,20.00)(60.00,17.33)(57.00,17.33)
\bezier{28}(57.00,17.33)(50.00,17.00)(50.00,17.00)
\bezier{32}(50.00,17.00)(45.00,16.00)(45.00,13.33)
\bezier{24}(30.00,20.00)(30.00,17.33)(33.00,17.33)
\bezier{28}(33.00,17.33)(40.00,17.00)(40.00,17.00)
\bezier{32}(40.00,17.00)(45.00,16.00)(45.00,13.33)
%\put(45.00,10.33){\makebox(0,0)[cc]{context}}
\put(45.00,5.33){\makebox(0,0)[cc]{$\{A,D,K\}$}}
\put(30.33,25.00){\line(1,0){30.00}}
%\emline(90.34,35.00)(60.34,25.00)
\multiput(90.34,35.00)(-0.36,-0.12){84}{\line(-1,0){0.36}}
%\end
\put(60.34,25.00){\circle{2.00}}
\put(90.34,35.00){\circle{2.00}}
\put(75.34,30.00){\circle{2.00}}
%\put(60.34,33.00){\makebox(0,0)[cc]{$x_3=x_3'$}}
%\put(75.34,26.00){\makebox(0,0)[cc]{$x_2'$}}
%\put(90.34,31.00){\makebox(0,0)[cc]{$x_1'$}}
\bezier{24}(90.67,20.00)(90.67,17.33)(87.67,17.33)
\bezier{28}(87.67,17.33)(80.67,17.00)(80.67,17.00)
\bezier{32}(80.67,17.00)(75.67,16.00)(75.67,13.33)
\bezier{24}(60.67,20.00)(60.67,17.33)(63.67,17.33)
\bezier{28}(63.67,17.33)(70.67,17.00)(70.67,17.00)
\bezier{32}(70.67,17.00)(75.67,16.00)(75.67,13.33)
%\put(75.67,10.33){\makebox(0,0)[cc]{context}}
\put(75.67,5.33){\makebox(0,0)[cc]{$\{K,L,M\}$}}
\end{picture}
\end{center}
}

\frame[shrink=2]{
\frametitle{More elaborate contexts in three dimensions}

Nonseparating set of two-valued probability measures:
For all two valued probability measures $P(x)\in\{0,1\}$, $P(a)=P(b)=1$,
there is no probability measure separating $a$ from $b$ through
$P(a)\neq P(b)$.

\begin{figure}
\begin{center}
%TexCad Options
%\grade{\on}
%\emlines{\off}
%\beziermacro{\off}
%\reduce{\on}
%\snapping{\off}
%\quality{2.00}
%\graddiff{0.01}
%\snapasp{1}
%\zoom{0.50}
\unitlength 0.50mm
\linethickness{0.4pt}
\begin{picture}(190.67,109.67)
%\emline(165.67,19.67)(145.67,39.67)
\multiput(165.67,19.67)(-0.12,0.12){167}{\line(0,1){0.12}}
%\end
%\emline(145.67,39.67)(145.67,79.67)
\put(145.67,39.67){\line(0,1){40.00}}
%\end
%\emline(145.67,79.67)(165.67,99.67)
\multiput(145.67,79.67)(0.12,0.12){167}{\line(0,1){0.12}}
%\end
%\emline(165.67,99.67)(185.67,79.67)
\multiput(165.67,99.67)(0.12,-0.12){167}{\line(1,0){0.12}}
%\end
%\emline(185.67,79.67)(185.67,39.67)
\put(185.67,79.67){\line(0,-1){40.00}}
%\end
%\emline(185.67,39.67)(165.67,19.67)
\multiput(185.67,39.67)(-0.12,-0.12){167}{\line(-1,0){0.12}}
%\end
%\emline(185.34,59.67)(145.67,59.67)
\put(185.34,59.67){\line(-1,0){39.67}}
%\end
\put(185.67,39.67){\circle{2.00}}
\put(185.67,59.67){\circle{2.00}}
\put(185.67,79.67){\circle{2.00}}
\put(145.67,39.67){\circle{2.00}}
\put(145.67,59.67){\circle{2.00}}
\put(145.67,79.67){\circle{2.00}}
\put(165.67,19.67){\circle{2.00}}
\put(165.67,99.67){\circle{2.00}}
%\emline(165.67,19.67)(95.67,59.67)
\multiput(165.67,19.67)(-0.21,0.12){334}{\line(-1,0){0.21}}
%\end
%\emline(95.67,59.67)(165.67,99.67)
\multiput(95.67,59.67)(0.21,0.12){334}{\line(1,0){0.21}}
%\end
\put(95.67,59.67){\circle{2.00}}
\put(95.67,74.67){\makebox(0,0)[cc]{$a_8=a_8'$}}
\put(165.67,109.67){\makebox(0,0)[cc]{$b=a_9=a_0'$}}
\put(140.67,79.67){\makebox(0,0)[cc]{$a_2'$}}
\put(140.67,59.67){\makebox(0,0)[cc]{$a_6'$}}
\put(140.67,39.67){\makebox(0,0)[cc]{$a_4'$}}
\put(190.67,39.67){\makebox(0,0)[cc]{$a_3'$}}
\put(190.67,59.67){\makebox(0,0)[cc]{$a_5'$}}
\put(190.67,80.00){\makebox(0,0)[cc]{$a_1'$}}
\put(165.67,9.67){\makebox(0,0)[cc]{$a_7'$}}
%\emline(25.00,19.67)(45.00,39.67)
\multiput(25.00,19.67)(0.12,0.12){167}{\line(0,1){0.12}}
%\end
%\emline(45.00,39.67)(45.00,79.67)
\put(45.00,39.67){\line(0,1){40.00}}
%\end
%\emline(45.00,79.67)(25.00,99.67)
\multiput(45.00,79.67)(-0.12,0.12){167}{\line(0,1){0.12}}
%\end
%\emline(25.00,99.67)(5.00,79.67)
\multiput(25.00,99.67)(-0.12,-0.12){167}{\line(-1,0){0.12}}
%\end
%\emline(5.00,79.67)(5.00,39.67)
\put(5.00,79.67){\line(0,-1){40.00}}
%\end
%\emline(5.00,39.67)(25.00,19.67)
\multiput(5.00,39.67)(0.12,-0.12){167}{\line(1,0){0.12}}
%\end
%\emline(5.33,59.67)(45.00,59.67)
\put(5.33,59.67){\line(1,0){39.67}}
%\end
\put(5.00,39.67){\circle{2.00}}
\put(5.00,59.67){\circle{2.00}}
\put(5.00,79.67){\circle{2.00}}
\put(45.00,39.67){\circle{2.00}}
\put(45.00,59.67){\circle{2.00}}
\put(45.00,79.67){\circle{2.00}}
\put(25.00,19.67){\circle{2.00}}
\put(25.00,99.67){\circle{2.00}}
%\emline(25.00,19.67)(95.00,59.67)
\multiput(25.00,19.67)(0.21,0.12){334}{\line(1,0){0.21}}
%\end
%\emline(95.00,59.67)(25.00,99.67)
\multiput(95.00,59.67)(-0.21,0.12){334}{\line(-1,0){0.21}}
%\end
\put(25.00,109.67){\makebox(0,0)[cc]{$a=a_0=a_9'$}}
\put(50.00,79.67){\makebox(0,0)[cc]{$a_2$}}
\put(50.00,59.67){\makebox(0,0)[cc]{$a_6$}}
\put(50.00,39.67){\makebox(0,0)[cc]{$a_4$}}
\put(-0.00,39.67){\makebox(0,0)[cc]{$a_3$}}
\put(-0.00,59.67){\makebox(0,0)[cc]{$a_5$}}
\put(-0.00,80.00){\makebox(0,0)[cc]{$a_1$}}
\put(25.00,9.67){\makebox(0,0)[cc]{$a_7$}}
\end{picture}
\end{center}
\end{figure}


An explosion requires 16 contexts and
could in principle be realized with some singlet state of 16 spin-one quanta.
Such a state  contains by far too many terms to satisfy the uniqueness property.
}


\subsection{Measurement of any arbitrary operator}
\frame[shrink=5]{
\frametitle{Measurement of any arbitrary operator}

Any matrix $A$ can be decomposed into two
self-adjoint components
$A_1, A_2$ as follows.
\begin{eqnarray}
&&A=A_1+iA_2\label{e-decom1}\nonumber \\
&&A_1={1\over 2}(A+A^\dagger) =:\Re A,\nonumber \\
&&A_2=-{i\over 2}(A-A^\dagger)=:\Im A.\nonumber
\end{eqnarray}
}


\section{Principles of context translation}
\frame[shrink=5]{
\frametitle{Principles of context translation}

\begin{itemize}
\item<+->
It is possible to encode into quanta
a certain finite amount of information by preparing them in a single context.
This amount is determined by the dimension of the associated Hilbert space.

\item<+->
(Counterfactual) Elements of physical reality which go beyond that single context
do not exist.

\item<+->
If quanta are measured within a different context,
that context may be translated by the measurement apparatus
into the context the quanta have been originally prepared for.
The capability of the measurement apparatus to translate the context
may depend on certain parameters, such as temperature.
\end{itemize}
}

\frame[shrink=2]{
\frametitle{Analogies $\ldots$}

\begin{itemize}
\item<+->
No agent, deterministic or other,
can be prepared to render answers to every conceivable question.
\item<+->
A ``silly'' example for this feature would be the attempt of a person
to enter the simple question
{\em ``is there enough oil in the car's engine?''} at
the command prompt of a desktop computer.
\item<+->
Yet nobody would come up with the suggestion that something strange,
bordering to the mysterious, or ``mindboggling'' is going on.
\end{itemize}
}


%\section*{Summary and open questions}
\frame[shrink=2]{
\frametitle{Summary and open questions}


\begin{itemize}
\item<+->
Measurements on the context (in)dependence
of two-context two quanta (in three dimensions per quantum) configurations
are feasible and need to be done.

\item<+->
It remains an open theoretical question whether or not
nonsinglet states exist which satisfy the noniqueness property for sufficiently many
different measurement ``directions'' or setups to allow for
``explosion views'' of more than two contexts.

\item<+->
The  issue of context translation remains
theoretically and experimentally unsettled.
Yet in principle context translation would be testable.
\end{itemize}
}

\frame[shrink=2]{
\begin{center}
Thank you for your attention!
\end{center}
}

%\bibliography{svozil}
%\bibliographystyle{apsrev}


\begin{thebibliography}{50}
\expandafter\ifx\csname natexlab\endcsname\relax\def\natexlab#1{#1}\fi
\expandafter\ifx\csname bibnamefont\endcsname\relax
  \def\bibnamefont#1{#1}\fi
\expandafter\ifx\csname bibfnamefont\endcsname\relax
  \def\bibfnamefont#1{#1}\fi
\expandafter\ifx\csname citenamefont\endcsname\relax
  \def\citenamefont#1{#1}\fi
\expandafter\ifx\csname url\endcsname\relax
  \def\url#1{\texttt{#1}}\fi
\expandafter\ifx\csname urlprefix\endcsname\relax\def\urlprefix{URL }\fi
\providecommand{\bibinfo}[2]{#2}
\providecommand{\eprint}[2][]{\url{#2}}

\bibitem[{\citenamefont{Clauser}(2000)}]{clauser-talkvie}
\bibinfo{author}{\bibfnamefont{J.}~\bibnamefont{Clauser}}
  (\bibinfo{year}{2000}), \bibinfo{note}{talk at `{Q}uantum [{U}n]Speakables,
  Conference in the commemmoration of {J}ohn {S}. {B}ell�, Nov. 10th,
  10.00-10.40}, \urlprefix\url{http://www.quantum.at/programs/Bell.htm}.

\bibitem[{\citenamefont{Svozil}(2002)}]{svozil-2002-noiq}
\bibinfo{author}{\bibfnamefont{K.}~\bibnamefont{Svozil}}
  (\bibinfo{year}{2002}), \eprint{quant-ph/0204168}.

\bibitem[{\citenamefont{Lakatos}(1978)}]{lakatosch}
\bibinfo{author}{\bibfnamefont{I.}~\bibnamefont{Lakatos}},
  \emph{\bibinfo{title}{Philosophical Papers. 1.~The Methodology of Scientific
  Research Programmes}} (\bibinfo{publisher}{Cambridge University Press},
  \bibinfo{address}{Cambridge}, \bibinfo{year}{1978}).

\bibitem[{\citenamefont{Svozil}(2004{\natexlab{a}})}]{svozil-2003-garda}
\bibinfo{author}{\bibfnamefont{K.}~\bibnamefont{Svozil}},
  \bibinfo{journal}{Journal of Modern Optics} \textbf{\bibinfo{volume}{51}},
  \bibinfo{pages}{811} (\bibinfo{year}{2004}{\natexlab{a}}),
  \eprint{quant-ph/0308110}.

\bibitem[{\citenamefont{Specker}(1960)}]{specker-60}
\bibinfo{author}{\bibfnamefont{E.}~\bibnamefont{Specker}},
  \bibinfo{journal}{Dialectica} \textbf{\bibinfo{volume}{14}},
  \bibinfo{pages}{175} (\bibinfo{year}{1960}), \bibinfo{note}{reprinted in
  \cite[pp. 175--182]{specker-ges}; {E}nglish translation: {\it The logic of
  propositions which are not simultaneously decidable}, reprinted in \cite[pp.
  135-140]{hooker}}.

\bibitem[{\citenamefont{Kamber}(1964)}]{kamber64}
\bibinfo{author}{\bibfnamefont{F.}~\bibnamefont{Kamber}},
  \bibinfo{journal}{Nachr. Akad. Wiss. G{\"{o}}ttingen}
  \textbf{\bibinfo{volume}{10}}, \bibinfo{pages}{103} (\bibinfo{year}{1964}).

\bibitem[{\citenamefont{Kamber}(1965)}]{kamber65}
\bibinfo{author}{\bibfnamefont{F.}~\bibnamefont{Kamber}},
  \bibinfo{journal}{Mathematische Annalen} \textbf{\bibinfo{volume}{158}},
  \bibinfo{pages}{158} (\bibinfo{year}{1965}).

\bibitem[{\citenamefont{Zierler and Schlessinger}(1965)}]{ZirlSchl-65}
\bibinfo{author}{\bibfnamefont{N.}~\bibnamefont{Zierler}} \bibnamefont{and}
  \bibinfo{author}{\bibfnamefont{M.}~\bibnamefont{Schlessinger}},
  \bibinfo{journal}{Duke Mathematical Journal} \textbf{\bibinfo{volume}{32}},
  \bibinfo{pages}{251} (\bibinfo{year}{1965}).

\bibitem[{\citenamefont{Bell}(1966)}]{bell-66}
\bibinfo{author}{\bibfnamefont{J.~S.} \bibnamefont{Bell}},
  \bibinfo{journal}{Reviews of Modern Physics} \textbf{\bibinfo{volume}{38}},
  \bibinfo{pages}{447} (\bibinfo{year}{1966}), \bibinfo{note}{reprinted in
  \cite[pp. 1-13]{bell-87}},
  \urlprefix\url{http://dx.doi.org/10.1103/RevModPhys.38.447}.

\bibitem[{\citenamefont{Kochen and Specker}(1967)}]{kochen1}
\bibinfo{author}{\bibfnamefont{S.}~\bibnamefont{Kochen}} \bibnamefont{and}
  \bibinfo{author}{\bibfnamefont{E.~P.} \bibnamefont{Specker}},
  \bibinfo{journal}{Journal of Mathematics and Mechanics}
  \textbf{\bibinfo{volume}{17}}, \bibinfo{pages}{59} (\bibinfo{year}{1967}),
  \bibinfo{note}{reprinted in \cite[pp. 235--263]{specker-ges}}.

\bibitem[{\citenamefont{Alda}(1980)}]{Alda}
\bibinfo{author}{\bibfnamefont{V.}~\bibnamefont{Alda}},
  \bibinfo{journal}{Aplik. mate.} \textbf{\bibinfo{volume}{25}},
  \bibinfo{pages}{373} (\bibinfo{year}{1980}).

\bibitem[{\citenamefont{Alda}(1981)}]{Alda2}
\bibinfo{author}{\bibfnamefont{V.}~\bibnamefont{Alda}},
  \bibinfo{journal}{Aplik. mate.} \textbf{\bibinfo{volume}{26}},
  \bibinfo{pages}{57} (\bibinfo{year}{1981}).

\bibitem[{\citenamefont{Peres}(1993)}]{peres}
\bibinfo{author}{\bibfnamefont{A.}~\bibnamefont{Peres}},
  \emph{\bibinfo{title}{Quantum Theory: Concepts and Methods}}
  (\bibinfo{publisher}{Kluwer Academic Publishers},
  \bibinfo{address}{Dordrecht}, \bibinfo{year}{1993}).

\bibitem[{\citenamefont{Mermin}(1993)}]{mermin-93}
\bibinfo{author}{\bibfnamefont{N.~D.} \bibnamefont{Mermin}},
  \bibinfo{journal}{Reviews of Modern Physics} \textbf{\bibinfo{volume}{65}},
  \bibinfo{pages}{803} (\bibinfo{year}{1993}),
  \urlprefix\url{http://dx.doi.org/10.1103/RevModPhys.65.803}.

\bibitem[{\citenamefont{Svozil and Tkadlec}(1996)}]{svozil-tkadlec}
\bibinfo{author}{\bibfnamefont{K.}~\bibnamefont{Svozil}} \bibnamefont{and}
  \bibinfo{author}{\bibfnamefont{J.}~\bibnamefont{Tkadlec}},
  \bibinfo{journal}{Journal of Mathematical Physics}
  \textbf{\bibinfo{volume}{37}}, \bibinfo{pages}{5380} (\bibinfo{year}{1996}),
  \urlprefix\url{http://dx.doi.org/10.1063/1.531710}.

\bibitem[{\citenamefont{Tkadlec}(2000)}]{tkadlec-00}
\bibinfo{author}{\bibfnamefont{J.}~\bibnamefont{Tkadlec}},
  \bibinfo{journal}{International Journal of Theoretical Physics}
  \textbf{\bibinfo{volume}{39}}, \bibinfo{pages}{921} (\bibinfo{year}{2000}),
  \urlprefix\url{http://dx.doi.org/10.1023/A:1003695317353}.

\bibitem[{\citenamefont{Peres}(1978)}]{peres222}
\bibinfo{author}{\bibfnamefont{A.}~\bibnamefont{Peres}},
  \bibinfo{journal}{American Journal of Physics} \textbf{\bibinfo{volume}{46}},
  \bibinfo{pages}{745} (\bibinfo{year}{1978}).

\bibitem[{\citenamefont{Krenn and A.Zeilinger}(1996)}]{krenn1}
\bibinfo{author}{\bibfnamefont{G.}~\bibnamefont{Krenn}} \bibnamefont{and}
  \bibinfo{author}{\bibnamefont{A.Zeilinger}}, \bibinfo{journal}{Physical
  Review A} \textbf{\bibinfo{volume}{54}}, \bibinfo{pages}{1793}
  (\bibinfo{year}{1996}),
  \urlprefix\url{http://link.aps.org/abstract/PRA/v54/p1793}.

\bibitem[{\citenamefont{Pitowsky}(1982)}]{pitowsky-82}
\bibinfo{author}{\bibfnamefont{I.}~\bibnamefont{Pitowsky}},
  \bibinfo{journal}{Physical Review Letters} \textbf{\bibinfo{volume}{48}},
  \bibinfo{pages}{1299} (\bibinfo{year}{1982}), \bibinfo{note}{cf. N. D.
  Mermin, {\sl Physical Review Letters} {\bf 49}, 1214 (1982); A. L. Macdonald,
  {\sl Physical Review Letters} {\bf 49}, 1215 (1982); Itamar Pitowsky, {\sl
  Physical Review Letters} {\bf 49}, 1216 (1982)},
  \urlprefix\url{http://dx.doi.org/10.1103/PhysRevLett.48.1299}.

\bibitem[{\citenamefont{Pitowsky}(1983)}]{pitowsky-83}
\bibinfo{author}{\bibfnamefont{I.}~\bibnamefont{Pitowsky}},
  \bibinfo{journal}{Physical Review D} \textbf{\bibinfo{volume}{27}},
  \bibinfo{pages}{2316} (\bibinfo{year}{1983}),
  \urlprefix\url{http://dx.doi.org/10.1103/PhysRevD.27.2316}.

\bibitem[{\citenamefont{Pitowsky}(1986)}]{pitowsky-86}
\bibinfo{author}{\bibfnamefont{I.}~\bibnamefont{Pitowsky}},
  \bibinfo{journal}{J. Math. Phys.} \textbf{\bibinfo{volume}{27}},
  \bibinfo{pages}{1556} (\bibinfo{year}{1986}).

\bibitem[{\citenamefont{Meyer}(1999)}]{meyer:99}
\bibinfo{author}{\bibfnamefont{D.~A.} \bibnamefont{Meyer}},
  \bibinfo{journal}{Physical Review Letters} \textbf{\bibinfo{volume}{83}},
  \bibinfo{pages}{3751} (\bibinfo{year}{1999}), \eprint{quant-ph/9905080},
  \urlprefix\url{http://dx.doi.org/10.1103/PhysRevLett.83.3751}.

\bibitem[{\citenamefont{Havlicek et~al.}(2001)\citenamefont{Havlicek, Krenn,
  Summhammer, and Svozil}}]{havlicek-2000}
\bibinfo{author}{\bibfnamefont{H.}~\bibnamefont{Havlicek}},
  \bibinfo{author}{\bibfnamefont{G.}~\bibnamefont{Krenn}},
  \bibinfo{author}{\bibfnamefont{J.}~\bibnamefont{Summhammer}},
  \bibnamefont{and} \bibinfo{author}{\bibfnamefont{K.}~\bibnamefont{Svozil}},
  \bibinfo{journal}{J. Phys. A: Math. Gen.} \textbf{\bibinfo{volume}{34}},
  \bibinfo{pages}{3071} (\bibinfo{year}{2001}),
  \urlprefix\url{http://dx.doi.org/10.1088/0305-4470/34/14/312}.

\bibitem[{\citenamefont{Bohr}(1949)}]{bohr-1949}
\bibinfo{author}{\bibfnamefont{N.}~\bibnamefont{Bohr}}, in
  \emph{\bibinfo{booktitle}{{A}lbert {E}instein: Philosopher-Scientist}}
  (\bibinfo{publisher}{The Library of Living Philosophers},
  \bibinfo{address}{Evanston, Ill.}, \bibinfo{year}{1949}), pp.
  \bibinfo{pages}{200--241},
  \urlprefix\url{http://www.emr.hibu.no/lars/eng/schilpp/Default.html}.

\bibitem[{\citenamefont{Heywood and Redhead}(1983)}]{hey-red}
\bibinfo{author}{\bibfnamefont{P.}~\bibnamefont{Heywood}} \bibnamefont{and}
  \bibinfo{author}{\bibfnamefont{M.~L.~G.} \bibnamefont{Redhead}},
  \bibinfo{journal}{Foundations of Physics} \textbf{\bibinfo{volume}{13}},
  \bibinfo{pages}{481} (\bibinfo{year}{1983}).

\bibitem[{\citenamefont{Redhead}(1990)}]{redhead}
\bibinfo{author}{\bibfnamefont{M.}~\bibnamefont{Redhead}},
  \emph{\bibinfo{title}{Incompleteness, Nonlocality, and Realism: A
  Prolegomenon to the Philosophy of Quantum Mechanics}}
  (\bibinfo{publisher}{Clarendon Press}, \bibinfo{address}{Oxford},
  \bibinfo{year}{1990}).

\bibitem[{\citenamefont{von Neumann}(1932)}]{v-neumann-49}
\bibinfo{author}{\bibfnamefont{J.}~\bibnamefont{von Neumann}},
  \emph{\bibinfo{title}{Mathematische Grundlagen der Quantenmechanik}}
  (\bibinfo{publisher}{Springer}, \bibinfo{address}{Berlin},
  \bibinfo{year}{1932}), \bibinfo{note}{{E}nglish translation: {\sl
  Mathematical Foundations of Quantum Mechanics}, Princeton University Press,
  Princeton, 1955}.

\bibitem[{\citenamefont{Halmos}(1974)}]{halmos-vs}
\bibinfo{author}{\bibfnamefont{P.~R.} \bibnamefont{Halmos}},
  \emph{\bibinfo{title}{Finite-dimensional vector spaces}}
  (\bibinfo{publisher}{Springer}, \bibinfo{address}{New York, Heidelberg,
  Berlin}, \bibinfo{year}{1974}).

\bibitem[{\citenamefont{Birkhoff and von Neumann}(1936)}]{birkhoff-36}
\bibinfo{author}{\bibfnamefont{G.}~\bibnamefont{Birkhoff}} \bibnamefont{and}
  \bibinfo{author}{\bibfnamefont{J.}~\bibnamefont{von Neumann}},
  \bibinfo{journal}{Annals of Mathematics} \textbf{\bibinfo{volume}{37}},
  \bibinfo{pages}{823} (\bibinfo{year}{1936}).

\bibitem[{\citenamefont{Greechie}(1971)}]{greechie:71}
\bibinfo{author}{\bibfnamefont{J.~R.} \bibnamefont{Greechie}},
  \bibinfo{journal}{Journal of Combinatorial Theory}
  \textbf{\bibinfo{volume}{10}}, \bibinfo{pages}{119} (\bibinfo{year}{1971}).

\bibitem[{\citenamefont{Pt{\'{a}}k and Pulmannov{\'{a}}}(1991)}]{pulmannova-91}
\bibinfo{author}{\bibfnamefont{P.}~\bibnamefont{Pt{\'{a}}k}} \bibnamefont{and}
  \bibinfo{author}{\bibfnamefont{S.}~\bibnamefont{Pulmannov{\'{a}}}},
  \emph{\bibinfo{title}{Orthomodular Structures as Quantum Logics}}
  (\bibinfo{publisher}{Kluwer Academic Publishers},
  \bibinfo{address}{Dordrecht}, \bibinfo{year}{1991}).

\bibitem[{\citenamefont{Svozil}(1998)}]{svozil-ql}
\bibinfo{author}{\bibfnamefont{K.}~\bibnamefont{Svozil}},
  \emph{\bibinfo{title}{Quantum Logic}} (\bibinfo{publisher}{Springer},
  \bibinfo{address}{Singapore}, \bibinfo{year}{1998}).

\bibitem[{\citenamefont{Kalmbach}(1983)}]{kalmbach-83}
\bibinfo{author}{\bibfnamefont{G.}~\bibnamefont{Kalmbach}},
  \emph{\bibinfo{title}{Orthomodular Lattices}} (\bibinfo{publisher}{Academic
  Press}, \bibinfo{address}{New York}, \bibinfo{year}{1983}).

\bibitem[{\citenamefont{Einstein et~al.}(1935)\citenamefont{Einstein, Podolsky,
  and Rosen}}]{epr}
\bibinfo{author}{\bibfnamefont{A.}~\bibnamefont{Einstein}},
  \bibinfo{author}{\bibfnamefont{B.}~\bibnamefont{Podolsky}}, \bibnamefont{and}
  \bibinfo{author}{\bibfnamefont{N.}~\bibnamefont{Rosen}},
  \bibinfo{journal}{Physical Review} \textbf{\bibinfo{volume}{47}},
  \bibinfo{pages}{777} (\bibinfo{year}{1935}),
  \urlprefix\url{http://dx.doi.org/10.1103/PhysRev.47.777}.

\bibitem[{\citenamefont{Kok et~al.}(2002)\citenamefont{Kok, Nemoto, and
  Munro}}]{kok-02}
\bibinfo{author}{\bibfnamefont{P.}~\bibnamefont{Kok}},
  \bibinfo{author}{\bibfnamefont{K.}~\bibnamefont{Nemoto}}, \bibnamefont{and}
  \bibinfo{author}{\bibfnamefont{W.~J.} \bibnamefont{Munro}}
  (\bibinfo{year}{2002}), \eprint{quant-ph/0201138}.

\bibitem[{\citenamefont{Mermin}(1990)}]{mermin1}
\bibinfo{author}{\bibfnamefont{N.~D.} \bibnamefont{Mermin}},
  \bibinfo{journal}{American Journal of Physics} \textbf{\bibinfo{volume}{58}},
  \bibinfo{pages}{731} (\bibinfo{year}{1990}).

\bibitem[{\citenamefont{Mair et~al.}(2001)\citenamefont{Mair, Vaziri, Weihs, ,
  and Zeilinger}}]{mvwz-2001}
\bibinfo{author}{\bibfnamefont{A.}~\bibnamefont{Mair}},
  \bibinfo{author}{\bibfnamefont{A.}~\bibnamefont{Vaziri}},
  \bibinfo{author}{\bibfnamefont{G.}~\bibnamefont{Weihs}}, , \bibnamefont{and}
  \bibinfo{author}{\bibfnamefont{A.}~\bibnamefont{Zeilinger}},
  \bibinfo{journal}{Nature} \textbf{\bibinfo{volume}{412}},
  \bibinfo{pages}{313} (\bibinfo{year}{2001}),
  \urlprefix\url{http://dx.doi.org/10.1038/35085529}.

\bibitem[{\citenamefont{Vaziri et~al.}(2002)\citenamefont{Vaziri, Weihs, , and
  Zeilinger}}]{vwz-2002}
\bibinfo{author}{\bibfnamefont{A.}~\bibnamefont{Vaziri}},
  \bibinfo{author}{\bibfnamefont{G.}~\bibnamefont{Weihs}}, , \bibnamefont{and}
  \bibinfo{author}{\bibfnamefont{A.}~\bibnamefont{Zeilinger}},
  \bibinfo{journal}{Physical Review Letters} \textbf{\bibinfo{volume}{89}},
  \bibinfo{pages}{240401} (\bibinfo{year}{2002}),
  \urlprefix\url{http://dx.doi.org/10.1103/PhysRevLett.89.240401}.

\bibitem[{\citenamefont{Riedmatten et~al.}(2002)\citenamefont{Riedmatten,
  Marcikic, Zbinden, and Gisin}}]{gisin-2002-d}
\bibinfo{author}{\bibfnamefont{H.~D.} \bibnamefont{Riedmatten}},
  \bibinfo{author}{\bibfnamefont{I.}~\bibnamefont{Marcikic}},
  \bibinfo{author}{\bibfnamefont{H.}~\bibnamefont{Zbinden}}, \bibnamefont{and}
  \bibinfo{author}{\bibfnamefont{N.}~\bibnamefont{Gisin}},
  \bibinfo{journal}{Quantum Information and Computing}
  \textbf{\bibinfo{volume}{2}}, \bibinfo{pages}{425} (\bibinfo{year}{2002}),
  \urlprefix\url{http://www.gap-optique.unige.ch/Publications/Pdf/QICfinale.pd%
f}.

\bibitem[{\citenamefont{Torres et~al.}(2003)\citenamefont{Torres, Deyanova,
  Torner, and Molina-Terriza}}]{tdtm-2003}
\bibinfo{author}{\bibfnamefont{J.~P.} \bibnamefont{Torres}},
  \bibinfo{author}{\bibfnamefont{Y.}~\bibnamefont{Deyanova}},
  \bibinfo{author}{\bibfnamefont{L.}~\bibnamefont{Torner}}, \bibnamefont{and}
  \bibinfo{author}{\bibfnamefont{G.}~\bibnamefont{Molina-Terriza}},
  \bibinfo{journal}{Physical Review A} \textbf{\bibinfo{volume}{67}},
  \bibinfo{pages}{052313} (\bibinfo{year}{2003}),
  \urlprefix\url{http://dx.doi.org/10.1103/PhysRevA.67.052313}.

\bibitem[{\citenamefont{Reck et~al.}(1994)\citenamefont{Reck, Zeilinger,
  Bernstein, and Bertani}}]{rzbb}
\bibinfo{author}{\bibfnamefont{M.}~\bibnamefont{Reck}},
  \bibinfo{author}{\bibfnamefont{A.}~\bibnamefont{Zeilinger}},
  \bibinfo{author}{\bibfnamefont{H.~J.} \bibnamefont{Bernstein}},
  \bibnamefont{and} \bibinfo{author}{\bibfnamefont{P.}~\bibnamefont{Bertani}},
  \bibinfo{journal}{Physical Review Letters} \textbf{\bibinfo{volume}{73}},
  \bibinfo{pages}{58} (\bibinfo{year}{1994}),
  \urlprefix\url{http://dx.doi.org/10.1103/PhysRevLett.73.58}.

\bibitem[{\citenamefont{Reck and Zeilinger}(1994)}]{reck-94}
\bibinfo{author}{\bibfnamefont{M.}~\bibnamefont{Reck}} \bibnamefont{and}
  \bibinfo{author}{\bibfnamefont{A.}~\bibnamefont{Zeilinger}}, in
  \emph{\bibinfo{booktitle}{Quantum Interferometry}}, edited by
  \bibinfo{editor}{\bibfnamefont{F.~D.} \bibnamefont{Martini}},
  \bibinfo{editor}{\bibfnamefont{G.}~\bibnamefont{Denardo}}, \bibnamefont{and}
  \bibinfo{editor}{\bibfnamefont{A.}~\bibnamefont{Zeilinger}}
  (\bibinfo{publisher}{World Scientific}, \bibinfo{address}{Singapore},
  \bibinfo{year}{1994}), pp. \bibinfo{pages}{170--177}.

\bibitem[{\citenamefont{Zukowski et~al.}(1997)\citenamefont{Zukowski,
  Zeilinger, and Horne}}]{zukowski-97}
\bibinfo{author}{\bibfnamefont{M.}~\bibnamefont{Zukowski}},
  \bibinfo{author}{\bibfnamefont{A.}~\bibnamefont{Zeilinger}},
  \bibnamefont{and} \bibinfo{author}{\bibfnamefont{M.~A.} \bibnamefont{Horne}},
  \bibinfo{journal}{Physical Review A} \textbf{\bibinfo{volume}{55}},
  \bibinfo{pages}{2564} (\bibinfo{year}{1997}),
  \urlprefix\url{http://dx.doi.org/10.1103/PhysRevA.55.2564}.

\bibitem[{\citenamefont{Svozil}(2004{\natexlab{b}})}]{svozil-2004-analog}
\bibinfo{author}{\bibfnamefont{K.}~\bibnamefont{Svozil}}
  (\bibinfo{year}{2004}{\natexlab{b}}), \eprint{quant-ph/0401113}.

\bibitem[{\citenamefont{Specker}(1990)}]{specker-ges}
\bibinfo{author}{\bibfnamefont{E.}~\bibnamefont{Specker}},
  \emph{\bibinfo{title}{Selecta}} (\bibinfo{publisher}{Birkh{\"{a}}user
  Verlag}, \bibinfo{address}{Basel}, \bibinfo{year}{1990}).

\bibitem[{\citenamefont{Hooker}(1975)}]{hooker}
\bibinfo{author}{\bibfnamefont{C.~A.} \bibnamefont{Hooker}}, in
  \emph{\bibinfo{booktitle}{The Logico-Algebraic Approach to Quantum Mechanics.
  Volume I: Historical Evolution}} (\bibinfo{publisher}{Reidel},
  \bibinfo{address}{Dordrecht}, \bibinfo{year}{1975}).

\bibitem[{\citenamefont{Bell}(1987)}]{bell-87}
\bibinfo{author}{\bibfnamefont{J.~S.} \bibnamefont{Bell}},
  \emph{\bibinfo{title}{Speakable and Unspeakable in Quantum Mechanics}}
  (\bibinfo{publisher}{Cambridge University Press},
  \bibinfo{address}{Cambridge}, \bibinfo{year}{1987}).

\bibitem[{\citenamefont{Feynman}(1965)}]{feynman-law}
\bibinfo{author}{\bibfnamefont{R.~P.} \bibnamefont{Feynman}},
  \emph{\bibinfo{title}{The Character of Physical Law}}
  (\bibinfo{publisher}{MIT Press}, \bibinfo{address}{Cambridge, MA},
  \bibinfo{year}{1965}).

\bibitem[{\citenamefont{Zeilinger}(1999)}]{zeil-99}
\bibinfo{author}{\bibfnamefont{A.}~\bibnamefont{Zeilinger}},
  \bibinfo{journal}{Foundations of Physics} \textbf{\bibinfo{volume}{29}},
  \bibinfo{pages}{631} (\bibinfo{year}{1999}).

\bibitem[{\citenamefont{Brukner and Zeilinger}(2003)}]{zeil-bruk-02}
\bibinfo{author}{\bibfnamefont{{\v{C}}.}~\bibnamefont{Brukner}}
  \bibnamefont{and}
  \bibinfo{author}{\bibfnamefont{A.}~\bibnamefont{Zeilinger}}, in
  \emph{\bibinfo{booktitle}{Time, Quantum, Information}}
  (\bibinfo{publisher}{Springer}, \bibinfo{year}{2003}),
  \eprint{quant-ph/0212084}.

\end{thebibliography}
\end{document}

