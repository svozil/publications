\documentclass[12pt]{article}
\usepackage[a4paper,left=3.5cm,right=2.5cm, includefoot]{geometry}


%{revtex4-1}
\usepackage[justification=centering]{caption}
\usepackage{floatrow}
\usepackage{etex}
%\usepackage[utf8]{inputenc}
%\usepackage[backend=biber, style=numeric]{biblatex}

%\addbibresource{anna.bib}

%%\RequirePackage[isolatin]{inputenc}
%\usepackage[T1]{fontenc}
%\usepackage{lmodern}
%%\usepackage[ngerman]{babel}
\usepackage{amsmath}
\usepackage{musixtex}
\usepackage{graphicx}
\usepackage{url}
\usepackage{tikz}

\usepackage{setspace}
\usetikzlibrary{decorations.fractals}

\usepackage[ngerman]{babel}
\usepackage[utf8x]{inputenc}
%\usepackage[utf8]{inputenc}

\usepackage[T1]{fontenc}
%\renewcommand{\lotname}{Tabellenverzeichnis}
%\renewcommand{\lofname}{Abbildungsverzeichnis}
%\renewcommand{\tocname}{Inhaltsangabe}

\usetikzlibrary{mindmap,trees}


%\bibliographystyle{ieeetr}

% Generates the index
%\usepackage{makeidx}
%\makeindex

%\renewcommand\thesection{\arabic{section}}
%\renewcommand\thesubsection{\thesection.\arabic{subsection}}
%\renewcommand\thesubsubsection{\thesection.\thesubsection.\arabic{subsubsection}}
%\renewcommand{\thesection}{\arabic{section}}
%\renewcommand{\thesubsection}{\thesection.\arabic{subsection}}
%\renewcommand{\thesubsubsection}{\thesubsection.\arabic{subsubsection}}
\begin{document}

\onehalfspacing
\begin{titlepage}
\begin{center}
  \bfseries
  \huge Algorithmisches Komponieren
  \vskip.2in
  \textsc{\LARGE Vorwissenschaftliche Arbeit}
  \vskip.2in
  \large Bundesgymnasium, Realgymnasium und Oberstufenrealgymnasium\\ Karajangasse 14 \\1200 Wien\\ 8H
  \vskip2in
\end{center}

\vskip1.2in

\begin{minipage}{.35\textwidth}
  \begin{flushleft}
    \bfseries\large Betreuungslehrer:\par Andreas Grimm
  \end{flushleft}
\end{minipage}
\hskip.4\textwidth
\begin{minipage}{.25\textwidth}
  \begin{flushleft}
    \bfseries\large Schüler:\par Anna Svozil
  \end{flushleft}
\end{minipage}
\vskip1.2in

\centering
\bfseries
\Large\today

\end{titlepage}

%\title{Algorithmisches Komponieren}

%\cdmtcsauthor{Karl Svozil}
%\cdmtcsaffiliation{Vienna University of Technology}
%\cdmtcstrnumber{407}
%\cdmtcsdate{September 2011}
%\coverpage

%\author{Anna Svozil}
%\emph{\\Bundesgymnasium, Realgymnasium und Oberstufenrealgymnasium\\ Karajangasse 14 \\1200 Wien\\ 8H Klasse}
%\affiliation{8H Klasse\\Bundesgymnasium, Realgymnasium und Oberstufenrealgymnasium\\ Karajangasse 14 \\1200 Wien\\ Betreuungslehrer: Prof. Mag. GRIMM Andreas\\ 14.Februar 2015}
%\email{anna.svozil@gmail.com} %\homepage[]{http://tph.tuwien.ac.at/~svozil}
%\emph{Betreuungslehrer:}{Prof. Mag. GRIMM Andreas}

%\maketitle
%\newpage
\section*{Abstract}
\addcontentsline{toc}{section}{Abstract}
Algorithmische Komposition setzt sich aus den Worten {\em Algorithmus} und {\em Komposition} zusammen.
Zu Beginn werden diese beiden Begriffe definiert und genauer erklärt.
In der Folge wird der Aufbau und die Entstehung einer algorithmischen Komposition detailliert beschrieben.
Die Beurteilung einer algorithmischen Komposition nach verschiedenen Kriterien wird diskutiert.
Ebenfalls behandelt werden Fragen im Zusammenhang mit dem Urheberrecht.
Um all diese Aspekte dem Leser zu verdeutlichen, und um diese beispielhaft zu demonstrieren
habe ich versucht, diverse algorithmische Kompositionstechniken konkret anzuwenden; das heißt ich habe kleine
Kompositionen algorithmisch nach den vorher beschriebenen Methoden erstellt.
Diese Methoden der Kompositionstechnik für algorithmische Musik beinhalten stochastische Algorithmen, fraktale Musik,
Markov-Ketten, wissensbasierte Systeme, genetische Algorithmen und musikalische Würfelspiele.
In dem  Themengebiet der fraktalen Musik gehe ich auf weißes, braunes und rosa (englisch {\em pink}es) Rauschen ein.
Diese jeweiligen Untergruppen werden mittels eines eigens durchgeführten Versuchs näher erklärt.
Im Themengebiet der genetischen Algorithmen und der musikalischen Würfelspiele beleuchte
ich ebenfalls den geschichtlichen Aspekt der Methode.
Alle Versuche und konkreten Beispiele sind so illustriert,
dass es dem Leser möglich sein sollte, die Versuche selbst und ohne weitere Literaturrecherchen nachzustellen.
Der Abschluss der Arbeit befasst sich mit meinen Ergebnissen,
sowie dem möglichen zukünftigen Nutzen der algorithmischen Kompositionstechnik in unserer Gesellschaft.
\newpage


%\maketitle

%\tableofcontents
%\addcontentsline{toc}{section}{Table of Contents}

\tableofcontents


\makeatletter
\let\toc@pre\relax
\let\toc@post\relax
\makeatother




\newpage
\section{Einleitung}
Meine Arbeit beschäftigt sich mit dem Thema der algorithmischen Komposition.
Dabei soll der Zusammenhang von Algorithmen und Musik möglichst allgemein dargestellt werden.
Ich möchte auch zeigen, dass die so entstehende ``mechanische''
Musik, welche durch algorithmische Kompositionstechniken erschaffen wurde,
zwar nicht immer gut klingen muss;
diese kann jedoch für die Zukunft ein interessanter Faktor der Musikproduktion darstellen.
In der Arbeit gehe ich wie folgt vor:
ich stelle die Art des Algorithmus vor und beschreibe den Zusammenhang mit der Musik.
Wenn es mir möglich ist einen Versuch darzustellen, führe ich einen solchen an,
um dem Leser einen besseren Einblick in die Thematik geben zu können.
Die Arbeit ist so aufgebaut, dass zu Beginn allgemein wichtige Begriffe geklärt werden.
Danach kommt der Hauptteil, in welchem ich die verschiedenen Algorithmen und die Techniken der algorithmischen Komposition
vorstelle.
Am Schluss erkläre ich, zu welchen Schlussfolgerungen ich gekommen bin,
und welche Folgen die algorithmischen Kompositionstechniken für uns als Menschen haben könnten.
Gleich vorweg möchte ich anführen, dass, was die verwendete Literatur betrifft, ich
die verschiedensten Quellen benutzt habe,
welche ich allesamt zitiere.
Insbesondere habe ich in dieser Arbeit auch einiges aus einer Diplomarbeit eines Absolventen der Musikuniversität
Wien~\cite{DiplomarbeitG},
sowie aus einer Diplomarbeit (die auf Englisch verfasst wurde) eines Absolventen der Technischen Universität Wien
entnommen~\cite{MagisterarbeitA}. All diese Zitate sind ebenfalls als solche dokumentiert.
Ich erhielt ebenfalls sehr viele Informationen und Denkanstöße durch meinen Bruder,
welcher Informatik an der Technischen Universität Wien studiert;
sowie durch meinen Vater, der als Professor für Theoretische Physik an der Technischen Universität Wien lehrt
und an einem Projekt der Europäischen Union über Zufall mitarbeitet.
Diese Arbeit wurde mit dem Textverarbeitungssystem \LaTeX\  verfasst.
\LaTeX\  ist ein von Wissenschaftlern selbst erarbeitetes Textverarbeitungssystem,
welches in den Formal- und Naturwissenschaften globale Verbreitung gefunden hat.
\LaTeX\ enthält einfache Formatierungsbefehle, welche die Verfassung eines strukturierten Textes mit methodisch korrekten
wissenschaftlichem Aufbau wesentlich erleichtern.
Außerdem hat \LaTeX, durch einige Zusatzmodule,
das Abbilden von Noten und Formatierungsarbeiten wesentlich einfacher gemacht hat.
Das Programm selbst zu bedienen ist mir anfangs nicht leicht gefallen;
jedoch konnte mir mein Bruder sehr oft helfen.

\section{Allgemeines}
In dem folgendem Kapitel werde ich auf die Begriffe {\em Algorithmus}
und {\em Komposition} eingehen, und auch {\em urheberrechtliche Fragestellungen} in diesem Zusammenhang
eingehen.

\subsection{Algorithmus}
%%\index{Algorithmus}
Das Wort Algorithmus kommt aus dem Arabischen und leitet sich von dem Namen des persischen Gelehrten al-Chwarismi ab.
Dieser lehrte am Hof eines Kalifen namens Al-Mamun.
Al-Chwarizmi  lebte von 787 bis 850. Er  führte die indische Ziffernschreibweise,
sowie  das dekadische Positionssystem in den arabischen Ländern ein.
Zudem veröffentlichte er ein  Lehrbuch in welchem Regeln (Handlungsvorschriften)
zum formalen Lösen von Gleichungen beschrieben sind.
Dies Handlungsvorschriften bekamen den abgeleiteten Namen des Autors~\cite{ALChwarizmi}.
Heute verstehen wir unter einem
Algorithmus eine endliche Abfolge von Befehlen.
Diese Befehle werden im Vorhinein von einem Programmierer, beziehungsweise in einer bestimmen Programmiersprache
wie {\em C++} oder {\em Java} vorausgesetzt und bestimmt.
Das Wichtige dabei ist, dass diese Befehle sehr präzise, klar zu interpretieren, und konsistent sind;
das heißt dass sich die Befehle nicht widersprechen.
Abfolgen solcher Befehle können sich wiederholen, solange der Befehl es auch zulässt.
Startet ein Algorithmus jedes Mal mit den gleichen Startbedingungen,
und endet derselbe auch immer mit dem gleichen Ergebnis, so spricht man von einem Algorithmus,
welcher {\em deterministisch} ist. Ein  deterministischer Algorithmus
hat durch eindeutigen Ablauf auch eindeutige Resultate.
%\index{deterministisch}
%\index{deterministischer Algorithmus}
Wenn es an einer Stelle des Algorithmus mehrere Möglichkeiten der Fortführung gibt,
und er nicht eindeutig determiniert ist,
so wird der Algorithmus als {\em nicht-deterministisch} bezeichnet.
%\index{nicht-deterministisch}
%\index{nicht-deterministischer Algorithmus}


\subsection{Komposition}
%\index{Komposition}
Das Wort {\em ``Komposition''} kommt aus dem lateinischen {\it composito} und heißt ``Zusammenstellung''.
In einer Komposition geht es einerseits um die {\em ``Schriftliche Festlegung von Entscheidungen''},
als auch um {\em ``wohllautende Ton-schriftliche Ausarbeitungen der Zusammensetzung
von Tönen im Sinne eines für sich bestehenden Werkes''}~\cite[S.~210]{eggebrecht1991musik}.
Eine Komposition ist in der Musik ein Zusammenstellung aus Noten, Takten, Pausen und
 anderen musikalischen Zeichen. Zusammen bilden diese eine Komposition.
Eine Komponistin hat die Macht über ihr Werk. Das heißt sie erschafft das Werk quasi ``aus sich heraus''.
Deshalb ist es vielen Komponisten wichtig, dass ihr Werk einen {\em wiedererkennbaren Wert} hat.
Wie bei einem Algorithmus ist die Komposition eine Reihe von Entscheidungen, welche insgesamt zu einem Ergebnis führen.
Als Komponistin hat man kaum Einschränkungen --
eine Komponistin kann als Erschaffer in ihrem Werk ihre eigenen Regeln bestimmen.
Mögliche Einschränkungen der Kompositionstechniken betreffen dennoch die {\em Rezeption}, die Notwendigkeit und
die Mühen bei der Entschlüsselung des Werkes
durch das Publikum; sowie die objektive Spielbarkeit desselben.
Wenngleich heutige Musikkonsumenten und Rezipienten -- zumindest im Vergleich mit früheren Musikliebhabern,
welche im Jahre 1913 sowohl in Wien als auch in Paris gegen neue Kompositionsstile
noch lauthals rebellierten -- einigermaßen tolerant auch offensichtlich schwer zu interpretierender Werke erscheinen,
sollte man wohl niemals vergessen, dass der subjektive Gesamteindruck einer Komposition
unter all zu großer Monotonie, als auch unter allzu großer Zufälligkeit des Klangbildes leidet.
Deshalb ist es wichtig, die Komposition ``mittig'' zwischen {\it Skylla} und {\it Charybdis},
also zwischen Monotonie und Chaos, anzusetzen.

\subsection{Algorithmische Komposition}
Der Begriff besteht aus den beiden Wörtern Algorithmus und Komposition.
Nun habe ich die beiden Begriffe oben schon erläutert; und ich möchte diese nun zusammen erklären.
Eine algorithmische Komposition besteht aus einen bestehenden{\em  Algorithmus, welcher eine Komposition erstellt.}
Dieser Algorithmus  hat exakt formulierte Anweisungen, die der Programmierer zuvor festgelegt hat,
um eine Komposition entstehen zu lassen.
Zur Erstellung einer algorithmischen Komposition braucht man ein gewisses Grundwissen
über Musik und ein relativ großes Wissen über programmieren und Algorithmen,
denn damit eine solche Komposition entstehen kann, muss sich der Programmierer sowohl
an die Regeln der Musik halten, als auch einen passenden Code für den Computer verfassen.


\subsection{Bewertung}
%\index{Bewertung}
Die Bewertung von Musik ist eine subjektive und kritische Angelegenheit.
Da jeder Mensch anders ist, ist auch der Zugang zur Musik,
und somit auch der Musikgeschmack der Menschen sehr vielfältig.
Ob die Komposition nun gut oder schlecht ist, muss wohl jeder für sich selbst entscheiden.
Musik ist ein Teil von Kunst und Kunst; und damit äußerst subjektiv.
Dennoch kann Musik auch einen geschichtlichen, beziehungsweise `objektiv'' ästhetischen, Wert besitzen,
was dem Künstler Anhaltspunkte beziehungsweise Ansätze zu einer ``objektiven'' Bewertung geben kann~\cite{DiplomarbeitG}.
Im folgenden Kapitel wird eine Möglichkeit, eine (algorithmische) Komposition zu bewerten, erklärt.

\subsubsection{Bewertung mittels Turing-Test}
%%\index{Bewertung mittels Turing-Test}
In den 1950er Jahren veröffentlichte Alan Turing %\index{Turing, Alan}
einen wissenschaftlichen Fachaufsatz mit dem Titel
{\it ``Computing Machinery and Intelligence''}~\cite{turing1948intelligent,Turing-Intelligent_Machinery}.
In diesem Aufsatz beschreibt er einen einfachen Test um die Intelligenz eines Gegenstandes herauszufinden.
Turing wollte herausfinden, ob man einen rechnenden Gegenstand -- einen Rechner, einen Automaten --
von einem Menschen unterscheiden kann; und zwar ausschließlich durch Kommunikation.
Der Test verläuft wie folgt:  ein Mensch befindet sich entweder mit einem anderen Menschen,
oder aber mit einem Computer und in einem Raum.
Zwischen dem Menschen und dem anderen Menschen beziehungsweise Computer ist eine Trennwand aufgebaut.
Der erste Mensch kann also nicht sehen, wer ihm gegenüber ist;
er kann nur mit seinem Gegenüber kommunizieren.
Der Turing-Test besteht darin, ob der Mensch erkennen kann,
wenn ihm ein Computer, eine Maschine, ein Automat gegenüber sitzt.
Gelingt dies dem Menschen nicht -- das heißt, der Mensch kann den Computer nicht von einem menschlichen Gegenüber unterscheiden --
dann hat der Computer den Turing-Test bestanden.
Wir schlagen nun vor, einen ähnlichen Test mit algorithmischen Kompositionen durchzuführen.
In diesem Sinne würde man entweder einem menschlichen Komponisten,
oder aber einem  Computer eine Komposition erstellen lassen;
in letzterem Falle mittels algorithmischen Kompositionstechniken komponieren lassen.
Ein Ensemble spielt die jeweilige Musik vom Blatt.
Auf der anderen Seite der Trennwand sitzt ein Mensch.
Wenn dieser den Unterschied zwischen der Computerkomposition und der von einem Menschen
komponierten Musik nicht erkennen kann, dann hätte das algorithmische Kompositionsprogramm
den Turing-Test bestanden.
Würde der Mensch den Computer aber entlarven, so wäre der Test nicht bestanden.
\subsection{Urheberrecht}
%\index{Urheberrecht}
{\em ``Es muss sich um eine eigentümliche (Stichwort Individualität) %\index{Individualität}
geistige Schöpfung handeln; und diese muss den Gebieten der Literatur,
der Tonkunst, der bildenden Künste oder der Filmkunst zuordbar sein~\cite{Urheberrecht}.''}
So wird die Frage, welche Werke urheberrechtlich geschützt sind, prinzipiell beantwortet.
Die interessante Frage ist nun: was, wenn der {``Komponist''} kein Mensch ist,
sondern ein Computer, der von einem Menschen programmiert wurde?
In ähnlicher Weise stellt sich die Frage, wer der wahre Komponist ist,
im Zusammenhang mit musikalischen Würfelspielen.
Denn bei Würfelspielen wurde die Vorlage von einem Komponisten geschrieben,
jedoch führt der {``Laie''} sie wieder in einer neuen Komposition zusammen.
Beim musikalischen Würfelspielen wird das Problem so gelöst,
dass man das Programm, den Programmierer,
oder die Vorlagen des verwendeten Komponisten anzugeben hat.
Ich werde bei den jeweiligen Kapitel noch näher darauf eingehen~\cite{DiplomarbeitG}.
\section{Verschiedene Algorithmen im Zusammenhang mit Musik}
Es gibt verschiedene Arten von Algorithmen. Ich werde in diesem Kapitel einige davon vorstellen.
Diese Art der Algorithmen sind entscheidend für die Art der algorithmische Komposition.
Gegenbefalls werde ich Beispiele zur Veranschaulichung anführen.


\subsection{Stochastische Algorithmen}
%\index{Stochastische Algorithmen}
Stochastische Algorithmen sind sehr beliebt bei algorithmischen Kompositionen.
Viele der Kompositionsprogramme verwenden diesen Ablauf der algorithmischen Komposition.
Geschichtlich gesehen besitzen stochastische Algorithmen
auch eine wichtige Funktion.
Beispielsweise wurden diese Algorithmen schon zur Zeit Hiller, Xenakis und Koenig verwendet.
Stochastische Algorithmen werden bei einer große Menge von Daten verwendet.
Wenn diese Menge untersucht werden muss,kann die Randomisierung einen
relevanten Einblick in die Materie geben~\cite{StochastischeAlgorithmenI.}.


\subsection{Markow-Ketten}
%\index{Markow-Ketten}
\label{Markow-Ketten}
Der russische Mathematiker Andrei Andrejewitsch Markow (1856–1922)
entwickelte die so genannten {\em  Markow-Ketten},
um stochastische Algorithmen besser kontrollieren zu können.
Mit Hilfe seines Konzeptes konnte er die zufälligen Ereignisse einschränken,
und somit wesentlich vorhersehbarer  machen.
In unserem Fall kann man mit Hilfe von Markow-Ketten berechnen,
mit welcher Wahrscheinlichkeit welche Note als nächstes erklingen wird.
Das kann man mit einer Tabelle veranschaulichen:
Jeder Ton in Tabelle \ref{my-label}  besitzt eine Wahrscheinlichkeit,
einen anderen Ton anzuschlagen.
Nach dem c kommt mit einer 50 prozentiger Wahrscheinlichkeit ein e oder ein g.
Die Null stellt das Ende der Melodie dar.
Wegen der Wahrscheinlichkeitsrechnung -- die Summe der Wahrscheinlichkeit des Eintritts
alle Fälle muss eins sein --  müssen sich die Zeilensummen auf eins addieren.
\begin{table}[h!]
\centering
\begin{tabular}{c | cccccccccc  }
\hline\hline
  \quad \quad \quad \quad &\quad c\quad &\quad d\quad &\quad e\quad &\quad f\quad &\quad g\quad &\quad 0\\ \hline
c &0&0&0,5&0&0,5&0 \\
d &0&0&0&1&0&0\\
e &0&1&0&0&0&0 \\
f &0,5&0&0&0&0&0,5 \\
g &0&1&0&0&0&0\\
\hline\hline
\end{tabular}
\caption{Markow-Ketten erster Ordnung. Die Zeilensummen sind eins.}
\label{my-label}
\end{table}
\\
Eine Frage stellt sich zuerst: mit welcher Note beginnt die Komposition?
Nach d kommt immer ein f.
Nach g kommt immer ein d.
Nach c kommt in der Hälfte der Fälle ein e oder ein g.
Nach f kommt in der Hälfte der Fälle ein e oder ein Abbruch.
Um diese Situation zu realisieren, habe ich der Einfachheit
halber mit der Note c begonnen, und jeweils  die Münze geworfen,
wenn es zu einem c oder f  gekommen ist.
Nun stellt sich die weitere Frage,
ob die Melodie nicht sehr schnell enden würde,
weil die Wahrscheinlichkeit, dass nach einem f die Melodie endet, 50-prozentig  ist --
das heißt in der Hälfte der Fälle endet die Melodie nach dem Ton F.
Diese Frage ist berechtigt; jedenfalls ist zu erwarten, dass aus den Festlegungen von Tabelle \ref{my-label}
keine allzu lange Komposition entstehen wird.
Man könnte die Null als Zeichen eines Taktstrichs darstellen.
Dies würde bei einer Melodie ohne Notenwerte und ohne Takt schon sinnvoller erscheinen.
Um nun die Markow-Ketten zu veranschaulichen,
werde ich mittels einer Münze eine Zufalls-folge generieren.
Wenn die Münze {\em Zahl} anzeigt ist die nachfolgende Note bei c ein e;
und bei f ein c.
Bei {\em Kopf} ist die nachfolgende Note bei c ein g und bei f endet die Melodie.
Das trifft in meinem Fall zu, weil ich die Position des Programmierers einnehme,
und somit die Voraussetzungen bestimmen darf.
Mit welcher Note beginnt die Komposition nun? Ich habe, wie schon gesagt,  der Einfachheit
halber mit der Note c begonnen, und jeweils  die Münze geworfen, wenn es zu einem c oder f  gekommen ist.

\subsubsection{Versuch Nr. 1}
Bei Versuch Nr. 1 werde ich dreimal eine Komposition erstellen,
bis ich durch Münzwurf zum Ende der Komposition gelange.
Das Ende ist in diesem Fall erreicht, wenn ich die Zahl Null erreiche.
Die vorher in dem Kapitel \ref{Markow-Ketten} (Seite \pageref{Markow-Ketten})
beschriebenen Regeln zum Versuch werde ich hier anwenden.
\begin{figure}[h!]
\centering
\begin{music}
\instrumentnumber{1}
\setstaffs1{1}

\startextract
\NOtes\zsong{c } \qa c \en
\NOtes\zsong{e } \qa e\en
\NOtes\zsong{d } \qa d\en
\NOtes \zsong{f }\qa f\en \endextract


\instrumentnumber{1}
\setstaffs1{1}

\startextract
\NOtes\zsong{c } \qa c \en
\NOtes\zsong{g } \qa g\en
\NOtes\zsong{d } \qa d\en
\NOtes \zsong{f }\qa f\en
\NOtes \zsong{c }\qa c\en
\NOtes \zsong{g}\qa g\en
\NOtes \zsong{d}\qa d\en
\NOtes \zsong{f }\qa f\en\endextract

\instrumentnumber{1}
\setstaffs1{1}

\startextract
\NOtes\zsong{c } \qa c \en
\NOtes\zsong{e } \qa e\en
\NOtes\zsong{d } \qa d\en
\NOtes \zsong{f }\qa f\en \endextract
\end{music}
\centering
\caption{Drei Musikkompositionen aus Markov-Ketten, definiert durch Tabelle \ref{my-label}.}
\label{my-fig1}
\end{figure}
\\
In dem gegebenen Beispiel, welches in der Abbildung \ref{my-fig1} dargestellt wird,
ist ein gewisses Muster aufgetreten: beispielsweise kommt nach dem d immer ein F.
Dieses Muster kann man durchbrechen, wenn man die Wahrscheinlichkeit der jeweiligen Zahlen verringert,
ein größeres Spektrum, das heißt mehr Töne, verwendet,
oder andere Wahrscheinlichkeiten in Tabelle \ref{my-label} annimmt.
Das Problem hierbei wäre abermals die Monotonie der musikalischen Komposition,
welche nicht ganz wegzubekommen ist.
In dem Beispiel ``Versuch Nr. 1''  klingt die Melodie subjektiv eher langweilig;
und zwar wegen der öfters auftretenden Wiederholungen.

\subsubsection{Versuch Nr. 2}
In Versuch Nr.2 werde ich einen Taktstrich einführen,
sobald ich zu der Zahl Null gelange.
Wenn ich bei Null angekommen bin, beginne ich nach dem Taktstrich wieder bei c.
Die Komposition wird nach dem Vierten Taktstrich enden.
Auch hier gelten die Vorgaben, welche ich bereits im Kapitel \ref{Markow-Ketten} (Seite \pageref{Markow-Ketten})
beschrieben habe. Die Wahrscheinlichkeitstabelle \ref{my-label} wird wieder angewandt.
\\
\begin{figure}[h!]
\centering
\begin{music}
\nobarnumbers
\instrumentnumber{1}
\setstaffs1{1}
\startextract
\NOtes\zsong{c } \qa c \en
\NOtes\zsong{e } \qa e\en
\NOtes\zsong{d } \qa d\en
\NOtes \zsong{f }\qa f\en \barre
\NOtes\zsong{c } \qa c \en
\NOtes\zsong{g } \qa g\en
\NOtes\zsong{d } \qa d\en
\NOtes \zsong{f }\qa f\en
\endextract
\startextract
\NOtes\zsong{c } \qa c \en
\NOtes\zsong{e } \qa e\en
\NOtes\zsong{d } \qa d\en
\NOtes \zsong{f }\qa f\en
\NOtes\zsong{c } \qa c \en
\NOtes\zsong{g } \qa g\en
\NOtes\zsong{d } \qa d\en
\NOtes \zsong{f }\qa f\en  \bar
\NOtes\zsong{c } \qa c \en
\NOtes\zsong{e } \qa e\en
\NOtes\zsong{d } \qa d\en
\NOtes \zsong{f }\qa f\en
\NOtes\zsong{c } \qa c \en
\endextract
\startextract
\NOtes\zsong{g } \qa g\en
\NOtes\zsong{d } \qa d\en
\NOtes \zsong{f }\qa f\en
\NOtes\zsong{c } \qa c \en
\NOtes\zsong{e } \qa e\en
\NOtes\zsong{d } \qa d\en
\NOtes \zsong{f }\qa f\en
\NOtes\zsong{c } \qa c \en
\NOtes\zsong{e } \qa e\en
\NOtes\zsong{d } \qa d\en
\NOtes \zsong{f }\qa f\en \bar
\NOtes\zsong{c } \qa c \en
\NOtes\zsong{g } \qa g\en
\NOtes\zsong{d } \qa d\en
\NOtes \zsong{f }\qa f\en
\endextract
\end{music}
\caption{Eine Musikkompositionen aus Markov-Ketten, definiert durch Tabelle \ref{my-label} mit Taktstrich.}
\label{my-fig2}
\end{figure}

Auch in der experimentellen Komposition, welche in Abbildung \ref{my-fig2}
dargestellt ist,
fällt auf, dass  ein gewisses Wiederholungsmuster der Komposition erkennbar wird.
Als weiteres Problem kommt noch dazu,
dass die Komposition nach jedem Taktstrich  wieder mit c beginnt.
Eine entscheidende Variation ist also nicht gegeben.
Das Stück würde also subjektiv  ebenfalls langweilig und monoton klingen.
Um das System zu verbessern, könnte man nach jedem Taktstrich mit einer anderen Note beginnen.
Ebenfalls ist es empfehlenswert,
die Anzahl der Noten zu erhöhen und die Wahrscheinlichkeiten zu verringern,
damit die Wahrscheinlichkeit, dass man die gleiche Note bekommt geringer ist.
\subsection{Fraktale}
%\index{Fraktale}
Ein Fraktal~\cite{mandelbrot-83,mandelbrot-77}
ist eine geometrische Struktur, Form oder Figur; ähnlich wie ein Kreis oder ein Dreieck,
aber mit sehr unterschiedlichen Eigenschaften.
Fraktale können sehr unregelmäßige, stochastische Formen bilden, welche   unendlich viele Details ausweisen.
Eines haben aber alle diese Strukturen gemeinsam: sie sehen auf allen betrachteten Skalen und
bei allen Messgenauigkeiten -- zum Beispiel
bei unterschiedlicher räumlicher oder zeitlicher Auflösung -- ``ähnlich'' aus.
Man sagt auch: Fraktale bilden {\em selbstähnliche Strukturen.}
Beispielsweise sieht die Küste Englands, von der Luft aus betrachten, ähnlich zerklüftet aus wie von zwei Metern Abstand;
oder auch wenn man sie mit der Lupe betrachtet.
Dies hat zur Folge, dass eine einfache Definition der Küstenlänge $L$
-- etwa das Produkt aus der Länge eines Maßstabes $l$
mal die Anzahl $n(l)$, wie oft man diesen Maßstab ablegen muss, um die Küste darzustellen, also
$L = l \cdot n(l)$ von der Masstabslänge abhängt.
Misst man konkret nach, dann wird diese Küstenlänge $L(l)$ immer größer je kleiner die
Maßstabslänge $l$ wird.
Diese Problematiken wurden schon früh von Mathematikern wie Felix Hausdorff (1868-1942)
erkannt und maßtheoretisch gedeutet~\cite{falconer2,falconer1}.
Fraktale Geometrien in der Natur wurden
erstmals von Benoit Mandelbrot vorgestellt,
um Figuren wie Wolken, Berge oder Küsten darstellen zu können.
Mandelbrot verwendete 1975 erstmals das englische Wort {\it ``fractal''}.
Die fraktalen Objekte wie die Kochsche Schneeflocke
war schon davor bekannt; zum Beispiel unter dem Namen {\it  ``monster curves''}
-- vielleicht wegen der Probleme, ihre Länge zu definieren~\cite{MagisterarbeitA}.
In der Schneeflocke sieht man sehr gut, was Mathematiker wie Mandelbrot mit Selbstähnlichkeit meinen.
%\index{Selbstähnlichkeit}
%\index{Schneeflocke}
%\index{Koch-Kurve}
Das Muster oder Erscheinungsbild, welches bei niedriger Auflösung zu erkennen ist,
wiederholt sich, wenn man genauer ``ins Bild hineinzoomt,'' und gleichzeitig die Auflösung erhöht.\\
\begin{figure}[h!]
\resizebox{12cm}{!}{
\centering
\begin{tikzpicture}[scale=3,decoration=Koch snowflake]
\draw decorate{ (0,0) -- (3,0) };
\draw decorate{ decorate{ (0,-1) -- (3,-1) }};
\draw decorate{ decorate{ decorate{ (0,-2) -- (3,-2) }}};
\draw decorate{ decorate{ decorate{ decorate{ (0,-3) -- (3,-3) }}}};
\end{tikzpicture}
}
\caption{Fraktale ``Schneeflocke'' oder Koch-Kurve.}
\label{my-fig3}
\end{figure}\\
An Abbildung~\ref{my-fig3} ist ersichtlich, dass die Schneeflocke mit einem einfachen mittigen Dreieck beginnt
und sich von diesem dann selbständig weiterbildet.
In jedem Bildungsschritt kommt statt einer jeden vormals geraden Linie nochmals ein Dreieck hinzu
beziehungsweise wird dieser geraden Linie durchbrochen und diese ein weiteres Dreieck darauf gesetzt;
und so weiter.
Wenn man nun ganz nahe heranzoomt, dann erkennt man, dass die Schneeflocke immer ein gleichbleibendes Muster hat
-- genau so ist sie ja konstruiert!
In der Natur kommt es selten zu regulären, streng sich wiederholenden, Selbstähnlichkeiten. Hier sind die natürlichen
Formen eher stochastisch; das heißt im Mittel großer Gruppen von Formen, selbstähnlich.
Man nennt diese Fraktale dann {\em stochastische Fraktale}~\cite{MagisterarbeitA}.

%\index{stochastische Fraktale}
\subsubsection{Fraktale Musik}
%\index{Fraktale Musik}
Ich habe angedeutet, wie Fraktale in der Geometrie vorkommen.
Diese Fraktale entstehen oft durch sehr einfache Bildungsgesetzte, welche man bei veränderten Masstab
immer wieder anwendet.
Wenn man Fraktale nun in der Musik betrachtet,
dann sind diese Funktionen jedoch nicht all zu einfach herstellbar.
Zuerst sollte man erwähnen, dass musikalische Fraktale {\em zeitliche Abfolgen} und Gebilde sind.
Basierend auf  immer wiederkehrenden Prozessen  kann man stochastische fraktale Tongebilde,
das heißt fraktale Musik,
mit chaotischen Systemen produzieren.
Bestimmte Arten der fraktalen Musik entstehen durch sogenannte {\em Iterierte Funktionensysteme.}
%\index{Iteriertes Funktionensystem}
In diesen Systemen werden Transformationen wie Rotationen und Übersetzungen selbst-rekursiv verwendet.
Das heißt, die Funktionen und Transformationen werden immer wieder auf sich selbst (bei verändertem Maßstab)
angewandt.
Diese Methode  ist jedoch relativ kompliziert und wird wenig verwendet.
Eine andere Möglichkeit wäre wichtige Informationen mit einer beständigen Skala zusammen zu führen.
Dafür ist die Rosa (englisch {\it pink})  Musik bekannt,
welche in einem der folgenden Kapitel \ref{pink} näher beschrieben wird~\cite{MagisterarbeitA}.
%\index{white noise}
\subsubsection{White noise-Weißes Rauschen}
%\index{weißes Rauschen}
Weißes Rauschen erscheint ``farblos'';
das heißt egal ob man es schnell oder langsam abspielt, es wirkt immer langweilig beziehungsweise subjektiv
unangenehm und störend.
Viele Leute kennen das Geräusch wahrscheinlich aus dem Fernseher oder aus dem Radio als
Bild oder Geräusch zwischen zwei Sendern.
Wenn zufällige Signale von einem elektronischen Bauteil oder von der Antenne in den Empfänger geraten,
dann entsteht dieses Geräusch.
Nun lässt sich weißes Rauschen auch mittels einfachster Methoden selbst erzeugen.
Man nimmt hierzu einen Zufallsgenerator --
also zum  Beispiel einen Würfel oder einen selbst gebastelten Drehgenerator.
Ich werde dies mit einer Oktav eines Klaviers (man kann mehrere Oktaven verwenden) demonstrieren.
Die Töne, welche ich verwende,
sind nur die weiße Tasten des Klaviers.
(Man könnte auch die schwarzen, also die Halbtöne, hinzunehmen.):
do, re, mi, fa, so, la, ti.
Wenn man, so wie ich in der Folge,
den Versuch mit einem Drehapparat macht,
muss man sich nun einen Kreis ausschneiden und diesen in sieben Segmente unterteilen,
um die Töne do, re, mi, fa, so, la, ti
der Oktav hineinzuschreiben.
Die Unterteilungen müssen nicht gleichmäßig sein.
So könnte die Unterteilung in der {fa} steht die größte sein,
und somit diese Note fa am wahrscheinlichsten auftreten.
In der Mitte realisiere ich mit Hilfe einer Stecknadel einen Dreher.
Um nun die weiße Melodie entstehen zu lassen,
muss man immer wieder an dem Dreher drehen und den Ton aufschreiben oder aufnehmen.
Um die Melodie etwas aufregender zu gestalten,
kann mehre Unterteilungen machen, und die schwarzen Tasten des Klaviers dazu nehmen~\cite{gard-78}.
\begin{figure}[h!]
\centering
\begin{music}
\instrumentnumber{1}
\setstaffs1{1}

\startextract
\NOtes\zsong{d } \qa d\en
\NOtes\zsong{h } \qa h\en
\NOtes\zsong{g } \qa g\en
\NOtes\zsong{a } \qa d\en
\NOtes\zsong{h } \qa h\en
\NOtes\zsong{a } \qa a\en
\NOtes \zsong{e }\qa e\en
\NOtes\zsong{c } \qa c\en
\NOtes\zsong{a } \qa a\en
\NOtes \zsong{a }\qa a\en
\NOtes \zsong{f }\qa f\en \endextract
\end{music}
\caption{Eine Musikkompositionen aus weißem Rauschen}
\label{my-fig4}
\end{figure}
\\
Um die Melodie in der Folge noch anspruchsvoller zu machen,
kann man noch einen Drehapparat nehmen, und diesen in vier Teile teilen.
Hierbei ist es wiederum egal, in welchem Verhältnis diese vier Unterteilungen zueinander stehen.
In die Unterteilungen kann man nun Notenwerte schreiben;
zum Beispiel 1, 1/2, 1/4 und 1/8.
Man dreht nun und fügt jedem Ton, welchen man vorhin generiert hat,
einen Notenwert hinzu.
Das Ergebnis wird zumeist subjektiv eher ``falsch'' klingen.
Dennoch hat man weißes Rauschen generiert~\cite{gard-78}.
\begin{figure}[h!]
\centering
\begin{music}
\instrumentnumber{1}
\setstaffs1{1}

\startextract
\NOtes\zsong{d } \ha d\en
\NOtes\zsong{h } \qa h\en
\NOtes\zsong{g } \qa g\en
\NOtes\zsong{a } \ca d\en
\NOtes\zsong{h } \ha h\en
\NOtes\zsong{a } \qa a\en
\NOtes \zsong{e }\ca e\en
\NOtes\zsong{c } \wh c\en
\NOtes\zsong{a } \ca a\en
\NOtes \zsong{a }\qa a\en
\NOtes \zsong{f }\ca f\en \endextract
\end{music}
\caption{Eine Musikkompositionen aus weißem Rauschen mit Noten werten}
\label{my-fig5}
\end{figure}

\subsubsection{Brownian Noise- Brown Noise- Braunes Rauschen}
%\index{brownian noise}
%\index{braunes Rauschen}
Bei dem Braunen Rauschen geht es darum,
dass sich der Generator den vorher angespielten Ton merkt, und von diesem ausgehend den nächsten Ton
erzeugt beziehungsweise anspielt.
Dies ist ein vollständig anderes Berufungsverfahren als beim weißen Rauschen, bei dem jedes Mal
ein komplett neuer Ton gewürfelt wird, welcher kein Verhältnis oder Korrelation zu einem der vorherigen Töne hat.
Um ein solches braunes Musikstück herzustellen  braucht man, wie schon beim weißen Rauschen beschrieben,
einen Kreis mit einer Drehscheibe.
Diese Drehscheibe wird wieder in sieben Teile unterteilt;
aber nun werden {\em Intervalle} eingefügt statt Noten.
Wieder ist es egal, in welchem Größenverhältnis die Unterteilungen zueinander stehen.
Die Intervalle bekommen jeweils noch ein ``Plus'' oder ein ``Minus'' hinzugefügt.
Plus heißt ``hinauf'' (aufsteigend), und Minus heißt ``hinunter'' (absteigend).
Nun beginnen wir das Stück am Klavier auf dem mittleren c.
Dreht man den Dreher auf der Drehscheibe, erhält man ein Intervall.
Dieses Intervall zeigt, ob man nach oben oder unter gehen muss;
und um wie viele Töne dies geschehen soll.
Mit dieser Tonbildungsvorschrift kann es passieren,
dass die Klaviertastatur sogar verlassen werden würde, wenn man den Tastaturendpunkt
nicht als Ende festlegt.
Man könnte diesen Punkt aber auch als {``elastische Barriere''} festlegen,
bei welcher der Ton ``zurückgeworfen'' wird.
Um dies möglich zu machen, müssen wir Regeln einfügen.
Diese Regel soll es möglich machen, dass,
je weiter der Ton sich von dem mittleren c entfernt,
desto eher kommt dieser wieder zurück~\cite{gard-78}.
\\
\begin{figure}[h!]
\centering
\begin{music}
\instrumentnumber{1}
\setstaffs1{1}

\startextract
\NOtes\zsong{c+1} \qa c\en
\NOtes\zsong{f-3} \qa f\en
\NOtes\zsong{c +1} \qa c\en
\NOtes\zsong{d-1} \qa d\en
\NOtes\zsong{c+1 } \qa c\en
\NOtes\zsong{f+3} \qa f\en
\NOtes \zsong{e-1}\qa e\en
\NOtes\zsong{c-2} \qa c\en
\NOtes\zsong{h-2} \qa a\en
\NOtes \zsong{d+3}\qa d\en
\NOtes \zsong{c-1}\qa c\en \endextract
\end{music}
\caption{Eine Musikkompositionen aus braunem Rauschen}
\label{my-fig6}
\centering
\end{figure}
\\
Um die braunschen Melodie nun anspruchsvoller zu machen,
kann man wie zuvor eine zweite Drehscheibe verwenden.
Diese zweite Drehscheibe würde wieder den Notendauer bestimmen.
Um aber wieder das Prinzip des brauschen Rauschens anzuwenden,
würde man wiederum von dem ausgehenden Wert entweder bestimmte Notendauer dazu geben oder abnehmen.
In dieser Weise wird man ein Musikstück erstellen,
das dem weißen Rauschen wenig ähnelt.
Die Melodie geht hinauf und hinunter,
aber der Klang erscheint nun subjektiv monoton und wird noch immer nicht als sehr schön empfunden.
Das Ergebnis klingt etwas kontrollierter, weil der Ton hinauf oder hinunter geht
--
die Töne sind also viel näher beieinander und ``springen nicht so extrem herum''~\cite{gard-78}.

\subsubsection{Pink Noise- Pinkes- Rosa Rauschen}
%\index{pink noise- Pinkes Rauschen}
%\index{rosa Rauschen}
\label{pink}
Wie bereits erwähnt, befindet sich rosa Rauschen in der Mitte von braunem und weißem Rauschen.
Man könnte sagen, dass weißes Rauschen, sowie braunes Rauschen die beiden Extreme darstellen.
Im weißen Rauschen ist kein Wiedererkennungswert, keine ``Erinnerung'' der Noten
untereinander,
während beim braunschem Rauschen eine zu starke Assoziation und  ``Erinnerung'' der Noten zueinander stattfindet.
Weißes Rauschen klingt langweilig,
weil es zu unberechenbar ist.
Andererseits klingt braunes Rauschen langweilig, weil es zu berechenbar ist.
In traditionellen Musikstücken wird zwischen diesen beiden Extremen der Mittelweg gefunden.
Der Komponist startet mit eine Idee, wie sein Stück klingen soll,
und überlegt sich, wie einzelne Teile aufgebaut werden sollen.
Langfristig gesehen führt das zu zeitlich korrelierenden Tönen,
mit kleineren Teilen von unabhängigen, zufälligen Tonfolgen.
Rosa Rauschen (oder 1/f-Rauschen) ist wesentlich aufwändiger herzustellen
als  braunes oder weißes Rauschen.
Die Hauptaufgabe liegt darin eine bestimmte Quantität von Zusammenhängen der Noten und ``Erinnerung''
derselben zu erzeugen.
Voss hat es mit einem Experiment geschafft,
rosa Rauschen herzustellen und nachvollziehbar zu machen~\cite{gard-78}.
Hierfür braucht man drei Würfel.
Jeder Würfel bekommt eine Farbe zugeordnet.
In unserem Fall wählen wir die Farben {\color{blue}blau}, {\color{green}grün} und {\color{red}rot}.
Die Summen, welche diese drei Würfel erwürfeln können,
liegt von 3 ($= 1+1+1$) bis 18 ($= 6+6+6$).
Wir wollen eine Melodie mit acht Tönen generieren,
in einer Skala von 16 Tönen, die frei ausgewählt sind.
Nun schreibe ich die ersten acht Nummern auf; die Nummer Null wird die erste sein:
0, 1, 2, 3, 4, 5, 6, 7. Ich kodiere sie jeweils im dem Binärsystem;
das heißt die Nummer Null bleibt die Nummer Null: $0\rightarrow 000$.
Danach kommen $1\rightarrow 001$,
$2\rightarrow 010$; und so weiter bis $7\rightarrow 111$.
Betrachten wir nun Tabelle \ref{my-label-t1}.
Jede Spalte bekommt nun einen Farbe zugewiesen~\cite{gard-78}.
\begin{table}[h!]
\centering
\begin{tabular}{ccccccccccc}
\hline\hline
  & {\color{blue}blau} &{\color{green}grün}& {\color{red}rot}\\ \hline
0 &{\color{blue}0}&{\color{green}0}&{\color{red}0}\\
1 &{\color{blue}0}&{\color{green}0}&{\color{red}1}\\
2 &{\color{blue}0}&{\color{green}1}&{\color{red}0}\\
3 &{\color{blue}0}&{\color{green}1}&{\color{red}1}\\
4 &{\color{blue}1}&{\color{green}0}&{\color{red}0}\\
5 &{\color{blue}1}&{\color{green}0}&{\color{red}1}\\
6 &{\color{blue}1}&{\color{green}1}&{\color{red}0}\\
7 &{\color{blue}1}&{\color{green}1}&{\color{red}1}\\
\hline\hline
\end{tabular}
\caption{Binäre Tabelle für die von Voss vorgeschlagenen Methode zur Erzeugung von rosa Rauschen.}
\label{my-label-t1}
\end{table}
\\
Die Würfelsumme die man erwürfelt wurde bereits einem Ton zugewiesen.
Um nun rosa Rauschen zu produzieren, beginnt man alle drei Würfel zu werfen.
Man addiert die Zahl, welche sich ergibt, und fügt die erste Note in dem Stück ein.
Wie man an der unten angeführten Tabelle sieht,
verändert sich in der Spalte vom roten Würfel die Zahl von Null auf eins.
Um nun die nächste Note zu generieren,
lasse ich den blauen und grünen Würfel liegen und merke mir die Summe;
den roten Würfel verwende ich um eine neue Zufalssvariabel zu erstellen.
Die Zahl, welche hierbei herauskommt, wird zu der Summe vom blauen und grünen Würfel addiert.
Hierbei kommt wieder eine Zahl heraus, die vorher  einer Note zugeordnet wurde.
Der Sinn dahinter ist, dass bei jeder Veränderung der entsprechende Würfel neu
gewürfelt wird;
und bei jeder gleichbleibenden Zahl bleibt der entsprechende Würfel liegen.
In der Spalte vom blauen Würfel verändert sich am wenigsten;
und in der Spalte vom roten Würfel ist die Veränderung am größten.
Um das Ganze nun noch anspruchsvoller zu gestalten, könnte man nun vier Würfel verwenden,
oder  die Anzahl der Töne erhöhen~\cite{gard-78}.
\\
\begin{figure}[h!]
\centering
\begin{music}
\instrumentnumber{1}
\setstaffs1{1}

\startextract
\NOtes\zsong{f } \qa f\en
\NOtes\zsong{c } \qa i\en
\NOtes\zsong{a } \qa a\en
\NOtes\zsong{f } \qa l\en
\NOtes\zsong{g } \qa m\en
\NOtes\zsong{a } \qa a\en
\NOtes \zsong{e }\qa e\en
\NOtes \zsong{c }\qa i\en \endextract
\end{music}
\caption{Eine Musikkompositionen aus rosa Rauschen}
\label{my-fig7}
\end{figure}

\subsection{Wissensbasierte Systeme}
%\index{Wissensbasierte Systeme}
Die meisten algorithmischen Systeme sind wissensbasiert.
Diese Systeme arbeiten nach Beschränkungen oder Regeln, welche sie einhalten müssen.
Wissensbasierte Systeme sind praktisch,
besonders wenn wir explizite Strukturen oder Regeln verwenden wollen.
Einer der größten Vorteile bei dieser Art von System ist, dass das System Gründe für seine Auswahl geben kann.
Man kann nachvollziehen, wieso der Computer zu dem Ergebnis gekommen ist~\cite{kurbel1992entwicklung}.
\subsubsection{CHORAL}
%\index{CHORAL}
Im Rahmen seiner Dissertation~\cite{ebcioglu1986expert}.schrieb
Kemal Ebcioglu ein Programm Namens CHORAL.
Er arbeitete an seinem Projekt weiter als er der IBM Thomas j. Watson Research Centre beitrat. Ebcioglu bekam mehrere Auszeichnungen für seine Arbeit, unter anderem 2 IBM Outstanding Technical Achievement Awards.\\ Das Programm beschreibt  die Harmonisierung von 4 Teiglingen Chorälen die von Johan Sebastian Bach Komponiert wurden. Es besteht aus ungefähr 350 Regeln die hauptsächlich mit Prädikatenlogik arbeiten. Diese Regeln beziehen sich auf diverse musikalische Aspekte; Akkorde, Melodien von individuelle Teilen oder den Stimmführungsregeln im Sopran und Bass. Das Programm verwendet eine {\em generiere und teste} Methoden sowie die sogenannte {\em backtracking} Methode. Mit Hilfe dieser Methoden werden Heuristiken verwendet um die best mögliche musikalische Lösung zu finden.
Um sein Programm kompakt darzustellen hat Ebicioglu  eine eigene Programmiersprache entwickelt.
\subsubsection{Problematik}
Wie auch bei den anderen Systemen treten auch hier Probleme auf.
Zum Beispiel ist es sehr zeitintensiv, Wissen aus dem Fachgebiet der Musik zu schöpfen.
Ebenfalls ist ein System nur so gut wie der Programmierer  oder die Programmiererin.
Man muss ein klares Konzept haben oder flexibel sein was die Darstellung des Systems betrifft.
Wenn man zu viele Ausnahmen für die Regel findet wird das System zu kompliziert~\cite{ebcioglu1986expert}.
\subsection{ Genetische Algorithmen}
%\index{Genetische Algorithmen}
In den folgenden Kapiteln erkläre ich wie Genetische Algorithmen in der Musik funktionieren, ebenso werden die Geschichte und die allgemeine Verwendung diese Algorithmus angeschnitten.
\subsubsection{Geschichte}
John H. Holland  entwickelte in den 60er Jahren an der Universität of Michigan,
USA, die heute bekannten {\em genetischen Algorithmen.}
Währenddessen  wurden Evolutionsstrategien und evolutionäres Programmieren entwickelt.
Im Jahre 1975 erschien das Buch {\em Adaptation in Natural and Artificial Systems}
von John H. Holland.
Goldberg, ein Student von Holland, entwickelt das Konzept seines Professors weiter und schrieb
ebenfalls ein Buch {\em Genetic Algorithms in Search, Optimization, and Machine Learning}~\cite{golberg1989genetic}.
Die Dissertation von Goldberg veranschaulicht die erste erfolgreiche Anwendung
eines solchen genetischen Algorithmus.
Seitdem werden Genetische Algorithmen immer weiter entwickelt und optimiert~\cite{GenetischeAlGesch}.
\subsubsection{Allgemeines}
Genetische Algorithmen sind ein Teil der evolutionären Algorithmen.
Diese evolutionären Algorithmen  sind, wie der Name es bereits verrät, evolutionär aufgebaut.
Bei der Fortpflanzung wird die Weitergabe der Erbinformation gesichert.
So wie bei uns Menschen, beziehungsweise wie bei Tieren,
können bei der Fortpflanzung {\em Mutationen}
%\index{Mutation}
(dies ist ist eine minimale Veränderung der Erbinformation) und {\em Rekombinationen}
(Gene  verschiedenen Individuen werden vermischt; und daraus entsteht ein neues Gen),
sowie {\em Selektion},
also die gezielte Auswahl  eines Individuums nach bestimmten Kriterien oder {\em ``Fitness''} entstehen.
%\index{Fitness}
Individuen mit höherem Fitnesswert werden eher ausgewählt.
Um nun eine Folgepopulation bei genetische Algorithmen entstehen zu lassen,
wird erstmal der sogenannte Fitnesswert, welcher die Brauchbarkeit einzelner Individuen bewertet,  gemessen.
Individuen mit besseren Genen haben bessere Chancen zu überleben;
im Gegensatz zu den Mitstreitern, welche nicht so gute Erbanlagen übernommen haben.
Aus dieser Variation und Selektion erklärt man sich heute die Artenvielfalt.
Diese Art von System ist besonders hilfreich bei komplexen und sehr großen Suchräumen.
In unserem Fall wäre das eine große Anzahl von computergenerierten Musikstücken.
Wenn man das nun in die Welt der Computer übertragen möchte, werden aus Individuen
Lösungskandidaten~\cite{GenetischeAl}.
\subsubsection{Genetische Algorithmen in der Musik}
Es gibt zwei Möglichkeiten, genetische Algorithmen in der Musik zu verwenden.
Zum einen können wie Individuen bestimmte Töne- und Klangfolgen nach beliebiger Länge zufällig
generiert werden und danach mittels genetischer Algorithmen
neu zusammengesetzt werden.
Die Gene in diesen Individuen können bestimmte Motive,
Klänge aber auch Töne sein.
Durch die Evolutionsprozesse werden diese klang beziehungsweise Tonabfolgen auseinander genommen und zu einem neuen Individuum zusammengebracht.
Der Fitnesswert berechnet wiederum, welches der  entstandenen Individuen am
besten in das vorher festgelegte Schema passt; und sortiert aus.
Das macht das Programm nun so lange,
bis man ihm die Anweisung zu stoppen gibt oder bis das vorher festgelegte Ziel erreicht wird.
Der Fitnesswert muss vorab bestimmt werden,
was in musikalischer Hinsicht im Gegensatz zur Mathematik und anderer weniger subjektiver Disziplinen als die Musik
ein Problem darstellen kann~\cite{DiplomarbeitG}.
\subsubsection{Crossover}
%\index{Crossover}
Das {\em Crossover} tritt meist bei Rekombination auf.
Wie in der unten angegeben Tabelle \ref{my-label-t3}
sieht man, dass bestimmte Teile des Individuums  vertauscht werden.
So ist in der ersten Notenzeile im ersten Takt ab dem f die gleiche Tonfolge;
und bei Takt zwei in der ersten Notenzeile die gleiche Abfolge wie bei der zweiten
Notenzeile im Takt eins ab dem Ton f.
Aus dem ursprünglichen Individuum entsteht eine neue Generation.
Diese wird von dem Fitnesstest noch einmal getestet,
welche in unserem Fall die meisten vorhin festgelegt Anforderungen erfüllt~\cite{DiplomarbeitG}.
\begin{table}[h!]
\centering
\begin{tabular}{l l}
\begin{music}
\generalsignature{0}
\startextract
\NOtes\qu c\en
\NOtes\qu h\en
\NOtes\qu f\en
\NOtes\qu g\en
\NOtes\qu g\en
\NOtes\qu h\en
\bar
\NOtes\qu c\en
\NOtes\qu h\en
\NOtes\qu f\en
\NOtes\qu g\en
\NOtes\qu c\en
\NOtes\qu f\en
\endextract
\end{music}
&\\
\begin{music}
\generalsignature{0}
\startextract
\NOtes\qu c\en
\NOtes\qu d\en
\NOtes\qu e\en
\NOtes\qu g\en
\NOtes\qu c\en
\NOtes\qu f\en
\bar
\NOtes\qu c\en
\NOtes\qu d\en
\NOtes\qu e\en
\NOtes\qu g\en
\NOtes\qu g\en
\NOtes\qu h\en
\endextract
\end{music}
\end{tabular}
\caption{Crossover}
\label{my-label-t3}
\end{table}

\subsubsection{Mutation}
%\index{Mutation}
Bei der {\em Mutation}
verändert sich ein Teil des Individuums zufällig.
Es gibt im musikalischen mehrere Möglichkeiten, wie eine solche Mutation aussehen kann.
Es kann zum Beispiel eine einfache Klangveränderung  oder auch eine rhythmische Veränderung sein.
In dem von mir unten angegebenen Beispiel in Tabelle \ref{my-label-t4}
hat sich das a, welches in der ersten Notenzeile eine Viertelnote war, zu einer Achtelnote verändert~\cite{DiplomarbeitG}.
\begin{table}[h!]
\centering
\begin{music}
\generalsignature{0}
\startextract
\NOtes\qu c\en
\NOtes\qu d\en
\NOtes\qu e\en
\NOtes\qu f\en
\NOtes\qu g\en
\NOtes\qu h\en
\endextract
\startextract
\NOtes\qu c\en
\NOtes\qu d\en
\NOtes\qu e\en
\NOtes\qu f\en
\NOtes\qu g\en
\NOtes\cu h\en
\endextract
\end{music}
\caption{Mutation}
\label{my-label-t4}
\end{table}

\subsection{Musikalische Würfelspiele}
%\index{Würfelspiele}
Bei musikalischen Würfelspielen dient der Würfel als Zufallsgenerator.
Früher wurden ebenfalls auch Spielkarten statt dem Würfel verwendet.
Trotz diesen Zufallsgeneratoren gilt das wenn man immer die gleichen Augenzahlen würfelt, so bekommt man immer das gleiche Ergebnis.
Denn die Augenzahlen sind einer bestimmten Sequenz des Stücks zugeordnet, welche sich nicht verändert.
\subsubsection{Geschichte}
Zu der Zeit, als Mozart lebte, waren Würfelspiele sehr beliebt.
Das Wunderkind selbst schrieb Vorlagen, damit  Laien auf diese Weise quasi ``componieren'' konnten.
Die Spiele kamen im Ende des 18. Jahrhunderts nach Europa
und wurden von nun an als lustiger Zeitvertreib genutzt.
Auf einmal konnte jeder, der ein bisschen Klavier spielen konnte,
auch eine eigene Komposition veröffentlichen.
Gegen Ende des 19. Jahrhundert wurde diese Methode immer seltener verwendet.
Erst in der zweiten Hälfte des 20. Jahrhundert wurden die Würfelspiele wieder beliebter~\cite{DiplomarbeitG}.
\subsubsection{Aufbau}
Die meisten musikalischen Würfelspiele sind periodisch und gleichförmig.
Sie haben den Aufbau eines Walzers oder eines Menuetts.
Grundsätzlich verändert sich nur der Rhythmus und die Melodie,
der harmonische Teil bleibt dabei immer gleich.
Die Stücke sind meistens für mindestens ein Melodieinstrument,
ein Bassinstrument, sowie  ein weiteres Tasteninstrument komponiert.
Wenn man sich die Anzahl der verschiedenen Kompositionsmöglichkeiten ausrechnen möchte,
so müsste man eine einfache Rechnung aufstellen.
Bei einem Würfel hat man sechs Möglichkeiten (Augenzahlen); bei zwei hat man schon 11 Möglichkeiten.
Eine gewürfelte Zahl ist ein Takt.
Bei einem $n$-Taktiken Stück müsste man also nur
$6^n$ oder $11^n$ berechnen,
um sich die verschiedenen Kompositionsmöglichkeiten auszurechnen.
Nun gibt es aber auch Takte, welche identisch sind;
trotzdem ist die Anzahl der verschiedenen Taktfolgen sehr groß~\cite{DiplomarbeitG}.
\subsubsection{Ablauf}
Um ein solches Stück zu komponieren, braucht man einen Würfel,
eine vom Autor vorgegeben Tabelle, und entsprechende Notenblätter.
Jede Zahl die gewürfelt werden kann, bekommt eine bestimmte Sequenz
der Komposition zugeordnet.
Wird die Zahl gewürfelt, so wird diese Sequenz in die neue Komposition eingefügt.
Dieser Vorgang wird  mehrmals wiederholt, bis eine bestimmte Taktanzahl erreicht wird.
Der Würfelnde hat somit seine eigne Komposition erschaffen.
\subsubsection{Vorteile und Nachteile der Würfelspiele}
Nachteile der musikalischen Würfelspiele sind die limitierten Möglichkeiten des Komponisten.
Der Komponist, in dem Fall die würfelnde Person, ist eingeschränkt in seiner Kreativität.
Er würfelt lediglich und kann seine Ideen nicht einbringen.
Hingegen ist genau diese Einfachheit ein Vorteil für komplette Laien.
Ohne jegliches Vorwissen der musikalischen Notation
ist es dem Laien möglich, eine Komposition zu erstellen.
Jedenfalls ist der Prozess nicht sehr zeitaufwendig.
\subsubsection{Urheberrecht}
%\index{Urheberrecht}
Der Urheber ist in einer solchen Komposition ebenfalls nicht komplett klar.
Der einzige Urheber ist eigentlich derjenige, welcher die Vorlage schreibt,
denn der Würfelnde schreibt  eigentlich immer wieder die Vorlage eines Komponisten ab;
allerdings mit dem Unterschied, dass der Takt immer an einer anderen Stelle steht~\cite{DiplomarbeitG}.
\subsection{Komponieren nach einem Regelwerk}
%\index{Regelwerk}
Menschen, welche zum ersten Mal ein Musikstück komponieren wollen,
greifen oftmals zu einem Regelwerk. Dieses verspricht,
solange man es strikt befolgt, dass die Anweisungen eine komplette Komposition ergeben.
Ein Problem, das ein solches Regelwerk jedoch mit sich bringt,
ist, dass es in sogenannte ``Sackgassen'' führen kann.
Das heißt, dass man die
Komposition nicht immer weiterführen kann.
Um einer solchen Sackgasse aus
dem Weg zu gehen, wird ein Verfahren abgewandt, welches den Namen {\em Backtracking-Verfahren} hat.
Dabei
geht man bis zu dem letzten Moment zurück, bei dem man eine sinnvolle
Wahlmöglichkeit hatte. Parallelen zu diesem Verfahren und  der Problemlösungsmethode
{``Trial and Error''}  sind jedenfalls vorhanden.
Ein weiteres Problem ist, dass nicht alle Aspekte,
die man in seine Komposition inkludieren möchte, in Regeln gefasst sind.
Zum Beispiel führen die Stimmführungsregeln zwar zu einer richtigen Stimmführung,
aber sie führt nicht implizit zu einer richtigen Form der Komposition,
was dazu führt, das man eine neue Regeln finden muss, welche die richtige Form der Komposition herbeiführt.
Zum weiteren wird, weil man das Regelwerk genauestens befolgt,
die eigene Komposition immer nur ein ``Abklatsch'' des {``Originalen''}  sein.
Wenn man noch nie etwas komponiert hat, ist diese Methode jedoch empfehlenswert,
weil man somit in das Thema eingeführt wird und ohne einem großen Wissenstand eine Komposition erstellen
kann~\cite{DiplomarbeitG}.
\section{Schlussfolgerung und Ausblick}
In den obigen Kapiteln habe ich verschiedene Arten von Algorithmen erklärt;
und wie sie zur Erzeugung einer musikalischen Form, oder gar zur Erzeugung einer Komposition verwendet werden können.
Manche dieser Algorithmen sind, rein subjektiv betrachtet, wohl akustisch weniger schön, und manche sind annehmbar.
Das spannende an algorithmischen Kompositionen ist,
dass ein vollkommen automatisierter Prozess Klänge erzeugt.
Diese können gut klingen aber auch für das Menschliche Ohr eine Folter darstellen.
Algorithmen müssen erst programmiert werden, damit sie eine Komposition erstellen können.
In unserem Gehirn passieren diese Prozesse beim Komponieren normalerweise vollkommen automatisch.
Bei Algorithmen müssen im Vorhinein Regeln bestimmt und überlegt werden.
Bei uns Menschen sind die meisten Vorgänge vorhersehbar.
Zum Beispiel wird jemand, der klassische Musik hört, niemals ein  Heavymetal-Stück komponieren.
Der Computer gibt dem Ganzen schon eher einen gewisse zufällige Note;
aber dennoch habe ich mich immer öfter während der Arbeit gefragt: Was ist Zufall eigentlich?
\subsection{Algorithmische Kompositionen in der Zukunft}
Individualismus ist das, wonach die meisten Menschen streben.
Unternehmen wollen diesen Individualismus ihren Kunden bieten;
jedoch fällt es schwer, für jede Privatperson ein individuelles Stück komponieren zu lassen.
Meist sind diese dann noch urheberrechtlich geschützt.
Algorithmische Komposition könnte dabei nun einen Fortschritt herbeiführen.
Nun wäre dieser Fortschritt für einen produzierenden Musiker eher weniger vorteilhaft, dieser würde für seine Arbeit weniger Tantiemen bekommen.
Es sind viel weniger Menschen involviert bei Algorithmischen Komposition, im besten Fall nur einer.
Der Musiker braucht Leute die ihre Instrumente beherrschen, die ein Aufnahme Möglichkeiten haben und die fähig sind das aufgenommene ab zu mischen.
Ebenfalls generiert der Computer für jeden ein individuelles Stück während ein Musiker
für die breite Masse aufwendig produziert~\cite{DiplomarbeitG}.
Was viele Menschen unterschätzen, sind die Kosten eines Musikstücks,
und wie viel Arbeit hinter einem solchen Werk steht.
Würden nun all die Menschen, die an einem Lied arbeiten,
auf einen Computer reduziert werden, dann wäre das revolutionär für diverse Industrien.
Ob dieser Fortschritt für Komponisten und Musiker eine Bereicherung ist, wage ich
allerdings zu bezweifeln.
\subsection{Problematiken}
Um nun Problematiken meiner Arbeit anzusprechen,
hatte ich bei dem Versuchsablauf der rosa Musik anfangs das Problem,
das ich diesen Zugang nicht ganz versandt; und somit keinen Versuch zur Erzeugung solcher Musik
machen konnte.
Zum Glück hat sich mein Vater mit dem selbigen Thema ebenfalls befasst
und mir erklären können, wie ich rosa Musik erstellen kann.
Generell ist es sehr faszinierend, wie durch Befehle, also Abläufe, welche
man zuvor bestimmt, Musik entstehen kann.
Natürlich muss man vorweg hinnehmen, dass die Musik nicht unbedingt gut klingt;
aber dennoch entsteht eine Abfolge von Tönen, die der Computer (oder andere Generatoren) generiert.
\newpage
\addcontentsline{toc}{section}{Abbildungsverzeichnis}
\listoffigures
\newpage
\addcontentsline{toc}{section}{Tabellenverzeichnis}
\listoftables
\newpage
\bibliographystyle{plain}
\addcontentsline{toc}{section}{Literatur}
%\printbibliography
\bibliography{anna}




\end{document}
