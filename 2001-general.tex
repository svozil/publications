%%tth:\begin{html}<LINK REL=STYLESHEET HREF="/~svozil/ssh.css">\end{html}
\documentclass[pra,preprint,showpacs,showkeys,amsfonts]{revtex4}
\usepackage{graphicx}
%\documentstyle[]{article}
 \RequirePackage{times}
%\RequirePackage{courier}
\RequirePackage{mathptm}
%\renewcommand{\baselinestretch}{1.3}
\begin{document}

\def\ttt{{\rm Tr} }
\def\diag{{\rm diag} }
%\def\frak{\cal }
%\def\Bbb{\bf }
%\sloppy




\title{Proposition-based measure of irreducibe $n$-ary quantum information}
\author{Niko Donath and Karl Svozil}
 \email{svozil@tuwien.ac.at}
\homepage{http://tph.tuwien.ac.at/~svozil}
\affiliation{Institut f\"ur Theoretische Physik, University of Technology Vienna,
Wiedner Hauptstra\ss e 8-10/136, A-1040 Vienna, Austria}

\begin{abstract}
Quantum information
is radically different from classical information
in that the quantum formalism
makes necessary the introduction of irreducible ``nits,''
$N$ being an arbitrary natural number.
We define a measure of quantum information which
is invariant with respect to
unitary transformation and furthermore is additive for (nonentangled) product states.
\end{abstract}


\pacs{03.65.Ta,03.67.-a}
\keywords{quantum information theory,quantum measurement theory}


\maketitle

\section{Introduction}
As pointed out many times by Landauer and others
(e.g., \cite{landauer,feynman-computation})
the formal concept of {\em information} is tied
to physics,
at least as far as applicability is a concern.
Thus it should come as no surprise that quantum mechanics
requires fundamentally new concepts of information
as compared to the ones appropriate for classical physics.
And indeed, research into quantum information and computation theory
has exploded in the last decade, bringing about a wealth of new
ideas, potential applications, and formalisms.

There seems to be one issue,
which, despite notable exceptions (e.g., \cite[Footnote 6]{zeil-99} and
\cite{Muthukrishnan}),
has not yet been acknowledged widely:
the principal irreducibility of $N$-ary quantum information
associated with the $N$-dimensionality of Hilbert space.
A physical configuration allowing for $N$ possible outcomes
has to be encoded quantum mechanically by an $N$-dimensional Hilbert space.
Any single one of the $N$ basis vector corresponds
to a onedimensional subspace spanned by that basis vector
which in turn corresponds to the following physical proposition:
{\em ``the physical system is in a pure state corresponding to the basis vector.''}
In more operational terms, a particle can be prepared
in a single one of $N$ possible states.
This particle then carries the information to
``be in a single one from $N$ different states.''
Subsequent measurements may confirm this statement.
Thus the most natural code basis for this configutaion is an $N$-ary code,
and not a binary one.


Classically, there is no preferred code basis whatsoever.
Every classical state is postulated to be determined by a
point in phase space.
Formally, this amounts to an infinite amount of information in whatever base,
since
with probability one, all points are random; i.e., algorithmically incompressible.
\cite{chaitin2,calude:94}.
Operationally, only a finite amount of classical information is accessible.
Yet, in which particular base this finite amount of classical information is coded
is purely conventional and depends on the particular choice of the experimental setup.



Shannon's measure of information has proven to be adequate in a variety of classical applications.
For genuine quantum mechanical systems though, the measure has been criticised recently
by Brukner and Zeilinger \cite{zeil-bruk-01}
mainly on the basis of its assumption of pre-existing, noncontextual elements
of physical reality, its ignorance of complementarity,
and its nonpreservation under unitary transformations.
In view of this criticism, we suggest a new measure of information which is based
on the operational arguments given by Brukner and Zeilinger  \cite{zeil-bruk-99a,zeil-bruk-01}
and has additional desirable properties; most notably separability for
product states.
Alternative measures have also been proposed by Rehacek and Hradil \cite{Reha-Hradil}.

Before discussing the quantum measure of information, let us mention some notation first.
Consider a particle which can be observed in a single one out of
a finite number
$N$ of possible operationally distinct and comeasurable properties.
The system's state is
formalized by a state operator (which is self-adjoint, positive and of
trace one)
$\rho$ of the $N$-dimensional Hilbert space. Every  operationally
distinct
property corresponds to the proposition $E_i$, $1\le i\le N$ that the
system, when measured, has the associated property.
By definition, propositions are dichotomic observables, since
$E_i^2=E_i$ is only satisfied for the eigenvalues $0$ and $1$
\cite{v-neumann-49}.
Any such complete, minimal, comeasurable and operationally distinct set
of $N$ propositions $E_i$ corresponds to an orthonormal set of base
states whose linear span is the $N$-dimensional Hilbert space.

The $n$ particle case can be formalized
by states $\rho$ in $N^n$-dimensional Hilbert spaces.
Any pure state
($\rho^2=\rho$ or equivalently, $\ttt (\rho )=1$, whether ``entangled'' or not)
can be obtained in two steps.
In the first step,
a system of pure basis vectors ${\cal B} = \{\rho_1,\ldots ,\rho_{N^n}\}$
is formed by taking the tensor products
$\rho^s_{i_1}\otimes \rho^s_{i_2}\otimes \cdots \otimes\rho^s_{i_n}$,
$1\le i_1,i_2,\ldots ,i_n \le N$,
of all the single-particle
projection operators
$\rho^s_{i_1}$, $\rho^s_{i_2}\ldots \rho^s_{i_n}$.
In the second step, this system of basis vector undergoes a unitary transformation;
i.e., consider the vector ${\bf B}=(\rho_1,\ldots ,\rho_{N^n})$
associated with the basis ${\cal B}$ and transform it by some
unitary
transformation $U(N^n)$ represented by a $N^n \times N^n$-matrix ${\bf U}$; i.e.,
\begin{equation}
\begin{array}{llll}
{\bf B} &\mapsto &{\bf B}' =   {\bf U} {\bf B} ,\\
\rho_i &\mapsto &\rho_i' = \sum_{j=1}^{N^n}{\bf U}_{ij}\rho_j, \; 1\le i\le N^n.
\end{array}
\label{2001-general-e0}
\end{equation}
The resulting system of base states ${\cal B}'$ is the new, transformed system of
``entangled'' (with respect to the previous base system composed from product states of single particle states)
base states.
Thus, at least formally, ``entangled'' state bases are
unitary equivalent.
As a consequence,  propositions, in particular optimally separating
ones, need no longer refer to attributes
or properties of single particles alone, but to joint properties of $k\le n$
particles
\cite{zeil-99,DonSvo01}.

In what follows, let us always consider a complete
system of base states  ${\cal B}$ associated with a unique ``context'' \cite{svozil-2001-cesena}
or ``communication frame''
 ${\cal F}=\{F_1,F_2, \ldots ,F_{n} \}$, which are a maximal set of
co-measurable $N$-ary observables.
For $N=2$,
their explicit form has been enumerated in  \cite{DonSvo01}.
In this particular case, the $F$'s can be identified with projection
operators,
whose two eigenvalues can be identified with the two states.
For three or more particles, this is no longer possible (see below).
A well-known theorem of linear algebra
(e.g.,
\cite{halmos-vs})
states that a single measurement exists which determines
all the observables associated with the context
$\{E_1,E_2, \ldots ,E_{n} \}$ at once.
The term ``communication frame'' is used here
as a synonym for a complete set of observables corresponding
to some pure orthonormal basis states.


\section{The case of two particles with three states}

Before proceeding to the most general case,
we shall consider the case of two particles
with three states per particle in all details.
The methods  developed \cite{DonSvo01} for case of $n$
particles
with two states per particle cannot be directly adopted here,
since the idempotence ($E_iE_i=E_i$) of the projection operators,
which maps the two states onto the two possible eigenvalues $0,1$ cannot
be generalized.

Instead we shall start by considering the optimal partitions of
states and construct the appropriate Hilbert space operators from there.
Assume that the first and second particle
has three orthogonal states labeled by
$a_1,b_1,c_1$
and
$a_2,b_2,c_2$,
respectively
(the subscript denotes the particle number).
Then nine product states can be formed and labeled from $1$ to $9$ in
lexicographic order; i.e.,
\begin{equation}
\begin{array}{llll}
a_1a_2 &\equiv&1,\\
a_1b_2 &\equiv&2,\\
a_1c_2 &\equiv&3,\\
b_1a_2 &\equiv&4,\\
  &\vdots& \\
c_3c_3 &\equiv&9.\\
\end{array}
\label{2001-general-ps3}
\end{equation}

Any maximal set of
co-measurable $3$-valued observables induces two state partitions
of the set of states $S=\{1,2,\cdots , 9\}$ with three partition
elements with the properties
that (i) the set theoretic intersection of any two elements of the two
partitions is a single state, and (ii) the union of all these nine
intersections is just the set of state $S$.
As can be easily checked, an example for such state partitions are
\begin{equation}
\begin{array}{llll}
F_1&=&\{\{1,2,3\},\{4,5,6\},\{7,8,9\}\},\\
F_2&=&\{\{1,4,7\},\{2,5,8\},\{3,6,9\}\}.\\
\end{array}
\label{2001-general-ps3e}
\end{equation}
Operationally, the ``trit'' $F_1$ can be obtained by
measuring the first particle state:
$\{1,2,3\}$ is associated with state $a_1$,
$\{4,5,6\}$ is associated with $b_1$,  and
$\{4,5,6\}$ is associated with $c_1$.
The ``trit'' $F_2$ can be obtained by
measuring the state ot the second particle:
$\{1,4,7\}$ is associated with state $a_2$,
$\{2,5,8\}$ is associated with $b_2$,  and
$\{3,6,9\}$ is associated with $c_2$.

A Hilbert space representation of this setting can be obtained as
follows.
Define the states in $S$ to be onedimensional linear subspaces of
nine-dimensional Hilbert space; e.g.,
\begin{equation}
\begin{array}{llll}
1 &\equiv& (1,0,0,0,0,0,0,0,0),\\
2 &\equiv& (0,1,0,0,0,0,0,0,0),\\
3 &\equiv& (0,0,1,0,0,0,0,0,0),\\
4 &\equiv& (0,0,0,1,0,0,0,0,0),\\
  &\vdots&\\
9 &\equiv& (0,0,0,0,0,0,0,0,1).\\
\end{array}
\label{2001-general-ps3a}
\end{equation}
The trit operators are given by (trit operators, observables and the
corresponding state partitions will be used synonymously)
\begin{equation}
\begin{array}{llll}
F_1&=& \diag (a,a,a,b,b,b,c,c,c),\\
F_2&=& \diag (a,b,c,a,b,c,a,b,c),\\
\end{array}
\label{2001-general-ps3top}
\end{equation}
for $a,b,c \in {\Bbb R}$, $a\neq b\neq c\neq a$.

If $F_2= \diag (d,e,f,d,e,f,d,e,f)$
and $a,b,c,d,e,f,$ are six different prime numbers,
then, due the uniqueness of prime decompositions,
the two trit operators
can be combined to a single
``context'' operator
\begin{equation}
C=F_1\cdot F_2=F_1\cdot F_1=
\diag (ad,ae,af,bd,be,bf,cd,ce,cf)
\label{2001-general-ps3pd}
\end{equation}
which acts on both particles and has nine different eigenvalues.



This Hilbert representations opens the way to more general ``entangled''
states and the optimal trit configurations for them by
unitary transformation $U(9)$.

\section{The general case}
A generalization to $n$ particles in $N$ states per particle is straightforward.
We obtain $n$ partition of the product states with $N$ elements per partition in such a way that
the every single product state is obtained by the set theoretic intersection of
$n$ elements of different partitions.
Every such partition can be interpreted as a nit.


\section{Measure of information}
We propose as the
measure of information for  $n$ $N$-state particles
\begin{equation}
I[\rho ,{\cal B} ]={\cal N}\sum_{i=1}^{N^n} \left(\ttt (E_i[{\cal B}]\;\rho) - {1 \over N} \right)^2,
\label{2001-general-e1}
\end{equation}
The sum extends over {\em all} projection operators $E_i$ which span the nits.
The additive constant $-1/N$ serves as a zero offset.
That is, a complete lack of information in a single experiment corresponds to  $p_i=1/2$ and yields $I=0$ \cite{zeil-99}.
${\cal N}$ stands for a normalization factor.

As both $\rho$ and $E_i$ behave in the same way under unitary transformations, the trace
and thus the measure (\ref{2001-general-e1}) is invariant under unitary transformations.
It should be stressed that the measure introduced in (\ref{2001-general-e1})
not only depends on the state being measured, but also on the questions being asked.
If there is a mismatch between these two, the resulting information gain may be zero,
although the state is pure; i.e., encodes  $n$ nits.
Thus, the communicating parties should agree beforehand, which communication frames are used.
This is a feature and necessity of quantum complementarity,
and is not changed by unitary transformations at all.




In the case of nonpure states (or a mixed set $R$ of pure and nonpure states)
we suggest
to attempt to decompose the  states $R =\{\rho^1,\ldots ,\rho^r\}$
into a mutually orthonormal set of pure base states  $\{ \psi_1, \ldots \psi_n\}$
such that
\begin{equation}
\rho^j =\sum_{i=1}^N p^j_i \vert \psi_i \rangle \langle \psi_i \vert ,
\quad 1\le j\le r.
\end{equation}
According to a recent result due to Nielsen  \cite{nielsen-2000}, this is only possible
if the vector formed by $p^j=(p^j_1,\ldots , p^j_N)$ is majorized by the
vector $\lambda^j =(\lambda^j_1, \ldots ,\lambda^j_n)$ of eigenvalues of $\rho^j$.
Thereby, we postulate to choose a set of base states which minimizes
the sum of the distances  between $\lambda^j$ and $p^j$.
This guarantees that the previously discussed case of a communication frame
with pure orthonormal base states is contained in the nonpure case.
If no such consistent decomposition of $R$ is possible, we can still
attempt to reconstruct $R$ by sums over noncommuting pure base states.
The worst case scenario is obtained by choosing
an equidecomposition of
states into at most $2^n$ pure ones \cite{nielsen-2000}.

\section{Two-state particles}

In what follows we shall consider a few cases for two-state particles.
In the case $n=1$, $N=2$ of a single  two-state particle,
Eq. (\ref{2001-general-e1-2}) reduces to $ [\ttt (E_1\rho)- 1/2]^2+[\ttt (E_2\rho)- 1/2]^2$.

For $n$ $2$-state particles.
Eq. (\ref{2001-general-e1}) reduces to
\begin{equation}
I[\rho ,{\cal B} ]= {\cal N}\sum_{i=1}^{2^n} \left(\ttt (E_i[{\cal B}]\;\rho) - {1 \over 2} \right)^2.
\label{2001-general-e1-2}
\end{equation}

\subsection{Zero information, worst case scenarios}

(i)
Total ignorance about the quantum state is formalized by a quantum
state which is proportional to the $2^n$-dimensional unit matrix ${\Bbb I}$; i.e., $\rho=(1/2^n){\Bbb I}$.
In such a case, any communication frame yields zero information measure.

(ii)
Assume the sender ``Alice'' wants to transfer a message to
the receiver ``Bob'', but Bob uses a different communication frame than Alice.
In this case, Bob irreversibly scrambles Alice's message due to complementarity.
In the worst case scenario, Alice's message
is totally randomized due to Bob's mismatch of her communication frame.
In this case, Bob and Alice use ``unbiased'' bases;
i.e., two sets of pure base states
$\vert A,\alpha \rangle$
and
$\vert B,\beta \rangle$,
$\alpha , \beta =1,  2$ such that
$\vert \langle A,\alpha \vert  B,\beta  \rangle \vert^2 = 1/2$.
For an example, take the two communication frames $\{\diag (0,1),\diag (1,0)\}$
and $\{({\Bbb I}+\sigma_1)/{2},({\Bbb I}+\sigma_1)/{2}  \}$,
yielding
$
I[\diag(0,1),\{({\Bbb I}+\sigma_1)/{2},({\Bbb I}+\sigma_1)/{2} \}]=
I[\diag(1,0),\{(({\Bbb I}+\sigma_1)/{2},({\Bbb I}+\sigma_1)/{2}\}]=0$.

Sender and receiver may stick with coding every bit into each particle separately;
i.e., one bit per particle.
But they may also decide to use entanglement, thereby
using effectively a basis set
which is rotated against the standard Cartesian basis
(recall that all basis states are unitary equivalent).
In the latter case, only measurements of joint properties
suffice to separate all states from one another.

\subsection{Additivity of information for product state}
Consider the case of two two-state
particles which are in a pure product state $\rho= \rho_{i_1}\otimes \rho_{i_2}$; i.e.,
$N=n=2$.
The optimal set of propositions has been discussed in
\cite{DonSvo01}.
For $n=2$ and a base set of product states, it consists of
%the unitary transforms of
the two projection operators
$E_1={\rm diag}(1,1,0,0)$
and
$E_2={\rm diag}(1,0,1,0)$.
The summation extends over all projection operators; i.e., also over
$E_3={\rm diag}(0,1,0,1)$,
$E_4={\rm diag}(0,0,1,1)$.
%The exact form of the unitary transformation
%is determined by the requirement that they transform the
%base states formed by the cartesian product states into
%a set of base states containing $\rho$.
In this case,  Eq. (\ref{2001-general-e1-2}) reduces to
\begin{equation}
I[\rho ]=
\left( \ttt (E_1 \rho)-{1\over 2} \right)^2
+ \left( \ttt (E_2 \rho)-{1\over 2}\right)^2
+ \left( \ttt (E_3 \rho)-{1\over 2}\right)^2
+ \left( \ttt (E_4 \rho)-{1\over 2}\right)^2
.
\end{equation}
Since
\begin{equation}
\rho=  \rho_{i_1}\otimes \rho_{i_2} =
\left(
\begin{array}{cccc}
(\rho_{i_1})_{11} (\rho_{i_2})_{11}&&&\cdots \\
&(\rho_{i_1})_{11} (\rho_{i_2})_{22}& &\\
&&(\rho_{i_1})_{22} (\rho_{i_2})_{11} &\\
\cdots &&&(\rho_{i_1})_{22} (\rho_{i_2})_{22}\\
\end{array}
\right).
\label{2001-general-xxx}
\end{equation}
Thus
\begin{equation}
\begin{array}{llll}
\ttt (E_1\rho)&=&(\rho_{i_1})_{11}(\rho_{i_2})_{11}+(\rho_{i_1})_{11}(\rho_{i_2})_{22}=(\rho_{i_1})_{11}\ttt \rho_{i_2}=(\rho_{i_1})_{11},\\
\ttt (E_2\rho)&=&(\rho_{i_2})_{11},
\quad
\ttt (E_3\rho)=(\rho_{i_2})_{22},
\quad
\ttt (E_2\rho)=(\rho_{i_1})_{22},
\end{array}
\label{2001-general-xxx1}
\end{equation}
and finally
\begin{equation}
\begin{array}{llll}
I[\rho_{i_1} \otimes \rho_{i_2} ]
&=&
\left((\rho_{i_1})_{11}-{1\over 2}\right)^2+
\left((\rho_{i_2})_{11}-{1\over 2}\right)^2+
\left((\rho_{i_2})_{22}-{1\over 2}\right)^2+
\left((\rho_{i_1})_{22}-{1\over 2}\right)^2
\\
&=&
\left(\ttt (F_1 \rho_{i_1})-{1\over 2}\right)^2+
\left(\ttt (F_1 \rho_{i_2})-{1\over 2}\right)^2+
\left(\ttt (F_2 \rho_{i_2})-{1\over 2}\right)^2+
\left(\ttt (F_2 \rho_{i_2})-{1\over 2}\right)^2
\\
&=&I[\rho_{i_1}]+I[\rho_{i_2}],
\end{array}
\end{equation}
where $F_1=\diag (1,0)$ and $F_2=\diag (0,1)$
are the corresponding single-particle projection operators.
This is the desired property that the information measure of
two particles in a product state adds up the single state information.
By induction it can be shown that this additivity holds for
the general case of $n$ two-state systems.



\section{Conclusion}
We introduced a general framework to calculate the information gained by any possible experiment.
This framework includes the recently proposed
measure of quantum information of Zeilinger and Brukner as well as Zeilinger's foundational principle and may easily be extended to the case of three-level-systems.

By a well known theorem for unitary operators
\cite{murnaghan},
any quantum measurement of an $n$-ary system can be decomposed
into binary measurements.
Also, it is possible to group the $n$ possible outcomes into binary filters of
ever finer resolution; calling the successive outcomes of these filter process
the ``binary code.''
Yet, all these attempts result in codes with undesirable features.
Unitary decompositions in general yield non-comeasurable observables and thus to
non-operationalizability. Filters are inefficient, and so may be binary codes
\cite{Cal-Cam-96}.

So far, the main emphasis in the area of quantum computation
has been in the area of binary decision problems.
It is suggested that these investigations should be extended to
$n$-ary decision problems (e.g., \cite[pp. 332-340]{kleene-52}),
for which quantum information theory seems
to be extraordinarily well equipped.



\bibliography{svozil}
\bibliographystyle{apsrev}

%\begin{thebibliography}{99}
%\bibitem{WooFie} W.~K.~Wootters and B.~D.~Fields: \emph{Optimal State-Determination by Mutually Unbiased Measurements}, Ann.Phys. 191,363-381 (1989)
%\end{thebibliography}
\end{document}
