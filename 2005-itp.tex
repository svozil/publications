\documentclass[pra,amssymb,showpacs,showkeys,preprint]{revtex4}
\usepackage[T1]{fontenc}
\usepackage{textcomp}
\usepackage{graphicx}
%\documentstyle[amsfonts]{article}
\RequirePackage{times}
%\RequirePackage{courier}
\RequirePackage{mathptm}
%\renewcommand{\baselinestretch}{1.3}
\RequirePackage[german]{babel}
\selectlanguage{german}
\RequirePackage[isolatin]{inputenc}
\begin{document}
%\sloppy


$\,$\\
http://tph.tuwien.ac.at/~svozil/publ/2005-itp.pdf
\\
http://tph.tuwien.ac.at/~svozil/publ/2005-itp.tex


\section{Quanteninformationstheorie}

Die Quanteninformationstheorie ist ein relativ junges, au�erordentlich aktives Forschungsgebiet
der Quantenphysik, dessen Haupteigenschaft die Verwendung einzelner Teilchen
zur Informationsbearbeitung
\cite{DonSvo01,svozil-2003-garda,svozil-2002-statepart-prl}
und verschl�sselten Informations�bertragung ist.
Hierbei werden verschiedene nichtklassische Eigenschaften der Quantenphysik ausgenutzt,
wie etwa die M�glichkeit, verschiedene Zust�nde, die klassisch nur konsekutiv bearbeitet
werden k�nnen, durch Superposition parallel zu entwickeln, was f�r gewisse
Optimierungsaufgaben und
Entscheidungsprobleme eine exponentielle zeitliche
Beschleunigung erg�be.
Weitere Eigenschaften von Quanteninformation sind in der
Komplementarit�t begr�ndet, sowie in der Nichtkopierbarkeit, und der m�glichen Verschr�nkung
von Vielteilchenzust�nden.

Schon Feynman \cite{feynman-computation,nielsen-book} hat darauf hingewiesen,
dass Entwicklung und Simulation von Quanteninformationsalgorithmen und Protokollen
auf klassischen Computern einen hohen ``r�umlichen'' Aufwand in Bezug
auf den erforderlichen Arbeitsspeicher
und damit die Rechnerleistung generell erfordern.
Dies bedingt au�erordentlich gro�e Ressourcen f�r die Entwicklung und �berpr�fung derselben.


\subsection{Entwicklung von Quantenalgorithmen}

Eine h�ufige Methode bei Quantenalgorithmen ist die Nutzung von gleich gewichteten
�berlagerungen von Vielteilchenzust�nden; ausgehend von einfach darzustellenden
Zust�nden durch verallgemeinerte Hadamardtransformationen.
In der Folge werden dann ``parallel'' die Einzelterme bearbeitet; und im letzten Schritt
per verallgemeinerter Hadamardtransformationen und einer darauf folgenden Messung
die Ergebnisse ausgelesen. Die Simulation dieser Prozesse erfordert einen
klassischen Simulationsaufwand,
welcher exponentiell mit der Anzahl der beteiligten Quantenbits steigt.
Eine interessante neue Entwicklung ist die Quantenberechnung durch Cluster-states
\cite{wrrswvaz-05}, deren Simulation ebenfalls einen erheblichen Aufwand erfordern.


Eines der Ziele ist dabei die Beschleunigung klassischer Optimierungs-
und Entscheidungsprobleme. Hier gibt es ganze Klassen von Problemstellungen,
deren funktionales Verhalten noch nicht aufgearbeitet wurden \cite{svozil-2002-statepart-prl}.
Ein weiteres Ziel ist die Stabilisierung von sehr kleinen Systemen, bei denen Quanteneffekte eine
Rolle spielen. Dieses Problemfeld wird mit der steigenden Miniaturisierung und
Integration von elektronischen Bauteilen immer wichtiger.
Entsprechende Simulationen sollen in beiden Forschungsfeldern
zu quantitativen Fortschritten f�hren.


\subsection{Entwicklung von quantenkryptographischen Protokollen}

Ziel der Quantenkryptographie ist die Verschl�sselung von Botschaften
oder das Erstellen und Vergr��ern von geheimen gleichen Zufallszahlen zwischen zwei
r�umlich getrennten Agenten.
Dies wird durch Elementarquanten, zum Beispiel einzelne Photonen erm�glicht,
welche in einem Quantenkanal �bermittelt werden.

Besonderes Augenmerk soll hier insbesondere auf die Entwicklung
Eutaktischer Codes \cite{svozil-2003-eu} und den darauf aufbauenden
quantenkryptographischen Protokollen
mit optischer Interferenz (Mach-Zehnder-Interferometer) gelegt werden.

Ein weiterer Forschungsschwerpunkt ist die Abh�rsicherheit quantumkryptographischer
Protokolle, insbesondere gegen�ber so genannter ``man-in-the-middle''-Angriffe;
etwa unter zu Hilfenahme von zeitlichen Interlocks bei der Signal�bertragung.
Auch in diesen F�llen ist die Simulation sehr aufwendig
und nur �ber gro�e Rechenkapazit�ten zu erreichen.

\subsection{Quanten-Korrelationspolytope}
In diesem Zusammenhang wichtig sind auch die Extremaleigenschaften
\cite{filipp-svo-04-qpoly,filipp-svo-04-qpoly-prl}, gewisser
Ungleichungen \cite{2000-poly,2001-cddif}, welche zur G�temessung von Quantenkan�len herangezogen werden
k�nnen. Dieses Problem is NP-vollst�ndig und bedarf, soweit �berhaupt l�stbar,
sehr gro�er Rechnerresourcen.

---> MAX KREUZER?

%\narrowtext
\bibliography{svozil}
\bibliographystyle{apsrev}

\end{document}


