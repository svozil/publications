%\documentclass[pra,showpacs,showkeys,amsfonts,amsmath,twocolumn]{revtex4}
\documentclass[amsmath,table,sans,amsfonts, handout]{beamer}
%\documentclass[pra,showpacs,showkeys,amsfonts]{revtex4}
\usepackage[T1]{fontenc}
%%\usepackage{beamerthemeshadow}
%%\usepackage[headheight=1pt,footheight=10pt]{beamerthemeboxes}
%%\addfootboxtemplate{\color{structure!80}}{\color{white}\tiny \hfill Karl Svozil (TU Vienna)\hfill}
%%\addfootboxtemplate{\color{structure!65}}{\color{white}\tiny \hfill mur.sat \hfill}
%%\addfootboxtemplate{\color{structure!50}}{\color{white}\tiny \hfill Graz, 2010-12-11\hfill}
%\usepackage[dark]{beamerthemesidebar}
%\usepackage[headheight=24pt,footheight=12pt]{beamerthemesplit}
%\usepackage{beamerthemesplit}
%\usepackage[bar]{beamerthemetree}
\usepackage{graphicx}
\usepackage{pgf}
%\usepackage{eepic}
%\usepackage[usenames]{color}
%\newcommand{\Red}{\color{Red}}  %(VERY-Approx.PANTONE-RED)
%\newcommand{\Green}{\color{Green}}  %(VERY-Approx.PANTONE-GREEN)

%\RequirePackage[german]{babel}
%\selectlanguage{german}
%\RequirePackage[isolatin]{inputenc}

%\pgfdeclareimage[height=0.5cm]{logo}{tu-logo}
%\logo{\pgfuseimage{logo}}
\beamertemplatetriangleitem
%\beamertemplateballitem

\beamerboxesdeclarecolorscheme{alert}{red}{red!15!averagebackgroundcolor}
%\begin{beamerboxesrounded}[scheme=alert,shadow=true]{}
%\end{beamerboxesrounded}

%\beamersetaveragebackground{yellow!10}

%\beamertemplatecircleminiframe

\begin{document}

\title{\bf \textcolor{blue}{Quantum Recursion Theory}}
%\subtitle{\textcolor{blue}Very Few Answers, Many Questions}
\subtitle{\textcolor{orange!60}{\small http://tph.tuwien.ac.at/$\sim$svozil/publ/2012-Cagliari-06-08-pres.pdf
%\\
%http://arxiv.org/abs/quant-ph/0503229v4
}}
\author{Karl Svozil}
\institute{ University of Technology Vienna and The University of Cagliari \\
Wiedner Hauptstra\ss e 8-10/136, A-1040 Vienna, Austria\\
svozil@tuwien.ac.at
%{\tiny Disclaimer: Die hier vertretenen Meinungen des Autors verstehen sich als Diskussionsbeitr�ge und decken sich nicht notwendigerweise mit den Positionen der Technischen Universit�t Wien oder deren Vertreter.}
}
\date{Cagliari, June 8th, 2012}
\maketitle


%\frame{
%\frametitle{Contents}
%\tableofcontents
%}


\section{Classical brightness of the Thomson lamp}


\frame{
\frametitle{Thomson process}
The Thomson process can formally be described at {\em intrinsic} (unsqueezed, unaccelerated)
discrete time steps $t$; i.e., by the partial sum
$$
s(t)=\sum_{n=0}^t (-1)^n = \left\{
\begin{array}{ll}
1 &\textrm{for even $t$,}
\\ 0 &\textrm{for odd $t$,}
\end{array}
\right.
$$
which can be interpreted as the result of all switching operations until time $t$.
The intrinsic time scale is related to an {\em extrinsic} (squeezed, accelerated) time scale
$$
 \tau_0=0,\;
 \tau_{t>0}=\sum_{n=1}^t 2^{-n}=2\left(1-2^{-t}\right)\quad .
$$
In the limit
 of infinite intrinsic time $t\rightarrow \infty$, the proper time
 $\tau_\infty = 2$ remains finite.
}


\frame{
\frametitle{Abel sum}
If one encodes the physical states of the Thomson lamp by ``0'' and ``1,''
associated with the lamp ``on'' and ``off,'' respectively, and the switching process with the concatenation of ``+1'' and ``-1'' performed so far, then
the divergent infinite series associated with the Thomson lamp is
the Leibniz series
$$
s = \sum_{n=0}^\infty (-1)^n=1-1+1-1+1-\cdots \stackrel{{\rm A}}{=} \frac{1}{2}.
$$
Here, ``A'' indicates the Abel sum obtained from a ``continuation'' of the geometric series, or alternatively, by
$s= 1-s$.
As this shows, formal summations of the Leibnitz type  require specifications
which could make them unique.
}


\frame{
\frametitle{Physical interpretation I}
In principle, the Thomson lamp could be perceived as a physical process governed by some associated differential equation,
such as $$y'(1-z)-1=0,$$ which has an exact solution $$f(z)=\log (1/1-x)$$
as well as a divergent series solution  $${\hat{f}}(z)=\sum_{n=0}^\infty (-z)^n/n$$
so that the first derivative of ${\hat{f}}(z)$, taken at $z=1$, yields the Leibniz series $s$.
}


\frame{
\frametitle{Physical interpretation II: Justification of the Abel sum}


The mean  brightness of the Thomson lamp can be operationalized by measuring the {\em average} time of the lamp to be ``on'' as compared to
being ``off'' in the time interval determined by the  temporal resolution of the measuring device.
This definition is equivalent to an idealized ``photographic'' plate or light-sensitive detector which additively collects light during the time the shutter is open.
Then, such a device would record the cycles of Thomson's lamp up to temporal resolution $\delta$, and would then integrate the remaining light emanating from it.
Suppose that $\delta = 1/2^t$, and that at the time $\sum_{n=1}^t 2^{-n}=2\left(1-2^{-t}\right)$, the Thomson lamp is in the ``off'' state.
Then the ``on'' and ``off'' periods sum up to
$$
\begin{array}{lclllllllllllllllllll}
s_0&=&  &    \frac{1}{4}   &+&    \frac{1}{16} &+&    \frac{1}{64}  &+& \cdots  &=&   \frac{1}{3}   \\
s_1&=&\frac{1}{2}   &+&    \frac{1}{8} &+&    \frac{1}{32}     &+& \cdots &\cdots  &=&   \frac{2}{3} \; .
\end{array}
$$
}

\frame{
\frametitle{Physical interpretation II: Justification of the Abel sum cntd.}


By averaging over the initial time, there is a 50:50 chance that the Thomson lamp is ``on'' or ``off'' at highest resolution --- which is equivalent to
repeating the experiment with different offsets -- resulting in an average brightness of $1/2$, which is identical to the Abel sum.
In this way, the Abel sum can be justified from a physical point of view.
The use of the Abel sum may not be perceived as totally convincing, as the Abel sum can be affected by changes to an initial segment
of the series.

In a strict sense, this classical physics treatment of the brightness of the Thomson lamp is of little use for predictions of the state of the Thomson lamp {\em after} the limit of the switching cycles.

}

\section{Quantum state of the Thomson lamp}
\frame{
\frametitle{Quantum state of the Thomson lamp}
Let $\vert 0\rangle = (1,0)$ and $\vert 1\rangle = (0,1)$ be the representations of the ``off'' and ``on''
states of the Thomson lamp, respectively.
Then, the switching process can be symbolized by the ``not'' operator
$\textsf{\textbf{X}}=
\left(
\begin{array}{cc}
0&1\\
1&0
\end{array}
\right)$, transforming $\vert 0\rangle$ into $\vert 1\rangle$ and {\it vice versa}.
Thus the quantum switching process of the Thomson lamp at time $t$ is the partial product
$$
\textsf{\textbf{S}}(t)
= \prod_{n=0}^t
\textsf{\textbf{X}}^t=   \left\{
\begin{array}{ll}
{\Bbb I}_2 &\textrm{for even $t$,}
\\ \textsf{\textbf{X}} &\textrm{for odd $t$.}
\end{array}
\right.
$$
In the limit, one obtains an infinite product $\textsf{\textbf{S}}$ of matrices with the two accumulation points mentioned above.

}


\frame{
\frametitle{Quantum state of the Thomson lamp cntd.}
The eigensystem of $\textsf{\textbf{S}}(t)$ is given by the two 50:50 mixtures of $\vert 0\rangle $ and $\vert 1\rangle $
with the two eigenvalues $1$ and $-1$; i.e.,
$$
\textsf{\textbf{S}}(t)\frac{1}{\sqrt{2}}\left( \vert 0\rangle \pm \vert 1\rangle   \right)
= \pm \frac{1}{\sqrt{2}}\left( \vert 0\rangle \pm \vert 1\rangle   \right) = \pm \vert \psi_\pm \rangle.
$$
In particular,  the state   $\vert \psi_+ \rangle$ associated with the eigenvalue $+1$
is a {\em fixed point} of the operator $\textsf{\textbf{S}}(t)$.
These are the two states which can be expected to emerge as the quantum state of the Thomson lamp
in the limit of infinity switching processes.
Note that, as for the classical case and for the formal Abel sum, they represent a fifty-fifty mixture of the ``on'' and ``off'' states.

}


\section{Quantum Fixed point of diagonalization operator}

\frame{
\frametitle{Hypothetical halting algorithm}

For the sake of contradiction,
assume that there exists a universal computer $U$ and an arbitrary algorithm
$B(X)$ whose input is a string of symbols $X$.  Assume that there exists
a ``halting algorithm'' $ h $ which is able to decide whether $B$
terminates on $X$ or not.

The domain of $ h $  is the set of legal programs.
The range of $ h $ are classical bits.
$ h $ outputs the code of a classical bit as follows:
$$
 h  ( B(X) ) =     \left\{
\begin{array}{ll}
0 &\textrm{ whenever $B(X)$ does not halt},  \\
1 &\textrm{  whenever $B(X)$ halts}.
\end{array}
\right.
$$
}

\frame[shrink=2]{
\frametitle{Classical recursion theory: algorithmic proof of the undecidability of the halting (forecasting) problem}

$\;$\\
\scriptsize

Using $ h (B(X))$ we shall construct another deterministic
computing agent $A$, which has as input any effective program $B$ and
which proceeds as follows: \\
$\;$\\
Upon reading the program $B$ as input, $A$
makes a copy of it.  This can be readily achieved, since the program $B$
is presented to $A$ in some encoded form
$\ulcorner B\urcorner $,
i.e., as a string of
symbols.  In the next step, the agent uses the code
$\ulcorner B\urcorner $
 as input
string for $B$ itself; i.e., $A$ forms  $B(\ulcorner B\urcorner )$,
henceforth denoted by
$B(B)$.  The agent now hands $B(B)$ over to its subroutine $ h $.
Then, $A$ proceeds as follows:  if $ h (B(B))$ decides that
$B(B)$ halts, then the agent $A$ does not halt; this can for instance be
realized by an infinite {\tt DO}-loop; if $ h (B(B))$ decides
that $B(B)$ does {\em not} halt, then $A$ halts. \\
$\;$\\
The agent $A$ will now be confronted with the following paradoxical
task:  take the own code as input and proceed to determine whether or not it halts.
Then, whenever $A(A)$
halts, $ h (A(A))$, by the definition of $A$, would force $A(A)$ not to halt.
Conversely,
whenever $A(A)$ does not halt, then $ h (A(A))$ would steer
$A(A)$ into the halting mode.  In both cases one arrives at a complete
contradiction.  \\
$\;$\\
Classically, this contradiction can only be consistently
avoided by assuming the nonexistence of $A$ and, since the only
nontrivial feature of $A$ is the use of the peculiar halting algorithm
$ h $, the impossibility of any such halting algorithm.


}

\frame{
\frametitle{Quantum Fixed point of diagonalization operator}
Suppose that the algorithm is not
restricted to classical bits of information.
Then
$$\vert \psi_+ \rangle = ({1}/{\sqrt{2}})\left( \vert 0\rangle + \vert 1\rangle   \right)$$
is the quantum fixed point state of the ``not'' operator,
which is essential for diagonal arguments, as
$$
\textsf{\textbf{X}}\vert \psi_+ \rangle = \vert \psi_+ \rangle
.
$$
Thus in quantum recursion theory,
the diagonal argument consistently goes through without leading to a contradiction,
as $ h (A(A))$ yielding $\vert \psi_+ \rangle$ still allows a consistent response
of $A$ by a coherent superposition of its halting and non-halting states.
}



\begin{frame}[fragile]
\frametitle{Is quantum jelly ontic or epistemic; and how can you possibly profit from epistemic uncertainty?}

E. Schr�dinger,  Die gegenw�rtige Situation in der Quantenmechanik.
Naturwissenschaften, 23, 807--812, 823--828,
844--849 (1935). \\
$\,$\\
E. Schr�dinger,  The Interpretation of Quantum Mechanics. {D}ublin Seminars (1949-1955) (Ox Bow Press, 1995)

\begin{center}
\includegraphics[width=70mm]{2012-WTCS2012-pres-schroedinger_0073_r.jpg}
% Erwin Schr�dinger and friends at the "baptism of the 'wave packet'," June 21, 1931. Reprinted with permission of Ruth Braunizer.
\end{center}


\end{frame}

\section{Quantum diagonalization}

\frame{
\frametitle{Quantum diagonalization}
Allow for
{\em non-classical diagonalization procedures}. Thereby, one could allow
the entire range of two-dimensional unitary transformations
$$
\textsf{\textbf{U}}_2(\omega ,\alpha ,\beta ,\varphi )=e^{-i\,\beta}\,
\left(
\begin{array}{cc}
{e^{i\,\alpha }}\,\cos \omega
&
{-e^{-i\,\varphi }}\,\sin \omega
\\
{e^{i\,\varphi }}\,\sin \omega
&
{e^{-i\,\alpha }}\,\cos \omega
 \end{array}
\right)
 \quad ,
$$
where $-\pi \le \beta ,\omega \le \pi$,
$-\, {\pi \over 2} \le  \alpha ,\varphi \le {\pi \over 2}$, to act on
the quantum bit.
}


\frame{
\frametitle{Quantum diagonalization cntd.}
A typical example of a non-classical operation on a quantum bit is
the ``square root of not''
($
\sqrt{\textsf{\textbf{X}}}
\cdot
\sqrt{\textsf{\textbf{X}}} =\textsf{\textbf{X}}$) gate operator
$$
\sqrt{\textsf{\textbf{X}}} =
{1 \over 2}
\left(
\begin{array}{cc}
1+i&1-i
\\
1-i&1+i
 \end{array}
\right)
\quad ,
$$
which again has the fixed point state $\vert \psi_+ \rangle$ associated with the eigenvalue $+1$.
Yet, not all of these unitary transformations have eigenvectors
associated with eigenvalues $+1$ and thus fixed points.
Indeed, only
unitary transformations of the form
$$
\begin{array}{c}
[\textsf{\textbf{U}}_2(\omega ,\alpha ,\beta ,\varphi )]^{-1}\,\mbox{diag}(1, e^{i\lambda
}) \textsf{\textbf{U}}_2(\omega ,\alpha ,\beta ,\varphi )=
\qquad
\qquad
\qquad
\qquad
\qquad  \\
=
\left(
\begin{array}{cc}
\cos^2 \omega  + e^{i\,\lambda} \,\sin^2 \omega
&
{1\over 2}
e^{-i\,\left(\alpha +\varphi \right) }
(e^{i\,\lambda}-1)
\sin (2\,\omega )
\\
{1\over 2}
 e^{i\,\left(\alpha +\varphi \right)}
(e^{i\,\lambda }-1)
\sin (2\,\omega )
&
e^{i\,\lambda }\,\cos^2 \omega  + \sin^2 \omega
 \end{array}
\right)
\end{array}
$$
for arbitrary $\lambda$ have fixed points.
}


\frame{
\frametitle{Quantum diagonalization cntd.}
Applying non-classical operations on quantum bits with no fixed points
$$
\begin{array}{c}
[\textsf{\textbf{U}}_2(\omega ,\alpha ,\beta ,\varphi )]^{-1}\,\mbox{diag}( e^{i\mu } ,
e^{i\lambda }) \textsf{\textbf{U}}_2(\omega ,\alpha ,\beta ,\varphi ) =
\qquad
\qquad
\qquad
\qquad
\qquad  \\
=
\left(
\begin{array}{cc}
e^{i\,\mu }\,\cos^2 \omega  + e^{i\,\lambda }\,\sin^2 \omega
&
{1\over 2}
e^{-i\,\left( \alpha  + \varphi \right) }
\left( e^{i\,\lambda } - e^{i\,\mu } \right)
\sin (2\,\omega )
\\
{1\over 2}
e^{i\,\left( \alpha  + \varphi \right)}
\left( e^{i\,\lambda } - e^{i\,\mu }  \right) \,\sin (2\,\omega )
&
e^{i\,\lambda } \cos^2 \omega  + e^{i\,\mu }\sin^2 \omega
 \end{array}
\right)
\end{array}
$$
with $\mu ,\lambda \neq 2n\pi$, $n\in {\Bbb N}_0$ gives rise to
eigenvectors which are not fixed points, and which acquire non-vanishing
phases $\mu , \lambda$ in the generalized diagonalization process.
}


\frame{
\frametitle{Physical realization}

{\small
The elementary diagonalization operator without a fixed point generalized
can be realized by a quantum optical beam splitter:
\begin{equation}
\begin{array}{rlcl}
P_1:&\vert {0}\rangle  &\rightarrow& \vert {0}\rangle e^{i(\alpha +\beta)}
 , \\
P_2:&\vert {1}\rangle  &\rightarrow& \vert {1}\rangle
e^{i \beta}
, \\
S:&\vert {0} \rangle
&\rightarrow& \sqrt{T}\,\vert {1}'\rangle  +i\sqrt{R}\,\vert {0}'\rangle
, \\
S:&\vert {1}\rangle  &\rightarrow& \sqrt{T}\,\vert {0}'\rangle  +i\sqrt{R}\,\vert
{1}'\rangle
, \\
P_3:&\vert {0}'\rangle  &\rightarrow& \vert {0}'\rangle e^{i
\varphi
} ,
\end{array}
\end{equation}

\begin{center}
%TeXCAD Picture [1.pic]. Options:
%\grade{\on}
%\emlines{\off}
%\epic{\off}
%\beziermacro{\on}
%\reduce{\on}
%\snapping{\off}
%\quality{8.000}
%\graddiff{0.005}
%\snapasp{1}
%\zoom{4.0000}
\unitlength .5mm % = 1.423pt
\linethickness{0.4pt}
\ifx\plotpoint\undefined\newsavebox{\plotpoint}\fi % GNUPLOT compatibility
\begin{picture}(120,80)(0,0)
\put(20,0){\framebox(80,80)[cc]{}}
\put(57.67,40){\line(1,0){5}}
\put(64.33,40){\line(1,0){5}}
\put(50.67,40){\line(1,0){5}}
\put(78.67,50){\dashbox{1,1}(8,4.33)[cc]{}}
\put(82.67,58){\makebox(0,0)[cc]{$P_3,\varphi$}}
\put(73.33,40){\makebox(0,0)[lc]{$S(T(\omega ))$}}
\put(8.33,65.67){\makebox(0,0)[cc]{$\vert 0\rangle$}}
\put(110.67,65.67){\makebox(0,0)[cc]{${\vert 0\rangle}'$}}
\put(110.67,25.67){\makebox(0,0)[cc]{${\vert 1\rangle}'$}}
\put(8,25.67){\makebox(0,0)[cc]{$\vert 1\rangle$}}
\put(24.33,75.67){\makebox(0,0)[lc]
{${\textsf{\textbf{U}}}^{bs}(\omega ,\alpha ,\beta ,\varphi )$}}
\put(0,59.67){\vector(1,0){20}}
\put(0,20){\vector(1,0){20}}
\put(100,60){\vector(1,0){20}}
\put(100,20){\vector(1,0){20}}
\put(20,20){\line(2,1){80}}
\put(20,60){\line(2,-1){80}}
\put(32.67,50){\dashbox{1,1}(8,4.33)[cc]{}}
\put(36.67,62){\makebox(0,0)[cc]{$P_1,\alpha +\beta $}}
\put(32.67,27){\dashbox{1,1}(8,4.33)[cc]{}}
\put(36.67,35){\makebox(0,0)[cc]{$P_2,\beta$}}
\end{picture}
\end{center}
}

}


\section{Summary}
\frame{
\frametitle{Summary}
\begin{itemize}
\item<1->
Some physical aspects related to the limit operations of the Thomson lamp have been discussed.
\item<1->
Operational interpretations are in agreement with the formal Abel sums of infinite series.
\item<1->
Formal analogies to accelerated (hyper-)computers and have discussed the recursion theoretic diagonal methods.
\item<1->
As quantum information is not bound by mutually exclusive states of classical bits,
it allows a consistent representation of fixed point states of the diagonal operator.
\item<1->
In an effort to reconstruct the self-contradictory feature of diagonalization and the resulting {\it reductio ad absurdum},
a generalized diagonal method allowing no quantum fixed points has been proposed.
\end{itemize}
}




\end{document}
