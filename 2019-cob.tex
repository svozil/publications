\documentclass[12pt]{article}
\usepackage[margin=0.5in]{geometry}
\usepackage{natbib}
\usepackage[breaklinks=true,colorlinks=true,anchorcolor=blue,citecolor=blue,filecolor=blue,menucolor=blue,pagecolor=blue,urlcolor=blue,linkcolor=blue]{hyperref}


\title{Physical aspects of Kelly James Clark's bowling {\it vs.} curling metaphor}
\author{Karl Svozil\\Vienna University of Technology\\Institute for Theoretical Physics\\
Wiedner Hauptstrasse 8-10/136\\
A-1040 Vienna, Austria, EU}

\begin{document}

\maketitle

\begin{abstract}
This is a very brief speculative excursion, or rather an update to Frank's {\em ``The Law of Causality and its Limits,''} on if and how transcendence could manifest itself in the physical world.
\end{abstract}

\begin{flushright}
Inspired by
Nicholas of Cusa ({\it aka} Nicolaus Cusanus,  1401 -- 1464),\\
author of {\it On Learned Ignorance}\\
who inevitably\footnote{{\em Squaring the circle} is an ancient geometric
challenge to construct a square exactly equal in area to a given circle
by finite means limited to the use of ruler and compass alone.
Because $\pi$, the ratio of the circumference to the diameter of a circle,
turned out to be a transcendental number, this task is provable
impossible.}
failed in constructing the {\em quadrature of the circle}~\citep[Chapter~4]{boehlandt-verborgeneZahl}.
\end{flushright}


\section{The historic context}

To spare the Reader a bloated introduction
I just name-drop a few previous references pertinent to the subject:
according to a familiar and influential narrative,
the European Enlightenment\footnote{For a subjective introduction to similar Greek periods
see Schr\"odinger's {\em ``Nature and the Greeks''}~\citep{schroed:natgr}.}
evolved as a courageous, thorough and highly successful\footnote{The criterion of success is taken relative to and in terms of full-spectrum dominance compared to alternative worldviews grounded in esoteric thought.}
exorcism of  transcendence;
in particular, the rejection of law-defying {\em miracles}~\citep{Swinburne-Miracle};
%by Hume~\citep[Section~X]{Hume-Enquiry}.
moreover, the empirical sciences ``established natural laws'' of regular, reliable tempo-spacial coincidences
which appear to be existentially trustworthy  --  think of air travel\footnote{I am unaware of any religious orthodoxy
refusing to board an airplane or any other water or ground vessel based on the belief that the
technology of this machinery is flawed because their respective deity disobeys such ``natural laws.''
(This does not exclude the occasional prayer for a safe journey; in particular, in case of emergencies.)
Indeed, one may argue that much of today's decline of competence of the religious orthodoxy originates in this and other
examples of hypocrisy: they act opportunistically when it suits their purposes but look the other way when
absorbing the respective consequences.}.

The denial of any direct breach or ``rupture''
of the laws of nature~\citep[Sect.~III,~10]{frank,franke}
has pushed the boundaries of conceivable transcendental real-time interventions,
and, in particular, divine providence, to the fringe
of ``gaps''~\citep[Sect.~III,~12]{frank,franke} in the laws of nature  --
indeterminate situations where applicable laws have not (yet?) been identified.

As effective as the formal~\citep{wigner:mb} and natural sciences are in terms of utility,
they turn out to be as means and context relative\footnote{Means relativity of an entity such as an idea is
the dependence (eg., validity, existence) of this entity on the means or assumptions employed.
Context relativity relates to whatever side and additional means and assumptions are employed.
Different means may lead to very different situations:
for instance, G\"odel's (in)completeness theorems state that
first order logic is complete~\citep{godel-1930-Vollstaendigkeit}
whereas ``stronger'' second order logic is incomplete~\citep{godel1}.}
as any construct of thought:
those imaginations of the human mind~\citep{berkeley,Goldschmidt2017-idealism}
cannot deliver any ``Archimedean point''
or ``ontological anchor'' upon which an ``objective reality''
(whatever that is) can be based\footnote{Anyone believing in the human capacity to grasp some ``ontology''
acts like the prisoners in Plato's {\em cave metaphor};
fools of which Goethe in {\it Elective Affinities}
once remarked, {\em ``no one is more a slave than the man who thinks himself free while he is not.''}}.
As Lakatos suggested~\citep{lakatosch} (and this Author concurs in the spirit of Berkeley~\citep{berkeley}),
all that we can ever hope for are successions of ``scientific research programs''
({\it aka} narratives) without a recognizable coherent conceptual convergence.
There is no ``certainty from rationality,'' and most likely, there never will be,
as all our findings are context and
means relative with respect to the assumptions and ego investments made,
and are, therefore, ``suspended in free human thought''\footnote{The desperation, if not
nihilism, that results from the deconstruction of long-held believes and narratives
has been very vividly decribed
by Schopenhauer~\citep{schopenhauer-dwawuv-VI},
as well as through Nietzsche's
{\it \"Ubermensch}~\citep{Nietzsche-ZarathustraI,Nietzsche-EcceHomo}
and Camus' {\it {S}isyphe}~\citep{camus-mos}.}.

Consider metamathematics, for example: Cantor took a bold step
in introducing informal ``entities of thought'' (Hilbert's paradise~\citep{hilbert-26,CBO9781139171519A019})
into the mathematical discourse~\citep{cantor-set}:
{\em ``By a `set' we mean any summary $M$
of certain well-differentiated objects $m$ of our outlook or thinking
(which are called the `elements' of $M$) into a whole.''}
Soon this turned out to be inconsistent; and yet, any remedy  --  at least insofar
it includes sufficiently\footnote{Sufficiency is
understood in terms of the capacity to universally compute.} ``strong'' formalizations  --
is provable incomplete in the sense that not all ``true'' statements; and, in particular, its consistency,
are derivable by intrinsic (within that respective formalization) means alone~\citep{godel1,turing-36,chaitin3}.

Furthermore, any formalization of physical (in)determinism by (in)computability,
as well as physical randomness as algorithmic incompressibility,
as well as general induction~\citep{go-67,blum75blum,angluin:83,ad-91,li:92}
would require transfinite means not available~\citep{gandy1}
in this Universe~\citep{svozil-93,svozil-unev,svozil-07-physical_unknowables}.
This is because the associated
formal proofs are blocked by the aforementioned
G\"odel-Turing-type incompleteness/incomputability results.

Therefore, one cannot expect that the formal and natural sciences
offer absolute corroboration of any type of semantic statements.
All they allow is a systematic exploitation of syntax and narratives
which are true relative to and for all means and practical purposes.


In what follows we shall first discuss what general options of existence can be imagined;
and then proceed with a discussion of their possible concrete physical {\it modi operandi.}



\section{Bowler type scenario of a clockwork universe}

The assumption of a ``clockwork universe''  --  that is, ``stuff'' such as
matter, energy and the like, together with its assorted evolution laws which are uniformly valid and unique (leaving no room for alternatives)  --
entails a ``bowler''-type god\footnote{Deity is understood as an entity creating existence; a sort of ``programmer of the Universe.''}.
Once this universe is created {\it ex nihilo} and put into motion
god does not in any way interfere with it; as all necessary and sufficient conditions exist to
determine its evolution uniquely and completely from a ``previous'' state into a ``later''
one\footnote{In such a scenario free will appears to be illusory and subjectively,
as per assumption choices are merely fictitious and
delusional\footnote{Never mind Molinism which to me appears a hopeless attempt to deny the inevitable.}.
For instance, addictions appear as the ``inner perception'' of the brain's dopamine system preferences;
in particular, the accumulation of $\Delta$FosB~\citep{Nestler11042}.}.

How could physics facilitate and support such a view?
\begin{itemize}

\item
The description of a unique physical state as a {\em function} of some operational physical quantity such as time  --
indeed, the very notion of {\em total function} (as opposed to partiality~\citep{Kleene1936}),
{\em Laplace's demon}, {\em causal~\citep{Norton-2003-cafs} determinism}
and the {\em principle of sufficient reason} are scientific tropes and schemes
signifying clockwork universes.
They were widely held in pre-statistical physics and quantum areas until around {\it fin de si\`ecle}.

In the context of ordinary differential equations of
classical continuum mechanics and classical electrodynamics
the semantic notion of ``determinism''
is formalized by the {\em uniqueness} of the solutions, which
are guaranteed by a Lipschitz continuity
condition\footnote{According to the Picard-Lindel\"of theorem %~\citep[Theorem~1.6.2]{nagy-ODE}
an {\em initial value problem}
defined by a first order ordinary differential equation of the form $y'(t)=f(t,y(t))$
and the initial value $y(t_0)=y_0$
has a {\em unique} solution if $f$  satisfies the
Lipschitz condition and is continuous as a function of $t$.
A mapping $f$ satisfies (global/local) {\em Lipschitz continuity} (or, used synonymously,   {\em Lipschitz condition})
with finite positive constant $0<k<\infty$ if
it increases the distance between any two points $y_1$ and $y_2$ (of its entire domain/some neighborhood)
by a factor at most $k$:
$
\vert f(t,y_2)-f(t,y_1) \vert \le k \vert y_2 - y_1 \vert
$.
$f$ may be nonlinear as long as it does not separate different points $y_1$ and $y_2$
``too much.''}~\citep[Chapter~17]{svozil-pac}.


\item
The quantum state evolution is postulated to be unique and deterministic\footnote{Formally it is represented by a unitary transformation, that is,
a generalized rotation mapping one orthonormal basis into another one.
Such a state evolution is one-to-one and thus reversible and unique.
However, if the preparation context differs from the measurement context,
the quantum state does not identify outcomes uniquely,
thereby allowing one particular kind of quantum indeterminacy.}.
But in general  --  that is, in the case of coherent superposition or mixed states  --
the quantum state is not operationally accessible.
Therefore this sort of
quantum determinacy cannot be given any direct empirical meaning.

\item
{\em Deterministic chaos} is characterized by a unique initial value  --
a ``seed''
supposed to be taken from the mathematical continuum and thus
incomputable and even random\footnote{Randomness of an infinite string
is taken to be algorithmically incompressible~\citep{martin-lof}.}
with probability one  --  whose information or digits are ``revealed''
by some suitable deterministic temporal evolution.
To be suitable a temporal evolution needs to be very sensitive to changes of
initial seeds such that very small
fluctuations may produce very large effects.
This is similar to Maxwell's gap scenario discussed later.

Like the quantum evolution, deterministic chaos might be considered both an argument
for as well as against classical determinism: because
the assumption of the continuum renders almost all seeds formally random~\citep{martin-lof},
thereby passing all statistical tests of randomness; in particular ``elementary'' test such as
Borel normality, certifying that all sequences of arbitrary length occur
with the expected frequency~\footnote{Unfortunately
Borel normality is no guarantee of randomness because very regular sequences,
for instance,
the Champernowne constant~\citep{Sloane_oeis.org/A033307} $C_{10}$ in base $10$
is just the sequence obtained by concatenating
successive numbers (encoded in base $10$),
turn out to be normal.}, but also much stronger ones.

In this respect classical machinery designed to utilize
extreme sensitivities of the temporal evolution to the initial seed,
 such as the Athenian~\citep{dow_aristotlekleroteria_1939}
$\kappa \lambda \eta \rho \omega \tau \eta \rho \iota o \nu$
({\it kleroterion}),
for all practical purposes is not inferior to a quantum oracle
for randomness, such as {\it QUANTIS}~\citep{Quantis},
based on the ``evangelical'' belief of irreducible quantum randomness~\citep{zeil-05_nature_ofQuantum}.


\item
In the context of system science or virtual physics, this modus could be referred to as {\em virtual reality,}
{\em computational gaming environment} or {\em simulation} ({\it aka} simulacrum); but only if there is
no interference from ``the outside'' ({\it aka} beyond): the respective universe is hermetic.
No participation is possible; only passive (without interference) observation.
\end{itemize}

How does physics contradict such a view?
\begin{itemize}


\item
Classical gaps are characterized by {\em instabilities}  at {\em singular points}, such that very small
fluctuations may produce very large effects.
To quote Maxwell~\citep[pp.~211,212]{Campbell-1882},
{\em ``for example, the rock loosed by frost and balanced on a singular point of the mountain-side, the little spark which
kindles the great forest~$\ldots$ At these
points, influences whose physical magnitude is too small to be taken account of by a finite being, may produce
results of the greatest importance.}

\item
In some physical situations the
Lipschitz continuity is violated, yielding no unique solutions.
The Norton dome~\citep{Norton-dome-2008,vanStrien2014} is a
contemporary example of such a situation.

\item
Spontaneous symmetry breaking,
a physical (re)source of nonuniqueness,
is a spontaneous process
by which a physical system in a symmetric state ends up in an asymmetric state.
This is facilitated by some appropriate ``Mexican hat'' potential,
not dissimilar to Norton's dome or
Maxwell's~\citep[pp.~211,212]{Campbell-1882}
{\em ``rock loosed by frost and balanced on a singular point''} mentioned earlier.

In particle physics the Higgs mechanism, the spontaneous symmetry breaking of gauge symmetries,
plays an important role in the origin of particle masses in
the standard model of particle physics.
All of these ruptures or breaches of uniqueness depend on the assumptions and models involved.

\item
Quantum indeterminacy, in particular, complementarity, contextuality ({\it aka} value indefiniteness),
and aspects (such as the exact decay time) of the occurrence of certain single events are postulated to signify indeterminism.
\end{itemize}

Because of both formal as well as empirical reasons these scenarios might no be interrelated and not separate:
for instance, one might suspect that Maxwell's
instabilities  at  singular points could be formalized by ``Mexican hat'' type potentials discussed in spontaneous symmetry breaking,
or by ordinary differential equations yielding Norton dome-type configurations.
One might even speculate that all violations of Lipschitz continuity amount to some kind of symmetry breakdown.

Empirically one might argue that, for all practical purposes~\citep{bell-a}, Maxwell's scenario as well as Norton dome-type configurations
(related to violations of Lipschitz continuity) or spontaneous symmetry breaking, never ``actually'' happen.
Because for all practical purposes a rock loosed by frost is never (with probability zero)
totally symmetrically balanced at a singular point; rather the position of its center of gravity will
fluctuate around the tip, thereby spoiling symmetry.
Also one may argue that, due to fluctuations related to the character of matter, singular points make no operational sense whatsoever; they
are (over)idealized concepts invented by the human mind for mere convenience.
In particular, microscopic quantum zero point fluctuations, as well as  thermal fluctuations~\citep{Smoluchovski-1912}
ultimately spoil symmetries.
Therefore, all such exploitations of such singularities confuse epistemic convenience with an ontology that has no physical, operational grounds.

\section{Scenario of a stochastic, disorganized universe}

The ``converse'' of a Laplacean determinism governed by a unique state evolution
``tied to'' previous states, as mentioned in the previous section, is one in which any given state is
independent\footnote{Two events $A$ and $B$
are statistically independent if their joint probability $P(A\cap B)$ can be written
as the product of their single probabilities $P(A)$ and $P(B)$; that is,
$P(A\cap B)= P(A)P(B)$.
It turns out that this results in a journey down a rabbit hole, as the concept of probability
is a nontrivial one~\citep{Uffink2011-UFFSPS}.}
of the respective previous (and future) states.
In such a most extreme scenario among many conceivable
degrees of stochasticity
the universe is ``completely'' stochastic and disorganized on the most fundamental level.
For the embedded observer's intrinsic perspective, due to irreducible contingency and chance,
it appears as if such a world is constantly created anew by throwing some sort of
dice\footnote{This may be considered an extreme form of {\it creatio continua.}
However, {\em extrinsically}  --  that is, from an external, extrinsic, perspective  --  this may be considered {\em creatio ex nihilo} as
no active, real-time participation is assumed.
Indeed, one may speculate that
if the temporal ordering of events (as well as causality) turns out to be epistemic  --
an intrinsically emerging concept/observable of  (self-)cognition/observation  --
then any differentiation
based on temporal creation  --  such as {\it creatio continua {\em versus} ex nihilo}  --  turns out to be a ``red herring.''
Alas, without granting ``time'' some ontology, also differentiations between a ``bowling'' or ``curling'' god collapse.\label{2019-cob-lcc-cen}}.

Whether and how some sort of structural continuity of existence can emerge and be maintained under such circumstances is a fascinating question.
As in such a scenario space and time, as much as notions of causality and the laws, are emergent concepts, continuity might emerge with them.

Indeed, one might speculate that ``the laws'' are some sort of
expressions of chaos~\footnote{This is not dissimilar to the
impossible choice not to communicate~\citep{Watzlawick-1967}.}, the formation of matter and genes are expressions of these laws,
the individuals carrying those genes are expressions thereof~\citep{Hamilton-1963}, and that the ideas about the world are expressions of these individuals.
In that transitive way, the Universe contemplates itself through our ideas  --  ideas such as religion, mathematics, ethics, and so on.

Contemporary physics supports such a view in postulating that many elementary events
 --  such as the spontaneous or stimulated emission of photons  --  occur acausally, irreducibly pure and simple~\citep{born-26-1,zeil-05_nature_ofQuantum}.
Indeed, both classical statistical physics at finite resolution\footnote{A Laplacian demon with unbounded resources might be able to determine
future states from present ones with arbitrary precision.}, as well as quantum mechanics, support such a view.

Already Exner~\citep{Hiebert2000,Stoeltzner-1999}, motivated by statistical physics
and the radiation law~\citep{schweidler-1905},
suggested that~\citep[p.~7,18]{Exner-1908}
%Gesetze existieren aber nicht in der Natur, die formuliert erst der Mensch, er bedient sich derselben als sprachlichen und rechnerischen Hilfsmittels und will damit nur sagen, dass die Vorg�nge in der Natur so verlaufen als w�rde die Materie, einem vern�nftigen Wesen gleich, diesen Gesetzen gehorchen.
{\em ``$\ldots$~laws do not exist in nature,
those are only formulated by man, he makes use of it
as a linguistic and computational aid
and only wants to say
that the processes in nature run as if matter, like a sentient being, would obey these laws.
%So m�ssen wir also alle sogenannten exakten Gesetze nur als Durchschnittsgesetze auffassen die nicht mit absoluter Sicherheit gelten, wohl aber mit um so gr��erer Wahrscheinlichkeit aus je mehr Einzelvorg�ngen sie sich ergeben.
$\ldots$
So we must understand all so-called exact laws
only as average laws, which are not valid with absolute certainty,
but with the higher probability
the more individual processes they result from.
% Alle physikalischen Gesetze gehen zur�ck auf molekulare Vorg�nge zuf�lliger Natur und aus ihnen folgt das Resultat nach den Gesetzen der Wahrscheinlichkeitsrechnung,
All physical laws
go back to molecular processes of random nature
and from them follows the result according to the laws
of probability calculus~$\ldots$~.''}

Even in totally ``random'' datasets, some sort of structure must necessarily emerge
by the law of large numbers:
for instance, if two dice are thrown sufficiently often, the number seven appears to be the most likely sum of their two faces.
Modern arguments for the emergence of laws from chaos employ,
among other methods~\citep{armstrong_1983,vanFraassen1989-VANLAS,calude-meyerstein,lawlses_rosen2010,calude2013theeinai,chaos_multiverse2017,Mueller-2017,Cabello-2018-BornRule},
Ramsey theory for structure formation and structural continuity through spurious correlations~\citep{svozil-2018-was}.
Thereby it is irrelevant whether or not these events occur ``absolutely randomly''  --  indeed,
as has been pointed out earlier, on an individual level and with finitistic means, ``absolute randomness''
appears to be a vacuous concept.

\section{The intermediate curler case}

Intuitively the curler case discussed here~\citep{Clark-2017-GodAsCurler}
is one in which the natural laws  --  whatever their form and origin  --  predominate,
but there are situations in which such laws do not exist, or if laws exist they are violated.
The first ``weak'' case of indeterminism can be realized by gaps\footnote{As mentioned earlier~\citep[Sect.~III,~10]{frank,franke}
``stronger'' forms of curling involve a ``rupture'' of the laws of nature, as they
are in direct violations of those laws
as mentioned in Voltaire's Philosophical Dictionary~\citep[Chapter~330]{voltaire-dict}.
Although nobody can {\it a priori}exclude such latter cases we shall henceforth stick with
Hume's attitude towards miracles~\citep[Section~X]{Hume-Enquiry}   and neglect them.}.

Theologically this could be perceived as a mild form of {\it creatio continua}\footnote{{\it Cf.} my earlier remarks on {\it creatio continua}
in footnote~\ref{2019-cob-lcc-cen}.}: god has created laws which are not violated,
but god also left  ``some room'' to communicate  {\it via} gaps.

A ``god of the gaps'' has been rephrased in many ways.
This concept is also quite popular since, on the one hand, the obvious regularities of experience and life express correlations or laws which appear evident:
the daily cycle of the sun, the yearly cycle of the seasons, life, death; apples and other stuff falling down and not up, and so on.
So denial of regularities appears futile; one way of integrating them into religion is to say that a ``bowling god'' made them so.
On the other hand, humans experience fate and uncontrollable circumstances quite often.
In a similar reaction, the primitive mind (re)interpreted such ``evidence'' as god's signal.

As more and more ``fateful'' behaviors became ``understood'' and even controllable\footnote{Think of medical treatments and also
volcanic eruptions,
floods or weather phenomena such as lightning and thunder.} it is not unreasonable to speculate that,
maybe, eventually, there will be no such gaps left  --  in which case
one recovers the bowler, {\em ex nihilo,} scenario.
Alternatively some ``pure'' gaps in the causal fabric of our universe
might ``turn out''  --  that is, relative to the assumptions and means employed  --
to be irreducible and final: those gaps cannot be eliminated and might remain forever.
In secular terms, this could be suspected to signify irreducible indeterminism or randomness~\citep{zeil-05_nature_ofQuantum}.
But there exist other, possibly transcendental, interpretations involving {\em intentionality} across gaps.

That these latter scenarios are not purely speculative can be demonstrated by an interactive gaming scenario:
If one is considering an interactive virtual reality environment~\citep{simula,permutationcity} one usually assumes that the virtual reality is
``sustained'' or ``supported'' by a computational process ``running'' on some kind of computer whose physical characteristics
are not directly related\footnote{To be feasible and nonmonotonic it can be assumed without loss of generality that both the
universe in which the simulation is implemented as well as the simulated universe are capable of universal computation in
the sense of Chuch-Turing.} to the simulacrum.
To be interactive the two universes need to be intertwined and connected
by some sort of (bidirectional) gap through which
information flows in both ``directions''\footnote{This could result in a sort of {\em dialogue} between those realms.
This could lead to a ``backflow'' from the simulacrum to the universe in which the
simulation takes place, such that the former simulacrum
performs ``empirical studies'' on the latter, thereby fully and actively participating in it.
In this very speculative scenario, ``transcendence becomes immanence.''
Think of evolving artificial intelligence in a computer simulation becoming aware of its situation and asking online players questions
about its situation and the general situation.
However, as symmetric as such an exchange through the interface may appear, it is asymmetric in one aspect: whereas the simulacrum cannot exist without the world
in which the simulation takes place
the latter can exist without the former.}.
For an intrinsic~\citep{svozil-94} observer embedded~\citep{toffoli:79}
in the virtual environment and bound by its operational capacities {\em the capacity to send an arbitrary signal through the interface
 --  from the simulating universe ({\it aka} ``the beyond'')  to the simulacrum  --  can only be realized by a gap.}
Because without a gap, the signal must remain immanent;
that is, it reduces to either lawful or chaotic behavior.

Gaps potentially allow some ``transcendental'' exchange of signals but do not necessarily imply such a
conversation or dialogue.
Therefore, gaps are a necessary but not a sufficient condition for transcendence  --  just because gaps have been located does not imply the existence of  ``active'' transcendental entities.

From a theological perspective, gaps can be employed to realize individual (human) soul/mind-body dualism~\citep{eccles:papal},
and also divine providence~\citep[Sect.~III,~9-16]{frank,franke}.

How does physics support gaps? Or can physics rule them out?
The following is an update and extension of Frank's discussion on physical gaps.
\begin{itemize}

\item
As has been mentioned earlier, in the classical domain of ordinary differential equations some breach of the
Lipschitz continuity condition~\citep[Chapter~17]{svozil-pac}
could result in nonunique solutions.
Often such types of gaps are identified with instabilities
at their singular points~\citep[pp.~211,212]{Campbell-1882}, \citep[Sect.~III,~13]{frank,franke}.

\item
Quantum complementarity, and, as an extension thereof, quantum contextuality ({\it aka} value indefiniteness) can
be interpreted as the impossibility to co-represent~\citep{peres,kochen1,2015-AnalyticKS}
certain (even finite) sets of  --  necessarily {\em counterfactual} because they are complementary
 --  quantum observables,
relative to the asssumptions\footnote{One assumption entering those proofs are the (context) independence
of outcomes of measurements for ``intertwine'' observables occurring in more than one context.
For reasons of being able to intertwine contexts formalized by orthonormal bases
this can only happen in vector spaces of dimension higher than two.}.
In my opinion, this is problematic as the corresponding experimental protocols
(``prepare a pure state and measure a different one'') seem to suggest that they
``reveal'' some pre-existing property  --  indicated by the (non)occurrence of a detector click.
Alas this could be misleading, as the respective click might
either be subject to debate and interpretation\footnote{A debate~\citep{Kimble-aposterioriQT,Bouwm-aposterioriQTReply}
on the alleged ``{\it a posteriori} teleportation'' is an example for such a nonunique
semantic perception of syntactically undisputed detector clicks.}
or merely signify the capacity of the measurement apparatus
to ``translate an improper question;'' thereby introducing stochastic noise~\citep{svozil-2003-garda}.
This appears to be related to notorious inconsistencies
in quantum physics proper~\citep{v-neumann-49,v-neumann-55,everett,wigner:mb,everett-collw} due to the
assumption of irreversible quantum measurements.

\item
Aspects of certain individual, single events in quantized systems such
as the time of emission or absorption of single quanta of light,
are postulated to be indeterministic.
\end{itemize}



\section{The (un)known (un)knowns}

The relativity of the considerations on the respective assumptions and means invested or taken for granted results in
an echo-chamber of sorts: whatever one puts in one gets out.
There does not seem to exist any ``firm (meta)physical ground,'' no  undisputable
``Archimedean ontological anchor'' upon which such speculations can be based.
And the tendency of the mind to rationalize, project~\citep{Jaynes1990,Freud-1912,Freud-1912-e} and empathically
embrace opinions which are favorable to one's ego-investments
increases delusions about particular beliefs and corroborations thereof even further.


At this point, the Reader might get frustrated: a negative message (akin to a negative theology)
has been delivered\footnote{One positive side effect might be the abandoning of what the Vienna Circle (in a Humean tradition) called
``meaningless pseudo-statements''~\citep{Hahn1930,Carnap1931,Carnap-1931-engl}
targeting a particular kind of hocus-pocus, abracadabra delusional (thought) rituals delivered by sophistic philosophers and an orthodox clergy.
However, one has to be very careful not to ``throw the baby out with the bathwater.''
Shortly after these seemingly bold rejections of metaphysical entities,
it turned out that their program based on empirical evidence and formal logic proposed could not be carried ous as completely as
desired~\citep{godel1,turing-36,smullyan-92,Smullyan1993-SMURTF,book:486992,chaitin3}.}.
Alas, unfortunately, this is all that can be safely stated.

Therefore we should accept the sober fact that there is certainty only in our uncertainty.
This has been expressed by many insightful individuals of many religions and at various times.
Aurelius Augustinus, for instance, writes~\citep[Book~XI, chapter~25.32]{Augustinus-Confessiones},
{\em ``Do I perhaps not know how to express what I do know?
Woe is me: I do not even know what it is I do not know!''}


\section*{Acknowledgments}
I kindly acknowledge discussions with  and suggestions by  Kelly James Clark, Silvia Jonas, Jeffrey Koperski, Irem Kurtsal and Emil Salim
in the context of the John Templeton Foundation's project on {\em Randomness and Providence.}
Some translations from German were performed by first using the {\em DeepL} translator.
All misunderstandings and mistakes are mine.


%  \bibliographystyle{newapa-doi-url}
%  \bibliography{svozil}



\begin{thebibliography}{}

\bibitem[\protect\citeauthoryear{Abbott, Calude \& Svozil}{Abbott
  et~al.}{2015}]{2015-AnalyticKS}
Abbott, A.~A., Calude, C.~S., \& Svozil, K. (2015).
\newblock A variant of the {K}ochen-{S}pecker theorem localising value
  indefiniteness.
\newblock {\em Journal of Mathematical Physics}, {\em 56\/}(10), 102201(1--17).
\newblock Available from: \url{https://doi.org/10.1063/1.4931658}, \href
  {http://arxiv.org/abs/arXiv:1503.01985} {\path{arXiv:arXiv:1503.01985}},
  \href {http://dx.doi.org/10.1063/1.4931658} {\path{doi:10.1063/1.4931658}}.

\bibitem[\protect\citeauthoryear{Adleman \& Blum}{Adleman \&
  Blum}{1991}]{ad-91}
Adleman, L.~M. \& Blum, M. (1991).
\newblock Inductive inference and unsolvability.
\newblock {\em The Journal of Symbolic Logic}, {\em 56}, 891--900.
\newblock Available from: \url{https://doi.org/10.2307/2275058}, \href
  {http://dx.doi.org/10.2307/2275058} {\path{doi:10.2307/2275058}}.

\bibitem[\protect\citeauthoryear{Angluin \& Smith}{Angluin \&
  Smith}{1983}]{angluin:83}
Angluin, D. \& Smith, C.~H. (1983).
\newblock Inductive inference: Theory and methods.
\newblock {\em ACM Computing Surveys}, {\em 15\/}(3), 237--269.
\newblock Available from: \url{https://doi.org/10.1145/356914.356918}, \href
  {http://dx.doi.org/10.1145/356914.356918} {\path{doi:10.1145/356914.356918}}.

\bibitem[\protect\citeauthoryear{Armstrong}{Armstrong}{1983}]{armstrong_1983}
Armstrong, D.~M. (1983).
\newblock {\em What is a Law of Nature?}
\newblock Cambridge Studies in Philosophy. Cambridge: Cambridge University
  Press.
\newblock Available from: \url{https://doi.org/10.1017/CBO9781139171700}, \href
  {http://dx.doi.org/10.1017/CBO9781139171700}
  {\path{doi:10.1017/CBO9781139171700}}.

\bibitem[\protect\citeauthoryear{{Augustine of Hippo}}{{Augustine of
  Hippo}}{2019}]{Augustinus-Confessiones}
{Augustine of Hippo} (2019).
\newblock {\em Confessions}.
\newblock Indianapolis, Indiana, USA: Hackett Publishing Company, Inc.
\newblock Williams edition, translated by Thomas Williams.
\newblock Available from:
  \url{https://www.hackettpublishing.com/confessions-williams-translation-4222}.

\bibitem[\protect\citeauthoryear{Bell}{Bell}{1990}]{bell-a}
Bell, J.~S. (1990).
\newblock Against `measurement'.
\newblock {\em Physics World}, {\em 3}, 33--41.
\newblock Available from: \url{https://doi.org/10.1088/2058-7058/3/8/26}, \href
  {http://dx.doi.org/10.1088/2058-7058/3/8/26}
  {\path{doi:10.1088/2058-7058/3/8/26}}.

\bibitem[\protect\citeauthoryear{Berkeley}{Berkeley}{1710}]{berkeley}
Berkeley, G. (1710).
\newblock {\em A Treatise Concerning the Principles of Human Knowledge}.
\newblock Available from: \url{http://www.gutenberg.org/etext/4723}.

\bibitem[\protect\citeauthoryear{Blum \& Blum}{Blum \& Blum}{1975}]{blum75blum}
Blum, L. \& Blum, M. (1975).
\newblock Toward a mathematical theory of inductive inference.
\newblock {\em Information and Control}, {\em 28\/}(2), 125--155.
\newblock Available from: \url{https://doi.org/10.1016/S0019-9958(75)90261-2},
  \href {http://dx.doi.org/10.1016/S0019-9958(75)90261-2}
  {\path{doi:10.1016/S0019-9958(75)90261-2}}.

\bibitem[\protect\citeauthoryear{B\"ohlandt}{B\"ohlandt}{2009}]{boehlandt-verborgeneZahl}
B\"ohlandt, M. (2009).
\newblock {\em {V}erborgene {Z}ahl -- {V}erborgener {G}ott. {M}athematik und
  {N}aturwissen im {D}enken des {N}ikolaus {C}usanus (1401-1464)}, volume~58 of
  {\em {S}udhoffs {A}rchiv -- {B}eihefte}.
\newblock Stuttgart: Franz Steiner Verlag.
\newblock Available from: \url{http://www.steiner-verlag.de/titel/57323.html}.

\bibitem[\protect\citeauthoryear{Born}{Born}{1926}]{born-26-1}
Born, M. (1926).
\newblock Zur {Q}uantenmechanik der {S}to{\ss}vorg{\"{a}}nge.
\newblock {\em Zeitschrift f\"{u}r Physik}, {\em 37\/}(12), 863--867.
\newblock Available from: \url{https://doi.org/10.1007/BF01397477}, \href
  {http://dx.doi.org/10.1007/BF01397477} {\path{doi:10.1007/BF01397477}}.

\bibitem[\protect\citeauthoryear{Bouwmeester, Pan, Daniell, Weinfurter,
  Zukowski \& Zeilinger}{Bouwmeester et~al.}{1998}]{Bouwm-aposterioriQTReply}
Bouwmeester, D., Pan, J.-W., Daniell, M., Weinfurter, H., Zukowski, M., \&
  Zeilinger, A. (1998).
\newblock Reply: {A} posteriori teleportation.
\newblock {\em Nature}, {\em 394}, 841.
\newblock Available from: \url{https://doi.org/10.1038/29678}, \href
  {http://dx.doi.org/10.1038/29678} {\path{doi:10.1038/29678}}.

\bibitem[\protect\citeauthoryear{Braunstein \& Kimble}{Braunstein \&
  Kimble}{1998}]{Kimble-aposterioriQT}
Braunstein, S.~L. \& Kimble, H.~J. (1998).
\newblock A posteriori teleportation.
\newblock {\em Nature}, {\em 394}, 840--841.
\newblock Available from: \url{https://doi.org/10.1038/29674}, \href
  {http://dx.doi.org/10.1038/29674} {\path{doi:10.1038/29674}}.

\bibitem[\protect\citeauthoryear{Cabello}{Cabello}{2019}]{Cabello-2018-BornRule}
Cabello, A. (2019).
\newblock Physical origin of quantum nonlocality and contextuality.
\newblock Available from: \url{https://arxiv.org/abs/1801.06347}, \href
  {http://arxiv.org/abs/arXiv:1801.06347} {\path{arXiv:arXiv:1801.06347}}.

\bibitem[\protect\citeauthoryear{Calude \& Meyerstein}{Calude \&
  Meyerstein}{1999}]{calude-meyerstein}
Calude, C. \& Meyerstein, F.~W. (1999).
\newblock Is the universe lawful?
\newblock {\em Chaos, Solitons \& Fractals}, {\em 10\/}(6), 1075--1084.
\newblock Available from:
  \url{http://dx.doi.org/10.1016/S0960-0779(98)00145-3}, \href
  {http://dx.doi.org/10.1016/S0960-0779(98)00145-3}
  {\path{doi:10.1016/S0960-0779(98)00145-3}}.

\bibitem[\protect\citeauthoryear{Calude, Meyerstein \& Salomaa}{Calude
  et~al.}{2013}]{calude2013theeinai}
Calude, C.~S., Meyerstein, W.~F., \& Salomaa, A. (2013).
\newblock The universe is lawless or ``panton chrematon metron anthropon
  einai''.
\newblock In H.~Zenil (Ed.), {\em Computable Universe: {U}nderstanding and
  Exploring Nature as Computation}  (pp.\ 525--537). Singapore: World
  Scientific.
\newblock Available from: \url{https://doi.org/10.1142/9789814374309\_0026},
  \href {http://dx.doi.org/10.1142/9789814374309\_0026}
  {\path{doi:10.1142/9789814374309\_0026}}.

\bibitem[\protect\citeauthoryear{Calude \& Svozil}{Calude \&
  Svozil}{2019}]{svozil-2018-was}
Calude, C.~S. \& Svozil, K. (2019).
\newblock Spurious, emergent laws in number worlds.
\newblock {\em Philosophies}, {\em 4\/}(2), 17.
\newblock Available from: \url{https://doi.org/10.3390/philosophies4020017},
  \href {http://arxiv.org/abs/arXiv:1812.04416}
  {\path{arXiv:arXiv:1812.04416}}, \href
  {http://dx.doi.org/10.3390/philosophies4020017}
  {\path{doi:10.3390/philosophies4020017}}.

\bibitem[\protect\citeauthoryear{Camus}{Camus}{1942}]{camus-mos}
Camus, A. (1942).
\newblock {\em Le Mythe de {S}isyphe (English translation: The Myth of
  Sisyphus)}.

\bibitem[\protect\citeauthoryear{Cantor}{Cantor}{1895}]{cantor-set}
Cantor, G. (1895).
\newblock Beitr{\"a}ge zur {B}egr{\"u}ndung der transfiniten {M}engenlehre.
  {E}rster {A}rtikel.
\newblock {\em Mathematische Annalen}, {\em 46\/}(4), 481--512.
\newblock Available from: \url{https://doi.org/10.1007/BF02124929}, \href
  {http://dx.doi.org/10.1007/BF02124929} {\path{doi:10.1007/BF02124929}}.

\bibitem[\protect\citeauthoryear{Carnap}{Carnap}{1931}]{Carnap1931}
Carnap, R. (1931).
\newblock {\"U}berwindung der {M}etaphysik durch logische {A}nalyse der
  {S}prache.
\newblock {\em Erkenntnis}, {\em 2\/}(1), 219--241.
\newblock Available from: \url{https://doi.org/10.1007/BF02028153}, \href
  {http://dx.doi.org/10.1007/BF02028153} {\path{doi:10.1007/BF02028153}}.

\bibitem[\protect\citeauthoryear{Carnap}{Carnap}{1959}]{Carnap-1931-engl}
Carnap, R. (1959).
\newblock The elimination of metaphysics through logical analysis of language.
\newblock In A.~J. Ayer (Ed.), {\em Logical Positivism}  (pp.\ 60--81). New
  York: Free Press.
\newblock translated by Arthur Arp.

\bibitem[\protect\citeauthoryear{Chaitin}{Chaitin}{2003}]{chaitin3}
Chaitin, G.~J. (1987,2003).
\newblock {\em Algorithmic Information Theory\/} (Revised edition ed.).
\newblock Cambridge Tracts in Theoretical Computer Science, Volume 1.
  Cambridge: Cambridge University Press.
\newblock Available from: \url{https://doi.org/10.1017/CBO9780511608858}, \href
  {http://dx.doi.org/10.1017/CBO9780511608858}
  {\path{doi:10.1017/CBO9780511608858}}.

\bibitem[\protect\citeauthoryear{Clark}{Clark}{2017}]{Clark-2017-GodAsCurler}
Clark, K.~J. (2017).
\newblock Is {G}od a bowler or a curler?
\newblock Presentation at the Randomness and Providence Workshop, May 9, 2017.

\bibitem[\protect\citeauthoryear{Dow}{Dow}{1939}]{dow_aristotlekleroteria_1939}
Dow, S. (1939).
\newblock {A}ristotle, the {K}leroteria, and the courts.
\newblock {\em Harvard Studies in Classical Philology}, {\em 50}, 1--34.
\newblock Available from: \url{https://doi.org/10.2307/310590}, \href
  {http://dx.doi.org/10.2307/310590} {\path{doi:10.2307/310590}}.

\bibitem[\protect\citeauthoryear{Eccles}{Eccles}{1990}]{eccles:papal}
Eccles, J.~C. (1990).
\newblock The mind-brain problem revisited: The microsite hypothesis.
\newblock In J.~C. Eccles \& O.~Creutzfeldt (Eds.), {\em The Principles of
  Design and Operation of the Brain}  (pp.\ 549--572). Berlin: Springer.
\newblock Available from: \url{https://doi.org/10.1017/S0960129510000344},
  \href {http://dx.doi.org/10.1017/S0960129510000344}
  {\path{doi:10.1017/S0960129510000344}}.

\bibitem[\protect\citeauthoryear{Egan}{Egan}{1994}]{permutationcity}
Egan, G. (1994).
\newblock {\em Permutation City}.
\newblock accessed January 4, 2017.
\newblock Available from:
  \url{http://www.gregegan.net/PERMUTATION/Permutation.html}.

\bibitem[\protect\citeauthoryear{{Everett III}}{{Everett III}}{1957}]{everett}
{Everett III}, H. (1957).
\newblock `{R}elative {S}tate' formulation of quantum mechanics.
\newblock {\em Reviews of Modern Physics}, {\em 29}, 454--462.
\newblock Available from: \url{https://doi.org/10.1103/RevModPhys.29.454},
  \href {http://dx.doi.org/10.1103/RevModPhys.29.454}
  {\path{doi:10.1103/RevModPhys.29.454}}.

\bibitem[\protect\citeauthoryear{{Everett III}}{{Everett
  III}}{2012}]{everett-collw}
{Everett III}, H. (2012).
\newblock In J.~A. Barrett \& P.~Byrne (Eds.), {\em The {E}verett
  Interpretation of Quantum Mechanics: Collected Works 1955-1980 with
  Commentary}. Princeton, NJ: Princeton University Press.
\newblock Available from: \url{http://press.princeton.edu/titles/9770.html}.

\bibitem[\protect\citeauthoryear{Exner}{Exner}{2016}]{Exner-1908}
Exner, F.~S. (1909, 2016).
\newblock {\em {\"U}ber {G}esetze in {N}aturwissenschaft und {H}umanistik:
  {I}naugurationsrede gehalten am 15. {O}ktober 1908}.
\newblock Vienna: H\"older, Ebooks on Demand Universit\"atsbibliothek Wien.
\newblock handle https://hdl.handle.net/11353/10.451413, o:451413, Uploaded:
  30.08.2016.
\newblock Available from: \url{http://phaidra.univie.ac.at/o:451413}.

\bibitem[\protect\citeauthoryear{Frank}{Frank}{1932}]{frank}
Frank, P. (1932).
\newblock {\em Das {K}ausalgesetz und seine {G}renzen}.
\newblock Vienna: Springer.

\bibitem[\protect\citeauthoryear{Frank \& {R. S. Cohen (Editor)}}{Frank \& {R.
  S. Cohen (Editor)}}{1997}]{franke}
Frank, P. \& {R. S. Cohen (Editor)} (1997).
\newblock {\em The Law of Causality and its Limits (Vienna Circle Collection)}.
\newblock Vienna: Springer.
\newblock Available from: \url{https://doi.org/10.1007/978-94-011-5516-8},
  \href {http://dx.doi.org/10.1007/978-94-011-5516-8}
  {\path{doi:10.1007/978-94-011-5516-8}}.

\bibitem[\protect\citeauthoryear{Freud}{Freud}{1958}]{Freud-1912-e}
Freud, S. (1912, 1958).
\newblock Recommendations to physicians practising psycho-analysis.
\newblock In A.~F.~A. Freud, A.~Strachey, \& A.~Tyson (Eds.), {\em The Standard
  Edition of the Complete Psychological Works of Sigmund Freud, Volume XII
  (1911-1913): The Case of {S}chreber, Papers on Technique and Other Works}
  (pp.\ 109--120). London: The Hogarth Press and the Institute of
  Psycho-Analysis.
\newblock Available from:
  \url{https://www.pep-web.org/document.php?id=se.012.0109a\#p0109}.

\bibitem[\protect\citeauthoryear{Freud}{Freud}{1999}]{Freud-1912}
Freud, S. (1912, 1999).
\newblock {R}atschl{\"{a}}ge f{\"{u}}r den {A}rzt bei der psychoanalytischen
  {B}ehandlung.
\newblock In A.~Freud, E.~Bibring, W.~Hoffer, E.~Kris, \& O.~Isakower (Eds.),
  {\em {G}esammelte {W}erke. {C}hronologisch geordnet. {A}chter {B}and. {W}erke
  aus den {J}ahren 1909--1913}  (pp.\ 376--387). Frankfurt am Main: Fischer.
\newblock Available from:
  \url{http://gutenberg.spiegel.de/buch/kleine-schriften-ii-7122/15}.

\bibitem[\protect\citeauthoryear{Galouye}{Galouye}{1964}]{simula}
Galouye, D.~F. (1964).
\newblock {\em Simulacron 3}.
\newblock New York: Bantam Books.

\bibitem[\protect\citeauthoryear{Gandy}{Gandy}{1982}]{gandy1}
Gandy, R.~O. (1982).
\newblock Limitations to mathematical knowledge.
\newblock In D.~van Dalen, D.~Lascar, \& J.~Smiley (Eds.), {\em Logic
  colloquium '80}  (pp.\ 129--146). Amsterdam: North Holland.
\newblock papers intended for the {E}uropean Summer Meeting of the Association
  for Symbolic Logic.
\newblock Available from:
  \url{https://www.elsevier.com/books/logic-colloquium-80/van-dalen/978-0-444-86465-9}.

\bibitem[\protect\citeauthoryear{G{\"{o}}del}{G{\"{o}}del}{1930}]{godel-1930-Vollstaendigkeit}
G{\"{o}}del, K. (1930).
\newblock {D}ie {V}ollst\"andigkeit der {A}xiome des logischen
  {F}unktionenkalk\"uls.
\newblock {\em Monatshefte f{\"{u}}r Mathematik und Physik}, {\em 37\/}(1),
  349--360.
\newblock Available from: \url{https://doi.org/10.1007/BF01696781}, \href
  {http://dx.doi.org/10.1007/BF01696781} {\path{doi:10.1007/BF01696781}}.

\bibitem[\protect\citeauthoryear{G{\"{o}}del}{G{\"{o}}del}{1931}]{godel1}
G{\"{o}}del, K. (1931).
\newblock {\"{U}}ber formal unentscheidbare {S}\"{a}tze der {P}rincipia
  {M}athematica und verwandter {S}ysteme.
\newblock {\em Monatshefte f{\"{u}}r Mathematik und Physik}, {\em 38\/}(1),
  173--198.
\newblock Available from: \url{https://doi.org/10.1007/s00605-006-0423-7},
  \href {http://dx.doi.org/10.1007/s00605-006-0423-7}
  {\path{doi:10.1007/s00605-006-0423-7}}.

\bibitem[\protect\citeauthoryear{Gold}{Gold}{1967}]{go-67}
Gold, M.~E. (1967).
\newblock Language identification in the limit.
\newblock {\em Information and Control}, {\em 10}, 447--474.
\newblock Available from: \url{https://doi.org/10.1016/S0019-9958(67)91165-5},
  \href {http://dx.doi.org/10.1016/S0019-9958(67)91165-5}
  {\path{doi:10.1016/S0019-9958(67)91165-5}}.

\bibitem[\protect\citeauthoryear{Goldschmidt \& Pearce}{Goldschmidt \&
  Pearce}{2018}]{Goldschmidt2017-idealism}
Goldschmidt, T. \& Pearce, K.~L. (2017, 2018).
\newblock {\em Idealism: New Essays in Metaphysics}.
\newblock Oxford, UK: Oxford University Press.
\newblock Available from:
  \url{https://doi.org/10.1093/oso/9780198746973.001.0001}, \href
  {http://dx.doi.org/10.1093/oso/9780198746973.001.0001}
  {\path{doi:10.1093/oso/9780198746973.001.0001}}.

\bibitem[\protect\citeauthoryear{Hahn}{Hahn}{1930}]{Hahn1930}
Hahn, H. (1930).
\newblock {D}ie {B}edeutung der wissenschaftlichen {W}eltauffassung,
  insbesondere f{\"u}r {M}athematik und {P}hysik.
\newblock {\em Erkenntnis}, {\em 1\/}(1), 96--105.
\newblock Available from: \url{https://doi.org/10.1007/BF00208612}, \href
  {http://dx.doi.org/10.1007/BF00208612} {\path{doi:10.1007/BF00208612}}.

\bibitem[\protect\citeauthoryear{Hamilton}{Hamilton}{1963}]{Hamilton-1963}
Hamilton, W.~D. (1963).
\newblock The evolution of altruistic behavior.
\newblock {\em The American Naturalist}, {\em 97\/}(896), 354--356.
\newblock Available from: \url{https://doi.org/10.1086/497114}, \href
  {http://dx.doi.org/10.1086/497114} {\path{doi:10.1086/497114}}.

\bibitem[\protect\citeauthoryear{Hiebert}{Hiebert}{2000}]{Hiebert2000}
Hiebert, E.~N. (2000).
\newblock Common frontiers of the exact sciences and the humanities.
\newblock {\em Physics in Perspective}, {\em 2\/}(1), 6--29.
\newblock Available from: \url{https://doi.org/10.1007/s000160050034}, \href
  {http://dx.doi.org/10.1007/s000160050034} {\path{doi:10.1007/s000160050034}}.

\bibitem[\protect\citeauthoryear{Hilbert}{Hilbert}{1926}]{hilbert-26}
Hilbert, D. (1926).
\newblock {\"{U}}ber das {U}nendliche.
\newblock {\em Mathematische Annalen}, {\em 95\/}(1), 161--190.
\newblock English translation in Ref. \cite{CBO9781139171519A019}.
\newblock Available from: \url{https://doi.org/10.1007/BF01206605}, \href
  {http://dx.doi.org/10.1007/BF01206605} {\path{doi:10.1007/BF01206605}}.

\bibitem[\protect\citeauthoryear{Hilbert}{Hilbert}{1984}]{CBO9781139171519A019}
Hilbert, D. (1984).
\newblock On the infinite.
\newblock In P.~Benacerraf \& H.~Putnam (Eds.), {\em Philosophy of
  mathematics\/} (Second ed.).  (pp.\ 183--201). Cambridge, UK: Cambridge
  University Press.
\newblock Available from: \url{https://doi.org/10.1017/CBO9781139171519.010},
  \href {http://dx.doi.org/10.1017/CBO9781139171519.010}
  {\path{doi:10.1017/CBO9781139171519.010}}.

\bibitem[\protect\citeauthoryear{Hume}{Hume}{2007}]{Hume-Enquiry}
Hume, D. (1748,2007).
\newblock {\em An enquiry concerning human understanding}.
\newblock {O}xford world's classics. Oxford University Press.
\newblock edited by Peter Millican.
\newblock Available from: \url{http://www.gutenberg.org/ebooks/9662}.

\bibitem[\protect\citeauthoryear{{ID Quantique SA}}{{ID Quantique
  SA}}{2010}]{Quantis}
{ID Quantique SA} (2001-2010).
\newblock {\em {QUANTIS}. Quantum number generator}.
\newblock Geneva, Switzerland: idQuantique.
\newblock accessed on Sep 8, 2019.
\newblock Available from:
  \url{https://www.idquantique.com/random-number-generation/products/quantis-random-number-generator/}.

\bibitem[\protect\citeauthoryear{Jaynes}{Jaynes}{1990}]{Jaynes1990}
Jaynes, E.~T. (1990).
\newblock Probability theory as logic.
\newblock In Foug{\`e}re, P.~F. (Ed.), {\em Maximum Entropy and {B}ayesian
  Methods}, (pp.\ 1--16)., Dordrecht. Springer Netherlands.
\newblock Available from: \url{https://doi.org/10.1007/978-94-009-0683-9\_1},
  \href {http://dx.doi.org/10.1007/978-94-009-0683-9\_1}
  {\path{doi:10.1007/978-94-009-0683-9\_1}}.

\bibitem[\protect\citeauthoryear{Kleene}{Kleene}{1936}]{Kleene1936}
Kleene, S.~C. (1936).
\newblock General recursive functions of natural numbers.
\newblock {\em Mathematische Annalen}, {\em 112\/}(1), 727--742.
\newblock Available from: \url{https://doi.org/10.1007/BF01565439}, \href
  {http://dx.doi.org/10.1007/BF01565439} {\path{doi:10.1007/BF01565439}}.

\bibitem[\protect\citeauthoryear{Kochen \& Specker}{Kochen \&
  Specker}{1967}]{kochen1}
Kochen, S. \& Specker, E.~P. (1967).
\newblock The problem of hidden variables in quantum mechanics.
\newblock {\em Journal of Mathematics and Mechanics (now Indiana University
  Mathematics Journal)}, {\em 17\/}(1), 59--87.
\newblock Available from: \url{https://doi.org/10.1512/iumj.1968.17.17004},
  \href {http://dx.doi.org/10.1512/iumj.1968.17.17004}
  {\path{doi:10.1512/iumj.1968.17.17004}}.

\bibitem[\protect\citeauthoryear{Lakatos}{Lakatos}{1978}]{lakatosch}
Lakatos, I. (1978).
\newblock {\em Philosophical Papers. 1. {T}he Methodology of Scientific
  Research Programmes}.
\newblock Cambridge: Cambridge University Press.

\bibitem[\protect\citeauthoryear{Li \& Vit{\'{a}}nyi}{Li \&
  Vit{\'{a}}nyi}{1992}]{li:92}
Li, M. \& Vit{\'{a}}nyi, P. M.~B. (1992).
\newblock Inductive reasoning and {K}olmogorov complexity.
\newblock {\em Journal of Computer and System Science}, {\em 44}, 343--384.
\newblock Available from: \url{https://doi.org/10.1016/0022-0000(92)90026-F},
  \href {http://dx.doi.org/10.1016/0022-0000(92)90026-F}
  {\path{doi:10.1016/0022-0000(92)90026-F}}.

\bibitem[\protect\citeauthoryear{Martin-L{\"{o}}f}{Martin-L{\"{o}}f}{1966}]{martin-lof}
Martin-L{\"{o}}f, P. (1966).
\newblock The definition of random sequences.
\newblock {\em Information and Control}, {\em 9\/}(6), 602--619.
\newblock Available from: \url{https://doi.org/10.1016/0030-4018(87)90271-9},
  \href {http://dx.doi.org/10.1016/S0019-9958(66)80018-9}
  {\path{doi:10.1016/S0019-9958(66)80018-9}}.

\bibitem[\protect\citeauthoryear{Maxwell}{Maxwell}{1999}]{Campbell-1882}
Maxwell, J.~C. (1882, 1999).
\newblock Does the progress of physical science tend to give any advantage to
  the opinion of necessity (or determinism) over that of the contingency of
  events and the freedom of the will?
\newblock In L.~Campbell \& W.~Garnett (Eds.), {\em The life of {J}ames {C}lerk
  {M}axwell. {W}ith a selection from his correspondence and occasional writings
  and a sketch of his contributions to science\/} (second ed.). London:
  MacMillan.
\newblock Available from:
  \url{https://archive.org/details/lifeofjamesclerk00camprich}.

\bibitem[\protect\citeauthoryear{Mueller}{Mueller}{2017}]{Mueller-2017}
Mueller, M.~P. (2017).
\newblock Could the physical world be emergent instead of fundamental, and why
  should we ask? (short version).
\newblock Available from: \url{https://arxiv.org/abs/1712.01816}, \href
  {http://arxiv.org/abs/arXiv:1712.01816} {\path{arXiv:arXiv:1712.01816}}.

\bibitem[\protect\citeauthoryear{Nestler, Barrot \& Self}{Nestler
  et~al.}{2001}]{Nestler11042}
Nestler, E.~J., Barrot, M., \& Self, D.~W. (2001).
\newblock {$\Delta$FosB}: {A} sustained molecular switch for addiction.
\newblock {\em Proceedings of the National Academy of Sciences}, {\em
  98\/}(20), 11042--11046.
\newblock Available from: \url{https://doi.org/10.1073/pnas.191352698}, \href
  {http://dx.doi.org/10.1073/pnas.191352698}
  {\path{doi:10.1073/pnas.191352698}}.

\bibitem[\protect\citeauthoryear{Nietzsche}{Nietzsche}{009
  a}]{Nietzsche-ZarathustraI}
Nietzsche, F. (1874, 1872, 1878, 2009-a).
\newblock {\em {A}lso sprach {Z}arathustra. {E}in {B}uch f\"ur {A}lle und
  {K}einen}.
\newblock Digital critical edition of the complete works and letters, based on
  the critical text by G. Colli and M. Montinari, Berlin/New York, de Gruyter
  1967-, edited by Paolo D'Iorio.
\newblock Available from: \url{http://www.nietzschesource.org/\#eKGWB/Za-I}.

\bibitem[\protect\citeauthoryear{Nietzsche}{Nietzsche}{009
  b}]{Nietzsche-EcceHomo}
Nietzsche, F. (1874, 1872, 1878, 2009-b).
\newblock {\em {E}cce homo. {W}ie man wird, was man ist}.
\newblock Digital critical edition of the complete works and letters, based on
  the critical text by G. Colli and M. Montinari, Berlin/New York, de Gruyter
  1967-, edited by Paolo D'Iorio.
\newblock Available from: \url{http://www.nietzschesource.org/\#eKGWB/EH}.

\bibitem[\protect\citeauthoryear{Norton}{Norton}{2003}]{Norton-2003-cafs}
Norton, J.~D. (2003).
\newblock Causation as folk science.
\newblock {\em Philosophers' Imprint}, {\em 3\/}(4), 1--22.
\newblock Available from:
  \url{http://hdl.handle.net/2027/spo.3521354.0003.004}.

\bibitem[\protect\citeauthoryear{Norton}{Norton}{2008}]{Norton-dome-2008}
Norton, J.~D. (2008).
\newblock The dome: An unexpectedly simple failure of determinism.
\newblock {\em Philosophy of Science}, {\em 75\/}(5), 786--798.
\newblock Available from: \url{https://doi.org/10.1086/594524}, \href
  {http://arxiv.org/abs/http://philsci-archive.pitt.edu/2943/}
  {\path{arXiv:http://philsci-archive.pitt.edu/2943/}}, \href
  {http://dx.doi.org/10.1086/594524} {\path{doi:10.1086/594524}}.

\bibitem[\protect\citeauthoryear{Peres}{Peres}{1993}]{peres}
Peres, A. (1993).
\newblock {\em Quantum Theory: Concepts and Methods}.
\newblock Dordrecht: Kluwer Academic Publishers.

\bibitem[\protect\citeauthoryear{Rosen}{Rosen}{2010}]{lawlses_rosen2010}
Rosen, J. (2010).
\newblock {\em Lawless Universe: Science and the Hunt for Reality}.
\newblock Balrtimreo, Maryland: The John Hopkins University Press.

\bibitem[\protect\citeauthoryear{Schopenhauer}{Schopenhauer}{1912}]{schopenhauer-dwawuv-VI}
Schopenhauer, A. (1819, 1912).
\newblock {\em {D}ie {W}elt als {W}ille und {V}orstellung. {E}rster {B}and\/}
  (third ed.).
\newblock M\"unchen: Georg M\"uller.
\newblock Available from:
  \url{https://archive.org/details/dieweltalswilleu00scho}.

\bibitem[\protect\citeauthoryear{Schr{\"{o}}dinger}{Schr{\"{o}}dinger}{2014}]{schroed:natgr}
Schr{\"{o}}dinger, E. (1954, 2014).
\newblock {\em Nature and the Greeks}.
\newblock Cambridge: Cambridge University Press.
\newblock Available from: \url{http://www.cambridge.org/9781107431836}.

\bibitem[\protect\citeauthoryear{Schweidler}{Schweidler}{1906}]{schweidler-1905}
Schweidler, E.~v. (1906).
\newblock {\em {\"U}ber {S}chwankungen der radioaktiven {U}mwandlung}, (pp.\
  German part, 1--3).
\newblock Paris: H. Dunod \& E. Pinat.
\newblock Available from:
  \url{https://archive.org/details/premiercongrsin03unkngoog}.

\bibitem[\protect\citeauthoryear{Sloane}{Sloane}{2019}]{Sloane_oeis.org/A033307}
Sloane, N. J.~A. (2019).
\newblock {A033307} {D}ecimal expansion of {C}hampernowne constant (or
  {M}ahler's number), formed by concatenating the positive integers.
\newblock accessed on Sep 8th, 2019.
\newblock Available from: \url{https://oeis.org/A033307}.

\bibitem[\protect\citeauthoryear{Smoluchowski}{Smoluchowski}{1912}]{Smoluchovski-1912}
Smoluchowski, M. (1912).
\newblock {E}xperimentell nachweisbare, der {\"u}blichen {T}hermodynamik
  widersprechende {M}olekularph{\"a}nomene.
\newblock {\em Physikalische Zeitschrift}, {\em 13}, 1069--1080.
\newblock Available from:
  \url{http://matwbn.icm.edu.pl/ksiazki/pms/pms2/pms2122.pdf}.

\bibitem[\protect\citeauthoryear{Smullyan}{Smullyan}{1992}]{smullyan-92}
Smullyan, R.~M. (1992).
\newblock {\em {G}{\"{o}}del's Incompleteness Theorems}.
\newblock Oxford Logic Guides 19. New York, Oxford: Oxford University Press.

\bibitem[\protect\citeauthoryear{Smullyan}{Smullyan}{1993}]{Smullyan1993-SMURTF}
Smullyan, R.~M. (1993).
\newblock {\em Recursion Theory for Metamathematics}.
\newblock Oxford Logic Guides 22. New York, Oxford: Oxford University Press.

\bibitem[\protect\citeauthoryear{Smullyan}{Smullyan}{1994}]{book:486992}
Smullyan, R.~M. (1994).
\newblock {\em Diagonalization and Self-Reference}.
\newblock Oxford Logic Guides 27. New York, Oxford: Clarendon Press.

\bibitem[\protect\citeauthoryear{St\"oltzner}{St\"oltzner}{1999}]{Stoeltzner-1999}
St\"oltzner, M. (1999).
\newblock {V}ienna indeterminism: {M}ach, {B}oltzmann, {E}xner.
\newblock {\em Synthese}, {\em 119}, 85--111.
\newblock Available from: \url{https://doi.org/10.1023/a:1005243320885}, \href
  {http://dx.doi.org/10.1023/a:1005243320885}
  {\path{doi:10.1023/a:1005243320885}}.

\bibitem[\protect\citeauthoryear{Svozil}{Svozil}{1993}]{svozil-93}
Svozil, K. (1993).
\newblock {\em Randomness \& Undecidability in Physics}.
\newblock Singapore: World Scientific.
\newblock Available from: \url{https://doi.org/10.1142/1524}, \href
  {http://dx.doi.org/10.1142/1524} {\path{doi:10.1142/1524}}.

\bibitem[\protect\citeauthoryear{Svozil}{Svozil}{1994}]{svozil-94}
Svozil, K. (1994).
\newblock Extrinsic-intrinsic concept and complementarity.
\newblock In H.~Atmanspacher \& G.~J. Dalenoort (Eds.), {\em Inside versus
  Outside}, volume~63 of {\em Springer Series in Synergetics}  (pp.\ 273--288).
  Berlin Heidelberg: Springer.
\newblock Available from: \url{https://doi.org/10.1007/978-3-642-48647-0\_15},
  \href {http://dx.doi.org/10.1007/978-3-642-48647-0\_15}
  {\path{doi:10.1007/978-3-642-48647-0\_15}}.

\bibitem[\protect\citeauthoryear{Svozil}{Svozil}{1996}]{svozil-unev}
Svozil, K. (1996).
\newblock Undecidability everywhere?
\newblock In J.~L. Casti \& A.~Karlquist (Eds.), {\em Boundaries and Barriers.
  On the Limits to Scientific Knowledge}  (pp.\ 215--237). Reading, MA:
  Addison-Wesley.

\bibitem[\protect\citeauthoryear{Svozil}{Svozil}{2004}]{svozil-2003-garda}
Svozil, K. (2004).
\newblock Quantum information via state partitions and the context translation
  principle.
\newblock {\em Journal of Modern Optics}, {\em 51}, 811--819.
\newblock Available from: \url{https://doi.org/10.1080/09500340410001664179},
  \href {http://arxiv.org/abs/arXiv:quant-ph/0308110}
  {\path{arXiv:arXiv:quant-ph/0308110}}, \href
  {http://dx.doi.org/10.1080/09500340410001664179}
  {\path{doi:10.1080/09500340410001664179}}.

\bibitem[\protect\citeauthoryear{Svozil}{Svozil}{2011}]{svozil-07-physical_unknowables}
Svozil, K. (2011).
\newblock Physical unknowables.
\newblock In M.~Baaz, C.~H. Papadimitriou, H.~W. Putnam, D.~S. Scott, \& C.~L.
  {Harper, Jr} (Eds.), {\em {K}urt {G}{\"o}del and the Foundations of
  Mathematics}  (pp.\ 213--251). Cambridge, UK: Cambridge University Press.
\newblock Available from: \url{https://doi.org/10.1017/CBO9780511974236.013},
  \href {http://arxiv.org/abs/arXiv:physics/0701163}
  {\path{arXiv:arXiv:physics/0701163}}, \href
  {http://dx.doi.org/10.1017/CBO9780511974236.013}
  {\path{doi:10.1017/CBO9780511974236.013}}.

\bibitem[\protect\citeauthoryear{Svozil}{Svozil}{2018}]{svozil-pac}
Svozil, K. (2018).
\newblock {\em Physical (A)Causality}, volume 192 of {\em Fundamental Theories
  of Physics}.
\newblock Cham, Heidelberg, New York, Dordrecht, London: Springer International
  Publishing.
\newblock Available from: \url{https://doi.org/10.1007/978-3-319-70815-7},
  \href {http://dx.doi.org/10.1007/978-3-319-70815-7}
  {\path{doi:10.1007/978-3-319-70815-7}}.

\bibitem[\protect\citeauthoryear{Swinburne}{Swinburne}{1970}]{Swinburne-Miracle}
Swinburne, R. (1970).
\newblock {\em The Concept of Miracle}.
\newblock New Studies in the Philosophy of Religion. London: Palgrave
  Macmillan.
\newblock Available from: \url{https://doi.org/10.1007/978-1-349-00776-9},
  \href {http://dx.doi.org/10.1007/978-1-349-00776-9}
  {\path{doi:10.1007/978-1-349-00776-9}}.

\bibitem[\protect\citeauthoryear{Toffoli}{Toffoli}{1978}]{toffoli:79}
Toffoli, T. (1978).
\newblock The role of the observer in uniform systems.
\newblock In G.~J. Klir (Ed.), {\em Applied General Systems Research: Recent
  Developments and Trends}  (pp.\ 395--400). New York, London, and Boston, MA:
  Plenum Press, Springer US.
\newblock Available from: \url{https://doi.org/10.1007/978-1-4757-0555-3\_29},
  \href {http://dx.doi.org/10.1007/978-1-4757-0555-3\_29}
  {\path{doi:10.1007/978-1-4757-0555-3\_29}}.

\bibitem[\protect\citeauthoryear{{T}uring}{{T}uring}{1937}]{turing-36}
{T}uring, A.~M. (1936-7 and 1937).
\newblock On computable numbers, with an application to the
  {E}ntscheidungsproblem.
\newblock {\em Proceedings of the London Mathematical Society, Series 2}, {\em
  42, 43}, 230--265, 544--546.
\newblock Available from: \url{https://doi.org/10.1112/plms/s2-42.1.230,
  https://doi.org/10.1112/plms/s2-43.6.544}, \href
  {http://dx.doi.org/10.1112/plms/s2-42.1.230, 10.1112/plms/s2-43.6.544}
  {\path{doi:10.1112/plms/s2-42.1.230, 10.1112/plms/s2-43.6.544}}.

\bibitem[\protect\citeauthoryear{Uffink}{Uffink}{2011}]{Uffink2011-UFFSPS}
Uffink, J. (2011).
\newblock Subjective probability and statistical physics.
\newblock In C.~Beisbart \& S.~Hartmann (Eds.), {\em Probabilities in Physics}
  (pp.\ 25--49). Oxford, UK: Oxford University Press.
\newblock Available from:
  \url{https://doi.org/10.1093/acprof:oso/9780199577439.003.0002}, \href
  {http://dx.doi.org/10.1093/acprof:oso/9780199577439.003.0002}
  {\path{doi:10.1093/acprof:oso/9780199577439.003.0002}}.

\bibitem[\protect\citeauthoryear{van Fraassen}{van
  Fraassen}{2003}]{vanFraassen1989-VANLAS}
van Fraassen, B.~C. (1989, 2003).
\newblock {\em Laws and Symmetry}.
\newblock Oxford University Press.
\newblock Available from: \url{https://doi.org/10.1093/0198248601.001.0001},
  \href {http://dx.doi.org/10.1093/0198248601.001.0001}
  {\path{doi:10.1093/0198248601.001.0001}}.

\bibitem[\protect\citeauthoryear{van Strien}{van Strien}{2014}]{vanStrien2014}
van Strien, M. (2014).
\newblock The {N}orton dome and the nineteenth century foundations of
  determinism.
\newblock {\em Journal for General Philosophy of Science}, {\em 45\/}(1),
  167--185.
\newblock Available from: \url{https://doi.org/10.1007/s10838-014-9241-0},
  \href {http://dx.doi.org/10.1007/s10838-014-9241-0}
  {\path{doi:10.1007/s10838-014-9241-0}}.

\bibitem[\protect\citeauthoryear{Voltaire}{Voltaire}{1764}]{voltaire-dict}
Voltaire (1764).
\newblock {\em A Philosophical Dictionary}.
\newblock Derived from The Works of Voltaire, A Contemporary Version, (New
  York: E.R. DuMont, 1901).
\newblock Available from:
  \url{https://ebooks.adelaide.edu.au/v/voltaire/dictionary/}.

\bibitem[\protect\citeauthoryear{{von Neumann}}{{von
  Neumann}}{1996}]{v-neumann-49}
{von Neumann}, J. (1932, 1996).
\newblock {\em {M}athematische {G}rundlagen der {Q}uantenmechanik\/} (second
  ed.).
\newblock Berlin, Heidelberg: Springer.
\newblock {E}nglish translation in Ref.~\cite{v-neumann-55}.
\newblock Available from: \url{https://doi.org/10.1007/978-3-642-61409-5},
  \href {http://dx.doi.org/10.1007/978-3-642-61409-5}
  {\path{doi:10.1007/978-3-642-61409-5}}.

\bibitem[\protect\citeauthoryear{{von Neumann}}{{von
  Neumann}}{1955}]{v-neumann-55}
{von Neumann}, J. (1955).
\newblock {\em Mathematical Foundations of Quantum Mechanics}.
\newblock Princeton, NJ: Princeton University Press.
\newblock {G}erman original in Ref.~\cite{v-neumann-49}.
\newblock Available from: \url{http://press.princeton.edu/titles/2113.html}.

\bibitem[\protect\citeauthoryear{Watzlawick, Beavin \& Jackson}{Watzlawick
  et~al.}{1967}]{Watzlawick-1967}
Watzlawick, P., Beavin, J.~H., \& Jackson, D.~D. (1967).
\newblock {\em Pragmatics of Human Communication: A Study of Interactional
  Patterns, Pathologies, and Paradoxes}.
\newblock New York: W. W. Norton \& Company.

\bibitem[\protect\citeauthoryear{Wigner}{Wigner}{1962}]{wigner:mb}
Wigner, E.~P. (1961,1962).
\newblock Remarks on the mind-body question.
\newblock In I.~J. Good (Ed.), {\em The Scientist Speculates}  (pp.\ 284--302).
  London and New York: Heinemann and Basic Books.
\newblock Available from: \url{https://doi.org/10.1007/978-3-642-78374-6\_20},
  \href {http://dx.doi.org/10.1007/978-3-642-78374-6\_20}
  {\path{doi:10.1007/978-3-642-78374-6\_20}}.

\bibitem[\protect\citeauthoryear{Yanofsky}{Yanofsky}{2017}]{chaos_multiverse2017}
Yanofsky, N. (2017).
\newblock Chaos makes the multiverse unnecessary.
\newblock {\em Nautilus}, 1--16.
\newblock Available from:
  \url{http://nautil.us/issue/49/the-absurd/chaos-makes-the-multiverse-unnecessary}.

\bibitem[\protect\citeauthoryear{Zeilinger}{Zeilinger}{2005}]{zeil-05_nature_ofQuantum}
Zeilinger, A. (2005).
\newblock The message of the quantum.
\newblock {\em Nature}, {\em 438}, 743.
\newblock Available from: \url{https://doi.org/10.1038/438743a}, \href
  {http://dx.doi.org/10.1038/438743a} {\path{doi:10.1038/438743a}}.

\end{thebibliography}

\end{document}
