%\documentclass[pra,showpacs,showkeys,amsfonts,amsmath,twocolumn]{revtex4}
\documentclass[amsmath,red]{beamer}
%\documentclass[pra,showpacs,showkeys,amsfonts]{revtex4}

%\usepackage{beamerthemeshadow}
%\usepackage[dark]{beamerthemesidebar}
%\usepackage[headheight=24pt,footheight=12pt]{beamerthemesplit}
\usepackage{beamerthemesplit}
%\usepackage[bar]{beamerthemetree}
\usepackage{graphicx}
\usepackage{pgf}

\pgfdeclareimage[height=0.6cm]{logo}{tu-logo}
\logo{\pgfuseimage{logo}}

\title{\bf The Diagonalization Method in Quantum Recursion Theory}
\subtitle{http://www.arxiv.org/abs/hep-th/9412048}
\author{Karl Svozil}
\institute{Institut f\"ur Theoretische Physik, University of Technology Vienna, \\
Wiedner Hauptstra\ss e 8-10/136, A-1040 Vienna, Austria}
\begin{document}
\maketitle

%\frame{\tableofcontents}


\section{Principle of computational adequacy}
\subsection{Statement}

\frame[shrink=2]{
\frametitle{Principle of computational adequacy}

In order for a computational model to be useful it should satisfy two criteria:

\begin{itemize}
\item<+-> {\em Operationalizability:} Every capacity of the model should be reflected in a physical capacity.

\item<+-> {\em Sufficiency:} Every physical capacity should be reflected in a theoretical entity.

\item<+-> $\Longrightarrow $ {\em Computational adequacy:}
``Good'' correspondence between a physical system and its theoretical, computational double.


\item<+-> {\em ``Every system is a perfect simulacron of itself ;)''}
\end{itemize}
 }

\subsection{Examples}
\frame[shrink=2]{
\frametitle{Examples of computational adequacy}

\begin{itemize}
\item<+-> Informal Algorithm $\longleftrightarrow$ Turing machine $\longleftrightarrow$ Recursive functions

\item<+-> Complementarity  $\longleftrightarrow$ Computational complementarity (E.~Moore,~1956)  $\longleftrightarrow$ finite automata partition logics  $\longleftrightarrow$ generalized urn models (Wright,~1978)
\end{itemize}
 }


\subsection{Known classical issues}
\frame[shrink=2]{
\frametitle{Known classical issues}

\begin{itemize}
\item<+->
Nonconstructive entities occur in physical theory: classical continuum theory (mechanics \& electrodynamics) and quantum mechanics (wave function, space \& time and other parameters)

\item<+->
Nonconstructive theoretical entities give rise to ``mind boggling physics;'' e.g., classical chaos through random initial values, Banach-Tarski paradoxes etc.
e.g., total randomness through algorithmic incompressibility

\item<+-> ``Breaking the Turing barrier'' via ``Zeno squeezing'' (Weyl,~1949);
i.e., geometric progression of intrinsic time cycles ``allows'' infinite computations in finite observer time.

\end{itemize}
 }

\subsection{Known quantum issues}
\frame[shrink=2]{
\frametitle{Known quantum issues}

\begin{itemize}


\item<+-> ``Breaking the Turing barrier'' quantum mechanically (Calude \& Pavlov,~2002, Kieu,~2002);
caused by Zeno squeezing of state space?

\item<+->  {\em ``There are good reasons to assume that nature cannot be represented by a continuous field. From quantum theory it could be inferred with certainty that a finite system with finite energy can be completely described by a finite number of (quantum) numbers. This seems not in accordance with continuum theory and has to render trials to describe reality with purely algebraic means. However, nobody has any idea of how one can find the basis of such a theory.''}
(Einstein,~1956)


\end{itemize}
 }


\section{Diagonalization}
\subsection{State of information represented by a complex continuum}
\frame[shrink=2]{
\frametitle{State of information represented by a complex continuum}
\begin{itemize}

\item<+->
Classical logical states are coded by
$\ulcorner t\urcorner =1$ and
$\ulcorner f\urcorner =0$ ($
\ulcorner
s
\urcorner$ stands for the code of $s$).


\item<+->
{\em Quantum bits} (qubits)
are physically represented by a coherent
superposition
of the two classical states $t$ and $f$.
The qbit states
$$
x_{\alpha ,\beta}  =\alpha t+\beta f
$$
They form a continuum, with
$ \vert \alpha \vert^2+\vert \beta \vert^2=1$, $\alpha ,\beta \in {\Bbb
C}$.

\item<+->
Cbits can then be coded by
$$
\ulcorner
x_{\alpha ,\beta }\urcorner  =(\alpha ,\beta )=
e^{i\varphi } (\sin \omega  ,e^{i\delta } \cos \omega )\quad ,
$$
with
$\omega ,\varphi ,\delta \in {\Bbb R}$.
Qbits can be identified with cbits as follows
$$
(1,0)\equiv 1
\mbox{ and }
(0,1)\equiv 0
\quad .
$$

\end{itemize}
}

\subsection{Classical diagonalization}
\frame[shrink=2]{
\frametitle{Classical diagonalization}
\begin{itemize}

\item<+->
For the sake of
contradiction, consider a universal computer $C$ and an arbitrary algorithm
$B(X)$ whose input is a string of symbols $X$.  Assume that there exists
a hypothetical ``halting algorithm'' ${\tt HALT}(B(X))$ which is able to decide whether $B$
terminates on $X$ or not.
The domain of ${\tt HALT}$  is the set of legal programs.
The range of ${\tt HALT}$ are cbits (classical case) and qubits (quantum
mechanical case).

\item<+->
Using ${\tt HALT}(B(X))$,  another deterministic
algorithm  $A$ can be constructed, which accepts as input any effective program $B$ and
which proceeds as follows:  Upon reading the program $B$ as input, $A$
makes a copy of it.  This can be readily achieved, since the program $B$
is presented to $A$ in some encoded form
$\ulcorner B\urcorner $,
i.e., as a string of
symbols.

\item<+->
In the next step, $A$ uses the code
$\ulcorner B\urcorner $
 as input
string for $B$ itself; i.e., $A$ forms  $B(\ulcorner B\urcorner )$,
henceforth denoted by
$B(B)$.

\item<+->
Then, $A$ hands $B(B)$ over to a hypothetical subroutine ${\tt HALT}$
which is assumed to be capable of deciding whether or not $B(B)$ converges; i.e., halts.

\item<+->
Finally, $A$ proceeds as follows:  if ${\tt HALT}(B(B))$ decides that
$B(B)$ converges, then $A$ does not halt.
(This can for instance be realized by an infinite {\tt DO}-loop.)
Alternatively, if ${\tt HALT}(B(B))$ decides
that $B(B)$ diverges, then $A$ halts.

\end{itemize}
}

\frame[shrink=2]{
\frametitle{Classical diagonalization (cntd.)}

\begin{itemize}

\item<+->
The agent $A$ will now be confronted with the following paradoxical
task:  take the own code as input and proceed.


\item<+->
 Assume that $A$ is
restricted to classical bits of information.
To be more specific,
assume that ${\tt HALT}$ outputs the code of a cbit as follows
($\uparrow$ and $\downarrow$ stands for divergence and convergence,
respectively):
$$
{\tt HALT} ( B(X) ) =\left\{
 \begin{array}{l}
0 \mbox{ if } B(X) \uparrow
\\
1 \mbox{ if } B(X) \downarrow \\
\end{array}
 \right.
\quad .
$$


\item<+->
Whenever $A(A)$
halts, ${\tt HALT}(A(A))$ outputs $1$ and forces $A(A)$ not to halt.
Conversely,
whenever $A(A)$ does not halt, then ${\tt HALT}(A(A))$ outputs $0$
and steers
$A(A)$ into the halting mode.  In both cases one arrives at a complete
contradiction.  Classically, this contradiction can only be consistently
avoided by assuming the nonexistence of $A$ and, since the only
nontrivial feature of $A$ is the use of the peculiar halting algorithm
${\tt HALT}$, the impossibility of any such halting algorithm.

\end{itemize}
}


\subsection{Quantum diagonalization}
\frame[shrink=2]{
\frametitle{Quantum diagonalization}

The task of the agent $A$
can be performed consistently if
$A$ is allowed to process quantum information.
Assume that the output of the hypothetical
``halting algorithm'' is a qbit
$$
{\tt HALT} ( B(X) ) = x_{\alpha , \beta}
\quad .
$$
We may think of   ${\tt HALT} ( B(X) )$ as a universal computer $C'$
simulating $C$ and containing a dedicated {\em halting bit}, which it
the output of $C'$
at every (discrete) time cycle. Initially (at time zero),
this halting bit is prepared to be a 50:50 mixture of the
classical halting and non-halting states $t$ and $f$; i.e.,
$x_{{1\over \sqrt{2} }, {1\over \sqrt{2} }}$. If later $C'$ finds that $C$ converges
(diverges) on $B(X)$, then the halting bit of $C'$ is set to the
classical value $t$ ($f$).


}


\frame[shrink=2]{
\frametitle{Quantum diagonalization by {\tt not} operators}

\begin{itemize}

\item<+->  Take
$$
t  \equiv
\left(
\begin{array}{c}
1 \\
0
 \end{array}
\right)
\mbox{ and }
f \equiv
\left(
\begin{array}{c}
0 \\
1
 \end{array}
\right) \quad .
$$

\item<+->
 The evolution representing diagonalization (effectively, agent
$A$'s task) can be expressed by the unitary operator $D$ by
$$
D t  =  f \mbox{ and }
D f  =  t\quad .
$$
Thus, $D$ acts essentially as a ${\tt not}$-gate.
In the above state basis, $D$ can be represented as follows:
$$
D=
\left(
\begin{array}{cc}
0 & 1\\
1 & 0
\end{array}
\right) \quad .
$$
$ D $ will be called {\em diagonalization} operator.

\item<+->
Suppose a hypothetical quantum halting algorithm ${\tt HALT} ( A(A) ) = x^\ast$
responds with the qubit $x^\ast$, which is the fixed point of $D$.
When applied to the classical diagonalization argument, $x^\ast$
does not give rise to inconsistencies.
For, if a quantum algorithm $A$ processes the fixed point state
$x ^\ast $ through the diagonalization
operator $D$, the same state
$x^\ast $ is recovered.
\end{itemize}
}



\frame[shrink=2]{
\frametitle{Is it helpful?}
\begin{itemize}

\item<+->
Any single measurement will yield an indeterministic result.
There is a 50:50 chance that
the fixed point state will be either in $t$ or $f$, since
$P_t(
x ^\ast)=
P_f(
x^\ast )= {1\over 2}$.
Thereby, classical undecidability is recovered.

\item<+->
Possible representation of inconsistent information in databases.
Thereby,
two contradicting cbits of information
$t$ and
$f $ are resolved by the qbit
$x^\ast =
{(t+f)/ \sqrt{2}}$.
Throughout the rest of the computation the coherence is maintained.
After the processing, the result is obtained by an irreversible
measurement. The processing of qbits requires an
exponential amount of space.
\end{itemize}
}

\subsection{Nonclassical diagonalization procedures}
\frame[shrink=2]{
\frametitle{Nonclassical diagonalization procedures}


Allow
the entire range of twodimensional unitary transformations
$$
U_2(\omega ,\alpha ,\beta ,\varphi )=e^{-i\,\beta}\,
\left(
\begin{array}{cc}
{e^{i\,\alpha }}\,\cos \omega
&
{-e^{-i\,\varphi }}\,\sin \omega
\\
{e^{i\,\varphi }}\,\sin \omega
&
{e^{-i\,\alpha }}\,\cos \omega
 \end{array}
\right)
 \quad ,
$$
where $-\pi \le \beta ,\omega \le \pi$,
$-\, {\pi \over 2} \le  \alpha ,\varphi \le {\pi \over 2}$, to act on
the qbit.
A typical example of a nonclassical operation on a qbit is
the ``square root of not'' gate
($
\sqrt{{\tt not}}
\sqrt{{\tt not}} =D$)
$$
\sqrt{{\tt not}} =
{1 \over 2}
\left(
\begin{array}{cc}
1+i&1-i
\\
1-i&1+i
 \end{array}
\right)
\quad .
$$
}

\frame[shrink=2]{
\frametitle{Nonclassical diagonalization procedures (cntd.)}
Not all these unitary transformations have eigenvectors
associated with eigenvalues $1$ and thus fixed points.
Indeed, it is not difficult to see that only
unitary transformations of the form
$$
\begin{array}{l}
[U_2(\omega ,\alpha ,\beta ,\varphi )]^{-1}\,\mbox{diag}(1, e^{i\lambda
}) U_2(\omega ,\alpha ,\beta ,\varphi )= \\
\qquad
\qquad
\left(
\begin{array}{cc}
{{\cos^2 \omega }} + {e^{i\,\lambda }}\,{{\sin^2 \omega }}&
{{{
{-1 + {e^{i\,\lambda
}}\over 2}
e^{-i\,\left(\alpha +\varphi \right) }}\,
\, \sin (2\,\omega )}} \\
{ -1 + {e^{i\,\lambda }}\over 2}
 {{{e^{i\,\left(\alpha
+\varphi \right) }}\,
 \sin
(2\,\omega )}}&
{e^{i\,\lambda }}\,{{\cos^2 \omega }} + {{\sin^2 \omega }}
 \end{array}
\right)
 \end{array}
$$
have fixed points.
}

\frame[shrink=2]{
\frametitle{Nonclassical diagonalization procedures (cntd.)}
Applying nonclassical operations on qbits with no fixed points
$$
\begin{array}{l}
[U_2(\omega ,\alpha ,\beta ,\varphi )]^{-1}\,\mbox{diag}( e^{i\mu } ,
e^{i\lambda }) U_2(\omega ,\alpha ,\beta ,\varphi )= \\
\left(
\begin{array}{cc}
  {e^{i\,\mu }}\,{{\cos^2 \omega }} +
     {e^{i\,\lambda }}\,{{\sin^2 \omega }}&
    {{{e^{-i\,\left( \alpha  + p \right) }\over 2}}\,
         \left( {e^{i\,\lambda }} - {e^{i\,\mu }} \right) \,\sin (2\,\omega )}
       \\
{{{e^{i\,\left( \alpha  + p \right) }\over 2}}\,
        \left( {e^{i\,\lambda }} - {e^{i\,\mu }}  \right) \,\sin (2\,\omega )}
       &{e^{i\,\lambda }}\,{{\cos^2 \omega }} +
     {e^{i\,\mu }}\,{{\sin^2 \omega }}
 \end{array}
\right)
 \end{array}
$$
with $\mu ,\lambda \neq n\pi$, $n\in {\Bbb N}_0$ gives rise to
eigenvectors which are not fixed points, but which acquire nonvanishing
phases $\mu , \lambda$ in the generalized diagonalization process.

}

\frame[shrink=4]{
\frametitle{Two designs of beam splitters yielding universal unitary transformations
$
{\bf T} (\alpha ,\beta ,\omega ,\varphi )=
i \, e^{i(\beta +{\omega \over 2})}\;\left(
\begin{array}{cc}
-e^{i(\alpha +  \varphi )}\sin {\omega \over 2}
&
e^{i  \varphi }\cos {\omega \over 2} \\
e^{i  \alpha }\cos {\omega \over 2}
&
\sin {{\omega }\over 2}
 \end{array}\right)
$}
%\begin{tabular}[tt]{p{300pt}p{300pt}}

\begin{center}
%TexCad Options
%\grade{\on}
%\emlines{\off}
%\beziermacro{\on}
%\reduce{\on}
%\snapping{\off}
%\quality{2.00}
%\graddiff{0.01}
%\snapasp{1}
%\zoom{0.50}
\unitlength 0.40mm
\linethickness{0.4pt}
\begin{picture}(270.00,94.67)
\put(20.00,14.67){\framebox(80.00,80.00)[cc]{}}
\put(57.67,54.67){\line(1,0){5.00}}
\put(64.33,54.67){\line(1,0){5.00}}
\put(50.67,54.67){\line(1,0){5.00}}
\put(78.67,64.67){\framebox(8.00,4.33)[cc]{}}
\put(82.67,72.67){\makebox(0,0)[cc]{$P_3,\varphi$}}
\put(73.33,54.67){\makebox(0,0)[lc]{$S(T)$}}
\put(8.33,78.34){\makebox(0,0)[cc]{${\bf 0}$}}
\put(110.67,78.34){\makebox(0,0)[cc]{${\bf 0}'$}}
\put(110.67,38.34){\makebox(0,0)[cc]{${\bf 1}'$}}
\put(8.00,38.34){\makebox(0,0)[cc]{${\bf 1}$}}
\put(24.33,100.34){\makebox(0,0)[lc]{${\bf T}^{bs}(\omega ,\alpha ,\beta ,\varphi )$}}
\put(0.00,74.34){\vector(1,0){20.00}}
\put(0.00,34.67){\vector(1,0){20.00}}
\put(100.00,74.67){\vector(1,0){20.00}}
\put(100.00,34.67){\vector(1,0){20.00}}
\put(170.00,14.67){\framebox(80.00,80.00)[cc]{}}
\put(170.00,34.67){\line(1,1){40.00}}
\put(210.00,74.67){\line(1,-1){40.00}}
\put(170.00,74.67){\line(1,-1){40.00}}
\put(210.00,34.67){\line(1,1){40.00}}
\put(205.00,74.67){\line(1,0){10.00}}
\put(205.00,34.67){\line(1,0){10.00}}
\put(187.67,54.67){\line(1,0){5.00}}
\put(194.33,54.67){\line(1,0){5.00}}
\put(180.67,54.67){\line(1,0){5.00}}
\put(227.67,54.67){\line(1,0){5.00}}
\put(234.33,54.67){\line(1,0){5.00}}
\put(220.67,54.67){\line(1,0){5.00}}
\put(238.67,64.67){\framebox(8.00,4.33)[cc]{}}
\put(243.67,73.67){\makebox(0,0)[rc]{$P_4,\varphi$}}
\put(210.00,80.67){\makebox(0,0)[cc]{$M$}}
\put(209.67,29.67){\makebox(0,0)[cc]{$M$}}
\put(178.67,57.67){\makebox(0,0)[rc]{$S_1$}}
\put(238.33,57.67){\makebox(0,0)[lc]{$S_2$}}
\put(158.33,78.34){\makebox(0,0)[cc]{${\bf 0}$}}
\put(260.67,78.34){\makebox(0,0)[cc]{${\bf 0}'$}}
\put(260.67,38.34){\makebox(0,0)[cc]{${\bf 1}'$}}
\put(158.00,38.34){\makebox(0,0)[cc]{${\bf 1}$}}
\put(199.00,39.67){\makebox(0,0)[cc]{$c$}}
\put(221.33,68.67){\makebox(0,0)[cc]{$b$}}
\put(174.33,100.34){\makebox(0,0)[lc]{${\bf T}^{MZ}(\alpha ,\beta ,\omega,\varphi )$}}
\put(150.00,74.34){\vector(1,0){20.00}}
\put(150.00,34.67){\vector(1,0){20.00}}
\put(250.00,74.67){\vector(1,0){20.00}}
\put(250.00,34.67){\vector(1,0){20.00}}
\put(198.67,64.67){\framebox(8.00,4.33)[cc]{}}
\put(206.67,60.67){\makebox(0,0)[lc]{$P_3,\omega$}}
\put(10.00,4.67){\makebox(0,0)[cc]{a)}}
\put(160.00,4.67){\makebox(0,0)[cc]{b)}}
\put(20.00,34.67){\line(2,1){80.00}}
\put(20.00,74.67){\line(2,-1){80.00}}
\put(32.67,64.67){\framebox(8.00,4.33)[cc]{}}
\put(36.67,76.67){\makebox(0,0)[cc]{$P_1,\alpha +\beta $}}
\put(174.67,64.67){\framebox(8.00,4.33)[cc]{}}
\put(174.67,73.67){\makebox(0,0)[lc]{$P_1,\alpha +\beta$}}
\put(174.67,41.67){\framebox(8.00,4.33)[cc]{}}
\put(181.34,35.67){\makebox(0,0)[cc]{$P_2,\beta$}}
\put(32.67,41.67){\framebox(8.00,4.33)[cc]{}}
\put(36.67,49.67){\makebox(0,0)[cc]{$P_2,\beta$}}
\end{picture}
\end{center}
}

\section{Verifiability \& Falsifiability}
\frame[shrink=2]{
\frametitle{Verifiability \& Falsifiability}

\begin{itemize}
\item<+->
Conjecture: by operational means it is not possible to go beyond
tests of hyper-NP-completeness.

\item<+->
As concerns harder cases with tractable verification:
Do there exist (decision) problems which are harder
than the known NP-complete cases,
possible having no recursively enumerable solution and proof methods,
whose results nevertheless are tractable verifiable?
\end{itemize}
}

\section*{Summary}
\frame[shrink=2]{
\frametitle{Summary}

\begin{itemize}
\item<+->
In quantum recursion theory, classical diagonalization does not yield to complete contradictions.
Reason: occurrence of fixed points.

\item<+->
Modification of recursion theory by
introduction of new diagonalization operators with no fixed points;
i.e., with all eigenvalues different from one.

\item<+->
Problems with verifiability \& falsifyiability.
\end{itemize}
}

\frame[shrink=2]{
\begin{center}
Thank you for your attention!
\end{center}
}

\bibliography{svozil}
\bibliographystyle{apsrev}


\end{document}
