%\documentclass[pra,showpacs,showkeys,amsfonts,amsmath,twocolumn]{revtex4}
%\documentclass[amsmath,red,handout]{beamer}
%\documentclass[amsmath,red,trans]{beamer}
\documentclass[amsmath,red]{beamer}
%\documentclass[pra,showpacs,showkeys,amsfonts]{revtex4}

%\usepackage{beamerthemeshadow}
%\usepackage[dark]{beamerthemesidebar}
%\usepackage[headheight=24pt,footheight=12pt]{beamerthemesplit}
\usepackage{beamerthemesplit}
%\usepackage[bar]{beamerthemetree}
\usepackage{graphicx}
\usepackage{pgf}

\pgfdeclareimage[height=0.6cm]{logo}{tu-logo}
\logo{\pgfuseimage{logo}}

\title{\bf Feyerabend and physics}
\subtitle{http://www.arxiv.org/abs/physics/0406079}
\author{Karl Svozil}
\institute{Institut f\"ur Theoretische Physik, University of Technology Vienna, \\
Wiedner Hauptstra\ss e 8-10/136, A-1040 Vienna, Austria}
\begin{document}
\maketitle

%\frame{\tableofcontents}



\section{General attitude}
%\subsection{Competence}

\frame[shrink=2]{
\frametitle{General attitude}

\begin{itemize}
\item<+->
Some papers on physics-related topics, in particular on the interpretation on quantum mechanics, on classical \& statistical physics.

\item<+->
In his autobiography he admitted to be no expert in this area.

\item<+-> Kurt Fischer:
`Apart from his stupidity --- he assured me --- nothing separated him from
being a physicist.


\item<+->
Unlike Popper's attempts to ``falsify the Copenhagen interpretation,''
much more humble, considerate \& self-critical
\end{itemize}
}


\frame[shrink=2]{
\frametitle{lasting message}
\begin{itemize}
\item<+->
Methodological openness

\item<+->
``Conquest of abundance''

\end{itemize}
 }


\section{Tower of Pisa example in ``Against Method''}

\subsection{Feyerabend's position}
\frame[shrink=2]{
\frametitle{Tower of Pisa example in ``Against Method''}

\begin{itemize}
\item<+->
Argument against earth rotation; e.g., put forward by Aristotle:
A stone from a high tower arrives at the foot of the tower without
any shift relative to the horizontal position of the release point on top of the tower
\item<+->
Feyerabend, as well as the historical reviewers never mentioned the contemporary physical situation,
in particular the Coriolis force \& the Kepler problem

\item<+->
Admittedly, Feyerabend has other objectives in mind (e.g., ``deceptions'' by Galileo, who ``brushes aside'' sense data and ``hides'' new ``absurd'' theoretical assumptions)
\end{itemize}
 }


\subsection{Historical question, still unsettled?}
\frame[shrink=2]{
\frametitle{Historical question, still unsettled?}

\begin{itemize}
\item<+->
No longitudinal \& latitudinal shifts also accepted  by Copernicus \& Galileo,
despite assumption of earth rotation (at equator 464 m/sec = 1670 km/hour !)

\item<+->
Hook \& Newton were interested (beginning of ``gravity''),
Gauss \& Laplace had (wrong) theoretical opinions

\item<+->
Edwin H. Hall's Harvard experiments 1902
 \cite{hall-1903a,hall-1903b}
http://dx.doi.org/10.1103/PhysRevSeriesI.17.179
http://dx.doi.org/10.1103/PhysRevSeriesI.17.245

\item<+->
Review by Angus Armitage \cite{armitage}:
{\em ``$\cdots$ Thus Newton's experimental test for the diurnal rotation
of the Earth may be said to have given positive results of the expected
order of magnitude, though the persistent occurrence of an unaccountable southward deviation
has continued to be a matter for inconclusive speculation.''}
\end{itemize}
 }



\subsection{Model calculation for Pisa}
\frame[shrink=2]{
\frametitle{Model calculation for Pisa}

\begin{itemize}

\item<+->  Model calculation by Martina Jedinger \&  Iva Brezinova (Projektarbeit TUW, 2004)
\begin{center}
%TexCad Options
%\grade{\off}
%\emlines{\off}
%\beziermacro{\on}
%\reduce{\on}
%\snapping{\off}
%\quality{2.00}
%\graddiff{0.01}
%\snapasp{1}
%\zoom{1.00}
\unitlength 0.20mm
\linethickness{0.4pt}
\begin{picture}(100.00,111.67)
%\circle(50.00,50.00){100.00}
\multiput(50.00,100.00)(1.60,-0.10){4}{\line(1,0){1.60}}
\multiput(56.39,99.59)(0.57,-0.11){11}{\line(1,0){0.57}}
\multiput(62.68,98.36)(0.36,-0.12){17}{\line(1,0){0.36}}
\multiput(68.76,96.35)(0.24,-0.12){24}{\line(1,0){0.24}}
\multiput(74.54,93.57)(0.18,-0.12){30}{\line(1,0){0.18}}
\multiput(79.91,90.07)(0.14,-0.12){35}{\line(1,0){0.14}}
\multiput(84.78,85.92)(0.12,-0.13){36}{\line(0,-1){0.13}}
\multiput(89.09,81.17)(0.12,-0.17){31}{\line(0,-1){0.17}}
\multiput(92.76,75.92)(0.12,-0.23){25}{\line(0,-1){0.23}}
\multiput(95.72,70.24)(0.12,-0.32){19}{\line(0,-1){0.32}}
\multiput(97.93,64.23)(0.12,-0.52){12}{\line(0,-1){0.52}}
\multiput(99.36,57.98)(0.10,-1.06){6}{\line(0,-1){1.06}}
\multiput(99.97,51.60)(-0.10,-3.20){2}{\line(0,-1){3.20}}
\multiput(99.77,45.20)(-0.11,-0.70){9}{\line(0,-1){0.70}}
\multiput(98.75,38.87)(-0.11,-0.38){16}{\line(0,-1){0.38}}
\multiput(96.92,32.73)(-0.12,-0.27){22}{\line(0,-1){0.27}}
\multiput(94.33,26.87)(-0.12,-0.20){28}{\line(0,-1){0.20}}
\multiput(91.01,21.39)(-0.12,-0.15){34}{\line(0,-1){0.15}}
\multiput(87.01,16.38)(-0.12,-0.12){38}{\line(-1,0){0.12}}
\multiput(82.41,11.93)(-0.16,-0.12){32}{\line(-1,0){0.16}}
\multiput(77.28,8.10)(-0.21,-0.12){27}{\line(-1,0){0.21}}
\multiput(71.69,4.95)(-0.28,-0.11){21}{\line(-1,0){0.28}}
\multiput(65.76,2.55)(-0.44,-0.12){14}{\line(-1,0){0.44}}
\multiput(59.56,0.92)(-0.91,-0.12){7}{\line(-1,0){0.91}}
\put(53.20,0.10){\line(-1,0){6.41}}
\multiput(46.80,0.10)(-0.91,0.12){7}{\line(-1,0){0.91}}
\multiput(40.44,0.92)(-0.44,0.12){14}{\line(-1,0){0.44}}
\multiput(34.24,2.55)(-0.28,0.11){21}{\line(-1,0){0.28}}
\multiput(28.31,4.95)(-0.21,0.12){27}{\line(-1,0){0.21}}
\multiput(22.72,8.10)(-0.16,0.12){32}{\line(-1,0){0.16}}
\multiput(17.59,11.93)(-0.12,0.12){38}{\line(-1,0){0.12}}
\multiput(12.99,16.38)(-0.12,0.15){34}{\line(0,1){0.15}}
\multiput(8.99,21.39)(-0.12,0.20){28}{\line(0,1){0.20}}
\multiput(5.67,26.87)(-0.12,0.27){22}{\line(0,1){0.27}}
\multiput(3.08,32.73)(-0.11,0.38){16}{\line(0,1){0.38}}
\multiput(1.25,38.87)(-0.11,0.70){9}{\line(0,1){0.70}}
\multiput(0.23,45.20)(-0.10,3.20){2}{\line(0,1){3.20}}
\multiput(0.03,51.60)(0.10,1.06){6}{\line(0,1){1.06}}
\multiput(0.64,57.98)(0.12,0.52){12}{\line(0,1){0.52}}
\multiput(2.07,64.23)(0.12,0.32){19}{\line(0,1){0.32}}
\multiput(4.28,70.24)(0.12,0.23){25}{\line(0,1){0.23}}
\multiput(7.24,75.92)(0.12,0.17){31}{\line(0,1){0.17}}
\multiput(10.91,81.17)(0.12,0.13){36}{\line(0,1){0.13}}
\multiput(15.22,85.92)(0.14,0.12){35}{\line(1,0){0.14}}
\multiput(20.09,90.07)(0.18,0.12){30}{\line(1,0){0.18}}
\multiput(25.46,93.57)(0.24,0.12){24}{\line(1,0){0.24}}
\multiput(31.24,96.35)(0.36,0.12){17}{\line(1,0){0.36}}
\multiput(37.32,98.36)(0.91,0.12){14}{\line(1,0){0.91}}
%\end
\bezier{32}(31.00,106.66)(28.33,103.33)(32.33,104.00)
\bezier{20}(32.33,104.00)(35.00,104.33)(36.66,105.66)
\bezier{20}(36.66,105.66)(38.33,107.66)(36.33,108.33)
\put(5.33,2.33){}
\bezier{188}(1.33,61.33)(-4.67,52.67)(32.00,51.67)
\bezier{140}(32.00,51.67)(53.00,53.00)(65.00,60.67)
\bezier{168}(65.00,60.67)(92.67,77.33)(86.00,84.67)
\put(60.33,58.00){\vector(2,1){31.67}}
\put(31.67,111.67){\line(1,-3){37.22}}
\put(36.00,108.67){\vector(-4,1){1.33}}
\end{picture}
\end{center}

\item<+->
9.6 cm towards South

\item<+->
0.6 cm towards East

\end{itemize}
 }

\subsection{Measurement of distant masses by Mach's principle}

\frame[shrink=2]{
\frametitle{Measurement of distant masses by Mach's principle}

\begin{itemize}

\item<+->
According to Ernst Mach, the inertia of a body is determined in relation to all
 other bodies in the universe
(in short, ``matter there governs inertia here'')
%[e.g., http://www.bun.kyoto-u.ac.jp/$\widetilde$suchii/mach.pr.html]


\item<+->
``Reverse argument:'' gravity pull \& shift of falling bodies known \& measurable
\item<+->
inertia is determined by the distant masses measurable by falling bodies

\end{itemize}
 }

\section{Quantum mechanics}
\subsection{Old topics in a new terminology: scholasticism, realism--idealism}

\frame[shrink=2]{
\frametitle{Old topics in a new terminology: realism--idealism}

\begin{itemize}
\item<+-> Realism: Some entities sometimes exist without being experienced by any finite mind.

\item<+-> Idealism:
$\ldots$
we have not the faintest reason for believing in the existence of
inexperienced entities
$\ldots$
[[Realism]] has been adopted
$\ldots$
solely because it simplifies our view of the universe
\cite{stace}
\end{itemize}
 }

\frame[shrink=2]{
\frametitle{Scholasticism in quantum physics}

%\begin{columns}
%\begin{column}{5cm}
%\pgfdeclareimage[height=2cm]{Specker}{specker}
%\pgfuseimage{Specker}
%\end{column}
%\begin{column}{9cm}
\begin{itemize}
\item<+->
Specker related the discussion on the foundations of quantum mechanics to scholastic philosophy;
in particular to scholastic speculations  about the existence of ``infuturabilities'' or
``counterfactuals.''

\item<+->
Question: Does
the omniscience (comprehensive knowledge) of God extend to events which
would have occurred if something  had happened which did not
happen?

\item<+->
Question: If so, can all events be pasted together to form a consistent whole?
\end{itemize}
%\end{column}
%\end{columns}
 }


\subsection{New features of quantum mechanics}
\frame[shrink=2]{
\frametitle{New features of quantum mechanics}

\begin{itemize}
\item<+-> Complementarity (nondistributive propositional structure not necessarily implies total abandonment of nonclassicality; e.g., automaton logic)

\item<+-> Value indefiniteness:
``not enough'' two-valued states to allow a faithful embedding into Boolean algebras.
E.g., nonseparable or unital set of two-valued states; even nonexistence of two-valued states on certain propositional structures (``Kochen-Specker theorem'')
\end{itemize}
 }

\subsection{Impossibility proof of classical physical existence}
\frame[shrink=2]{
\frametitle{Impossibility proof of classical physical existence
\cite[$\Gamma_3$]{kochen1}: Nonseparable set of two-valued states ($P(a) = P(b)$)}

\begin{center}
%TexCad Options
%\grade{\on}
%\emlines{\off}
%\beziermacro{\off}
%\reduce{\on}
%\snapping{\off}
%\quality{2.00}
%\graddiff{0.01}
%\snapasp{1}
%\zoom{0.50}
\unitlength 0.50mm
\linethickness{0.4pt}
\begin{picture}(190.67,109.67)
%\emline(165.67,19.67)(145.67,39.67)
\multiput(165.67,19.67)(-0.12,0.12){167}{\line(0,1){0.12}}
%\end
%\emline(145.67,39.67)(145.67,79.67)
\put(145.67,39.67){\line(0,1){40.00}}
%\end
%\emline(145.67,79.67)(165.67,99.67)
\multiput(145.67,79.67)(0.12,0.12){167}{\line(0,1){0.12}}
%\end
%\emline(165.67,99.67)(185.67,79.67)
\multiput(165.67,99.67)(0.12,-0.12){167}{\line(1,0){0.12}}
%\end
%\emline(185.67,79.67)(185.67,39.67)
\put(185.67,79.67){\line(0,-1){40.00}}
%\end
%\emline(185.67,39.67)(165.67,19.67)
\multiput(185.67,39.67)(-0.12,-0.12){167}{\line(-1,0){0.12}}
%\end
%\emline(185.34,59.67)(145.67,59.67)
\put(185.34,59.67){\line(-1,0){39.67}}
%\end
\put(185.67,39.67){\circle{2.00}}
\put(185.67,59.67){\circle{2.00}}
\put(185.67,79.67){\circle{2.00}}
\put(145.67,39.67){\circle{2.00}}
\put(145.67,59.67){\circle{2.00}}
\put(145.67,79.67){\circle{2.00}}
\put(165.67,19.67){\circle{2.00}}
\put(165.67,99.67){\circle{2.00}}
%\emline(165.67,19.67)(95.67,59.67)
\multiput(165.67,19.67)(-0.21,0.12){334}{\line(-1,0){0.21}}
%\end
%\emline(95.67,59.67)(165.67,99.67)
\multiput(95.67,59.67)(0.21,0.12){334}{\line(1,0){0.21}}
%\end
\put(95.67,59.67){\circle{2.00}}
\put(95.67,74.67){\makebox(0,0)[cc]{$a_8=a_8'$}}
\put(165.67,109.67){\makebox(0,0)[cc]{$b=a_9=a_0'$}}
\put(140.67,79.67){\makebox(0,0)[cc]{$a_2'$}}
\put(140.67,59.67){\makebox(0,0)[cc]{$a_6'$}}
\put(140.67,39.67){\makebox(0,0)[cc]{$a_4'$}}
\put(190.67,39.67){\makebox(0,0)[cc]{$a_3'$}}
\put(190.67,59.67){\makebox(0,0)[cc]{$a_5'$}}
\put(190.67,80.00){\makebox(0,0)[cc]{$a_1'$}}
\put(165.67,9.67){\makebox(0,0)[cc]{$a_7'$}}
%\emline(25.00,19.67)(45.00,39.67)
\multiput(25.00,19.67)(0.12,0.12){167}{\line(0,1){0.12}}
%\end
%\emline(45.00,39.67)(45.00,79.67)
\put(45.00,39.67){\line(0,1){40.00}}
%\end
%\emline(45.00,79.67)(25.00,99.67)
\multiput(45.00,79.67)(-0.12,0.12){167}{\line(0,1){0.12}}
%\end
%\emline(25.00,99.67)(5.00,79.67)
\multiput(25.00,99.67)(-0.12,-0.12){167}{\line(-1,0){0.12}}
%\end
%\emline(5.00,79.67)(5.00,39.67)
\put(5.00,79.67){\line(0,-1){40.00}}
%\end
%\emline(5.00,39.67)(25.00,19.67)
\multiput(5.00,39.67)(0.12,-0.12){167}{\line(1,0){0.12}}
%\end
%\emline(5.33,59.67)(45.00,59.67)
\put(5.33,59.67){\line(1,0){39.67}}
%\end
\put(5.00,39.67){\circle{2.00}}
\put(5.00,59.67){\circle{2.00}}
\put(5.00,79.67){\circle{2.00}}
\put(45.00,39.67){\circle{2.00}}
\put(45.00,59.67){\circle{2.00}}
\put(45.00,79.67){\circle{2.00}}
\put(25.00,19.67){\circle{2.00}}
\put(25.00,99.67){\circle{2.00}}
%\emline(25.00,19.67)(95.00,59.67)
\multiput(25.00,19.67)(0.21,0.12){334}{\line(1,0){0.21}}
%\end
%\emline(95.00,59.67)(25.00,99.67)
\multiput(95.00,59.67)(-0.21,0.12){334}{\line(-1,0){0.21}}
%\end
\put(25.00,109.67){\makebox(0,0)[cc]{$a=a_0=a_9'$}}
\put(50.00,79.67){\makebox(0,0)[cc]{$a_2$}}
\put(50.00,59.67){\makebox(0,0)[cc]{$a_6$}}
\put(50.00,39.67){\makebox(0,0)[cc]{$a_4$}}
\put(-0.00,39.67){\makebox(0,0)[cc]{$a_3$}}
\put(-0.00,59.67){\makebox(0,0)[cc]{$a_5$}}
\put(-0.00,80.00){\makebox(0,0)[cc]{$a_1$}}
\put(25.00,9.67){\makebox(0,0)[cc]{$a_7$}}
\end{picture}
\end{center}

}


\subsection{Science is the continuation of philosophy by different means}
\frame[shrink=2]{
\frametitle{Science is the continuation of philosophy by different means}

\begin{itemize}
\item<+->
Concepts and formalizations require too much mathematical and physical techniques \& methods
for the majority of philosophers; most are at a total loss to understand the theory even at its most elementary level

\item<+->
Feyerabend in ``How to Defend Society Against Science:''
{\em ``$\ldots$ Kuhn encourages people who have no idea why a stone falls to the ground to talk with assurance about scientific method. Now I have no objection to incompetence but I do object when incompetence is accompanied by boredom and self-righteousness.
And this is exactly what happens. $\ldots$''}


\end{itemize}
}

\frame[shrink=2]{
\frametitle{Physicists at their own $\ldots$}

Physicists are left alone to interpret the ``meaning'' of the formalism


\begin{itemize}
\item<+->
Copenhagen interpretation (Bohr)

\item<+->
Many-worlds interpretation  (Everett)


\item<+->
Bohm interpretation

\item<+->
``Realistic'' interpretation  (Einstein, De Broglie, Schr\"odinger)


\item<+->
some physicists ``go wild'' and pretend that the transient status of their science reflects
final truth of the world; they
tell fairy tales about the first three minutes of the Universe, short histories of time and what not

\end{itemize}
}

\section{Some personal remarks}
\subsection{Personal account \& impact on the scientific community}
\frame[shrink=2]{
\frametitle{Some personal remarks}

\begin{itemize}
\item<+->
In 1982, when I met him, Feyerabend made a sad but rebellious impression,
best described by the German word ``unerf\"ullt'' (``unrealised, unfulfilled'')

\item<+->
Despite his fame, no one even considered taking his ideas on science financing seriously;
e.g.,  lay judges for science assessment \& financing
\end{itemize}
}

\frame[shrink=2]{
\frametitle{distribute the funding for research projects}
I would like to suggest to distribute the funding for research projects in the following manner:
\begin{itemize}
\item<+->
First, make a very rough plausibility check to eliminate
applications which are obviously fraudulent, inconsistent or otherwise impossible,
unlawful or catastrophic.
This could be done by lay judges.
\item<+->
Then, in the first round of money distribution,
choose 10\%
projects for funding
totally at random.
\item<+->
Choose the next 20\%
by a system of lay judges, as suggested by Feyerabend.
\item<+->
Finally, distribute the remaining 70\%
via the conventional peer review process.
\item<+->
After five years, publish the outcome of the projects funded by all three selection catagories
and adjust the relative magnitude of these categories accordingly.
\end{itemize}
}

\subsection{Heritage}
\frame[shrink=2]{
\frametitle{Heritage}

Despite all his (admitted) ignorance and imperfections,
his biggest heritage may be two messages:
\begin{itemize}
\item<+->
Enlightenment (cf. Kant): try on your own, let not others decide what you think;
do not stop where other people, authorities \& mandarins tell you to halt
\item<+->
The Conquest of Abundance:
The pursuit of science is one of the greatest passions of life,
and our capabilities to recognize and manipulate the physical world
may only be limited by our phantasy.
Maybe one hopefully happy day we will be able to {\em tune} the world according to our will alone.
\end{itemize}
}


\frame[shrink=2]{
\begin{center}
Thank you for your attention!
\end{center}
}

\bibliography{svozil}
\bibliographystyle{apsrev}


\end{document}


