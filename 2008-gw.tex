\documentclass[prl,amsfonts,showpacs,showkeys,preprint]{revtex4}
\usepackage{graphicx}% Include figure files
\RequirePackage{times}
\RequirePackage{mathptm}
\begin{document}


\title{Possible Effects of Thermal Activities Inside of the Earth on Climatic Changes and Global Warming}


\author{Karl Svozil}
\email{svozil@tuwien.ac.at}
\homepage{http://tph.tuwien.ac.at/~svozil}
\affiliation{Institut f\"ur Theoretische Physik, University of Technology Vienna,
Wiedner Hauptstra\ss e 8-10/136, A-1040 Vienna, Austria}


\begin{abstract}
The thermal activities inside earth may contribute substantially to global warming and cooling.
\end{abstract}



\pacs{92.70.Mn,92.70.Np,92.70.Bc,92.70.Pq}
\keywords{global warming, Earth mantle and core, global climate modeling}


\maketitle

Sir!

Among the many factors for global climatic changes and heating including emissions of
greenhouse gases from human and solar activities,
one possible cause has hardly been discussed:
the effect of the energy flows and the dynamics of the planet's core, mantle and crust on its surface and atmosphere.
While the energy exchange of the atmosphere with outer space and the sun can to some degree be influenced by human activities
such as outlined in the Kyoto protocol,
inner-earth effects are totally uncontrollable by our present capabilities.

Indeed, the effect of the intrinsic energy density of a planet on the temperature of its atmosphere can hardly be underestimated.
A very crude estimate of the heat transfer in equilibrium between the soil and the atmosphere
can be obtained by energy conservation and comparing the volumes of the mantle with the active core,
assuming uniformity of the earth's entire interior.
Every change in the earth core temperature translates into one fifth of that change in the temperature of the mantle.
Temperature changes resulting from the rather small region of the core could also explain the rather dramatic
climatic changes which were recorded in the past.

Although much more careful heat transfer estimates and measurements are necessary,
the effect of inner earth processes on the atmosphere should therefore be carefully considered.
We suggest also to examine tectonic activities such as earthquakes and volcanoes with climatic changes.



\section *{Acknowledgements}

This text is submitted to the {\em Correspondence} section of {\em Nature}.



%\bibliography{svozil}
%\bibliographystyle{osa}
%\bibliographystyle{apsrev}




\end {document}
