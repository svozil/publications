
a1 = {a11, a12} /. a11 -> Prime[1] /. a12 -> Prime[2];
a2 = {a21, a22} /. a21 -> Prime[3] /. a22 -> Prime[4];
a3 = {a31, a32} /. a31 -> Prime[5] /. a32 -> Prime[6];
a4 = {a41, a42} /. a41 -> Prime[7] /. a42 -> Prime[8];


V = {};
Do[
  Do[
   Do[
    Do[AppendTo[
      V, {"1   ",a1[[i1]] a3[[i3]], a1[[i1]] a4[[i4]], a2[[i2]] a3[[i3]],
       a2[[i2]] a4[[i4]]}]
     , {i4, 1, Length[a4]}]
    , {i3, 1, Length[a3]}]
   , {i2, 1, Length[a2]}]
  , {i1, 1, Length[a1]}];

TraditionalForm[V]


###################################################################################

* four variables:
* p13, p14, p23, p24
*
V-representation
begin
16   4   integer
1       22      34      55      85
1       22      38      55      95
1       26      34      65      85
1       26      38      65      95
1       22      34      77      119
1       22      38      77      133
1       26      34      91      119
1       26      38      91      133
1       33      51      55      85
1       33      57      55      95
1       39      51      65      85
1       39      57      65      95
1       33      51      77      119
1       33      57      77      133
1       39      51      91      119
1       39      57      91      133





import cdd;
mat = cdd.Matrix([
[ 1 ,      22  ,    34  ,    55  ,    85    ] ,
[ 1 ,      22  ,    38  ,    55  ,    95    ] ,
[ 1 ,      26  ,    34  ,    65  ,    85    ] ,
[ 1 ,      26  ,    38  ,    65  ,    95    ] ,
[ 1 ,      22  ,    34  ,    77  ,    119   ] ,
[ 1 ,      22  ,    38  ,    77  ,    133   ] ,
[ 1 ,      26  ,    34  ,    91  ,    119   ] ,
[ 1 ,      26  ,    38  ,    91  ,    133   ] ,
[ 1 ,      33  ,    51  ,    55  ,    85    ] ,
[ 1 ,      33  ,    57  ,    55  ,    95    ] ,
[ 1 ,      39  ,    51  ,    65  ,    85    ] ,
[ 1 ,      39  ,    57  ,    65  ,    95    ] ,
[ 1 ,      33  ,    51  ,    77  ,    119   ] ,
[ 1 ,      33  ,    57  ,    77  ,    133   ] ,
[ 1 ,      39  ,    51  ,    91  ,    119   ] ,
[ 1 ,      39  ,    57  ,    91  ,    133   ] ] ,
number_type='fraction');
mat.rep_type = cdd.RepType.GENERATOR;
poly = cdd.Polyhedron(mat);
ine= poly.get_inequalities();
print(ine);
f = open('C:/mytex/cdd/2-2-prim.ine', 'w');
s=str(ine);
f.write(s);
f.close();


H-representation
begin
 28 5 rational
 0 85/39 -5/3 -17/13 1
 0 -95/33 5/3 19/11 -1
 -85/2 -85/22 5/2 51/22 -1
 -85 0 0 0 1
 -300/17 -1315/374 5/2 19/11 -1
 -425/21 -85/21 65/21 34/21 -1
 -85/3 85/33 -5/3 -34/33 1
 -55 0 0 1 0
 -50/3 95/22 -5/2 -47/33 1
 -5100/299 85/26 -1195/598 -17/13 1
 -34 -85/22 7/2 17/11 -1
 -22 1 0 0 0
 -34 119/22 -5/2 -17/11 1
 -34 0 1 0 0
 0 133/22 -7/2 -19/11 1
 0 -119/26 7/2 17/13 -1
 665/27 -266/81 154/81 38/27 -1
 133 0 0 0 -1
 7980/253 133/33 -2191/759 -19/11 1
 133/2 133/26 -7/2 -57/26 1
 91 0 0 -1 0
 70/3 119/39 -7/3 -61/39 1
 91/2 -7/2 91/38 1 -39/38
 420/17 -2443/663 7/3 17/13 -1
 57 0 -1 0 0
 38 95/39 -7/3 -19/13 1
 38 -133/39 5/3 19/13 -1
 39 -1 0 0 0
end
