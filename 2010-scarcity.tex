
\documentclass[pra,amsfonts,showpacs,showkeys,preprint]{revtex4}%,twocolumn]{revtex4}
%\documentclass[rmp,amsfonts,showpacs,showkeys,twocolumn]{revtex4}%,twocolumn]{revtex4}
\usepackage{url}
\usepackage{xcolor}
\usepackage{eepic}
\usepackage{graphicx}% Include figure files

\usepackage{bm}% bold math
\RequirePackage{times}
\RequirePackage{mathptm}
\RequirePackage{courier}

\usepackage{longtable}
\usepackage{dcolumn}% Align table columns on decimal point

\renewcommand{\labelenumi}{(\roman{enumi})}
\bibpunct{[}{]}{,}{a}{}{;}

\usepackage[left]{eurosym}
%\usepackage{arydshln}


%\usepackage{cdmtcs}

\begin{document}


%\cdmtcsauthor{Karl Svozil}
%\cdmtcstitle{Feasibility is limited by fantasy alone}
%\cdmtcsaffiliation{Vienna University of Technology}
%\cdmtcstrnumber{346}
%\cdmtcsdate{February 2009}predicability
%\colourcoverpage


\title{Classical irrepresentability by scarcity of two-valued probability measures}

\author{Karl Svozil}
\email{svozil@tuwien.ac.at}
\homepage{http://tph.tuwien.ac.at/~svozil}
\affiliation{Institute for Theoretical Physics, Vienna University of Technology,  \\
Wiedner Hauptstra\ss e 8-10/136, A-1040 Vienna, Austria}

\author{Josef Tkadlec}
\email{tkadlec@fel.cvut.cz }
\homepage{http://math.feld.cvut.cz/tkadlec/}
\affiliation{Department of Mathematics, Faculty of Electrical Engineering, Czech Technical University,  \\
Zikova 4, 166 27 Praha, Czech Republic}

\begin{abstract}
Classical irrepresentability and value indefiniteness of quantum logics occurs already at a stage when two-valued measures still exist, but are too ``scarce'' to separate propositions.
\end{abstract}

\pacs{03.65.Ta,03.65.Ud}
\keywords{nonseparating set of two valued probability measures, quantum value indefiniteness}

\maketitle


%\tableofcontents

\section{Introduction}

For dimensions higher than two, Gleason's theorem~\cite{Gleason,r:dvur-93,pitowsky:218} derives the Born rule of quantum probability from elementary axioms~\cite[Section~7-2.]{peres},
in particular from the assumptions  that
(i) within {\em classical subalgebras}, also called {\em blocks}~\cite{greechie:71,nav:91,pulmannova-91}
and represented by {\em maximal operators}~\cite[\S~84]{halmos-vs},
the quantum probabilities  behave classically~\cite{svozil-2008-ql}, and that
(ii) for any given physical probability measure (state), the expectation value of an observable is context independent; i.e.,
it does not depend in the choice of the basis in which it is included~\cite{svozil:040102}.
The latter assumption takes effect only for dimensions higher than two,
since only in these cases orthogonal basis may be interconnected and intertwined~\cite[p.~193]{peres}.

Gleason's findings stimulated many researchers~\cite{ZirlSchl-65,Alda,Alda2,kamber64,kamber65} to work out the consequences of the techniques developed therein,
most notably Bell~\cite{bell-87,mermin-93}, Specker~\cite{specker-60} and Kochen \& Specker~\cite{kochen1,svozil-tkadlec}.
Very early, the issue was raised if three- and higher-dimensional Hilbert spaces allowed the construction of
{\em two-valued frame functions} on the set of observables.
In the Gleason setup, frame functions are identifiable with probabilities; hence
two-valued probability measures can be interpreted as classical truth assignments.
Incidentally, by their convex combination~\cite{ziegler}, ``sufficiently many'' two-valued probability measures can be used to construct classical probability measures~\cite{pitowsky}.
The nonexistence of two-valued probability measures thus would have far-reaching consequences for the classical irrepresentability of quantum observables,
and would indicate a value indefineteness~\cite{peres222} already encountered in Bell-Bool type reformulations~\cite{pitowsky-89a} of  Einstein-Podolski-Rosen type experiments~\cite{epr}.


In 1960, Specker announced the impossibility to construct two-valued frame functions in quantum mechanics~\cite{specker-60},
followed by a detailed proof on certain finite collections of quantum observables~\cite{kochen1}.
The latter paper by Kochen \& Specker already presented concrete examples of collections of observables which would still allow the existence of two-valued probability measures,
yet which nevertheless ``suffer'' from a ``scarcity'' of two-valued probability measures.
One example, henceforth called the  ``bug'' diagram~\cite{Specker-priv} (see also Ref.~\cite[Fig.~2.4.6]{pulmannova-91})
and depicted in the form of a (Greechie) orthogonality diagram~\cite{greechie:71,pulmannova-91} (representing propositions as points
and bases as smooth lines connecting them) in Fig.~\ref{2010-scarcity-f1}a),
represents a configuration of seven intertwined bases of three-dimensional Hilbert space, which is a subgraph of Graph $\Gamma_1$ in Ref.~\cite{kochen1}.
This diagram implies that, whenever the observable proposition $K$ is true, the observable proposition $E$ must be false, and {\em vice versa}.
For, if both $K$ and $E$ were true, then
$A$,
$C$,
$G$,
$I$ must all be false; hence $H$ and
$B$ must be true, which cannot be, because the latter two points belong to a single basis.

For the sake of making matters worse, two ``bug'' diagrams can be tightly intertwined to render a diagram,
depicted in Fig.~\ref{2010-scarcity-f1}b), which does not allow a {\em separable} set of two-valued probability measures.
Separability means that any two different points, say  $K$ and $K'$, can be distinguished by their probability;
in particular for two-valued measures, by their truth and falsity.
As Kochen \& Specker in their diagram $\Gamma_3$, which is equivalent to the orthogonality diagram in Fig.~\ref{2010-scarcity-f1}b), point out  that
whenever $K$ is true, both $E$ and $N$ must be false, so $K'$ must be true as well, and {\em vice versa}.

Suppose, for the sake of contradiction, that two atoms of a Boolean algebra $A,B$ are nonseparable by any two-valued probability measure.
That is, $f_i(A)=f_i(B)$ for all two-falued frame functions $f_i$ in the sense of Gleason~\cite{Gleason}.
This, however, would contradict Stone's representation theorem for Boolean algebras stating that every Boolean algebra is isomorphic to a field of sets~\cite{stone}
and its  equivalent Boolean prime ideal theorem.

Separability is a necessary (and sufficient? proof?) criterion for homeomorphic imbeddability~\cite{kochen1,CalHerSvo}.
Thus, a configuration of observables aligned as the ones in the ``twin bug'' in Fig.~\ref{2010-scarcity-f1}b)
has no classical interpretation; i.e., it is value indefinite.
(In a different terminology~\cite{bohr-1949,bell-66,hey-red,redhead,cabello:210401,Bartosik-09} it may be said to require ``contextuality.'')

\begin{figure}%[p]
\begin{center}
\begin{tabular}{ccc}
%TeXCAD Picture [1.pic]. Options:
%\grade{\on}
%\emlines{\off}
%\epic{\off}
%\beziermacro{\on}
%\reduce{\on}
%\snapping{\off}
%\quality{8.00}
%\graddiff{0.01}
%\snapasp{1}
%\zoom{5.6569}
\unitlength .3mm % = 1.42pt
\linethickness{0.8pt}
\ifx\plotpoint\undefined\newsavebox{\plotpoint}\fi % GNUPLOT compatibility
\begin{picture}(120.92,114.73)(0,0)
%\emline(86.57,102.14)(111.57,58.64)
\multiput(86.57,102.14)(.11961722,-.20813397){209}{\line(0,-1){.20813397}}
%\end
%\emline(86.57,15.14)(111.57,58.64)
\multiput(86.57,15.14)(.11961722,.20813397){209}{\line(0,1){.20813397}}
%\end
%\emline(36.65,102.14)(11.65,58.64)
\multiput(36.65,102.14)(-.11961722,-.20813397){209}{\line(0,-1){.20813397}}
%\end
%\emline(36.65,15.14)(11.65,58.64)
\multiput(36.65,15.14)(-.11961722,.20813397){209}{\line(0,1){.20813397}}
%\end
\put(86.57,101.89){\line(-1,0){50}}
\put(86.57,15.39){\line(-1,0){50}}
\put(86.46,101.94){\circle{4}}
\put(86.46,15.34){\circle{4}}
\put(111.39,58.63){\circle{4}}
\put(11.74,58.63){\circle{4}}
\put(61.77,58.63){\circle{4}}
\put(36.52,101.94){\circle{4}}
\put(61.62,101.94){\circle{4}}
\put(61.62,15.44){\circle{4}}
\put(97.68,82.85){\circle{4}}
\put(25.71,82.85){\circle{4}}
\put(98.74,36.35){\circle{4}}
\put(24.65,36.35){\circle{4}}
\put(36.52,15.34){\circle{4}}
\put(61.69,101.82){\line(0,-1){86.27}}
\put(30.41,2.65){\makebox(0,0)[cc]{$A$}}
\put(61.87,2.3){\makebox(0,0)[cc]{$B$}}
\put(91.93,2.48){\makebox(0,0)[cc]{$C$}}
\put(110.84,30.94){\makebox(0,0)[cc]{$D$}}
\put(120.92,57.98){\makebox(0,0)[lc]{$E$}}
\put(108.41,88.92){\makebox(0,0)[cc]{$F$}}
\put(91.93,114.2){\makebox(0,0)[cc]{$G$}}
\put(61.7,114.73){\makebox(0,0)[cc]{$H$}}
\put(30.41,114.02){\makebox(0,0)[cc]{$I$}}
\put(13.56,87.86){\makebox(0,0)[cc]{$J$}}
\put(1.77,57.98){\makebox(0,0)[rc]{$ K$}}
\put(14.67,30.05){\makebox(0,0)[rc]{$L$}}
\put(72,55.51){\makebox(0,0)[cc]{$M$}}
%\put(71.34,9.19){\makebox(0,0)[cc]{$a$}}
%\put(107.91,40.35){\makebox(0,0)[cc]{$b$}}
%\put(98.53,95.32){\makebox(0,0)[cc]{$c$}}
%\put(54.46,108.01){\makebox(0,0)[cc]{$d$}}
%\put(15.03,78.14){\makebox(0,0)[cc]{$e$}}
%\put(21.56,27.06){\makebox(0,0)[cc]{$f$}}
%\put(67.88,75.51){\makebox(0,0)[cc]{$g$}}
\end{picture}
&
$\qquad$
$\qquad$
&
%TeXCAD Picture [1.pic]. Options:
%\grade{\on}
%\emlines{\off}
%\epic{\off}
%\beziermacro{\on}
%\reduce{\on}
%\snapping{\off}
%\quality{8.00}
%\graddiff{0.01}
%\snapasp{1}
%\zoom{4.0000}
\unitlength .3mm % = 1.42pt
\linethickness{0.8pt}
\ifx\plotpoint\undefined\newsavebox{\plotpoint}\fi % GNUPLOT compatibility
\begin{picture}(320.85,118.44)(0,0)
%\emline(105.32,33.64)(61.82,8.64)
\multiput(105.32,33.64)(-.20813397,-.11961722){209}{\line(-1,0){.20813397}}
%\end
%\emline(308.26,33.64)(264.76,8.64)
\multiput(308.26,33.64)(-.20813397,-.11961722){209}{\line(-1,0){.20813397}}
%\end
%\emline(18.32,33.64)(61.82,8.64)
\multiput(18.32,33.64)(.20813397,-.11961722){209}{\line(1,0){.20813397}}
%\end
%\emline(221.26,33.64)(264.76,8.64)
\multiput(221.26,33.64)(.20813397,-.11961722){209}{\line(1,0){.20813397}}
%\end
%\emline(105.32,83.56)(61.82,108.56)
\multiput(105.32,83.56)(-.20813397,.11961722){209}{\line(-1,0){.20813397}}
%\end
%\emline(308.26,83.56)(264.76,108.56)
\multiput(308.26,83.56)(-.20813397,.11961722){209}{\line(-1,0){.20813397}}
%\end
%\emline(18.32,83.56)(61.82,108.56)
\multiput(18.32,83.56)(.20813397,.11961722){209}{\line(1,0){.20813397}}
%\end
%\emline(221.26,83.56)(264.76,108.56)
\multiput(221.26,83.56)(.20813397,.11961722){209}{\line(1,0){.20813397}}
%\end
\put(105.07,33.64){\line(0,1){50}}
\put(308.01,33.64){\line(0,1){50}}
\put(18.57,33.64){\line(0,1){50}}
\put(221.51,33.64){\line(0,1){50}}
\put(105.12,33.75){\circle{4}}
\put(308.06,33.75){\circle{4}}
\put(18.52,33.75){\circle{4}}
\put(221.46,33.75){\circle{4}}
\put(61.81,8.82){\circle{4}}
\put(264.75,8.82){\circle{4}}
\put(61.81,108.47){\circle{4}}
\put(264.75,108.47){\circle{4}}
\put(61.81,58.44){\circle{4}}
\put(264.75,58.44){\circle{4}}
\put(105.12,83.69){\circle{4}}
\put(308.06,83.69){\circle{4}}
\put(105.12,58.59){\circle{4}}
\put(308.06,58.59){\circle{4}}
\put(18.62,58.59){\circle{4}}
\put(221.56,58.59){\circle{4}}
\put(86.03,22.53){\circle{4}}
\put(288.97,22.53){\circle{4}}
\put(86.03,94.5){\circle{4}}
\put(288.97,94.5){\circle{4}}
\put(39.53,21.47){\circle{4}}
\put(242.47,21.47){\circle{4}}
\put(39.53,95.56){\circle{4}}
\put(242.47,95.56){\circle{4}}
\put(18.52,83.69){\circle{4}}
\put(221.46,83.69){\circle{4}}
\put(163.21,58.69){\circle{4}}
\put(105,58.52){\line(-1,0){86.27}}
\put(307.94,58.52){\line(-1,0){86.27}}
\put(5.83,89.8){\makebox(0,0)[]{$A$}}
\put(208.77,89.8){\makebox(0,0)[]{$A'$}}
\put(5.48,58.34){\makebox(0,0)[]{$B$}}
\put(208.42,58.34){\makebox(0,0)[]{$B'$}}
\put(5.66,28.28){\makebox(0,0)[]{$C$}}
\put(208.6,28.28){\makebox(0,0)[]{$C'$}}
\put(34.12,9.37){\makebox(0,0)[]{$D$}}
\put(237.06,9.37){\makebox(0,0)[]{$D'$}}
\put(61.16,-.71){\makebox(0,0)[t]{$E$}}
\put(264.1,-.71){\makebox(0,0)[t]{$E'$}}
\put(92.1,11.8){\makebox(0,0)[]{$F$}}
\put(295.04,11.8){\makebox(0,0)[]{$F'$}}
\put(117.38,28.28){\makebox(0,0)[]{$G$}}
\put(320.32,28.28){\makebox(0,0)[]{$G'$}}
\put(117.91,58.51){\makebox(0,0)[]{$H$}}
\put(320.85,58.51){\makebox(0,0)[]{$H'$}}
\put(117.2,89.8){\makebox(0,0)[]{$I$}}
\put(320.14,89.8){\makebox(0,0)[]{$I'$}}
\put(91.04,106.65){\makebox(0,0)[]{$J$}}
\put(293.98,106.65){\makebox(0,0)[]{$J'$}}
\put(61.16,118.44){\makebox(0,0)[b]{$ K$}}
\put(264.1,118.44){\makebox(0,0)[b]{$ K'$}}
\put(33.23,105.54){\makebox(0,0)[b]{$L$}}
\put(236.17,105.54){\makebox(0,0)[b]{$L'$}}
\put(58.69,48.33){\makebox(0,0)[]{$M$}}
\put(261.63,48.33){\makebox(0,0)[]{$M'$}}
%\put(12.37,48.87){\makebox(0,0)[]{$a$}}
%\put(215.31,48.87){\makebox(0,0)[]{$a'$}}
%\put(43.53,12.3){\makebox(0,0)[]{$b$}}
%\put(246.47,12.3){\makebox(0,0)[]{$b'$}}
%\put(98.5,21.68){\makebox(0,0)[]{$c$}}
%\put(301.44,21.68){\makebox(0,0)[]{$c'$}}
%\put(111.19,65.75){\makebox(0,0)[]{$d$}}
%\put(314.13,65.75){\makebox(0,0)[]{$d'$}}
%\put(81.32,105.18){\makebox(0,0)[]{$e$}}
%\put(284.26,105.18){\makebox(0,0)[]{$e'$}}
%\put(30.24,98.65){\makebox(0,0)[]{$f$}}
%\put(233.18,98.65){\makebox(0,0)[]{$f'$}}
%\put(38.69,52.33){\makebox(0,0)[]{$g$}}
%\put(281.63,52.33){\makebox(0,0)[]{$g'$}}
%\emline(61.75,8.75)(264.5,108.5)
\multiput(61.75,8.75)(.243687933,.1198908573){832}{\line(1,0){.243687933}}
%\end
%\emline(61.75,108.25)(264.75,8.75)
\multiput(61.75,108.25)(.2445763352,-.1198785485){830}{\line(1,0){.2445763352}}
%\end
\put(163,46.25){\makebox(0,0)[cc]{$N$}}
%\put(182.75,78.25){\makebox(0,0)[cc]{$h$}}
%\put(182.5,41.25){\makebox(0,0)[cc]{$i$}}
\qbezier(61.75,58.5)(104.62,30)(163,58.5)
\qbezier(264.25,58.5)(221.37,87)(163,58.5)
%\put(241.75,76.75){\makebox(0,0)[cc]{$j$}}
\end{picture}
\\
a)&&b)
\end{tabular}
\end{center}
\caption{Orthogonality diagrams of graphs with a ``scarce'' set of two-valued probability measures:
%
a) the ``bug'' diagram;
%
b) two intertwined bug diagrams with a nonseparating set of two-valued probability measures.
\label{2010-scarcity-f1} }
\end{figure}


\section{Representability of orthogonality diagrams in vector space}

Just as in the case of the Kochen \& Specker theorem regarding finite collections of quantum observables which have no two-valued probability measures,
one may investigate finite collections of quantum observables whose set of two-valued probability measures is not empty but so ``meager'' and ``scarce'' that they do not allow a homeomorphic imbeddability
into a Boolean algebra; i.e., pointedly stated, they do not allow a classical representation.
As already mentioned, the first example has been given by Kochen \& Specker in their Diagram $\Gamma_3$.


In order to allow a systematic investigation of configurations, it would be helpful to have criteria
to check whether or not a particular  orthogonality diagram allows a vector space representation;
in the positive case constructive methods would be useful.
[[[cf. Cabello]]]
We shall call a (Greechie) orthogonality diagram which can be realized in some vector space {\em representable.}

For three-dimensional vector spaces, one may speculate that every nonrepresentable orthogonality diagram contains the ``bug'' as subdiagram.
This, however, can be disproved by the example depicted in Fig.~\cite{2010-scarcity-f-nonbug}.

\begin{figure}%[p]
\begin{center}
N.N.
\end{center}
\caption{Orthogonality diagram allowing a representation in three-dimensional vector space without ``bug'' subdiagram.
\label{2010-scarcity-f-nonbug} }
\end{figure}

\section{Orthogonality diagrams with a nonseparable set of probability measures}



As a conjecture one could state that, for three-dimensional vector spaces, every nonseparable set of two-valued probability measures contains the ``bug'' as subdiagram.


As another conjecture one could state that, for three-dimensional vector spaces, The ``twin bug'' diagram $\Gamma_3$~\cite{kochen1}
and depicted in Fig.~\ref{2010-scarcity-f1}b) is the diagram with the smallest number of atoms and subalgebras not allowing a separable set of two-valued probability measures.

\bibliography{svozil}
\bibliographystyle{osa}



\end{document}

