%%tth:\begin{html}<LINK REL=STYLESHEET HREF="http://tph.tuwien.ac.at/~svozil/ssh.css">\end{html}
\documentclass[pra,preprint,showpacs,showkeys,amsfonts]{revtex4}
\usepackage{graphicx}
%\documentstyle[]{article}
 \RequirePackage{times}
%\RequirePackage{courier}
\RequirePackage{mathptm}
%\renewcommand{\baselinestretch}{1.3}
\begin{document}

\def\ttt{{\rm Tr} }
\def\diag{{\rm diag} }
%\def\frak{\cal }
%\def\Bbb{\bf }
%\sloppy




\title{Quantum information in base $n$ defined by state partitions }
\author{Karl Svozil}
 \email{svozil@tuwien.ac.at}
\homepage{http://tph.tuwien.ac.at/~svozil}
\affiliation{Institut f\"ur Theoretische Physik, University of Technology Vienna,
Wiedner Hauptstra\ss e 8-10/136, A-1040 Vienna, Austria}

\begin{abstract}
%Quantum information
%makes necessary the introduction of irreducible ``nits,''
%$n$ being an arbitrary natural number associated with the possible one-particle states.
We define a measure of quantum information which
is based on state partitions.
Properties of this measure for  entangled many-particle states are discussed.
$k$ particles specify $k$ ``nits'' in such a way that $k$ mutually commuting
measurements of observables with $n$ possible outcomes are necessary to determine the information.
\end{abstract}


\pacs{03.65.Ta,03.67.-a}
\keywords{quantum information theory,quantum measurement theory}


\maketitle

%\section{Introduction}
As pointed out many times by Landauer and others
(e.g., \cite{landauer,feynman-computation})
the formal concept of {\em information} is tied
to physics,
at least as far as applicability is a concern.
Thus it should come as no surprise that quantum mechanics
requires fundamentally new concepts of information
as compared to the ones appropriate for classical physics.
And indeed, research into quantum information and computation theory
has exploded in the last decade, bringing about a wealth of new
ideas, potential applications, and formalisms.

Yet, there seems to be one issue,
which, despite notable exceptions (e.g., \cite[Footnote 6]{zeil-99} and
\cite{Muthukrishnan}),
has not yet been acknowledged widely:
the principal irreducibility of quantum information in base $n$.
An $n$-state particle  can be prepared in a single one of $n$ possible states.
Then, this particle carries one ``nit'' of information, namely to
``be in a single one from $n$ different states.''
Subsequent measurements may confirm this statement.
The most natural code basis for such a configuration is a code in base $n$,
and not a binary or decimal one.


Classically, there is no preferred code basis whatsoever.
Every classical state is postulated to be determined by a
point in phase space.
Formally, this amounts to an infinite amount of information in whatever base,
since
with probability one, all points are random; i.e., algorithmically incompressible.
\cite{chaitin2,calude:94}.
Operationally, only a finite amount of classical information is accessible.
Yet, the particular base in which this finite amount of classical information is represented
is purely conventional.


Let us mention some notation and setup first.
Consider a particle which can be observed in a single one of
a finite number
$n$ of possible operationally distinct and comeasurable properties.
The system's state is
formalized by a self-adjoint, positive
state operator  of trace class $\rho$ of an $n$-dimensional Hilbert space.
Any pure state, whether ``entangled'' or not,
is characterized by a onedimensional subspace, which in turn
corresponds to a onedimensional projection operator $\rho^2=\rho$.

Every  operationally distinct property corresponds to the proposition
(we shall use identical symbols to denote logical propositions and operators)
$E_i$, $1\le i\le k$ that the
system, when measured, has the associated property.
Propositions are formalized by projection operators,
which are dichotomic observables, since
$E_i^2=E_i$ is only satisfied for the eigenvalues $0$ and $1$
\cite{v-neumann-49}.

Thus, any complete set of $n$ base vectors has a dual interpretation:
either as an orthonormal set of base states
whose linear span is the $n$-dimensional Hilbert space,
or as a maximally comeasurable and operationally distinct set of observables
corresponding to propositions such as,
{\em ``the physical system is in a pure state corresponding to the respective basis vector.''}
Any such proposition is operationalized by a measurement, ideally by registering a
click in a particle counter.

All bases corresponding to the $k$-particle case are obtained in two steps:
(i)
In the first step,
a system of pure basis vectors $\{\rho_1,\ldots ,\rho_{n^k}\}$
is formed by taking the tensor products
$\rho^s_{i_1}\otimes \rho^s_{i_2}\otimes \cdots \otimes\rho^s_{i_k}$,
$1\le i_1,i_2,\ldots ,i_k \le n$,
of all the single-particle
projection operators
$\rho^s_{i_1}$, $\rho^s_{i_2}\ldots \rho^s_{i_k}$.
(ii)
In the second step, this system of basis vector undergoes a unitary transformation
$U(n^k)$ represented by a $n^k \times n^k$-matrix.
The transformed states need no longer decompose into a product of single particle states,
a property called ``entanglement'' by Schr{\"{o}}dinger \cite{schrodinger}.
From this point of view,  entangled state bases are
unitary equivalent to nonentangled ones.
As a consequence,  propositions need no longer refer to attributes
or properties of single particles alone, but to joint properties of
particles \cite{zeil-99,DonSvo01}.

In what follows, let us always consider a complete
system of base states  ${\cal B}$ associated with a unique ``context'' \cite{svozil-2001-cesena}
or ``communication frame''
 ${\cal F}=\{F_1,F_2, \ldots ,F_{n} \}$, which are defined as a minimal set of
co-measurable observables with $n$ possible outcomes.
For $n=2$,
their explicit form has been enumerated in  \cite{DonSvo01}.
In this particular case, the $F$'s can be identified with certain projection
operators from the set of all possible mutually orthogonal ones,
whose two eigenvalues can be identified with the two states.
For three or more particles, this is no longer possible (see below).
A well-known theorem of linear algebra
(e.g.,
\cite{halmos-vs})
states that there exists a single ``context operator''
such that all the $F_i\in {\cal F}$ are just polynomials of it.
That is, a single measurement exists which determines
all the observables associated with the context
at once.



%\section{State partitions of one or more particles in a product state}

%\subsection{The general single particle case}
For a single $n$-state particle, the nit can be formalized by as a state partition
which is fine grained into $n$ elements, one state per element.
That is,
if the set of states is represented by $\{1,\ldots ,n\}$,
then the nit is defined by
\begin{equation}
\{\{1\} ,\ldots ,\{n\}\}
.
\label{2002-statepart-snit}
\end{equation}
Of course, any labeling would suffice, as long as the structure is preserved.
It does not matter
whether one calls the states, for instance, ``+,'' ``0'' and  ``-'', or ``1,'' ``2'' and ``3'',
resulting in a trit represented by
$\{\{+\} ,\{0\} ,\{-\}\}$ or
$\{\{1\} ,\{2\} ,\{3\}\}$.
Thus, nits are defined modulo isomorphisms (i.e., one-to-one translations)
of the state labels.
To complete the setup of the single particle case, let us
recall that any such state set would correspond to an orthonormal
basis of $n$-dimensional Hilbert space.

%\subsection{The case of two particles with three states}

Before proceeding to the most general case,
we shall consider the case of two particles
with three states per particle in all details.
The methods  developed \cite{DonSvo01} for case of $k$
particles
with two states per particle cannot be directly adopted here,
since the idempotence ($E_iE_i=E_i$) of the projection operators,
which maps the two states onto the two possible eigenvalues $0,1$ cannot
be generalized.

Instead we shall start by considering the optimal partitions of
states and construct the appropriate Hilbert space operators from there.
Assume that the first and second particle
has three orthogonal states labeled by
$a_1,b_1,c_1$
and
$a_2,b_2,c_2$,
respectively
(the subscript denotes the particle number;
$a_1a_2$ stands for
$
\vert a_1\rangle \otimes \vert a_2\rangle
=
\vert a_1 a_2\rangle
$).
Then nine product states can be formed and labeled from $1$ to $9$ in
lexicographic order; i.e.,
\begin{equation}
\begin{array}{llll}
a_1a_2 &\equiv&1,\\
a_1b_2 &\equiv&2,\\
a_1c_2 &\equiv&3,\\
b_1a_2 &\equiv&4,\\
  &\vdots& \\
c_1c_2 &\equiv&9.\\
\end{array}
\label{2002-statepart-ps3}
\end{equation}

Any maximal set of
co-measurable $3$-valued observables induces two state partitions
of the set of states $S=\{1,2,\ldots , 9\}$ with three partition
elements with the properties
that (i) the set theoretic intersection of any two elements of the two
partitions is a single state, and (ii) the union of all these nine
intersections is just the set of state $S$.
As can be easily checked, an example for such state partitions are
\begin{equation}
\begin{array}{llllll}
F_1&=&\{\{1,2,3\},\{4,5,6\},\{7,8,9\}\}&\equiv& \{\{a_1\},\{b_1\},\{c_1\}\},\\
F_2&=&\{\{1,4,7\},\{2,5,8\},\{3,6,9\}\}&\equiv& \{\{a_2\},\{b_2\},\{c_2\}\}.\\
\end{array}
\label{2002-statepart-ps3e}
\end{equation}
Operationally, the ``trit'' $F_1$ can be obtained by
measuring the first particle state:
$\{1,2,3\}$ is associated with state $a_1$,
$\{4,5,6\}$ is associated with $b_1$,  and
$\{4,5,6\}$ is associated with $c_1$.
The ``trit'' $F_2$ can be obtained by
measuring the state of the second particle:
$\{1,4,7\}$ is associated with state $a_2$,
$\{2,5,8\}$ is associated with $b_2$,  and
$\{3,6,9\}$ is associated with $c_2$.
This amounts to the operationalization of the trits (\ref{2002-statepart-ps3e})
as state filters.
In the above case, the filters are ``local'' and can be realized on single particles,
one trit per particle.
In the more general case of rotated ``entangled'' states (cf. below),
the trits (more generally, nits) become
inevitably associated with joint properties of ensembles of particles.
Measurement of the propositions,
{\em ``the particle is in state $\{1,2,3\}$''}
and,
{\em ``the particle is in state $\{3,6,9\}$''}
can be evaluated by taking the set theoretic intersection of the respective sets; i.e., by
the proposition,
{\em ``the particle is in state $\{1,2,3\}\cap \{3,6,9\} = 3$.''}
In figure \ref{2002-statepart1},
the state partitions are drawn as cells of a twodimensional square spanned
by the single cells of the two three-state particles.
\begin{figure}
\begin{tabular}{ccccccc}
%TexCad Options
%\grade{\off}
%\emlines{\off}
%\beziermacro{\on}
%\reduce{\on}
%\snapping{\off}
%\quality{2.00}
%\graddiff{0.01}
%\snapasp{1}
%\zoom{1.00}
\unitlength 0.37mm
\linethickness{0.4pt}
\begin{picture}(110.00,125.00)
\put(9.67,25.00){\framebox(90.33,90.00)[cc]{}}
\put(40.00,115.00){\line(0,-1){90.00}}
\put(70.00,115.00){\line(0,-1){90.00}}
\put(9.67,55.00){\line(1,0){90.33}}
\put(100.00,85.00){\line(-1,0){90.33}}
\put(0.00,40.00){\makebox(0,0)[cc]{$c_2$}}
\put(0.00,70.00){\makebox(0,0)[cc]{$b_2$}}
\put(0.00,100.00){\makebox(0,0)[cc]{$a_2$}}
\put(25.00,125.00){\makebox(0,0)[cc]{$a_1$}}
\put(55.00,125.00){\makebox(0,0)[cc]{$b_1$}}
\put(85.00,125.00){\makebox(0,0)[cc]{$c_1$}}
\put(25.33,15.00){\makebox(0,0)[cc]{$a_1$}}
\put(55.33,15.00){\makebox(0,0)[cc]{$b_1$}}
\put(85.33,15.00){\makebox(0,0)[cc]{$c_1$}}
\put(110.00,40.00){\makebox(0,0)[cc]{$c_2$}}
\put(110.00,70.00){\makebox(0,0)[cc]{$b_2$}}
\put(110.00,100.00){\makebox(0,0)[cc]{$a_2$}}
\put(55.00,100.00){\oval(86.00,26.00)[]}
\put(25.00,100.00){\makebox(0,0)[cc]{$1$}}
\put(55.00,100.00){\makebox(0,0)[cc]{$2$}}
\put(85.00,100.00){\makebox(0,0)[cc]{$3$}}
\put(55.00,70.00){\oval(86.00,26.00)[]}
\put(55.00,40.00){\oval(86.00,26.00)[]}
\put(25.00,70.00){\makebox(0,0)[cc]{$4$}}
\put(55.00,70.00){\makebox(0,0)[cc]{$5$}}
\put(85.00,70.00){\makebox(0,0)[cc]{$6$}}
\put(25.00,40.00){\makebox(0,0)[cc]{$7$}}
\put(55.00,40.00){\makebox(0,0)[cc]{$8$}}
\put(85.00,40.00){\makebox(0,0)[cc]{$9$}}
\put(55.00,0.00){\makebox(0,0)[cc]{$F_1$}}
\end{picture}
&
\quad
\unitlength 0.50mm
\linethickness{0.4pt}
\begin{picture}(0.00,105.00)
\put(0.00,52.00){\makebox(0,0)[cc]{$\;$}}
\end{picture}
&
%TexCad Options
%\grade{\off}
%\emlines{\off}
%\beziermacro{\on}
%\reduce{\on}
%\snapping{\off}
%\quality{2.00}
%\graddiff{0.01}
%\snapasp{1}
%\zoom{1.00}
\unitlength 0.37mm
\linethickness{0.4pt}
\begin{picture}(110.00,120.00)
\put(10.00,25.00){\framebox(90.33,90.00)[cc]{}}
\put(40.33,115.00){\line(0,-1){90.00}}
\put(70.33,115.00){\line(0,-1){90.00}}
\put(10.00,55.00){\line(1,0){90.33}}
\put(100.33,85.00){\line(-1,0){90.33}}
\put(0.00,40.00){\makebox(0,0)[cc]{$c_2$}}
\put(0.00,70.00){\makebox(0,0)[cc]{$b_2$}}
\put(0.00,100.00){\makebox(0,0)[cc]{$a_2$}}
\put(25.00,125.00){\makebox(0,0)[cc]{$a_1$}}
\put(55.00,125.00){\makebox(0,0)[cc]{$b_1$}}
\put(85.00,125.00){\makebox(0,0)[cc]{$c_1$}}
\put(25.33,15.00){\makebox(0,0)[cc]{$a_1$}}
\put(55.33,15.00){\makebox(0,0)[cc]{$b_1$}}
\put(85.33,15.00){\makebox(0,0)[cc]{$c_1$}}
\put(110.00,40.00){\makebox(0,0)[cc]{$c_2$}}
\put(110.00,70.00){\makebox(0,0)[cc]{$b_2$}}
\put(110.00,100.00){\makebox(0,0)[cc]{$a_2$}}
\put(25.33,70.00){\oval(26.00,86.00)[]}
\put(55.33,70.00){\oval(26.00,86.00)[]}
\put(85.33,70.00){\oval(26.00,86.00)[]}
\put(27.33,100.00){\makebox(0,0)[cc]{$1$}}
\put(27.33,70.00){\makebox(0,0)[cc]{$4$}}
\put(27.33,40.00){\makebox(0,0)[cc]{$7$}}
\put(57.33,100.00){\makebox(0,0)[cc]{$2$}}
\put(57.33,70.00){\makebox(0,0)[cc]{$5$}}
\put(57.33,40.00){\makebox(0,0)[cc]{$8$}}
\put(87.33,100.00){\makebox(0,0)[cc]{$3$}}
\put(87.33,70.00){\makebox(0,0)[cc]{$6$}}
\put(87.33,40.00){\makebox(0,0)[cc]{$9$}}
\put(55.00,0.00){\makebox(0,0)[cc]{$F_2$}}
\end{picture}
&
\quad
\unitlength 0.50mm
\linethickness{0.4pt}
\begin{picture}(0.00,105.00)
\put(0.00,52.00){\makebox(0,0)[cc]{$\;$}}
\end{picture}
&
%TexCad Options
%\grade{\off}
%\emlines{\off}
%\beziermacro{\on}
%\reduce{\on}
%\snapping{\off}
%\quality{2.00}
%\graddiff{0.01}
%\snapasp{1}
%\zoom{1.00}
\unitlength 0.37mm
\linethickness{0.4pt}
\begin{picture}(110.0,120.00)
\put(10.00,25.00){\framebox(90.33,90.00)[cc]{}}
\put(40.33,115.00){\line(0,-1){90.00}}
\put(70.33,115.00){\line(0,-1){90.00}}
\put(10.00,55.00){\line(1,0){90.33}}
\put(100.33,85.00){\line(-1,0){90.33}}
\put(0.00,40.00){\makebox(0,0)[cc]{$c_2$}}
\put(0.00,70.00){\makebox(0,0)[cc]{$b_2$}}
\put(0.00,100.00){\makebox(0,0)[cc]{$a_2$}}
\put(25.00,125.00){\makebox(0,0)[cc]{$a_1$}}
\put(55.00,125.00){\makebox(0,0)[cc]{$b_1$}}
\put(85.00,125.00){\makebox(0,0)[cc]{$c_1$}}
\put(25.33,15.00){\makebox(0,0)[cc]{$a_1$}}
\put(55.33,15.00){\makebox(0,0)[cc]{$b_1$}}
\put(85.33,15.00){\makebox(0,0)[cc]{$c_1$}}
\put(110.00,40.00){\makebox(0,0)[cc]{$c_2$}}
\put(110.00,70.00){\makebox(0,0)[cc]{$b_2$}}
\put(110.00,100.00){\makebox(0,0)[cc]{$a_2$}}
\put(25.00,100.00){\oval(26.00,26.00)[]}
\put(25.00,100.00){\makebox(0,0)[cc]{1}}
\put(55.00,100.00){\oval(26.00,26.00)[]}
\put(85.00,100.00){\oval(26.00,26.00)[]}
\put(55.00,100.00){\makebox(0,0)[cc]{2}}
\put(85.00,100.00){\makebox(0,0)[cc]{3}}
\put(25.00,70.00){\oval(26.00,26.00)[]}
\put(25.00,40.00){\oval(26.00,26.00)[]}
\put(25.00,70.00){\makebox(0,0)[cc]{4}}
\put(25.00,40.00){\makebox(0,0)[cc]{7}}
\put(55.00,70.00){\oval(26.00,26.00)[]}
\put(55.00,40.00){\oval(26.00,26.00)[]}
\put(85.00,70.00){\oval(26.00,26.00)[]}
\put(85.00,40.00){\oval(26.00,26.00)[]}
\put(55.00,70.00){\makebox(0,0)[cc]{5}}
\put(55.00,40.00){\makebox(0,0)[cc]{8}}
\put(85.00,70.00){\makebox(0,0)[cc]{6}}
\put(85.00,40.00){\makebox(0,0)[cc]{9}}
\put(55.00,0.00){\makebox(0,0)[cc]{$F_1\wedge F_2$}}
\end{picture}
\end{tabular}
 \caption{
Representation of
state partitions as cells of a twodimensional square spanned
by the single cells of the two three-state particles.
}
\label{2002-statepart1}
\end{figure}

A Hilbert space representation of this setting can be obtained as
follows.
Define the states in $S$ to be onedimensional linear subspaces of
nine-dimensional Hilbert space; e.g.,
\begin{equation}
\begin{array}{llll}
1 &\equiv& (1,0,0,0,0,0,0,0,0),\\
  &\vdots&\\
9 &\equiv& (0,0,0,0,0,0,0,0,1).\\
\end{array}
\label{2002-statepart-ps3a}
\end{equation}
The trit operators are given by (trit operators, observables and the
corresponding state partitions will be used synonymously)
\begin{equation}
\begin{array}{llll}
F_1&=& \diag (a,a,a,b,b,b,c,c,c),\\
F_2&=& \diag (a,b,c,a,b,c,a,b,c),\\
\end{array}
\label{2002-statepart-ps3top}
\end{equation}
for $a,b,c \in {\Bbb R}$, $a\neq b\neq c\neq a$.

If $F_2= \diag (d,e,f,d,e,f,d,e,f)$
and $a,b,c,d,e,f,$ are six different prime numbers,
then, due to the uniqueness of prime decompositions,
the two trit operators
can be combined to a single
``context'' operator
\begin{equation}
C=F_1\cdot F_2=F_2\cdot F_1=
\diag (ad,ae,af,bd,be,bf,cd,ce,cf)
\label{2002-statepart-ps3pd}
\end{equation}
which acts on both particles and has nine different eigenvalues.
Just as for the two-particle case \cite{DonSvo01},
there exist $2^3!=9!=362880$ permutations of operators
which are all able to separate the nine states.
They are obtained by forming a $(2\times 9)$-matrix
whose rows are the diagonal components of $F_1$ and $F_2$
from Eq. (\ref{2002-statepart-ps3top})
and permuting all the columns.
The resulting new operators $F_1'$ and $F_2'$ are also trit operators.





%\subsection{The general case}
A generalization to $k$ particles in $n$ states per particle is straightforward.
We obtain
(i) $k$ partitions of the product states with
(ii) $n$ elements per partition in such a way that
(iii) every single product state is obtained by the set theoretic intersection of
$k$ elements of all the different partitions.

Every single such partition can be interpreted as a nit.
All such sets are generated by permuting the set of states,
which amounts to $n^k!$ equivalent sets of state partitions.
However, since they are mere one-to-one translations,
they represent the same trits.
This equivalence, however, does not concern the property of (non)entanglement,
since the permutations take entangled states into nonentangled ones.
We shall give an example below.

Again, the standard orthonormal basis of
$n^k$-dimensional Hilbert space is identified with the set of states $S=\{1,2,\ldots ,n^k\}$; i.e.,
(superscript ``$T$'' indicates transposition)
\begin{equation}
\begin{array}{llll}
1 &\equiv& (1,\ldots,0)^T\equiv \mid 1,\ldots ,1\rangle = \mid 1\rangle \otimes \cdots \otimes \mid 1\rangle ,\\
  &\vdots&\\
n^k &\equiv& (0,\ldots,1)^T\equiv \mid n,\ldots ,n\rangle = \mid n\rangle \otimes \cdots \otimes \mid n\rangle .\\
\end{array}
\label{2002-statepart-psma}
\end{equation}
The single-particle states are also labeled by $1$ through $n$,
and the tensor product states are formed and ordered lexicographically ($0<1$).


The nit operators are defined via diagonal matrices
which contain equal amounts $n^{k-1}$ of mutually $n$ different numbers
such as different primes $q_1,\ldots ,q_n$; i.e.,
\begin{equation}
\begin{array}{llll}
F_1&=& \diag (\underbrace{\underbrace{q_1,\ldots ,q_1}_{n^{k-1}\;{\rm times}},\ldots ,\underbrace{q_n,\ldots ,q_n}_{n^{k-1}\;{\rm times}}}_{n^0\;{\rm times}}),\\
F_2&=& \diag (\underbrace{\underbrace{q_1,\ldots ,q_1}_{n^{k-2}\;{\rm times}},\ldots ,\underbrace{q_n,\ldots ,q_n}_{n^{k-2}\;{\rm times}}}_{n^1\;{\rm times}}),\\
  &\vdots&\\
F_k&=& \diag (\underbrace{q_1,\ldots ,q_n}_{n^{k-1}\;{\rm times}}).
\end{array}
\label{2002-statepart-nitopgen}
\end{equation}
The operators implement an $n$-ary search strategy,
filtering the search space into $n$ equal partitions of states,
such that a successive applications of all such filters
renders a single state.

There exist $n^k!$  sets of nit operators,
which are
are obtained by forming a $(n^k \times n^k)$-matrix
whose rows are the diagonal components of $F_1,\ldots,F_k$  from Eq.
(\ref{2002-statepart-nitopgen})
and permuting all the columns.
The resulting new operators $F_1',\ldots,F_k'$ are also nit operators.

All partitions discussed so far are equally weighted and well balanced,
as all elements of them contain an equal number of states.
In principle, one could also consider nonbalanced partitions.
For example, one could take the partition
$\overline{F}_1=\{\{1\},\{2,3\},\{4,5,6,7,8,9\}\}$
instead of $F_1$ in (\ref{2002-statepart-ps3e}),
represented the by trit diagonal operator
$\diag (a,b,b,c,c,c,c,c,c)$.
Yet any such attempt would result
in a deviation from the optimal $n$-ary search strategy, and
in an nonoptimal measurement procedures.
Another, more principal, disadvantage would be the fact that such a state separation
could not reflect the inevitable $n$-arity of the quantum choice.



%\section{The entangled case and diagonal bases}

%Consider a particular orthogonal basis of a
%$k$-particle state such as (\ref{2002-statepart-psma}),
%in which all elements correspond to tensor products of the single particle states.
%Such tensor products are also called nonentangled states.
%Entangled states correspond to the weighted sum of two or more tensor product states.
%That is, let
%$
%\vert q_{i_1}^1 \cdots q_{i_k}^k\rangle
%=
%\vert q_{i_1}^1\rangle
%\otimes
%\cdots
%\otimes
%\vert q_{i_k}^k\rangle
%$, $1\le i_1,\ldots,i_k \le n$,
%be orthonormal tensor products of the single particle states $\vert q_{i_j}^j\rangle$, $1\le j\le m$.
%Then  a general $k$-particle state can be written as
%\begin{equation}
%\vert \Psi \rangle
%=
%\sum_{1\le i_1,\ldots,i_k \le n}
%c_{i_1 \ldots i_k}
%\vert q_{i_1}^1 \cdots q_{i_k}^k\rangle  ,\quad  c_{i_1 \ldots i_k}\in {\Bbb C},\quad  \sum_{1\le i_1,\ldots,i_k \le n} \vert c_{i_1 \ldots i_k}\vert^2=1.
%\label{2002-statepart-epstate}
%\end{equation}
%
%If the sum (\ref{2002-statepart-epstate}) cannot be rewritten as the tensor product
%of $k$ particle states which have been rotated in the respective single-particle
%subspace $U(n)$,
%the state is entangled.

The most important feature of entangled states is that the nits,
or propositions corresponding to $F_i$, are no longer single-particle properties,
but in general depend on the joint properties of all the $k$ particles.
Formally,
this feature of entanglement is reflected in the isometric (unitary) transformation
of the standard state basis (\ref{2002-statepart-psma}), such that
this unitary transformation leads outside the domain of single particle
unitary transformations.
One could define a measure of entanglement by considering the sum total
of all parameter differences from the nearest nonentangled states.
This amounts to a nontrivial group theoretic question \cite{dirl-svozil-2002}.
In terms of partitions, a different approach is possible.
Entanglement occurs for diagonal or antidiagonal arrangements of states
which do not add up to completed blocks.

Take, for example, the state partition scheme of Fig. \ref{2002-statepart1},
which results in nonentangled states and state measurements.
A modified, entangled scheme can be established by just grouping the states
into diagonal and counterdiagonal groups as drawn in Fig.
\ref{2002-statepart2}.
The corresponding trits are
\begin{equation}
\begin{array}{llll}
F_1&=&\{\{1,5,9\},\{2,6,7\},\{3,4,8\}\},\\
F_2&=&\{\{1,6,8\},\{2,4,9\},\{3,5,7\}\}.\\
\end{array}
\label{2002-statepart-ps3eentan}
\end{equation}
\begin{figure}
\begin{tabular}{ccccccc}
%TexCad Options
%\grade{\off}
%\emlines{\off}
%\beziermacro{\on}
%\reduce{\on}
%\snapping{\off}
%\quality{2.00}
%\graddiff{0.01}
%\snapasp{1}
%\zoom{1.00}
\unitlength 0.37mm
\linethickness{0.4pt}
\begin{picture}(110.00,125.00)
\put(9.67,25.00){\framebox(90.33,90.00)[cc]{}}
\put(40.00,115.00){\line(0,-1){90.00}}
\put(70.00,115.00){\line(0,-1){90.00}}
\put(9.67,55.00){\line(1,0){90.33}}
\put(100.00,85.00){\line(-1,0){90.33}}
\put(0.00,40.00){\makebox(0,0)[cc]{$c_2$}}
\put(0.00,70.00){\makebox(0,0)[cc]{$b_2$}}
\put(0.00,100.00){\makebox(0,0)[cc]{$a_2$}}
\put(25.00,125.00){\makebox(0,0)[cc]{$a_1$}}
\put(55.00,125.00){\makebox(0,0)[cc]{$b_1$}}
\put(85.00,125.00){\makebox(0,0)[cc]{$c_1$}}
\put(25.33,15.00){\makebox(0,0)[cc]{$a_1$}}
\put(55.33,15.00){\makebox(0,0)[cc]{$b_1$}}
\put(85.33,15.00){\makebox(0,0)[cc]{$c_1$}}
\put(110.00,40.00){\makebox(0,0)[cc]{$c_2$}}
\put(110.00,70.00){\makebox(0,0)[cc]{$b_2$}}
\put(110.00,100.00){\makebox(0,0)[cc]{$a_2$}}
\put(12.00,113.00){\line(0,-1){13.00}}
\put(12.00,100.00){\line(1,-1){73.00}}
\put(85.00,27.00){\line(1,0){13.00}}
\put(98.00,27.00){\line(0,1){13.00}}
\put(98.00,40.00){\line(-1,1){73.00}}
\put(25.00,113.00){\line(-1,0){13.00}}
\put(12.00,83.00){\line(0,-1){13.00}}
\put(12.00,53.00){\line(0,-1){13.00}}
\put(25.00,83.00){\line(-1,0){13.00}}
\put(25.00,53.00){\line(-1,0){13.00}}
\put(55.00,27.00){\line(1,0){13.00}}
\put(25.00,27.00){\line(1,0){13.00}}
\put(68.00,27.00){\line(0,1){13.00}}
\put(38.00,27.00){\line(0,1){13.00}}
\put(38.00,40.00){\line(-1,1){13.00}}
\put(55.00,27.00){\line(-1,1){43.00}}
\put(68.00,40.00){\line(-1,1){43.00}}
\put(25.00,27.00){\line(-1,1){13.00}}
\put(42.00,113.00){\line(1,0){13.00}}
\put(72.00,113.00){\line(1,0){13.00}}
\put(42.00,100.00){\line(0,1){13.00}}
\put(72.00,100.00){\line(0,1){13.00}}
\put(98.00,70.00){\line(0,-1){13.00}}
\put(98.00,100.00){\line(0,-1){13.00}}
\put(98.00,57.00){\line(-1,0){13.00}}
\put(98.00,87.00){\line(-1,0){13.00}}
\put(85.00,87.00){\line(-1,1){13.00}}
\put(98.00,70.00){\line(-1,1){43.00}}
\put(85.00,57.00){\line(-1,1){43.00}}
\put(98.00,100.00){\line(-1,1){13.00}}
\put(25.00,100.00){\makebox(0,0)[cc]{1}}
\put(55.00,100.00){\makebox(0,0)[cc]{2}}
\put(85.00,100.00){\makebox(0,0)[cc]{3}}
\put(25.00,70.00){\makebox(0,0)[cc]{4}}
\put(25.00,40.00){\makebox(0,0)[cc]{7}}
\put(55.00,70.00){\makebox(0,0)[cc]{5}}
\put(55.00,40.00){\makebox(0,0)[cc]{8}}
\put(85.00,70.00){\makebox(0,0)[cc]{6}}
\put(85.00,40.00){\makebox(0,0)[cc]{9}}
\put(55.00,0.00){\makebox(0,0)[cc]{$F_1$}}
\end{picture}
&
\quad
\unitlength 0.50mm
\linethickness{0.4pt}
\begin{picture}(0.00,105.00)
\put(0.00,52.00){\makebox(0,0)[cc]{$\;$}}
\end{picture}
&
%TexCad Options
%\grade{\off}
%\emlines{\off}
%\beziermacro{\on}
%\reduce{\on}
%\snapping{\off}
%\quality{2.00}
%\graddiff{0.01}
%\snapasp{1}
%\zoom{1.00}
\unitlength 0.37mm
\linethickness{0.4pt}
\begin{picture}(110.00,125.00)
\put(9.67,25.00){\framebox(90.33,90.00)[cc]{}}
\put(40.00,115.00){\line(0,-1){90.00}}
\put(70.00,115.00){\line(0,-1){90.00}}
\put(9.67,55.00){\line(1,0){90.33}}
\put(100.00,85.00){\line(-1,0){90.33}}
\put(0.00,40.00){\makebox(0,0)[cc]{$c_2$}}
\put(0.00,70.00){\makebox(0,0)[cc]{$b_2$}}
\put(0.00,100.00){\makebox(0,0)[cc]{$a_2$}}
\put(25.00,125.00){\makebox(0,0)[cc]{$a_1$}}
\put(55.00,125.00){\makebox(0,0)[cc]{$b_1$}}
\put(85.00,125.00){\makebox(0,0)[cc]{$c_1$}}
\put(25.33,15.00){\makebox(0,0)[cc]{$a_1$}}
\put(55.33,15.00){\makebox(0,0)[cc]{$b_1$}}
\put(85.33,15.00){\makebox(0,0)[cc]{$c_1$}}
\put(110.00,40.00){\makebox(0,0)[cc]{$c_2$}}
\put(110.00,70.00){\makebox(0,0)[cc]{$b_2$}}
\put(110.00,100.00){\makebox(0,0)[cc]{$a_2$}}
\put(98.00,113.00){\line(-1,0){13.00}}
\put(85.00,113.00){\line(-1,-1){73.00}}
\put(12.00,40.00){\line(0,-1){13.00}}
\put(12.00,27.00){\line(1,0){13.00}}
\put(25.00,27.00){\line(1,1){73.00}}
\put(98.00,100.00){\line(0,1){13.00}}
\put(68.00,113.00){\line(-1,0){13.00}}
\put(38.00,113.00){\line(-1,0){13.00}}
\put(68.00,100.00){\line(0,1){13.00}}
\put(38.00,100.00){\line(0,1){13.00}}
\put(12.00,70.00){\line(0,-1){13.00}}
\put(12.00,100.00){\line(0,-1){13.00}}
\put(12.00,57.00){\line(1,0){13.00}}
\put(12.00,87.00){\line(1,0){13.00}}
\put(25.00,87.00){\line(1,1){13.00}}
\put(12.00,70.00){\line(1,1){43.00}}
\put(25.00,57.00){\line(1,1){43.00}}
\put(12.00,100.00){\line(1,1){13.00}}
\put(98.00,83.00){\line(0,-1){13.00}}
\put(98.00,53.00){\line(0,-1){13.00}}
\put(85.00,83.00){\line(1,0){13.00}}
\put(85.00,53.00){\line(1,0){13.00}}
\put(55.00,27.00){\line(-1,0){13.00}}
\put(85.00,27.00){\line(-1,0){13.00}}
\put(42.00,27.00){\line(0,1){13.00}}
\put(72.00,27.00){\line(0,1){13.00}}
\put(72.00,40.00){\line(1,1){13.00}}
\put(55.00,27.00){\line(1,1){43.00}}
\put(42.00,40.00){\line(1,1){43.00}}
\put(85.00,27.00){\line(1,1){13.00}}
\put(25.00,100.00){\makebox(0,0)[cc]{1}}
\put(55.00,100.00){\makebox(0,0)[cc]{2}}
\put(85.00,100.00){\makebox(0,0)[cc]{3}}
\put(25.00,70.00){\makebox(0,0)[cc]{4}}
\put(25.00,40.00){\makebox(0,0)[cc]{7}}
\put(55.00,70.00){\makebox(0,0)[cc]{5}}
\put(55.00,40.00){\makebox(0,0)[cc]{8}}
\put(85.00,70.00){\makebox(0,0)[cc]{6}}
\put(85.00,40.00){\makebox(0,0)[cc]{9}}
\put(55.00,0.00){\makebox(0,0)[cc]{$F_2$}}
\end{picture}
&
\quad
\unitlength 0.50mm
\linethickness{0.4pt}
\begin{picture}(0.00,105.00)
\put(0.00,52.00){\makebox(0,0)[cc]{$\;$}}
\end{picture}
&
%TexCad Options
%\grade{\off}
%\emlines{\off}
%\beziermacro{\on}
%\reduce{\on}
%\snapping{\off}
%\quality{2.00}
%\graddiff{0.01}
%\snapasp{1}
%\zoom{1.00}
\unitlength 0.37mm
\linethickness{0.4pt}
\begin{picture}(110.0,120.00)
\put(10.00,25.00){\framebox(90.33,90.00)[cc]{}}
\put(40.33,115.00){\line(0,-1){90.00}}
\put(70.33,115.00){\line(0,-1){90.00}}
\put(10.00,55.00){\line(1,0){90.33}}
\put(100.33,85.00){\line(-1,0){90.33}}
\put(0.00,40.00){\makebox(0,0)[cc]{$c_2$}}
\put(0.00,70.00){\makebox(0,0)[cc]{$b_2$}}
\put(0.00,100.00){\makebox(0,0)[cc]{$a_2$}}
\put(25.00,125.00){\makebox(0,0)[cc]{$a_1$}}
\put(55.00,125.00){\makebox(0,0)[cc]{$b_1$}}
\put(85.00,125.00){\makebox(0,0)[cc]{$c_1$}}
\put(25.33,15.00){\makebox(0,0)[cc]{$a_1$}}
\put(55.33,15.00){\makebox(0,0)[cc]{$b_1$}}
\put(85.33,15.00){\makebox(0,0)[cc]{$c_1$}}
\put(110.00,40.00){\makebox(0,0)[cc]{$c_2$}}
\put(110.00,70.00){\makebox(0,0)[cc]{$b_2$}}
\put(110.00,100.00){\makebox(0,0)[cc]{$a_2$}}
\put(25.00,100.00){\oval(26.00,26.00)[]}
\put(25.00,100.00){\makebox(0,0)[cc]{1}}
\put(55.00,100.00){\oval(26.00,26.00)[]}
\put(85.00,100.00){\oval(26.00,26.00)[]}
\put(55.00,100.00){\makebox(0,0)[cc]{2}}
\put(85.00,100.00){\makebox(0,0)[cc]{3}}
\put(25.00,70.00){\oval(26.00,26.00)[]}
\put(25.00,40.00){\oval(26.00,26.00)[]}
\put(25.00,70.00){\makebox(0,0)[cc]{4}}
\put(25.00,40.00){\makebox(0,0)[cc]{7}}
\put(55.00,70.00){\oval(26.00,26.00)[]}
\put(55.00,40.00){\oval(26.00,26.00)[]}
\put(85.00,70.00){\oval(26.00,26.00)[]}
\put(85.00,40.00){\oval(26.00,26.00)[]}
\put(55.00,70.00){\makebox(0,0)[cc]{5}}
\put(55.00,40.00){\makebox(0,0)[cc]{8}}
\put(85.00,70.00){\makebox(0,0)[cc]{6}}
\put(85.00,40.00){\makebox(0,0)[cc]{9}}
\put(55.00,0.00){\makebox(0,0)[cc]{$F_1\wedge F_2$}}
\end{picture}
\end{tabular}
 \caption{
Entangled schemes through diagonalization and counterdiagonalization
of the states.
}
\label{2002-statepart2}
\end{figure}

We can now introduce new $2\times 3$  basis vectors  grouped into the two bases
$\{a_1',b_1',c_1'\}$
and
$\{a_2',b_2',c_2'\}$ by
\begin{equation}
\begin{array}{llll}
\vert a_1'  \rangle &=&{1\over \sqrt{3}}
   (\vert a_1a_2\rangle +  \vert b_1b_2\rangle  + \vert c_1c_2\rangle )
%   \equiv {1\over \sqrt{3}} (1,0,0,0,1,0,0,0,1)
,\\
\vert b_1'  \rangle &=&{1\over \sqrt{3}}
   (\vert a_1b_2\rangle +  \vert b_1c_2\rangle  + \vert c_1a_2\rangle )
   %\equiv {1\over \sqrt{3}} (0,1,0,0,0,1,1,0,0)
,\\
\vert c_1'  \rangle &=&{1\over \sqrt{3}}
   (\vert a_1c_2\rangle +  \vert b_1a_2\rangle  + \vert c_1b_2\rangle )
   %\equiv {1\over \sqrt{3}} (0,0,1,1,0,0,0,1,0)
,\\
\vert a_2'  \rangle &=&{1\over \sqrt{3}}
   (\vert a_1a_2\rangle +  \vert b_1c_2\rangle  + \vert c_1b_2\rangle )
   %\equiv {1\over \sqrt{3}} (1,0,0,0,0,1,0,1,0)
,\\
\vert b_2'  \rangle &=&{1\over \sqrt{3}}
   (\vert a_1b_2\rangle +  \vert b_1a_2\rangle  + \vert c_1c_2\rangle )
   %\equiv {1\over \sqrt{3}} (0,1,0,1,0,0,0,0,1)
,\\
\vert c_2'  \rangle &=&{1\over \sqrt{3}}
   (\vert a_1c_2\rangle +  \vert b_1b_2\rangle  + \vert c_1a_2\rangle )
   %\equiv {1\over \sqrt{3}} (0,0,1,0,1,0,1,0,0)
.
\end{array}
\label{2002-statepart-notb}
\end{equation}
The new orthonormal basis states are ``entangled'' with respect to the old bases
and {\em vice versa}.
Their tensor products generate a complete set of basis states in a new
nine-dimensional Hilbert space.
In terms of the new basis states, the trits can be written as
$F_1\equiv \{\{a_1'\},\{b_1'\},\{c_1'\}\}$
and
$F_2\equiv \{\{a_2'\},\{b_2'\},\{c_2'\}\}$.
The associated bases will be called {\em diagonal bases}.
Note that the permutation which produces the entangled case
(\ref{2002-statepart-ps3eentan})
the nonentangled
(\ref{2002-statepart-ps3e})
one is
$1\rightarrow 1$,
$2\rightarrow 9$,
$3\rightarrow 5$,
$4\rightarrow 6$,
$5\rightarrow 2$,
$6\rightarrow 7$,
$7\rightarrow 8$,
$8\rightarrow 4$,
$9\rightarrow 3$, or $(1)(2,9,3,5)(4,6,7,8)$ in cycle form.


A generalization to diagonal bases associated with
an arbitrary number of nits is straightforward.
We conjecture that the  basis generated by the diagonal  bases are
``maximally entangled'' with respect to group theoretic measures of entanglement.




%\section{Discussion and summary}

If Hilbert space is taken for granted as a valid formalization of quantum mechanics,
then Gleason's theorem \cite{Gleason} and many later results
\cite{kamber64,kamber65,ZirlSchl-65,CalHerSvo}, most notably the theorems by
Bell \cite{bell,mermin-93},
Kochen \& Specker \cite{kochen1},
and Greenberger, Horne \& Zeilinger  \cite{ghsz}
state the impossibility of (noncontextual) value definiteness.
That is, there does not seem to exist elements of reality to every conceivable
observable but a single (complete) one.

One possible consequence (among many others, not necessarily compatible ones)
of this fact may be the assumption that a quantum state of $k$ $n$-state particles
carries just
$k$ nits, and no more information.
It should be stressed that,
from this point of view,
all other conceivable information relating to counterfactual elements
of physical reality
corresponding to state bases different
from the encoded one are nonexistent.
The measurement outcomes in such improper communication frames
are randomized (maybe because the measurement apparatus
serves as an randomizing interface to the proper communication frame,
maybe because the information cannot be interpreted correctly).
If interrogated ``correctly'' (i.e., in the proper communication frame),
it will pass on this information unambiguously.
In such a frame, but only in this particular one,
this information is pseudo-classical and therefore can
also be copied or ``cloned.''
Such a point of view is suggested, to some extend, by Peres' statement
that ``unperformed experiments have no results'' \cite{peres222},
and is compatible to Zeilinger's foundational principle for quantum mechanics
\cite{zeil-99} stating that {\em ``the most elementary quantum system represents the truth value
of one proposition.''}


The quantum logical description allows for two possible representations.
(i) Every single partition
$F_i$ generates a subalgebra of the Boolean algebra of states.
All the subalgebras corresponding to the $k$ partitions and their
can then be pasted together to form a nonboolean lattice.
(ii) All partitions can be used to generate the context operator (cf. above) which
generates the finest partition. Every element of the
latter partition can then be identified with the atoms of a Boolean subalgebra.

Every such system can be modeled
by a finite automaton
partition logic \cite{svozil-93,schaller-96,dvur-pul-svo,cal-sv-yu,svozil-ql}
or a generalized urn model   \cite{wright:pent,wright,svozil-2001-eua}.
All filters combined render a Boolean algebra $2^{n^k}$ with ${n^k}$ atoms.

As regards the binary codes and their relation to codes in base $n$,
by a well known theorem for unitary operators
\cite{murnaghan},
any corresponding quantum measurement can be decomposed
into binary measurements.
Also, it is possible to group the $k$ possible outcomes into binary filters of
ever finer resolution; calling the successive outcomes of these filter process
the ``binary code.''
Yet, all these attempts result in codes with undesirable features.
Unitary decompositions in general yield noncomeasurable observables and thus to
nonoperationalizability. Filters are inefficient, and so may be binary codes
\cite{Cal-Cam-96}.

So far, the main emphasis in the area of quantum computation
has been in the area of binary decision problems.
It is suggested that these investigations should be extended to
decision problems with $n$ alternatives (e.g., \cite[pp. 332-340]{kleene-52}),
for which quantum information theory seems
to be extraordinarily well equipped.


\begin{thebibliography}{38}
\expandafter\ifx\csname natexlab\endcsname\relax\def\natexlab#1{#1}\fi
\expandafter\ifx\csname bibnamefont\endcsname\relax
  \def\bibnamefont#1{#1}\fi
\expandafter\ifx\csname bibfnamefont\endcsname\relax
  \def\bibfnamefont#1{#1}\fi
\expandafter\ifx\csname citenamefont\endcsname\relax
  \def\citenamefont#1{#1}\fi
\expandafter\ifx\csname url\endcsname\relax
  \def\url#1{\texttt{#1}}\fi
\expandafter\ifx\csname urlprefix\endcsname\relax\def\urlprefix{URL }\fi
\providecommand{\bibinfo}[2]{#2}
\providecommand{\eprint}[2][]{\url{#2}}

\bibitem[{\citenamefont{Landauer}(1991)}]{landauer}
\bibinfo{author}{\bibfnamefont{R.}~\bibnamefont{Landauer}},
  \bibinfo{journal}{Physics Today} \textbf{\bibinfo{volume}{44}},
  \bibinfo{pages}{23} (\bibinfo{year}{1991}).

\bibitem[{\citenamefont{Feynman}(1996)}]{feynman-computation}
\bibinfo{author}{\bibfnamefont{R.~P.} \bibnamefont{Feynman}},
  \emph{\bibinfo{title}{The Feynman lectures on computation}}
  (\bibinfo{publisher}{Addison-Wesley Publishing Company},
  \bibinfo{address}{Reading, MA}, \bibinfo{year}{1996}), \bibinfo{note}{edited
  by A.J.G. Hey and R. W. Allen}.

\bibitem[{\citenamefont{Zeilinger}(1999)}]{zeil-99}
\bibinfo{author}{\bibfnamefont{A.}~\bibnamefont{Zeilinger}},
  \bibinfo{journal}{Foundations of Physics} \textbf{\bibinfo{volume}{29}},
  \bibinfo{pages}{631} (\bibinfo{year}{1999}).

\bibitem[{\citenamefont{Muthukrishnan and C.~R.~Strod}(2000)}]{Muthukrishnan}
\bibinfo{author}{\bibfnamefont{A.}~\bibnamefont{Muthukrishnan}}
  \bibnamefont{and}
  \bibinfo{author}{\bibfnamefont{J.}~\bibnamefont{C.~R.~Strod}},
  \bibinfo{journal}{Physical Review} \textbf{\bibinfo{volume}{A62}},
  \bibinfo{pages}{052309} (\bibinfo{year}{2000}).

\bibitem[{\citenamefont{Chaitin}(1990)}]{chaitin2}
\bibinfo{author}{\bibfnamefont{G.~J.} \bibnamefont{Chaitin}},
  \emph{\bibinfo{title}{Information, Randomness and Incompleteness}}
  (\bibinfo{publisher}{World Scientific}, \bibinfo{address}{Singapore},
  \bibinfo{year}{1990}), \bibinfo{edition}{2nd} ed., \bibinfo{note}{this is a
  collection of G. Chaitin's publications}.

\bibitem[{\citenamefont{Calude}(1994)}]{calude:94}
\bibinfo{author}{\bibfnamefont{C.}~\bibnamefont{Calude}},
  \emph{\bibinfo{title}{Information and Randomness---An Algorithmic
  Perspective}} (\bibinfo{publisher}{Springer}, \bibinfo{address}{Berlin},
  \bibinfo{year}{1994}).

\bibitem[{\citenamefont{von Neumann}(1932)}]{v-neumann-49}
\bibinfo{author}{\bibfnamefont{J.}~\bibnamefont{von Neumann}},
  \emph{\bibinfo{title}{Mathematische Grundlagen der Quantenmechanik}}
  (\bibinfo{publisher}{Springer}, \bibinfo{address}{Berlin},
  \bibinfo{year}{1932}), \bibinfo{note}{english translation: {\sl Mathematical
  Foundations of Quantum Mechanics}, Princeton University Press, Princeton,
  1955}.

\bibitem[{\citenamefont{Schr{\"{o}}dinger}(1935)}]{schrodinger}
\bibinfo{author}{\bibfnamefont{E.}~\bibnamefont{Schr{\"{o}}dinger}},
  \bibinfo{journal}{Naturwissenschaften} \textbf{\bibinfo{volume}{23}},
  \bibinfo{pages}{807} (\bibinfo{year}{1935}), \bibinfo{note}{english
  translation in \cite{trimmer} and \cite[pp. 152-167]{wheeler-Zurek:83}; a
  copy can be found at www.emr.hibu.no/lars/eng/cat/}.

\bibitem[{\citenamefont{Donath and Svozil}(2002{\natexlab{a}})}]{DonSvo01}
\bibinfo{author}{\bibfnamefont{N.}~\bibnamefont{Donath}} \bibnamefont{and}
  \bibinfo{author}{\bibfnamefont{K.}~\bibnamefont{Svozil}},
  \bibinfo{journal}{Physical Review A} \textbf{\bibinfo{volume}{65}},
  \bibinfo{pages}{044302} (\bibinfo{year}{2002}{\natexlab{a}}),
  \bibinfo{note}{{\tt arXiv:quant-ph/0105046}, also published in
  \cite{DonSvo01vjqi}}.

\bibitem[{\citenamefont{Svozil}(2001)}]{svozil-2001-cesena}
\bibinfo{author}{\bibfnamefont{K.}~\bibnamefont{Svozil}}
  (\bibinfo{year}{2001}), \eprint{quant-ph/0012066}.

\bibitem[{\citenamefont{Halmos}(1974)}]{halmos-vs}
\bibinfo{author}{\bibfnamefont{P.~R.} \bibnamefont{Halmos}},
  \emph{\bibinfo{title}{Finite-Dimensional Vector spaces}}
  (\bibinfo{publisher}{Springer}, \bibinfo{address}{New York, Heidelberg,
  Berlin}, \bibinfo{year}{1974}).

\bibitem[{\citenamefont{Dirl and Svozil}(2002)}]{dirl-svozil-2002}
\bibinfo{author}{\bibfnamefont{R.}~\bibnamefont{Dirl}} \bibnamefont{and}
  \bibinfo{author}{\bibfnamefont{K.}~\bibnamefont{Svozil}}
  (\bibinfo{year}{2002}), \bibinfo{note}{in preparation}.

\bibitem[{\citenamefont{Gleason}(1957)}]{Gleason}
\bibinfo{author}{\bibfnamefont{A.~M.} \bibnamefont{Gleason}},
  \bibinfo{journal}{Journal of Mathematics and Mechanics}
  \textbf{\bibinfo{volume}{6}}, \bibinfo{pages}{885} (\bibinfo{year}{1957}).

\bibitem[{\citenamefont{Kamber}(1964)}]{kamber64}
\bibinfo{author}{\bibfnamefont{F.}~\bibnamefont{Kamber}},
  \bibinfo{journal}{Nachr. Akad. Wiss. G{\"{o}}ttingen}
  \textbf{\bibinfo{volume}{10}}, \bibinfo{pages}{103} (\bibinfo{year}{1964}).

\bibitem[{\citenamefont{Kamber}(1965)}]{kamber65}
\bibinfo{author}{\bibfnamefont{F.}~\bibnamefont{Kamber}},
  \bibinfo{journal}{Mathematische Annalen} \textbf{\bibinfo{volume}{158}},
  \bibinfo{pages}{158} (\bibinfo{year}{1965}).

\bibitem[{\citenamefont{Zierler and Schlessinger}(1965)}]{ZirlSchl-65}
\bibinfo{author}{\bibfnamefont{N.}~\bibnamefont{Zierler}} \bibnamefont{and}
  \bibinfo{author}{\bibfnamefont{M.}~\bibnamefont{Schlessinger}},
  \bibinfo{journal}{Duke Mathematical Journal} \textbf{\bibinfo{volume}{32}},
  \bibinfo{pages}{251} (\bibinfo{year}{1965}).

\bibitem[{\citenamefont{Calude et~al.}(1999)\citenamefont{Calude, Hertling, and
  Svozil}}]{CalHerSvo}
\bibinfo{author}{\bibfnamefont{C.}~\bibnamefont{Calude}},
  \bibinfo{author}{\bibfnamefont{P.}~\bibnamefont{Hertling}}, \bibnamefont{and}
  \bibinfo{author}{\bibfnamefont{K.}~\bibnamefont{Svozil}},
  \bibinfo{journal}{Foundations of Physics} \textbf{\bibinfo{volume}{29}},
  \bibinfo{pages}{349} (\bibinfo{year}{1999}).

\bibitem[{\citenamefont{Bell}(1964)}]{bell}
\bibinfo{author}{\bibfnamefont{J.~S.} \bibnamefont{Bell}},
  \bibinfo{journal}{Physics} \textbf{\bibinfo{volume}{1}}, \bibinfo{pages}{195}
  (\bibinfo{year}{1964}), \bibinfo{note}{reprinted in \cite[pp.
  403-408]{wheeler-Zurek:83} and in \cite[pp. 14-21]{bell-87}}.

\bibitem[{\citenamefont{Mermin}(1993)}]{mermin-93}
\bibinfo{author}{\bibfnamefont{N.~D.} \bibnamefont{Mermin}},
  \bibinfo{journal}{Reviews of Modern Physics} \textbf{\bibinfo{volume}{65}},
  \bibinfo{pages}{803} (\bibinfo{year}{1993}).

\bibitem[{\citenamefont{Kochen and Specker}(1967)}]{kochen1}
\bibinfo{author}{\bibfnamefont{S.}~\bibnamefont{Kochen}} \bibnamefont{and}
  \bibinfo{author}{\bibfnamefont{E.~P.} \bibnamefont{Specker}},
  \bibinfo{journal}{Journal of Mathematics and Mechanics}
  \textbf{\bibinfo{volume}{17}}, \bibinfo{pages}{59} (\bibinfo{year}{1967}),
  \bibinfo{note}{reprinted in \cite[pp. 235--263]{specker-ges}}.

\bibitem[{\citenamefont{Greenberger et~al.}(1990)\citenamefont{Greenberger,
  Horne, Shimony, and Zeilinger}}]{ghsz}
\bibinfo{author}{\bibfnamefont{D.~M.} \bibnamefont{Greenberger}},
  \bibinfo{author}{\bibfnamefont{M.~A.} \bibnamefont{Horne}},
  \bibinfo{author}{\bibfnamefont{A.}~\bibnamefont{Shimony}}, \bibnamefont{and}
  \bibinfo{author}{\bibfnamefont{A.}~\bibnamefont{Zeilinger}},
  \bibinfo{journal}{American Journal of Physics} \textbf{\bibinfo{volume}{58}},
  \bibinfo{pages}{1131} (\bibinfo{year}{1990}).

\bibitem[{\citenamefont{Peres}(1978)}]{peres222}
\bibinfo{author}{\bibfnamefont{A.}~\bibnamefont{Peres}},
  \bibinfo{journal}{American Journal of Physics} \textbf{\bibinfo{volume}{46}},
  \bibinfo{pages}{745} (\bibinfo{year}{1978}).

\bibitem[{\citenamefont{Svozil}(1993)}]{svozil-93}
\bibinfo{author}{\bibfnamefont{K.}~\bibnamefont{Svozil}},
  \emph{\bibinfo{title}{Randomness \& Undecidability in Physics}}
  (\bibinfo{publisher}{World Scientific}, \bibinfo{address}{Singapore},
  \bibinfo{year}{1993}).

\bibitem[{\citenamefont{Schaller and Svozil}(1996)}]{schaller-96}
\bibinfo{author}{\bibfnamefont{M.}~\bibnamefont{Schaller}} \bibnamefont{and}
  \bibinfo{author}{\bibfnamefont{K.}~\bibnamefont{Svozil}},
  \bibinfo{journal}{International Journal of Theoretical Physics}
  \textbf{\bibinfo{volume}{35}}, \bibinfo{pages}{911} (\bibinfo{year}{1996}).

\bibitem[{\citenamefont{Dvure{\v{c}}enskij
  et~al.}(1995)\citenamefont{Dvure{\v{c}}enskij, Pulmannov{\'{a}}, and
  Svozil}}]{dvur-pul-svo}
\bibinfo{author}{\bibfnamefont{A.}~\bibnamefont{Dvure{\v{c}}enskij}},
  \bibinfo{author}{\bibfnamefont{S.}~\bibnamefont{Pulmannov{\'{a}}}},
  \bibnamefont{and} \bibinfo{author}{\bibfnamefont{K.}~\bibnamefont{Svozil}},
  \bibinfo{journal}{Helvetica Physica Acta} \textbf{\bibinfo{volume}{68}},
  \bibinfo{pages}{407} (\bibinfo{year}{1995}).

\bibitem[{\citenamefont{Calude et~al.}(1997)\citenamefont{Calude, Calude,
  Svozil, and Yu}}]{cal-sv-yu}
\bibinfo{author}{\bibfnamefont{C.}~\bibnamefont{Calude}},
  \bibinfo{author}{\bibfnamefont{E.}~\bibnamefont{Calude}},
  \bibinfo{author}{\bibfnamefont{K.}~\bibnamefont{Svozil}}, \bibnamefont{and}
  \bibinfo{author}{\bibfnamefont{S.}~\bibnamefont{Yu}},
  \bibinfo{journal}{International Journal of Theoretical Physics}
  \textbf{\bibinfo{volume}{36}}, \bibinfo{pages}{1495} (\bibinfo{year}{1997}).

\bibitem[{\citenamefont{Svozil}(1998)}]{svozil-ql}
\bibinfo{author}{\bibfnamefont{K.}~\bibnamefont{Svozil}},
  \emph{\bibinfo{title}{Quantum Logic}} (\bibinfo{publisher}{Springer},
  \bibinfo{address}{Singapore}, \bibinfo{year}{1998}).

\bibitem[{\citenamefont{Wright}(1978)}]{wright:pent}
\bibinfo{author}{\bibfnamefont{R.}~\bibnamefont{Wright}}, in
  \emph{\bibinfo{booktitle}{Mathematical Foundations of Quantum Theory}},
  edited by \bibinfo{editor}{\bibfnamefont{A.~R.} \bibnamefont{Marlow}}
  (\bibinfo{publisher}{Academic Press}, \bibinfo{address}{New York},
  \bibinfo{year}{1978}), pp. \bibinfo{pages}{255--274}.

\bibitem[{\citenamefont{Wright}(1990)}]{wright}
\bibinfo{author}{\bibfnamefont{R.}~\bibnamefont{Wright}},
  \bibinfo{journal}{Foundations of Physics} \textbf{\bibinfo{volume}{20}},
  \bibinfo{pages}{881} (\bibinfo{year}{1990}).

\bibitem[{\citenamefont{Svozil}(2002)}]{svozil-2001-eua}
\bibinfo{author}{\bibfnamefont{K.}~\bibnamefont{Svozil}}
  (\bibinfo{year}{2002}), \bibinfo{note}{e-print {\tt arXiv:quant-ph/in
  preparation}}.

\bibitem[{\citenamefont{Murnaghan}(1962)}]{murnaghan}
\bibinfo{author}{\bibfnamefont{F.~D.} \bibnamefont{Murnaghan}},
  \emph{\bibinfo{title}{The Unitary and Rotation Groups}}
  (\bibinfo{publisher}{Spartan Books}, \bibinfo{address}{Washington},
  \bibinfo{year}{1962}).

\bibitem[{\citenamefont{Calude and C\^{a}mpeanu}(1996)}]{Cal-Cam-96}
\bibinfo{author}{\bibfnamefont{C.}~\bibnamefont{Calude}} \bibnamefont{and}
  \bibinfo{author}{\bibfnamefont{C.}~\bibnamefont{C\^{a}mpeanu}},
  \bibinfo{journal}{Complexity} \textbf{\bibinfo{volume}{5}},
  \bibinfo{pages}{47} (\bibinfo{year}{1996}).

\bibitem[{\citenamefont{Kleene}(1952)}]{kleene-52}
\bibinfo{author}{\bibfnamefont{S.~C.} \bibnamefont{Kleene}}
  (\bibinfo{publisher}{North-Holland}, \bibinfo{address}{Amsterdam},
  \bibinfo{year}{1952}).

\bibitem[{\citenamefont{Trimmer}(1980)}]{trimmer}
\bibinfo{author}{\bibfnamefont{J.~D.} \bibnamefont{Trimmer}},
  \bibinfo{journal}{Proc. Am. Phil. Soc.} \textbf{\bibinfo{volume}{124}},
  \bibinfo{pages}{323} (\bibinfo{year}{1980}), \bibinfo{note}{reprinted in
  \cite[pp. 152-167]{wheeler-Zurek:83}.}

\bibitem[{\citenamefont{Wheeler and Zurek}(1983)}]{wheeler-Zurek:83}
\bibinfo{author}{\bibfnamefont{J.~A.} \bibnamefont{Wheeler}} \bibnamefont{and}
  \bibinfo{author}{\bibfnamefont{W.~H.} \bibnamefont{Zurek}},
  \emph{\bibinfo{title}{Quantum Theory and Measurement}}
  (\bibinfo{publisher}{Princeton University Press},
  \bibinfo{address}{Princeton}, \bibinfo{year}{1983}).

\bibitem[{\citenamefont{Donath and Svozil}(2002{\natexlab{b}})}]{DonSvo01vjqi}
\bibinfo{author}{\bibfnamefont{N.}~\bibnamefont{Donath}} \bibnamefont{and}
  \bibinfo{author}{\bibfnamefont{K.}~\bibnamefont{Svozil}},
  \bibinfo{journal}{Virtual Journal of Quantum Information}
  \textbf{\bibinfo{volume}{2}} (\bibinfo{year}{2002}{\natexlab{b}}).

\bibitem[{\citenamefont{Bell}(1987)}]{bell-87}
\bibinfo{author}{\bibfnamefont{J.~S.} \bibnamefont{Bell}},
  \emph{\bibinfo{title}{Speakable and Unspeakable in Quantum Mechanics}}
  (\bibinfo{publisher}{Cambridge University Press},
  \bibinfo{address}{Cambridge}, \bibinfo{year}{1987}).

\bibitem[{\citenamefont{Specker}(1990)}]{specker-ges}
\bibinfo{author}{\bibfnamefont{E.}~\bibnamefont{Specker}},
  \emph{\bibinfo{title}{Selecta}} (\bibinfo{publisher}{Birkh{\"{a}}user
  Verlag}, \bibinfo{address}{Basel}, \bibinfo{year}{1990}).

\end{thebibliography}

%\bibliography{svozil}
%\bibliographystyle{apsrev}

%\end{thebibliography}
\end{document}
