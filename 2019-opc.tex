\documentclass[a4paper,12pt]{article}


\usepackage[ngerman,english]{babel}
\usepackage[utf8]{inputenc}
\usepackage{amsmath}
\usepackage{amsfonts}
\usepackage{ dsfont }
\setlength{\headheight}{15pt}

\usepackage[doublespacing]{setspace}

\usepackage[autostyle]{csquotes}
\usepackage{hyperref}
\usepackage{enumitem}

\usepackage[
top    = 3cm,
bottom = 3cm,
left   = 2.5cm,
right  = 2.5cm]{geometry}

\usepackage{xcolor}
\newcommand{\silvi}[1]{\textcolor{red}{#1 (Silvi)}}
\newcommand{\karli}[1]{\textcolor{blue}{#1 (Karli)}}

\usepackage{fancyhdr}
\pagestyle{fancy}
\fancyhf{}
\rhead{Silvia Jonas and Karl Svozil}
\lhead{\textbf{Order, Patterns, \karli{Symmetries}, Correlations}}
\rfoot{Page \thepage}

\usepackage{natbib}
  \bibliographystyle{chicago}


\title{Order, Patterns, \karli{Symmetries}, Correlations: When may empirical regularities legitimately cry out for meta-empirical explanation?}

\author{Silvia Jonas and Karl Svozil}



\date{\today}

\begin{document}
\maketitle

\renewcommand{\abstractname}{Abstract}

\begin{abstract}
\footnotesize
\singlespacing

The recognition of striking order, \karli{symmetries}, patterns, and correlations---in short, regularities---in the physical world plays a major role in the justification of hypotheses and the development of new theories both in the natural sciences and in philosophy. However, while scientists consider only strictly empirical hypotheses to explain such regularities, philosophers also explore meta-empirical hypotheses. One example is mathematical realism, which proposes the existence of abstract mathematical entities as an explanation for the applicability of mathematics in the sciences. Another example is theism, which offers the existence of a supernatural being as an explanation for the design-like appearance of the physical cosmos. \karli{Maybe motivated by pragmatism in favor of an open, less dogmatic stance}, many philosophers consider only the former, but not the latter kind of meta-empirical explanation an admissible (if defeasible) hypothesis. This may seem like a double-standard, given that neither the existence of mathematical entities nor the existence of a cosmic designer are empirically testable hypotheses. Nevertheless, there is a strong intuition that some meta-empirical explanations are more warranted than others. The goal of this paper is to sharpen this intuition into a clear criterion for the (in)admissibility of meta-empirical explanations for empirical facts. Drawing on recent debates about the indispensability of mathematics and teleological arguments for the existence of God, we argue that a meta-empirical explanation is admissible just in case the explanation refers to an entity that, though not itself causally efficacious, guarantees the instantiation of a causally efficacious entity that is an actual cause of the regularity.\footnote{Thanks to *** for helpful comments and discussions. Work on this project has been supported by *** .}
\end{abstract}

\section{The legitimacy of meta-empirical hypotheses}

The recognition of striking order, patterns, and correlations lies at the heart of the natural sciences and is a core element of scientific progress. It often receives its initial impetus through reasoning by analogy, an indispensable heuristic tool in the development of new theories that enables the extraction of general categories and rules from complex patterns, lends justification to new hypotheses by relating our past experience to new problems in order to demonstrate parallels with already accepted hypotheses, and provides a basis for the unification of theories about distinct phenomena.\footnote{See \cite{Bartha2010} for a thorough investigation of the formal characteristics of analogical reasoning. For an excellent discussion of the role of analogical reasoning in major breakthroughs in $18^{th}$ and $19^{th}$ century physics, see \cite{Steiner1989,Steiner1998}.}

When scientists observe unexplained regularities, they begin to investigate possible explanations. However, not all logically possible explanations count as legitimate objects of scientific study. For example, in order to explain the spiral-shaped arrangement of seeds in sunflower heads, a biologist would not consider the hypothesis that the seeds were arranged by beauty-loving angels. Rather, empirical scientists investigate only strictly empirical hypotheses.

This is different with philosophers. The investigation of striking order, patterns, and correlations also play a major role in philosophy, but philosophers don't restrict their attention to empirical hypotheses only. Rather, they also develop `meta-empirical' explanations, i.e. explanations that transcend the realm of observation and experimentation. For example, moral realists posit the existence of abstract, non-observable moral entities (such as reasons or values), which they sometimes even describe as having quasi-causal powers. Typically, meta-empirical hypotheses come into play when empirical methods offer no suitable way of investigating a striking order, pattern, or correlation (below we introduce four classical cases). Such meta-empirical hypotheses are then judged according to purely theoretical criteria (e.g. internal consistency, coherence with accepted philosophical views, explanatory power, etc.).

However, no matter how well they meet such theoretical criteria, not all meta-empirical hypotheses are considered prima facie equally legitimate. Some appear to enjoy more initial credibility than others. For example, no matter how much explanatory power the hypothesis of theism has, many philosophers do not consider the existence of God a legitimate explanation for striking empirical regularities. By contrast, many philosophers and mathematicians are perfectly happy to subscribe to some form of mathematical realism, i.e. the view that mathematical statements are about mind-independent mathematical truths and have objective, determinate truth-values, even if this evidently implies some kind of mathematical ontology. This seems like a double-standard. After all, given that the entities they posit all defy empirical confirmation, all meta-empirical hypotheses ought to be considered at least prima facie equally legitimate.

Nevertheless, there is a strong intuition that some meta-empirical explanations are prima facie more warranted than others. The way this intuition is usually substantiated is by reference to the particular features of the entities in question and the problems these features raise in a particular context. For example, most metaphysicians reject Lewisian modal realism, according to which there exist infinitely many concrete yet causally disconnected possible worlds. The reason they reject this view is of course not that the view failed empirical testing. Rather, besides its counter-intuitiveness and its vastly inflated ontology, modal realism raises a number of issues that are problematic within the greater context of the philosophy of modality, for example, Kaplan's paradox or the problem of island universes.

\karli{[[Das verstehe ich nicht; davon weiss ich nichts. Ev. erkl\"arst Du es mir?]]}

A problem with this way of deciding whether a meta-empirical posit is admissible or not is that it requires entering the particulars of a given debate in great depth. Taking modal realism for example again, one would have to be familiar with the structure of Kaplan's paradox; the reasons why the concept of island universes is inconsistent; the various positions in the debate about counterfactuals; and so on.

We aim to offer a way to explain why some meta-empirical hypotheses are considered prima facie legitimate (and others aren't) that \textit{abstracts} from the particulars of a specific philosophical debate or domain. The goal of this paper is thus to sharpen the intuition that some meta-empirical hypotheses are more warranted than others into a clear criterion for the (in)admissibility of meta-empirical explanations of empirical facts. We argue that a meta-empirical explanation is admissible just in case the explanation refers to an entity that, though not itself causally efficacious, guarantees the instantiation of a causally efficacious entity that is an actual cause of the regularity.

\section{Meta-empirical explanations of empirical regularities}

Here are four examples of meta-empirical hypotheses that have been offered in a variety of philosophical contexts in order to explain particular kinds of order, patterns, or correlations observable in the physical world. We have sorted the examples in descending order of intuitive `acceptability:'

\begin{description}

\item[Mathematical entities:]

Ancient Greek philosophers like Pythagoras, Plato, and Euclid, transformed mathematics,
hitherto nothing but a tool for the solution of practical problems,
into an abstract science whose clarity, precision and rigor soon became a standard for all other sciences.
However, this abstract conception of mathematics also raises profound philosophical questions.
For example, how can a purely abstract system be applied to the empirical world (and be incredibly successful at that)? As Wigner puts it: `the enormous usefulness of mathematics in the natural sciences is something bordering on the mysterious' \cite[p.~2]{Wigner1960}. One way to explain this \textit{correlation between our mathematical and empirical facts} is to endorse `mathematical Platonism,' a philosophical view at the core of which stands the \textit{meta-empirical hypothesis that there exist irreducibly abstract, mind-independent mathematical entities}.\footnote{The view has had many prominent adherents, including Plato himself, but also mathematicians like Gödel and Erdös, and contemporary philosophers like Quine, Putnam, and, most recently, \cite{Colyvan2001}.}

\karli{
Another mathematical entities related topic
which is increasingly popular among scientists working in the foundations of (quantum) physics
contemplates that, insofar ``the universe is numbers'' (and we are living in a simulation)
\cite{schroed:natgr,mg1968,zuse-69,zuse-70,fredkin,toffoli:79,margolus-billard,svozil-93,wolfram-2002,svozil-2005-cu,tegmark2007,calude2013theeinai,tegmark2014,Bostrom-sim},
it is stochastic on its most fundamental level
\cite{armstrong_1983,vanFraassen1989-VANLAS,calude-meyerstein,lawlses_rosen2010,calude2013theeinai,chaos_multiverse2017,Mueller-2017,Cabello-2018-BornRule,svozil-2018-was}.
This is in accord with {\it fin de si\'ecle} speculations \cite{Hiebert2000,Stoeltzner1999}
that all sciences are based on randomness and chance \cite{Exner-1908,schrodinger-1929,book:16081,born-26-1}.
Already Hume speculated that causation is merely an epistemic imagination; and that natural laws cannot be proven to be necessary, but rather emerge
from (spurious) correlations and conjunctions~\cite{Hume-Treatise} -- in modern systems theory terminology, natural law may be {\em emergent}.
}

\item[Moral entities and properties:]

Also, metaethicists have offered meta-empirical hypotheses to explain empirical regularities. For instance, robust moral realism, i.e. the view according to which there exist objective, mind-independent moral entities (e.g. reasons and values) and properties (e.g. rightness or wrongness), has been suggested as an explanation for the empirical fact that people tend to take moral matters seriously, or more specifically, that in the face of moral disagreement, they tend to reason and act in ways analogous to the ones we use to resolve disagreements about strictly empirical facts.\footnote{See, for example, \cite[p.~23f]{Enoch2011_Taking}.}

\item[Possible worlds and propositions:] As mentioned above, we also find meta-empirical explanations of empirical regularities in the area of metaphysics. The most notorious example here are probably possible worlds (and propositions, possible worlds' sidekicks, that are defined as sets of possible worlds). In order to explain the fact that our reasoning about what is and isn't possible is so extraordinarily useful across a wide range of different contexts, modal realists have proposed the hypothesis that the modal statements employed in such reasoning are true, namely, true of concrete physical worlds that are spatiotemporally and causally disconnected from our actual world, but in all other respects just like our actual world.\footnote{The classic account of modal realism is, of course, \cite{Lewis1986}.}

\item[God:] The oldest meta-empirical hypothesis is, of course, theism, which posits the existence of God. Different versions of this hypothesis have been offered for thousands of years as an explanation for various kinds of regularities in the empirical world. Such `arguments from design,' or `teleological arguments,' begin with a premise that points out some type of order observable in the physical cosmos, argue that this type of order would not exist had it not been intentionally created, and end with a conclusion that proclaims the (likely) existence of an intelligent being or `designer' who created the physical cosmos.\footnote{Comprehensive introductions to Design Arguments are \cite{Sedley2007} (focusing on discussions from Antiquity) and \cite{Jantzen2014} (covering arguments from Antiquity to Modernity).}
Despite the fact that the kinds of order or regularity that are relevant for arguments from design have been a topic of philosophical discussion for over two thousand years,\footnote{The versions that are currently considered to have most (intuitive) force are fine-tuning arguments. These arguments start with a premise according to which the universe is evidently fine-tuned for the existence of life, and that the probability of the laws of physics corresponding exactly to the conditions necessary to enable life is incredibly small \cite{x}.}, and despite the fact that many of the orders, patterns, or correlations once thought to cry out for meta-empirical explanation can now be explained in purely empirical terms\footnote{give examples}, and despite the fact that no formulation of the design argument has yet been found that is accepted by all participants in the debate,\footnote{Design Arguments are standardly classified as analogical arguments, though there have been efforts to give them a deductive or abductive (IBE) structure. See, for example, \cite{x}.} design arguments continue to persist, and even philosophers who would reject theistic conclusions of any sort admit that the intricate functional organization of our cosmos is striking.

\end{description}

Generally speaking, though, philosophers are much more comfortable positing the existence of mathematical entities or truths than they are with positing the existence of God. Is there any domain-independent fact, i.e. a fact independent of the particular objections occurring in each of the individual debates, that can account for our different intuitions regarding meta-empirical hypotheses?

In order to answer this question, we now turn to the current debate about mathematical realism, more specifically, to arguments from explanatory indispensability cashed out in terms of the concept of `program explanation.' The concept of program explanation illuminates how mathematics can play a genuinely explanatory role in our best scientific theories. Contrasting the case of mathematics with the case of theism, we argue that positing the existence of God as a meta-empirical explanation for cosmic fine-tuning is less warranted than positing the existence of mathematical entities as a meta-empirical explanation for the `unreasonable effectiveness' of mathematics in the empirical sciences because God does not play a genuinely explanatory role in the sense of `program explanation' in our best scientific theories of cosmic fine-tuning.

\karli{
There is another, more pragmatic argument for favoring mathematics over deities:
it can be expected and corroborated by historic contemplations
that the non-dogmatic, flexible systems of varying mathematical formalizations
-- even if they are inconsistent; such as Cantorian set theory \cite{hilbert-26}
-- allow for a relatively easy succession of scientific ``revolutions''
of ever-changing research programs \cite{lakatosch}
adapting to internal demands as well as to increasing empirical content.
At least from the times of G\"odel, Turing, Kleene, Tarski, and many others
the transitory character of mathematical formalisms is a constituent feature of modern metamathematical thought
\cite{smullyan-92,Smullyan1993-SMURTF,book:486992}.
Thereby, devoid of any theistic overhead and it's potentially harmful
dogmatic content, mathematics may not contribute towards an approximation of ``truth;'' but at least
renders better predictability and manipulability of physical phenomena.
In that unpretentious sense, mathematics can be perceived as an auxiliary,
transitory construction of
the human mind, serving as one pillar
of the ongoing search for capacities for ``programming nature'' by human desires.
}

\section{Mathematical explanations of empirical facts}

For some time now, debates about mathematical realism have focused on issues concerning how best to understand mathematical explanation. At the center of these discussions is the question whether, at least in some specific cases, mathematics plays an \textit{explanatorily} indispensable role in the scientific explanation of particular empirical phenomena. Mathematical realists argue that the scientific explanations of those phenomena contain mathematical elements whose role in the scientific explanation cannot be reduced to a mere representation of empirical regularities. Here are three examples:

\begin{description}

\item[Honeycombs:] Bees build their honeycombs out of hexagonal cells. This striking fact calls for explanation. Darwin argued that minimizing the amount of energy and wax used for the construction of honeycombs generates an evolutionary advantage, such that the bees who are most efficient in the use of energy and wax will be selected \cite{Darwin1859}. In 1999, Thomas Hales proved that `a hexagonal grid is the most efficient way to divide a Euclidean plane into regions of equal area with least total perimeter' \cite[p.~4]{Hales2001}.

\item[Cicadas:] Certain types of North American cicada, \textit{Magicicadas}, have prime-numbered life cycles. They emerge from the ground every 13 and 17 years. This striking fact calls for explanation. Due to a number of ecological constraints, for example the need to minimise intersection with periodic predators with different cycle periods, having a prime-numbered life-cycle is advantageous for the \textit{Magicicada} (\citealp{Baker2005,Baker2009}; \citealp[p.33]{Goles_etal2001}). The reason for this is the number-theoretic fact that prime numbers maximize their lowest common multiple relative to all lower numbers.\footnote{For proofs of the relevant lemmas, see \cite{Landau1958}.}

\item[Sunflower seeds:] Sunflowers arrange their seeds in a striking spiral pattern. The explanation for this is again a combination of evolutionary with mathematical facts. Fitting as many seeds as possible into the circular flowerhead constitutes an evolutionary advantage. Sunflowers grow their seeds at the centre of the flowerhead, with new seeds pushing older ones outwards. Whenever a new seed develops, it does so at some angle of rotation from the older one. Given this way of growing seeds, the optimal rotation angle, i.e. the rotation angle at which the highest amount of seeds can be fitted into the flowerhead, can be made mathematically precise: it is an irrational fraction of 360 degrees, roughly 137.5 degrees, which is equivalent to the complement of 360 $\varphi$ mod 360 (where $\varphi$ is the Golden Ratio; \citealp[p.~4]{Lyon2012}).

\end{description}


What distinguishes those examples from other scientific theories featuring mathematics is that the mathematical part of the explanation plays an explanatory role \textit{of its own}, i.e. it constitutes part of the explanation of the empirical phenomenon. Moreover, it plays this role \textit{indispensably}, i.e. there is no way of formulating the respective scientific hypotheses in a non-mathematical way.

It is easy to see how the fact that mathematics is explanatorily indispensable to the formulation of our best scientific theories supports mathematical realism: if we believe, following \cite{Quine1981} and \cite{Putnam1979}, that we ought to be ontologically committed to all and only those entities that figure ineliminably in our best scientific theories, then the fact that mathematical entities play this role in some---perhaps many---scientific theories commits us to the belief in the existence of mathematical entities, or at least, renders the belief in the existence of those entities pro tanto admissible.

Could it be possible for other meta-empirical posits, such as possible worlds, moral entities, or God, to play an analogously indispensable role in the explanation of empirical regularities, such that, once we recognize this role, we are committed to believe in the existence of those entities? In order to understand this question, we need to develop a more precise idea of what exactly makes the explanations in the above example mathematical, i.e. how exactly the mathematics featuring in the above examples explains the empirical facts in question.


\section{Explanations featuring mathematics}

The natural sciences are full of mathematics: it is an indispensable tool for the formulation of theories about the physical world.
In many of those theories, the mathematics involved plays a merely representational role, for example by representing quantities.

\karli{
In such instances, one could maintain that the mathematical entities involved are mere simulacr\ae, doubles
\cite{Arthaud} of nature, and tailor-made to follow the latter in as many aspects as possible.
As Hertz~\cite[Introduction]{hertz-94,hertz-94e} noted:
\blockquote{ ``We form for ourselves images [[chimera, the German original is {\em Scheinbild}]] or symbols of external objects;
and the form which we give them is such that the necessary
consequences of the images in thought are always the images of
the necessary consequents in nature of the things pictured.
$\ldots$~we do not
know, nor have we any means of knowing, whether our conceptions
of things are in conformity with [[the things]] in any other
than this one fundamental respect.''}
}

However, the mathematical elements involved in the examples above are themselves an integral part of the explanation:
Hale's theorem explains why bees build their honeycombs out of hexagonal cells; the number-theoretic properties of prime numbers explain the length of the life-cycles of \textit{Magicicadas}; and the irrational number equivalent to the Golden Ratio explains the elegant arrangement of sunflower seeds. How exactly is it possible for a purely mathematical fact, i.e. a fact that is causally inefficacious, to explain a strictly empirical regularity? In order to answer this question, it will be useful to look at two kinds of explanation of empirical regularities that involve mathematics without it playing an explanatory role.

\subsection{Empirical instantiations of mathematical truths}
It is a theorem of mathematics that the sum of the internal angles of any triangle in a Euclidean space is 180 degrees. This mathematical truth is \textit{instantiated} in every physically existing triangle (setting aside unavoidable discrepancies between mathematical ideals and corresponding physical approximations). However, it does not explain anything about why some particular physical triangle exists; such an explanation would involve some story about how a chain of physical events led to the creation of a physical triangle.

Similarly, Ramsey's Theorem (\citeyear{Ramsey1930}) tells us that, if the number of objects in a set is sufficiently large and each pair of objects has one of a number of relations, then there is always a subset containing a certain number of objects where each pair has the same relation. In other words,
there exists a certain degree of order in all sets above a certain size, regardless of their composition and the properties of its members. Any sufficiently large collection of numbers, footballs, people, or pebbles will thus exhibit orderly substructures. For example, $n$ pigeons sitting in $m<n$ holes result in at least one hole being filled with at least two pigeons. Given enough stars, one can always find a group that forms a particular pattern, for example, a line, a rectangle, or even a big dipper. And at a gathering of any six people, some three of them will either be mutually acquainted or complete strangers to each other \cite{Greenwood-Gleason-55,Bostwick-1959}.
As the collection of data grows, the number of such relational properties grows---regularities thus `emerge' in any set of (physical and abstract) entities. However, the mathematical theorem that tells us this does not explain anything about the occurrence of any \textit{particular} order, pattern, or correlation. Any explanation of an individual regularity would involve some story about how a chain of physical events caused the emergence of that particular regularity.


\subsection{Mathematical representations of empirical facts}

Probably the most famous physical formula is Einstein's mass-energy equivalence E=m$c^2$, which states that the energy \textit{E}
of any \karli{physical} entity with a mass is equivalent to its mass \textit{m} multiplied by the speed of light \textit{c} squared. Clearly, this formula is a mathematical equation. However, the variables and constants featuring in it represent physical quantities: an object's mass and energy as well as the speed with which light travels; the formula employs mathematical language for the sole purpose of \textit{representing physical relations} in a convenient and economical way. Hence, even though the formula clearly \textit{features} mathematics, the mathematical elements do not contribute to the explanation of the relations holding between the physical properties objects.

\karli{
If one demands {\em necessity} rather than {\em sufficiency} then one has to
talk about inevitability and {\em uniqueness} of the mathematical prediction or description.
The uniqueness of solutions of differential equations has been a vast area of research in the 19th century
\cite[Chapter~17]{svozil-2016-pu-book}.
Criteria for uniqueness have been identified by continuity considerations such as
the Lipschitz condition. They play an important role in the discussion of free will and Providence
(as in ``God of the gaps'' \cite{frank,franke}) as well as (in)determinism even to this day
\cite{Norton-2003-cafs,Norton-dome-2008}.
}

\subsection{Mathematical explanations of empirical facts}

So how exactly is it possible for a purely mathematical fact to explain an empirical regularity? What distinguishes the honeycomb, the \textit{Magicicada}, and the sunflower examples from mere instantiations of mathematical truths in the physical world on the one hand, and from explanations using mathematical language for purely representational purposes on the other? A theory first introduced by Jackson and Pettit carves out an important distinction between two ways in which properties can play an explanatory role in empirical theories:

\blockquote{It appears then that there are at least two distinct ways in which a property can be causally relevant: through being efficacious in the production of whatever is in question, or through programming for the presence of an efficacious property. \cite[p.~115]{JacksonPettit1990}}

This distinction can be illustrated using Putnam's classic peg-hole example \cite[pp.~295ff]{Putnam1975}. Imagine a wooden board with two holes, one circular with a diameter of one inch, the other square with a side-length of one inch. What explains the fact that a cubical peg with a side-length of 15/16ths of an inch on each side will fit through the square hole but not the round hole? In our answer to this question, we will most certainly refer to mathematical properties. For example, we might say that any peg with a side-length of 15/16ths of an inch is too large for any hole with a one-inch diameter.

But there is something odd about this. Strictly speaking, an explanation of what \textit{caused} the peg to bump into the board rather than pass through the round hole ought to refer to peg's as well as the board's micro-physical properties (its spatiotemporal coordinates, the forces acting on the bodies, their molecular structure, fundamental components, etc.). In other words, it ought to refer to causally efficacious properties.

However, there is a strong sense in which the peg's micro-physical properties provide only part of the explanation of the peg's failure to pass through the board. A full explanation would also mention the peg's and the board's geometrical properties respectively. Using Jackson's and Pettit's terminology, the microphysical properties of peg and board are causally relevant to the peg's failure to pass the board by being \textit{efficacious} in the production of the `bump;' the geometrical properties of peg and board are causally relevant through \textit{programming} for the presence of the relevant microphysical properties:

\blockquote{The analogy is with a computer program which ensures that certain things will happen---things satisfying certain descriptions---though all the work of producing those things goes on at a lower, mechanical level. \cite[p.~114]{JacksonPettit1990}}

Also, the honeycombs, the \textit{Magicadas}, and the sunflower seeds can be analyzed in this way: the explanation of the respective empirical regularities (hexagonal cell shapes, prime-numbered life-cycles, Golden Ratio rotation angles) involve a purely mathematical element as well as a causal element that works on the microphysical level.

\blockquote{Roughly, a process explanation is one that gives a detailed account of the actual causes that led to the event to be explained. A program explanation, on the other hand, is one that cites a property or entity that, although not causally efficacious, ensures the instantiation of a causally efficacious property or entity that is an actual cause of the explanandum. \cite[p.~565f]{Lyon2012}}

What distinguishes empirical \textit{instantiations} of mathematical truths as well as mathematical \textit{representations} of empirical facts from mathematical \textit{explanations} is thus the way in which the mathematical element of a hypothesis contributes to the explanation of the empirical phenomenon in question, i.e. by playing a \textit{programming} role. Through this role, it is possible for a purely mathematical fact to explain an empirical regularity. And if, at least sometimes, mathematics plays the programming role \textit{indispensably}, it is admissible---perhaps even necessary---to draw ontological conclusions from this fact, i.e. to posit the existence of mathematical entities.

\karli{At this point, a {\it caveat} or at least a warning seems in order:
As already Hume has pointed out, even indispensable necessities (uniqueness of consequences), as programmed as they may appear,
maybe a mere consequences of (spurious) correlations.
Hume contemplated miracles, a category largely abandoned in present discussions;
but the same reservations may apply also for apparently programmed entities in nature.
}

\section{Theistic explanations of empirical facts}

We now turn back to our earlier question: Could it be possible also for other meta-empirical posits, for example possible worlds, moral entities, or God, to play an analogously indispensable programming role in the explanation of empirical regularities, such that, once we recognize this role, it is prima facie legitimate for us to believe in the existence of those entities?

Let's consider the most controversial case, i.e. theistic explanations of empirical regularities as they feature in arguments from design. One of the earliest arguments from design can be found in Cicero's \textit{The Nature of the Gods}:

\blockquote{If the first sight of the universe happened to throw [philosophers] into confusion, once they observed its measured, steady movements, and noted that all its parts were governed by established order and unchangeable regularity, they ought to have realised that in this divine dwelling in the heavens was one who was not merely a resident but also a ruler, controller, and so to say the architect of this great structural project. (\citealp[II.90,~p.~79]{Cicero1998}; see also \citealp[p.~37]{Jantzen2014})}


Since Cicero's times, numerous variations of the argument from design have been suggested, but their main structure is always roughly like this:

\begin{enumerate}[noitemsep]
\item We perceive regularities (of some striking kind) in the physical world.
\item There is no plausible way of explaining these independent of deliberate intent.
\item Deliberate intent implies a designer.
\item The designer is God.
\end{enumerate}

Arguments with this structure can be attacked in different ways, depending on which premise is considered implausible. However, premise 2 is the one that has come under attack most frequently, viz. whenever scientists developed purely natural explanations for phenomena that, at one point, seemed to call for supernatural explanations.\footnote{Add examples.} And this makes sense, of course: The more empirically tractable allegedly design-like properties are, the less we accept them as actual marks of design and purpose. Despite the many successes of empirical science, however, attempts to account for striking regularities in theistic terms never completely vanished from the philosophical landscape. In fact, some versions of the design argument, most notably the argument from cosmic fine-tuning, have drawn a lot of attention lately and have developed the argument in great detail.\footnote{Examples}

We will now briefly outline the main structure of fine-tuning arguments. We will then apply the concept of `program' explanation in order to investigate whether the designer-hypothesis plays an indispensable programming-role in those arguments, such that recognition of this role makes it prima facie legitimate to posit the existence of God.

Fine-tuning arguments cash out premise 1, i.e. the observation that there are striking regularities in the empirical world, in terms of (a) the fine-tuning of the physical cosmos, more precisely, the fact that the universe appears to be functionally organized in a way that makes life possible, and (b) the improbability of the physical constants falling exactly into the range required for the development of life.

For example, if the cosmological constant $\Lambda$, the parameter representing the expansion rate of the universe, were only slightly smaller than it is, then the universe would have collapsed back onto itself shortly after the Big Bang. If $\Lambda$ were only slightly greater than it is, stars could not have developed. And since stars are the only known sources in the universe capable of producing large quantities of the elements on which all living organisms crucially depend---oxygen, carbon, hydrogen, etc.---life without stars would arguably not be possible. Taking into consideration all of the fine-tuning examples relevant to the formation of stars, the chance of stars existing in the universe has been estimated by theoretical physicists to be 1 in $10^{229}$.\footnote{See \citealp[p.~45]{Smolin1999}: `In my opinion, a probability this tiny is not something we can let go unexplained. Luck will certainly not do here; we need some rational explanation of how something this unlikely turned out to be the case.' Importantly, though, Smolin does not think that the required rational explanation will feature an intelligent creator \cite[p.~282]{Huberman2006}.} There are various other examples of physical constants being just as life needs them to be.\footnote{Add more examples} However, the point on which all fine-tuning examples converge is the strikingly low probability of the universe being exactly as life needs it to be.

Naturally, fine-tuning arguments have inspired a number of counter-arguments. Some have used the anthropic principle to argue that the existence of life in our universe is not at all surprising; others have argued that mathematical probability distributions are undefined over infinitely large space of possible outcomes (i.e. possible universes).\footnote{e.g. \cite{McGrew_etal2001}} It has also been argued that there is no reason to believe that science will not find a natural explanation for fine-tuning, just as it managed to find natural explanations for other striking regularities in the past.\footnote{e.g. \cite{Harnik_etal2006}} Finally, some have suggested that the universe is, in fact, a multiverse, consisting of vastly many or even infinitely many universes, which would increase the odds of there being one life-permitting universe.\footnote{e.g. \cite{Kraay2014}}

We will not enter these specific debates. Rather, we are interested in the question of whether or not there is an \textit{in principle} reason to be skeptical about the legitimacy of theistic explanations of empirical facts. To answer this question, we will now investigate whether God can be argued to play an indispensable programming role in the explanation of cosmic fine-tuning that is analogous to the role of mathematics in the explanation of other empirical regularities. If a case can be made that God does play such a role, then it seems at least prima facie legitimate to posit the existence of God as an explanation of cosmic fine-tuning. However, if such a case cannot be made, then this could be argued to constitute an in principle reason to be skeptical about theistic explanations of empirical regularities.

\section{Does God `program' cosmic fine-tuning?}

Recall the examples of mathematical explanations introduced above.
What distinguishes the honeycombs, the \textit{Magicicadas}
and the sunflower seeds are the genuinely explanatory contribution of the mathematics featuring in the explanation.
\karli{
Stated differently, it is the uniqueness, the necessity, the unavoidability of certain mathematical features which
suggest programming thereof.
}
For simplicity, let's stick to one of the examples, say, the honeycombs.

On the face of it, there are two elements to the explanation of the phenomenon of hexagonal honeycombs. The first element is purely causal: every cell in a honeycomb exists because one or several bees extracted wax from their abdomens, manipulated it with their antennae, mandibles, or legs, and finally built it. In other words, every honeycomb exists due to a causal chain of physical events leading up from a group of bees manipulating wax to the finished honeycomb. However, the purely causal explanation leaves a crucial question open, namely, the question of why the cells of the honeycomb have their striking hexagonal shape. In order to answer this question, the explanation needs to be supplemented.

The second element of the explanation, then, is purely mathematical: the most efficient way to divide a plane into regions of equal area with least total perimeter is by dividing it into regular hexagons. The geometrical properties of hexagons thus add the information needed to explain the striking shape of honeycomb shells, which is missing in the purely causal explanation.\footnote{In fact, these two elements neither suffice to make the explanation complete. Notice that the additional information the mathematical element provides is only relevant once we establish two additional connecting claims, namely that (a) the survival of species is contingent on their fitness and efficiency, and (b) survival is what all living organisms strive for. Only with this extra-pieces of information in place does the explanation become complete.}


Let's now look at the case of fine-tuning arguments for theism. The phenomenon calling for explanation is the precise attunement of physical constants, such that life becomes possible. For every `ordinary' physical event---a hurricane, a supernova, an atomic fission---there arguably exists a purely causal explanation involving a chain of prior physical events leading up to it.
The theory of the evolution of species by random mutation and natural selection can even account for much of the intricate functional organization of organisms on an individual as well as on a collective level. However, the physical phenomenon of, say, the cosmological constant $\Lambda$ having the value it has cannot be explained by a purely causal story; it is a brute fact. Again, the purely causal explanation leaves a crucial question open, namely, why $\Lambda$ has precisely the value necessary for the existence of life.\footnote{Is there even reason to believe that this question is in principle empirically intractable?}

In this case, the gap in the explanation cannot be filled mathematically; there is no theorem that can account for the value of $\Lambda$. Nevertheless, without an explanation of the value of $\Lambda$, our explanation of the apparent fine-tuning of the physical cosmos is incomplete.

Recall that in the case of the honeycombs, a mathematical theorem is put forth as a meta-empirical explanation in order to fill the gap in the purely causal explanation; the argument from the explanatory indispensability of mathematics to our best scientific theories is then argued to support mathematical realism.

At this point, two questions arise:


\begin{enumerate}
\item Is there a meta-empirical hypothesis that can explain the value of $\Lambda$ and thus complete our explanation of the apparent fine-tuning of the physical cosmos?

\item Could such a hypothesis be argued to be explanatorily indispensable to our best scientific theories of the apparent fine-tuning of the physical cosmos?
\end{enumerate}


The answer to the question is yes, of course. There are numerous possible meta-empirical hypotheses that could be offered as an explanation of the value of $\Lambda$ (angels, demons, aliens), but the two main competitors (at least currently) are that the value of $\Lambda$ was intended by a designer capable of bringing $\Lambda$ about, or that chance brought it about against all odds.

Let's now turn to question 2: Could one of the two hypotheses be argued to be explanatorily indispensable to our best theories of fine-tuning? Recall the distinction between process and program explanations introduced above. Process explanations operate on the purely causal level by giving exact accounts of the empirical causes that led up to a particular physical event. A program explanation, on the other hand, is an explanation that refers to properties or entities that are themselves not causally efficacious, but that ensures the instantiation of a causally efficacious property or entity that is an actual cause of the explanandum.

Which one of the two hypotheses, the designer-hypothesis, and the chance-hypothesis, could be argued to play a programming role in the required sense? It seems evident to us that only the designer-hypothesis could play this role, and thereby complete the explanation of cosmic fine-tuning.

A designer who intended $\Lambda$ to have the value it has and who is capable of bringing the value of $\Lambda$ about would ensure the instantiation of the precise value $\Lambda$ has. In this sense, the designer-hypothesis would guarantee the value of $\Lambda$.

Chance, on the other hand, could have brought about any possible value for $\Lambda$.
In particular, it could have brought about a whole range of values for $\Lambda$ that would have made life impossible.
In this sense, the chance-hypothesis does not guarantee the value of $\Lambda$ and thus, cannot be \karli{argued}
to play an explanatorily indispensable role to our best scientific theories of cosmic fine-tuning.

\karli{Alas}, as we demonstrated above using the case of mathematics,
a meta-empirical explanation of an empirical regularity is \karli{considered}
admissible \karli{only}
in case the explanation refers to an entity \karli{which},
though not itself causally efficacious, guarantees the instantiation of a causally efficacious property
that is an actual cause of the regularity.
As we have just argued, \karli{relative to our assumptions}
only the designer-hypothesis, but not the chance-hypothesis, meets this criterion.
Therefore, the designer-hypothesis, but not the chance-hypothesis,
is an admissible hypothesis for the explanation of cosmic fine-tuning.

\karli{
Nevertheless, within another contextual framework,
it could be circularly argued that (or capacity of) cognition is an emergent property
which only substantiates in ``proper'' universes with ``fitting'' values of $\Lambda$.
Therefore, only in those special universes, cognition happens --
and thereby any cognitive agent or entity necessarily ``finds itself to be  immersed
in a universe with fitting $\Lambda$.''
Stated pointedly, relative to the assumptions of a conceivable existence of such universes,
this circular argument ``explains'' why cognitive individuals may register merely ``proper'' universes with ``fitting'' parameters.
}


\newpage

\footnotesize
\singlespacing
\bibliography{silviasbib,svozil}



\end{document}

