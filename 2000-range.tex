
\documentclass{article}
%%%%%%%%%%%%%%%%%%%%%%%%%%%%%%%%%%%%%%%%%%%%%%%%%%%%%%%%%%%%%%%%%%%%%%%%%%%%%%%%%%%%%%%%%%%%%%%%%%%%%%%%%%%%%%%%%%%%%%%%%%%%
\usepackage{graphicx}
\usepackage{amsmath}

%TCIDATA{OutputFilter=LATEX.DLL}
%TCIDATA{Created=Sun Nov 12 10:32:31 2000}
%TCIDATA{LastRevised=Sun Nov 12 12:09:16 2000}
%TCIDATA{<META NAME="GraphicsSave" CONTENT="32">}
%TCIDATA{<META NAME="DocumentShell" CONTENT="General\Blank Document">}
%TCIDATA{Language=American English}
%TCIDATA{CSTFile=LaTeX article (bright).cst}

\newtheorem{theorem}{Theorem}
\newtheorem{acknowledgement}[theorem]{Acknowledgement}
\newtheorem{algorithm}[theorem]{Algorithm}
\newtheorem{axiom}[theorem]{Axiom}
\newtheorem{case}[theorem]{Case}
\newtheorem{claim}[theorem]{Claim}
\newtheorem{conclusion}[theorem]{Conclusion}
\newtheorem{condition}[theorem]{Condition}
\newtheorem{conjecture}[theorem]{Conjecture}
\newtheorem{corollary}[theorem]{Corollary}
\newtheorem{criterion}[theorem]{Criterion}
\newtheorem{definition}[theorem]{Definition}
\newtheorem{example}[theorem]{Example}
\newtheorem{exercise}[theorem]{Exercise}
\newtheorem{lemma}[theorem]{Lemma}
\newtheorem{notation}[theorem]{Notation}
\newtheorem{problem}[theorem]{Problem}
\newtheorem{proposition}[theorem]{Proposition}
\newtheorem{remark}[theorem]{Remark}
\newtheorem{solution}[theorem]{Solution}
\newtheorem{summary}[theorem]{Summary}
\newenvironment{proof}[1][Proof]{\textbf{#1.} }{\ \rule{0.5em}{0.5em}}
\input{tcilatex}

\begin{document}


\section{$\protect\bigskip $ The Range Problem}

\QTP{Body Math}
We are interested on the set of all vectors $%
p=(p_{1},p_{1},...,p_{n},p_{n+1},...,p_{2n},...,p_{ij},...)$ which have the
following representation: There exist a Hilbert space $H$, projections $%
E_{1},...,E_{n},F_{1},...F_{n}$ on $H$, and a statistical operator $W$ on $%
H\otimes H$, such that

\QTP{Body Math}
\begin{equation}
p_{i}=tr[W(E_{i}\otimes I)]\qquad \qquad p_{n+j}=tr[W(I\otimes F_{j})]\qquad
p_{ij}=tr[W(E_{i}\otimes F_{j})]\qquad i,j=1,2,...n  \tag{1}
\end{equation}

\QTP{Body Math}
Denote the set of all vectors with the representation (1) by $b(n).$

\QTP{Body Math}
The first thing to see is that $b(n)$ is convex. For that assume that  $%
p=(p_{1},p_{1},...,p_{n},p_{n+1},...,p_{2n},...,p_{ij},...)$ has the
representation in terms of $H$ projections $E_{1},...,E_{n},F_{1},...F_{n}$
and state $W$. Similarly assume that  $%
p^{,}=(p_{1}^{,},p_{1}^{,},...,p_{n}^{,},p_{n+1}^{,},...,p_{2n}^{,},...,p_{ij}^{,},...)
$ has a representation in terms of $H^{,},\
E_{1}^{,},...,E_{n}^{,},F_{1}^{,},...F_{n}^{,},\ W^{,}$.We have to show that
$\lambda p+(1-\lambda )p^{,}\in b(n)$ for $0\leq \lambda \leq 1$.

\QTP{Body Math}
we shall take the Hilbert space to be $H\oplus H^{,}$ (direct sum) the
projections to be $E_{i}\oplus E_{i}^{,}$ on the ''left'' and

\QTP{Body Math}
$F_{j}\oplus F_{j}^{,}$ on the ''right''. To complete the proof it is left
is to define the state on $\ (H\oplus H^{,})\otimes (H\oplus H^{,}).$ Assume
that

\QTP{Body Math}
\begin{equation}
W=\sum_{\mu \nu }\lambda _{\mu \nu }P_{\mu }\otimes Q_{\nu }\qquad \lambda
_{\mu \nu }\geq 0,\quad \sum ' \lambda _{\mu \nu }=1  \tag{2}
\end{equation}

\QTP{Body Math}
Where $P_{\mu },Q_{\nu }$ projections on $H$ with $P_{\mu }\otimes Q_{\nu
}\perp P_{\mu `}\otimes Q_{\nu `}$ for $(\mu ,\nu )\neq (\mu `,\nu `)$.
Similarly:

\QTP{Body Math}
\begin{equation*}
W^{,}=\sum_{\mu \nu }\lambda _{\mu \nu }^{,}P_{\mu }^{,}\otimes Q_{\nu
}^{,}\qquad \lambda _{\mu \nu }^{,}\geq 0,\quad \sum ' \lambda _{\mu \nu
}^{,}=1
\end{equation*}

\QTP{Body Math}
Now let the new statistical operator be

\QTP{Body Math}
\begin{equation}
\lambda \sum_{\mu \nu }\lambda _{\mu \nu }(P_{\mu }\oplus 0^{,})\otimes
(Q_{\nu }\oplus 0^{,})+(1-\lambda )\sum_{\mu \nu }\lambda _{\mu \nu
}^{,}(0\oplus P_{\mu }^{,})\otimes (0\oplus Q_{\nu }^{,})  \tag{3}
\end{equation}

\QTP{Body Math}
Where $0^{,}$ is just the null subspace of $H^{,}$, so that $P_{\mu }\oplus
0^{,}$ is the projection in $H\oplus H^{,}$ on $P_{\mu }(H)\oplus 0^{,}$,
etc. It is easy to see that this will do the job.

\QTP{Body Math}
The central question is whether $b(n)$ is a polytope. As a first iteration
consider the inequalities

\QTP{Body Math}
\begin{equation}
0\leq p_{ij}\leq p_{i},p_{j+n}\qquad p_{i}+p_{j+n}-p_{ij}\leq 1\ \qquad \
i,j=1,2,...,n  \tag{4}
\end{equation}

\QTP{Body Math}
Which are obviously satisfied by every $p\in b(n)$.

\QTP{Body Math}
QUESTION: What are the vertices of the polytope which is defined by
inequalities (4)?

\QTP{Body Math}
If we can (a) find the vertices, and (b) show that every vertex is an
element of $b(n)$, then by the convexity, we have proved that $b(n)$ is
given by (4).

\QTP{Body Math}
Perhaps we can start by the $2\times 2$ and $3\times 3$ case and proceed
from there?

\end{document}
