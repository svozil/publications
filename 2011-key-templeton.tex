\documentclass[%
 reprint,
 %superscriptaddress,
 %groupedaddress,
 %unsortedaddress,
 %runinaddress,
 %frontmatterverbose,
 %preprint,
 showpacs,
 showkeys,
 preprintnumbers,
 %nofootinbib,
 %nobibnotes,
 %bibnotes,
 amsmath,amssymb,
 aps,
 prl,
 % pra,
 % prb,
 % rmp,
 %prstab,
 %prstper,
  longbibliography,
 %floatfix,
 %lengthcheck,%
 ]{revtex4-1}

\usepackage[breaklinks=true,colorlinks=true,anchorcolor=blue,citecolor=blue,filecolor=blue,menucolor=blue,pagecolor=blue,urlcolor=blue,linkcolor=blue]{hyperref}
\usepackage{graphicx}% Include figure files
\usepackage{xcolor}

 \begin{document}

\title{Key researchable issues in physics that relate to randomness and miracles}


\author{Karl Svozil}
\affiliation{Institute of Theoretical Physics, Vienna
    University of Technology, Wiedner Hauptstra\ss e 8-10/136, A-1040
    Vienna, Austria}
\email{svozil@tuwien.ac.at} \homepage[]{http://tph.tuwien.ac.at/~svozil}


\date{\today}

\begin{abstract}
Physical randomness can be grouped into three categories~\cite{svozil-07-physical_unknowables}:
(I) extreme sensitivity to initial values in classical continuum theories;
(II) recursion theoretical undecidability applicable to physics (``by reduction'' to some recursion theoretic incompleteness);
(III) quantum randomness, which may be grouped into three subcategories:
(i) the random occurrence of single events (e.g., particle decays, scattering of single quanta, beamsplitters);
(ii) quantum complementarity; and
(iii) quantum value indefiniteness (through Kochen-Specker- and Boole-Bell-type theorems).
My subjective evaluations of the progressiveness of the research area will be given in terms of the common {\em Long Term Bond Ratings} from AAA to D.
% http://www.bondsonline.com/asp/research/bondratings.asp
\end{abstract}

\pacs{03.65.Ta, 03.65.Ud}
\keywords{randomness, undecidability, quantum  measurement theory, quantum contextuality, counterfactual observables}
%\preprint{CDMTCS preprint nr. 372/2009}
\maketitle




\section{Extreme sensitivity to initial values in classical continuum theories (CCC)}

The field of ``deterministic chaos'' or ``symbolic dynamics''has been extensively searched for years now.
I believe it is degenative in the sense of Lakatos~\cite{lakatosch}.


Already in 1873, Maxwell mentioned \cite[211-212]{Campbell-1882}:
\begin{quote}
{
When an infinitely small variation in the present state may bring about a finite difference in the state of the
system in a finite time, the condition of the system is said to be unstable.
It is manifest that the existence of unstable conditions renders impossible the prediction of future events, if our
knowledge of the present state is only approximate, and not accurate.
}
\end{quote}
Maxwell also discussed unstable states of high potential energy whose
spontaneous \citep{frank}  decay or change \cite[212]{Campbell-1882}
{``requires an expenditure of work, which in certain cases may be
infinitesimally small, and in general bears no definite proportion to the energy developed in consequence thereof.''}

In Poincar{\'e}'s words \cite[chapt.~4, sect.~2,  56--57]{poincare14},
\begin{quote}
{
If we would know the laws of nature and the state of the Universe precisely
for a certain time,
we would be able to predict with certainty
the state of the Universe for any later time.
But
$\ldots$
it can be the case that small differences in the initial values
produce great differences in the later phenomena;
a small error in the former may result in a large error in the latter.
The prediction becomes impossible and we have a ``random phenomenon.''}
\end{quote}


\section{Recursion theoretical undecidability applicable to physics (BBB+)}

This research area offers a plethora of formal results.
It might be judged ``progressive'' but eventually
might dry out and falter due to exhaustion.
Maybe also some (pre-)G\"odelian connections to quantum mechanics will be found.

\subsection{Undecidability of the general prediction problem}

For any deterministic system strong enough to support
universal computation,  the general forecast or prediction
problem is provable unsolvable.
This proposition will be argued by reduction to the halting problem, which is provable unsolvable.
A straightforward embedding of a universal computer
into a physical system results in the fact that,
owing to the reduction to the recursive undecidability of the halting problem,
certain future events cannot be predicted
and are thus provable indeterministic.

A related question is the upper bound of running time  --  or,
alternatively, recurrence time  --  of a program of length $n$ bits before
terminating  or, alternatively, recurring?
An answer to this question explains just how long we have to
wait for the most time-consuming program of length $n$ bits to
halt. That, of course, is a worst-case scenario.
We mention without proof \citep{chaitin-ACM,chaitin-bb}  that
this bound can be represented by the busy beaver function, and therefore grows faster than any computable
function of $n$.

\subsection{Undecidability of the general induction problem}

Induction, in physics, is the inference of general rules
dominating and generating physical behaviors from these behaviors alone.
For any deterministic system strong enough to support
universal computation, the general induction problem
is provable unsolvable.
Induction is thereby reduced to the unsolvability of
the rule inference problem \citep{go-67,blum75blum,angluin:83,ad-91,li:92}
of identifying a rule or law reproducing the behavior of a deterministic system
by observing its input-output performance by purely algorithmic means
(not by intuition).

\subsection{Impossibility}

Physical tasks which would result in paradoxical
behavior \citep{hilbert-26} are impossible to perform.
One such task is the solution of the general halting problem, as discussed earlier.
Thus omnipotence appears infeasible, at least as long as one sticks to the usual
formal rules opposing inconsistencies \cite[163]{hilbert-26}.

Another such paradoxical task (requiring substitution and self-reference) can be forced upon  {\it La Bocca della Verit\'a} (Mouth of Truth),
located in the {\it portico} of the church of {\it Santa Maria in Cosmedin} in Rome.
It is believed that if one tells a lie with one's hand in the mouth of the sculpture,
the hand will be bitten off; another less violent legend has it that anyone sticking a hand in the mouth while
uttering a false statement will never be able to pull the hand back out.
\citet[178]{rucker} once allegedly put in his hand in the sculpture's mouth uttering, {``I will not be able to pull my hand back out.''}
The author leaves it to the reader to imagine {\it La Bocca della Verit\'a}'s confusion when confronted with such as statement!

There is a {\em pandemonium} of conceivable physical tasks \citep{barrow-impossibilities},
some quite entertaining \citep{smullyan-78}, which would result in paradoxical behavior and
are thus impossible to perform.
Some of these tasks are pre-G\"odelian and merely require {\em substitution}.

\subsection{Results in classical recursion theory with implications for theoretical physics}


The following theorems of recursive  analysis \citep{aberth-80,Weihrauch} have some
implications for theoretical physics \citep{kreisel}:
(1)
There exist recursive monotone bounded sequences of rational numbers
whose limit is no computable number
\citep{Specker49}.
A concrete example of such a number is Chaitin's Omega number \citep{chaitin3,calude:02,calude-dinneen06},
the halting probability for a computer (using prefix-free code),
which can be defined by a sequence of rational numbers
with no computable rate of convergence.
(2)
There exist a recursive real function which has its maximum in the unit interval
at no recursive real number \citep{Specker57}.
This has implications for the principle of least action.
(3)
There exists a real number $r$ such that $G(r) = 0$ is recursively undecidable for $G(x)$
in a class of functions which involves polynomials and the sine function
\citep{wang}.
This, again, has some bearing on  the principle of least action.
(4)
There exist incomputable solutions of the wave equations for computable initial values
\citep{pr1,bridges1}.
(5)
On the basis of theorems of recursive analysis \citep{Scarpellini-63,richardson68},
many questions in dynamical systems theory are provable undecidable \citep{1985cfd..book.....F,dc-d93,Stewart-91,calude:037103}.
%\end{description}


\section{Quantum randomness (A+)}

I believe that our perception of the quantum phenomena will change dramatically, but I cannot predict when and where.

\subsection{Random occurrence of single events}

In 1926, \citet[866]{born-26-1} [see an English translation in \citet[54]{wheeler-Zurek:83}] postulated that
\begin{quote}
{``from the standpoint of our quantum mechanics, there is no quantity
which in any individual case causally fixes the consequence of the collision;
but also experimentally we have so far no reason to believe that there are some inner properties of the atom
which condition a definite outcome for the collision.
Ought we to hope later to discover such properties $\ldots$  and determine them in individual cases?
Or ought we to  believe that the agreement of theory and experiment  --  as to the impossibility
of prescribing conditions? I myself am inclined  to give up determinism in the world of atoms.''
}
\end{quote}


Furthermore, Born suggested that, though {\em individual particles behave irreducibly indeterministic},
the {\em quantum state evolves deterministically}  in a strictly Laplacian causal way.
Indeed, between (supposedly irreversible) measurements the (unitary) quantum state evolution
is even reversible, that is, one-to-one, and amounts to a  generalized (distance preserving) rotation in complex Hilbert space.
In Born's \citeyearpar[804]{born-26-2} [see an English translation in \citet[302]{jammer:89}] own words,
\begin{quote}
{the motion of particles conforms to the laws of probability, but the probability itself
is propagated in accordance with the law of causality.
[This means that knowledge of a state in all points in a given time determines the distribution of
the state at all later times.]
}
\end{quote}

At present, indeterminism is clearly favored, the canonical position being expressed by \citet{zeil-05_nature_ofQuantum}:
{``The discovery that individual events are
irreducibly random is probably one of the
most significant findings of the twentieth
century. $\ldots$~For the individual event in quantum physics,
not only do we not know the cause, there is no cause.''}

\subsection{Quantum conundrum (AAA)}

This distinction between a reversible, deterministic evolution of the quantum state, on one hand,
and the irreversible measurement, on the other hand, has left some physicists with an uneasy feeling;
in particular, because of the possibility to erase
\citep{PhysRevD.22.879,PhysRevA.25.2208,greenberger2,Nature351,Zajonc-91,PhysRevA.45.7729,PhysRevLett.73.1223,PhysRevLett.75.3783,hkwz}
measurements by reconstructing the quantum state,
accompanied by a complete loss of the information obtained from the quantum state before the (undone) measurement --
unlike in classical reversible computation \citep{bennett-73,bennett-82,maxwell-demon},
which still allows copying, that is, one-to-many operations, the quantum state evolution is strictly one-to-one.
Indeed, the possibility to undo measurements on quantum states appears to be  not bound by any fundamental principle,
and limited merely by the experimenter's  technological capacities.
Stated pointedly, it would in principle be possible to undo all measurements,
yet this cannot be accomplished  most of the time (for almost all measurements)
for all practical purposes~\cite{bell:a1}.
But then, one could speculate, Born's statement seems to suggest that the deterministic state evolution uniformly prevails.
Pointedly stated, if, at least in principle, there is no such thing as an irreversible measurement,
and the quantum state evolves uniformly deterministically, why should there exist indeterministic individual events?
In this view, the insistence in irreversible measurements as well as in an irreducible indeterminism associated with individual quantum events
appears to be an idealistic, subjective illusion -- in fact, this kind of indeterminism depends
on measurement irreversibility and decays into thin air if the latter is denied.

Alas, the deterministic evolution of the quantum state could result in the
{\em superposition} of classically contradictory states.
One of the mind-boggling, perplexing and counterintuitive consequences associated with this coexistence of classical contradictions is
Schr\"odinger's \citeyearpar[812]{schrodinger} cat paradox
implying the simultaneous coexistence of death and life of a macroscopic object such as a mammal.
Another one is Everett's \citeyearpar{everett} aforementioned many-worlds interpretation suggesting that
our universe perpetually branches
off into zillions of consistent alternatives.

Thus one is faced with a {\em dilemma:}
either to accept a somehow spurious nonuniformity in the evolution of the quantum state
during (irreversible) measurement processes -- an {\it ad hoc} assumption challenged by quantum erasure experiments --
or being confronted with the counterintuitive decay of quantum states into superpositions of
classically mutually exclusive states -- a sort of jelly -- not backed by our everday
experience as conscious beings (although often ambivalent we usually dont reside in mental ambiguity for too long).
\citet[19--20]{schroedinger-interpretation} sharply addressed the difficulties of a quantum theorist
coping with this aspect of the quantum formalism:
\begin{quote}
{
The idea that  [the alternate measurement outcomes] be not alternatives but {\em all} really happening simultaneously
seems lunatic to [the quantum theorist], just {\em impossible.}
He thinks that if the laws of nature took {\em this} form for,
let me say,
a quarter of an hour, we should find our surroundings rapidly turning into a quagmire, a sort of a featureless jelly or plasma,
all contours becoming blurred, we ourselves probably becoming jelly fish.
It is strange that he should believe this.
For I understand he grants that unobserved nature does behave this way -- namely according to the wave equation.
$\ldots$ according to the quantum theorist, nature is prevented from rapid
jellification only by our perceiving or observing it.
}
\end{quote}


\subsection{Quantum complementarity}

{\em Complementarity} is the impossibility of measuring
two or more complementary observables
with arbitrary precision simultaneously.
In 1933, \citet[7]{pauli:58} gave the first explicit definition of {\em complementarity} stating that
[see the partial English translation in \cite[369]{jammer:89}]
\begin{quote}
{
in the case of  an indeterminacy of a property of a system at a certain configuration
(at a certain state of a system), any attempt to measure the respective property (at least partially)
annihilates the influence of the previous knowledge of the system on the (possibly statistical) propositions
about possible later measurement results.
$\ldots$
The impact
on the system by the  measurement apparatus for momentum (position) is such that
within the limits of the uncertainty relations
the value of the knowledge of the previous position (momentum) for the
prediction of later measurements of position and momentum is lost.
}
\end{quote}

Einstein, Podolsky, and Rosen \citeyearpar{epr} challenged quantum complementarity (and doubted the completeness of quantum theory)
by utilizing a configuration
of two entangled \citep{schrodinger,CambridgeJournals:1737068,CambridgeJournals:2027212} particles.
They
claimed to be able to empirically infer two different complementary contexts counterfactually simultaneously,
thus circumventing quantum complementarity.

\subsection{Quantum value indefiniteness}

Still another quantum unknowable results from the fact that no global
(in the sense of all or at least certain finite sets of  complementary observables)
classical truth
assignment exists which is consistent with even a finite number of local (in the sense of comeasurable) ones,
that is, no consistent classical truth table can be given by pasting together the possible outcomes of measurements of certain complementary observables.
This phenomenon is also known as {\em value indefiniteness} or, by an option to interpret this result, {\em contextuality}  (see later).
Here the term {\em local} refers to a particular context \citep{svozil-2008-ql}
that, operationally, should be thought of as the collection of all comeasurable or copreparable \citep{zeil-99} observables.
The structure of quantum propositions \citep{birkhoff-36,kochen3,kalmbach-83,kalmbach-86,pulmannova-91,nav:91,svozil-ql}
can be obtained by pasting contexts together.

As by definition, only {\em one} such context is directly measurable,
arguments based on more than one context must necessarily involve counterfactuals \citep{svozil-2006-omni,vaidman:2009}.
A {\em counterfactual} is a would-be-observable or
{\em contrary-to-fact conditional}
\citep{chisholm-46}
which has not been measured but potentially could have been measured
if an observer would have decided to do so; alas the observer decided to measure a different, presumably complementary, observable.

Already scholastic philosophy,
for instance, Thomas Aquinas,
considered similar questions such as whether God has knowledge of
non-existing  things \cite[part one, question 14, article 9]{Aquinas} or things
that are not yet \cite[part one, question 14, article 13]{Aquinas};
see also Specker's \citeyearpar[243]{specker-60}  reference to {\it infuturabilities}.
Classical omniscience, at least its naive expression that,
if a proposition is true, then an omniscient agent (such as God) knows that it is true,
is plagued by controversies and paradoxes.


\section{Miracles due to gaps in causal description (BB)}

This topic is related to all others discussed so far. It is highly speculative and offers also
potential scenarios for a dualistic mind-body link.

A different issue, discussed by \citet{frank},
is the possible occurrence of miracles in the presence of {\em gaps} of physical determinism.
Already Maxwell has considered {\em singular points} \cite[212--213]{Campbell-1882}, {``where prediction,
except from absolutely perfect data, and guided by the omniscience of contingency, becomes impossible.''}
One might perceive individual events occurring
outside the validity of classical and quantum physics without any apparent cause as miracles.
For if there is no cause to an event,
why should such an event occur altogether rather than not occur?

Although such thoughts remain highly speculative, miracles
could be the basis for an operator-directed evolution in otherwise deterministic physical systems.
Similar models have  been applied to dualistic models of the mind \citep{popper-eccles,Eccles22051986,eccles:papal}.
The objection that this scenario is unnecessarily complicating an otherwise monistic model
should be carefully reevaluated in view of computer-generated {\em virtual realities} \citep{descartes-meditation,putnam:81,svozil-nat-acad}.
In such algorithmic universes, there are computable evolution laws as well as inputs from interfaces.
From the intrinsic perspective \citep{svozil-94}, the inputs cannot be causally accounted for,
and hence they remain irreducibly transcendental with respect to the otherwise algorithmic universe.

\section{Harnessing unknowables and indeterminism}

Physical indeterminism need not necessarily be perceived negatively as the absence of causal laws
but rather as a {\em valuable resource.}
Indeed, ingenious quasi-programs to compute the {\em halting probability} \citep{chaitin3,calude-dinneen06,rtx100200236p}
through summation of series without any computable rate of convergence could,
at least in principle, and in the limit of unbounded computational resources,
be interpreted as generating  provable random sequences.
However, as has already been expressed by  \citet[768]{von-neumann1},
{``anyone who considers arithmetical methods of producing random digits is, of course, in a state of sin.''}


Besides recursion-theoretic undecidability,
there appear to be at least two principal sources of indeterminism and randomness in physics:
(1) one scenario is associated with instabilities of classical physical systems
and with a strong dependence of future behaviors on the initial value, and
(2) quantum indeterminism, which can be subdivided into three subcategories, including  random outcomes of individual events,
 complementarity, and
value indefiniteness.

The production of random numbers by  physical generators has a long history \citep{rand-55}.
The similarities and differences between classical and quantum randomness can be conceptualized
in terms of two  black boxes: the first  of them,  called the {\em ``Poincar{\'e} box,''}
containing a classical, deterministic, chaotic source of randomness and
the second,  called the {\em ``Born box,''}
containing a quantum source of randomness.

A Poincar{\'e} box could be realized by operating a classical dynamical system in the shift map region.
Major principles for  Born boxes utilizing beam
splitters or parametric down conversion
include the following:
(1) there should be at least three mutually exclusive outcomes to ensure value indefiniteness
\citep{PhysRevLett.85.3313,2008-cal-svo,svozil-2009-howto,1367-2630-12-1-013019,10.1038/nature09008};
(2) the states prepared and measured should be pure and in mutually  [possibly interlinked \citep{svozil:040102}]
unbiased bases or contexts; and
(3) events should be independent  to be able to apply proper normalization procedures \citep{von-neumann1,Samuelson-1968}.


\section{Metaphysical status of (in)determinism}

Hilbert's \citeyearpar{hilbert-1900e} sixth problem is about the axiomatization of physics.
Regardless of whether this goal is achievable,
omniscience cannot be gained
via the formalized, syntactic route,
which will remain blocked forever by the paradoxical self-reference
to which intrinsic observers and operational methods are bound.
Even if the universe were a computer \citep{zuse-70,fredkin,wolfram-2002,svozil-2005-cu},
we would intrinsically experience unpredictability and complementarity.


With regard to conjectures about the (in)deterministic evolution of physical events,
the situation is unsettled and can be expected to remain unsettled forever.
The reason for this is the provable impossibility to formally prove (in)determinism:
it is not possible to ensure that physical behaviors are causal and will remain so forever,
nor is it possible to exclude all causal behaviors.

The opportunistic approach that (as historically,  many ingenious scientists have failed to come up with a causal description)
indeterminism will prevail appears to be anecdotal, at best, and  misleading, at worst.
Likewise, the advice of authoritative researchers to
avoid asking questions related
to completing a theory,
or to avoid thinking about the meaning of quantum mechanics or any  kind of rational interpretation,
and to avoid searching for causal laws for phenomena which are, at the same time,
postulated to occur indeterministically by the same authorities --  even wisely and benevolently posted  --
hardly qualify as proof.

Any kind of lawlessness can thus be claimed only
{\em with reference to,} and {\em relative to,} certain criteria, laws, or quantitative statistical or algorithmic tests.
For instance, randomness could be established merely {\em with respect to} certain tests,
such as some batteries of tests of randomness, for instance, {\em diehard} \citep{diehard}, {\it NIST} \citep{Rukhin-nist},  {\it TestU01} \citep{1268777},
or algorithmic \citep{calude-dinneen05,PhysRevA.82.022102} tests.
Note, however, that even the decimal expansion of $\pi$, the ratio between the circumference and the diameter of an ideal circle \citep{bailey97,bailey05},
behaves reasonably random \citep{PhysRevA.82.022102};
$\pi$ might even be a good source of randomness for many Monte Carlo calculations.


Thus, both from a  formal as well as from an operational point of view,
any rational investigation into, or claim of, absolute (in)determinism is metaphysical and can only be proved
{\em relative to} a limited number of statistical or algorithmic tests
which some specialists happen to choose;
with very limited validity for the formal and the natural sciences.






\bibliography{svozil}

\end{document}

