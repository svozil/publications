\documentclass{article}
\usepackage{amsmath}
\usepackage{amsfonts}
\usepackage{geometry}
\geometry{a4paper, margin=1in}
\begin{document}

\section*{Joint Operators and Eigensystems for CHSH Inequality}

This document provides the joint measurement operators and their eigensystems for the optimal angles achieving the maximal CHSH violation of \( 2\sqrt{2} \). The operators are defined for Alice's measurements at \( \theta_A = 0^\circ \) (\( \sigma_z \)) and \( 90^\circ \) (\( \sigma_x \)), and Bob's measurements at \( \theta_B = 45^\circ \) and \( -45^\circ \).

\subsection*{Single-Qubit Operators}
\begin{itemize}
    \item \( A_1 = \sigma_z = \begin{pmatrix} 1 & 0 \\ 0 & -1 \end{pmatrix} \).
    \item \( A_2 = \sigma_x = \begin{pmatrix} 0 & 1 \\ 1 & 0 \end{pmatrix} \).
    \item \( B_1 = \frac{1}{\sqrt{2}} (\sigma_z + \sigma_x) = \begin{pmatrix} \frac{1}{\sqrt{2}} & \frac{1}{\sqrt{2}} \\ \frac{1}{\sqrt{2}} & -\frac{1}{\sqrt{2}} \end{pmatrix} \).
    \item \( B_2 = \frac{1}{\sqrt{2}} (\sigma_z - \sigma_x) = \begin{pmatrix} \frac{1}{\sqrt{2}} & -\frac{1}{\sqrt{2}} \\ -\frac{1}{\sqrt{2}} & -\frac{1}{\sqrt{2}} \end{pmatrix} \).
\end{itemize}

\subsection*{Joint Operators}
The joint operators are tensor products acting on the two-qubit Hilbert space (basis \( |00\rangle, |01\rangle, |10\rangle, |11\rangle \)).

\begin{itemize}
    \item \( A_1 \otimes B_1 \):
    \[
    \begin{pmatrix}
    \frac{1}{\sqrt{2}} & \frac{1}{\sqrt{2}} & 0 & 0 \\
    \frac{1}{\sqrt{2}} & -\frac{1}{\sqrt{2}} & 0 & 0 \\
    0 & 0 & -\frac{1}{\sqrt{2}} & -\frac{1}{\sqrt{2}} \\
    0 & 0 & -\frac{1}{\sqrt{2}} & \frac{1}{\sqrt{2}}
    \end{pmatrix}
    \]
    \item \( A_1 \otimes B_2 \):
    \[
    \begin{pmatrix}
    \frac{1}{\sqrt{2}} & -\frac{1}{\sqrt{2}} & 0 & 0 \\
    -\frac{1}{\sqrt{2}} & -\frac{1}{\sqrt{2}} & 0 & 0 \\
    0 & 0 & -\frac{1}{\sqrt{2}} & \frac{1}{\sqrt{2}} \\
    0 & 0 & \frac{1}{\sqrt{2}} & \frac{1}{\sqrt{2}}
    \end{pmatrix}
    \]
    \item \( A_2 \otimes B_1 \):
    \[
    \begin{pmatrix}
    0 & 0 & \frac{1}{\sqrt{2}} & \frac{1}{\sqrt{2}} \\
    0 & 0 & \frac{1}{\sqrt{2}} & -\frac{1}{\sqrt{2}} \\
    \frac{1}{\sqrt{2}} & \frac{1}{\sqrt{2}} & 0 & 0 \\
    \frac{1}{\sqrt{2}} & -\frac{1}{\sqrt{2}} & 0 & 0
    \end{pmatrix}
    \]
    \item \( A_2 \otimes B_2 \):
    \[
    \begin{pmatrix}
    0 & 0 & \frac{1}{\sqrt{2}} & -\frac{1}{\sqrt{2}} \\
    0 & 0 & -\frac{1}{\sqrt{2}} & -\frac{1}{\sqrt{2}} \\
    \frac{1}{\sqrt{2}} & -\frac{1}{\sqrt{2}} & 0 & 0 \\
    -\frac{1}{\sqrt{2}} & -\frac{1}{\sqrt{2}} & 0 & 0
    \end{pmatrix}
    \]
\end{itemize}

\subsection*{Eigensystems}
Each joint operator has eigenvalues \( +1 \) and \( -1 \), each with multiplicity 2. The eigenvectors are constructed from the single-qubit eigenvectors:
\begin{itemize}
    \item \( \sigma_z \): \( \lambda = +1 \): \( |0\rangle = \begin{pmatrix} 1 \\ 0 \end{pmatrix} \); \( \lambda = -1 \): \( |1\rangle = \begin{pmatrix} 0 \\ 1 \end{pmatrix} \).
    \item \( \sigma_x \): \( \lambda = +1 \): \( |+\rangle = \frac{1}{\sqrt{2}} \begin{pmatrix} 1 \\ 1 \end{pmatrix} \); \( \lambda = -1 \): \( |-\rangle = \frac{1}{\sqrt{2}} \begin{pmatrix} 1 \\ -1 \end{pmatrix} \).
    \item \( B_1 \): \( \lambda = +1 \): \( |v_1\rangle = \frac{1}{\sqrt{4-2\sqrt{2}}} \begin{pmatrix} 1 \\ \sqrt{2}-1 \end{pmatrix} \); \( \lambda = -1 \): \( |v_2\rangle = \frac{1}{\sqrt{4+2\sqrt{2}}} \begin{pmatrix} 1 \\ -1-\sqrt{2} \end{pmatrix} \).
    \item \( B_2 \): \( \lambda = +1 \): \( |w_1\rangle = \frac{1}{\sqrt{4-2\sqrt{2}}} \begin{pmatrix} 1 \\ 1-\sqrt{2} \end{pmatrix} \); \( \lambda = -1 \): \( |w_2\rangle = \frac{1}{\sqrt{4+2\sqrt{2}}} \begin{pmatrix} 1 \\ 1+\sqrt{2} \end{pmatrix} \).
\end{itemize}

\begin{itemize}
    \item \( A_1 \otimes B_1 \):
    \begin{itemize}
        \item Eigenvalue \( +1 \):
        \[
        \frac{1}{\sqrt{4-2\sqrt{2}}} \begin{pmatrix} 1 \\ \sqrt{2}-1 \\ 0 \\ 0 \end{pmatrix}, \quad \frac{1}{\sqrt{4+2\sqrt{2}}} \begin{pmatrix} 0 \\ 0 \\ 1 \\ -1-\sqrt{2} \end{pmatrix}.
        \]
        \item Eigenvalue \( -1 \):
        \[
        \frac{1}{\sqrt??? \sqrt{4+2\sqrt{2}}} \begin{pmatrix} 1 \\ -1-\sqrt{2} \\ 0 \\ 0 \end{pmatrix}, \quad \frac{1}{\sqrt{4-2\sqrt{2}}} \begin{pmatrix} 0 \\ 0 \\ 1 \\ \sqrt{2}-1 \end{pmatrix}.
        \]
    \end{itemize}
    \item \( A_1 \otimes B_2 \):
    \begin{itemize}
        \item Eigenvalue \( +1 \):
        \[
        \frac{1}{\sqrt{4-2\sqrt{2}}} \begin{pmatrix} 1 \\ 1-\sqrt{2} \\ 0 \\ 0 \end{pmatrix}, \quad \frac{1}{\sqrt{4+2\sqrt{2}}} \begin{pmatrix} 0 \\ 0 \\ 1 \\ 1+\sqrt{2} \end{pmatrix}.
        \]
        \item Eigenvalue \( -1 \):
        \[
        \frac{1}{\sqrt{4+2\sqrt{2}}} \begin{pmatrix} 1 \\ 1+\sqrt{2} \\ 0 \\ 0 \end{pmatrix}, \quad \frac{1}{\sqrt{4-2\sqrt{2}}} \begin{pmatrix} 0 \\ 0 \\ 1 \\ 1-\sqrt{2} \end{pmatrix}.
        \]
    \end{itemize}
    \item \( A_2 \otimes B_1 \):
    \begin{itemize}
        \item Eigenvalue \( +1 \):
        \[
        \frac{1}{\sqrt{8-4\sqrt{2}}} \begin{pmatrix} 1 \\ \sqrt{2}-1 \\ 1 \\ \sqrt{2}-1 \end{pmatrix}, \quad \frac{1}{\sqrt{8+4\sqrt{2}}} \begin{pmatrix} 1 \\ -1-\sqrt{2} \\ -1 \\ 1+\sqrt{2} \end{pmatrix}.
        \]
        \item Eigenvalue \( -1 \):
        \[
        \frac{1}{\sqrt{8+4\sqrt{2}}} \begin{pmatrix} 1 \\ -1-\sqrt{2} \\ 1 \\ -1-\sqrt{2} \end{pmatrix}, \quad \frac{1}{\sqrt{8-4\sqrt{2}}} \begin{pmatrix} 1 \\ \sqrt{2}-1 \\ -1 \\ 1-\sqrt{2} \end{pmatrix}.
        \]
    \end{itemize}
    \item \( A_2 \otimes B_2 \):
    \begin{itemize}
        \item Eigenvalue \( +1 \):
        \[
        \frac{1}{\sqrt{8-4\sqrt{2}}} \begin{pmatrix} 1 \\ 1-\sqrt{2} \\ 1 \\ 1-\sqrt{2} \end{pmatrix}, \quad \frac{1}{\sqrt{8+4\sqrt{2}}} \begin{pmatrix} 1 \\ 1+\sqrt{2} \\ -1 \\ -1-\sqrt{2} \end{pmatrix}.
        \]
        \item Eigenvalue \( -1 \):
        \[
        \frac{1}{\sqrt{8+4\sqrt{2}}} \begin{pmatrix} 1 \\ 1+\sqrt{2} \\ 1 \\ 1+\sqrt{2} \end{pmatrix}, \quad \frac{1}{\sqrt{8-4\sqrt{2}}} \begin{pmatrix} 1 \\ 1-\sqrt{2} \\ -1 \\ \sqrt{2}-1 \end{pmatrix}.
        \]
    \end{itemize}
\end{itemize}



\section*{Comparison of Eigenvectors Across CHSH Joint Operator Eigensystems}

This analysis examines whether eigenvectors from the eigensystems of the joint operators \( A_1 \otimes B_1 \), \( A_1 \otimes B_2 \), \( A_2 \otimes B_1 \), and \( A_2 \otimes B_2 \) are identical or orthogonal across different eigensystems, for the optimal CHSH measurement angles achieving the maximal violation \( 2\sqrt{2} \).

\subsection*{Eigenvectors}
The eigenvectors for each operator are as follows (in the basis \( |00\rangle, |01\rangle, |10\rangle, |11\rangle \)):

\begin{itemize}
    \item \( A_1 \otimes B_1 \):
    \begin{itemize}
        \item Eigenvalue \( +1 \):
        \[
        |v_{11}^+\rangle = \frac{1}{\sqrt{4-2\sqrt{2}}} \begin{pmatrix} 1 \\ \sqrt{2}-1 \\ 0 \\ 0 \end{pmatrix}, \quad |v_{12}^+\rangle = \frac{1}{\sqrt{4+2\sqrt{2}}} \begin{pmatrix} 0 \\ 0 \\ 1 \\ -1-\sqrt{2} \end{pmatrix}.
        \]
        \item Eigenvalue \( -1 \):
        \[
        |v_{11}^-\rangle = \frac{1}{\sqrt{4+2\sqrt{2}}} \begin{pmatrix} 1 \\ -1-\sqrt{2} \\ 0 \\ 0 \end{pmatrix}, \quad |v_{12}^-\rangle = \frac{1}{\sqrt{4-2\sqrt{2}}} \begin{pmatrix} 0 \\ 0 \\ 1 \\ \sqrt{2}-1 \end{pmatrix}.
        \]
    \end{itemize}
    \item \( A_1 \otimes B_2 \):
    \begin{itemize}
        \item Eigenvalue \( +1 \):
        \[
        |v_{21}^+\rangle = \frac{1}{\sqrt{4-2\sqrt{2}}} \begin{pmatrix} 1 \\ 1-\sqrt{2} \\ 0 \\ 0 \end{pmatrix}, \quad |v_{22}^+\rangle = \frac{1}{\sqrt{4+2\sqrt{2}}} \begin{pmatrix} 0 \\ 0 \\ 1 \\ 1+\sqrt{2} \end{pmatrix}.
        \]
        \item Eigenvalue \( -1 \):
        \[
        |v_{21}^-\rangle = \frac{1}{\sqrt{4+2\sqrt{2}}} \begin{pmatrix} 1 \\ 1+\sqrt{2} \\ 0 \\ 0 \end{pmatrix}, \quad |v_{22}^-\rangle = \frac{1}{\sqrt{4-2\sqrt{2}}} \begin{pmatrix} 0 \\ 0 \\ 1 \\ 1-\sqrt{2} \end{pmatrix}.
        \]
    \end{itemize}
    \item \( A_2 \otimes B_1 \):
    \begin{itemize}
        \item Eigenvalue \( +1 \):
        \[
        |v_{31}^+\rangle = \frac{1}{\sqrt{8-4\sqrt{2}}} \begin{pmatrix} 1 \\ \sqrt{2}-1 \\ 1 \\ \sqrt{2}-1 \end{pmatrix}, \quad |v_{32}^+\rangle = \frac{1}{\sqrt{8+4\sqrt{2}}} \begin{pmatrix} 1 \\ -1-\sqrt{2} \\ -1 \\ 1+\sqrt{2} \end{pmatrix}.
        \]
        \item Eigenvalue \( -1 \):
        \[
        |v_{31}^-\rangle = \frac{1}{\sqrt{8+4\sqrt{2}}} \begin{pmatrix} 1 \\ -1-\sqrt{2} \\ 1 \\ -1-\sqrt{2} \end{pmatrix}, \quad |v_{32}^-\rangle = \frac{1}{\sqrt{8-4\sqrt{2}}} \begin{pmatrix} 1 \\ \sqrt{2}-1 \\ -1 \\ 1-\sqrt{2} \end{pmatrix}.
        \]
    \end{itemize}
    \item \( A_2 \otimes B_2 \):
    \begin{itemize}
        \item Eigenvalue \( +1 \):
        \[
        |v_{41}^+\rangle = \frac{1}{\sqrt{8-4\sqrt{2}}} \begin{pmatrix} 1 \\ 1-\sqrt{2} \\ 1 \\ 1-\sqrt{2} \end{pmatrix}, \quad |v_{42}^+\rangle = \frac{1}{\sqrt{8+4\sqrt{2}}} \begin{pmatrix} 1 \\ 1+\sqrt{2} \\ -1 \\ -1-\sqrt{2} \end{pmatrix}.
        \]
        \item Eigenvalue \( -1 \):
        \[
        |v_{41}^-\rangle = \frac{1}{\sqrt{8+4\sqrt{2}}} \begin{pmatrix} 1 \\ 1+\sqrt{2} \\ 1 \\ 1+\sqrt{2} \end{pmatrix}, \quad |v_{42}^-\rangle = \frac{1}{\sqrt{8-4\sqrt{2}}} \begin{pmatrix} 1 \\ 1-\sqrt{2} \\ -1 \\ \sqrt{2}-1 \end{pmatrix}.
        \]
    \end{itemize}
\end{itemize}

\subsection*{Identical Vectors}
No eigenvectors from different eigensystems are identical (up to a phase). The vectors have distinct component structures due to the different measurement bases (\( \sigma_z \), \( \sigma_x \), \( B_1 \), \( B_2 \)), with differing normalization constants and non-zero component patterns (e.g., \( A_1 \)-based vectors have two zeros, while \( A_2 \)-based vectors have four non-zero components).

\subsection*{Orthogonal Vectors}
Representative inner product calculations show that eigenvectors from different eigensystems are not orthogonal. For example:
\begin{itemize}
    \item \( \langle v_{11}^+ | v_{21}^+ \rangle \approx 1.366 \neq 0 \).
    \item \( \langle v_{11}^+ | v_{31}^+ \rangle \approx 0.686 \neq 0 \).
    \item \( \langle v_{31}^+ | v_{41}^+ \rangle \approx 1.354 \neq 0 \).
    \item \( \langle v_{31}^+ | v_{42}^- \rangle \approx 0.323 \neq 0 \).
\end{itemize}
The non-zero inner products arise because the eigenvectors involve components like \( \sqrt{2}-1 \) and \( 1-\sqrt{2} \), which do not cancel out across different operators. The operators do not commute in general, confirming that their eigenspaces are not orthogonal.

\subsection*{Conclusion}
There are no identical or orthogonal eigenvectors across the eigensystems of different joint operators, due to the distinct measurement angles and resulting tensor product structures.

\section*{Eigensystem of the CHSH Operator}

The CHSH operator is defined as \( S = A_1 \otimes B_1 + A_1 \otimes B_2 + A_2 \otimes B_1 - A_2 \otimes B_2 \), where:
\begin{itemize}
    \item \( A_1 = \sigma_z = \begin{pmatrix} 1 & 0 \\ 0 & -1 \end{pmatrix} \),
    \item \( A_2 = \sigma_x = \begin{pmatrix} 0 & 1 \\ 1 & 0 \end{pmatrix} \),
    \item \( B_1 = \frac{1}{\sqrt{2}} (\sigma_z + \sigma_x) = \begin{pmatrix} \frac{1}{\sqrt{2}} & \frac{1}{\sqrt{2}} \\ \frac{1}{\sqrt{2}} & -\frac{1}{\sqrt{2}} \end{pmatrix} \),
    \item \( B_2 = \frac{1}{\sqrt{2}} (\sigma_z - \sigma_x) = \begin{pmatrix} \frac{1}{\sqrt{2}} & -\frac{1}{\sqrt{2}} \\ -\frac{1}{\sqrt{2}} & -\frac{1}{\sqrt{2}} \end{pmatrix} \).
\end{itemize}

\subsection*{CHSH Operator}
The operator \( S \) in the two-qubit basis \( |00\rangle, |01\rangle, |10\rangle, |11\rangle \) is:
\[
S = \begin{pmatrix}
\sqrt{2} & 0 & 0 & \sqrt{2} \\
0 & -\sqrt{2} & \sqrt{2} & 0 \\
0 & \sqrt{2} & -\sqrt{2} & 0 \\
\sqrt{2} & 0 & 0 & \sqrt{2}
\end{pmatrix}.
\]

\subsection*{Eigensystem}
The eigenvalues and corresponding eigenvectors are:
\begin{itemize}
    \item Eigenvalue \( \lambda = 2\sqrt{2} \):
    \[
    |v_1\rangle = \frac{1}{2\sqrt{3}} \begin{pmatrix} 1 \\ 1 \\ 3 \\ 1 \end{pmatrix}.
    \]
    \item Eigenvalue \( \lambda = -2\sqrt{2} \):
    \[
    |v_2\rangle = \frac{1}{2\sqrt{3}} \begin{pmatrix} -3 \\ 1 \\ -1 \\ 1 \end{pmatrix}.
    \]
    \item Eigenvalue \( \lambda = 0 \) (multiplicity 2):
    \[
    |v_3\rangle = \frac{1}{\sqrt{2}} \begin{pmatrix} -1 \\ 0 \\ 0 \\ 1 \end{pmatrix}, \quad |v_4\rangle = \frac{1}{\sqrt{2}} \begin{pmatrix} 0 \\ 1 \\ 1 \\ 0 \end{pmatrix}.
    \]
\end{itemize}

\end{document}

