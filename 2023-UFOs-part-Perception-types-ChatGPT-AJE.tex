%%%%%%%%%%%%%%%%%%%%% chapter.tex %%%%%%%%%%%%%%%%%%%%%%%%%%%%%%%%%
%
% sample chapter
%
% Use this file as a template for your own input.
%
%%%%%%%%%%%%%%%%%%%%%%%% Springer-Verlag %%%%%%%%%%%%%%%%%%%%%%%%%%
%\motto{Use the template \emph{chapter.tex} to style the various elements of your chapter content.}
\chapter{UFO perception in the USA and elsewhere}
\label{2023-UFO-part-Perception-types} % Always give a unique label
% use \chaptermark{}
% to alter or adjust the chapter heading in the running head


\abstract*{It is conjectured that in 1948, after Project Sign in the USA, a secret program was initiated that focused on advanced topics such as anti-gravity and spaceship development. Meanwhile, Project Grudge and later Project Blue Book were publicly promoted as attempts to downplay the significance of these findings and mislead the public. The individuals involved, such as Ruppelt, Condon, and Hynek, are portrayed as unknowingly participating in a government-led drama to deny both the existence of these advanced technologies and the extraterrestrial origin of a tiny fraction of UFO sightings.}



\abstract{It is conjectured that in 1948, after Project Sign in the USA, a secret program was initiated that focused on advanced topics such as anti-gravity and spaceship development. Meanwhile, Project Grudge and later Project Blue Book were publicly promoted as attempts to downplay the significance of these findings and mislead the public. The individuals involved, such as Ruppelt, Condon, and Hynek, are portrayed as unknowingly participating in a government-led drama to deny both the existence of these advanced technologies and the extraterrestrial origin of a tiny fraction of UFO sightings.}




\section{United States of America}
\label{2023-UFO-part-Perception-types-USA}

My basic assumption, although unproven and speculative, is that the US government's executive branch, intelligence, and military entities follow a two-tier strategy concerning UFOs:
\begin{enumerate}
\item
Denial to the public while occasionally conducting hearings, committees, and reports that appear to investigate the matter. This approach satisfies Congress, the press, and the general public. Project Blue Book, the Robertson Panel, the Condon Report, AAWSAP/AATIP, and the recent All-domain Anomaly Resolution Office (AARO) are examples of such ``outside cover-up investigations.'' This strategy is comparable to adults playing Christmas for their children: the children blissfully remain unaware~\cite{Loewe-Christmas}.
\item
It cannot be outrightly excluded that there may be an ongoing exploitation and sequestration of crashed UFOs, though this remains to be corroborated by factual evidence. It is possible that a select few insiders, with ties to influential think tanks and military-industrial complexes, may have access to crashed UFOs and their occupants, if indeed there any.
Battelle Memorial Institute, a private nonprofit company that specializes in technology development, is one example of a think tank that some individuals suspect may have ties to this area. Another possible connection could be Lockheed Martin Corporation. Furthermore, there are those who speculate that private contractors may be more integral to the government's activities in this field due to governmental transparency and oversight laws such as the Freedom of Information Act, as well as occasional presidential or congressional reviews.
\end{enumerate}

Imagine this situation as an iceberg: it may appear small and insignificant on the surface, but it is deep and vast underneath. For laypersons, the scarcity of official information about UFOs can be frustrating. However, this lack of information is often accompanied by a large volume of claims, reports, and anecdotal evidence, making it difficult to determine what is credible. The sheer volume of UFO sightings of varying quality can make it challenging to wade through the data and separate credible reports from false or misidentified ones. ``Clutter'' and noise may inhibit the meaning and message due to their sheer volume.

The interface between the ``surface'' and the ``deep state'' parts seems almost impenetrable. There may have been individuals, driven by the knowledge they acquired during their function as ``official and open'' investigators, attempting to access what is going on inside. In particular, Josef Allen Hynek~\cite{Hynek:53,Hynek_1969,Hynek1972,Hynek1975Dec,Hynek1977Jan}\index{Hynek, J. Allen} ``changed camps'' and became increasingly weary of the debunking dramaturgy he had to follow. In more recent times, members of AATIP and Christopher Karl Mellon~\cite{Mellon2018Mar,Mellon2022Dec} have come forward, promoting disclosure of the deeper layers of UFO activities.




For the normal citizen who is considered not to be in the need to know, the signal-to-noise ratio in UFO sighting reports can be low,
making it difficult to separate credible reports from false or misidentified ones---to separate the wheat from the chaff.
This can make investigating UFO sightings frustrating, time-consuming, and resource-intensive.

As a result, many interested laypeople may turn to government agencies such as the military---because of the sensors deployed---or
Congress---because of constitutional principles and oversight---in hopes of obtaining more information. Additionally, some people may use Freedom of Information Act (FOIA) requests to try to access information about UFOs.


%One may compare this to the Budhist concepts of Mahayana (great vehicle) Hinayana (little vehicle)

In July 1952, following the Washington flyovers, Major General John Samford, the Air Force's director of intelligence, presented a mantra during what was arguably the largest press conference since the end of the Second World War~\cite{Lewis-Kraus2021Apr}. This mantra would be repeatedly handed out as official canon, as documented in the archives~\cite{Archives1952}:
\begin{svgraybox}
``Our basic difficulty in dealing with these is that there is no measurement of [[the flying saucers]]
that makes it possible for us to put them in any pattern that would be profitable for a deliberate custom sort of analysis to take the next step.''

``We have as a date come to only one firm conclusion with respect to this remaining percentage,
and that is that it does not contain any pattern or purpose or of consistency
that we can relate [[$\ldots$]] to any conceivable threat to the United States.''

``We can say that the recent sightings are in no way connected with any secret development by any department of the United States.''
\end{svgraybox}

The multi-authored book ``{UFO}s and Government: {A} Historical Inquiry''~\cite{Swords2012Jul}
by Michael Swords,  Robert Powell,  Clas Svahn, Vicente-Juan Ballester Olmos,
Bill Chalker,  Barry Greenwood,
Richard Thieme,  Jan Aldrich, and Steve Purcell gives a detailed analysis of the people involved.
The following is a sketch of what might have happened.

\subsection{Project Sign, late 1947--1948}
\label{2023-UFO-part-Perception-types-USA-Sign}
\index{Project Saucer}
\index{Project Sign}
\index{Sign}


Project Saucer, later renamed Project Sign, was a covert US Air Force investigation into unidentified flying objects (UFOs)
established by Air Force General Nathan Farragut Twining, head of the Air Technical Service Command,
at the end of 1947 or early 1948. The initial name of the project was Project Saucer, and its goal was to collect, evaluate, and distribute within the government all information relating to UFO sightings on the premise that they might pose a national security concern. On April 27, 1949, a paper prepared by the Intelligence Division of the Air Material Command at Wright-Patterson Field, Ohio, was publicly released by the US Air Force. The paper stated that, while some UFOs appeared to represent actual aircraft, there were not enough data to determine their origin. Although almost all cases were explained by ordinary causes, the report recommended continuing the investigation of all sightings.

Retired Air Force Captain Edward J. Ruppelt, who later directed Project Blue Book, first mentioned Project Sign in his 1956 book~\cite{Ruppelt2011May}. Ruppelt~\cite[Chapter~3]{Ruppelt2011May} claimed that Project Sign had produced an ``Estimate of the Situation,'' a document that was never published. According to Ruppelt, the document endorsed an interplanetary explanation for UFOs:
\begin{svgraybox}
``The people at ATIC [[Air Technical Intelligence Center at Wright-Patterson Air Force Base, Ohio]]
\index{ATIC}
\index{Air Technical Intelligence Center}
decided that the time had arrived to make an Estimate of the Situation.
The situation was the UFO's; the estimate was that they were interplanetary!

It was a rather thick document with a black cover and it was printed on legal sized paper. Stamped across the front were the words TOP SECRET.

It contained the Air Force's analysis  of many  of
the incidents I have told you about plus many similar ones.
All of them had come from scientists, pilots, and other equally credible observers, and each one was an unknown.


      The document pointed out that the reports hadn't actually started with the Arnold Incident
[[see Section~\ref{2023-UFO-part-History-chapter-post-1945-pre-1947-KA}]].
\index{Arnold, Kenneth}
Belated reports from a weather observer in Richmond, Virginia, who observed a
``silver disk'' through his theodolite telescope; an F47 pilot and three pilots in his formation who saw a ``silver flying wing,''
and the English ``ghost airplanes'' that had been picked up on radar early in 1947 proved this point.
Although reports on them were not received until after the
Arnold sighting, these incidents all had taken place earlier.''
\end{svgraybox}

There were also several dramatic reports of pilot dogfights with UFOs that later turned out to be balloons.
Therefore, the situation might have been ambivalent: on the one hand, once commissioned to investigate UFO cases,
ATIC was vigorously investigating such sightings and, as one might have expected this from such organizations,
produced reports that favoured further studies.
On the other hand, many of these alleged flying saucers turned out to have prosaic origins.
Therefore, despite the aforementioned report,
General Hoyt Sanford Vandenberg~\cite{VandenbergUSAF}, the Chief of Staff of the Air Force,
``wouldn't buy interplanetary vehicles. The report lacked proof. A group from ATIC went
to the Pentagon to bolster their position but had no luck, the Chief of Staff just couldn't be convinced''~\cite[Chapter~3]{Ruppelt2011May}.
Project Sign was shut down on the grounds of an alleged lack of evidence.

In summary, Project Sign was established to investigate UFO sightings and determine whether they posed a national security threat.
However, at least the official verdict was that the project's findings were not convincing enough for the Air Force
to continue the investigation.
Indeed, most sightings could be attributed to misidentifications of natural phenomena or conventional aircraft such as balloons.
As a result, conclusions were drawn that UFOs did not pose a threat to national security, and further investigation was unnecessary.

This decision of the US Air Force caused heated public debates. UFOs continued to be seen, even on radar.
We may suppose that the press played a role in pushing the narrative and continuing investigations into these sightings.
Flying saucers captured much attention, which increased paper circulation volume and revenues.
Clearly, the US press would not let this good opportunity go to waste,
even after the Air Force claimed that the underlying phenomena were mundane.

Authors who had past experience with fantastic stories were lured into the general flying saucer hype. Let me mention two examples: in 1947 and 1948, the British journalist Harold T. Wilkins was still publishing about
``Mysteries of Ancient South America''~\cite{Wilkins1946} and ``Mysteries and Monsters of the Deep''~\cite{Wilkins2017Aug}
(see also his 1929 Popular Mechanics piece ``Secrets of Ancient Torture Chambers''~\cite{Wilkins1929}),
but he then shifted his attention, switching to UFO mysteries and writing books entitled
``Flying Saucers on the Attack''~\cite{Wilkins1954Jan} or ``Flying Saucers Uncensored''~\cite{Wilkins1955}.

Donald Edward Keyhoe managed a coast-to-coast tour with Charles Lindbergh in 1927 and wrote his first book, ``Flying With Lindbergh,''
in 1928~\cite{Keyhoe1928}. During World War II, he served in the Naval Aviation Training Division and attained the rank of Major.
Before this, he had written fantasy stories for pulp magazines and glossies in the 1920s and 1930s, such as
``The Grim Passenger''~\cite{WeirdTalesKeyhoe-25}, which involved supernatural forces, magic, and extrasensory perception.
Allegedly, Keyhoe had friends and good contacts in the military and the Pentagon.
Later in life, he became passionate about the flying saucer mystery.
He suggested that ``the saucers are not a joke''~\cite{KeyhoeFSAR1950} and ``The Flying Saucers Are Real''~\cite{KeyhoeFSAR1950,Keyhoe1950}
and need to be investigated further.
Afterwards, Keyhoe claimed that there was a conspiratorial cover-up going on~\cite{Keyhoe1955}.

At that point ATIC, the Air Technical Intelligence Center at Wright-Patterson Air Force Base, Ohio, seemed to give in to the demands of
Vandenberg, the Chief of Staff of the Air Force.
In 1971, Australian physicist Oliver Harry Turner, who was working as a senior nuclear scientist for the Australian government at the time, compiled an intelligence analysis of the US's initial attempts to address the flying saucer or UFO phenomenon~\cite{TurnerAustralia1971}:
\begin{svgraybox}
``In June of 1947 the Air Technical Intelligence Centre
(ATIC) near Dayton, Ohio, assumed a responsibility to investigate
the initial reports of `flying saucers'. Within a month it was
considered that the phenomena were real and probably of Soviet
origin. By the end of the year, when ATIC was officially authorised
to investigate under the project code name of SIGN and with a high
priority, most of the investigators were focussing on an interplanetary
rather than a Soviet origin. These opinions were
crystallized into a [[untraceable or lost]] written estimate [[of the situation]] that was sent to the Pentagon
in September 1948. When the interplanetary conclusions were rejected
on the grounds of insufficient hard evidence, a reaction set in at
ATIC against trying to unravel the UFO problem.''
\end{svgraybox}


Despite indications of an ongoing saucer craze fueled by the press and some public figures, UFO sightings from various military and public channels continued to stream in.

\subsection{Project Grudge, 1948-1952}
\label{2023-UFO-part-Perception-types-USA-Grudge}
\index{Project Sign}
\index{Sign}

Project Sign officially ended in late 1948. It was succeeded by a new project known as Project Grudge, which was met with criticism, as it was believed to have a mandate to debunk UFO sightings rather than genuinely investigate them.

The head of Project Grudge, Edward Ruppelt, later referred to the period during which Project Grudge was active as the ``dark ages''~\cite[Chapter~5]{Ruppelt2011May} of early USAF UFO investigation:
\begin{svgraybox}
``New people took over Project Grudge. ATIC's top intelligence specialists who had been so eager to work on Project Sign
were no longer working on Project Grudge. Some of them had drastically and hurriedly changed their minds about
      UFOs  when they thought that the Pentagon was no longer sympathetic to the UFO cause.
They were now directing their talents toward more socially acceptable projects.
 Other charter members of Project Sign had been ``purged.'' These were the people who had refused to change their original opinions about UFOs.


      With the new name and the new personnel came the new objective: get rid of the UFOs.
It was never specified this way in writing, but it didn't take much effort to see that this was the goal of Project Grudge.
This unwritten objective was reflected in every memo, report, and directive.
\end{svgraybox}


Project Grudge concluded that all UFO sightings could be explained as natural phenomena or misinterpretations of known objects~\cite[Chapter~4]{Jacobsen2011}. However, the report also admitted that 23 percent of the reports could not be explained by any known means. Despite this finding, the project concluded that UFOs did not pose a threat to national security and that there was no need for further investigation. As a result, Project Grudge was discontinued. It was not until the creation of the Project Blue Book in 1952 that the government conducted an official investigation into UFOs.

\subsection{Insert: Was there a breakaway agency parallel to Project Grudge attempting to gather design data on interplanetary spaceships?}
\label{2023-UFO-part-Perception-types-USA-baa}

Before continuing with this public history, I invite the reader to contemplate the possibility that Turner's intelligence analysis of 1971 had come to nuanced conclusions. It may even have hinted that, at that point in time, the USA had initiated a ``breakaway utilization'' of UFO material while publicly holding on to Project Grudge~\cite{TurnerAustralia1971}:
\begin{svgraybox}
``2.~In February 1949, ATIC personnel of Project SIGN were
replaced with new personnel to form Project GRUDGE. A definite
attempt was made during 1949 to use Project GRUDGE to destroy any
acceptance of UFO's. The motives for this are not clear: possibly
Air Force embarrassment at being incapable of controlling the
situation and/or a fear of national panic prompted USAF to try and
remove the problem by denying its existence. Another possible
motive may have been to provide a breathing space for another
``investigative agency'' to reach some conclusion; the [[this other investigative]] agency had been
assisting ATIC through 1948 and, contrary to official USAF policy,
was maintaining a high level of interest during 1949. This governmental
agency was not the FBI, and had rocket, nuclear and intelligence
experts; their purpose was to study UFO reports in an effort to gather
design data on interplanetary spaceships.''
\end{svgraybox}
The Australian report speculates that ``in the light of later developments, this agency was almost certainly the CIA.''

I prefer to leave the question of who exactly was the ``investigative agency'' that handled the ``effort to gather design data on interplanetary spaceships'' open until new evidence comes forward, if ever. According to Richard Dolan and Bryce Zabel, there is a group referred to as ``The Breakaway Group''~\cite{DolanZabel2012May}. This group includes what Donald Keyhoe identified as ``super-secret groups'' or the ``Silence Group'' in the 1950s~\cite[Foreword]{Keyhoe1955}. If certain leaked documents from the 1980s are credible, which is highly controversial, this group was also called ``Majestic-12'' or ``MJ-12.''

Turner's estimation of the situation is corroborated by another source, coming from Canada, the Smith memo~\cite{SmithWilbertMemo},
summarized in Section~\ref{2023-UFO-part-Perception-crash-retreivals-SM}, which points to two US scientists:
Robert Sarbacher and Eric Walker. In a letter dated November 29, 1983, Sarbacher confirmed the following names~\cite{Sarbacher-83}:
John von Neumann, Vannever Bush, and maybe Robert Oppenheimer. Sarbacher mentions that his
``association with the Research and Development Board under Doctor Compton during the Eisenhower administration was rather limited,''
so that, although he had been invited to participate in several discussions associated with the reported recoveries (UFO crash retrievals), he
could not personally attend the meetings. Sarbacher was sure that they would have asked Wernher von Braun.


One could speculate that, as time passed, the gap between the official, openly accessible US flying saucer efforts such as Project Blue Book or the Condon Report on the one hand, and these breakaway hidden projects on the other hand, widened, with communication exchange becoming increasingly absent.

If one believes Corso's narratives, as summarized in Section~\ref{2023-UFO-part-Perception-crash-retreivals-CM},
material from crash retrievals was presented to third parties such as Batelle or Lockheed Martin and locked away,
as mentioned in the contemporary security state lingo in Appendix~\ref{2023-UFO-part-Perception-crash-retreivals--cousap},
as carve-out Unacknowledged Special Access Programs (USAPs),
or Controlled Access Programs (CAPs) or DOD SAPs.



\subsection{Project Blue Book, 1952--1969---a data volume trap---drowning in data}
\label{2023-UFO-part-Perception-types-USA-PBB}

It is important to note that, based on my hypothesis, investigations of UFOs following Project Grudge
were primarily for the purpose of diverting attention away from related strange phenomena, and comforting and educating the public.
They were not conducted with the intention of fully understanding the UFO phenomenon, but rather to give the appearance that the military and intelligence community were taking the phenomenon seriously.

According to Ruppelt, by the end of 1951, several high-ranking, very influential USAF generals were so dissatisfied
with the state of Air Force UFO investigations that they dismantled Project Grudge and replaced it with Project Blue Book in March 1952.
One of these men was General Charles P. Cabell.
Another pro-UFO change came when General William Garland~\cite{GarlandUSAF,GarlandProject1947} joined Cabell's staff:
Garland was an ``experiencer'' himself, as he had witnessed a UFO, and thought the UFO question deserved serious scrutiny.

The Blue Book team was authorized to interview any and all military personnel who witnessed UFOs and was not required to follow the chain of command. Each US Air Force Base had a Blue Book officer to collect UFO reports and forward them to Ruppelt.

The new name, Project Blue Book, was selected to refer to the blue booklets used for testing at some colleges and universities.
Ruppelt was the first head of the project, an experienced airman with
an aeronautics degree. He officially coined the term ``Unidentified Flying Object,'' to replace the term ``flying saucer'' the military had previously used.

Ruppelt streamlined the manner in which UFOs were reported to (and by) military officials, partly in hopes of alleviating the stigma and ridicule associated with UFO witnesses.
Ruppelt also ordered the development of a standard questionnaire for UFO witnesses---Air
Force Letter 200-5 (AFL 200-5, 29-APR-1952)~\cite{AFL200-5},
Air Force Regulation 200-2 (or AFR 200-2) version August 12, 1954~\cite{AFR200-2} (superseding versions August 26, 1953, including Change 200-2A, November 2, 1953),
a technical information sheet~\cite[Appendix~II]{PBB-SR8-1952}
and Air Force Form 112---hoping to uncover data that could be subject to statistical analysis.

For this task, he commissioned the Battelle\index{Batelle} Memorial Institute to create the questionnaire and computerize the data. Using case reports and computerized data, Battelle then conducted a massive scientific and statistical study of all Air Force UFO cases, completed in 1954 and known as ``Project Blue Book Special Report No. 14.'' The Summary sections basically stated that concerns or efforts with regard to UFOs were a waste of resources~\cite[p.~ix]{ATIC1955May-SR14}:
\begin{svgraybox}
``All available data were included in this study which was prepared by
a panel of scientists both in and out of the Air Force. On the basis of this
study it is believed that all the unidentified aerial objects could have been
explained if more complete observational data had been available. Insofar
as the reported aerial objects which still remain unexplained are concerned,
there exists little information other than the impressions and interpretations
of their observers. As these impressions and interpretations have been
replaced by the use of improved methods of investigation and reporting,
and by scientific analysis, the number of unexplained cases has decreased rapidly
towards the vanishing point.

Therefore, on the basis of this evaluation of the information, it is
considered to be highly improbable that reports of unidentified aerial objects
examined in this study represent observations of technological developments
outside of the range of present-day scientific knowledge. It is emphasized
that there has been a complete lack of any valid evidence of physical matter
in any case of a reported unidentified aerial object.''
\end{svgraybox}


The Data Section reveals some very personal frustration with the Blue Book data~\cite[p.~93]{ATIC1955May-SR14}:
\begin{svgraybox}
``The reaction, mentioned above, that after reading a few reports, the
reader is convinced that `flying saucers' are real and are some form of
sinister contrivance, is very misleading. As more and more of the reports
are read, the feeling that `saucers' are real fades, and is replaced by a
feeling of skepticism regarding their existence. The reader eventually
reaches a point of saturation, after which the reports contain no new information
at all and are no longer of any interest. This feeling of surfeit was
universal among the personnel who worked on this project, and continually
necessitated a conscious effort on their part to remain objective.''
\end{svgraybox}

One can sense the genuine frustration of these analysts and programmers. However, they may not have recognized
that they were using an inappropriate methodology for this particular kind of problem:
in a regime where only a tiny fraction of the data indicates a new phenomenon,
these researchers were overwhelmed by the amount of data they had generated and reported on.
To quote Einstein in a different context~\cite{einstei-letter-to-schr,Meyenn-2011,Howard1985171,Howard1990}: the main point ``was buried by the erudition.''
In my opinion, the lesson learned from the Batelle investigation initiated and paid for by the US Air Force was that it makes little sense to cope with this phenomenon through ``big data.'' There is too much clutter to reveal the signal; even advanced AI techniques may fail to separate the wheat from the chaff.

Ruppelt tried to avoid uncorroborated speculation and a resulting escalation of controversies related to the extraterrestrial hypothesis in his own group: he allegedly fired three personnel very early in the project because they were either ``too pro'' or ``too con'' on the extraterrestrial hypothesis.

The astronomer J. Allen Hynek\index{Hynek, J. Allen} was the scientific consultant for the project.
In their attempt to find prosaic causes for, or even ``explain away,'' all sorts of UFO sightings,
oftentimes justified but sometimes far-fetched, with occasional implausible causes,
Hynek and Blue Book became increasingly at odds with the public. On March 25, 1966,
this culminated in a press conference by Hynek, where he rather firmly stated that a rash of UFO sightings across Michigan in the mid-1960s,
in particular, the incidents at the Mannor farm and Hillsdale College, were related to swamp gas as a result of rotting vegetation in
lowland areas. The vegetation created gases that were trapped in winter. Then during spring, the gases were released.
This so-called ``swamp gas'' phenomenon could produce light and even sound.
``A dismal swamp is a most unlikely place for a visit from outer space,'' Hynek said at the press conference.
Moreover, Hynek assured the public that some strange photos taken at related events were
``trails made as a result of a camera time exposure of the rising crescent moon~[[$\ldots$]] and the planet Venus''---so, effectively, nothing to see here~\cite{Zielin-swampgas}.

However, with UFO sightings pouring in in ever-increasing numbers, as documented by the Blue Book protocols~\cite{bluebook-directory-listing}, and probably about five percent of these cases remaining hard to explain away by prosaic causes, Hynek started to get doubts---his inclinations toward the extraterrestrial or even stranger hypotheses can be deduced from the timeline of his publications~\cite{Hynek:53,Hynek_1969,Hynek1972,Hynek1975Dec,Hynek1977Jan}.


Moreover, on the one hand, he was pressured by the public at large to come up with more plausible, if not spectacular, explanations. Politicians such as the later President of the United States and then Michigan Congressman, Gerald Rudolph Ford Jr., called Hynek ``flippant'' and proposed that either the Science and Astronautics Committee or the Armed Services Committee of the House schedule UFO hearings and ``invite testimony from both the executive branch of the government and some of the persons who claim to have seen UFO's''~\cite{Logan2016Apr}.

On the other hand, the US Air Force and the CIA had very different problems, officially claiming that ``continued emphasis on UFO reporting might threaten `the orderly functioning' of the government by clogging the channels of communication with irrelevant reports and by inducing `hysterical mass behavior' harmful to constituted authority''~\cite{Haines-CIA-UFO}. This was where the Robertson Panel came in.

\subsection{Robertson Panel, 1952--53}
\label{2023-UFO-part-Perception-types-USA-RP}
\index{Robertson Panel}

Again, it is important to note that, based on my speculations, investigations of UFOs following Project Grudge were primarily
for the purpose of educating---or, in more sinister terms, distracting---the public. They were not conducted with
the intention of fully understanding the UFO phenomenon,
but rather to give the appearance that the military and intelligence community were ``dealing with it appropriately''
and, if necessary, taking the phenomenon seriously.

The CIA had noted that the British were in a similar situation: in 1951, upon mounting public sightings and interest,
the secret British Flying Saucer Working Party (FSWP) mentioned in Section~\ref{2023-UFO-part-Perception-types-UK-flwp1950-1}
had created a report amounting to ``nothing to see here''~\cite{FSWP1951}.
Marshall H.Chadwell, the CIA's chief scientist and Assistant Director of CIA's Office of Scientific Intelligence (OSI), was a guest of FSWP. Chadwell was responsible for the US government's strategy of dismissing UFO reports as a means of eliminating the perceived danger that belief in UFOs presented during the Cold War. The CIA aimed to decrease the subject's media attention through an ``educational'' program~\cite{ClarkeFSWP}.
\index{Chadwell, Marshall H.}

In December 1952, the Intelligence Advisory Committee discussed UFOs and decided to enlist the help of selected scientists.
In January 1953, Chadwell\index{Chadwell, Marshall H.}
and Howard P. Robertson, a physicist from the California Institute of Technology,
assembled a panel of distinguished nonmilitary scientists to examine the issue of UFOs.
The panel was led by Robertson and included other notable scientists, such as Samuel A. Goudsmit,
a nuclear physicist from the Brookhaven National Laboratories, Luis Alvarez,
a high-energy physicist, Thornton Page, the deputy director of the Johns Hopkins Operations Research Office and an expert in radar and electronics,
and Lloyd Berkner, a director of the Brookhaven National Laboratories and a specialist in geophysics.



Neither the panel chair, Robertson, nor Alvarez were new to the study of UFOs,
as they had been involved with the study of foo fighters and UFOs in World War II,
 as mentioned in the introduction of Chester's book~\cite{Chester2007May}.
This is similar to David T. Griggs, a professor of geophysics at the University of California at Los Angeles, who was also mentioned in the
Durant Report~\cite{RobertsonPanelDurantReport} of the Robertson Panel.

Thornton Page, a panel member, later revealed that before the main panel meetings, the panel members had an informal ``mission'' meeting with no outsiders present~\cite{Page1992}:
\begin{svgraybox}
``Robertson told us in the first private (no outsiders) session that our job was to reduce public concern,
and show that UFO reports could be explained by conventional reasoning.''
\end{svgraybox}

The panel's official task~\cite{Haines-CIA-UFO} was a careful review of the available evidence on UFOs and evaluation of the potential risks to US national security.
The panel met from January 14 to 18, 1953, and examined Air Force data on UFO case histories.
After studying the phenomena for 12 hours, the panel determined that reasonable explanations could be provided for most,
if not all, sightings.
The panel reached a unanimous conclusion that there was no evidence of a direct threat to national security from the UFO sightings,
nor was there any evidence that the objects sighted were extraterrestrial in origin. However, they were concerned that continued emphasis on UFO reporting could disrupt the orderly functioning of the government by clogging communication channels with irrelevant reports and inducing ``hysterical mass behavior'' that could harm the authorities.
The panel also worried that potential enemies might exploit the UFO phenomena and use them to disrupt US air defenses.

To address these issues, the panel recommended that the National Security Council (NSC) work to debunk UFO reports
and implement a public education campaign to reassure the public that there was no evidence to support the existence of UFOs.
They suggested utilizing mass media, advertising, business clubs, schools, and even the Disney corporation to disseminate this message.
Given the political climate at the time, the panel also recommended monitoring
private UFO groups such as the Civilian Flying Saucer Investigators in Los Angeles and the
Aerial Phenomena Research Organization in Wisconsin for any so-called ``subversive'' activities.

The conclusions of the Robertson panel were similar to those of earlier Air Force project reports on UFOs,
as well as the CIA's own Office of Scientific Intelligence (OSI) Study Group, which found that UFO reports posed no direct threat to national security
and that there was no evidence of extraterrestrial visits.
Following the panel's findings, CIA officials stated that no further consideration of the subject appeared necessary,
but they continued to monitor sightings for potential national security concerns~\cite{RobertsonPanelDurantReport}.

In the letter quoted earlier, Page also recalls~\cite{Page1992}:
\begin{svgraybox}
``As indicated in `UFOs, A Scientific Debate'~\cite{Sagan1974Jan}, Sagan and I later became convinced that E.U. Condon's scientific `Study of Unidentified Flying Objects'~\cite{Condon-report}
(Bantam Books 1969 = `The Condon report')
was neither scientific nor rational, concluding, as it did, that 15 celebrated sightings were not worthy of further investigation.''
\end{svgraybox}

Interestingly, many of the prominent scientists on the Robertson Panel,
who were tasked with investigating UFOs in 1953, had previously been involved in the examination of unconventional
objects such as foo fighters during World War II~\cite{Chester2007May,Rendall2021Aug}. This included Robertson,
the chairman of the panel,
as well as Luis W. Alvarez from the University of California.

Notably, as mentioned earlier, the Durant Robertson Report also identified David T. Griggs, a geophysics professor at the University of California at Los Angeles,
as the individual with the most knowledge about foo fighters. These scientists had been observing and
analyzing this phenomenon for a considerable amount of time before the panel was formed.
John Archibald Wheeler was also contacted by Durant ``as a consultant in the CIA attack on the `flying saucers' problem'' but politely refused,
stating ``that he might not be able to actively assist even after March 1953''
but ``would be pleased at any time to discuss this matter briefly''~\cite{WheelerFOIACIA0005515945}.

The Durant Report of the Robertson Panel contains the following passage~\cite{RobertsonPanelDurantReport}:
\begin{svgraybox}
The Panel concluded unanimously that there was no evidence of a direct threat to national security in the objects sighted.
Instances of ``Foo Fighters'' were cited.
These were unexplained phenomena sighted by aircraft pilots during World War II in both European and Far East theaters
of operation wherein ``balls of light'' would fly near or with the aircraft and maneuver rapidly.
They were believed to be electrostatic (similar to St. Elmo's fire) or
electromagnetic phenomena or possibly light reflections from ice crystals in the air,
but their exact cause or nature was never defined.
Both Robertson and Alvares had been concerned in the investigation of these phenomena,
but David T. Griggs (Professor of Geophysics at theUniversity of California at Los Angeles)
is believed to have been the most knowledgeable person on this subject.
If the term ``flying saucers'' had been popular in 1943--1945, these objects would
have been so labeled.
It was interesting that in at least two cases reviewed that the object sighted was categorized by Robertson and Alvarez as probably
``Foo Fighters,'' to date unexplained but not dangerous; they were not happy thus to dismiss the sightings by calling them names.
It was their feeling that these phenomena are not beyond the domain of present knowledge of physical sciences, however.
\end{svgraybox}

Therefore, what were---or are---these strange ``foo fighters''?

Turner's analysis of 1971 characterized the situation as follows~\cite{TurnerAustralia1971}:
\begin{svgraybox}
``3. Project GRUDGE failed to eliminate the UFO problem. UFO
reports in 1949 actually exceeded the number in 1948, and several
people [[Donald Edward Keyhoe and James Edward McDonald?]]
who had gained access to earlier official reports were able
to contradict the USAF. Journalists generally felt that GRUDGE
reporting represented a cover to a more serious knowledge. Eventually,
USAF intelligence decided that a fresh approach to the problem was
necessary. Between September 1951 and the establishment of Project
BLUE BOOK in March 1952, UFO investigation regained adequate financial
and administrative support to once again analyse the collected data.
Project BLUE BOOK was able to process the data from 3200 reports into
a form suitable for their consultants to be able to use
IBM card-sorting machines.''

``4. The summer of 1952 saw a more than twenty-fold rise in
the normal rate of reporting and included the two extensive July
sightings involving Washington D.C. This marked increase in sightings
had diverse effects. A component of USAF intelligence considered
that UFO's were interplanetary spaceships which were about to make
closer ccntact. To prepare the public for this possibility, 41
previously classified reports were released for publication between
August 1952 end February 1953. These reports contradicted the
earlier official USAF policy of dlismissing the reports as
misidentifications, etc. On the other hand, the CIA regarded
the summer
UFO activity as a threat to national security mainly because the
resulting crowded communications and defence forces involvement
lessened the level of national alertness against possible enemy attack.''

``5. A scientific panel chaired by H.P. Robertson was convened by
the Office of Scientific Intelligence of the CIA during mid-January 1953
for the purpose of recommending future action on the UFO problem.
Briefings were made both by the CIA and USAF. ATIC personnel showed the
then classified two movie films of UFO's and the early results of
statistical analysis of 3200 reports. Because of the vital issues
involved, the panel felt restricted to recommending that the
investigation be continued, but with increased personnel and equipment.
The USAF responded promptly with an instruction to comply with these
recommendations.''

``6. The CIA, however, in a report dated 16 February 1953 showed
a preference to publicly abandon the investigation whilst intensifying
the collection of data. By September 1953 the CIA position had been
largely achieved with Project BLUE BOOK reduced from a staff of ten
qualified personnel operating at a top secret level to a virtually
inactive project involving one airman. The investigating component
had been transferred to the 4602nd Air Intelligence Service Squadron [[(AISS)]]
which was trained in rapid intelligence procurement and reported to
Air Defence Command and USAF Intelligence Washington rather than
BLUE BOOK. Direct access between the 4602nd AISS and all USAF units
was authorized by AFR 200-2 [[~\cite{AFR200-2}~]] whereas previously this privilege had
been given to BLUE BOOK. Although only the airman (first-class)
remained in September 1943, BLUE BOOK was later built up to one
officer, one sergeant, one secretary, and a part-time consultant
Dr J. Allen Hynek, staying at about this level until it was closed
down in December 1969. During this time BLUE BOOK served mainly
as a means of supplying unclassified summaries of UFO identifications
to the public, and did not form a vital link in collection or
serious analysis.''

%\end{svgraybox}
%
%
%%\subsection{Post Blue Book}
%%\label{2023-UFO-part-Perception-types-USA-pbbera}
%
%The Australian intelligence analysis of 1971 characterized the
%Project Blue Book situation as follows~\cite{TurnerAustralia1971}:
%\begin{svgraybox}
``7. Control of public awareness of the UFO situation was
tightened by the issuing of JANAP~146  in 1953
[[JANAP~146C~\cite{JANAP_146C} in 1954]]
which prohibited
service personnel from discussing UPF's by threatening defaulters
with up to 10 years gaol and up to a {\$}\,10\,000 fine. When service
personnel resigned or retired, however, it was possible to reveal
USAF attitudes or opinions even if actual data was still restricted.
In this way many Intelligence Officers associated with the UFO problem,
including Major D. Fournet, who was BLUE BOOK Project Officer at the
Pentagon until late 1952, Captain E. Ruppelt, who headed Project
GRUDGE and Project BLUE BOOK until September 1953, and Admiral
Hillenkoetter, who directed CIA from its inception until October 1950,
on retiring from these services, all publicly stated that the US
Government knew UFO's were extraterrestial but was withholding
this fact from the public.''

``8. When the National Investigation Committee on Aerial Phenomena
(NICAP) was formed in 1956 to counter the publicly suppressed USAF
investigation of UFO's, the first Chairman was Admiral D.S. Fahrney,
who had directed the Navy's guided missile programme from its
inception. Apart from Admiral Hillenkoetter, Major Keyhoe, and
Major Fournet, other Directors have included Rear Admiral H.B. Knowles,
General A. Wedemeyer and Col J.J. Bryan (who was a special assistant
to the Secretary of the Air Force). To reduce the effect of these
and similar defections from official policy after retirement, the
revised JANAP~146E
[[JANAP~146E~\cite{JANAP_146E} in 1977]], passed in 1960, made it an offence under the
Espionage Act if data on UFO's were revealed.''

``9. The change in style of USAF reporting before and after the
Robertson panel meeting is clearly indicated in the Project BLUE BOOK
Special Report No. 14 [[~\cite{BlueBookSR14-1955}~]]. The body of the report prepared between
March 1952 and early 1953, although biased in favour of a natural
explanation for UFO's, nevertheless showed mathematically that the
evidence favoured an explanation that was scientifically unknown.
This section of the 316 page report was not released to the public
other than as a copy to be consulted, assuming the reader knew of
its existence. Public distribution was made, however, of a so-called
``summary,'' which in fact did not summarise, nor scarcely
allude to the 1947--52 data, but concentrated on 1953--55 reporting,
which was clearly designed to reduce the residual unknowns to an
insignificant number, no matter how senseless the identification
became.''

``10. Within the body of the difficult-to-obtain report there
is an interesting diagram. The product of the estimated observer
reliability and the report reliability became the sighting
reliability. The percentage of reports that had to be registered
as `unknown' (i.e., incapable of being even approximately identified
as a known object) increased as the sighting reliability improved.
Conversely, the percentage listed as `insufficient information'
decreased with improving reliability.''

\begin{center}
\setlength\tabcolsep{6pt}
\begin{tabular}{ l l l r@{}l}
\hline\noalign{\smallskip}
Sighting&Number of&Unknown (\%)&\multicolumn{2}{l}{Insufficient}\\[-1pt]
reliability&reports& & \multicolumn{2}{l}{information (\%)}\\\noalign{\smallskip}\hline\noalign{\smallskip}
Poor & 435         &  16.6  &     21.&4 \\
Doubtful & 794     &  13.0  &    14.&0  \\
Good & 757         &  24.8  &    3.&6   \\
Excellent & 213    &  33.3  &   4.&2    \\\noalign{\smallskip}\hline\noalign{\smallskip}
\end{tabular}
\end{center}




 ``11. Throughout the years of the UFO phenomenon, there has been
a persistent form of official pronouncements which state that the
percentage of unknowns would be reduced if more data were available.
The above table contradicts that statement. Reports of excellent
reliability generally stem from astronomers, pilots, scientists,
surveyors, meteorologists, radar operators, etc., complete with
instrumented values and accurately detailed accounts. The
introduction of good reliable reporting prevents the ready prosaic
interpretation. In all probability the overall average percentage
of unknowns (19.7{\%}) would have been substantially increased if the
data had been more reliable.''

``12. Project BLUE BOOK consultants statistically tested the
unknown object population to determine the likelihood that it was
similar to the population of identified objects and found that the
probability was less than one in $10^{28}$ (i.e., using the American
system, the odds were ten thousand trillion trillion to one
against the unknowns being the same as the knowns). Since the
consultants had arbitrarily called all green fireballs and short
duration (i.e., less than five seconds) night-time sightings as
known astronomical objects, there was an undue preponderance in
that category. Hence, assuming that no astronomical objects were
left in the unknowns, the statistical tests were repeated with
astronomical identifications removed. The odds were reduced to
ten trillion trillion to one. The analysts could not find a way
to reduce those odds sufficiently further to warrant additional
testing, and irrationally considered the results to be `inconclusive'.''

``13. While PROJECT BLUE BOOK endeavoured to reduce the official
number of unknowns---in 1957 they claimed only 14 out of 1006
sightings remained unidentified---the covert programme expanded
considerably. The government agency (almost certainly the CIA) that
had been collecting data on UPO performance and propulsion methods
during 1948--52 presumably influenced US governmental funding of
certain advanced projects. One project was the Canadian Avro saucer.
A drawing of this saucer released in October 1955, showed a typical
flying disc as described in many UFO reports. The Secretary of the
Air Force, D.A. Quarles, appeared moderately confident that such
a vehicle would be successfully developed by the U.S.''
\end{svgraybox}


\subsection{Futile attempts toward breakaway gravity research starting in 1955}
\label{2023-UFO-part-Perception-types-USA-bapbbera}

Turner's analysis of 1971 continues with what he calls ``astounding'' investments made by the US into anti-gravity research.
During the early years of his career, in 1956, Turner came across the US anti-gravity project through a note that was posted on a board at Harwell,
which was Britain's inaugural Atomic Energy Research Establishment, according to some biographical notes by   Dominic McNamara and Bill Chalker~\cite{Turner-bio-Chalker}:
``The opinion of staff at Harwell was that this was odd, as we didn't know what gravity itself was, let alone researching anti-gravity.''
This research faced an insurmountable explanation trap~\cite{TurnerAustralia1971}:
\begin{svgraybox}
``14. A more astounding decision on the part of the U.S. Government
was to allocate considerable funds to investigate gravity and a
means of controlling gravity. Despite the fact that science had
not attained a level of competence to deal with either gravity or
anti-gravity problems and the only theory that might be applicable
was Einstein's Unified Field Theory which was still incomplete at the
time of his death, the U.S. chose to support six universities and
government agencies in an all-out drive to conquer the problem.
It is significant that at this time the current theories on UFO
propulsion were a mixture of gravity control and electro-magnetic
propulsion.''

``15. During 1955, because insufficient staff could be recruited
for the project, recourse was made to an urgent appeal for theoretical
physicists and mathematicians from AERE Harwell, U.K. The six
Gravity Research Centres being established were at the Institute for
Advanced Study (Princeton, N.J.), Princeton University, University
of Indiana, Purdue University Research Foundation, University of
North Carolina and the Massachusetts Institute of Technology
through the (Roger Babson) Gravity Research Institute (New Boston
N.H.). The latter institute is a non-profit organization founded
in 1949 with George M. Rideout as President. It was believed that
to make a gravity motor, a gravity differential was required which
necessitated the discovery of an insulator, deflector or absorber
of gravity. By 1955, 485 essays had been written on this subject
and awards totalling {\$}\,10\,600 made for original contributions.''

``16. The scientists involved included Teller from the University
of California, Oppenheimer and F.J. Dyson of the Institute of
Advanced Studies, J.A. Wheeler and Richard Arnowitt of Princeton,
Vaclay Hlavaty of University of Indiana (who had worked with
Einstein in Prague) and Stanley Deser. The objective was to control
gravity. During 1955 the following firms entered into gravity
and/or electromagnetic programmes: Glenn L. Martin Aviation Co.
(specifically Dr B. Heirn from Goettingen University and Dr P.
Jordan from Hamburg University), Convair of San Diego, Bell Aircraft
of Buffalo, Sikorsky Division, Lear Inc. of Santa Monica, Clarke
Electronics of Palm Springs, California, and Sperry Gyroscope
Division of Great Neck, Long Island, N.Y.''

``17. Such an intensive onslaught on the gravity enigma was
entirely irrational from the standpoint of conventional science,
and can only be rationalized within the context of a firm belief
that UFOs were real and that the intelligences behind them knew
how to control gravity. The drive to harness this power before
the USSR could do so would be a strong incentive for the U.S.
Government to fully support an anti-gravity programme. By 1966,
46 separate projects of this nature were being financially supported,
33 of which were under the supervision of the U.S. Air Force.
Although details of most of these projects have been kept classified
it would appear that generally they have not been successful. Work
on gravitational waves by J. Weber and his associates under USAF
Cambridge Research Laboratory jurisdiction has been reported
fairly extensively since 1966.''
\end{svgraybox}

Perhaps Turner's observations and suspicions were partly stimulated by the private initiative
of two wealthy eccentric businessmen -- Roger Babson and Agnew Bahnson.
They envisioned and privately financed research into the ``shielding of gravity''
or some form of anti-gravity~\cite{Kaiser_2018}.
Babson's Gravity Research Foundation, founded in 1948, sponsored annual ``Essays on Gravitation'' competitions.
The winners included researchers like Stephen Hawking, who later became well-known scientists.
Bahnson played an instrumental role in the founding years of the Institute of Field Physics, Inc.
at the University of North Carolina, Chapel Hill, in 1955~\cite{Rickles_2021},
which hosted the 1957 Chapel Hill Conference on
``The Role of Gravitation in Physics''~\cite{Rickles2011Feb}.

\subsection{Condon Committee, 1966--1968}
\label{2023-UFO-part-Perception-types-USA-CC}
\index{Condon Committee}
\index{Condon Report}

Once more, it is important to keep in mind that, based on my hypothesis, investigations of UFOs following Project Grudge were primarily for the purpose of educating the public. They were not conducted with the intention of fully understanding the UFO phenomenon. Instead, they aimed to create the illusion that the military and intelligence community were taking the phenomenon seriously.

%As time passed the small group of ``new'' people that constituted the Project Blue Book team still collected UFO sightings.
%By 1966, the US Air Force had been examining UFO reports for almost 20 years, with over 10,000 cases being investigated.
%It was drowning in data. With regards to the perception of the subject not much had changed, as controversies lingered on.
%It appears that at this point the Air Force sought to put an end to the costly and time-consuming task of collecting and analyzing UFO evidence.
%It is possible that the Air Force came to the conclusion that it was necessary to conduct a definitive investigation of UFOs, or at least one that would justify discontinuing Project Blue Book.

As the years passed, the small team working on Project Blue Book---consisting of only one officer,
a sergeant, and a secretary~\cite{Condon-report,Condon-report-Bantam,Condon-report-Dutton,BibEntry2023Jan}---continued to collect
an increasing amount of data on UFO sightings. By 1966, the US Air Force had gathered a vast amount of information from nearly 20
years of investigating over 10\,000 cases.
Despite all of this,
public perception of UFOs had not greatly evolved, and controversies surrounding the topic persisted.
It appears that the Air Force, faced with the daunting task of collecting and analyzing an ever increasing amount of data,
sought to put an end to the futile, tedious, and time-consuming endeavor.
They may have believed that conducting a final, definitive investigation or one that
could legitimize ending Project Blue Book, would be the solution to resolve the issue.

Finding a reputable ``lead'' scientist, as well as an academic institution to conduct a study on UFOs, was
difficult. The US Air Force (USAF) struggled with this search for ``suitable and willing candidates''
in the spring and summer of 1966.
They needed a renowned academic for the job and approached well-known scientific institutions such as
the Massachusetts Institute of Technology, but none of them were willing to undertake such an effort. Meanwhile, James E. McDonald, an atmospheric physicist from the University of Arizona, was actively campaigning to secure the contract. McDonald was well respected by his peers but faced a major drawback, as his not-so-hidden opinions went against the Air Force's intentions: McDonald already strongly believed that some UFOs were of extraterrestrial origin~\cite{Klass2019Apr}.

In 1966, the University of Colorado agreed to undertake a study on UFOs with funding from the Air Force,
led by physics professor Edward Uhler Condon. The project was met with mixed reactions from the university faculty,
but was ultimately accepted, possibly due to the Air Force's involvement.
Condon, a well-known public figure and science functionary, had previously directed the National Bureau of Standards
and held positions as president of the American Physical Society and the American Association for the Advancement of Science.
On Condon's behalf, Robert J. Low, an assistant dean of the university's graduate program,
served as coordinator, with David Saunders of the pro-extraterrestrial hypothesis,
National Investigations Committee on Aerial Phenomena (NICAP)\index{NICAP} and astronomer Franklin E. Roach as coprincipal investigators.
Saunders and NICAP would eventually leave the study team~\cite{Saunders1968Jan}
(aka ``Dr. Norman Levine and Dr. David Saunders had been summarily fired by Dr. Condon''~\cite[Chapter~11, p.~199]{CraigCondon1995}).

While it is speculative to assume that both Condon and Low were specifically chosen for the task and had a ``hidden agenda'' according to the Air Force's intentions, it is worth noting that Low and Condon appeared to have preconceived views on UFOs before the committee began its work. On August 9, 1966, Low wrote a memo in which he reassured two University of Colorado administrators that there was nothing to fear in terms of ridicule or negative repercussions from the scientific community, which could harm the University of Colorado's reputation~\cite[pp.~33,34]{Hoyt2000Apr}:
\begin{svgraybox}
``The trick would be, I think, to describe the project so that, to
the public, it would appear to be a totally objective study but, to the scientific community,
would present the image of a group of nonbelievers trying their best to be objective but
having an almost zero expectation of finding a saucer.''
\end{svgraybox}
Low's memo to E. James Archer, Dean of the Graduate School, and Thurston E. Manning, Vice President and Dean of Faculties,
regarding the Air Force proposed UFO study, presents a thorough analysis of the pros and cons and is highly recommended~\cite[pp.~33,34]{Hoyt2000Apr}.
Additionally, highly recommended is ``UFOs: An Insider's View of the Official Quest for Evidence''~\cite{CraigCondon1995},
a personal account by Roy Craig,
Associate Professor and Coordinator of Physical Science
at the University of Colorado's Division of Integrated Studies,
describing his participation in the Condon study.

During his speech on January 25, 1967, at a Sigma Xi chapter in Corning, New York, Condon, who was known for his breezy and anecdotal style, reportedly expressed preconceived opinions. The Elmira Star-Gazette reported on these opinions, in which he made similar comments:
``UFOs are not the business of the Air Force. [[$\ldots$]]
It is my inclination to recommend right now that the government get out of this business.
My attitude right now is that there is nothing to it.'' Allegedly with a smile, he added:
``But I'm not supposed to reach a conclusion for another year''~\cite{Hoyt2000Apr,Fuller1968}.

The Committee faced additional controversies due to these statements and the loose correlation between case evaluations and conclusions (see later).
The pro-extraterrestrial faction, represented by McDonald and Keyhoe, attempted to undermine the work of the Committee by claiming that the results were biased due to the attitudes of its main proponents.

Regardless, the outcome of the investigation was not surprising, as it was predetermined by the US Air Force, who commissioned the report. The executive summary of the Condon Report~\cite{Condon-report1,Condon-report2,Condon-report3,Condon-report,Condon-report-Bantam,Condon-report-Dutton}, specifically Section 1 titled ``Conclusions and Recommendations'' and authored by Edward U. Condon, confirms this:
\begin{svgraybox}
``[[$\ldots$]] we think that all of the agencies of the federal government, and the private foundations as well,
ought to be willing to consider UFO research proposals along with the others submitted to them on an open-minded, unprejudiced basis.
 While we do not think at present that anything worthwhile is likely to come of such research, each individual case ought
to be carefully considered on its own merits.

This formulation carries with it the corollary that we do not think that at this time the federal government ought
to set up a major new agency, as some have suggested, for the scientific study of UFOs.
This conclusion may not be true for all time. If, by the progress of research based on new ideas in this field,
it then appears worthwhile to create such an agency, the decision to do so may be taken at that time.''

[[$\ldots$]]

``Our general conclusion is that nothing has come from the study of UFOs in the past 21 years that has added to scientific knowledge.
Careful consideration of the record as it is available to us leads us to conclude that further extensive study of
UFOs probably cannot be justified in the expectation that science will be advanced thereby.''

[[$\ldots$]]

``The question remains as to what, if anything, the federal government should do about the UFO reports it receives from the general public.
We are inclined to think that nothing should be done with them in the expectation that they are going to contribute to the advance of science.

This question is inseparable from the question of the national defense interest of these reports.
The history of the past 21 years has repeatedly led Air Force officers to the conclusion that none of the things seen,
or thought to have been seen, which pass by the name of UFO reports, constituted any hazard or threat to national security.''



[[$\ldots$]]

``It is our impression that the defense function could be performed within the framework
established for intelligence and surveillance operations without the continuance of a special unit such as Project Blue Book,
but this is a question for defense specialists rather than research scientists.''



[[$\ldots$]]

``The subject of UFOs has been widely misrepresented to the public by a small number of individuals who have given sensationalized presentations
in writings and public lectures.
So far as we can judge, not many people have been misled by such irresponsible behavior,
but whatever effect there has been has been bad.''
\end{svgraybox}

The report concluded that public attention should be directed to the UFO ``miseducation in our schools.'' After the report's publication, a scientific dispute broke out about its consistency, methodological adequacy, and validity. A review panel of the National Academy of Sciences expressed its satisfaction with the report and certified it to be ``adequate to its purpose'' and ``in accordance with accepted standards of scientific investigation''~\cite{PNAS-Condon-1969}.

Both sides argue that Condon's involvement was not substantial enough to conduct a deep analysis of the investigated cases, and Low lacked the necessary experience to lead such a complex investigation effectively~\cite{Klass2019Apr}. The connection between the first two sections, ``Conclusions and Recommendations'' and ``Summary of the Study,'' to the main body of the report appears spurious. Most case studies were conducted by junior staff, with little participation from senior staff and the director. Notably, the analysis of evidence indicates substantial differences between the findings of the project staff and the director. While both the director and staff were careful not to make definite statements, the staff tended to focus on difficult cases and unresolved issues, while the director stressed the difficulty of further research and the likelihood that no new scientific information could be obtained~\cite{Sturrock-Condon-87}.

It has been argued from both sides of the debate about the extraterrestrial origins of UFOs that,
despite all of the Condon Committee's efforts to find prosaic explanations, those who prepared the Condon Report
ended up with about a dozen, approximately 15{\%},
of their cases in their ``Unexplained'' category~\cite{Klass2019Apr,Hynek_1969,Page_1969,McDonald-Condon-69,Sturrock-Condon-87}.
This alleged deficiency of the conclusion and summary parts of the Condon Report was clearly expressed in an appraisal of the
UFO problem by the American Institute of Aeronautics and Astronautics (AIAA)~\cite{AA-Condon-1970}.
In 1967, the Technical Committee on Atmospheric Environment and the Technical Committee on Space and Atmospheric Physics of AIAA
jointly formed a UFO Subcommittee, which published a discussion of the Condon Report.
They stated that there existed a fraction of UFO reports, perhaps less than one percent (i.e., less than one out of a hundred),
which might be called ``hard cases.'' Those hard cases have high credibility, as they are:
``observations by multiple independent witnesses of known and reliable background or by multiple independent sensing systems
(reported by multiple independent operators) or both; high abnormality or strangeness,
when no known natural phenomena whatsoever seem to fit the observations.''
\begin{svgraybox}
``Taking all evidence which has come to the Subcommittee's attention into account,
we find it difficult to ignore the small residue of
well-documented but unexplainable cases which form the hard core of the UFO controversy.''

[[$\ldots$]]

``In fact, the Subcommittee finds that the opposite conclusion could have been drawn from its content, namely,
that a phenomenon with such a high ratio of unexplained cases (about 30{\%}
[[in the Condon Report]])
should arouse sufficient scientific curiosity to continue its study.''

``The issue seems to boil down to the question:
Are we justified to extrapolate from 0.99 to 1.00, implying that if 99{\%}
of all observations can be explained, the remaining 1{\%}
could also be explained; or do we face a severe problem of signal-to-noise ratio
(order of magnitude 0.01)?''
\end{svgraybox}

These concerns of the AIAA align with my own impression of the Air Force's Project Blue Book, which was overwhelmed by a large amount of data with a weak signal and significant noise, as reported by the Condon Report, which was staffed with only one officer, a sergeant, and a secretary. This made it difficult to extract meaningful information from the incoming stream of alleged UFO sightings. However, it remains unclear whether this was intentional or simply a way to save taxpayers' money while communicating to UFO enthusiasts and the interested public that the ``Air Force cared.'' Regardless, the Condon report ultimately brought these official efforts to an end.


\subsection{Turner's view of the Condon Report and the end of Project Blue Book}
\label{2023-UFO-part-Perception-types-USA-avecreopbb}

Turner's analysis of 1971
presents the following evaluation of the finalization of any official
US efforts to cope with flying saucers~\cite{TurnerAustralia1971}:
\begin{svgraybox}
``18. During August of 1965 Project BLUE BOOK received 262
reports which was about six times the average number for a month
and was twice any previous month since November 1957. On
28 September 1965 Maj. Gen. LeBailly, Director of Information,
formally requested the Air Force Scientific Advisory Board to
review Project BLUE BOOK. The review suggested that the limited
Project BLUE BOJK staff and the official investigating officers
did not possess the technical competence to properly identify the
phenomena and that university teams should be appointed to
investigate selected sightings. This conclusion was supported by
the House Armed Services Committee which met on April 5th 1966
in the shadow of a public furore consequent to the USAF identifying
the well-publicized Michigan sightings as being swamp gas. The
Colorado University was selected for the task, and Dr. Edward U. Condon
appointed to lead the project with an initial allocation of {\$}\,313\,000
later raised to {\$}\,525\,000.''

``19. The Colorado project became discredited when Dr Condon
stated publicly on 25 January 1967 that `my attitude right now
is that there is nothing to it, but I'm not supposed to reach a
conclusion for another year'. The revealing of a memorandum
outlining a method to trick the public, combined with a general
dissatisfaction at Condon's biased attitude, led to the dismissal
and resignation of most of the staff after most of the investigations
had been made but not completely written up. The final report of 965
pages lacked coherence. Condon's conclusions were at variance with
individual staff conclusions, although only Condon's conclusions were
publicised. As a result of the Condon report, USAP closed down
Project BLUE BOOK shortly before the American Association for the
Advancement of Science held a special meeting to counter-act the
effect of the Condon report. The Chairman of the Special Committee,
Dr Thornton Page, was one of the signatories to the Robertson report.''

``20. Dr J. Allen Hynek, scientific consultant to Project Blue
Book 1948--69, began his association with a conviction that all
sightings could be conventionally explained. Even though doubts
grew in his mind, he found himself obliged to support official USAF
public policy. Since 1966, however, he has become more outspoken
against the USAF attitude and has assisted to convene both
congressional hearings and scientific symposia on the subject.
Although initially supporting the Condon Committee, he became
disillusioned and critical of it with the passage of time. It is
quite clear that Dr Hynek along with many other reputable scientists
do not accept the USAF explanation of misidentification, hysteria, or
hoax.''
\end{svgraybox}


\subsection{Ad hoc initiatives after Project Blue Book}
\label{2023-UFO-part-Perception-types-USA-ahpbbera}

Since the cessation of Project Blue Book, initiatives by interested individuals
such as John B. Alexander's Advanced Theoretical Physics Working Group~\cite{Alexander2023Jan,Omega_Point2022Nov},
Laurance S. Rockefeller's Initiative~\cite{Berliner2000Jun}, and Steven M. Greer's Disclosure Project~\cite{Greer-dp}
have erupted in an ad hoc and ad hominem manner time and again.

More recently, the Advanced Aerospace Weapon Systems Applications Program (AAWSAP) was started by Harry Reid (D-Nevada),
then the Senate Majority Leader, at the initiative of a Nevada-based journalist, George Knapp, and Robert Bigelow, a Nevada businessman and governmental contractor. The program, which studied unexplained aerial phenomena (UAP), was supported by Senators Ted Stevens (R-Alaska) and Daniel Inouye (D-Hawaii) and began in 2007 with a budget of {\$}22 million for its five-year operation. In January 2008, Bigelow established Bigelow Aerospace Advanced Space Studies (BAASS), a company that undertook the AAWSAP contract for the DIA~\cite{Lacatski-2021}. On June 24, 2009, Senator Reid attempted to create a ``Restricted Special Access Program (SAP) with a Bigoted Access List for specific portions of the AATIP,'' but this initiative failed~\cite{Lacatski-2021}.

As far as I can resolve the connections, the Advanced Aerospace Threat Identification Program (AATIP) was a colocated program associated with AAWSAP in the Defense Intelligence Agency (DIA), the intelligence agency and combat support agency of the United States Department of Defense (DOD). The connection between AAWSAP and AATIP is not entirely clear to me. Additionally, some personal affiliations are not entirely certain. I conjecture that James T. Lacatski was the first Defense Intelligence Agency program manager of AAWSAP. Luis Elizondo was involved in AATIP and had a leadership role, possibly as a successor of Lacatski, and ``investigating UAPs as the head of AATIP''~\cite{ReidLetterApril2021}.

AAWSAP/AATIP was a rather ambitious, maybe ``reconnaissance/pathfinder'' project---started by outsiders (Bigelow) and insiders (Reid, Christopher Mellon)---to ``get as far as possible to the bottom'' of the Pentagon's and defense contractors' knowledge of UFOs and to make it public. They also studied a farm in Utah that allegedly exhibited strange performances~\cite{Lacatski-2021}. AATIP produced DIA products listed after a FOIA request~\cite{FOIA-AATIP-2019Jan}, including reports entitled ``Advanced Space Propulsion Based on Vacuum (space time metric) Engineering, Dr. Hal Puthoff, EarthTech International,'' ``Invisibility Cloaking, Dr. Ulf Leonhardt, Univ. of St. Andrews,'' ``Traversable Wormholes, Stargates, and Negative Energy, Dr. Eric Davis, EarthTech International,'' ``Field Effects on Biological Tissues, Dr. Kit Green, Wayne State University,'' and ``Space-Communication Implications of Quantum Entanglement and Nonlocality, Dr. J. Cramer, Univ. of Washington.'' These are available from the DOI FOIA reading room~\cite{DOI-FRR2023Jan}.

Another alleged attempt to initiate the public to the alleged secret machinations of aliens and the USA's reaction to it comes from Tom DeLonge,
who has created a group of advisors and scientists and published two book series on ``secret machines'' with Andrew James Hartley and Peter Levenda~\cite{DeLongeHartley-SM1,DeLongeLevenda-Gods,DeLongeHartley-SM2,DeLongeLevenda-Men}.


\section{United Kingdom}
\label{2023-UFO-part-Perception-types-UK}

\index{Rendlesham Forest incident}

The situation in the United Kingdom is not entirely different from that in the USA. Officially, the UK conducted two investigations into flying saucers: the Flying Saucer Working Party and the Condign Report. The following is a very incomplete description of the UK's efforts to cope with the phenomenon. One important case that took place in Suffolk, England, the Rendlesham Forest incident in 1980, is mentioned in Section~\ref{2023-UFO-chapter-History--1953-2016-rf1980}.


\subsection{Flying Saucer Working Party, 1950--1951}
\label{2023-UFO-part-Perception-types-UK-flwp1950-1}
\index{Flying Saucer Working Party}

A team dedicated to studying flying saucers was established in 1950 and worked closely with the CIA. The Ministry of Defence (MOD) denied the existence of this official study of UFOs until minutes of the meetings were discovered at the National Archives in 1999. However, the report produced by this committee, known as the ``Holy Grail'' to those who believe in a cover-up of UFO evidence, could not be found and was said to have ``not survived the passage of time.'' Despite this, the report is considered an important piece of the history of the Cold War.

The study of ``Flying Saucers'' was initiated in 1950 by Sir Henry Tizard, a trusted scientific advisor of Churchill, due to pressure from a pro-saucer newspaper campaign supported by Lord Louis Mountbatten and other high-ranking officials who believed the saucers to be advanced craft from outer space. The study, called the Flying Saucer Working Party, was led by members of the Technical Intelligence branches of the Air Ministry, Admiralty, War Office, and Ministry of Defence (MOD) and held its first meeting in 1950. The RAF and Royal Navy were asked to submit sighting reports for investigation.

The Flying Saucer Working Party produced their final report entitled ``MoD DSI/JTIC Report No 7 Unidentified Flying Objects''~\cite{FSWP1951} and stamped SECRET in 1951. Although this document was lost, it has been recovered by UFO researcher David Clarke~\cite{ClarkeFSWP}. The report recommended debunking sightings and implementing a tight security clampdown to prevent puzzling cases from reaching the public.

When the Flying Saucer Working Party produced their final report in 1951~\cite{FSWP1951},
the CIA's chief scientist
and Assistant Director of the CIA's Office of Scientific Intelligence (OSI), Marshall H. Chadwell, was present~\cite{Haines-CIA-UFO}. This is also reflected by parallels between the UK Flying Saucer Working Party and the USA Robertson Panel mentioned in Section \ref{2023-UFO-part-Perception-types-USA-RP}, which was established by the CIA in 1952--1983 with a similar goal of debunking UFO sightings and ensuring that UFOs posed no threat to national security and did not escalate into public hysteria.
\index{Chadwell, Marshall H.}

\subsection{Condign Report of 1996}
\label{2023-UFO-part-Perception-types-UK-cr1996}
\index{Condign Report}

The following paragraph is based on a podcast with and posts by David Clarke~\cite{Clarke2022Nov,BibEntry2018May},
as well as on an article by Clarke and Anthony~\cite{Clarke-Anthony-IUR-Condign} on the Condign Report.

Just as in the UK Flying Saucer Working Party report of 1951~\cite{FSWP1951} relates to
(correlates with) the US Durant Report of the Robertson Panel\index{Robertson Panel} of 1953~\cite{RobertsonPanelDurantReport},
so does the UK Condign Report of 1996~\cite{CondignReport} relate to
(correlate with) the US Condon report of 1968~\cite{Condon-report,Condon-report-Bantam,Condon-report-Dutton,BibEntry2023Jan}.
(The name ``Condign''---meaning ``a severe and well deserved punishment''---was chosen with the help of a randomization algorithm to not signify anything;
its similarity with ``Condon'' is purely accidental~\cite{Clarke2022Nov}.)

One difference between the Condon and the Condign reports was that, whereas
the Condign Report of 1996 was a classified study commissioned by the British government and directed toward its own military and intelligence community,
the earlier unclassified and highly marketed Condon report was an effort to eliminate Project Blue Book and the public attention of UFOs as quickly and for as long as possible.
Both formulated an official ``surface'' stance---a manufactured consent~\cite{Herman2002Jan}---on the subject but directed toward different recipients.

Here is the official reason for the Condign report, as laid out by the Ministry of Defence (MoD)~\cite{CondignReport}:
\begin{svgraybox}
``During a policy review in 1996 into the handling of Unidentified Aerial Phenomena sighting reports received by the Ministry of Defence,
 a study was undertaken to determine the potential value, if any, of such reports to Defence Intelligence.
Consistent with Ministry of Defence policy, the available data was studied principally to ascertain whether
there is any evidence of a threat to the UK, and secondly, should the opportunity arise, to identify any potential military technologies of interest.''
\end{svgraybox}
Both the Condon and the Condign reports concluded that the majority of UFO sightings could be explained as misidentifications of natural phenomena or man-made objects,
and that there was no evidence of a threat to national security.

Three executive branches were involved in the Condign investigation: the Royal Air Force (RAF) ``Ground'' section, which received data from radar stations and the Airborne Warning and Control System (AWACS); the DI55 partition of the Defence Intelligence Staff (DIS), now known as Defence Intelligence (DI), which focuses on gathering and analyzing military intelligence; and the MoD's civilian Secretariat (Air Staff), abbreviated as Sec(AS), also known as MoD's ``UFO desk.'' Allegedly, DI55 believed that civilians were ``leaky'',
that is, more likely to disclose classified information, leading to the decision to exclude Sec(AS) from the Condign investigation~\cite{BibEntry2018May}.

According to Davies, the officially undisclosed lead author of the Condign Report was Ron Haddow, a government scientist~\cite{Clarke2022Nov}. Haddow's task was to assess any potential threats to the UK and the possibility of acquiring new technology, rather than investigating claims such as alien abductions~\cite{BibEntry2018May}.

An executive summary of the Condign Report was provided in a letter dated December 4, 2000~\cite{Clarke2022Oct}, in line with the ``official UK/US position'' mentioned earlier. The letter contains excerpts from the report's summary:
\begin{svgraybox}
``2. The main conclusion of the Study [[Condign Report]] is that the sighting reports provide nothing of value to the
DIS in our assessment of threat weapon systems. Taken together with other evidence, we believe that
many of the sightings can be explained as mis-reporting of man-made vehicles; natural but not
unusual phenomena, and natural but relatively rare and not completely understood phenomena. It is
for these reasons that we have taken the decision to do no further work on the subject and will no
longer receive copies of sighting reports.''
\end{svgraybox}
The executive summary expressed interest in the Condign Report's findings regarding (i) flight safety staff and (ii) possible novel phenomena associated with plasma formations, which may have potential applications to novel weapon technology.

As of the time of writing, the Ministry of Defense appears to have misplaced their copy of the Condign Report. Additionally, no destruction certificate signed by a senior Ministry of Defense official has been discovered~\cite{Clarke2022Nov}. As a former civil servant in Austria's Ministry of Research, I can sympathize with MoD staff who may be struggling with this situation.
[Civil servants in Austria have a saying: ``etwas ist in Versto{\ss} geraten,'' meaning that
``something has fallen into the pile'' of forget-me(-not).]

\subsection{Discrete enquiries for the Duke of Edinburgh}
\label{2023-UFO-part-Perception-types-UK-dede}

From 1953 to 1956, Air Marshal Sir Beresford Peter Torrington Horsley served as Equerry to Prince Philip, the ``Royal Flying Saucerer''~\cite{Clarke-Philip-TRFS}, and, as such, was part of the British Royal Household.
In an amazing chapter entitled ``Visitors''~\cite[Chapter~10]{Horsley1998Jan}, he describes discreet investigations into flying saucers.

Horsley suggests four general categories of reported sightings: those from (i) objective observers with no self-interest, (ii) a growing body of people promoting sightings for mercenary reasons or self-advertisement, (iii) observers suffering from flying saucer psychosis or mental illness, and finally (iv) practical jokers.
He describes a great number of sightings of the first category, some of which he reported to the Duke of Edinburgh.


One took place on November 3, 1953, when two Royal Air Force officers, Flying Officer T.S. Johnson and Flying Officer G. Smythe,
reported seeing a circular, brightly illuminated object flying at over 1000 miles an hour in the London Metropolitan Sector.
Despite the official explanation that it was likely to be a balloon,
the officers believed it was not, due to the object's high speed and unusual behavior.
The officers were medically examined and found to be fit for operational flying,
and there were no known high-speed experimental aircraft in the area at the time.
On the same day, an object was tracked
by radar behaving in a similar manner, moving from stationary to high
speed over a considerable height band. As Horsley had not interviewed the operators, he
discounted this report as merely corroborative evidence.

Another ``Little Rissington'' case is a rare incident where a visual and radar report of a UFO matched up.
The incident occurred when an instructor and his pupil-instructor were flying a tandem-seater Meteor Mk VII high above Little Rissington,
a central flying school responsible for training flying instructors.
The front pilot observed a circular object dead ahead, filling about three inches of his windscreen,
which he initially assumed to be an oxygen failure.
 The instructor took over control of the aircraft, turned it through forty-five degrees,
and saw the object, confirming its description with the pupil-instructor.
 They reported the incident to Little Rissington Air Traffic Control who instructed them to approach closer.
They turned again towards the object, and when the circular object filled half their windscreen, it suddenly turned on its side,
emitting an iridescent light around its edges, and climbed away out of sight at great speed.
The object's size could not be estimated as they did not know its distance.
Meanwhile, in Fighter Command, Southern Sector radar reported an unidentified aircraft travelling through the sector.
The Meteor appeared on the radar tube and closed in on the unidentified blip, which then moved rapidly across the screen at an estimated speed of over 1000 miles an hour.
However, the final distance between the two was never measured, so the size of the UFO was never estimated.
A pair of standby fighters were scrambled from Tangmere but never made contact with it.
The incident was unusual as both the visual and radar reports matched up, providing evidence for the existence of an unidentified flying object.
Horsley never interviewed the persons involved, but his narrative stimulated an investigation by David Clarke~\cite{Clarke-Mainbrace-2020Dec,Team2022Sep}
also mentioned in Section~\ref{2023-UFO-chapter-History--1953-2016-brafpst}.



%Horsley also recalls an amazing meeting---maybe a rare kind of fraternization---with ``Janus,''
%who seemed to be at home among earthlings and also appeared to be able to explain a
%lot of the history and agenda of visitors from outer space.
%Janus requested an audience with the Duke of Edinburgh and then, while seeming to be able to read Horsley's thoughts,
%proceeded with a long rant about the context of the ``visitations'' by extraterrestrial craft.
%
%Janus reflected on historical analogies to the advancement of propulsion techniques.
%Considering only the modes of transportation mankind was familiar with just three hundred years ago---feet, horses, carriages, and ships---travel to the Moon was inconceivable.
%It took men of great intellect and vision, such as Leonardo da Vinci, Jules Verne, and Wells, who had the imagination to project their thoughts into the distant future,
%to make the impossible seem possible. Therefore, if we projected our present knowledge into the future, we too might catch a glimpse of what lay ahead,
%maybe even speeds of travel faster than the speed of light.
%However, just as these visionaries were considered cranks in their time, anybody who contemplated space travel might face a similar fate.
%
%Janus also explained that
%the traffic of extraterrestrial beings to Earth was only a small trickle in the vast highways of the universe.
%And despite the fact that Earth was a galactic backwater inhabited by only semi-civilized and dangerous humans,
%explorers still wished to find out more about it, much like people on Earth who travelled to uncomfortable and dangerous places.
%To ensure the safety of their probes and keep their existence hidden from most of Earth's population,
%some extraterrestrial vehicles were manned by beings with highly developed mental, such as telepathic, powers.
%These beings only made contact with select individuals where secrecy could be maintained and were careful
%not to interfere with the natural development of life on Earth.
% They had been studying the planet for a long time and were equipped with advanced medical technology to allow their bodies to operate normally on Earth.
%Contact with higher forms of life and intelligent beings had to be conducted with great secrecy and responsibility,
%as the dangers of fear and misunderstanding were high.
%These observers did not use weapons and relied solely on their mental powers to protect themselves.
%In societies where secrecy could be maintained, contact could be made more easily, particularly in England and America,
%but not in police states or dictatorships.
%The basic principle of responsible space exploration was to not interfere with the natural development and order of life in the universe.
%It was important for Earth to learn its responsibilities for the preservation of life before it embarked upon deep space travel.
%
%After this meeting, Horsley wondered whether Janus could be part of some kind of plot. If this were the case, it would have been his duty to report the meeting to the security authorities, especially if it had anything to do with the Royal Family. However, when he tried to contact Janus again, Janus himself and all the contacts surrounding him seemed to vanish into thin air---``the curtain had dropped.''


Horsley recounts a meeting with someone who was said to be named ``Janus'' and who seemed to him knowledgeable about the history and agenda of extraterrestrial visitors.
Janus believed that the Earth was a small part of the universe and that extraterrestrial traffic to Earth was minimal.
These extraterrestrials had developed means of superluminal travel.

Despite the Earth allegedly being a galactic backwater inhabited only by semi-civilized and dangerous humans,
explorers still wished to find out more about it, much like people on Earth who travel to uncomfortable and dangerous places.
Therefore, interference with life on Earth was kept to a minimum.
Some extraterrestrial vehicles were manned by beings with advanced mental powers who made contact with select individuals.
These beings did not use weapons and were equipped with medical technology to operate on Earth.
Horsley suspected that Janus might be part of a plot, but when he tried to contact him again, Janus and his contacts had disappeared.


\section{Former USSR, Russia, and Ukraine}
\label{2023-UFO-part-Perception-types-Russia}

The situation in Russia regarding UFOs appears to be confusing, as reliable information is scarce.
Two sources provide insight into the subject: Vall\'ee's
``UFO Chronicles of the Soviet Union. A Cosmic Samizdat''~\cite{Vallee1992Feb}, and a 1968 RAND study which suggests that
``Even the Soviets, who previously refused even to discuss the subject now admit to having a study group with good qualifications''~\cite[p.~6]{Kocher-RAND-1968Jan}.

There are also unconfirmed and highly uncertain online resources, such as ``UFO Facts and Documents'' by G.K. Kolchin,
who is allegedly a retired colonel and deputy chairman of the Commission on Anomalous Phenomena of the Geographical
 Society of the USSR~\cite{Kolchin2018Feb}. This resource is briefly outlined in a Reddit post~\cite{TypewriterTourist2023Jan}.
 Additionally, there exist review articles of unknown depth, reliability, and accuracy on UFO research in the former USSR~\cite{Alexeyev97,PlatovSokolov-2000} and in Ukraine~\cite{Gershtein2015,Bilyk2016Sep}.


\subsection{UFOs over Soviet nuclear bases}

The Soviet UFO missile crises\index{Soviet UFO missile crises}
is reviewed by Hastings~\cite[Chapter~23]{Hastings2008Jan} (see also~\cite{ABC-transcript-1994,Huneeus2011Jan,Hvar2012})
as retold in Section~\ref{2023-UFO-chapter-History--1953-2016-sumc}.


\subsection{Bizarre statement by Russian prime minister Dmitry Medvedev}

There is a bizarre anecdote involving the then Russian prime minister, Dmitry Medvedev, making some outlandish remarks while television cameras were still rolling. When a journalist inquired whether the President of Russia is provided with classified information on aliens, upon receiving the briefcase required to initiate the country's nuclear arsenal, Medvedev responded~\cite{Sky2012Dec}:
\begin{svgraybox}
``I tell you the first and last time.
Together with the briefcase with nuclear codes, the president of the country is given a special `top secret' folder.
This folder is entirely devoted to the strangers who visited our planet.''

``The report is provided by the secret special service that handles the control over aliens in our country [[$\ldots$]]
More detailed information on this topic can be obtained from the documentary `Men in Black' [$\ldots$]''

``I will not tell you how many of them are among us because it may cause panic.''
\end{svgraybox}
Note that when referring to ``Men in Black,'' Medvedev might have had the Russian documentary~\cite{Ufodisclosure20162013Feb} in mind.

\subsection{Handwaving and handraising UFO responses}

The Soviet Union reportedly had the ability to ``summon'' UFOs to at least one of their military bases, as recounted in an interview with Major-General Vasily Alexeyev of the Russian Air Force, Space Communications Centre, Moscow, 1997~\cite{Hesemann2000,Uvarov_2000}.

Observers in the Soviet Union utilized a technique to intentionally create the appearance of a UFO by increasing military activity, including the transportation of ``special loads,'' which may have been nuclear devices. This orchestrated activity resulted in a corresponding appearance of a UFO, establishing a correlation between military actions and UFO sightings. As a result, it was concluded that whoever was responsible for the UFOs possessed advanced intelligence and sensitivity to such activities.

At certain testing sites, presumed to be nuclear bases, individuals were able to establish communication with the UFOs beyond simply summoning them. The UFOs typically appeared as spheres, although they occasionally manifested themselves in different forms. Communication occurred through physical signals or gestures. For example, pointing an arm in various directions caused the UFO sphere to flatten in the corresponding direction. Raising arms three times elicited a response from the UFO, causing it to flatten out in the vertical direction three times as well~\cite{Hesemann2000,Uvarov_2000}.

\section{France}
\label{2023-UFO-part-Perception-types-France}

\subsection{GEIPAN}

GEIPAN---Groupe d'Etude et d'Information sur les Phenom\`enes Aerospatiaux Non-identifi\'es
(Group for Study and Information on Unidentified Aerospace Phenomena)---is the official investigation unit of the
French Space Agency (CNES) responsible for the study and investigation of unidentified flying objects (UFOs) and other aerial phenomena. Established in 1977 and based at the CNES headquarters in Paris, GEIPAN conducts research, collects and analyzes data, and provides information to the public on its findings. Additionally, the group maintains the official French government archive of UFO reports and related information.

According to its official front webpage~\cite{Geipan}, GEIPAN has collected 99 ``hard cases'' [in their terminology ``D --
unidentified (after investigation)''] out of a total of 2978 cases, which is just over 3 percent of all cases.
This estimate appears to be consistent with the less than 1{\%}
estimate for ``hard cases'' based on the Condon Report and Project Blue Book of the UFO
Subcommittee of the American Institute of Aeronautics and Astronautics (AIAA) in 1967~\cite{AA-Condon-1970}. Everything else, 97{\%}, appears to be clutter.

Institutions or initiatives such as GEIPAN are repeatedly faced with the challenge of a small signal-to-noise ratio. This makes it difficult and expensive to distinguish the ``wheat from the chaff,'' the signal from the noise and clutter of unnecessary, disorganized, or obstructive items. This creates a sense of chaos, disorientation, and frustration and makes it difficult to find what one is looking for or to focus on the task at hand.

\subsection{COMETA}\index{COMETA}

COMETA, an association comprised of experts in various fields, including physics, life sciences, human sciences, engineering, and senior military officers, published a comprehensive report on UFOs in July 1999 and submitted it to the President and Prime Minister~\cite{COMETA2003Jun,COMETA-English}. According to the report, the possibility of UFOs having an extraterrestrial origin could not be dismissed. One of the conclusions, or rather accusations, of the COMETA report was that the US government was engaged in a major cover-up and disinformation campaign, particularly with respect to the Roswell incident. \index{Roswell incident}

\section{Australia}
\label{2023-UFO-part-Perception-types-Australia}

For information on the Australian situation regarding UFO sightings and ongoing discussions, see Keith Basterfield~\cite{Basterfield1981Jan,Basterfield-Blog}, Bill Chalker~\cite{Chalker-Ozfiles,Swords2012Jul}, and Ross Coulthart~\cite{Coulthart2021Aug}.

In 1971, Oliver Harry Turner,\index{Turner, Oliver Harry}, an Australian nuclear physicist,
made a spectacular attempt to comprehend the UFO phenomenon~\cite{Basterfield-Blog,Swords2012Jul,Chalker22}, including US efforts to study flying saucers, which were later called UFOs~\cite{TurnerAustralia1971}.
I have fully quoted much of the summary part in earlier sections.

After reading Menzel's ``debunking'' book on UFOs~\cite{Menzel_1953}, Turner became interested in the subject and investigated local UFO sightings in Melbourne suburbs. He was later recruited by the Royal Australian Air Force (RAAF) to investigate these sightings and analyze RAAF files on the subject. Turner found physical evidence in one of his investigations and prepared a report on the Dandenong sightings for the RAAF.

After that episode, Turner traveled to the UK to work at Harwell, a British nuclear research establishment. In 1956, a job posting at Harwell sought scientists to work on anti-gravity research in the United States. He and other scientists at Harwell took notice, as anti-gravity appeared to be outlandish~\cite{Turner-bio-Chalker}. Why should the USA be interested in this exotic subject?

Upon his return to Australia, Turner was stationed at Maralinga, a nuclear test site in South Australia. Again, he covered an investigation into a UFO sighting on an ad hoc basis, as one official report states~\cite[p.~74]{Turner-WREMa}: ``Mr. Oliver Harry Turner, Health Physics Officer, who possesses an inquiring mind, made an independent investigation and extensive calculations. He is of the opinion that the light was not the result of a natural phenomenon but caused by an unidentified flying object, either a cone from a satellite or a `flying saucer.'\,''

As he progressed through the defense department, he became a scientific analyst working in the Directorate of Scientific and Technical Intelligence (DSTI) of the Joint Intelligence Bureau (JIB) of the Australian Government's Department of Defense. It has been estimated that during his tenure, Turner devoted most of his time to researching UFOs~\cite{Chalker22}.

In the 1970s, Turner pushed for a UFO investigation team. He presented an analysis stating that the UFO reports were real phenomena with flight characteristics so advanced that only an extraterrestrial origin could be envisaged~\cite{TurnerAustralia1971}. After consulting with the USA, the Australians eventually turned these initiatives down.


The 1971 document by Turner, like many others before the FOIA period, was never intended for public release.
However, thanks to the efforts of Keith Basterfield and Bill Chalker, along with a group of investigators, the Turner files were located,
digitized, and published between 2003 and 2008. We owe them the rediscovery of these files.

As far as I am aware, Turner himself never reported a UFO sighting. Two Australians,
Keith Basterfield~\cite{Basterfield-Blog} and Paul Dean~\cite{PaulDean-blog},
provide a wealth of material through their blogs, including information on Australian cases.
Ross Coulthart is an internationally renowned investigative television journalist and author
who recently concentrated on UFOs and related subjects~\cite{Coulthart2021Aug}.


\section{Canada}
\label{2023-UFO-part-Perception-types-Canada}

I may be totally ignorant of Canadian efforts regarding flying saucers/UFOs/UAPs, but my basic assumption is that the Canadian government was mostly hesitant, if not unwilling, to deal with such issues. They relied on the USA to handle it.

The Smith--Sarbacher--Walker connection, discussed in Section~\ref{2023-UFO-part-Perception-crash-retreivals-SM}, was extemporaneous, ad hoc, and anecdotal rather than systematic, and it was dependent on a single person, Wilbert Brockhouse Smith, who became increasingly involved personally. In my opinion, he discredited his own initial efforts to investigate UFOs.

In more recent times, the closest thing Canada had to an official ``UFO desk''~\cite{Otis2022Mar} was civilian researcher Chris Rutkowski~\cite{Rutkowski2022May}. His surveys were based on UFO reports obtained through interviews with witnesses by the Royal Canadian Mounted Police in the 1980s and 1990s~\cite{Rutkowsky2022Aug}. However, he can no longer access sources from the Department of National Defence and Transport Canada.

%Grant Cameron


\section{China, India and many other states}
\label{2023-UFO-part-Perception-types-China}

I have no knowledge about attempts to investigate sightings and alleged crashes that have occurred in various other regions of the planet. My supposition is that there are efforts to investigate such incidents, but to what extent, I cannot say. For certain countries, there are overviews available~\cite{Swords2012Jul}.



\section{Germany}
\label{2023-UFO-part-Perception-types-Germany}

Just as a reference point, I mention
Illobrand von Ludwiger's ``Best {UFO} Cases--{E}urope''~\cite{VonLudwigerNIDS}
and
Andreas  M\"uller's ``{D}eutschlands {UFO}-Akten''~\cite{Muller2021Nov},
as well as the cases of two chief pilots of both Lufthansa and Austrian airlines mentioned in Section~\ref{2023-UFO-chapter-History--1953-2016-scpla}.
Robert Fleischer is a very active journalist and producer of ExomagazinTV with numerous interesting contributions~\cite{ExoMagazinTV2022Oct}.
