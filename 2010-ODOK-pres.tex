%\documentclass[pra,showpacs,showkeys,amsfonts,amsmath,twocolumn,handou]{revtex4}
\documentclass[amsmath,red,table,sans,handout]{beamer}
%\documentclass[pra,showpacs,showkeys,amsfonts]{revtex4}
%\documentclass[pra,showpacs,showkeys,amsfonts]{revtex4}
\usepackage[T1]{fontenc}
%%\usepackage{beamerthemeshadow}
\usepackage[headheight=1pt,footheight=10pt]{beamerthemeboxes}
\addfootboxtemplate{\color{structure!80}}{\color{white}\tiny \hfill Karl Svozil (TU Vienna)\hfill}
\addfootboxtemplate{\color{structure!65}}{\color{white}\tiny \hfill Verbuchhalterung\hfill}
\addfootboxtemplate{\color{structure!50}}{\color{white}\tiny \hfill Dokumententag, Sept 21, 2010, Leoben\hfill}
%\usepackage[dark]{beamerthemesidebar}
%\usepackage[headheight=24pt,footheight=12pt]{beamerthemesplit}
%\usepackage{beamerthemesplit}
%\usepackage[bar]{beamerthemetree}
\usepackage{graphicx}
\usepackage{pgf}
%\usepackage{eepic}
%\usepackage[x11names]{xcolor}
%\newcommand{\Red}{\color{Red}}  %(VERY-Approx.PANTONE-RED)
%\newcommand{\Green}{\color{Green}}  %(VERY-Approx.PANTONE-GREEN)



\RequirePackage[german]{babel}
\selectlanguage{german}
\RequirePackage[isolatin]{inputenc}

\pgfdeclareimage[height=0.5cm]{logo}{tu-logo}
\logo{\pgfuseimage{logo}}
\beamertemplatetriangleitem
%\beamertemplateballitem

\beamerboxesdeclarecolorscheme{alert}{red}{red!15!averagebackgroundcolor}
\beamerboxesdeclarecolorscheme{alert2}{orange}{orange!15!averagebackgroundcolor}
%\begin{beamerboxesrounded}[scheme=alert,shadow=true]{}
%\end{beamerboxesrounded}

%\beamersetaveragebackground{green!10}

%\beamertemplatecircleminiframe

%\usepackage{feynmf}             %Package for feynman diagrams.

\begin{document}


\title{\bf \textcolor{red}{Verbuchhalterung der Universit\"aten}}
%\subtitle{Naturwissenschaftlich-Humanisticher Tag am BG 19\\Weltbild und Wissenschaft\\http://tph.tuwien.ac.at/\~{}svozil/publ/2005-BG18-pres.pdf}
\subtitle{\textcolor{orange!60}{\small http://tph.tuwien.ac.at/$\sim$svozil/publ/2010-ODOK-pres.pdf}
}
\author{Karl Svozil}
\institute{Institut f\"ur Theoretische Physik, Vienna University of Technology, \\
Wiedner Hauptstra\ss e 8-10/136, A-1040 Vienna, Austria\\
svozil@tuwien.ac.at \\
{\tiny Disclaimer: Die hier vertretenen Meinungen des Autors verstehen sich als Diskussionsbeitr�ge und decken sich nicht notwendigerweise mit den Positionen der Technischen Universit�t Wien oder deren Vertreter.}
}
\date{Sept 21, 2010, Montanuniversit\"at Leoben, Steiermark}
\maketitle



%\frame{
%\frametitle{Contents}
%\tableofcontents
%}


%\section{}
\frame{
\frametitle{``Exectutive Summary''}
{
\begin{itemize}

\item<1->
Hauptthese I: die Notwendigkeit, den Einsatz �ffentlicher Mittel zu rechtfertigen,
und die damit einhergehenden
Rechtfertigungsstrategien (``checks \& balances'', ``accountability'')
f�hren in demokratischen Republiken
zunehmend zu einer �berbordenden B�roratisierung und Verbuchhalterung der Universit�ten.

\item<1->
Hauptthese II: Das intuitive Konzept der ``wissenschaftlichen Leistung'' ist nicht
formalisierbar.
\begin{itemize}

\item<1->
 Tut man es trotzdem, f�hrt dies in Plutokratien (USA, UK,~$\ldots$) zum
``Taylorismus'' und zu Flie�bandtechniken.

\item<1->
Im Sowjet-Totalitarismus (UdSSR) f�hrte dies zur ``Stachanov-Bewegung''
(Eugene Garfield's ``Freunde'').
\end{itemize}

\item<1->
Das System passt sich an; dadurch bekommt
 das, was man abfragt:
Pseudo-Performance und Leistungsabfall durch �berbordende Verbuchhalterung
{ (EPS: {\it Bibliometric evaluation of individual researchers: not even right... not even wrong!}''
http://dx.doi.org/10.1051/epn/2009704). }

\end{itemize}
}
}

\frame{
\frametitle{Ein Bild sagt mehr als 100 Worte~$\ldots$}
{
\begin{center}
\includegraphics<1>[height=6.5cm]{2010-ODOK-pres-f1}
\end{center}
}
}

\frame{
\frametitle{Forderungen}
{
\begin{itemize}

\item<1->
``Bologna neu'':
R�ckkehr zum Humboldt'schen System durch Totalentr�mpelung
der Studien- und Wissenschaftssysteme;
dadurch  automatisch Internationalit�t \& Interoperabilit�t,
allerdings ohne  b�rokratischen Overhead.

\item<1->
Keine Anst vor Mi�brauch! (``Wanderung Rekawinkel $\rightarrow$ Salmannsdorf'')
\item<1->
Neubewertung der Wissenschaften in der �ffentlichkeit --- weg von infantilen Bed�rfnisbefriedigungsanstalten, vom ``fun factor'', ``science fairs'', ``Kinderunis'',
``Lange N�chte der Wissenschaft'' und jahrmarkt�hnlichen Vermarktungsstrategien.

\item<1->
Entlastung des Einzelwissenschafters von buchhalterischen Eiferern
und Rechtfertigungsstrategen;
Mut zum Risiko und zur Vision (Kennedy: ``man on the moon'')

\item<1->
M�glichkeit zur Forschung in Freiheit \& Einsamkeit.

\end{itemize}
}
}


\frame{

\centerline{\Large Danke f�r Ihre Aufmerksamkeit!}

\begin{center}
$\widetilde{\qquad \qquad }$
$\widetilde{\qquad \qquad}$
$\widetilde{\qquad \qquad }$
\end{center}
 }

\end{document}
