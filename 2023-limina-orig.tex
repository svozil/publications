\newif\ifws
%\wstrue
\ifws

\documentclass{article}

\usepackage{graphicx}        % standard LaTeX graphics tool
\usepackage{xcolor}

\usepackage{hyperref}
\hypersetup{
    colorlinks,
    linkcolor={blue!80!black},
    citecolor={red!75!black},
    urlcolor={blue!80!black}
}

% Damit die Verwendung der deutschen Sprache nicht ganz so umst\"andlich wird,
% sollte man die folgenden Pakete einbinden:


%German
%\usepackage[latin1]{inputenc}% erm\"oglich die direkte Eingabe der Umlaute
%\usepackage[T1]{fontenc} % das Trennen der Umlaute
%\usepackage{ngerman} % hiermit werden deutsche Bezeichnungen genutzt und
                     % die W\"orter werden anhand der neue Rechtschreibung
                     % automatisch getrennt.
\title{Unidentified Anomalous Phenomena complience and exlpanation traps}
\author{Karl Svozil \\
        Institute for Theoretical Physics,
Vienna  University of Technology,  \\
Wiedner Hauptstrasse 8-10/136,
1040 Vienna,  Austria
        }

\date{\today}
% Hinweis: \title{um was auch immer es geht}, \author{wer es auch immer
% geschrieben hat} und  \date{wann auch immer das war} k\"onnen vor
% oder nach dem  Kommando \begin{document} stehen
% Aber der \maketitle Befehl mu\ss{} nach dem \begin{document} Kommando stehen!
\begin{document}

\maketitle


\begin{abstract}
The article presents arguments for the potential scientific and technological benefits of studying Unidentified Anomalous Phenomena (UAPs), which some believe could shortcut human development. It discusses the history of government investigations into UAP sightings and suggests that the mainstream media may have overlooked the extent of government involvement. The article also explores the challenges of understanding advanced technology and scientific knowledge between civilizations with different levels of technological development. It draws on the ideas of philosophers of science such as Kuhn and Lakatos, who presented distinct viewpoints on the progression of scientific knowledge. The article proposes solutions to overcome traps that can hinder the study of extraterrestrial phenomena, including the delusion of expert competence and the data volume trap, and also discusses the challenges of accessing and penetrating an alien computer system.
\end{abstract}


\else
\documentclass[%
 %reprint,
  twocolumn,
 %superscriptaddress,
 %groupedaddress,
 %unsortedaddress,
 %runinaddress,
 %frontmatterverbose,
 % preprint,
 showpacs,
 showkeys,
 preprintnumbers,
 %nofootinbib,
 %nobibnotes,
 %bibnotes,
 amsmath,amssymb,
 aps,
 % prl,
  pra,
 % prb,
 % rmp,
 %prstab,
 %prstper,
  longbibliography,
 floatfix,
 %lengthcheck,%
 ]{revtex4-1}

%\usepackage{cdmtcs-pdf}

\usepackage{mathptmx}% http://ctan.org/pkg/mathptmx

\usepackage{amssymb,amsthm,amsmath}

\usepackage{tikz}
\usepackage{graphicx}% Include figure files
%\usepackage{url}

\usepackage{xcolor}

\usepackage{hyperref}
\hypersetup{
    colorlinks,
    linkcolor={blue},
    citecolor={red!75!black},
    urlcolor={blue}
}


\begin{document}


\title{Unidentified Anomalous Phenomena complience and exlpanation traps}


\author{Karl Svozil}
\email{svozil@tuwien.ac.at}
\homepage{http://tph.tuwien.ac.at/~svozil}

\affiliation{Institute for Theoretical Physics,
Vienna  University of Technology,
Wiedner Hauptstrasse 8-10/136,
1040 Vienna,  Austria}



\date{\today}

\begin{abstract}
The article presents arguments for the potential scientific and technological benefits of studying Unidentified Anomalous Phenomena (UAPs), which some believe could shortcut human development. It discusses the history of government investigations into UAP sightings and suggests that the mainstream media may have overlooked the extent of government involvement. The article also explores the challenges of understanding advanced technology and scientific knowledge between civilizations with different levels of technological development. It draws on the ideas of philosophers of science such as Kuhn and Lakatos, who presented distinct viewpoints on the progression of scientific knowledge. The article proposes solutions to overcome traps that can hinder the study of extraterrestrial phenomena, including the delusion of expert competence and the data volume trap, and also discusses the challenges of accessing and penetrating an alien computer system.
\end{abstract}

%\pacs{03.65.Aa, 03.65.Ta, 03.65.Ud, 03.67.-a}
\keywords{UAP, UFO, data volume trap, explanation trap, anti-gravity}
%\preprint{CDMTCS preprint nr. x}

\maketitle

\fi


%\section{Caveat}
%A caveat for the soon-to-be perplexed:
%The following thoughts are highly speculative and are stimulated by previous research on the modeling in cognitive science
%by James L. McClelland~\cite{McClelland_2009},
%Paul  Thagard~\cite{thagard_2022,Thagard2019Feb,thagard_2012,Thagard_2010,Thagard2005Feb},  and
%Daniel C.  Dennett~\cite{Dennett1992Oct,Dennett_2005},
%among others.
%It is also inspired by the recent advances of Large Language Models (LLMs) of Artificial intelligence (AI),
%in particular, Generative Pre-trained Transformer (GPT)~\cite{OpenAI-GTP-4-2023Mar,Anshu2023Mar}.

\section{Caveat}

%The following ideas are highly speculative.
%Maybe UAPs are the big nothingburger of the past eighty years.

Over the past eighty years, countless reports of Unidentified Aerial Phenomena (UAPs) have been made by people all around the world. Despite this, there has been little concrete evidence to support the idea that these sightings are evidence of extraterrestrial life visiting Earth. In fact, some people believe that UAPs are nothing more than a fanciful concept with no real substance~\cite{Menzel_1953,Klass1975Jan,Klass1983Jan,Dick2015,West2018Apr}.

While the overwhelming number of UAP sightings can be easily explained as natural phenomena or human-made objects, there are still a significant number of reports that remain unexplained. However, the fact that these sightings remain unexplained does not necessarily mean that they are evidence of extraterrestrial life. It is entirely possible that they could be caused by other, as-yet-unknown phenomena.

There are also those who believe that the idea of UAPs has been kept alive by conspiracy theorists and people looking to make a quick buck off of books, documentaries, and other media related to the topic. They argue that there is little to no evidence to support the idea that UAPs are anything more than a figment of people's imaginations or a product of sensationalist media coverage.

Of course, the idea that UAPs are a big nothingburger is just one possibility among many. It is entirely possible that there is more to the phenomenon than we currently understand.


%UAPs may be one channel to scientific and technological progress.
%Maybe they are shortcuts to capacities which, under nondisruptive conditions, would have developed by our own comprehension and cognition after centuries and millenia have passed.

\section{UAPs as scientific and technological inspiration}

One possibility is that UAPs (Unidentified Aerial Phenomena) may serve as a resource of scientific and technological inspiration, potentially providing a shortcut to centuries or even millennia of human development. Some argue that the advanced technology behind UAP sightings may be beyond our current capabilities, and could represent a major shift in our understanding of the universe and our place in it.

While this idea remains highly speculative, it is important to consider the potential implications of such a breakthrough.
Developing a comprehension of recovered UAP material or craft would depend largely on our current understanding of science and technology.
Ideally we may be able to study and reverse engineer any recovered material or craft.
However, if the technology is based on principles beyond our current understanding, it may prove difficult or impossible to comprehend.

By considering analogies from our past, we can explore these possibilities further.


\section{Similarities and differences of processing UAP and NAZI technology in the USA}

During World War II, Nazi Germany was known for its advanced technology, which included the development of jet engines and rocketry.
The US government was quick to study and reverse engineer these technologies after the war ended, with the help of Operation Paperclip~\cite{Jacobsen2014}.
This program allowed the US to recruit and employ German scientists who had worked on these technologies,
in order to gain an advantage in the Cold War against the Soviet Union.

Similarly, the Soviet Union was also heavily involved in studying and reverse engineering foreign technologies, including those of Nazi Germany and the US.
This led to an arms race between the two superpowers, as each sought to gain an edge in military technology.
The Soviet Union was particularly interested in rocketry and atomic secrets, which they saw as a means of achieving military equality and superiority over the US.

Given this history of studying and reverse engineering foreign technologies,
 it is not unreasonable to speculate that the US government may have attempted to "process" alien artifacts similarly to how they studied Nazi
and other Earth-bound technologies. If such artifacts did exist, the US government would likely view them as a potential source of technological advantage
over other countries.

%It might also be plausible that, as time went by, so did ancient recovered artifacts were buried deeper and deeper in the respective security apparatus of states.

As time went by, it is possible that any crashed craft or other artifacts recovered by the US government or other nations would have been buried deeper and deeper within their respective security apparatus. This is because governments often become more secretive and protective of sensitive information over time, especially when it comes to matters of national security.

If the US government or other nations did recover ancient artifacts, they would likely be classified as top-secret and heavily guarded.
As time passes and the government changes hands, new administrations might not even be aware of the existence of these artifacts, unless they were specifically briefed on them.

Furthermore, the government may have chosen to bury the artifacts deeper within the security apparatus to prevent Freedom of Information Act disclosure or  unauthorized access or tampering. This is because the knowledge or technology contained within the artifacts could be incredibly powerful and could potentially pose a threat to national security if it fell into the wrong hands.

It is also worth noting that the recovery and possession of ancient artifacts may be controversial, both legally and ethically. Some may argue that these artifacts should be returned to their country of origin, or that they should be studied by a more neutral and international body to prevent any one nation from gaining an unfair advantage.

Overall, if any ancient artifacts were recovered by the US government or other nations, it is plausible that they would have been buried deeper and deeper within the security apparatus as time went on. This would be done to prevent unauthorized access and to protect national security. However, the legal and ethical implications of possessing such artifacts are complex and would need to be carefully considered.

One possibility to ``carve out'' space for such retreived craft would be carve-out Unacknowledged Special Access Programs with very limited and almost totally restricted congressional oversight.

\subsection{The Five Eyes colonials I: Canada and the Wilbert Smith memo}

Wilbert Brockhouse Smith was a Canadian government scientist who became interested in UAPs after reading reports in the late 1940s. In 1950, he indirectly met with American physicist and defense consultant Robert Irving Sarbacher, who informed him that UAPs were real, highly classified in the US government, and their technology was advanced. Smith wrote a memo about their interview, which is believed to be a recollection of the interview rather than an official document. The memo is not part of the national archives collection but can be found in the archives at the University of Ottawa as part of the Arthur Bray fonds. It is uncertain whether Smith actually met with Sarbacher in person or communicated with him through intermediaries.

In 1950, Smith also wrote a memorandum to the Controller of Telecommunications stating that he had received information that flying saucers exist and that the US authorities are investigating several lines related to the saucers. Smith also mentioned that if Canada is doing anything in geo-magnetics, the US authorities would welcome a discussion with suitably accredited Canadians.

In a letter written in 1983, Sarbacher confirmed his meeting with Smith and stated that John von Neumann and Vannevar Bush were involved in the recovery of flying saucers. He also mentioned that materials from the reported flying saucer crashes were extremely light and very tough, and that instruments or people operating the machines were also of very light weight. Sarbacher expressed confusion about the high level of classification given to the topic and the denial of the existence of these devices.

Sarbacher led researchers to Eric Walker, who confirmed the existence of crash retrievals and recoveries of bodies and attended meetings on the subject. He also verified the existence of an organization similar to MJ-12. However, he warned researchers to stop their investigation, stating that it was an area that could not be pursued.

\subsection{The Five Eyes  colonials II: Australia and the Oliver Harry Turner memo}


Twenty years later, and independent of Smith, Oliver Harry Turner, in the Australian outpost of the Five Eyes Community, came to very similar conclusions as Smith.
In a secret and possibly unsolicited report to the Australian government~\cite[Sections~14-17]{TurnerAustralia1971}, Turner meticulously reviewed past UAP activities in the United States.

Turner's analysis in 1971 revealed the significant investments made by the US government into anti-gravity research.
Despite the lack of scientific understanding in the field of gravity and anti-gravity, six universities and government
agencies were supported in a pursuit to solve the problem.
According to Turner the research was prompted by the belief that UAPs were real and that their intelligent controllers had mastered gravity control. By 1966, 46 separate projects were being financially supported, 33 of which were under the supervision of the US Air Force. However, most of the details of these projects were classified, and in general, they were unsuccessful.

According to the biographical notes of Turner by Dominic McNamara and Bill Chalker~\cite{Turner-bio-Chalker}, during the early years of his career in 1956, Turner had stumbled upon the US anti-gravity project through a note that was posted on a board at Harwell, which was Britain's inaugural Atomic Energy Research Establishment. As per the opinion of the staff at Harwell, the idea of anti-gravity research was considered peculiar since the concept of gravity itself was not fully understood.


\subsection{The Wilson memo}

The memo allegedly records a conversation between Eric Davis~\cite{WilsonNotes-Davis}, a physicist and contractor for the Department of Defense, and Admiral Tom Wilson, the former Director of the Defense Intelligence Agency, about the possible existence of crashed and retrieved UAPs or UAPs (Unidentified Aerial Phenomena).

According to the memo, Davis met with Wilson in 2002 and asked him about his previous attempts to access information on a secret program that was reportedly involved in reverse engineering alien technology from a crashed UAP. Wilson told Davis that he had learned about the program in 1997 from Will Miller, a retired naval officer and UAP enthusiast, who had also introduced him to Edgar Mitchell and Steven Greer, the founder of the Disclosure Project.

The memo alleges that Wilson was frustrated with his initial difficulties accessing the program(s), which he believed was illegal and unconstitutional. He later was told of the alleged existence of a recovered UAP, which he described as {\it ``was not of this Earth - not made by man - not by human hands.''} He was not told where the UAP came from, how old it was, or how it worked. Efforts to reverse engineer the craft had reportedly failed.
The memo also states that Wilson was threatened with career penalties if he did not stop his inquiries.

Wilson himself called the notes ``fiction'' and stated, ``I wouldn't know Eric Davis if he walked in right now''~\cite{cox20,Cox2021Aug}.
However, one might speculate that denials as protective measures to safeguard core secrets is mandatory~\cite{DCSA-SAPguide,DODDirective5205.07,vanderReijden2005}.
Other authors have indirectly hinted on the authenticity of the Wilson memo~\cite{Mellon2022Dec,ValleeFS5}.

\subsection{Uncorking the UAP bottle}


%Suppose for a moment the legacy media in 2017 got it all wrong: they mistook an ad hoc attempt of semi-outsiders---AAWSAP/AATIP---for the real effort, staring at the top of the iceberg, but not at its huge extent deep down in the water.


%Two groups of experts, the Robertson Panel and the Condon Committee,
%had been tasked with calming the hysteria and convincing the public that there was no threat to national security posed by UAPs.
%The Condon report from the latter group had also recommended that although a tiny fraction of ``hard cases'' remained unexplained and ``strange,''
%the opportunity costs of ongoing reporting were unjustified. Therefore, as intended and planned,
%the recommendation had been made for the US Air Force to cease any official investigations into the constantly growing number of reports.
%
%Despite occasional efforts from external sources, such as the Rockefeller Initiative or the Disclosure Project by Greer,
%the US Air Force has been able to maintain a peaceful state of noninvolvement until the present day.
%
%A congressional initiative, started by a UAP enthusiast and real estate tycoon with strong connections to a powerful senator,
%led to the creation of a small private contract and an associated section in the Pentagon.
% Despite not achieving Special Access Program status, this project aggressively attempted to uncover what
%its proponents believed to be secret government programs related to UAPs, such as a screwdriver trying to open a corked bottle.
%They sent out one of their own, who wrote a secret memorandum of a conversation with a high-ranking military officer who suffered a
%bureaucratic defeat while attempting to determine what was inside that UAP bottle.
%Those in-the-know let the officer taste one bottle but refused access to more of their collection in their wine cellar.


Let us imagine, just for a moment, that in 2017 the mainstream media got it all wrong.
They may have mistakenly identified  an ad hoc attempt of semi-outsiders of AAWSAP/AATIP as the real effort and only looked at the tip of the iceberg,
missing the vast extent of the movement deep below the surface since the late 1940's.

Two groups of experts, the Robertson Panel and the Condon Committee, were appointed to quell the public hysteria and convince them
that UAPs did not pose a threat to national security. The Condon report, in particular, suggested that even though a small number of ``hard cases''
were still strange and warranted some small-scale qualified research, the costs of continued investigation were not justified.
This recommendation led to the US Air Force officially ceasing all investigations into the increasing number of UAP reports, as originally intended and planned.

Despite some external initiatives like the Rockefeller Initiative~\cite{Berliner2000Jun} or the Disclosure Project by Greer~\cite{Greer-dp},
the US Air Force has successfully maintained a state of non-involvement until the present day.

However, a congressional initiative spearheaded by a UAP enthusiast and real estate tycoon with strong connections to a powerful senator,
led to the creation of a small private contract and a corresponding section in the Pentagon.
Although it did not attain Special Access Program status, this project aggressively sought to uncover what its supporters
believed to be secret government programs related to UAPs, akin to a screwdriver trying to open a corked bottle.
They dispatched one of their own, who documented a private conversation with a high-ranking military
officer who faced bureaucratic defeat while attempting to determine what was inside that ``UAP bottle.''
Those who knew about it let the high-ranking military officer sample one bottle, but refused access to more, especially those stored in their wine cellar.

One of the key mysteries surrounding alleged UAP technology is the energy source and propulsion system, which remain unsolved to this day.
The propulsion system may be based on changing inertia, anti-gravity, or another unknown technology.
Therefore, the US government may have obtained advanced crafts from recovered UAPs, but has been unable to replicate their technology on a larger scale.
It remains not totally unplausible that salvaged propulsion systems from these crafts have been used to construct new, self-made crafts ``around''
the propulsion drives.
It goes without saying that all of these speculations are highly uncertain and lack any tangible evidence for verification.

\section{Explanation trap through scientific overreach}

\subsection{Succession of scientific revolutions}

Kuhn and Lakatos are two notable philosophers of science who presented supplementary viewpoints on the progression of scientific knowledge.
Kuhn contended that science is not a linear and cumulative process, but rather a series of sudden, disruptive shifts or revolutions.
He maintained that science is guided by a dominant paradigm, which includes a set of assumptions, methods, and problems that direct typical scientific research.
Normal science is characterized by puzzle-solving activities that aim to expand and refine the paradigm.
However, anomalies or phenomena may arise that cannot be explained by the paradigm.
When these anomalies accumulate and challenge the paradigm's validity, a crisis ensues that may result in a scientific revolution.
A revolution occurs when a new paradigm replaces the old one and redefines the standards and criteria of scientific practice.
Kuhn argued that paradigms are incommensurable, implying that they cannot be objectively compared or evaluated since they have distinct conceptual frameworks and values.

On the other hand, Lakatos criticized Kuhn's relativism and his emphasis on the social and psychological factors that influence scientific change.
He suggested a more rational and objective approach to evaluate scientific theories and programs.
While he agreed with Kuhn that science is not based on isolated theories but rather on research programs consisting of a hard core of fundamental assumptions
and a protective belt of auxiliary hypotheses, he disagreed with Kuhn's view that paradigms are incommensurable and that revolutions are irrational.
He posited that a research program can be judged as progressive or degenerating based on its empirical and theoretical performance.
A program is progressive if it predicts new facts, solves empirical problems, generates testable hypotheses, and broadens the scope of inquiry.
 A program is degenerating if it fails to do so and relies on ad hoc adjustments or immunizing strategies to protect its core from falsification.
He argued that scientists should select the most progressive program among competing alternatives and abandon the degenerating ones,
but it may be challenging to identify progressive versus degenerative research because core ideas are often ``shielded'' by auxiliary theories and assumptions.
He also emphasized the importance of logical and methodological criteria in assessing scientific theories and programs.


Both Kuhn and Lakatos, and to some extent Feyerabend~\cite{feyerabend,fey-philpapers1,fey-philpapers2}), share a common perspective that:
\begin{enumerate}
\item During prolonged periods, there exist beliefs, hard-core assumptions, and practices that form a dominant scientific program.
\item Any dominant scientific program comprises core semantical concepts that are expressed through theoretical, syntactic formalizations.
\item Eventually, a dominant scientific program will be overturned by another scientific program.
\item Paradigms are incommensurable~\cite{sep-incommensurability}, meaning that the semantical concepts of competing or successive scientific programs are un(cor)related. However, their theoretical and syntactic formalizations might coincide to some extent.
\end{enumerate}



Hence, it is crucial to consider the temporal aspect of scientific progress, which is often taken for granted in historical contexts. This is particularly important when civilizations with vastly different backgrounds, scientific and technological capabilities confront each other.

\subsection{Spread of scientific and technologic advancements}

%One important consequence of the aforementioned temporal increment of scientific research is
%incommensurability and incomprehensibility of ``very advanced'' science and technology
%with respect to the means of understanding of ``less advanced'' civilizations.
%
%Individuals or groups pursuing a specific research program will not be able to understand or reconstruct phenomena and technology associated with a program more than one step ahead. This means that if they encounter a technology that is based on a scientific revolution that they have not experienced yet, they will likely fail to comprehend its principles and mechanisms. Therefore, trying to understand advanced technology from a civilization with more than one scientific revolution ahead will likely fail.
%Such technology would appear incomprehensible, mysterious or even magical to them.

%This results in an insurmountable (from the point of view of low-tec civilizations) spread of scientific and technologic advancements.

One important consequence of the rapid pace of scientific research is that it can lead to incommensurability and incomprehensibility
between civilizations with different levels of technological development.
When individuals or groups are engaged in a particular research program,
they may not be able to understand or reconstruct phenomena and technology associated with a program more than one step ahead.
This means that if they encounter a technology that is based on a scientific revolution that they have not experienced yet, they will likely fail to comprehend its principles and mechanisms. As a result, attempting to understand advanced technology from a civilization that is more than one scientific revolution ahead may prove to be futile.

Consequently, the spread of scientific and technological advancements may become insurmountable from the point of view of less advanced civilizations.
Such technology would appear incomprehensible, mysterious, or even magical~\cite{Clarke2000Jan} to them.

The above statement is only accurate in the context of a closed system,
where civilizations have no means of accessing outside sources of information
about advanced technology.
In such a closed system, civilizations would not have access to some ``alien Prometheus''
who could provide them with a detailed explanation of the principles and mechanisms underlying advanced technology, such as the ``crashed phenotype.''

However, if civilizations have access to outside intelligence resources, the situation may be different. In this case, they could potentially leverage the knowledge and expertise of more advanced civilizations to better understand advanced technology.
This may involve collaboration, knowledge-acquisision, or even reverse engineering of existing technology.

While knowledge-sharing and collaboration between civilizations could potentially help bridge the gap in scientific development, the motivation for technology transfer is a crucial factor to consider. It is essential to question what motivates more advanced civilizations to share their knowledge and technology with less advanced ones.

Some potential motivations could include economic gain, political influence, or even altruistic motives such as a desire to improve the well-being of all humanity. In some cases, technology transfer could also be motivated by a desire to build strategic partnerships and alliances.

However, it is also important to acknowledge that technology transfer can have negative consequences. It may lead to the loss of intellectual property and market advantages for the more advanced civilization. It could also result in the destabilization of social and economic systems in the less advanced civilization, or the potential for the technology to be misused or weaponized.

We need to ask ourselves: why should ``These Others'' communicate with ``Us?''
To quote a passage from Charles Fort's ``The Book of the Damned''~\cite{FortBotD},
``Would we, if we could, educate and sophisticate pigs, geese, cattle?
Would it be wise to establish diplomatic relation with the hen that now functions, satisfied with mere sense of achievement by way of compensation?
I think we're property.''

\subsection{The delusion of  expert competence}


It is possible that relying solely on contemporary experts, such as theoretical physicists or rocket scientists, may not always be the most effective approach in understanding complex phenomena. These experts may be biased and invested in their respective fields, leading them to emphasize their current thinking and potentially overlook alternative perspectives or hypotheses.

In the context of Reich's Segmental Armouring Theory, experts may carry and apply their current expertise like a vendor's tray, surrounded by an impenetrable armor of alleged wisdom. This can make it difficult for them to consider alternative viewpoints or to recognize the limitations of their own understanding.

When it comes to evaluating very advanced technology or progressive research programs, the thinking and belief systems of contemporary experts may be inappropriate or even distracting. In some cases, their understanding may be outrightly wrong, leading to missed opportunities and wasted costs. To illustrate this point, consider asking a shaman medicine man of Borneo to explain a WWII airplane flying overhead. The shaman's belief system and understanding of the world may not align with modern physics and technology, leading to an inaccurate or incomplete explanation.



\subsection{Overcoming the explanation trap by suspended attention}

The ``mind projection fallacy'' or ``explanation trap'' is a cognitive error that occurs
when individuals assume that their perceptions of the world reflect objective reality.
This fallacy arises when someone projects their own mental constructs or subjective experiences onto the external world,
as if these constructs were objective features of the world. This can lead to confusion and misunderstandings in various fields of science and philosophy.

The term ``mind projection fallacy'' was first coined by Edwin Thompson Jaynes~\cite{jaynes-90}, a physicist and Bayesian philosopher,
who argued that this fallacy is particularly prevalent in the study of probability and statistics.
Jaynes pointed out that individuals often assume that randomness and disorder are objective features of the world, rather than subjective judgments based on their own lack of knowledge.

However, the mind projection fallacy is not limited to probability and statistics; it can also occur in other areas, such as philosophy, psychology, and neuroscience. For example, individuals might assume that their own subjective experiences of consciousness or free will are universal features of human experience, without considering the possibility that these experiences are influenced by cultural, historical, or individual factors.

To avoid the mind projection fallacy, I propose adopting an analytical approach called ``evenly-suspended attention''~\cite{Freud-itp,Freud-itpe}.
This involves observing phenomena without projecting one's own mental constructs or biases onto them. By suspending judgment and preconceived ideas,
individuals can allow their observations to settle in and reduce the influence of their own imagination or ignorance on their perception of reality.

Evenly-suspended attention is not an easy task, as it requires a high degree of mindfulness and self-awareness. However, it can be a valuable tool for improving critical thinking and avoiding cognitive biases. By acknowledging the limitations of their own knowledge and perceptions, individuals can open themselves to new possibilities and gain a more nuanced understanding of the world.


\subsection{Could we hack us into their database?}

The idea of accessing and penetrating an alien computer system, specifically the information storage and handling facility in a recovered craft, may seem like a viable option for obtaining information about their technology, means of propulsion, and energy source. However, given the vast technological differences between our civilizations, attempting to hack into their systems may prove fruitless.

To identify potential storage facilities in a recovered flying saucer, we would first need to determine its hardware.
It is likely that any such storage would be highly advanced and thus have a complex material layer with a high information density stored per unit of material. This would make any memory device appear irregular and randomly aligned, similar to white noise.
However, this paradoxical feature of highly organized, dense coding also makes it difficult to decipher.

Therefore, we would need to look for areas of the craft that appear structurally irregular and chaotic to locate potential storage devices.
We could also use advanced scanning techniques to analyze the material at an atomic level and identify any areas of high information density.
Overall, a multi-faceted approach combining structure analysis of the material with algorithmic pattern analysis would be necessary
to successfully identify and cryptanalyze storage facilities in a recovered flying saucer.

As a side note I would like to point out that a conventional search for extraterrestrial intelligence (SETI) may not be effective
in detecting advanced civilizations that have moved beyond using narrowband radio signals.
As technology progresses and broadcasts become digitized,
the emitted signals tend to approach white noise, making them much more difficult to analyze and decipher.
Therefore, alternative methods, such as the identification and analysis of recovered alien technology,
 may be necessary to obtain information about advanced civilizations.

\subsection{Overcoming the data volume trap by automated prost-processing}

I believe it can be safely stated that the US Air Force Project Blue Book was overwhelmed by the amount and variety of data on UAP sightings
that it had to deal with. As the years passed, the small team working on Project Blue Book---consisting of only one officer,
a sergeant, and a secretary~\cite{Condon-report,Condon-report-Bantam,Condon-report-Dutton,BibEntry2023Jan}---continued to collect an increasing amount of data on UAP sightings. By 1966, the US Air Force had gathered a vast amount of information from nearly 20 years of investigating over 10,000 cases.

However, this data was not analyzed in a comprehensive, and even less so rigorous, way.
One of the challenges is tho ``separate the wheat from the chaff,'' the overhelming numbe ---some estimates range from
97{\%}~\cite{Geipan} to more than 99{\%}~\cite{AA-Condon-1970}---of
mundane sightings from the ``hard cases.''

The American Institute of Aeronautics and Astronautics (AIAA) presented a critique of the Condon Report~\cite{AA-Condon-1970} and
thereby raised a fundamental question that summarizes the core issue.
The question can be stated as follows: ``Is it reasonable to assume that if 99{\%}
of observations can be accounted for, then the remaining 1{\%}
can also be explained?''
Alternatively, the question can be phrased as,
``Do we have a significant signal-to-noise ratio problem, with an order of magnitude of 0.01?''

Given a similarly very limited allocation of resources, I suggest:
\begin{enumerate}
\item
using (possibly downgraded for security concerns) existing military sensors to allocate uncorrelated targets, and additionally,

\item
using automated (artificial intelligence) methods to ``separate the wheat from the chaff,'' i.e., to locate the "hard cases."

\item
post-selecting the remaining cases by  consulting  critical experts, such as, for instance, with Metabunk~\cite{Metabunk2023Mar} and Mick West~\cite{West2018Apr}.
\end{enumerate}



\section{Autonomy or (blissful) ignorance}

It may be tempting to speculate that even if some individuals in positions of power were aware of ``alien visitations,''
they may not have the autonomy to publicly acknowledge it due to military, societal and political pressures.
Governments may be attempting to deal with the phenomenon
while maintaining a facade of (blissful?) normalcy for the general population.
I would also consider the potential consequences for projects aimed at disclosing their existence,
in particular, also if they would have to cope with the possibility of telepathic abilities of the extraterrestrial beings
to influence or manipulate humans.

Additionally, it may not be in the best interest of the extraterrestrial beings to interact with humans,
as their goals and motives may not align with human interests.

From analogues of historical European expansion and colonization,
I infer that centers of empires eventually suffer from a phenomenon called ``imperial backflow.''
This refers to the tendency of former colonies to negatively impact the centers in terms of autonomy.
There tends to be an ``osmosis'' effect where individuals from colonies gravitate towards richer and technologically advanced civilization hubs.

To prevent this from happening, one solution could be to impose a policy of strict nonfraternization~\cite{GulfOfSilencey2020}.
This policy prohibits or limits social interactions between those in power and those who are not,
including relationships of any informal kind that could create a potential connection.
The goal is to maintain an impartial environment and avoid actions that could undermine the integrity or effectiveness of those in power.

As noted by Fort~\cite{FortBotD}, Bramley~\cite{Bramley1993Mar},
and others, we may have little to offer advanced extraterrestrial civilizations visiting us.
The asymmetry between their advanced technology and our own would make ``their'' potential gains from
contact relatively marginal compared to the potential risks and losses ``we'' might cause ``them.''
Moreover, our own civilizations may decline.
This consideration is likely to dominate future principles of compliance
with less technologically advanced civilizations, particularly as we approach a saturation point in space exploration and colonization.

While we might hope for an ``alien Prometheus'' who gifts us with scientific and technological achievements,
there is no guarantee that such a gift would occur.

In the absence of external help, we may have no choice but to rely on our own efforts to comprehend the phenomenon.
However, we need to be careful and  proceed with caution and avoid mind projection fallacies and falling into the trap of premature or unfounded explanations.


Acknowledgements: This paper was post-processed by OpenAI's ChatGTP-3.5, as well as Microsoft Bing (GTP-4).
\bibliography{svozil,ufo}
\ifws

\bibliographystyle{spmpsci}

\else
 \bibliographystyle{apsrev}

\fi

\end{document}
