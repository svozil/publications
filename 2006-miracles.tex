Unknowability in Physics

Karl Svozil

Although G�del expressed doubts about the relevance of his independence theorems for quantum physics, G�delian incompleteness can be formulated for physical systems strong enough to support universal computation. A straightforward embedding of a universal computer into a physical system results in the fact that, due to the reduction to the recursive undecidability of the halting problem, certain future events cannot be forecasted and are thus provable indeterministic. Furthermore, for any such deterministic system, due to the recursive unsolvability of the rule inference problem, the general induction problem is provable unsolvable. It remains to be seen whether or not these principal G�delian issues are relevant for the practical development of physics proper.

Another type of physical randomness associated with unknowability occurs in classical deterministic physical systems. Algorithmic measures indicate that "almost all" elements of the classical (real or complex) continuum are not only noncomputable, but random; i.e., algorithmically incompressible. If the reality of the physical continuum is postulated, then "almost all" possible initial values are random. Classical, deterministic chaos results from "unfolding" of such a random initial value drawn from the "continuum urn" by a recursive, deterministic function. A weaker form of deterministic chaos just expresses the fact that a linear increase of the inevitable error in the determination of the system's initial value results in exponential divergences in the possible future behaviours consistent with the measured initial values.

A third group of physical unknowables arises in the quantum context. The prevalent interpretations of the quantum formalism postulate the principal indeterministic occurrence of certain measurement outcomes, such as the "quantum coin toss" resulting from a mismatch between preparation and measurement. Another quantum indeterminism is complementarity; i.e., the principal impossibility to measure two or more complementary observables with arbitrary precision simultaneously. Still another quantum unknowable results from the fact that in general no "global" classical truth assignment exists which is consistent with even a finite number of "local" ones. That is, no consistent classical truth table can be given by pasting together the outcomes of noncommeasurable blocks of commeasurable observables. This phenomenon is also known as "value indefiniteness" or "noncontextuality."

A different issue, discussed by Philipp Frank, is the possible occurrence of miracles in the presence of gaps of physical determinism. In this view, singular events could also be interpreted as miracles occurring within the bounds of classical and quantum indeterminism. For, if there is no cause for an event, why should such an event occur altogether rather than not occur? Although such thoughts remain highly speculative, single miracles could be the basis for a directed evolution in indeterministic physical systems, if they exist.



