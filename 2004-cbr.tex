\documentclass[prl,amsfonts,showpacs,showkeys]{revtex4}
%\documentclass[pra,showpacs,showkeys,amsfonts]{revtex4}
\usepackage{graphicx}
%\documentstyle[amsfonts]{article}
\RequirePackage{times}
%\RequirePackage{courier}
\RequirePackage{mathptm}
%\renewcommand{\baselinestretch}{1.3}

\begin{document}
%\sloppy



\title{On the possible local foreground origin of microwave background redshifts}

\author{Karl Svozil}
\email{svozil@tuwien.ac.at}
\homepage{http://tph.tuwien.ac.at/~svozil}
\affiliation{Institut f\"ur Theoretische Physik, University of Technology Vienna,
Wiedner Hauptstra\ss e 8-10/136, A-1040 Vienna, Austria}


\begin{abstract}
Correlation-induced Doppler-like frequency shifts of spectral lines
in scattering media are discussed as a possible origin of the recently discovered strong
correlations of Cosmic Microwave Background (CMB) ratiation with the orientation of the solar system.
\end{abstract}

\pacs{42.50.Ar, 03.80.+r, 05.40.+j, 98.50.-v}
\keywords{Cosmic Microwave Background, redshift, Doppler-like frequency shift}
\maketitle

The recently discovered \cite{oliviera-cost,schwarz}
amazing coincidence between the orientation of CMB,
as observed by the Cosmic Background Explore (COBE) and by
the Wilkinson Microwave Anisotropy Probe (WMAP),
with the geometry and motion of the solar system
casts doubts on its cosmic origin.

One possible interpretation of the data could be in terms of the {\em Wolf effect}
associated with a scattering media,
as suggested in the Quasar context by Emil Wolf
\cite{Wolf-Nature-87,Wolf-PRL-87,Wolf-PRL-89,Bh-Jai}.
If indeed some medium whose macroscopic properties fluctuate randomly both in space and time
accounts for the directions of the multipole vectors, then this
medium, possibly interstellar dust, needs to be identified.

\bibliography{svozil}
\bibliographystyle{apsrev}
\end{document}
