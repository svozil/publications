\documentstyle[12pt]{article}
\begin{document}

\section{Publications}

\section{Lectures \& posters}

1.\\
K. Svozil\\
``Quantenlogik''\\
Atominstitut\\
22. 11. 1995


2.\\
K. Svozil\\
``Some physical aspects of future communication technology''\\
Universisad Federal do Rio De Janairo/Brasilien \\
28. 11. 1995



\section{G\"aste}


\section{Forschungst\"atigkeit}

\subsection{Quantencomputer}
(Svozil)\\
Die Ma\ss einheit der quantenmechanischen Information, das
``Quanten-bit'', ist die koh\"arente \"Uberlagerung der beiden
klassischen Bitzust\"ande.
Dies erlaubt die formale Darstellung von inkonsistenter klassischer
Information, z. B. in Datenbanken oder Expertensystemen, in
Quantencomputer.
Paper:
\ref{maryland},
\ref{qct1},
\ref{qct2},
\ref{omega}
\\
Vortrag:
\ref{1-svoz6},
\ref{2-svoz8},
\ref{2-svoz11},
\ref{2-svoz14},
\\

\subsection{Automatenlogik}
(Svozil)\\
Durch Ein- und Ausgabeexperimente wird der Anfangszustand eines
endlichen deterministischen Automaten gemessen. Die Propositionsstruktur
kann man algebraisch mit Hilfe der in diesem Zusammenhang entwickelten
sogennanten
``Partitionslogik'' beschreiben.
Es wird untersucht, inwieweit die so gefundene Automaten- oder
Partitionslogik mit der Quanenlogik \"ubereinstimmt.
Paper:
\ref{maryland},
\ref{complexity}
\\
Vortrag:
\ref{1-svoz3},
\ref{2-svoz4},
\ref{2-svoz13}
\\



\subsection{Fraktale Signalcodierung}
(Svozil)\\
Die prim\"aren nervlichen Erregungsmuster, welche die Ohrnerven
erzeugen, haben eine stochastische fraktale geometrische Struktur. Es
kann gezeigt werden, da\ss $\,$ schon kleine \"Anderungen der Dimension
dieser prim\"aren Erregungsmuster durch geeignete Verarbeitung
(Stichwort ``Konvergenz und Divergenz'' der Nervenbahnen) zu gro\ss en
\"Anderungen in der Dimension von s\"akunderen Erregungsmustern
f\"uhren.
Paper:
\\
Vortrag:
\ref{2-svoz12},
\ref{2-svoz2}
\\

\end{document}
