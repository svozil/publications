%%Vorlage ``Buch'', v1.06 %%

\documentclass[ebook,12pt,oneside,openany]{memoir}


%\documentclass[a5paper  %Papierformat
%,DIV=13         %Einstellungen für den Satzspiegel
%,twoside                %zweiseitiger Satzspiegel; bei der
%                                                %book-Dokumentklasse automatisch
%                                                %so eingestellt, dass neue Kapitel
%                                                %immer auf der rechten (ungeraden)
%                                                %Seite beginnen
%,11pt % Schriftgr��?e
%,headsepline    %erzeugt eine Trennlinie in der
%                                                %Kopfzeile. Gibts natürlich auch
%                                                %als ``footsepline''
%]{scrbook}
%\usepackage[utf8]{inputenc}
%\usepackage[latin1]{inputenc}
%\usepackage[ngerman]{babel}
%\usepackage[T1]{fontenc}
\usepackage{blindtext} %zur Erzeugung von unsinnigen Textkolonnen
\usepackage[colorlinks=false,pdfborder={0 0 0},bookmarksnumbered]{hyperref}
\usepackage{lettrine}   %Initiale
\usepackage{setspace}
\usepackage{ellipsis}   %verbesserter Abstand bei 3 Punkten ...
%\usepackage[osf]{libertinus}
                                                %Die Option ``osf'' (old school font)
                                                %bewirkt, dass alle Ziffern als
                                                %Minuskelziffern (Mediävalziffern)
                                                %gesetzt werden.
\usepackage{microtype}
\usepackage{graphicx}
\usepackage[a5paper]{geometry}

%\usepackage[OT1]{fontenc}\usepackage{times}
\usepackage[latin1]{inputenc}
\usepackage{times}

\usepackage{titlesec}
\assignpagestyle{\chapter}{empty}

\begin{document}
 \pagestyle{empty}


\mainmatter % Gliederungsm�glichkeit bei Dokumentklasse Book für den Hauptteil

\begin{flushleft}
I M P R E S S U M\\
Bora Bora on the Air Mattress\\
Karl Svozil\\
� Funzl Verlag 2024\\
All rights reserved\\
Autor: Karl Svozil\\
Wasnergasse 13/20 1200 Wien, �sterreich\\
office@funzl.at\\
ISBN: 9783951969688
\end{flushleft}









\chapter{First Impressions of the Night}

From Auckland, New Zealand's largest city, it is a five-hour flight northeast to Fa'a'{a}, near Papeete on Tahiti. The night flight with Air New Zealand is nearly empty, giving me the luxury to stretch out and relax.
Upon arrival at one o'clock in the morning, we are greeted by intense heat---Tahiti's sole international airport lacks both air conditioning and a skybridge.

In the immigration area, a Polynesian trio greets us with South Sea melodies, strumming ukuleles and swaying their hips gracefully. Like figures in a mechanical music box, they begin to play as soon as the first passengers approach. Their movements, reminiscent of dancing ballerinas, freeze abruptly when they run out of their flow of music---or the arrival of passengers.

Afterward, the small crowd of arrivals disperses into the island's dim constellation of airport lights, swallowed by the surrounding Polynesian darkness. A sense of tropical despair, emptiness, and scatteredness prevails.
Most passengers linger in the humid night air, awaiting their morning connections.

Unable to secure a transfer to the ferry, I wait along with the others and eventually board a very early bus bound for the quiet, slumbering Papeete---a half-hour drive to its idle port. A young couple joins me on this nocturnal journey.

Carrying my suitcase through Papeete's deserted streets, from the bus station to the ferry terminal, feels surreal---a journey reminiscent of a ride on the Grottenbahn in Vienna's Wurstelprater.
Along the harbor, I pass a homeless man asleep, stretched out on a bench, both exposing himself at critical parts, as well as appearing exposed and vulnerable.
The tropical climate allows life to spill outdoors, leading to states of neglect unimaginable in colder regions.


In hindsight, I am grateful that it did not rain during this interlude between the airport and the ferry terminal. We wait another hour on the narrow sidewalk before the gates open, bracing ourselves for a seven-hour crossing in rough weather aboard a comfortable yet chilly ship.

\chapter{Rainy Arrival in Paradise: The Salzkammergut in Bora Bora}

The rain begins softly at 6 a.m. in Papeete, then intensifies steadily. By the time we board the ferry \textit{Apetahi Express}, it has become a full-blown downpour.
A pervasive gloom, driven by massive rolling waves and an unbroken expanse of gray clouds, dominates the atmosphere.
During one of the first stops in Huahine, I see locals with umbrellas; some resign themselves to the rain, allowing themselves to get drenched as they greet their relatives or guests.
One scene stands out---a grandmother, beaming with joy as she warmly welcomes a young man with a \textit{lei}, the traditional Polynesian flower necklace. Both are utterly soaked in the process, their cheerful reunion undiminished by the downpour.

A friend in Auckland, upon receiving some videos of my passage through the wild, rumbling, grayish ocean under an equally gray, stormy sky, texts me back rather annoyed, asking me not to send any more videos of that sort.
He has had more than enough of this weather back in New Zealand---and, as it turns out,  will have to endure much more to come.

The sea remains turbulent until we reach Raiatea, the second-to-last stop. As we pass Taha'a, the final island before Bora Bora, the Polynesian universe feels enveloped in gloom and cloud---a stark contrast to the glossy brightness and endless bluish expanses depicted in postcards and podcasts.

Approaching Bora Bora, the island looms as a dark silhouette against lighter clouds, evoking memories of Austria's Salzkammergut region---a South Sea version of alpine melancholy.
The scene stirs recollections of 1986, descending into Moscow's overcast skies aboard a Hungarian Malev plane. Clouds blanketed everything---a universe of gray!

Holidaymakers often arrive in Bora Bora drenched and bedraggled, resembling ``drowned poodles''.
In this whimsical sense, Bora Bora morphs into the \textit{Sound of Music}-laden Salzkammergut or a grim Moscow, and vice versa---all merging into a watery tapestry of somber experiences.

Vaitape, the island's main town, cannot be reached directly from Raiatea or Taha'a, as the ship would risk running aground on the edges of their reefs.
BBora Bora's main island reclines lazily, like a bather in a warm tub, encircled by smaller islets known as \textit{motus} and a jagged stretch of broken coral forming a protective reef.
A coral reef surrounds the island, creating a tranquil lagoon.

These \textit{motus} and the reef create a nearly perfect ring, similar to Mauritius but unlike Hawaii or New Zealand. Inside the lagoon, the waters remain calm.
Outside, the Pacific roars with untamed force. The reef is never far away---from any vantage point on the main island, you can see the distant spray as waves crash against it. Albert Camus might have described it as ``the sea releasing its dogs''. But these waves are not the Mediterranean's gentle laps; they are towering, rolling mountains of the wild South Seas! From my hotel, their thunderous growls and snapping are ever-present, restrained by the reef's firm grip. Even in daylight, their spray rises high, creating a perpetual white mist on the horizon.

The lagoon has an inlet and outlet, a single opening in the reef situated in front of Vaitape, the island's main town on the west side. Entering the lagoon requires a journey along the southwest reef, ultimately leading to Vaitape.

As we near the lagoon's entrance, heavy clouds cling to Mont Otemanu, the island's central volcanic peak, creating an ominous spectacle. The sight resembles a colossal mushroom cloud or the aftermath of a volcanic eruption. Dark clouds churn above Mont Otemanu like a swirling mass of volcanic ash or a radioactive plume.

\includegraphics[keepaspectratio]{image-20231213_132128-contentaware-rotated-A4.jpg}

Once inside the atoll, a surprising calm prevails. The ferry glides into dock smoothly, as if it had never endured the ferocious waves of the South Sea.

I leave the icy, over-air-conditioned ship's cabin and am immediately enveloped by the steamy, warm, and humid tropical air. Sweat beads on my skin almost instantly.

To my surprise, no one seems to be awaiting my arrival. I circle the jetty repeatedly until a plump lady---almost all the women here are plump to varying degrees---finally locates me.
She ushers me and my luggage onto a bus with wooden benches and a sturdy, wood-framed chassis.
It is comfortably filled with many arrivals from my boat, though not overcrowded.
During the transfer to the hotel, I awkwardly insist on buying a phone card, which, according to my online research, should be available en route to Matira Beach. As the aforementioned plump lady astutely predicted, this proves to be a futile endeavor: the shop in question has long since been converted into a bar.

To my delight, the hotel offers excellent internet connectivity, allowing me to stay in touch with the broader world. A 4500-kilometer-long Honotua optical fiber submarine cable links Vaitape in Bora Bora to Papeete and onward to Hawaii's Big Island. This connection, maintained by the US network operator Hurricane Electric LLC, provides my current link to Europe and the USA. Connections to Auckland in New Zealand also route via Hawaii, facilitated by DRFortress, LLC in Honolulu. Additionally, the 3600-kilometer-long Manatua One Polynesia Cable connects Bora Bora to the Cook Islands, further enhancing the region's connectivity.

\includegraphics[keepaspectratio]{image-2023-bb-20231214_164131-contentaware.jpg}



\chapter{Island Crossing with Dog Bite and Mango Refreshment}
In my experience, the best way to get to know a place is to explore it on foot. Walking creates an immediate and visceral exposure to the environment, quickly revealing its true character.

This intimate exploration indeed becomes my downfall on Bora Bora. My planned route appears straightforward on my maps: crossing the island's narrow side from Vaitape, the main town, along a path less than ten kilometers wide.

Starting from Vaitape on the west side, I intend to ascend a steep slope, traverse the ridge, descend into the ``Valley of the Kings,'' cross a low pass, and finally emerge on the east coast. On the maps, the route looks deceptively simple---even accounting for roots and bushes, the path seems navigable. By Vienna standards, this would be a leisurely afternoon stroll, easily managed in spring, autumn, or winter.

My digital maps, all sourced from Open Street Map, precisely indicate the route's beginning: a long dead-end road at Vaitape's northern end, lined with traditional hut dwellings, leading directly to a path ascending into the bush. As soon as I take my first steps down this alley, however, the landscape's true nature becomes apparent.

Dogs dominate the environment, barking from every direction. Some emerge from gateless fences, advancing aggressively. Locals have explained these dogs are kept ``for protection'' and ``for security,'' though this seems at odds with widespread claims that French Polynesia---at least outside Papeete---is entirely safe and largely crime-free. Is this protective stance reserved only for locals? Or perhaps something more complex is at play.

My speculation turns toward two potential explanations. First, a pragmatic approach: residents might simply want to discourage potential intrusions, creating a psychological barrier through their canine sentinels. Alternatively, these dogs could represent an externalized aggression---a ``territorial'' valve through which suppressed tensions are released. Everywhere, ``tapu'' (taboo) signs reinforce boundaries, prohibiting entry to various spaces and properties.

Undeterred, I continue walking, passing growling dogs emerging from gardens. Suddenly, at the alley's end, a pack of different dogs circles me, barking menacingly. They seem emboldened by their numbers, making me feel hunted. A sharp pain strikes my right leg.

The confrontation abruptly ends when a young Polynesian sharply calls off his dogs. By then, I have already been bitten---a small wound that does not bleed profusely. The local simply shrugs and retreats, leaving one dog still growling in the middle of the road. My anger rises, and I fantasize about retaliating with my heavy hiking boots.

Higher up the slope, I examine the wound: a bruise and a tiny blood droplet. My American jeans have fortunately dampened the bite, which seems more a deterrent than a serious attack. A barefoot encounter would have been far more dangerous.

I later contact the Bora Bora ambulance and am greeted by a reassuring French doctor who speaks English. She confirms no dog rabies on the island but warns about potential infection risks. Her advice is pragmatic: monitor small wounds or seek immediate treatment if concerned. I choose to observe, treating the bite areas with zinc ointment and tracking the developing hematoma.

Later, seeking understanding, I also contact the Tahitian tourist office to discuss ``the apparent dog situation''. They confirm the widespread problem of dogs allowed to roam freely. The issue is more serious than a mere nuisance---at one point, a person had been killed by these uncontrolled packs, and serious injuries occurred with disturbing frequency. My minor bite, it seems, is just one symptom of a larger, more dangerous urban wildlife problem that local authorities have yet to effectively address.

With the dog encounter behind me, I venture into the bushes. The initially clear path quickly deteriorates---trash litters the muddy slope, obscuring the route to the ridge. In an upper clearing, I consult my map, and---perhaps in revenge for the dog bite---pluck a massive avocado hanging from a nearby tree, adding unexpected weight to my already heavy backpack

\includegraphics[keepaspectratio]{image-2023-bb-20231218_134531.jpg}

Where I am supposed to go according to the trail data on the map, there is only dense undergrowth. To make matters worse, I start sweating from every pore without cooling down.

Due to the previous rain and the sun, the water on the ground and on the leafs now begins to evaporate, causing the air's water vapor saturation to reach one hundred percent. As a result, the cooling mechanism of the skin through evaporation no longer works. People like me simply do not cool down.

This has nothing to do with my birthday today, which catapults me into an unimaginably Methuselah-like age. Evil tongues claim that this is precisely why I drove close to the international date line---to stay eleven hours younger than I actually am. In Auckland, it would have been the other way around!

Be that as it may, I am seriously considering, in the middle of the climb, whether I will even manage to climb this mini slope of about 150 meters. I am experiencing the feeling of age, or of not being able to walk anymore. Just last summer, I experienced something similar with a tourist on the way down from the Schafberg: the poor woman had taken the cable car up and then wanted to walk the 1300 meters downhill herself. You hardly ever allow yourself anything else! As I dashed past, balancing over hill and dale, her partner tried to give her a piggyback ride for a few meters! I do not know how she ever found her way back to St. Wolfgang in that state. I certainly did not read anything about it in the newspaper!

That is more or less how I feel as I climb up this little slope, pausing every five to ten meters, in 100 percent humidity. Meanwhile, a muddy track coincides with the route marked on the map. That is why I follow it. In particularly steep and slippery places, ropes are attached, which I gratefully use for support.

I come across a tree with giant avocados, under which there are rusting oil drums. Out of a feeling of ``revenge'' on everything, I pick one that is probably still unripe, but temptingly hanging in front of me, and I stuff it into my already quite full backpack.

You would not believe it: after not long and several breaks, I reach the ridge and turn right, following a well-trodden path.

Suddenly, completely unexpectedly, like manna to the Israelites, a bright yellow mango appears on the ground. Then another and another. The whole ground is covered with medium-sized, bright golden yellow mangos! Some of them have already been nibbled on by ants and other insects, but most still have their skin intact, and the flesh is revealed once you peel off the skin. I eat almost all of the ones I find in the area---there must be about twenty of them---and pack three more in my backpack for later. The hike is starting to be fun!

\includegraphics[keepaspectratio]{image-2023-bb-20231218_143939.jpg}

I do not want to go into any more detail here, just this much: the path is good at first, up to a point with a view of Fa'anui, the second largest town on the island. Then it becomes problematic again because it is hardly used. After a bit of back and forth, I find the path again. On the way down from the ridge, I pass a large rock with a small waterfall quite far down. Below it is a brownish natural pool. I hear the happy laughter of children from far away. In fact, a few young boys and two dogs are climbing around on the rock. I take off my jeans and swim a little in the murky water. There is no cooling off again, but at least I feel like I have bathed.

The guys give me water---I do not have enough with me---and offer me two grapefruits. They then shake a papaya tree until it reveals its ripe fruit. This, too, disappears into my backpack as a donation.

I then part ways with the nice company that is leading me to the Valley of the Kings. After a while, I come to a clearing and take a small side path. It takes me a while to realize that at the end of it, there is a huge banyan tree growing over black granite rocks, said to have grown around and integrated several royal mummies buried upright over time. I slide further across lush meadows with fruit trees and then come to a flat place, a marae, where the Polynesians used to live. Then, after a small detour, I continue along the power line to the Traversale pass and finally down to the sea on the east side of the island.

A friendly employee of a luxury resort picks me up after only a short time standing on the circulation road in her car. She cannot believe the tour I have just done.

\includegraphics[keepaspectratio]{image-2023-bb-Banjon-20231223_134921.jpg}

\chapter{Distant Island Sounds in the Luxury Overwater Bungalow: Roosters Crowing and Dogs Barking}
I find myself contemplating the auditory landscape experienced by luxury tourists nestled in their extravagant overwater bungalows, perched above the crystalline waters with the main island spread before them---a picturesque tableau
ideally only occasionally, but realistically oftentimes obscured by wandering clouds. What sonic fragments drift across the misty banks into these outrageously expensive retreats?

The dawn inevitably announces itself through a raucous rooster symphony. During the entire night you can hear roosters crowing, but it gets more intense at daybreak.
These feathered vocalists seem locked in an absurd competition, each attempting to out-screech the others with piercing calls that puncture the morning stillness.
 One can almost imagine them puffing out their chests, each believing themselves the most accomplished herald of daybreak.

Accompanying this avian chorus is the persistent counterpoint of dogs barking---a soundscape that, while perhaps less refined, remains somewhat morew tolerable than the mechanical cacophony one might encounter elsewhere.
I cannot help but contrast this with the thunderous boat engines that transform places like Railay Beach in Thailand's Krabi into sonic nightmares. Those ubiquitous photographs of seemingly pristine, tourist-free beaches raise intriguing questions. Are they the result of carefully orchestrated photo sessions where beaches are temporarily cleared? Or perhaps the magic of digital editing has simply whisked away the human presence, creating an illusion of untouched paradise.

It seems improbable that the glossy marketing materials for Bora Bora's luxury resorts would candidly advertise: ``Immerse yourself in an authentic island soundscape---wake to the melodious crowing of roosters, punctuated by the enthusiastic barking of local dogs!'' And yet, these sounds represent a genuine, unfiltered slice of island life, far more authentic than any digitally sanitized postcard view.

\includegraphics[keepaspectratio]{image-2023-bb-20231220_190915.jpg}

\chapter{A High-Quality Yellow Air Mattress Costs the Same as a Kilo of Locally Grown Tomatoes}

I survey the swimming equipment offerings at the two supermarkets and the surprisingly expansive clothing store near Matira Beach---not in Vaitape itself, but close by. The larger supermarket displays an inflatable life buoy that bears witness to economic shifts.
 Its price has dramatically jumped from 2150 Pacific francs to the current 2900 (approximately 24 Euros), a change so recent that the old price still faintly visible beneath the new price tags tells a story of inflation.

The smaller supermarket presents a different temporal snapshot. Its price tags have yellowed with age, suggesting the goods might be as brittle as the paper displaying their prices.
After careful consideration, I decide to purchase a yellow air mattress that appears surprisingly robust, priced at a reasonable 950 Pacific francs---roughly 8 Euros. A quick internet search reveals that online retailers in Europe would offer the identical air mattress for 12 Euros, with the enticing bonus of free shipping on orders over 39 Euros.
What a fascinating glimpse into the economic time capsule of local retail!

An amusing coincidence strikes me: the price of this air mattress matches precisely the cost of a kilogram of locally grown tomatoes. In this small moment of economic observation, the local market reveals its own unique pricing logic---where an inflatable recreational item and a staple food item can command the same monetary value.

\chapter{Inflating the Air Mattress Requires the Help of an American Hotel Chain}

That evening, my attempt to inflate the air mattress ends in total defeat. This mattress clearly demands a pump---unless one wishes to risk a catastrophic stroke from manual inflation! My hotel proves utterly unhelpful, lacking even a simple bicycle pump.

The neighboring establishment rents out decrepit, rust-eaten bicycles for around 13 Euros daily and possesses a pump---but it is a pitiful excuse for inflation equipment.
Reluctantly, the owner emerges with the pump. Her attempts prove disastrous; she comprehends neither the mattress valve's intricate mechanism nor the delicate art of inflation.
Missing an entire pump arm and compatible with only one valve type, the device seems more a relic of mechanical failure than a functional tool. I feel a genuine pity for the owner of such a sorry enterprise.


After several futile attempts, I abandon hope. At this point the mattress seems destined to remain a flat, useless rectangle.

Salvation arrives unexpectedly from a nearby hotel---an international chain with distinctly American origins. In their repair shed, I find unexpected assistance. An employee immediately abandons his tasks, retrieving a compressor from another storage area. The machine roars to life, yet initially fails to inflate the mattress. The young worker struggles, pressing the hose against the valve with such misguided enthusiasm that he effectively seals the air's escape route.

Fortunately, an older employee---clearly the junior's supervisor---intervenes. With practiced ease, he connects an appropriate valve to the compressor. A first attempt fails as air escapes through two suspicious holes. In a moment of desperation, I take control. Gripping the holes tightly over the valve, I create an impromptu seal. Suddenly, the mattress inflates with remarkable speed.

Triumphant grins spread across everyone's faces. I offer heartfelt thanks and head toward the sea, my hard-won air mattress in tow. Once again, an American company has rescued me from potential disaster---a recurring theme, as future stories will reveal.

\chapter{The Penetration of the Coconut with a Swiss Army Knife from Namibia}

I do not intend to dwell too much on the topic of coconuts, but I will say this much: in Vaitape, I had the pleasure of enjoying a delicious, plump coconut for a reasonable price of 300 Pacific Francs (about two and a half Euros). It was purchased and opened by an older gentleman, who made the process look effortless.

However, my experiences with the coconuts found scattered on the beaches, fallen from the palm trees, have been far less enjoyable. While it is easy enough to pick one up, opening it is an entirely different matter. Back home, I manage to cut through the first fibrous layer, and then, with the help of a corkscrew and a sharp point, I manage to drill a hole into the coconut. But the watery liquid that escapes is a pale imitation of what I enjoyed in Vaitape, both in taste and in quantity. The yield is far smaller, and the liquid, though refreshing, lacks the full flavor of the fresh coconut. In the end, I discard the coconut I brought back from the beach, a sad reminder of the disparity between expectations and reality.

By the way, these falling coconuts are not without danger. I can already picture the headline in an Austrian tabloid: ``Viennese Pensioner Killed by a Coconut on Bora Bora!''



\chapter{The Corpulence of the Polynesians is Striking: A Sign of Physical Strength or Poor Nutrition?}

A commentary from in a respected publication notes that the parliamentary opposition of the island nation has described the health situation as ``catastrophic''. Indeed, I have never encountered such widespread obesity anywhere, across both genders.

Remarkably, the physical differences between men and women---or indeed, between different gender identities---are often hard to discern due to the overwhelming presence of body mass. Gender is usually only identifiable by clothing. We are fast approaching the Rae-Raes, which will be discussed later.

This obesity is further exacerbated by the oppressive heat and a general reluctance to hide or downplay the extra pounds. Indefinable bulges of fat sit in the water along the beach, a stark contrast to the idyllic surroundings.


\chapter{Begging on the Dream Beach}

Thilo Sarrazin, during his tenure as Berlin's finance senator, allegedly tested with his wife whether and how one could survive on social security benefits in Germany. Specifically, the two of them devised a savings plan that they adhered to for some time.

I have undertaken something similar here in Bora Bora. Friends in Auckland had warned me that eating in Bora Bora would bankrupt me. Fortunately, that has not been the case.

First, it is important to understand what is expensive and what is not. For example, baguettes are cheaper here than in many other places, as they are subsidized by the government. Consequently, some of these baguettes, supposedly, even end up being fed to livestock.

What else can you eat besides a baguette?
The answer is: almost everything is expensive. But there are interesting exceptions! For instance, I often pair baguette with olive oil and tomatoes or avocados. Local tomatoes, in particular, are quite costly, costing around 8 Euros per kilo---though
I cannot quite explain why. Perhaps it is because there is little alternative but to buy from very few supermarkets, or maybe it is because everyone grows their own tomatoes, leading to what I can only describe as ``tourist prices,''
 a phenomenon that was rumored in Austria during the post-war period. I encountered the same issue in New Zealand, where a small 400-gram bag of tomatoes easily costs two and a half Euros.
One has to wonder whether there exists a price point beyond which everyone who can will grow their own tomatoes, or consumption simply collapses!

However, if you go hiking---such as in Vaitape, where I was bitten by a dog, or in the Valley of the Kings near Fa'anui---you occasionally find avocados for free, without having to steal them. You also come across giant grapefruits lying by the side of the path, which help weigh down your backpack. Mango trees, laden with fallen, sometimes burst, ripe mangoes, are everywhere. The abundance is so overwhelming that, even though I often swim for long periods and shower with soap every day, I wake up in the morning to the smell of mangoes emanating from my skin, a peculiar yet sweet scent of perspiration. I have also found papayas and green bananas, which I place in the sun to ripen and turn yellow.

A liter of organic Spanish olive oil costs only was affordable 10 Euros at the U-Mart in Vaitape. I promptly grab a bottle. A week later, when I check the shelf again, there is only one left.

Currently, I also have a stash of Tyrolean jams, packaged in small cubes with one side of aluminum that you can easily tear open. These little fruit delights from a much colder country somehow made their way here---a testament to the Austrian-German post-war export miracle!

But how does one secure good, inexpensive protein here? The insider tip is freshly caught tuna! There are two types: one, redder and more flavorful, and the other, a whiter, more tender variety.
Both types of tuna fillets are sold whole, priced around 11 Euros per kilo everywhere. I stumble  upon the tuna after seeing two elderly women in the center of Vaitape, standing in the blazing heat with a refrigerated container in front of them,
a sign reading ``tun''. I ask them if I could have a look at the goods; they hesitantly show  me a huge tuna fillet, well-wrapped in cling film. They explained that this piece would cost 1,600 Pacific Francs (around 14 Euros).
After withdrawing cash from an ATM, I return and buy the fish for 1,500 Pacific Francs, having misunderstood the price earlier. Later, I see these fresh tuna fillets for sale in supermarkets throughout the island. This solves my protein dilemma here!

Additionally, if you have a hot water kettle in your hotel (or bring a portable immersion heater), you can easily make thin noodles with just hot water. Tomato paste, which you can pour over the noodles (perhaps with a little seasoning), is also widely available. This reminds me of an elderly German man I once met in the Sinai Peninsula in Egypt, following a rather unpleasant episode of vomiting and diarrhea. He explained to me that he only fed himself boiled noodles with tomato sauce.

Rice, however, is a more complicated issue. You can soak it for an extended period, as the Iranians do with great skill. But for that, you'll need a gas stove and at least a small cooking pot. If you bring your cooking utensils with you, butane gas cartridges (type 206) are readily available here, and they cost only a little more than they would at home.

So, I believe that even on a modest German welfare budget, you could live fairly well on food here in Bora Bora---provided you do not factor in accommodation and avoid booking into a luxury hotel on a remote motu. The overnight rates at those resorts are so high that food prices almost become irrelevant!

To put it in perspective: A night in a luxury over-water bungalow on the east side will soon cost as much as my three and a half weeks spent in a bungalow on Matira Beach!

\includegraphics{image-2023-bb-20231231_122109.jpg}

\chapter{Life with and on the Air Mattress is Beautiful!}

On Christmas Day, December 25th, the sky unexpectedly clears after a week of intermittent rain. As I receive my breakfast, I can hardly believe my luck: I can even see blue skies!

This, of course, motivates me to make the most of the day with water activities. So, I quickly slurp down my magnesium-vitamin C water and the somewhat peculiar coffee offered to me---one I don't want to discard out of convenience. After successfully relieving myself, I don my long black New Zealand sports trousers, a long-sleeved black sports T-shirt, diving shoes (which prove very important later), and two hats---one being a wide-brimmed fisherman's hat from China (everything I wear and use is from China anyway, including the air mattress). I tuck the yellow air mattress under my arm.

Before heading out, I apply a generous amount of UV sunscreen, followed by extra titanium dioxide on areas not covered by clothing, ensuring I am well-protected from the sun's rays.

I make my way towards ``Matira Point,'' located at the tip of a spike of land jutting into the sea, towards a motu (small island) known for its ``coral garden''. I have already seen the bay and the surrounding path from above while walking to a cellphone tower on a hill. I figure I can reach the spot nearly without swimming, simply by ``water walking''.

Matira Point itself is marked off with private signs saying ``tapu'' (forbidden), ``off-limits,'' and ``dogs''. So, I enter the water just before the point, at a boat dock on the left. I wouldn't recommend attempting this water walk without diving shoes or at least running shoes with thick street soles, as there are an abundance of sea urchins, and I'm constantly stepping on coral.

However, the landscape is absolutely breathtaking when the sea is calm! The water is crystal clear, transitioning from light green to azure blue, with the bottom often covered in bright white sand. On the left, the volcanic mountains rise, their peaks often shrouded in clouds. The highest of these is Mont Otemanu, with its almost square top and steep, nearly vertical edges. To the right, the islands of Raiatea and Taha'a are visible. The sight is truly magnificent in such weather: calm, gentle waves and sunshine. I'm also wearing tinted polycarbonate glasses, closed on the sides, similar to those used by workers in New Zealand. Without these, I'd barely be able to tolerate the intense light! Alternatively, one could wear regular sunglasses along with untinted polycarbonate safety glasses that block UV light.

On the air mattress, I have brought a bottle of fresh water---one I plan to greedily drink later---as well as two pairs of diving goggles, one of which comes with a snorkel.

The spot I'm aiming for is easy to locate, as there are several small and large observation boats anchored nearby. As I draw closer, I see snorkelers frolicking in the water.

However, about two-thirds of the way there (approximately 500 meters in total), the water has risen nearly to my neck. Maybe I misjudged something, or perhaps I did not follow the shallow part of the bay? Either way, the water is lukewarm, and I swim slowly towards the boats. Or rather, I hold onto the ends of the air mattress, using my feet to paddle through the water.

As I approach, I hear the loud, happy shrieks of my fellow snorkelers. I pull my goggles from the air mattress, place them on, and slip on my snorkel. Almost immediately, I spot the first ``boulders'' or submerged elevations, with a variety of fish swimming around.

At first, I think to myself: ``Well, this is not much better than anywhere else''. But the longer I snorkel, the more diverse and beautiful the scene becomes. The coral garden is vibrant, with many species of fish in various patterns and colors. The name truly fits: it is like a large aquarium!

Groups of tourists are constantly being ushered in and out of the water, often with crude comments like, ``Fish nice, eh?'' or ``Bora Bora nice, you like?'' which does not bother me in the slightest.

By the way, there are some very cheeky little fish here that are not shy about nibbling at your legs if you remain still for too long. They are small, flattened, black with a golden shimmer on their sides. Here, it seems, everyone eats whatever they can grab. Next time,
I intend to bring a baguette with me, as a tour operator here suggested. In Thailand, they throw fish food into the sea to give tourists something to talk about when they get home. Austrian chocolate slices were also quite popular with the fish there!

Once I have had enough of swimming among the coral reefs, I decide to visit a small motu that my host recommended to me, which is not too far away. She claimed there are beautiful corals there as well.
Unfortunately, this information turns out to be inaccurate, but the journey and the destination prove interesting all the same.

I have to swim farther than I anticipated until the ground becomes shallow enough for me to walk. It is tiring, though, as each step is met with resistance from the water---a kind of ``snow walk!'' But corals are everywhere on the sandy bottom.
Without diving shoes or thick-soled sneakers, I would not be able to manage!

The coral reefs here are smaller and simply a reflection of the aquarium I just left. As I approach the motu, a local woman points to a fin sticking out of the water and calls, ``Shark!'' I hastily try to reach dry land.

Sure enough, she later shows me the young shark swimming lazily around. It is likely there to ``clean up'' discarded mussels, which the locals extract from the rocks with flathead screwdrivers. A small ray also appears. From my observation, rays are often present when there's something to clean up!

The motu is only about 20 meters from the reef, and I notice two people walking along the reef. That is exactly where I want to go! The reef forms an interface or boundary between the deep South Pacific and the shallow atoll. Good shoes are essential here, as the bottom is covered with coral and sharp edges. I place my air mattress in a pool of water with rather dull edges and set off on a short walk. As the local man returning from his own walk explains, it is a calm day, and he has gathered a good number of mussels---his back pocket is overflowing with them. But, from my perspective, the water here seems anything but calm. At one point, I lose my footing due to the force of the water lapping against the atoll. Thankfully, I manage to steady myself without injury. Sharp, jagged edges are everywhere!

Finally, I reach the ridge, about five meters wide. It sinks slightly when I stand on it, with the sea roaring in from Raiatea and Taha'a. Thick clouds hang over both islands, and I can see streams of rainwater cascading down the slopes. It is even possible to do a reef hike from here to the next motu---the reef is flat and gentle, though likely yielding. One wonders what it would be like to be washed out to open sea from here!

I turn around and slide back toward the motu, retrieve my air mattress, and ask a local if it is okay to walk or swim towards Matira Beach. He does not understand me at first but eventually tells me that I can swim first, as it is still deep enough, and then I can walk in the water. So, I leave the motu, heading towards the hotel, happily letting the tailwind carry me along, especially as the clouds from Raiatea and Taha'a grow thicker.

Upon reaching the beach, I spot a French tourist, her face bright red, taking photos with her mobile phone.

The next day, I attempt to visit the coral garden again. This time, however, the wind is so strong that the waves are half a meter high even within the atoll, and my air mattress stands up like a sail. With considerable effort, I manage to get close to the coral garden, but then I'm picked up by an American from San Diego, who is using two boats from a luxury hotel with his family. He kindly takes me back to the hotel.

Water has seeped into my cellphone case without causing any noticeable damage. When I attempt to climb up the boat ladder, I injure my finger. I realize: once again, Americans have ``saved'' me, making possible what I would not have been able to do on my own.


\includegraphics[keepaspectratio]{image-2023-bb-20231216_094817.jpg}

\chapter{Evening Scene in the Neighboring Bungalow}

A remarkable scene unfolds next door: a family---father, mother, and two daughters, around 9 and 12 years old---are enjoying two pizzas together for dinner in front of their bungalow.

The daughters have made themselves comfortable on the most plush, padded loungers, relaxing without a care. The mother, however, sits on a simple wooden chair that offers little in the way of comfort. The father, who is probably the one funding the entire experience, does not even have a proper seat at first. During the meal, he settles on a narrow stone staircase. The way it is positioned forces him to twist his body unnaturally in order to face his family.

After dinner, he rises and moves to sit on the armrest of one of his daughter's beach chairs. He chats cheerfully, relaxed as can be---one might imagine him, to paraphrase Camus, as a happy and contented person.

Then, suddenly, the mood shifts: one of the daughters begins to cry. It is not a quiet sob, but an intense, guttural cry, filled with raw emotion and grief. She is practically shaking with sorrow.
Meanwhile, the other daughter, seemingly unaffected, giggles happily to herself, her carefree laughter in stark contrast to her sister's distress.
Eventually, both emotions fade as they drift off to sleep.

\chapter{Bora Bora was Taken Over by the French After the Germans Wanted ``A Place in the Sun'' There Too}

In the 19th century, the then remaining dominating maritime colonial powers, Great Britain and France, divided some of the last paradisiacal corners of our planet through tough negotiations. Oil and energy were not yet a major concern. Those fortunate enough to be ``neutralized'' by these powers were spared from direct colonization. Bora Bora, much like Hawaii, initially remained neutral in this regard. However, the colonial powers treated the indigenous population as astutely described by Brecht: ``When it rained and they encountered a new race, be it brown or pale, they might make their beefsteak tartare out of it''.

This system held until other states started asserting their claims. The United States, in the case of Hawaii, was far too powerful to be challenged without significant losses. Germany, in particular, began pushing into the remaining territories and forgotten corners of the world, such as German South West Africa, now known as Namibia. In a speech to the Reichstag on December 6, 1897, Bernhard von B�low made Germany's intentions clear: ``We do not wish to overshadow anyone, but we demand our place in the sun''. He likely included Polynesian islands like Bora Bora in his expansionist vision.

However, France had already taken precautions. Bora Bora, known locally as ``Pora Pora,'' was annexed on April 17, 1888, and the last queen, Teriimaevarua III, was allowed to ``rule'' until 1895.
After her abdication, she kept her court until her death, but the French replaced her with a bureaucrat who adopted the title of vice president.
Thus, there was no need for a reconquista after the German ambitions were swiftly diminished from the outset, colliding with the Franco-Polynesian reefs.

Since then, French has been the official language, with occasional Polynesian expressions, such as ``maruuru'' for ``thank you''.

\includegraphics[keepaspectratio]{image-2023-bb-20240101_113626.jpg}

\chapter{Water Sitting and Lagoon Drifting}

Speaking of a place in the sun: A friend from Upper Austria recently asked me why there's no one on the beach in my photos---are the beaches here really empty?

When the sun shines on Matira Beach, it is so intense that it becomes nearly impossible to sunbathe for long. To quote Albert Camus again: the beach is ``black with sun''.

What is possible, however, is drifting around in the lagoon's water. On one hand, the water temperature is about 28�C when overcast, and it gets even warmer when the sun is out. When the sun beats down all day, the water near the shore becomes almost too hot, prompting me to wade into slightly deeper areas where the water reaches my navel. The buoyancy is so great that competitive swimming feels almost impossible.

Obese locals, seemingly unaffected by the heat, sometimes sit with only their lower bodies submerged in the water, gossiping with their boyfriends or girlfriends.

\includegraphics[keepaspectratio]{image-2023-bb-20231217_162207.jpg}

\chapter{Deconstruction}
The situation has hardly changed since Gauguin: Many Polynesians live in corrugated iron huts, which, together with the ``front gardens'', appear incredibly neglected. The French are ``running'' the island, which the Polynesians turn up their noses at, but they know that they are better off in the end than if they were in charge of the administration alone.

The only international airport in this island kingdom, Fa'a'{a}, has no air conditioning. Occasionally, fans swirl the hot air, which is completely saturated with water vapor. Papeete's town hall has a clock tower that probably stopped at some point after the French handed the islands over to the locals in 1984. Apparently, no one cares what time it is: the clock shows 7:38, and since then time has stood immovably still there.

This jammed clock is probably a symbol and warning sign of the general depravity that the locals are struggling with. Nobody cares about the dogs that are roaming around everywhere, often leading a miserable existence, about the screeching roosters and the poultry, about the manure in the gardens and courtyards: what is considered ``hoarding'' in Europe is the norm here.

I think the French have given up on colonialism with a shudder and now only take what they need, without any pretense of statehood. In the end, French Polynesia was probably just too expensive for them and they got rid of it like a hot potato. Officially, however, they were politically correct in their decolonial and deconstructive attitude. Derrida and Foucault would certainly be delighted with that and would babble about it in a weighty, sophistical way!

\includegraphics[keepaspectratio]{image-2023-bb-20231215_151852.jpg}


\chapter{Aeolian Harps}
No one can tell me what I am hearing. But after I arrive, I hear wind harp songs behind my bungalow for a while, as if the siren there, which is supposed to sound tsunami warnings, is being played by the wind in a ``civilian job''.

\includegraphics[keepaspectratio]{image-2023-bb-20231215_140758.jpg}

\chapter{Island of Species Cooperation}
While in Australia the strategy of ``biting and being bitten'' has become the norm over the course of species evolution, on Bora Bora a different strategy prevails, namely that of ``getting out of each other's way'' and cooperation.
This means that one can safely go into the bush or into the grass at night without having to worry about being bitten by a Western Brown. Tiny lizards scurry about everywhere, but their agility is more a sign of their defensive attitude than their aggression.
 Instead of Robert Hughes' ``The Fatal Shore'' there is freedom of movement everywhere within the atoll.

\includegraphics[keepaspectratio]{image-2023-bb-20231231_141824.jpg}


\chapter{Fleeting Encounters with Mahus and Rae Raes}
Immediately after his arrival, the Polynesians thought Paul Gauguin was a \textit{Mahu}, a ``mediator'' or ``woman-man'', because of his long, flowing hair and his eccentric clothing style.
Gauguin, who at first did not know what to do with this and often associated with young Polynesian women, allegedly had his hair cut immediately.

What strikes me immediately after my arrival, as I wait for the ferry with other passengers: a masculine-looking lady! She is not dressed particularly conspicuously, wearing a cotton dress with bright floral patterns. I try not to stare at her or look at her. But just as surely as my mind identified her as a ``gentleman pretending to be a lady'', she identified me just as quickly as someone who is looking at her. Otherwise she would not stand out; everyone else acts as if it were the most natural thing in the world.

Then in Huahine another ``lady'' boards the boat, this time wearing a striking feather sash and an orange knitted dress with no shoulder covering. She looks a little conspicuous and seems to know everyone and is treated with respect and kindness.

The next day on the beach at Matira I see a somewhat implausible ``she'' in a tight bikini, together with another lady, both of them usually obese, and a small child---maybe teirs.

Then I see again in front of the church in Vaitape a ``she like that'' in a colorful dress.

In the 2022 film ``Pacifiction,'' Pahoa Mahagafanau plays a trans woman named Shannah who is an ally of sorts to the French High Commissioner.

It takes some time for me to realize that these ladies here are called \textit{Rae Raes} and are treated with great respect, like Mahus who were honored or at least treated with respect in ancient Polynesian culture.

I wonder: What is the percentage of such inclinations or determinations here?

Assuming a frequency of five to ten percent of homosexual orientation in the total population, then the transgender proportion is reduced again by a factor of ten. I imagine that this is traditionally practiced here, rather than suppressed like in Western cultures.

The existence and open display of transvestism also shows how little Christian missionary activity has ``taken effect'' here in the South Seas. Evil tongues claim that the gentle South Sea chants that one seems to hear in Christian churches in Polynesia are the only reason why the locals visit these churches: to come together and sing. Or rather, as I observed at the Christmas mass in Vaitape, to shout rather than sing.

I am thinking of a documentary about the missionary who was killed by the Sentinelese not so long ago.
It features a former missionary who lost his faith because of ``his'' indigenous people, who asked him ``plausible'' questions about the origins of his religion. Here the ``primitive'' appears as an ``enlightener''.
Maybe these indigenous people also remain in a kind of limbo: because contact with the Europeans also caused the disintegration of their one faith. What remains are the eternal metaphysical questions: why does something exist rather than not, and is there anything that ``survives''
or ``escapes'' the disintegration of the body?

\chapter{Other Islands of the Society Islands Archipelago}

The Society Islands are divided into two subgroups: the Leeward Islands, which include Bora Bora, Raiatea, Taha'a, Huahine, and Maupiti, and the Windward Islands, which encompass Moorea, Tahiti, and Tetiaroa.

I visited two of these islands: Tahiti, with its extension Tahiti-Iti, where the international airport Fa'a'{a} is located near the capital Papeete, and Moorea, a short ferry ride from Papeete.

I circumnavigated the largest island, Tahiti, twice. First, I took two buses for a basic overview. For some inexplicable reason, there are no buses that fully circle the island. Instead, two buses each travel halfway around the island in opposite directions, and passengers must transfer at the common terminus in Taravao. Taravao is also home to a giant Carrefour supermarket.

From Taravao, you can explore the smaller island of Tahiti-Iti, which is separated from Tahiti by a narrow isthmus. Unlike the main island, Tahiti-Iti cannot be completely circumnavigated.

The coastal ring road of Tahiti can easily be traveled in one day, even with a few stops at tourist attractions such as a water hole, a giant waterfall, and various parks and caves. I did this the following day with a rental car, which seemed a bit expensive but worth it.

Once you venture inland, you're immediately met with an impenetrable rainforest. The jungle is thick and wild, with only a few poor roads leading into it. The coastal road itself is lined with homes that appear somewhat rundown---after all, not everyone wants to show the side of their house facing the main road.

Near Papeete, I found a lovely beach, though a sign warned of jellyfish. Despite the warning, many young people were out surfing the heavy waves.
Before reaching the beach, I passed an avenue of mango trees, their fruits scattered on the ground in various stages of ripeness and decay.

I did not come across any other beaches that were inviting enough to swim at. Indeed, after circling the island twice, I felt I had seen enough.

Moorea remains unforgettable to me because of an unexpected downpour I encountered, completely unprepared. I hitchhiked and then walked a nicely laid out path to the Belvedere viewpoint, where I could see two bays with a massive mountain rock formation in between.
I got only moderately wet during the light drizzle, and the path was interesting---it led through indigenous settlements, which steamed in the thick, tropical vegetation. Afterward, I visited another viewpoint with palm trees and a swing hanging from two of those trees.

Despite warnings from a local French cyclist, who advised me to return to the valley due to the weather, I decided to ignore his advice. My online hiking map showed a direct route along a wooded mountainside leading to the ferry.

This decision proved problematic: The path was partially washed out, longer than expected, and became a muddy trail due to the heavy rain. In the middle of my hike, the downpour intensified, and the trail became so slippery I was sliding around. Luckily, two local youths noticed me and did not just pass by. They stayed with me, guiding me carefully down to the valley.

After a final, perilous slide down a steep, mud-covered slope, I made it safely to the ferry. My fisherman's hat, however, was lost somewhere in the water, but that was a small sacrifice considering what could have happened!
\backmatter

\chapter{Something More}

The construction of paradise through the Pacific island world, embodied here in Bora Bora, succeeds only partially. The island's neglect is palpable:
shabby shacks dot the landscape, while oppressive humidity induces instant perspiration with no respite.
Roosters pierce the hot nights with their strident calls, and a cacophony of canine sounds accompanies them---packs of dogs emerging from driveways, blocking paths with threatening postures, teeth bared and growls menacing.

In short: If this is Bora Bora's paradise, then I long for nirvana!

The weather proves maddeningly capricious. During the rainy season, a two-week vacation can dissolve into an endless grey tableau.

 I observed two Americans wading halfheartedly through the shallow lagoon, the sky a solid canopy of clouds obscuring the sun since early afternoon.
Their expressions betrayed a universal disappointment: Is this why we traveled halfway across the world?


Yet when the sun finally emerges, both locals and tourists justifiably speak of paradisiacal conditions---provided one can endure the scorching heat.
The lagoon transforms into a mesmerizing canvas: azure blue deepening with water depth, white-green shallows giving way to profound blue expanses.
One becomes enveloped in a surreal luminescence, surrounded by what poets might call ``everywhere and forever the distance shines bright and blue''!

\chapter{Afterword}

The travelogue Besides Paul Gauguin's ``Noa Noa'' (Fragrant Scent), ``The Happy Isles of Oceania: Paddling the Pacific''  by Paul Theroux is probably still standard South Seas literature. This travel diary through the South Seas was recommended to me by my esteemed friend Herbert Gottweis, who was torn from this world far too soon by a malignant brain tumor into the unknown, ``from which no man ever returned''---except perhaps Jesus, the anointed ``Christ'', at least if one is to believe today's theologians. Their guild once branded and persecuted those of different faiths as ``heretics'': After the Council of Nicaea, all who doubted the divinity of Jesus of Nazareth and his resurrection were persecuted, banished, and if necessary, burned.

The abundance of Theroux's impressions, which he experienced while quasi paddling through the Pacific, is impressive. The chapter ``New Zealand: Sloshing through the South Island'' alone lacks no clarity and coincides with much of what this author has experienced on the North Island of Aotearoa.

However, Theroux's journey is now also over thirty years ago and allows for additions and supplements. This little book is such a supplement. May it delight and inspire the readership! For man is very often much freer than he believes himself to be, and summer in winter is just like the light in the heavy, the bright in the dark, and the dark in the radiant.

\chapter{Acknowledgments}

At this point, I would like to thank the ancien r\'egime of my university, which left no stone unturned---almost lovingly---in detail to dismiss their ``elders'', and thus me; and the European Court of Justice, which in its decision clearly stated that it is perfectly acceptable in the EU area to discriminate against older people [ECJ judgment of 21.07.2011---joined cases C-159/10 (Fuchs) and C-160/10 (K�hler)], even contrary to EU Directive 2000/78/EC regarding the prohibition of discrimination on grounds of age: I was sent into compulsory retirement by both.

In contrast, my colleagues, for example in the USA, New Zealand, and elsewhere, are not forced into retirement but can continue to research and teach without restriction at their universities. But they must or should do so! For had the authorities not collectively banished me to retirement, I would probably not have had the opportunity to travel to Polynesia, but would still be teaching mathematical methods of theoretical physics in cold countries in the lecture hall!

So I owe it to my former rector and the European high court judges that I was able to get to know this beautiful spot on Earth. Albert Camus spoke of the absurd freedom that I now enjoy, whether I like it or not.

\newpage


\thispagestyle{empty} % Remove page number

\includegraphics{image-2023-bb-VeroHiti.jpg}

\end{document}


