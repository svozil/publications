%%%%%%%%%%%%%%%%%%%%%%%%%%%%%%%%%%%%%%%%%%%%%%%%%%%%%%%%%%%%%%%%%%%%%%%%%%
%% For support : Yolande Koh, <ykoh@wspc.com.sg>
%% Technical assistance : D. Rajesh Babu, <rajesh@wspc.com.sg>
%% Master File for Book (updated on 01-09-2016)
%% Book Trim Size: 9.61in x 6.69in
%% Text Area: 8in (include runningheads) x 5in
%% Main Text: 10/13pt
%%
%% The content, structure, format and layout of this style file is
%% the property of World Scientific Publishing Co. Pte. Ltd.
%% Copyright 2016 by World Scientific Publishing Co.
%% All rights are reserved.
%%%%%%%%%%%%%%%%%%%%%%%%%%%%%%%%%%%%%%%%%%%%%%%%%%%%%%%%%%%%%%%%%%%%%%%%%%

%\documentclass[wsdraft]{ws-book961x669} % to draw border line around text area - wsdraft option
\documentclass{ws-book961x669}
\usepackage{ws-book-thm}      % comment this line when `amsthm / theorem / ntheorem` package is used
\usepackage{ws-book-har}      % Author-Date citation model
\usepackage[pdfpagelabels=false]{hyperref}  % \label, \ref and \cite are recommended


\usepackage[dvipsnames]{xcolor}



\renewcommand{\frak}{\cal}
\newcommand{\marginnote}{\footnote}
\newcommand{\sidenote}[2][]{\footnote{#1}}
\newcommand{\newthought}{}
\newenvironment{marginfigure}{\begin{figure}}{\end{figure}}
%\usepackage{amssymb,amsmath}

\usepackage{curves}
%\usepackage{eepic}
%\usepackage{dingbat}
\usepackage{wasysym}
\usepackage{fourier}
%\newcommand{\bexamples}{{\Huge $\ulcorner$}}
%\newcommand{\eexamples}{{\Huge $\lrcorner$}}
%\newcommand{\bexample}{{\large $\lceil$}}
%\newcommand{\eexample}{{\large $\rfloor$}}
%\newcommand{\bproof}{{\large $\lceil$}}
%\newcommand{\eproof}{{\large $\rfloor$}}
\newcommand{\bexample}{ }
\newcommand{\eexample}{ }
\newcommand{\bproof}{ }
\newcommand{\eproof}{ }
\usepackage{tikz}
\usetikzlibrary{positioning}
% The face style, can be changed
\tikzset{face/.style={shape=circle,minimum size=4ex,shading=radial,outer sep=0pt,
        inner color=white!50!yellow,outer color= yellow!70!orange}}

\usepackage{pgfplots}

%% Some commands to make the code easier
\newcommand{\emoticon}[1][]{%
  \node[face,#1] (emoticon) {};
  %% The eyes are fixed.
  \draw[fill=white] (-1ex,0ex) ..controls (-0.5ex,0.2ex)and(0.5ex,0.2ex)..
        (1ex,0.0ex) ..controls ( 1.5ex,1.5ex)and( 0.2ex,1.7ex)..
        (0ex,0.4ex) ..controls (-0.2ex,1.7ex)and(-1.5ex,1.5ex)..
        (-1ex,0ex)--cycle;}
\newcommand{\pupils}{
  %% standard pupils
  \fill[shift={(0.5ex,0.5ex)},rotate=80]
       (0,0) ellipse (0.3ex and 0.15ex);
  \fill[shift={(-0.5ex,0.5ex)},rotate=100]
       (0,0) ellipse (0.3ex and 0.15ex);}

\newcommand{\emoticonname}[1]{
  \node[below=1ex of emoticon,font=\footnotesize,
        minimum width=4cm]{#1};}


%\definecolor{lightgrey}{rgb}{0.95,0.95,0.95}



\title{Mathematical Methods of Theoretical Physics}         % your book title here for even page running header

\makeindex

\begin{document}


\titlepages                        % pls. do not remove this line

%\begin{dedication}
%\large Dedication Page \\[13pt]    % input your dedication data here
%\large (optional)
%\end{dedication}

%\blankpage                        % blank page with no running heads

\input 2018-mm-ch-preface.tex                  %\begin{preface}...\end{preface}

%\input foreword.tex               %\begin{foreword}...\end{foreword}

%\input acknowledgments.tex        %\begin{acknowledgments}...\end{acknowledgments}

\tableofcontents

%\listoffigures                    % list of figures, optional

%\listoftables                     % list of tables, optional

\setcounter{page}{1}



%\part{Second Part}{}              % divider page, optional

\part{Linear vector spaces}{}
%\part{Linear vector spaces \vskip 9 true cm \begin{center}\includegraphics[width=0.7\textwidth, angle=-20]{2018-mm-swimmer.png}\end{center}}
\input 2018-mm-ch-fdvs.tex
\input 2018-mm-ch-tensor.tex
\input 2018-mm-ch-projgeom.tex
\input 2018-mm-ch-gt.tex
\part{Functional analysis}{}
%\part{Functional analysis \vskip 9 true cm \begin{center}\includegraphics[width=0.7\textwidth, angle=-20]{2018-mm-swimmer.png}\end{center}}
\input 2018-mm-ch-ca.tex
\input 2018-mm-ch-fa.tex
\input 2018-mm-ch-di.tex
\part{Differential equations}{}
%\part{Differential equations \vskip 9 true cm \begin{center}\includegraphics[width=0.7\textwidth, angle=-20]{2018-mm-swimmer.png}\end{center}}
\input 2018-mm-ch-gf.tex
\input 2018-mm-ch-sl.tex
\input 2018-mm-ch-sv.tex
\input 2018-mm-ch-sf.tex
\input 2018-mm-ch-ds.tex

% for BibTeX users
\bibliographystyle{ws-book-har}    % Bibliography: Author-Date system
\bibliography{svozil}      % pls. call your database here

% for non-BibTeX users
% \input bibliography.tex

\printindex

\end{document}
