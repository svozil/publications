%\documentclass[pra,showpacs,showkeys,amsfonts,amsmath,twocolumn,handou]{revtex4}
\documentclass[amsmath,red,table,sans,handout]{beamer}
%\documentclass[pra,showpacs,showkeys,amsfonts]{revtex4}
\usepackage[T1]{fontenc}
%%\usepackage{beamerthemeshadow}
\usepackage[headheight=1pt,footheight=10pt]{beamerthemeboxes}
\addfootboxtemplate{\color{structure!80}}{\color{white}\tiny \hfill Karl Svozil\hfill}
\addfootboxtemplate{\color{structure!65}}{\color{white}\tiny \hfill QCrypt\hfill}
\addfootboxtemplate{\color{structure!50}}{\color{white}\tiny \hfill Dec 2008\hfill}
%\usepackage[dark]{beamerthemesidebar}
%\usepackage[headheight=24pt,footheight=12pt]{beamerthemesplit}
%\usepackage{beamerthemesplit}
%\usepackage[bar]{beamerthemetree}
\usepackage{graphicx}
\usepackage{pgf}
%\usepackage[usenames]{color}
%\newcommand{\Red}{\color{Red}}  %(VERY-Approx.PANTONE-RED)
%\newcommand{\Green}{\color{Green}}  %(VERY-Approx.PANTONE-GREEN)

%\RequirePackage[german]{babel}
%\selectlanguage{german}
%\RequirePackage[isolatin]{inputenc}

%\pgfdeclareimage[height=0.5cm]{logo}{tu-logo}
%\logo{\pgfuseimage{logo}}
\beamertemplatetriangleitem
%\beamertemplateballitem

\beamerboxesdeclarecolorscheme{alert}{red}{red!15!averagebackgroundcolor}
\beamerboxesdeclarecolorscheme{alert2}{purple}{orange!15!averagebackgroundcolor}
%\begin{beamerboxesrounded}[scheme=alert,shadow=true]{}
%\end{beamerboxesrounded}

%\beamersetaveragebackground{green!10}

%\beamertemplatecircleminiframe
\newcounter{nc}[part]
\setcounter{nc}{1}
\begin{document}

\title{\bf \textcolor{red}{Quantum Cryptography \\With Chocolate Balls}}
%\subtitle{Naturwissenschaftlich-Humanisticher Tag am BG 19\\Weltbild und Wissenschaft\\http://tph.tuwien.ac.at/\~{}svozil/publ/2005-BG18-pres.pdf}
\subtitle{\textcolor{orange!60}{\small http://tph.tuwien.ac.at/$\sim$svozil/publ/2008-QCrypt-pres.pdf}}
\author{Karl Svozil}
\institute{Institut f\"ur Theoretische Physik, Vienna University of Technology, \\
Wiedner Hauptstra\ss e 8-10/136, A-1040 Vienna, Austria\\
svozil@tuwien.ac.at
%{\tiny Disclaimer: Die hier vertretenen Meinungen des Autors verstehen sich als Diskussionsbeitr�ge und decken sich nicht notwendigerweise mit den Positionen der Technischen Universit�t Wien oder deren Vertreter.}
}
\date{December 2008}
\maketitle



%\frame{
%\frametitle{Contents}
%\tableofcontents
%}

\section{Principles of Quantum Information}

\frame{
\begin{center}\Huge
{\color{purple}    Part \Roman{nc}:  \\
Principles of Quantum Information}
\end{center}
\addtocounter{nc}{1}
\begin{center}{\color{lime}
$\widetilde{\qquad \qquad }$
$\widetilde{\qquad \qquad}$
$\widetilde{\qquad \qquad }$ }
\end{center}
 }


\subsection{Concepts}
\frame{
\frametitle{Concepts}

Qubits are the fundamental units of quantum information.
They refer to quantum states and observables which behave nonclassically;
in particular they are capable of
\begin{itemize}
\item<1->
complementarity (incompleteness),
\item<1->
value indefiniteness (randomness),
\item<1->
coherent superpositions (parallel co-representation of classically mutually excluding cases), and
\item<1->
entanglement
(information spread over a multitude of particles or observables),
\item<1->
``interaction-free'' counterfactual potentiality.
\end{itemize}
At the same time they are subject to {\em reversibility} in-between ``irreversible'' measurements.

All of these features can be used to efficiently and securely compute, distribute and transfer information.


}

\subsection{Status}
\frame{
\frametitle{Status}

\begin{itemize}
\item<1->
Quantum cryptography is implemented:
bbn.com/DARPA (US), idquantique.com (Switzerland), magiqtech.com (US/Australia),
qinetiq.com (UK), NEC (Japan), Siemens (Austria/Germany), $\ldots$;
\item<1-> Quantum computer
hardware not (yet?) existent; e.g., problems with maintaining coherence;
still needed: a ``quantum transistor;''
\item<1->
Quantum algorithms:
\begin{itemize}
\item<1-> Deutsch-type algorithm (counterexample: parity is qcomp-hard; gain only factor 2);
\item<1-> Factoring (Shor's algorithm): speedup may or may not be exponential;
\item<1-> Grover's search algorithm (quadratic speedup).
\end{itemize}
\end{itemize}


}


\subsection{Counterfactual quantum  computation}


\frame{
\frametitle{Counterfactual computation (Elitzur and Vaidman, 1993)}
Mach-Zehnder interferometer
\begin{center}
%TexCad Options
%\grade{\off}
%\emlines{\off}
%\beziermacro{\off}
%\reduce{\on}
%\snapping{\off}
%\quality{0.20}
%\graddiff{0.01}
%\snapasp{1}
%\zoom{2.00}
\unitlength 0.70mm
%\linethickness{0.4pt}
\thicklines
\begin{picture}(78.67,51.00)
{\color{orange}
\put(57.67,30.00){\line(0,-1){25.00}}
\put(5.00,45.00){\color{yellow}\makebox(0,0)[cc]{$L$}}
\put(5.00,45.00){\color{yellow}\circle{10.00}}
\put(10.00,45.00){\line(1,0){40.00}}
\put(39.67,45.00){\line(1,0){18.00}}
\put(20.00,45.00){\line(0,-1){25.00}}
\put(20.00,20.00){\line(1,0){13.00}}
\put(57.67,45.00){\line(0,-1){25.00}}
\put(57.67,20.00){\line(-1,0){35.00}}
\put(57.67,20.00){\line(1,0){13.00}}
\put(57.67,20.00){\line(0,-1){13.00}}
\put(75.17,20.00){\color{purple}\oval(7.00,8.00)[r]}
\put(78.67,26.00){\color{purple}\makebox(0,0)[cc]{$D_1$}}
\put(23.00,32.00){\makebox(0,0)[cc]{$c$}}
\put(20.00,51.00){\color{violet}\makebox(0,0)[cc]{$S_1$}}
\put(51.67,28.00){\color{violet}\makebox(0,0)[cc]{$S_2$}}
\put(13.00,41.00){\makebox(0,0)[cc]{$a$}}
\put(65.67,23.00){\makebox(0,0)[cc]{$d$}}
\put(57.67,3.33){\color{purple}\oval(8.67,8.00)[b]}
\put(65.00,-1.00){\color{purple}\makebox(0,0)[cc]{$D_2$}}
\put(60.33,10.33){\makebox(0,0)[cc]{$e$}}
\put(38.00,51.00){\makebox(0,0)[cc]{$b$}}
\put(57.00,51.00){\color{violet}\makebox(0,0)[cc]{$M$}}
\put(22.33,11.00){\color{violet}\makebox(0,0)[cc]{$M$}}
\put(34.10,16.70){\color{red}\dashbox{1.00}(9.57,6.83)[cc]{}}
\put(38.62,12.91){\color{red}\makebox(0,0)[cc]{$B$}}
\put(25.00,15.00){\color{violet}\line(-1,1){10.00}}
\put(62.67,40.00){\color{violet}\line(-1,1){10.00}}
\put(20.00,45.00){\color{violet}\circle{1.00}}
\put(22.00,43.00){\color{violet}\circle{1.00}}
\put(24.00,41.00){\color{violet}\circle{1.00}}
\put(16.00,49.00){\color{violet}\circle{1.00}}
\put(18.00,47.00){\color{violet}\circle{1.00}}
\put(57.67,20.00){\color{violet}\circle{1.00}}
\put(59.67,18.00){\color{violet}\circle{1.00}}
\put(61.67,16.00){\color{violet}\circle{1.00}}
\put(53.67,24.00){\color{violet}\circle{1.00}}
\put(55.67,22.00){\color{violet}\circle{1.00}}
}
\end{picture}
\end{center}

A single quantum (photon, neutron, electron {\it etc}) is emitted in $L$
and meets a lossless beam splitter (half-silvered mirror) $S_1$, after
which its wave function
is in a coherent superposition of $  b $ and $  c $.  The two beams are then recombined at a second lossless
beam splitter (half-silvered
mirror) $S_2$. The quant is detected at either $D_1$ or $D_2$,
corresponding to the states $d $ and $ e $, respectively.

}

\frame{
\frametitle{Counterfactual computation cntd.}

The computer is prepared to solve a decision problem, such that ---
{\em if} the particle takes pass $c$ towards the computer ---
activates it and lets the photon pass through ("yes"),
or blocks it ("no"); corresponding to the absence or presence of a bomb.
\begin{center}
$\widetilde{\qquad \qquad }$
$\widetilde{\qquad \qquad}$
$\widetilde{\qquad \qquad }$
\end{center}
Case 1:
Decision problem yields ``yes'' and thus path is free:
The computation proceeds by successive substitution (transition) of
states:
let ``$i$'' stand for a phase factor from the beam reflection at 90 degrees, then
\begin{eqnarray*}
S_1:\; a  &\rightarrow& ( b  +i c )/\sqrt{2}, \\
S_2:\; b  &\rightarrow& ( e  + i d )/\sqrt{2} ,\quad
S_2:\; c  \rightarrow ( d  + i e )/\sqrt{2}.
\end{eqnarray*}
The resulting transition is (normalization factors omitted)
$$
  a  \rightarrow b+ic \rightarrow id  + 0 \cdot e =id \quad .
$$
Thus, the emitted quant is always detected $D_1$, never in  $D_2$.
}

\frame{
\frametitle{Counterfactual computation cntd.}

Case 2: Decision problem yields ``no'' and thus path is blocked:
``Bomb'' $B$ presence is implemented by setting $c=0$; i.e.,
\begin{eqnarray*}
S_1:\; a  &\rightarrow& ( b  +i c )/\sqrt{2}\quad , \\
B:  \; c  &\rightarrow& 0,\\
S_2:\; b  &\rightarrow& ( e  + i d )/\sqrt{2}\quad ,
\end{eqnarray*}
The resulting transition is (normalization factors omitted)
$$
  a  \rightarrow b \rightarrow e + id  \quad .
$$
The emitted quant --- necessarily having taken path $b$ {\em without} activating the computer~(!) --- is detected with 50:50 chance in $D_1$ or $D_2$.
Thus, if $D_2$ clicks, we have certainty that the decision problem yields ``no''
without even having started the computation.

}


\section{Quantum cryptography}

\frame{
\begin{center}\Huge
{\color{purple}    Part \Roman{nc}:  \\
Quantum cryptography}
\end{center}
\addtocounter{nc}{1}
\begin{center}{\color{lime}
$\widetilde{\qquad \qquad }$
$\widetilde{\qquad \qquad}$
$\widetilde{\qquad \qquad }$ }
\end{center}
 }

\subsection{History}
\frame{
\frametitle{History}

\begin{itemize}
\item<+-> [1970]
Stephen Wiesner, {\em ``Conjugate coding:''}
noisy transmission of two or more ``complementary messages'' by using single photons in
two or more complementary polarization directions/bases.

\item<+-> [1984]
BB84 Protocol: key growing via quantum channel \& additional classical bidirectional communication channel

\item<+-> [1991]
EPR-Ekert protocol: maximally entangled state, three complementary polarization directions;
additional security confirmation by violation of Bell-type inequality
through data which cannot be directly used for coding


\end{itemize}
 }

\frame{
\frametitle{Man-in-the-middle attacks}
 {\footnotesize
\begin{itemize}
\item<+->
Not save against man--in--the--middle attacks.
\item<+->
Due to complementarity and value indefiniteness save against eavesdropping on the (quantum) channel.
\item<+->
Compare: ``Standard quantum key distribution protocols are provably secure against eavesdropping attacks, if quantum theory is correct.''
(from http://arxiv.org/abs/quant-ph/0405101).

\item<+->
 ``The need for the public (non-quantum) channel in this scheme to be immune to active eavesdropping can be
relaxed if the Alice and Bob have agreed beforehand on a small secret [[classical cryptographic]]  key,.."
(from BB84: C. H. Bennett and G. Brassard, 1984), pp. 175-179.)

\item<+->  More realistic:
``In accordance with our general philosophy that QKD forms a part of an overall cryptographic architecture, and not an
entirely novel architecture of its own, the DARPA Quantum Network currently employs the standardized authentication
mechanisms built into the Internet security architecture (IPsec), and in particular those provided by the Internet Key
Exchange (IKE) protocol.''
(from http://arxiv.org/abs/quant-ph/0503058)
\end{itemize}
}
}
\subsection{BB84 Protocol}
\frame{
\frametitle{BB84 Protocol}
%$\longrightarrow$ time\\
\includegraphics[height=8cm]{2005-qcrypt-pres-BBBSS92.pdf}
}




\frame{
\frametitle{Literature}
\begin{itemize}
\item<1->
Introductory: N. David Mermin,
 {\em ``Quantum Computer Science''}
  (Cambridge University Press, 2007)   \\
http://people.ccmr.cornell.edu/~mermin/qcomp/CS483.html
\item<1->
Extended:
M. A. Nielsen and I. L. Chuang,
{\em ``Quantum Computation and Quantum Information''}
(Cambridge University Press, 2000)

\item<1->
John Preskill's Caltech lecture notes, available at URL\\
http://www.theory.caltech.edu/people/preskill/ph229/

\end{itemize}

 }




\frame{
\centerline{\Huge {\color{purple}Thank you for your attention!}}
\begin{center}{\color{lime}
$\widetilde{\qquad \qquad }$
$\widetilde{\qquad \qquad}$
$\widetilde{\qquad \qquad }$ }
\end{center}
 }


\end{document}

