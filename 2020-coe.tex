\documentclass[%
 reprint,
  % twocolumn,
 %superscriptaddress,
 %groupedaddress,
 %unsortedaddress,
 %runinaddress,
 %frontmatterverbose,
 %preprint,
 showpacs,
 showkeys,
 preprintnumbers,
 %nofootinbib,
 %nobibnotes,
 %bibnotes,
 amsmath,amssymb,
 aps,
 % prl,
  pra,
 % prb,
 % rmp,
 %prstab,
 %prstper,
  longbibliography,
 floatfix,
 %lengthcheck,%
 ]{revtex4-1}


\usepackage{xcolor}

\usepackage{hyperref}
\hypersetup{
    colorlinks,
    linkcolor={blue!80!black},
    citecolor={red!75!black},
    urlcolor={blue!80!black}
}


\begin{document}


\title{Our Beloved Coronavirus}
%\thanks{This is a largely revised and enlarged version of a paper which has been published somewhat hidden in Section~12.9 of my recent book on (A)Causality~\cite{svozil-pac}.}


\author{Karl Svozil}
\email{svozil@gmail.com}

\affiliation{1200 Vienna,Austria}


\date{\today}

\begin{abstract}
COVID-19, commonly known as ``the coronavirus'', is on everyone's lips; and many interested parties are rallying around this phenomenon:
who the coronavirus benefits and who it harms; and what strategies seem appropriate.\end{abstract}


\maketitle

\section{Media and attention economy during a pandemic}

Over the years as a theoretical physicist I have experienced that certain topics were hyped: they were carried into the public, inflated and
marketed to serve parts of physics with special means.
This targeted funding economy is not evil in itself, since it still serves to increase knowledge.
But if the success of such hypes is disproportionately large and they are inappropriately enforced, then
such partial interests are not in the public interest because they distort the public perception of urgencies
and   lead  to a misallocation of societal resources: misinvestment is caused by opportunity costs: ``bubbles'' arise.
I now fear that in the case of the coronavirus, due to similar mechanisms, the collateral damage is disproportionately much greater than the damage,
which could be caused by the virus itself.

Examples of such hypes were and are quantum information technology or quantum computers, where an unprecedented
\href{http://doi.org/10.3354/esep00171}{Hocus-Pocus} is ignited to ensure attention and resources:
the backers are promised -- like a carrot to a rabbit -- an El Dorado of new technologies and applications,
if they only invest enough time and enough money in this part of quantum physics.
Another problem area is the so-called ``climate sciences'', which alternately \href{https://doi.org/10.1126/science.173.3992.138}{a new ice age}
or the exact opposite, namely a \href{https://drive.google.com/file/d/1p_chp9vO0nefV9976q-C_MWTxdqmBCaJ/view}{fatal global warming},
combined with the melting of the polar caps,
predict the demise of paradise islands and mass migration due to drought or rainfall.
The resulting politically prescribed countermeasures are likely to cause serious damage to the automotive (supplier) industry in Germany and Austria,
and energy prices are extremely expensive.

The pattern is always the same: it's about attention and financial resources.
The resulting \href{https://www.imdb.com/title/tt0432232/}{Propaganda}
is a narrative that either conjures up a threat scenario or promises redemption.
Threat scenarios are likely to be more effective the more personal, ``nearer'' they become.
The supposed sinking of the Maldives triggers less emotion --
rather the tendency to enjoy this beautiful, \href{https://youtu.be/15N9K3wXh0Y}{corrupt} island kingdom once more before it (allegedly) sinks in the Indian Ocean --
than the idea that invisible pathogens
could take over their own metabolism.

However, while the assertion of particular interests of scientific subgroups is still essentially a matter of
to quasi ``pull journalism and politicians onto their side'' to
\href{http://www.iemar.tuwien.ac.at/publications/Franck_1993a.pdf}{public attention}
in the case of the Corona pandemic, you are virtually sitting ``in the same boat'' with the politicians and their journalists:
especially the last professional group has been under so much pressure lately for various reasons,
that a sudden flare-up of public focus on the mainstream media
is equivalent to a supposed liberating blow: finally again unrestricted
A hearing for quality journalism!
Here apparently still Chomsky's \href{https://youtu.be/EuwmWnphqII}{fabricated consensus} works perfectly and without restrictions: the media ``humble''
like corona fever.
The massive outflow of advertising money into the new, digital media seems to have been forgotten;
hardly any dissent on reporting in the state media, which in Europe are partly financed by compulsory fees,
and the threatening erosion of their importance by filling in content which the ``gap press'' does not represent.

Everything now hangs on the lips of the moderators and the experts presented by them.
This is therefore also the hour of the virologists and epidemiologists:
they are in demand as rarely.
But these ladies and gentlemen are acting in good faith -- that is what is subordinated to them and what they are allowed to do -- but by no means without their own interests.
For on the one hand, a virologist attaches great importance to viruses -- just as a quantum physicist attaches great importance to quanta -- but often without keeping an eye on the big picture
and to consider the proportionality of resources and urgency.



And on the other hand, scientists have a natural need to be able to communicate.
Malicious tongues say yes -- and I see this assessment confirmed after many years of scientific practice --
that the narcissism of individual scientists cannot be overestimated, it is that high: the narcissism of scientists is omnipresent
and sometimes dominates the sciences.

That's not surprising. Georg Franck once described it in his
\href{http://www.iemar.tuwien.ac.at/publications/Franck_1993a.pdf}{``Economy of Attention''} put it this way:
{\em
``What is more pleasant than the benevolent attention of other people
what could be more soothing than their participatory empathy? What is more inspiring than speaking to inflamed ears, what more captivating than your own exercise of fascination? What is more exciting than an entire hall of tense glances, what is more captivating than the applause you receive?
After all, what is equal to the magic that the rapt attention
of those we ourselves are enchanted by? - The attention
of other people is the most irresistible of all drugs. Their reference stings
any other income. Therefore fame is above power, therefore wealth fades next to celebrity.''}

And the governments are also satisfied, because everywhere people flock around them. The opposition fades into the background;
daily problems and dissent fade away.
Our politicians are served by servile editors when they need new isolation measures,
and severe restrictions in social and economic life.
What a feeling this must be for such an elected politician!
For what is better for a government than a well-managed catastrophe, or, as Orwell suggests in ``1984'', even a war
against an external enemy?
And there's a war going on everywhere against the coronavirus.

\section{Strategies for Pandemics}

Now one could argue that this is all well and good, but only the media accompaniment of a very serious threat.
This almost unavoidable ``concentration on the essentials'' makes the measures taken to contain the underlying pandemic
necessary social measures.
They make it easier for politicians to take the necessary, far-reaching measures associated with this.
At a press conference on Friday, 13 March 2020, for example, the Austrian Federal Chancellor expressly thanked the media for this.
Everything is running virtually like clockwork here: informing the media, the government is taking action, the population nods, buys to reassure
toilet paper, and follow the posted instructions.

This always under the condition that the threat scenario is also present in the dramatic form announced everywhere in the media!
And what if not?
However, there are serious doubters about the allegedly widespread narrative:
A \href{https://www.zdf.de/politik/frontal-21/corona-zwischen-panik-und-pandemie-100.html}{ZDF contribution}
names some doubters of the subject, among them \href{https://www.wodarg.com/}{Wolfgang Wodarg}, a former pulmonologist and SPD politician,
and Tom Jefferson of the Nordic Cochrane Institute.
Virologist M\"olling warns against alarmism...
questions the \href{https://www.youtube.com/watch?v=ycum0vtVwi4}{Appropriateness of extreme measures};
as does her colleague
\href{https://www.faz.net/aktuell/gesellschaft/gesundheit/coronavirus/neue-corona-symptome-entdeckt-virologe-hendrik-streeck-zum-virus-16681450.html?printPagedArticle=true#pageIndex_3}{Hendrik Streeck}
and
\href{https://www.statnews.com/2020/03/17/a-fiasco-in-the-making-as-the-coronavirus-pandemic-takes-hold-we-are-making-decisions-without-reliable-data/}{John P.A. Ioannidis}
(\href{https://doi.org/10.1111/eci.13222}{and DOI doi.org/10.1111/eci.13222}).

First of all, it seems important to me to emphasize that in any case the data situation is not particularly consolidated.
As stressed by Chris Whitty, Chief Scientific Adviser to the British Prime Minister \href{https://www.youtube.com/watch?v=cMtE1ppmkM0&feature=youtu.be&t=747}{emphasised:}
we lack the knowledge about the level of infestation of the populations.


Assuming the claim of Christian Drosten/Charite Berlin is true and his
\href{https://www.sueddeutsche.de/medien/corona-drosten-virologe-1.4843374}{Test} is coronavirus-specific.
But you always have to distinguish between ontology and epistemology: only where you look you see something; and those in the dark you don't see.
Where there was no testing, there were and are also few recognized diseases (e.g. in India).
So the first (probably wrong) assumption is: the spread of the coronavirus is proportional to the number of positive ``drost tests''.

Further: with the supraregional diseases it probably behaves in the same way as with petrol
(as Hans Werner Sinn argues: everything that is not consumed in Germany goes through the exhaust pipe elsewhere, for example in the USA or China):
even if we were able to eliminate the coronavirus completely from Europe -- i.e. we would be ``rost-test''-negative;
then it would immediately attack again when we open the borders.

A global campaign therefore seems hopeless, at least in the longer term. Therefore we have to accept the contamination at least in the longer term.


It is probably the case that cases admitted to hospitals are more severe than mild symptoms,
whereby one can assume that in Wuhan itself, at least at the beginning, the triage after heavy
inpatient cases and minor untreated cases that were cared for at home were more likely to be
``very heavy'' went relative to the other regions of China that were later affected.

But this gives us a completely different casuistry, which is difficult to compare statistically:
because, of course, ``very seriously ill'' patients die to a higher degree than those who are 'slightly ill'.
This probably has the effect that one says wrongly: in Wuhan more people died in percentage terms than elsewhere,
where one treated better (intubated or ECMO ventilation).
In addition, in Wuhan in the early stages probably much fewer but more seriously ill patients did the ``Drost Test'' than later.

Therefore I would recommend a statement (from \href{https://youtu.be/Qac5Kk1dKqU}{Hans Werner Sinn} in an email from 19.3.2020) like ``Corona 5\% lethality without ventilators and 0,5\% with such ventilators and other intensive care equipment''.
because it is assumed that the severity of the diseases of the patients admitted is the same everywhere,
and that the ''rust-test'' is applied homogeneously.


A much better statistical parameter would be: given an extensive infestation of the population with the coronavirus (i.e. about 50-70\%):
what is then the overall mortality rate of all those who had the coronavirus in their blood?
According to the above, such a parameter cannot be operationalized (measured), because we simply do not know the infestation.

For a crude assessment of the threat, one could also consider the flu deaths -- for example in Italy.
According to \href{https://doi.org/10.1016/j.ijid.2019.08.003}{this article}, the development of influenza in Italy in the past four influenza seasons gives the following picture:
{\em ``Conclusions: Over 68,000 deaths were attributable to influenza epidemics in the study period (from 2013/14 to 2016/17 in Italy)''} --
Interesting also the following statement: {\em ``Italy showed a higher influenza attributable excess mortality compared to other European countries.
especially in the elderly.''}
This is over 68,000/4 = 17,000 deaths/flu season in Italy.
In comparison, the current 3500 deaths which had the coronavirus in their blood according to the ``Drost Test'' --
only the gods know what other co-morbidities these dead people had and what the real cause of their deaths was -- rather little
Why the coronavirus caused such upheavals in Italy therefore seems unclear.
 One reason could be the media coverage;
another is that, due to the organizational structure of the Italian health care system, too many of them could easily
Sick people visited the hospitals;
a third that the isolation measures
has politically wilfully brought down the Italian economy and thus the entire stress stability of society.




My assumption is that here in Central Europe.
in the end -- after weeks of these isolation measures -- one has to return to normal life anyway, completely unnerved
and the attack of the coronavirus;
politically, this will be sold as a relative success of the previous draconian measures;
Until then, secondary effects...
more people will have been harmed or even died
than would ever have been fatally affected by the corona virus.
Let's take an anecdotally as a ``light'' example my 94-year-old mother who has Alzheimer's disease and needs 24/7 care by Hungarian nurses.
She is now at risk because of the isolation measures. What is to be done if the local care for these very old people on the verge of extinction is no longer available?
Do they crowd together in death clinics and hand them over to the coronavirus virus?

Let me also mention the role of the UN or the WHO.
These institutions seem to be pursuing an increasingly dangerous policy:
in order to get money and influence, I see their bureaucracies and committees (e.g. the IPCC; but in this case the WHO) increasingly
-- with the authority of the community of states -- pursue goals that are simply harmful to the broad world population.

And there would also have to be considered the costs of the consequences for the economy and the restriction of social life.
We are currently experiencing a regression of economies to levels
in which it is not clear how much they damage the well-being of all of us -- also in social and health terms -- in the long run.
It is generally assumed that a kind of fasting period until Easter, a kind of holiday for society as a whole, would not have many disadvantages.
A kind of stock market storm, then things would start to pick up again. Besides, all the omissions, structural problems and economic misallocations
that were used to fight the coronavirus

But what about the longer term?
No one is talking about the huge opportunity cost of the measures just imposed.
And of the fact that market-based fine-tuning is being interrupted by supposedly necessary political measures.
But is all this necessary, appropriate or sensible?
It is easy to wilfully destroy an economy. But it is less easy to reactivate it.

Let us consider the following alternative extreme scenario for Austria and imagine
we would decide that we would simply ``dip through'' the corona pandemic.
The \href{https://taz.de/Niederlande-Anderer-Ansatz-zu-Corona/!5672956/}{Dutch} and English governments seemed to have been determined until recently,
to let the virus ``burn out'' in order to protect the vulnerable people via the herd immunity thus obtained.
What consequences would that have?

The decisive factor is that the coronavirus causes relatively mild symptoms in the vast majority of infected people; often they do not feel any symptoms at all.
A simple calculation example results in the following risk assessment:
Let's assume that in Austria -- including illegal residents -- there are about 10 million people
Let us further assume that --
\href{https://www.theatlantic.com/health/archive/2020/02/covid-vaccine/607000/}{as claimed by many epidemiologists}
-- a large proportion, i.e. about 70 percent of these people will be infected with the coronavirus.
That would add up to about 7 million cases of infection.
Of those, and this is where the divide is...
it can be assumed that at least 0.5 percent, but at most 4 percent, will die.
As previously indicated, these estimates are so inaccurate because, among other things, nobody knows how high the actual infestation of the population will be,
where there have been deaths caused by the coronavirus in the past.
\href{https://doi.org/10.1038/s41591-020-0822-7}{New findings} estimate the death rate from disease at approximately 1.4\%.
So we are talking about about 35,000 to 280,000 in any case, according to recent estimates just under 100,000 pandemic victims.
Similar estimates come from
\href{https://doi.org/10.1111/eci.13222}{Ioannidis}.

As long as only symptomatic and palliative measures can be taken and no treatment of the coronavirus disease itself is possible,
these numbers of victims will probably be completely independent of whether and what measures are taken: if you dive through and do nothing at all
to prevent exponential spread, these diseases only find faster,
\href{https://topdocumentaryfilms.com/arithmetic-population-and-energy-lecture/}{something explosive}.
The downside, of course, is that such an exponential increase is a severe traumatic shock -- for our health care system, among others --
for all concerned.
The advantage would be that this epidemic would pass quickly and not be extended into \href{https://medium.com/@joschabach/flattening-the-curve-is-a-deadly-delusion-eea324fe9727}{relative eternities}.

Now one could say: maybe there will be viral drugs or vaccinations soon. Therefore it makes sense to slow down the spread until then; because after that
the coronavirus would no longer pose a threat.
It is also often argued that our health system can cope much better with the pandemic shock if it is not exponential or at least delayed
comes from.

First hope would be to say that this is probably highly speculative.
The second would probably have to be agreed to, but one would have to admit
that medicine currently knows only a few symptomatic palliative measures to treat dangerous
to treat disease progressions that often focus on the lungs as a vulnerable organ:
the addition of antibiotics to limit bacterial side effects, the administration of oxygen,
to ensure the oxygen concentration in the blood of the sick person,
and antipyretic agents.
Extreme cases require intubation and (if available) the use of so-called
\href{https://uihc.org/health-topics/family-guide-ecmo}{Extracorporeal membrane oxygenation (ECMO)}-machines for long-term
``external respiration'' if the lung tissue is severely damaged.
ECMO machines would be urgently needed; possibly by applying patent law in this area (with compensation)
and creates components in 3-D printers according to open access patterns.
Possibly the treatment of extremely severe disease progressions should
with a large number of small, mobile and delocalized task forces
and the people are treated at home if possible; possibly accompanied by 24/7 care on site by trained staff
People who could take over nursing admissions.
In particular, such measures would be easier to manage in a relatively unrestrictedly operating company than in a corporation,
whose economic capacity was deliberately 'run down'.



In any event, the \href{https://lex.substack.com/p/slowing-gdp-growth-by-15-is-like}{costs of opportunity} of all the measures taken would have to be taken into account
and compare them with each other:
it should be clear to everyone that there is no ``solution'' here, but rather the management of various forms of misery;
and the reduction of the same relative to openly expressed criteria.
All the measures would have to be compared with each other and it would have to be made clear what the long-term consequences would be.
I believe that after such a comparison it would seem appropriate not to impose certain social and economic restrictions,
but to consciously accept the infection; always under the condition that one prepares and
would set predefined measures.

Of course, from a tactical point of view, it is opportune and advisable for the politicians concerned to take exactly the same hard measures as those proposed by virologists and epidemiologists
are often demanded: because woe betide the politician who is accused of a concrete course of disease,
by an elderly patient, gasping and gasping for air, succumbing to the corona virus in front of a running camera!
The ad hominem club of the Journaille is too seductive not to unpack it.
Already Johann Wolfgang von Goethe spoke in a similar context of \href{http://woerterbuchnetz.de/GWB/call_wbgui_py_from_form?sigle=GWB&mode=Volltextsuche&hitlist=&patternlist=&lemid=JL00944}{``Lazaret[t]-Poesie''}.

In the end, however, the current political-economic exit strategy could appear no less irrational
like the global hoarding of toilet paper; as if these rolls magically had anti-viral properties
if you stash just enough of them.

%\bibliography{svozil,csvo}

\end{document}
