%\documentclass[amsmath,table,sans,amsfonts, handout]{beamer}
\documentclass[amsmath,table,sans,amsfonts]{beamer}
\usepackage[T1]{fontenc}
%%\usepackage{beamerthemeshadow}
%%\usepackage[headheight=1pt,footheight=10pt]{beamerthemeboxes}
%%\addfootboxtemplate{\color{structure!80}}{\color{white}\tiny \hfill Karl Svozil (TU Vienna)\hfill}
%%\addfootboxtemplate{\color{structure!65}}{\color{white}\tiny \hfill mur.sat \hfill}
%%\addfootboxtemplate{\color{structure!50}}{\color{white}\tiny \hfill Graz, 2010-12-11\hfill}
%\usepackage[dark]{beamerthemesidebar}
%\usepackage[headheight=24pt,footheight=12pt]{beamerthemesplit}
%\usepackage{beamerthemesplit}
%\usepackage[bar]{beamerthemetree}
\usepackage{graphicx}

%Global Background must be put in preamble
{%
%\usebackgroundtemplate%      \includegraphics[width=\paperwidth,height=\paperheight]{HaK-Urkaos-da}%
}
 \setbeamercolor{background canvas}{bg=black}

%\usepackage{eepic}
%\usepackage[usenames]{color}
%\newcommand{\Red}{\color{Red}}  %(VERY-Approx.PANTONE-RED)
%\newcommand{\Green}{\color{Green}}  %(VERY-Approx.PANTONE-GREEN)


%\RequirePackage[german]{babel}
%\selectlanguage{german}
%\RequirePackage[isolatin]{inputenc}

%\pgfdeclareimage[height=0.5cm]{logo}{tu-logo}
%\logo{\pgfuseimage{logo}}
\beamertemplatetriangleitem
%\beamertemplateballitem

\beamerboxesdeclarecolorscheme{alert}{red}{red!15!averagebackgroundcolor}
%\begin{beamerboxesrounded}[scheme=alert,shadow=true]{}
%\end{beamerboxesrounded}

%\beamersetaveragebackground{yellow!10}

%\beamertemplatecircleminiframe

\newtheorem{question}{Question}
\newtheorem{conjecture}[question]{Principle}
\newtheorem{challenge}[question]{Challenge}
\usepackage{tikz}
\newcommand{\bra}[1]{\left< #1 \right|}
\newcommand{\ket}[1]{\left| #1 \right>}

\newcommand{\iprod}[2]{\langle #1 | #2 \rangle}
\newcommand{\mprod}[3]{\langle #1 | #2 | #3 \rangle}
\newcommand{\oprod}[2]{| #1 \rangle\langle #2 |}

\newcommand{\proj}[3]{\begin{smallmatrix} #1 & #2 & #3 \end{smallmatrix}}
\newcommand{\projbf}[3]{\begin{smallmatrix} \mathbf{#1} & \mathbf{#2} & \mathbf{#3} \end{smallmatrix}}

\sloppy
\parskip .7em %vskip between paragraphs

\newcommand{\seq}[1]{\mathbf{#1}}
\newcommand{\floor}[1]{\left\lfloor #1 \right\rfloor}
\newcommand{\ceil}[1]{\left\lceil #1 \right\rceil}
\newcommand{\m}[1]{\widetilde{#1}}
\newcommand{\p}[1]{\scriptsize\textcolor{black}{$[#1]$}}


        \definecolor{orange(webcolor)}{rgb}{1.0, 0.65, 0.0}
\setbeamercolor{normal text}{fg=orange}
\setbeamercolor{structure}{fg=orange}


\begin{document}




\title{\bf \textcolor{orange!100}{Quantum Hocus Pocus}}
\subtitle{\textcolor{orange!100}{\small http://tph.tuwien.ac.at/$\sim$svozil/publ/2017-Svozil-FQMT-pres.pdf
%\\
%Nature-Springer, in press 2017; drafts on demand
}}
\author{\textcolor{orange!100}{Karl Svozil}}
\institute{\textcolor{orange!100}{ITP/Vienna University of Technology, Austria\\
\& CS/University of Auckland, NZ  \\
svozil@tuwien.ac.at
\\
$\,$\\
{\tiny
Background picture: Hilma af Klint (1862-1944, Stockholm), Urkaos, nr 16, 1906-1907 Ur: Serie WU/ROSEN. Grupp 1 (c) Stiftelsen Hilma af Klints Verk
}
}
%{\tiny Disclaimer: Die hier vertretenen Meinungen des Autors verstehen sich als Diskussionsbeitr�ge und decken sich nicht notwendigerweise mit den Positionen der Technischen Universit�t Wien oder deren Vertreter.}
}
\date{\textcolor{orange!100}{V\"axj\"o, Sweden, EU, June 13th, 2017}}
\maketitle



% \frame{
% \frametitle{}
%
% }

 \frame{
 \frametitle{\textcolor{orange!100}{I shall not talk about this:}}
\includegraphics[width=\paperwidth]{2015-AnalyticKS-j_Page_02}
 }

 \frame{
 \frametitle{\textcolor{orange!100}{A little bit about this:}}
\includegraphics[width=\paperwidth]{2016-quantum-hokus-pokus-j_Page_1}

 }

 \frame{
 \frametitle{\textcolor{orange!100}{$\ldots$~and this:}}
\includegraphics[width=\paperwidth]{2016-quantum-Abracadabra-j}

 }

 \frame{
 \frametitle{QM measurement conundrum}

Setting the stage by a practical example:
A (lossless) beam splitter is represented by a (perfectly) unitary --
that is, one-to-one, norm preserving -- and thus deterministic and reversible evolution of the state.

In general, von Neumann (1932), Schr\"odinger (cat papers, 1935), London and Bauer (1939),
Everett (1957), Wigner (1961) pointed out that
a {\em nesting argument} yields a uniform unitary -- that is, one-to-one, norm preserving -- state evolution,
leaving no room for irreversibility (many-to-one) and indeterminism (one-to-many state evolution).

Cf. quantum erasure (Scully \& Dr\"uhl, 1982, Greenberger \& YaSin 1989, Ma {\it et al}, 2013)

So is the solution ``against `measurement' (Bell, 1990)?''
That is, are measurements merely fapp and means relative to the resources invested?


 }


 \frame{
 \frametitle{Where exactly does quantum randomness reside?}

With such epistemic ``measurements'' there goes the irreducible randomness.

Similar situation as for the 2nd law of thermodynamics,
which appears to be means dependent (Maxwell, 1878, Myrvold, 2011).


}

 \frame{
 \frametitle{One ``revolutionary'' (according to Schr\"odinger) resolution}
%\includegraphics[width=\paperwidth]{HaK-Urkaos-vf}
In his 1908 inauguration address as  Rektor of the University of Vienna, Franz Serafin Exner stated,
{\em ``$\ldots$~in the region of the small, in time as in space, the physical
laws are probably invalid
$\ldots$~we have to perceive all so-called exact laws  as probabilistic which are not valid with absolute certainty;
but the more individual processes are involved the higher the certainty.
All physical laws can be traced back to random processes on the  molecular level,
and from them the result follows according to the laws of probability theory$\ldots$''}

German original:
{\em ``$\ldots$~im kleinen, der Zeit wie dem Raume nach,
gelten die physikalischen Gesetze voraussichtlich nicht
$\ldots$
So m\"ssen wir also alle sogenannten exakten Gesetze
nur als Durchschnittsgesetze auffassen die nicht
mit absoluter Sicherheit gelten, wohl aber mit um so
gr\"o{\ss}erer Wahrscheinlichkeit aus je mehr Einzelvorg\"angen
sie sich ergeben. Alle physikalischen Gesetze
gehen zur\"uck auf molekulare Vorg\"ange zuf\"alliger Natur
und aus ihnen folgt das Resultat nach den Gesetzen
der Wahrscheinlichkeitsrechnung$\ldots$''}

}
\frame{
\frametitle{Exner was explicitly mentioned in Schr\"odinger's 1922 inaugural lecture in Zurich, and also later}

{\em ``Long before modern quantum mechanics
made their quantitative statements with respect to
the degree of inaccuracy, it was possible, although it
was not necessary, to doubt the justification of
determinism from a far more general point of view.
In fact, such doubts were raised in 1918 by Franz
Exner, nine years before Heisenberg set up his relation of indeterminacy. Little attention was paid to
them, however, and if support was given to them,
as by the author in his inaugural dissertation at
Zurich, they met with considerable shakings of heads.''}
\begin{center}
{\color{orange}
$\widetilde{\qquad \qquad }$
$\widetilde{\qquad \qquad}$
$\widetilde{\qquad \qquad }$
}
\end{center}
{\color{magenta} What if the Schr�dinger equation is inaccurate for very few particles or scales,
just as Schweidler's 1905 suggestion deviations from Rutherford's decay law (1902) for small sample seizes?
}
}

\frame{
\frametitle{Another promising resolution:
Schr\"odinger on individuation,  entanglement and qm object-observer relation}



Schr\"odinger again (cat papers, 1935) wrt entanglement and individuality:
{\em ``The whole is in a definite state, the parts taken individually are not.''}

German original: {\em ``Das Ganze ist in einem bestimmten Zustand,
die Teile f\"ur sich genommen nicht.''}
\begin{center}{\color{orange}
$\widetilde{\qquad \qquad }$
$\widetilde{\qquad \qquad}$
$\widetilde{\qquad \qquad }$}
\end{center}
{\color{magenta}
Suppose through measurement the object \& observer (measurement apparatus) interact and become entangled.
Then none of them appers to be in a definite state {\em individually} any longer, even if both of them were
in a definite individual state before the measurement.
The initial information got re-encoded into relational properties.
This was already discussed by
von Neumann (1932), Schr\"odinger (cat papers, 1935) \& London and Bauer (1939), among others.
}
}








\frame{
\frametitle{Breathing in \& and out of individuality \& entanglement}


Toy example involving the Cartesian standard basis
$ \begin{pmatrix}
\vert {\bf e}_1 \rangle ,
\vert {\bf e}_2 \rangle ,
\vert {\bf e}_3 \rangle ,
\vert {\bf e}_4 \rangle  \end{pmatrix}
$
(for individuation)
and the Bell basis
$ \begin{pmatrix}
\vert \Psi^- \rangle ,
\vert \Psi^+ \rangle ,
\vert \Phi^- \rangle ,
\vert \Phi^+ \rangle  \end{pmatrix}
$
(for entanglement). Then,
\begin{equation*}
\begin{split}
\textsf{\textbf{U}} =
\vert \Psi^- \rangle \langle {\bf e}_1  \vert  +
\vert \Psi^+ \rangle \langle {\bf e}_2  \vert  +
\vert \Phi^- \rangle \langle {\bf e}_3  \vert  +
\vert \Phi^+ \rangle \langle {\bf e}_4  \vert
=
\\
=
\begin{pmatrix}
\vert \Psi^- \rangle ,
\vert \Psi^+ \rangle ,
\vert \Phi^- \rangle ,
\vert \Phi^+ \rangle  \end{pmatrix}
=   \frac{1}{\sqrt{2}}
\begin{pmatrix}
0 & 0 &   1 &   1 \\
1 & 1 &   0 &   0 \\
-1& 1 &   0 &   0 \\
0 & 0 &  -1 &   1
 \end{pmatrix}
.
\\
%\textsf{\textbf{U}} =
%\vert \Psi^- \rangle \langle {\bf e}_1  \vert  +
%\vert \Psi^+ \rangle \langle {\bf e}_2  \vert  +
%\vert \Phi^- \rangle \langle {\bf e}_3  \vert  +
%\vert \Phi^+ \rangle \langle {\bf e}_4  \vert  ,\\
\textsf{\textbf{V}} =
\vert {\bf e}_2 \rangle \langle  \Psi^-  \vert  +
\vert {\bf e}_3 \rangle \langle  \Psi^+  \vert  +
\vert {\bf e}_4 \rangle \langle  \Phi^-  \vert  +
\vert {\bf e}_1 \rangle \langle  \Phi^+  \vert
=
\\
=
\begin{pmatrix}
 \langle \Phi^+ \vert \\
 \langle \Psi^- \vert \\
 \langle \Psi^+ \vert \\
 \langle \Phi^- \vert  \end{pmatrix}
=   \frac{1}{\sqrt{2}}
\begin{pmatrix}
1 & 0 &   0 &   1 \\
0 & 1 &   -1 &   0 \\
0& 1 &   1 &   0 \\
1 & 0 &  0 &  -1
 \end{pmatrix}
.
\\
\vert {\bf e}_1 \rangle
\stackrel{\textsf{\textbf{U}}}{\mapsto}
\vert \Psi^- \rangle
\stackrel{\textsf{\textbf{V}}}{\mapsto}
\vert {\bf e}_2 \rangle
\stackrel{\textsf{\textbf{U}}}{\mapsto}
\vert \Psi^+ \rangle
\stackrel{\textsf{\textbf{V}}}{\mapsto}
\vert {\bf e}_3 \rangle
\stackrel{\textsf{\textbf{U}}}{\mapsto}
\vert \Phi^- \rangle
\stackrel{\textsf{\textbf{V}}}{\mapsto}
\vert {\bf e}_4 \rangle
\stackrel{\textsf{\textbf{U}}}{\mapsto}
\vert \Phi^+ \rangle
\stackrel{\textsf{\textbf{V}}}{\mapsto}
\vert {\bf e}_1 \rangle
.
\end{split}
\end{equation*}
}



\frame{
 \frametitle{Planck, The seventeenth Guthrie Lecture, delivered on June 17, 1932}
``$\ldots$~the law of causality is neither right nor
wrong, it can be neither generally proved nor generally disproved. It is rather a
heuristic principle, a sign-post (and to my mind the most valuable sign-post we
possess) to guide us in the motley confusion of events and to show us the direction
in which scientific research must advance in order to attain fruitful results. As the
law of causality immediately seizes the awakening soul of the child and causes him
indefatigably to ask ``Why?'' so it accompanies the investigator through his whole
life and incessantly sets him new problems. For science does not mean contemplative
rest in possession of sure knowledge, it means untiring work and steadily
advancing development.''
}

\usebackgroundtemplate{\includegraphics[width=\paperwidth,height=\paperheight]{HaK-Urkaos-da}}

\frame{



\centerline{\Large {\color{magenta} Thank you for your attention!}}

\begin{center}\color{orange}
$\widetilde{\qquad \qquad }$
$\widetilde{\qquad \qquad}$
$\widetilde{\qquad \qquad }$
\end{center}
 }
 \end{document}

\end{document}

\frame{
 \frametitle{Feynman, 1965 \& {\color{magenta}Zeilinger, 2005}}
$\ldots$ the ``perpetual torment that results
from [[the question]], `But how can it be like that?' which
is a reflection of uncontrolled but utterly vain desire to see
[[quantum mechanics]] in terms of an analogy with something familiar.''
Therefore, Feynman advises,
``do not keep saying to yourself, if you can possibly avoid it,
`But how can it be like that?'
because you will get `down the drain', into a blind alley from which nobody has yet
escaped.''
\begin{center}{\color{orange}
$\widetilde{\qquad \qquad }$
$\widetilde{\qquad \qquad}$
$\widetilde{\qquad \qquad }$}
\end{center}
{\color{magenta}``The discovery that individual events are
irreducibly random is probably one of the
most significant findings of the twentieth
century. $\ldots$~For the individual event in quantum physics,
not only do we not know the cause, there is no cause.'' }

}



\section{Classical [in]determinism}

 \frame{
 \frametitle{Classical [in]determinism}


\begin{itemize}
%\pause
\item dependent on assumptions; eg. classical (nonconstructive) continua; means relativity

\item
deterministic chaos (strong dependence on initial values; ``unfolding'' of the algorithmic information content therein)

%\pause
\item instabilities and weak solutions of ordinary differential equations (not Lipschitz continuous):
discussion about gaps for free will by Poisson in 1806, Duhamel in 1845, Bertrand in 1878, Boussinesq in 1879,
and in 1873 by Maxwell; modern version ``Norton dome''

\item exotic constructions: Kreisel, Pour-El \& Richards, $\ldots$

%\pause
\end{itemize}
}

\section{Quantum [in]determinism}

 \frame{
 \frametitle{Quantum [in]determinism}


\begin{itemize}
%\pause
\item single events: creatio continua (spontaneous \& stimulated emissions)

%\pause
\item complementarity
%\pause
\item value-indefiniteness/contextuality a la Bell, Kochen-Specker

\item entanglement: individuality versus relationality in multipartite situations

\item unitarity (one-to-one-ness            permutation) of the quantum evolution versus irreversible (?) measurements:
quantum erasure experiments; nesting (von Neumann, Everett, Wigner)
\end{itemize}
}


 \frame{
 \frametitle{Where exactly does quantum randomness reside?}

A (lossless) beam splitter is represented by a (perfectly) unitary (that is, one-to-one) evolution of the state.

 }



\frame{

\centerline{\Large {\color{magenta} Thank you for your attention!}}

\begin{center}\color{orange}
$\widetilde{\qquad \qquad }$
$\widetilde{\qquad \qquad}$
$\widetilde{\qquad \qquad }$
\end{center}
 }
 \end{document}

\section{Contents}

 \frame{
 \frametitle{Contents}

{\Huge

\begin{itemize}

%\pause
\item provable [un]provables
%\pause
\item classical [in]determinism
%\pause
\item quantum [in]determinism

\end{itemize}
}
}

\frame{
 \frametitle{Born \& {\color{magenta}Einstein in a letter to Born,
dated December~12, 1926}}

``from the standpoint of our quantum mechanics, there is no quantity
which in any individual case causally fixes the consequence of the collision;
but also experimentally we have so far no reason to believe that there are some inner properties of the atom
which condition a definite outcome for the collision.
Ought we to hope later to discover such properties $\ldots$  and determine them in individual cases?
Or ought we to  believe that the agreement of theory and experiment  --  as to the impossibility
of prescribing conditions? I myself am inclined  to give up determinism in the world of atoms.''
\begin{center}{\color{orange}
$\widetilde{\qquad \qquad }$
$\widetilde{\qquad \qquad}$
$\widetilde{\qquad \qquad }$}
\end{center}
{\color{magenta} ``In any case I am convinced that he [the Old One] does not throw dice.''}
}

\frame{
 \frametitle{Zeilinger, 2005}
``The discovery that individual events are
irreducibly random is probably one of the
most significant findings of the twentieth
century. $\ldots$~For the individual event in quantum physics,
not only do we not know the cause, there is no cause.''

}

 \frame{
 \frametitle{}

 }

 \frame{
 \frametitle{}

 }

 \frame{
 \frametitle{}

 }

 \frame{
 \frametitle{}

 }

 \frame{
 \frametitle{}

 }

 \frame{
 \frametitle{}

 }

 \frame{
 \frametitle{}

 }

 \frame{
 \frametitle{}

 }

 \frame{
 \frametitle{}

 }

 \frame{
 \frametitle{}

 }

 \frame{
 \frametitle{}

 }

 \frame{
 \frametitle{}

 }

 \frame{
 \frametitle{}

 }

 \frame{
 \frametitle{}

 }

 \frame{
 \frametitle{}

 }

 \frame{
 \frametitle{}

 }

 \frame{
 \frametitle{}

 }

