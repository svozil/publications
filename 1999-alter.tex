\documentstyle{article}
\renewcommand{\baselinestretch}{2}
\begin{document}
\title{Aging and Complexity in Equilibrium Dynamics}
\author{
K. Ehrenberger \\
{\small ENT Department,
University of Vienna}\\
{\small W\"ahringer G\"urtel 18-20,
A-1090 Vienna, Austria}\\
{\small and}\\
K. Svozil\\
 {\small Institut f\"ur Theoretische Physik}  \\
  {\small University of Technology Vienna }
  {\small Wiedner Hauptstra\ss e 8-10/136}    \\
  {\small A-1040 Vienna, Austria   }
  {\small e-mail: svozil@tph.tuwien.ac.at}\\
  {\small www: http://tph.tuwien.ac.at/$\widetilde{\;\;}\,$svozil}}
\date{ }
\maketitle

\begin{flushright}
{\scriptsize
http://tph.tuwien.ac.at/$\widetilde{\;\;}\,$svozil/publ/alter.tex}
\end{flushright}

\begin{abstract}
With increasing age, the fractal dimension of the graph of
postural
stability
decreases.
\end{abstract}

From the many aspects of aging, the possible loss of complexity of
certain physiological functions and processes is seldomly discussed
\cite{lip-gold,lip-mie-mo-gol} and less well understood. We report here
findings related to the loss of complexity, as documented by the
decrease of the dimension associated with the graph of equilibrium
dynamics with increasing age.



Graphs of
postural
stability
 are quite easily obtained by placing the test persons on
a platform and asking to stand
in an upright straight position.
The upright position of humans is constantly endangered by small drifts
and motions. To counter this tendency to leave the upright position,
the tilt-dependent bodyweight difference between the stressed left or
right feet is registered by a posturographic platform.
The platform is a device which records the force exerted by
the test persons  to
keep standing still and upright. The vector of the force difference
stressing the left and right feet
describes twodimensional temporal graphs orthogonal to
the upright position of the test persons' body in space.


The fractal dimension
of the graphs can be specified to the dimensional parameter obtained by
the usual box counting methods
\cite{mandelbrot-77,falconer1,falconer2,schuster1}.


In figure
\ref{f-cfrmg}
several graphs of equilibrium dynamics and their associated dimensions
are drawn. In figure
\ref{f-cfrmg2}
the dimensions are plotted against the year of birth of the respective
test persons.
\begin{figure}[htd]
\vskip 9 true cm
\caption{\label{f-cfrmg}
Several posturographs of different test persons and the
associated dimensions
$D$.}
\end{figure}
\begin{figure}[htd]
\vskip 9 true cm
\caption{\label{f-cfrmg2}
Plot of the dimension $D$ of  posturographs versus the
year of birth of the respective test persons. The solid line represents
a least-squares linear fit.
}
\end{figure}


In summary, while the results are preliminar and more data are necessary
to draw a definite conclusion,  the  data reveal a clear
anticorrelation between age and dimension and,
if dimension is taken as a reasonable measure for complexity, also a
clear anticorrelation between age and complexity of equilibrium
dynamics.
One conceivable reason
for this loss of complexity is the overall
functional and structural loss of sensory and neuro-muscular elements
involved in the maintenance of the postural stability in space.

%\bibliography{svozil}
%\bibliographystyle{plain}

\begin{thebibliography}{1}

\bibitem{falconer1}
Kenneth~J. Falconer.
\newblock {\em The Geometry of Fractal Sets}.
\newblock Cambridge University Press, Cambridge, 1985.

\bibitem{falconer2}
Kenneth~J. Falconer.
\newblock {\em Fractal Geometry}.
\newblock John Wiley \& Sons, Chichester, 1990.

\bibitem{lip-gold}
Lewis~A. Lipsitz and Ary~L. Goldberger.
\newblock Loss of `complexity' and aging.
\newblock {\em JAMA}, 267(13):1806--1809, 1992.

\bibitem{lip-mie-mo-gol}
Lewis~A. Lipsitz, Joseph Mietus, George~B. Moody, and Ary~L. Goldberger.
\newblock Spectral characteristics of heart rate variability before and during
  postural tilt. {R}elations to aging and risc of syncope.
\newblock {\em Circulation}, 81(6):1803--1810, 1990.

\bibitem{mandelbrot-77}
B.~B. Mandelbrot.
\newblock {\em Fractals: Form, Chance and Dimension}.
\newblock Freeman, San Francisco, 1977.

\bibitem{schuster1}
H.~G. Schuster.
\newblock {\em Deterministic Chaos}.
\newblock Physik Verlag, Weinheim, 1984.

\end{thebibliography}
\end{document}
