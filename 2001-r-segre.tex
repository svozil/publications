%%tth:\begin{html}<LINK REL=STYLESHEET HREF="/~svozil/ssh.css">\end{html}
%\documentstyle{article}
\documentstyle[amsfonts,11pt]{article}
\RequirePackage{graphicx}
%\RequirePackage{times}
\RequirePackage{courier}
\RequirePackage{mathptm}
\RequirePackage{bookman}
\renewcommand{\baselinestretch}{1.1}
\begin{document}

%\def\frak{\cal }
%\def\Bbb{\bf }
%\sloppy



\title{Evaluation of\\
Algorithmic Information Theoretic Issues in Quantum Mechanics\\
PhD thesis by Gavriel Segre}
\author{Karl Svozil\\
 {\small Institut f\"ur Theoretische Physik, University of Technology Vienna }     \\
  {\small Wiedner Hauptstra\ss e 8-10/136,}
  {\small A-1040 Vienna, Austria   }            \\
  {\small e-mail: svozil@tuwien.ac.at}}
\date{ }
\maketitle

%\begin{flushright}
%{\scriptsize http://tph.tuwien.ac.at/$\widetilde{\;\;}\,$svozil/publ/2000-conventions,tex$\}$}
%\end{flushright}

\begin{abstract}
The work contains a wealth of very interesting and original scientific thoughts.
It is also based based on a solid study of the literature and the past history of the subject.
Therefore, I recommend the acceptance of the PhD Thesis.
\end{abstract}

\section{Scientific appropriateness}

At present, the area of quantum information is a ``hot spot'' in physical research
which does not deal with already settled, ``canonical'' findings.
Several groups are working on related subjects and competition is relatively high.
Moreover, the topics are within the interface between the computer sciences and
(quantum) physics, which complicates matters further.

Given these provisos, Gavriel Segre contributes in many substantial ways.
As correctly claimed, he deals with the following topics, among others:
\begin{itemize}
\item the Uspensky approach in general;
\item the analysis of the applicability of Uspensky's
 approach to  Classical Algorithmic
 Information Theory;
 \item the proposition that the
 Algorithmic Approach to Chaos Theory is equivalent to the
 standard one only in weak sense
\item the analysis of Von Neumann's concept of continuous
dimension from the point of view of Quantum Logic
\item the suggestion  that one-dimensional noncommutative
differential calculus may be used to derive new properties of
Mandelbrot sets;
\item the analysis of the Einstein-Podolski-Rosen setting in
terms of Noncommutative Bayesian Statistical Inference;
 \item the idea  of the Quantum-Lisp programming language;
 \item the analysis of the concept of quantum-formal-system
 with respect to the Moore-Crutchfield's Quantum-Chomsky-Hierarchy;
 \item a concrete model illustrating Calude-Dinnen-Svozil's
 observation on the possibility of overcoming the Turing Barrier;
 \item the formalization of the constraint on the notion of
quantum algorithmic randomness obtained considering independent and free
 tosses of a noncommutative coin;
 \item formalization of  of Quantum Algorithmic
 Information Theory as a particular instance of Uspensky's
 abstract approach;
 \item introduction of the Law of Censorship of Quantum Gambling
 Systems;
 \item formalization of the constraint on the notion of quantum
 algorithmic randomness given by the imposition of the Law of Censorship of Quantum Gambling
 Systems.
\end{itemize}

These issues are highly innovative and constitute independent interesting research results.

\section{Formal appropriateness}

The Thesis is in the form of a scientific monograph.
It is divided into several chapters with
a very extensive bibliography of more than 250 references.
The style is very structured and in accordance with the mathematical
nature of the material. Descriptions of brief contents of the
theorems and definitions could be written in (square) brackets immediately
after the numbering rather than in capital letters.
Typos could best be eliminated with a spell checker.

\section{Summary}

Gavriel Segre has produced a proper and extensive review of
the present status of algorithmic randomness.
He has added many substantial contributions to the field.
From his writings it becomes evident that the author
is a very  careful and humble person who takes the sciences seriously
and is able to foster their progress.
I therefore can warmly recommend the acceptance of the PhD thesis.

\end{document}
