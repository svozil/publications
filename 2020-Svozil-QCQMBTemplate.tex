\documentclass[english]{article}
\usepackage[T1]{fontenc}
\usepackage[latin9]{inputenc}
\setlength{\parskip}{\smallskipamount}
\setlength{\parindent}{0pt}

\makeatletter
\@ifundefined{date}{}{\date{}}
\makeatother

%\usepackage[numbers]{natbib}

\usepackage{babel}
\begin{document}
\thispagestyle{empty}


\section*{Karl Svozil}



{Institute for Theoretical Physics, Vienna  University of Technology,\\
Wiedner Hauptstrasse 8-10/136,
1040 Vienna,  Austria} \\
{svozil@tuwien.ac.at},
{http://tph.tuwien.ac.at/~svozil}, \\
ORCHID ID: orcid.org/0000-0001-6554-2802

\section*{Generalized probabilities from Cauchy's Functional Equation}


One interpretation of Gleason's theorem is in terms of solutions of a Cauchy Functional Equation -- but not with scalar arguments in its usual and most elementary, classical form.
Instead, the  arguments of Cauchy Functional Equation are taken to be unit vectors/one-dimensional orthogonal projections, and only  (mutually) orthogonal vectors in a Hilbert space are allowed to enter the sum.
Indeed, as Victoria J. Wright and Stefan Weigert write~[1] {\em ``Gleason-type theorems $\ldots$ can be viewed as results about the solutions of Cauchy's functional equation for vector-valued arguments: additive functions on subsets of a real vector space, subject to some additional constraints, are necessarily linear.''}
To quote the first two paragraphs of Gleason~[2] this is guaranteed by the Pythagorean property if one takes the square of the length of the orthogonal projections of some pure state vector onto elements of some orthonormal basis. In that way, a ``linear'' function(al) $f(x+y)=f(x)+f(y)$ can be constructed via a state vector $\psi$,  with $f_\psi(x)= \vert \langle \psi | x\rangle\vert^2$ for $x,y$ in some orthonormal basis: summation over the absolute squares of  the inner products of such (mutually orthogonal) vectors with a state vector guarantees positivity and additivity up to value one.

One straightforward generalization is to consider Cauchy's Functional Equation with {\em tensor arguments} (multilinear forms), or to consider arguments which are other entities and more general formal objects not necessarily associated with Hilbert spaces. The only remaining ``underlying'' structure may be intertwined Boolean algebras or (mutual) exclusivity; maybe forming orthomodular lattices.

It might also be quite natural to ``invert'' the question by ``fixing'' the functional form of $f$, say, to some elementary analytic function; and then ask: (i) what are the mathematical entities $x,y$ satisfying Cauchy's Functional Equation $f(x+y) = f(x)+f(y)$? (ii) Which empirical data might require such an analysis?

\bigskip{}


[1] \textsc{Wright}, Victoria~J.;
   {Weigert}, Stefan:
 {G}leason-Type Theorems from {C}auchy's {F}unctional {E}quation.
 {In: }\emph{Foundations of Physics} 49 (2019), Jun, Nr. 6, 594-606.
  {http://dx.doi.org/10.1007/s10701-019-00275-x}. --
 DOI 10.1007/s10701--019--00275--x. --
 ISSN 1572--9516

[2]  {Gleason}, Andrew~M.:
 Measures on the closed subspaces of a {H}ilbert space.
 {In: }\emph{Journal of Mathematics and Mechanics (now Indiana
  University Mathematics Journal)} 6 (1957), Nr. 4, 885-893.
  {http://dx.doi.org/10.1512/iumj.1957.6.56050}. --
 DOI 10.1512/iumj.1957.6.56050. --
 ISSN 0022--2518

%\bibliographystyle{natdin}
%\bibliography{csvo,svozil}


\end{document}
