 \documentstyle[12pt,pslatex,amsfonts]{article}
%\documentstyle[amsfonts]{article}
%\documentstyle{article}
%\renewcommand{\baselinestretch}{2}
\begin{document}

 \def\Bbb{\bf }
% \def\frak{\cal }

\title{ONE-TO-ONE}
\author{K. Svozil\\
 {\small Institut f\"ur Theoretische Physik,}
  {\small University of Technology Vienna }     \\
  {\small Wiedner Hauptstra\ss e 8-10/136,}
  {\small A-1040 Vienna, Austria   }            \\
  {\small e-mail: svozil@tph.tuwien.ac.at}\\
  {\small www: http://tph.tuwien.ac.at/$\widetilde{\;\;}\,$svozil}}
\date{ }
\maketitle

\begin{flushright}
{\scriptsize http://tph.tuwien.ac.at/$\widetilde{\;\;}\,$svozil/publ/oto.$\{$ps,tex$\}$}
\end{flushright}

\begin{abstract}
Reversible computation is a great metaphor for the foundations of physics.
\end{abstract}

\subsection*{General discussion}
A reversible computation is a computation which can be reversed
completely.  That is,  after
insertion of the input into a reversible computer, the
reversible computer generates some output (if ever).
In such a case one may run the entire computation backward by inserting
the output as new input, thereby obtaining the input one started with.
The computation can flow back and forth an arbitrary number of times.
The implicit time symmetry spoils the very notion of
``result,'' since what is a valuable output is purely determined by the
subjective meaning the observer associates with it and is devoid of any
syntactic relevance.

 In more formal terms, reversible computation can be
 characterized by one-to-one operations, by a
reversible, bijective evolution of the computer states onto
themselves
\cite{landauer:61,bennett-73,fred-tof-82,bennett-82,landauer-94,maxwell-demon}.
If only a finite number of such states are involved, this
amounts to their permutation.

In such a scheme, not a single bit
gets lost, and any piece of information (including the trash) remains in
the
computer forever. That may be good news for the case of decay and loss
of information, but it is bad news with respect to waste management.
There is no way of trashing garbage bits other than cleverly
compressing them and pile them ``high and deep.''
Stated differently:
in this restricted regime,
many-to-one operations such as deletion of bits are
not allowed.

In the strict sense of reversibility discussed here,
one-to-many operations such as copying are forbidden as well.
Any computation can be embedded into a reversible one. The trick is
to provide markers in order to make back-tracking possible, which amounts
to
memorizing the past states of the system. If no copying is allowed, this
may amount to large space overheads as compared to irreversible
computations.



The flow diagram
 depicted in Figure \ref{f-rev-comp} was introduced by Landauer
\cite{landauer-94}.
It illustrates
differences between one-to-one, many-to-one and one-to-many
information flows.
\begin{figure}
\begin{center}
%TexCad Options
%\grade{\off}
%\emlines{\off}
%\beziermacro{\off}
%\reduce{\on}
%\snapping{\off}
%\quality{0.20}
%\graddiff{0.01}
%\snapasp{1}
%\zoom{0.10}
\unitlength 0.50mm
\linethickness{0.4pt}
\begin{picture}(191.00,95.00)
\put(10.00,10.00){\circle*{2.00}}
\put(20.00,10.00){\circle*{2.00}}
\put(30.00,10.00){\circle*{2.00}}
\put(40.00,10.00){\circle*{2.00}}
\put(10.00,20.00){\circle{2.00}}
\put(10.00,11.00){\vector(0,1){8.00}}
\put(10.00,30.00){\circle{2.00}}
\put(10.00,21.00){\vector(0,1){8.00}}
\put(10.00,40.00){\circle{2.00}}
\put(10.00,31.00){\vector(0,1){8.00}}
\put(10.00,50.00){\circle{2.00}}
\put(10.00,41.00){\vector(0,1){8.00}}
\put(10.00,60.00){\circle{2.00}}
\put(10.00,51.00){\vector(0,1){8.00}}
\put(10.00,70.00){\circle{2.00}}
\put(10.00,61.00){\vector(0,1){8.00}}
\put(10.00,80.00){\circle{2.00}}
\put(10.00,71.00){\vector(0,1){8.00}}
\put(10.00,90.00){\circle{2.00}}
\put(10.00,81.00){\vector(0,1){8.00}}
\put(20.00,20.00){\circle{2.00}}
\put(20.00,11.00){\vector(0,1){8.00}}
\put(20.00,30.00){\circle{2.00}}
\put(20.00,21.00){\vector(0,1){8.00}}
\put(20.00,40.00){\circle{2.00}}
\put(20.00,31.00){\vector(0,1){8.00}}
\put(20.00,50.00){\circle{2.00}}
\put(20.00,41.00){\vector(0,1){8.00}}
\put(20.00,60.00){\circle{2.00}}
\put(20.00,51.00){\vector(0,1){8.00}}
\put(20.00,70.00){\circle{2.00}}
\put(20.00,61.00){\vector(0,1){8.00}}
\put(20.00,80.00){\circle{2.00}}
\put(20.00,71.00){\vector(0,1){8.00}}
\put(20.00,90.00){\circle{2.00}}
\put(20.00,81.00){\vector(0,1){8.00}}
\put(30.00,20.00){\circle{2.00}}
\put(30.00,11.00){\vector(0,1){8.00}}
\put(30.00,30.00){\circle{2.00}}
\put(30.00,21.00){\vector(0,1){8.00}}
\put(30.00,40.00){\circle{2.00}}
\put(30.00,31.00){\vector(0,1){8.00}}
\put(30.00,50.00){\circle{2.00}}
\put(30.00,41.00){\vector(0,1){8.00}}
\put(30.00,60.00){\circle{2.00}}
\put(30.00,51.00){\vector(0,1){8.00}}
\put(30.00,70.00){\circle{2.00}}
\put(30.00,61.00){\vector(0,1){8.00}}
\put(30.00,80.00){\circle{2.00}}
\put(30.00,71.00){\vector(0,1){8.00}}
\put(30.00,90.00){\circle{2.00}}
\put(30.00,81.00){\vector(0,1){8.00}}
\put(40.00,20.00){\circle{2.00}}
\put(40.00,11.00){\vector(0,1){8.00}}
\put(40.00,30.00){\circle{2.00}}
\put(40.00,21.00){\vector(0,1){8.00}}
\put(40.00,40.00){\circle{2.00}}
\put(40.00,31.00){\vector(0,1){8.00}}
\put(40.00,50.00){\circle{2.00}}
\put(40.00,41.00){\vector(0,1){8.00}}
\put(40.00,60.00){\circle{2.00}}
\put(40.00,51.00){\vector(0,1){8.00}}
\put(40.00,70.00){\circle{2.00}}
\put(40.00,61.00){\vector(0,1){8.00}}
\put(40.00,80.00){\circle{2.00}}
\put(40.00,71.00){\vector(0,1){8.00}}
\put(40.00,90.00){\circle{2.00}}
\put(40.00,81.00){\vector(0,1){8.00}}
\put(50.00,10.00){\circle*{2.00}}
\put(50.00,20.00){\circle{2.00}}
\put(50.00,11.00){\vector(0,1){8.00}}
\put(50.00,30.00){\circle{2.00}}
\put(50.00,21.00){\vector(0,1){8.00}}
\put(50.00,40.00){\circle{2.00}}
\put(50.00,31.00){\vector(0,1){8.00}}
\put(50.00,50.00){\circle{2.00}}
\put(50.00,41.00){\vector(0,1){8.00}}
\put(50.00,60.00){\circle{2.00}}
\put(50.00,51.00){\vector(0,1){8.00}}
\put(50.00,70.00){\circle{2.00}}
\put(50.00,61.00){\vector(0,1){8.00}}
\put(50.00,80.00){\circle{2.00}}
\put(50.00,71.00){\vector(0,1){8.00}}
\put(50.00,90.00){\circle{2.00}}
\put(50.00,81.00){\vector(0,1){8.00}}
\put(80.00,10.00){\circle*{2.00}}
\put(90.00,10.00){\circle*{2.00}}
\put(100.00,10.00){\circle*{2.00}}
\put(110.00,10.00){\circle*{2.00}}
\put(80.00,20.00){\circle{2.00}}
\put(80.00,11.00){\vector(0,1){8.00}}
\put(80.00,30.00){\circle{2.00}}
\put(80.00,21.00){\vector(0,1){8.00}}
\put(90.00,20.00){\circle{2.00}}
\put(90.00,11.00){\vector(0,1){8.00}}
\put(90.00,30.00){\circle{2.00}}
\put(90.00,21.00){\vector(0,1){8.00}}
\put(100.00,20.00){\circle{2.00}}
\put(100.00,11.00){\vector(0,1){8.00}}
\put(100.00,30.00){\circle{2.00}}
\put(100.00,21.00){\vector(0,1){8.00}}
\put(100.00,40.00){\circle{2.00}}
\put(100.00,31.00){\vector(0,1){8.00}}
\put(100.00,50.00){\circle{2.00}}
\put(100.00,41.00){\vector(0,1){8.00}}
\put(100.00,60.00){\circle{2.00}}
\put(100.00,51.00){\vector(0,1){8.00}}
\put(100.00,70.00){\circle{2.00}}
\put(100.00,61.00){\vector(0,1){8.00}}
\put(100.00,80.00){\circle{2.00}}
\put(100.00,71.00){\vector(0,1){8.00}}
\put(100.00,90.00){\circle{2.00}}
\put(100.00,81.00){\vector(0,1){8.00}}
\put(110.00,20.00){\circle{2.00}}
\put(110.00,11.00){\vector(0,1){8.00}}
\put(110.00,30.00){\circle{2.00}}
\put(110.00,21.00){\vector(0,1){8.00}}
\put(110.00,40.00){\circle{2.00}}
\put(110.00,31.00){\vector(0,1){8.00}}
\put(110.00,50.00){\circle{2.00}}
\put(110.00,41.00){\vector(0,1){8.00}}
\put(120.00,10.00){\circle*{2.00}}
\put(120.00,20.00){\circle{2.00}}
\put(120.00,11.00){\vector(0,1){8.00}}
\put(120.00,30.00){\circle{2.00}}
\put(120.00,21.00){\vector(0,1){8.00}}
\put(120.00,40.00){\circle{2.00}}
\put(120.00,31.00){\vector(0,1){8.00}}
\put(120.00,50.00){\circle{2.00}}
\put(120.00,41.00){\vector(0,1){8.00}}
\put(80.00,31.00){\vector(2,1){18.67}}
\put(90.00,31.00){\vector(1,1){8.67}}
\put(110.00,51.00){\vector(-1,1){8.67}}
\put(119.67,51.00){\vector(-2,1){18.00}}
\put(10.00,5.00){\makebox(0,0)[cc]{$p_1$}}
\put(20.00,5.00){\makebox(0,0)[cc]{$p_2$}}
\put(30.00,5.00){\makebox(0,0)[cc]{$p_3$}}
\put(40.00,5.00){\makebox(0,0)[cc]{$p_4$}}
\put(50.00,5.00){\makebox(0,0)[cc]{$p_5$}}
\put(80.00,5.00){\makebox(0,0)[cc]{$p_1$}}
\put(90.00,5.00){\makebox(0,0)[cc]{$p_2$}}
\put(100.00,5.00){\makebox(0,0)[cc]{$p_3$}}
\put(110.00,5.00){\makebox(0,0)[cc]{$p_4$}}
\put(120.00,5.00){\makebox(0,0)[cc]{$p_5$}}
\put(5.00,95.00){\makebox(0,0)[cc]{a)}}
\put(75.00,95.00){\makebox(0,0)[cc]{b)}}
\put(145.00,95.00){\makebox(0,0)[cc]{c)}}
\put(170.00,90.00){\circle{2.00}}
\put(170.00,80.00){\circle{2.00}}
\put(170.00,70.00){\circle{2.00}}
\put(170.00,60.00){\circle{2.00}}
\put(170.00,50.00){\circle{2.00}}
\put(170.00,40.00){\circle{2.00}}
\put(170.00,30.00){\circle{2.00}}
\put(170.00,20.00){\circle{2.00}}
\put(170.00,10.00){\circle*{2.00}}
\put(170.00,5.00){\makebox(0,0)[cc]{$p_1$}}
\put(170.00,11.00){\vector(0,1){8.00}}
\put(170.00,21.00){\vector(0,1){8.00}}
\put(170.00,31.00){\vector(0,1){8.00}}
\put(170.00,41.00){\vector(0,1){8.00}}
\put(170.00,51.00){\vector(0,1){8.00}}
\put(170.00,61.00){\vector(0,1){8.00}}
\put(170.00,71.00){\vector(0,1){8.00}}
\put(170.00,81.00){\vector(0,1){8.00}}
\put(180.00,70.00){\circle{2.00}}
\put(190.00,70.00){\circle{2.00}}
\put(180.00,80.00){\circle{2.00}}
\put(180.00,71.00){\vector(0,1){8.00}}
\put(180.00,90.00){\circle{2.00}}
\put(180.00,81.00){\vector(0,1){8.00}}
\put(190.00,80.00){\circle{2.00}}
\put(190.00,71.00){\vector(0,1){8.00}}
\put(190.00,90.00){\circle{2.00}}
\put(190.00,81.00){\vector(0,1){8.00}}
\put(150.00,50.00){\circle{2.00}}
\put(150.00,60.00){\circle{2.00}}
\put(150.00,51.00){\vector(0,1){8.00}}
\put(150.00,70.00){\circle{2.00}}
\put(150.00,61.00){\vector(0,1){8.00}}
\put(150.00,80.00){\circle{2.00}}
\put(150.00,71.00){\vector(0,1){8.00}}
\put(150.00,90.00){\circle{2.00}}
\put(150.00,81.00){\vector(0,1){8.00}}
\put(160.00,50.00){\circle{2.00}}
\put(160.00,60.00){\circle{2.00}}
\put(160.00,51.00){\vector(0,1){8.00}}
\put(160.00,70.00){\circle{2.00}}
\put(160.00,61.00){\vector(0,1){8.00}}
\put(160.00,80.00){\circle{2.00}}
\put(160.00,71.00){\vector(0,1){8.00}}
\put(160.00,90.00){\circle{2.00}}
\put(160.00,81.00){\vector(0,1){8.00}}
\put(168.67,39.67){\vector(-2,1){18.00}}
\put(169.00,40.33){\vector(-1,1){8.33}}
\put(170.67,60.33){\vector(1,1){8.67}}
\put(170.67,59.67){\vector(2,1){18.67}}
%\put(174.00,50.00){\makebox(0,0)[cc]{$s$}}
%\put(163.00,50.00){\makebox(0,0)[cc]{$s$}}
%\put(153.00,50.00){\makebox(0,0)[cc]{$s$}}
%\put(194.00,70.00){\makebox(0,0)[cc]{$t$}}
%\put(183.00,70.00){\makebox(0,0)[cc]{$t$}}
%\put(173.00,70.00){\makebox(0,0)[cc]{$t$}}
\end{picture}
\end{center}
\caption{In this flow diagram, the lowest ``root'' represents the
initial state of the computer. Forward computation represents
upwards motion
through a sequence of states represented by open circles. Different
symbols $p_i$ correspond to different computer states.
a) One-to-one computation.
b) Many-to-one junction which is information discarding. Several
computational paths, moving upwards, merge into one.
c) One-to-many computation is allowed only
 if no information is
created and discarded; e.g., in copy-type operations on blank memory.
From Landauer \protect\cite{landauer-94}.
\label{f-rev-comp}
}
\end{figure}


Classical continuum mechanics and electrodynamics are reversible
``at heart.'' That means that all  equations of motion are invariant
with
respect to reversing the arrow of time. Also quantum theory
postulates a unitary evolution of the state between (irreversible)
measurements, which per definition is reversible.
The
no-copy feature of reversible computation is for instance reflected by
the no-cloning theorem of quantum theory.
Therefore, in quantum computations it is not possible to copy arbitrary
bits.


There is an undeniable potential technological advantage
of reversible computers over irreversible ones. It lies in the fact that
reversible computation is not necessarily associated with energy
consumption and heat dissipation while the latter one is
\cite{maxwell-demon}.
And since heat dissipation per computation step can be kept at arbitrary
low levels, when ``scaled up to very large sizes,'' reversible
computation outperforms irreversible computation in that
regime \cite{fr-kn-mar}.
Moreover, after all, physics at very small scales {\it is} reversible.
At this year's UMC'98 conference in Auckland, New Zealand,
an MIT group headed by Thomas Knight presented silicon prototypes of
reversible computers \cite{fr-kn-mar2,kn-su,vieri}.

\subsection*{Reversible finite automata}
We shall concentrate on a particular class of reversible {\em finite}
automata which were first discussed in the UMC'98 workshop in Auckland
\cite{sv-aut-rev}.
Just as irreversible Mealy automata \cite{hopcroft,brauer-84},
reversible ones will be
characterized by the following properties:
\begin{description}
\item[$\bullet$]
a finite set
$S$
of states,
\item[$\bullet$]
a finite input  alphabet $I$,
\item[$\bullet$]
a finite output alphabet $O$,
\item[$\bullet$]
temporal evolution function
$\delta :S\times I\rightarrow S$,
\item[$\bullet$]
output function
$\lambda :S\times I\rightarrow O$.
\end{description}


We additionally require one-to-one
reversibility,
which we interpret in this context as follows.  Assume that the set of
input and output symbols is identical, i.e.,
$$I=O.$$
Assume further that a reasonable formalization of reversibility is that
the combined (state and output) temporal evolution is associated with a
one-to-one
(bijective) map
\begin{equation}
U:(s,i)\rightarrow (\delta(s,i),\lambda (s,i)),
\label{t-e-l}
\end{equation}
with
$s\in S$ and $i\in I$.
As will be discussed below,
neither $\delta$ nor $\lambda$ needs to be a bijection.


The elements of the Cartesian product
$S\times I$ can be arranged as a linear list of length
$n$, just like a vector. In this sense,
$U$ corresponds to a $n\times n$-matrix.
In some analogy to quantum theory, we shall call this matrix $U$ {\em
transition matrix.} Let $\Psi_i$
be the $i$'th element in the vectorial representation of some
$(s,i)$, and let
$U_{ij}$ be the element
of
$U$ in the $i$'th row and the $j$'th column.
Equation (\ref{t-e-l}) can in be rewritten as
\begin{equation}
\Psi_i (t+1)= U_{ij}\Psi_j (t),
\label{t-e-l2a}
\quad \textrm{or just}\quad
\Psi (t+1)= U\Psi (t).
\label{t-e-l3}
\end{equation}
$t$ is a discrete time parameter.
Thus in general,
the discrete temporal evolution (\ref{t-e-l}) can, in matrix
notation, be represented by
\begin{equation}
\Psi (t+1)=U\Psi (t)=U^{N+1}\Psi (0),
\label{t-e-l2}
\end{equation}
where again $t=0,1,2,3,\ldots$ is a discrete time parameter.

Now we shall specify the form of the transition matrix $U$. Due to
of determinism, uniqueness and invertibility, we require
\begin{description}
\item[$\bullet$]
$U_{ij}\in \{0,1\}$,
\item[$\bullet$]
orthogonality:
 $U^{-1}=U^t$ (superscript $t$ means transposition) and
$(U^{-1})_{ij}=U_{ji}$,
\item[$\bullet$]
doubly stochasticity:
the sum of each row and column is one. That is,
$\sum_{i=1}^n U_{ij}= \sum_{j=1}^n U_{ij}=1$ \cite{lande73,peres,louck}.
\end{description}
Since $U$ is a square matrix whose elements are either one or zero and
which has exactly one nonzero entry in each row and exactly one in each
column, it is a {\em permutation matrix}.

Let ${\cal P}_n$ denote the set of all $n\times n$ permutation matrices.
 ${\cal P}_n$ forms the {\em permutation group} (sometimes referred to
as the
{\em symmetric group}) of degree $n$ \cite[chapter VII]{lomont}. (The
product of two permutation matrices is
a permutation matrix, the inverse is the transpose and the identity
${\bf
1}$ belongs to  ${\cal P}_n$.)
${\cal P}_n$ has $n!$ elements.


The simplest case is
$n=1$. It is just the identity ${\cal P}_1=\{1\}$.

The first nontrivial case
is $n=2$. The permutation matrices of $${\cal P}_2=\left\{
\left(
\begin{array}{cc}
1&0\\
0&1
\end{array}
\right)
,
\left(
\begin{array}{cc}
0&1\\
1&0
\end{array}
\right)
\right\}$$
correspond to the identity and the {\it not}-gate.

Before we shall consider more examples, let us mention the connection
between permutation matrices and reversible automata.
In fact, the correspondence between permutation matrices and
reversible automata is straightforward.\footnote{
Indeed, by taking the pairs $(s,i)\in S\times I$ as  states of a new
finite automaton (with empty output), the permutation matrix is just the
adjacency matrix of the transition diagram of this automaton
\cite{LindMarcus95,Eilenberg74,Biggs93}.}
Per definition
[cf. Equation
(\ref{t-e-l})], every reversible automaton is representable by some
permutation matrix. That every $n\times n$ permutation matrix
corresponds to an automaton can be demonstrated by
considering the simplest
case of a one state automaton with $n$ input/output symbols.
In this particular but rather trivial case, the transition function is
many-to-one (in fact, $n$-to-one) but the output function is one-to-one
(in fact, $n$-to-$n$).

There exist less trivial identifications. For example,
let
\begin{eqnarray*}
&{U_1}={\Bbb I}=
\left(
\begin{array}{cccc}
1&0&0&0 \\
0&1&0&0\\
0&0&1&0\\
0&0&0&1
\end{array}
\right)
,\quad
{U_2}=
\left(
\begin{array}{cccc}
1&0&0&0\\
0&0&1&0 \\
0&1&0&0\\
0&0&0&1
\end{array}
\right)
&, \\
&{U_3}=
\left(
\begin{array}{cccc}
0&1&0&0        \\
1&0&0&0 \\
0&0&1&0\\
0&0&0&1
\end{array}
\right)
,\quad
{U_4}= U_2U_3=
\left(
\begin{array}{cccc}
0&1&0&0\\
0&0&1&0 \\
1&0&0&0\\
0&0&0&1
\end{array}
\right)&
.
\end{eqnarray*}
The transition and output functions of the four corresponding
reversible automaton are listed in Table
\ref{t-ra}.
\begin{table}
\begin{center}
\begin{tabular}{|c|cc|cc|}
 \hline\hline
 &$\delta$ & & $\lambda$&\\
$S\backslash I$ &1&2& 1&2\\
 \hline\hline
\multicolumn{5}{|c|}{$M_1$}\\
 \hline
$s_1$&$s_1$&$s_1$ & 1&2\\
$s_2$&$s_2$&$s_2 $& 1&2\\
 \hline\hline
\multicolumn{5}{|c|}{$M_2$}\\
 \hline
$s_1$&$s_1$&$s_2$ & 1&1\\
$s_2$&$s_1$&$s_2 $& 2&2\\
 \hline\hline
\multicolumn{5}{|c|}{$M_3$}\\
 \hline
$s_1$&$s_1$&$s_1$ & 2&1\\
$s_2$&$s_2$&$s_2 $& 1&2\\
 \hline\hline
\multicolumn{5}{|c|}{$M_4$}\\
 \hline
$s_1$&$s_1$&$s_2$ & 2&1\\
$s_2$&$s_1$&$s_2 $& 1&2\\
 \hline\hline
\end{tabular}
\end{center}
\caption{Transition and output table of four reversible
automata $M_1,M_2,M_3$ and $M_4$ with two states $S=\{s_1, s_2\}$ and two
input/output symbols $I= \{1,2\}$.
For $M_1$, the transition as well as the output function is one-to-one.
For $M_2$, the transition function is many-to-one but the output
function is one-to-one.
For $M_3$, the transition function is one-to-one but the output
function is many-to-one.
$M_4$ is a concatenation of $M_3$ and $M_2$. Both its transition
function as well as its output function is many-to-one.
\label{t-ra}}
\end{table}
The associated flow diagrams are drawn in Figure \ref{f-fdia}.
\begin{figure}
\begin{center}
%TexCad Options
%\grade{\off}
%\emlines{\off}
%\beziermacro{\off}
%\reduce{\on}
%\snapping{\off}
%\quality{2.00}
%\graddiff{0.01}
%\snapasp{1}
%\zoom{4.00}
\unitlength 1.30mm
\linethickness{0.4pt}
\begin{picture}(30.75,110.75)
\put(0.00,100.00){\circle{1.50}}
\put(10.00,100.00){\circle{1.50}}
\put(20.00,100.00){\circle{1.50}}
\put(30.00,100.00){\circle{1.50}}
\put(0.00,110.00){\circle{1.50}}
\put(10.00,110.00){\circle{1.50}}
\put(20.00,110.00){\circle{1.50}}
\put(30.00,110.00){\circle{1.50}}
\put(0.00,95.00){\makebox(0,0)[cc]{$(s_1,1)$}}
\put(10.00,95.00){\makebox(0,0)[cc]{$(s_1,2)$}}
\put(20.00,95.00){\makebox(0,0)[cc]{$(s_2,1)$}}
\put(30.00,95.00){\makebox(0,0)[cc]{$(s_2,2)$}}
%\put(10.00,12.00){\vector(0,0){0.33}}
\put(0.00,70.00){\circle{1.50}}
\put(0.00,40.00){\circle{1.50}}
\put(10.00,70.00){\circle{1.50}}
\put(10.00,40.00){\circle{1.50}}
\put(20.00,70.00){\circle{1.50}}
\put(20.00,40.00){\circle{1.50}}
\put(30.00,70.00){\circle{1.50}}
\put(30.00,40.00){\circle{1.50}}
\put(0.00,80.00){\circle{1.50}}
\put(0.00,50.00){\circle{1.50}}
\put(10.00,80.00){\circle{1.50}}
\put(10.00,50.00){\circle{1.50}}
\put(20.00,80.00){\circle{1.50}}
\put(20.00,50.00){\circle{1.50}}
\put(30.00,80.00){\circle{1.50}}
\put(30.00,50.00){\circle{1.50}}
\put(0.00,65.00){\makebox(0,0)[cc]{$(s_1,1)$}}
\put(0.00,35.00){\makebox(0,0)[cc]{$(s_1,1)$}}
\put(10.00,65.00){\makebox(0,0)[cc]{$(s_1,2)$}}
\put(10.00,35.00){\makebox(0,0)[cc]{$(s_1,2)$}}
\put(20.00,65.00){\makebox(0,0)[cc]{$(s_2,1)$}}
\put(20.00,35.00){\makebox(0,0)[cc]{$(s_2,1)$}}
\put(30.00,65.00){\makebox(0,0)[cc]{$(s_2,2)$}}
\put(30.00,35.00){\makebox(0,0)[cc]{$(s_2,2)$}}
\put(0.00,100.50){\vector(0,1){8.83}}
\put(10.00,100.50){\vector(0,1){8.83}}
\put(20.00,100.50){\vector(0,1){8.83}}
\put(30.00,100.50){\vector(0,1){8.83}}
\put(0.00,70.50){\vector(0,1){8.83}}
\put(30.00,70.50){\vector(0,1){8.83}}
\put(10.50,70.50){\vector(1,1){8.83}}
\put(19.50,70.50){\vector(-1,1){8.83}}
\put(0.50,40.50){\vector(1,1){8.83}}
\put(9.50,40.50){\vector(-1,1){8.83}}
\put(20.00,40.50){\vector(0,1){8.83}}
\put(30.00,40.50){\vector(0,1){8.83}}
\put(0.00,30.00){\makebox(0,0)[cc]{c) $M_3$}}
\put(0.00,60.00){\makebox(0,0)[cc]{b) $M_2$}}
\put(0.00,90.00){\makebox(0,0)[cc]{a) $M_1$}}
\put(0.00,10.00){\circle{1.50}}
\put(10.00,10.00){\circle{1.50}}
\put(20.00,10.00){\circle{1.50}}
\put(30.00,10.00){\circle{1.50}}
\put(0.00,20.00){\circle{1.50}}
\put(10.00,20.00){\circle{1.50}}
\put(20.00,20.00){\circle{1.50}}
\put(30.00,20.00){\circle{1.50}}
\put(0.00,5.00){\makebox(0,0)[cc]{$(s_1,1)$}}
\put(10.00,5.00){\makebox(0,0)[cc]{$(s_1,2)$}}
\put(20.00,5.00){\makebox(0,0)[cc]{$(s_2,1)$}}
\put(30.00,5.00){\makebox(0,0)[cc]{$(s_2,2)$}}
\put(0.50,10.50){\vector(1,1){8.83}}
\put(30.00,10.50){\vector(0,1){8.83}}
\put(0.00,0.00){\makebox(0,0)[cc]{d) $M_4$}}
\put(10.50,10.50){\vector(1,1){8.83}}
\put(19.42,10.25){\vector(-2,1){18.50}}
\end{picture}
\end{center}
\caption{Flow diagram of one evolution cycle of the reversible automata
listed in Table
\protect\ref{t-ra}.
\label{f-fdia}
}
\end{figure}


The final example is based upon the perturbation matrix
$$
{U}=
\left(
\begin{array}{cccccc}
1&0&0&0&0&0 \\
0&1&0&0&0&0 \\
0&0&0&0&1&0 \\
0&0&0&1&0&0 \\
0&0&0&0&0&1 \\
0&0&1&0&0&0
\end{array}
\right).
$$
It can be realized by a reversible automaton which is represented in
Table
\ref{t-ra22}.
\begin{table}
\begin{center}
\begin{tabular}{|c|ccc|ccc|}
 \hline\hline
 &$\delta$ & && $\lambda$&&\\
$S\backslash I$ &1&2&3& 1&2&3\\
 \hline
$s_1$&$s_1$&$s_1$ &$s_2$ & 1&2&2\\
$s_2$&$s_2$&$s_2$ &$s_1$ & 1&3&3\\
 \hline\hline
\end{tabular}
\end{center}
\caption{Transition and output table of a reversible
automaton  with two states $S=\{s_1, s_2\}$ and three
input/output symbols $I= \{1,2,3\}$.
Neither its  transition nor its output function is one-to-one.
\label{t-ra22}}
\end{table}
Neither its
evolution function nor its transition function is one-to-one, since for
example
$
\delta (s_1,3)
=
\delta (s_2,1)=s_2
$
and
$
\lambda (s_1,2)
=
\lambda (s_1,3)=2
$.
Its flow diagram throughout five evolution steps is depicted in Figure
\ref{f-fdia22}.
\begin{figure}
\begin{center}
%TexCad Options
%\grade{\off}
%\emlines{\off}
%\beziermacro{\off}
%\reduce{\on}
%\snapping{\off}
%\quality{6.00}
%\graddiff{0.01}
%\snapasp{1}
%\zoom{2.38}
\unitlength 1.30mm
\linethickness{0.4pt}
\begin{picture}(61.36,57.78)
\put(10.00,5.06){\circle{2.11}}
\put(20.00,5.06){\circle{2.11}}
\put(30.00,5.06){\circle{2.11}}
\put(40.00,5.06){\circle{2.11}}
\put(50.00,5.06){\circle{2.11}}
\put(60.00,5.06){\circle{2.11}}
\put(9.96,15.35){\circle{2.11}}
\put(20.10,15.35){\circle{2.11}}
\put(29.96,15.35){\circle{2.11}}
\put(39.96,15.35){\circle{2.11}}
\put(49.96,15.35){\circle{2.11}}
\put(59.96,15.35){\circle{2.11}}
\put(10.00,6.11){\vector(0,1){8.02}}
\put(20.05,6.11){\vector(0,1){8.02}}
\put(10.00,0.01){\makebox(0,0)[cc]{$(s_1,1)$}}
\put(20.05,0.01){\makebox(0,0)[cc]{$(s_1,2)$}}
\put(30.11,0.01){\makebox(0,0)[cc]{$(s_1,3)$}}
\put(40.17,0.01){\makebox(0,0)[cc]{$(s_2,1)$}}
\put(50.22,0.01){\makebox(0,0)[cc]{$(s_2,2)$}}
\put(60.28,0.01){\makebox(0,0)[cc]{$(s_2,3)$}}
%\bezier{76}(59.96,6.15)(61.36,8.67)(45.01,10.06)
\put(59.96,6.15){\line(0,1){0.64}}
\multiput(60.02,6.79)(-0.11,0.12){5}{\line(0,1){0.12}}
\multiput(59.47,7.40)(-0.23,0.11){5}{\line(-1,0){0.23}}
\multiput(58.30,7.96)(-0.36,0.11){5}{\line(-1,0){0.36}}
\multiput(56.52,8.49)(-0.48,0.10){5}{\line(-1,0){0.48}}
\multiput(54.12,8.98)(-0.75,0.11){4}{\line(-1,0){0.75}}
\multiput(51.11,9.42)(-1.02,0.11){6}{\line(-1,0){1.02}}
%\end
%\bezvec{76}(45.01,10.06)(28.79,11.32)(30.05,14.26)
\put(30.05,14.26){\vector(0,1){0.2}}
\multiput(45.01,10.06)(-0.99,0.09){4}{\line(-1,0){0.99}}
\multiput(41.04,10.42)(-0.84,0.10){4}{\line(-1,0){0.84}}
\multiput(37.68,10.84)(-0.69,0.12){4}{\line(-1,0){0.69}}
\multiput(34.93,11.32)(-0.43,0.11){5}{\line(-1,0){0.43}}
\multiput(32.78,11.85)(-0.31,0.12){5}{\line(-1,0){0.31}}
\multiput(31.23,12.45)(-0.16,0.11){6}{\line(-1,0){0.16}}
\multiput(30.29,13.10)(-0.08,0.39){3}{\line(0,1){0.39}}
%\end
%\bezier{56}(30.05,6.15)(28.51,8.39)(39.98,10.06)
\multiput(30.05,6.15)(-0.07,0.39){2}{\line(0,1){0.39}}
\multiput(29.91,6.93)(0.12,0.12){6}{\line(0,1){0.12}}
\multiput(30.61,7.68)(0.25,0.12){6}{\line(1,0){0.25}}
\multiput(32.13,8.39)(0.39,0.11){6}{\line(1,0){0.39}}
\multiput(34.49,9.06)(0.61,0.11){9}{\line(1,0){0.61}}
%\end
%\bezvec{56}(39.98,10.06)(51.16,12.16)(50.04,14.26)
\put(50.04,14.26){\vector(0,1){0.2}}
\multiput(39.98,10.06)(0.51,0.11){7}{\line(1,0){0.51}}
\multiput(43.58,10.81)(0.40,0.11){7}{\line(1,0){0.40}}
\multiput(46.40,11.56)(0.29,0.11){7}{\line(1,0){0.29}}
\multiput(48.43,12.31)(0.18,0.11){7}{\line(1,0){0.18}}
\multiput(49.68,13.06)(0.09,0.30){4}{\line(0,1){0.30}}
%\end
%\bezier{36}(50.04,6.15)(48.50,7.55)(54.93,10.06)
\multiput(50.04,6.15)(-0.08,0.29){3}{\line(0,1){0.29}}
\multiput(49.80,7.01)(0.11,0.11){9}{\line(0,1){0.11}}
\multiput(50.79,8.05)(0.24,0.12){17}{\line(1,0){0.24}}
%\end
%\bezvec{32}(54.93,10.06)(60.24,11.60)(59.96,14.12)
\put(59.96,14.12){\vector(1,3){0.2}}
\multiput(54.93,10.06)(0.31,0.12){9}{\line(1,0){0.31}}
\multiput(57.70,11.12)(0.15,0.11){11}{\line(1,0){0.15}}
\multiput(59.38,12.37)(0.12,0.35){5}{\line(0,1){0.35}}
%\end
\put(40.04,6.11){\vector(0,1){8.02}}
\put(9.82,25.69){\circle{2.11}}
\put(9.68,36.03){\circle{2.11}}
\put(9.54,46.38){\circle{2.11}}
\put(9.40,56.72){\circle{2.11}}
\put(19.96,25.69){\circle{2.11}}
\put(19.82,36.03){\circle{2.11}}
\put(19.68,46.38){\circle{2.11}}
\put(19.54,56.72){\circle{2.11}}
\put(29.82,25.69){\circle{2.11}}
\put(29.68,36.03){\circle{2.11}}
\put(29.54,46.38){\circle{2.11}}
\put(29.40,56.72){\circle{2.11}}
\put(39.82,25.69){\circle{2.11}}
\put(39.68,36.03){\circle{2.11}}
\put(39.54,46.38){\circle{2.11}}
\put(39.40,56.72){\circle{2.11}}
\put(49.82,25.69){\circle{2.11}}
\put(49.68,36.03){\circle{2.11}}
\put(49.54,46.38){\circle{2.11}}
\put(49.40,56.72){\circle{2.11}}
\put(59.82,25.69){\circle{2.11}}
\put(59.68,36.03){\circle{2.11}}
\put(59.54,46.38){\circle{2.11}}
\put(59.40,56.72){\circle{2.11}}
\put(9.86,16.45){\vector(0,1){8.02}}
\put(9.72,26.79){\vector(0,1){8.02}}
\put(9.58,37.14){\vector(0,1){8.02}}
\put(9.44,47.48){\vector(0,1){8.02}}
\put(19.91,16.45){\vector(0,1){8.02}}
\put(19.77,26.79){\vector(0,1){8.02}}
\put(19.63,37.14){\vector(0,1){8.02}}
\put(19.49,47.48){\vector(0,1){8.02}}
%\bezier{76}(59.82,16.49)(61.22,19.01)(44.87,20.41)
\put(59.82,16.49){\line(0,1){0.64}}
\multiput(59.88,17.13)(-0.11,0.12){5}{\line(0,1){0.12}}
\multiput(59.33,17.74)(-0.23,0.11){5}{\line(-1,0){0.23}}
\multiput(58.16,18.30)(-0.36,0.11){5}{\line(-1,0){0.36}}
\multiput(56.38,18.83)(-0.48,0.10){5}{\line(-1,0){0.48}}
\multiput(53.98,19.32)(-0.75,0.11){4}{\line(-1,0){0.75}}
\multiput(50.97,19.77)(-1.02,0.11){6}{\line(-1,0){1.02}}
%\end
%\bezier{76}(59.68,26.84)(61.08,29.35)(44.73,30.75)
\put(59.68,26.84){\line(0,1){0.64}}
\multiput(59.74,27.48)(-0.11,0.12){5}{\line(0,1){0.12}}
\multiput(59.19,28.08)(-0.23,0.11){5}{\line(-1,0){0.23}}
\multiput(58.02,28.65)(-0.36,0.11){5}{\line(-1,0){0.36}}
\multiput(56.24,29.17)(-0.48,0.10){5}{\line(-1,0){0.48}}
\multiput(53.84,29.66)(-0.75,0.11){4}{\line(-1,0){0.75}}
\multiput(50.83,30.11)(-1.02,0.11){6}{\line(-1,0){1.02}}
%\end
%\bezier{76}(59.54,37.18)(60.94,39.70)(44.59,41.09)
\put(59.54,37.18){\line(0,1){0.64}}
\multiput(59.60,37.82)(-0.11,0.12){5}{\line(0,1){0.12}}
\multiput(59.05,38.43)(-0.23,0.11){5}{\line(-1,0){0.23}}
\multiput(57.88,38.99)(-0.36,0.11){5}{\line(-1,0){0.36}}
\multiput(56.10,39.52)(-0.48,0.10){5}{\line(-1,0){0.48}}
\multiput(53.70,40.01)(-0.75,0.11){4}{\line(-1,0){0.75}}
\multiput(50.69,40.45)(-1.02,0.11){6}{\line(-1,0){1.02}}
%\end
%\bezier{76}(59.40,47.52)(60.80,50.04)(44.45,51.44)
\put(59.40,47.52){\line(0,1){0.64}}
\multiput(59.46,48.16)(-0.11,0.12){5}{\line(0,1){0.12}}
\multiput(58.91,48.77)(-0.23,0.11){5}{\line(-1,0){0.23}}
\multiput(57.74,49.33)(-0.36,0.11){5}{\line(-1,0){0.36}}
\multiput(55.96,49.86)(-0.48,0.10){5}{\line(-1,0){0.48}}
\multiput(53.56,50.35)(-0.75,0.11){4}{\line(-1,0){0.75}}
\multiput(50.55,50.80)(-1.02,0.11){6}{\line(-1,0){1.02}}
%\end
%\bezvec{76}(44.87,20.41)(28.65,21.67)(29.91,24.60)
\put(29.91,24.60){\vector(0,1){0.2}}
\multiput(44.87,20.41)(-0.99,0.09){4}{\line(-1,0){0.99}}
\multiput(40.90,20.77)(-0.84,0.10){4}{\line(-1,0){0.84}}
\multiput(37.54,21.19)(-0.69,0.12){4}{\line(-1,0){0.69}}
\multiput(34.79,21.66)(-0.43,0.11){5}{\line(-1,0){0.43}}
\multiput(32.64,22.20)(-0.31,0.12){5}{\line(-1,0){0.31}}
\multiput(31.09,22.79)(-0.16,0.11){6}{\line(-1,0){0.16}}
\multiput(30.15,23.44)(-0.08,0.39){3}{\line(0,1){0.39}}
%\end
%\bezvec{76}(44.73,30.75)(28.51,32.01)(29.77,34.94)
\put(29.77,34.94){\vector(0,1){0.2}}
\multiput(44.73,30.75)(-0.99,0.09){4}{\line(-1,0){0.99}}
\multiput(40.76,31.11)(-0.84,0.10){4}{\line(-1,0){0.84}}
\multiput(37.40,31.53)(-0.69,0.12){4}{\line(-1,0){0.69}}
\multiput(34.65,32.00)(-0.43,0.11){5}{\line(-1,0){0.43}}
\multiput(32.50,32.54)(-0.31,0.12){5}{\line(-1,0){0.31}}
\multiput(30.95,33.13)(-0.16,0.11){6}{\line(-1,0){0.16}}
\multiput(30.01,33.78)(-0.08,0.39){3}{\line(0,1){0.39}}
%\end
%\bezvec{76}(44.59,41.09)(28.37,42.35)(29.63,45.29)
\put(29.63,45.29){\vector(0,1){0.2}}
\multiput(44.59,41.09)(-0.99,0.09){4}{\line(-1,0){0.99}}
\multiput(40.62,41.45)(-0.84,0.10){4}{\line(-1,0){0.84}}
\multiput(37.26,41.87)(-0.69,0.12){4}{\line(-1,0){0.69}}
\multiput(34.51,42.35)(-0.43,0.11){5}{\line(-1,0){0.43}}
\multiput(32.36,42.88)(-0.31,0.12){5}{\line(-1,0){0.31}}
\multiput(30.81,43.48)(-0.16,0.11){6}{\line(-1,0){0.16}}
\multiput(29.87,44.13)(-0.08,0.39){3}{\line(0,1){0.39}}
%\end
%\bezvec{76}(44.45,51.44)(28.23,52.69)(29.49,55.63)
\put(29.49,55.63){\vector(-1,3){0.2}}
\multiput(44.45,51.44)(-1.32,0.12){3}{\line(-1,0){1.32}}
\multiput(40.48,51.80)(-0.84,0.10){4}{\line(-1,0){0.84}}
\multiput(37.12,52.21)(-0.69,0.12){4}{\line(-1,0){0.69}}
\multiput(34.37,52.69)(-0.43,0.11){5}{\line(-1,0){0.43}}
\multiput(32.22,53.22)(-0.31,0.12){5}{\line(-1,0){0.31}}
\multiput(30.67,53.82)(-0.16,0.11){6}{\line(-1,0){0.16}}
\multiput(29.73,54.47)(-0.08,0.39){3}{\line(0,1){0.39}}
%\end
%\bezier{56}(29.91,16.49)(28.37,18.73)(39.84,20.41)
\multiput(29.91,16.49)(-0.07,0.39){2}{\line(0,1){0.39}}
\multiput(29.77,17.27)(0.12,0.12){6}{\line(0,1){0.12}}
\multiput(30.47,18.02)(0.25,0.12){6}{\line(1,0){0.25}}
\multiput(31.99,18.73)(0.39,0.11){6}{\line(1,0){0.39}}
\multiput(34.35,19.40)(0.61,0.11){9}{\line(1,0){0.61}}
%\end
%\bezier{56}(29.77,26.84)(28.23,29.07)(39.70,30.75)
\multiput(29.77,26.84)(-0.07,0.39){2}{\line(0,1){0.39}}
\multiput(29.63,27.62)(0.12,0.12){6}{\line(0,1){0.12}}
\multiput(30.33,28.36)(0.25,0.12){6}{\line(1,0){0.25}}
\multiput(31.85,29.07)(0.39,0.11){6}{\line(1,0){0.39}}
\multiput(34.21,29.75)(0.61,0.11){9}{\line(1,0){0.61}}
%\end
%\bezier{56}(29.63,37.18)(28.09,39.42)(39.56,41.09)
\multiput(29.63,37.18)(-0.07,0.39){2}{\line(0,1){0.39}}
\multiput(29.49,37.96)(0.12,0.12){6}{\line(0,1){0.12}}
\multiput(30.19,38.71)(0.25,0.12){6}{\line(1,0){0.25}}
\multiput(31.71,39.42)(0.39,0.11){6}{\line(1,0){0.39}}
\multiput(34.07,40.09)(0.61,0.11){9}{\line(1,0){0.61}}
%\end
%\bezier{56}(29.49,47.52)(27.95,49.76)(39.42,51.44)
\multiput(29.49,47.52)(-0.07,0.39){2}{\line(0,1){0.39}}
\multiput(29.35,48.30)(0.12,0.12){6}{\line(0,1){0.12}}
\multiput(30.05,49.05)(0.25,0.12){6}{\line(1,0){0.25}}
\multiput(31.57,49.76)(0.39,0.11){6}{\line(1,0){0.39}}
\multiput(33.93,50.43)(0.61,0.11){9}{\line(1,0){0.61}}
%\end
%\bezvec{56}(39.84,20.41)(51.02,22.50)(49.90,24.60)
\put(49.90,24.60){\vector(1,3){0.2}}
\multiput(39.84,20.41)(0.51,0.11){7}{\line(1,0){0.51}}
\multiput(43.44,21.16)(0.40,0.11){7}{\line(1,0){0.40}}
\multiput(46.26,21.90)(0.29,0.11){7}{\line(1,0){0.29}}
\multiput(48.29,22.65)(0.18,0.11){7}{\line(1,0){0.18}}
\multiput(49.54,23.40)(0.09,0.30){4}{\line(0,1){0.30}}
%\end
%\bezvec{56}(39.70,30.75)(50.88,32.85)(49.76,34.94)
\put(49.76,34.94){\vector(0,1){0.2}}
\multiput(39.70,30.75)(0.51,0.11){7}{\line(1,0){0.51}}
\multiput(43.30,31.50)(0.40,0.11){7}{\line(1,0){0.40}}
\multiput(46.12,32.25)(0.29,0.11){7}{\line(1,0){0.29}}
\multiput(48.15,33.00)(0.18,0.11){7}{\line(1,0){0.18}}
\multiput(49.40,33.74)(0.09,0.30){4}{\line(0,1){0.30}}
%\end
%\bezvec{56}(39.56,41.09)(50.74,43.19)(49.62,45.29)
\put(49.62,45.29){\vector(0,1){0.2}}
\multiput(39.56,41.09)(0.51,0.11){7}{\line(1,0){0.51}}
\multiput(43.16,41.84)(0.40,0.11){7}{\line(1,0){0.40}}
\multiput(45.98,42.59)(0.29,0.11){7}{\line(1,0){0.29}}
\multiput(48.01,43.34)(0.18,0.11){7}{\line(1,0){0.18}}
\multiput(49.26,44.09)(0.09,0.30){4}{\line(0,1){0.30}}
%\end
%\bezvec{56}(39.42,51.44)(50.60,53.53)(49.48,55.63)
\put(49.48,55.63){\vector(0,1){0.2}}
\multiput(39.42,51.44)(0.51,0.11){7}{\line(1,0){0.51}}
\multiput(43.02,52.19)(0.40,0.11){7}{\line(1,0){0.40}}
\multiput(45.84,52.93)(0.29,0.11){7}{\line(1,0){0.29}}
\multiput(47.87,53.68)(0.18,0.11){7}{\line(1,0){0.18}}
\multiput(49.12,54.43)(0.09,0.30){4}{\line(0,1){0.30}}
%\end
%\bezier{36}(49.90,16.49)(48.36,17.89)(54.79,20.41)
\multiput(49.90,16.49)(-0.08,0.29){3}{\line(0,1){0.29}}
\multiput(49.66,17.35)(0.11,0.12){9}{\line(0,1){0.12}}
\multiput(50.65,18.39)(0.24,0.12){17}{\line(1,0){0.24}}
%\end
%\bezier{36}(49.76,26.84)(48.22,28.23)(54.65,30.75)
\multiput(49.76,26.84)(-0.08,0.29){3}{\line(0,1){0.29}}
\multiput(49.52,27.70)(0.11,0.11){9}{\line(0,1){0.11}}
\multiput(50.51,28.73)(0.24,0.12){17}{\line(1,0){0.24}}
%\end
%\bezier{36}(49.62,37.18)(48.08,38.58)(54.51,41.09)
\multiput(49.62,37.18)(-0.08,0.29){3}{\line(0,1){0.29}}
\multiput(49.38,38.04)(0.11,0.11){9}{\line(0,1){0.11}}
\multiput(50.37,39.08)(0.24,0.12){17}{\line(1,0){0.24}}
%\end
%\bezier{36}(49.48,47.52)(47.94,48.92)(54.37,51.44)
\multiput(49.48,47.52)(-0.08,0.29){3}{\line(0,1){0.29}}
\multiput(49.24,48.38)(0.11,0.12){9}{\line(0,1){0.12}}
\multiput(50.23,49.42)(0.24,0.12){17}{\line(1,0){0.24}}
%\end
%\bezvec{32}(54.79,20.41)(60.10,21.94)(59.82,24.46)
\put(59.82,24.46){\vector(1,4){0.2}}
\multiput(54.79,20.41)(0.31,0.12){9}{\line(1,0){0.31}}
\multiput(57.56,21.46)(0.15,0.11){11}{\line(1,0){0.15}}
\multiput(59.24,22.71)(0.12,0.35){5}{\line(0,1){0.35}}
%\end
%\bezvec{32}(54.65,30.75)(59.96,32.29)(59.68,34.80)
\put(59.68,34.80){\vector(1,3){0.2}}
\multiput(54.65,30.75)(0.31,0.12){9}{\line(1,0){0.31}}
\multiput(57.42,31.81)(0.15,0.11){11}{\line(1,0){0.15}}
\multiput(59.10,33.05)(0.12,0.35){5}{\line(0,1){0.35}}
%\end
%\bezvec{32}(54.51,41.09)(59.82,42.63)(59.54,45.15)
\put(59.54,45.15){\vector(1,3){0.2}}
\multiput(54.51,41.09)(0.31,0.12){9}{\line(1,0){0.31}}
\multiput(57.28,42.15)(0.15,0.11){11}{\line(1,0){0.15}}
\multiput(58.96,43.40)(0.12,0.35){5}{\line(0,1){0.35}}
%\end
%\bezvec{32}(54.37,51.44)(59.68,52.97)(59.40,55.49)
\put(59.40,55.49){\vector(1,3){0.2}}
\multiput(54.37,51.44)(0.31,0.12){9}{\line(1,0){0.31}}
\multiput(57.14,52.49)(0.15,0.11){11}{\line(1,0){0.15}}
\multiput(58.82,53.74)(0.12,0.35){5}{\line(0,1){0.35}}
%\end
\put(39.90,16.45){\vector(0,1){8.02}}
\put(39.76,26.79){\vector(0,1){8.02}}
\put(39.62,37.14){\vector(0,1){8.02}}
\put(39.48,47.48){\vector(0,1){8.02}}
\end{picture}
\end{center}
\caption{Flow diagram of five evolution cycles of the reversible
automaton listed in Table
\protect\ref{t-ra22}.
\label{f-fdia22}
}
\end{figure}


%If we represent the elements of $S\times I$ as the vertex of a graph,
%$U$ can be interpreted as an adjacency matrix.
%Due to reversibility, there exist automata such that a cycle which
%includes every vertex of the graph, bringing the considerations close
%to Hamiltonian graphs \cite{brunner-pc}.


%\item[$\bullet$]
%\item[$\bullet$]
%\item[$\bullet$]
%\item[$\bullet$]
%\item[$\bullet$]
%\end{description}



\subsection*{Measurements}
Let us now attempt  to  model the
measurement
process within a system whose states evolve according  to a one-to-one
evolution. This is distinct from the orthodox quantum
mechanical
conception of an irreversible measurement associated with the reduction
of the state vector or with the notorious ``wave function collapse.''


In what follows we shall artificially divide a reversible system into an
``inside'' and an
``outside'' region
(cf. Refs.
\cite{bos,toffoli:79,svo-83,svo5,svo-86,roessler-87,roessler-92,weibel:92}
and
\cite[chapter 6]{svozil-93}).
This can be suitably represented by introducing a black box which
contains the ``inside'' region
--- the subsystem to be measured, whereas
the
remaining ``outside'' region is interpreted as the measurement
apparatus.
An input and an output interface mediate all interactions of the
``inside'' with the ``outside,'' of the ``observed'' and the
``observer'' by symbolic exchange. Let us assume that, despite such
symbolic exchanges via the interfaces
(for all practical purposes),
to an outside
observer
 what happens inside the black box is a hidden,
 inaccessible
  arena. This establishes a (arguable artificial) {\em cut} between the
observer and the observed.


Throughout temporal evolution, not only is information transformed
one-to-one
(bijectively, isomorphically) inside the black box,
but
this information is handled  one-to-one {\em after} it appeared on the
black box
interfaces. It might  seem evident at first glance that the symbols
appearing on the interfaces should be treated as classical information.
That is, they could in principle be copied. The possibility to copy the
experiment (input and output) enables the application of Bennett's
strategy
\cite{bennett-73}:
in such a case, one keeps the experimental finding by copying
it, reverts the system evolution and starts with a ``fresh'' black
box system in its original initial state. The result is a
classical Boolean calculus with no computational complementarity
\cite{ca-ca-st}.

The scenario is drastically changed, however, if we assume a
one-to-one evolution also for the environment at and outside
of the black box. That is, one deals with a
homogeneous
and uniform one-to-one evolution ``inside'' and ``outside'' of the black
box, thereby
assuming that the experimenter also evolves
one-to-one and not classically.
In our toy automaton model, this could for instance be realized by some
automaton corresponding to a permutation operator $U$ inside the black
box, and another reversible automaton corresponding to another $U'$
outside of it. Conventionally, $U$ and $U'$ correspond to the measured
system and the measurement device, respectively.

In such a case, as there is no copying due to one-to-one evolution,
in order to set
back the system to its original initial state, the experimenter would
have to invest all knowledge bits of information acquired so far.
The experiment would have to evolve back to the initial state of the
measurement device and the measured system prior to the measurement.
This is similar to the
opening,
closing and reopening of Schr\"odinger's catalogue of expectation values
(cf. \cite[p. 823]{schrodinger} as well as \cite{greenberger2,hkwz}).

As a result, the representation of measurement results in one-to-one
reversible systems may cause a sort of complementarity due to
the impossibility of measuring all variants of the representation
at once.

%%%%%%%%%%%%%%%%%%%%%%%%%%%%%%%%%%%%%%%%%%%%%%%%%%%%%%%%%%%%%%%%

\subsection*{Afterthoughts}
Let me close this review with a few afterthoughts.
One algorithmic aspect of reversibility seems disturbing.
If reversible computation is just a
rephrasing, a permutation of the input, then what use is it anyway?
The ``garbage in-garbage out'' metaphor is particularly pressing here.

This issue seems to be
somewhat related to an old question in proof theory, in which sense a
proof of a statement is ``better'' then the mere knowledge that this
statement is true.
It may be that the term
``informative'' can only be given a subjective, idealistic meaning
devoid of any formal rigor.


%\bibliography{svozil}
%\bibliographystyle{aalpha}

\newcommand{\etalchar}[1]{$^{#1}$}
\ifx\undefined\bysame
\newcommand{\bysame}{\leavevmode\hbox to3em{\hrulefill}\,}
\fi
\begin{thebibliography}{VAW{\etalchar{+}}98}

\bibitem[Ben73]{bennett-73}
Bennett, Charles~H., {\em Logical reversibility of computation}, IBM Journal of
  Research and Development {\bf 17} (1973), 525--532, Reprinted in \cite[pp.
  197-204]{maxwell-demon}.

\bibitem[Ben82]{bennett-82}
Bennett, Charles~H., {\em The thermodynamics of computation---a review}, In
  International Journal of Theoretical Physics \cite{maxwell-demon}, 905--940,
  Reprinted in \cite[pp. 213-248]{maxwell-demon}.

\bibitem[Big93]{Biggs93}
Biggs, Norman, {\em Algebraic graph theory}, second ed., Cambridge University
  Press, Cambridge, 1993.

\bibitem[Bos55]{bos}
Boskovich, R.~J., {\em De spacio et tempore, ut a nobis cognoscuntur}, Vienna,
  1755, English translation in \cite{bos1}.

\bibitem[Bos66]{bos1}
Boskovich, R.~J., {\em De spacio et tempore, ut a nobis cognoscuntur}, A Theory
  of Natural Philosophy (Cambridge, MA) (Child, J.~M., ed.), Open Court (1922)
  and MIT Press, Cambridge, MA, 1966, pp.~203--205.

\bibitem[Bra84]{brauer-84}
Brauer, W., {\em Automatentheorie}, Teubner, Stuttgart, 1984.

\bibitem[CC{\c{S}}98]{ca-ca-st}
Calude, Cristian, Calude, Elena, and {\c{S}}tef{\u a}nescu, C{\u a}t{\u a}lina,
  {\em Computational complementarity for {M}ealy automata}, CDMTCS preprint,
  April 1998.

\bibitem[Eil74]{Eilenberg74}
Eilenberg, Samuel, {\em Automata, languages, and machines}, vol.~A, Academic
  Press, New York -- London, 1974.

\bibitem[FKM98]{fr-kn-mar}
Frank, Michael, Knight, Thomas, and Margolus, Norman~H., {\em Reversibility in
  optimally scalable computer architectures}, Unconventional Models of
  Computation (Singapore) (Calude, Cristian~S., Casti, John, and Ninneen,
  Michael~J., eds.), Springer, 1998, pp.~165--182.

\bibitem[FT82]{fred-tof-82}
Fredkin, E. and Toffoli, T., {\em Conservative logic}, International Journal of
  Theoretical Physics {\bf 21} (1982), 219--253.

\bibitem[FVA{\etalchar{+}}98]{fr-kn-mar2}
Frank, Michael, Vieri, Carlin, Ammer, Josephine, Love, Nicole, Margolus,
  Norman~H., and Knight, Thomas, {\em A scalable reversible computer in
  silicon}, Unconventional Models of Computation (Singapore) (Calude,
  Cristian~S., Casti, John, and Ninneen, Michael~J., eds.), Springer, 1998,
  pp.~183--200.

\bibitem[GW92]{weibel:92}
Gerbel, K. and Weibel, P. (eds.), {\em Die welt von innen---endo \& nano (the
  world from within -- endo \& nano)}, Linz, Austria, PVS Verleger, 1992.

\bibitem[GY89]{greenberger2}
Greenberger, Daniel~B. and YaSin, A., {\em ``{H}aunted'' measurements in
  quantum theory}, Foundation of Physics {\bf 19} (1989), no.~6, 679--704.

\bibitem[HKWZ95]{hkwz}
Herzog, Thomas~J., Kwiat, Paul~G., Weinfurter, Harald, and Zeilinger, Anton,
  {\em Complementarity and the quantum eraser}, Physical Review Letters {\bf
  75} (1995), no.~17, 3034--3037.

\bibitem[HU79]{hopcroft}
Hopcroft, J.~E. and Ullman, J.~D., {\em Introduction to automata theory,
  languages, and computation}, Addison-Wesley, Reading, MA, 1979.

\bibitem[KS98]{kn-su}
Knight, Thomas and Sussman, Gerald~Jay, {\em Cellular gate technology},
  Unconventional Models of Computation (Singapore) (Calude, Cristian~S., Casti,
  John, and Ninneen, Michael~J., eds.), Springer, 1998, pp.~257--272.

\bibitem[Lan61]{landauer:61}
Landauer, R., {\em Irreversibility and heat generation in the computing
  process}, IBM Journal of Research and Development {\bf 3} (1961), 183--191,
  Reprinted in \cite[pp. 188-196]{maxwell-demon}.

\bibitem[Lan73]{lande73}
Land{\'{e}}, Alfred, {\em Quantum mechanics in a new key}, Exposition Press,
  New York, 1973.

\bibitem[Lan94]{landauer-94}
Landauer, R., {\em Zig-zag path to understanding}, Proceedings of the Workshop
  on Physics and Computation PHYSCOMP '94 (Los Alamitos, CA), IEEE Computer
  Society Press, 1994, pp.~54--59.

\bibitem[LM95]{LindMarcus95}
Lind, Douglas and Marcus, Brian, {\em An introduction to symbolic dynamics and
  coding}, Cambridge University Press, Cambridge -- New York -- Melbourne,
  1995.

\bibitem[Lom59]{lomont}
Lomont, J.~S., {\em Applications of finite groups}, Academic Press, New York,
  1959.

\bibitem[Lou97]{louck}
Louck, James~D., {\em Doubly stochastic matrices in quantum mechanics},
  Foundations of Physics {\bf 27} (1997), 1085--1104.

\bibitem[LR90]{maxwell-demon}
Leff, H.~S. and Rex, A.~F., {\em Maxwell's demon}, Princeton University Press,
  Princeton, 1990.

\bibitem[Per93]{peres}
Peres, Asher, {\em Quantum theory: Concepts and methods}, Kluwer Academic
  Publishers, Dordrecht, 1993.

\bibitem[R{\"{o}}s87]{roessler-87}
R{\"{o}}ssler, Otto~E., {\em Endophysics}, Real Brains, Artificial Minds (New
  York) (Casti, John~L. and Karlquist, A., eds.), North-Holland, New York,
  1987, p.~25.

\bibitem[R{\"{o}}s92]{roessler-92}
R{\"{o}}ssler, Otto~E., {\em Endophysics, die welt des inneren beobachters},
  Merwe Verlag, Berlin, 1992, With a foreword by Peter Weibel.

\bibitem[Sch35]{schrodinger}
Schr{\"{o}}dinger, Erwin, {\em Die gegenw{\"{a}}rtige {S}ituation in der
  {Q}uantenmechanik}, Naturwissenschaften {\bf 23} (1935), 807--812, 823--828,
  844--849, English translation in \cite{trimmer} and \cite[pp.
  152-167]{wheeler-Zurek:83}.

\bibitem[Svo83]{svo-83}
Svozil, Karl, {\em On the setting of scales for space and time in arbitrary
  quantized media}, Lawrence Berkeley Laboratory preprint {\bf LBL-16097}
  (1983).

\bibitem[Svo86a]{svo5}
Svozil, Karl, {\em Connections between deviations from lorentz transformation
  and relativistic energy-momentum relation}, Europhysics Letters {\bf 2}
  (1986), 83--85.

\bibitem[Svo86b]{svo-86}
Svozil, Karl, {\em Operational perception of space-time coordinates in a
  quantum medium}, Il Nuovo Cimento {\bf 96B} (1986), 127--139.

\bibitem[Svo93]{svozil-93}
Svozil, Karl, {\em Randomness \& undecidability in physics}, World Scientific,
  Singapore, 1993.

\bibitem[Svo98]{sv-aut-rev}
Svozil, Karl, {\em The {C}hurch-{T}uring thesis as a guiding principle for
  physics}, Unconventional Models of Computation (Singapore) (Calude,
  Cristian~S., Casti, John, and Ninneen, Michael~J., eds.), Springer, 1998,
  pp.~371--385.

\bibitem[Tof78]{toffoli:79}
Toffoli, T., {\em The role of the observer in uniform systems}, Applied General
  Systems Research (New York, London) (Klir, G., ed.), Plenum Press, New York,
  London, 1978.

\bibitem[Tri80]{trimmer}
Trimmer, J.~D., {\em The present situation in quantum mechanics: a translation
  of {S}chr{\"{o}}dinger's ``cat paradox''}, Proc. Am. Phil. Soc. {\bf 124}
  (1980), 323--338, Reprinted in \cite[pp. 152-167]{wheeler-Zurek:83}.

\bibitem[VAW{\etalchar{+}}98]{vieri}
Vieri, Carlin, Ammer, Josephine, Wakefield, Amory, Svensson, Lars, Athas,
  William, and Knight, Thomas, {\em Designing reversible memory},
  Unconventional Models of Computation (Singapore) (Calude, Cristian~S., Casti,
  John, and Ninneen, Michael~J., eds.), Springer, 1998, pp.~386--405.

\bibitem[WZ83]{wheeler-Zurek:83}
Wheeler, John~Archibald and Zurek, Wojciech~Hubert, {\em Quantum theory and
  measurement}, Princeton University Press, Princeton, 1983.

\end{thebibliography}

\end{document}
