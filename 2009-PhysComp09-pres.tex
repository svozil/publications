%\documentclass[pra,showpacs,showkeys,amsfonts,amsmath,twocolumn,handou]{revtex4}
\documentclass[amsmath,red,table,sans,handout]{beamer}
%\documentclass[pra,showpacs,showkeys,amsfonts]{revtex4}
%\documentclass[pra,showpacs,showkeys,amsfonts]{revtex4}
\usepackage[T1]{fontenc}
%%\usepackage{beamerthemeshadow}
\usepackage[headheight=1pt,footheight=10pt]{beamerthemeboxes}
\addfootboxtemplate{\color{structure!80}}{\color{white}\tiny \hfill Karl Svozil (TU Vienna)\hfill}
\addfootboxtemplate{\color{structure!65}}{\color{white}\tiny \hfill Test of Quantum Contextuality\hfill}
\addfootboxtemplate{\color{structure!50}}{\color{white}\tiny \hfill Ponta Dalgada,  Sept. 7, 2009\hfill}
%\usepackage[dark]{beamerthemesidebar}
%\usepackage[headheight=24pt,footheight=12pt]{beamerthemesplit}
%\usepackage{beamerthemesplit}
%\usepackage[bar]{beamerthemetree}
\usepackage{graphicx}
\usepackage{pgf}
%\usepackage[usenames]{color}
%\newcommand{\Red}{\color{Red}}  %(VERY-Approx.PANTONE-RED)
%\newcommand{\Green}{\color{Green}}  %(VERY-Approx.PANTONE-GREEN)

%\RequirePackage[german]{babel}
%\selectlanguage{german}
%\RequirePackage[isolatin]{inputenc}

\pgfdeclareimage[height=0.5cm]{logo}{tu-logo}
\logo{\pgfuseimage{logo}}
\beamertemplatetriangleitem
%\beamertemplateballitem

\beamerboxesdeclarecolorscheme{alert}{red}{red!15!averagebackgroundcolor}
\beamerboxesdeclarecolorscheme{alert2}{orange}{orange!15!averagebackgroundcolor}
%\begin{beamerboxesrounded}[scheme=alert,shadow=true]{}
%\end{beamerboxesrounded}

%\beamersetaveragebackground{green!10}

%\beamertemplatecircleminiframe

\begin{document}

\title{\bf \textcolor{red}{Proposed Direct Test of Quantum Contextuality}}
%\subtitle{Naturwissenschaftlich-Humanisticher Tag am BG 19\\Weltbild und Wissenschaft\\http://tph.tuwien.ac.at/\~{}svozil/publ/2005-BG18-pres.pdf}
\subtitle{\textcolor{orange!60}{\small http://tph.tuwien.ac.at/$\sim$svozil/publ/2009-PhysComp09-pres.pdf}\\
\textcolor{gray!60}{\footnotesize http://arxiv.org/abs/quant-ph/0609209 \& 0401112}
}
\author{Karl Svozil}
\institute{Institut f\"ur Theoretische Physik, Vienna University of Technology, \\
Wiedner Hauptstra\ss e 8-10/136, A-1040 Vienna, Austria\\
svozil@tuwien.ac.at
%{\tiny Disclaimer: Die hier vertretenen Meinungen des Autors verstehen sich als Diskussionsbeitr�ge und decken sich nicht notwendigerweise mit den Positionen der Technischen Universit�t Wien oder deren Vertreter.}
}
\date{Physics and Computation 2009\\
Ponta Dalgada, Azores, Portugal, Sept. 7-11, 2009}
\maketitle



%\frame{
%\frametitle{Contents}
%\tableofcontents
%}


\section{Quantum Logic}


\frame{
\frametitle{``Mini-course in quantum Logic''}
\begin{itemize}
\item<1->
Formally a quantum (yes-no) {\em proposition} corresponds to a {\em projector} in, or equivalently, a {\em linear subspace} of a Hilbert space.
\item<1->
The number of mutually exclusive outcomes determines the dimension of the Hilbert space.
\item<1->
Operations \& relations:
\begin{center}
{\tiny
 \begin{tabular}{|ccccc|} \hline\hline
 generic lattice  &  order relation   & ``meet''
&
``join''  & ``complement''\\
\hline
propositional&implication&disjunction&conjunction&negation\\
calculus&$\rightarrow$&``and'' $\wedge$&``or'' $\vee$&``not'' $\neg$\\
\hline
``classical'' lattice  &  subset $\subset $  & intersection $\cap$ &
union
$\cup$ & complement\\
of subsets&&&&\\
of a set&&&&\\
\hline
Hilbert & subspace& intersection of & closure of     & orthogonal \\
lattice & relation& subspaces $\cap$&  linear& subspace   \\
        & $\subset$ &                 & span $\oplus$  &  $\perp$   \\

\hline
lattice of& $E_1E_2=E_1$& $E_1E_2$& $E_1+E_2-E_1E_2$& orthogonal\\
commuting&&&&projection\\
\{noncommuting\}&&\{$\displaystyle\lim_{n\rightarrow \infty}(E_1E_2)^n$\}&&\\
projection&&&&\\
operators&&&&\\
 \hline\hline
 \end{tabular}
}
\end{center}
\end{itemize}
}



\section{Quantum contexts as blocks}
\frame{
\frametitle{Quantum contexts as blocks}
{
\begin{itemize}

\item<1->
All that is operationally knowable for a given quantized system is a {\em single block}
represented by a maximal collection of mutually co-measurable observables.
Thus, single blocks or, in another terminology, Boolean subalgebras of Hilbert lattices,
will be identified with {\em quantum contexts}.

\item<1->
Equivalently, a quantum context can  be formalized
by a single (nondegenerate) ``maximal'' self-adjoint operator
such that all commuting, compatible co-measurable observables are functions thereof.
\end{itemize}
$\;$\\
%$\;$\\



}
}



\frame{
\frametitle{Pasting construction}


The blocks can be joined or pasted together by a {\em pasting} construction as follows.

\begin{itemize}

\item<1->
The tautologies of all blocks are identified.
\item<1->
The absurdities of all blocks are identified.
\item<1->
Identical elements in different blocks are identified.
\item<1->
The logical and algebraic structures of all blocks remain intact.
\end{itemize}

Every single block represents some ``maximal collection of co-measurable observables''
which will be identified with some quantum {\em context}.

Hilbert lattices can be thought of as the pasting of a continuity of such blocks or contexts:
As Hilbert lattices are pastings of a continuity of blocks or contexts, contexts are the building blocks of quantum logics.

}


\section{Counterfactuals}
\frame{
\frametitle{``Counterfactually measuring'' different contexts}
{\footnotesize
Einstein-Podolsky-Rosen (EPR) type arguments
claim to be able to infer two different contexts counterfactually.
One context is measured on one side of the setup, the other context on the other side of it.
By the uniqueness property  of certain two-particle states,
knowledge of a property of one particle entails the certainty
that, if this property were measured on the other particle as well, the outcome of the measurement would be
a unique function of the outcome of the measurement performed.
\\
$\;$\\
This makes possible the measurement of one context, {\em as well as} the {\em simultaneous counterfactual inference} of a different context.
Because, one could argue, that although one has actually measured on one side a different, incompatible context compared to the context measured on the other side,
{\em if} on both sides the same  context {\em would be measured}, the outcomes on both sides {\em would be uniquely correlated}.
Hence measurement of one context per side is sufficient, for the outcome could be counterfactually inferred from the measurements on the other side.
}
}


\frame{
\frametitle{Realism-Idealism in philosophy \& theology}

\begin{itemize}
\item<+-> Realism: Some entities sometimes exist without being experienced by any finite mind.

\item<+-> Idealism:
$\ldots$
we have not the faintest reason for believing in the existence of
unexperienced entities
$\ldots$
[[Realism]] has been adopted
$\ldots$
solely because it simplifies our view of the universe.
\\
(W.T. Stace, Mind {\bf 53}, 1934 \& ``Readings in Philosophical
Analysis'', ed. by Feigl \& Sellars).
\end{itemize}
 }



\frame{
\frametitle{Scholasticism in quantum physics}

%\begin{columns}
%\begin{column}{5cm}
%\pgfdeclareimage[height=2cm]{Specker}{specker}
%\pgfuseimage{Specker}
%\end{column}
%\begin{column}{9cm}
\begin{itemize}
\item<+->
In 1960, Specker related the discussion on the foundations of quantum mechanics to scholastic philosophy;
in particular to scholastic speculations  about the existence of ``infuturabilities'' or
``counterfactuals.''

\item<+->
Question: Does
the omniscience (comprehensive knowledge) of God extend to events which
would have occurred if something  had happened which did not
happen?

\item<+->
Question: If so, can all conceivable outcomes (regardless of whether or not the measurements have ``actually'' been porformed) be pasted together to form a consistent whole?
\end{itemize}
%\end{column}
%\end{columns}
 }



\section{Kochen-Specker Constructions}
\frame[squeeze]{
\frametitle{Answer: ``No!'' given; e.g., by Kochen-Specker constructions}
{\small
\begin{center}
%TeXCAD Picture [4.pic]. Options:
%\grade{\on}
%\emlines{\off}
%\epic{\off}
%\beziermacro{\on}
%\reduce{\on}
%\snapping{\off}
%\quality{8.00}
%\graddiff{0.01}
%\snapasp{1}
%\zoom{5.6569}
\unitlength 0.3mm % = 2.85pt
\linethickness{0.8pt}
\ifx\plotpoint\undefined\newsavebox{\plotpoint}\fi % GNUPLOT compatibility
\begin{picture}(134.09,125.99)(0,0)
%\emline(86.39,101.96)(111.39,58.46)
\multiput(86.39,101.96)(.119617225,-.208133971){209}{{\color{green}\line(0,-1){.208133971}}}
%\end
%\emline(86.39,14.96)(111.39,58.46)
\multiput(86.39,14.96)(.119617225,.208133971){209}{{\color{red}\line(0,1){.208133971}}}
%\end
%\emline(36.47,101.96)(11.47,58.46)
\multiput(36.47,101.96)(-.119617225,-.208133971){209}{{\color{brown}\line(0,-1){.208133971}}}
%\end
%\emline(36.47,14.96)(11.47,58.46)
\multiput(36.47,14.96)(-.119617225,.208133971){209}{{\color{pink}\line(0,1){.208133971}}}
%\end
\color{blue}\put(86.39,15.21){\color{blue}\line(-1,0){50}}
\put(86.39,101.71){\color{violet}\line(-1,0){50}}
%
\put(36.34,15.16){\color{pink}\circle{6}}
\put(36.34,15.16){\color{blue}\circle{4}}
\put(52.99,15.16){\color{blue}\circle{4}}
\put(52.99,15.16){\color{cyan}\circle{6}}
\put(69.68,15.16){\color{blue}\circle{4}}
\put(69.68,15.16){\color{orange}\circle{6}}
\put(86.28,15.16){\color{blue}\circle{4}}
\put(86.28,15.16){\color{red}\circle{6}}
%
\put(93.53,27.71){\color{red}\circle{4}}
\put(93.53,27.71){\color{orange}\circle{6}}
\put(102.37,43.44){\color{red}\circle{4}}
\put(102.37,43.44){\color{olive}\circle{6}}
\put(111.21,58.45){\color{red}\circle{4}}
\color{green}\put(111.21,58.45){\circle{6}}
%
\put(102.37,73.47){\color{green}\circle{4}}
\put(102.37,73.47){\color{olive}\circle{6}}
\put(93.53,89.21){\color{green}\circle{4}}
\put(93.53,89.21){\color{cyan}\circle{6}}
\put(86.28,101.76){\color{green}\circle{4}}
\put(86.28,101.76){\color{violet}\circle{6}}
%
\put(69.68,101.76){\color{violet}\circle{4}}
\put(69.68,101.76){\color{cyan}\circle{6}}
\put(52.99,101.76){\color{violet}\circle{4}}
\put(52.99,101.76){\color{orange}\circle{6}}
\put(36.34,101.76){\color{violet}\circle{4}}
\put(36.34,101.76){\color{brown}\circle{6}}
%
\put(29.24,89.21){\color{brown}\circle{4}}
\put(29.24,89.21){\color{orange}\circle{6}}
\put(20.4,73.47){\color{brown}\circle{4}}
\put(20.4,73.47){\color{olive}\circle{6}}
\put(11.56,58.45){\color{brown}\circle{4}}
\put(11.56,58.45){\color{pink}\circle{6}}

\put(20.4,43.44){\color{pink}\circle{4}}
\put(20.4,43.44){\color{olive}\circle{6}}
\put(29.24,27.71){\color{pink}\circle{4}}
\put(29.24,27.71){\color{cyan}\circle{6}}

\color{cyan}
\qbezier(29.2,27.73)(23.55,-5.86)(52.99,15.24)
\qbezier(29.2,27.88)(36.93,75)(69.63,101.91)
\qbezier(52.69,15.24)(87.47,40.96)(93.72,89.27)
\qbezier(93.72,89.27)(98.4,125.99)(69.49,102.06)
\color{orange}
\qbezier(93.57,27.73)(99.22,-5.86)(69.78,15.24)
\qbezier(93.57,27.88)(85.84,75)(53.13,101.91)
\qbezier(70.08,15.24)(35.3,40.96)(29.05,89.27)
\qbezier(29.05,89.27)(24.37,125.99)(53.28,102.06)
\color{olive}
\qbezier(20.15,73.72)(-11.67,58.52)(20.15,43.31)
\qbezier(20.33,73.72)(61.34,93.16)(102.36,73.72)
\qbezier(102.36,73.72)(134.09,58.52)(102.53,43.31)
\qbezier(102.53,43.31)(60.99,23.43)(20.15,43.49)
{\color{black} \tiny
\put(12.41,116.02){\makebox(0,0)[rc]{$(0,1,-1,0)$}}
\put(12.41,2.65){\makebox(0,0)[rc]{$(0,0,1,-1)$}}
\put(58.68,116.38){\makebox(0,0)[rc]{$(1,0,0,1)$}}
\put(58.68,2.3){\makebox(0,0)[rc]{$(1,-1,0,0)$}}
\put(115.93,116.2){\makebox(0,0)[lc]{$(-1,1,1,1)$}}
\put(115.93,2.48){\makebox(0,0)[lc]{$(1,1,1,1)$}}
\put(65.65,116.38){\makebox(0,0)[lc]{$(1,1,1,-1)$}}
\put(65.65,2.3){\makebox(0,0)[lc]{$(1,1,-1,-1)$}}
\put(108.24,94.22){\makebox(0,0)[lc]{$(1,1,-1,1)$}}
\put(17.45,94.22){\makebox(0,0)[rc]{$(0,1,1,0)$}}
\put(108.24,22.45){\makebox(0,0)[lc]{$(1,-1,1,-1)$}}
\put(16.45,22.45){\makebox(0,0)[rc]{$(0,0,1,1)$}}
\put(114.13,77.96){\makebox(0,0)[lc]{$(1,0,1,0)$}}
\put(8.55,77.96){\makebox(0,0)[rc]{$(0,0,0,1)$}}
\put(114.13,38.72){\makebox(0,0)[lc]{$(1,0,-1,0)$}}
\put(8.55,38.72){\makebox(0,0)[rc]{$(0,1,0,0)$}}
\put(120.92,57.98){\makebox(0,0)[lc]{$(0,1,0,-1)$}}
\put(1.77,57.98){\makebox(0,0)[rc]{$(1,0,0,0)$}}
}
\end{picture}
\end{center}
{ \tiny
Greechie diagram of a ``short'' proof of the Kochen-Specker theorem by {\it Cabello et al.} (PRL 101 (2008) 210401;
arXiv:0808.2456) in four dimensions.
The graph cannot be colored by the two colors red (associated with truth)
and green (associated with falsity) such that every context contains exactly one red and three green points.
For
in a table containing the points of the contexts as columns
and the enumeration of contexts as rows,
every red point occurs in exactly {\em two} contexts, and
there should be an {\em even} number of red points.
On the other hand, there are nine contexts involved; thus by the rules it follows that there
should be an {\em odd} number (nine) of red points in this table (exactly one per context).
}
}
}


\section{Contextuality and its alternatives}
\frame{
\frametitle{Contextuality and its alternatives}

\begin{itemize}
\item<+->  Abandonment of classical omniscience:
It is wrong to assume that
all observables which could in principle (``potentially'') have been measured also co--exist,
irrespective of whether or not they have or even could have been actually measured.
Realism might still be assumed for a {\em single} context, in particular the one in which the system was prepared;


\item<+->    Abandonment of realism: It is wrong to assume that physical entities exist
even without being experienced by any finite mind.
Quite literary, with this assumption, the proofs of KS and similar decay into thin air because
there are no counterfactuals or unobserved physical observables
or inferred (rather than measured) elements of physical reality.


\item<+->  Contextuality; i.e., the abandonment of context independence of measurement outcomes:
It is wrong to assume that the
result of an observation is independent
of the complete disposition  of the apparatus.

\item<+-> $\ldots$
\end{itemize}
}

\frame{
\frametitle{Historical notion of contextuality envisioned by Bohr \& Bell}

\begin{itemize}
\item<+->
Bohr 1949:
{\em ``the impossibility of any sharp separation
between the behavior of atomic objects and the interaction with the measuring instruments which serve to define
the conditions under which the phenomena appear.''}

\item<+->
Bell 1966: the {\em ``$\ldots$
result of an observation may reasonably depend
not only on the state of the system  $\ldots$
but also on the complete disposition  of the apparatus.''}

\item<+->
Stated pointedly, the outcome of the measurement of an observable  $A$
might depend on which other observables
from systems of maximal observables
are measured alongside with $A$.
\end{itemize}
}



\section{Direct tests of contextuality}
\frame{
\frametitle{Direct test of contextuality I: $L_{12}$}

\begin{center}
%TexCad Options
%\grade{\off}
%\emlines{\off}
%\beziermacro{\on}
%\reduce{\on}
%\snapping{\off}
%\quality{2.00}
%\graddiff{0.01}
%\snapasp{1}
%\zoom{1.00}
\unitlength1.2mm
\thicklines %\linethickness{0.4pt}
\begin{picture}(61.33,36.00)
%\emline(0.33,20.00)(30.33,25.00)
\multiput(0.33,20.00)(0.36,-0.12){84}{{\color{red}\line(1,0){0.36}}}
%\end
%\emline(30.33,10.00)(60.33,35.00)
\multiput(30.33,10.00)(0.36,0.12){84}{{\color{blue}\line(1,0){0.36}}}
%\end
\put(0.33,20.00){{\color{red}\circle{2.50}}}
\put(15.33,15.00){{\color{red}\circle{2.50}}}
\put(30.33,10.00){{\color{red}\circle{2.50}} }
\put(30.33,10.00){{\color{blue}\circle{4.00}} }
\put(45.33,15.00){{\color{blue}\circle{2.50}} }
\put(60.33,20.00){{\color{blue}\circle{2.50}} }
\put(30.33,2.00){\makebox(0,0)[cc]{$(0, 1, 0)$}}
\put(30.33,17.00){\makebox(0,0)[cc]{{\color{red}$\alpha$},{\color{blue}$\delta$}}}
\put(0.33,14.00){\makebox(0,0)[cc]{{\color{red}$\left(-\frac{1}{\sqrt{2}}, 0, \frac{1}{\sqrt{2}}\right)$}}}
\put(0.33,26.00){\makebox(0,0)[cc]{{\color{red}$\gamma$}}}
\put(15.33,8.00){\makebox(0,0)[cc]{{\color{red}$\left(\frac{1}{\sqrt{2}}, 0, \frac{1}{\sqrt{2}}\right)$}}}
\put(15.33,20.00){\makebox(0,0)[cc]{{\color{red}$\beta$}}}
\put(45.33,8.00){\makebox(0,0)[cc]{{\color{blue}$\left(-\frac{i}{\sqrt{2}}, 0, \frac{1}{\sqrt{2}}\right)$}}}
\put(45.33,20.00){\makebox(0,0)[cc]{{\color{blue}$\epsilon$}}}
\put(60.33,14.00){\makebox(0,0)[cc]{{\color{blue}$\left(\frac{i}{\sqrt{2}}, 0, \frac{1}{\sqrt{2}}\right)$}}}
\put(60.33,26.00){\makebox(0,0)[cc]{{\color{blue}$\zeta$}}}
\end{picture}
\end{center}
Diagrammatical representation of two interlinked Kochen-Specker contexts:
 Greechie (orthogonality) diagram representing two tripods with a common leg: points stand for individual basis vectors, and entire contexts
--- in this case the one-dimensional linear subspaces spanned by the vectors of the orthogonal tripods
--- are drawn as smooth curves.
}

\frame{
\frametitle{Direct test of contextuality I: $L_{12}$ cntd.}

\begin{itemize}
\item<+->
For the two spin-one particle singlet state
$\left|  \left. \varphi_s \right\rangle  \right. =(1/\sqrt{3})\left(-|00\rangle+|-+\rangle+|+-\rangle\right)$\\
contextuality predicts that there exist outcomes associated with $\alpha$ on one context which
are accompanied by the outcomes $\epsilon$ or $\zeta$  for the other context; likewise $\delta$ should also occur with
$\beta$ and $\gamma$.

\item<+->
The quantum mechanical expectation values can be obtained from
$$
\begin{array}{cc}
&
\text{Tr}\left\{ \bigl| \varphi_s \right\rangle \left\langle \varphi_s  \bigr|
\;\cdot \;
\left[C_{KS}(\alpha , \beta , \gamma)\otimes C'_{KS}( \delta ,\epsilon , \zeta )\right]\right\}
\\
&
\qquad
\qquad
\qquad
\qquad
\qquad
=\frac{1}{6} \left[2 \alpha  \delta + (\beta + \gamma) (\epsilon + \zeta) \right]
.
\end{array}
$$
As a consequence, the outcomes
$\alpha$--$\epsilon $,
$\alpha$--$\zeta  $, as well as
$\beta $--$ \delta $ and
$\gamma $--$ \delta $ indicating contextuality do not occur.
This is in contradiction to the contextuality hypothesis.
\end{itemize}
}

\frame{
\frametitle{Direct test of contextuality II: Two Contexts interlinked in two observables}
\begin{center}
%TexCad Options
%\grade{\off}
%\emlines{\off}
%\beziermacro{\on}
%\reduce{\on}
%\snapping{\off}
%\quality{2.00}
%\graddiff{0.01}
%\snapasp{1}
%\zoom{1.00}
\unitlength 0.60mm
\thicklines %\linethickness{0.4pt}
\begin{picture}(103.67,92.00)
\put(0.00,80.33){{\color{red}\line(2,-1){40.00}}}
{{\color{red} \bezier{104}(40.00,60.33)(50.33,55.00)(50.00,40.33)}}
\put(50.00,40.33){{\color{red}\line(0,-1){40.00}}}
\put(102.00,80.33){{\color{blue}\line(-2,-1){40.00}}}
{{\color{blue} \bezier{104}(62.00,60.33)(51.67,55.00)(52.00,40.33)}}
\put(52.00,40.33){{\color{blue}\line(0,-1){40.00}}}
\put(0.00,80.33){{\color{red}\circle{3.33}}}
\put(40.00,60.33){{\color{red}\circle{3.33}}}
\put(51.00,40.33){{\color{red}\circle{3.33}}}
\put(51.00,40.33){{\color{blue}\circle{5}}}
\put(51.00,0.33){{\color{red}\circle{3.33}}}
\put(51.00,0.33){{\color{blue}\circle{5}}}
\put(102.00,80.33){{\color{blue}\circle{3.33}}}
\put(62.00,60.33){{\color{blue}\circle{3.33}}}
\put(45.00,40.33){\makebox(0,0)[rc]{{\color{red}$\gamma$},{\color{blue}$\eta$}}}
\put(57.00,40.33){{\makebox(0,0)[lc]{$(0,0,1,0)$}}}
\put(45.00,0.33){\makebox(0,0)[rc]{{\color{red}$\delta$},{\color{blue}$\nu$}}}
\put(57.00,0.33){{\makebox(0,0)[lc]{$(0,0,0,1)$}}}
\put(0.00,72.66){{\color{red}\makebox(0,0)[rc]{$(1,0,0,0)$}}}
\put(0.00,87.66){{\color{red}\makebox(0,0)[cc]{$\alpha$}}}
\put(32.00,55.66){{\color{red}\makebox(0,0)[rc]{$(0,1,0,0)$}}}
\put(40.00,70.66){{\color{red}\makebox(0,0)[cc]{$\beta$}}}
\put(70.00,55.66){{\color{blue}\makebox(0,0)[lc]{$\left(\frac{1}{\sqrt{2}},\frac{1}{\sqrt{2}},0,0\right)$}}}
\put(62.00,70.66){{\color{blue}\makebox(0,0)[cc]{$\zeta$}}}
\put(105.00,72.00){{\color{blue}\makebox(0,0)[lc]{$\left(-\frac{1}{\sqrt{2}},\frac{1}{\sqrt{2}},0,0\right)$}}}
\put(102.00,87.00){{\color{blue}\makebox(0,0)[cc]{$\epsilon$}}}
\end{picture}
\end{center}
Greechie diagram of two contexts in four-dimensional Hilbert space interconnected by two link observables.
 }

\frame{
\frametitle{Direct test of contextuality II cntd.}

\begin{itemize}
\item<+->
For the singlet state
$\left|  \left. \psi_s \right\rangle  \right. =
\frac{1}{2} \left(
\left| \left. \frac{3}{2}, -\frac{3}{2}\right\rangle \right.
 - \left| \left.  -\frac{3}{2}, \frac{3}{2}\right\rangle    \right.
- \left| \left.  \frac{1}{2}, -\frac{1}{2}\right\rangle  \right.
+ \left| \left.  -\frac{1}{2}, \frac{1}{2}\right\rangle   \right.
\right)
$ \\
of two spin 2/3-particles, by counterfactual inference,
if the contexts measured on both sides are identical,
whenever $\alpha$ or $\beta$, and $\gamma$ or $\delta$ is registered on one side, $\nu$ or $\eta$, and  $\zeta$ or $\epsilon$
is measured on the other side, respectively, and {\it vice versa}.

\item<+->
Contextuality predicts totally uncorrelated outcomes associated with a maximal unbias of the two common link observables,
resulting in the equal occurrence of the joint outcomes
$\gamma$--$\eta$,
$\gamma$--$\nu$,
$\delta$--$\eta$, and
$\delta$--$\nu$.

\item<+->
The quantum mechanical predictions are based on the expectation values
$$
\begin{array}{cc}
&
\text{Tr}\left\{ \bigl| \psi_s \right\rangle \left\langle \psi_s  \bigr|
\;\cdot \;
\left[C(\alpha , \beta , \gamma , \delta )\otimes C'(\epsilon , \zeta , \eta , \nu )\right]\right\} \\
&
\qquad
\qquad
\qquad
\qquad
\qquad
=\frac{1}{8} \left[(\gamma + \delta) (\epsilon + \zeta ) + 2 (\beta \eta  + \alpha  \nu )\right]
.
\end{array}
$$

As a consequence, there are no outcomes
$\gamma$--$\eta$,
$\gamma$--$\nu$,
$\delta$--$\eta$, and
$\delta$--$\nu$, which is in contradiction to the contextuality postulate.
\end{itemize}
}

\section{Addendum: Complementarity $\neq$ value indefiniteness}

\frame{
\frametitle{Addendum I: Complementarity $\neq$ value indefiniteness. Bertlmann's Box.}
\pgfdeclareimage[height=6cm]{2009-qchocolate-box}{2009-qchocolate-box}
\pgfuseimage{2009-qchocolate-box}
\begin{center}
\end{center}
}


\frame{
\frametitle{Addendum II: Complementarity $\neq$ value indefiniteness. Specker's ``bug'' diagram. }
{\footnotesize
\begin{center}
%TexCad Options
%\grade{\off}
%\emlines{\off}
%\beziermacro{\off}
%\reduce{\on}
%\snapping{\off}
%\quality{0.20}
%\graddiff{0.01}
%\snapasp{1}
%\zoom{1.00}
\unitlength 0.6mm
\linethickness{1pt}
\thicklines
\begin{picture}(108.00,55.00)
\put(25.00,7.33){\color{orange}\line(1,0){60.00}}
\put(25.00,47.33){\color{red}\line(1,0){60.00}}
\put(55.00,7.33){\color{cyan}\line(0,1){40.00}}
\put(25.00,7.33){\color{blue}\line(-1,1){20.00}}
\put(5.00,27.33){\color{green}\line(1,1){20.00}}
\put(85.00,7.33){\color{magenta}\line(1,1){20.00}}
\put(105.00,27.33){\color{gray}\line(-1,1){20.00}}
\put(24.67,55.00){\makebox(0,0)[rc]{$a_3=\{10,11,12,13,14\}$}}
\put(55.33,55.00){\makebox(0,0)[cc]{$a_4=\{2,6,7,8\}$}}
\put(85.33,55.00){\makebox(0,0)[lc]{$a_5=\{1,3,4,5,9\}$}}
\put(9.00,40.00){\makebox(0,0)[rc]{$a_2=\{4,5,6,7,8,9\}$}}
\put(99.33,40.00){\makebox(0,0)[lc]{$a_6=\{2,6,8,11,12,14\}$}}
\put(0.00,26.33){\makebox(0,0)[rc]{$a_1=\{1,2,3\}$}}
\put(108.00,26.33){\makebox(0,0)[lc]{$a_7=\{7,10,13\}$}}
\put(60.33,31.33){\makebox(0,0)[lc]{$a_{13}=$}}
\put(60.33,26.33){\makebox(0,0)[lc]{$\{1,4,5,10,11,12\}$}}
\put(9.00,13.33){\makebox(0,0)[rc]{$a_{12}=\{4,6,9,12,13,14\}$}}
\put(99.67,13.33){\makebox(0,0)[lc]{$a_8=\{3,5,8,9,11,14\}$}}
\put(24.67,-0.05){\makebox(0,0)[rc]{$a_{11}=\{5,7,8,10,11\}$}}
\put(55.33,-0.05){\makebox(0,0)[cc]{$a_{10}=\{3,9,13,14\}$}}
\put(85.33,-0.05){\makebox(0,0)[lc]{$a_9=\{1,2,4,6,12\}$}}
\put(15.00,17.09){\color{blue}\circle{2.00}}
\put(25.00,7.33){\color{blue}\circle{2.00}}
\put(25.00,7.33){\color{orange}\circle{3.00}}
\put(55.00,27.33){\color{cyan}\circle{2.00}}
\put(85.00,7.33){\color{orange}\circle{2.00}}
\put(85.00,7.33){\color{magenta}\circle{3.00}}
\put(95.00,17.33){\color{magenta}\circle{2.00}}
\put(5.00,27.33){\color{green}\circle{2.00}}
\put(5.00,27.33){\color{blue}\circle{3.0}}
\put(15.00,37.33){\color{green}\circle{2.00}}
\put(25.00,47.33){\color{green}\circle{2.00}}
\put(25.00,47.33){\color{red}\circle{3.00}}
\put(55.00,47.33){\color{red}\circle{2.00}}
\put(55.00,47.33){\color{cyan}\circle{3.00}}
\put(85.00,47.33){\color{red}\circle{2.00}}
\put(85.00,47.33){\color{gray}\circle{3.00}}
\put(55.00,7.33){\color{orange}\circle{2.00}}
\put(55.00,7.33){\color{cyan}\circle{3.00}}
\put(104.76,27.33){\color{gray}\circle{2.00}}
\put(104.76,27.33){\color{magenta}\circle{3.00}}
\put(95.00,37.33){\color{gray}\circle{2.00}}
\end{picture}
\end{center}
}

}


\frame{
\frametitle{Addendum II cntd.: Realization in terms of a generalized urn model}

\begin{center}
{\tiny
\setlength{\tabcolsep}{3pt}
\begin{tabular}{|c|ccccccccccccc||ccccccc|}
%\begin{tabular}{|c|c@{}c@{}c@{}c@{}c@{}c@{}c@{}c@{}c@{}c@{}c@{}c@{}c||c@{}c@{}c@{}c@{}c@{}c@{}c|}
\hline\hline
&\multicolumn{13}{c||}{(a) lattice atoms}&\multicolumn{7}{|c|}{(b) colors}\\
%\cline{2-14}
\raisebox{1.5ex}[0cm][0cm]{$m_r$ and}&$a_1$&$a_2$&$a_3$&$a_4$&$a_5$&$a_6$&$a_7$&$a_8$&$a_9$&$a_{10}$&$a_{11}$&$a_{12}$&$a_{13}$&\color{green}$c_1$&\color{red}$c_2$&\color{gray}$c_3$&\color{magenta}$c_4$&$\color{orange}c_5$&$\color{blue}c_6$&$\color{cyan}c_7$\\
\raisebox{1.5ex}[0cm][0cm]{ball type}&&&&&&&&&&&&&&&&&&&&\\
\hline
1  &1&0&0&0&1&0&0&0&1&0&0&0&1&  \color{green}1&\color{red}1&\color{gray} 1&\color{magenta} 1&\color{orange} 1&\color{blue} 1&\color{cyan}1          \\
2  &1&0&0&1&0&1&0&0&1&0&0&0&0&  \color{green}1&\color{red}2&\color{gray} 2&\color{magenta} 1&\color{orange} 1&\color{blue} 1&\color{cyan}2           \\
3  &1&0&0&0&1&0&0&1&0&1&0&0&0&  \color{green}1&\color{red}1&\color{gray} 1&\color{magenta} 2&\color{orange} 2&\color{blue} 1&\color{cyan}3          \\
4  &0&1&0&0&1&0&0&0&1&0&0&1&1&  \color{green}2&\color{red}1&\color{gray} 1&\color{magenta} 1&\color{orange} 1&\color{blue} 2&\color{cyan}1          \\
5  &0&1&0&0&1&0&0&1&0&0&1&0&1&  \color{green}2&\color{red}1&\color{gray} 1&\color{magenta} 2&\color{orange} 3&\color{blue} 3&\color{cyan}1          \\
6  &0&1&0&1&0&1&0&0&1&0&0&1&0&  \color{green}2&\color{red}2&\color{gray} 2&\color{magenta} 1&\color{orange} 1&\color{blue} 2&\color{cyan}2           \\
7  &0&1&0&1&0&0&1&0&0&0&1&0&0&  \color{green}2&\color{red}2&\color{gray} 3&\color{magenta} 3&\color{orange} 3&\color{blue} 3&\color{cyan}2           \\
8  &0&1&0&1&0&1&0&1&0&0&1&0&0&  \color{green}2&\color{red}2&\color{gray} 2&\color{magenta} 2&\color{orange} 3&\color{blue} 3&\color{cyan}2           \\
9  &0&1&0&0&1&0&0&1&0&1&0&1&0&  \color{green}2&\color{red}1&\color{gray} 1&\color{magenta} 2&\color{orange} 2&\color{blue} 2&\color{cyan}3          \\
10 &0&0&1&0&0&0&1&0&0&0&1&0&1&  \color{green}3&\color{red}3&\color{gray} 3&\color{magenta} 3&\color{orange} 3&\color{blue} 3&\color{cyan}1          \\
11 &0&0&1&0&0&1&0&1&0&0&1&0&1&  \color{green}3&\color{red}3&\color{gray} 2&\color{magenta} 2&\color{orange} 3&\color{blue} 3&\color{cyan}1          \\
12 &0&0&1&0&0&1&0&0&1&0&0&1&1&  \color{green}3&\color{red}3&\color{gray} 2&\color{magenta} 1&\color{orange} 1&\color{blue} 2&\color{cyan}1          \\
13 &0&0&1&0&0&0&1&0&0&1&0&1&0&  \color{green}3&\color{red}3&\color{gray} 3&\color{magenta} 3&\color{orange} 2&\color{blue} 2&\color{cyan}3          \\
14 &0&0&1&0&0&1&0&1&0&1&0&1&0&  \color{green}3&\color{red}3&\color{gray} 2&\color{magenta} 2&\color{orange} 2&\color{blue} 2&\color{cyan}3          \\
\hline\hline
\end{tabular}
}
\end{center}
(a) Dispersion-free states  of the Kochen-Specker ``bug'' logic with 14 dispersion-free states
and (b) the associated generalized urn model.

}



\section{Summary}
\frame{
\frametitle{Summary \& ``mind map'' representing the present situation with regards to (quantum) contexts.}
\begin{center}
{\tiny
 %TeXCAD Picture [1.pic]. Options:
%\grade{\on}
%\emlines{\off}
%\epic{\off}
%\beziermacro{\on}
%\reduce{\on}
%\snapping{\off}
%\pvinsert{% Your \input, \def, etc. here}
%\quality{8.000}
%\graddiff{0.005}
%\snapasp{1}
%\zoom{4.0000}
\unitlength .5mm % = 1.565pt
\linethickness{0.4pt}
\ifx\plotpoint\undefined\newsavebox{\plotpoint}\fi % GNUPLOT compatibility
\begin{picture}(303.75,133.25)(0,0)
\put(106.125,125.25){\oval(51.25,16)[]}
\put(25.375,87.25){\oval(51.25,16)[]}
\put(105.625,87.5){\oval(51.25,16)[]}
\put(178.375,87.75){\oval(51.25,16)[]}
\put(105.5,125){\makebox(0,0)[cc]{context}}
\put(105,90.25){\makebox(0,0)[cc]{block}}
\put(105,85.25){\makebox(0,0)[cc]{Boolean subalgebra}}
\put(24,87.25){\makebox(0,0)[cc]{Boolean algebra}}
\put(177.75,90.25){\makebox(0,0)[cc]{block}}
\put(177.75,85.25){\makebox(0,0)[cc]{Boolean subalgebra}}
\put(80.25,117.5){\makebox(0,0)[cc]{}}
%\vector[middle](80.75,117.5)(31,98.75)
\put(55.875,108.125){\vector(-3,-1){.128}}\multiput(80.75,117.5)(-.1625816993,-.0612745098){306}{\line(-1,0){.1625816993}}
%\end
%\vector[middle](105,115.25)(105,99.5)
\put(105,107.375){\vector(0,-1){.128}}\put(105,115.25){\line(0,-1){15.75}}
%\end
%\vector[middle](131.75,117.5)(170,101.75)
\put(150.875,109.625){\vector(3,-1){.128}}\multiput(131.75,117.5)(.1488326848,-.0612840467){257}{\line(1,0){.1488326848}}
%\end
%\dottedbox(6.5,55.25)(33.25,13)
\put(6.5,55.25){\makebox(33.25,13)[cc]{}}
\multiput(6.372,68.122)(0,-.92857){15}{{\rule{.4pt}{.4pt}}}
\multiput(6.372,55.122)(.977941,0){35}{{\rule{.4pt}{.4pt}}}
\multiput(6.372,68.122)(.977941,0){35}{{\rule{.4pt}{.4pt}}}
\multiput(39.622,68.122)(0,-.92857){15}{{\rule{.4pt}{.4pt}}}
%\end
\put(23,64){\makebox(0,0)[cc]{two-valued}}
\put(23,59){\makebox(0,0)[cc]{measures}}
%\dottedbox(87.5,55.5)(33.25,13)
\put(87.5,55.5){\makebox(33.25,13)[cc]{}}
\multiput(87.372,68.372)(0,-.92857){15}{{\rule{.4pt}{.4pt}}}
\multiput(87.372,55.372)(.977941,0){35}{{\rule{.4pt}{.4pt}}}
\multiput(87.372,68.372)(.977941,0){35}{{\rule{.4pt}{.4pt}}}
\multiput(120.622,68.372)(0,-.92857){15}{{\rule{.4pt}{.4pt}}}
%\end
\put(103,64){\makebox(0,0)[cc]{two-valued}}
\put(103,59){\makebox(0,0)[cc]{measures}}
%\dottedbox(168.75,55.5)(33.25,13)
\put(168.75,55.5){\makebox(33.25,13)[cc]{}}
\multiput(168.622,68.372)(0,-.92857){15}{{\rule{.4pt}{.4pt}}}
\multiput(168.622,55.372)(.977941,0){35}{{\rule{.4pt}{.4pt}}}
\multiput(168.622,68.372)(.977941,0){35}{{\rule{.4pt}{.4pt}}}
\multiput(201.872,68.372)(0,-.92857){15}{{\rule{.4pt}{.4pt}}}
%\end
\put(185.5,64){\makebox(0,0)[cc]{no two-valued}}
\put(185.5,59){\makebox(0,0)[cc]{measure}}
\put(46.75,113.25){\makebox(0,0)[rc]{classical}}
\put(109,107.25){\makebox(0,0)[lc]{generalized urns}}
\put(109,102.25){\makebox(0,0)[lc]{automata}}
\put(157.25,116){\makebox(0,0)[lc]{quantized}}
%\vector[middle](165.25,77.25)(165.25,47.75)
\put(165.25,62.5){\vector(0,-1){.128}}\put(165.25,77.25){\line(0,-1){29.5}}
%\end
%\circle(165.25,22.5){41.869}
\put(186.185,22.5){\line(0,1){.9537}}
\put(186.163,23.454){\line(0,1){.9517}}
\put(186.098,24.405){\line(0,1){.9478}}
\multiput(185.989,25.353)(-.05051,.31396){3}{\line(0,1){.31396}}
\multiput(185.838,26.295)(-.04857,.2335){4}{\line(0,1){.2335}}
\multiput(185.643,27.229)(-.05916,.23104){4}{\line(0,1){.23104}}
\multiput(185.407,28.153)(-.055696,.182485){5}{\line(0,1){.182485}}
\multiput(185.128,29.066)(-.053293,.149799){6}{\line(0,1){.149799}}
\multiput(184.808,29.965)(-.060062,.147215){6}{\line(0,1){.147215}}
\multiput(184.448,30.848)(-.057177,.123708){7}{\line(0,1){.123708}}
\multiput(184.048,31.714)(-.05491,.105853){8}{\line(0,1){.105853}}
\multiput(183.609,32.561)(-.059675,.103241){8}{\line(0,1){.103241}}
\multiput(183.131,33.387)(-.05717,.089258){9}{\line(0,1){.089258}}
\multiput(182.617,34.19)(-.061177,.086561){9}{\line(0,1){.086561}}
\multiput(182.066,34.969)(-.058552,.075316){10}{\line(0,1){.075316}}
\multiput(181.481,35.722)(-.056293,.065973){11}{\line(0,1){.065973}}
\multiput(180.861,36.448)(-.05924,.06334){11}{\line(0,1){.06334}}
\multiput(180.21,37.144)(-.062064,.060575){11}{\line(-1,0){.062064}}
\multiput(179.527,37.811)(-.064759,.057685){11}{\line(-1,0){.064759}}
\multiput(178.815,38.445)(-.074052,.060142){10}{\line(-1,0){.074052}}
\multiput(178.074,39.047)(-.076715,.056706){10}{\line(-1,0){.076715}}
\multiput(177.307,39.614)(-.088021,.059058){9}{\line(-1,0){.088021}}
\multiput(176.515,40.145)(-.09062,.054986){9}{\line(-1,0){.09062}}
\multiput(175.699,40.64)(-.10466,.057151){8}{\line(-1,0){.10466}}
\multiput(174.862,41.097)(-.122463,.059798){7}{\line(-1,0){.122463}}
\multiput(174.005,41.516)(-.12506,.054157){7}{\line(-1,0){.12506}}
\multiput(173.129,41.895)(-.14863,.056471){6}{\line(-1,0){.14863}}
\multiput(172.237,42.234)(-.181258,.059569){5}{\line(-1,0){.181258}}
\multiput(171.331,42.532)(-.183784,.051249){5}{\line(-1,0){.183784}}
\multiput(170.412,42.788)(-.23241,.05353){4}{\line(-1,0){.23241}}
\multiput(169.483,43.002)(-.31281,.05718){3}{\line(-1,0){.31281}}
\multiput(168.544,43.174)(-.31509,.04287){3}{\line(-1,0){.31509}}
\put(167.599,43.302){\line(-1,0){.9501}}
\put(166.649,43.388){\line(-1,0){.9531}}
\put(165.696,43.43){\line(-1,0){.954}}
\put(164.742,43.428){\line(-1,0){.9529}}
\put(163.789,43.383){\line(-1,0){.9499}}
\multiput(162.839,43.295)(-.31496,-.04381){3}{\line(-1,0){.31496}}
\multiput(161.894,43.164)(-.31264,-.05811){3}{\line(-1,0){.31264}}
\multiput(160.956,42.989)(-.23225,-.05422){4}{\line(-1,0){.23225}}
\multiput(160.027,42.773)(-.18363,-.051798){5}{\line(-1,0){.18363}}
\multiput(159.109,42.514)(-.181079,-.06011){5}{\line(-1,0){.181079}}
\multiput(158.204,42.213)(-.148461,-.056914){6}{\line(-1,0){.148461}}
\multiput(157.313,41.872)(-.124897,-.05453){7}{\line(-1,0){.124897}}
\multiput(156.439,41.49)(-.122284,-.060164){7}{\line(-1,0){.122284}}
\multiput(155.583,41.069)(-.104489,-.057463){8}{\line(-1,0){.104489}}
\multiput(154.747,40.609)(-.090455,-.055257){9}{\line(-1,0){.090455}}
\multiput(153.933,40.112)(-.087844,-.05932){9}{\line(-1,0){.087844}}
\multiput(153.142,39.578)(-.076545,-.056935){10}{\line(-1,0){.076545}}
\multiput(152.377,39.008)(-.073872,-.060363){10}{\line(-1,0){.073872}}
\multiput(151.638,38.405)(-.064587,-.057878){11}{\line(-1,0){.064587}}
\multiput(150.927,37.768)(-.061883,-.06076){11}{\line(-1,0){.061883}}
\multiput(150.247,37.1)(-.05905,-.063516){11}{\line(0,-1){.063516}}
\multiput(149.597,36.401)(-.056096,-.06614){11}{\line(0,-1){.06614}}
\multiput(148.98,35.674)(-.058326,-.07549){10}{\line(0,-1){.07549}}
\multiput(148.397,34.919)(-.060919,-.086743){9}{\line(0,-1){.086743}}
\multiput(147.849,34.138)(-.056904,-.089429){9}{\line(0,-1){.089429}}
\multiput(147.336,33.333)(-.059367,-.103419){8}{\line(0,-1){.103419}}
\multiput(146.861,32.506)(-.054593,-.106016){8}{\line(0,-1){.106016}}
\multiput(146.425,31.658)(-.056808,-.123878){7}{\line(0,-1){.123878}}
\multiput(146.027,30.79)(-.059623,-.147394){6}{\line(0,-1){.147394}}
\multiput(145.669,29.906)(-.052846,-.149957){6}{\line(0,-1){.149957}}
\multiput(145.352,29.006)(-.055151,-.182651){5}{\line(0,-1){.182651}}
\multiput(145.076,28.093)(-.05847,-.23122){4}{\line(0,-1){.23122}}
\multiput(144.843,27.168)(-.04787,-.23364){4}{\line(0,-1){.23364}}
\multiput(144.651,26.234)(-.04957,-.31411){3}{\line(0,-1){.31411}}
\put(144.502,25.291){\line(0,-1){.9481}}
\put(144.397,24.343){\line(0,-1){.9519}}
\put(144.334,23.391){\line(0,-1){2.859}}
\put(144.408,20.532){\line(0,-1){.9475}}
\multiput(144.519,19.585)(.05144,-.3138){3}{\line(0,-1){.3138}}
\multiput(144.674,18.643)(.04927,-.23335){4}{\line(0,-1){.23335}}
\multiput(144.871,17.71)(.05985,-.23086){4}{\line(0,-1){.23086}}
\multiput(145.11,16.787)(.056241,-.182318){5}{\line(0,-1){.182318}}
\multiput(145.391,15.875)(.05374,-.149639){6}{\line(0,-1){.149639}}
\multiput(145.714,14.977)(.060502,-.147035){6}{\line(0,-1){.147035}}
\multiput(146.077,14.095)(.057546,-.123537){7}{\line(0,-1){.123537}}
\multiput(146.48,13.23)(.055225,-.105689){8}{\line(0,-1){.105689}}
\multiput(146.922,12.385)(.059983,-.103063){8}{\line(0,-1){.103063}}
\multiput(147.401,11.56)(.057436,-.089087){9}{\line(0,-1){.089087}}
\multiput(147.918,10.758)(.055292,-.07774){10}{\line(0,-1){.07774}}
\multiput(148.471,9.981)(.058776,-.075141){10}{\line(0,-1){.075141}}
\multiput(149.059,9.23)(.056489,-.065804){11}{\line(0,-1){.065804}}
\multiput(149.68,8.506)(.059429,-.063163){11}{\line(0,-1){.063163}}
\multiput(150.334,7.811)(.062244,-.06039){11}{\line(1,0){.062244}}
\multiput(151.019,7.147)(.064931,-.057491){11}{\line(1,0){.064931}}
\multiput(151.733,6.514)(.074231,-.059921){10}{\line(1,0){.074231}}
\multiput(152.475,5.915)(.076884,-.056477){10}{\line(1,0){.076884}}
\multiput(153.244,5.35)(.088197,-.058795){9}{\line(1,0){.088197}}
\multiput(154.038,4.821)(.090784,-.054716){9}{\line(1,0){.090784}}
\multiput(154.855,4.329)(.10483,-.056838){8}{\line(1,0){.10483}}
\multiput(155.694,3.874)(.122641,-.059433){7}{\line(1,0){.122641}}
\multiput(156.552,3.458)(.125221,-.053784){7}{\line(1,0){.125221}}
\multiput(157.429,3.081)(.148798,-.056027){6}{\line(1,0){.148798}}
\multiput(158.321,2.745)(.181435,-.059028){5}{\line(1,0){.181435}}
\multiput(159.229,2.45)(.183936,-.050701){5}{\line(1,0){.183936}}
\multiput(160.148,2.197)(.23257,-.05284){4}{\line(1,0){.23257}}
\multiput(161.079,1.985)(.31298,-.05625){3}{\line(1,0){.31298}}
\multiput(162.018,1.817)(.31522,-.04193){3}{\line(1,0){.31522}}
\put(162.963,1.691){\line(1,0){.9504}}
\put(163.914,1.608){\line(1,0){.9532}}
\put(164.867,1.569){\line(1,0){.954}}
\put(165.821,1.573){\line(1,0){.9528}}
\put(166.773,1.621){\line(1,0){.9496}}
\multiput(167.723,1.712)(.31483,.04475){3}{\line(1,0){.31483}}
\multiput(168.668,1.846)(.31246,.05905){3}{\line(1,0){.31246}}
\multiput(169.605,2.023)(.23209,.05492){4}{\line(1,0){.23209}}
\multiput(170.533,2.243)(.183474,.052346){5}{\line(1,0){.183474}}
\multiput(171.451,2.505)(.180899,.06065){5}{\line(1,0){.180899}}
\multiput(172.355,2.808)(.14829,.057357){6}{\line(1,0){.14829}}
\multiput(173.245,3.152)(.124734,.054903){7}{\line(1,0){.124734}}
\multiput(174.118,3.537)(.122103,.060528){7}{\line(1,0){.122103}}
\multiput(174.973,3.96)(.104317,.057775){8}{\line(1,0){.104317}}
\multiput(175.807,4.422)(.09029,.055526){9}{\line(1,0){.09029}}
\multiput(176.62,4.922)(.087667,.059582){9}{\line(1,0){.087667}}
\multiput(177.409,5.458)(.076375,.057163){10}{\line(1,0){.076375}}
\multiput(178.173,6.03)(.073692,.060583){10}{\line(1,0){.073692}}
\multiput(178.91,6.636)(.064414,.05807){11}{\line(1,0){.064414}}
\multiput(179.618,7.275)(.061701,.060945){11}{\line(1,0){.061701}}
\multiput(180.297,7.945)(.058861,.063692){11}{\line(0,1){.063692}}
\multiput(180.944,8.646)(.055898,.066308){11}{\line(0,1){.066308}}
\multiput(181.559,9.375)(.058101,.075664){10}{\line(0,1){.075664}}
\multiput(182.14,10.132)(.060659,.086925){9}{\line(0,1){.086925}}
\multiput(182.686,10.914)(.056636,.089598){9}{\line(0,1){.089598}}
\multiput(183.196,11.72)(.059058,.103596){8}{\line(0,1){.103596}}
\multiput(183.668,12.549)(.054277,.106179){8}{\line(0,1){.106179}}
\multiput(184.103,13.399)(.056438,.124047){7}{\line(0,1){.124047}}
\multiput(184.498,14.267)(.059182,.147571){6}{\line(0,1){.147571}}
\multiput(184.853,15.152)(.052398,.150114){6}{\line(0,1){.150114}}
\multiput(185.167,16.053)(.054606,.182815){5}{\line(0,1){.182815}}
\multiput(185.44,16.967)(.05778,.23139){4}{\line(0,1){.23139}}
\multiput(185.671,17.893)(.04717,.23378){4}{\line(0,1){.23378}}
\multiput(185.86,18.828)(.04863,.31425){3}{\line(0,1){.31425}}
\put(186.006,19.771){\line(0,1){.9484}}
\put(186.109,20.719){\line(0,1){1.781}}
%\end
\put(166,25){\makebox(0,0)[cc]{Gleason theorem}}
\put(165.25,20){\makebox(0,0)[cc]{(Born rule)}}
\put(90,20.75){\framebox(29.75,20.25)[]{}}
\put(8,20.75){\framebox(29.75,20.25)[]{}}
\put(105,34){\makebox(0,0)[cc]{convex}}
\put(105,29){\makebox(0,0)[cc]{sum}}
\put(23,34){\makebox(0,0)[cc]{convex}}
\put(23,29){\makebox(0,0)[cc]{sum}}
\put(111.5,48.25){\makebox(0,0)[lc]{probabilities}}
\put(29.5,48.25){\makebox(0,0)[lc]{probabilities}}
\put(169.75,48.25){\makebox(0,0)[lc]{probabilities}}
\put(58.75,6){\framebox(28.5,23)[cc]{}}
\put(73.5,19){\makebox(0,0)[cc]{finite pasting}}
\put(73.5,14){\makebox(0,0)[cc]{of blocks}}
\put(190,6){\framebox(29,23)[cc]{}}
\put(206,23){\makebox(0,0)[cc]{continuous }}
\put(206,18){\makebox(0,0)[cc]{pasting}}
\put(206,13){\makebox(0,0)[cc]{of blocks}}
%\qbezvec[middle](80,79.75)(67.875,55.75)(73.25,32.75)
\put(72.25,56){\vector(-1,-4){.128}}\qbezier(80,79.75)(67.875,55.75)(73.25,32.75)
%\end
%\qbezvec[middle](204.5,79.75)(221.5,59.75)(209.5,33.75)
\put(214.25,58.25){\vector(1,-4){.128}}\qbezier(204.5,79.75)(221.5,59.75)(209.5,33.75)
%\end
%\qbezvec(31.5,74.75)(45.625,68.375)(45.25,76.5)
\put(45.25,76.5){\vector(0,1){.128}}\qbezier(31.5,74.75)(45.625,68.375)(45.25,76.5)
%\end
\put(76,47.25){\makebox(0,0)[lc]{logic}}
\put(208,37.25){\makebox(0,0)[rc]{logic}}
\put(56,69){\makebox(0,0)[rc]{logic}}
\put(44.5,70.75){\makebox(0,0)[lc]{}}
\put(44.25,71.75){\makebox(0,0)[rc]{}}
\put(46.25,71){\makebox(0,0)[lc]{}}
%\vector[middle](23.75,77)(23.75,71)
\put(23.75,74){\vector(0,-1){1.5}}\put(23.75,77){\line(0,-1){6}}
%\end
%\vector[middle](23.5,50.5)(23.5,44.5)
\put(23.5,47.5){\vector(0,-1){1.5}}\put(23.5,50.5){\line(0,-1){6}}
%\end
%\vector[middle](105,77)(105,71)
\put(105,74){\vector(0,-1){1.5}}\put(105,77){\line(0,-1){6}}
%\end
%\vector[middle](104.75,50.5)(104.75,44.5)
\put(104.75,47.5){\vector(0,-1){1.5}}\put(104.75,50.5){\line(0,-1){6}}
%\end
%\vector[middle](186.25,77)(186.25,71)
\put(186.25,74){\vector(0,-1){1.5}}\put(186.25,77){\line(0,-1){6}}
%\end
\end{picture}
}
\end{center}
}


\frame{



\centerline{\Large Thank you for your attention!}

\begin{center}
$\widetilde{\qquad \qquad }$
$\widetilde{\qquad \qquad}$
$\widetilde{\qquad \qquad }$
\end{center}
 }

\end{document}
