\documentclass[%
  %reprint,
  %twocolumn,
 %superscriptaddress,
 %groupedaddress,
 %unsortedaddress,
 %runinaddress,
 %frontmatterverbose,
   preprint,
 showpacs,
 showkeys,
 preprintnumbers,
 %nofootinbib,
 %nobibnotes,
 %bibnotes,
 amsmath,amssymb,
 aps,
 % prl,
  pra,
 % prb,
 % rmp,
 %prstab,
 %prstper,
  %longbibliography,
 %floatfix,
 %lengthcheck,%
 ]{revtex4-1}


\usepackage[dvipsnames]{xcolor}

\usepackage{mathptmx}% http://ctan.org/pkg/mathptmx

\usepackage{amssymb,amsthm,amsmath}

\usepackage{tikz}
\usetikzlibrary{calc}

%\usepackage[breaklinks=true,colorlinks=true,anchorcolor=blue,citecolor=blue,filecolor=blue,menucolor=blue,pagecolor=blue,urlcolor=blue,linkcolor=blue]{hyperref}
\usepackage{graphicx}% Include figure files
\usepackage{url}

\usepackage{hyperref}
\hypersetup{
    colorlinks=true, %set true if you want colored links
    linktoc=all,     %set to all if you want both sections and subsections linked
    linkcolor=blue,  %choose some color if you want links to stand out
    citecolor=blue,
    filecolor=red,
    urlcolor=blue
}


\begin{document}

\title{No ``quantum supremacy using a programmable superconducting processor''}


\author{Karl Svozil}
\email{svozil@tuwien.ac.at}
\homepage{http://tph.tuwien.ac.at/~svozil}

\affiliation{Institute for Theoretical Physics,
Vienna  University of Technology,
Wiedner Hauptstrasse 8-10/136,
1040 Vienna,  Austria}



\date{\today}

\begin{abstract}
Insofar as certain quantum systems such as the one realized by {\it Google AI}'s Sycamore processor~\cite{Arute2019} have no algorithmically or computationally compressible classical double they perform similarly to classical chaotic systems. And insofar as they are ``programmable'' they are a very far cry from being universal in the sense of Church, Kleene, and Turing. Claims that any such quantum device constitutes a (quantum) supremacy (I prefer the term ``advantage'') beyond their very restricted specification are misleading and signifying a new kind of scientific sensationalism akin to marketing, perception management, and product placement.
\end{abstract}



\keywords{quantum advantage, quantum parallelism, quantum random number generators,classical chaotic systems}

\maketitle

Every system is a perfect and oftentimes efficient {\em simulacrum} of itself.
Undoubtedly this seems to be the case for the particular task {\it Google AI}'s Sycamore processor~\cite{Arute2019}
has been designed for: sampling the output of a pseudo-random quantum circuit.

But the same is also true for chaotic systems exhibiting an exponential sensitivity to changes of the initial state.
Such systems are well known since Maxwell and Poincare;
they have been popularized under the name ``deterministic chaos.''
To paraphrase claims made by {\it Google AI}'s Sycamore processor group report:
a physically realizable classical chaotic system represents a polynomial-time computing machinery
for which no efficient method is known to exist for any universal computing machinery; both quantum and classical.
This qualifies as a violation of the extended Church-Turing thesis formulated by Bernstein and Vazirani,
in a similar way as quoted for quantum sampling.

Indeed, if one postulates the validity of quantum mechanics, in particular, complementarity and value indefiniteness (aka contextuality), then a genuine quantum advantage beyond the violation of the extended Church-Turing thesis
-- namely a violation of the original Church-Turing thesis --
has already been realized by numerous quantum random number generators. Such devices are even commercially available.
Because of the irreducible randomness of quantum outcomes, these quantum oracles for random numbers offer an absolute advantage over any classical or quantum pseudorandom algorithms.

The crucial question in all of the above claims of advantage seems to be this:
which kind of computational tasks can be efficiently performed relative to
which kind of computational architecture?
In the context of Von Neumann-Zuse and also universal Turing machine type architectures two tasks appear as rather natural:
One task, inspired by computational complexity theory, is the assumption -- often referred to as Cook-Karp thesis --
that NP-complete problems are computationally intractable.
The other task is the effective simulation of universal computation in the sense of Church, Turing and Kleene.


{\it Google AI}'s Sycamore processor, as well as classical chaotic systems, yield no advantage for both of these tasks.
One of the reasons for this is that the former present very specialized computational capacities
which fail to render effective methods for solving NP-complete problems; and even less so ``true'' universal computation.
In particular, the claim~\cite{Google-2019-qs} that Sycamore is a {\em ``fully programmable 54-qubit processor''}
is unfounded if ``fully'' is understood in the usual universally computable sense.

\bibliography{svozil}

\end{document}
