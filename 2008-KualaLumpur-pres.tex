%\documentclass[pra,showpacs,showkeys,amsfonts,amsmath,twocolumn,handou]{revtex4}
\documentclass[amsmath,red,table,sans,handout]{beamer}
%\documentclass[pra,showpacs,showkeys,amsfonts]{revtex4}
%\documentclass[pra,showpacs,showkeys,amsfonts]{revtex4}
\usepackage[T1]{fontenc}
%%\usepackage{beamerthemeshadow}
\usepackage[headheight=1pt,footheight=10pt]{beamerthemeboxes}
\addfootboxtemplate{\color{structure!80}}{\color{white}\tiny \hfill Karl Svozil (TU Vienna)\hfill}
\addfootboxtemplate{\color{structure!65}}{\color{white}\tiny \hfill Quantum Logic\hfill}
\addfootboxtemplate{\color{structure!50}}{\color{white}\tiny \hfill EQuaLS2, November 2008\hfill}
%\usepackage[dark]{beamerthemesidebar}
%\usepackage[headheight=24pt,footheight=12pt]{beamerthemesplit}
%\usepackage{beamerthemesplit}
%\usepackage[bar]{beamerthemetree}
\usepackage{graphicx}
\usepackage{pgf}
%\usepackage[usenames]{color}
%\newcommand{\Red}{\color{Red}}  %(VERY-Approx.PANTONE-RED)
%\newcommand{\Green}{\color{Green}}  %(VERY-Approx.PANTONE-GREEN)

%\RequirePackage[german]{babel}
%\selectlanguage{german}
%\RequirePackage[isolatin]{inputenc}

\pgfdeclareimage[height=0.5cm]{logo}{tu-logo}
\logo{\pgfuseimage{logo}}
\beamertemplatetriangleitem
%\beamertemplateballitem

\beamerboxesdeclarecolorscheme{alert}{red}{red!15!averagebackgroundcolor}
\beamerboxesdeclarecolorscheme{alert2}{green}{green!15!averagebackgroundcolor}
%\begin{beamerboxesrounded}[scheme=alert,shadow=true]{}
%\end{beamerboxesrounded}

%\beamersetaveragebackground{green!10}

%\beamertemplatecircleminiframe

\begin{document}

\title{\bf \textcolor{red}{Quantum Logic}}
%\subtitle{Naturwissenschaftlich-Humanisticher Tag am BG 19\\Weltbild und Wissenschaft\\http://tph.tuwien.ac.at/\~{}svozil/publ/2005-BG18-pres.pdf}
\subtitle{\textcolor{orange!60}{\footnotesize http://tph.tuwien.ac.at/$\sim$svozil/publ/2008-KualaLumpur-pres.pdf}
\\
\textcolor{orange!60}{\small http://arxiv.org/abs/quant-ph/0609209}\\ \tiny{$\;$}\\
\textcolor{gray!60}{\small in memory of Andrew Gleason} \\
\textcolor{gray!60}{\small $\ast$November 4, 1921 $-$ $\dagger$October 17, 2008}
}
\author{Karl Svozil}
\institute{Institut f\"ur Theoretische Physik, Vienna University of Technology, \\
Wiedner Hauptstra\ss e 8-10/136, A-1040 Vienna, Austria\\
svozil@tuwien.ac.at
%{\tiny Disclaimer: Die hier vertretenen Meinungen des Autors verstehen sich als Diskussionsbeitr�ge und decken sich nicht notwendigerweise mit den Positionen der Technischen Universit�t Wien oder deren Vertreter.}
}
\date{Kuala Lumpur, Malaysia, November 2008}
\maketitle



%\frame{
%\frametitle{Contents}
%\tableofcontents
%}


\section{Introduction}
\subsection{History}
\frame{
\frametitle{History}

\begin{itemize}
\item<1->
Around 1900 precision measurements of the blackbody radiation at the {\em Kaiser Wilhelm Institut} in {\em Berlin;}


\item<1->
December 14th, 1900 {\em Max Planck},  {\em Lecture at the Deutsche Physikalische Gesellschaft}
{\em ``{\"{U}}ber eine {V}erbesserung der {W}ien'schen {S}pectralgleichung''} --- a ``desperate step;''

\item<1->
1905 {\em Albert Einstein's} light quantum hypothesis

\item<1->
{\em Heisenberg, Born, Sommerfeld, Bohr,} $\ldots$, {Erwin Schr{\"{o}}dinger's} computation spectral lines of hydrogen;
1935 {\em ``Die gegenw{\"{a}}rtige {S}ituation in der {Q}uantenmechanik;''}
\end{itemize}
}

\frame{
\frametitle{History cntd.}

\begin{itemize}

\item<1->
1936 {\em Garrett Birkhoff} and {\em John von  Neumann},
{\em ``The Logic of Quantum Mechanics''}


\item<1->
1957 {\em Andrew M. Gleason's}
 {\em ``Measures on the closed subspaces of a {H}ilbert space,''}
based on {\em George W. Mackey's} {\em ``Quantum mechanics and {H}ilbert space;''}


\item<1->
1964/6 {\em John S. Bell's} {\em ``On the {E}instein {P}odolsky {R}osen paradox''}/
 {\em ``On the Problem of hidden variables in quantum mechanics;''}

\item<1->
1967 {\em Simon Kochen} and {\em Ernst P. Specker's}
{\em ``The Problem of Hidden Variables in Quantum Mechanics.''}


\end{itemize}
}


\subsection{``Nonclassical'' features of quantum mechanics}

\frame{
\frametitle{``Nonclassical'' features of quantum mechanics}

\begin{itemize}

\item<1->
Complementarity (empirical non-omniscience);

\item<1->
Random occurrence of outcomes;

\item<1->
Value indefiniteness (often referred to as ``contextuality''):
nonexistence of classical truth assignments for cerain (even finite) sets of observables simultaneously;

\item<1->
Entanglement (Schr\"odinger's ``Verschr\"ankung'');

\item<1->
Nonlocality,
parameter independence of outcomes \& outcome dependence for entangled quantum states;

\item<1->
Stronger-than-classical quantum correlations;

\item<1->
Not all classical tautologies are quantum tautologies.

\end{itemize}

}


\subsection{What can be learned from quantum logic?}

\frame{
\frametitle{What can be learned from quantum logic?}
\begin{beamerboxesrounded}[scheme=alert,shadow=true]{}
\begin{itemize}

\item<1->
Any quantum logic is the ``pasting'' of ``quasi-classical'' logics, which are synonymously called ``blocks,''
``Boolean subalgebras'' or ``maximal (operators) observables.''

\item<1->
The probabilities and expectation values associated with quantized systems can be derived from the above assumption (via Gleason's theorem).

\item<1->
Classical omniscience does not hold for quantized systems: not all quantum observables which {\em could} be measured
{\em can} operationally be measured synonymously (due to complementarity and the Kochen-Specker theorem).

\end{itemize}
\end{beamerboxesrounded}
}

\section{Classical Logic}

\frame{



\centerline{\Huge Part I:  Classical Logic}

\begin{center}
$\widetilde{\qquad \qquad }$
$\widetilde{\qquad \qquad}$
$\widetilde{\qquad \qquad }$
\end{center}
 }


\subsection{Boolean Algebra as classical logic}
\frame{
\frametitle{Boolean Algebra as classical logic}

\begin{beamerboxesrounded}[scheme=alert,shadow=true]{}
A classical logic is a Boolean algebra ${\mathfrak B}$ is a set endowed with
two binary operations $\wedge$ (called ``and'')  and  $\vee$ (called ``or''),
as well as a unary operation ``~$'$~'' (called ``complement'' or ``negation'').
It also contains two elements $1$ (called ``true'') and $0$ (called ``false'') satisfying
associativity,   commutativity, the  absorption law and   distributivity.
Every element has a unique complement.
\end{beamerboxesrounded}
{\footnotesize
\begin{itemize}

\item<1->
A typical example of a Boolean algebra is set theory. The operations
are identified with the set theoretic intersection, union, and complement, respectively.
The implication relation is identified with the subset relation.

\item<1->
A classical Boolean algebra is the representation
of all possible ``propositions'' or ``knowables.''
Every knowable can be combined with every other one by the standard logical operations
``and'' and ``or.''
Operationally,
all knowables are in principle knowable simultaneously.
Stated differently:
within the Boolean ``universe,'' the knowables are
all consistently co-knowable.

\end{itemize}
}
}

\frame{
\frametitle{Representation of propositions in terms of vectors}

Suppose there are $d$ mutually different outcomes $o_1,o_2,\ldots ,\o_d$ of an experiment.
Then, identify [cf. Mermin, {\em Quantum Computer Science} (CUP 2007)]
every such outcome with one unique  {\em base vector} of an orthonormal (Cartesian standard) basis of  {\em $d$-dimensional vector space;} i.e.,
$$
\begin{array}{rcl}
{\bf o}_1 &=&(1,0,\ldots ,0),\\
{\bf o}_2 &=&(0,1,\ldots ,0),\\
&\cdots & \\
{\bf o}_d &=&(0,0,\ldots ,1).
\end{array}
$$


\begin{itemize}

\item<1->  For dichotomic outcomes (e.g., ``true'' and ``false''), we obtain two vectors in two-dimensional vector space:
$
{\bf o}_1 =(1,0)$ and $
{\bf o}_2 =(0,1)
$.

\item<1->  For $n$-partite systems, the outcomes can be presented as the product of the single outcome vectors.

\item<1->  Generalizable to quantized systems. In this case, the vectors span linear subspaces of a Hilbert space.

\end{itemize}

}

\subsection{Classical probabilities}
\frame{
\frametitle{Classical probabilities}
{\small
Classical probabilities and joint probabilities can be represented as
points of a {\em convex polytope}
spanned by all possible ``extreme cases'' of the classical Boolean algebra; more formally:
by all two-valued measures on the Boolean algebra. Two-valued measures, also
called dispersionless measures or valuations, acquire only the values ``0'' and ``1,''
interpretable as  falsity and truth, respectively.
If some events are independently measured, then their joint probability $pq\cdots$ can be expressed as the product of their individual probabilities
$p$, $q$, $\ldots$.

\begin{beamerboxesrounded}[scheme=alert,shadow=true]{}
The associated {\em correlation polytope}
is  spanned by a convex combination  of vertices, which are vectors of the form $(p,q, \ldots, pq,\ldots )$,
where the components are the individual probabilities of independent events which take on the values $0$ and $1$,
together with their joint probabilities, which are the products of the individual probabilities.
The polytope faces impose  ``inside--outside'' distinctions.
The associated inequalities must be obeyed
by all classical probability measures; they are  bounds on classical (joint) probabilities
termed {\em ``conditions of possible experience''} by Boole.
\end{beamerboxesrounded}
}
}



\frame{
\frametitle{Example I: Two-event ``1--1'' case}
{\small
The simplest nontrivial example are two propositions as well as their joint proposition
\begin{quote}
$E\equiv${\em ``a particle detector aligned along direction ${\bf a}$ clicks,''}
\\
$F\equiv${\em ``a particle detector aligned along direction ${\bf b}$ clicks,''}
\\
$E\wedge F \equiv$ {\em ``the two particle detectors
aligned along directions ${\bf a}$ and ${\bf b}$ click.''}
\end{quote}
The notation ``1--1'' alludes to the experimental setup, in which
the two events are registered by detectors located at two ``adjacent sites.''

For three measurements of  $E_1,E_2,E_3$ on the ``left,'' and three measurements of  $F_1,F_2,F_3$ on the ``right'' hand side, along directions $\theta_i$:
{  \begin{center}
\includegraphics[width=70mm]{2006-ql-eifj}
  \end{center} }

}
}


\frame{
\frametitle{Example I cntd.}
{\small
Construction of the correlation polytope for two  events:  the four possible cases are represented by
the truth table, whose rows can be interpreted as three-dimensional vectors forming the vertices of the correlation polytope.\\
$\;$  \\
\centerline{
\qquad\begin{tabular}{|c|ccc|}
\hline\hline
& $\quad E\quad $ & $\quad F\quad $ & $\quad E\wedge F \equiv E \cdot F\quad $\\
\hline
$1$& 0&0&0   \\
$2$& 0&1&0   \\
$3$ & 1&0&0  \\
$4$ & 1&1&1  \\
\hline\hline
\end{tabular}
 }
$\;$  \\
The correlation polytope for the probabilities $p$, $q$
and the joint probabilities $pq$ of an occurrence of $E$, $F$, and both $E\& F$ can be defined by
$$
\kappa_1 (0,0,0)+
 \kappa_2 (0,1,0)+
 \kappa_3 (1,0,0)+
 \kappa_4 (1,1,1)
=(\kappa_3+\kappa_4, \kappa_2+\kappa_4,\kappa_4)$$
which is spanned by the convex sum $ \kappa_1 +
 \kappa_2 +
 \kappa_3 +
 \kappa_4 =1$
of these four vectors.
$\kappa_i$ can be interpreted as the normalized weight for event $i$ to occur.
}
}

\frame{
\frametitle{Example I cntd.}
 {\small


The resulting four faces of the polytope are characterized by half-spaces
which are obtained by solving the so-called {\em hull problem;} the solution is
\centerline{
\begin{tabular}{|c|c|}
\hline\hline
 & full facet inequality\\
\hline
$1$& $pq \ge 0$ \\
$2$& $p \ge pq$   \\
$3$ & $p \ge pq$  \\
$4$ & $pq \ge p+q-1$  \\
\hline\hline
\end{tabular}
}

One of the consequences are bounds on joint occurrences of events.
Suppose, for example, that
the probability of a click in detector aligned along direction ${\bf a}$ is 0.9, and
the probability of a click in the second detector aligned along direction ${\bf b}$ is 0.7.
Then inequality 4 forces us to accept that the probability
that both detector register clicks cannot be smaller than $0.9+0.7-1= 0.6$.
If, for instance, somebody comes up with a joint probability of $0.4$, we would
know that this result is flawed,
possibly by fundamental measurement errors, or by cheating, or by (quantum) ``magic.''
}
}

\frame{
\frametitle{Example II: Four-event ``2--2'' case}

A configuration  discussed in quantum mechanics is one with four events
grouped into two equal parts $E_1,E_2$ and $F_1,F_2$.
There are $2^4$ different cases  of occurrence or nonoccurrence of these four events
enumerated in the following Table:
$\;$  \\
$\;$  \\
\centerline{
{\tiny
\begin{tabular}{|c|cccccccc|ccccc}
\hline\hline
& $ E_1$ & $ E_2$ & $F_1$& $F_2$& $ E_1F_1$ & $ E_1F_2$ & $E_2F_1$& $E_2F_2$\\
\hline
1   &0  &0  &0  &0  &0  &0  &0  &0    \\
2   &0  &0  &0  &1  &0  &0  &0  &0    \\
3   &0  &0  &1  &0  &0  &0  &0  &0    \\
4   &0  &0  &1  &1  &0  &0  &0  &0    \\
5   &0  &1  &0  &0  &0  &0  &0  &0    \\
6   &0  &1  &0  &1  &0  &0  &0  &1    \\
7   &0  &1  &1  &0  &0  &0  &1  &0    \\
8   &0  &1  &1  &1  &0  &0  &1  &1    \\
9   &1  &0  &0  &0  &0  &0  &0  &0    \\
10   &1  &0  &0  &1  &0  &1  &0  &0    \\
11   &1  &0  &1  &0  &1  &0  &0  &0    \\
12   &1  &0  &1  &1  &1  &1  &0  &0    \\
13   &1  &1  &0  &0  &0  &0  &0  &0    \\
14   &1  &1  &0  &1  &0  &1  &0  &1    \\
15   &1  &1  &1  &0  &1  &0  &1  &0    \\
16   &1  &1  &1  &1  &1  &1  &1  &1    \\
\hline\hline
\end{tabular}
}
}

}


\frame{
\frametitle{Example II cntd.}

By solving the hull problem, one obtains a set of conditions of possible experience
which represent the bounds on classical probabilities enumerated in the Table below:
$\;$  \\ $\;$  \\

\centerline{
{\tiny
\begin{tabular}{|c|cc|}
\hline\hline
 & full facet inequality &  inequality for $p_1=p_2=q_1=q_2={1\over 2}$\\
\hline
 1 &     $p_1q_1 \ge 0$  &
 $ p_1q_1\ge 0$   \\
$\cdots$&$\cdots$&$\cdots$\\
%
%
%
 13  &     $p_1q_1 \ge   p_1 + q_1 - 1$  &
 $p_1q_1 \ge  0$   \\
%
$\cdots$&$\cdots$&$\cdots$\\
%
%
%
 17 &     $0 \ge   p_1q_1 + p_1q_2 + p_2q_1 - p_2q_2 - p_1 - q_1$  &
 $1 \ge  + p_1q_1 + p_1q_2 + p_2q_1 - p_2q_2$   \\
%
 18 &     $p_1q_1 + p_1q_2 + p_2q_1 - p_2q_2 - p_1 - q_1 \ge -1$  &
 $ p_1q_1 + p_1q_2 + p_2q_1 - p_2q_2 \ge 0$   \\
%
%
 19 &     $0 \ge   p_1q_1 + p_1q_2 - p_2q_1 + p_2q_2- p_1 - q_2$  &
 $1 \ge  + p_1q_1 + p_1q_2 - p_2q_1 + p_2q_2$   \\
%
 20  &     $p_1q_1 + p_1q_2 - p_2q_1 + p_2q_2- p_1 - q_2 \ge -1$  &
 $p_1q_1 + p_1q_2 - p_2q_1 + p_2q_2 \ge 0$   \\
%
%
 21 &     $ 0\ge p_1q_1 - p_1q_2 + p_2q_1 + p_2q_2- p_2 - q_1$  &
 $1 \ge   p_1q_1 - p_1q_2 + p_2q_1 + p_2q_2$   \\
%
 22&     $p_1q_1 - p_1q_2 + p_2q_1 + p_2q_2- p_2 - q_1 \ge -1$  &
 $p_1q_1 - p_1q_2 + p_2q_1 + p_2q_2 \ge 0$   \\
%
%
 23  &     $0 \ge   - p_1q_1 + p_1q_2 + p_2q_1 + p_2q_2- p_2 - q_2$  &
 $1 \ge  - p_1q_1 + p_1q_2 + p_2q_1 + p_2q_2$   \\
%
 24  &     $- p_1q_1 + p_1q_2 + p_2q_1 + p_2q_2- p_2 - q_2 \ge -1$  &
 $  - p_1q_1 + p_1q_2 + p_2q_1 - p_2q_2 \ge 0$   \\
 \hline\hline
\end{tabular}
}
}
$\;$  \\
For historical reasons, the bounds 17-18, 19-20, 21-22, and 23-24
are  called the Clauser-Horne inequalities.
They are equivalent (up to permutations of $p_i,q_i$),
and are the only additional inequalities structurally different from the two-event ``1--1''
case.
}


\subsection{Classical correlations}

\frame{
\frametitle{General setup for two-particle correlations}

\begin{itemize}
\item<1->
Two measurement directions ${ a}$ \& ${ b}$
of two
dichotomic observables
$O({ a})$ \&
$O({ b})$
with values ``-1'' or ``1''
at two spatially separated locations.
\item<1->
The measurement direction ${a}$ at ``Alice's location''
is unknown to an observer ``Bob'' measuring ${ b}$ and {\it vice versa}.
\item<1->
A two-particle correlation function $E(a,b )$
is defined by averaging over the product of the outcomes $O({ a})_i, O({ b} )_i\in \{-1,1\}$
in the $i$th experiment for a total of $N$ experiments; i.e.,  $$E(a,b )={1\over N}\sum_{i=1}^N O({ a})_i O({ b})_i$$
\end{itemize}
}




\frame{
\frametitle{Classical correlations for two-particle ``perfect correlation''}
Assume uniform  distribution of (opposite) ``angular momenta''
of the two particles; Alice measuring along angle $a$, Bob measuring along $b$:

\begin{center}
%TexCad Options
%\grade{\on}
%\emlines{\off}
%\beziermacro{\on}
%\reduce{\on}
%\snapping{\off}
%\quality{8.00}
%\graddiff{0.01}
%\snapasp{1}
%\zoom{0.60}
\unitlength 0.40mm
\linethickness{0.4pt}
\begin{picture}(220.35,68.50)
(0,0)
%\circle(30.25,29.75){61.53}
\put(30.25,60.52){\line(1,0){1.23}}
\put(31.48,60.49){\line(1,0){1.22}}
\multiput(32.70,60.42)(0.61,-0.06){2}{\line(1,0){0.61}}
\multiput(33.92,60.30)(0.61,-0.09){2}{\line(1,0){0.61}}
\multiput(35.14,60.12)(0.60,-0.11){2}{\line(1,0){0.60}}
\multiput(36.34,59.91)(0.40,-0.09){3}{\line(1,0){0.40}}
\multiput(37.54,59.64)(0.40,-0.10){3}{\line(1,0){0.40}}
\multiput(38.73,59.32)(0.29,-0.09){4}{\line(1,0){0.29}}
\multiput(39.90,58.96)(0.29,-0.10){4}{\line(1,0){0.29}}
\multiput(41.06,58.55)(0.28,-0.11){4}{\line(1,0){0.28}}
\multiput(42.20,58.10)(0.22,-0.10){5}{\line(1,0){0.22}}
\multiput(43.32,57.60)(0.22,-0.11){5}{\line(1,0){0.22}}
\multiput(44.42,57.06)(0.22,-0.12){5}{\line(1,0){0.22}}
\multiput(45.50,56.47)(0.18,-0.10){6}{\line(1,0){0.18}}
\multiput(46.55,55.84)(0.17,-0.11){6}{\line(1,0){0.17}}
\multiput(47.58,55.17)(0.17,-0.12){6}{\line(1,0){0.17}}
\multiput(48.58,54.46)(0.14,-0.11){7}{\line(1,0){0.14}}
\multiput(49.55,53.71)(0.13,-0.11){7}{\line(1,0){0.13}}
\multiput(50.49,52.92)(0.13,-0.12){7}{\line(1,0){0.13}}
\multiput(51.40,52.10)(0.11,-0.11){8}{\line(1,0){0.11}}
\multiput(52.27,51.23)(0.12,-0.13){7}{\line(0,-1){0.13}}
\multiput(53.11,50.34)(0.11,-0.13){7}{\line(0,-1){0.13}}
\multiput(53.91,49.41)(0.11,-0.14){7}{\line(0,-1){0.14}}
\multiput(54.68,48.45)(0.10,-0.14){7}{\line(0,-1){0.14}}
\multiput(55.40,47.46)(0.11,-0.17){6}{\line(0,-1){0.17}}
\multiput(56.09,46.45)(0.11,-0.17){6}{\line(0,-1){0.17}}
\multiput(56.74,45.40)(0.10,-0.18){6}{\line(0,-1){0.18}}
\multiput(57.34,44.33)(0.11,-0.22){5}{\line(0,-1){0.22}}
\multiput(57.90,43.24)(0.10,-0.22){5}{\line(0,-1){0.22}}
\multiput(58.41,42.13)(0.12,-0.28){4}{\line(0,-1){0.28}}
\multiput(58.89,41.00)(0.11,-0.29){4}{\line(0,-1){0.29}}
\multiput(59.31,39.85)(0.09,-0.29){4}{\line(0,-1){0.29}}
\multiput(59.69,38.68)(0.11,-0.39){3}{\line(0,-1){0.39}}
\multiput(60.02,37.50)(0.10,-0.40){3}{\line(0,-1){0.40}}
\multiput(60.31,36.30)(0.12,-0.60){2}{\line(0,-1){0.60}}
\multiput(60.55,35.10)(0.09,-0.61){2}{\line(0,-1){0.61}}
\multiput(60.74,33.89)(0.07,-0.61){2}{\line(0,-1){0.61}}
\put(60.88,32.67){\line(0,-1){1.22}}
\put(60.97,31.45){\line(0,-1){1.23}}
\put(61.01,30.22){\line(0,-1){1.23}}
\put(61.01,28.99){\line(0,-1){1.23}}
\put(60.95,27.77){\line(0,-1){1.22}}
\multiput(60.85,26.54)(-0.08,-0.61){2}{\line(0,-1){0.61}}
\multiput(60.70,25.33)(-0.10,-0.61){2}{\line(0,-1){0.61}}
\multiput(60.49,24.12)(-0.08,-0.40){3}{\line(0,-1){0.40}}
\multiput(60.25,22.91)(-0.10,-0.40){3}{\line(0,-1){0.40}}
\multiput(59.95,21.72)(-0.11,-0.39){3}{\line(0,-1){0.39}}
\multiput(59.61,20.55)(-0.10,-0.29){4}{\line(0,-1){0.29}}
\multiput(59.22,19.38)(-0.11,-0.29){4}{\line(0,-1){0.29}}
\multiput(58.78,18.24)(-0.10,-0.23){5}{\line(0,-1){0.23}}
\multiput(58.30,17.11)(-0.11,-0.22){5}{\line(0,-1){0.22}}
\multiput(57.77,16.00)(-0.11,-0.22){5}{\line(0,-1){0.22}}
\multiput(57.20,14.91)(-0.10,-0.18){6}{\line(0,-1){0.18}}
\multiput(56.59,13.85)(-0.11,-0.17){6}{\line(0,-1){0.17}}
\multiput(55.93,12.81)(-0.12,-0.17){6}{\line(0,-1){0.17}}
\multiput(55.24,11.80)(-0.11,-0.14){7}{\line(0,-1){0.14}}
\multiput(54.50,10.82)(-0.11,-0.14){7}{\line(0,-1){0.14}}
\multiput(53.73,9.87)(-0.12,-0.13){7}{\line(0,-1){0.13}}
\multiput(52.92,8.95)(-0.11,-0.11){8}{\line(0,-1){0.11}}
\multiput(52.07,8.06)(-0.11,-0.11){8}{\line(-1,0){0.11}}
\multiput(51.19,7.21)(-0.13,-0.12){7}{\line(-1,0){0.13}}
\multiput(50.27,6.39)(-0.14,-0.11){7}{\line(-1,0){0.14}}
\multiput(49.32,5.61)(-0.14,-0.11){7}{\line(-1,0){0.14}}
\multiput(48.35,4.87)(-0.17,-0.12){6}{\line(-1,0){0.17}}
\multiput(47.34,4.17)(-0.17,-0.11){6}{\line(-1,0){0.17}}
\multiput(46.30,3.51)(-0.18,-0.10){6}{\line(-1,0){0.18}}
\multiput(45.25,2.89)(-0.22,-0.12){5}{\line(-1,0){0.22}}
\multiput(44.16,2.31)(-0.22,-0.11){5}{\line(-1,0){0.22}}
\multiput(43.06,1.78)(-0.23,-0.10){5}{\line(-1,0){0.23}}
\multiput(41.93,1.29)(-0.29,-0.11){4}{\line(-1,0){0.29}}
\multiput(40.79,0.85)(-0.29,-0.10){4}{\line(-1,0){0.29}}
\multiput(39.63,0.45)(-0.39,-0.12){3}{\line(-1,0){0.39}}
\multiput(38.45,0.10)(-0.40,-0.10){3}{\line(-1,0){0.40}}
\multiput(37.26,-0.21)(-0.40,-0.09){3}{\line(-1,0){0.40}}
\multiput(36.06,-0.46)(-0.60,-0.10){2}{\line(-1,0){0.60}}
\multiput(34.85,-0.67)(-0.61,-0.08){2}{\line(-1,0){0.61}}
\put(33.64,-0.83){\line(-1,0){1.22}}
\put(32.41,-0.94){\line(-1,0){1.23}}
\put(31.19,-1.00){\line(-1,0){1.23}}
\put(29.96,-1.01){\line(-1,0){1.23}}
\put(28.74,-0.98){\line(-1,0){1.22}}
\multiput(27.51,-0.89)(-0.61,0.07){2}{\line(-1,0){0.61}}
\multiput(26.29,-0.76)(-0.61,0.09){2}{\line(-1,0){0.61}}
\multiput(25.08,-0.58)(-0.60,0.12){2}{\line(-1,0){0.60}}
\multiput(23.87,-0.35)(-0.40,0.09){3}{\line(-1,0){0.40}}
\multiput(22.68,-0.07)(-0.39,0.11){3}{\line(-1,0){0.39}}
\multiput(21.49,0.26)(-0.29,0.09){4}{\line(-1,0){0.29}}
\multiput(20.33,0.63)(-0.29,0.10){4}{\line(-1,0){0.29}}
\multiput(19.17,1.05)(-0.28,0.12){4}{\line(-1,0){0.28}}
\multiput(18.04,1.51)(-0.22,0.10){5}{\line(-1,0){0.22}}
\multiput(16.92,2.02)(-0.22,0.11){5}{\line(-1,0){0.22}}
\multiput(15.83,2.58)(-0.21,0.12){5}{\line(-1,0){0.21}}
\multiput(14.75,3.17)(-0.17,0.11){6}{\line(-1,0){0.17}}
\multiput(13.71,3.81)(-0.17,0.11){6}{\line(-1,0){0.17}}
\multiput(12.68,4.49)(-0.14,0.10){7}{\line(-1,0){0.14}}
\multiput(11.69,5.21)(-0.14,0.11){7}{\line(-1,0){0.14}}
\multiput(10.73,5.97)(-0.13,0.11){7}{\line(-1,0){0.13}}
\multiput(9.80,6.77)(-0.13,0.12){7}{\line(-1,0){0.13}}
\multiput(8.90,7.60)(-0.11,0.11){8}{\line(0,1){0.11}}
\multiput(8.03,8.47)(-0.12,0.13){7}{\line(0,1){0.13}}
\multiput(7.20,9.38)(-0.11,0.13){7}{\line(0,1){0.13}}
\multiput(6.40,10.31)(-0.11,0.14){7}{\line(0,1){0.14}}
\multiput(5.65,11.28)(-0.12,0.17){6}{\line(0,1){0.17}}
\multiput(4.93,12.27)(-0.11,0.17){6}{\line(0,1){0.17}}
\multiput(4.25,13.30)(-0.11,0.17){6}{\line(0,1){0.17}}
\multiput(3.62,14.35)(-0.12,0.21){5}{\line(0,1){0.21}}
\multiput(3.03,15.42)(-0.11,0.22){5}{\line(0,1){0.22}}
\multiput(2.48,16.52)(-0.10,0.22){5}{\line(0,1){0.22}}
\multiput(1.97,17.64)(-0.12,0.28){4}{\line(0,1){0.28}}
\multiput(1.51,18.77)(-0.10,0.29){4}{\line(0,1){0.29}}
\multiput(1.10,19.93)(-0.09,0.29){4}{\line(0,1){0.29}}
\multiput(0.73,21.10)(-0.11,0.39){3}{\line(0,1){0.39}}
\multiput(0.41,22.28)(-0.09,0.40){3}{\line(0,1){0.40}}
\multiput(0.13,23.48)(-0.11,0.60){2}{\line(0,1){0.60}}
\multiput(-0.10,24.68)(-0.09,0.61){2}{\line(0,1){0.61}}
\multiput(-0.27,25.90)(-0.06,0.61){2}{\line(0,1){0.61}}
\put(-0.40,27.12){\line(0,1){1.22}}
\put(-0.48,28.34){\line(0,1){1.23}}
\put(-0.51,29.57){\line(0,1){1.23}}
\put(-0.50,30.80){\line(0,1){1.23}}
\put(-0.43,32.02){\line(0,1){1.22}}
\multiput(-0.32,33.24)(0.08,0.61){2}{\line(0,1){0.61}}
\multiput(-0.15,34.46)(0.11,0.60){2}{\line(0,1){0.60}}
\multiput(0.06,35.67)(0.09,0.40){3}{\line(0,1){0.40}}
\multiput(0.32,36.87)(0.10,0.40){3}{\line(0,1){0.40}}
\multiput(0.63,38.05)(0.12,0.39){3}{\line(0,1){0.39}}
\multiput(0.98,39.23)(0.10,0.29){4}{\line(0,1){0.29}}
\multiput(1.38,40.39)(0.11,0.29){4}{\line(0,1){0.29}}
\multiput(1.83,41.53)(0.10,0.22){5}{\line(0,1){0.22}}
\multiput(2.32,42.65)(0.11,0.22){5}{\line(0,1){0.22}}
\multiput(2.86,43.76)(0.12,0.22){5}{\line(0,1){0.22}}
\multiput(3.44,44.84)(0.10,0.18){6}{\line(0,1){0.18}}
\multiput(4.06,45.90)(0.11,0.17){6}{\line(0,1){0.17}}
\multiput(4.73,46.93)(0.12,0.17){6}{\line(0,1){0.17}}
\multiput(5.43,47.93)(0.11,0.14){7}{\line(0,1){0.14}}
\multiput(6.18,48.91)(0.11,0.13){7}{\line(0,1){0.13}}
\multiput(6.96,49.85)(0.12,0.13){7}{\line(0,1){0.13}}
\multiput(7.78,50.76)(0.11,0.11){8}{\line(0,1){0.11}}
\multiput(8.64,51.64)(0.11,0.11){8}{\line(1,0){0.11}}
\multiput(9.53,52.49)(0.13,0.12){7}{\line(1,0){0.13}}
\multiput(10.45,53.30)(0.14,0.11){7}{\line(1,0){0.14}}
\multiput(11.40,54.07)(0.14,0.10){7}{\line(1,0){0.14}}
\multiput(12.39,54.80)(0.17,0.12){6}{\line(1,0){0.17}}
\multiput(13.40,55.49)(0.17,0.11){6}{\line(1,0){0.17}}
\multiput(14.44,56.14)(0.18,0.10){6}{\line(1,0){0.18}}
\multiput(15.51,56.75)(0.22,0.11){5}{\line(1,0){0.22}}
\multiput(16.60,57.32)(0.22,0.10){5}{\line(1,0){0.22}}
\multiput(17.71,57.84)(0.28,0.12){4}{\line(1,0){0.28}}
\multiput(18.84,58.32)(0.29,0.11){4}{\line(1,0){0.29}}
\multiput(19.98,58.75)(0.29,0.10){4}{\line(1,0){0.29}}
\multiput(21.15,59.14)(0.39,0.11){3}{\line(1,0){0.39}}
\multiput(22.33,59.48)(0.40,0.10){3}{\line(1,0){0.40}}
\multiput(23.52,59.77)(0.40,0.08){3}{\line(1,0){0.40}}
\multiput(24.72,60.01)(0.61,0.10){2}{\line(1,0){0.61}}
\multiput(25.93,60.21)(0.61,0.07){2}{\line(1,0){0.61}}
\put(27.15,60.36){\line(1,0){1.22}}
\put(28.37,60.46){\line(1,0){1.88}}
%\end
\put(30.25,30.25){\line(0,1){30.50}}
%\dottedline(1.75,235.75)(2,235.25)
\multiput(1.68,235.68)(.125,-.25){3}{{\rule{.4pt}{.4pt}}}
\put(0.00,0.00){}
%\dottedline(2,235.25)(63.5,234.75)
\multiput(1.93,235.18)(.99194,-.00806){63}{{\rule{.4pt}{.4pt}}}
\put(0.00,0.00){}
%\dottedline(9.5,255.5)(55.75,214.25)
\multiput(9.43,255.43)(.72266,-.64453){65}{{\rule{.4pt}{.4pt}}}
\put(0.00,0.00){}
\put(30.25,68.50){\makebox(0,0)[cc]{$a$}}
%\emline(0.00,30.00)(61.00,30.00)
\put(0.00,30.00){\line(1,0){61.00}}
%\end
%\circle(109.92,29.75){61.53}
\put(109.92,60.52){\line(1,0){1.23}}
\put(111.15,60.49){\line(1,0){1.22}}
\multiput(112.37,60.42)(0.61,-0.06){2}{\line(1,0){0.61}}
\multiput(113.59,60.30)(0.61,-0.09){2}{\line(1,0){0.61}}
\multiput(114.81,60.12)(0.60,-0.11){2}{\line(1,0){0.60}}
\multiput(116.01,59.91)(0.40,-0.09){3}{\line(1,0){0.40}}
\multiput(117.21,59.64)(0.40,-0.10){3}{\line(1,0){0.40}}
\multiput(118.40,59.32)(0.29,-0.09){4}{\line(1,0){0.29}}
\multiput(119.57,58.96)(0.29,-0.10){4}{\line(1,0){0.29}}
\multiput(120.73,58.55)(0.28,-0.11){4}{\line(1,0){0.28}}
\multiput(121.87,58.10)(0.22,-0.10){5}{\line(1,0){0.22}}
\multiput(122.99,57.60)(0.22,-0.11){5}{\line(1,0){0.22}}
\multiput(124.09,57.06)(0.22,-0.12){5}{\line(1,0){0.22}}
\multiput(125.17,56.47)(0.18,-0.10){6}{\line(1,0){0.18}}
\multiput(126.22,55.84)(0.17,-0.11){6}{\line(1,0){0.17}}
\multiput(127.25,55.17)(0.17,-0.12){6}{\line(1,0){0.17}}
\multiput(128.25,54.46)(0.14,-0.11){7}{\line(1,0){0.14}}
\multiput(129.22,53.71)(0.13,-0.11){7}{\line(1,0){0.13}}
\multiput(130.16,52.92)(0.13,-0.12){7}{\line(1,0){0.13}}
\multiput(131.07,52.10)(0.11,-0.11){8}{\line(1,0){0.11}}
\multiput(131.94,51.23)(0.12,-0.13){7}{\line(0,-1){0.13}}
\multiput(132.78,50.34)(0.11,-0.13){7}{\line(0,-1){0.13}}
\multiput(133.58,49.41)(0.11,-0.14){7}{\line(0,-1){0.14}}
\multiput(134.35,48.45)(0.10,-0.14){7}{\line(0,-1){0.14}}
\multiput(135.07,47.46)(0.11,-0.17){6}{\line(0,-1){0.17}}
\multiput(135.76,46.45)(0.11,-0.17){6}{\line(0,-1){0.17}}
\multiput(136.41,45.40)(0.10,-0.18){6}{\line(0,-1){0.18}}
\multiput(137.01,44.33)(0.11,-0.22){5}{\line(0,-1){0.22}}
\multiput(137.57,43.24)(0.10,-0.22){5}{\line(0,-1){0.22}}
\multiput(138.08,42.13)(0.12,-0.28){4}{\line(0,-1){0.28}}
\multiput(138.56,41.00)(0.11,-0.29){4}{\line(0,-1){0.29}}
\multiput(138.98,39.85)(0.09,-0.29){4}{\line(0,-1){0.29}}
\multiput(139.36,38.68)(0.11,-0.39){3}{\line(0,-1){0.39}}
\multiput(139.69,37.50)(0.10,-0.40){3}{\line(0,-1){0.40}}
\multiput(139.98,36.30)(0.12,-0.60){2}{\line(0,-1){0.60}}
\multiput(140.22,35.10)(0.09,-0.61){2}{\line(0,-1){0.61}}
\multiput(140.41,33.89)(0.07,-0.61){2}{\line(0,-1){0.61}}
\put(140.55,32.67){\line(0,-1){1.22}}
\put(140.64,31.45){\line(0,-1){1.23}}
\put(140.68,30.22){\line(0,-1){1.23}}
\put(140.68,28.99){\line(0,-1){1.23}}
\put(140.62,27.77){\line(0,-1){1.22}}
\multiput(140.52,26.54)(-0.08,-0.61){2}{\line(0,-1){0.61}}
\multiput(140.37,25.33)(-0.10,-0.61){2}{\line(0,-1){0.61}}
\multiput(140.16,24.12)(-0.08,-0.40){3}{\line(0,-1){0.40}}
\multiput(139.92,22.91)(-0.10,-0.40){3}{\line(0,-1){0.40}}
\multiput(139.62,21.72)(-0.11,-0.39){3}{\line(0,-1){0.39}}
\multiput(139.28,20.55)(-0.10,-0.29){4}{\line(0,-1){0.29}}
\multiput(138.89,19.38)(-0.11,-0.29){4}{\line(0,-1){0.29}}
\multiput(138.45,18.24)(-0.10,-0.23){5}{\line(0,-1){0.23}}
\multiput(137.97,17.11)(-0.11,-0.22){5}{\line(0,-1){0.22}}
\multiput(137.44,16.00)(-0.11,-0.22){5}{\line(0,-1){0.22}}
\multiput(136.87,14.91)(-0.10,-0.18){6}{\line(0,-1){0.18}}
\multiput(136.26,13.85)(-0.11,-0.17){6}{\line(0,-1){0.17}}
\multiput(135.60,12.81)(-0.12,-0.17){6}{\line(0,-1){0.17}}
\multiput(134.91,11.80)(-0.11,-0.14){7}{\line(0,-1){0.14}}
\multiput(134.17,10.82)(-0.11,-0.14){7}{\line(0,-1){0.14}}
\multiput(133.40,9.87)(-0.12,-0.13){7}{\line(0,-1){0.13}}
\multiput(132.59,8.95)(-0.11,-0.11){8}{\line(0,-1){0.11}}
\multiput(131.74,8.06)(-0.11,-0.11){8}{\line(-1,0){0.11}}
\multiput(130.86,7.21)(-0.13,-0.12){7}{\line(-1,0){0.13}}
\multiput(129.94,6.39)(-0.14,-0.11){7}{\line(-1,0){0.14}}
\multiput(128.99,5.61)(-0.14,-0.11){7}{\line(-1,0){0.14}}
\multiput(128.02,4.87)(-0.17,-0.12){6}{\line(-1,0){0.17}}
\multiput(127.01,4.17)(-0.17,-0.11){6}{\line(-1,0){0.17}}
\multiput(125.97,3.51)(-0.18,-0.10){6}{\line(-1,0){0.18}}
\multiput(124.92,2.89)(-0.22,-0.12){5}{\line(-1,0){0.22}}
\multiput(123.83,2.31)(-0.22,-0.11){5}{\line(-1,0){0.22}}
\multiput(122.73,1.78)(-0.23,-0.10){5}{\line(-1,0){0.23}}
\multiput(121.60,1.29)(-0.29,-0.11){4}{\line(-1,0){0.29}}
\multiput(120.46,0.85)(-0.29,-0.10){4}{\line(-1,0){0.29}}
\multiput(119.30,0.45)(-0.39,-0.12){3}{\line(-1,0){0.39}}
\multiput(118.12,0.10)(-0.40,-0.10){3}{\line(-1,0){0.40}}
\multiput(116.93,-0.21)(-0.40,-0.09){3}{\line(-1,0){0.40}}
\multiput(115.73,-0.46)(-0.60,-0.10){2}{\line(-1,0){0.60}}
\multiput(114.52,-0.67)(-0.61,-0.08){2}{\line(-1,0){0.61}}
\put(113.31,-0.83){\line(-1,0){1.22}}
\put(112.08,-0.94){\line(-1,0){1.23}}
\put(110.86,-1.00){\line(-1,0){1.23}}
\put(109.63,-1.01){\line(-1,0){1.23}}
\put(108.41,-0.98){\line(-1,0){1.22}}
\multiput(107.18,-0.89)(-0.61,0.07){2}{\line(-1,0){0.61}}
\multiput(105.96,-0.76)(-0.61,0.09){2}{\line(-1,0){0.61}}
\multiput(104.75,-0.58)(-0.60,0.12){2}{\line(-1,0){0.60}}
\multiput(103.54,-0.35)(-0.40,0.09){3}{\line(-1,0){0.40}}
\multiput(102.35,-0.07)(-0.39,0.11){3}{\line(-1,0){0.39}}
\multiput(101.16,0.26)(-0.29,0.09){4}{\line(-1,0){0.29}}
\multiput(100.00,0.63)(-0.29,0.10){4}{\line(-1,0){0.29}}
\multiput(98.84,1.05)(-0.28,0.12){4}{\line(-1,0){0.28}}
\multiput(97.71,1.51)(-0.22,0.10){5}{\line(-1,0){0.22}}
\multiput(96.59,2.02)(-0.22,0.11){5}{\line(-1,0){0.22}}
\multiput(95.50,2.58)(-0.21,0.12){5}{\line(-1,0){0.21}}
\multiput(94.42,3.17)(-0.17,0.11){6}{\line(-1,0){0.17}}
\multiput(93.38,3.81)(-0.17,0.11){6}{\line(-1,0){0.17}}
\multiput(92.35,4.49)(-0.14,0.10){7}{\line(-1,0){0.14}}
\multiput(91.36,5.21)(-0.14,0.11){7}{\line(-1,0){0.14}}
\multiput(90.40,5.97)(-0.13,0.11){7}{\line(-1,0){0.13}}
\multiput(89.47,6.77)(-0.13,0.12){7}{\line(-1,0){0.13}}
\multiput(88.57,7.60)(-0.11,0.11){8}{\line(0,1){0.11}}
\multiput(87.70,8.47)(-0.12,0.13){7}{\line(0,1){0.13}}
\multiput(86.87,9.38)(-0.11,0.13){7}{\line(0,1){0.13}}
\multiput(86.07,10.31)(-0.11,0.14){7}{\line(0,1){0.14}}
\multiput(85.32,11.28)(-0.12,0.17){6}{\line(0,1){0.17}}
\multiput(84.60,12.27)(-0.11,0.17){6}{\line(0,1){0.17}}
\multiput(83.92,13.30)(-0.11,0.17){6}{\line(0,1){0.17}}
\multiput(83.29,14.35)(-0.12,0.21){5}{\line(0,1){0.21}}
\multiput(82.70,15.42)(-0.11,0.22){5}{\line(0,1){0.22}}
\multiput(82.15,16.52)(-0.10,0.22){5}{\line(0,1){0.22}}
\multiput(81.64,17.64)(-0.12,0.28){4}{\line(0,1){0.28}}
\multiput(81.18,18.77)(-0.10,0.29){4}{\line(0,1){0.29}}
\multiput(80.77,19.93)(-0.09,0.29){4}{\line(0,1){0.29}}
\multiput(80.40,21.10)(-0.11,0.39){3}{\line(0,1){0.39}}
\multiput(80.08,22.28)(-0.09,0.40){3}{\line(0,1){0.40}}
\multiput(79.80,23.48)(-0.11,0.60){2}{\line(0,1){0.60}}
\multiput(79.57,24.68)(-0.09,0.61){2}{\line(0,1){0.61}}
\multiput(79.40,25.90)(-0.06,0.61){2}{\line(0,1){0.61}}
\put(79.27,27.12){\line(0,1){1.22}}
\put(79.19,28.34){\line(0,1){1.23}}
\put(79.16,29.57){\line(0,1){1.23}}
\put(79.17,30.80){\line(0,1){1.23}}
\put(79.24,32.02){\line(0,1){1.22}}
\multiput(79.35,33.24)(0.08,0.61){2}{\line(0,1){0.61}}
\multiput(79.52,34.46)(0.11,0.60){2}{\line(0,1){0.60}}
\multiput(79.73,35.67)(0.09,0.40){3}{\line(0,1){0.40}}
\multiput(79.99,36.87)(0.10,0.40){3}{\line(0,1){0.40}}
\multiput(80.30,38.05)(0.12,0.39){3}{\line(0,1){0.39}}
\multiput(80.65,39.23)(0.10,0.29){4}{\line(0,1){0.29}}
\multiput(81.05,40.39)(0.11,0.29){4}{\line(0,1){0.29}}
\multiput(81.50,41.53)(0.10,0.22){5}{\line(0,1){0.22}}
\multiput(81.99,42.65)(0.11,0.22){5}{\line(0,1){0.22}}
\multiput(82.53,43.76)(0.12,0.22){5}{\line(0,1){0.22}}
\multiput(83.11,44.84)(0.10,0.18){6}{\line(0,1){0.18}}
\multiput(83.73,45.90)(0.11,0.17){6}{\line(0,1){0.17}}
\multiput(84.40,46.93)(0.12,0.17){6}{\line(0,1){0.17}}
\multiput(85.10,47.93)(0.11,0.14){7}{\line(0,1){0.14}}
\multiput(85.85,48.91)(0.11,0.13){7}{\line(0,1){0.13}}
\multiput(86.63,49.85)(0.12,0.13){7}{\line(0,1){0.13}}
\multiput(87.45,50.76)(0.11,0.11){8}{\line(0,1){0.11}}
\multiput(88.31,51.64)(0.11,0.11){8}{\line(1,0){0.11}}
\multiput(89.20,52.49)(0.13,0.12){7}{\line(1,0){0.13}}
\multiput(90.12,53.30)(0.14,0.11){7}{\line(1,0){0.14}}
\multiput(91.07,54.07)(0.14,0.10){7}{\line(1,0){0.14}}
\multiput(92.06,54.80)(0.17,0.12){6}{\line(1,0){0.17}}
\multiput(93.07,55.49)(0.17,0.11){6}{\line(1,0){0.17}}
\multiput(94.11,56.14)(0.18,0.10){6}{\line(1,0){0.18}}
\multiput(95.18,56.75)(0.22,0.11){5}{\line(1,0){0.22}}
\multiput(96.27,57.32)(0.22,0.10){5}{\line(1,0){0.22}}
\multiput(97.38,57.84)(0.28,0.12){4}{\line(1,0){0.28}}
\multiput(98.51,58.32)(0.29,0.11){4}{\line(1,0){0.29}}
\multiput(99.65,58.75)(0.29,0.10){4}{\line(1,0){0.29}}
\multiput(100.82,59.14)(0.39,0.11){3}{\line(1,0){0.39}}
\multiput(102.00,59.48)(0.40,0.10){3}{\line(1,0){0.40}}
\multiput(103.19,59.77)(0.40,0.08){3}{\line(1,0){0.40}}
\multiput(104.39,60.01)(0.61,0.10){2}{\line(1,0){0.61}}
\multiput(105.60,60.21)(0.61,0.07){2}{\line(1,0){0.61}}
\put(106.82,60.36){\line(1,0){1.22}}
\put(108.04,60.46){\line(1,0){1.88}}
%\end
%\circle(189.58,29.75){61.53}
\put(189.58,60.52){\line(1,0){1.23}}
\put(190.81,60.49){\line(1,0){1.22}}
\multiput(192.03,60.42)(0.61,-0.06){2}{\line(1,0){0.61}}
\multiput(193.25,60.30)(0.61,-0.09){2}{\line(1,0){0.61}}
\multiput(194.47,60.12)(0.60,-0.11){2}{\line(1,0){0.60}}
\multiput(195.67,59.91)(0.40,-0.09){3}{\line(1,0){0.40}}
\multiput(196.87,59.64)(0.40,-0.10){3}{\line(1,0){0.40}}
\multiput(198.06,59.32)(0.29,-0.09){4}{\line(1,0){0.29}}
\multiput(199.23,58.96)(0.29,-0.10){4}{\line(1,0){0.29}}
\multiput(200.39,58.55)(0.28,-0.11){4}{\line(1,0){0.28}}
\multiput(201.53,58.10)(0.22,-0.10){5}{\line(1,0){0.22}}
\multiput(202.65,57.60)(0.22,-0.11){5}{\line(1,0){0.22}}
\multiput(203.75,57.06)(0.22,-0.12){5}{\line(1,0){0.22}}
\multiput(204.83,56.47)(0.18,-0.10){6}{\line(1,0){0.18}}
\multiput(205.88,55.84)(0.17,-0.11){6}{\line(1,0){0.17}}
\multiput(206.91,55.17)(0.17,-0.12){6}{\line(1,0){0.17}}
\multiput(207.91,54.46)(0.14,-0.11){7}{\line(1,0){0.14}}
\multiput(208.88,53.71)(0.13,-0.11){7}{\line(1,0){0.13}}
\multiput(209.82,52.92)(0.13,-0.12){7}{\line(1,0){0.13}}
\multiput(210.73,52.10)(0.11,-0.11){8}{\line(1,0){0.11}}
\multiput(211.60,51.23)(0.12,-0.13){7}{\line(0,-1){0.13}}
\multiput(212.44,50.34)(0.11,-0.13){7}{\line(0,-1){0.13}}
\multiput(213.24,49.41)(0.11,-0.14){7}{\line(0,-1){0.14}}
\multiput(214.01,48.45)(0.10,-0.14){7}{\line(0,-1){0.14}}
\multiput(214.73,47.46)(0.11,-0.17){6}{\line(0,-1){0.17}}
\multiput(215.42,46.45)(0.11,-0.17){6}{\line(0,-1){0.17}}
\multiput(216.07,45.40)(0.10,-0.18){6}{\line(0,-1){0.18}}
\multiput(216.67,44.33)(0.11,-0.22){5}{\line(0,-1){0.22}}
\multiput(217.23,43.24)(0.10,-0.22){5}{\line(0,-1){0.22}}
\multiput(217.74,42.13)(0.12,-0.28){4}{\line(0,-1){0.28}}
\multiput(218.22,41.00)(0.11,-0.29){4}{\line(0,-1){0.29}}
\multiput(218.64,39.85)(0.09,-0.29){4}{\line(0,-1){0.29}}
\multiput(219.02,38.68)(0.11,-0.39){3}{\line(0,-1){0.39}}
\multiput(219.35,37.50)(0.10,-0.40){3}{\line(0,-1){0.40}}
\multiput(219.64,36.30)(0.12,-0.60){2}{\line(0,-1){0.60}}
\multiput(219.88,35.10)(0.09,-0.61){2}{\line(0,-1){0.61}}
\multiput(220.07,33.89)(0.07,-0.61){2}{\line(0,-1){0.61}}
\put(220.21,32.67){\line(0,-1){1.22}}
\put(220.30,31.45){\line(0,-1){1.23}}
\put(220.34,30.22){\line(0,-1){1.23}}
\put(220.34,28.99){\line(0,-1){1.23}}
\put(220.28,27.77){\line(0,-1){1.22}}
\multiput(220.18,26.54)(-0.08,-0.61){2}{\line(0,-1){0.61}}
\multiput(220.03,25.33)(-0.10,-0.61){2}{\line(0,-1){0.61}}
\multiput(219.82,24.12)(-0.08,-0.40){3}{\line(0,-1){0.40}}
\multiput(219.58,22.91)(-0.10,-0.40){3}{\line(0,-1){0.40}}
\multiput(219.28,21.72)(-0.11,-0.39){3}{\line(0,-1){0.39}}
\multiput(218.94,20.55)(-0.10,-0.29){4}{\line(0,-1){0.29}}
\multiput(218.55,19.38)(-0.11,-0.29){4}{\line(0,-1){0.29}}
\multiput(218.11,18.24)(-0.10,-0.23){5}{\line(0,-1){0.23}}
\multiput(217.63,17.11)(-0.11,-0.22){5}{\line(0,-1){0.22}}
\multiput(217.10,16.00)(-0.11,-0.22){5}{\line(0,-1){0.22}}
\multiput(216.53,14.91)(-0.10,-0.18){6}{\line(0,-1){0.18}}
\multiput(215.92,13.85)(-0.11,-0.17){6}{\line(0,-1){0.17}}
\multiput(215.26,12.81)(-0.12,-0.17){6}{\line(0,-1){0.17}}
\multiput(214.57,11.80)(-0.11,-0.14){7}{\line(0,-1){0.14}}
\multiput(213.83,10.82)(-0.11,-0.14){7}{\line(0,-1){0.14}}
\multiput(213.06,9.87)(-0.12,-0.13){7}{\line(0,-1){0.13}}
\multiput(212.25,8.95)(-0.11,-0.11){8}{\line(0,-1){0.11}}
\multiput(211.40,8.06)(-0.11,-0.11){8}{\line(-1,0){0.11}}
\multiput(210.52,7.21)(-0.13,-0.12){7}{\line(-1,0){0.13}}
\multiput(209.60,6.39)(-0.14,-0.11){7}{\line(-1,0){0.14}}
\multiput(208.65,5.61)(-0.14,-0.11){7}{\line(-1,0){0.14}}
\multiput(207.68,4.87)(-0.17,-0.12){6}{\line(-1,0){0.17}}
\multiput(206.67,4.17)(-0.17,-0.11){6}{\line(-1,0){0.17}}
\multiput(205.63,3.51)(-0.18,-0.10){6}{\line(-1,0){0.18}}
\multiput(204.58,2.89)(-0.22,-0.12){5}{\line(-1,0){0.22}}
\multiput(203.49,2.31)(-0.22,-0.11){5}{\line(-1,0){0.22}}
\multiput(202.39,1.78)(-0.23,-0.10){5}{\line(-1,0){0.23}}
\multiput(201.26,1.29)(-0.29,-0.11){4}{\line(-1,0){0.29}}
\multiput(200.12,0.85)(-0.29,-0.10){4}{\line(-1,0){0.29}}
\multiput(198.96,0.45)(-0.39,-0.12){3}{\line(-1,0){0.39}}
\multiput(197.78,0.10)(-0.40,-0.10){3}{\line(-1,0){0.40}}
\multiput(196.59,-0.21)(-0.40,-0.09){3}{\line(-1,0){0.40}}
\multiput(195.39,-0.46)(-0.60,-0.10){2}{\line(-1,0){0.60}}
\multiput(194.18,-0.67)(-0.61,-0.08){2}{\line(-1,0){0.61}}
\put(192.97,-0.83){\line(-1,0){1.22}}
\put(191.74,-0.94){\line(-1,0){1.23}}
\put(190.52,-1.00){\line(-1,0){1.23}}
\put(189.29,-1.01){\line(-1,0){1.23}}
\put(188.07,-0.98){\line(-1,0){1.22}}
\multiput(186.84,-0.89)(-0.61,0.07){2}{\line(-1,0){0.61}}
\multiput(185.62,-0.76)(-0.61,0.09){2}{\line(-1,0){0.61}}
\multiput(184.41,-0.58)(-0.60,0.12){2}{\line(-1,0){0.60}}
\multiput(183.20,-0.35)(-0.40,0.09){3}{\line(-1,0){0.40}}
\multiput(182.01,-0.07)(-0.39,0.11){3}{\line(-1,0){0.39}}
\multiput(180.82,0.26)(-0.29,0.09){4}{\line(-1,0){0.29}}
\multiput(179.66,0.63)(-0.29,0.10){4}{\line(-1,0){0.29}}
\multiput(178.50,1.05)(-0.28,0.12){4}{\line(-1,0){0.28}}
\multiput(177.37,1.51)(-0.22,0.10){5}{\line(-1,0){0.22}}
\multiput(176.25,2.02)(-0.22,0.11){5}{\line(-1,0){0.22}}
\multiput(175.16,2.58)(-0.21,0.12){5}{\line(-1,0){0.21}}
\multiput(174.08,3.17)(-0.17,0.11){6}{\line(-1,0){0.17}}
\multiput(173.04,3.81)(-0.17,0.11){6}{\line(-1,0){0.17}}
\multiput(172.01,4.49)(-0.14,0.10){7}{\line(-1,0){0.14}}
\multiput(171.02,5.21)(-0.14,0.11){7}{\line(-1,0){0.14}}
\multiput(170.06,5.97)(-0.13,0.11){7}{\line(-1,0){0.13}}
\multiput(169.13,6.77)(-0.13,0.12){7}{\line(-1,0){0.13}}
\multiput(168.23,7.60)(-0.11,0.11){8}{\line(0,1){0.11}}
\multiput(167.36,8.47)(-0.12,0.13){7}{\line(0,1){0.13}}
\multiput(166.53,9.38)(-0.11,0.13){7}{\line(0,1){0.13}}
\multiput(165.73,10.31)(-0.11,0.14){7}{\line(0,1){0.14}}
\multiput(164.98,11.28)(-0.12,0.17){6}{\line(0,1){0.17}}
\multiput(164.26,12.27)(-0.11,0.17){6}{\line(0,1){0.17}}
\multiput(163.58,13.30)(-0.11,0.17){6}{\line(0,1){0.17}}
\multiput(162.95,14.35)(-0.12,0.21){5}{\line(0,1){0.21}}
\multiput(162.36,15.42)(-0.11,0.22){5}{\line(0,1){0.22}}
\multiput(161.81,16.52)(-0.10,0.22){5}{\line(0,1){0.22}}
\multiput(161.30,17.64)(-0.12,0.28){4}{\line(0,1){0.28}}
\multiput(160.84,18.77)(-0.10,0.29){4}{\line(0,1){0.29}}
\multiput(160.43,19.93)(-0.09,0.29){4}{\line(0,1){0.29}}
\multiput(160.06,21.10)(-0.11,0.39){3}{\line(0,1){0.39}}
\multiput(159.74,22.28)(-0.09,0.40){3}{\line(0,1){0.40}}
\multiput(159.46,23.48)(-0.11,0.60){2}{\line(0,1){0.60}}
\multiput(159.23,24.68)(-0.09,0.61){2}{\line(0,1){0.61}}
\multiput(159.06,25.90)(-0.06,0.61){2}{\line(0,1){0.61}}
\put(158.93,27.12){\line(0,1){1.22}}
\put(158.85,28.34){\line(0,1){1.23}}
\put(158.82,29.57){\line(0,1){1.23}}
\put(158.83,30.80){\line(0,1){1.23}}
\put(158.90,32.02){\line(0,1){1.22}}
\multiput(159.01,33.24)(0.08,0.61){2}{\line(0,1){0.61}}
\multiput(159.18,34.46)(0.11,0.60){2}{\line(0,1){0.60}}
\multiput(159.39,35.67)(0.09,0.40){3}{\line(0,1){0.40}}
\multiput(159.65,36.87)(0.10,0.40){3}{\line(0,1){0.40}}
\multiput(159.96,38.05)(0.12,0.39){3}{\line(0,1){0.39}}
\multiput(160.31,39.23)(0.10,0.29){4}{\line(0,1){0.29}}
\multiput(160.71,40.39)(0.11,0.29){4}{\line(0,1){0.29}}
\multiput(161.16,41.53)(0.10,0.22){5}{\line(0,1){0.22}}
\multiput(161.65,42.65)(0.11,0.22){5}{\line(0,1){0.22}}
\multiput(162.19,43.76)(0.12,0.22){5}{\line(0,1){0.22}}
\multiput(162.77,44.84)(0.10,0.18){6}{\line(0,1){0.18}}
\multiput(163.39,45.90)(0.11,0.17){6}{\line(0,1){0.17}}
\multiput(164.06,46.93)(0.12,0.17){6}{\line(0,1){0.17}}
\multiput(164.76,47.93)(0.11,0.14){7}{\line(0,1){0.14}}
\multiput(165.51,48.91)(0.11,0.13){7}{\line(0,1){0.13}}
\multiput(166.29,49.85)(0.12,0.13){7}{\line(0,1){0.13}}
\multiput(167.11,50.76)(0.11,0.11){8}{\line(0,1){0.11}}
\multiput(167.97,51.64)(0.11,0.11){8}{\line(1,0){0.11}}
\multiput(168.86,52.49)(0.13,0.12){7}{\line(1,0){0.13}}
\multiput(169.78,53.30)(0.14,0.11){7}{\line(1,0){0.14}}
\multiput(170.73,54.07)(0.14,0.10){7}{\line(1,0){0.14}}
\multiput(171.72,54.80)(0.17,0.12){6}{\line(1,0){0.17}}
\multiput(172.73,55.49)(0.17,0.11){6}{\line(1,0){0.17}}
\multiput(173.77,56.14)(0.18,0.10){6}{\line(1,0){0.18}}
\multiput(174.84,56.75)(0.22,0.11){5}{\line(1,0){0.22}}
\multiput(175.93,57.32)(0.22,0.10){5}{\line(1,0){0.22}}
\multiput(177.04,57.84)(0.28,0.12){4}{\line(1,0){0.28}}
\multiput(178.17,58.32)(0.29,0.11){4}{\line(1,0){0.29}}
\multiput(179.31,58.75)(0.29,0.10){4}{\line(1,0){0.29}}
\multiput(180.48,59.14)(0.39,0.11){3}{\line(1,0){0.39}}
\multiput(181.66,59.48)(0.40,0.10){3}{\line(1,0){0.40}}
\multiput(182.85,59.77)(0.40,0.08){3}{\line(1,0){0.40}}
\multiput(184.05,60.01)(0.61,0.10){2}{\line(1,0){0.61}}
\multiput(185.26,60.21)(0.61,0.07){2}{\line(1,0){0.61}}
\put(186.48,60.36){\line(1,0){1.22}}
\put(187.70,60.46){\line(1,0){1.88}}
%\end
\put(189.58,30.25){\line(0,1){30.50}}
\put(189.58,68.50){\makebox(0,0)[cc]{$a$}}
\put(133.61,62.94){\makebox(0,0)[cc]{$b$}}
\put(213.28,62.94){\makebox(0,0)[cc]{$b$}}
\put(182.83,13.50){\makebox(0,0)[cc]{$-$}}
\put(210.08,40.50){\makebox(0,0)[cc]{$-$}}
\put(165.08,38.00){\makebox(0,0)[cc]{$+$}}
\put(211.58,21.25){\makebox(0,0)[cc]{$+$}}
%\emline(84.33,46.33)(135.67,13.00)
\multiput(84.33,46.33)(0.18,-0.12){278}{\line(1,0){0.18}}
%\end
%\emline(164.00,46.33)(215.33,13.00)
\multiput(164.00,46.33)(0.18,-0.12){278}{\line(1,0){0.18}}
%\end
%\emline(159.33,30.00)(220.33,30.00)
\put(159.33,30.00){\line(1,0){61.00}}
%\end
%\emline(110.00,30.00)(128.33,54.00)
\multiput(110.00,30.00)(0.12,0.16){153}{\line(0,1){0.16}}
%\end
%\emline(189.44,30.00)(207.78,54.00)
\multiput(189.44,30.00)(0.12,0.16){153}{\line(0,1){0.16}}
%\end
\put(18.89,42.78){\makebox(0,0)[cc]{$+$}}
\put(29.44,12.22){\makebox(0,0)[cc]{$-$}}
\put(110.56,46.67){\makebox(0,0)[cc]{$-$}}
\put(99.44,17.22){\makebox(0,0)[cc]{$+$}}
\bezier{44}(189.44,45.00)(195.00,46.11)(198.89,42.22)
\bezier{56}(171.67,30.00)(167.78,36.11)(172.78,40.00)
\bezier{60}(206.11,30.00)(210.00,23.33)(202.78,21.11)
\end{picture}
\end{center}

$$
\begin{array}{lll}
E(a,b) &=& {A_+(a,b)-A_-(a,b)\over 2\pi}= {2A_+(a,b) -2\pi \over 2\pi}=
\\
&&  \qquad \qquad \qquad \qquad  ={2\over \pi}\vert a-b\vert - 1={2\over \pi}\theta - 1
 \end{array}
$$

 }



\subsection{Classical correlations with a single bit exchange}

 \frame{
\frametitle{``Stronger'' correlations}

\begin{itemize}
\item<1->
More anti-coincidences of detector clicks between $0< \theta < \pi /2$;
more coincidences of detector clicks between $\pi /2< \theta < \pi $;
identical for $\theta = 0,\pi /2 ,\pi$.


\item<1->
Clauser-Horne-Shimony-Holt  inequality
$$
\vert
E({ a} ,{ b} )+
E({ a} ,{ b} ' )+
E({ a}' ,{ b} )-
E({a} ',{ b} ')
\vert
\le 2
$$
can be violated algebraically by 4,
and larger
than the quantum Tsirelson bound for quantum violations $2\sqrt{2}$.

\item<1->
Maximal algebraic violation is
not forbidden by relativistic causality, as long as there is parameter independence and mere outcome dependence.
``Why does Nature not violate the Clauser-Horne-Shimony-Holt inequality maximally?''
\end{itemize}
 }


%%%%%%%%%%%%%%%%%%%%%%%%%%%%%%%%%%%%%%%%%%%%%%%%%%%%%%%%%%%%%%%%%%%%%%%%%%%%%%%%%%%%

 \frame{
\frametitle{Classical correlations with a single bit exchange
which breaks the Bell barrier maximally [PRA 66 050302(R) (2005)]}
\begin{itemize}
\item<1->
Consider a single share ${ \lambda }$,
and an additional direction ${ \Delta} (\delta )$, which is obtained by rotating
${ \lambda }$ clockwise around the origin by an angle $\delta$ which is a constant
shift for all experiments.
That is, ${\Delta} (\delta )=\lambda +\delta$.
\item<1->
Alice's observable is given by
$$\alpha  = {\rm sgn}({\vec  a} \cdot {\vec  \lambda } )
={\rm sgn}\left[\cos ({a} - { \lambda } )\right].$$
\item<1->
The bit communicated by Alice is given by
$$
c(\delta) =
{\rm sgn}({\vec  a} \cdot {\vec  \lambda } )
{\rm sgn}\left[{\vec  a} \cdot {\vec  \Delta} (\delta)\right]=
{\rm sgn}\left[\cos ({ a} - { \lambda } )\right]
{\rm sgn}\cos \left[{ a} - { \Delta} (\delta)\right].$$
\item<1->
Bob's observable is  defined by
$$\beta (\delta )=  {\rm sgn}[{\vec  b} \cdot ({\vec  \lambda } +c(\delta){\vec  \Delta} (\delta))].$$
%Motivated by B. F. Toner and D. Bacon, PRL 91, 187904 (2003).
\end{itemize}


 }

 \frame[shrink=2]{
\frametitle{Shift mechanism at work}

\centering
%TexCad Options
%\grade{\off}
%\emlines{\off}
%\beziermacro{\off}
%\reduce{\on}
%\snapping{\off}
%\quality{6.00}
%\graddiff{0.01}
%\snapasp{1}
%\zoom{2.00}
\unitlength 1.00mm
\linethickness{0.5pt}
\begin{picture}(63.00,71.00)
%\circle(30.00,30.00){60.00}
\put(30.00,60.00){\line(1,0){1.61}}
\multiput(31.61,59.96)(0.80,-0.06){2}{\line(1,0){0.80}}
\multiput(33.22,59.83)(0.80,-0.11){2}{\line(1,0){0.80}}
\multiput(34.81,59.61)(0.53,-0.10){3}{\line(1,0){0.53}}
\multiput(36.39,59.31)(0.39,-0.10){4}{\line(1,0){0.39}}
\multiput(37.96,58.93)(0.39,-0.12){4}{\line(1,0){0.39}}
\multiput(39.50,58.46)(0.30,-0.11){5}{\line(1,0){0.30}}
\multiput(41.01,57.91)(0.25,-0.11){6}{\line(1,0){0.25}}
\multiput(42.50,57.27)(0.24,-0.12){6}{\line(1,0){0.24}}
\multiput(43.94,56.56)(0.20,-0.11){7}{\line(1,0){0.20}}
\multiput(45.35,55.78)(0.17,-0.11){8}{\line(1,0){0.17}}
\multiput(46.71,54.92)(0.16,-0.12){8}{\line(1,0){0.16}}
\multiput(48.02,53.98)(0.14,-0.11){9}{\line(1,0){0.14}}
\multiput(49.28,52.98)(0.13,-0.12){9}{\line(1,0){0.13}}
\multiput(50.49,51.91)(0.11,-0.11){10}{\line(1,0){0.11}}
\multiput(51.64,50.78)(0.11,-0.12){10}{\line(0,-1){0.12}}
\multiput(52.72,49.59)(0.11,-0.14){9}{\line(0,-1){0.14}}
\multiput(53.74,48.34)(0.12,-0.16){8}{\line(0,-1){0.16}}
\multiput(54.69,47.04)(0.11,-0.17){8}{\line(0,-1){0.17}}
\multiput(55.57,45.69)(0.12,-0.20){7}{\line(0,-1){0.20}}
\multiput(56.37,44.30)(0.10,-0.21){7}{\line(0,-1){0.21}}
\multiput(57.10,42.86)(0.11,-0.25){6}{\line(0,-1){0.25}}
\multiput(57.75,41.39)(0.11,-0.30){5}{\line(0,-1){0.30}}
\multiput(58.33,39.88)(0.10,-0.31){5}{\line(0,-1){0.31}}
\multiput(58.82,38.35)(0.10,-0.39){4}{\line(0,-1){0.39}}
\multiput(59.22,36.79)(0.11,-0.53){3}{\line(0,-1){0.53}}
\multiput(59.54,35.21)(0.12,-0.80){2}{\line(0,-1){0.80}}
\multiput(59.78,33.62)(0.08,-0.80){2}{\line(0,-1){0.80}}
\put(59.93,32.01){\line(0,-1){1.61}}
\put(60.00,30.40){\line(0,-1){1.61}}
\put(59.98,28.79){\line(0,-1){1.61}}
\multiput(59.87,27.18)(-0.10,-0.80){2}{\line(0,-1){0.80}}
\multiput(59.67,25.59)(-0.09,-0.53){3}{\line(0,-1){0.53}}
\multiput(59.39,24.00)(-0.09,-0.39){4}{\line(0,-1){0.39}}
\multiput(59.03,22.43)(-0.11,-0.39){4}{\line(0,-1){0.39}}
\multiput(58.58,20.88)(-0.11,-0.30){5}{\line(0,-1){0.30}}
\multiput(58.05,19.36)(-0.10,-0.25){6}{\line(0,-1){0.25}}
\multiput(57.44,17.87)(-0.12,-0.24){6}{\line(0,-1){0.24}}
\multiput(56.75,16.42)(-0.11,-0.20){7}{\line(0,-1){0.20}}
\multiput(55.98,15.00)(-0.11,-0.17){8}{\line(0,-1){0.17}}
\multiput(55.14,13.63)(-0.11,-0.17){8}{\line(0,-1){0.17}}
\multiput(54.22,12.30)(-0.11,-0.14){9}{\line(0,-1){0.14}}
\multiput(53.24,11.03)(-0.12,-0.14){9}{\line(0,-1){0.14}}
\multiput(52.19,9.81)(-0.11,-0.12){10}{\line(0,-1){0.12}}
\multiput(51.07,8.64)(-0.12,-0.11){10}{\line(-1,0){0.12}}
\multiput(49.89,7.54)(-0.14,-0.12){9}{\line(-1,0){0.14}}
\multiput(48.66,6.51)(-0.14,-0.11){9}{\line(-1,0){0.14}}
\multiput(47.37,5.54)(-0.17,-0.11){8}{\line(-1,0){0.17}}
\multiput(46.03,4.64)(-0.20,-0.12){7}{\line(-1,0){0.20}}
\multiput(44.65,3.82)(-0.20,-0.11){7}{\line(-1,0){0.20}}
\multiput(43.22,3.07)(-0.24,-0.11){6}{\line(-1,0){0.24}}
\multiput(41.76,2.40)(-0.30,-0.12){5}{\line(-1,0){0.30}}
\multiput(40.26,1.81)(-0.31,-0.10){5}{\line(-1,0){0.31}}
\multiput(38.73,1.30)(-0.39,-0.11){4}{\line(-1,0){0.39}}
\multiput(37.18,0.87)(-0.52,-0.11){3}{\line(-1,0){0.52}}
\multiput(35.61,0.53)(-0.53,-0.09){3}{\line(-1,0){0.53}}
\multiput(34.02,0.27)(-0.80,-0.09){2}{\line(-1,0){0.80}}
\put(32.41,0.10){\line(-1,0){1.61}}
\put(30.81,0.01){\line(-1,0){1.61}}
\put(29.19,0.01){\line(-1,0){1.61}}
\multiput(27.59,0.10)(-0.80,0.09){2}{\line(-1,0){0.80}}
\multiput(25.98,0.27)(-0.53,0.09){3}{\line(-1,0){0.53}}
\multiput(24.39,0.53)(-0.52,0.11){3}{\line(-1,0){0.52}}
\multiput(22.82,0.87)(-0.39,0.11){4}{\line(-1,0){0.39}}
\multiput(21.27,1.30)(-0.31,0.10){5}{\line(-1,0){0.31}}
\multiput(19.74,1.81)(-0.30,0.12){5}{\line(-1,0){0.30}}
\multiput(18.24,2.40)(-0.24,0.11){6}{\line(-1,0){0.24}}
\multiput(16.78,3.07)(-0.20,0.11){7}{\line(-1,0){0.20}}
\multiput(15.35,3.82)(-0.20,0.12){7}{\line(-1,0){0.20}}
\multiput(13.97,4.64)(-0.17,0.11){8}{\line(-1,0){0.17}}
\multiput(12.63,5.54)(-0.14,0.11){9}{\line(-1,0){0.14}}
\multiput(11.34,6.51)(-0.14,0.12){9}{\line(-1,0){0.14}}
\multiput(10.11,7.54)(-0.12,0.11){10}{\line(-1,0){0.12}}
\multiput(8.93,8.64)(-0.11,0.12){10}{\line(0,1){0.12}}
\multiput(7.81,9.81)(-0.12,0.14){9}{\line(0,1){0.14}}
\multiput(6.76,11.03)(-0.11,0.14){9}{\line(0,1){0.14}}
\multiput(5.78,12.30)(-0.11,0.17){8}{\line(0,1){0.17}}
\multiput(4.86,13.63)(-0.11,0.17){8}{\line(0,1){0.17}}
\multiput(4.02,15.00)(-0.11,0.20){7}{\line(0,1){0.20}}
\multiput(3.25,16.42)(-0.12,0.24){6}{\line(0,1){0.24}}
\multiput(2.56,17.87)(-0.10,0.25){6}{\line(0,1){0.25}}
\multiput(1.95,19.36)(-0.11,0.30){5}{\line(0,1){0.30}}
\multiput(1.42,20.88)(-0.11,0.39){4}{\line(0,1){0.39}}
\multiput(0.97,22.43)(-0.09,0.39){4}{\line(0,1){0.39}}
\multiput(0.61,24.00)(-0.09,0.53){3}{\line(0,1){0.53}}
\multiput(0.33,25.59)(-0.10,0.80){2}{\line(0,1){0.80}}
\put(0.13,27.18){\line(0,1){1.61}}
\put(0.02,28.79){\line(0,1){1.61}}
\put(0.00,30.40){\line(0,1){1.61}}
\multiput(0.07,32.01)(0.08,0.80){2}{\line(0,1){0.80}}
\multiput(0.22,33.62)(0.12,0.80){2}{\line(0,1){0.80}}
\multiput(0.46,35.21)(0.11,0.53){3}{\line(0,1){0.53}}
\multiput(0.78,36.79)(0.10,0.39){4}{\line(0,1){0.39}}
\multiput(1.18,38.35)(0.10,0.31){5}{\line(0,1){0.31}}
\multiput(1.67,39.88)(0.11,0.30){5}{\line(0,1){0.30}}
\multiput(2.25,41.39)(0.11,0.25){6}{\line(0,1){0.25}}
\multiput(2.90,42.86)(0.10,0.21){7}{\line(0,1){0.21}}
\multiput(3.63,44.30)(0.12,0.20){7}{\line(0,1){0.20}}
\multiput(4.43,45.69)(0.11,0.17){8}{\line(0,1){0.17}}
\multiput(5.31,47.04)(0.12,0.16){8}{\line(0,1){0.16}}
\multiput(6.26,48.34)(0.11,0.14){9}{\line(0,1){0.14}}
\multiput(7.28,49.59)(0.11,0.12){10}{\line(0,1){0.12}}
\multiput(8.36,50.78)(0.11,0.11){10}{\line(1,0){0.11}}
\multiput(9.51,51.91)(0.13,0.12){9}{\line(1,0){0.13}}
\multiput(10.72,52.98)(0.14,0.11){9}{\line(1,0){0.14}}
\multiput(11.98,53.98)(0.16,0.12){8}{\line(1,0){0.16}}
\multiput(13.29,54.92)(0.17,0.11){8}{\line(1,0){0.17}}
\multiput(14.65,55.78)(0.20,0.11){7}{\line(1,0){0.20}}
\multiput(16.06,56.56)(0.24,0.12){6}{\line(1,0){0.24}}
\multiput(17.50,57.27)(0.25,0.11){6}{\line(1,0){0.25}}
\multiput(18.99,57.91)(0.30,0.11){5}{\line(1,0){0.30}}
\multiput(20.50,58.46)(0.39,0.12){4}{\line(1,0){0.39}}
\multiput(22.04,58.93)(0.39,0.10){4}{\line(1,0){0.39}}
\multiput(23.61,59.31)(0.53,0.10){3}{\line(1,0){0.53}}
\multiput(25.19,59.61)(0.80,0.11){2}{\line(1,0){0.80}}
\multiput(26.78,59.83)(1.61,0.09){2}{\line(1,0){1.61}}
%\end
\put(30.00,30.00){\vector(2,1){26.33}}
\put(16.50,57.00){\line(1,-2){26.83}}
\put(30.00,65.84){\makebox(0,0)[cc]{${\vec  b}$}}
\put(63.00,46.00){\makebox(0,0)[cc]{${\vec  a}$}}
\put(46.00,51.83){\makebox(0,0)[cc]{$+1$}}
\put(7.67,43.01){\makebox(0,0)[cc]{$-1$}}
\put(52.83,17.17){\makebox(0,0)[cc]{$-1$}}
\put(11.83,10.67){\makebox(0,0)[cc]{$+1$}}
%\circle(30.00,30.00){30.00}
\put(30.00,45.00){\line(1,0){1.00}}
\put(31.00,44.97){\line(1,0){0.99}}
\multiput(31.99,44.87)(0.49,-0.08){2}{\line(1,0){0.49}}
\multiput(32.97,44.70)(0.49,-0.11){2}{\line(1,0){0.49}}
\multiput(33.94,44.47)(0.32,-0.10){3}{\line(1,0){0.32}}
\multiput(34.90,44.18)(0.31,-0.12){3}{\line(1,0){0.31}}
\multiput(35.83,43.82)(0.23,-0.10){4}{\line(1,0){0.23}}
\multiput(36.73,43.40)(0.22,-0.12){4}{\line(1,0){0.22}}
\multiput(37.61,42.93)(0.17,-0.11){5}{\line(1,0){0.17}}
\multiput(38.45,42.39)(0.16,-0.12){5}{\line(1,0){0.16}}
\multiput(39.25,41.80)(0.13,-0.11){6}{\line(1,0){0.13}}
\multiput(40.02,41.16)(0.12,-0.12){6}{\line(1,0){0.12}}
\multiput(40.74,40.47)(0.11,-0.12){6}{\line(0,-1){0.12}}
\multiput(41.41,39.74)(0.10,-0.13){6}{\line(0,-1){0.13}}
\multiput(42.03,38.96)(0.11,-0.16){5}{\line(0,-1){0.16}}
\multiput(42.60,38.14)(0.10,-0.17){5}{\line(0,-1){0.17}}
\multiput(43.11,37.28)(0.11,-0.22){4}{\line(0,-1){0.22}}
\multiput(43.57,36.40)(0.10,-0.23){4}{\line(0,-1){0.23}}
\multiput(43.96,35.48)(0.11,-0.31){3}{\line(0,-1){0.31}}
\multiput(44.30,34.54)(0.09,-0.32){3}{\line(0,-1){0.32}}
\multiput(44.57,33.58)(0.10,-0.49){2}{\line(0,-1){0.49}}
\multiput(44.77,32.60)(0.07,-0.49){2}{\line(0,-1){0.49}}
\put(44.91,31.62){\line(0,-1){0.99}}
\put(44.99,30.62){\line(0,-1){1.00}}
\put(45.00,29.63){\line(0,-1){1.00}}
\multiput(44.94,28.63)(-0.06,-0.49){2}{\line(0,-1){0.49}}
\multiput(44.81,27.64)(-0.09,-0.49){2}{\line(0,-1){0.49}}
\multiput(44.62,26.66)(-0.08,-0.32){3}{\line(0,-1){0.32}}
\multiput(44.37,25.70)(-0.11,-0.32){3}{\line(0,-1){0.32}}
\multiput(44.05,24.75)(-0.09,-0.23){4}{\line(0,-1){0.23}}
\multiput(43.67,23.83)(-0.11,-0.22){4}{\line(0,-1){0.22}}
\multiput(43.23,22.94)(-0.10,-0.17){5}{\line(0,-1){0.17}}
\multiput(42.73,22.07)(-0.11,-0.17){5}{\line(0,-1){0.17}}
\multiput(42.18,21.24)(-0.10,-0.13){6}{\line(0,-1){0.13}}
\multiput(41.57,20.45)(-0.11,-0.12){6}{\line(0,-1){0.12}}
\multiput(40.91,19.71)(-0.12,-0.12){6}{\line(-1,0){0.12}}
\multiput(40.20,19.00)(-0.13,-0.11){6}{\line(-1,0){0.13}}
\multiput(39.45,18.35)(-0.13,-0.10){6}{\line(-1,0){0.13}}
\multiput(38.65,17.75)(-0.17,-0.11){5}{\line(-1,0){0.17}}
\multiput(37.82,17.20)(-0.17,-0.10){5}{\line(-1,0){0.17}}
\multiput(36.95,16.71)(-0.22,-0.11){4}{\line(-1,0){0.22}}
\multiput(36.06,16.28)(-0.23,-0.09){4}{\line(-1,0){0.23}}
\multiput(35.13,15.90)(-0.32,-0.10){3}{\line(-1,0){0.32}}
\multiput(34.18,15.59)(-0.32,-0.08){3}{\line(-1,0){0.32}}
\multiput(33.22,15.35)(-0.49,-0.09){2}{\line(-1,0){0.49}}
\put(32.24,15.17){\line(-1,0){0.99}}
\put(31.25,15.05){\line(-1,0){1.00}}
\put(30.25,15.00){\line(-1,0){1.00}}
\put(29.25,15.02){\line(-1,0){0.99}}
\multiput(28.26,15.10)(-0.49,0.07){2}{\line(-1,0){0.49}}
\multiput(27.27,15.25)(-0.49,0.11){2}{\line(-1,0){0.49}}
\multiput(26.30,15.46)(-0.32,0.09){3}{\line(-1,0){0.32}}
\multiput(25.34,15.74)(-0.31,0.11){3}{\line(-1,0){0.31}}
\multiput(24.40,16.08)(-0.23,0.10){4}{\line(-1,0){0.23}}
\multiput(23.49,16.49)(-0.22,0.12){4}{\line(-1,0){0.22}}
\multiput(22.61,16.95)(-0.17,0.10){5}{\line(-1,0){0.17}}
\multiput(21.76,17.47)(-0.16,0.12){5}{\line(-1,0){0.16}}
\multiput(20.94,18.04)(-0.13,0.10){6}{\line(-1,0){0.13}}
\multiput(20.17,18.67)(-0.12,0.11){6}{\line(-1,0){0.12}}
\multiput(19.44,19.35)(-0.11,0.12){6}{\line(0,1){0.12}}
\multiput(18.75,20.07)(-0.11,0.13){6}{\line(0,1){0.13}}
\multiput(18.12,20.84)(-0.12,0.16){5}{\line(0,1){0.16}}
\multiput(17.54,21.65)(-0.11,0.17){5}{\line(0,1){0.17}}
\multiput(17.01,22.50)(-0.12,0.22){4}{\line(0,1){0.22}}
\multiput(16.54,23.38)(-0.10,0.23){4}{\line(0,1){0.23}}
\multiput(16.13,24.29)(-0.12,0.31){3}{\line(0,1){0.31}}
\multiput(15.78,25.22)(-0.10,0.32){3}{\line(0,1){0.32}}
\multiput(15.50,26.18)(-0.11,0.49){2}{\line(0,1){0.49}}
\multiput(15.27,27.15)(-0.08,0.49){2}{\line(0,1){0.49}}
\put(15.12,28.13){\line(0,1){0.99}}
\put(15.03,29.13){\line(0,1){1.00}}
\put(15.00,30.12){\line(0,1){1.00}}
\put(15.04,31.12){\line(0,1){0.99}}
\multiput(15.15,32.11)(0.09,0.49){2}{\line(0,1){0.49}}
\multiput(15.32,33.09)(0.12,0.48){2}{\line(0,1){0.48}}
\multiput(15.56,34.06)(0.10,0.32){3}{\line(0,1){0.32}}
\multiput(15.86,35.01)(0.09,0.23){4}{\line(0,1){0.23}}
\multiput(16.23,35.94)(0.11,0.23){4}{\line(0,1){0.23}}
\multiput(16.65,36.84)(0.10,0.17){5}{\line(0,1){0.17}}
\multiput(17.14,37.71)(0.11,0.17){5}{\line(0,1){0.17}}
\multiput(17.68,38.55)(0.12,0.16){5}{\line(0,1){0.16}}
\multiput(18.27,39.35)(0.11,0.13){6}{\line(0,1){0.13}}
\multiput(18.92,40.11)(0.12,0.12){6}{\line(0,1){0.12}}
\multiput(19.62,40.82)(0.12,0.11){6}{\line(1,0){0.12}}
\multiput(20.36,41.49)(0.13,0.10){6}{\line(1,0){0.13}}
\multiput(21.14,42.11)(0.16,0.11){5}{\line(1,0){0.16}}
\multiput(21.97,42.67)(0.17,0.10){5}{\line(1,0){0.17}}
\multiput(22.83,43.17)(0.22,0.11){4}{\line(1,0){0.22}}
\multiput(23.72,43.62)(0.23,0.10){4}{\line(1,0){0.23}}
\multiput(24.64,44.01)(0.31,0.11){3}{\line(1,0){0.31}}
\multiput(25.58,44.33)(0.32,0.09){3}{\line(1,0){0.32}}
\multiput(26.54,44.60)(0.49,0.10){2}{\line(1,0){0.49}}
\multiput(27.52,44.79)(0.49,0.07){2}{\line(1,0){0.49}}
\put(28.51,44.93){\line(1,0){1.49}}
%\end
\put(30.00,30.00){\vector(4,-1){14.33}}
\put(30.00,30.00){\vector(-1,-4){3.58}}
\put(30.00,30.00){\line(1,4){3.63}}
\put(30.00,30.00){\line(-4,1){14.50}}
\put(42.33,33.67){\makebox(0,0)[cc]{$-$}}
\put(37.33,20.50){\makebox(0,0)[cc]{$+$}}
\put(19.17,24.17){\makebox(0,0)[cc]{$-$}}
\put(22.67,40.17){\makebox(0,0)[cc]{$+$}}
\put(47.00,25.67){\makebox(0,0)[cc]{${\vec  \lambda}$}}
\put(25.67,12.00){\makebox(0,0)[cc]{${\vec  \lambda}^\perp$}}
\put(48.00,44.50){\makebox(0,0)[cc]{${\vec  \lambda}-{\vec  \lambda}^\perp$}}
\put(30.00,30.02){\vector(3,2){18.60}}
\put(44.73,26.36){\vector(1,4){3.95}}
\put(30.00,30.00){\vector(0,1){30.00}}
\put(0.00,30.00){\line(1,0){60.00}}
\end{picture}


 }

 \frame[shrink=2]{
\frametitle{Resulting ``stronger'' correlations}

\begin{itemize}
\item<1->
The associated correlation function for $\theta =\vert a-b\vert$ is
\begin{equation}
E(\theta ,\delta )= \left\{
\begin{array}{ll}
-1 & \text{ for } \;\; 0\le \theta \le {\delta \over 2} ,  \\
-1 +{2\over \pi}(\theta -{\delta \over 2} )&\text{ for } \;\; {\delta \over 2} < \theta \le {1 \over 2}(\pi - \delta) ,   \\
-2(1-{2 \over \pi } \theta ) &\text{ for } \; \; {1 \over 2}(\pi - \delta) < \theta \le {1 \over 2}(\pi   + \delta ) , \\
1+ {2\over \pi }(\theta-\pi +{\delta \over 2} ) &\text{ for } \;\; {1 \over 2}(\pi   + \delta )  < \theta \le \pi - {\delta \over 2} , \\
1 & \text{ for } \;\; \pi - {\delta \over 2} < \theta \le \pi .
\end{array}        \right.
\nonumber
\end{equation}

\item<1-> Strongest correlation for $\delta= \pi /2$
\begin{equation}
\begin{array}{lll}
E\left(\theta ,{\pi \over 2}\right)
&=&
H\left(\theta - {3 \pi\over 4}\right)
 - H\left({\pi \over 4} - \theta \right)  - \\
&&
\qquad
\qquad
\qquad
  2\left(1 - {2\over \pi}\theta \right)
H\left(\theta - {\pi\over 4}\right) H\left({3 \pi\over 4} - \theta \right)
.
\end{array}
\nonumber
\end{equation}

\item<1-> Averaged correlation between $0\le \delta \le \pi /2$
\begin{equation}
{2\over \pi }\int_0^{\pi /2} d\delta \; E(\theta,\delta )
=
{4\over \pi^2}
\left[
\left(\theta ^2 -{\pi^2 \over 4}\right)
-2
H\left(\theta -{\pi \over 2}\right)
\left(\theta -{\pi \over 2} \right)^2
\right].
\nonumber
\end{equation}

\end{itemize}
 }

 \frame[shrink=2]{
\frametitle{Resulting ``stronger'' correlations cntd.}
$\;$\\
$\;$\\
\centering
 \includegraphics[width=70mm]{2004-brainteaser-f2a}

 }

 \frame{
\frametitle{Properties of ``stronger'' correlations}
\begin{itemize}
\item<1->
For all nonzero $\delta$, $E(\theta ,\delta )$ correlates stronger than quantized systems
for some values of $\theta$.

\item<1->
For $\delta =\pi /2$
and
for
${a} ={\pi \over 4}$, ${a}'= {3\pi \over 4}$
${b} ={\pi \over 2}$ , ${b}'=0$,
the Clauser-Horne-Shimony-Holt  inequality
is violated maximally by
$$
\vert
E({\pi \over 4} ,{\pi \over 2})+
E({\pi \over 4} ,0)+
E({3\pi \over 4} ,{\pi \over 2})-
E({3\pi \over 4},0)
\vert  = 4,
$$
which is larger
than the Tsirelson bound for quantum violations $2\sqrt{2}$.

\item<1->
For $\delta =0$, the classical linear correlation function
$E(\theta )=  2\theta /\pi -1$ is recovered, as can be expected.

\end{itemize}
 }


%%%%%%%%%%%%%%%%%%%%%%%%%%%%%%%%%%%%%%%%%%%%%%%%%%%%%%%%%%%%%%%%%%%%%%%%%%%%%%%%%%%%
%%%%%%%%%%%%%%%%%%%%%%%%%%%%%%%%%%%%%%%%%%%%%%%%%%%%%%%%%%%%%%%%%%%%%%%%%%%%%%%%%%%%
%%%%%%%%%%%%%%%%%%%%%%%%%%%%%%%%%%%%%%%%%%%%%%%%%%%%%%%%%%%%%%%%%%%%%%%%%%%%%%%%%%%%

\section{Quantum logic}


\frame{



\centerline{\Huge Part II:  Quantum Logic}

\begin{center}
$\widetilde{\qquad \qquad }$
$\widetilde{\qquad \qquad}$
$\widetilde{\qquad \qquad }$
\end{center}
 }


\frame{
\frametitle{Principles}
{\small
\frametitle{Principles of quantum logic}

\begin{beamerboxesrounded}[scheme=alert,shadow=true]{}
Quantum logic has been introduced by
Garrett Birkhoff and John von Neumann in the thirties.
They  organized it {\em top-down},
starting from the Hilbert space
formalism of quantum mechanics.
Certain entities of Hilbert spaces are identified
with propositions,
partial order relations and lattice operations.
These relations and operations are
identified with the logical
implication relation and operations such as ``and,'' ``or,'' and the negation.

If theoretical physics is assumed to be a faithful
representation of our experience, such an ``empirical,'' ``operational''
logic derives
its justification by the phenomena themselves.
In this sense,
one of the main justifications for quantum logic is the
construction of the logical and algebraic order of events based on empirical findings.
\end{beamerboxesrounded}

Kochen and Specker suggested to consider only relations and operations among compatible, co-measurable observables;
i.e., within Boolean subalgebras, which will be identified with blocks and contexts
of Hilbert lattices.
Nevertheless, some of their theorems formally take into account ensembles of contexts
 for which a multitude of incompatible observables contribute.
}
}

\subsection{Identifications}
\frame{
\frametitle{Quantum proposition}

Any closed linear
subspace of --- or, equivalently, any
projection operator on ---  a Hilbert space corresponds to an elementary
proposition. The elementary {``true''}--{``false''} proposition can in
English be spelled out explicitly as

\begin{quote}
``The physical system has a property corresponding to the associated
closed linear subspace.''
\end{quote}

{\footnotesize
\begin{itemize}
\item<1->  Note that this identification with logical propositions comes quite naturally, as projectors $E$ are {\em idempotent} linear operators; i.e.,
$E^2=E$ with two eigenvalues  $0$ and $1$.



\item<1->  Note that projectors can be composed from normalized vectors
via their dyadic product;
$\vert 0\rangle =(0,1)^T$, then $E_0=\vert 0\rangle \langle 0\vert
= (0,1)^T\otimes (0,1) =  \left(
\begin{array}{cc}
0&0\\
0&1
\end{array}
\right)$.



\end{itemize}
}
}

\frame{
\frametitle{Logical operations among propositions}

  The logical {``and''} operation is identified with the set
theoretical intersection of two propositions ``$\cap$''; i.e., with the
intersection of two subspaces.
It is denoted by the symbol ``$\wedge$''.
So, for two
propositions $p$ and $q$ and their associated closed linear
subspaces
${\mathfrak M}_p$ and
${\mathfrak M}_q$,
$$
{\mathfrak M}_{p\wedge q} = \{x \mid x\in
{\mathfrak M}_p, \;
x\in {\mathfrak M}_q\} .$$

}


\frame{
\frametitle{Logical operations among propositions cntd.}
  The logical {``or''} operation is identified with the closure of the
linear span ``$\oplus$'' of the subspaces corresponding to the two
propositions.
 It is denoted by the symbol ``$\vee$''.
So, for two
propositions $p$ and $q$ and their associated closed linear
subspaces
${\mathfrak M}_p$ and
${\mathfrak M}_q$,
$$
{\mathfrak M}_{p\vee q} =
{\mathfrak M}_{p} \oplus
{\mathfrak M}_{q} =
 \{x \mid x=\alpha y+\beta z,\; \alpha,\beta \in {\mathbb C},\; y\in
{\mathfrak M}_p, \;
z\in {\mathfrak M}_q\} .$$



Notice that
a vector of Hilbert space may be an element of
$
{\mathfrak M}_{p} \oplus
{\mathfrak M}_{q}
$
without being an element of either
$
{\mathfrak M}_{p} $ or
${\mathfrak M}_{q}
$, since
$
{\mathfrak M}_{p} \oplus
{\mathfrak M}_{q}
$
includes all the vectors in
$
{\mathfrak M}_{p} \cup
{\mathfrak M}_{q}
$, as well as all of their linear combinations (superpositions) and
their limit vectors.
}


\frame{
\frametitle{Logical operations among propositions cntd.}
 The logical {``not''}-operation, or ``negation'' or ``complement,''
is
identified with operation of taking the orthogonal subspace ``$\perp$''.
It is denoted by the symbol ``~$'$~''.
In particular, for a
proposition $p$ and its associated closed linear
subspace
${\mathfrak M}_p$, the negation $p'$ is associated with
$$
{\mathfrak M}_{p'} =
 \{x \mid (x,y)=0,\; y\in
{\mathfrak M}_p
\} ,$$
where $(x,y)$ denotes the scalar product of $x$ and $y$.



}


\frame{
\frametitle{Logical relation among propositions}

  The logical {``implication''} relation is identified with the set
theoretical subset relation ``$\subset$''.
It is denoted by the symbol ``$\rightarrow$''.
So, for two
propositions $p$ and $q$ and their associated closed linear
subspaces
${\mathfrak M}_p$ and
${\mathfrak M}_q$,
$$
{p\rightarrow q} \Longleftrightarrow
{\mathfrak M}_{p} \subset
{\mathfrak M}_{q}.$$


}


\frame{
\frametitle{Extrema}

\begin{itemize}

\item<1->  A trivial statement which is always {``true''} is denoted by $1$.
It is represented by the entire Hilbert space $\mathfrak H$.
So, $${\mathfrak M}_1=\mathfrak H.$$

\item<1->
An absurd statement which is always {``false''} is denoted by $0$.
It is represented by the zero vector $0$.
So, $${\mathfrak M}_0= 0.$$







\end{itemize}

}


\frame{
\frametitle{Comparison of quantum versus classical relations \& operations}
\begin{beamerboxesrounded}[scheme=alert,shadow=true]{}
\begin{center}
{\tiny
 \begin{tabular}{|ccccc|} \hline\hline
 generic lattice  &  order relation   & ``meet''
&
``join''  & ``complement''\\
\hline
propositional&implication&disjunction&conjunction&negation\\
calculus&$\rightarrow$&``and'' $\wedge$&``or'' $\vee$&``not'' $\neg$\\
\hline
``classical'' lattice  &  subset $\subset $  & intersection $\cap$ &
union
$\cup$ & complement\\
of subsets&&&&\\
of a set&&&&\\
\hline
Hilbert & subspace& intersection of & closure of     & orthogonal \\
lattice & relation& subspaces $\cap$&  linear& subspace   \\
        & $\subset$ &                 & span $\oplus$  &  $\perp$   \\
\hline
lattice of& $E_1E_2=E_1$& $E_1E_2$& $E_1+E_2-E_1E_2$& orthogonal\\
commuting&&&&projection\\
\{noncommuting\}&&\{$\displaystyle\lim_{n\rightarrow \infty}(E_1E_2)^n$\}&&\\
projection&&&&\\
operators&&&&\\
 \hline\hline
 \end{tabular}
}
\end{center}
\end{beamerboxesrounded}

}


\subsection{Diagrammatical representation, blocks}
\frame{
\frametitle{Diagrammatical representation, blocks}
Propositional structures are often represented by
Hasse and Greechie diagrams.

A {\em Hasse diagram} is a convenient representation of the
logical implication,
as well as of the {``and''} and {``or''}
operations
among propositions.
 Points
``~$\bullet$~'' represent propositions. Propositions
which are implied by other ones are drawn higher than the other ones.
Two propositions are connected by a line if one implies the other.
Atoms are propositions which ``cover'' the least element $0$; i.e.,
they lie ``just above'' $0$ in a Hasse diagram of the partial order.




}


\frame{
\frametitle{Diagrammatical representation, blocks cntd.}

A much more compact representation of the propositional calculus can be
given in terms of
its {\em Greechie diagram}.
In this representation, the emphasis is on Boolean subalgebras.
Points ``~$\circ$~'' represent the atoms.
If they belong to the same Boolean subalgebra, they are connected by edges or smooth curves.
The collection of all atoms and elements belonging to the same Boolean subalgebra is called {\em block};
i.e., every block represents a Boolean subalgebra within a nonboolean structure.

}


\frame{
\frametitle{Pasting construction}
The blocks can be joined or pasted together as follows.

\begin{beamerboxesrounded}[scheme=alert,shadow=true]{}
\begin{itemize}

\item<1->
The tautologies of all blocks are identified.
\item<1->
The absurdities of all blocks are identified.
\item<1->
Identical elements in different blocks are identified.
\item<1->
The logical and algebraic structures of all blocks remain intact.
\end{itemize}
\end{beamerboxesrounded}

This construction is often referred to as {\em pasting} construction.
If the blocks are only pasted together at the tautology and
the absurdity, one calls the resulting logic a {\em horizontal
sum}.




Every single block represents some ``maximal collection of co-measurable observables''
which will be identified with some quantum {\em context}.
Hilbert lattices can be thought of as the pasting of a continuity of such blocks or contexts.

}

\subsection{Quantum contexts as blocks}
\frame{
\frametitle{Quantum contexts as blocks}
{
\begin{beamerboxesrounded}[scheme=alert,shadow=true]{}
\begin{itemize}

\item<1->
All that is operationally knowable for a given quantized system is a {\em single block}
representing co-measurable observables.
Thus, single blocks or, in another terminology, maximal Boolean subalgebras of Hilbert lattices,
will be identified with quantum contexts.

\item<1->
Equivalently, a quantum context can  be formalized
by a single (nondegenerate) ``maximal'' self-adjoint operator
such that all commuting, compatible co-measurable observables are functions thereof.
\end{itemize}
\end{beamerboxesrounded}
$\;$\\
%$\;$\\
\begin{beamerboxesrounded}[scheme=alert2,shadow=true]{Hilbert lattices as pastings of contexts}
As Hilbert lattices are pastings of a continuity of blocks or contexts, contexts are the building blocks of quantum logics.
\end{beamerboxesrounded}



}
}



\subsection{Counterfactuals}
\frame{
\frametitle{``Counterfactually measuring'' different contexts}
{\footnotesize
Einstein-Podolski-Rosen (EPR) type arguments  utilizing a configuration
sketched before
claim to be able to infer two different contexts counterfactually.
One context is measured on one side of the setup, the other context on the other side of it.
By the uniqueness property  of certain two-particle states,
knowledge of a property of one particle entails the certainty
that, if this property were measured on the other particle as well, the outcome of the measurement would be
a unique function of the outcome of the measurement performed.
\\
$\;$\\
\begin{beamerboxesrounded}[scheme=alert,shadow=true]{}
This makes possible the measurement of one context, {\em as well as} the {\em simultaneous counterfactual inference} of a different context.
Because, one could argue, although one has actually measured on one side a different, incompatible context compared to the context measured on the other side,
if on both sides the same  context {\em would be measured}, the outcomes on both sides {\em would be uniquely correlated}.
Hence measurement of one context per side is sufficient, for the outcome could be counterfactually inferred on the other side.
\end{beamerboxesrounded}
}
}


\frame{
\frametitle{Realism-Idealism in philosophy \& theology}

\begin{itemize}
\item<+-> Realism: Some entities sometimes exist without being experienced by any finite mind.

\item<+-> Idealism:
$\ldots$
we have not the faintest reason for believing in the existence of
unexperienced entities
$\ldots$
[[Realism]] has been adopted
$\ldots$
solely because it simplifies our view of the universe.
\\
(W.T. Stace, Mind {\bf 53}, 1934 \& ``Readings in Philosophical
Analysis'', ed. by Feigl \& Sellars).
\end{itemize}
 }



\frame{
\frametitle{Scholasticism in quantum physics}

%\begin{columns}
%\begin{column}{5cm}
%\pgfdeclareimage[height=2cm]{Specker}{specker}
%\pgfuseimage{Specker}
%\end{column}
%\begin{column}{9cm}
\begin{itemize}
\item<+->
In 1960, Specker related the discussion on the foundations of quantum mechanics to scholastic philosophy;
in particular to scholastic speculations  about the existence of ``infuturabilities'' or
``counterfactuals.''

\item<+->
Question: Does
the omniscience (comprehensive knowledge) of God extend to events which
would have occurred if something  had happened which did not
happen?

\item<+->
Question: If so, can all events be pasted together to form a consistent whole?
\end{itemize}
%\end{column}
%\end{columns}
 }


\frame{
\frametitle{Example I: Measurement of an ``observable'' corresponding to arbitrary operator}

In the counterfactual sense, the measurement of observables associated with arbitrary operators
becomes feasible, since formally any matrix $A$ can be uniquely decomposed into two
self-adjoint components
$A_1, A_2$ as follows (e.g., Halmos, {\em ``Finite-dimensional vector spaces:''}):
$$
\begin{array}{lllllll}
A&=&A_1+iA_2 \\
A_1&=&{1\over 2}(A+A^\dagger) =:\Re A,\;   \;
A_2=-{i\over 2}(A-A^\dagger)=:\Im A.
\end{array}  \label{e-decom1}
$$
By assuming the uniqueness property,  $A_1, A_2$ can be measured along two different entangled
particles, respectively, and subsequently counterfactually ``completed.''

}

\frame{
\frametitle{Example II: the spin one-half system}
{\small
As an example, we shall paste together observables of the spin
one-half systems.
We have associated a propositional system
$$L({\bf a})= \{ 0, E, E', 1 \}, $$
corresponding to the outcomes of a measurement of the spin states
along some arbitrary direction ${\bf a}$.
If the spin states would be measured along a different spatial
direction, say
${\bf b}\neq \pm {\bf a} $, an identical propositional
system
$$L({\bf b})= \{ 0, F, F', 1 \} $$
would have resulted, with the propositions $E$ and $F$ explicitly expressed before.
The  two-dimensional Hilbert space representation of this configuration is depicted as follows:
\begin{center}
%TexCad Options
%\grade{\off}
%\emlines{\off}
%\beziermacro{\on}
%\reduce{\on}
%\snapping{\off}
%\quality{2.00}
%\graddiff{0.01}
%\snapasp{1}
%\zoom{1.00}
\unitlength 0.4mm
\thicklines %\linethickness{0.4pt}
\begin{picture}(40.00,49.67)
%\put(60.33,15.00){\circle{0.00}}
%\put(60.33,15.00){\circle{2.00}}
%\put(45.33,10.00){\circle{2.00}}
%\put(30.33,5.00){\circle{2.00}}
%\put(15.33,10.00){\circle{2.00}}
%\put(0.33,16.00){\circle{0.00}}
%\put(0.33,15.00){\circle{2.00}}
\put(14.70,45.00){\color{green}\line(0,-1){30.00}}
\put(15.30,45.00){\color{red}\line(0,-1){30.00}}
\put(15.00,15.00){\color{red}\line(-1,-1){15.00}}
\put(15.00,15.00){\color{red}\line(1,0){25.00}}
\put(15.00,15.00){\color{green}\line(3,-4){11.00}}
\put(15.00,15.00){\color{green}\line(5,3){16.67}}
\put(3.33,-1.67){\makebox(0,0)[cc]{$B$}}
\put(30.00,0.00){\makebox(0,0)[cc]{$D$}}
\put(40.00,11.33){\makebox(0,0)[cc]{$C$}}
\put(35.00,23.67){\makebox(0,0)[cc]{$K$}}
\put(19.33,49.67){\makebox(0,0)[cc]{$A$}}
%\bezvec{60}(7.67,6.67)(14.33,2.67)(20.00,6.67)
\put(20.00,6.67){\vector(2,1){0.2}}
\bezier{60}(7.67,6.67)(14.33,2.67)(20.00,6.67)
%\end
%\bezvec{36}(29.67,16.00)(32.33,19.67)(30.00,23.00)
\put(30.00,23.00){\vector(-1,2){0.2}}
\bezier{36}(29.67,16.00)(32.33,19.67)(30.00,23.00)
%\end
\put(13.33,1.00){\makebox(0,0)[cc]{$\varphi$}}
\put(40.00,18.67){\makebox(0,0)[rc]{$\varphi$}}
\end{picture}
\end{center}
}
}


\frame{
\frametitle{Example II: the spin one-half system cntd.}

$L({\bf a})$ and $L({\bf b})$ can be joined by pasting them together.
In particular, we identify their tautologies and absurdities; i.e., $0$ and $1$.
All the other propositions remain distinct.
We then obtain a propositional structure
$$L({\bf a})\oplus L({\bf b}) =
MO_2$$
whose Hasse diagram is of the ``Chinese lantern'' form.
Here, the ``$O$'' stands for {\em orthocomplementation,}
expressing the fact that for every element there exists an orthogonal complement.
The term ``$M$''
stands for {\em modularity}; i.e., for all $x\rightarrow b$, $x \vee (a\wedge b)=(x\vee a)\wedge b$.
The subscript ``2'' stands for the pasting of
two Boolean subalgebras $2^2$.
}


\frame{
\frametitle{Example II: the spin one-half system cntd.}
Since all possible directions ${\bf a}\in {\mathbb R}^3$ form a continuum,
the Hilbert lattice is a continuum of pastings of subalgebras of the form $L({\bf a})$:

\begin{center}
\begin{tabular}{ccc}
%TexCad Options
%\grade{\off}
%\emlines{\off}
%\beziermacro{\off}
%\reduce{\on}
%\snapping{\off}
%\quality{0.20}
%\graddiff{0.01}
%\snapasp{1}
%\zoom{1.00}
\unitlength 0.40mm
\thicklines %\linethickness{0.4pt}
\begin{picture}(125.00,60.73)
\put(10.00,30.00){\color{red}\circle{1.5}}
\put(30.00,30.00){\color{red}\circle{1.5}}
\put(90.00,30.00){\color{green}\circle{1.5}}
\put(110.00,30.00){\color{green}\circle{1.5}}
\put(10.00,30.00){\color{red}\circle{3}}
\put(30.00,30.00){\color{red}\circle{3}}
\put(90.00,30.00){\color{green}\circle{3}}
\put(110.00,30.00){\color{green}\circle{3}}
\put(60.00,0.00){\color{red}\line(-1,1){30.00}}
\put(30.00,30.00){\color{red}\line(1,1){30.00}}
\put(60.00,60.00){\color{green}\line(1,-1){30.00}}
\put(90.00,30.00){\color{green}\line(-1,-1){30.00}}
\put(69.00,0.00){\makebox(0,0)[lc]{$0=1'$}}
\put(69.00,60.00){\makebox(0,0)[lc]{$1=\color{red}E\vee E'=\color{green} F \vee F'$}}
\put(30.00,25.00){\makebox(0,0)[cc]{\color{red}$E'$}}
\put(90.00,25.00){\makebox(0,0)[cc]{\color{green}$F$}}
\put(60.00,0.00){\color{red}\line(-5,3){50.00}}
\put(10.00,30.00){\color{red}\line(5,3){50.00}}
\put(60.00,60.00){\color{green}\line(5,-3){50.00}}
\put(110.00,30.00){\color{green}\line(-5,-3){50.00}}
\put(10.00,25.00){\makebox(0,0)[cc]{\color{red}$E$}}
\put(110.00,25.00){\makebox(0,0)[cc]{\color{green}$ F'$}}
\put(25.00,50.00){\makebox(0,0)[cc]{\color{red}$L(x)$}}
\put(95.00,50.00){\makebox(0,0)[cc]{\color{green}$L(\overline x)$}}
\put(0.00,15.00){\color{red}\dashbox{2.00}(45.00,30.00)[cc]{}}
\put(80.00,15.00){\color{green}\dashbox{2.00}(45.00,30.00)[cc]{}}
\put(60.00,0.00){\circle*{3}}
\put(60.00,59.67){\circle*{3}}
\end{picture}
&
\qquad
&
%TexCad Options
%\grade{\off}
%\emlines{\off}
%\beziermacro{\off}
%\reduce{\on}
%\snapping{\off}
%\quality{0.20}
%\graddiff{0.01}
%\snapasp{1}
%\zoom{1.00}
\unitlength 0.30mm
\thicklines %\linethickness{0.4pt}
\begin{picture}(141.06,65.00)
\put(0.00,55.00){\color{red}\circle{6}}
\put(60.00,55.00){\color{red}\circle{6}}
\put(0.00,45.00){\color{red}\makebox(0,0)[cc]{$E$}}
\put(60.00,45.00){\color{red}\makebox(0,0)[cc]{$E'$}}
\put(0.00,55.00){\color{red}\line(1,0){60.00}}
\put(80.00,55.00){\color{green}\circle{6}}
\put(140.00,55.00){\color{green}\circle{6}}
\put(80.00,45.00){\color{green}\makebox(0,0)[cc]{$ F $}}
\put(140.00,45.00){\color{green}\makebox(0,0)[cc]{$ F'$}}
\put(80.00,55.00){\color{green}\line(1,0){60.00}}
\put(40.00,65.00){\color{red}\makebox(0,0)[cc]{$L(x)$}}
\put(110.00,65.00){\color{green}\makebox(0,0)[cc]{$L(\overline x)$}}
\end{picture}
\\    \\
``Chinese lantern'' Hasse diagram&&Greechie
diagram
\end{tabular}
\end{center}
}


\frame{
\frametitle{Example II: the spin one-half system cntd.}
{\footnotesize
The propositional system obtained is  not a
classical Boolean algebra,
since the distributive laws are not satisfied; i.e.,
\begin{eqnarray*}
\begin{array}[b]{ccc}
F \vee (E \wedge E')&\stackrel{?}{=}&(F\vee E) \wedge (F \vee E')\\
F \vee 0&\stackrel{?}{=}&1 \wedge 1\\
F &\neq &1,
\end{array}
\\
\begin{array}[b]{ccc}
F \wedge (E \vee E')&\stackrel{?}{=}&(F\wedge E) \vee (F \wedge E')\\
F \wedge 1&\stackrel{?}{=}&0 \vee 0\\
F &\neq &0.
\end{array}
\end{eqnarray*}


Notice that the expressions can be easily evaluated by using the Hasse
diagram:
For any $a,b$,
$a\vee b$ is just the least element which is connected
by $a$ and $b$;
$a\wedge b$ is just the highest element connected
to $a$ and $b$. Intermediates which are not connected to both $a$
and $b$ do not count. That is,
\begin{center}
%TeXCAD Picture [1.pic]. Options:
%\grade{\off}
%\emlines{\off}
%\epic{\off}
%\beziermacro{\off}
%\reduce{\on}
%\snapping{\off}
%\quality{2.000}
%\graddiff{0.010}
%\snapasp{1}
%\zoom{6.7272}
\unitlength .4mm % = .854pt
\linethickness{0.4pt}
\ifx\plotpoint\undefined\newsavebox{\plotpoint}\fi % GNUPLOT compatibility
\begin{picture}(155.095,20.945)(0,0)
\put(0,0){\line(1,1){20}}
\put(20,20){\line(1,-1){20}}
\put(110.095,20){\line(1,-1){20}}
\put(130.095,0){\line(1,1){20}}
\put(150.095,20){\circle*{1.89}}
\put(110.095,20){\circle*{1.89}}
\put(20,20){\circle*{1.89}}
\put(130.095,0){\circle*{1.89}}
\put(40,0){\circle*{1.89}}
\put(0,0){\circle*{1.89}}
\put(5,0){\makebox(0,0)[cc]{$a$}}
\put(45,0){\makebox(0,0)[cc]{$b$}}
\put(115.095,20){\makebox(0,0)[cc]{$a$}}
\put(155.095,20){\makebox(0,0)[cc]{$b$}}
\put(25,20){\makebox(0,0)[lc]{$a\vee b$}}
\put(135.425,0){\makebox(0,0)[lc]{$a \wedge b$}}
\end{picture}
\end{center}
$a\vee b$ is called a least upper bound of $a$ and $b$.
\index{least upper bound}
$a\wedge b$ is called a greatest lower bound of $a$ and $b$.
\index{greatest lower bound}

}
}




\frame{
\frametitle{Example III: $L_{12}$}

\begin{itemize}

\item<1->  In two dimensional Hilbert space, interlinked contexts do not exist,
since every context is fixed by the assumption of one property.
The entire context is just this property, together with its negation,
which corresponds to the orthogonal ray (which spans a one dimensional subspace)
or projection associated with the ray corresponding to the property.



\item<1->
The simplest nontrivial configuration of interlinked contexts exists in three-dimensional Hilbert space.
Consider an arrangement
of five observables $A$, $B$, $C$, $D$, $K$ with two systems
of operators
$\{A,B,C\}$
and
$\{D,K,A\}$,
the  contexts,
which are interconnected by $A$.
Within a context, the operators commute and the associated observables are co-measurable.
For two different contexts, operators outside the link operators do not commute.
$A$ is a link observable.

\end{itemize}
}


\frame{
\frametitle{Example III: $L_{12}$ cntd.}

This propositional structure (also known as $L_{12}$) can be represented in three-dimensional Hilbert space
by two tripods with a single common leg.
$\;$\\
$\;$\\

%\begin{center}
\begin{tabular}{ccccc}
%TexCad Options
%\grade{\off}
%\emlines{\off}
%\beziermacro{\on}
%\reduce{\on}
%\snapping{\off}
%\quality{2.00}
%\graddiff{0.01}
%\snapasp{1}
%\zoom{1.00}
\unitlength 0.6mm
\thicklines %\linethickness{0.4pt}
\begin{picture}(40.00,49.67)
%\put(60.33,15.00){\circle{0.00}}
%\put(60.33,15.00){\circle{2.00}}
%\put(45.33,10.00){\circle{2.00}}
%\put(30.33,5.00){\circle{2.00}}
%\put(15.33,10.00){\circle{2.00}}
%\put(0.33,16.00){\circle{0.00}}
%\put(0.33,15.00){\circle{2.00}}
\put(14.70,45.00){\color{green}\line(0,-1){30.00}}
\put(15.30,45.00){\color{red}\line(0,-1){30.00}}
\put(15.00,15.00){\color{red}\line(-1,-1){15.00}}
\put(15.00,15.00){\color{red}\line(1,0){25.00}}
\put(15.00,15.00){\color{green}\line(3,-4){11.00}}
\put(15.00,15.00){\color{green}\line(5,3){16.67}}
\put(3.33,-1.67){\makebox(0,0)[cc]{$B$}}
\put(30.00,0.00){\makebox(0,0)[cc]{$D$}}
\put(40.00,11.33){\makebox(0,0)[cc]{$C$}}
\put(35.00,23.67){\makebox(0,0)[cc]{$K$}}
\put(19.33,49.67){\makebox(0,0)[cc]{$A$}}
%\bezvec{60}(7.67,6.67)(14.33,2.67)(20.00,6.67)
\put(20.00,6.67){\vector(2,1){0.2}}
\bezier{60}(7.67,6.67)(14.33,2.67)(20.00,6.67)
%\end
%\bezvec{36}(29.67,16.00)(32.33,19.67)(30.00,23.00)
\put(30.00,23.00){\vector(-1,2){0.2}}
\bezier{36}(29.67,16.00)(32.33,19.67)(30.00,23.00)
%\end
\put(13.33,1.00){\makebox(0,0)[cc]{$\varphi$}}
\put(40.00,18.67){\makebox(0,0)[rc]{$\varphi$}}
\end{picture}
&
\qquad
&
%TexCad Options
%\grade{\off}
%\emlines{\off}
%\beziermacro{\on}
%\reduce{\on}
%\snapping{\off}
%\quality{2.00}
%\graddiff{0.01}
%\snapasp{1}
%\zoom{1.00}
\unitlength 0.6mm
\thicklines %\linethickness{0.4pt}
\begin{picture}(61.33,36.00)
%\emline(0.33,35.00)(30.33,25.00)
\multiput(0.33,35.00)(0.36,-0.12){84}{\color{red}\line(1,0){0.36}}
%\end
%\emline(30.33,25.00)(60.33,35.00)
\multiput(30.33,25.00)(0.36,0.12){84}{\color{green}\line(1,0){0.36}}
%\end
%\put(60.33,15.00){\circle{0.00}}
%\put(60.33,15.00){\circle{2.00}}
%\put(45.33,10.00){\circle{2.00}}
%\put(30.33,5.00){\circle{2.00}}
%\put(15.33,10.00){\circle{2.00}}
%\put(0.33,16.00){\circle{0.00}}
%\put(0.33,15.00){\circle{2.00}}
\put(0.33,35.00){\color{red}\circle{2.00}}
\put(15.33,30.00){\color{red}\circle{2.00}}
\put(30.33,25.00){\color{red}\circle{2.00}}
\put(30.33,25.00){\color{green}\circle{3.80}}
\put(45.33,30.00){\color{green}\circle{2.00}}
\put(60.33,35.00){\color{green}\circle{2.00}}
\put(60.33,31.00){\makebox(0,0)[cc]{$K$}}
\put(45.33,26.00){\makebox(0,0)[cc]{$D$}}
\put(30.33,30.00){\makebox(0,0)[cc]{$A$}}
\put(15.33,26.00){\makebox(0,0)[cc]{$C$}}
\put(0.33,31.00){\makebox(0,0)[cc]{$B$}}
\bezier{24}(0.00,20.00)(0.00,17.33)(3.00,17.33)
\bezier{28}(3.00,17.33)(10.00,17.00)(10.00,17.00)
\bezier{32}(10.00,17.00)(15.00,16.00)(15.00,13.33)
\bezier{24}(30.00,20.00)(30.00,17.33)(27.00,17.33)
\bezier{28}(27.00,17.33)(20.00,17.00)(20.00,17.00)
\bezier{32}(20.00,17.00)(15.00,16.00)(15.00,13.33)
%\put(15.00,10.33){\makebox(0,0)[cc]{context}}
\put(15.00,5.33){\color{red}\makebox(0,0)[cc]{$\{B,C,A\}$}}
\bezier{24}(60.00,20.00)(60.00,17.33)(57.00,17.33)
\bezier{28}(57.00,17.33)(50.00,17.00)(50.00,17.00)
\bezier{32}(50.00,17.00)(45.00,16.00)(45.00,13.33)
\bezier{24}(30.00,20.00)(30.00,17.33)(33.00,17.33)
\bezier{28}(33.00,17.33)(40.00,17.00)(40.00,17.00)
\bezier{32}(40.00,17.00)(45.00,16.00)(45.00,13.33)
%\put(45.00,10.33){\makebox(0,0)[cc]{context}}
\put(45.00,5.33){\color{green}\makebox(0,0)[cc]{$\{A,D,K\}$}}
\end{picture}
&
\qquad
&
%TexCad Options
%\grade{\off}
%\emlines{\off}
%\beziermacro{\on}
%\reduce{\on}
%\snapping{\off}
%\quality{2.00}
%\graddiff{0.01}
%\snapasp{1}
%\zoom{1.00}
\unitlength 0.60mm
\thicklines %\linethickness{0.4pt}
\begin{picture}(51.37,10.00)
\put(0.00,25.00){\color{red}\circle{2.75}}
\put(30.00,25.00){\color{green}\circle{2.75}}
\put(1.33,25.00){\line(1,0){27.33}}
\put(15.00,30.00){\makebox(0,0)[cc]{$A$}}
%\put(30.00,20.00){\makebox(0,0)[cc]{context}}
\put(30.00,15.00){\color{green}\makebox(0,0)[cc]{$\{A,D,K\}$}}
%\put(0.00,20.00){\makebox(0,0)[cc]{context}}
\put(0.00,15.00){\color{red}\makebox(0,0)[cc]{$\{B,C,A\}$}}
\end{picture}
\\              \\
${\mathbb R}^3$&&Greechie diagram&&Tkadlec diagram\\
\end{tabular}
%\end{center}
 }


\frame{
\frametitle{Example III: $L_{12}$ cntd.}
The operators  $B,C,A$ and $D,K,A$ can be identified with the projectors corresponding
to the two bases
$$
\begin{array}{lcl}
B_{B-C-A}&=&
\{
(1,0,0)^T,
(0,1,0)^T,
(0,0,1)^T
\}
,
\\
B_{D-K-A}&=&
\{
(\cos \varphi , \sin \varphi ,0)^T,
(-\sin \varphi ,\cos \varphi , 0)^T,
(0,0,1)^T
\},
\end{array}
\label{e-vaxjo1}
$$
(the superscript ``$T$'' indicates transposition).
Their matrix representation is the  dyadic product of every vector with itself.

Physically, the union of contexts $\{B,C,A\}$ and $\{D,K,A\}$ interlinked along $A$ does not have any direct
operational meaning; only a single context can be measured along a single quantum at a time;
the other being irretrievably lost if no reconstruction of the original state is possible.
Thus, in a direct way, testing the value of observable $A$ against different
contexts $\{B,C,A\}$ and $\{D,K,A\}$ is metaphysical.



}


\frame{
\frametitle{Example III: $L_{12}$ cntd.}
  It is, however, possible to counterfactually retrieve information
about the two different contexts of a single quantum indirectly
by considering a singlet state
$
\vert \Psi_2 \rangle
= ({1/ \sqrt{3}})(
\vert + -\rangle
+
\vert - +\rangle
-
\vert 0 0\rangle
)$
via the ``explosion view'' Einstein-Podolsky-Rosen type of argument.
Since the state is form invariant with respect to variations of the measurement angle
and at the same time satisfies the uniqueness property,
one may retrieve the first context
$\{B,C,A\}$ from the first quantum
and the second context $\{D,K,A\}$ from the second quantum.
(This is a standard procedure in Bell type arguments with two spin one-half quanta.)


}


\frame{
\frametitle{Other examples}
More tightly interlinked contexts such as
$\{A,B,C\}-\{C,D,E\}-\{E,F,A\}$, whose Greechie diagram is a triangle with the edges $A$, $C$ and $E$,
or
$\{A,B,C\}-\{C,D,E\}-\{E,F,G\}-\{G,H,A\}$, whose Greechie diagram is a quadrangle with the edges $A$, $C$, $E$ and $G$,
cannot be represented in Hilbert space and thus have no realization in quantum logics.
The five contexts
$\{A,B,C\}-\{C,D,E\}-\{E,F,G\}-\{G,H,I\}-\{I,J,A\}$
whose Greechie diagrams is a pentagon with the edges $A$, $C$, $E$, $G$ and $I$ have realizations in ${\mathbb R}^3$.



}

\subsection{Quantum probability theory}

\frame{
\frametitle{Quantum probability theory}

\begin{itemize}

\item<1->  The quantum propositional calculus
(the algebra of quantum logical propositions from 3-dimensional Hilbert space onwards) does not allow a two-valued state
interpretable as truth assignment (Specker 1960, Kochen-Specker theorem 1967).



\item<1->  Hence the classical strategy of basing probabilities on the convex sum of two-valued probability measures fails.

\item<1->  Gleason's theorem (Gleason 1957) adopts another strategy --- analoguous to the pasting construction of Hilbert lattices ---
by basing quantum probabilities on quasi-classical probabilities
{\em within} blocks or contexts.

\end{itemize}

}







\subsection{``Scarcity'' of two--valued states}

\frame{
\frametitle{Principle of explosion}

\begin{itemize}
\item<1->
The  {\em principle of explosion}:  {\it ``ex falso quodlibet,''} or {\it ``contradictione sequitur quodlibet''}
amounts to ``anything follows from a contradiction.''
\item<1->
Due to the pasting construction of Hilbert lattices, the principle of explosion holds also in quantum logic.
\end{itemize}

}





\frame{
\frametitle{Example I: The one--zero rule}

\begin{center}
\begin{tabular}{ccc}
%TeXCAD Picture [1.pic]. Options:
%\grade{\on}
%\emlines{\off}
%\epic{\off}
%\beziermacro{\on}
%\reduce{\on}
%\snapping{\off}
%\quality{8.00}
%\graddiff{0.01}
%\snapasp{1}
%\zoom{5.6569}
\unitlength .3mm % = 1.42pt
\linethickness{0.8pt}
\ifx\plotpoint\undefined\newsavebox{\plotpoint}\fi % GNUPLOT compatibility
\begin{picture}(120.92,114.73)(0,0)
%\emline(86.57,102.14)(111.57,58.64)
\multiput(86.57,102.14)(.11961722,-.20813397){209}{\line(0,-1){.20813397}}
%\end
%\emline(86.57,15.14)(111.57,58.64)
\multiput(86.57,15.14)(.11961722,.20813397){209}{\line(0,1){.20813397}}
%\end
%\emline(36.65,102.14)(11.65,58.64)
\multiput(36.65,102.14)(-.11961722,-.20813397){209}{\line(0,-1){.20813397}}
%\end
%\emline(36.65,15.14)(11.65,58.64)
\multiput(36.65,15.14)(-.11961722,.20813397){209}{\line(0,1){.20813397}}
%\end
\put(86.57,101.89){\line(-1,0){50}}
\put(86.57,15.39){\line(-1,0){50}}
\put(86.46,101.94){\circle{4}}
\put(86.46,15.34){\circle{4}}
\put(111.39,58.63){\circle{4}}
\put(11.74,58.63){\circle{4}}
\put(61.77,58.63){\circle{4}}
\put(36.52,101.94){\circle{4}}
\put(61.62,101.94){\circle{4}}
\put(61.62,15.44){\circle{4}}
\put(97.68,82.85){\circle{4}}
\put(25.71,82.85){\circle{4}}
\put(98.74,36.35){\circle{4}}
\put(24.65,36.35){\circle{4}}
\put(36.52,15.34){\circle{4}}
\put(61.69,101.82){\line(0,-1){86.27}}
\put(30.41,2.65){\makebox(0,0)[cc]{$A$}}
\put(61.87,2.3){\makebox(0,0)[cc]{$B$}}
\put(91.93,2.48){\makebox(0,0)[cc]{$C$}}
\put(110.84,30.94){\makebox(0,0)[cc]{$D$}}
\put(120.92,57.98){\makebox(0,0)[lc]{$E$}}
\put(108.41,88.92){\makebox(0,0)[cc]{$F$}}
\put(91.93,114.2){\makebox(0,0)[cc]{$G$}}
\put(61.7,114.73){\makebox(0,0)[cc]{$H$}}
\put(30.41,114.02){\makebox(0,0)[cc]{$I$}}
\put(13.56,87.86){\makebox(0,0)[cc]{$J$}}
\put(1.77,57.98){\makebox(0,0)[rc]{$ K$}}
\put(14.67,30.05){\makebox(0,0)[rc]{$L$}}
\put(67.88,55.51){\makebox(0,0)[cc]{$M$}}
\put(71.34,9.19){\makebox(0,0)[cc]{$a$}}
\put(107.91,40.35){\makebox(0,0)[cc]{$b$}}
\put(98.53,95.32){\makebox(0,0)[cc]{$c$}}
\put(54.46,108.01){\makebox(0,0)[cc]{$d$}}
\put(15.03,78.14){\makebox(0,0)[cc]{$e$}}
\put(21.56,27.06){\makebox(0,0)[cc]{$f$}}
\put(67.88,75.51){\makebox(0,0)[cc]{$g$}}
\end{picture}
&
$\qquad$
&

%TeXCAD Picture [1.pic]. Options:
%\grade{\on}
%\emlines{\off}
%\epic{\off}
%\beziermacro{\on}
%\reduce{\on}
%\snapping{\off}
%\quality{8.00}
%\graddiff{0.01}
%\snapasp{1}
%\zoom{5.6569}
\unitlength .3mm % = 1.42pt
\linethickness{0.8pt}
\ifx\plotpoint\undefined\newsavebox{\plotpoint}\fi % GNUPLOT compatibility
\begin{picture}(120.92,114.2)(0,0)
%\emline(86.57,102.14)(111.57,58.64)
\multiput(86.57,102.14)(.11961722,-.20813397){209}{\line(0,-1){.20813397}}
%\end
%\emline(86.57,15.14)(111.57,58.64)
\multiput(86.57,15.14)(.11961722,.20813397){209}{\line(0,1){.20813397}}
%\end
%\emline(36.65,102.14)(11.65,58.64)
\multiput(36.65,102.14)(-.11961722,-.20813397){209}{\line(0,-1){.20813397}}
%\end
%\emline(36.65,15.14)(11.65,58.64)
\multiput(36.65,15.14)(-.11961722,.20813397){209}{\line(0,1){.20813397}}
%\end
\put(86.57,101.89){\line(-1,0){50}}
\put(86.57,15.39){\line(-1,0){50}}
\put(86.46,101.94){\circle*{4}}
\put(86.46,15.34){\circle*{4}}
\put(111.39,58.63){\circle*{4}}
\put(11.74,58.63){\circle*{4}}
\put(36.52,101.94){\circle*{4}}
\put(36.52,15.34){\circle*{4}}
\put(61.77,58.63){\circle*{4}}
%\emline(86.44,102)(36.59,15.56)
\multiput(86.44,102)(-.119831731,-.207788462){416}{\line(0,-1){.207788462}}
%\end
%-
\put(30.41,2.65){\makebox(0,0)[cc]{$a$}}
\put(91.93,2.48){\makebox(0,0)[cc]{$b$}}
\put(120.92,57.98){\makebox(0,0)[lc]{$c$}}
\put(91.93,114.2){\makebox(0,0)[cc]{$d$}}
\put(30.41,114.02){\makebox(0,0)[cc]{$e$}}
\put(1.77,57.98){\makebox(0,0)[rc]{$f$}}
\put(67.88,55.51){\makebox(0,0)[cc]{$g$}}
\end{picture}
\\
a)&&b)
\end{tabular}
\end{center}
{\footnotesize
Configuration of observables in three-dimensional Hilbert space implying
that whenever $K$ is true, $E$ must be false.
The seven interconnected contexts
$a=\{A,B,C\}$,
$b=\{C,D,E\}$,
$c=\{E,F,G\}$,
$d=\{G,H,I\}$,
$e=\{I,J,K\}$,
$f=\{K,L,A\}$,
$g=\{B,H,M\}$,
consist of the 13 projectors associated with the one dimensional subspaces spanned by
$ A= ( 1,\sqrt{2},-1)      $,
$ B= ( 1,0,1)   $,
$ C= ( -1,\sqrt{2},1)    $,
$ D= ( -1,\sqrt{2},-3)    $,
$  E=( \sqrt{2},1,0) $,
$  F=( 1,-\sqrt{2},-3)            $,
$  G=( -1,\sqrt{2},-1)           $,
$  H=( 1,0,-1)    $,
$  I=( 1,\sqrt{2},1)   $,
$ J= ( 1,\sqrt{2},-3)     $,
$ K=( \sqrt{2},-1,0)    $,
$ L=( 1,\sqrt{2},3)     $,
$ M=(0,1,0)    $.
%
%   kp[a1_, a2_, a3_, b1_, b2_, b3_] =   {a2  b3 - a3 b2, a3 b1 - a1 b3, a1 b2 - a2 b1}
%
}
}

\frame{
\frametitle{Example I: The one--one rule}

\begin{center}
\begin{tabular}{c}
%TeXCAD Picture [1.pic]. Options:
%\grade{\on}
%\emlines{\off}
%\epic{\off}
%\beziermacro{\on}
%\reduce{\on}
%\snapping{\off}
%\quality{8.00}
%\graddiff{0.01}
%\snapasp{1}
%\zoom{4.0000}
\unitlength .30mm % = 1.42pt
\linethickness{0.8pt}
\ifx\plotpoint\undefined\newsavebox{\plotpoint}\fi % GNUPLOT compatibility
\begin{picture}(320.85,118.44)(0,0)
%\emline(105.32,33.64)(61.82,8.64)
\multiput(105.32,33.64)(-.20813397,-.11961722){209}{\line(-1,0){.20813397}}
%\end
%\emline(308.26,33.64)(264.76,8.64)
\multiput(308.26,33.64)(-.20813397,-.11961722){209}{\line(-1,0){.20813397}}
%\end
%\emline(18.32,33.64)(61.82,8.64)
\multiput(18.32,33.64)(.20813397,-.11961722){209}{\line(1,0){.20813397}}
%\end
%\emline(221.26,33.64)(264.76,8.64)
\multiput(221.26,33.64)(.20813397,-.11961722){209}{\line(1,0){.20813397}}
%\end
%\emline(105.32,83.56)(61.82,108.56)
\multiput(105.32,83.56)(-.20813397,.11961722){209}{\line(-1,0){.20813397}}
%\end
%\emline(308.26,83.56)(264.76,108.56)
\multiput(308.26,83.56)(-.20813397,.11961722){209}{\line(-1,0){.20813397}}
%\end
%\emline(18.32,83.56)(61.82,108.56)
\multiput(18.32,83.56)(.20813397,.11961722){209}{\line(1,0){.20813397}}
%\end
%\emline(221.26,83.56)(264.76,108.56)
\multiput(221.26,83.56)(.20813397,.11961722){209}{\line(1,0){.20813397}}
%\end
\put(105.07,33.64){\line(0,1){50}}
\put(308.01,33.64){\line(0,1){50}}
\put(18.57,33.64){\line(0,1){50}}
\put(221.51,33.64){\line(0,1){50}}
\put(105.12,33.75){\circle{4}}
\put(308.06,33.75){\circle{4}}
\put(18.52,33.75){\circle{4}}
\put(221.46,33.75){\circle{4}}
\put(61.81,8.82){\circle{4}}
\put(264.75,8.82){\circle{4}}
\put(61.81,108.47){\circle{4}}
\put(264.75,108.47){\circle{4}}
\put(61.81,58.44){\circle{4}}
\put(264.75,58.44){\circle{4}}
\put(105.12,83.69){\circle{4}}
\put(308.06,83.69){\circle{4}}
\put(105.12,58.59){\circle{4}}
\put(308.06,58.59){\circle{4}}
\put(18.62,58.59){\circle{4}}
\put(221.56,58.59){\circle{4}}
\put(86.03,22.53){\circle{4}}
\put(288.97,22.53){\circle{4}}
\put(86.03,94.5){\circle{4}}
\put(288.97,94.5){\circle{4}}
\put(39.53,21.47){\circle{4}}
\put(242.47,21.47){\circle{4}}
\put(39.53,95.56){\circle{4}}
\put(242.47,95.56){\circle{4}}
\put(18.52,83.69){\circle{4}}
\put(221.46,83.69){\circle{4}}
\put(163.21,58.69){\circle{4}}
\put(105,58.52){\line(-1,0){86.27}}
\put(307.94,58.52){\line(-1,0){86.27}}
\put(5.83,89.8){\makebox(0,0)[]{$A$}}
\put(208.77,89.8){\makebox(0,0)[]{$A'$}}
\put(5.48,58.34){\makebox(0,0)[]{$B$}}
\put(208.42,58.34){\makebox(0,0)[]{$B'$}}
\put(5.66,28.28){\makebox(0,0)[]{$C$}}
\put(208.6,28.28){\makebox(0,0)[]{$C'$}}
\put(34.12,9.37){\makebox(0,0)[]{$D$}}
\put(237.06,9.37){\makebox(0,0)[]{$D'$}}
\put(61.16,-.71){\makebox(0,0)[t]{$E$}}
\put(264.1,-.71){\makebox(0,0)[t]{$E'$}}
\put(92.1,11.8){\makebox(0,0)[]{$F$}}
\put(295.04,11.8){\makebox(0,0)[]{$F'$}}
\put(117.38,28.28){\makebox(0,0)[]{$G$}}
\put(320.32,28.28){\makebox(0,0)[]{$G'$}}
\put(117.91,58.51){\makebox(0,0)[]{$H$}}
\put(320.85,58.51){\makebox(0,0)[]{$H'$}}
\put(117.2,89.8){\makebox(0,0)[]{$I$}}
\put(320.14,89.8){\makebox(0,0)[]{$I'$}}
\put(91.04,106.65){\makebox(0,0)[]{$J$}}
\put(293.98,106.65){\makebox(0,0)[]{$J'$}}
\put(61.16,118.44){\makebox(0,0)[b]{$ K$}}
\put(264.1,118.44){\makebox(0,0)[b]{$ K'$}}
\put(33.23,105.54){\makebox(0,0)[b]{$L$}}
\put(236.17,105.54){\makebox(0,0)[b]{$L'$}}
\put(58.69,52.33){\makebox(0,0)[]{$M$}}
\put(261.63,52.33){\makebox(0,0)[]{$M'$}}
\put(12.37,48.87){\makebox(0,0)[]{$a$}}
\put(215.31,48.87){\makebox(0,0)[]{$a'$}}
\put(43.53,12.3){\makebox(0,0)[]{$b$}}
\put(246.47,12.3){\makebox(0,0)[]{$b'$}}
\put(98.5,21.68){\makebox(0,0)[]{$c$}}
\put(301.44,21.68){\makebox(0,0)[]{$c'$}}
\put(111.19,65.75){\makebox(0,0)[]{$d$}}
\put(314.13,65.75){\makebox(0,0)[]{$d'$}}
\put(81.32,105.18){\makebox(0,0)[]{$e$}}
\put(284.26,105.18){\makebox(0,0)[]{$e'$}}
\put(30.24,98.65){\makebox(0,0)[]{$f$}}
\put(233.18,98.65){\makebox(0,0)[]{$f'$}}
\put(38.69,52.33){\makebox(0,0)[]{$g$}}
\put(281.63,52.33){\makebox(0,0)[]{$g'$}}
%\emline(61.75,8.75)(264.5,108.5)
\multiput(61.75,8.75)(.243687933,.1198908573){832}{\line(1,0){.243687933}}
%\end
%\emline(61.75,108.25)(264.75,8.75)
\multiput(61.75,108.25)(.2445763352,-.1198785485){830}{\line(1,0){.2445763352}}
%\end
\put(163,46.25){\makebox(0,0)[cc]{$N$}}
\put(182.75,78.25){\makebox(0,0)[cc]{$h$}}
\put(182.5,41.25){\makebox(0,0)[cc]{$i$}}
\qbezier(61.75,58.5)(104.62,30)(163,58.5)
\qbezier(264.25,58.5)(221.37,87)(163,58.5)
\put(241.75,76.75){\makebox(0,0)[cc]{$j$}}
\end{picture}
\end{tabular}
\end{center}
{\footnotesize
Configuration of observables implying that the occurrences of $K$ and $K'$ coincide.
%
Greechie diagram representing atoms by points, and  contexts by maximal smooth, unbroken curves.
The coordinates of the ``primed'' points $A'$--$M'$ are obtained by interchanging the first and the
second components of the unprimed coordinates $A$--$M$ ;
and $N=(0,0,1)$.
The two contexts $h$ and $i$ linking the primed with the unprimed observables allow the following argument:
Whenever $K$ occurs, then by the one-zero rule $E$ cannot occur;
moreover $N$ cannot occur, hence $K'$ must occur.
Conversely, by symmetry whenever $K'$ occurs, $K$ must occur.
}

}

\renewcommand{\baselinestretch}{0.75}

\subsection{Kochen-Specker Constructions}
\frame{
\frametitle{Kochen-Specker Construction I}
{\small
\begin{center}
\begin{tabular}{cc}
 %\includegraphics[width=50mm]{cabello}
%TeXCAD Picture [1.pic]. Options:
%\grade{\on}
%\emlines{\off}
%\epic{\off}
%\beziermacro{\on}
%\reduce{\on}
%\snapping{\off}
%\quality{8.000}
%\graddiff{0.010}
%\snapasp{1}
%\zoom{5.6569}
\unitlength .3mm % = 1.423pt
\thicklines %\linethickness{0.8pt}
\ifx\plotpoint\undefined\newsavebox{\plotpoint}\fi % GNUPLOT compatibility
\begin{picture}(134.09,125.99)(0,0)

%\emline(86.39,101.96)(111.39,58.46)
\multiput(86.39,101.96)(.119617225,-.208133971){209}{{\color{green}\line(0,-1){.208133971}}}
%\end
%\emline(86.39,14.96)(111.39,58.46)
\multiput(86.39,14.96)(.119617225,.208133971){209}{{\color{red}\line(0,1){.208133971}}}
%\end
%\emline(36.47,101.96)(11.47,58.46)
\multiput(36.47,101.96)(-.119617225,-.208133971){209}{{\color{yellow}\line(0,-1){.208133971}}}
%\end
%\emline(36.47,14.96)(11.47,58.46)
\multiput(36.47,14.96)(-.119617225,.208133971){209}{{\color{magenta}\line(0,1){.208133971}}}
%\end
\color{blue}\put(86.39,15.21){\color{blue}\line(-1,0){50}}
\put(86.39,101.71){\color{violet}\line(-1,0){50}}
%
\put(36.34,15.16){\color{magenta}\circle{6}}
\put(36.34,15.16){\color{blue}\circle{4}}
\put(52.99,15.16){\color{blue}\circle{4}}
\put(52.99,15.16){\color{cyan}\circle{6}}
\put(69.68,15.16){\color{blue}\circle{4}}
\put(69.68,15.16){\color{orange}\circle{6}}
\put(86.28,15.16){\color{blue}\circle{4}}
\put(86.28,15.16){\color{red}\circle{6}}
%
\put(93.53,27.71){\color{red}\circle{4}}
\put(93.53,27.71){\color{orange}\circle{6}}
\put(102.37,43.44){\color{red}\circle{4}}
\put(102.37,43.44){\color{olive}\circle{6}}
\put(111.21,58.45){\color{red}\circle{4}}
\color{green}\put(111.21,58.45){\circle{6}}
%
\put(102.37,73.47){\color{green}\circle{4}}
\put(102.37,73.47){\color{olive}\circle{6}}
\put(93.53,89.21){\color{green}\circle{4}}
\put(93.53,89.21){\color{cyan}\circle{6}}
\put(86.28,101.76){\color{green}\circle{4}}
\put(86.28,101.76){\color{violet}\circle{6}}
%
\put(69.68,101.76){\color{violet}\circle{4}}
\put(69.68,101.76){\color{cyan}\circle{6}}
\put(52.99,101.76){\color{violet}\circle{4}}
\put(52.99,101.76){\color{orange}\circle{6}}
\put(36.34,101.76){\color{violet}\circle{4}}
\put(36.34,101.76){\color{yellow}\circle{6}}
%
\put(29.24,89.21){\color{yellow}\circle{4}}
\put(29.24,89.21){\color{orange}\circle{6}}
\put(20.4,73.47){\color{yellow}\circle{4}}
\put(20.4,73.47){\color{olive}\circle{6}}
\put(11.56,58.45){\color{yellow}\circle{4}}
\put(11.56,58.45){\color{magenta}\circle{6}}

\put(20.4,43.44){\color{magenta}\circle{4}}
\put(20.4,43.44){\color{olive}\circle{6}}
\put(29.24,27.71){\color{magenta}\circle{4}}
\put(29.24,27.71){\color{cyan}\circle{6}}

\color{cyan}
\qbezier(29.2,27.73)(23.55,-5.86)(52.99,15.24)
\qbezier(29.2,27.88)(36.93,75)(69.63,101.91)
\qbezier(52.69,15.24)(87.47,40.96)(93.72,89.27)
\qbezier(93.72,89.27)(98.4,125.99)(69.49,102.06)
\color{orange}
\qbezier(93.57,27.73)(99.22,-5.86)(69.78,15.24)
\qbezier(93.57,27.88)(85.84,75)(53.13,101.91)
\qbezier(70.08,15.24)(35.3,40.96)(29.05,89.27)
\qbezier(29.05,89.27)(24.37,125.99)(53.28,102.06)
\color{olive}
\qbezier(20.15,73.72)(-11.67,58.52)(20.15,43.31)
\qbezier(20.33,73.72)(61.34,93.16)(102.36,73.72)
\qbezier(102.36,73.72)(134.09,58.52)(102.53,43.31)
\qbezier(102.53,43.31)(60.99,23.43)(20.15,43.49)
{\color{black}
\put(30.41,114.02){\makebox(0,0)[cc]{$M$}}
\put(30.41,2.65){\makebox(0,0)[cc]{$A$}}
\put(52.68,114.38){\makebox(0,0)[cc]{$L$}}
\put(52.68,2.3){\makebox(0,0)[cc]{$B$}}
\put(91.93,114.2){\makebox(0,0)[cc]{$J$}}
\put(91.93,2.48){\makebox(0,0)[cc]{$D$}}
\put(69.65,114.38){\makebox(0,0)[cc]{$K$}}
\put(73.65,2.3){\makebox(0,0)[cc]{$C$}}
\put(103.24,94.22){\makebox(0,0)[cc]{$I$}}
\put(17.45,94.22){\makebox(0,0)[cc]{$ N$}}
\put(106.24,22.45){\makebox(0,0)[cc]{$E$}}
\put(17.45,22.45){\makebox(0,0)[cc]{$ R$}}
\put(115.13,77.96){\makebox(0,0)[cc]{$H$}}
\put(8.55,77.96){\makebox(0,0)[cc]{$ O$}}
\put(115.13,38.72){\makebox(0,0)[cc]{$F$}}
\put(10.55,38.72){\makebox(0,0)[cc]{$ Q$}}
\put(120.92,57.98){\makebox(0,0)[l]{$ G$}}
\put(1.77,57.98){\makebox(0,0)[rc]{$  P$}}
}
\put(61.341,9.192){\color{blue}\makebox(0,0)[cc]{$a$}}
\put(102.883,35.355){\color{red}\makebox(0,0)[cc]{$b$}}
\put(102.53,84.322){\color{green}\makebox(0,0)[cc]{$c$}}
\put(60.457,108.01){\color{violet}\makebox(0,0)[cc]{$d$}}
\put(18.031,84.145){\color{yellow}\makebox(0,0)[cc]{$e$}}
\put(18.561,33.057){\color{magenta}\makebox(0,0)[cc]{$f$}}
\put(61.341,39.774){\color{olive}\makebox(0,0)[cc]{$g$}}
\put(72.124,67.882){\color{orange}\makebox(0,0)[cc]{$h$}}
\put(48.79,67.705){\color{cyan}\makebox(0,0)[cc]{$i$}}
\end{picture}
&
%TeXCAD Picture [1.pic]. Options:
%\grade{\on}
%\emlines{\off}
%\epic{\off}
%\beziermacro{\on}
%\reduce{\on}
%\snapping{\off}
%\quality{8.000}
%\graddiff{0.010}
%\snapasp{1}
%\zoom{5.6569}
\unitlength .3mm % = 1.423pt
\thicklines %\linethickness{0.8pt}
\ifx\plotpoint\undefined\newsavebox{\plotpoint}\fi % GNUPLOT compatibility
\begin{picture}(119.854,112.606)(0,0)
%\emline(86.567,102.137)(111.567,58.637)
\multiput(86.567,102.137)(.119617225,-.208133971){209}{\line(0,-1){.208133971}}
%\end
%\emline(86.567,15.137)(111.567,58.637)
\multiput(86.567,15.137)(.119617225,.208133971){209}{\line(0,1){.208133971}}
%\end
%\emline(36.647,102.137)(11.647,58.637)
\multiput(36.647,102.137)(-.119617225,-.208133971){209}{\line(0,-1){.208133971}}
%\end
%\emline(36.647,15.137)(11.647,58.637)
\multiput(36.647,15.137)(-.119617225,.208133971){209}{\line(0,1){.208133971}}
%\end
%\emline(86.087,73.357)(86.447,101.997)
\multiput(86.087,73.357)(.12,9.54667){3}{\line(0,1){9.54667}}
%\end
%\emline(86.267,73.537)(111.187,58.687)
\multiput(86.267,73.537)(.200967742,-.119758065){124}{\line(1,0){.200967742}}
%\end
%\emline(86.087,73.357)(11.667,58.517)
\multiput(86.087,73.357)(-.60016129,-.119677419){124}{\line(-1,0){.60016129}}
%\end
%\emline(86.087,73.187)(36.237,15.207)
\multiput(86.087,73.187)(-.1198317308,-.139375){416}{\line(0,-1){.139375}}
%\end
%\emline(36.951,15.376)(61.341,29.696)
\multiput(36.951,15.376)(.20325,.119333333){120}{\line(1,0){.20325}}
%\end
%\emline(61.341,29.696)(86.801,15.376)
\multiput(61.341,29.696)(.212166667,-.119333333){120}{\line(1,0){.212166667}}
%\end
%\emline(60.991,29.696)(36.591,101.296)
\multiput(60.991,29.696)(-.119607843,.350980392){204}{\line(0,1){.350980392}}
%\end
%\emline(61.161,29.526)(86.801,101.646)
\multiput(61.161,29.526)(.119813084,.337009346){214}{\line(0,1){.337009346}}
%\end
\put(11.844,58.69){\line(3,2){22.804}}
%\emline(34.648,73.892)(36.593,101.823)
\multiput(34.648,73.892)(.1144118,1.643){17}{\line(0,1){1.643}}
%\end
%\emline(86.62,15.38)(34.472,73.716)
\multiput(86.62,15.38)(-.1198804598,.1341057471){435}{\line(0,1){.1341057471}}
%\end
%\emline(34.472,73.716)(111.369,58.513)
\multiput(34.472,73.716)(.605488189,-.119708661){127}{\line(1,0){.605488189}}
%\end
\put(86.567,101.887){\line(-1,0){50}}
\put(86.567,15.387){\line(-1,0){50}}
%
\put(86.457,101.937){\color{red}\circle{3.4}}
\put(86.457,101.937){\color{red}\circle{1.3}}
\put(86.457,15.337){\color{violet}\circle{3.4}}
\put(86.457,15.337){\color{violet}\circle{1.3}}
\put(85.927,73.127){\color{olive}\circle{3.4}}
\put(85.927,73.127){\color{olive}\circle{1.3}}
\put(34.486,73.849){\color{cyan}\circle{3.4}}
\put(34.486,73.849){\color{cyan}\circle{1.3}}
\put(111.387,58.627){\color{green}\circle{3.4}}
\put(111.387,58.627){\color{green}\circle{1.3}}
\put(11.737,58.627){\color{magenta}\circle{3.4}}
\put(11.737,58.627){\color{magenta}\circle{1.3}}
\put(36.517,101.937){\color{blue}\circle{3.4}}
\put(36.517,101.937){\color{blue}\circle{1.3}}
\put(36.517,15.337){\color{yellow}\circle{3.4}}
\put(36.517,15.337){\color{yellow}\circle{1.3}}
\put(60.941,29.586){\color{orange}\circle{3.4}}
\put(60.941,29.586){\color{orange}\circle{1.3}}
%
\put(35.885,112.606){\color{blue}\makebox(0,0)[cc]{$a$}}
\put(86.09,111.722){\color{red}\makebox(0,0)[cc]{$b$}}
\put(119.854,55.861){\color{green}\makebox(0,0)[cc]{$c$}}
\put(86.266,6.364){\color{violet}\makebox(0,0)[cc]{$d$}}
\put(35.885,5.834){\color{yellow}\makebox(0,0)[cc]{$e$}}
\put(3.359,58.689){\color{magenta}\makebox(0,0)[cc]{$f$}}
\put(60.634,22.45){\color{orange}\makebox(0,0)[cc]{$h$}}
\put(93.161,77.074){\color{olive}\makebox(0,0)[cc]{$g$}}
\put(28.814,76.544){\color{cyan}\makebox(0,0)[cc]{$i$}}
\end{picture}
\end{tabular}
\end{center}
{ \footnotesize
Greechie \& Tkadlec diagrams of a ``short'' proof of the Kochen-Specker theorem by {\it Cabello et al.} (drawing from  Adan Cabello,  Phys. Rev. Lett. 101 (2008) 210401;
arXiv:0808.2456) in four-dimensional real vector space.
The nine tightly interconnected contexts
$a$--$i$
consist of the 18 projectors associated with the one dimensional subspaces.
Every observable proposition occurs in exactly two contexts.
Thus, in an enumeration of the four observable propositions of each of the nine contexts,
there appears to be an {\em even} number of true propositions.
Yet, as there is an odd number of contexts,
there should be an {\em odd} number (actually nine) of true propositions.
}
}
}

\frame{
\frametitle{Kochen-Specker Construction II}

\includegraphics[width=100mm]{2008-ks-3}

}


\frame{
\frametitle{Kochen-Specker Construction II cntd.}
{\small
Let us prove that there is no two-valued probability measure or ``red-green-green'' coloring.
Due to the symmetry of the problem, we can  choose a particular
coordinate axis such that, without loss of generality,
$P(100)=1$.
Furthermore, we may assume (case 1) that
$P(21\bar{1}) = 1$.
It immediately follows that $P(001) = P(010) =
P(102) = P(\bar{1}20) = 0$.
A second glance shows that $P(20\bar{1}) = 1$, $P(1\bar{1}2) = P(112) = 0$.

Let us now suppose (case 1a)
 that $P(201) = 1$. Then  we obtain $P(\bar{1}12) = P(\bar{1}\bar{1}2)
= 0$. We are forced to accept $P(110)
= P(1\bar{1}0)  = 1$ ---  a contradiction, since $(110)$ and
$(1\bar{1}0)$ are
orthogonal to each other and lie on one edge.

Hence we have to assume (case 1b) that $P(201) = 0$.
This gives immediately
$P(\bar{1}02)=1$ and
$P(211) =0$.  Since $P(01\bar{1})=0$, we obtain $P(2\bar{1}\bar{1})=1$ and thus
$P(120)=0$.
This requires $P(2\bar{1}0)=1$ and therefore $P(12\bar{1})=P(121)=0$.
Observe that $P(210) = 1$, and thus $P(\bar{1}2\bar{1}) = P(\bar{1}21) = 0$.
In the following step, we notice that $P(10\bar{1}) =
P(101) = 1$ ---  a contradiction,
since $(101)$ and $(10\bar{1})$ are
orthogonal to each other and lie on one edge.


Thus we are forced to assume (case 2) that
$P(2\bar{1}1) = 1$. There is no third alternative, since $P(011)=0$ due to the
orthogonality with $(100)$. Now we can repeat the argument for case 1 in
its mirrored form.
}

}




\frame{
\frametitle{Contextuality and its alternatives}

{\small

\begin{itemize}\item[ ]
\begin{beamerboxesrounded}[scheme=alert2,shadow=true]{(i) Abandonment of classical omniscience:}
It is wrong to assume that
all observables which could in principle (``potentially'') have been measured also co--exist,
irrespective of whether or not they have or even could have been actually measured.
Realism might still be assumed for a {\em single} context, in particular the one in which the system was prepared;
\end{beamerboxesrounded}

\item[(ii)]   Abandonment of realism: It is wrong to assume that physical entities exist
even without being experienced by any finite mind.
Quite literary, with this assumption, the proofs of KS and similar decay into thin air because
there are no counterfactuals or unobserved physical observables
or inferred (rather than measured) elements of physical reality.


\item[(iii)] Contextuality; i.e., the abandonment of context independence of measurement outcomes:
It is wrong to assume that the
result of an observation is independent
not only of the state of the system
but also of the complete disposition  of the apparatus ---
possible test in the $\{A,B,C\}-\{C,D,E\}$ system of observables with an
explosion type EPR setup \& a singlet state of two spin--one particles.

\item[$\ldots$]
\end{itemize}

}
}

\frame{
\frametitle{Historical notion of contextuality envisioned by Bohr \& Bell}

\begin{itemize}
\item<+->
Bohr 1949:
{\em ``the impossibility of any sharp separation
between the behavior of atomic objects and the interaction with the measuring instruments which serve to define
the conditions under which the phenomena appear.''}

\item<+->
Bell 1966): the {\em ``$\ldots$
result of an observation may reasonably depend
not only on the state of the system  $\ldots$
but also on the complete disposition  of the apparatus.''}

\item<+->
Stated pointedly, the outcome of the measurement of an observable  $A$
might depend on which other observables
from systems of maximal observables
are measured alongside with $A$.
\end{itemize}
}


\subsection{Gleason's theorem}
\frame{
\frametitle{Gleason's theorem}
{\small
In view of the nonexistence of classical two-valued states on even finite superstructures of blocks or contexts associated with quantized systems,
one could still resort to classicality {\em within} blocks or contexts.
According to Gleason's theorem, this is exactly the {\em ``via regia,''}
to the quantum probabilities, in particular to the Born rule.
\\
$\;$ \\
{\footnotesize
A state, or (countably additive) probability measure, is a real function $P$
on  $L$ with the following properties:
\begin{itemize}
\item<1->  $P(0)=0$ and $P(y)\ge 0$  for all $y\in L$;
\item<1->  if $x_1,x_2,\ldots ,x_n$ are  atoms which can be represented by
an orthogonal basis of $n$-dimensional Hilbert space, then
$\sum_{i=1}^n P(x_i)=1$.
\end{itemize}
}
%$\;$ \\
\begin{beamerboxesrounded}[scheme=alert2,shadow=true]{Gleason's theorem}
Given a (countably additive) probability measure $P$ on a Hilbert lattice of dimension $\geq 3$; then there is an Hermitian,
non-negative operator $\rho$ whose trace is unity, such that
$P(x)={\rm tr}\left(\rho  E_{\bf {x}}\right) $
for all atoms $x\in L $, where
$E_{\bf {x}}={\bf {x}}\otimes{\bf {x}}^T\equiv \vert {\bf {x}}\rangle \langle {\bf {x}}\vert$
is the projector associated with the unit vector ${\bf {x}}$.
In particular, if some ${x_0}\in L$ satisfies $P(x_{0})=1$ then
$\rho ={\bf {x}}_0\otimes{\bf {x}}_0^T\equiv \vert {\bf {x}}_0\rangle \langle {\bf {x}}_0\vert$ and
$P(x)=\left\vert \langle{\bf {x}}_{0}\vert {\bf {x}}\rangle \right\vert ^{2}$
for all $x\in L$, where $\langle \cdot \vert \cdot \rangle $\ is the inner product (Born's rule).
\end{beamerboxesrounded}

}
}



\subsection{Quantum correlations of two particles}

\frame{
\frametitle{Example I: Quantum correlation of singlet state of two spin-1/2 particles}

$$
\vert \Psi_{2,s} \rangle = \frac{1}{\sqrt{ 2}}
\bigl(
\vert +- \rangle -
\vert -+ \rangle
\bigr)
$$
with
$$
\vert +\rangle
\equiv {\hat {\bf e}}_1 =(1,0)
\textrm{ and}
\vert -\rangle \equiv {\hat {\bf e}}_2 =(0,1)
$$
\begin{eqnarray}
\vert \Psi_{2,s} \rangle &=& {1\over \sqrt{2}}\bigl({\hat {\bf e}}_1\otimes {\hat {\bf e}}_2-{\hat {\bf e}}_2\otimes {\hat {\bf e}}_1 \bigr)\nonumber \\
&\equiv&\left( 0,\frac{1}{\sqrt{2}},- \frac{1}{\sqrt{2}} ,  0\right)  \nonumber \\
\rho_s&\equiv&\left(
\begin{array}{cccc}
0&0&0&0\\
0&\frac{1}{2}&- \frac{1}{2} &  0 \\
0&-\frac{1}{2}& \frac{1}{2} &  0 \\
0&0&0&0
\end{array}
\right)  \nonumber
\end{eqnarray}
}



 \frame{
\frametitle{Example I cntd.:Invariance and uniqueness property of two-partite singlet states}

\begin{itemize}
\item<1->
Singlet states are form invariant with respect to arbitrary unitary
transformations in the single-particle Hilbert spaces and thus
also rotationally invariant in configuration space,
in particular under the rotations
$$
\vert + \rangle =
e^{ i{\frac{\varphi}{2}} }
\left[
\cos (\frac{\theta}{2}) \vert +'  \rangle
-
\sin (\frac{\theta}{2}) \vert -'   \rangle
\right]
$$
and
$$
\vert - \rangle =
e^{ -i{\frac{\varphi}{2}} }
\left[
\sin (\frac{\theta}{2}) \vert +'   \rangle
+
\cos (\frac{\theta}{2}) \vert -'  \rangle
\right]
$$
in spherical coordinates $\theta , \varphi$.

\item<1->
The states satisfy a uniqueness property in the sense that knowledge
of a spin state observable for one particle is  sufficient
for the simultaneous (counterfactual) determination of
spin state properties for all other three particles.
\end{itemize}
}


 \frame{
\frametitle{Example I cntd.: Operators}

The projection operators $F$
corresponding to a two~spin-${1\over 2}$~particle joint measurement
aligned (``$+$'') or antialigned  (``$-$'') along the angles $\theta_i,\varphi_i$ are
\begin{equation}
\begin{array}{lll}
 F_{\pm \pm} ({\hat \theta},{\hat \varphi} ) =
{\frac{1}{2}}\left[{\mathbb I}_2 \pm {\bf \sigma}( \theta_1,\varphi_1 )\right]
\otimes
{\frac{1}{2}}\left[{\mathbb I}_2 \pm {\bf \sigma}( \theta_2,\varphi_2 )\right]
\end{array}
\end{equation}
with
$$
{\bf \sigma}( \theta ,\varphi )=
\left(
\begin{array}{cc} \cos \theta  &e^{-i\varphi} \sin \theta   \\
  e^{i\varphi}\sin \theta  & -\cos \theta
  \end{array}
\right).
$$

For example, $F_{- + } ({\hat \theta},{\hat \varphi} )$ stands for the proposition
\begin{quote}
{\it `The spin state of the first particle measured along $\theta_1,\varphi_1$ is ``$-$'', and
      the spin state of the second particle measured along $\theta_2,\varphi_2$ is ``$+$''.'
}
\end{quote}

}



 \frame{
\frametitle{Example I cntd.: Born Rule}

The joint probability to register the spins of the two particles
in state $\rho_s$
aligned or antialigned along the directions defined by $\theta_1,\varphi_1 $ on one side, and
$\theta_2,\varphi_2 $ on the other side
can be evaluated by a straightforward calculation
of
$$
P_{\rho_{ s }\,\pm \pm } ({\hat \theta},{\hat \varphi} )=
{\rm Tr}\left[\rho_s \cdot F_{\pm \pm} \left({\hat \theta},{\hat \varphi} \right)\right].
$$

}


\frame{
\frametitle{Example I cntd.: Expectation function}

The expectation function is defined by
$$E=P_{=}-P_{\neq} =  P_{++}+P_{--} - P_{+-}- P_{-+},$$
with  $P_{++}+P_{--} + P_{+-}+ P_{-+}=P_{=}+P_{\neq}=1$.
Let
$$
P_==   P_{++}+P_{--} = {1\over2}\left(1 + E \right)
\; \textrm{ and }\;
P_{\neq} =  P_{+-}+P_{-+} = {1\over2}\left(1 - E \right).
$$
An explicit calculation yields
%\begin{beamerboxesrounded}[scheme=alert,shadow=true]{}
$$
\begin{array}{l}
E(\theta_1,\theta_2,\varphi_1 , \varphi_2)=
-\cos \theta_1 \cos \theta_2 - \cos (\varphi_1 - \varphi_2) \sin \theta_1 \sin \theta_2
\end{array}
$$
%\end{beamerboxesrounded}

or, more specifically, for $\theta_{1,2}={\pi\over 2}$,\\

%\begin{beamerboxesrounded}[scheme=alert,shadow=true]{}
$$
\begin{array}{l}
E({\pi\over 2},{\pi\over 2},\varphi_1 , \varphi_2)= - \cos (\varphi_1 - \varphi_2)
\end{array}
$$
%\end{beamerboxesrounded}
}


\frame{
\frametitle{Example I cntd.: Functional form of expectations}
\begin{center}
%TexCad Options
%\grade{\off}
%\emlines{\off}
%\beziermacro{\off}
%\reduce{\on}
%\snapping{\off}
%\quality{4.00}
%\graddiff{0.01}
%\snapasp{1}
%\zoom{1.00}
\unitlength .70mm
\linethickness{0.4pt}
\begin{picture}(102.00,102.00)
%\emline(10.00,10.00)(10.00,100.00)
\put(10.00,10.00){\line(0,1){90.00}}
%\end
\put(10.00,55.00){\line(1,0){45.00}}
\put(55.00,55.00){\line(1,0){45.00}}
\put(10.00,10.00){\line(1,1){90.00}}
\put(100.00,100.00){\line(-1,0){45.00}}
\put(10.00,10.00){\line(1,0){45.00}}
%\bezier{284}(10.00,10.00)(30.00,10.00)(55.00,55.00)
\put(10.00,10.00){\line(1,0){1.41}}
\multiput(11.41,10.06)(0.71,0.08){2}{\line(1,0){0.71}}
\multiput(12.84,10.22)(0.48,0.09){3}{\line(1,0){0.48}}
\multiput(14.28,10.50)(0.36,0.10){4}{\line(1,0){0.36}}
\multiput(15.73,10.89)(0.29,0.10){5}{\line(1,0){0.29}}
\multiput(17.20,11.39)(0.25,0.10){6}{\line(1,0){0.25}}
\multiput(18.67,12.01)(0.21,0.10){7}{\line(1,0){0.21}}
\multiput(20.16,12.73)(0.21,0.12){7}{\line(1,0){0.21}}
\multiput(21.66,13.57)(0.19,0.12){8}{\line(1,0){0.19}}
\multiput(23.18,14.52)(0.17,0.12){9}{\line(1,0){0.17}}
\multiput(24.70,15.58)(0.15,0.12){10}{\line(1,0){0.15}}
\multiput(26.24,16.75)(0.14,0.12){11}{\line(1,0){0.14}}
\multiput(27.79,18.03)(0.13,0.12){12}{\line(1,0){0.13}}
\multiput(29.36,19.43)(0.12,0.12){13}{\line(1,0){0.12}}
\multiput(30.93,20.94)(0.11,0.12){14}{\line(0,1){0.12}}
\multiput(32.52,22.55)(0.11,0.12){14}{\line(0,1){0.12}}
\multiput(34.12,24.28)(0.12,0.13){14}{\line(0,1){0.13}}
\multiput(35.74,26.12)(0.12,0.14){14}{\line(0,1){0.14}}
\multiput(37.36,28.08)(0.12,0.15){14}{\line(0,1){0.15}}
\multiput(39.00,30.14)(0.12,0.16){14}{\line(0,1){0.16}}
\multiput(40.65,32.32)(0.12,0.16){14}{\line(0,1){0.16}}
\multiput(42.31,34.60)(0.12,0.17){14}{\line(0,1){0.17}}
\multiput(43.99,37.00)(0.11,0.17){15}{\line(0,1){0.17}}
\multiput(45.67,39.51)(0.11,0.17){15}{\line(0,1){0.17}}
\multiput(47.37,42.14)(0.11,0.18){15}{\line(0,1){0.18}}
\multiput(49.09,44.87)(0.11,0.19){15}{\line(0,1){0.19}}
\multiput(50.81,47.72)(0.12,0.20){15}{\line(0,1){0.20}}
\multiput(52.55,50.67)(0.12,0.21){21}{\line(0,1){0.21}}
%\end
%\bezier{284}(55.00,55.00)(80.00,100.00)(100.00,100.00)
\multiput(55.00,55.00)(0.12,0.21){15}{\line(0,1){0.21}}
\multiput(56.75,58.11)(0.12,0.20){15}{\line(0,1){0.20}}
\multiput(58.50,61.11)(0.12,0.19){15}{\line(0,1){0.19}}
\multiput(60.23,64.00)(0.11,0.19){15}{\line(0,1){0.19}}
\multiput(61.94,66.78)(0.11,0.18){15}{\line(0,1){0.18}}
\multiput(63.65,69.45)(0.11,0.17){15}{\line(0,1){0.17}}
\multiput(65.34,72.01)(0.12,0.17){14}{\line(0,1){0.17}}
\multiput(67.02,74.45)(0.12,0.17){14}{\line(0,1){0.17}}
\multiput(68.69,76.78)(0.12,0.16){14}{\line(0,1){0.16}}
\multiput(70.34,79.00)(0.12,0.15){14}{\line(0,1){0.15}}
\multiput(71.99,81.11)(0.12,0.14){14}{\line(0,1){0.14}}
\multiput(73.62,83.11)(0.12,0.13){14}{\line(0,1){0.13}}
\multiput(75.23,84.99)(0.11,0.13){14}{\line(0,1){0.13}}
\multiput(76.84,86.77)(0.11,0.12){14}{\line(0,1){0.12}}
\multiput(78.43,88.43)(0.12,0.12){13}{\line(1,0){0.12}}
\multiput(80.01,89.98)(0.13,0.12){12}{\line(1,0){0.13}}
\multiput(81.58,91.42)(0.13,0.11){12}{\line(1,0){0.13}}
\multiput(83.14,92.75)(0.14,0.11){11}{\line(1,0){0.14}}
\multiput(84.68,93.97)(0.15,0.11){10}{\line(1,0){0.15}}
\multiput(86.21,95.07)(0.17,0.11){9}{\line(1,0){0.17}}
\multiput(87.73,96.06)(0.19,0.11){8}{\line(1,0){0.19}}
\multiput(89.24,96.94)(0.21,0.11){7}{\line(1,0){0.21}}
\multiput(90.73,97.71)(0.25,0.11){6}{\line(1,0){0.25}}
\multiput(92.21,98.37)(0.29,0.11){5}{\line(1,0){0.29}}
\multiput(93.68,98.92)(0.36,0.11){4}{\line(1,0){0.36}}
\multiput(95.14,99.36)(0.48,0.11){3}{\line(1,0){0.48}}
\multiput(96.58,99.68)(0.72,0.11){2}{\line(1,0){0.72}}
\put(98.02,99.89){\line(1,0){1.98}}
%\end
\put(5.00,100.00){\makebox(0,0)[cc]{$+1$}}
\put(5.00,10.00){\makebox(0,0)[cc]{$-1$}}
\put(5.00,55.00){\makebox(0,0)[cc]{$0$}}
\put(58.00,50.00){\makebox(0,0)[cc]{$\pi /2$}}
\put(100.00,50.00){\makebox(0,0)[cc]{$\pi$}}
\put(102.00,59.00){\makebox(0,0)[cc]{$\theta$}}
\put(14.00,102.00){\makebox(0,0)[cc]{$E$}}
\put(30.00,38.00){\makebox(0,0)[cc]{$E_c(\theta )$}}
\put(46.00,28.00){\makebox(0,0)[cc]{$E_{qm}(\theta )$}}
\put(35.00,13.00){\makebox(0,0)[cc]{$E_s(\theta )$}}
\put(55.00,55.00){\circle*{2.00}}
\end{picture}
\end{center}

 }


\section{Partition logic}

\frame{

\centerline{\Huge Part III:  Partition logic}

\begin{center}
$\widetilde{\qquad \qquad }$
$\widetilde{\qquad \qquad}$
$\widetilde{\qquad \qquad }$
\end{center}
 }

\subsection{Generalized urn models}
\frame{
\frametitle{Generalized urn models}

A generalized urn model (Wright 1990)
${\cal U}=\langle U,C,L,\Lambda \rangle $ is
characterized as follows.
Consider an ensemble of balls with black background color.
Printed on these balls are some color symbols from a symbolic alphabet $L$.
The colors are elements of a set of colors $C$.
A particular ball type is associated with a unique combination of mono-spectrally
(no mixture of wavelength) colored symbols
printed on the black ball background.
Let $U$ be the set of ball types.
We shall assume that every ball contains
just one single symbol per color.
(Not all types of balls; i.e., not all color/symbol combinations, may be present in
the ensemble, though.)

Let
$\vert U\vert $ be the number of different types of balls,
$\vert C\vert $ be the number of different mono-spectral colors,
$\vert L\vert $ be the number of different output symbols.

}


\frame{
\frametitle{Generalized urn models cntd.}

Consider the deterministic ``output'' or ``lookup''
function $\Lambda (u,c)=v$,
$u\in U$,
$c\in C$,
$v\in L$,
which returns one symbol per ball type and color.
One interpretation of this lookup function $\Lambda$ is as follows.
Consider a set of $\vert C\vert $ eyeglasses build from filters for the
$\vert C\vert $ different colors.
Let us assume that these mono-spectral filters are
``perfect'' in that they totally absorb light of all other colors
but a particular single one.
In that way, every color can be associated with a particular eyeglass and vice versa.
}


\frame{
\frametitle{Generalized urn models cntd.}
When a spectator looks at a particular ball through such an eyeglass,
the only operationally recognizable symbol will be the one in the particular
color which is transmitted through the eyeglass.
All other colors are absorbed, and the symbols printed in them will appear black
and therefore cannot be differentiated from the black background.
Hence the ball appears to carry a different ``message'' or symbol,
depending on the color at which it is viewed.

Consider, for the sake of demonstration, a single color and its associated partition of the set of ball types
(ball types within a given element of the partition cannot be differetiated by that color).
In the generalized urn model, an element $a$ of this partition is a set of ball types which corresponds to an elementary proposition
\begin{quote}
{\em ``the ball drawn from the urn is of the type contained in $a$.''}
\end{quote}

}

\subsection{Finite automata}
\frame{
\frametitle{Finite automata}
A (Mealy type) automaton
${\cal A}=\langle S,I,O,\delta ,\lambda \rangle$ is characterized
by the set of states $S$,
by the set of input symbols $I$,
and by the set of output symbols $O$.
$\delta (s,i)=s'$ and
$\lambda (s,i)=o$,
$s,s'\in S$,
$i\in I$
and $o\in O$
represent the transition and the output functions, respectively.
The restriction to Mealy automata is for convenience only.


In the analysis of a {\em state identification problem}, a typical automaton experiment aims at
\index{state identification problem}
an operational determination of an {\em unknown initial state}
by the input of some symbolic sequence and the observation of the resulting output symbols.

}


\frame{
\frametitle{Finite automata cntd.}

A typical proposition in the automaton model refers to a partition element $a$  containing automaton states which cannot be distinguished
by the analysis of the strings of input and output symbols; i.e., it can be expressed by
\begin{quote}
{\em ``the automaton is initially in a state which is contained in $a$.''}
\end{quote}

}


\frame{
\frametitle{Automaton partition logic
with a nonfull set of dispersion-free measures}
{\footnotesize
\begin{center}
%TexCad Options
%\grade{\off}
%\emlines{\off}
%\beziermacro{\off}
%\reduce{\on}
%\snapping{\off}
%\quality{0.20}
%\graddiff{0.01}
%\snapasp{1}
%\zoom{1.00}
\unitlength 0.6mm
\linethickness{1pt}
\thicklines
\begin{picture}(108.00,55.00)
\put(25.00,7.33){\color[rgb]{0.965,0.785,0.89}\line(1,0){60.00}}
\put(25.00,47.33){\color{red}\line(1,0){60.00}}
\put(55.00,7.33){\color{cyan}\line(0,1){40.00}}
\put(25.00,7.33){\color{blue}\line(-1,1){20.00}}
\put(5.00,27.33){\color{green}\line(1,1){20.00}}
\put(85.00,7.33){\color{magenta}\line(1,1){20.00}}
\put(105.00,27.33){\color{yellow}\line(-1,1){20.00}}
\put(24.67,55.00){\makebox(0,0)[rc]{$a_3=\{10,11,12,13,14\}$}}
\put(55.33,55.00){\makebox(0,0)[cc]{$a_4=\{2,6,7,8\}$}}
\put(85.33,55.00){\makebox(0,0)[lc]{$a_5=\{1,3,4,5,9\}$}}
\put(9.00,40.00){\makebox(0,0)[rc]{$a_2=\{4,5,6,7,8,9\}$}}
\put(99.33,40.00){\makebox(0,0)[lc]{$a_6=\{2,6,8,11,12,14\}$}}
\put(0.00,26.33){\makebox(0,0)[rc]{$a_1=\{1,2,3\}$}}
\put(108.00,26.33){\makebox(0,0)[lc]{$a_7=\{7,10,13\}$}}
\put(60.33,31.33){\makebox(0,0)[lc]{$a_{13}=$}}
\put(60.33,26.33){\makebox(0,0)[lc]{$\{1,4,5,10,11,12\}$}}
\put(9.00,13.33){\makebox(0,0)[rc]{$a_{12}=\{4,6,9,12,13,14\}$}}
\put(99.67,13.33){\makebox(0,0)[lc]{$a_8=\{3,5,8,9,11,14\}$}}
\put(24.67,-0.05){\makebox(0,0)[rc]{$a_{11}=\{5,7,8,10,11\}$}}
\put(55.33,-0.05){\makebox(0,0)[cc]{$a_{10}=\{3,9,13,14\}$}}
\put(85.33,-0.05){\makebox(0,0)[lc]{$a_9=\{1,2,4,6,12\}$}}
\put(15.00,17.09){\color{blue}\circle{2.00}}
\put(25.00,7.33){\color{blue}\circle{2.00}}
\put(25.00,7.33){\color[rgb]{0.965,0.785,0.89}\circle{3.00}}
\put(55.00,27.33){\color{cyan}\circle{2.00}}
\put(85.00,7.33){\color[rgb]{0.965,0.785,0.89}\circle{2.00}}
\put(85.00,7.33){\color{magenta}\circle{3.00}}
\put(95.00,17.33){\color{magenta}\circle{2.00}}
\put(5.00,27.33){\color{green}\circle{2.00}}
\put(5.00,27.33){\color{blue}\circle{3.0}}
\put(15.00,37.33){\color{green}\circle{2.00}}
\put(25.00,47.33){\color{green}\circle{2.00}}
\put(25.00,47.33){\color{red}\circle{3.00}}
\put(55.00,47.33){\color{red}\circle{2.00}}
\put(55.00,47.33){\color{cyan}\circle{3.00}}
\put(85.00,47.33){\color{red}\circle{2.00}}
\put(85.00,47.33){\color{yellow}\circle{3.00}}
\put(55.00,7.33){\color[rgb]{0.965,0.785,0.89}\circle{2.00}}
\put(55.00,7.33){\color{cyan}\circle{3.00}}
\put(104.76,27.33){\color{yellow}\circle{2.00}}
\put(104.76,27.33){\color{magenta}\circle{3.00}}
\put(95.00,37.33){\color{yellow}\circle{2.00}}
\end{picture}
\end{center}
}

}


\frame{
\frametitle{Realization in terms of a generalized urn model}

\begin{center}
{\tiny
\setlength{\tabcolsep}{3pt}
\begin{tabular}{|c|ccccccccccccc||ccccccc|}
%\begin{tabular}{|c|c@{}c@{}c@{}c@{}c@{}c@{}c@{}c@{}c@{}c@{}c@{}c@{}c||c@{}c@{}c@{}c@{}c@{}c@{}c|}
\hline\hline
&\multicolumn{13}{c||}{(a) lattice atoms}&\multicolumn{7}{|c|}{(b) colors}\\
%\cline{2-14}
\raisebox{1.5ex}[0cm][0cm]{$m_r$ and}&$a_1$&$a_2$&$a_3$&$a_4$&$a_5$&$a_6$&$a_7$&$a_8$&$a_9$&$a_{10}$&$a_{11}$&$a_{12}$&$a_{13}$&\color{green}$c_1$&\color{red}$c_2$&\color{yellow}$c_3$&\color{magenta}$c_4$&$\color[rgb]{0.965,0.785,0.89}c_5$&$\color{blue}c_6$&$\color{cyan}c_7$\\
\raisebox{1.5ex}[0cm][0cm]{ball type}&&&&&&&&&&&&&&&&&&&&\\
\hline
1  &1&0&0&0&1&0&0&0&1&0&0&0&1&  \color{green}1&\color{red}1&\color{yellow} 1&\color{magenta} 1&\color[rgb]{0.965,0.785,0.89} 1&\color{blue} 1&\color{cyan}1          \\
2  &1&0&0&1&0&1&0&0&1&0&0&0&0&  \color{green}1&\color{red}2&\color{yellow} 2&\color{magenta} 1&\color[rgb]{0.965,0.785,0.89} 1&\color{blue} 1&\color{cyan}2           \\
3  &1&0&0&0&1&0&0&1&0&1&0&0&0&  \color{green}1&\color{red}1&\color{yellow} 1&\color{magenta} 2&\color[rgb]{0.965,0.785,0.89} 2&\color{blue} 1&\color{cyan}3          \\
4  &0&1&0&0&1&0&0&0&1&0&0&1&1&  \color{green}2&\color{red}1&\color{yellow} 1&\color{magenta} 1&\color[rgb]{0.965,0.785,0.89} 1&\color{blue} 2&\color{cyan}1          \\
5  &0&1&0&0&1&0&0&1&0&0&1&0&1&  \color{green}2&\color{red}1&\color{yellow} 1&\color{magenta} 2&\color[rgb]{0.965,0.785,0.89} 3&\color{blue} 3&\color{cyan}1          \\
6  &0&1&0&1&0&1&0&0&1&0&0&1&0&  \color{green}2&\color{red}2&\color{yellow} 2&\color{magenta} 1&\color[rgb]{0.965,0.785,0.89} 1&\color{blue} 2&\color{cyan}2           \\
7  &0&1&0&1&0&0&1&0&0&0&1&0&0&  \color{green}2&\color{red}2&\color{yellow} 3&\color{magenta} 3&\color[rgb]{0.965,0.785,0.89} 3&\color{blue} 3&\color{cyan}2           \\
8  &0&1&0&1&0&1&0&1&0&0&1&0&0&  \color{green}2&\color{red}2&\color{yellow} 2&\color{magenta} 2&\color[rgb]{0.965,0.785,0.89} 3&\color{blue} 3&\color{cyan}2           \\
9  &0&1&0&0&1&0&0&1&0&1&0&1&0&  \color{green}2&\color{red}1&\color{yellow} 1&\color{magenta} 2&\color[rgb]{0.965,0.785,0.89} 2&\color{blue} 2&\color{cyan}3          \\
10 &0&0&1&0&0&0&1&0&0&0&1&0&1&  \color{green}3&\color{red}3&\color{yellow} 3&\color{magenta} 3&\color[rgb]{0.965,0.785,0.89} 3&\color{blue} 3&\color{cyan}1          \\
11 &0&0&1&0&0&1&0&1&0&0&1&0&1&  \color{green}3&\color{red}3&\color{yellow} 2&\color{magenta} 2&\color[rgb]{0.965,0.785,0.89} 3&\color{blue} 3&\color{cyan}1          \\
12 &0&0&1&0&0&1&0&0&1&0&0&1&1&  \color{green}3&\color{red}3&\color{yellow} 2&\color{magenta} 1&\color[rgb]{0.965,0.785,0.89} 1&\color{blue} 2&\color{cyan}1          \\
13 &0&0&1&0&0&0&1&0&0&1&0&1&0&  \color{green}3&\color{red}3&\color{yellow} 3&\color{magenta} 3&\color[rgb]{0.965,0.785,0.89} 2&\color{blue} 2&\color{cyan}3          \\
14 &0&0&1&0&0&1&0&1&0&1&0&1&0&  \color{green}3&\color{red}3&\color{yellow} 2&\color{magenta} 2&\color[rgb]{0.965,0.785,0.89} 2&\color{blue} 2&\color{cyan}3          \\
\hline\hline
\end{tabular}
}
\end{center}
(a) Dispersion-free states  of the Kochen-Specker ``bug'' logic with 14 dispersion-free states
and (b) the associated generalized urn model.

}


\frame{
\frametitle{Probability theory}

There are enough two-valued states to be able to apply the classical strategy.

}

\section{Summary}
\frame{
\frametitle{Summary \& ``mind map'' representing the use of contexts to build up logics and construct probabilities.}
\begin{center}
{\tiny
 %TeXCAD Picture [1.pic]. Options:
%\grade{\on}
%\emlines{\off}
%\epic{\off}
%\beziermacro{\on}
%\reduce{\on}
%\snapping{\off}
%\pvinsert{% Your \input, \def, etc. here}
%\quality{8.000}
%\graddiff{0.005}
%\snapasp{1}
%\zoom{4.0000}
\unitlength .5mm % = 1.565pt
\linethickness{0.4pt}
\ifx\plotpoint\undefined\newsavebox{\plotpoint}\fi % GNUPLOT compatibility
\begin{picture}(303.75,133.25)(0,0)
\put(106.125,125.25){\oval(51.25,16)[]}
\put(25.375,87.25){\oval(51.25,16)[]}
\put(105.625,87.5){\oval(51.25,16)[]}
\put(178.375,87.75){\oval(51.25,16)[]}
\put(105.5,125){\makebox(0,0)[cc]{context}}
\put(105,90.25){\makebox(0,0)[cc]{block}}
\put(105,85.25){\makebox(0,0)[cc]{Boolean subalgebra}}
\put(24,87.25){\makebox(0,0)[cc]{Boolean algebra}}
\put(177.75,90.25){\makebox(0,0)[cc]{block}}
\put(177.75,85.25){\makebox(0,0)[cc]{Boolean subalgebra}}
\put(80.25,117.5){\makebox(0,0)[cc]{}}
%\vector[middle](80.75,117.5)(31,98.75)
\put(55.875,108.125){\vector(-3,-1){.128}}\multiput(80.75,117.5)(-.1625816993,-.0612745098){306}{\line(-1,0){.1625816993}}
%\end
%\vector[middle](105,115.25)(105,99.5)
\put(105,107.375){\vector(0,-1){.128}}\put(105,115.25){\line(0,-1){15.75}}
%\end
%\vector[middle](131.75,117.5)(170,101.75)
\put(150.875,109.625){\vector(3,-1){.128}}\multiput(131.75,117.5)(.1488326848,-.0612840467){257}{\line(1,0){.1488326848}}
%\end
%\dottedbox(6.5,55.25)(33.25,13)
\put(6.5,55.25){\makebox(33.25,13)[cc]{}}
\multiput(6.372,68.122)(0,-.92857){15}{{\rule{.4pt}{.4pt}}}
\multiput(6.372,55.122)(.977941,0){35}{{\rule{.4pt}{.4pt}}}
\multiput(6.372,68.122)(.977941,0){35}{{\rule{.4pt}{.4pt}}}
\multiput(39.622,68.122)(0,-.92857){15}{{\rule{.4pt}{.4pt}}}
%\end
\put(23,64){\makebox(0,0)[cc]{two-valued}}
\put(23,59){\makebox(0,0)[cc]{measures}}
%\dottedbox(87.5,55.5)(33.25,13)
\put(87.5,55.5){\makebox(33.25,13)[cc]{}}
\multiput(87.372,68.372)(0,-.92857){15}{{\rule{.4pt}{.4pt}}}
\multiput(87.372,55.372)(.977941,0){35}{{\rule{.4pt}{.4pt}}}
\multiput(87.372,68.372)(.977941,0){35}{{\rule{.4pt}{.4pt}}}
\multiput(120.622,68.372)(0,-.92857){15}{{\rule{.4pt}{.4pt}}}
%\end
\put(103,64){\makebox(0,0)[cc]{two-valued}}
\put(103,59){\makebox(0,0)[cc]{measures}}
%\dottedbox(168.75,55.5)(33.25,13)
\put(168.75,55.5){\makebox(33.25,13)[cc]{}}
\multiput(168.622,68.372)(0,-.92857){15}{{\rule{.4pt}{.4pt}}}
\multiput(168.622,55.372)(.977941,0){35}{{\rule{.4pt}{.4pt}}}
\multiput(168.622,68.372)(.977941,0){35}{{\rule{.4pt}{.4pt}}}
\multiput(201.872,68.372)(0,-.92857){15}{{\rule{.4pt}{.4pt}}}
%\end
\put(185.5,64){\makebox(0,0)[cc]{no two-valued}}
\put(185.5,59){\makebox(0,0)[cc]{measure}}
\put(46.75,113.25){\makebox(0,0)[rc]{classical}}
\put(109,107.25){\makebox(0,0)[lc]{generalized urns}}
\put(109,102.25){\makebox(0,0)[lc]{automata}}
\put(157.25,116){\makebox(0,0)[lc]{quantized}}
%\vector[middle](165.25,77.25)(165.25,47.75)
\put(165.25,62.5){\vector(0,-1){.128}}\put(165.25,77.25){\line(0,-1){29.5}}
%\end
%\circle(165.25,22.5){41.869}
\put(186.185,22.5){\line(0,1){.9537}}
\put(186.163,23.454){\line(0,1){.9517}}
\put(186.098,24.405){\line(0,1){.9478}}
\multiput(185.989,25.353)(-.05051,.31396){3}{\line(0,1){.31396}}
\multiput(185.838,26.295)(-.04857,.2335){4}{\line(0,1){.2335}}
\multiput(185.643,27.229)(-.05916,.23104){4}{\line(0,1){.23104}}
\multiput(185.407,28.153)(-.055696,.182485){5}{\line(0,1){.182485}}
\multiput(185.128,29.066)(-.053293,.149799){6}{\line(0,1){.149799}}
\multiput(184.808,29.965)(-.060062,.147215){6}{\line(0,1){.147215}}
\multiput(184.448,30.848)(-.057177,.123708){7}{\line(0,1){.123708}}
\multiput(184.048,31.714)(-.05491,.105853){8}{\line(0,1){.105853}}
\multiput(183.609,32.561)(-.059675,.103241){8}{\line(0,1){.103241}}
\multiput(183.131,33.387)(-.05717,.089258){9}{\line(0,1){.089258}}
\multiput(182.617,34.19)(-.061177,.086561){9}{\line(0,1){.086561}}
\multiput(182.066,34.969)(-.058552,.075316){10}{\line(0,1){.075316}}
\multiput(181.481,35.722)(-.056293,.065973){11}{\line(0,1){.065973}}
\multiput(180.861,36.448)(-.05924,.06334){11}{\line(0,1){.06334}}
\multiput(180.21,37.144)(-.062064,.060575){11}{\line(-1,0){.062064}}
\multiput(179.527,37.811)(-.064759,.057685){11}{\line(-1,0){.064759}}
\multiput(178.815,38.445)(-.074052,.060142){10}{\line(-1,0){.074052}}
\multiput(178.074,39.047)(-.076715,.056706){10}{\line(-1,0){.076715}}
\multiput(177.307,39.614)(-.088021,.059058){9}{\line(-1,0){.088021}}
\multiput(176.515,40.145)(-.09062,.054986){9}{\line(-1,0){.09062}}
\multiput(175.699,40.64)(-.10466,.057151){8}{\line(-1,0){.10466}}
\multiput(174.862,41.097)(-.122463,.059798){7}{\line(-1,0){.122463}}
\multiput(174.005,41.516)(-.12506,.054157){7}{\line(-1,0){.12506}}
\multiput(173.129,41.895)(-.14863,.056471){6}{\line(-1,0){.14863}}
\multiput(172.237,42.234)(-.181258,.059569){5}{\line(-1,0){.181258}}
\multiput(171.331,42.532)(-.183784,.051249){5}{\line(-1,0){.183784}}
\multiput(170.412,42.788)(-.23241,.05353){4}{\line(-1,0){.23241}}
\multiput(169.483,43.002)(-.31281,.05718){3}{\line(-1,0){.31281}}
\multiput(168.544,43.174)(-.31509,.04287){3}{\line(-1,0){.31509}}
\put(167.599,43.302){\line(-1,0){.9501}}
\put(166.649,43.388){\line(-1,0){.9531}}
\put(165.696,43.43){\line(-1,0){.954}}
\put(164.742,43.428){\line(-1,0){.9529}}
\put(163.789,43.383){\line(-1,0){.9499}}
\multiput(162.839,43.295)(-.31496,-.04381){3}{\line(-1,0){.31496}}
\multiput(161.894,43.164)(-.31264,-.05811){3}{\line(-1,0){.31264}}
\multiput(160.956,42.989)(-.23225,-.05422){4}{\line(-1,0){.23225}}
\multiput(160.027,42.773)(-.18363,-.051798){5}{\line(-1,0){.18363}}
\multiput(159.109,42.514)(-.181079,-.06011){5}{\line(-1,0){.181079}}
\multiput(158.204,42.213)(-.148461,-.056914){6}{\line(-1,0){.148461}}
\multiput(157.313,41.872)(-.124897,-.05453){7}{\line(-1,0){.124897}}
\multiput(156.439,41.49)(-.122284,-.060164){7}{\line(-1,0){.122284}}
\multiput(155.583,41.069)(-.104489,-.057463){8}{\line(-1,0){.104489}}
\multiput(154.747,40.609)(-.090455,-.055257){9}{\line(-1,0){.090455}}
\multiput(153.933,40.112)(-.087844,-.05932){9}{\line(-1,0){.087844}}
\multiput(153.142,39.578)(-.076545,-.056935){10}{\line(-1,0){.076545}}
\multiput(152.377,39.008)(-.073872,-.060363){10}{\line(-1,0){.073872}}
\multiput(151.638,38.405)(-.064587,-.057878){11}{\line(-1,0){.064587}}
\multiput(150.927,37.768)(-.061883,-.06076){11}{\line(-1,0){.061883}}
\multiput(150.247,37.1)(-.05905,-.063516){11}{\line(0,-1){.063516}}
\multiput(149.597,36.401)(-.056096,-.06614){11}{\line(0,-1){.06614}}
\multiput(148.98,35.674)(-.058326,-.07549){10}{\line(0,-1){.07549}}
\multiput(148.397,34.919)(-.060919,-.086743){9}{\line(0,-1){.086743}}
\multiput(147.849,34.138)(-.056904,-.089429){9}{\line(0,-1){.089429}}
\multiput(147.336,33.333)(-.059367,-.103419){8}{\line(0,-1){.103419}}
\multiput(146.861,32.506)(-.054593,-.106016){8}{\line(0,-1){.106016}}
\multiput(146.425,31.658)(-.056808,-.123878){7}{\line(0,-1){.123878}}
\multiput(146.027,30.79)(-.059623,-.147394){6}{\line(0,-1){.147394}}
\multiput(145.669,29.906)(-.052846,-.149957){6}{\line(0,-1){.149957}}
\multiput(145.352,29.006)(-.055151,-.182651){5}{\line(0,-1){.182651}}
\multiput(145.076,28.093)(-.05847,-.23122){4}{\line(0,-1){.23122}}
\multiput(144.843,27.168)(-.04787,-.23364){4}{\line(0,-1){.23364}}
\multiput(144.651,26.234)(-.04957,-.31411){3}{\line(0,-1){.31411}}
\put(144.502,25.291){\line(0,-1){.9481}}
\put(144.397,24.343){\line(0,-1){.9519}}
\put(144.334,23.391){\line(0,-1){2.859}}
\put(144.408,20.532){\line(0,-1){.9475}}
\multiput(144.519,19.585)(.05144,-.3138){3}{\line(0,-1){.3138}}
\multiput(144.674,18.643)(.04927,-.23335){4}{\line(0,-1){.23335}}
\multiput(144.871,17.71)(.05985,-.23086){4}{\line(0,-1){.23086}}
\multiput(145.11,16.787)(.056241,-.182318){5}{\line(0,-1){.182318}}
\multiput(145.391,15.875)(.05374,-.149639){6}{\line(0,-1){.149639}}
\multiput(145.714,14.977)(.060502,-.147035){6}{\line(0,-1){.147035}}
\multiput(146.077,14.095)(.057546,-.123537){7}{\line(0,-1){.123537}}
\multiput(146.48,13.23)(.055225,-.105689){8}{\line(0,-1){.105689}}
\multiput(146.922,12.385)(.059983,-.103063){8}{\line(0,-1){.103063}}
\multiput(147.401,11.56)(.057436,-.089087){9}{\line(0,-1){.089087}}
\multiput(147.918,10.758)(.055292,-.07774){10}{\line(0,-1){.07774}}
\multiput(148.471,9.981)(.058776,-.075141){10}{\line(0,-1){.075141}}
\multiput(149.059,9.23)(.056489,-.065804){11}{\line(0,-1){.065804}}
\multiput(149.68,8.506)(.059429,-.063163){11}{\line(0,-1){.063163}}
\multiput(150.334,7.811)(.062244,-.06039){11}{\line(1,0){.062244}}
\multiput(151.019,7.147)(.064931,-.057491){11}{\line(1,0){.064931}}
\multiput(151.733,6.514)(.074231,-.059921){10}{\line(1,0){.074231}}
\multiput(152.475,5.915)(.076884,-.056477){10}{\line(1,0){.076884}}
\multiput(153.244,5.35)(.088197,-.058795){9}{\line(1,0){.088197}}
\multiput(154.038,4.821)(.090784,-.054716){9}{\line(1,0){.090784}}
\multiput(154.855,4.329)(.10483,-.056838){8}{\line(1,0){.10483}}
\multiput(155.694,3.874)(.122641,-.059433){7}{\line(1,0){.122641}}
\multiput(156.552,3.458)(.125221,-.053784){7}{\line(1,0){.125221}}
\multiput(157.429,3.081)(.148798,-.056027){6}{\line(1,0){.148798}}
\multiput(158.321,2.745)(.181435,-.059028){5}{\line(1,0){.181435}}
\multiput(159.229,2.45)(.183936,-.050701){5}{\line(1,0){.183936}}
\multiput(160.148,2.197)(.23257,-.05284){4}{\line(1,0){.23257}}
\multiput(161.079,1.985)(.31298,-.05625){3}{\line(1,0){.31298}}
\multiput(162.018,1.817)(.31522,-.04193){3}{\line(1,0){.31522}}
\put(162.963,1.691){\line(1,0){.9504}}
\put(163.914,1.608){\line(1,0){.9532}}
\put(164.867,1.569){\line(1,0){.954}}
\put(165.821,1.573){\line(1,0){.9528}}
\put(166.773,1.621){\line(1,0){.9496}}
\multiput(167.723,1.712)(.31483,.04475){3}{\line(1,0){.31483}}
\multiput(168.668,1.846)(.31246,.05905){3}{\line(1,0){.31246}}
\multiput(169.605,2.023)(.23209,.05492){4}{\line(1,0){.23209}}
\multiput(170.533,2.243)(.183474,.052346){5}{\line(1,0){.183474}}
\multiput(171.451,2.505)(.180899,.06065){5}{\line(1,0){.180899}}
\multiput(172.355,2.808)(.14829,.057357){6}{\line(1,0){.14829}}
\multiput(173.245,3.152)(.124734,.054903){7}{\line(1,0){.124734}}
\multiput(174.118,3.537)(.122103,.060528){7}{\line(1,0){.122103}}
\multiput(174.973,3.96)(.104317,.057775){8}{\line(1,0){.104317}}
\multiput(175.807,4.422)(.09029,.055526){9}{\line(1,0){.09029}}
\multiput(176.62,4.922)(.087667,.059582){9}{\line(1,0){.087667}}
\multiput(177.409,5.458)(.076375,.057163){10}{\line(1,0){.076375}}
\multiput(178.173,6.03)(.073692,.060583){10}{\line(1,0){.073692}}
\multiput(178.91,6.636)(.064414,.05807){11}{\line(1,0){.064414}}
\multiput(179.618,7.275)(.061701,.060945){11}{\line(1,0){.061701}}
\multiput(180.297,7.945)(.058861,.063692){11}{\line(0,1){.063692}}
\multiput(180.944,8.646)(.055898,.066308){11}{\line(0,1){.066308}}
\multiput(181.559,9.375)(.058101,.075664){10}{\line(0,1){.075664}}
\multiput(182.14,10.132)(.060659,.086925){9}{\line(0,1){.086925}}
\multiput(182.686,10.914)(.056636,.089598){9}{\line(0,1){.089598}}
\multiput(183.196,11.72)(.059058,.103596){8}{\line(0,1){.103596}}
\multiput(183.668,12.549)(.054277,.106179){8}{\line(0,1){.106179}}
\multiput(184.103,13.399)(.056438,.124047){7}{\line(0,1){.124047}}
\multiput(184.498,14.267)(.059182,.147571){6}{\line(0,1){.147571}}
\multiput(184.853,15.152)(.052398,.150114){6}{\line(0,1){.150114}}
\multiput(185.167,16.053)(.054606,.182815){5}{\line(0,1){.182815}}
\multiput(185.44,16.967)(.05778,.23139){4}{\line(0,1){.23139}}
\multiput(185.671,17.893)(.04717,.23378){4}{\line(0,1){.23378}}
\multiput(185.86,18.828)(.04863,.31425){3}{\line(0,1){.31425}}
\put(186.006,19.771){\line(0,1){.9484}}
\put(186.109,20.719){\line(0,1){1.781}}
%\end
\put(166,25){\makebox(0,0)[cc]{Gleason theorem}}
\put(165.25,20){\makebox(0,0)[cc]{(Born rule)}}
\put(90,20.75){\framebox(29.75,20.25)[]{}}
\put(8,20.75){\framebox(29.75,20.25)[]{}}
\put(105,34){\makebox(0,0)[cc]{convex}}
\put(105,29){\makebox(0,0)[cc]{sum}}
\put(23,34){\makebox(0,0)[cc]{convex}}
\put(23,29){\makebox(0,0)[cc]{sum}}
\put(111.5,48.25){\makebox(0,0)[lc]{probabilities}}
\put(29.5,48.25){\makebox(0,0)[lc]{probabilities}}
\put(169.75,48.25){\makebox(0,0)[lc]{probabilities}}
\put(58.75,6){\framebox(28.5,23)[cc]{}}
\put(73.5,19){\makebox(0,0)[cc]{finite pasting}}
\put(73.5,14){\makebox(0,0)[cc]{of blocks}}
\put(190,6){\framebox(29,23)[cc]{}}
\put(206,23){\makebox(0,0)[cc]{continuous }}
\put(206,18){\makebox(0,0)[cc]{pasting}}
\put(206,13){\makebox(0,0)[cc]{of blocks}}
%\qbezvec[middle](80,79.75)(67.875,55.75)(73.25,32.75)
\put(72.25,56){\vector(-1,-4){.128}}\qbezier(80,79.75)(67.875,55.75)(73.25,32.75)
%\end
%\qbezvec[middle](204.5,79.75)(221.5,59.75)(209.5,33.75)
\put(214.25,58.25){\vector(1,-4){.128}}\qbezier(204.5,79.75)(221.5,59.75)(209.5,33.75)
%\end
%\qbezvec(31.5,74.75)(45.625,68.375)(45.25,76.5)
\put(45.25,76.5){\vector(0,1){.128}}\qbezier(31.5,74.75)(45.625,68.375)(45.25,76.5)
%\end
\put(76,47.25){\makebox(0,0)[lc]{logic}}
\put(208,37.25){\makebox(0,0)[rc]{logic}}
\put(56,69){\makebox(0,0)[rc]{logic}}
\put(44.5,70.75){\makebox(0,0)[lc]{}}
\put(44.25,71.75){\makebox(0,0)[rc]{}}
\put(46.25,71){\makebox(0,0)[lc]{}}
%\vector[middle](23.75,77)(23.75,71)
\put(23.75,74){\vector(0,-1){1.5}}\put(23.75,77){\line(0,-1){6}}
%\end
%\vector[middle](23.5,50.5)(23.5,44.5)
\put(23.5,47.5){\vector(0,-1){1.5}}\put(23.5,50.5){\line(0,-1){6}}
%\end
%\vector[middle](105,77)(105,71)
\put(105,74){\vector(0,-1){1.5}}\put(105,77){\line(0,-1){6}}
%\end
%\vector[middle](104.75,50.5)(104.75,44.5)
\put(104.75,47.5){\vector(0,-1){1.5}}\put(104.75,50.5){\line(0,-1){6}}
%\end
%\vector[middle](186.25,77)(186.25,71)
\put(186.25,74){\vector(0,-1){1.5}}\put(186.25,77){\line(0,-1){6}}
%\end
\end{picture}
}
\end{center}
}


\frame{



\centerline{\Large Thank you for your attention!}

\begin{center}
$\widetilde{\qquad \qquad }$
$\widetilde{\qquad \qquad}$
$\widetilde{\qquad \qquad }$
\end{center}
 }

\end{document}
