\documentclass[%
 preprint,
 %reprint,
 % twocolumn,
 %superscriptaddress,
 %groupedaddress,
 %unsortedaddress,
 %runinaddress,
 %frontmatterverbose,
 % preprint,
 showpacs,
 showkeys,
 preprintnumbers,
 %nofootinbib,
 %nobibnotes,
 %bibnotes,
 amsmath,amssymb,
 aps,
 % prl,
  pra,
 % prb,
 % rmp,
 %prstab,
 %prstper,
  longbibliography,
 %floatfix,
 %lengthcheck,%
 ]{revtex4-1}

%\usepackage{cdmtcs-pdf}

%\usepackage{mathptmx}% http://ctan.org/pkg/mathptmx


\usepackage{natbib}
%\bibliographystyle{abbrvnat}
\bibliographystyle{newapa-doi-url}
\setcitestyle{authoryear,open={(},close={)}}

\usepackage{tikz}
\usepackage[breaklinks=true,colorlinks=true,anchorcolor=blue,citecolor=blue,filecolor=blue,menucolor=blue,pagecolor=blue,urlcolor=blue,linkcolor=blue]{hyperref}
\usepackage{graphicx}% Include figure files
\usepackage{url}

\usepackage{xcolor}


\usepackage{eurosym}

\usepackage[none]{hyphenat}

\begin{document}
\sloppy

\title{Quantum Abracadabra}

%\cdmtcsauthor{Karl Svozil}
%\cdmtcsaffiliation{Vienna University of Technology}
%\cdmtcstrnumber{407}
%\cdmtcsdate{September 2011}
%\coverpage


\author{Karl Svozil}
\affiliation{Institute for Theoretical Physics, Vienna
    University of Technology, Wiedner Hauptstra\ss e 8-10/136, A-1040
    Vienna, Austria}
\affiliation{Department of Computer Science, University of Auckland, Private Bag 92019,  Auckland 1142, New Zealand}
\email{svozil@tuwien.ac.at} \homepage[]{http://tph.tuwien.ac.at/~svozil}


%\pacs{03.67.-a}
%\keywords{quantum computation, quantum information}
%\preprint{CDMTCS preprint nr. 407/2011}

%\begin{abstract}
%\end{abstract}

\maketitle

In private conversations, the late Swiss mathematician Ernst Specker related ``Jesuit lies''
to distorting facts without lying explicitly --
by either
issuing ambiguous statements,
or
by stating only the convenient facts while omitting inopportune ones,
%or by preaching preliminary knowledge as absolute, eternal truth,
or
by allowing outrageous claims without correcting them
--
if only it is advantageous.
%As unethical and uncommon as this may appear, even presidential candidates in secluded circles commit that ``$\ldots$ you need both a public and a private position.''
It should come as no surprise that these sort of issues also occur in science; in particular,
if resources and big money are involved.
What makes this worrisome is the fact that, unlike politicians and just as theologians, scientists have great authority regarding the pursuit of truth.
As a consequence, the public, and political bodies and institutions,
tend to uncritically adapt exaggerated, biased claims as matters of fact; in particular, if it serves some profits and the public's desire for
fairy tales.

A striking example are the nuclear power plants in Three Mile Island, Chernobyl, and Fukushima falsifying bold claims that nuclear fission technology is ``safe beyond doubt.''
%Actually, one could still hold that it might be safe if only human misjudgement, bad maintenance as well as natural disasters, such as earthquakes and tsunamis, and all other factors making them unsafe can be excluded.

In a similar fashion, the alleged mysteries of the quantum has been sold to the public for quite some time now.
The ``quantum mechanics is magic'' tour, expressed for instance in
Europe's quantum community's {\it Quantum Manifesto,} has, among other, previous initiatives world-wide,
recently launched a European \euro{}1 billion quantum technologies flagship initiative in quantum technology.
The campaign promises to initiate nothing less than a second quantum revolution.
%by {\em ``taking quantum theory to its technological consequences.''}

I have no doubts that \euro{}1 billion spent on the quantum are wisely invested, and that something worthy will come out of it.
Alas what leaves me worried is the deceptive and potentially harmful way this, and similar, quantum related initiatives are marketed.
While many of the  Quantum Manifesto's short- and medium-term goals appear feasible,
%some remain means relative at best, and
some of the long-term goals might not even be achievable in principle.
And when it comes to quantum random number generators and quantum cryptography, certain goals are provable impossible.

Let us, for the moment, contemplate on the Manifesto's call to
{\em ``build a universal quantum computer able to demonstrate the resolution of a problem that, with
current techniques on a supercomputer, would take longer than the age of the universe.''}
I am at a loss of imagining what that could be; given the rather sober situation regarding the capacity of quantum computers.
Although NIST's {\em Quantum Algorithmic Zoo} enumerates a growing number of potential speedups,
no substantial ``killer-apps'' have been suggested in the last years,
and
there does not even exist a consolidated view about what exactly could make quantum computation superior over classical computation.
Most observers seem to agree that one advantage might be quantum parallelism:
based on coherently superposing
classically distinct and mutually exclusive computational states, the capacity to push all of them
through a quantum computer simultaneously.
Alas, there's a ``Hamletian rub:'' the operator has no direct access to the state processed,
and has to analyse the output state subject to complementarity.
It seems that this strategy is applicable only in particular instances,
in which certain properties can be encoded into suitable orthogonal subspaces.
In such cases, relational information about the input-output behaviour can be extracted from such states
without the need (or possibility) to analyse the single cases contributing to the correlations.

It also remains unclear whether quantum computation is scalable in the sense that an increase in quantum bits needs
no excessive, possibly exponential, overhead in resources creating and maintaining the additional bits.

Another quantum asset is the use of entanglement for communication involving multiple particles
across arbitrary distances.
Such a system could be in a definite collective state,
defined solely in terms of relational properties or correlations among the constituents,
whereas the states of the single constituents remain totally undefined.
While in this regard exponential speedups have been proposed,
there is again no common understanding of the issues involved.

With regards to quantum random number generators, the situation is confused, to say the least.
Indeed, it is not even clear where exactly quantum randomness resides: it cannot originate
from elements such as lossless beam splitters, because these are merely permuting the quantum state.
If measurements were the source of randomness, then it would be means relative at best.

Moreover, because of incompleteness theorems
such as the recursive unsolvability of the halting and the rule inference (induction) problem,
any statement regarding the {\it ex nihilo} creation of empirical bit sequences are provable unprovable.
Thus regardless of what we may be inclined to believe,
and whatever authoritative certificates are issued,
such claims remain strictly metaphysical and conjectural.
%They will remain conjectures; their truth is relative to the our own suppositions and narratives.

Finally, contrary to publicized claims,
quantum cryptography is insecure and can be successfully cryptanalyzed through man-in-the-middle attacks.
Already Bennett and Brassard acknowledged this possibility by discussing active eavesdropping.
As a consequence, such quantum cryptographic protocols, in order to be safe,
require both an uncompromised classical as well as quantum communication channel.
With these provisos one may ask, what exactly is the advantage and the ``added security?''
Is quantum cryptography presenting itself as the solution of a problem
while at the same time requiring the absence of this threat it purports to resolve?
If you push the experts with these kind of questions, they respond that, rather than generating a key out of the blue,
within certain error bounds, they could ``enlarge'' an existing key.
This is the type of confidence that is implied by ``unconditional security'' in many of these papers.
%Secure indeed, but only in the absence of certain cryptanalytic attacks!

Let me finish by suggesting that, besides all the aforementioned quantum challenges,
we desperately need an entirely different initiative;
this one requiring much higher ``whatever it takes'' investments:
thermonuclear fusion might be
sustainable at moderate operating costs and perils.
The sooner we seriously start investigating this potential energy resource,
the smoother physics might be able to provide solutions
to the upcoming energy crisis, with depleting cheap crude oil.


\acknowledgments{
This work was supported in part by the European Union, Research Executive Agency (REA),
Marie Curie FP7-PEOPLE-2010-IRSES-269151-RANPHYS grant.


Responsibility for the information and views expressed in this article lies entirely with the author.
The content therein does not reflect the official opinion of the Vienna University of Technology or the University of Auckland.


The author declares no conflict of interest and, in particular, no involvement in nuclear fusion research.

}



%\bibliography{svozil}

\end{document}
