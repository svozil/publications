\documentclass[amsfonts,a4,12pt]{article}
\RequirePackage[english]{babel}
\RequirePackage{times}
\RequirePackage{courier}
\RequirePackage{mathptm}
%\RequirePackage{bookman}
%\RequirePackage{helvetic}
%\RequirePackage{times}
\selectlanguage{english}
\RequirePackage[isolatin]{inputenc}
\makeindex
\begin{document}
\pagestyle{plain}


Dear Cris,

In what follows I'll list a few principles
of Albert Einstein's special theory of relativity which are important in this respect.

Assume two observers $S_1$ and $S_2$.
Suppose $S_2$ travels with a reference frame of the accelerated particles at speed $v$
with respect to an observer $S_1$ at rest.
Suppose further that the accelerated observer  $S_2$
measures a time interval $\Delta t_2$.
According to Einstein's special theory of relativity,
this time interval, when measured in $S_1$,  is dilated (``shrinked'')
%%     http://en.wikipedia.org/wiki/Time_dilation
by  the Lorentz factor $\gamma (v) \ge 1$; i.e.,
%%     http://en.wikipedia.org/wiki/Lorentz_factor
\begin{equation}
\begin{array}{ccl}
\Delta t_2 &=& \Delta t_1 / \gamma (v) \le \Delta t_1, \\
\gamma (v) &=& \left[ 1- \left({v\over c}\right)^2\right]^{-{1\over 2}}.
\end{array}
\end{equation}
That is, clocks and computational processes in $S_2$ are ``slowed down'' with respect
to clocks and computational processes in $S_1$.

This has been proven by high energy particle decay experiments from cosmic sources.

To illustrate this fact by a simple example:
If the velocity of $S_2$ reaches 99\% of the speed of light,
$\gamma (0.99c) \approx 7$,
and the clocks in $S_2$ are slowed down by a factor of seven
with respect to the clocks in $S_1$.
Stated differently, one time cycle $\Delta t_2=1$ in $S_2$ corresponds to  seven a time cycles
$\Delta t_1 = 1 \times \gamma (0.99c) \approx 7$ in $S_1$.
All processes in $S_2$, including computation, are temporally decelerated by a factor of about seven
with respect to $S_1$.

Hope that makes sense.

Best,
Karl

\end{document}
