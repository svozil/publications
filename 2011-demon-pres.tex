%\documentclass[pra,showpacs,showkeys,amsfonts,amsmath,twocolumn]{revtex4}
\documentclass[amsmath,table,sans]{beamer}
%\documentclass[pra,showpacs,showkeys,amsfonts]{revtex4}
\usepackage[T1]{fontenc}
%%\usepackage{beamerthemeshadow}
%%\usepackage[headheight=1pt,footheight=10pt]{beamerthemeboxes}
%%\addfootboxtemplate{\color{structure!80}}{\color{white}\tiny \hfill Karl Svozil (TU Vienna)\hfill}
%%\addfootboxtemplate{\color{structure!65}}{\color{white}\tiny \hfill mur.sat \hfill}
%%\addfootboxtemplate{\color{structure!50}}{\color{white}\tiny \hfill Graz, 2010-12-11\hfill}
%\usepackage[dark]{beamerthemesidebar}
%\usepackage[headheight=24pt,footheight=12pt]{beamerthemesplit}
%\usepackage{beamerthemesplit}
%\usepackage[bar]{beamerthemetree}
\usepackage{graphicx}
\usepackage{pgf}
%\usepackage[usenames]{color}
%\newcommand{\Red}{\color{Red}}  %(VERY-Approx.PANTONE-RED)
%\newcommand{\Green}{\color{Green}}  %(VERY-Approx.PANTONE-GREEN)

%\RequirePackage[german]{babel}
%\selectlanguage{german}
%\RequirePackage[isolatin]{inputenc}

\pgfdeclareimage[height=0.5cm]{logo}{tu-logo}
\logo{\pgfuseimage{logo}}
\beamertemplatetriangleitem
%\beamertemplateballitem

\beamerboxesdeclarecolorscheme{alert}{red}{red!15!averagebackgroundcolor}
%\begin{beamerboxesrounded}[scheme=alert,shadow=true]{}
%\end{beamerboxesrounded}

%\beamersetaveragebackground{yellow!10}

%\beamertemplatecircleminiframe

\begin{document}

\title{\bf \textcolor{blue}{What can Maxwell's Demon do for you?}}
%\subtitle{Naturwissenschaftlich-Humanisticher Tag am BG 19\\Weltbild und Wissenschaft\\http://tph.tuwien.ac.at/\~{}svozil/publ/2005-BG18-pres.pdf}
\subtitle{\textcolor{orange!60}{\small http://tph.tuwien.ac.at/$\sim$svozil/publ/2011-demon-pres.pdf\\
http://arxiv.org/abs/1105.4768}}
\author{Karl Svozil}
\institute{Institut f\"ur Theoretische Physik, University of Technology Vienna, \\
Wiedner Hauptstra\ss e 8-10/136, A-1040 Vienna, Austria\\
svozil@tuwien.ac.at
%{\tiny Disclaimer: Die hier vertretenen Meinungen des Autors verstehen sich als Diskussionsbeitr�ge und decken sich nicht notwendigerweise mit den Positionen der Technischen Universit�t Wien oder deren Vertreter.}
}
\date{StatPhysI, May 25th, 2011}
\maketitle

\frame{

\centerline{\Large Part I: }
\centerline{ }
\centerline{\Large  \color{blue} Introduction of the Demon}

\begin{center}
{\color{orange}
$\widetilde{\qquad \qquad }$
$\widetilde{\qquad \qquad}$
$\widetilde{\qquad \qquad }$
}
\end{center}
}

\begin{frame}[fragile]
\frametitle{Maxwell's Theory of Heat}

\small
\begin{verbatim}
@BOOK{Maxwell-1871,
  title = {Theory of Heat},
  year = {1871},
  author = {James Clerk Maxwell},
  url={http://www.archive.org/details/theoryheat02maxwgoog}
}
\end{verbatim}


\end{frame}


\frame[shrink=1.01]{
\frametitle{Maxwell's Demon}
\begin{center}
\includegraphics{2011-emtech-MaxwellsDemonPicture2.pdf}
\end{center}
}

\frame[shrink=1.01]{
\frametitle{Maxwell's Demon}
\begin{center}
\includegraphics{2011-emtech-MaxwellsDemonPicture3.pdf}
\end{center}
}


\frame{
\frametitle{What could go wrong?}
\begin{center}
\Huge ?
\end{center}
}

\frame{

\centerline{\Large Part II: }
\centerline{ }
\centerline{\Large  \color{blue} Early explanations \& exploitations}

\begin{center}
{\color{orange}
$\widetilde{\qquad \qquad }$
$\widetilde{\qquad \qquad}$
$\widetilde{\qquad \qquad }$
}
\end{center}
}

\begin{frame}[fragile]
\frametitle{Early explanations \& exploitations}
\small
\begin{verbatim}
@article{Szilard-1929,
   author = {Le\'o Szil\'ard},
   affiliation = {Berlin},
   title = {{\"{U}}ber die {E}ntropieverminderung
              in einem thermodynamischen {S}ystem bei
              {E}ingriffen intelligenter {W}esen},
   journal = {Zeitschrift f{\"u}r Physik},
   publisher = {Springer Berlin / Heidelberg},
   issn = {0939-7922},
   keyword = {Physics and Astronomy},
   pages = {840-856},
   volume = {53},
   issue = {11},
   url = {http://dx.doi.org/10.1007/BF01341281},
   doi = {10.1007/BF01341281},
   year = {1929},
   note={English translation in \cite[pp.~110-119]{maxwell-demon2}}
}
\end{verbatim}

\end{frame}

\frame{

\centerline{\Large Part III: }
\centerline{ }
\centerline{\Large  \color{blue} Present understanding}

\begin{center}
{\color{orange}
$\widetilde{\qquad \qquad }$
$\widetilde{\qquad \qquad}$
$\widetilde{\qquad \qquad }$
}
\end{center}
}

\frame{
\frametitle{Information theoretic ``solution'' to Maxwell's Demon}

R. Landauer,  {Irreversibility and Heat Generation in the Computing Process},
{IBM Journal of Research and Development} {\bf 3},  {183-191} (1961)

\begin{itemize}

\item<+->
 logical irreversibility in connection with information-discarding processes
---
``cleared'' memory can be from a variety of previous states
---
due to a two-to-one change of state volume in ``phase space'' associated with $\Delta S= k_B {\rm log}2$

\item<+->
Each logical step must somehow correspond to a physical state

\item<+-> (``the bad news'')
logical irreversibility is associated with physical ``heat dissipation''
and ``entropy increase''  ;-(

\item<+-> (``the good news'')
logically reversibile operations need not be associated with physical ``heat dissipation''
and ``entropy increase'' ;-)   $\Rightarrow$ (reversible) dissipationless universal computation possible!

\end{itemize}


}

\begin{frame}[fragile]
\frametitle{Modern-day ``solution'' of Maxwell's question}
\small
\begin{verbatim}
@ARTICLE{bennett-82,
  author = {Charles H. Bennett},
  title = {The Thermodynamics of Computation---A Review},
  journal = {International Journal of Theoretical Physics},
  year = {1982},
  volume = {21},
  pages = {905-940},
  note = {Reprinted in Ref.~\cite[pp. 283-318]{maxwell-demon2}},
  doi = {10.1007/BF02084158},
  url = {http://dx.doi.org/10.1007/BF02084158}
}
\end{verbatim}

\end{frame}

\frame[shrink=1.01]{
\frametitle{Modern-day ``solution'' of Maxwell's question}
\begin{center}
\includegraphics{2011-emtech-emtech-demon-op.pdf}
\end{center}
}


\frame{

\centerline{\Large Part IV: }
\centerline{ }
\centerline{\Large  \color{blue} Brownian ratchet}
\centerline{\Large  \color{blue} (Molekulare Ratsche nach Smoluchowski-Feynman )}

\begin{center}
{\color{orange}
$\widetilde{\qquad \qquad }$
$\widetilde{\qquad \qquad}$
$\widetilde{\qquad \qquad }$
}
\end{center}
}


\begin{frame}[fragile]
\frametitle{Modern-day ``solution'' of Maxwell's question}
\small
\begin{verbatim}
@ARTICLE{Smoluchovski-1912,
  author = {Marian Smoluchowski},
  title = {{E}xperimentell nachweisbare,
            der {\"u}blichen {T}hermodynamik
            widersprechende {M}olekularph{\"a}nomene},
  journal = {Physikalische Zeitschrift},
  year = {1912},
  volume = {13},
  pages = {1069-1080},
  url1 = {http://www.physik.uni-augsburg.de/theo1/hanggi/History/PhysZeitschrift.pdf},
  url = {http://matwbn.icm.edu.pl/ksiazki/pms/pms2/pms2122.pdf}
}
\end{verbatim}

\end{frame}


\frame[shrink=1.01]{
\frametitle{Brownian ratchet (Molekulare Ratsche)}
\begin{center}
\includegraphics{2011-demon-pres-Molekularratsche.pdf}
\end{center}
}


\frame{
\frametitle{What could go wrong?}
\begin{center}
\Huge ?
\end{center}
}


\frame{
\frametitle{Impossibility of $\ldots$}
\begin{center}
$\ldots$ perpetual motion machines of (the first, and of) the second kind.
\end{center}
\begin{center}
$\ldots$ the construction of a device which {\em permanently} --
that is, in continuation -- solely extracts work from a heat reservoir.
In ``small'' time scales, such an extraction seems possible.
\end{center}
}


\end{document}
