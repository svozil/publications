\newif\ifws
%\wstrue
\ifws

\documentclass{article}

\usepackage{graphicx}        % standard LaTeX graphics tool
\usepackage{xcolor}

\usepackage{hyperref}
\hypersetup{
    colorlinks,
    linkcolor={blue!80!black},
    citecolor={red!75!black},
    urlcolor={blue!80!black}
}

% Damit die Verwendung der deutschen Sprache nicht ganz so umst\"andlich wird,
% sollte man die folgenden Pakete einbinden:


%German
%\usepackage[latin1]{inputenc}% erm\"oglich die direkte Eingabe der Umlaute
%\usepackage[T1]{fontenc} % das Trennen der Umlaute
%\usepackage{ngerman} % hiermit werden deutsche Bezeichnungen genutzt und
                     % die W\"orter werden anhand der neue Rechtschreibung
                     % automatisch getrennt.
\title{Why Computation?}
\author{Karl Svozil \\
        Institute for Theoretical Physics,
Vienna  University of Technology,  \\
Wiedner Hauptstrasse 8-10/136,
1040 Vienna,  Austria \\
        \and
        Department of Computer Science, University of Auckland,  \\
Private Bag 92019, Auckland, New Zealand
        }

\date{\today}
% Hinweis: \title{um was auch immer es geht}, \author{wer es auch immer
% geschrieben hat} und  \date{wann auch immer das war} k\"onnen vor
% oder nach dem  Kommando \begin{document} stehen
% Aber der \maketitle Befehl mu\ss{} nach dem \begin{document} Kommando stehen!
\begin{document}

\maketitle


\begin{abstract}
This paper reviews my personal inclinations and fascination with the area of unconventional computing. Computing can be perceived as an inscription in a ``Rosetta Stone,'' a category akin to physics, and therefore as a form of comprehension of nature. I also address the question of why there is computation, and sketch a research program based on primordial chaos, out of which order and even self-referential perception emerges by way of evolution.
\end{abstract}


\else
\documentclass[%
 %reprint,
  twocolumn,
 %superscriptaddress,
 %groupedaddress,
 %unsortedaddress,
 %runinaddress,
 %frontmatterverbose,
 % preprint,
 showpacs,
 showkeys,
 preprintnumbers,
 %nofootinbib,
 %nobibnotes,
 %bibnotes,
 amsmath,amssymb,
 aps,
 % prl,
  pra,
 % prb,
 % rmp,
 %prstab,
 %prstper,
  longbibliography,
 floatfix,
 %lengthcheck,%
 ]{revtex4-1}

%\usepackage{cdmtcs-pdf}

\usepackage{mathptmx}% http://ctan.org/pkg/mathptmx

\usepackage{amssymb,amsthm,amsmath}

\usepackage{tikz}
\usepackage{graphicx}% Include figure files
%\usepackage{url}

\usepackage{xcolor}

\usepackage{hyperref}
\hypersetup{
    colorlinks,
    linkcolor={blue},
    citecolor={red!75!black},
    urlcolor={blue}
}


\begin{document}


\title{Why Computation?\footnote{Contribution to a special issue on Integral Biomathics: In Paths to Unconventional Computing, The Necessary Conjunction of the Western and Eastern Thought Traditions for Exploring the Nature of Mind and Life}}


\author{Karl Svozil}
\email{svozil@tuwien.ac.at}
\homepage{http://tph.tuwien.ac.at/~svozil}

\affiliation{Institute for Theoretical Physics,
Vienna  University of Technology,
Wiedner Hauptstrasse 8-10/136,
1040 Vienna,  Austria}

\affiliation{Department of Computer Science, University of Auckland,
Private Bag 92019, Auckland, New Zealand}

\date{\today}

\begin{abstract}
This paper reviews my personal inclinations and fascination with the area of unconventional computing. Computing can be perceived as an inscription in a ``Rosetta Stone,'' one category akin to physics, and therefore as a form of comprehension of nature. I also address the question of why there is computation, and sketch a research program based on primordial chaos, out of which order and even self-referential perception emerges by way of evolution.
\end{abstract}

%\pacs{03.65.Aa, 03.65.Ta, 03.65.Ud, 03.67.-a}
\keywords{Church-Turing thesis, unconventional computing, primordial chaos, self-referential perception ,emergence, evolution}
%\preprint{CDMTCS preprint nr. x}

\maketitle

\fi

\subsection*{Computation and physics as categories}

Nowadays I might be able to express my long time intuition in a category theoretical form~\cite{Yanofsky-mc}: in short,
computation and physics are both categories linked by functors.
Thereby category theory serves as a sort of Rosetta Stone~\cite{Baez2011}, making possible a translation among very similar, possibly equivalent, structures --
with the functors serving as translators back and forth between the physical and the computational universes.
One may even enlarge this picture by other categories like mathematics, and the natural transformations between the possible functors.
In what follows I shall rant about computation as a metaphysical as well as metamathematical metaphor.
At the same time, computation could also be understood as a narrative designed to navigate and manipulate the impression of what we experience as physical world.

First it should be acknowledged that,
on the one hand,
although conceptualized with paper-and-pencil operations in mind~\cite[p.~34]{Turing-Intelligent_Machinery},
the category of computation, as many structures invented by our minds,
including mathematics and theology~\cite{jonas-msot}
or our money~\cite{svozil-2011-apology-f-m}, appears to be ``suspended in free thought'' --  and solely grounded in our belief in it.

On the other hand, there appears to be ``physical stuff out there'' which at first peek appears to be rather solid and ``material.''
Alas, the deeper we have looked into it, and the better our means to spatially resolve matter became,
the more this stuff looked like
an emptiness containing point particles of zero extension.
Moreover, throughout the history of natural sciences, there appears to be no convergence of ``causes,''
but rather a succession of alternating narrations and (re)presentations as to why this stuff interacts:
take what we today call gravity, turning from mythology to Ptolemaian geometry to Newtonian force back to Einsteinian space-time geometry~\cite{lakatosch}.
And this is a far cry from explaining why something exists at all -- even if this something might turn out to be primordial chaos,
or an initial singularity (possibly hiding other cycles of other universes).

Indeed, it can be expected that, for an embedded observer~\index{toffoli:79} in a virtual reality,
the computational intrinsic ``phenomenology'' supporting such an agent appears just as ``material,'' and even ``quantum complementary
like''~\cite{svozil-2008-ql}, as our own universe is experienced by us.
A surreal feeling is expressed by Prospero in Shakespeare's Tempest, claiming that {\em ``we are such stuff as dreams are made on.''}
(Some~\cite{camus-mos} have therefore concluded that science cannot offer much
anchor from which to comprehend and cope with the absurdities of our existence.)
Ought we therefore not be allowed to assume that the category subsumed under the name ``physics''
contains entities and structures which are not dissimilar to computation?


Second, consider the functors which -- like a function -- assigns to each entity in the physical world an entity in the computational universe.
More specifically, the Church-Turing thesis, interpreted as functor between physics and computation,
specifies that every capacity in the physical world is reflected
by some computational, algorithmic capacity of what is known today as a partial recursive function,
or universal Turing computability.
This is a highly nontrivial claim which needs to be corroborated or falsified with every physical capacity we discover.
It is, so to say, under ``permanent attack'' from physics.
Although highly likely, nobody can guarantee that it will survive the next day.
To give one exotic and highly speculative example:
maybe someone eventually comes up with a clever way of building infinity machines with some Zeno squeezed cycles.
It is also interesting to note that one might be able to resolve the seemingly contradicting claims of
``information is physical'' by Landauer, as well as ``it from bit'' by Wheeler, through perceiving both physics and computation as categories linked by functors.

A universal computer, hooked up to a quantum random number generator (serving as an oracle for randomness)
is supposed to be (relative to the validity of orthodox quantum mechanics) a machine transcending  universal computational capacities.
Claims of computational capacities beyond Turing's universal computability may turn out to be difficult to (dis)prove.
One way might involve zero-knowledge proofs or zero-knowledge protocols; but I am unaware of any such criterion~\cite{2007-hc}.
%By the way, in his later years, Ernst Specker became very interested in zero-knowledge proofs; but as far as I remember not in the context mentioned.
Unfortunately, some such instances,
in particular ``true randomness'' or ``true (in)determinism'' as claimed by quantum information theory,
due to reductions to the halting and rule inference problems,
are provable impossible to prove.


The converse functor, mapping entities from universal computation into entities in the physical universe is considered unproblematic.
After all, in principle, given enough stuff, universal computers could be physically realized; at least up to some finite means.
These finite physical means induce bounds on universal computability~\cite{gandy2}.

In any case, this category theoretical view
could immediately resolve {\em ``the unreasonable effectiveness of mathematics in the natural sciences''}~\cite{wigner}
by essentially identifying the two categories; more precisely, by tying them together with the Church-Turing thesis,
or generalizations thereof, should the latter become necessary.
As Yanofsky has pointed out throughout our conversations, any such identification
is not immediately wrong only {\em ``if one adopts the constructivist/Bishop philosophy of mathematics. $\ldots$
But, if you do not follow that school, and you are either a
Platonist or a nominalist, then mathematics is more than computation. In that case ``physics$=$computation'' still does not answer Wigner's mystery.''}
For these latter cases one might argue that evolution has given us the gift to imagine ways and forms of mathematics (such as the continuum)
which go beyond the constructivist/Bishop philosophy of mathematics.
And even then, in order to be substantiated
this ``physis$=$computation'' metaphor would have involve an infinite universe (to operate a universal Turing machine),
as well as some (metaphysical) ``corroboration'' (such as an ineffable~\cite{Jonas-ineffability} believe therein)
that the Church-Turing thesis is correct.

Thus, speaking about computation might be like speaking about physics. And any capacity of one category has to show up in the other one as well.
In view of this it is highly questionable if nonconstructive entities such as continua are more than a formal convenience,
if not a distractive misrepresentation, of physical capacities.

\subsection*{Optimized dissipation of energy through computation}

Let me, in the second part, come to a sketch of the semantic aspect of the categories compared earlier; and just how and why they could have formed.

Suppose that there exist (we do not attempt here to explain why this should be so; for instance due to fluctuations or initial values)
two regions in space with a difference in temperature, or, more generally, energy (density).
Suppose further that there is some interface, such as empty space, or material structure, or agent,
allowing physical dissipative flows from one region into the other, connecting these two regions.
Then, as expressed by the second law of thermodynamics~\cite{Myrvold2011237}, there will be an exchange of energy, whereby statistically energy flows
from hot to cold through the interface. So far, this is a purely physical process.

Let us concentrate on the interface. More specifically, let us consider a variety of interfaces, and look at their relative efficiency or ``fitness''
(we are slowly entering an evolution type domain here).
Undoubtedly, all things equal, the type of interface with the highest throughput rate of energy per time will dominate the dissipation process:
it can ``grab the biggest piece of the cake.''
Finding good or even optimal interfaces might be facilitated through random mutation; thereby roaming through an
abstract space of possible interface states and configurations.
The situation will become even more dynamic if the relative magnitude of the various processes can change over time.
In particular, if a very efficient process (which needs not be the most efficient) can self-replicate.
Then a regime emerges which is
dominated by the {M}atthew effect~\cite{merton-68} of compound interest: the population of the strongest interface will increase relative to less effective interfaces
by the rate of compound interest -- which is effectively exponential. This means that the growth rates will at first look linear (and thus sustainable),
but later grow faster and faster until either all the energy is distributed or other side conditions limit this growth.
Now, if we identify certain interfaces with biological entities
we end up with a sort of biological evolution driven by physical processes;
in particular, by energy dissipation~\cite{England-PhysRevX.6.021036}.


How does computation come into this picture?
Actually, quite straightforwardly, if we are willing to continue this speculative path:
systems which compute can serve as, and even construct and produce,
better interfaces for energy dissipation than systems without algorithmics.
Thus, through mutation, that is trial-and-error driven by random walks through roaming configurations and state space,
the universe, and in particular, self-reproducing agents and units,
have learned to compute.
This is, essentially, a scenario for the emergence of mathematics and of universal computation.

One could go one step further and speculate that the perception and representation of the self;
that is, self-awareness and consciousness, emerged in the same way: driven to dissipate -- that is,
in moralistic terms, use and waste -- as much energy as possible.
As depicted somewhat ironically in Fig.~\ref{2017-ptuc-f}, computation, mathematics,
the human mind, emerge as -- and have evolved because of, and are still driven by --  mere facilities and vehicles for
optimal heat exchange.
Stated differently, the universe attempts to ``perceive'' and ``understand'' itself better;
the goal being optimized ``self-digestion;'' that is, dissipation of energy.
\ifws
\begin{figure}
\begin{center}
\includegraphics[width=5cm,angle=0]{2017-ptuc-f1s}
\hspace{3mm}
\includegraphics[width=5cm,angle=0]{2017-ptuc-f2s}
\end{center}
\caption{
Evolution of species and computation, driven by the second law of thermodynamics (as inspired by England's {\it et al} approach~\cite{England-PhysRevX.6.021036}):
(i) interface between hot and cold regions is empty space;
(ii) plant interface capable of more dissipation than empty space;
(iii) animals and, in particular, humans (drawn political correctly) present interfaces with improved (over plants and emptyness) energy dissipation;
(iv) humans equipped with engines and universal computation capacities (indicated by ``T'' for ``universal Turing machine'') can consume even more energy than standalone.
\label{2017-ptuc-f}
}
\end{figure}
\else
\begin{figure*}
\begin{center}
\includegraphics[width=7cm,angle=0]{2017-ptuc-f1s}
\hspace{3mm}
\includegraphics[width=7cm,angle=0]{2017-ptuc-f2s}
\end{center}
\caption{
Evolution of species and computation, driven by the second law of thermodynamics (as inspired by England's {\it et al} approach~\cite{England-PhysRevX.6.021036}):
(i) interface between hot and cold regions is empty space;
(ii) plant interface capable of more dissipation than empty space;
(iii) animals and, in particular, humans (drawn political correctly) present interfaces with improved (over plants and emptyness) energy dissipation;
(iv) humans with engines and universal computation capacities (indicated by ``T'' for ``universal Turing machine'') can consume even more energy than standalone.
\label{2017-ptuc-f}
}
\end{figure*}
\fi

This might be perceived as rather sobering, even dystopian perspective, from which
the evolution of species, consciousness,  mathematics and computation appear as means to ``waste more energy'' than without these emergent utilities.
Questions of ethics and ultimately divinity are pertinent.
Yet all we can say is that even in a universe of primordial chaos
-- note that, just like Egon von Schweidler in 1905 speculated that the laws of radioactive decay are merely probabilistic~\cite{schweidler-1905},
so might all of our natural laws be merely probabilistic
(an old and almost forgotten proposal by Exner~\cite{Exner-1908,Hanley-1979}, also reviewed by Schr\"odinger~\cite{schrodinger-1929,book:16081}),
and therefore subject to deviations from their standard form for ``small'' scales --
the gods, and thus law and order, might have evolved as principles of social conditioning,
in particular when, very late in this chain of events,
societies formed and secular and religious powers underwent a symbiosis.
This provides a context to
Nietzsche's critique of slave morality (transgressed by Marxism)~\cite{Nietzsche-GM}.
Alas, his obvious failure to suggest  alternatives beyond Greek thinking
(see also the  {\em Milian dialogue}~\cite[Chapter~V, \S~84-116]{Thucydides}) might have been due to the nonexistence
of formal techniques at the time.
Nowadays game theory~\cite{Braithwaite-gt,Nowak:1995:AMH} seems to be well equipped for a relativized view of good and evil.
A relativized view of morality seems to have also been endorsed by Paul Dirac in a forgotten lecture~\cite{dirac-81}.
This lecture on the futility of (nuclear) war, which I had the privilege to attend,
deeply impressed my thinking on these matters ever since.

The physical ``underpinning'' or ``layer'' argued earlier is in the spirit of Landauer's ``information is physical,''
thereby extending it to ``consciousness is physical.''
It requires no god of miracles or god of the gaps; yet it cannot explain the initial boot-up of the universe.

%~~~~~~~~~~~~~~~~~~~~~~~~


Let me end this very brief perspective with the acknowledgement that one constant source of inspiration and motivation has been
the cooperation with Cristian Calude and his tolerant and open-minded approach,
as well as his genuine interest in the physical layers of information and computation.
This brief account partly presents my recollection of, and reflections on, recent discussions with Noson Yanofsky at his Brooklyn home.
I am also deeply
thankful for many pertinent discussions with Gregory Chaitin, Alexander Leitsch, the late Ernst Specker,
among many other patient and passionate fellow researchers
sharing a common goal: the pursuit of what every individual one of us, in (in)effable ways~\cite{Jonas-ineffability} calls truth.
This work was supported in part by  the John Templeton Foundation's {\em  Randomness and Providence: an Abrahamic Inquiry Project}.

\bibliography{svozil}
\ifws

\bibliographystyle{spmpsci}

\else
%\bibliographystyle{apsrev}

\fi

\end{document}
