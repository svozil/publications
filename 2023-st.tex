\documentclass[%
  reprint,
 %twocolumn,
 %onecolumn,
 %superscriptaddress,
 %groupedaddress,
 %unsortedaddress,
 %runinaddress,
 %frontmatterverbose,
 %preprint,
 showpacs,
 showkeys,
 preprintnumbers,
 nofootinbib,
 %nobibnotes,
 %bibnotes,
 amsmath,amssymb,
 aps,
 % prl,
 pra,
 % prb,
 % rmp,
 %prstab,
 %prstper,
  longbibliography,
 %floatfix,
 %lengthcheck,%
 ]{revtex4-2}

%\usepackage{cdmtcs-pdf}

\usepackage[dvipsnames]{xcolor}

\usepackage{mathptmx}% http://ctan.org/pkg/mathptmx

\usepackage{amssymb,amsthm,amsmath}

\usepackage{tikz}
\usetikzlibrary{calc,decorations.pathreplacing,decorations.markings,positioning,shapes}

\usepackage[breaklinks=true,colorlinks=true,anchorcolor=blue,citecolor=blue,filecolor=blue,menucolor=blue,pagecolor=blue,urlcolor=blue,linkcolor=blue]{hyperref}
\usepackage{graphicx}% Include figure files
\usepackage{url}


\begin{document}

\title{It-From-Click (Re)Construction of Quantum Space-Time Frames Requires Parameter Dependence}


\author{Karl Svozil}
\email{svozil@tuwien.ac.at}
\homepage{http://tph.tuwien.ac.at/~svozil}

\affiliation{Institute for Theoretical Physics,
TU Wien,
\\
Wiedner Hauptstrasse 8-10/136,
1040 Vienna,  Austria}


\date{\today}

\begin{abstract}
Quantum entanglement restricts the physical means to generate space-time frames for entangled states. In particular, space and time protocols for entangled systems may result in scales that differ from the ones used by Poincar\'e-Einstein synchronization.
\end{abstract}

\keywords{space-time frames, synchronization, induced relativity, quantum space-time}


\maketitle

%\begin{widetext}

\section{Conventions and the nessity of choice}

Physical categories and conceptualizations, such as time and space,
are formed in minds in accordance with the operational means available to observers.
They are thus idealistic~\cite{stace1} and epistemic, and therefore historic, preliminary, contextual, and not absolute.


Operationalists like Bridgman~\cite{bridgman36}, Zeilinger~\cite{sv1,zeil-99}, or Summhammer~\cite{Summhammer_1994}
have emphasized the empirical aspect of physical category formation~\cite{Hardy_2007}.
Hertz~\cite{hertz-94e} also highlighted the idealistic nature of physical `images' (or mental categories)
that we construct to represent external objects,
and how these formal structures should remain consistent with, and connected to, the empirical events or outcomes:
'We form for ourselves images or symbols of external objects; and the form which we give them is such that the necessary
consequences of the images in thought always mirror the images of the necessary consequences in nature of the things pictured.'
In these perspectives, physical theories may seem to reflect ontology, but at their core, they are essentially epistemic.

In the subsequent discussion, our focus will be on the construction of space-time frames, not in a (perhaps) Newtonian or Kantian sense,
portrayed as premeditated `as they are', but rather in a Leibnizian sense, constructing them as they can be by the available operational means~\cite{Ballard_1960}.
As stated by Leibniz\cite[p.~14]{Leibniz2000Mar}, ``space
[[is]] something purely relative, as time is---[[space is]] an order of coexistences, as time is an order of successions.''
The forthcoming argument will contend that entangled quantum states do not appear to provide the means
for such spatial order of coexistences, nor for any order of successions.


We also need to be aware of the conventions involved in constructing space-time frames.
One such convention is the frame-independent determination of the velocity of light~\cite{pet-83,peres-84},
which means that light cones remain unchanged.
Alongside the assumption of bijective mappings of space-time point labels in distinct coordinate frames,
this convention, preserving the quadratic distance (Minkowski metric) of zero,
leads to affine Lorentzian transformations~\cite{alex1,lester}.

These conventions formally imply and define the Lorentz transformations of the theory of special relativity.
They are inspired by physics but lack inherent physical content themselves.
Their physical significance arises from the preservation of the form invariance of equations of motion, such as Maxwell's equations,
under Lorentz transformations that include (the conventionally defined constant and frame-independent)
velocity of light.

When it comes to reconstructing space-time frames from quantum events,
it is essential to keep in mind that quantum measurements essentially result in
`(ir)reversible'~\cite{PhysRevA.25.2208,greenberger2,Ma22012013} detections in detectors.
As a result, we must (re)construct quantum space-time frames primarily based on these detections.

As long as those detections are statistically independent, we can synchronize time at different locations using
radar (`round-trip', `two-way') coordinates obtained by sending a (light-in-vacuum) signal back and forth (in a zig-zag manner)
between the respective locations~\cite{ein-05,Einstein_1910,Jammer2006Nov,Minguzzi_2011}.
A formal expression of the statistical independence of two events, outcomes, or observables $L$ and $R$
is the fact that their joint state $\Psi_{LR}$ can be written as the product
of their individual states $\Psi_{L}$ and $\Psi_{R}$, i.e., $\Psi_{LR} = \Psi_{L} \Psi_{R}$.
These states are then nonentangled and separable with respect to the observables $L$ and $R$.

\section{Nonfactorization and the lack of mutual, relational choice}

But what about entangled states? In this case, independence cannot be assumed, as
by definition, the joint state is not a product of the constituent states.
Quantum entangled states are encoded relationally~\cite{schrodinger-gwsidqm2,zeil-99,zeil-Zuk-bruk-01}.
They lack distinctness between their constituents.
A formal expression of such a quantum relational encoding is the outcome dependence
of two respective events, outcomes, or observations $L$ and $R$ belonging to the registrations of those entangled particle pairs.

However, outcomes on either side $L$ or $R$ maintain their statistical parameter independence,
which means that any parameter measured at $L$ does not affect the outcome
or any other operationally accessible observable at $R$, and vice versa.
In Shimony's terminology~\cite{shimony2,shimony_1993}, ``an experimenter
at $R$, for example, cannot affect the statistics of outcomes at $L$ by selective measurements''.
This can be ensured by the value indefiniteness of the respective outcomes, which appear irreducibly random
with respect to all physical operational means deployable by an intrinsic observer.


Since the product rule does not hold for quantum entangled states, we cannot assume that the respective individual outcomes
are guaranteed to be mutually separate or mutually distinct in these observables.
However, the factorization of states guarantees a specific feature that is crucial for radar coordinates: choice.
Simultaneity conventions require the capacity to independently select space-time labels for both types of measurements (parameter independence) and their outcomes,
regardless of what is being measured and recorded elsewhere.
Outcome independence, along with the resulting temporal and spatial distinctiveness,
is essential for establishing any internally operational space-time scale.
%However, without such temporal and spatial distinctiveness and the associated possibility of choice, no space-time scale, in particular, no clock, can be generated.

Without the freedom to make choices regarding spatio-temporal labeling,
the concept of clocks and the measurement of space and time they provide becomes unattainable.
Indeed, distinct labels require a distinction among entities to be labeled.
However, for quantum entangled states that have traded individuality for relationality,
there is no distinction concerning the respective observables.


Suppose, for the moment, an isolated mini-universe composed
of entangled states, such as the singlet Bell state $\vert \Psi^-(12) \rangle$ from the Bell basis
\begin{equation}
\begin{split}
\vert \Psi^\pm (12) \rangle
=
\frac{1}{2}\left(
\vert 0_1 1_2 \rangle
\pm
\vert 1_1 0_2 \rangle
\right)
,\\
\vert \Phi^\pm (12) \rangle
=
\frac{1}{2}\left(
\vert 0_1 0_2 \rangle
\pm
\vert 1_1 1_2 \rangle
\right)
.
\end{split}
\label{2023-st-EPRBstates}
\end{equation}
In maintaining the order of the terms---the first and second (from left to right) entries
refer to the first and second constituents, respectively---we may omit the subscripts.

Typically, these constituents are understood to be spatially separated, often under strict Einstein locality conditions.
For example, Einstein, Podolsky, and Rosen (EPR) employed such spatially separated configurations to argue against the `completeness' of
quantum mechanics~\cite{epr,Howard1985171}.
However, as we do not wish to confine ourselves to spacelike entanglement,
we aim to also encompass timelike entanglement. This type of entanglement can---in the customary space-time frames that
we assume to be ad hoc creations of certain non-entangled elements, like light rays of classical optics,
in the standard Poincar\'e-Einstein protocols mentioned earlier---be generated through processes such as delayed-choice entanglement swapping.
Formally, achieving this involves reordering the product $\vert \Psi^-(12) \Psi^-(34) \rangle$,
expressed in terms of the four individual product states
$\vert \Psi^+(14)   \Psi^+(23)  \rangle$,
$\vert \Psi^-(14)   \Psi^-(23)  \rangle$,
$\vert \Phi^+(14)   \Phi^+(23)  \rangle$, and
$\vert \Phi^-(14)   \Phi^-(23)  \rangle$
of the Bell bases of the ``outer'' (14) and ``inner'' (23)
particles~\cite{Zuk-1993-entanglementswapping,Megidish_2013,peres-DelayedChoiceEntanglementSwapping,svozil-2016-sampling}.
Bell state measurements of the latter, `inner' particles yield a rescrambling of the `outer'
correllations, which can then be postselected into the desired `outer'
Bell states, respectively.

Delay lines serve as essential components for temporal entanglement.
These delay lines could, in principle, also lead to mixed temporal-spatial quantum correlations, where,
for instance, pairs (12) are spatially entangled while pairs (34) are temporally entangled,
resulting in an `outer' pair (14) that is both spatially and temporally entangled.

We note in passing that temporally entangled shares (as well as mixed temporal-spatial ones) could lead to standard
violations of Bell-Boole type inequalities---for instance, at a single point in space but at different times.
The derivation seems to be straightforward:
all that is required is a respective Hull computation
of the classical correlation polytope~\cite{froissart-81,pitowsky-86},
followed by the (maximal) quantum violation of the inequalities that represent the edges of
the classical polytope~\cite{cirelson:80,filipp-svo-04-qpoly-prl}.
One of the reasons for the seamless transfer of spatial as well as temporal variables
is their interoperability and their realization using delay lines, when necessary.

While considering the question of whether and how such entangled shares could lead to space-time scales, and ultimately frames,
we make three observations:
Firstly, the two `constituents' of the relationally entangled share reveal themselves, if compelled
into individual events, through two random outcomes that are mutually dependent due to quantum correlations
in the form of the quantum cosine expectation laws.
These single individual outcomes are expected to be independent of the experiments or parameters
applied on the respective `other side' or at the `other time'.

Secondly, these correlations surpass the classical linear correlations~\cite{Peres222}
for almost all relative measurement directions (except for collinear and orthogonal ones).
However, since the outcomes are only dependent on outcomes and not on parameters, this does not lead to
inconsistencies with classical space-time scales generated by the conventional classical Poincar\'e-Einstein
synchronization convention. Indeed, even `stronger-than-quantum' correlations such as a Heaviside
correlation function would, under these conditions, not result in violations of causality through faster-than-light signaling.
One could also argue that, given outcome dependence but parameter independence,
the space-time labeling is arbitrary.

Thirdly, since individual outcomes cannot be controlled, any synchronization convention and protocol
that depends on controlled outcomes cannot be carried out with entangled shares,
as there is no means of transmitting (arrival and departure) information
'across those shares.'
Signaling from one space-time point to another assumes choice,
yet the form of relational value definiteness that comes at the expense of individual value definiteness,
originating from the unitarity of quantum evolution,
between two or more constituents of a quantum entangled share
prevents signaling across its constituents.

%\section{Patchwork of space-time frames}



\section{Construction of internal operational spatio-temporal frames}

Although entanglement does not provide a means for scale synchronization, it can be utilized for synchronizing directions as well as orthogonality among different frames.

Suppose that all observers agree to `measure the same observable.'
It's important to note that, at this stage, we have not yet established a spatial frame. Therefore, for example, an observable
like the `direction of spin' (or, for photons, linear polarization) is initially undefined.
It must be defined in terms of quantum mechanical entities, such as the state~(\ref{2023-st-EPRBstates}), and observables.
Ultimately, this process involves the interpretation of detections in a detector.

Directional synchronization of spatio-temporal frames can be established, for instance, through the state~(\ref{2023-st-EPRBstates})
by employing successive measurements of particles in that state.
In this manner, the directions can be synchronized by maximizing correlations.

Three- and four-dimensionality can also be established by exploiting correlations---either by mutually minimizing them through
mutually unbiased bases~\cite{Schwinger.60}, or by maximizing them
through the establishment of mutual spatial orthogonality by systematically varying measurements in configuration space.


\section{Controllable nonlocality and parameter dependece of outcomes due to nonlinearity of quantum field theory?}

We could hope that the addition of non-linearity via interactions or statistical effects---for example, higher-order perturbation expansions---might help overcome the parameter independence of outcomes in an EPR-type setup.
However, as of now, there is no indication of any violation of Einstein locality in field theory~\cite{shirokov,Hegerfeldt_1998,Perez_PhysRevD.16.315,Svidzinsky-PhysRevResearch.3.013202}.


In my earlier publications~\cite{svozil-slash}, I have speculated that if one constituent of an EPR pair were to enter a region
of high or low density of a particular particle type---for instance, 'boxes of particles in state $\vert 0 \rangle$'---then stimulated emission might
encourage the corresponding state of the constituent to 'materialize' with a higher or lower probability.
This, in turn, could be a scenario for parameter dependence of outcomes, even under strict Einstein locality conditions.


\section{Summary and afterthoughts}

As argued earlier, there is no independent choice among the individual outcomes of entangled particles:
an observer at the 'one constituent end' of an entangled share has no ability to select or establish a specific time as a pointer reading.

These considerations are not directly related to the 'problem of (lapse of) time' that has led to the notion of a
fictitious stationary 'external' versus an 'intrinsic' time~\cite{Page_1983,Wootters_1984,Moreva_2014}
by equating it with the measurement problem in quantum mechanics.

The adage that ``If $\ldots$ two spacetime regions are spacelike separated,
then the operators should commute''~\cite{Hardy_2007}
implicitly supposes two assumptions:

\begin{itemize}
\item[(i)] Firstly, Einstein's separation criterion (German 'Trennungsprinzip'~\cite[537-539]{Meyenn-2011}),
which states that relativity theory, and in particular its causal structure determined by light cones,
applies to observables formalized as operators.
A Newtonian space-time frame, even in the modified versions proposed by Poincar\'e
and Einstein, is not applicable.
Therefore, we cannot depend on a pre-existing space-time structure for operators to commute,
and consequently, for observables to be independent.

\item[(ii)] Secondly, it assumes that states are distinct from operators, even though
pure states can be re-interpreted as the formalization of observables; specifically, as the assertion that the system is in the respective state.
\end{itemize}


Our approach diverges from Einstein, who, in a letter to Schr\"odinger~\cite{Meyenn-2011,Howard1985171}, emphasizes that
following a collision that entangles the constituents $L$ and $R$, the actual state of $LR$
comprises the actual state of $L$ and the actual state of $R$, which are unrelated,
and that the real state of $L$ (due to possible spacelike separation)
cannot be influenced by the type of measurement conducted on $R$.
It is important to note that not all observables of a collection of particles may be entangled; some could be factorizable. In this case, the latter type of observables may still be applicable for
the creation of relativistic space-time frames, unlike the entangled ones.

Since Poincar\'e-Einstein synchronization via radar coordinates requires a choice and thus parameter dependence, the utilization of entangled states becomes impossible.
Hence, we are restricted to separable states. The separability and value definiteness of components within a physical system ultimately boils down to the measurement problem in quantum mechanics.
This measurement problem, which involves understanding how an entangled system experiences 'individualization' under strictly unitary transformations, with associated value definite information on individual components of the system, remains notoriously unresolved.
We must acknowledge that, at least for now,
Poincar\'e-Einstein synchronization for quantized systems can be carried out for all practical purposes (FAPP),
but it remains fundamentally unresolved.

We observe that, in the case of relationally encoded entangled quantum states, there is no spatio-temporal resolution. However, due to parameter independence, this type of 'nonlocality' cannot be exploited for signaling or radar coordination.
Without individuation and measurement, there can be no operational significance assigned to space-time.
From this perspective, quantum coordinatization reduces to quantum measurements, which,
at least in the author's view, remains unresolved.


\begin{acknowledgments}
This research was funded in whole, or in part, by the Austrian Science Fund (FWF), Project No. I 4579-N.

The author declares no conflict of interest.
\end{acknowledgments}

\bibliography{svozil}

%\end{widetext}
\end{document}

Then, a time scale in $n$ different locations can be established by labeling points on those frames
according to the arrivals of the respective constituents of those successive states.
We may state that the time of the specific space-time point (among $n$)
when the constituent of the first share is detected (by a click in one of the detectors labeled ``0'' or ``1'',
regardless of which one) is temporally labeled as "0".

Sequentially, the specific space-time point when the constituent of the second (and subsequent) share
is detected is temporally labeled as "1" (and so on).
In fact, if all clocks are identical,
we only require one share to determine the time offset accurately.

Alternativey, we my chose the temporal labels, in a BB84~\cite{benn-84} type protocol,
not by the arrival time(s) but by the values observed, that is, by the value of the detector clicks.
Such synchronization conventions need not be restricted to quantum mechanical entangled shares,
but for quantum shares there is no apparent other way of how to proceed, as points appear not separated.
Indeed, if


\subsection{Construction of internal operational temporal scales}

In order to create time scales we need to assume that some form of temporal succession or order can be established.
This assumption presupposes temporal factorizability of ``successive'' states.
But any such order of the shares is not inherent in the quantum states.
Indeed, temporally entangled states
exhibit indefinite causal order~\cite{Oreshkov_2012,GoswamiPhysRevLett.121.090503}.

For the sake of a classical reconstruction of time frames we shall assume that there exists a temporal succession of states.
Suppose the $n$ observers have identical clocks.
We could, for instance, nominally fixate
two or more of entangled shares of the type of the ones in Equation~(\ref{2023-st-GHZstates})
in a pre-arranged order.

(* Mathematica code *)

psi12m = (1/Sqrt[2]) (H1*V2 - V1*H2);
psi12p = (1/Sqrt[2]) (H1*V2 + V1*H2);
phi12m = (1/Sqrt[2]) (H1*H2 - V1*V2);
phi12p = (1/Sqrt[2]) (H1*H2 + V1*V2);

psi34m = (1/Sqrt[2]) (H3*V4 - V3*H4);
psi34p = (1/Sqrt[2]) (H3*V4 + V3*H4);
phi34m = (1/Sqrt[2]) (H3*H4 - V3*V4);
phi34p = (1/Sqrt[2]) (H3*H4 + V3*V4);

psi14m = (1/Sqrt[2]) (H1*V4 - V1*H4);
psi14p = (1/Sqrt[2]) (H1*V4 + V1*H4);
phi14m = (1/Sqrt[2]) (H1*H4 - V1*V4);
phi14p = (1/Sqrt[2]) (H1*H4 + V1*V4);

psi23m = (1/Sqrt[2]) (H2*V3 - V2*H3);
psi23p = (1/Sqrt[2]) (H2*V3 + V2*H3);
phi23m = (1/Sqrt[2]) (H2*H3 - V2*V3);
phi23p = (1/Sqrt[2]) (H2*H3 + V2*V3);




FullSimplify[  psi12m*psi34m]

FullSimplify[  psi12m*psi34m == (1/2) (
psi14p*psi23p
- psi14m*psi23m
- phi14p*phi23p
+ phi14m*phi23m
) ]


(* magic basis *)

psi12m = (1/Sqrt[2]) (H1*V2 - V1*H2);
psi12p = (1/Sqrt[2]) (H1*V2 + V1*H2);
phi12m = (1/Sqrt[2]) (H1*H2 - V1*V2);
phi12p = (1/Sqrt[2]) (H1*H2 + V1*V2);

psi34m = (1/Sqrt[2]) (H3*V4 - V3*H4);
psi34p = (1/Sqrt[2]) (H3*V4 + V3*H4);
phi34m = (1/Sqrt[2]) (H3*H4 - V3*V4);
phi34p = (1/Sqrt[2]) (H3*H4 + V3*V4);

psi14m = (1/Sqrt[2]) (H1*V4 - V1*H4);
psi14p = (1/Sqrt[2]) (H1*V4 + V1*H4);
phi14m = (1/Sqrt[2]) (H1*H4 - V1*V4);
phi14p = (1/Sqrt[2]) (H1*H4 + V1*V4);

psi23m = (1/Sqrt[2]) (H2*V3 - V2*H3);
psi23p = (1/Sqrt[2]) (H2*V3 + V2*H3);
phi23m = (1/Sqrt[2]) (H2*H3 - V2*V3);
phi23p = (1/Sqrt[2]) (H2*H3 + V2*V3);




FullSimplify[  psi12m*psi34m]


Solve[  psi12m*psi34m == (1/2) (
ap*psi14p*psi23p
+ am*psi14m*psi23m
+ bp*phi14p*phi23p
+ bm*phi14m*phi23m
),{ap,am,bp,bm}]

