%\documentclass[amsmath,table,sans,amsfonts, handout]{beamer}
\documentclass[amsmath,table,sans,amsfonts]{beamer}
\usepackage[T1]{fontenc}
%%\usepackage{beamerthemeshadow}
%%\usepackage[headheight=1pt,footheight=10pt]{beamerthemeboxes}
%%\addfootboxtemplate{\color{structure!80}}{\color{white}\tiny \hfill Karl Svozil (TU Vienna)\hfill}
%%\addfootboxtemplate{\color{structure!65}}{\color{white}\tiny \hfill mur.sat \hfill}
%%\addfootboxtemplate{\color{structure!50}}{\color{white}\tiny \hfill Graz, 2010-12-11\hfill}
%\usepackage[dark]{beamerthemesidebar}
%\usepackage[headheight=24pt,footheight=12pt]{beamerthemesplit}
%\usepackage{beamerthemesplit}
%\usepackage[bar]{beamerthemetree}
\usepackage{graphicx}
\usepackage{pgf}
%\usepackage{eepic}
%\usepackage[usenames]{color}
%\newcommand{\Red}{\color{Red}}  %(VERY-Approx.PANTONE-RED)
%\newcommand{\Green}{\color{Green}}  %(VERY-Approx.PANTONE-GREEN)


%\RequirePackage[german]{babel}
%\selectlanguage{german}
%\RequirePackage[isolatin]{inputenc}

%\pgfdeclareimage[height=0.5cm]{logo}{tu-logo}
%\logo{\pgfuseimage{logo}}
\beamertemplatetriangleitem
%\beamertemplateballitem

\beamerboxesdeclarecolorscheme{alert}{red}{red!15!averagebackgroundcolor}
%\begin{beamerboxesrounded}[scheme=alert,shadow=true]{}
%\end{beamerboxesrounded}

%\beamersetaveragebackground{yellow!10}

%\beamertemplatecircleminiframe

\newtheorem{question}{Question}
\newtheorem{conjecture}[question]{Principle}
\newtheorem{challenge}[question]{Challenge}
\usepackage{tikz}
\newcommand{\bra}[1]{\left< #1 \right|}
\newcommand{\ket}[1]{\left| #1 \right>}

\newcommand{\iprod}[2]{\langle #1 | #2 \rangle}
\newcommand{\mprod}[3]{\langle #1 | #2 | #3 \rangle}
\newcommand{\oprod}[2]{| #1 \rangle\langle #2 |}

\newcommand{\proj}[3]{\begin{smallmatrix} #1 & #2 & #3 \end{smallmatrix}}
\newcommand{\projbf}[3]{\begin{smallmatrix} \mathbf{#1} & \mathbf{#2} & \mathbf{#3} \end{smallmatrix}}

\sloppy
\parskip .7em %vskip between paragraphs

\newcommand{\seq}[1]{\mathbf{#1}}
\newcommand{\floor}[1]{\left\lfloor #1 \right\rfloor}
\newcommand{\ceil}[1]{\left\lceil #1 \right\rceil}
\newcommand{\m}[1]{\widetilde{#1}}
\newcommand{\p}[1]{\scriptsize\textcolor{black}{$[#1]$}}

\begin{document}

\title{\bf \textcolor{blue}{Quantum Random Number Generators}}
\subtitle{\textcolor{purple!60}{\small http://tph.tuwien.ac.at/$\sim$svozil/publ/2017-Svozil-Shaping the future Quantum Technology Flagship - WS, 18.1.2017-pres.pdf
\\
DOI: 10.3354/esep00171, arXiv:1605.08569
}}
\author{Karl Svozil}
\institute{ITP/Vienna University of Technology, Austria\\
\& CS/University of Auckland, NZ  \\
svozil@tuwien.ac.at
%{\tiny Disclaimer: Die hier vertretenen Meinungen des Autors verstehen sich als Diskussionsbeitr�ge und decken sich nicht notwendigerweise mit den Positionen der Technischen Universit�t Wien oder deren Vertreter.}
}
\date{Vienna, January 18th, 2017}
\maketitle


% \frame{
% \frametitle{Contents}
% \tableofcontents
% }

\section{Principles}

 \frame{
 \frametitle{Early QRNGs using quantum complementarity}


\begin{itemize}
%\pause
\item Quantum coin toss: prepare in a (pure) state, measure in another (non-orthogonal) direction
[KS,  1990, DOI: 10.1016/0375-9601(90)90408-G;~$\ldots$]

%\pause
\item Realizations by various groups (eg,
Jennewein, Zeilinger et al, 2000, DOI: {10.1063/1.1150518};
Stefanov, Gisin et al., 2000, DOI:   10.1080/095003400147908;
F\"{u}rst, Weinfurter et al, 2010, DOI: {10.1364/OE.18.013029};
Quantis TRNG (True Random Number Generator) - ID Quantique, 2000--2017;~$\ldots$)
%\pause
\item Potential problem: complementarity has classical models; no guarantee for value indefiniteness (eg, KS, 2009, DOI: 10.1016/B978-0-444-52869-8.50015-3)
\end{itemize}
}

 \frame{
 \frametitle{QRNG featuring quantum value indefiniteness}


\begin{itemize}
%\pause
\item  Quantum value indefiniteness in higher dimensional ($D\ge 3$) systems; no classical {\it double}
(eg,
Pitowski, 1998, DOI:10.1063/1.532334;
KS, 20109, DOI: 10.1103/PhysRevA.79.054306;
Abbot et al, 2012-2015, DOI: 10.1103/PhysRevA.86.062109, 10.1103/PhysRevA.89.032109, 10.1063/1.4931658;~$\ldots$);

%\pause
\item Realizations by various groups (eg,
Hai-Qiang et al, 2004, DOI: {10.1088/0256-307X/21/10/027};
Pironio et al., 2010, DOI:   10.1038/nature09008;~$\ldots$)
%\pause
\item Challenge: dim-3 (qtrits); GHZ-type realization (strict nonstochastic violation of classical predictions).
\end{itemize}
}

\section{Questions and challenges}

 \frame{
 \frametitle{Questions and challenges}


\begin{itemize}
%\pause
\item Normalization of bias of (non)independent events
(eg, von Neumann,  1951, Various Techniques Used in Connection With Random Digits;~$\ldots $)

\item
Where exactly is the randomness located/grounded? Beam splitters are unitary (one-to-one) elements; nesting argument of Everett DOI: 10.1103/RevModPhys.29.454
\& Wigner DOI: 10.1007/978-3-642-78374-6\_20

%\pause
\item Claims of absolute and irreducible randomness are provable unprovable (by reduction to the recursive unsolvability of the halting and the rule inference problem).
%\pause
\end{itemize}
}

\section{General questions and challenges}

 \frame{
 \frametitle{General questions and challenges}


\begin{itemize}
%\pause
\item What is the particular source of quantum speedups (eg, entanglement, coherence$=$parallelism, partitioning of information)?

%\pause
\item How to cope with man-in-the middle attacks on quantum cryptography? How much authentification is needed for key growing?
Claims of ``absolute security'' wrt qc; Specker's ``Jesuit lies''
%\pause
\item Where is a "killer-app" in the zoo of qantum algorithms http://math.nist.gov/quantum/zoo/~?

\item Ethics and certification issues related to science marketing; in particular the ``quantum mechanis is magic/hocus pocus/abracadabra'' tour:
KS, Ethics in Science and Environmental Politics (ESEP),
 DOI: 10.3354/esep00171, arXiv:1605.08569
\end{itemize}
}



\frame{

\centerline{\Large {\color{magenta} Thank you for your attention!}}

\begin{center}\color{orange}
$\widetilde{\qquad \qquad }$
$\widetilde{\qquad \qquad}$
$\widetilde{\qquad \qquad }$
\end{center}
 }
 \end{document}
