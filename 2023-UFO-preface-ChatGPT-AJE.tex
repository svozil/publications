%%%%%%%%%%%%%%%%%%%%%%%%preface.tex%%%%%%%%%%%%%%%%%%%%%%%%%%%%%%%%%%%%%%%%%
% sample preface
%
% Use this file as a template for your own input.
%
%%%%%%%%%%%%%%%%%%%%%%%%%%%%%%%%%%%%%%%%%%%%%%%%%%%%%%

\preface

\index{preface}

This may well be a book about a nonexistent subject.

I know that I may not make friends with this statement. UFO enthusiasts will be upset because I seemingly throw their subject of study under the bus.

Conventional, conformist scientists will ask: ``So why all the effort? Does he have no other things to do than chase illusions?''
The answer I may give is that the claims are extraordinary, if not outrageous, and the evidence is not rich.

However, if only part of what has been observed, experienced, and claimed is true, the consequences and expectations are formidable.
Take, for example, the kinetic flight characteristics of some of these objects.
They exhibit a zigzag-style motion with abrupt and powerful changes in direction and velocity.
These movements can be observed both by the naked eye and through radar or infrared sensors.
The mere possibility of such seemingly ``inertialess'' motion is stunning and encourages radically
novel ways of thinking about physics.

However, before I get carried away, I should remember that other scientific fields may experience similar situations.
A friend in mathematics shared a story with me about a PhD thesis in abstract algebra that had promising representation theorems.
Unfortunately, after the thesis was submitted, it was revealed that the only example was the empty set.

With that in mind, I acknowledge that I may be writing extensively about a category with no objects or only prosaic and mundane objects.
Undoubtedly, ``UFOs' or ``UAPs''---in the literal meaning of Unidentified (Flying) Objects or Unidentified Anomalous Phenomena---exist.
There will always be objects or uncategorized entities populating the atmosphere, cruising the oceans, or navigating space.
However, they may have very prosaic, mundane explanations and may be man-made phenomena such as weather events such as lenses of air or
ice crystals in the sky~\cite{Menzel_1953}, aircraft, or other objects such as balloons, outdoor laser lights, and xenon aerial searchlights.
Indeed, upon closer inspection and
depending on the veracity of reports and the quality of sensor data, more than 95{\%}
(and sometimes even more than 99{\%}~\cite{AA-Condon-1970})
of UFO sightings are found to have rather mundane origins.

Nevertheless, despite all the prosaic explanations and debunking, some sightings of ``hard cases''
remain unexplained and are the subject of speculation and debate.
These remaining cases are anomalies or phenomenologic outliers:
objects or phenomena in the sky that cannot be identified or explained by conventional scientific knowledge.
However,
just because they appear sporadically,
cannot be summoned or replicated easily and present a form of a phenomenological outlier or anomaly,
does not necessarily mean that one should dismiss flying saucers.
Rather than look the other way and disregard those ``hard cases,'' one needs to see them as challenges that may,
at some point, greatly enhance if not revolutionize both our knowledge as well as our technologies.
Methodologically and optimistically, I suppose that UFOs and UAPs represent an epistemic challenge,
not an ontological one: we shall pursue the thesis that we do not know what they are,
but in principle, this ignorance can be overcome by scientific methods.


Scientists often have encountered strange phenomena which later turned out to be instrumental
for entirely new capacities and capabilities.
Take, for instance, the case of electric rays or torpedo fish,
delivering to its prey what is nowadays conceptualized by electric charge.
This phenomenon had been known since antiquity and later, with John Walsh, John Hunter, Henry Cavendish, and others' research,
turned out to be an instance of ``animal electricity'' or bioelectricity.
To draw another historic analogy: disregarding the ``peculiar'' (for contemporaries) and incomprehensible form of the blackbody
radiation in fin de si\`ecle statistical physics would have delayed the development of quantum mechanics.
Planck's ``act of desperation''~\cite[p.~23]{Hermann1974Nov}, later extended by Einstein's hypothesis on light quantum,
initiated a pioneering research program that dominated the upcoming century of physics.


Why should we care about UFOs?
Because caring about them may align
with a highly successful principle that has often been wrongly~\cite{Kleinert_2009} attributed to Galileo Galilei~\cite[p.~139]{weyl:49}:
``to measure what is measurable and to try to render  measurable what is not so as yet.''
However, we should be concerned about what  exactly  is still unmeasurable.
It could turn out to be a great ``nothingburger,'' or, even worse,  ``bullshit''~\cite{Frankfurt-OnBullshit}.

To quote Major General John Samford, the Air Force's director of intelligence, who,
in reaction to the July 1952 Washington DC flyovers,
started the arguably largest press conference since the end of the Second World War~\cite{Lewis-Kraus2021Apr},
in a maybe well-prepared, enacted serious, grave, and sober demeanor with the words~\cite{Archives1952,Samford1952},
\begin{svgraybox}
``I am here to discuss the so-called flying saucers.
Air Force interest in this problem has been due to our feeling of an obligation to identify and analyze to the best of our ability,
anything in the air that may have the possibility of threat or menace to the United States.
In pursuit of this obligation since 1947,
we have received and analyzed between one and two thousand reports that have come to us from all kinds of sources.
Of this great mass of reports we have been able adequately to explain the great bulk of them, explain them to our own satisfaction.
We've been able to explain them as hoaxes, as erroneously identified friendly aircraft,
as meteorological or electronic phenomena, or as light aberration.''

``However, there have been a certain percentage of this volume of reports that have been made by credible observers of relatively incredible things.''
\end{svgraybox}

There is a long history of coping with ``relatively incredible things,'' or in more graphical terms, ``the woo,'' in a rational way.
Hume discussed how to cope with miracles~\cite{Hume-Enquiry,frank,franke}.
Laplace offered both encouragement and a critical warning~\cite{Laplace1814,Flournoy1900}:
\begin{svgraybox}
``We are thus far from recognizing all the agents of nature and their diverse modes of action
that it would be unphilosophical to deny the phenomena solely because they are inexplicable in the present state of our knowledge.
However, we ought
to examine them with an attention as much the more scrupulous as it appears the more difficult to admit them~$\ldots$''
\end{svgraybox}




Later, Russell, in a related mood, created a pragmatic teapot metaphor, often referred to as ``Russel's teapot,''\index{Russel's teapot}
for categories lacking empirical foundations (such as God,
but that does not exclude direct knowledge by believe and providence~\cite{Sagan-Contact})~\cite{Russell1952}:
\begin{svgraybox}
``Many orthodox people speak as though it were the business of sceptics
to disprove received dogmas rather than of dogmatists to prove them. This
is, of course, a mistake. If I were to suggest that between the Earth and
Mars there is a china teapot revolving about the sun in an elliptical orbit,
nobody would be able to disprove my assertion provided I were careful to
add that the teapot is too small to be revealed even by our most powerful
telescopes. But if I were to go on to say that, since my assertion cannot be
disproved, it is intolerable presumption on the part of human reason to
doubt it, I should rightly be thought to be talking nonsense.''
\end{svgraybox}

Russell's teapot analogy argues that the burden of proof is on the believer in the absence of evidence
and that it is not rational to believe in something just because it cannot be disproved.
He compares the existence of a small, undetectable teapot in orbit around the sun to religious beliefs
that lack evidence and suggests that just as the teapot's existence is highly unlikely, the same can be said for these religious beliefs.
The fact that they cannot be disproved does not make them any more likely to be true.

In their book ``The Demon-Haunted World: Science as a Candle in the Dark''~\cite{Sagan1997Feb}  Carl Sagan  and Ann Druyan
argue similarly: their ``baloney detection kit'' contains advice what not to do. One of these wrong principles is
what they call ``appeal to ignorance:'' the claim that whatever has not been proved false must be true.
However, absence of evidence is not evidence of absence.

Laplace and Russell's arguments can be transcribed into the position that
believing in the existence of flying saucers solely based on unconfirmed anecdotal hearsay and the inclinations
of their proponents is unreasonable, even if immediate dismissal cannot be justified.
Support of highly speculative claims should be provided by the claimants.
They must produce substantial, rather than just circumstantial, evidence.


In regard to acceptance of those phenomena, the situation is not entirely dissimilar to that of a cult:
We are expected to give credence to hearsay and commitment to authority.
This is true for those who are ``in-the-know' of UFOs---most likely so-called ``experiencers' that might just be a tuned-down form of ``abductees.'
There have been many reported sightings of UFOs throughout history,
but there is no openly available conclusive scientific evidence
to support the existence of extraterrestrial spacecraft or otherworldly beings visiting Earth.
There are many speculative theories and hypotheses about the nature of UFOs and the possibility of extraterrestrial life,
but these are largely based on speculation and are not supported by scientific evidence.
The scientific community generally takes a skeptical approach to the existence of extraterrestrial life and does not consider
``raw'' UFO sightings as evidence of their existence.

However, cult-like adherence is often also implied by deniers,
for representatives of organizations, corporations, and executives of the state and of the military who continuously argue
that there is nothing to see here, and all efforts going into researching that phenomenon---even if it exists---yield nothing useful and of value,
and thus are wasteful and cause unnecessary opportunity costs.


Certainly, the fact that certain phenomena may appear anomalous or strange to us and cannot be explained
by our current scientific understanding does not necessarily imply the involvement of aliens or entities from some unknown realm.
However, should we throw out the proverbial ``baby with the bathwater'' and discard these phenomenologic outliers entirely?
This is an issue that most official committees of experts in the USA and UK had to cope with.
Almost always, their recommendations were paradoxical.
While they weakly recommended further research, they mostly suggested the cessation of existing studies.
However, discarding empirical anomalies might save us from much effort, costs, and trouble, but it could potentially block the advancements of science. In any case, straightforward answers to these issues are not apparent,
and we should be fully aware of the consequences of our decisions in this regard.



Let me finally make two confessions: first, I have never experienced a ``live UFO broadcast'' in real-time.
Therefore, I am not among those who are strong UFO supporters ``in-the-know'' or damned~\cite{FortBotD}
and do not require any further evidence.

Second, I may suffer from a cargo mindset, hoping that at some point, either voluntarily or forcefully,
some inhabitants of these vehicles, if they exist, will divulge the secrets encoded in their propulsion drive.
This would enable humankind to free itself from the bounds of Earth and travel quickly and conveniently to the stars,
experiencing a smooth ride through the difficulties---``per aspera ad astra.''

It could be enjoyable to cultivate these sciences or technologies independently and autonomously,
without any ``alien Prometheus'' who, similar to the figure in Greek mythology, might disobey nonfraternization and gift the ``fire''
of their knowledge  to humanity.
However, I do not believe that at this point we need to, or can, take this step easily and purely on our own, without any help, advice or guidance.

The quest for interstellar technologies may prove to be a lengthy and tiresome one.
While traveling between planets within our solar system may be challenging yet endurable,
embarking on a journey to a remote solar system without rapid propulsion methods may not be a prudent decision.
It may be more sensible to wait for technology to advance beyond our current capabilities before launching a
spacecraft to a far-off destination.

Nevertheless, we can still ponder over the peculiar flight features of unidentified flying objects
and attempt to unravel the secrets behind their abilities.
At the very least, exploring the potential of such capabilities could motivate us to push the boundaries of what is conceivable and achievable,
and inspire further research.

In the worst-case scenario, our quest to understand the UFO phenomenon could evolve into a scientifically oriented cargo cult
in its own right, driven by an unrealistic ambition to attain flight characteristics, methods,
and capabilities that are either impossible to achieve in this universe or by far exceed our current technological capabilities.
In this ``cargo--driven'' sense, the UFO phenomenon embodies our hopes, aspirations, and objectives, encompassing all that we endeavor to achieve,
and we must be mindful of this, whatever the outcome.


This is why I believe that it is important to approach the topic of UFOs and extraterrestrial life with an open mind and healthy skepticism and to base any conclusions on solid scientific conduct---regardless where it leads us.
%% Please write your preface here
%Use the template \emph{preface.tex} together with the document class SVMono (monograph-type books) or SVMult (edited books) to style your preface.

%A preface is a book's preliminary statement, usually written by the \textit{author or editor} of a work, which states its origin, scope, purpose, plan, and intended audience, and which sometimes includes afterthoughts and acknowledgments of assistance.

%When written by a person other than the author, it is called a foreword. The preface or foreword is distinct from the introduction, which deals with the subject of the work.

%Customarily \textit{acknowledgments} are included as last part of the preface.


\vspace{\baselineskip}
\begin{flushright}\noindent
March 2023\hfill {\it Karl  Svozil}\\
%month year\hfill {\it Firstname  Surname}\\
\end{flushright}
