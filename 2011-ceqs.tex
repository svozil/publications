\documentclass[%
 %reprint,
 %superscriptaddress,
 %groupedaddress,
 %unsortedaddress,
 %runinaddress,
 %frontmatterverbose,
 preprint,
 showpacs,
 showkeys,
 preprintnumbers,
 %nofootinbib,
 %nobibnotes,
 %bibnotes,
 amsmath,amssymb,
 aps,
 % prl,
 pra,
 % prb,
 % rmp,
 %prstab,
 %prstper,
  longbibliography,
 %floatfix,
 %lengthcheck,%
 ]{revtex4-1}

 \usepackage{graphicx}% Include figure files

 \begin{document}

\title{Cogito ergo quantum sum}

\author{Daniel M. Greenberger}
\affiliation{Department of Physics, City College of the City University of New York     \\
  New York, NY 10031, USA}
\email{greenbgr@sci.ccny.cuny.edu}

\author{Karl Svozil}
\affiliation{Institute of Theoretical Physics, Vienna
    University of Technology, Wiedner Hauptstra\ss e 8-10/136, A-1040
    Vienna, Austria}
\email{svozil@tuwien.ac.at} \homepage[]{http://tph.tuwien.ac.at/~svozil}


\date{\today}

\begin{abstract}
If the unitary quantum mechanical state evolution is universally valid, quantized systems evolve uniformly, deterministically, reversible; that is, even one-to-one. Hence, what might be considered a measurement is a purely subjective, conventional, and a convenient approximation of the situation that, although in principal totally reversible, ``fapp'' (e.g. for all practical purposes) measurements cannot be undone. If this is granted, then Schr\"odinger's ``quantum jellification'' arises because of the inevitability of the physical co-existence of classically mutually exclusive states through quantum coherence. We suggest to take our rather unique personal experience of nature as evidence that, at least at the level of our apperception, quantum jellification is rather weak and can be ignored fapp, although in principle perception should be ambivalent on a fundamental level of cognition.
\end{abstract}

\pacs{03.65.Ta, 03.65.Ud}
\keywords{quantum  measurement theory}
%\preprint{CDMTCS preprint nr. 372/2009}
\maketitle


\bibliography{svozil}

\end{document}
