%%tth:\begin{html}<LINK REL=STYLESHEET HREF="/~svozil/ssh.css">\end{html}
\documentstyle[12pt]{article}
%\documentstyle[amsfonts]{article}
%\RequirePackage{times}
%\RequirePackage{courier}
%\RequirePackage{mathptm}
%\renewcommand{\baselinestretch}{1.3}
\begin{document}

%\def\frak{\cal }
%\def\Bbb{\bf }
\sloppy



\title{On possible extensions of quantum mechanics}
\author{Karl Svozil\\
 {\small Institut f\"ur Theoretische Physik, University of Technology Vienna }     \\
  {\small Wiedner Hauptstra\ss e 8-10/136,}
  {\small A-1040 Vienna, Austria   }            \\
  {\small e-mail: svozil@tuwien.ac.at}}
\date{ }
\maketitle

\begin{flushright}
{\scriptsize http://tph.tuwien.ac.at/$\widetilde{\;\;}\,$svozil/publ/2000-exte.$\{$htm,ps,tex$\}$}
\end{flushright}

\begin{abstract}
We propose to generalize the probability axiom of quantum mechanics
to self-adjoint positive operators of  trace one.
Furthermore, we discuss the Cartesian and polar decomposition of arbitrary normal operators
and the possibility to operationalize the corresponding observables.
Finally we review and emphasize the use of observables which maximally represent the context.
\end{abstract}

Since Planck's introduction of the quantum exactly 100 years ago \cite{planck:1901-v},
quantum mechanics has developed into a phantastically successful theory
which appears to be stronger than ever.
Despite its obvious relevance and gratifying predictive power, the question of how to proceed
to theories beyond the quantum is not totally unjustified and is also asked
by eminent and prominent researchers in the area \cite{greenberger-talk-99,shimony-talk-2000}.
In what follows, a humble approach to this problem is pursued.
Von Neumann's Hilbert space formalism \cite{v-neumann-49} to quantum mechanics
is extended
by considering more general forms of operators as proper realizations of physical observables.
From the point of view of vector space theory, these extensions reflect well-known properties of the
algebraic structures arising in quantum mechanics \cite{halmos-vs,svozil-ql}.
It may nevertheless be worthwhile to review them for a proper understanding of
the underlying physics.

The quantum  probability $P(\psi , A)$ of a proposition $A$ given a state $\psi$
is usually introduced as
the trace of the product of the state operators $\rho_\psi $
and the projection operator $E_A$; i.e.,
$P(\psi ,A)={\rm Tr}(\rho_\psi  E_A)$.
(This ``axiom'' of quantum probability has been derived
for Hilbert spaces of dimension larger than two
from reasonable basic assumptions by Gleason
\cite{Gleason}.)
The state operator $\rho_\psi$ must be
(i)  self-adjoint; i.e., $\rho_\psi =\rho_\psi ^\dagger$,
(ii) positive; i.e., $(\rho_\psi  x,x) =\langle x\mid \rho_\psi  \mid x\rangle \ge 0$ for all $x$, and
(iii) of trace one; i.e., ${\rm Tr} (\rho_\psi )=1$.
Since one criterion for a pure state is its idempotence; i.e., $\rho_\varphi \rho_\varphi =\rho_\varphi$,
one way to interpret $E_A$ is a measurement apparatus in a pure state $\rho_\varphi= E_A$.
But while pure states can be interpreted as a system being in a given property,
not every state is pure and thus corresponds to a projection.

We propose here to generalize quantum probabilities to properties corresponding
also to nonpure states, such that the general form of quantum probabilities can be
written as
\begin{equation}
P(\psi ,\varphi)={\rm Tr}(\rho_\psi  \rho_\varphi),
\end{equation}
where again we require that $\rho_\varphi$ is self-adjoint, positive and of trace one.
One immediate advantage is the equal treatment of the object and measurement apparatus.
They appear interchangeably, stressing the conventionality of the measurement process
\cite{svozil-2000interface}.

One property of the extended probability measure is its positive definiteness and boundedness;
i.e., $0\le P(\psi ,\varphi) \le 1$. The former
bound follows from positivity.
The latter bound by $1$ can be easily
proved for finite dimensions by making a unitary basis transformation such that $\rho_\varphi$ (or $\rho_\psi$) is
diagonal.
%Then,
%because of positivity and boundedness of $\rho_\varphi $, which acquires a diagonal form
%\begin{eqnarray}
%P(\psi ,\varphi)&=&{\rm Tr}(\rho_\psi  \rho_\varphi)\nonumber \\
%&=& \sum_i [\rho_\psi  \rho_\varphi]_{ii}\nonumber \\
%&=& \sum_{i,j} [\rho_\psi]_{ij}  [\rho_\varphi]_{ji}\nonumber \\
%&=& \sum_{i} [\rho_\psi]_{ii}  [\rho_\varphi]_{ii}\nonumber \\
%&\le & \sum_{i} [\rho_\psi]_{ii} \le 1. \nonumber
%\end{eqnarray}


A very simple example of an extended probability is the case of the total ignorance of the
state of the measurement apparatus and of the measured system and $n$ nondegenerate outcomes for
the apparatus and the state, then $\rho_\psi=\rho_\varphi={\bf 1}/n=(1/ n)\,{\rm diag}(1,1,\ldots, 1)$,
then the probability to find any combination thereof is
$P(\psi ,\varphi)=1/n^2$.
The extended probability reduces to the standard form
if one assumes total knowledge of the state
of the measurement apparatus, since then $\rho_\varphi$ is pure and thus a projection.


%%%%%%%%%%%%%%%%%%%%%%%%%%%%%%%%%%%%%%%%%%%%%%%%%%%%%%%%%%%%%%%%%%%%%%%%%%%%%%%%%%%%

Let us from now on consider finite dimensional Hilbert spaces.

Another well known fact is the Cartesian and polar decomposition of an arbitrary
operator $A$ into operators $B,C$ and $D,E$ such that
\begin{eqnarray}
A&=&B+iC\nonumber \\
&=&DE  \nonumber \\
B&=&{A+A^\dagger \over 2} \nonumber \\
C&=&{A-A^\dagger \over 2i},\nonumber  \\
E&=&\sqrt{A^\dagger A},\nonumber  \\
D&=&{AE^{-1}},\nonumber
\end{eqnarray}
where $B,C$ are self-adjoint, $E$ is positive and $D$ is unitary (i.e., an isometry).
The last two equations are for invertible operators $A$.
These are just the matrix equivalents of the decompositions of complex numbers.

If $A$ is a {\em normal} operator;
i.e.,  $AA^\dagger=A^\dagger A$, then
$B$ and $C$ commute (i.e., $[B,C]=0$)
and are thus co-measurable.
(All unitary and self-adjoint operators are normal.)
In this case, also the operators of the polar decomposition $D$ and $E$
are unique and commute;
i.e.,    $[D,E]=0$, and are thus co-measurable.
We have thus reduced the issue of operationalizability of normal operators
to the self-adjoint case; an issue which has been solved positively
\cite{rzbb}.

Hence,  normal operators are  operationalizable
either by a simultaneous measurement of the summands in the Cartesian decomposition or
or of the factors in a polar decomposition
(cf. also \cite{neumaier-pr,appleby-pr}).
Indeed, all operators are ``measurable''
if one assumes EPR's elements of counterfactual
physical reality   \cite[p. 108f]{svozil-ql}.
In this case, one makes use of the Cartesian decomposition, where $B$ and $C$
not necessarily
can be diagonalized simultaneously and thus need not commute.
Nevertheless, one may devise a singlet state of two particles with respect to the
observables $B$ and $C$, and measure $B$ on one particle and $C$ on the other one.

As an example for the case of a normal operator which is neither self-adjoint nor
unitary, consider

\begin{eqnarray}
{\rm diag}(2,i)
&=&
{\bf 1}+\sigma_3 +{i\over 2}({\bf 1} - \sigma_3) \nonumber \\
&=&
\left[
{1+i\over 2} {\bf 1}
+
{1-i\over 2} \sigma_3
\right]
\left[
{3\over 2} {\bf 1}
+
{1\over 2} \sigma_3
\right]
\nonumber
\end{eqnarray}

where $\sigma_3={\rm diag}(1,-1)$ and
both summands and factors commute and thus are co-measurable.

%%%%%%%%%%%%%%%%%%%%%%%%%%%%%%%%%%%%%%%%%%%%%%%%%%%%%%%%%%%%%

Co-measurability
is an important issue in the theory of partial algebras \cite{kochen3,kochen1},
where, in accordance with quantum mechanics,
operations are only allowed between mutually commuting operators corresponding to
co-measurable observables.
In particular, let us define the {\em context}
as the set of all co-measurable properties of a physical system.
By a well-known theorem, any context has associated with it a
single (though not unique) observable represented by a self-adjoint operator $C$
such that all other observables represented by self-adjoint $A_i$
within a given context
are merely functions (in finite dimensions polynomials) $A_i=f_i(C)$
thereof. We shall call $C$ the {\em context operator}.
Context operators are maximal in the sense that they exhaust their context
but they are not unique, since any one-to-one transformation of $C$
such as an isometry yields a context operator as well.

Different operators $A_i$ may belong to different contexts.
Actually, the proof of Kochen and Specker \cite{kochen1}
(of the nonexistence of consistent global truth values by associating such valuations
locally) is based on a finite chain of contexts linked together at one
operator per junction which belongs to the two contexts forming that junction.
This fact suggests that---rather than considering single operators
which may belong to different contexts---it is more appropriate
to  consider context operators instead. By definition, they carry the
entire context and thus cannot belong to different ones.
A graphical representation of context operators has been given
by Tkadlec \cite{tkadlec-96}, who suggested to consider dual Greechie diagrams
which represent context operators as vertices and links between different contexts
by edges.
A typical application would be the measurement of all the $N$ contexts
necessary for a Kochen-Specker contradiction
in an entangle $N$ particle singlet state.
In such a case, there should exist at least one observable
belonging to two different contexts whose outcomes are different
(cf also \cite{hey-red} for a similar reasoning).

In summary, there exist extensions of quantum mechanics
guided by Hilbert space theory
which may be considered
as generalizations of the standard formalism.
All these extensions are operationalizable and may thus contribute
to a better understanding of the quantum phenomena.


\newpage


%\bibliography{svozil}
%\bibliographystyle{unsrt}
%\bibliographystyle{plain}

\begin{thebibliography}{10}

\bibitem{planck:1901-v}
Max Planck.
\newblock Ueber eine {V}erbesserung der {W}ien'schen {S}pectralgleichung.
\newblock Lecture at the Deutsche Physikalische Gesellschaft, December 14th,
  1900, published in \cite[p. 698]{planck:1901vv}.

\bibitem{greenberger-talk-99}
Daniel~M. Greenberger.
\newblock Solar eclipse Workshop, Vienna, August 1999.

\bibitem{shimony-talk-2000}
Abner Shimony.
\newblock Recollections and reflections on bell�s theorem.
\newblock Talk given at the Quantum [Un]speakables Conference in commemoration
  of John S. Bell, Vienna, November 10, 2000.

\bibitem{v-neumann-49}
John von Neumann.
\newblock {\em Mathematische Grundlagen der Quantenmechanik}.
\newblock Springer, Berlin, 1932.
\newblock English translation: {\sl Mathematical Foundations of Quantum
  Mechanics}, Princeton University Press, Princeton, 1955.

\bibitem{halmos-vs}
Paul~R.. Halmos.
\newblock {\em Finite-Dimensional Vector spaces}.
\newblock Springer, New York, Heidelberg, Berlin, 1974.

\bibitem{svozil-ql}
K.~Svozil.
\newblock {\em Quantum Logic}.
\newblock Springer, Singapore, 1998.

\bibitem{Gleason}
Andrew~M. Gleason.
\newblock Measures on the closed subspaces of a {H}ilbert space.
\newblock {\em Journal of Mathematics and Mechanics}, 6:885--893, 1957.

\bibitem{svozil-2000interface}
Karl Svozil.
\newblock Quantum interfaces.
\newblock e-print {\tt arXiv:quant-ph/0001064} available
  {http://arxiv.org/abs/quant-ph/0001064}, 2000.

\bibitem{rzbb}
M.~Reck, Anton Zeilinger, H.~J. Bernstein, and P.~Bertani.
\newblock Experimental realization of any discrete unitary operator.
\newblock {\em Physical Review Letters}, 73:58--61, 1994.
\newblock See also \cite{murnaghan}.

\bibitem{neumaier-pr}
Arnold Neumaier.
\newblock Private communication, April 2000.

\bibitem{appleby-pr}
David~Marcus Appleby.
\newblock Seminar Talk, ESI, Vienna, September 2000.

\bibitem{kochen3}
Simon Kochen and Ernst~P. Specker.
\newblock The calculus of partial propositional functions.
\newblock In {\em Proceedings of the 1964 International Congress for Logic,
  Methodology and Philosophy of Science, Jerusalem}, pages 45--57, Amsterdam,
  1965. North Holland.
\newblock Reprinted in \cite[pp. 222--234]{specker-ges}.

\bibitem{kochen1}
Simon Kochen and Ernst~P. Specker.
\newblock The problem of hidden variables in quantum mechanics.
\newblock {\em Journal of Mathematics and Mechanics}, 17(1):59--87, 1967.
\newblock Reprinted in \cite[pp. 235--263]{specker-ges}.

\bibitem{tkadlec-96}
Josef Tkadlec.
\newblock Greechie diagrams of small quantum logics with small state spaces.
\newblock Jan 1998.

\bibitem{hey-red}
Peter Heywood and Michael L.~G. Redhead.
\newblock Nonlocality and the kochen-specker paradox.
\newblock {\em Foundations of Physics}, 13(5):481--499, 1983.

\bibitem{planck:1901vv}
Max Planck.
\newblock {\em {P}hysikalische {A}bhandlungen und {V}ortr{\"{a}}ge. {B}and
  1-3}.
\newblock Viehweg und Sohn, Braunschweig, 1958.

\bibitem{murnaghan}
F.~D. Murnaghan.
\newblock {\em The Unitary and Rotation Groups}.
\newblock Spartan Books, Washington, 1962.

\bibitem{specker-ges}
Ernst Specker.
\newblock {\em Selecta}.
\newblock Birkh{\"{a}}user Verlag, Basel, 1990.

\end{thebibliography}


\end{document}
