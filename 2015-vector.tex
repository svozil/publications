\newif\ifws
%\wstrue
\wsfalse
\ifws
%  LaTeX support: latex@mdpi.com
%  In case you need support, please attach all files that are necessary for compiling as well as the log file, and specify the details of your LaTeX setup (which operating system and LaTeX version / tools you are using).

% You need to save the "mdpi.cls" and "mdpi.bst" files into the same folder as this template file.

%=================================================================

\documentclass[journal,article,accept,oneauthor,pdftex,10pt,a4paper]{mdpi}
%--------------------
% Class Options:
%--------------------
% journal
%----------
% Choose between the following MDPI journals:
% actuators, admsci, aerospace, agriculture, agronomy, algorithms, animals, antibiotics, antibodies, antioxidants, applsci, arts, atmosphere, atoms, axioms, batteries, behavsci, beverages, bioengineering, biology, biomedicines, biomimetics, biomolecules, biosensors, brainsci, buildings, c, cancers, catalysts, cells, challenges, chemosensors, children, chromatography, climate, coatings, computation, computers, condensedmatter, cosmetics, cryptography, crystals, data, dentistry, diagnostics, diseases, diversity, econometrics, economies, education, electronics, energies, entropy, environments, epigenomes, fermentation, fibers, fishes, fluids, foods, forests, futureinternet, galaxies, games, gels, genealogy, genes, geosciences, geriatrics, healthcare, horticulturae, humanities, hydrology, informatics, information, infrastructures, inorganics, insects, instruments, ijerph, ijfs, ijms, ijgi, inventions, jcdd, jcm, jdb, jfb, jfmk, jimaging, jof, jintelligence, jlpea, jmse, jpm, jrfm, jsan, land, languages, laws, life, literature, lubricants, machines, magnetochemistry, marinedrugs, materials, mathematics, medsci, medicines, membranes, metabolites, metals, microarrays, micromachines, microorganisms, minerals, molbank, molecules, mps, nanomaterials, ncrna, neonatalscreening, nutrients, particles, pathogens, pharmaceuticals, pharmaceutics, pharmacy, philosophies, photonics, plants, polymers, processes, proteomes, publications, recycling, religions, remotesensing, resources, risks, robotics, safety, sensors, sexes, sinusitis, socsci, societies, soils, sports, standards, sustainability, symmetry, systems, technologies, toxics, toxins, universe, vaccines, vetsci, viruses, water
%---------
% article
%---------
% The default type of manuscript is article, but can be replaced by:
% addendum, article, bookreview, briefreport, casereport, changes, comment, commentary, communication, conceptpaper, correction, conferencereport, meetingreport, creative, datadescriptor, discussion, editorial, essay, erratum, hypothesis, interestingimage, letter, newbookreceived, opinion, obituary, projectreport, reply, retraction, review, shortnote, supfile, technicalnote
% supfile = supplementary materials
%----------
% submit
%----------
% The class option "submit" will be changed to "accept" by the Editorial Office when the paper is accepted. This will only make changes to the frontpage (e.g. the logo of the journal will get visible), the headings, and the copyright information. Journal info and pagination for accepted papers will also be assigned by the Editorial Office.
%------------------
% moreauthors
%------------------
% If there is only one author the class option oneauthor should be used. Otherwise use the class option moreauthors.
%---------
% pdftex
%---------
% The option "pdftex" is for use with pdfLaTeX. If eps figure are used, use the option "dvipdfmx", with LaTeX and dvi2pdf.

%=================================================================
\firstpage{1} %%MDPI internal note: new layout%%
\makeatletter %%MDPI internal note: new layout%%
\setcounter{page}{\@firstpage} %%MDPI internal note: new layout%%
\makeatother %%MDPI internal note: new layout%%
\articlenumber{x}
\doinum{10.3390/------}
\pubvolume{18}
\pubyear{2016}
\copyrightyear{2016}
\externaleditor{Academic Editors: Gregg Jaeger, Andrei Khrennikov and Kevin H. Knuth}
\history{Received: 26 January 2016; Accepted: 19 April 2016; Published: date}
%------------------------------------------------------------------
% The following line should be uncommented if the LaTeX file is uploaded to arXiv.org
%\pdfoutput=1

%=================================================================

% Add packages and commands here. The following packages are loaded in our class file: fontenc, calc, indentfirst, fancyhdr, graphicx, lastpage, ifthen, lineno, float, amsmath, setspace, enumitem, mathpazo, eulervm, booktabs, titlesec, amsthm, hyphenat, natbib, hyperref, footmisc, geometry, caption, url,

\usepackage{amssymb}
\usepackage[usenames,dvipsnames]{xcolor}
\usepackage{soul}
\usepackage{microtype}

%=================================================================
%% Please use the following mathematics environments:
 \theoremstyle{mdpi}
 \newcounter{thm}
 \setcounter{thm}{0}
 \newcounter{ex}
 \setcounter{ex}{0}
 \newcounter{re}
 \setcounter{re}{0}

 \newtheorem{Theorem}[thm]{Theorem}
 \newtheorem{Lemma}[thm]{Lemma}
 \newtheorem{Corollary}[thm]{Corollary}
 \newtheorem{Proposition}[thm]{Proposition}

 \theoremstyle{mdpidefinition}
 \newtheorem{Characterization}[thm]{Characterization}
 \newtheorem{Property}[thm]{Property}
 \newtheorem{Problem}[thm]{Problem}
 \newtheorem{Example}[ex]{Example}
 \newtheorem{ExamplesandDefinitions}[ex]{Examples and Definitions}
 \newtheorem{Remark}[re]{Remark}
 \newtheorem{Definition}[thm]{Definition}
%% For proofs, please use the proof environment (the amsthm package is loaded by the MDPI class).

%\usepackage[breaklinks=true,colorlinks=true,anchorcolor=blue,citecolor=blue,filecolor=blue,menucolor=blue,urlcolor=blue,linkcolor=blue]{hyperref}
%\usepackage{url}

%\usepackage{mathptmx}% http://ctan.org/pkg/mathptmx
%\usepackage{graphicx}% Include figure files
%
%\usepackage[usenames,dvipsnames]{xcolor}

%=================================================================
% Full title of the paper (Capitalized)
\Title{Orthogonal Vector Computations}

% Authors (Add full first names)
\Author{Karl Svozil $^{1,2}$}
%\AuthorNames{} %%MDPI internal note: new layout%%

% Affiliations / Addresses (Add [1] after \address if there is only one affiliation.)
\address{%
$^{1}$\quad Institute for Theoretical Physics, Vienna University of Technology,
Wiedner Hauptstrasse 8-10/136, \linebreak Vienna 1040, Austria;  svozil@tuwien.ac.at; Tel.: +43-1-58801  (ext. 13614)
$^{2}$\quad Department of Computer Science, University of Auckland, Private Bag 92019,  Auckland 1142, New Zealand

}

% Contact information of the corresponding author
%\corres{Correspondence: svozil@tuwien.ac.at; Tel.: +43 1 58801x13614}

% Current address and/or shared authorship
%\firstnote{Current address: Affiliation 3} %%MDPI internal note: new layout%% with dagger
%\secondnote{These authors contributed equally to this work.} %%MDPI internal note: new layout%% with ddagger

% Simple summary
%\simplesumm{} %%MDPI internal note: new layout%%

% Abstract (Do not use inserted blank lines, i.e. \\)
\abstract{Quantum computation is the suitable orthogonal encoding of possibly holistic functional properties into state vectors, followed by a projective measurement.}

% Keywords
\keyword{quantum theory; probability theory; quantum logic; quantum algorithms; parity}

% The fields PACS, MSC, and JEL may be left empty or commented out if not applicable
\PACS{03.65.Ca, 02.50.-r, 02.10.-v, 03.65.Aa, 03.65.Aa, 03.67.Ac}
%\MSC{}
%\JEL{}

% If this is an expanded version of a conference paper, please cite it here: enter the full citation of your conference paper, and add $^\S$ in the end of the title of this article.
%\conference{}

%%%%%%%%%%%%%%%%%%%%%%%%%%%%%%%%%%%%%%%%%%
% Only for the journal Data:

%\dataset{DOI number or link to the deposited data set in cases where the data set is published or set to be published separately. If the data set is submitted and will be published as a supplement to this paper in the journal Data, this field will be filled by the editors of the journal. In this case, please make sure to submit the data set as a supplement when entering your manuscript into our manuscript editorial system.}

%\datasetlicense{license under which the data set is made available (CC0, CC-BY, CC-BY-SA, CC-BY-NC, etc.)}

%%%%%%%%%%%%%%%%%%%%%%%%%%%%%%%%%%%%%%%%%%

\begin{document}

\else


\documentclass[%
 10pt,
 reprint,
  twocolumn,
 %superscriptaddress,
 %groupedaddress,
 %unsortedaddress,
 %runinaddress,
 %frontmatterverbose,
 % preprint,
 showpacs,
 showkeys,
 preprintnumbers,
 %nofootinbib,
 %nobibnotes,
 %bibnotes,
 amsmath,amssymb,
 aps,
 % prl,
  pra,
 % prb,
 % rmp,
 %prstab,
 %prstper,
  longbibliography,
 %floatfix,
 %lengthcheck,%
 ]{revtex4-1}

\usepackage[breaklinks=true,colorlinks=true,anchorcolor=blue,citecolor=blue,filecolor=blue,menucolor=blue,urlcolor=blue,linkcolor=blue]{hyperref}
\usepackage{url}

%\usepackage{mathptmx}% http://ctan.org/pkg/mathptmx
\usepackage{graphicx}% Include figure files

\usepackage[usenames,dvipsnames]{xcolor}

\begin{document}

\title{Orthogonal vector computations}

%\cdmtcsauthor{Karl Svozil}
%\cdmtcsaffiliation{Vienna University of Technology}
%\cdmtcstrnumber{407}
%\cdmtcsdate{September 2011}
%\coverpage

\author{Karl Svozil}
\affiliation{Institute for Theoretical Physics, Vienna
    University of Technology, Wiedner Hauptstra\ss e 8-10/136, A-1040
    Vienna, Austria}
\email{svozil@tuwien.ac.at} \homepage[]{http://tph.tuwien.ac.at/~svozil}


\pacs{03.65.Ca, 02.50.-r, 02.10.-v, 03.65.Aa, 03.65.Aa, 03.67.Ac}
\keywords{quantum theory, probability theory, quantum logic, quantum algorithms, parity}
%\preprint{CDMTCS preprint nr. 407/2011}


\begin{abstract}
Quantum computation is the suitable orthogonal encoding of possibly holistic functional properties into state vectors, followed by a projective measurement.
\end{abstract}

\maketitle
\fi


\section{Identifying Quantum Physical Means for Computation}

The hypothesis pursued in this paper is that
the power of quantum computation solely resides in a proper ``translation'' of ``holistic'' properties of  functions---manifesting themselves in the {\em relational}
%According to our layout rule, please remove "italic" if they are not necessary. please check similar use throughout.
 values for different elements of their domain, or of their entire image---into orthogonal subspaces
and their associated perpendicular projections.
This is usually facilitated by {\em {quantum parallelism}}---the possibility to co-represent and co-encode classically distinct and mutually exclusive clauses
into simultaneous coherent superpositions thereof.
However, in order to take advantage of parallelism, it is necessary to be able to analyse the resulting quantum state by suitably chosen
orthogonal projections.
In what follows we shall therefore attempt to enumerate conditions under which a given algorithmic task can be quantum mechanically encoded into
orthogonal subspaces, thereby identifying criteria for potential quantum speedups.

This paper is organized as follows: after a brief and somewhat iconoclastic exposition of the standard quantum formalism
(which may be amusing to some but considered to be superfluous by others)
strategies to encode properties of computable functions into orthogonal subspaces are discussed.
This involves the possibility to orthogonalise non-orthogonal vectors of some initial Hilbert space by interpreting them as
orthogonal projections of mutually orthogonal vectors in a Hilbert space of greater dimension.


%Claims of an advantage of certain physical resources for
%(not necessarily universal) computational purposes
%need to be substantiated by concretely locating the particular operational, physical feature(s)
%which ``make a difference'' relative to previously employed physical means.
%Thereby it is important to keep in mind that merely formally postulated features
%introduced for convenience (and ``corroborated'' by a lack of empirical falsification)
%-- such as, for instance, the classical continuum, or ``merely'' infinite divisibility;
%and infinities of all kinds -- ``responsible''
%for speedups, or even hyper-Turing computability,
%have little physical and technological relevance.
%(They me be a very successful in terms of public relations and marketing, though.)
%
%In quantum mechanics, in particular,
%when it comes to contemplating and promising stronger-than-classical capacities
%it is not uncommon to vaguely alleging mind boggling, almost ``magical'' features, such as
%quantum parallelism, contextuality, entanglement, superposition, complementarity,
%{\it ex nihilo} occurrences of measurement outcomes,
%nonlocality,
%or interference,
%without relating to concrete situations motivating and exploiting these resources from operationally supported  principles.
%
%%
%%
%Consider two examples: although it is often argued that
%``quantum cryptography is unconditionally secure,''
%it is seldom stated that, as has been pointed out initially~\cite{benn-84}, this ``unconditional security''
%requires an uncompromised classical channel
%(unless active eavesdropping such as middlemen attacks are {\it ad hoc} excluded) --
%and yet, quantum cryptography have often been advocated as remedy against compromised classical channels.
%
%A second example is the computation of the parity of a binary function of $n$ bits.
%This is a classically ``hard'' problem that could (erroneously) be expected to serve as
%an ideal candidate for the exploitation of quantum parallelism along the following scheme:
%generate a
%(equally weighted, such as through some generalized Hadamard transformation of a single classical bit state)
%superposition of all $2^n$ conceivable (classically mutually exclusive) input states of $n$ classical bits,
%push it through a suitable quantum oracle encoding some binary function of $n$ bits,
%and, from its output, read off a single yes-no answer determining its parity.
%Alas, as has been shown~\cite{Farhi-98}, relative to the assumptions made,
%there is not much to be gained quantum mechanically;
%indeed the optimal (constant polynomial) speedup is a factor of $2$,
%as exploited in Deutsch's problem -- the determination
%of the parity of a binary function of one bit.
%


\section{Locating Quantum Resources}

Starting with Plack's~\cite{hermann} (p. 31)  ``act of despair''
it may not be totally unreasonable to state that the newly discovered quantum capacities,
unheard of in classical continuum mechanics,
caused shock \linebreak and awe
among some of the most prominent creators of quantum mechanics.
Most notably Schr\"odinger struggled with
quantum coherence, today known as quantum parallelism, throughout his entire life,
{bringing forward seemingly absurd consequences of the formalism, such as the cat paradox,
or quantum} jellification~\cite{schroedinger-interpretation}.
In the same series of papers~\cite{schrodinger} he also mentioned the capacity that
multiple quanta may be entangled;
a resource related to quantum non-locality, which was put forward by Einstein on
various occasions~\cite{einstei-letter-to-schr,Howard1985171}
to question the consistency of quantum and relativity theory.

On the formal side, Hilbert space quantum mechanics~\cite{dirac} and, in particular,
quantum logic have been conceived~\cite{birkhoff-36} as attempts
to systematically identify and study the valid, feasible, operational resources and capacities
of quantized physical systems in terms of algebraic structures arising in {quantum mechanics.
Thereby, logical entities are identified with quantum theoretical terms;
\linebreak a typical }example being the identification with a binary logical yes-no proposition or property
with an orthogonal projection operator;
or, equivalently, with the associated linear subspace, which, in the one-dimensional case,
is spanned by a single (unit) vector.
In what follows some of these identifications will explicitly be enumerated.

Subsequently, and possibly motivated by single-quantum experiments which have become experimentally feasible,
quantum information and communication research attempts to exploit mostly non-classical capacities of
quantized systems, such as parallelism, entanglement, or other quantum resources~\cite{be-va,benn:97,fortnov-03,Castagnoli}.
(A {\it{caveat:}} we refrain here from a detailed, more comprehensive list of references,
as~this might transcend this brief review.)


\subsection{State as Context}

In the Dirac-von Neumann formalism of quantum mechanics a {\em{pure state}}  is completely
characterized by a unit vector
$\vert \psi \rangle$ (or, equivalently, by the associated one-dimensional subspace
%$\mathfrak{M}_\psi = \textrm{span}\left(\vert \psi \rangle \right)$
spanned by $\vert \psi \rangle$;
or by the orthogonal projection operator $\textsf{\textbf{E}}_\psi = \vert \psi \rangle \langle \psi \vert$)
% onto $\mathfrak{M}_\psi$ --
in a Hilbert space, which~essentially
(disregarding completeness) is
a vector space equipped with a scalar product.
In what follows we shall disregard mixed states completely---according to the {\em {Church of the Larger Hilbert Space}} any mixed state is purely epistemic (but not ontic),
and originates from some unknown pure state which can be reconstructed, subject to some (usually non-unique) purification
by ``reverting'' (by disclosing ``additional information'') the introduction of ignorance
(enforced by a non-unitary many-to-one state evolution)
through partial traces.
Every state will thus assumed to be pure.


Alternatively it may be preferable~\cite{svozil-2013-omelette} to define a state { via} a context, or, equivalently,  an~orthonormal basis,
or a maximal observable~\cite{halmos-vs} (\S~84, Theorem~1) whose spectral sum is~non-degenerate.
Any such context represents a Boolean subalgebra (formed by mutually orthogonal and thus commuting projections) of a quantum logic;
a kind of ``greatest classical mini-universe'' which is co-prepareable.

\subsection{Observable as Context}

Every maximal {\em {observable}} or context corresponds to the non-degenerate spectral sum of one-dimensional mutually orthogonal (and thus commuting)
projections whose sum is a resolution of the identity.
Within operational capacities, those bases can be chosen freely by the experimenter.\linebreak
They correspond to quasi-classical mini-universes of mutually commuting, compatible observables.

Once a particular maximal observable is chosen, the respective orthonormal basis\linebreak
$\mathfrak{B}_n =\left\{ \vert {\bf e}_1 \rangle, \vert {\bf e}_2 \rangle, \ldots , \vert {\bf e}_n \rangle
\right\}$ facilitates a {\em {particular view}} on the state $\vert \psi \rangle$.
If one basis element coincides with the state $\vert \psi \rangle$,
then the measurement of $\mathfrak{B}_n$ is deterministic:
a single proposition associated with the orthogonal projection operator identical to
$\vert \psi \rangle \langle \psi \vert$
is true; and all the other propositions represented by the remaining basis elements are false.
In this case we may say that this observable
reflects a physical {\em {property}} of that state~\cite{2015-AnalyticKS}
(thereby affirming the dual use of orthogonal projections both for pure states as well as for propositions).

Otherwise, if there is a mismatch between the state preparation and the query, {\em {context translation}} introduces stochasticity
through the many degrees of freedom of quasi-classical measurement devices~\cite{svozil-2003-garda}.
The epistemic interpretation of maximal observables as views on the state $\vert \psi \rangle$
alleviates
Schr\"odinger's continued concerns regarding the ontological existence of coherent superpositions,
so~vividly put forward in the cat paradox or by quantum jellification
mentioned in his lesser known
Dublin seminars \cite{schroedinger-interpretation}.

\subsection{Probability}

Gleason's theorem~\cite{Gleason} states that (for dimensions larger than two, allowing intertwining orthogonal bases) the quantum probabilities
follow from the requirement that on all contexts the quantum probabilities should be identical with their respective classical probabilities
(in particular, $\sigma$-additivity of probabilities of mutually exclusive events). That~is, within classical mini-universes of a quantum logic,
classical probabilities rule.

Although most of his paper is devoted to proving necessity of the Born rule (including mixed states) from this assumption,
already in the second paragraph Gleason observed that,
as long as one is concerned merely about pure states
it is possible to construct a {\em {probability measure}} on the Hilbert space
{via} an elementary geometric construction, suggesting the squared norm of the orthogonal projections
of these pure states onto the vectors of some orthonormal basis
as quantum probability measure.
One could again say that the pure state is ``viewed from'' the maximal operator corresponding to the aforementioned basis.

In contrast to the full proof of Gleason's theorem,
the argument for the sufficiency of the squared norm  probability measures (satisfying classicality within contexts)
is elementary;
thereby the absolute squares of the probability amplitudes
$\langle  \psi \vert {\bf e}_i  \rangle$ result from the Pythagorean theorem:
since $\vert \psi \rangle$ is a unit vector and
the  $\vert {\bf e}_i  \rangle$ represent vectors of an orthonormal basis,
all absolute squares of $\langle  \psi \vert {\bf e}_i  \rangle$  sum up to~one;
that is,
$\sum_{i=1}^n \vert\langle  \psi \vert {\bf e}_i  \rangle \vert^2=1$.
A construction of such a Gleason-type measure
in two-dimensional real vector space is depicted in Figure~\ref{2015-m-fdlvs-vv}.
\begin{figure}
\begin{center}
% This is a LaTeX picture output by TeXCAD.
% File name: [1.pic].
% Version of TeXCAD: 4.3
% Reference / build: 30-Jun-2012 (rev. 105)
% For new versions, check: http://texcad.sf.net/
% Options on the following lines.
%\grade{\on}
%\emlines{\off}
%\epic{\off}
%\beziermacro{\on}
%\reduce{\on}
%\snapping{\off}
%\pvinsert{% Your \input, \def, etc. here}
%\quality{8.000}
%\graddiff{0.005}
%\snapasp{1}
%\zoom{4.0000}
\unitlength 0.35mm
\linethickness{0.4pt}
\ifx\plotpoint\undefined\newsavebox{\plotpoint}\fi % GNUPLOT compatibility
\begin{picture}(178.25,188)(25,30)
%\dashline{1}(167.25,110)(167.25,81.5)
\multiput(167.18,109.93)(0,-2){15}{\color{Red}\line(0,-1){0.5}}
%%%
%\end
%\dashline{1}(167,110)(71.75,110)
\multiput(166.93,109.93)(-2,0){48}{\color{Red}\line(-1,0){0.5}}
%\end
%\dashline{1}(167.25,110)(104,48.5)
\multiput(166.93,109.93)(-2,-2){32}{\color{ForestGreen}\line(-1,0){0.5}}
%\end
%\dashline{1}(167.5,110)(133.75,142.25)
\multiput(166.93,109.93)(-2,2){17}{\color{ForestGreen}\line(-1,0){0.5}}
%\end
%
%
\put(71.5,81.25){\color{Red}\vector(1,0){110}}
\put(71.5,81.25){\color{Red}\vector(0,1){100}}
%\vector(71.5,81.25)(167,109.75)
\put(167,109.75){\color{blue}\vector(3,1){.07}}\multiput(71.5,81.25)(.11301775148,.03372781065){845}{\color{blue}\line(1,0){.11301775148}}
%\end
%\vector(71.5,81.25)(142.5,151.25)
%%\put(142.5,151.25){\vector(1,1){.07}}\multiput(71.5,81.25)(.03421686747,.033734939759){2075}{\line(1,0){.03421686747}}
\put(71.5,81.25){\color{ForestGreen}\vector(1,1){70}}
%\end
%\vector(71.5,81.25)(.5,151.25)
%%\put(.5,151.25){\vector(-1,1){.07}}\multiput(71.5,81.25)(-.03421686747,.033734939759){2075}{\line(-1,0){.03421686747}}
\put(71.5,81.25){\color{ForestGreen}\vector(-1,1){50}}
%\end
%\vector(71.5,81.25)(141.5,10.25)
%%\put(141.5,10.25){\vector(1,-1){.07}}\multiput(71.5,81.25)(.033734939759,-.03421686747){2075}{\line(0,-1){.03421686747}}
\put(71.5,81.25){\color{ForestGreen}\vector(1,-1){50}}
%\end
\put(175,111.25){\makebox(0,0)[cc]{\color{blue}\scriptsize $\vert \psi  \rangle$}}
\put(195,81.5){\makebox(0,0)[cc]{\color{Red}\scriptsize $\vert {\bf e}_1 \rangle$}}
\put(147.75,157){\makebox(0,0)[cc]{\scriptsize $\vert {\bf f}_1 \rangle$}}
\put(71.5,188){\makebox(0,0)[cc]{\color{Red}\scriptsize $\vert {\bf e}_2 \rangle$}}
\put(20,140){\makebox(0,0)[cc]{\color{ForestGreen}\scriptsize $\vert {\bf f}_2 \rangle$}}
\put(125,25){\makebox(0,0)[cc]{\color{ForestGreen}\scriptsize $- \vert {\bf f}_2 \rangle$}}
\put(135,120){\makebox(0,0)[cc]
{\color{Red}\scriptsize $\vert \langle \psi  \vert {\bf e}_1 \rangle \vert$}}
\put(185,97){\makebox(0,0)[cc]
{\color{Red}\scriptsize $\vert \langle \psi  \vert {\bf e}_2 \rangle \vert$}}
\put(145,64.5){\makebox(0,0)[cc]
{\color{ForestGreen}\scriptsize $\vert \langle \psi  \vert {\bf f}_1 \rangle \vert$}}
\put(170,135){\makebox(0,0)[cc]
{\color{ForestGreen}\scriptsize $\vert \langle \psi  \vert {\bf f}_2 \rangle \vert$}}
\end{picture}
\end{center}
\caption{Different orthonormal bases
{\color{black}
$\{
\vert {\bf e}_1 \rangle ,
\vert {\bf e}_2 \rangle
\}$}
and
{\color{black}
$\{
\vert {\bf f}_1 \rangle ,
\vert {\bf f}_2 \rangle
\}$}
offer different ``views''
on the pure state {\color{black} $\vert \psi  \rangle$}.
As {\color{black} $\vert \psi  \rangle$} is a unit vector
it follows  from the Pythagorean theorem that
${\color{black}
\vert \langle \psi  \vert {\bf e}_1 \rangle \vert^2
+
\vert \langle \psi  \vert {\bf e}_2 \rangle \vert^2}=
{\color{black}
\vert \langle \psi  \vert {\bf f}_1 \rangle \vert^2
+
\vert \langle \psi  \vert {\bf f}_2 \rangle \vert^2}
=1
$, thereby
motivating the use of the abolute value (modulus) squared of the amplitude for
quantum probabilities on pure states.}
  \label{2015-m-fdlvs-vv}
\end{figure}

By construction, this probability measure satisfies
the requirement of the validity of classical probabilities within all classical mini-universes.
This is remarkable because of the absence of (sufficiently many) global two-valued measures~\cite{kochen1}
whose positive convex combinations form all classical probability measures~\cite{pitowsky-89a}.

\subsection{Entanglement}

Pointedly stated, quantum entanglement is a consequence of the inseparability of not-so-individual quanta.
In general, quantum states do not allow the decomposition into
product states of constituents.
For instance, a general three partite state of three bits
$\vert \psi \rangle =\sum_{i,j,k=0}^1 \alpha_{ijk} \vert ijk\rangle$
is entangled if and only if it cannot be written as a single product state of the three constituents;\linebreak
a non-entangled product state of three particles requires~\cite{mermin-07} (p.~18) that
all of the following equations are satisfied:
$\alpha_{000} \alpha_{011} = \alpha_{001} \alpha_{010}$,
$\alpha_{000} \alpha_{101} = \alpha_{001} \alpha_{100}$,
$\alpha_{000} \alpha_{110} = \alpha_{010} \alpha_{100}$,
$\alpha_{000} \alpha_{111} = \alpha_{011} \alpha_{100}$,
$\alpha_{001} \alpha_{110} = \alpha_{011} \alpha_{100}$,
$\alpha_{001} \alpha_{111} = \alpha_{011} \alpha_{101}$,
$\alpha_{010} \alpha_{101} = \alpha_{011} \alpha_{100}$,
$\alpha_{010} \alpha_{111} = \alpha_{011} \alpha_{110}$,
$\alpha_{100} \alpha_{111} = \alpha_{101} \alpha_{110}$.

As a consequence, unlike classical properties, entangled states
encode (not purely individual but) {\em {relational}} properties of multiple quanta~\cite{Zeilinger-97,zeil-99}.
It is thus futile to request, as Einstein maintained in a~letter communicated to Schr\"odinger~\cite{einstei-letter-to-schr,Einstein-48,Howard1985171}, that
the sub-state of an entangled state is not affected by any change of another sub-state if the two involved entangled sub-systems are
spatially separated.

\subsection{Evolution as Permutation}

The quantum evolution is postulated to be an isometry (that is, a distance preserving map);\linebreak
or, stated differently, a permutation---a one-to-one mapping---preserving the scalar product.
This~can be formalized by unitary transformations~\cite{Schwinger.60}:
suppose
$\mathfrak{B}_n= \{\vert {\bf e}_1\rangle , \vert  {\bf e}_2\rangle , \ldots , \vert {\bf e}_n\rangle \}$
and\linebreak
$\mathfrak{B}'_n= \{\vert {\bf f}_1\rangle , \vert  {\bf f}_2\rangle , \ldots , \vert {\bf f}_n\rangle \}$
are orthonormal bases, then $
\textsf{\textbf{U}}_{fe}=  \sum_{i=1}^n  \vert {\bf f}_i\rangle \langle {\bf e}_i \vert
$
yields a unitary transformation; conversely, any unitary transformation can be represented by such a change of orthonormal bases.
So, stated pointedly, the quantum evolution amounts to a (generalized) rotation either of the state vector (the active, Schr\"odinger representation),
or of the coordinate frame (passive, Heisenberg picture) relative to which the state is viewed.
%By a sort of quantum relativity principle -- there is no {\it a priori} preference one way or the other -- the difference between both perceptions is indiscernible and thus operationally equivalent.

Much of the conceptual progress in the foundations of quantum mechanics
owes to the exploitation of this one-to-one permutation scheme in various scenarios:
to name just three examples, take quantum teleportation~\cite{BBCJPW},
induced coherence~\cite{zou-wang-mandel:91a},
and the quantum erasure~\cite{PhysRevD.22.879,PhysRevA.25.2208,engrt-sg-I}.
Indeed,
as~has already been pointed out by Everett~\cite{everett} (p.~454) (see also Wigner~\cite{wigner:mb})
there is no such thing as an
``irreversible measurement'' if quantum theory is assumed universally and uniformly valid:
what~we fapp~\cite{bell-a} call ``measurement'' is means relative~\cite{Myrvold2011237}, epistemic,
and could, at least in principle, be undone if our operational capacities permit.

\subsection{Computational Resources}

Quantum theory offers or identifies a single state vector, or orthonormal systems of vectors, which~can be rotated or viewed from various
perspectives,
and on which projection measurements can be made,
as the quantum resource for computations.
Related alleged capacities, such as  ``quantum parallelism'' associated with coherent superposition induced by views on this vector which do not coincide with it,
or {\it {ex nihilo}}
randomness induced by the quasi-stochasticity of the resulting measurement results through context translations,
or entanglement originating from the pretension that the state of subsystems which are spatially apart must be separable,
are implications
which need to be carefully revised~\cite{jaynes-89,jaynes-90}
in order to re-evaluate the quantum capacities for computation.

\section{Renditions via Orthogonal Subspaces}

So far the stage has been set for concrete proposals to utilize quantized systems for ``optimized'' computation:
in particular, by ``looking at a pure state from a proper perspective;''
that is, \linebreak from an orthonormal basis of which it is not an element.
In such a basis, the pure state has a~ composition in terms of more than one basis vectors---in quantum mechanical terms, the state is in a {\em {coherent superposition}} of these basis vectors.
If the basis ist associated with classical bit states, such as  $\mathfrak{B}_2=\{\vert 0 \rangle ,\vert 1\rangle \}$,
one could claim
that a quantum state such as
$\alpha_0 \vert 0 \rangle + \alpha_1\vert 1\rangle$ with non-vanishing $\alpha_0$, $\alpha_1$
is ``in~both states $\vert 0 \rangle$ and $\vert 1\rangle$ simultaneously.''
This may nurture hopes that the simultaneous co-processing of an arbitrary amount of data or clauses
might become quantum feasible
by introducing a ``sufficient'' amount of bits in superposition.


One fundamental issue which has to be accounted for is the fact that
any recursive function is completely characterized by its input-output behaviour.
In order to perform an evaluation (of the recursive function), this behaviour must be coded into a physical operation {\it {simulacron}}~\cite{simula}.

However, there is a fundamental limit to quantum computation
which resides in one of the basic relations between vectors in Hilbert space: their (non-)orthogonality.
This translates into the criterion that a query is quantum mechanically feasible if and only if it can be encoded into orthogonal~subspaces.
As the unitary evolution preserves angles, and thus orthogonality,
within the same Hilbert space there is nothing one can do about ``orthogonally separating'' two non-orthogonal subspaces,
\linebreak because a Gram-Schmidt orthogonalisation is quantum infeasible
when it is most needed; that is, if the initial vectors are not orthogonal
(just as the cloning of an arbitrary quantum state is impossible for the same reason---unitarity of the quantum state evolution).

\subsection{Example}

Recall
the supposition (or rather speculation) mentioned earlier:
that the result of any kind of (binary) decision problem can be obtained by taking ``enough'' qubits to ``cover all cases''
through the coherent superposition of all classical cases by a single quantum query.
Indeed, for the sake of temporal speedups, we could be so bold as to attempt to quantum encode an algorithm as the ``parallel''
(weighted by relative phases and amplitudes) sum (not the product); that is,  the coherent superposition, of all conceivable classical clauses.
(Never mind computation space; that is, the amount of qubits, as~long as this resource is bounded.
Memory space is also required in order to render the computation one-one; that is, a permutation,
if the evaluated functions are not invertible.)

For the sake of an example of this strategy, consider Deutsch's problem for computing the parity of a binary function of a single bit.
there, two bits---the input bit and an auxiliary bit (to take care of injectivity in case the function is constant)---which are initially classical,
are spread out by Hadamard gates into a proper (the relative phases are relevant) coherent superposition of all classical states.

At the heart of this algorithm is a way to subdivide the four-dimensional state space of the two~bits
into two orthogonal subspaces associated with parity $\pm 1$, respectively,
thereby rendering parity in a single query. (See ~\cite{Farhi-98} for a different finding.)
The quantum {\it versus} classical trade-off is the ignorance of the precise function (out of four) after this single quantum query.

How can this exactly be achieved?
Suppose we identify with the classical bit states $0$ and $1$ the quantum states, represented by the two vectors of a Cartesian basis $\mathfrak{B}_2$,
$0 \equiv \vert 0 \rangle \equiv \begin{pmatrix} 1\\0\end{pmatrix}$
and
$1 \equiv \vert 1 \rangle \equiv \begin{pmatrix} 0\\1\end{pmatrix}$,
respectively.
Suppose further that the four binary functions of a single classical bit are  the two constant  functions
$f_0(x)=0$ and
$f_3(x)=1$,
as well as the identity $f_1(x) = x$ and the negation $f_2(x) = 1 \oplus x$, where $x \in \{0,1\}$ and $\oplus$ represents the addition modulo $2$.
By forming a unitary quantum oracle $\textsf{\textbf{U}}_{f_i} \vert {\bf x} {\bf y} \rangle = \vert {\bf x} \left[{\bf y}\oplus f_i({\bf x})\right] \rangle$, $i\in \{0,1,2,3\}$,
and by taking $\vert {\bf x}  \rangle = \vert {\bf y}  \rangle = \left(1/\sqrt{2}\right)\vert 0-1 \rangle $
one obtains   (by omitting normalization factors; the superscript $^T$ indicates transposition, and ``$\pm$'' stands for ``$\pm 1$,'' respectively)
\setlength{\medmuskip}{1mu}
\begin{equation}
\begin{split}
\textsf{\textbf{U}}_{f_0}\vert (0-1)(0-1) \rangle = \vert +00 - 01 - 10 + 11 \rangle \equiv  \begin{pmatrix} +--+\end{pmatrix}^T   \\
\textsf{\textbf{U}}_{f_1}\vert (0-1)(0-1) \rangle = \vert +00 - 01 + 10 - 11 \rangle \equiv  \begin{pmatrix} +-+-\end{pmatrix}^T  \\
\textsf{\textbf{U}}_{f_2}\vert (0-1)(0-1) \rangle = \vert -00 + 01 - 10 + 11 \rangle \equiv  \begin{pmatrix} -+-+\end{pmatrix}^T \\
\textsf{\textbf{U}}_{f_3}\vert (0-1)(0-1) \rangle = \vert -00 + 01 + 10 - 11 \rangle \equiv  \begin{pmatrix} -++-\end{pmatrix}^T
\end{split}
\label{2016-vector-e-dp}
\end{equation}
\setlength{\medmuskip}{3mu}

If one is being presented with an unknown function $f_i$
and is given the task to identify~\cite{e-f-moore,svozil-2001-eua} it,
the following two-(binary)-valued query is associated with functional parity:
in terms of subspaces in fourdimensional Hilbert space
with the basis
$\mathfrak{B}_2 \equiv \{\vert {\bf e}_1\rangle , \vert  {\bf e}_2\rangle \}
\equiv
\left\{
(1/2) \begin{pmatrix} 1,-1,-1,1\end{pmatrix}^T ,
(1/2) \begin{pmatrix} 1,-1,1,-1\end{pmatrix}^T
\right\}
$, and written as spectral sum
$
\textsf{\textbf{P}} =
 \vert {\bf e}_1 \rangle \langle {\bf e}_1 \vert
-
 \vert {\bf e}_2 \rangle \langle {\bf e}_2 \vert
$
corresponding to a partitioning~\cite{DonSvo01,svozil-2002-statepart-prl,2007-tkadlec-svozil-springer}
of functions
$
\{\{f_0,f_3\},\{f_1,f_2\}\}
$.
%Any such system could be ``completed'' by the introduction of additional basis vectors
%$\mathfrak{B}'_2 \equiv \{\vert {\bf e}_3\rangle , \vert  {\bf e}_4\rangle \}
%\equiv
%\{
%(1/\sqrt{2}) \begin{pmatrix} 1,-1,0,0\end{pmatrix}^T ,
%(1/\sqrt{2}) \begin{pmatrix} 0,0,1,-1\end{pmatrix}^T
%\}
%$,
%so that
%$\mathfrak{B}_4= \mathfrak{B}_2 \cup \mathfrak{B}'_2
%$
%represents a basis of $\mathbb{C}^4$.

It might not be too critical to call the coincidence
between the particular ``modulo two'' oracle and parity of a binary function of one bit
{\it ad hoc.}



\subsection{Generalized Gram-Schmidt Process}

In the previous example the input-output behaviour
$\{\{0,0\},\{0,1\},\{1,0\},\{1,1\}\}$
of a functional class
has been translated into the appropriate orthogonal subspaces;
the trade-off being the enlargement of Hilbert space by two extra dimensions.
The question arises if it is always possible to do so.

As an illustration consider two non-orthogonal and non-collinear planar vectors depicted in Figure~\ref{2016-vector-f2}.
Suppose we ``extend'' those vectors into the three-dimensional space such that the
direction and length of the two new non-planar vectors is chosen such that
(i) they are orthogonal,
and at the same time
(ii) their orthogonal projections onto the plain
are identical with the original planar vectors.

\begin{figure}
\begin{center}
% This is a LaTeX picture output by TeXCAD.
% File name: [1.pic].
% Version of TeXCAD: 4.3
% Reference / build: 30-Jun-2012 (rev. 105)
% For new versions, check: http://texcad.sf.net/
% Options on the following lines.
%\grade{\on}
%\emlines{\off}
%\epic{\off}
%\beziermacro{\on}
%\reduce{\on}
%\snapping{\off}
%\pvinsert{% Your \input, \def, etc. here}
%\quality{8.000}
%\graddiff{0.005}
%\snapasp{1}
%\zoom{4.0000}
\unitlength 0.5mm % = 2.845pt
\linethickness{0.4pt}
\ifx\plotpoint\undefined\newsavebox{\plotpoint}\fi % GNUPLOT compatibility
\begin{picture}(118,100)(0,25)
\thinlines
%\put(50,50){\color{Red}\vector(1,-2){25}}
\put(50,50){\color{Red}\vector(1,-1){25}}
\put(82,25){\makebox(0,0)[cc]{\color{Red}\scriptsize $\vert {\bf f}_2 \rangle$}}
\thicklines
%\vector(50,50)(75,62.25)
\put(75,62.25){\color{Red}\vector(2,1){.07}}\multiput(50,50)(.0686813187,.0336538462){364}{\color{Red}\line(1,0){.0686813187}}
\put(82,65){\makebox(0,0)[cc]{\color{Red}\scriptsize $\vert {\bf e}_2 \rangle$}}
%\end
\thinlines
\put(50,50){\color{ForestGreen}\vector(1,1){50}}
\put(107,100){\makebox(0,0)[cc]{\color{ForestGreen}\scriptsize $\vert {\bf f}_1 \rangle$}}
\thicklines
\put(50,50){\color{ForestGreen}\vector(1,0){50}}
\put(106,50){\makebox(0,0)[cc]{\color{ForestGreen}\scriptsize $\vert {\bf e}_1 \rangle$}}
\thinlines
%\dottedline(100,50)(100,100)
\multiput(99.93,49.93)(0,.980392){52}{{\color{ForestGreen}\rule{.4pt}{.4pt}}}
%\end
%\dottedline(75,62.226)(75,25.332)
\multiput(75,62.156)(0,-.997144){38}{{\color{Red}\rule{.4pt}{.4pt}}}
%\end
%\emline(51.25,35)(118,44.5)
\multiput(51.25,35)(.2367021277,.0336879433){282}{\line(1,0){.2367021277}}
%\end
%\emline(11.25,75)(78,84.5)
\multiput(11.25,75)(.2367021277,.0336879433){282}{\line(1,0){.2367021277}}
%\end
\put(118,44.5){\line(-1,1){40}}
\put(50.75,35){\line(-1,1){40}}
%
\qbezier(73.158,60.755)(81.094,57.707)(78.729,50.033)
%\qbezier(54.658,54.343)(58.495,50.191)(54.553,45.408)
\put(82.618,56.235){\makebox(0,0)[lc]{$\varphi$}}
\put(62,44.8){\makebox(0,0)[lc]{$\frac{\pi}{2}$}}
\thinlines
%\emline(56.041,55.893)(61.096,50.913)
\multiput(56.041,55.893)(.034149529,-.03364733){148}{\line(1,0){.034149529}}
%\end
%\emline(61.096,50.913)(55.149,44.893)
\multiput(61.096,50.913)(-.033593421,-.034013339){177}{\line(0,-1){.034013339}}
%\end
\end{picture}
\end{center}
\caption{Demonstration of the orthogonalisation by dimensional lifting for two planar vectors
{\color{black}$\vert {\bf e}_1 \rangle$}
and {\color{black} $\vert {\bf e}_2 \rangle$}
which are ``lifted'' into two orthogonal vectors
{\color{black} $\vert {\bf f}_1 \rangle$}
and
{\color{black} $\vert {\bf f}_2 \rangle$} in three dimensions which orthogonally project
onto {\color{black}$\vert {\bf e}_1 \rangle$}
and {\color{black} $\vert {\bf e}_2 \rangle$}.
}
  \label{2016-vector-f2}
\end{figure}


In the following we shall prove the formal (but not quantum physical)
possibility of {\em {orthogonalisation by dimensional lifting:}}
%\begin[Orthogonalisation lemma]{theorem}
An arbitrary number $k$ of nonzero vectors
$\{ \vert {\bf e}_1 \rangle ,\ldots ,\vert {\bf e}_k \rangle \}$
of $n$-dimensional Hilbert space can be (non-uniquely)
interpreted as the orthogonal projections (onto the original Hilbert space)
of a set of mutually orthogonal
(\emph{i.e.}, $\langle {\bf f}_i \vert {\bf f}_j \rangle =0$ for $i\neq j$)
vectors  $\{\vert {\bf f}_1 \rangle ,\ldots \vert {\bf f}_k \rangle  \}$
in an Hilbert space of dimension $m\ge n$.
%\end{theorem}


For the sake of an explicit example, suppose that, for two vectors (\emph{i.e.}, $k=2$) and arbitrary finite dimension $n$,
we start with
$\vert {\bf e}_1 \rangle \equiv \begin{pmatrix}x_{11},\ldots ,x_{1n}\end{pmatrix}^T$,
and
$\vert {\bf e}_2 \rangle \equiv \begin{pmatrix}x_{21},\ldots ,x_{2n}\end{pmatrix}^T$.
It is not too difficult to construct two vectors
\begin{equation}
\begin{split}
\vert {\bf f}_1 \rangle \equiv \begin{pmatrix}x_{11},\ldots ,x_{1n},1\end{pmatrix}^T   \\
\vert {\bf f}_2 \rangle \equiv \begin{pmatrix}x_{21},\ldots ,x_{2n},- \left[x_{11} x_{21} + \cdots  + x_{1n} x_{2n}\right]\end{pmatrix}^T
\end{split}
\end{equation}
which are orthogonal vectors in $n+1$ dimensions and project onto
$\vert {\bf e}_1 \rangle$ and
$\vert {\bf e}_2 \rangle$,
respectively.
The~configuration for $n=2$ is depicted in Figure~\ref{2016-vector-f2}.


The general proof by construction is iterative and follows an ``inverse'' (in the sense of ``adding'' {some dimension rather than ``subtracting'' existing vector projections)
generalized Gram-Schmidt process:}
suppose all the $k$ vectors of $n$-dimensional space are non-collinear and non-orthogonal.
\linebreak Then,  by a series of  combinations, whose number is determined by the binomial coefficient $\begin{pmatrix} k \\ 2 \end{pmatrix}$, \linebreak of two vector orthogonalisations as above,
while at the same time, by taking the new vector component of all the other $k-2$ vectors zero,
one can construct a set of $k$ mutually orthogonal vectors in
$\left[k + \begin{pmatrix} k \\ 2 \end{pmatrix}\right]$-dimensional vector space whose orthogonal projections on the original space are the original vectors.
Another orthogonalisation by dimensional lifting requiring merely $n+k$ dimensions due to Havlicek~\cite{havlicek-priv2}
will be discussed in a forthcoming paper.

A necessary and sufficient condition for a generalized Gram-Schmidt process to be representable by a quantum evolution is
its  correspondence to some unitary operation from the initial into \linebreak the final state.
Unfortunately, by design this generalized Gram-Schmidt process is not unitary {\it {per se}}.
(However, this also is true for the functional encoding of bits before the introduction of \linebreak a quantum oracle:
{\it a priori} the introduction of auxiliary bits is not guaranteed to be representable by a unitary transformation.)
Recall that, as has been mentioned earlier, a necessary and sufficient condition for a unitary transformation is that, by
$
\textsf{\textbf{U}}_{fe}=  \sum_{i=1}^n  \vert {\bf f}_i\rangle \langle {\bf e}_i \vert
$
any orthonormal basis
$\mathfrak{B}_n \equiv \{\vert {\bf e}_1\rangle , \vert  {\bf e}_2\rangle , \ldots , \vert {\bf e}_n\rangle \}$ is transformed
into another one
$\mathfrak{B}'_n \equiv \{\vert {\bf f}_1\rangle , \vert  {\bf f}_2\rangle , \ldots , \vert {\bf f}_n\rangle \}$~\cite{Schwinger.60}.
However, in our case, neither are the original vectors orthogonal, nor needs the target system to be a basis.\linebreak
(This latter deficiency could, in principle, be remedied by adding basis vectors,
thereby creating an~eutactic star~\cite{schlaefli-1901,hasse-stachel96}.)
So, unless the vectors resulting from (i) are not already properly orthogonal (for the particular quantum query),
the generalized Gram-Schmidt process (ii) has no quantum mechanical realization.

Moreover, unitary transformations preserve the angles among vectors.
Therefore, if the functional behaviour, subjected to a quantum oracle,
cannot separate different functional properties by orthogonal subspaces,
not much can be gained by a straightforward generalized Gram-Schmidt processes.

Nevertheless, let us, for the sake of another example of dimensional lifting,
come back to \linebreak Deutsch's problem, and
consider another {\it ad hoc} quantum oracle, namely
$\textsf{\textbf{V}}_{f_i} \vert {\bf x} {\bf y} \rangle = \vert {\bf x}  f_i({\bf y}) \rangle$, $i\in \{0,1,2,3\}$,
resulting in
\setlength{\medmuskip}{1mu}
\begin{equation}
\begin{split}
\textsf{\textbf{V}}_{f_0}\vert (0+1)(0+1) \rangle = \vert +00 + 10 \rangle \equiv  \begin{pmatrix} +0+0\end{pmatrix}^T    \\
\textsf{\textbf{V}}_{f_1}\vert (0+1)(0+1) \rangle = \vert +00 + 11 \rangle \equiv  \begin{pmatrix} +00+\end{pmatrix}^T  \\
\textsf{\textbf{V}}_{f_2}\vert (0+1)(0+1) \rangle = \vert +01 + 10 \rangle \equiv  \begin{pmatrix} 0++0\end{pmatrix}^T   \\
\textsf{\textbf{V}}_{f_3}\vert (0+1)(0+1) \rangle = \vert +01 + 11 \rangle \equiv  \begin{pmatrix} 0+0+\end{pmatrix}^T
\end{split}
\label{2016-vector-e-dv}
\end{equation}
\setlength{\medmuskip}{3mu}

By appending four extra dimensions it is not difficult to {\it ad hoc} orthogonalise this set of vectors (two pairs are already orthogonal);
namely
\begin{equation}
\begin{split}
\begin{pmatrix} +0+0+000\end{pmatrix}^T   \\
\begin{pmatrix} +00+-+00\end{pmatrix}^T  \\
\begin{pmatrix} 0++0--++\end{pmatrix}^T  \\
\begin{pmatrix} 0+0+0---\end{pmatrix}^T
\end{split}
\label{2016-vector-e-dvo}
\end{equation}
so that, provided a suitable quantum oracle could be found which maps the functional behaviour into
the orthogonal subspaces spanned by the vectors in {Equation} (\ref{2016-vector-e-dvo}),
an unknown function $f_i$ could be identified uniquely
by a single query (encoding the orthogonal projections corresponding to this latter set of orthogonal vectors).

\subsection{Counterexample and (In-)Sufficiency}

Our hypothesis has been that it is not totally unreasonable
to speculate that any functional behaviour can somehow be mapped into orthogonal subspaces
through the application of a single functional call whose argument is the (equal-weighted) superposition of all possible classical variations of arguments.
More specifically, if reversibility is guaranteed by the introduction of auxiliary~bits,
{different functional behaviours on such arguments result in mutually different (non-orthogonal) vectors;} which then might be
``lifted''  into mutually orthogonal subspaces in Hilbert spaces of greater dimensions than the original Hilbert space.
Any query, performed on this higher-dimensional {Hilbert~space, could be realized by ``bundling''
(formally by adding)
the associated projections appropriately.}

Indeed, it might be possible (in contrast to findings using different assumptions and oracles~\cite{Farhi-98})
with this method to solve the parity problem by a single quantum query, through separating the parity even and odd functions into
two orthogonal subspaces, whose direct sum is the entire Hilbert space.

However, the unitary quantum state evolution imposes strong restrictions on such queries, because it essentially amounts to isometries; that is,
to rotations of state vectors:
the functional properties must be encoded into these rotations; and
the subspaces associated with the particular functional classes
which need to be distinguished from one another must be
rotated in such ways that their relative separation amounts to orthogonality.


%As has been mentioned earlier, a collection of functions which map,
%through some quantum oracle, one and the same input quantum state
%(which is the coherent superposition of all mutually exclusive classical states)
%into a final state
%can be deterministically (that is, with certainty)
%distinguished from one another if and only if the resulting output states are orthogonal.
%This quantum oracle must be unitary;
%that is, it must be representable by a unique mapping of one orthonormal basis (representing the input state) into another one (representing the output state),
%thereby preserving the angles of the input and output states.

Therefore there exist collections of functions which, relative to some particular quantum oracles,
in particular,
$\textsf{\textbf{U}}_i( \vert {\bf x}_1,\ldots , {\bf x}_n, {\bf y} \rangle ) =
\vert {\bf x}_1,\ldots , {\bf x}_n, {\bf y}\oplus f_i(   {\bf x}_1,\ldots , {\bf x}_n)\rangle$,
cannot be orthogonally separated.

For an indication consider~\cite{svozil-2015-AUC-svozil}, as a generalization of
Deutsch's problem discussed earlier, two~particular binary functions of two bits with different parity,
namely
$f_0(xy)=0$ for all $x,y\in \{0,1\}$,
and
$f_{8}(00)=f_{8}(01)=f_{8}(10)=0$, as well as $f_{8}(11)=1$.
A similar calculation as in Equation ~(\ref{2016-vector-e-dp})~yields
\begin{equation}
\begin{split}
\textsf{\textbf{U}}_{f_0}\vert (0-1)(0-1)(0-1) \rangle \\
= \vert + 000 - 001 - 010 + 011 - 100 + 101 + 110 - 111 \rangle \\
\equiv  \begin{pmatrix} +--+-++-\end{pmatrix}^T    \\
\textsf{\textbf{U}}_{f_8}\vert (0-1)(0-1)(0-1) \rangle \\
= \vert + 000 - 001 - 010 + 011 - 100 + 101 - 110 + 111 \rangle \\
\equiv  \begin{pmatrix} +--+-+-+\end{pmatrix}^T
\end{split}
\label{2016-vector-e-dpgen}
\end{equation}
$f_0$ and $f_8$ represent functional parity $+1$ and $-1$; and yet
their respective quantum oracles  $\textsf{\textbf{U}}_{f_0}$ and $\textsf{\textbf{U}}_{f_8}$
do not map the state $\vert (0-1)(0-1)(0-1) \rangle$ into orthogonal states
(indeed, non-orthogonality may come as no surprise).

Note that, although the relative phases of the coherent superposition of the three input bits have been chosen to be $\pi$,
other phases could result in orthogonal vectors as desired.
For the sake of demonstration, consider the general case (with $\vert a_i \vert^2  + \vert b_i \vert^2 = 1$)
\begin{widetext}
\begin{equation}
\begin{split}
\textsf{\textbf{U}}_{f_0}
\left( a_1\vert 0 \rangle + b_1\vert 1 \rangle \right)
\left( a_2\vert 0 \rangle + b_2\vert 1 \rangle \right)
\left( a_3\vert 0 \rangle + b_3\vert 1 \rangle \right)
\\
=
a_1a_2a_3 \vert 000 \rangle  +
a_1a_2b_3 \vert 001 \rangle  +
a_1b_2a_3 \vert 010 \rangle  +
a_1b_2b_3 \vert 011 \rangle  + \\
b_1a_2a_3 \vert 100 \rangle  +
b_1a_2b_3 \vert 101 \rangle  +
b_1b_2a_3 \vert 110 \rangle  +
b_1b_2b_3 \vert 111 \rangle \\
\equiv
\begin{pmatrix}
a_1a_2a_3 ,
a_1a_2b_3 ,
a_1b_2a_3 ,
a_1b_2b_3 ,
b_1a_2a_3 ,
b_1a_2b_3 ,
b_1b_2a_3 ,
b_1b_2b_3
\end{pmatrix}^T  \\
\textsf{\textbf{U}}_{f_8}
\left( a_1\vert 0 \rangle + b_1\vert 1 \rangle \right)
\left( a_2\vert 0 \rangle + b_2\vert 1 \rangle \right)
\left( a_3\vert 0 \rangle + b_3\vert 1 \rangle \right)
\\
=
a_1a_2a_3 \vert 000 \rangle  +
a_1a_2b_3 \vert 001 \rangle  +
a_1b_2a_3 \vert 010 \rangle  +
a_1b_2b_3 \vert 011 \rangle  +  \\
b_1a_2a_3 \vert 100 \rangle  +
b_1a_2b_3 \vert 101 \rangle  +
b_1b_2b_3 \vert 110 \rangle  +
b_1b_2a_3 \vert 111 \rangle \\
\equiv
\begin{pmatrix}
a_1a_2a_3 ,
a_1a_2b_3 ,
a_1b_2a_3 ,
a_1b_2b_3 ,
b_1a_2a_3 ,
b_1a_2b_3 ,
b_1b_2b_3 ,
b_1b_2a_3
\end{pmatrix}^T
\end{split}
\label{2016-vector-e-dpgen1}
\end{equation}
\end{widetext}

The scalar product of these vector vanishes for the particular value assignments
 $a_1=a_2=a_3=0$ and thus $b_1=b_2=b_3=1$.
Alas, this particular value assignment is improper to separate other cases
such as, for instance, $f_8$ and $f_{15} = 1$
%yielding the state $\begin{pmatrix}
%a_1a_2b_3 ,
%a_1a_2a_3 ,
%a_1b_2b_3 ,
%a_1b_2a_3 ,
%b_1a_2b_3 ,
%b_1a_2a_3 ,
%b_1b_2b_3 ,
%b_1b_2a_3
%\end{pmatrix}^T$,
orthogonally.

\section{Summary}

The main goal of this paper has been a discussion of, and the enumeration of criteria for encoding some algorithmic task into quantum mechanically feasible orthogonal subspaces.
This is by no means a trivial task.

With regards to somehow ``mapping'' a functional behaviour into Hilbert space,
such that certain  relational or  holistic functional properties---manifesting themselves  in the {\em {relational}} values for different elements of their domain,
and only after the evaluation of more than one classical input---reveal themselves through efficient quantum queries,
the situation is what may be called ``ambivalent.''

On the one hand, relative to some quantum oracle, it is sometimes impossible to decide certain
relational or functional properties,
simply because this particular quantum oracle is unsuitable to render the appropriate orthogonal subspaces.

On the other, more promising, hand, by the orthogonalisation via dimensional lifting of non-orthogonal vectors,
it cannot be excluded that any functional property can be mapped into orthogonal subspaces of some Hilbert space of higher dimension
than the space spanned by the original non-orthogonal vectors.
In this sense, any functional property, including parity, might be decidable by a single quantum query.

The situation is not dissimilar to the purification of a mixed state, where the
``missing information'' is (non-uniquely) supplemented by fitting hypotheses or conjectures (about the original pure state).
However, this generalized Gram-Schmidt process is not unitary in general, and thus does not correspond to any quantum capacity.

So far, all quantum speedups based on orthogonalisation have been {\it ad hoc};
no method to convert or translate a general functional behavior encoded by a coherent superposition of functional clauses
(and made injective by adding bits) into orthogonal subspaces of a Hilbert space has been found.
Also~the method of dimensional lifting introduced here is quantum infeasible because it is non-unitary (translating non-orthogonal vector into orthogonal ones).
Therefore, the question remains of whether a~general unitary method of dimensional lifting, possibly utilizing auxiliary bits, can be found.



\acknowledgments{\textbf{Acknowledgments:} This work was supported in part by the European Union, Research Executive Agency (REA),
Marie Curie FP7-PEOPLE-2010-IRSES-269151-RANPHYS grant. }

%\conflictofinterests{\textbf{Conflicts of Interest:} The author declares no conflict of interest. }

\ifws
\bibliographystyle{mdpi}
\fi

%%%%\bibliography{svozil}

%merlin.mbs apsrev4-1.bst 2010-07-25 4.21a (PWD, AO, DPC) hacked
%Control: key (0)
%Control: author (0) dotless jnrlst
%Control: editor formatted (1) identically to author
%Control: production of article title (0) allowed
%Control: page (1) range
%Control: year (0) verbatim
%Control: production of eprint (0) enabled
\begin{thebibliography}{45}%
\makeatletter
\providecommand \@ifxundefined [1]{%
 \@ifx{#1\undefined}
}%
\providecommand \@ifnum [1]{%
 \ifnum #1\expandafter \@firstoftwo
 \else \expandafter \@secondoftwo
 \fi
}%
\providecommand \@ifx [1]{%
 \ifx #1\expandafter \@firstoftwo
 \else \expandafter \@secondoftwo
 \fi
}%
\providecommand \natexlab [1]{#1}%
\providecommand \enquote  [1]{``#1''}%
\providecommand \bibnamefont  [1]{#1}%
\providecommand \bibfnamefont [1]{#1}%
\providecommand \citenamefont [1]{#1}%
\providecommand \href@noop [0]{\@secondoftwo}%
\providecommand \href [0]{\begingroup \@sanitize@url \@href}%
\providecommand \@href[1]{\@@startlink{#1}\@@href}%
\providecommand \@@href[1]{\endgroup#1\@@endlink}%
\providecommand \@sanitize@url [0]{\catcode `\\12\catcode `\$12\catcode
  `\&12\catcode `\#12\catcode `\^12\catcode `\_12\catcode `\%12\relax}%
\providecommand \@@startlink[1]{}%
\providecommand \@@endlink[0]{}%
\providecommand \url  [0]{\begingroup\@sanitize@url \@url }%
\providecommand \@url [1]{\endgroup\@href {#1}{\urlprefix }}%
\providecommand \urlprefix  [0]{URL }%
\providecommand \Eprint [0]{\href }%
\providecommand \doibase [0]{http://dx.doi.org/}%
\providecommand \selectlanguage [0]{\@gobble}%
\providecommand \bibinfo  [0]{\@secondoftwo}%
\providecommand \bibfield  [0]{\@secondoftwo}%
\providecommand \translation [1]{[#1]}%
\providecommand \BibitemOpen [0]{}%
\providecommand \bibitemStop [0]{}%
\providecommand \bibitemNoStop [0]{.\EOS\space}%
\providecommand \EOS [0]{\spacefactor3000\relax}%
\providecommand \BibitemShut  [1]{\csname bibitem#1\endcsname}%
\let\auto@bib@innerbib\@empty
%</preamble>
\bibitem [{\citenamefont {Hermann}(1969)}]{hermann}%
  \BibitemOpen
  \bibfield  {author} {\bibinfo {author} {\bibfnamefont {Armin}\ \bibnamefont
  {Hermann}},\ }\href@noop {} {\emph {\bibinfo {title} {{F}r\"uhgeschichte der
  {Q}uantentheorie (1899-1913)}}}\ (\bibinfo  {publisher} {{P}hysik {V}erlag},\
  \bibinfo {address} {Mosbach in Baden},\ \bibinfo {year} {1969})\ \bibinfo
  {note} {{H}abilitationsschrift, Naturwissenschaftliche Fakult\"at der
  Universit\"at M\"unchen}\BibitemShut {NoStop}%
\bibitem [{\citenamefont
  {Schr{\"{o}}dinger}(1995)}]{schroedinger-interpretation}%
  \BibitemOpen
  \bibfield  {author} {\bibinfo {author} {\bibfnamefont {Erwin}\ \bibnamefont
  {Schr{\"{o}}dinger}},\ }\href@noop {} {\emph {\bibinfo {title} {The
  Interpretation of Quantum Mechanics. {D}ublin Seminars (1949-1955) and Other
  Unpublished Essays}}}\ (\bibinfo  {publisher} {Ox Bow Press},\ \bibinfo
  {address} {Woodbridge, Connecticut},\ \bibinfo {year} {1995})\BibitemShut
  {NoStop}%
\bibitem [{\citenamefont {Schr{\"{o}}dinger}(1935)}]{schrodinger}%
  \BibitemOpen
  \bibfield  {author} {\bibinfo {author} {\bibfnamefont {Erwin}\ \bibnamefont
  {Schr{\"{o}}dinger}},\ }\bibfield  {title} {\enquote {\bibinfo {title} {Die
  gegenw{\"{a}}rtige {S}ituation in der {Q}uantenmechanik},}\ }\href {\doibase
  10.1007/BF01491891, 10.1007/BF01491914, 10.1007/BF01491987} {\bibfield
  {journal} {\bibinfo  {journal} {Naturwissenschaften}\ }\textbf {\bibinfo
  {volume} {23}},\ \bibinfo {pages} {807--812, 823--828, 844--849} (\bibinfo
  {year} {1935})}\BibitemShut {NoStop}%
\bibitem [{\citenamefont {Einstein}(1935)}]{einstei-letter-to-schr}%
  \BibitemOpen
  \bibfield  {author} {\bibinfo {author} {\bibfnamefont {Albert}\ \bibnamefont
  {Einstein}},\ }\href {http://alberteinstein.info/vufind1/Record/EAR000024019}
  {\enquote {\bibinfo {title} {Letter to {S}chr\"odinger},}\ } (\bibinfo {year}
  {1935}),\ \bibinfo {note} {old Lyme, dated 19.6.35, Einstein Archives 22-047
  (searchable by document nr. 22-47)}\BibitemShut {NoStop}%
\bibitem [{\citenamefont {Howard}(1985)}]{Howard1985171}%
  \BibitemOpen
  \bibfield  {author} {\bibinfo {author} {\bibfnamefont {Don}\ \bibnamefont
  {Howard}},\ }\bibfield  {title} {\enquote {\bibinfo {title} {{E}instein on
  locality and separability},}\ }\href {\doibase 10.1016/0039-3681(85)90001-9}
  {\bibfield  {journal} {\bibinfo  {journal} {Studies in History and Philosophy
  of Science Part A}\ }\textbf {\bibinfo {volume} {16}},\ \bibinfo {pages}
  {171--201} (\bibinfo {year} {1985})}\BibitemShut {NoStop}%
\bibitem [{\citenamefont {Dirac}(1930)}]{dirac}%
  \BibitemOpen
  \bibfield  {author} {\bibinfo {author} {\bibfnamefont {Paul A.~M.}\
  \bibnamefont {Dirac}},\ }\href@noop {} {\emph {\bibinfo {title} {The
  Principles of Quantum Mechanics}}}\ (\bibinfo  {publisher} {Oxford University
  Press},\ \bibinfo {address} {Oxford},\ \bibinfo {year} {1930})\BibitemShut
  {NoStop}%
\bibitem [{\citenamefont {Birkhoff}\ and\ \citenamefont {{von
  Neumann}}(1936)}]{birkhoff-36}%
  \BibitemOpen
  \bibfield  {author} {\bibinfo {author} {\bibfnamefont {Garrett}\ \bibnamefont
  {Birkhoff}}\ and\ \bibinfo {author} {\bibfnamefont {John}\ \bibnamefont {{von
  Neumann}}},\ }\bibfield  {title} {\enquote {\bibinfo {title} {The logic of
  quantum mechanics},}\ }\href {\doibase 10.2307/1968621} {\bibfield  {journal}
  {\bibinfo  {journal} {Annals of Mathematics}\ }\textbf {\bibinfo {volume}
  {37}},\ \bibinfo {pages} {823--843} (\bibinfo {year} {1936})}\BibitemShut
  {NoStop}%
\bibitem [{\citenamefont {Bernstein}\ and\ \citenamefont
  {Vazirani}(1993)}]{be-va}%
  \BibitemOpen
  \bibfield  {author} {\bibinfo {author} {\bibfnamefont {E.}~\bibnamefont
  {Bernstein}}\ and\ \bibinfo {author} {\bibfnamefont {U.}~\bibnamefont
  {Vazirani}},\ }\bibfield  {title} {\enquote {\bibinfo {title} {Quantum
  complexity theory},}\ }in\ \href {\doibase 10.1137/S0097539796300921} {\emph
  {\bibinfo {booktitle} {Proceedings of the 25th Annual ACM Symposium on Theory
  of Computing, San Diego, California, May 16-18, 1993}}}\ (\bibinfo
  {publisher} {ACM Press},\ \bibinfo {year} {1993})\ pp.\ \bibinfo {pages}
  {11--20}\BibitemShut {NoStop}%
\bibitem [{\citenamefont {Bennett}\ \emph {et~al.}(1997)\citenamefont
  {Bennett}, \citenamefont {Bernstein}, \citenamefont {Brassard},\ and\
  \citenamefont {Vazirani}}]{benn:97}%
  \BibitemOpen
  \bibfield  {author} {\bibinfo {author} {\bibfnamefont {Charles~H.}\
  \bibnamefont {Bennett}}, \bibinfo {author} {\bibfnamefont {Ethan}\
  \bibnamefont {Bernstein}}, \bibinfo {author} {\bibfnamefont {Gilles}\
  \bibnamefont {Brassard}}, \ and\ \bibinfo {author} {\bibfnamefont {Umesh}\
  \bibnamefont {Vazirani}},\ }\bibfield  {title} {\enquote {\bibinfo {title}
  {Strengths and weaknesses of quantum computing},}\ }\href {\doibase
  10.1137/S0097539796300933} {\bibfield  {journal} {\bibinfo  {journal} {SIAM
  Journal on Computing}\ }\textbf {\bibinfo {volume} {26}},\ \bibinfo {pages}
  {1510--1523} (\bibinfo {year} {1997})},\ \Eprint
  {http://arxiv.org/abs/arXiv:quant-ph/9701001} {arXiv:quant-ph/9701001}
  \BibitemShut {NoStop}%
\bibitem [{\citenamefont {Fortnow}(2003)}]{fortnov-03}%
  \BibitemOpen
  \bibfield  {author} {\bibinfo {author} {\bibfnamefont {Lance}\ \bibnamefont
  {Fortnow}},\ }\bibfield  {title} {\enquote {\bibinfo {title} {One complexity
  theorist's view of quantum computing},}\ }\href {\doibase
  10.1016/S0304-3975(01)00377-2} {\bibfield  {journal} {\bibinfo  {journal}
  {Theoretical Computer Science}\ }\textbf {\bibinfo {volume} {292}},\ \bibinfo
  {pages} {597--610} (\bibinfo {year} {2003})}\BibitemShut {NoStop}%
\bibitem [{\citenamefont {Castagnoli}(2013)}]{Castagnoli}%
  \BibitemOpen
  \bibfield  {author} {\bibinfo {author} {\bibfnamefont {Giuseppe}\
  \bibnamefont {Castagnoli}},\ }\href@noop {} {\enquote {\bibinfo {title}
  {Highlighting the mechanism of the quantum speedup by time-symmetric and
  relational quantum mechanics},}\ } (\bibinfo {year} {2013}),\ \Eprint
  {http://arxiv.org/abs/arXiv:1308.5077} {arXiv:1308.5077} \BibitemShut
  {NoStop}%
\bibitem [{\citenamefont {Svozil}(2014)}]{svozil-2013-omelette}%
  \BibitemOpen
  \bibfield  {author} {\bibinfo {author} {\bibfnamefont {Karl}\ \bibnamefont
  {Svozil}},\ }\bibfield  {title} {\enquote {\bibinfo {title} {Unscrambling the
  quantum omelette},}\ }\href {\doibase 10.1007/s10773-013-1995-3} {\bibfield
  {journal} {\bibinfo  {journal} {International Journal of Theoretical
  Physics}\ }\textbf {\bibinfo {volume} {53}},\ \bibinfo {pages} {3648--3657}
  (\bibinfo {year} {2014})},\ \Eprint {http://arxiv.org/abs/arXiv:1206.6024}
  {arXiv:1206.6024} \BibitemShut {NoStop}%
\bibitem [{\citenamefont {Halmos}(1974)}]{halmos-vs}%
  \BibitemOpen
  \bibfield  {author} {\bibinfo {author} {\bibfnamefont {Paul~R..}\
  \bibnamefont {Halmos}},\ }\href@noop {} {\emph {\bibinfo {title}
  {Finite-dimensional Vector Spaces}}}\ (\bibinfo  {publisher} {Springer},\
  \bibinfo {address} {New York, Heidelberg, Berlin},\ \bibinfo {year}
  {1974})\BibitemShut {NoStop}%
\bibitem [{\citenamefont {Abbott}\ \emph {et~al.}(2015)\citenamefont {Abbott},
  \citenamefont {Calude},\ and\ \citenamefont {Svozil}}]{2015-AnalyticKS}%
  \BibitemOpen
  \bibfield  {author} {\bibinfo {author} {\bibfnamefont {Alastair~A.}\
  \bibnamefont {Abbott}}, \bibinfo {author} {\bibfnamefont {Cristian~S.}\
  \bibnamefont {Calude}}, \ and\ \bibinfo {author} {\bibfnamefont {Karl}\
  \bibnamefont {Svozil}},\ }\bibfield  {title} {\enquote {\bibinfo {title} {A
  variant of the {K}ochen-{S}pecker theorem localising value indefiniteness},}\
  }\href {\doibase 10.1063/1.4931658} {\bibfield  {journal} {\bibinfo
  {journal} {Journal of Mathematical Physics}\ }\textbf {\bibinfo {volume}
  {56}},\ \bibinfo {eid} {102201} (\bibinfo {year} {2015})},\ \Eprint
  {http://arxiv.org/abs/arXiv:1503.01985} {arXiv:1503.01985} \BibitemShut
  {NoStop}%
\bibitem [{\citenamefont {Svozil}(2004)}]{svozil-2003-garda}%
  \BibitemOpen
  \bibfield  {author} {\bibinfo {author} {\bibfnamefont {Karl}\ \bibnamefont
  {Svozil}},\ }\bibfield  {title} {\enquote {\bibinfo {title} {Quantum
  information via state partitions and the context translation principle},}\
  }\href {\doibase 10.1080/09500340410001664179} {\bibfield  {journal}
  {\bibinfo  {journal} {Journal of Modern Optics}\ }\textbf {\bibinfo {volume}
  {51}},\ \bibinfo {pages} {811--819} (\bibinfo {year} {2004})},\ \Eprint
  {http://arxiv.org/abs/arXiv:quant-ph/0308110} {arXiv:quant-ph/0308110}
  \BibitemShut {NoStop}%
\bibitem [{\citenamefont {Gleason}(1957)}]{Gleason}%
  \BibitemOpen
  \bibfield  {author} {\bibinfo {author} {\bibfnamefont {Andrew~M.}\
  \bibnamefont {Gleason}},\ }\bibfield  {title} {\enquote {\bibinfo {title}
  {Measures on the closed subspaces of a {H}ilbert space},}\ }\href {\doibase
  10.1512/iumj.1957.6.56050"} {\bibfield  {journal} {\bibinfo  {journal}
  {Journal of Mathematics and Mechanics (now Indiana University Mathematics
  Journal)}\ }\textbf {\bibinfo {volume} {6}},\ \bibinfo {pages} {885--893}
  (\bibinfo {year} {1957})}\BibitemShut {NoStop}%
\bibitem [{\citenamefont {Kochen}\ and\ \citenamefont
  {Specker}(1967)}]{kochen1}%
  \BibitemOpen
  \bibfield  {author} {\bibinfo {author} {\bibfnamefont {Simon}\ \bibnamefont
  {Kochen}}\ and\ \bibinfo {author} {\bibfnamefont {Ernst~P.}\ \bibnamefont
  {Specker}},\ }\bibfield  {title} {\enquote {\bibinfo {title} {The problem of
  hidden variables in quantum mechanics},}\ }\href {\doibase
  10.1512/iumj.1968.17.17004} {\bibfield  {journal} {\bibinfo  {journal}
  {Journal of Mathematics and Mechanics (now Indiana University Mathematics
  Journal)}\ }\textbf {\bibinfo {volume} {17}},\ \bibinfo {pages} {59--87}
  (\bibinfo {year} {1967})}\BibitemShut {NoStop}%
\bibitem [{\citenamefont {Pitowsky}(1989)}]{pitowsky-89a}%
  \BibitemOpen
  \bibfield  {author} {\bibinfo {author} {\bibfnamefont {Itamar}\ \bibnamefont
  {Pitowsky}},\ }\bibfield  {title} {\enquote {\bibinfo {title} {From {G}eorge
  {B}oole to {J}ohn {B}ell: The origin of {B}ell's inequality},}\ }in\ \href
  {\doibase 10.1007/978-94-017-0849-4\_6} {\emph {\bibinfo {booktitle}
  {{B}ell's Theorem, Quantum Theory and the Conceptions of the Universe}}},\
  \bibinfo {series} {Fundamental Theories of Physics}, Vol.~\bibinfo {volume}
  {37},\ \bibinfo {editor} {edited by\ \bibinfo {editor} {\bibfnamefont
  {Menas}\ \bibnamefont {Kafatos}}}\ (\bibinfo  {publisher} {Springer
  (Kluwer)},\ \bibinfo {address} {Dordrecht},\ \bibinfo {year} {1989})\ pp.\
  \bibinfo {pages} {37--49}\BibitemShut {NoStop}%
\bibitem [{\citenamefont {Mermin}(2007)}]{mermin-07}%
  \BibitemOpen
  \bibfield  {author} {\bibinfo {author} {\bibfnamefont {David~N.}\
  \bibnamefont {Mermin}},\ }\href
  {http://people.ccmr.cornell.edu/~mermin/qcomp/CS483.html} {\emph {\bibinfo
  {title} {Quantum Computer Science}}}\ (\bibinfo  {publisher} {Cambridge
  University Press},\ \bibinfo {address} {Cambridge},\ \bibinfo {year}
  {2007})\BibitemShut {NoStop}%
\bibitem [{\citenamefont {Zeilinger}(1997)}]{Zeilinger-97}%
  \BibitemOpen
  \bibfield  {author} {\bibinfo {author} {\bibfnamefont {Anton}\ \bibnamefont
  {Zeilinger}},\ }\bibfield  {title} {\enquote {\bibinfo {title} {Quantum
  teleportation and the non-locality of information},}\ }\href {\doibase
  10.1098/rsta.1997.0138} {\bibfield  {journal} {\bibinfo  {journal}
  {Philosophical Transactions of the Royal Society of London A}\ }\textbf
  {\bibinfo {volume} {355}},\ \bibinfo {pages} {2401--2404} (\bibinfo {year}
  {1997})}\BibitemShut {NoStop}%
\bibitem [{\citenamefont {Zeilinger}(1999)}]{zeil-99}%
  \BibitemOpen
  \bibfield  {author} {\bibinfo {author} {\bibfnamefont {Anton}\ \bibnamefont
  {Zeilinger}},\ }\bibfield  {title} {\enquote {\bibinfo {title} {A
  foundational principle for quantum mechanics},}\ }\href {\doibase
  10.1023/A:1018820410908} {\bibfield  {journal} {\bibinfo  {journal}
  {Foundations of Physics}\ }\textbf {\bibinfo {volume} {29}},\ \bibinfo
  {pages} {631--643} (\bibinfo {year} {1999})}\BibitemShut {NoStop}%
\bibitem [{\citenamefont {Einstein}(1948)}]{Einstein-48}%
  \BibitemOpen
  \bibfield  {author} {\bibinfo {author} {\bibfnamefont {A.}~\bibnamefont
  {Einstein}},\ }\bibfield  {title} {\enquote {\bibinfo {title}
  {{Q}uanten-{M}echanik und {W}irklichkeit},}\ }\href {\doibase
  10.1111/j.1746-8361.1948.tb00704.x} {\bibfield  {journal} {\bibinfo
  {journal} {Dialectica}\ }\textbf {\bibinfo {volume} {2}},\ \bibinfo {pages}
  {320324} (\bibinfo {year} {1948})}\BibitemShut {NoStop}%
\bibitem [{\citenamefont {Schwinger}(1960)}]{Schwinger.60}%
  \BibitemOpen
  \bibfield  {author} {\bibinfo {author} {\bibfnamefont {Julian}\ \bibnamefont
  {Schwinger}},\ }\bibfield  {title} {\enquote {\bibinfo {title} {Unitary
  operators bases},}\ }\href {\doibase 10.1073/pnas.46.4.570} {\bibfield
  {journal} {\bibinfo  {journal} {Proceedings of the National Academy of
  Sciences (PNAS)}\ }\textbf {\bibinfo {volume} {46}},\ \bibinfo {pages}
  {570--579} (\bibinfo {year} {1960})}\BibitemShut {NoStop}%
\bibitem [{\citenamefont {Bennett}\ \emph {et~al.}(1993)\citenamefont
  {Bennett}, \citenamefont {Brassard}, \citenamefont {Cr{\'{e}}peau},
  \citenamefont {Jozsa}, \citenamefont {Peres},\ and\ \citenamefont
  {Wootters}}]{BBCJPW}%
  \BibitemOpen
  \bibfield  {author} {\bibinfo {author} {\bibfnamefont {Charles~H.}\
  \bibnamefont {Bennett}}, \bibinfo {author} {\bibfnamefont {Gilles}\
  \bibnamefont {Brassard}}, \bibinfo {author} {\bibfnamefont {Claude}\
  \bibnamefont {Cr{\'{e}}peau}}, \bibinfo {author} {\bibfnamefont {Richard}\
  \bibnamefont {Jozsa}}, \bibinfo {author} {\bibfnamefont {Asher}\ \bibnamefont
  {Peres}}, \ and\ \bibinfo {author} {\bibfnamefont {William~K.}\ \bibnamefont
  {Wootters}},\ }\bibfield  {title} {\enquote {\bibinfo {title} {Teleporting an
  unknown quantum state via dual classical and {E}instein-{P}odolsky-{R}osen
  channels},}\ }\href {\doibase 10.1103/PhysRevLett.70.1895} {\bibfield
  {journal} {\bibinfo  {journal} {Physical Review Letters}\ }\textbf {\bibinfo
  {volume} {70}},\ \bibinfo {pages} {1895--1899} (\bibinfo {year}
  {1993})}\BibitemShut {NoStop}%
\bibitem [{\citenamefont {Zou}\ \emph {et~al.}(1991)\citenamefont {Zou},
  \citenamefont {Wang},\ and\ \citenamefont {Mandel}}]{zou-wang-mandel:91a}%
  \BibitemOpen
  \bibfield  {author} {\bibinfo {author} {\bibfnamefont {X.~Y.}\ \bibnamefont
  {Zou}}, \bibinfo {author} {\bibfnamefont {L.~J.}\ \bibnamefont {Wang}}, \
  and\ \bibinfo {author} {\bibfnamefont {L.}~\bibnamefont {Mandel}},\
  }\bibfield  {title} {\enquote {\bibinfo {title} {Induced coherence and
  indistinguishability in optical interference},}\ }\href {\doibase
  10.1103/PhysRevLett.67.318} {\bibfield  {journal} {\bibinfo  {journal}
  {Physical Review Letters}\ }\textbf {\bibinfo {volume} {67}},\ \bibinfo
  {pages} {318--321} (\bibinfo {year} {1991})}\BibitemShut {NoStop}%
\bibitem [{\citenamefont {Peres}(1980)}]{PhysRevD.22.879}%
  \BibitemOpen
  \bibfield  {author} {\bibinfo {author} {\bibfnamefont {Asher}\ \bibnamefont
  {Peres}},\ }\bibfield  {title} {\enquote {\bibinfo {title} {Can we undo
  quantum measurements?}}\ }\href {\doibase 10.1103/PhysRevD.22.879} {\bibfield
   {journal} {\bibinfo  {journal} {Physical Review D}\ }\textbf {\bibinfo
  {volume} {22}},\ \bibinfo {pages} {879--883} (\bibinfo {year}
  {1980})}\BibitemShut {NoStop}%
\bibitem [{\citenamefont {Scully}\ and\ \citenamefont
  {Dr\"uhl}(1982)}]{PhysRevA.25.2208}%
  \BibitemOpen
  \bibfield  {author} {\bibinfo {author} {\bibfnamefont {Marlan~O.}\
  \bibnamefont {Scully}}\ and\ \bibinfo {author} {\bibfnamefont {Kai}\
  \bibnamefont {Dr\"uhl}},\ }\bibfield  {title} {\enquote {\bibinfo {title}
  {Quantum eraser: A proposed photon correlation experiment concerning
  observation and ``delayed choice'' in quantum mechanics},}\ }\href {\doibase
  10.1103/PhysRevA.25.2208} {\bibfield  {journal} {\bibinfo  {journal}
  {Physical Review A}\ }\textbf {\bibinfo {volume} {25}},\ \bibinfo {pages}
  {2208--2213} (\bibinfo {year} {1982})}\BibitemShut {NoStop}%
\bibitem [{\citenamefont {Englert}\ \emph {et~al.}(1988)\citenamefont
  {Englert}, \citenamefont {Schwinger},\ and\ \citenamefont
  {Scully}}]{engrt-sg-I}%
  \BibitemOpen
  \bibfield  {author} {\bibinfo {author} {\bibfnamefont {Berthold-Georg}\
  \bibnamefont {Englert}}, \bibinfo {author} {\bibfnamefont {Julian}\
  \bibnamefont {Schwinger}}, \ and\ \bibinfo {author} {\bibfnamefont
  {Marlan~O.}\ \bibnamefont {Scully}},\ }\bibfield  {title} {\enquote {\bibinfo
  {title} {Is spin coherence like {H}umpty-{D}umpty? {I}. {S}implified
  treatment},}\ }\href {\doibase 10.1007/BF01909939} {\bibfield  {journal}
  {\bibinfo  {journal} {Foundations of Physics}\ }\textbf {\bibinfo {volume}
  {18}},\ \bibinfo {pages} {1045--1056} (\bibinfo {year} {1988})}\BibitemShut
  {NoStop}%
\bibitem [{\citenamefont {{Everett III}}(1957)}]{everett}%
  \BibitemOpen
  \bibfield  {author} {\bibinfo {author} {\bibfnamefont {Hugh}\ \bibnamefont
  {{Everett III}}},\ }\bibfield  {title} {\enquote {\bibinfo {title}
  {`{R}elative {S}tate' formulation of quantum mechanics},}\ }\href {\doibase
  10.1103/RevModPhys.29.454} {\bibfield  {journal} {\bibinfo  {journal}
  {Reviews of Modern Physics}\ }\textbf {\bibinfo {volume} {29}},\ \bibinfo
  {pages} {454--462} (\bibinfo {year} {1957})}\BibitemShut {NoStop}%
\bibitem [{\citenamefont {Wigner}(1961)}]{wigner:mb}%
  \BibitemOpen
  \bibfield  {author} {\bibinfo {author} {\bibfnamefont {Eugene~P.}\
  \bibnamefont {Wigner}},\ }\bibfield  {title} {\enquote {\bibinfo {title}
  {Remarks on the mind-body question},}\ }in\ \href
  {http://www.phys.uu.nl/igg/jos/foundQM/wigner.pdf} {\emph {\bibinfo
  {booktitle} {The Scientist Speculates}}},\ \bibinfo {editor} {edited by\
  \bibinfo {editor} {\bibfnamefont {I.~J.}\ \bibnamefont {Good}}}\ (\bibinfo
  {publisher} {Heinemann and Basic Books},\ \bibinfo {address} {London and New
  York},\ \bibinfo {year} {1961})\ pp.\ \bibinfo {pages} {284--302}\BibitemShut
  {NoStop}%
\bibitem [{\citenamefont {Bell}(1990)}]{bell-a}%
  \BibitemOpen
  \bibfield  {author} {\bibinfo {author} {\bibfnamefont {John~S.}\ \bibnamefont
  {Bell}},\ }\bibfield  {title} {\enquote {\bibinfo {title} {Against
  `measurement'},}\ }\href
  {http://physicsworldarchive.iop.org/summary/pwa-xml/3/8/phwv3i8a26}
  {\bibfield  {journal} {\bibinfo  {journal} {Physics World}\ }\textbf
  {\bibinfo {volume} {3}},\ \bibinfo {pages} {33--41} (\bibinfo {year}
  {1990})}\BibitemShut {NoStop}%
\bibitem [{\citenamefont {Myrvold}(2011)}]{Myrvold2011237}%
  \BibitemOpen
  \bibfield  {author} {\bibinfo {author} {\bibfnamefont {Wayne~C.}\
  \bibnamefont {Myrvold}},\ }\bibfield  {title} {\enquote {\bibinfo {title}
  {Statistical mechanics and thermodynamics: A {M}axwellian view},}\ }\href
  {\doibase 10.1016/j.shpsb.2011.07.001} {\bibfield  {journal} {\bibinfo
  {journal} {Studies in History and Philosophy of Science Part B: Studies in
  History and Philosophy of Modern Physics}\ }\textbf {\bibinfo {volume}
  {42}},\ \bibinfo {pages} {237--243} (\bibinfo {year} {2011})}\BibitemShut
  {NoStop}%
\bibitem [{\citenamefont {Jaynes}(1989)}]{jaynes-89}%
  \BibitemOpen
  \bibfield  {author} {\bibinfo {author} {\bibfnamefont {Edwin~Thompson}\
  \bibnamefont {Jaynes}},\ }\bibfield  {title} {\enquote {\bibinfo {title}
  {Clearing up mysteries - the original goal},}\ }in\ \href
  {http://bayes.wustl.edu/etj/articles/cmystery.pdf} {\emph {\bibinfo
  {booktitle} {Maximum-Entropy and Bayesian Methods: : Proceedings of the 8th
  Maximum Entropy Workshop, held on August 1-5, 1988, in St. John's College,
  Cambridge, England}}},\ \bibinfo {editor} {edited by\ \bibinfo {editor}
  {\bibfnamefont {John}\ \bibnamefont {Skilling}}}\ (\bibinfo  {publisher}
  {Kluwer},\ \bibinfo {address} {Dordrecht},\ \bibinfo {year} {1989})\ pp.\
  \bibinfo {pages} {1--28}\BibitemShut {NoStop}%
\bibitem [{\citenamefont {Jaynes}(1990)}]{jaynes-90}%
  \BibitemOpen
  \bibfield  {author} {\bibinfo {author} {\bibfnamefont {Edwin~Thompson}\
  \bibnamefont {Jaynes}},\ }\bibfield  {title} {\enquote {\bibinfo {title}
  {Probability in quantum theory},}\ }in\ \href
  {http://bayes.wustl.edu/etj/articles/prob.in.qm.pdf} {\emph {\bibinfo
  {booktitle} {Complexity, Entropy, and the Physics of Information: Proceedings
  of the 1988 Workshop on Complexity, Entropy, and the Physics of Information,
  held May - June, 1989, in Santa Fe, New Mexico}}},\ \bibinfo {editor} {edited
  by\ \bibinfo {editor} {\bibfnamefont {Wojciech~Hubert}\ \bibnamefont
  {Zurek}}}\ (\bibinfo  {publisher} {Addison-Wesley},\ \bibinfo {address}
  {Reading, MA},\ \bibinfo {year} {1990})\ pp.\ \bibinfo {pages}
  {381--404}\BibitemShut {NoStop}%
\bibitem [{\citenamefont {Galouye}(1964)}]{simula}%
  \BibitemOpen
  \bibfield  {author} {\bibinfo {author} {\bibfnamefont {Daniel~F.}\
  \bibnamefont {Galouye}},\ }\href@noop {} {\emph {\bibinfo {title} {Simulacron
  3}}}\ (\bibinfo  {publisher} {Bantam Books},\ \bibinfo {address} {New York,
  NY},\ \bibinfo {year} {1964})\BibitemShut {NoStop}%
\bibitem [{\citenamefont {Farhi}\ \emph {et~al.}(1998)\citenamefont {Farhi},
  \citenamefont {Goldstone}, \citenamefont {Gutmann},\ and\ \citenamefont
  {Sipser}}]{Farhi-98}%
  \BibitemOpen
  \bibfield  {author} {\bibinfo {author} {\bibfnamefont {Edward}\ \bibnamefont
  {Farhi}}, \bibinfo {author} {\bibfnamefont {Jeffrey}\ \bibnamefont
  {Goldstone}}, \bibinfo {author} {\bibfnamefont {Sam}\ \bibnamefont
  {Gutmann}}, \ and\ \bibinfo {author} {\bibfnamefont {Michael}\ \bibnamefont
  {Sipser}},\ }\bibfield  {title} {\enquote {\bibinfo {title} {Limit on the
  speed of quantum computation in determining parity},}\ }\href {\doibase
  10.1103/PhysRevLett.81.5442} {\bibfield  {journal} {\bibinfo  {journal}
  {Physical Review Letters}\ }\textbf {\bibinfo {volume} {81}},\ \bibinfo
  {pages} {5442--5444} (\bibinfo {year} {1998})},\ \Eprint
  {http://arxiv.org/abs/arXiv:quant-ph/9802045} {arXiv:quant-ph/9802045}
  \BibitemShut {NoStop}%
\bibitem [{\citenamefont {Moore}(1956)}]{e-f-moore}%
  \BibitemOpen
  \bibfield  {author} {\bibinfo {author} {\bibfnamefont {Edward~F.}\
  \bibnamefont {Moore}},\ }\bibfield  {title} {\enquote {\bibinfo {title}
  {Gedanken-experiments on sequential machines},}\ }in\ \href@noop {} {\emph
  {\bibinfo {booktitle} {Automata Studies}}},\ \bibinfo {editor} {edited by\
  \bibinfo {editor} {\bibfnamefont {C.~E.}\ \bibnamefont {Shannon}}\ and\
  \bibinfo {editor} {\bibfnamefont {J.}~\bibnamefont {McCarthy}}}\ (\bibinfo
  {publisher} {Princeton University Press},\ \bibinfo {address} {Princeton,
  NJ},\ \bibinfo {year} {1956})\ pp.\ \bibinfo {pages} {129--153}\BibitemShut
  {NoStop}%
\bibitem [{\citenamefont {Svozil}(2005)}]{svozil-2001-eua}%
  \BibitemOpen
  \bibfield  {author} {\bibinfo {author} {\bibfnamefont {Karl}\ \bibnamefont
  {Svozil}},\ }\bibfield  {title} {\enquote {\bibinfo {title} {Logical
  equivalence between generalized urn models and finite automata},}\ }\href
  {\doibase 10.1007/s10773-005-7052-0} {\bibfield  {journal} {\bibinfo
  {journal} {International Journal of Theoretical Physics}\ }\textbf {\bibinfo
  {volume} {44}},\ \bibinfo {pages} {745--754} (\bibinfo {year} {2005})},\
  \Eprint {http://arxiv.org/abs/arXiv:quant-ph/0209136}
  {arXiv:quant-ph/0209136} \BibitemShut {NoStop}%
\bibitem [{\citenamefont {Donath}\ and\ \citenamefont
  {Svozil}(2002)}]{DonSvo01}%
  \BibitemOpen
  \bibfield  {author} {\bibinfo {author} {\bibfnamefont {Niko}\ \bibnamefont
  {Donath}}\ and\ \bibinfo {author} {\bibfnamefont {Karl}\ \bibnamefont
  {Svozil}},\ }\bibfield  {title} {\enquote {\bibinfo {title} {Finding a state
  among a complete set of orthogonal ones},}\ }\href {\doibase
  10.1103/PhysRevA.65.044302} {\bibfield  {journal} {\bibinfo  {journal}
  {Physical Review A}\ }\textbf {\bibinfo {volume} {65}},\ \bibinfo {pages}
  {044302} (\bibinfo {year} {2002})},\ \Eprint
  {http://arxiv.org/abs/arXiv:quant-ph/0105046} {arXiv:quant-ph/0105046}
  \BibitemShut {NoStop}%
\bibitem [{\citenamefont {Svozil}(2002)}]{svozil-2002-statepart-prl}%
  \BibitemOpen
  \bibfield  {author} {\bibinfo {author} {\bibfnamefont {Karl}\ \bibnamefont
  {Svozil}},\ }\bibfield  {title} {\enquote {\bibinfo {title} {Quantum
  information in base $n$ defined by state partitions},}\ }\href {\doibase
  10.1103/PhysRevA.66.044306} {\bibfield  {journal} {\bibinfo  {journal}
  {Physical Review A}\ }\textbf {\bibinfo {volume} {66}},\ \bibinfo {pages}
  {044306} (\bibinfo {year} {2002})},\ \Eprint
  {http://arxiv.org/abs/arXiv:quant-ph/0205031} {arXiv:quant-ph/0205031}
  \BibitemShut {NoStop}%
\bibitem [{\citenamefont {Svozil}\ and\ \citenamefont
  {Tkadlec}(2009)}]{2007-tkadlec-svozil-springer}%
  \BibitemOpen
  \bibfield  {author} {\bibinfo {author} {\bibfnamefont {Karl}\ \bibnamefont
  {Svozil}}\ and\ \bibinfo {author} {\bibfnamefont {Josef}\ \bibnamefont
  {Tkadlec}},\ }\bibfield  {title} {\enquote {\bibinfo {title} {On the solution
  of trivalent decision problems by quantum state identification},}\ }\href
  {\doibase 10.1007/s11047-009-9112-5} {\bibfield  {journal} {\bibinfo
  {journal} {Natural Computing}\ }\textbf {\bibinfo {volume} {in print}},\
  \bibinfo {pages} {in print} (\bibinfo {year} {2009})}\BibitemShut {NoStop}%
\bibitem [{\citenamefont {Havlicek}()}]{havlicek-priv2}%
  \BibitemOpen
  \bibfield  {author} {\bibinfo {author} {\bibfnamefont {Hans}\ \bibnamefont
  {Havlicek}},\ }\href@noop {} {}\bibinfo {note} {Private
  communication}\BibitemShut {NoStop}%
\bibitem [{\citenamefont {Schl{\"{a}}fli}(1901)}]{schlaefli-1901}%
  \BibitemOpen
  \bibfield  {author} {\bibinfo {author} {\bibfnamefont {L.}~\bibnamefont
  {Schl{\"{a}}fli}},\ }\href {\doibase 10.1007/978-3-0348-4118-4\_13} {\emph
  {\bibinfo {title} {{T}heorie der vielfachen {K}ontinuit{\"{a}}t}}}\ (\bibinfo
   {publisher} {Z{\"{u}}rcher \& Furrer},\ \bibinfo {address} {Z{\"{u}}rich},\
  \bibinfo {year} {1901})\ pp.\ \bibinfo {pages} {134--139},\ \bibinfo {note}
  {ed. by J. H. Graf for the {D}enkschriften-{K}ommission der {S}chweizerischen
  {N}aturforschenden {G}esellschaft. {S}eparatabdruck aus den {D}enkschriften
  der {S}chweizerischen {N}aturforschenden {G}esellschaft. {B}and {XXXVIII}.
  1st half}\BibitemShut {NoStop}%
\bibitem [{\citenamefont {Haase}\ and\ \citenamefont
  {Stachel}(1996)}]{hasse-stachel96}%
  \BibitemOpen
  \bibfield  {author} {\bibinfo {author} {\bibfnamefont {Dietrich}\
  \bibnamefont {Haase}}\ and\ \bibinfo {author} {\bibfnamefont {Hellmuth}\
  \bibnamefont {Stachel}},\ }\bibfield  {title} {\enquote {\bibinfo {title}
  {Almost-orthonormal vector systems},}\ }\href
  {http://www.emis.de/journals/BAG/vol.37/no.2/} {\bibfield  {journal}
  {\bibinfo  {journal} {Beitr. Algebra Geom.}\ }\textbf {\bibinfo {volume}
  {37}},\ \bibinfo {pages} {367--381} (\bibinfo {year} {1996})}\BibitemShut
  {NoStop}%
\bibitem [{\citenamefont {Svozil}(2015)}]{svozil-2015-AUC-svozil}%
  \BibitemOpen
  \bibfield  {author} {\bibinfo {author} {\bibfnamefont {Karl}\ \bibnamefont
  {Svozil}},\ }\href@noop {} {\enquote {\bibinfo {title} {Quantum queries
  associated with equi-partitioning of states and multipartite relational
  encoding across space-time},}\ } (\bibinfo {year} {2015})\BibitemShut
  {NoStop}%
\end{thebibliography}%

\end{document}


Eliminate[{
a1*a2*a3 == \[Alpha]000,
a1*a2*b3 == \[Alpha]001,
a1*b2*a3 == \[Alpha]010,
b1*a2*a3 == \[Alpha]100,
a1*b2*b3 == \[Alpha]011,
b1*a2*b3 == \[Alpha]101,
b1*b2*a3 == \[Alpha]110,
b1*b2*b3 == \[Alpha]111},
{a1,a2,a3,b1,b2,b3}]


(* Definition of the Dyadic Product *)


DyadicProduct2Vec[x_, y_] :=
  Flatten[Table[x[[i]] Conjugate[y[[j]]], {i, 1, Length[x]}, {j, 1, Length[y]}]];


AA = DyadicProduct2Vec[a1{1,0}+b1{0,1},DyadicProduct2Vec[a2{1,0}+b2{0,1},a3{1,0}+b3{0,1}]]

a1=1;
a2=1;
a3=1;
b1=-1;
b2=-1;
b3=-1;

f0[x_,y_] := 0;
f8[x_,y_] := If[x*y == 1,1,0];


Reduce[{
a1 * a2 * a3 * a1 * a2 * a3  +
a1 * a2*Sqrt[1-a3^2] * a1 * a2*Sqrt[1-a3^2]  +
a1*Sqrt[1-a2^2]a3 * a1*Sqrt[1-a2^2]a3  +
a1*Sqrt[1-a2^2]*Sqrt[1-a3^2] * a1*Sqrt[1-a2^2]*Sqrt[1-a3^2]  +
Sqrt[1-a1^2]*a2 * a3 * Sqrt[1-a1^2]*a2 * a3  +
Sqrt[1-a1^2]*a2*Sqrt[1-a3^2] * Sqrt[1-a1^2]*a2*Sqrt[1-a3^2]  +
Sqrt[1-a1^2]*Sqrt[1-a2^2]*a3 * Sqrt[1-a1^2]*Sqrt[1-a2^2]*Sqrt[1-a3^2]  +
Sqrt[1-a1^2]*Sqrt[1-a2^2]*Sqrt[1-a3^2] * Sqrt[1-a1^2]*Sqrt[1-a2^2]*a3
== 0,a1==0},{a1,a2, a3}]



\begin{equation}
\begin{array}{c|ccccc}
f_i&P(f_i)&f_i(00)&f_i(01)&f_i(10)&f_i(11)\\
\hline
f_0& 0 &    -1 & -1 & -1 & -1 \\
f_1& 0 &    -1 & -1 & +1 & +1 \\
f_2& 0 &    -1 & +1 & -1 & +1 \\
f_3& 0 &    -1 & +1 & +1 & -1 \\
f_4& 0 &    +1 & -1 & -1 & +1 \\
f_5& 0 &    +1 & -1 & +1 & -1 \\
f_6& 0 &    +1 & +1 & -1 & -1 \\
f_7& 0 &    +1 & +1 & +1 & +1 \\
f_8& 1 &    -1 & -1 & -1 & +1 \\
f_9& 1 &    -1 & -1 & +1 & -1 \\
f_{10}& 1 & -1 & +1 & -1 & -1 \\
f_{11}& 1 & -1 & +1 & +1 & +1 \\
f_{12}& 1 & +1 & -1 & -1 & -1 \\
f_{13}& 1 & +1 & -1 & +1 & +1 \\
f_{14}& 1 & +1 & +1 & -1 & +1 \\
f_{15}& 1 & +1 & +1 & +1 & -1  \\
\end{array}
\end{equation}



f_0& 0 &    -1 & -1 & -1 & -1 \\
f_1& 0 &    -1 & -1 & +1 & +1 \\
f_2& 0 &    -1 & +1 & -1 & +1 \\
f_3& 0 &    -1 & +1 & +1 & -1 \\
f_8& 1 &    -1 & -1 & -1 & +1 \\
f_9& 1 &    -1 & -1 & +1 & -1 \\
f_{10}& 1 & -1 & +1 & -1 & -1 \\
f_{11}& 1 & -1 & +1 & +1 & +1 \\

AA = {
{-1 , -1 , -1 , -1 , 1 , 0 , 0, 0 , 0 , 0 , 0 , 0 },
{-1 , -1 , +1 , +1 , x206 , 1 , 0, 0 , 0 , 0 , 0 , 0 },
{-1 , +1 , -1 , +1 , x306 , x307 , 1, 0 , 0 , 0 , 0 , 0 },
{-1 , +1 , +1 , -1 , x406 , x407 , x408, 1 , 0 , 0 , 0 , 0 },
{-1 , -1 , -1 , +1 , x506 , x507 , x508, x509 , 1 , 0 , 0 , 0 },
{-1 , -1 , +1 , -1 , x606 , x607 , x608, x609 , x610 , 1 , 0 , 0 },
{-1 , +1 , -1 , -1 , x706 , x707 , x708, x709 , x710 , x711 , 1 , 0 },
{-1 , +1 , +1 , +1 , x806 , x807 , x808, x809 , x810 , x811 , x812 , 1 }}

Reduce[{
AA[[1]].AA[[2]]==0,
AA[[1]].AA[[3]]==0,
AA[[1]].AA[[4]]==0,
AA[[1]].AA[[5]]==0,
AA[[1]].AA[[6]]==0,
AA[[1]].AA[[7]]==0,
AA[[1]].AA[[8]]==0,
AA[[2]].AA[[3]]==0,
AA[[2]].AA[[4]]==0,
AA[[2]].AA[[5]]==0,
AA[[2]].AA[[6]]==0,
AA[[2]].AA[[7]]==0,
AA[[2]].AA[[8]]==0,
AA[[3]].AA[[4]]==0,
AA[[3]].AA[[5]]==0,
AA[[3]].AA[[6]]==0,
AA[[3]].AA[[7]]==0,
AA[[3]].AA[[8]]==0,
AA[[4]].AA[[5]]==0,
AA[[4]].AA[[6]]==0,
AA[[4]].AA[[7]]==0,
AA[[4]].AA[[8]]==0,
AA[[5]].AA[[6]]==0,
AA[[5]].AA[[7]]==0,
AA[[5]].AA[[8]]==0,
AA[[6]].AA[[7]]==0,
AA[[6]].AA[[8]]==0
},{x206,
 x306 , x307 ,
 x406 , x407 , x408,
 x506 ,  x507 , x508, x509 ,
 x606 , x607 , x608, x609 ,
 x706 , x707 , x708, x709 , x710 , x711 ,
 x806 , x807 , x808, x809 , x810 , x811 , x812}]

x610 == 0 &&
x206 == 0 &&
x306 == 0 &&
x307 == 0 &&
x406 == 0 &&
 x407 == 0 &&
x408 == 0 &&
x506 == -2 &&
x507 == -2 &&
x508 == -2 &&
 x509 == 2 &&
x606 == -2 &&
x607 == -2 &&
x608 == 2 &&
x609 == -2 &&
 x706 == -2 &&
x707 == 2 &&
x708 == -2 &&
x709 == -2 &&
x710 == 0 &&
x711 == 0 &&
x806 == 2 &&
x807 == -2 &&
x808 == -2 &&
x809 == -2 &&
 x810 == 0 &&
x811 == 0

AAA = {
{-1 , -1 , -1 , -1 , 1 , 0 , 0, 0 , 0 , 0 , 0 , 0 },
{-1 , -1 , +1 , +1 , 0 , 1 , 0, 0 , 0 , 0 , 0 , 0 },
{-1 , +1 , -1 , +1 , 0 , 0 , 1, 0 , 0 , 0 , 0 , 0 },
{-1 , +1 , +1 , -1 , 0 , 0 , 0, 1 , 0 , 0 , 0 , 0 },
{-1 , -1 , -1 , +1 , -2 , -2 , -2, -2 , 1 , 0 , 0 , 0 },
{-1 , -1 , +1 , -1 , -2 , -2 , -2, -2 , -2 , 1 , 0 , 0 },
{-1 , +1 , -1 , -1 , -2 , -2 , -2, -2 , 0 , 0 , 1 , 0 },
{-1 , +1 , +1 , +1 , -2 , -2 , -2, -2 , 0 , 0 , -2 , 1 }}


