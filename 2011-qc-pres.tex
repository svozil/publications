%\documentclass[pra,showpacs,showkeys,amsfonts,amsmath,twocolumn]{revtex4}
\documentclass[amsmath,table,sans,amsfonts]{beamer}
%\documentclass[pra,showpacs,showkeys,amsfonts]{revtex4}
\usepackage[T1]{fontenc}
%%\usepackage{beamerthemeshadow}
%%\usepackage[headheight=1pt,footheight=10pt]{beamerthemeboxes}
%%\addfootboxtemplate{\color{structure!80}}{\color{white}\tiny \hfill Karl Svozil (TU Vienna)\hfill}
%%\addfootboxtemplate{\color{structure!65}}{\color{white}\tiny \hfill mur.sat \hfill}
%%\addfootboxtemplate{\color{structure!50}}{\color{white}\tiny \hfill Graz, 2010-12-11\hfill}
%\usepackage[dark]{beamerthemesidebar}
%\usepackage[headheight=24pt,footheight=12pt]{beamerthemesplit}
%\usepackage{beamerthemesplit}
%\usepackage[bar]{beamerthemetree}
\usepackage{graphicx}
\usepackage{pgf}
%\usepackage{eepic}
%\usepackage[usenames]{color}
%\newcommand{\Red}{\color{Red}}  %(VERY-Approx.PANTONE-RED)
%\newcommand{\Green}{\color{Green}}  %(VERY-Approx.PANTONE-GREEN)

%\RequirePackage[german]{babel}
%\selectlanguage{german}
%\RequirePackage[isolatin]{inputenc}

\pgfdeclareimage[height=0.5cm]{logo}{tu-logo}
\logo{\pgfuseimage{logo}}
\beamertemplatetriangleitem
%\beamertemplateballitem

\beamerboxesdeclarecolorscheme{alert}{red}{red!15!averagebackgroundcolor}
%\begin{beamerboxesrounded}[scheme=alert,shadow=true]{}
%\end{beamerboxesrounded}

%\beamersetaveragebackground{yellow!10}

%\beamertemplatecircleminiframe

\begin{document}

\title{\bf \textcolor{blue}{Classical and Quantum Correlations}}
%\subtitle{Naturwissenschaftlich-Humanisticher Tag am BG 19\\Weltbild und Wissenschaft\\http://tph.tuwien.ac.at/\~{}svozil/publ/2005-BG18-pres.pdf}
\subtitle{\textcolor{orange!60}{\small http://tph.tuwien.ac.at/$\sim$svozil/publ/2011-qc-pres.pdf\\
http://arxiv.org/abs/quant-ph/0503229v4}}
\author{Karl Svozil}
\institute{Institut f\"ur Theoretische Physik, University of Technology Vienna, \\
Wiedner Hauptstra\ss e 8-10/136, A-1040 Vienna, Austria\\
svozil@tuwien.ac.at
%{\tiny Disclaimer: Die hier vertretenen Meinungen des Autors verstehen sich als Diskussionsbeitr�ge und decken sich nicht notwendigerweise mit den Positionen der Technischen Universit�t Wien oder deren Vertreter.}
}
\date{StatPhysI, June 1st, 2011}
\maketitle

\frame{

\centerline{\Large Part I: }
\centerline{ }
\centerline{\Large  \color{blue} Setup of two-particle  correlations}

\begin{center}
{\color{orange}
$\widetilde{\qquad \qquad }$
$\widetilde{\qquad \qquad}$
$\widetilde{\qquad \qquad }$
}
\end{center}
}

\begin{frame}[fragile]
\frametitle{Two-particle  correlations}

\begin{verbatim}
@ARTICLE{peres222,
  author = {Asher Peres},
  title = {Unperformed experiments have no results},
  journal = {American Journal of Physics},
  year = {1978},
  volume = {46},
  pages = {745-747},
  doi = {10.1119/1.11393},
  url = {http://dx.doi.org/10.1119/1.11393}
}
\end{verbatim}


\end{frame}


\begin{frame}[fragile]
\frametitle{Frequency definition of two-particle correlations}

Consider two particles or quanta. On each one of the two quanta, certain measurements
(such as the spin state or polarization) of
(dichotomic) observables
$O({ a})$ and
$O({ b})$
along the directions $a$ and $b$, respectively, are performed.
The individual outcomes are
encoded or labeled by the
values ``$-\lambda$'' and ``$+\lambda$;'' e.g., ``$-1$'' and ``$+1$'' (or, alternatively, by the symbols``$-$'' and  ``$+$,'' or ``0'' and ``1'') are recorded along
the directions ${ a}$ for the first particle, and  ${ b}$ for the second particle, respectively.

A two-particle correlation function $E(a,b )$
is defined by averaging over the product of the outcomes $O({ a})_i, O({ b} )_i\in \{-\lambda,+\lambda\}$
in the $i$th experiment for a total of $N$ experiments; i.e.,
$$
E(a,b )={1\over N}\sum_{i=1}^N O({ a})_i O({ b})_i.
$$


\end{frame}

\begin{frame}[fragile]
\frametitle{Two-particle correlations}

\begin{figure}
\begin{center}
%TeXCAD Picture [1.pic]. Options:
%\grade{\off}
%\emlines{\off}
%\epic{\on}
%\beziermacro{\on}
%\reduce{\on}
%\snapping{\off}
%\quality{2.000}
%\graddiff{0.010}
%\snapasp{1}
%\zoom{9.5137}
\unitlength 0.9mm % = 2.845pt
\thicklines %\linethickness{0.4pt}
\ifx\plotpoint\undefined\newsavebox{\plotpoint}\fi % GNUPLOT compatibility
\begin{picture}(120,25.01)(0,0)
\put(56,9.086){\line(4,3){8}}
\put(64,9.086){\line(-4,3){8}}
\put(5,5.01){\oval(10,10)[l]}
\put(5,.01){\line(0,1){10}}
\put(2.5,5.01){\makebox(0,0)[cc]{$-$}}
\put(5,20.01){\oval(10,10)[l]}
\put(5,15.01){\line(0,1){10}}
\put(2.5,20.01){\makebox(0,0)[cc]{$+$}}
\put(10,5.01){\framebox(10,15)[cc]{$\theta_1,\varphi_1$}}
\put(115,5.01){\oval(10,10)[r]}
\put(115,.01){\line(0,1){10}}
\put(117.5,5.01){\makebox(0,0)[cc]{$-$}}
\put(115,20.01){\oval(10,10)[r]}
\put(115,15.01){\line(0,1){10}}
\put(117.5,20.01){\makebox(0,0)[cc]{$+$}}
\put(100,5.01){\framebox(10,15)[cc]{$\theta_2,\varphi_2$}}
\put(60.019,11.983){\circle{9.727}}
%\vector[middle]{\line}
\put(65.379,12.088){\line(1,0){33.846}}\put(82.302,12.088){\vector(1,0){.07}}
%\end
%\vector[middle]{\line}
\put(54.658,12.088){\line(-1,0){33.846}}\put(37.735,12.088){\vector(-1,0){.07}}
%\end
\end{picture}
\end{center}
\caption{Simultaneous spin state measurement of
the two-partite state.
Boxes indicate spin state analyzers such as Stern-Gerlach apparatus
oriented along the directions $\theta_1,\varphi_1 $ and
$\theta_2,\varphi_2 $;
their two output ports are occupied with detectors  associated
with the outcomes
``$+$''
and
``$-$'',
respectively.
}
\end{figure}


\end{frame}


\begin{frame}[fragile]
\frametitle{Two-particle correlations}


\begin{figure}
\begin{center}
%TeXCAD Picture [1.pic]. Options:
%\grade{\on}
%\emlines{\off}
%\epic{\off}
%\beziermacro{\on}
%\reduce{\on}
%\snapping{\on}
%\pvinsert{% Your \input, \def, etc. here}
%\quality{8.000}
%\graddiff{0.005}
%\snapasp{1}
%\zoom{6.7272}
\unitlength .6mm % = 1.707pt
%\thicklines
\linethickness{0.4pt}
\ifx\plotpoint\undefined\newsavebox{\plotpoint}\fi % GNUPLOT compatibility
\begin{picture}(120,102)(0,0)
%\emline(0,8)(41,30)
\multiput(0,8)(.1045918367,.056122449){392}{\line(1,0){.1045918367}}
%\end
\put(41,29.5){\line(0,1){72.5}}
%\emline(41,102)(0,80)
\multiput(41,102)(-.1045918367,-.056122449){392}{\line(-1,0){.1045918367}}
%\end
\put(0,80.5){\line(0,-1){72.5}}
{\color{blue}
\put(20,51){\vector(1,0){100}}
\put(20,51){\vector(0,1){34}}
\put(20,51){\vector(3,2){20}}
}
{
%\bezvec{618}[middle](45,51)(50,62)(35,61)
\put(45,59){\color{red}\vector(-2,3){.117}}\color{red}\bezier{618}(45,51)(50,62)(35,61)
%\end
%\bezvec{487}[middle](35,61)(28.5,72)(20,71)
\put(28,69){\color{red}\vector(-4,3){.117}}\color{red}\bezier{487}(35,61)(28.5,72)(20,71)
%\end
}
{\color{blue}
\put(20,45){\makebox(0,0)[cc]{$0$}}
\put(106,43){\makebox(0,0)[cc]{$Z$}}
\put(23,84){\makebox(0,0)[cc]{$Y$}}
\put(37,68){\makebox(0,0)[cc]{$X$}}
}
{\color{red}
\put(49,63){\makebox(0,0)[cc]{$\theta$}}
\put(29,75){\makebox(0,0)[cc]{$\varphi$}}
}
\end{picture}
\end{center}
\caption{Coordinate system for measurements of particles travelling along $0Z$}
\end{figure}


\end{frame}


\frame{

\centerline{\Large Part II: }
\centerline{ }
\centerline{\Large  \color{blue} Classical two-particle  quantum correlations}

\begin{center}
{\color{orange}
$\widetilde{\qquad \qquad }$
$\widetilde{\qquad \qquad}$
$\widetilde{\qquad \qquad }$
}
\end{center}
}


\frame{
\frametitle{Two-particle classical correlations}
{\tiny
\begin{figure}
\begin{center}
%
%TeXCAD Picture [2.pic]. Options:
%\grade{\on}
%\emlines{\off}
%\epic{\off}
%\beziermacro{\on}
%\reduce{\on}
%\snapping{\off}
%\quality{8.000}
%\graddiff{0.010}
%\snapasp{1}
%\zoom{5.7082}
\unitlength .2mm % = 1.138pt
%\thicklines
\linethickness{0.4pt}
\ifx\plotpoint\undefined\newsavebox{\plotpoint}\fi % GNUPLOT compatibility
%\begin{picture}(220.345,235.75)(0,0)
\begin{picture}(220.345,70)(0,0)
{\color{blue}
\put(30.25,29.75){\circle{61.53}}
%
\put(30.00,68.5){\makebox(0,0)[cc]{$a$}}
\put(30.25,30.25){\line(0,1){30.5}}
\put(-.091,29.825){\line(1,0){61}}
%\dottedline(1.75,235.75)(2,235.25)
%\multiput(1.574,235.574)(.125,-.25){3}{{\rule{.4pt}{.4pt}}}
%\end
\put(18.89,42.78){\makebox(0,0)[cc]{$+$}}
\put(29.44,12.22){\makebox(0,0)[cc]{$-$}}
}
{\color{red}
\put(109.92,29.75){\circle{61.53}}
%
%\emline(110,30)(128.33,54)
\multiput(110,30)(.084082569,.110091743){218}{\line(0,1){.110091743}}
%\end
%\emline(85.59,48.466)(134.056,11.196)
\multiput(85.59,48.466)(.1096526165,-.0843225288){442}{\line(1,0){.1096526165}}
%\end
\put(133.61,62.94){\makebox(0,0)[cc]{$b$}}
\put(110.56,46.67){\makebox(0,0)[cc]{$-$}}
\put(99.44,17.22){\makebox(0,0)[cc]{$+$}}
}
\put(189.58,29.75){\circle{61.53}}
%
\put(165.08,38){\makebox(0,0)[cc]{$+$}}
\put(182.83,13.5){\makebox(0,0)[cc]{{\color{blue}$-$}$\cdot${\color{red}$+$}$=-$}}
\put(227.5,35.5){\makebox(0,0)[cc]{{\color{blue}$+$}$\cdot${\color{red}$-$}$=-$}}
\put(211.58,21.25){\makebox(0,0)[cc]{$+$}}
{\color{blue}
\put(189.58,30.25){\line(0,1){30.5}}
\put(159.33,30){\line(1,0){61}}
\put(189.58,68.5){\makebox(0,0)[cc]{$a$}}
}

%\emline(189.44,30)(207.78,54)
\multiput(189.44,30)(.08412844,.110091743){218}{\color{red}\line(0,1){.110091743}}
%\end
%\emline(165.125,48.642)(213.591,11.371)
\multiput(165.125,48.642)(.1096526165,-.0843225288){442}{\color{red}\line(1,0){.1096526165}}
%\end

{\color{red}
\put(213.28,62.94){\makebox(0,0)[cc]{$b$}}
}
\put(179.00,34.00){\makebox(0,0)[rc]{$\theta$}}
\put(194.00,40.00){\makebox(0,0)[cc]{$\theta$}}
\put(198.00,26.00){\makebox(0,0)[lc]{$\theta$}}
\bezier{44}(189.44,45)(195,46.11)(198.89,42.22)
\bezier{106}(172.209,29.957)(171.946,35.651)(175.538,40.293)
\bezier{94}(204.794,29.782)(204.444,24.526)(201.29,20.672)
\end{picture}
\end{center}
\end{figure}

}
By considering the length  $A_+(a,b)$ and $A_-(a,b)$ of the positive and negative contributions to expectation function,
one obtains for
$0\le \theta=\vert a-b\vert \le \pi$,
$$
\begin{array}{rcl}
E_{\text{cl},2,2}(\theta )&=&  E_{\text{cl},2,2}(a,b)= \frac{1}{2\pi} \left[A_+(a,b)-A_-(a,b)\right]\\
&& \quad =  \frac{1}{2\pi} \left[2A_+(a,b) -2\pi \right]=
{2\over \pi}\vert a-b\vert - 1 = {2\theta \over \pi} - 1,
\end{array}
$$
where the subscripts stand for the number of mutually exclusive measurement outcomes per particle, and
for the number of particles, respectively.
Note that $A_+(a,b)+A_-(a,b)=2\pi$.


}


\frame{

\centerline{\Large Part III: }
\centerline{ }
\centerline{\Large  \color{blue} Quantum two-particle  quantum correlations}

\begin{center}
{\color{orange}
$\widetilde{\qquad \qquad }$
$\widetilde{\qquad \qquad}$
$\widetilde{\qquad \qquad }$
}
\end{center}
}


\frame{
\frametitle{Definitions}
Let
$
\vert +\rangle
$
denote the pure state corresponding to
$ {\hat {\bf e}}_1 =(0,1)
$,
and
$
\vert -\rangle $ denote the orthogonal pure state
corresponding to
${\hat {\bf e}}_2 =(1,0)
$.
The superscript
``$T$,''
``$\ast$'' and
``$\dagger$'' stand for transposition, complex and hermitian conjugation, respectively.

In finite-dimensional Hilbert space, the matrix representation of projectors $E_{\bf a}$
from normalized vectors ${\bf a}=(a_1,a_2,\ldots ,a_n)^T$ with respect to some basis of $n$-dimensional Hilbert space
is obtained by taking the dyadic product; i.e., by
$$
\begin{array}{l}
E_{\bf a}= \left[{\bf a},{\bf a}^\dagger\right]=\left[{\bf a},({\bf a}^\ast)^T\right]=
{\bf a}\otimes {\bf a}^\dagger =
\left(
\begin{array}{cccccccccc}
a_1{\bf a}^\dagger \\
a_2{\bf a}^\dagger \\
\ldots  \\
a_n{\bf a}^\dagger
\end{array}
\right)
=   \\
\qquad =
\left(
\begin{array}{cccccccccc}
a_1a_1^\ast & a_1a_2^\ast & \ldots & a_1a_n^\ast \\
a_2a_1^\ast & a_2a_2^\ast & \ldots & a_2a_n^\ast \\
\ldots & \ldots & \ldots & \ldots \\
a_na_1^\ast & a_na_2^\ast & \ldots & a_na_n^\ast
\end{array}
\right)
.
\end{array}
$$


%%%%%%%%%%%%%%%%%%%%%%%%%%%%%%%%%%%%%%%%%
}

\frame{
%%%%%%%%%%%%%%%%%%%%%%%%%%%%%%%%%%%%%%%%%

The tensor or Kronecker product of two vectors ${\bf a}$ and ${\bf b} =(b_1,b_2,\ldots ,b_m)^T$ can be represented by
$$
{\bf a} \otimes {\bf b} = (a_1{\bf b},a_2{\bf b},\ldots ,a_n{\bf b})^T = (a_1b_1,a_1b_2,\ldots ,a_nb_m)^T
$$
The tensor or Kronecker product of some operators
$$
A=
\left(
\begin{array}{cccccccccc}
a_{11} & a_{12} & \ldots & a_{1n} \\
a_{21} & a_{22} & \ldots & a_{2n} \\
\ldots & \ldots & \ldots & \ldots \\
a_{n1} & a_{n2} & \ldots & a_{nn}
\end{array}
\right)
\text{ and  }B=
\left(
\begin{array}{cccccccccc}
b_{11} & b_{12} & \ldots & b_{1m} \\
b_{21} & b_{22} & \ldots & b_{2m} \\
\ldots & \ldots & \ldots & \ldots \\
b_{m1} & b_{m2} & \ldots & b_{mm}
\end{array}
\right)
$$
is represented by an $n\times n$-matrix  $A\otimes B
=$
$$
\left(
\begin{array}{cccccccccc}
a_{11} B& a_{12} B& \ldots & a_{1n}B \\
a_{21} B& a_{22} B& \ldots & a_{2n}B \\
\ldots & \ldots & \ldots & \ldots \\
a_{n1} B& a_{n2} B& \ldots & a_{nn}B
\end{array}
\right)
=
\left(
\begin{array}{cccccccccc}
a_{11} b_{11}& a_{11} b_{12} & \ldots & a_{1n}b_{1m} \\
a_{11} b_{21}& a_{11} b_{22}& \ldots & a_{2n} b_{2m}\\
\ldots & \ldots & \ldots & \ldots \\
a_{nn} b_{m1}& a_{nn} b_{m2}& \ldots & a_{nn} b_{mm}
\end{array}
\right)
.
$$


%%%%%%%%%%%%%%%%%%%%%%%%%%%%%%%%%%%%%%%%%
}

\frame{
%%%%%%%%%%%%%%%%%%%%%%%%%%%%%%%%%%%%%%%%%
\frametitle{Observables}

Let us start with the spin one-half angular momentum observables of {\em a single} particle along an arbitrary direction
in spherical co-ordinates $\theta$ and $\varphi$
in units of $\hbar$; i.e.,
$$
M_x=
\frac{1}{2}
\left(
\begin{array}{cccccccccc}
0&1\\
1&0
\end{array}
\right),
\;
M_y=
\frac{1}{2}
\left(
\begin{array}{cccccccccc}
0&-i\\
i&0
\end{array}
\right),
\;
M_z=
\frac{1}{2}
\left(
\begin{array}{cccccccccc}
1&0\\
0&-1
\end{array}
\right).
$$
The angular momentum operator in arbitrary direction
$\theta$, $\varphi$ is given by its spectral decomposition
$$
\begin{array}{lcl}
S_\frac{1}{2} (\theta ,\varphi) =
xM_x
+
yM_y
+
zM_z
 \\
=  M_x  \sin \theta \cos \varphi
+
M_y   \sin \theta \sin \varphi
+
M_z   \cos \theta
\\
=   \frac{1}{2}\sigma (\theta ,\varphi)=
{1\over 2}
\left(\begin{array}{rcl}
\cos \theta &  e^{-i \varphi }\sin \theta \\
e^{i \varphi }\sin \theta & - \cos \theta
\end{array}
\right)\\
=
-
\frac{1}{2}
\left(
\begin{array}{cc}
 \sin ^2 \frac{\theta }{2} & -\frac{1}{2} e^{-i \varphi } \sin \theta  \\
 -\frac{1}{2} e^{i \varphi } \sin \theta  & \cos ^2\frac{\theta  }{2}
\end{array}
\right)
+
\frac{1}{2}
 \left(
\begin{array}{cc}
 \cos ^2 \frac{\theta }{2} & \frac{1}{2} e^{-i \varphi } \sin \theta  \\
 \frac{1}{2} e^{i \varphi } \sin \theta  & \sin ^2 \frac{\theta }{2}
\end{array}
\right)\\
=
-
\frac{1}{2}
\left\{
\frac{1}{2}
\left[
{\Bbb I}_2 - \sigma (\theta ,\varphi)
\right]
\right\}
+
\frac{1}{2}
\left\{
\frac{1}{2}
\left[
{\Bbb I}_2 + \sigma (\theta ,\varphi)
\right]
\right\}
.
\end{array}
$$

%%%%%%%%%%%%%%%%%%%%%%%%%%%%%%%%%%%%%%%%%
}

\frame{
%%%%%%%%%%%%%%%%%%%%%%%%%%%%%%%%%%%%%%%%%
The  orthonormal eigenstates (eigenvectors)  associated with the eigenvalues $-\frac{1}{2}$ and $+\frac{1}{2}$ of
$S_\frac{1}{2}(\theta , \varphi )$
are
$$
\begin{array}{cccc}
\vert -\rangle_{\theta ,\varphi} \equiv {\bf x}_{-\frac{1}{2}}(\theta ,\varphi)&=e^{i\delta_{+}}& \left(-
e^{-\frac{i\varphi}{2}} \sin{\theta \over 2} ,e^{\frac{i\varphi}{2}}  \cos{\theta \over 2}
\right),\\
\vert +\rangle_{\theta ,\varphi} \equiv {\bf x}_{+\frac{1}{2}}(\theta ,\varphi)&=e^{i\delta_{-}}& \left(
e^{-\frac{i\varphi}{2}} \cos{\theta \over 2}, e^{\frac{i\varphi}{2}}\sin{\theta \over 2}
\right) ,
\end{array}
$$
respectively. $\delta_{+}$ and $\delta_{-}$ are arbitrary phases.
These orthogonal unit vectors correspond to the two orthogonal projectors
$$
F_\mp (\theta ,\varphi ) =
\frac{1}{2}
\left[
{\Bbb I}_2 \mp \sigma (\theta ,\varphi)
\right]
$$
for the spin down and up states along $\theta $ and $\varphi$, respectively.
By setting all the phases and angles to zero, one obtains the original
orthonormalized basis $\{\vert -\rangle,\vert +\rangle\}$.

%%%%%%%%%%%%%%%%%%%%%%%%%%%%%%%%%%%%%%%%%
}

\frame{
%%%%%%%%%%%%%%%%%%%%%%%%%%%%%%%%%%%%%%%%%




If we are only interested in spin state measurements with the associated
outcomes of spin states in units of $\hbar$,
the previous formula can be rewritten to include all possible cases at once; i.e.,
$$
 S_{\frac{1}{2} \frac{1}{2} } ({\hat \theta},{\hat \varphi} ) =
S_{\frac{1}{2} }( \theta_1,\varphi_1 )
\otimes
S_{\frac{1}{2} }( \theta_2,\varphi_2 ).
$$





The two-particle projectors
$F_{\pm \pm }$ or, by another notation, $F_{\pm_1 \pm_2 }$
to indicate the outcome on the first or the second particle,
corresponding to a two~spin-${1\over 2}$~particle joint measurement
aligned (``$+$'') or antialigned  (``$-$'') along arbitrary directions are
$$
 F_{\pm_1 \pm_2 } ({\hat \theta},{\hat \varphi} ) =
{\frac{1}{2}}\left[{\mathbb I}_2 \pm_1 {\bf \sigma}( \theta_1,\varphi_1 )\right]
\otimes
{\frac{1}{2}}\left[{\mathbb I}_2 \pm_2 {\bf \sigma}( \theta_2,\varphi_2 )\right];
$$
where ``$\pm_i$,'' $i=1,2$ refers to the outcome on the $i$'th particle,
and the notation ${\hat \theta},{\hat \varphi}$ is used to indicate all angular parameters.

%%%%%%%%%%%%%%%%%%%%%%%%%%%%%%%%%%%%%%%%%
}

\frame{
%%%%%%%%%%%%%%%%%%%%%%%%%%%%%%%%%%%%%%%%%


To demonstrate its physical interpretation, let us consider as a concrete example
a spin state measurement on two quanta:
$F_{- +  } ({\hat \theta},{\hat \varphi} )$ stands for the proposition
$\;$\\
$\;$\\
\begin{quote}
{\em `The spin state of the first particle measured along $\theta_1,\varphi_1$ is ``$-$''
      and
      the spin state of the second particle measured along $\theta_2,\varphi_2$ is ``$+$''~.'
}
\end{quote}

%%%%%%%%%%%%%%%%%%%%%%%%%%%%%%%%%%%%%%%%%
}

\frame{
%%%%%%%%%%%%%%%%%%%%%%%%%%%%%%%%%%%%%%%%%

More generally, we will consider two different numbers
$\lambda_+$ and $\lambda_-$,
and the generalized single-particle operator
$$
R_{\frac{1}{2}} (\theta ,\varphi) =
\lambda_-
\left\{
\frac{1}{2}
\left[
{\Bbb I}_2 - \sigma (\theta ,\varphi)
\right]
\right\}
+
\lambda_+
\left\{
\frac{1}{2}
\left[
{\Bbb I}_2 + \sigma (\theta ,\varphi)
\right]
\right\}
,
$$
as well as the resulting two-particle operator
$$
\begin{array}{l}
R_{\frac{1}{2} \frac{1}{2}} ({\hat \theta},{\hat \varphi} ) =
R_{\frac{1}{2}}( \theta_1,\varphi_1 )
\otimes
R_{\frac{1}{2}} ( \theta_2,\varphi_2 )\\
=
\lambda_- \lambda_- F_{--} +
\lambda_- \lambda_+ F_{-+} +
\lambda_+ \lambda_- F_{+-} +
\lambda_+ \lambda_+ F_{++}
.
\end{array}
$$

%%%%%%%%%%%%%%%%%%%%%%%%%%%%%%%%%%%%%%%%%
}

\frame{
%%%%%%%%%%%%%%%%%%%%%%%%%%%%%%%%%%%%%%%%%

\frametitle{Singlet state}




In what follows, singlet states $\vert \Psi_{d,n,i} \rangle$ will be labeled by three numbers $d$, $n$ and $i$,
denoting
the number $d$ of outcomes associated with the dimension of Hilbert space per particle,
the number $n$ of participating particles,
and the state count $i$ in an enumeration of all possible singlet states of $n$ particles of spin $j=(d-1)/2$, respectively.
For $n=2$, there is only one singlet state, and $i=1$ is always one.

%%%%%%%%%%%%%%%%%%%%%%%%%%%%%%%%%%%%%%%%%
}

\frame{
%%%%%%%%%%%%%%%%%%%%%%%%%%%%%%%%%%%%%%%%%

Consider the {\em singlet} ``Bell'' state of two spin-${1\over 2}$
particles
$$
\vert \Psi_{2,2,1} \rangle
=
 {1\over \sqrt{2}}
\bigl(
\vert +- \rangle -
\vert -+ \rangle
\bigr)
.
$$

With the identifications
$
\vert +\rangle
\equiv {\hat {\bf e}}_1 =(1,0)
$
and
$
\vert -\rangle \equiv {\hat {\bf e}}_2 =(0,1)
$ as before,
the Bell state has a vector representation as
$$
\begin{array}{l}
\vert  \Psi_{2,2,1}\rangle
 \equiv
{1\over \sqrt{2}}\left({\hat {\bf e}}_1\otimes {\hat {\bf e}}_2-{\hat {\bf e}}_2\otimes {\hat {\bf e}}_1 \right)\\
= {1\over \sqrt{2}}\left[ (1,0)\otimes (0,1) - (0,1) \otimes (1,0)\right]\\
=\left( 0,\frac{1}{\sqrt{2}},- \frac{1}{\sqrt{2}} ,  0 \right).
\end{array}
$$
%%%%%%%%%%%%%%%%%%%%%%%%%%%%%%%%%%%%%%%%%
}

\frame{
%%%%%%%%%%%%%%%%%%%%%%%%%%%%%%%%%%%%%%%%%
\frametitle{Density operator}
The density operator $\rho_{\Psi_{2,2,1}}$
is just the projector of the dyadic product of this vector, corresponding to the one-dimensional
linear subspace spanned by  $\vert  \Psi_{2,2,1}\rangle $; i.e.,
$$
\begin{array}{lll}
\rho_{\Psi_{2,2,1}} = \vert  \Psi_{2,2,1}\rangle \langle  \Psi_{2,2,1} \vert    \\
=
\left[ \vert  \Psi_{2,2,1}\rangle ,\vert  \Psi_{2,2,1}\rangle^\dagger \right]    \\
=
\frac{1}{2}
 \left(
\begin{array}{rrrr}
0&0&0&0\\
0&1&-1&0\\
0&-1&1&0\\
0&0&0&0
\end{array}
\right)
.
\end{array}
$$

%%%%%%%%%%%%%%%%%%%%%%%%%%%%%%%%%%%%%%%%%
}

\frame{
%%%%%%%%%%%%%%%%%%%%%%%%%%%%%%%%%%%%%%%%%
\frametitle{Form invariance of singlet states}

Singlet states are form invariant with respect to arbitrary unitary
transformations in the single-particle Hilbert spaces and thus
also rotationally invariant in configuration space,
in particular under the rotations
$$
\vert + \rangle =
e^{ i{\frac{\varphi}{2}} }
\left(
\cos \frac{\theta}{2} \vert +'  \rangle
-
\sin \frac{\theta}{2} \vert -'   \rangle
\right)
$$
and
$$
\vert - \rangle =
e^{ -i{\frac{\varphi}{2}} }
\left(
\sin \frac{\theta}{2} \vert +'   \rangle
+
\cos \frac{\theta}{2} \vert -'  \rangle
\right)
$$
in the spherical coordinates $\theta , \varphi$ defined above.

%%%%%%%%%%%%%%%%%%%%%%%%%%%%%%%%%%%%%%%%%
}

\frame{
%%%%%%%%%%%%%%%%%%%%%%%%%%%%%%%%%%%%%%%%%

The Bell singlet state is unique in the sense that the outcome of a spin state measurement
along a particular direction on one particle ``fixes'' also the opposite outcome of a spin state measurement
along {\em the same} direction on its ``partner'' particle: (assuming lossless devices)
\begin{itemize}
\item<+->
whatever the common direction of spin (intrinsic angular momentum) state measurement,
\item<+->
whenever a ``plus'' or a ``minus'' is recorded on one side,
\end{itemize}
a ``minus'' or a ``plus'' is recorded on the other side, and {\it vice versa.}


%%%%%%%%%%%%%%%%%%%%%%%%%%%%%%%%%%%%%%%%%
}

\frame{
%%%%%%%%%%%%%%%%%%%%%%%%%%%%%%%%%%%%%%%%%

\frametitle{Results}

We now turn to the calculation of quantum predictions.
The joint probability to register the spins of the two particles
in state $\rho_{\Psi_{2,2,1}}$
aligned or antialigned along the directions defined by
($\theta_1$, $\varphi_1 $) and
($\theta_2$, $\varphi_2 $)
can be evaluated by the {\color{orange} Born formula}
$$
\begin{array}{l}
P_{{ \Psi_{2,2,1}}\,\pm_1 \pm_2 } ({\hat \theta},{\hat \varphi} )\\
=
{\rm Tr}\left[\rho_{ \Psi_{2,2,1}} \cdot F_{\pm_1 \pm_2 } \left({\hat \theta},{\hat \varphi} \right)\right] \\
=\frac{1}{4} \left\{ 1-(\pm_1 1)( \pm_2 1) \left[\cos \theta_1 \cos \theta_2 + \sin \theta_1 \sin \theta_2 \cos (\varphi_1-\varphi_2) \right]\right\}
.
\end{array}
$$
Again, ``$\pm_i$,'' $i=1,2$ refers to the outcome on the $i$'th particle.

%%%%%%%%%%%%%%%%%%%%%%%%%%%%%%%%%%%%%%%%%
}

\frame{
%%%%%%%%%%%%%%%%%%%%%%%%%%%%%%%%%%%%%%%%%

Since $P_= + P_{\neq} = 1$ and $E= P_= - P_{\neq}$, the joint probabilities to find the two particles
in an even or in an odd number of
spin-``$-\frac{1}{2}$''-states when measured along
($\theta_1$, $\varphi_1 $) and
($\theta_2$, $\varphi_2 $)
are in terms of the expectation function given by
$$
\begin{array}{l}
P_= = P_{++}+P_{--} \\
= {1\over2}\left(1 + E  \right)\\
=\frac{1}{2} \left\{ 1- \left[\cos \theta_1 \cos \theta_2 - \sin \theta_1 \sin \theta_2 \cos (\varphi_1-\varphi_2) \right]\right\}
,
\end{array}
$$
$$
\begin{array}{l}
\\
P_{\neq} = P_{+-}+P_{-+}\\
 = {1\over2}\left(1 - E \right)\\
=\frac{1}{2} \left\{ 1+ \left[\cos \theta_1 \cos \theta_2 + \sin \theta_1 \sin \theta_2 \cos (\varphi_1-\varphi_2) \right]\right\}
.
\end{array}
$$
%%%%%%%%%%%%%%%%%%%%%%%%%%%%%%%%%%%%%%%%%
}

\frame{
%%%%%%%%%%%%%%%%%%%%%%%%%%%%%%%%%%%%%%%%%
Finally, the quantum mechanical expectation function is obtained by  the difference $P_= -P_{\neq }$; i.e.,
$$
\begin{array}{l}
E_{{ \Psi_{2,2,1}}\,-1,+1  }(\theta_1,\theta_2,\varphi_1 , \varphi_2)\\
=
-\left[\cos \theta_1 \cos \theta_2 + \cos (\varphi_1 - \varphi_2) \sin \theta_1 \sin \theta_2\right]
.
\end{array}
$$



By setting either the azimuthal angle differences equal to zero,
or by assuming measurements in the plane perpendicular to the direction of particle propagation,
i.e., with $\theta_1=\theta_2 =\frac{\pi}{2}$,
one obtains
$$
\begin{array}{rcl}
E_{{ \Psi_{2,2,1}}\,-1,+1  }(\theta_1,\theta_2)&=& -\cos (\theta_1 - \theta_2),\\
E_{{ \Psi_{2,2,1}}\,-1,+1  }(\frac{\pi}{2},\frac{\pi}{2},\varphi_1 , \varphi_2) &=& - \cos (\varphi_1 - \varphi_2).
\end{array}
$$

%%%%%%%%%%%%%%%%%%%%%%%%%%%%%%%%%%%%%%%%%
}

\frame{
%%%%%%%%%%%%%%%%%%%%%%%%%%%%%%%%%%%%%%%%%



A more ``natural'' choice of $\lambda_\pm$ would be in terms of the spin state observables in units of $\hbar$;
i.e., $  \lambda_+ = -  \lambda_- =\frac{1}{2}$.
The expectation function of  these observables can be directly calculated {\it via} $S_{\frac{1}{2}}$; i.e.,
$$
\begin{array}{l}
E_{{ \Psi_{2,2,1}}\,-\frac{1}{2},+ \frac{1}{2} } ({\hat \theta},{\hat \varphi} )\\
=
{\rm Tr}\left\{ \rho_{ \Psi_{2,2,1}} \cdot \left[ S_{\frac{1}{2}}(\theta_1,\varphi_1) \otimes S_{\frac{1}{2}}(\theta_2,\varphi_2)\right]\right\} \\
=
\frac{1}{4} \left[\cos
    \theta_1  \cos  \theta_2 +\cos ( \varphi_1 - \varphi_2 ) \sin \theta_1  \sin  \theta_2 \right]\\
= \frac{1}{4}E_{{ \Psi_{2,2,1}}\,-1 ,+1 } ({\hat \theta},{\hat \varphi} )
.
\end{array}
$$
}


\begin{frame}[fragile]
\frametitle{Plot of classical and quantum ``singlet'' two-particle correlations:
more different clicks between $(0,\pi /2)$, and more equals between $(\pi /2,\pi)$~!}

\begin{center}
\includegraphics[width=60mm]{2011-qc-kl}
\end{center}


\end{frame}


%%%%%%%%%%%%%%%%%%%%%%%%%%%%%%%%%%%%%%%%%%%%%%%%%%%%%%%%%%%%%%%%%%%%%%%%%%%%%%%%%%%%%%%%%%%%%%%%%%%%%%%%%%%%%%%%%%%%%%%%%%%%%%%%%%%%%%%%%%%%%%%

\frame{

\centerline{\Large Part IV: }
\centerline{ }
\centerline{\Large  \color{blue} Boole's ``conditions of physical existence''}
\centerline{\Large  \color{blue} -- aka Bell-type inequalities}

\begin{center}
{\color{orange}
$\widetilde{\qquad \qquad }$
$\widetilde{\qquad \qquad}$
$\widetilde{\qquad \qquad }$
}
\end{center}
}

\begin{frame}[fragile]
\frametitle{Would you believe?}

\begin{itemize}

\item<+->
Proposition \#1 ($P1$): ``It rains in Vienna, Austria, with probability 0.1.''

\item<+->
Proposition \#2 ($P2$): ``It rains in Auckland, New Zealand, with probability 0.1.''

\item<+->
Proposition \#3 (joint \# 1 and \# 2, $P12$): ``It simultaneously rains in Auckland as well as in Vienna with probability 0.9.''

\item<+->   {\color{orange} Exactly when would you believe?}
{ \color{blue}  --
Boole's {\em Laws of Thought} (1958), and {\em On the Theory of Probabilities} (1863)}

\end{itemize}

\end{frame}

\begin{frame}[fragile]
\frametitle{Truth table}

Suppose \#1 and \#2 are independent, then the joint probability is just the product of the single probabilities:


\begin{center}
\begin{tabular}{cccc}
\hline\hline
two-valued&$P1$ & $P2$ & $P12=P1 \cdot P2$\\
probability measure  \\
interpreted as vector\\
\hline
$p_1$&(0,&0,&0)\\
$p_2$&(0,&1,&0)\\
$p_3$&(1,&0,&0)\\
$p_4$&(1,&1,&1)\\
\hline\hline
\end{tabular}
\end{center}

All possible classical (joint) probabilities  can be represented by the following {\em  \color{orange} correlation polytope}:
$$
\begin{array}{l}
\Large\{ (x,y,z) \mid (x,y,z)= \lambda_1  p_1  +   \lambda_2  p_2  +  \lambda_3  p_3  +  \lambda_4  p_4 ;\\
  \qquad \qquad
  \qquad
 {\rm with}\;
   \lambda_1 + \lambda_2 + \lambda_3 + \lambda_4 =1; \lambda_1, \ldots ,\lambda_4 \in {\Bbb R}^+ \cup \{0\} \Large\}
\end{array}
$$



\end{frame}

\begin{frame}[fragile]
\frametitle{Bell-type inequalities represented by half-spaces (faces) of the correlation polytope}

{\color{orange} Weyl-Minkowski representation theorem: a convex polytope can either be represented by its vertices, or by the inequalities characterizing its half-spaces
(and bounded by its ``faces''). }

The problem to find the polytope faces is NP-complete [Pitowski, 1991, http://dx.doi.org/10.1007/BF01594946]; yet for a small number of vertices it is tractable.

\tiny
\begin{verbatim}
URL http://www.ifor.math.ethz.ch/~fukuda/cdd_home/index.html:
\end{verbatim}

\color{orange}

\begin{verbatim}
V-representation
begin
   4  4  integer
1  0  0  0
1  0  1  0
1  1  0  0
1  1  1  1
end
hull
\end{verbatim}
\end{frame}

\begin{frame}[fragile]
\frametitle{Facet inequalities -- the hull problem}

\tiny
\color{blue}

\begin{verbatim}
* cdd+: Double Description Method in C++:Version 0.76a1 (June 8, 1999)
* Copyright (C) 1999, Komei Fukuda, fukuda@ifor.math.ethz.ch
* Compiled for Floating-Point Arithmetic
*Input File:2011-VIEAKL.ext(4x4)
*HyperplaneOrder: LexMin
*Degeneracy preknowledge for computation: None (possible degeneracy)
*Hull computation is chosen.
*Zero tolerance = 1e-06
*Computation starts     at Tue May 31 12:22:44 2011
*            terminates at Tue May 31 12:22:44 2011
*Total processor time = 0 seconds
*                     = 0h 0m 0s
*Since hull computation is chosen, the output is a minimal inequality system
*FINAL RESULT:
*Number of Facets = 4
H-representation
begin
4  4  real
 1 -1 -1 1
 0 1 0 -1
 0 0 1 -1
 0 0 0 1
end
\end{verbatim}
\end{frame}

\begin{frame}[fragile]
\frametitle{Facet inequalities -- the hull problem cntd.}


\begin{center}
\begin{tabular}{cccccc}
\hline\hline
\#& inequality\\
$i_1$:&$ 1 P1 +1 P2 -1 P12  \le  1 $    &  $\rightarrow$ & $P1 +  P2 -  P12  \le  1$     \\
$i_2$:&$-1 P1 +0 P2 +1 P12  \le  0 $    &  $\rightarrow$ &     $ P1 \ge   P12 $              \\
$i_3$:&$ 0 P1 -1 P2 +1 P12  \le  0 $     &  $\rightarrow$ &    $   P2   \ge  P12 $             \\
$i_4$:&$ 0 P1 +0 P2 -1 P12  \le   0$     &  $\rightarrow$ &    $  P12  \ge   0$                    \\
\hline\hline
\end{tabular}
\end{center}

$i_1,\ldots, i_4$ render  conditions on classical probabilities; thus
you could believe $P12$ if an only if it claims that ``It simultaneously rains in Auckland as well as in Vienna with probability less than 0.1 ($i_2$ and $i_3$).''

[[Other claim:
``It rains in Auckland  with probability  0.9.''
``It rains in Vienna  with probability  0.7.''
``It simultaneously rains in Auckland as well as in Vienna with probability greater than 0.6 ($i_1$) but less than 0.7 ($i_3$).'']]



\end{frame}

\begin{frame}[fragile]
\frametitle{Clauser-Horne-Shimony-Holt (CHSH) inequality}


Four  observables (e.g., polarization measurements on photons, spin state measurements on electrons) --
two observables on  ``Alice's'' and ``Bob's'' side: $A1,A2, B1,B2$

{\tiny

\setlength{\tabcolsep}{1pt}
\begin{center}
\begin{tabular}{cccccccccccccccccc}
\hline\hline
two-valued&$E(A1)$ & $E(A2)$ & $E(B1)$& $E(B2)$& $E(A1,B1)$& $E(A2,B1)$& $E(A1,B2)$& $E(A2,B2)$\\
expectations  \\
\hline
$p_1$&(-1,&-1,&-1,&-1,&+1,&+1,&+1,&+1)\\
$p_2$&(-1,&-1,&-1,&+1,&+1,&+1,&-1,&-1)\\
$p_3$&(-1,&-1,&+1,&-1,&-1,&-1,&+1,&+1)\\
$p_4$&(-1,&-1,&+1,&+1,&-1,&-1,&-1,&-1)\\
$\cdot$& $\cdot$ &$\cdot$ &$\cdot$ &$\cdot$ &$\cdot$ &$\cdot$ &$\cdot$ &$\cdot$ \\
$\cdot$& $\cdot$ &$\cdot$ &$\cdot$ &$\cdot$ &$\cdot$ &$\cdot$ &$\cdot$ &$\cdot$ \\
$p_{16}$&(+1,&+1,&+1,&+1,&+1,&+1,&+1,&+1)\\
\hline\hline
\end{tabular}
\end{center}
\setlength{\tabcolsep}{6pt}
}

Solving the hull problem for this configuration yields some type of
``new'' nontrivial (CHSH) inequalities for the joint expectation values (in the case of equidistributed $E(A1)=$ $E(A2)=$ $E(B3)=E(B4)=0$):
$$-2\le E(A1,B1)+E(A1,B2)+E(A2,B1)-E(A2,B2) \le 2$$
\end{frame}

\begin{frame}[fragile]
\frametitle{Tsirelson bound for CHSH}

For
$\angle (A1)= {\pi \over 2}$,
$\angle (A2)= {0}$,
$\angle (B1)= {\pi \over 4}$,
$\angle (B2)= {3\pi \over 4}$,
and with the quantum correlations $E(Ai,Bj) = -\cos[\angle (Aj) - \angle (Bj)]$,
$$\begin{array}{l}
\vert
E(A1,B1)+E(A1,B2)+E(A2,B1)-E(A2,B2)
\vert \\
\qquad \qquad \qquad \qquad \qquad \qquad = 4\cos(\pi /4) =2\sqrt{2},
\end{array}$$
which represents a (without proof: maximal) {\color{orange} violation} of Boole's conditions of classical experience!

Note 1: The classical expectation function
$E(\theta ) = {2\theta / \pi} - 1
$ could never violate CHSH.)

Note 2: due to quantum complementarity measurement of each one of the four joint expectations entails a separate measurement (``breakfast-lunch-tea-dinner'').

Note 3: quantum mechanics does not violate the CHSH bounds maximally; i.e., by the algebraic maximum of 4.

\end{frame}

\begin{frame}[fragile]
\frametitle{How bad could it get quantum mechanically?}

{\color{orange} Kochen-Specker theorem:}    from three mutually exclusive outcomes onward (3-dim. Hilbert space),
and for certain finite configurations of quantum observables,
there does not exist any two-valued measure or (simultaneous) truth table.

That is very bad, as classically two-valued measures are used to derive the set of all probabilities.

{\color{orange} Gleason's theorem:}   from three mutually exclusive outcomes onward (3-dim. Hilbert space),    Born's rule
for quantum probabilities and expectations can be derived by assuming classical probabilities on contexts (``maximal sets of commeasurable observables'' equivalent to maximal operators).


\end{frame}

\begin{frame}[fragile]
\frametitle{Diagrammatic proof of the Kochen-Specker theorem}

\begin{figure}
\begin{center}
%TeXCAD Picture [1.pic]. Options:
%\grade{\on}
%\emlines{\off}
%\epic{\off}
%\beziermacro{\on}
%\reduce{\on}
%\snapping{\off}
%\quality{8.000}
%\graddiff{0.010}
%\snapasp{1}
%\zoom{5.6569}
\unitlength .5mm % = 1.423pt
\thicklines %\linethickness{0.8pt}
\ifx\plotpoint\undefined\newsavebox{\plotpoint}\fi % GNUPLOT compatibility
\begin{picture}(134.09,125.99)(0,0)

%\emline(86.39,101.96)(111.39,58.46)
\multiput(86.39,101.96)(.119617225,-.208133971){209}{{\color{green}\line(0,-1){.208133971}}}
%\end
%\emline(86.39,14.96)(111.39,58.46)
\multiput(86.39,14.96)(.119617225,.208133971){209}{{\color{red}\line(0,1){.208133971}}}
%\end
%\emline(36.47,101.96)(11.47,58.46)
\multiput(36.47,101.96)(-.119617225,-.208133971){209}{{\color{yellow}\line(0,-1){.208133971}}}
%\end
%\emline(36.47,14.96)(11.47,58.46)
\multiput(36.47,14.96)(-.119617225,.208133971){209}{{\color{magenta}\line(0,1){.208133971}}}
%\end
\color{blue}\put(86.39,15.21){\color{blue}\line(-1,0){50}}
\put(86.39,101.71){\color{violet}\line(-1,0){50}}
%
\put(36.34,15.16){\color{magenta}\circle{6}}
\put(36.34,15.16){\color{blue}\circle{4}}
\put(52.99,15.16){\color{blue}\circle{4}}
\put(52.99,15.16){\color{cyan}\circle{6}}
\put(69.68,15.16){\color{blue}\circle{4}}
\put(69.68,15.16){\color{orange}\circle{6}}
\put(86.28,15.16){\color{blue}\circle{4}}
\put(86.28,15.16){\color{red}\circle{6}}
%
\put(93.53,27.71){\color{red}\circle{4}}
\put(93.53,27.71){\color{orange}\circle{6}}
\put(102.37,43.44){\color{red}\circle{4}}
\put(102.37,43.44){\color{olive}\circle{6}}
\put(111.21,58.45){\color{red}\circle{4}}
\color{green}\put(111.21,58.45){\circle{6}}
%
\put(102.37,73.47){\color{green}\circle{4}}
\put(102.37,73.47){\color{olive}\circle{6}}
\put(93.53,89.21){\color{green}\circle{4}}
\put(93.53,89.21){\color{cyan}\circle{6}}
\put(86.28,101.76){\color{green}\circle{4}}
\put(86.28,101.76){\color{violet}\circle{6}}
%
\put(69.68,101.76){\color{violet}\circle{4}}
\put(69.68,101.76){\color{cyan}\circle{6}}
\put(52.99,101.76){\color{violet}\circle{4}}
\put(52.99,101.76){\color{orange}\circle{6}}
\put(36.34,101.76){\color{violet}\circle{4}}
\put(36.34,101.76){\color{yellow}\circle{6}}
%
\put(29.24,89.21){\color{yellow}\circle{4}}
\put(29.24,89.21){\color{orange}\circle{6}}
\put(20.4,73.47){\color{yellow}\circle{4}}
\put(20.4,73.47){\color{olive}\circle{6}}
\put(11.56,58.45){\color{yellow}\circle{4}}
\put(11.56,58.45){\color{magenta}\circle{6}}

\put(20.4,43.44){\color{magenta}\circle{4}}
\put(20.4,43.44){\color{olive}\circle{6}}
\put(29.24,27.71){\color{magenta}\circle{4}}
\put(29.24,27.71){\color{cyan}\circle{6}}

\color{cyan}
\qbezier(29.2,27.73)(23.55,-5.86)(52.99,15.24)
\qbezier(29.2,27.88)(36.93,75)(69.63,101.91)
\qbezier(52.69,15.24)(87.47,40.96)(93.72,89.27)
\qbezier(93.72,89.27)(98.4,125.99)(69.49,102.06)
\color{orange}
\qbezier(93.57,27.73)(99.22,-5.86)(69.78,15.24)
\qbezier(93.57,27.88)(85.84,75)(53.13,101.91)
\qbezier(70.08,15.24)(35.3,40.96)(29.05,89.27)
\qbezier(29.05,89.27)(24.37,125.99)(53.28,102.06)
\color{olive}
\qbezier(20.15,73.72)(-11.67,58.52)(20.15,43.31)
\qbezier(20.33,73.72)(61.34,93.16)(102.36,73.72)
\qbezier(102.36,73.72)(134.09,58.52)(102.53,43.31)
\qbezier(102.53,43.31)(60.99,23.43)(20.15,43.49)
{\color{black}
\put(30.41,114.02){\makebox(0,0)[cc]{$M$}}
\put(30.41,2.65){\makebox(0,0)[cc]{$A$}}
\put(52.68,114.38){\makebox(0,0)[cc]{$L$}}
\put(52.68,2.3){\makebox(0,0)[cc]{$B$}}
\put(91.93,114.2){\makebox(0,0)[cc]{$J$}}
\put(91.93,2.48){\makebox(0,0)[cc]{$D$}}
\put(69.65,114.38){\makebox(0,0)[cc]{$K$}}
\put(73.65,2.3){\makebox(0,0)[cc]{$C$}}
\put(103.24,94.22){\makebox(0,0)[cc]{$I$}}
\put(17.45,94.22){\makebox(0,0)[cc]{$ N$}}
\put(106.24,22.45){\makebox(0,0)[cc]{$E$}}
\put(17.45,22.45){\makebox(0,0)[cc]{$ R$}}
\put(115.13,77.96){\makebox(0,0)[cc]{$H$}}
\put(8.55,77.96){\makebox(0,0)[cc]{$ O$}}
\put(115.13,38.72){\makebox(0,0)[cc]{$F$}}
\put(10.55,38.72){\makebox(0,0)[cc]{$ Q$}}
\put(120.92,57.98){\makebox(0,0)[l]{$ G$}}
\put(1.77,57.98){\makebox(0,0)[rc]{$  P$}}
}
\put(61.341,9.192){\color{blue}\makebox(0,0)[cc]{$a$}}
\put(102.883,35.355){\color{red}\makebox(0,0)[cc]{$b$}}
\put(102.53,84.322){\color{green}\makebox(0,0)[cc]{$c$}}
\put(60.457,108.01){\color{violet}\makebox(0,0)[cc]{$d$}}
\put(18.031,84.145){\color{yellow}\makebox(0,0)[cc]{$e$}}
\put(18.561,33.057){\color{magenta}\makebox(0,0)[cc]{$f$}}
\put(61.341,39.774){\color{olive}\makebox(0,0)[cc]{$g$}}
\put(72.124,67.882){\color{orange}\makebox(0,0)[cc]{$h$}}
\put(48.79,67.705){\color{cyan}\makebox(0,0)[cc]{$i$}}
\end{picture}
\end{center}
\caption{Greechie diagram of a finite subset of the continuum of blocks or contexts embeddable in
four-dimensional real Hilbert space without a two-valued probability measure [Cabello, 1996, URL http://dx.doi.org/10.1016/0375-9601(96)00134-X].
}
\end{figure}


\end{frame}

\begin{frame}[fragile]
\frametitle{Diagrammatic proof of the Kochen-Specker theorem cntd.}
{\small
The proof of the Kochen-Specker theorem  uses  nine tightly interconnected contexts
$\color{blue}a=\{A,B,C,D\}$,
$\color{red}b=\{D,E,F,G\}$,
$\color{green}c=\{G,H,I,J\}$,
$\color{violet}d=\{J,K,L,M\}$,
$\color{yellow}e=\{M,N,O,P\}$,
$\color{magenta}f=\{P,Q,R,A\}$,
$\color{orange}g=\{B,I,K,R\}$,
$\color{olive}h=\{C,E,L,N\}$,
$\color{cyan}i=\{F,H,O,Q\}$
consisting of the 18 projectors associated with the one dimensional subspaces spanned by
$ A=(0,0,1,-1)    $,
$ B=(1,-1,0,0)    $,
$ C=(1,1,-1,-1)   $,
$ D=(1,1,1,1)     $,
$  E=(1,-1,1,-1)  $,
$  F=(1,0,-1,0)   $,
$  G=(0,1,0,-1)   $,
$  H=(1,0,1,0)    $,
$  I=(1,1,-1,1)   $,
$ J=(-1,1,1,1)    $,
$ K=(1,1,1,-1)    $,
$ L=(1,0,0,1)     $,
$ M=(0,1,-1,0)    $,
$  N=(0,1,1,0)    $,
$  O=(0,0,0,1)    $,
$  P=(1,0,0,0)    $,
$  Q=(0,1,0,0)    $,
$  R=(0,0,1,1)    $.

Greechie diagram representing atoms by points, and  contexts by maximal smooth, unbroken curves.
Every observable proposition occurs in exactly two contexts.
Thus, in an enumeration of the four observable propositions of each of the nine contexts,
there appears to be an {\em \color{red} even} number of true propositions.
Yet, as there is an odd number of contexts,
there should be an {\em \color{green} odd} number (actually nine) of true propositions.
}

\end{frame}


\end{document}

Plot[{2 tt/Pi - 1, -Cos[tt]}, {tt, 0, Pi},
 PlotStyle -> {{Hue[0.65, 1, 1], Dashing[{0.04, 0.02}],
    Thickness[0.01]}, {Red, Thickness[0.01]}, {Hue[0, 1, 1],
    Dashing[{0.04, 0.02, 0.005, 0.02}], Thickness[0.01]}, {Green,
    Thickness[0.01], Dashing[{0.02, 0.02}]}, {Orange, Thickness[0.01],
     Dashing[{0.01, 0.01}]}, {Magenta, Thickness[0.01]}},
 PlotRange -> {-1.03, 1.03}, PlotRange -> {-1.03, 1.03},
 Frame -> True,
 FrameLabel -> {Style["\[Theta] [rad]", FontSize -> 26],
   Style["E(\[Theta])", FontSize -> 26]}, AspectRatio -> 1,
 BaseStyle -> {FontFamily -> "Times", FontSize -> 24},
 FrameTicks -> {{{0.0001, "0"}, \[Pi]/4, \[Pi]/2,
    3 \[Pi]/4, \[Pi]}, {-1, -1/2, {0.0001, "0"}, 1/2, 1}, None, None}]

Export["c:/mytex/2011-qc-kl.pdf", %, "PDF", ImageSize -> 500]
