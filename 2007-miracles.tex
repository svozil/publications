\documentclass[rmp,amsfonts,showpacs,showkeys,twocolumn]{revtex4}
%\documentclass[pra,showpacs,showkeys,amsfonts]{revtex4}
\usepackage{graphicx}% Include figure files
\usepackage{dcolumn}% Align table columns on decimal point
\usepackage{bm}% bold math
\usepackage{eepic}
\usepackage{xcolor}
\bibpunct{[}{]}{,}{a}{}{;}

\RequirePackage{times}
%\RequirePackage{courier}
\RequirePackage{mathptm}




\begin{document}
%\sloppy

\title{Physical unknowables\footnote{Contribution to
the international symposium ``Horizons of Truth''
celebrating the 100th birthday of Kurt G\"odel
at  the University of Vienna from April 27th-29th, 2006}}

\author{Karl Svozil}
\email{svozil@tuwien.ac.at}
\homepage{http://tph.tuwien.ac.at/~svozil}
\affiliation{Institut f\"ur Theoretische Physik, University of Technology Vienna,
Wiedner Hauptstra\ss e 8-10/136, A-1040 Vienna, Austria}


\begin{abstract}
Different types of physical unknowables are discussed.
Provable unknowables are derived from reduction to problems which are known to be recursively unsolvable.
Recent series solutions to the $n$-body problem and related to it, chaotic systems, may have no computable radius of convergence.
Quantum unknowables include the random occurrence of single events, complementarity and value indefiniteness.
\end{abstract}


\pacs{01.70.+w,01.65.+g,02.30.Lt,03.65.Ta}
\keywords{Unknowables, Church-Turing thesis, induction and forecast, n-body problem, quantum indeterminism}

\maketitle

\tableofcontents


\begin{quote}
\begin{flushright}
{\footnotesize
ei mihi, qui nescio saltem quid nesciam!\\
$[$Alas for me, that I do not at least know the extent of my own ignorance!$]$  \\
{\em Aurelius Augustinus, 354--430, ``Confessiones (Confessions),'' (Book XI, Chapter 25)}
%http://www.stoa.org/hippo/text11.html
%http://de.wikipedia.org/wiki/Confessiones
%http://en.wikipedia.org/wiki/Confessiones
%http://www.ccel.org/a/augustine/confessions/confessions.html
%http://www.ccel.org/ccel/augustine/confessions.xiv.html
%http://www.newadvent.org/fathers/110111.htm
\\ $\;$
\\ $\;$
$\ldots$ as we know, there are known knowns; \\
there are things we know we know. \\
We also know there are known unknowns; \\
that is to say we know there are some things we do not know. \\
But there are also unknown unknowns --\\
the ones we don't know we don't know.  \\
{\em United States Secretary of Defense Donald H. Rumsfeld \\
at a Department of Defense news briefing on February 12, 2002}
% http://www.defenselink.mil/Transcripts/Transcript.aspx?TranscriptID=2636
 }
\end{flushright}
\end{quote}

\section{Toward explanation and feasibility}

Throughout history,
the human desire to foresee and manipulate the physical world according to personal wishes and  fantasies
has been confronted with the inability to predict and manipulate
large portions of the habitat.
As time passed, people have figured out various ways to tune
ever increasing fragments of the world according to their needs.
From a purely behavioral perspective, this is brought about in the way of
pragmatic quasi-causal conditional rules of the following kind,
``if we do this, we obtain that.''
A typical example of such a rule is, ``if I rub my hands, they get warmer.''


How do we arrive at those kinds of rules?
Guided by our suspicions, thoughts, formalisms and by pure chance,
we ``fiddle'' and ``roam around,'' inspecting portions of our world
and examining their behavior.
We observe repeating patterns of behavior and pin them down by reproducing them.
A physical behavior is anything that can be observed and thus operationally obtained and measured;
e.g.,
the rise and fall of the sun, the ignition of fire, the formation and the melting of ice.
Note that, due to the finiteness of the resolution, all kinds of physical behaviors,
even the ones that appear continuous, can be discretized.
Ultimately, all physical experiences can be broken down into yes-no propositions
representable by zeroes and ones, by sequences of single clicks in detectors.


As we observe physical behaviors,
we attempt to ``understand'' them by trying to figure out
the ``cause''~\cite{frank} or ``reason'' for the behavioral  patterns.
That is, we invent virtual parallel worlds of thoughts
and intellectual concepts such as ``electric field'' or ``mechanical force''
to ``explain''  and manipulate the behavioral patterns.
We call these creations of our minds ``physical theories.''
Contemporary physical theories are heavily formalized and spelled out in the language of mathematics.
A ``good theory'' provides us with the feeling of a ``key'' unlocking new ways of world comprehension and manipulation.
Ideally, an ``explanation'' should be as compact as possible
and should apply to as many behavioral patterns as possible.

The methods we employ are pretty reliable.
Reliability yields a feeling of security and consolation.
It strengthens the belief in the applicability and the overall validity of the method.

As we are able to manipulate more and more fragments
of our habitat, some of us get the feeling that we are converging to some ``final truth.''
Ultimately, we seek theories of everything~\cite{barrow-TOE}
predicting and manipulating the phenomena at large.

In the extreme form, we envision ourselves as becoming empowered with omniscience and omni-influence:
we presume that our ability to manipulate and tune the world is limited by our own fantasies alone,
and any constraints whatsoever can be bypassed or overcome one way or another.
Indeed, some of what in the past has been called magic, mystery and the beyond has been realized in everyday life.
Many wonders of witchcraft have been transferred into the realm of the physical sciences.
Take, for example, our abilities to fly,
the capability to transmute mercury into gold~\cite{PhysRev.60.473},
to listen and speak to far away friends,
or to cure bacterial diseases with a few pills of antibiotics.

Thereby, we not only trust the rules merely syntactically in the operational sense,
but we often take for granted the semantic significance of the physical theories that ``let us understand'' the
behavioral patterns and even lead us to novel predictions of behaviors.
Stated pointedly, we not only accept physical theories as pure abstractions and constructions of our own mind,
but we associate meaning and truth to them.
We grant absolute status to our own constructions of mind,
purporting that they somehow are metaphysically real and eternal;
so much so that only very reluctantly we admit their preliminary character.

Alas, the possibility to formulate theories per se;
and in particular the applicability of formal, mathematical models, comes as a ``mindboggling'' surprise.
There appears to be an unreasonable effectiveness of mathematics in the natural sciences~\cite{wigner}
which seems difficult to explain within science proper.
It is not too unreasonable to speculate that any such reasoning might be metaphysical.



\section{Provable physical unknowables}

In the past century, unknowability has been formally defined and derived along the notion of unprovability,
accompanied by a precise meaning of provability~\cite{rogers1,odi:89,odi:99}.
In formal logic~\cite{godel1} and the foundations of mathematics~\cite{tarski:32,tarski:56}
as well as theoretical computer sciences~\cite{turing-36,chaitin3},
unprovability has been established as a proper concept.
Those theoretical frameworks proved strong enough to derive some of their own limitations;
among them their incompleteness and the impossibility to determine their consistency.

This is a remarkable departure from informal suspicions and observations regarding the limitations
of our worldview.
No longer is one reduced to informal, heuristic contemplations and comparisons about what one knows or can do versus
one's unpredictability and incapability.
Formal unknowability is about formal proofs of unpredictability and impossibility.

Almost since its discovery, attempts~\cite{popper-50i,popper-50ii} have been made to translate
formal incompleteness into the physical science,
mostly by reduction to the halting problem~\cite{moore,casti:94-onlimits_book,casti:96-onlimits}.
Here reduction means that physical undecidability is linked or reduced to logical undecidability.
A typical example for such a reduction is the embedding of a Turing machine or any type of computer capable of
universal computation into a physical system.
As a consequence, the physical system inherits
any type of unsolvability derivable for universal computers, such as the
unsolvability of the halting problem:
since the computer is part of the physical system, so are its behavioral patterns.


A clear distinction should be made from the onset regarding two different types of unknowables in
the natural sciences: unknowables about physical systems and their phenomena and behaviors on one hand,
and unknowables of the formal theoretical descriptions and models on the other hand.
This section will mainly be concerned with the first type of physical unknowability;
the one which is associated with deterministic physical systems.





\subsection{Intrinsic self-referential observers}



Every physical observation is essentially (i) discrete, (ii) finite and (iii) self-referential.
There is a definite distinction between finite and discrete, in particular in physics.
Discrete systems may still be unbounded. Examples are linear chain models, or quantization.
In the latter bosonic case, there can only be a discrete number of particles per field mode,
but this number is bounded by physical constraints such as the finiteness of energy.
On the other hand, finite systems such as the classical electric field in any bounded volume
of space-time may be continuous.
Whereas finiteness and discreteness have been briefly mentioned earlier,
self-referentiality is a seldom recognized system science aspect of physical world perception.

Let us start with the assumption that there exist observers measuring objects, and that
observer and object are distinct from one another.
That is, there exists a ``cut'' between observer and object.
Through that cut, information is exchanged.

If we insist on idealized one-way observation,
information is transferred from the object to the observer via the cut.
%, as sketched in Fig.~\ref{2007-miracles-fc}a).
In this scenario, the object is a transmitter,
and the observer is the receiver.

Symbolically, we may regard the object as an agent contained in a ``black box,''
whose only relevant emanations are representable by finite strings of zeroes and ones
appearing on the cut, which can be modeled by any kind of screen or display.
In this purely syntactic point of view,
a physical theory should be able to render identical symbols like the ones appearing through the cut.
That is, a physical theory should be able to mimic or emulate the black box it purports to apply to.
This view is often adapted in quantum mechanics,
where the question regarding any ``meaning'' (e.g., Ref.~\cite[p. 129]{feynman-law}, see below) of the quantum formalism is notorious.


A sharp distinction between a physical object and an extrinsic,
outside observer is a rarely affordable abstraction.
Examples are astronomy, blackbody radiation and classical physical configurations
allowing an infinitely small (relative to the entire system) subsystem to convey the information transfer.

We are mostly interested in another scenario, in which the observer is part of the system to be observed.
In such a case,
the measurement process is modeled symmetrically,
and information is exchanged between observer and object bidirectionally.

The symmetrical configuration makes a distinction between observer and object purely conventional.
The cut is constituted by the information exchanged.
We tend to associate with the ``measurement apparatus''
one of the two subsystems which in comparison is ``larger'' and ``more classical''
and up-linked with some conscious observer.
The rest of the system we call the ``measured object.''


Intrinsic observers face all kinds of self-referential situations.
Among the most interesting are paradoxical self-referential statements.
These have been known
both informally as puzzling amusement and artistic perplexity,
as well as a formalized scientifically valuable resource.
There is an English phrase stating that one should not bite the hand that feeds oneself.
In German, the saying amounts to the advice not to cut the very tree branch one is sitting on.
The liar paradox is already mentioned in the Bible's Epistle to Titus, 1:12 stating that,
``one of Crete's own prophets has said it: `Cretans are always liars, evil brutes, lazy gluttons.'
He has surely told the truth.''

In what follows, paradoxical self-referentiality will be used to argue
against the solvability of the general induction problem,
as well as for a pandemonium of undecidabilities related to physical systems
and their behaviors. All of them are based on intrinsic observers embedded
in the system they observe.

It is not totally unreasonable to speculate that the
limits of ``intrinsic self-expression'' seems to be
what G\"odel himself
considered the gist of his incompleteness theorems.
In a reply to a letter by Burks
(reprinted in Ref.~\cite[p. 55]{v-neumann-66}; see also Ref.~\cite[p.554]{fef-84}),
G\"odel states,
 \begin{quote}
 {\em
 ``$\ldots$ that a complete epistemological description
 of a language $A$ cannot be given in the same language $A$, because
 the concept of truth of sentences of $A$ cannot be defined in $A$. It
 is this theorem which is the true reason for the existence of
 undecidable propositions in the formal systems containing arithmetic.''}
 \end{quote}







\subsection{What is an acceptable form of proof?}

There exist ancient and informal notions of proof.
An example~\cite{baats1} is the Babylonian notion to ``prove'' arithmetical statements
by considering ``large number'' cases
of algebraic formulae such as~\cite[Chapter V]{neugeb},
$\sum_i^n i^2 = (1/3)(1+2n)\sum_i^n i$.
Another example  is knowledge acquired by revelation or by authority.
Oracles occur in modern computer science,
but only as idealized concepts whose physical realization is highly
questionable if not forbidden.

The contemporary notion of proof is formalized and algorithmic.
Around 1930 mathematicians could still hope for a
``mathematical theory of everything''
which consists of a finite number of axioms and algorithmic derivation rules
by which all true mathematical statements could formally be derived.
In particular, as expressed in Hilbert's 2nd problem,
it should be possible to prove the consistency of the axioms of arithmetic.

Shortly afterwards, G\"odel~\cite{godel1}, Tarski~\cite{tarski:32}, and Turing
\cite{turing-36} put an end to this formalist program.
They first formalized the concepts of proof and computation in general,
equating them with algorithmic content.
Then, they translated self-referential statements of the
kind mentioned above into the formalism.


From a purely syntactic point of view,
every formal system of mathematics  can be identified with a computation
and {\it vice versa}.
Indeed, as stated by K. G\"odel in a {\sl Postscript,} dated from  June 3rd, 1964~\cite[pp. 369-370]{godel-ges1},
 \begin{quote}
 {\em
 $\ldots $ due to A. M. Turing's work,
 a precise and unquestionably
 adequate definition of the general concept of formal system can now be
 given, the existence of undecidable arithmetical propositions and the
 non-demonstrability of the consistency of a system in the same system
 can now be proved rigorously for {\em every} consistent formal system
 containing a certain amount of finitary number theory.

 Turing's work gives an analysis of the
 concept of ``mechanical
 procedure'' (alias ``algorithm'' or ``computation procedure'' or
 ``finite combinatorial procedure''). This concept is shown to be
 equivalent with that of a ``Turing machine.'' A formal system can
 simply be defined to be any mechanical procedure for producing
 formulas, called provable formulas.}
 \end{quote}

What is an algorithm? In Turing's own words~\cite{turing-36},
 \begin{quote}
{\em
``whatever  can (in principle) be calculated on a
 sheet of paper by  the usual rules is  computable.''}
\end{quote}

These constraints limit the expressiveness of any formalism,
for either it is too restricted to allow the representation of rich patterns of behavior,
or it is bounded by self-referentiality.
They, however, do not exclude revelations
and knowledge of truth transcending the algorithmic formalism~\cite{kreisel-80}.



\subsection{Undecidability of the general forecasting problem}


Logical, mathematical and algorithmic undecidabilities are based on intrinsic paradoxical self-reference.
Can we make use of paradoxical self-reference in physics?
Is it possible to find physical expressions corresponding to,
for instance, the liar paradox?
Can we apply the ``G\"odelian program'' to physics?

Indeed, we can argue that for any deterministic system strong enough to support
universal computation,  the general forecast or prediction
problem is provable unsolvable.
This will be shown by reduction to the halting problem.

G\"odel himself had doubts about the relevance of formal incompleteness to physics,
in particular to quantum mechanics.
The author was told by professor Wheeler that this resentment
(also mentioned in Ref.~\cite[pp. 140-141]{bernstein})
may have been due to Einstein's negative opinion of quantum theory;
to the extend that Einstein may have ``brainwashed'' G\"odel
into believing that all efforts in this direction were in vain.

One of the first researchers getting interested in the application
of paradoxical self-reference to physics
was the philosopher Popper,
who published two almost forgotten papers
\cite{popper-50i,popper-50ii}
discussing, among other issues, Russell's Paradox of
Tristram Shandy~\cite{sterne}:
In Volume 1, Chapter XIV, Shandy finds that he could publish
two volumes of his life every year,
covering a time span far smaller than the time it took him to write
these volumes. This de-synchronization, Shandy concedes,
will rather increase than diminish as he advances; and one may thus have serious doubts
whether he will ever complete his autobiography.

More recently, there have been attempts to bring together researchers
interested in the relevance of G\"odelian incompleteness in physics.
One of those meetings took place in Santa Fe
\cite{casti:94-onlimits_book}, another one in Abisko
\cite{casti:96-onlimits}.

A straightforward embedding of a universal computer
into a physical system results in the fact that,
due to the reduction to the recursive undecidability of the halting problem,
certain future events cannot be forecasted
and are thus provable indeterministic.
Here reduction again means that physical undecidability is linked or reduced
to logical undecidability.

For the sake of getting an (algorithmic) taste
of what paradoxical self-reference is like,
we present the sketch of an algorithmic proof (by contradiction)
of the unsolvability of the halting problem.
Consider a universal computer $U$ and an arbitrary algorithm
$B(X)$ whose input is a string of symbols $X$.  Assume that there exists
a ``halting algorithm'' ${\tt HALT}$ which is able to decide whether $B$
terminates on $X$ or not.
The domain of ${\tt HALT}$  is the set of legal programs.
The range of ${\tt HALT}$ are classical bits.

Using ${\tt HALT}(B(X))$ we shall construct another deterministic
computing agent $A$, which has as input any effective program $B$ and
which proceeds as follows:  Upon reading the program $B$ as input, $A$
makes a copy of it.  This can be readily achieved, since the program $B$
is presented to $A$ in some encoded form
$\ulcorner B\urcorner $,
i.e., as a string of
symbols.  In the next step, the agent uses the code
$\ulcorner B\urcorner $
 as input
string for $B$ itself; i.e., $A$ forms  $B(\ulcorner B\urcorner )$,
henceforth denoted by
$B(B)$.  The agent now hands $B(B)$ over to its subroutine ${\tt HALT}$.
Then, $A$ proceeds as follows:  if ${\tt HALT}(B(B))$ decides that
$B(B)$ halts, then the agent $A$ does not halt; this can for instance be
realized by an infinite {\tt DO}-loop; if ${\tt HALT}(B(B))$ decides
that $B(B)$ does {\em not} halt, then $A$ halts.

The agent $A$ will now be confronted with the following paradoxical
task:  take the own code as input and proceed to determine whether or not it halts.
Then, whenever $A(A)$
halts, ${\tt HALT}(A(A))$, by the definition of $A$, would force $A(A)$ not to halt.
Conversely,
whenever $A(A)$ does not halt, then ${\tt HALT}(A(A))$ would steer
$A(A)$ into the halting mode.  In both cases one arrives at a complete
contradiction.  Classically, this contradiction can only be consistently
avoided by assuming the nonexistence of $A$ and, since the only
nontrivial feature of $A$ is the use of the peculiar halting algorithm
${\tt HALT}$, the impossibility of any such halting algorithm.


A universal computer can in principle be embedded into or realized by
physical systems~\cite{moore}.
An example for such a physical system is the computer
on which I am currently typing this manuscript.
It follows by reduction that there exist physical observables,
in particular forecasts about whether or not such computer will ever
halt in the sense sketched above,
which are provable undecidable.




\subsection{The busy beaver function as the maximal recurrence time}

The busy beaver function~\cite{rado,chaitin-ACM,dewdney,brady}
addresses the following
question: given a finite system;
i.e., a system whose algorithmic description is of finite length.
What is the biggest number producible by such a system before halting?

Let $\Sigma (n)$ denote the busy beaver function of $n$.
 Originally, T. Rado~\cite{rado}
 asked how
 many $1$'s a Turing machine with $n$ possible states and an empty
 input tape
 could print on that tape before halting.
 The first values of the Turing busy beaver function $\Sigma _T(x)$
 are known or estimated by~\cite{dewdney,brady}:
  $\Sigma _T(1)=1$,
 $\Sigma _T(2)= 4$,
  $\Sigma _T(3)=6$,
 $\Sigma _T(4)= 13$,
 $\Sigma _T(5) \ge 1915$,
 $\Sigma_T(7)\ge 22961$,
 $\Sigma_T(8)\ge 3\cdot (7\cdot 3^{92}-1)/2$.

Consider a related question: what is the upper bound of running time --- or,
alternatively, recurrence time --- of a program of length $n$ bits before
terminating, or, alternatively, recurring?
An answer to that question confers a feeling of how long we have to
wait for the most time-consuming program of length $n$ bits to
hold. That, of course, is a worst-case scenario. Many programs of
length $n$ bits will have halted long  before the maximal halting time.

We mention without proof~\cite{chaitin-ACM,chaitin-bb}  that
this bound can be represented by the busy beaver function:
${\tt TMAX}(n)=\Sigma (n+O(1))$ is the minimum time at which all
programs of size smaller than or equal to $n$ bits which halt have done so.

Knowledge of ${\tt TMAX}$ would ``solve'' the halting
problem quantitatively.
Because if the maximal halting time would be known
and bounded by any computable function of the program size of $n$ bits,
one would have to wait
just a little bit longer than ${\tt TMAX}(n)$ to make sure
that every program of length $n$ --- also this particular program ---
would have terminated.
Otherwise, the program would run forever.
In this sense, knowledge of ${\tt TMAX}$ is equivalent to  a
perfect predictor.
Since due to the recursive unsolvability of the halting problem the latter one does not exist,
we may expect that ${\tt TMAX}$ cannot be a computable function.
Indeed, for large values of $n$, $\Sigma (n)$
grows faster than any computable function  of $n$.


By reduction we obtain upper bounds for the recurrence of any kind of physical behavior:
for deterministic systems representable by $n$ bits,
the recurrence time grows faster than any computable number
of $n$.
This bound from below for possible behaviors may be interpreted as a qualitative measure
of the impossibility to predict and forecast such behaviors by algorithmic means.


\subsection{Undecidability of the induction problem}

Induction in physics is the inference of general rules
dominating and generating physical behaviors from these behaviors.
For any deterministic system strong enough to support
universal computation, the general induction problem
is provable unsolvable.
Induction is thereby reduced to the unsolvability of
the rule inference problem~\cite{go-67,blum75blum,angluin:83,ad-91,li:92}
of identifying a rule or law reproducing the behavior of a deterministic system
by observing its input/output performance by purely algorithmic means
(not by intuition).

Informally, the algorithmic idea of the proof is to take any sufficiently powerful
rule or method of induction and, by using it, to define some
functional behavior which is not identified by it.
This amounts
to constructing an algorithm which
(passively!)
 ``fakes'' the ``guesser'' by simulating some particular function $\varphi $
until the guesser
pretends to guess this function correctly.
In a second,  diagonalization step, the ``faking'' algorithm then switches to a different
 function $\varphi^\ast  \neq \varphi $, such that the guesser's guesses become incorrect.

%
%More formally, assume two (universal) computers $U$ and $U'$.
%Suppose that the second computer $U'$ executes an arbitrary
%algorithm $p$ unknown to computer $U$, the ``guesser.''
% The task of $U$,
% which is called the rule inference problem,
% is to conjecture the ``law'' or algorithm $p$ by analysing the
% behavior of $U'(p)$.
% The recursive unsolvability of the rule inference problem~\cite{go-67} states that this task cannot be
% performed by any effective computation.
%
%For the sake of contradiction, assume~\cite{li:92}
%that there exists a ``perfect guesser'' $U$ which can identify
%all total recursive functions (wrong).
%Then it is possible to construct a function $\varphi^\ast:{\Bbb N} \rightarrow
%\{0,1\}$, such that the guesses of $U$ are wrong infinitely often,
%thereby contradicting the above assumption.
%
%Define $\varphi^\ast (0)=0$.
%One may construct $\varphi^\ast $ by simulating $U$. Suppose the values
%$\varphi^\ast (0)$, $\varphi^\ast (1)$, $\varphi^\ast (2)$, $\cdots$,
%$\varphi^\ast (n-1)$ have already been constructed. Then, on input $n$,
%simulate $U$, based on the previous series
%$
%\{0, \varphi^\ast (0)\}$,
%$
%\{1, \varphi^\ast (1)\}$,
%$
%\{2, \varphi^\ast (2)\},
%\cdots ,
%\{n-1, \varphi^\ast (n-1)\}$,
% and define
%$\varphi^\ast (n)$ equal to 1 plus the guess of $U$ of
%$\varphi^\ast (n)$ mod 2. In this way, $U$ can never guess
%$\varphi^\ast $ correctly; thereby making an infinite number of mistakes.

One can also interpret this result in terms of the recursive
unsolvability of the halting problem, which in turn is related to the busy beaver function:
there is no recursive bound on the
time the guesser has to wait in order to make sure that his guess is
correct.


\subsection{Results in classical recursion theory with implications for theoretical physics}


The following theorems of recursive (i.e., computable) analysis~\cite{aberth-80,Weihrauch} have some
implications to theoretical physics~\cite{kreisel}.
(i)
There exist recursive monotone bounded sequences of rational numbers
whose limit is no computable number
\cite{Specker49}.
A concrete example of such a number is Chaitin's Omega number~\cite{chaitin3,calude:02,calude-dinneen06},
the ``halting probability'' for a computer (using prefix-free code),
which can be defined by a sequence of rational numbers
with no computable rate of convergence.

(ii)
There exist a recursive real function which has its maximum in the unit interval
at no recursive real number~\cite{Specker57}.
This has implication for the principle of least action.

(iii)
There exists a real number $r$ such that $G(r) = 0$ is recursively undecidable for $G(x)$
in a class of functions which involves polynomials and the sine function
\cite{wang}.
This again has some bearing on  the principle of least action.

(iv)
There exist incomputable solutions of the wave equations for computable initial values
\cite{pr1,bridges1}.

%\end{description}



\section{Behavior of three or more classical bodies}

With regard to planetary and other mechanistic systems without collisions,
an extreme deterministic position
was formulated by Laplace,  stating that~\cite[Chapter II]{laplace-prob}
\begin{quote}
{\em
Present events are connected with preceding ones
by a tie based upon the evident principle that a thing
cannot occur without a cause which produces it. This
axiom, known by the name of the principle of sufficient
reason, extends even to actions which are considered
indifferent $\ldots$


We ought then to regard the present state of the
universe as the effect of its anterior state and as the
cause of the one which is to follow. Given for one
instant an intelligence which could comprehend all the
forces by which nature is animated and the respective
situation of the beings who compose it an intelligence
sufficiently vast to submit these data to analysis it
would embrace in the same formula the movements of
the greatest bodies of the universe and those of the
lightest atom; for it, nothing would be uncertain and
the future, as the past, would be present to its eyes.
}
\end{quote}

In the late 18th hundred,
the issue seemed worthy and pressing enough to establish a prize by
King Oscar II of Sweden, advised by Martin Leffler, who published the following
question formulated by Weierstrass:
\begin{quote}
{\em
Given a system of arbitrarily many mass points that attract each
according to Newton's law, under the assumption that no two points ever collide,
try to find a representation of the coordinates of each point
as a series in a variable that is some known function of time and for
all of whose values the series converges uniformly.
}
\end{quote}
Poincar{\'e}'s original prize--winning contribution contained errors.
The necessary corrections led the author to the conclusion that sometimes small
variations in the initial values could lead to huge variations in the
evolution of a physical system in later times.
In Poincar{\'e}'s own words~\cite[Chapter 4, Sect. II, pp. 56-57]{poincare14}\footnote{
W\"urden wir die Gesetze der Natur und den Zustand des Universums
f\"ur einen gewissen Zeitpunkt genau kennen, so
k\"onnten wir den Zustand dieses Universums f\"ur
irgendeinen sp\"ateren Zeitpunkt genau voraussagen.
Aber
[[~$\ldots$~]]
 es kann der Fall eintreten,
da\ss $\;$ kleine Unterschiede in den Anfangsbedingungen
gro\ss e Unterschiede in den sp\"ateren Erscheinungen bedingen;
ein kleiner Irrtum in den ersteren kann einen au\ss erordentlich gro\ss en
Irrtum f\"ur den letzteren nach sich ziehen.
Die Vorhersage wird unm\"oglich und wir haben eine
``zuf\"allige Erscheinung''.}:
\begin{quote}
{\em
If we would know the laws of Nature and the state of the Universe precisely
for a certain time,
we would be able to predict with certainty
the state of the Universe for any later time.
But
[[~$\ldots$~]]
it can be the case that small differences in the initial values
produce great differences in the later phenomena;
a small error in the former may result in a large error in the latter.
The prediction becomes impossible and we have a ``random phenomenon.''}
\end{quote}

\subsection{Deterministic chaos}

Poincar{\'e}'s recognition of possible instabilities
in $n$-body problems was the first indication of what today is called ``deterministic chaos.''
In chaotic systems it is practically impossible to specify
the initial value precise enough to allow long-term predictions.

A stronger assumption supposes that the initial values are elements of
a continuum, and thus ``almost all'' (of measure one) of them
are not representable by any algorithmically compressible number;
in short, that they are random~\cite{calude:02}.
Classical, deterministic chaos results from ``unfolding'' such a random initial value
drawn from the ``continuum urn'' by a recursive, deterministic function.

A weaker form of deterministic chaos just expresses the fact
that linear deviations of initial values which lie within the measurement precision
result in ``large,'' e.g., exponential, divergences in the future evolution of the system.
For further discussions, the interested reader is referred to
Refs.~\cite{shaw,crutchFaPaShaw,schuster1,bricmont,Crutchfield90}.


\subsection{Convergence of the general solution}

More than one hundred years after its formulation as quoted above,
the $n$--body problem has been solved by Wang~\cite{Wang91,Diacu96,Wang01}.
The $3$--body problem was already solved in 1912~\cite{Sundman12}.
The solutions are given in terms of power series.

Yet, in order to be practically applicable,
the rate of convergence of the series must be computable and even ``reasonably good.''
We might already expect from deterministic chaos
that these series solution have a ``very slow'' convergence.
Even the prediction of behaviors in insignificantly short times
may require the summation of a huge number of terms,
making these series unusable for any practical work
\cite{Diacu96}.

Alas, the complications regarding convergence of the series solutions are far more serious.
Suppose we are able to construct a universal computer based on the $n$--body problem.
This can, for instance, be achieved by ballistic computation, such as the
``Billiard Ball'' model of computation
\cite{fred-tof-82,margolus-02}
which effectively ``embeds'' a universal computer into a system of $n$--bodies~\cite{svozil-2007-cestial}.
It follows by reduction that certain predictions are impossible.

What are the consequences of this reduction for the convergence of the series solutions?
It can be expected that not only do the series converge ``very slowly,''
like in deterministic chaotic systems,
but that in general there does not exist any computable rate of convergence
for the series solutions of particular observables.
This is very similar to Chaitin's Omega number~\cite{chaitin3,calude:02}
representing the halting probability of a universal computer, or the busy beaver function.
The Omega number can be ``enumerated''
by series solutions from ``quasi-algorithms''
computing its very first digits~\cite{calude-dinneen06}.
Yet, due to the incomputable growth of the time
required to determine whether or not certain summation terms corresponding to halting programs possibly contribute,
the series lack any computable rate of convergence.

While it may be possible to evaluate
the state of the $n$ bodies by Wang's power series solution
for any finite time with a computable rate of convergence,
global observables, referring to (recursively) unbounded times, may be incomputable.
Examples of global observables correspond to solutions of certain decision problems,
like for instance the stability of the solar system (we do not claim that this is provable incomputable) or the
halting problem.

This, of course,
stems from the metaphor and robustness of universal computation
and the capacity of the $n$-body problem to implement universality.
It is no particularity and peculiarity of Wang's power series solution.
Indeed, the ``troubles'' reside in the capabilities to implement Peano arithmetic and
universal computation by $n$-body problems.
Because of this capacity, there cannot exist other formalizable methods,
analytic solutions or approximations capable to decide and compute certain decision problems
or observables for the $n$-body problem.

\section{Quantum unknowables}


A third group of physical unknowables arise in the quantum domain.
Throughout its development, and although it has turned out to be a highly successful theory,
quantum mechanics, in particular its interpretation and meaning,
has been received controversially within the community.
Some of its founding fathers, such as Schr\"odinger,
De Broglie and in particular Einstein\footnote{
Recall Einstein's {\it dictum} in a letter to Born, dated December~12th, 1926~\cite[p.~113]{born-69},
``In any case I am convinced that he [[the Old One]] does not throw dice.''
(In German: ``Jedenfalls bin ich {\"{u}}berzeugt, dass der [[Alte]] nicht w{\"{u}}rfelt.'')} had a very critical view on its
validity and considered quantum mechanics a preliminary theory which should give
way to a more complete one~\cite{epr}.
Others, among them Bohr and Heisenberg,
claimed that quantum unknowables will stay with us forever.
Over the years, the latter view seems to have prevailed
\cite{fuchs-peres}; although not totally unchallenged
\cite{jammer:66,jammer1,jammer-92}.
Already Sommerfeld warned his students not to get
into the ``meaning behind'' quantum mechanics,
and, as mentioned by Clauser~\cite{clauser-talkvie},
not long ago scientists working in that field
had to be very careful not to become discredited as ``quacks.''
Richard Feynman~\cite[p. 129]{feynman-law}
once mentioned the
\begin{quote}
{\em ``$\ldots$ perpetual torment that results
from [[the question]], `But how can it be like that?' which
is a reflection of uncontrolled but utterly vain desire to see
[[quantum mechanics]] in terms of an analogy with something familiar
$\ldots$
Do not keep saying to yourself, if you can possibly avoid it,
`But how can it be like that?'
because you will get `down the drain', into a blind alley from which nobody has yet
escaped.''}
\end{quote}

In what follows, we shall discuss three main quantum unknowables:
(i) randomness of single events,
(ii) complementarity, and
(iii) value indefiniteness.

\subsection{Random events}

In 1926, Born postulated that (cf. Ref.~\cite[p.~866]{born-26-1}, English translation in Ref.~\cite[p.~54]{wheeler-Zurek:83})\footnote{
{ ``Vom Standpunkt unserer Quantenmechanik gibt es keine Gr\"o\ss e, die im {\em Einzelfalle} den Effekts eines Sto\ss es
kausal festlegt; aber auch in der Erfahrung haben wir keinen Anhaltspunkt daf\"ur, da\ss~ es innere Eigenschaften
der Atome gibt, die einen bestimmten Sto\ss erfolg bedingen.
Sollen wir hoffen, sp\"ater solche Eigenschaften
[[$\ldots$]] zu entdecken und im Einzelfalle zu bestimmen?
Oder sollen wir glauben, dass die \"Ubereinstimmung von Theorie und Erfahrung
in der Unf\"ahigkeit, Bedingungen f\"ur den kausalen Ablauf anzugeben, eine pr\"astabilisierte Harmonie ist,
die auf der Nichtexistenz solcher Bedingungen beruht?
Ich selber neige dazu,die Determiniertheit in der atomaren Welt aufzugeben.''
}
}
\begin{quote}
{  ``From the standpoint of our quantum mechanics, there is no quantity
which in any individual case causally fixes the consequence of the collision;
but also experimentally we have so far no reason to believe that there are some inner properties of the atom
which condition a definite outcome for the collision.
Ought we to hope later to discover such properties [[$\ldots$]]  and determine them in individual cases?
Or ought we to  believe that the agreement of theory and experiment --- as to the impossibility
of prescribing conditions? I myself am inclined  to give up determinism in the world of atoms.''
}
\end{quote}
Furthermore, Born simultaneously claimed that individual particles behave indeterministically
as well as acknowledged a deterministic evolution of the wave function
(cf. Ref.~\cite[p.~804]{born-26-2}, English translation in Ref.~\cite[p.~302]{jammer:89})\footnote{
{  ``Die Bewegung der Partikel folgt Wahrscheinlichkeitsgesetzen,
die Wahrscheinlichkeit selbst aber breitet sich im Einklang mit dem Kausalgesetz  aus.
[Das hei\ss t, da\ss~ die Kenntnis des Zustandes in allen Punkten in einem Augenblick
die Verteilung des Zustandes zu allen sp{\"a}teren Zeiten festlegt.]''
}
}
\begin{quote}
{  ``The motion of particles conforms to the laws of probability, but the probability itself
is propagated in accordance with the law of causality.
[This means that knowledge of a state in all points in a given time determines the distribution of
the state at all later times.]''
}
\end{quote}

Born did not specify the formal notion of ``indeterminism'' he was relating to.
So far, no mathematical characterization of quantum randomness has been proven~\cite{2008-cal-svo}.
In the absence of any indication to the contrary, it is mostly implicitly assumed
that quantum randomness is of the strongest possible type;
which amounts to postulating that the associated sequences are algorithmically incompressible.


In general, the quantum formalism does not predict the outcome of single events
when there is a mismatch between the context in which a state was prepared,
and the context in which it is measured.
Here, the term ``context''~\cite{svozil-2006-omni,svozil-2008-ql}
denotes a maximal collection of commeasurable observables,
or, more technically,
the maximal operator from which all commuting operators can be functionally derived
\cite[\S 84]{halmos-vs}.
Ideally, a quantized system can be prepared
to yield exactly one answer in exactly one context~\cite{zeil-99,DonSvo01,svozil-2002-statepart-prl}.
Other outcomes associated with other contexts occur indeterministically~\cite{2008-cal-svo}.

In the absence of other explanations, one is thus lead to the conclusion
that these single events occur without any causation
and thus at random.
Such random ``quantum coin tosses''~\cite{svozil-qct,rarity-94,zeilinger:qct,stefanov-2000,0256-307X-21-10-027,wang:056107,fiorentino:032334,svozil-2009-howto}
have been used for various purposes, among them delayed choice experiments
\cite{wjswz-98,zeilinger:qct}.
Commercial interface cards~\cite{Quantis} perform at a rate of 4 to 16 Mbit/s.

Note that randomness of this type~\cite{Cris04,calude-dinneen05}
is postulated rather than proven.
This is necessarily so, for any claim of randomness can only be corroborated
with respect to a more or less large class of laws or behaviors;
it is impossible to inspect the hypothesis against an infinity of conceivable laws.
To rephrase a statement about computability~\cite[p. 11]{davis-58}, how can we ever exclude the possibility of our
presented, some day (perhaps by some extraterrestrial visitors), with a (perhaps
extremely complex) device  that ``computes'' and ``predicts''
a certain type of hitherto ``random'' physical behavior?


\subsection{Complementarity}

Another quantum indeterminism is complementarity.
Complementarity is the principal impossibility to measure
two or more complementary observables
with arbitrary precision simultaneously.

In 1933, Pauli gave the first explicit definition of complementarity stating that (cf. \cite[p.~7]{pauli:58},
partial English translation in \cite[p.~369]{jammer:89})\footnote{
{  ``Bei der Unbestimmtheit einer Eigenschaft eines Systems bei einer bestimmten Anordnung
(bei einem bestimmten Zustand eines Systems) vernichtet jeder Versuch, die betreffende Eigenschaft zu messen,
(mindestens teilweise) den Einflu\ss~
der fr{\"u}heren Kenntnisse vom System auf die (eventuell statistischen) Aussagen
{\"u}ber sp{\"a}tere m{\"o}gliche Messungsergebnisse.
[[$\ldots$]]
So m{\"u}ssen, um den Ort eines Teilchens zu bestimmen und um seinen Impuls zu bestimmen,
{\em einander ausschlie\ss ende Versuchsanordnungen benutzt werden.}
[[$\ldots$]]
Die Beeinflussung des Systems durch den Messaparat f{\"u}r den Impuls (Ort)
ist eine solche, da\ss~ innerhalb der durch die Ungenauigkeitsrelationen gegebenen Grenzen
die Benutzbarkeit der fr{\"u}heren Orts- (Impuls-)
Kenntnis f{\"u}r die Voraussagbarkeit der Ergebnisse sp{\"a}terer Orts- (Impuls-) Messungen verlorengegangen ist.
Wenn aus diesem Grunde die Benutzbarkeit {\em eines} klassischen Begriffes in einem
ausschlie\ss enden Verh{\"a}ltnis zu einem {\em anderen} steht, nennen wir diese beiden Begriffe (z.B. Orts- und
Impulskoordinaten eines Teilchens) mit Bohr {\em komplement{\"a}r.}
[[$\ldots$]]
Man wird sehen, dass diese ``Komplementarit{\"a}t'' kein Analogon in
der klassischen Gastheorie besitzt, die ja auch mit statistischen
Gesetzm\"a\ss igkeiten operiert.
Diese Theorie enth{\"a}lt n{\"a}mlich nicht die erst durch die Endlichkeit des Wirkungsquantums
geltend werdende Aussage, da\ss~ durch Messungen an einem System die durch fr{\"u}here Messungen gewonnenen Kenntnisse
{\"u}ber das System unter Umst{\"a}nden notwendig verlorengehen m{\"u}ssen, d.h. nicht mehr verwertet werden k{\"o}nnen.''
}
}
\begin{quote}
{  ``In the case of  an indeterminacy of a property of a system at a certain configuration
(at a certain state of a system), any attempt to measure the respective property (at least partially)
annihilates the influence of the previous knowledge of system on the (possibly statistical) propositions
about possible later measurement results.
[[$\ldots$]]
The impact
on the system by the  measurement apparatus for momentum (position) is such that
within the limits of the uncertainty relations
the value of the knowledge of the previous position (momentum) for the
prediction of later measurements of position and momentum is lost.
If, for this reason, the applicability of {\em one} classical concept stands in the relation of
exclusion to that of {\em another}, we call both of these
concepts (e.g., the position and momentum coordinates of a particle) with Bohr {\em complementary.}
[[$\ldots$]]
One will see that this ``complementarity''
has no analogy in the classical statistical theory of gases,
which also operates with statistical laws.
This theory does not contain the assertion --- which is only valid through the finiteness of the
quantum of action --- that the measurement of a system may necessarily result in a loss
of knowledge acquired through previous measurements; i.e., the previous
measurements can no longer be used.''
}
\end{quote}

Einstein-Podolski-Rosen~\cite{epr} type arguments  utilizing a configuration
of two ``entangled'' particles~\cite{schrodinger,CambridgeJournals:1737068,CambridgeJournals:2027212}
claim to be able to infer two different contexts counterfactually simultaneously, thus circumventing complementarity.
Thereby, one context is measured on one side of the setup, the other context on the other side of it.
By the uniqueness property  of certain two-particle states,
knowledge of a property of one particle entails the certainty
that, if this property were measured on the other particle as well, the outcome of the measurement would be
a unique function of the outcome of the measurement performed.

This makes possible the measurement of one context, {\em as well as} the {\em simultaneous counterfactual inference} of a different context.
Because, one could argue, although one has actually measured on one side a different, incompatible context compared to the context measured on the other side,
if on both sides the same  context {\em would be measured}, the outcomes on both sides {\em would be uniquely correlated}.
Hence measurement of one context per side is sufficient, for the outcome could be counterfactually inferred on the other side.

Complementarity was first encountered in quantum mechanics,
but it is a phenomenon also observable in the classical world.
To get a better intuition for complementarity, we shall consider generalized urn models~\cite{wright,wright:pent} or,
equivalently~\cite{svozil-2001-eua},
finite Moore and Mealy automata~\cite{e-f-moore,schaller-96,dvur-pul-svo,cal-sv-yu}.
Both quasi-classic examples mimic complementarity to the extent that even quasi-quantum cryptography
can be performed with them~\cite{svozil-2005-ln1e}.


A generalized urn model is
characterized by an ensemble of balls with black background color.
Printed on these balls are some color symbols from a symbolic alphabet.
The colors are elements of a set of colors.
A particular ball type is associated with a unique combination of mono-spectrally
(no mixture of wavelength) colored symbols
printed on the black ball background.
Every ball contains just one single symbol per color.

Assume further some mono-spectral filters or eyeglasses which are
``perfect'' by totally absorbing light of all other colors
but a particular single one.
In that way, every color can be associated with a particular eyeglass and vice versa.
%This, of course, is a system science trick related to intrinsic color perception.

When a spectator looks at a particular ball through such an eyeglass,
the only operationally recognizable symbol will be the one in the particular
color which is transmitted through the eyeglass.
All other colors are absorbed, and the symbols printed in them will appear black
and therefore cannot be differentiated from the black background.
Hence the ball appears to carry a different ``message'' or symbol,
depending on the color at which it is viewed.

The difference between the balls and the quanta is the possibility
to view all the different symbols on the balls
in all different colors by taking off the eyeglasses.
Quantum mechanics does not provide us with such a possibility to ``look across the quantum veil.''
On the contrary, there are strong formal arguments suggesting
that the assumption of a simultaneous
physical existence of such complementary observables yields a complete contradiction.
These issues will be discussed next.


\subsection{Value indefiniteness versus omniscience}

Still another quantum unknowable results from the fact that no ``global'' classical truth
assignment exists which is consistent with even a finite number of ``local'' ones.
That is, no consistent classical truth table can be given by pasting together commeasurable observables.
This phenomenon is also known as value indefiniteness or, by an option to interpret this result, contextuality.

Already scholastic philosophy~\cite{specker-60},
for instance Thomas Aquinas  Ref.~\cite{Aquinas},
considered questions such as whether God has knowledge of non-existing things
(Part 1, Question 14, Article 9) or things
that are not yet (Part 1, Question 14, Article 13).
Classical omniscience, at least its naive expression that
``if a proposition is true, then an omniscient agent (such as God) knows that it is true''
is plagued by controversies and paradoxes.

The empirical sciences implement classical omniscience by assuming that
in principle all observables of classical physics are (co-)measurable without any restrictions,
and regardless of whether they are actually measured or not.
No ontological distinction is made between an observable obtained by an ``actual'' and a ``potential'' or ``counterfactual'' measurement.
(In contrast compare Schr\"odinger's own epistemological interpretation of the wave function~\cite[\S  7]{schrodinger} as a
``catalogue of expectations.'')
Precision and commeasurability are limited only by the technical capacities of the experimenter.
The principle of empirical classical omniscience has given rise to the realistic believe that
all observables ``exist'' regardless of their observation; i.e., regardless and independent of
any particular measurement.
Physical (co-)existence is thereby related to the realistic assumption~\cite{stace}
(sometimes referred to as the ``ontic''~\cite{atman:05} viewpoint) that such physical entities exist
even without being experienced by any finite mind.

The formal expression of classical omniscience is the Boolean algebra of observable propositions~\cite{Boole},
and in particular the ``abundance'' of two-valued states interpretable
as omniscience about the system.
Thereby, any such ``dispersionless'' quasi-classical two-valued state --- associated with a ``truth table''
--- can be defined on all observables,
regardless of whether they have been actually observed or not.

Historically, the discovery of the uncertainty principle and complementarity
seem to have been first indications of the decline of classical omniscience.
A formal expression of complementarity is the nondistributive algebra of quantum observables.
Alas,
nondistributivity of the empirical logical structure is no sufficient
condition for the impossibility of omniscience.
The generalized urn
as well as equivalent finite automaton models discussed above
possess two-valued states
interpretable as omniscience.

A further blow to quantum omniscience came from Boole's
``conditions of possible experience''~\cite{Boole-62,Pit-94} for quantum probabilities and
expectation functions.
In particular, Bell was the first to point to experiments which,
based on counterfactually inferred elements of physical reality
discussed by Einstein, Podolsky and Rosen~\cite{epr},
seemed to indicate the impossibility
to faithfully embed quantum observables into classical Boolean algebras.
To state the issues pointedly, under some (presumably mild) side assumptions,
``unperformed experiments have no results''
\cite{peres222} --- there cannot exist a table enumerating all
actual and hypothetical experimental
outcomes consistent with the observed quantum frequencies~\cite{zeilinger-epr-98}.
Any such table could be interpreted as omniscience with respect to
the observables in the Bell-type experiments.
The impossibility to construct such tables appears to be a very serious indication against
omniscience in the quantum domain.

The reason for the impossibility to describe all quantum observables
simultaneously by classical tables of experimental outcomes
can be understood in terms of a ``stronger'' result stating that,
for quantum systems whose Hilbert space is of dimension greater than two,
there does not exist any ``dispersionless'' quasi-classical, two-valued state
interpretable as truth table.
This result, which is known as the Kochen-Specker theorem
\cite{specker-60,kochen1,ZirlSchl-65,Alda,Alda2,kamber64,kamber65,svozil-tkadlec,cabello-96},
has a finitistic proof by contradiction.
Proofs of the Kochen-Specker theorem  amount to  brain teasers in graph coloring:
show that no coloring of certain sets of points (representing quantum observables) in Kochen-Specker type
diagrams exists which include only a single red point
per smooth, unbroken curve; all other points should be colored green.


The violations of conditions of possible classical experience or
the Kochen-Specker theorem do not exclude realism restricted to a single context,
but realistic omniscience beyond it.
It is not totally unreasonable to suspect that the assumption of \mbox{(pre-)determined} observables ``outside''
of a single context is unjustified and anachronistic. The current mainstream interpretation
of the Kochen-Specker theorem is in terms of quantum contextuality; i.e., through a dependence of the outcome
of a single observable on what other observables are actually measured, or could have been in principle measured, alongside of it.


\section{Miracles due to gaps in causal description}

A different issue, discussed by Philipp Frank~\cite{frank},
is the possible occurrence of miracles in the presence of gaps of physical determinism.
One might perceive singular events occurring
within the bounds of classical and quantum physics without any apparent cause as miracles.
For, if there is no cause to an event,
why should such an event occur altogether rather than not occur?

Although such thoughts remain highly speculative, miracles,
if they exist,
could be the basis for an ``operator-directed'' evolution in otherwise deterministic physical systems.
Since in this scenario
the actions of the operator as well as the operating agent itself are ``located outside'' of the physical domain,
this presents a viable option for a transcendental influence in an otherwise deterministic universe~\cite{svozil-nat-acad}.
Similar models have also been applied to dualistic models of the mind~\cite{eccles:papal,popper-eccles}.

There exist bounds on miracles and on behavioral patterns in general due to the self-referential
perception of intrinsic observers endowed with free will:
if such an observer is omniscient and has absolute predictive power,
then free will could counteract omniscience, and in particular the own predictions.
The only consistent alternative seems either to abandon free will,
stating that it is an idealistic illusion,
or to accept that omniscience and absolute predictive power is bound by paradoxical self-reference.

\section{Concluding thoughts}



Hilbert's sixth problem is the axiomatization of physics.
We still do not know whether or not this goal is achievable.
All we know is that even if it could be achieved, omniscience cannot be gained
via the formalized, syntactic route to infer and predict physical behaviors.
It will remain blocked forever by paradoxical self-reference
which intrinsic observers and operational methods are bound to.


The postulate of indeterministic behavior in physics or elsewhere is impossible to prove by
considering a finite operationally obtained encoded phenotype, such as a finite sequence of (supposedly random) bits,
alone.
Recall that, to rephrase a dictum of von Neumann~\cite{von-neumann1},
everybody claiming to be able to prove the randomness of a constructive sequence,
is in a state of sin.
Randomness is an asymptotic property of infinite sequences.
Only in very rare circumstances it is possible to formally prove the randomness of a concrete mathematical object
such as the halting probability or the busy beaver function
(and not just an ``element of the continuum'' selected by the axiom of choice~\cite{svozil-set}).

Besides recursion theoretic undecidability,
there appear to be at least two principal sources of indeterminism and randomness in physics:
(i) the deterministic chaos associated with instabilities of classical physical systems,
and with the strong dependence of their future behavior on the initial value;
(ii) quantum indeterminism, which can be subdivided into three subcategories:  random outcomes of individual events,
 complementarity, and
value indefiniteness.

The similarities and differences between classical and quantum randomness can probably be best conceptualized
in terms of two ``black boxes:'' the first one of them --- called the {\em ``Poincar{\'e} box''} ---
containing a classical, deterministic chaotic, source of randomness;
the second  --- called the {\em ``Born box''} ---
containing a quantum source of randomness, such as a quantized system including a beam splitter.
Suppose an agent is being presented with both boxes without any label on, or hint about, them;
i.e., the origin of indeterminism
is unknown to the agent.
In a modified Turing test, the agent's task would be to find out which is the Born and which is
the Poincar{\'e} box by solely observing their output.
As far as is presently known, there should not exist any operational criterion, method or procedure
discriminating amongst these boxes.
Moreover, both types of indeterminism appear to be based on metaphysical assumptions:
in the classical case it is the existence of continua and the possibility to ``choose''
elements thereof, representing the initial values;
in the quantum case it is the irreducible indeterminism of single events~\cite{zeil-05_nature_ofQuantum}.

\bibliography{svozil}
\bibliographystyle{osa}

\end{document}

\begin{thebibliography}{99}

%******** REFERENCES *************************************
%         -- Enter your references below.
%         -- For convenience, we have included sample entries showing the citation
%            format that should be used for a few common types of works.
%         -- Remove these samples (and the notes that follow)
%            when entering your references.









%
%
% SAMPLE REFERENCES:
%
\bibitem{citenameA}
Imhoff, M.L., Bounoua, L., Ricketts, T., et al.
(2004).
Global patterns in human consumption of net primary production.
\textit{Nature}
429:870-3.
%     Type of work:
      \underline{(Sample journal article)}

\bibitem{citenameB}
Barrow, J.D. and Tipler, F.
(1988).
\textit{The Anthropic Cosmological Principle}.
Oxford: Oxford University Press.
%     Type of work:
      \underline{(Sample book)}

\bibitem{citenameC}
Dyson, F.J.
(2004).
Thought-experiments in honor of John Archibald Wheeler.
In \textit{Science and Ultimate Reality: Quantum Theory, Cosmology and Complexity},
ed. J.D. Barrow, P.C.W. Davies, and C.L. Harper Jr.
72-89.
Cambridge: Cambridge University Press.
%     Type of work:
      \underline{(Sample book chapter)}
%
\bibitem{citenameD}
U.S. Census Bureau (2000).
Health insurance coverage status and type of coverage by sex, race, and Hispanic origin.
Health Insurance Historical Table 1.
http://www.census.gov/hhes/hlthins/historic/hihisttl.html.
%     Type of work:
      \underline{(Sample online document)}
%
\bibitem{citenameE}
Doyle, B.
(2002).
Howling like dogs: Metaphorical language in Psalm 59.
Paper presented at the annual international meeting for the Society of Biblical Literature,
June 19-22, Berlin, Germany.
%     Type of work:
      \underline{(Sample paper presented at a meeting or conference)}
%
%
%
% NOTES FOR AUTHORS:
\newline
\newline
\textbf{Notes:}
\begin{itemize}
\item{Although the LaTeX ``book" documentclass labels this section as a ``Bibliography," the section will be labeled ``References" when it is incorporated into the published book. Thus, it should include \textit{only} references to works cited in this chapter. If other works (i.e., those not cited in the chapter) are to be referenced, please present them in an Additional Readings list.}
\item{References should appear in a list at the end of your chapter, not as footnotes or endnotes. Footnotes are used for explanatory or additional information only; do not use endnotes at all.}
\item{Simlarly, do not include explanatory or additional information in the References list; such information should instead be presented in footnotes at appropriate locations in the text.}
\item{Each entry should reference only a single work. References to multiple works, even if related, should be listed as separate entries.}
\item{If a work has more than three authors, list the names of only the first three authors followed by ``et al."}
\item{Although our LaTeX template is set up to use LaTeX's default numerical format, authors are encouraged to instead use the author-date format if it would be more appropriate for their content.}
\end{itemize}
% END NOTES FOR AUTHORS
%
%
%******** END REFERENCES ******************
\end{thebibliography}
%
% -------------------------------------------------------------- [END DOCUMENT]


















