\documentclass[10pt,a4paper]{article}
\textwidth=16cm
\textheight=24.7cm
\oddsidemargin= 0mm
\topmargin -17mm
\begin{document}

\title{Chromatic Quantum Contextuality}
\author{Karl Svozil\thanks{Institute for Theoretical Physics, TU Wien, Vienna, Austria, EU; e-mail: karl.svozil@tuwien.ac.at}}
\date{}
\maketitle
\pagenumbering{gobble}

I present a framework for understanding physical reality based on \textit{contexts,} defined as maximal sets of co-measurable, mutually exclusive observables~\cite{svozil-2025-color}.
In quantum mechanics, contexts correspond to orthonormal bases of Hilbert space or maximal observables, representing the maximal information obtainable from a system upon measurement.
A key aspect of this \textit{operator-valued} framework is the emphasis on the entirety of a context---the potential outcomes that could have been observed alongside the measured one.
This moves beyond standard two-valued measures in quantum logic, which only distinguish between an observed outcome and its complement. Instead, each potential outcome within a context is assigned a unique label.
These labels can be conceptualized as colors, with each color appearing exactly once in every context. Non-contextuality is then defined as the requirement that an observable's assigned color is independent of the context in which it is measured.
This coloring scheme imposes a strong restriction on classical assignments: only two-valued measures derivable by mapping one color to 1 and all others to 0 are permitted.
Consequently, certain two-valued measures typically allowed by standard exclusivity and completeness rules are excluded.
This has significant implications for constructing classical probability distributions, usually built as convex sums of such measures.
This can be demonstrated by analyzing the pentagon/pentagram hypergraph, identifying a two-valued measure excluded by the coloring approach and deriving consequences for associated Boole-Bell type inequalities, specifically the Klyachko inequality.
I also introduce a generalized chromatic-type Kochen-Specker theorem, demonstrated by the Yo-Oh uniform hypergraph. This hypergraph serves as a crucial example as it is both set representable within a partition logic and yet possesses a chromatic number exceeding its clique number.
In summary, this work proposes a tightened notion of classical physical existence, constrained by the maximal knowledge provided by quantum observables.
This leads to a more restrictive criterion for valid two-valued measures and the resulting classical probability distributions relevant to contextuality tests.

%%Acknowledgments can be added here.
%%\medskip
%%\noindent \textbf{Acknowledgments}


\begin{thebibliography}{9}

\bibitem{svozil-2025-color} Karl Svozil, ``Chromatic Quantum Contextuality'', Entropy 27(4), 387 (2025) [DOI: 10.3390/e27040387]

\end{thebibliography}

\end{document}
