%\documentclass[pra,showpacs,showkeys,amsfonts,amsmath,twocolumn]{revtex4}
\documentclass[amsmath,blue,handout]{beamer}
%\documentclass[pra,showpacs,showkeys,amsfonts]{revtex4}
\usepackage[T1]{fontenc}
\usepackage{beamerthemeshadow}
%\usepackage[dark]{beamerthemesidebar}
%\usepackage[headheight=24pt,footheight=12pt]{beamerthemesplit}
%\usepackage{beamerthemesplit}
%\usepackage[bar]{beamerthemetree}
\usepackage{graphicx}
\usepackage{pgf}

%\RequirePackage[german]{babel}
%\selectlanguage{german}
%\RequirePackage[isolatin]{inputenc}

\pgfdeclareimage[height=0.5cm]{logo}{tu-logo}
\logo{\pgfuseimage{logo}}
\beamertemplatetriangleitem
\begin{document}

\title{\bf Quantum information \& computation}
%\subtitle{Naturwissenschaftlich-Humanisticher Tag am BG 19\\Weltbild und Wissenschaft\\http://tph.tuwien.ac.at/\~{}svozil/publ/2005-BG18-pres.pdf}
\subtitle{http://tph.tuwien.ac.at/$\sim$svozil/publ/2005-stpoelten-pres.pdf}
\author{Karl Svozil}
\institute{Institut f\"ur Theoretische Physik, University of Technology Vienna, \\
Wiedner Hauptstra\ss e 8-10/136, A-1040 Vienna, Austria\\
svozil@tuwien.ac.at
%{\tiny Disclaimer: Die hier vertretenen Meinungen des Autors verstehen sich als Diskussionsbeitr�ge und decken sich nicht notwendigerweise mit den Positionen der Technischen Universit�t Wien oder deren Vertreter.}
}
\date{Nov. 8, 2005}
\maketitle

\frame{\tableofcontents}


\section{Quantum information}


\subsection{Basics \& differences to classical information}
\frame[shrink=2]{
\frametitle{Basics \& differences to classical information}

\begin{itemize}
\item<+->
Elementary unit of classical information is the classical bit (``cbit'')
which is in one of the two classical states
``0'' or ``no'' or ``false'' and
``1'' or ``yes'' or ``true,'' respectively.

\item<+->
Elementary unit of quantum information is the quantum bit (``qubit'')
which can be in a {\em coherent superposition}
$$\vert \Psi\rangle =
a_0 \vert 0\rangle
+
a_1 \vert 1\rangle , \qquad {\rm with}\quad
\vert a_0 \vert^2
+
\vert a_1 \vert^2
=1$$
of the classical states
``0''  and ``1.''

\item<+->
A single qubit ``embodies'' two classically contradictory states at once.
This is the basis of ``quantum parallelism.''

\item<+->
$n$ single qubits ``embody'' $2^n$ classically contradictory states at once.
A linear increase of quantum information
is associated with an exponential increase of embodied classically states -- ``quantum parallelism.''

\end{itemize}

}


\frame{
\frametitle{Representations of cbits \& qubits}

\begin{itemize}
\item<+->
Representation of the two cbits as orthogonal vectors in a two-dimensional vector space:


$0\equiv
\left(
\begin{array}{c}
0
\\
1
 \end{array}
\right)
,
\qquad 1\equiv
\left(
\begin{array}{c}
1
\\
0
 \end{array}
\right)
$.


\item<+->
Representation of qubits as normalized vectors in a two-dimensional vector space:
$$\vert \Psi\rangle =
a_0 \vert 0\rangle
+
a_1 \vert 1\rangle \equiv
\left(
\begin{array}{c}
a_1
\\
a_0
 \end{array}
\right)
, \; \; {\rm with}\:
\vert a_0 \vert^2
+
\vert a_1 \vert^2
=1.$$

\end{itemize}

}


%\section{}
\subsection{Quantum state evolution: one-to-one}
\frame[shrink=2]{
\frametitle{Quantum state evolution: one-to-one}


\begin{itemize}
\item<+->
Classical reversible computation associated with permutations of the classical states
associated with permutation matrices (only a single entry ``1'' per row \& column, else ``0'').

\item<+->
Inbetween measurements, quantum states follow reversible deterministic, unitary state evolution:
$$\vert \Psi_{{\rm later}} \rangle = U \vert \Psi_{{\rm former}} \rangle .$$

\item<+->
$U$ is a unitary matrix: $UU^\dagger= U[(U^\ast)^T]=1$; \\
i.e., $U^\dagger =U^{-1}$. Here,

\begin{itemize}
\item<+->
``$\ast$'' stands for ``complex conjugate,''
\item<+->
``$T$'' stands for ``transposition,'' and
\item<+->
``$\dagger$'' stands for ``hermitean conjugate'' (``$= \ast \& T$''),
respectively.
\end{itemize}
\end{itemize}
}

\frame{
\frametitle{Quantum state evolution: Examples}

\begin{itemize}
\item<+->
The identity  defined by
$\vert  {\bf 0}  \rangle  \rightarrow  \vert  {\bf 0}  \rangle $,
$\vert  {\bf 1}  \rangle  \rightarrow  \vert  {\bf 1}  \rangle $:
$ {\Bbb I}_2=
\left(
\begin{array}{cc}
1&0
\\
0&1
 \end{array}
\right)
$,
\item<+->
the ``${\tt not}$''
defined by
$\vert  {\bf 0}  \rangle  \rightarrow  \vert  {\bf 1}  \rangle $,
$\vert  {\bf 1}  \rangle  \rightarrow  \vert  {\bf 0}  \rangle $:
$$  \textsf{\textbf{X}} =
\left(
\begin{array}{cc}
0&1
\\
1&0
 \end{array}
\right)
,$$

\item<+->
the Hadamard ``$\textsf{\textbf{H}}=\sqrt{{\Bbb I}_2}$,''
defined by
$\vert  {\bf 0}  \rangle  \rightarrow  (\vert  {\bf 0}  \rangle + \vert  {\bf 1}  \rangle )(1/\sqrt{2})$,
$\vert  {\bf 1}  \rangle  \rightarrow  (\vert  {\bf 0}  \rangle - \vert  {\bf 1}  \rangle )(1/\sqrt{2})$:
$$  \textsf{\textbf{H}}=
{1 \over \sqrt{2}}
\left(
\begin{array}{cc}
1&1
\\
1&-1
 \end{array}
\right)
$$ with $\sqrt{{\Bbb I}_2}\cdot \sqrt{{\Bbb I}_2} = {\Bbb I}_2$,

\item<+->
the $\sqrt{{\tt not}}$:  $
{1 \over 2}
\left(
\begin{array}{cc}
1+i&1-i
\\
1-i&1+i
 \end{array}
\right)
$
with
$
\sqrt{{\tt not}}
\sqrt{{\tt not}} = {\tt not}$.

\end{itemize}


}



%\section{}
\subsection{Mach-Zehnder interferometer}
\frame[shrink=2]{
\frametitle{Mach-Zehnder interferometer}
%TexCad Options
%\grade{\off}
%\emlines{\off}
%\beziermacro{\off}
%\reduce{\on}
%\snapping{\off}
%\quality{0.20}
%\graddiff{0.01}
%\snapasp{1}
%\zoom{1.00}
\unitlength 0.70mm
\linethickness{0.4pt}
\begin{picture}(83.67,61.00)
%\emline(62.67,40.00)(62.67,15.00)
\put(62.67,40.00){\line(0,-1){25.00}}
%\end
\put(10.00,55.00){\makebox(0,0)[cc]{$L$}}
\put(10.00,55.00){\circle{10.00}}
%\emline(15.00,55.00)(55.00,55.00)
\put(15.00,55.00){\line(1,0){40.00}}
%\end
%\emline(44.67,55.00)(62.67,55.00)
\put(44.67,55.00){\line(1,0){18.00}}
%\end
%\emline(25.00,55.00)(25.00,30.00)
\put(25.00,55.00){\line(0,-1){25.00}}
%\end
%\emline(25.00,30.00)(38.00,30.00)
\put(25.00,30.00){\line(1,0){13.00}}
%\end
%\emline(62.67,55.00)(62.67,30.00)
\put(62.67,55.00){\line(0,-1){25.00}}
%\end
%\emline(62.67,30.00)(27.67,30.00)
\put(62.67,30.00){\line(-1,0){35.00}}
%\end
%\emline(62.67,30.00)(75.67,30.00)
\put(62.67,30.00){\line(1,0){13.00}}
%\end
%\emline(62.67,30.00)(62.67,17.00)
\put(62.67,30.00){\line(0,-1){13.00}}
%\end
%\emline(58.67,59.00)(66.67,51.00)
\multiput(58.67,59.00)(0.12,-0.12){67}{\line(0,-1){0.12}}
%\end
%\emline(21.00,35.00)(29.00,25.00)
\multiput(21.00,35.00)(0.12,-0.15){67}{\line(0,-1){0.15}}
%\end
\put(80.17,30.00){\oval(7.00,8.00)[r]}
\put(83.67,36.00){\makebox(0,0)[cc]{$D_1$}}
\put(28.00,42.00){\makebox(0,0)[cc]{$c$}}
\put(25.00,61.00){\makebox(0,0)[cc]{$S_1$}}
\put(56.67,38.00){\makebox(0,0)[cc]{$S_2$}}
%\emline(24.00,56.00)(26.00,54.00)
\multiput(24.00,56.00)(0.12,-0.12){17}{\line(0,-1){0.12}}
%\end
%\emline(23.00,57.00)(21.00,59.00)
\multiput(23.00,57.00)(-0.12,0.12){17}{\line(0,1){0.12}}
%\end
%\emline(27.00,53.00)(29.00,51.00)
\multiput(27.00,53.00)(0.12,-0.12){17}{\line(0,-1){0.12}}
%\end
%\emline(61.67,31.00)(63.67,29.00)
\multiput(61.67,31.00)(0.12,-0.12){17}{\line(0,-1){0.12}}
%\end
%\emline(60.67,32.00)(58.67,34.00)
\multiput(60.67,32.00)(-0.12,0.12){17}{\line(0,1){0.12}}
%\end
%\emline(64.67,28.00)(66.67,26.00)
\multiput(64.67,28.00)(0.12,-0.12){17}{\line(0,-1){0.12}}
%\end
\put(18.00,51.00){\makebox(0,0)[cc]{$a$}}
\put(60.67,41.00){\framebox(4.00,5.00)[cc]{}}
\put(68.67,43.00){\makebox(0,0)[cc]{$\varphi$}}
\put(58.67,43.00){\makebox(0,0)[cc]{$P$}}
\put(70.67,33.00){\makebox(0,0)[cc]{$d$}}
\put(62.67,13.33){\oval(8.67,8.00)[b]}
\put(70.00,9.00){\makebox(0,0)[cc]{$D_2$}}
\put(65.33,20.33){\makebox(0,0)[cc]{$e$}}
\put(43.00,61.00){\makebox(0,0)[cc]{$b$}}
\put(62.00,61.00){\makebox(0,0)[cc]{$M$}}
\put(27.33,21.00){\makebox(0,0)[cc]{$M$}}
\end{picture}\\
Mach-Zehnder interferometer.
A single quantum (photon, neutron, electron {\it etc}) is emitted in $L$
and meets a lossless beam splitter (half-silvered mirror) $S_1$, after
which its wave function
is in a coherent superposition of $  b $ and $  c $. In beam
path $b$ a phase shifter shifts the phase of state $  b $ by
$\varphi$. The two beams are then recombined at a second lossless
beam splitter (half-silvered
mirror) $S_2$. The quant is detected at either $D_1$ or $D_2$,
corresponding to the states $d $ and $ e $, respectively.
}


\frame[shrink=2]{
\frametitle{Mach-Zehnder interferometer cntd.}
\begin{tabular}{cc}
%TexCad Options
%\grade{\off}
%\emlines{\off}
%\beziermacro{\off}
%\reduce{\on}
%\snapping{\off}
%\quality{0.20}
%\graddiff{0.01}
%\snapasp{1}
%\zoom{1.00}
\unitlength 0.70mm
\linethickness{0.4pt}
\begin{picture}(83.67,61.00)
%\emline(62.67,40.00)(62.67,15.00)
\put(62.67,40.00){\line(0,-1){25.00}}
%\end
\put(10.00,55.00){\makebox(0,0)[cc]{$L$}}
\put(10.00,55.00){\circle{10.00}}
%\emline(15.00,55.00)(55.00,55.00)
\put(15.00,55.00){\line(1,0){40.00}}
%\end
%\emline(44.67,55.00)(62.67,55.00)
\put(44.67,55.00){\line(1,0){18.00}}
%\end
%\emline(25.00,55.00)(25.00,30.00)
\put(25.00,55.00){\line(0,-1){25.00}}
%\end
%\emline(25.00,30.00)(38.00,30.00)
\put(25.00,30.00){\line(1,0){13.00}}
%\end
%\emline(62.67,55.00)(62.67,30.00)
\put(62.67,55.00){\line(0,-1){25.00}}
%\end
%\emline(62.67,30.00)(27.67,30.00)
\put(62.67,30.00){\line(-1,0){35.00}}
%\end
%\emline(62.67,30.00)(75.67,30.00)
\put(62.67,30.00){\line(1,0){13.00}}
%\end
%\emline(62.67,30.00)(62.67,17.00)
\put(62.67,30.00){\line(0,-1){13.00}}
%\end
%\emline(58.67,59.00)(66.67,51.00)
\multiput(58.67,59.00)(0.12,-0.12){67}{\line(0,-1){0.12}}
%\end
%\emline(21.00,35.00)(29.00,25.00)
\multiput(21.00,35.00)(0.12,-0.15){67}{\line(0,-1){0.15}}
%\end
\put(80.17,30.00){\oval(7.00,8.00)[r]}
\put(83.67,36.00){\makebox(0,0)[cc]{$D_1$}}
\put(28.00,42.00){\makebox(0,0)[cc]{$c$}}
\put(25.00,61.00){\makebox(0,0)[cc]{$S_1$}}
\put(56.67,38.00){\makebox(0,0)[cc]{$S_2$}}
%\emline(24.00,56.00)(26.00,54.00)
\multiput(24.00,56.00)(0.12,-0.12){17}{\line(0,-1){0.12}}
%\end
%\emline(23.00,57.00)(21.00,59.00)
\multiput(23.00,57.00)(-0.12,0.12){17}{\line(0,1){0.12}}
%\end
%\emline(27.00,53.00)(29.00,51.00)
\multiput(27.00,53.00)(0.12,-0.12){17}{\line(0,-1){0.12}}
%\end
%\emline(61.67,31.00)(63.67,29.00)
\multiput(61.67,31.00)(0.12,-0.12){17}{\line(0,-1){0.12}}
%\end
%\emline(60.67,32.00)(58.67,34.00)
\multiput(60.67,32.00)(-0.12,0.12){17}{\line(0,1){0.12}}
%\end
%\emline(64.67,28.00)(66.67,26.00)
\multiput(64.67,28.00)(0.12,-0.12){17}{\line(0,-1){0.12}}
%\end
\put(18.00,51.00){\makebox(0,0)[cc]{$a$}}
\put(60.67,41.00){\framebox(4.00,5.00)[cc]{}}
\put(68.67,43.00){\makebox(0,0)[cc]{$\varphi$}}
\put(58.67,43.00){\makebox(0,0)[cc]{$P$}}
\put(70.67,33.00){\makebox(0,0)[cc]{$d$}}
\put(62.67,13.33){\oval(8.67,8.00)[b]}
\put(70.00,9.00){\makebox(0,0)[cc]{$D_2$}}
\put(65.33,20.33){\makebox(0,0)[cc]{$e$}}
\put(43.00,61.00){\makebox(0,0)[cc]{$b$}}
\put(62.00,61.00){\makebox(0,0)[cc]{$M$}}
\put(27.33,21.00){\makebox(0,0)[cc]{$M$}}
\end{picture}
&
$
\begin{array}{ccl}
S_1:\; a  &\rightarrow& ( b  +i c
)/\sqrt{2}\quad , \\
P:\; b  &\rightarrow&  b e^{i \varphi
}\quad ,\\
S_2:\; b  &\rightarrow& ( e  + i
d )/\sqrt{2}\quad ,\\
S_2:\; c  &\rightarrow& ( d  + i
e )/\sqrt{2}\quad .
\end{array}
$\\
\end{tabular}

$$ a  \rightarrow \psi =i\left( {e^{i\varphi} +1\over 2}\right)
d  +
\left( {e^{i\varphi} -1\over 2}\right)
e  .
$$
$\varphi =0$, i.e., there is no phase shift at all:
$ a  \rightarrow i d $, and the emitted quant is detected
only by $D_1$.\\
$\varphi =\pi $:
$ a  \rightarrow -  e  $, and the emitted quant is detected
only by $D_2$.\\
For general phase shift $\varphi$:
$$
P_{D_1}(\varphi )=\vert ( d, \psi ) \vert^2=\cos^2({\varphi
\over 2})
\quad ,
\quad
P_{D_2}(\varphi )=\vert ( e, \psi ) \vert^2=\sin^2({\varphi
\over 2})
\quad .
$$
}


\frame{
\frametitle{Alternative representations}

Alternatively, the action of a lossless beam splitter may be
described by the unitary matrix
$$
\left(
\begin{array}{cc}
i \, \sqrt{R(\omega )}& \sqrt{T(\omega )}
\\
\sqrt{T(\omega )}&  i\, \sqrt{R(\omega )}
 \end{array}
\right)
=
\left(
\begin{array}{cc}
i \, \sin \omega  & \cos \omega
\\
\cos \omega&  i\, \sin \omega
 \end{array}
\right)
.
$$
A phase shifter in two-dimensional Hilbert space is represented by
either the unitary matrix
$${\rm  diag}\left(
e^{i\varphi },1
\right)
\qquad
{\rm or}
\qquad
{\rm  diag}
\left(
1,e^{i\varphi }
\right)
.$$


}



%\section{}
\subsection{``Quantum mindbogglers''}
\frame[shrink=2]{
\frametitle{``Interaction-free'' measurement}

%TexCad Options
%\grade{\off}
%\emlines{\off}
%\beziermacro{\off}
%\reduce{\on}
%\snapping{\off}
%\quality{0.20}
%\graddiff{0.01}
%\snapasp{1}
%\zoom{1.00}
\unitlength 0.70mm
\linethickness{0.4pt}
\begin{picture}(83.67,61.00)
%\emline(62.67,40.00)(62.67,15.00)
\put(62.67,40.00){\line(0,-1){25.00}}
%\end
\put(10.00,55.00){\makebox(0,0)[cc]{$L$}}
\put(10.00,55.00){\circle{10.00}}
%\emline(15.00,55.00)(55.00,55.00)
\put(15.00,55.00){\line(1,0){40.00}}
%\end
%\emline(44.67,55.00)(62.67,55.00)
\put(44.67,55.00){\line(1,0){18.00}}
%\end
%\emline(25.00,55.00)(25.00,30.00)
\put(25.00,55.00){\line(0,-1){25.00}}
%\end
%\emline(25.00,30.00)(38.00,30.00)
\put(25.00,30.00){\line(1,0){13.00}}
%\end
%\emline(62.67,55.00)(62.67,30.00)
\put(62.67,55.00){\line(0,-1){25.00}}
%\end
%\emline(62.67,30.00)(27.67,30.00)
\put(62.67,30.00){\line(-1,0){35.00}}
%\end
%\emline(62.67,30.00)(75.67,30.00)
\put(62.67,30.00){\line(1,0){13.00}}
%\end
%\emline(62.67,30.00)(62.67,17.00)
\put(62.67,30.00){\line(0,-1){13.00}}
%\end
%\emline(58.67,59.00)(66.67,51.00)
\multiput(58.67,59.00)(0.12,-0.12){67}{\line(0,-1){0.12}}
%\end
%\emline(21.00,35.00)(29.00,25.00)
\multiput(21.00,35.00)(0.12,-0.15){67}{\line(0,-1){0.15}}
%\end
\put(80.17,30.00){\oval(7.00,8.00)[r]}
\put(83.67,36.00){\makebox(0,0)[cc]{$D_1$}}
\put(28.00,42.00){\makebox(0,0)[cc]{$c$}}
\put(25.00,61.00){\makebox(0,0)[cc]{$S_1$}}
\put(56.67,38.00){\makebox(0,0)[cc]{$S_2$}}
%\emline(24.00,56.00)(26.00,54.00)
\multiput(24.00,56.00)(0.12,-0.12){17}{\line(0,-1){0.12}}
%\end
%\emline(23.00,57.00)(21.00,59.00)
\multiput(23.00,57.00)(-0.12,0.12){17}{\line(0,1){0.12}}
%\end
%\emline(27.00,53.00)(29.00,51.00)
\multiput(27.00,53.00)(0.12,-0.12){17}{\line(0,-1){0.12}}
%\end
%\emline(61.67,31.00)(63.67,29.00)
\multiput(61.67,31.00)(0.12,-0.12){17}{\line(0,-1){0.12}}
%\end
%\emline(60.67,32.00)(58.67,34.00)
\multiput(60.67,32.00)(-0.12,0.12){17}{\line(0,1){0.12}}
%\end
%\emline(64.67,28.00)(66.67,26.00)
\multiput(64.67,28.00)(0.12,-0.12){17}{\line(0,-1){0.12}}
%\end
\put(18.00,51.00){\makebox(0,0)[cc]{$a$}}
\put(68.67,43.00){\makebox(0,0)[cc]{$P$}}
\put(58.67,43.00){\makebox(0,0)[cc]{$$}}
\put(70.67,33.00){\makebox(0,0)[cc]{$d$}}
\put(62.67,13.33){\oval(8.67,8.00)[b]}
\put(70.00,9.00){\makebox(0,0)[cc]{$D_2$}}
\put(65.33,20.33){\makebox(0,0)[cc]{$e$}}
\put(43.00,61.00){\makebox(0,0)[cc]{$b$}}
\put(62.00,61.00){\makebox(0,0)[cc]{$M$}}
\put(27.33,21.00){\makebox(0,0)[cc]{$M$}}
\put(60.67,41.00){\rule{4.00\unitlength}{5.00\unitlength}}
\end{picture}

Case \#1:
Suppose, $P$ is not a beam splitter, but a perfect absorber.
Then, the beam path $b$ is blocked entirely, leaving open only beam path $c$,
resulting in a 50:50 chance that detectors $D_1$ and $D_2$ fire.


Case \#2:
Suppose, $P$ is a transparent medium (no absorber): since $\varphi \equiv 0$: only $D_1$ fires.

Hence: if we want to know whether or not an absorber is in beam path $b$, then whenever $D_2$ fires
(in 1/4 of all cases), we know that the absorber is present although the quant ``has not touched it.''
We als say that ``no interachtion has taken place between the absorber \& the quant.''
Has it not ;-)

SINGLE QUANT (QUBIT) EFFECT!!!!!


}


%\section{}
\frame[shrink=2]{
\frametitle{``Delayed choice'' measurements}

%TexCad Options
%\grade{\off}
%\emlines{\off}
%\beziermacro{\off}
%\reduce{\on}
%\snapping{\off}
%\quality{0.20}
%\graddiff{0.01}
%\snapasp{1}
%\zoom{1.00}
\unitlength 0.70mm
\linethickness{0.4pt}
\begin{picture}(83.67,61.00)
%\emline(62.67,40.00)(62.67,15.00)
\put(62.67,40.00){\line(0,-1){25.00}}
%\end
\put(10.00,55.00){\makebox(0,0)[cc]{$L$}}
\put(10.00,55.00){\circle{10.00}}
%\emline(15.00,55.00)(55.00,55.00)
\put(15.00,55.00){\line(1,0){40.00}}
%\end
%\emline(44.67,55.00)(62.67,55.00)
\put(44.67,55.00){\line(1,0){18.00}}
%\end
%\emline(25.00,55.00)(25.00,30.00)
\put(25.00,55.00){\line(0,-1){25.00}}
%\end
%\emline(25.00,30.00)(38.00,30.00)
\put(25.00,30.00){\line(1,0){13.00}}
%\end
%\emline(62.67,55.00)(62.67,30.00)
\put(62.67,55.00){\line(0,-1){25.00}}
%\end
%\emline(62.67,30.00)(27.67,30.00)
\put(62.67,30.00){\line(-1,0){35.00}}
%\end
%\emline(62.67,30.00)(75.67,30.00)
\put(62.67,30.00){\line(1,0){13.00}}
%\end
%\emline(62.67,30.00)(62.67,17.00)
\put(62.67,30.00){\line(0,-1){13.00}}
%\end
%\emline(58.67,59.00)(66.67,51.00)
\multiput(58.67,59.00)(0.12,-0.12){67}{\line(0,-1){0.12}}
%\end
%\emline(21.00,35.00)(29.00,25.00)
\multiput(21.00,35.00)(0.12,-0.15){67}{\line(0,-1){0.15}}
%\end
\put(80.17,30.00){\oval(7.00,8.00)[r]}
\put(83.67,36.00){\makebox(0,0)[cc]{$D_1$}}
\put(28.00,42.00){\makebox(0,0)[cc]{$c$}}
\put(25.00,61.00){\makebox(0,0)[cc]{$S_1$}}
\put(56.67,38.00){\makebox(0,0)[cc]{$S_2$}}
%\emline(24.00,56.00)(26.00,54.00)
\multiput(24.00,56.00)(0.12,-0.12){17}{\line(0,-1){0.12}}
%\end
%\emline(23.00,57.00)(21.00,59.00)
\multiput(23.00,57.00)(-0.12,0.12){17}{\line(0,1){0.12}}
%\end
%\emline(27.00,53.00)(29.00,51.00)
\multiput(27.00,53.00)(0.12,-0.12){17}{\line(0,-1){0.12}}
%\end
%\emline(61.67,31.00)(63.67,29.00)
\multiput(61.67,31.00)(0.12,-0.12){17}{\line(0,-1){0.12}}
%\end
%\emline(60.67,32.00)(58.67,34.00)
\multiput(60.67,32.00)(-0.12,0.12){17}{\line(0,1){0.12}}
%\end
%\emline(64.67,28.00)(66.67,26.00)
\multiput(64.67,28.00)(0.12,-0.12){17}{\line(0,-1){0.12}}
%\end
\put(18.00,51.00){\makebox(0,0)[cc]{$a$}}
\put(68.67,43.00){\makebox(0,0)[cc]{$P$}}
\put(58.67,43.00){\makebox(0,0)[cc]{$$}}
\put(70.67,33.00){\makebox(0,0)[cc]{$d$}}
\put(62.67,13.33){\oval(8.67,8.00)[b]}
\put(70.00,9.00){\makebox(0,0)[cc]{$D_2$}}
\put(65.33,20.33){\makebox(0,0)[cc]{$e$}}
\put(43.00,61.00){\makebox(0,0)[cc]{$b$}}
\put(62.00,61.00){\makebox(0,0)[cc]{$M$}}
\put(27.33,21.00){\makebox(0,0)[cc]{$M$}}
\put(60.67,41.00){\rule{4.00\unitlength}{5.00\unitlength}}
\end{picture}

Suppose we block beam path $b$ with an absorber at $P$ only {\em after} the quant has ``passed''
the first 50:50 mirror at $S_1$ and is ``somewhere inbetween $S_1$ and $P$.''

Would this make any difference as compared to blocking the path $b$ beforehand; i.e., before the quant has ``passed''
the first 50:50 mirror at $S_1$?

Guess what happens ;-)

SINGLE QUANT (QUBIT) EFFECT!!!!!


}


%\section{}
\subsection{Classical \& quantum correlations and the Boole-Bell inequalities}

\frame{
\frametitle{General setup}

\begin{itemize}
\item<+->
Two measurement directions ${ a}$ and ${ b}$
of two
dichotomic observables with values ``-1'' and ``1''
at two spatially separated locations.
\item<+->
The measurement direction ${a}$ at ``Alice's location''
is unknown to an observer ``Bob'' measuring ${ b}$ and {\it vice versa}.
\item<+->
A two-particle correlation function $E(\theta )$
with $\theta =\vert {a} - { b}\vert $
is defined by averaging the product of the outcomes $O({ a})_i, O({ b} )_i\in {-1,1}$
in the $i$th experiment; i.e.,  $E(\theta )=(1/N)\sum_{i=1}^N O({ a})_i O({ b})_i$.
\end{itemize}
}


\frame{
\frametitle{Classical correlations for two-particle ``perfectly correlated'' state}
Assume uniform  distribution of (opposite) ``angular momentum'' of the two particles; Alice measuring along angle $a$, Bob measuring along $b$:
$$
\begin{array}{l}
E(a,b) = {A_+(a,b)-A_-(a,b)\over 2\pi}= {2A_+(a,b) -2\pi \over 2\pi}=
\\
\qquad \qquad \qquad \qquad  ={2\over \pi}\vert a-b\vert - 1={2\over \pi}\theta - 1
 \end{array}
$$

%TexCad Options
%\grade{\on}
%\emlines{\off}
%\beziermacro{\on}
%\reduce{\on}
%\snapping{\off}
%\quality{8.00}
%\graddiff{0.01}
%\snapasp{1}
%\zoom{0.60}
\unitlength 0.40mm
\linethickness{0.4pt}
\begin{picture}(220.35,68.50)
(0,0)
%\circle(30.25,29.75){61.53}
\put(30.25,60.52){\line(1,0){1.23}}
\put(31.48,60.49){\line(1,0){1.22}}
\multiput(32.70,60.42)(0.61,-0.06){2}{\line(1,0){0.61}}
\multiput(33.92,60.30)(0.61,-0.09){2}{\line(1,0){0.61}}
\multiput(35.14,60.12)(0.60,-0.11){2}{\line(1,0){0.60}}
\multiput(36.34,59.91)(0.40,-0.09){3}{\line(1,0){0.40}}
\multiput(37.54,59.64)(0.40,-0.10){3}{\line(1,0){0.40}}
\multiput(38.73,59.32)(0.29,-0.09){4}{\line(1,0){0.29}}
\multiput(39.90,58.96)(0.29,-0.10){4}{\line(1,0){0.29}}
\multiput(41.06,58.55)(0.28,-0.11){4}{\line(1,0){0.28}}
\multiput(42.20,58.10)(0.22,-0.10){5}{\line(1,0){0.22}}
\multiput(43.32,57.60)(0.22,-0.11){5}{\line(1,0){0.22}}
\multiput(44.42,57.06)(0.22,-0.12){5}{\line(1,0){0.22}}
\multiput(45.50,56.47)(0.18,-0.10){6}{\line(1,0){0.18}}
\multiput(46.55,55.84)(0.17,-0.11){6}{\line(1,0){0.17}}
\multiput(47.58,55.17)(0.17,-0.12){6}{\line(1,0){0.17}}
\multiput(48.58,54.46)(0.14,-0.11){7}{\line(1,0){0.14}}
\multiput(49.55,53.71)(0.13,-0.11){7}{\line(1,0){0.13}}
\multiput(50.49,52.92)(0.13,-0.12){7}{\line(1,0){0.13}}
\multiput(51.40,52.10)(0.11,-0.11){8}{\line(1,0){0.11}}
\multiput(52.27,51.23)(0.12,-0.13){7}{\line(0,-1){0.13}}
\multiput(53.11,50.34)(0.11,-0.13){7}{\line(0,-1){0.13}}
\multiput(53.91,49.41)(0.11,-0.14){7}{\line(0,-1){0.14}}
\multiput(54.68,48.45)(0.10,-0.14){7}{\line(0,-1){0.14}}
\multiput(55.40,47.46)(0.11,-0.17){6}{\line(0,-1){0.17}}
\multiput(56.09,46.45)(0.11,-0.17){6}{\line(0,-1){0.17}}
\multiput(56.74,45.40)(0.10,-0.18){6}{\line(0,-1){0.18}}
\multiput(57.34,44.33)(0.11,-0.22){5}{\line(0,-1){0.22}}
\multiput(57.90,43.24)(0.10,-0.22){5}{\line(0,-1){0.22}}
\multiput(58.41,42.13)(0.12,-0.28){4}{\line(0,-1){0.28}}
\multiput(58.89,41.00)(0.11,-0.29){4}{\line(0,-1){0.29}}
\multiput(59.31,39.85)(0.09,-0.29){4}{\line(0,-1){0.29}}
\multiput(59.69,38.68)(0.11,-0.39){3}{\line(0,-1){0.39}}
\multiput(60.02,37.50)(0.10,-0.40){3}{\line(0,-1){0.40}}
\multiput(60.31,36.30)(0.12,-0.60){2}{\line(0,-1){0.60}}
\multiput(60.55,35.10)(0.09,-0.61){2}{\line(0,-1){0.61}}
\multiput(60.74,33.89)(0.07,-0.61){2}{\line(0,-1){0.61}}
\put(60.88,32.67){\line(0,-1){1.22}}
\put(60.97,31.45){\line(0,-1){1.23}}
\put(61.01,30.22){\line(0,-1){1.23}}
\put(61.01,28.99){\line(0,-1){1.23}}
\put(60.95,27.77){\line(0,-1){1.22}}
\multiput(60.85,26.54)(-0.08,-0.61){2}{\line(0,-1){0.61}}
\multiput(60.70,25.33)(-0.10,-0.61){2}{\line(0,-1){0.61}}
\multiput(60.49,24.12)(-0.08,-0.40){3}{\line(0,-1){0.40}}
\multiput(60.25,22.91)(-0.10,-0.40){3}{\line(0,-1){0.40}}
\multiput(59.95,21.72)(-0.11,-0.39){3}{\line(0,-1){0.39}}
\multiput(59.61,20.55)(-0.10,-0.29){4}{\line(0,-1){0.29}}
\multiput(59.22,19.38)(-0.11,-0.29){4}{\line(0,-1){0.29}}
\multiput(58.78,18.24)(-0.10,-0.23){5}{\line(0,-1){0.23}}
\multiput(58.30,17.11)(-0.11,-0.22){5}{\line(0,-1){0.22}}
\multiput(57.77,16.00)(-0.11,-0.22){5}{\line(0,-1){0.22}}
\multiput(57.20,14.91)(-0.10,-0.18){6}{\line(0,-1){0.18}}
\multiput(56.59,13.85)(-0.11,-0.17){6}{\line(0,-1){0.17}}
\multiput(55.93,12.81)(-0.12,-0.17){6}{\line(0,-1){0.17}}
\multiput(55.24,11.80)(-0.11,-0.14){7}{\line(0,-1){0.14}}
\multiput(54.50,10.82)(-0.11,-0.14){7}{\line(0,-1){0.14}}
\multiput(53.73,9.87)(-0.12,-0.13){7}{\line(0,-1){0.13}}
\multiput(52.92,8.95)(-0.11,-0.11){8}{\line(0,-1){0.11}}
\multiput(52.07,8.06)(-0.11,-0.11){8}{\line(-1,0){0.11}}
\multiput(51.19,7.21)(-0.13,-0.12){7}{\line(-1,0){0.13}}
\multiput(50.27,6.39)(-0.14,-0.11){7}{\line(-1,0){0.14}}
\multiput(49.32,5.61)(-0.14,-0.11){7}{\line(-1,0){0.14}}
\multiput(48.35,4.87)(-0.17,-0.12){6}{\line(-1,0){0.17}}
\multiput(47.34,4.17)(-0.17,-0.11){6}{\line(-1,0){0.17}}
\multiput(46.30,3.51)(-0.18,-0.10){6}{\line(-1,0){0.18}}
\multiput(45.25,2.89)(-0.22,-0.12){5}{\line(-1,0){0.22}}
\multiput(44.16,2.31)(-0.22,-0.11){5}{\line(-1,0){0.22}}
\multiput(43.06,1.78)(-0.23,-0.10){5}{\line(-1,0){0.23}}
\multiput(41.93,1.29)(-0.29,-0.11){4}{\line(-1,0){0.29}}
\multiput(40.79,0.85)(-0.29,-0.10){4}{\line(-1,0){0.29}}
\multiput(39.63,0.45)(-0.39,-0.12){3}{\line(-1,0){0.39}}
\multiput(38.45,0.10)(-0.40,-0.10){3}{\line(-1,0){0.40}}
\multiput(37.26,-0.21)(-0.40,-0.09){3}{\line(-1,0){0.40}}
\multiput(36.06,-0.46)(-0.60,-0.10){2}{\line(-1,0){0.60}}
\multiput(34.85,-0.67)(-0.61,-0.08){2}{\line(-1,0){0.61}}
\put(33.64,-0.83){\line(-1,0){1.22}}
\put(32.41,-0.94){\line(-1,0){1.23}}
\put(31.19,-1.00){\line(-1,0){1.23}}
\put(29.96,-1.01){\line(-1,0){1.23}}
\put(28.74,-0.98){\line(-1,0){1.22}}
\multiput(27.51,-0.89)(-0.61,0.07){2}{\line(-1,0){0.61}}
\multiput(26.29,-0.76)(-0.61,0.09){2}{\line(-1,0){0.61}}
\multiput(25.08,-0.58)(-0.60,0.12){2}{\line(-1,0){0.60}}
\multiput(23.87,-0.35)(-0.40,0.09){3}{\line(-1,0){0.40}}
\multiput(22.68,-0.07)(-0.39,0.11){3}{\line(-1,0){0.39}}
\multiput(21.49,0.26)(-0.29,0.09){4}{\line(-1,0){0.29}}
\multiput(20.33,0.63)(-0.29,0.10){4}{\line(-1,0){0.29}}
\multiput(19.17,1.05)(-0.28,0.12){4}{\line(-1,0){0.28}}
\multiput(18.04,1.51)(-0.22,0.10){5}{\line(-1,0){0.22}}
\multiput(16.92,2.02)(-0.22,0.11){5}{\line(-1,0){0.22}}
\multiput(15.83,2.58)(-0.21,0.12){5}{\line(-1,0){0.21}}
\multiput(14.75,3.17)(-0.17,0.11){6}{\line(-1,0){0.17}}
\multiput(13.71,3.81)(-0.17,0.11){6}{\line(-1,0){0.17}}
\multiput(12.68,4.49)(-0.14,0.10){7}{\line(-1,0){0.14}}
\multiput(11.69,5.21)(-0.14,0.11){7}{\line(-1,0){0.14}}
\multiput(10.73,5.97)(-0.13,0.11){7}{\line(-1,0){0.13}}
\multiput(9.80,6.77)(-0.13,0.12){7}{\line(-1,0){0.13}}
\multiput(8.90,7.60)(-0.11,0.11){8}{\line(0,1){0.11}}
\multiput(8.03,8.47)(-0.12,0.13){7}{\line(0,1){0.13}}
\multiput(7.20,9.38)(-0.11,0.13){7}{\line(0,1){0.13}}
\multiput(6.40,10.31)(-0.11,0.14){7}{\line(0,1){0.14}}
\multiput(5.65,11.28)(-0.12,0.17){6}{\line(0,1){0.17}}
\multiput(4.93,12.27)(-0.11,0.17){6}{\line(0,1){0.17}}
\multiput(4.25,13.30)(-0.11,0.17){6}{\line(0,1){0.17}}
\multiput(3.62,14.35)(-0.12,0.21){5}{\line(0,1){0.21}}
\multiput(3.03,15.42)(-0.11,0.22){5}{\line(0,1){0.22}}
\multiput(2.48,16.52)(-0.10,0.22){5}{\line(0,1){0.22}}
\multiput(1.97,17.64)(-0.12,0.28){4}{\line(0,1){0.28}}
\multiput(1.51,18.77)(-0.10,0.29){4}{\line(0,1){0.29}}
\multiput(1.10,19.93)(-0.09,0.29){4}{\line(0,1){0.29}}
\multiput(0.73,21.10)(-0.11,0.39){3}{\line(0,1){0.39}}
\multiput(0.41,22.28)(-0.09,0.40){3}{\line(0,1){0.40}}
\multiput(0.13,23.48)(-0.11,0.60){2}{\line(0,1){0.60}}
\multiput(-0.10,24.68)(-0.09,0.61){2}{\line(0,1){0.61}}
\multiput(-0.27,25.90)(-0.06,0.61){2}{\line(0,1){0.61}}
\put(-0.40,27.12){\line(0,1){1.22}}
\put(-0.48,28.34){\line(0,1){1.23}}
\put(-0.51,29.57){\line(0,1){1.23}}
\put(-0.50,30.80){\line(0,1){1.23}}
\put(-0.43,32.02){\line(0,1){1.22}}
\multiput(-0.32,33.24)(0.08,0.61){2}{\line(0,1){0.61}}
\multiput(-0.15,34.46)(0.11,0.60){2}{\line(0,1){0.60}}
\multiput(0.06,35.67)(0.09,0.40){3}{\line(0,1){0.40}}
\multiput(0.32,36.87)(0.10,0.40){3}{\line(0,1){0.40}}
\multiput(0.63,38.05)(0.12,0.39){3}{\line(0,1){0.39}}
\multiput(0.98,39.23)(0.10,0.29){4}{\line(0,1){0.29}}
\multiput(1.38,40.39)(0.11,0.29){4}{\line(0,1){0.29}}
\multiput(1.83,41.53)(0.10,0.22){5}{\line(0,1){0.22}}
\multiput(2.32,42.65)(0.11,0.22){5}{\line(0,1){0.22}}
\multiput(2.86,43.76)(0.12,0.22){5}{\line(0,1){0.22}}
\multiput(3.44,44.84)(0.10,0.18){6}{\line(0,1){0.18}}
\multiput(4.06,45.90)(0.11,0.17){6}{\line(0,1){0.17}}
\multiput(4.73,46.93)(0.12,0.17){6}{\line(0,1){0.17}}
\multiput(5.43,47.93)(0.11,0.14){7}{\line(0,1){0.14}}
\multiput(6.18,48.91)(0.11,0.13){7}{\line(0,1){0.13}}
\multiput(6.96,49.85)(0.12,0.13){7}{\line(0,1){0.13}}
\multiput(7.78,50.76)(0.11,0.11){8}{\line(0,1){0.11}}
\multiput(8.64,51.64)(0.11,0.11){8}{\line(1,0){0.11}}
\multiput(9.53,52.49)(0.13,0.12){7}{\line(1,0){0.13}}
\multiput(10.45,53.30)(0.14,0.11){7}{\line(1,0){0.14}}
\multiput(11.40,54.07)(0.14,0.10){7}{\line(1,0){0.14}}
\multiput(12.39,54.80)(0.17,0.12){6}{\line(1,0){0.17}}
\multiput(13.40,55.49)(0.17,0.11){6}{\line(1,0){0.17}}
\multiput(14.44,56.14)(0.18,0.10){6}{\line(1,0){0.18}}
\multiput(15.51,56.75)(0.22,0.11){5}{\line(1,0){0.22}}
\multiput(16.60,57.32)(0.22,0.10){5}{\line(1,0){0.22}}
\multiput(17.71,57.84)(0.28,0.12){4}{\line(1,0){0.28}}
\multiput(18.84,58.32)(0.29,0.11){4}{\line(1,0){0.29}}
\multiput(19.98,58.75)(0.29,0.10){4}{\line(1,0){0.29}}
\multiput(21.15,59.14)(0.39,0.11){3}{\line(1,0){0.39}}
\multiput(22.33,59.48)(0.40,0.10){3}{\line(1,0){0.40}}
\multiput(23.52,59.77)(0.40,0.08){3}{\line(1,0){0.40}}
\multiput(24.72,60.01)(0.61,0.10){2}{\line(1,0){0.61}}
\multiput(25.93,60.21)(0.61,0.07){2}{\line(1,0){0.61}}
\put(27.15,60.36){\line(1,0){1.22}}
\put(28.37,60.46){\line(1,0){1.88}}
%\end
\put(30.25,30.25){\line(0,1){30.50}}
%\dottedline(1.75,235.75)(2,235.25)
\multiput(1.68,235.68)(.125,-.25){3}{{\rule{.4pt}{.4pt}}}
\put(0.00,0.00){}
%\dottedline(2,235.25)(63.5,234.75)
\multiput(1.93,235.18)(.99194,-.00806){63}{{\rule{.4pt}{.4pt}}}
\put(0.00,0.00){}
%\dottedline(9.5,255.5)(55.75,214.25)
\multiput(9.43,255.43)(.72266,-.64453){65}{{\rule{.4pt}{.4pt}}}
\put(0.00,0.00){}
\put(30.25,68.50){\makebox(0,0)[cc]{$a$}}
%\emline(0.00,30.00)(61.00,30.00)
\put(0.00,30.00){\line(1,0){61.00}}
%\end
%\circle(109.92,29.75){61.53}
\put(109.92,60.52){\line(1,0){1.23}}
\put(111.15,60.49){\line(1,0){1.22}}
\multiput(112.37,60.42)(0.61,-0.06){2}{\line(1,0){0.61}}
\multiput(113.59,60.30)(0.61,-0.09){2}{\line(1,0){0.61}}
\multiput(114.81,60.12)(0.60,-0.11){2}{\line(1,0){0.60}}
\multiput(116.01,59.91)(0.40,-0.09){3}{\line(1,0){0.40}}
\multiput(117.21,59.64)(0.40,-0.10){3}{\line(1,0){0.40}}
\multiput(118.40,59.32)(0.29,-0.09){4}{\line(1,0){0.29}}
\multiput(119.57,58.96)(0.29,-0.10){4}{\line(1,0){0.29}}
\multiput(120.73,58.55)(0.28,-0.11){4}{\line(1,0){0.28}}
\multiput(121.87,58.10)(0.22,-0.10){5}{\line(1,0){0.22}}
\multiput(122.99,57.60)(0.22,-0.11){5}{\line(1,0){0.22}}
\multiput(124.09,57.06)(0.22,-0.12){5}{\line(1,0){0.22}}
\multiput(125.17,56.47)(0.18,-0.10){6}{\line(1,0){0.18}}
\multiput(126.22,55.84)(0.17,-0.11){6}{\line(1,0){0.17}}
\multiput(127.25,55.17)(0.17,-0.12){6}{\line(1,0){0.17}}
\multiput(128.25,54.46)(0.14,-0.11){7}{\line(1,0){0.14}}
\multiput(129.22,53.71)(0.13,-0.11){7}{\line(1,0){0.13}}
\multiput(130.16,52.92)(0.13,-0.12){7}{\line(1,0){0.13}}
\multiput(131.07,52.10)(0.11,-0.11){8}{\line(1,0){0.11}}
\multiput(131.94,51.23)(0.12,-0.13){7}{\line(0,-1){0.13}}
\multiput(132.78,50.34)(0.11,-0.13){7}{\line(0,-1){0.13}}
\multiput(133.58,49.41)(0.11,-0.14){7}{\line(0,-1){0.14}}
\multiput(134.35,48.45)(0.10,-0.14){7}{\line(0,-1){0.14}}
\multiput(135.07,47.46)(0.11,-0.17){6}{\line(0,-1){0.17}}
\multiput(135.76,46.45)(0.11,-0.17){6}{\line(0,-1){0.17}}
\multiput(136.41,45.40)(0.10,-0.18){6}{\line(0,-1){0.18}}
\multiput(137.01,44.33)(0.11,-0.22){5}{\line(0,-1){0.22}}
\multiput(137.57,43.24)(0.10,-0.22){5}{\line(0,-1){0.22}}
\multiput(138.08,42.13)(0.12,-0.28){4}{\line(0,-1){0.28}}
\multiput(138.56,41.00)(0.11,-0.29){4}{\line(0,-1){0.29}}
\multiput(138.98,39.85)(0.09,-0.29){4}{\line(0,-1){0.29}}
\multiput(139.36,38.68)(0.11,-0.39){3}{\line(0,-1){0.39}}
\multiput(139.69,37.50)(0.10,-0.40){3}{\line(0,-1){0.40}}
\multiput(139.98,36.30)(0.12,-0.60){2}{\line(0,-1){0.60}}
\multiput(140.22,35.10)(0.09,-0.61){2}{\line(0,-1){0.61}}
\multiput(140.41,33.89)(0.07,-0.61){2}{\line(0,-1){0.61}}
\put(140.55,32.67){\line(0,-1){1.22}}
\put(140.64,31.45){\line(0,-1){1.23}}
\put(140.68,30.22){\line(0,-1){1.23}}
\put(140.68,28.99){\line(0,-1){1.23}}
\put(140.62,27.77){\line(0,-1){1.22}}
\multiput(140.52,26.54)(-0.08,-0.61){2}{\line(0,-1){0.61}}
\multiput(140.37,25.33)(-0.10,-0.61){2}{\line(0,-1){0.61}}
\multiput(140.16,24.12)(-0.08,-0.40){3}{\line(0,-1){0.40}}
\multiput(139.92,22.91)(-0.10,-0.40){3}{\line(0,-1){0.40}}
\multiput(139.62,21.72)(-0.11,-0.39){3}{\line(0,-1){0.39}}
\multiput(139.28,20.55)(-0.10,-0.29){4}{\line(0,-1){0.29}}
\multiput(138.89,19.38)(-0.11,-0.29){4}{\line(0,-1){0.29}}
\multiput(138.45,18.24)(-0.10,-0.23){5}{\line(0,-1){0.23}}
\multiput(137.97,17.11)(-0.11,-0.22){5}{\line(0,-1){0.22}}
\multiput(137.44,16.00)(-0.11,-0.22){5}{\line(0,-1){0.22}}
\multiput(136.87,14.91)(-0.10,-0.18){6}{\line(0,-1){0.18}}
\multiput(136.26,13.85)(-0.11,-0.17){6}{\line(0,-1){0.17}}
\multiput(135.60,12.81)(-0.12,-0.17){6}{\line(0,-1){0.17}}
\multiput(134.91,11.80)(-0.11,-0.14){7}{\line(0,-1){0.14}}
\multiput(134.17,10.82)(-0.11,-0.14){7}{\line(0,-1){0.14}}
\multiput(133.40,9.87)(-0.12,-0.13){7}{\line(0,-1){0.13}}
\multiput(132.59,8.95)(-0.11,-0.11){8}{\line(0,-1){0.11}}
\multiput(131.74,8.06)(-0.11,-0.11){8}{\line(-1,0){0.11}}
\multiput(130.86,7.21)(-0.13,-0.12){7}{\line(-1,0){0.13}}
\multiput(129.94,6.39)(-0.14,-0.11){7}{\line(-1,0){0.14}}
\multiput(128.99,5.61)(-0.14,-0.11){7}{\line(-1,0){0.14}}
\multiput(128.02,4.87)(-0.17,-0.12){6}{\line(-1,0){0.17}}
\multiput(127.01,4.17)(-0.17,-0.11){6}{\line(-1,0){0.17}}
\multiput(125.97,3.51)(-0.18,-0.10){6}{\line(-1,0){0.18}}
\multiput(124.92,2.89)(-0.22,-0.12){5}{\line(-1,0){0.22}}
\multiput(123.83,2.31)(-0.22,-0.11){5}{\line(-1,0){0.22}}
\multiput(122.73,1.78)(-0.23,-0.10){5}{\line(-1,0){0.23}}
\multiput(121.60,1.29)(-0.29,-0.11){4}{\line(-1,0){0.29}}
\multiput(120.46,0.85)(-0.29,-0.10){4}{\line(-1,0){0.29}}
\multiput(119.30,0.45)(-0.39,-0.12){3}{\line(-1,0){0.39}}
\multiput(118.12,0.10)(-0.40,-0.10){3}{\line(-1,0){0.40}}
\multiput(116.93,-0.21)(-0.40,-0.09){3}{\line(-1,0){0.40}}
\multiput(115.73,-0.46)(-0.60,-0.10){2}{\line(-1,0){0.60}}
\multiput(114.52,-0.67)(-0.61,-0.08){2}{\line(-1,0){0.61}}
\put(113.31,-0.83){\line(-1,0){1.22}}
\put(112.08,-0.94){\line(-1,0){1.23}}
\put(110.86,-1.00){\line(-1,0){1.23}}
\put(109.63,-1.01){\line(-1,0){1.23}}
\put(108.41,-0.98){\line(-1,0){1.22}}
\multiput(107.18,-0.89)(-0.61,0.07){2}{\line(-1,0){0.61}}
\multiput(105.96,-0.76)(-0.61,0.09){2}{\line(-1,0){0.61}}
\multiput(104.75,-0.58)(-0.60,0.12){2}{\line(-1,0){0.60}}
\multiput(103.54,-0.35)(-0.40,0.09){3}{\line(-1,0){0.40}}
\multiput(102.35,-0.07)(-0.39,0.11){3}{\line(-1,0){0.39}}
\multiput(101.16,0.26)(-0.29,0.09){4}{\line(-1,0){0.29}}
\multiput(100.00,0.63)(-0.29,0.10){4}{\line(-1,0){0.29}}
\multiput(98.84,1.05)(-0.28,0.12){4}{\line(-1,0){0.28}}
\multiput(97.71,1.51)(-0.22,0.10){5}{\line(-1,0){0.22}}
\multiput(96.59,2.02)(-0.22,0.11){5}{\line(-1,0){0.22}}
\multiput(95.50,2.58)(-0.21,0.12){5}{\line(-1,0){0.21}}
\multiput(94.42,3.17)(-0.17,0.11){6}{\line(-1,0){0.17}}
\multiput(93.38,3.81)(-0.17,0.11){6}{\line(-1,0){0.17}}
\multiput(92.35,4.49)(-0.14,0.10){7}{\line(-1,0){0.14}}
\multiput(91.36,5.21)(-0.14,0.11){7}{\line(-1,0){0.14}}
\multiput(90.40,5.97)(-0.13,0.11){7}{\line(-1,0){0.13}}
\multiput(89.47,6.77)(-0.13,0.12){7}{\line(-1,0){0.13}}
\multiput(88.57,7.60)(-0.11,0.11){8}{\line(0,1){0.11}}
\multiput(87.70,8.47)(-0.12,0.13){7}{\line(0,1){0.13}}
\multiput(86.87,9.38)(-0.11,0.13){7}{\line(0,1){0.13}}
\multiput(86.07,10.31)(-0.11,0.14){7}{\line(0,1){0.14}}
\multiput(85.32,11.28)(-0.12,0.17){6}{\line(0,1){0.17}}
\multiput(84.60,12.27)(-0.11,0.17){6}{\line(0,1){0.17}}
\multiput(83.92,13.30)(-0.11,0.17){6}{\line(0,1){0.17}}
\multiput(83.29,14.35)(-0.12,0.21){5}{\line(0,1){0.21}}
\multiput(82.70,15.42)(-0.11,0.22){5}{\line(0,1){0.22}}
\multiput(82.15,16.52)(-0.10,0.22){5}{\line(0,1){0.22}}
\multiput(81.64,17.64)(-0.12,0.28){4}{\line(0,1){0.28}}
\multiput(81.18,18.77)(-0.10,0.29){4}{\line(0,1){0.29}}
\multiput(80.77,19.93)(-0.09,0.29){4}{\line(0,1){0.29}}
\multiput(80.40,21.10)(-0.11,0.39){3}{\line(0,1){0.39}}
\multiput(80.08,22.28)(-0.09,0.40){3}{\line(0,1){0.40}}
\multiput(79.80,23.48)(-0.11,0.60){2}{\line(0,1){0.60}}
\multiput(79.57,24.68)(-0.09,0.61){2}{\line(0,1){0.61}}
\multiput(79.40,25.90)(-0.06,0.61){2}{\line(0,1){0.61}}
\put(79.27,27.12){\line(0,1){1.22}}
\put(79.19,28.34){\line(0,1){1.23}}
\put(79.16,29.57){\line(0,1){1.23}}
\put(79.17,30.80){\line(0,1){1.23}}
\put(79.24,32.02){\line(0,1){1.22}}
\multiput(79.35,33.24)(0.08,0.61){2}{\line(0,1){0.61}}
\multiput(79.52,34.46)(0.11,0.60){2}{\line(0,1){0.60}}
\multiput(79.73,35.67)(0.09,0.40){3}{\line(0,1){0.40}}
\multiput(79.99,36.87)(0.10,0.40){3}{\line(0,1){0.40}}
\multiput(80.30,38.05)(0.12,0.39){3}{\line(0,1){0.39}}
\multiput(80.65,39.23)(0.10,0.29){4}{\line(0,1){0.29}}
\multiput(81.05,40.39)(0.11,0.29){4}{\line(0,1){0.29}}
\multiput(81.50,41.53)(0.10,0.22){5}{\line(0,1){0.22}}
\multiput(81.99,42.65)(0.11,0.22){5}{\line(0,1){0.22}}
\multiput(82.53,43.76)(0.12,0.22){5}{\line(0,1){0.22}}
\multiput(83.11,44.84)(0.10,0.18){6}{\line(0,1){0.18}}
\multiput(83.73,45.90)(0.11,0.17){6}{\line(0,1){0.17}}
\multiput(84.40,46.93)(0.12,0.17){6}{\line(0,1){0.17}}
\multiput(85.10,47.93)(0.11,0.14){7}{\line(0,1){0.14}}
\multiput(85.85,48.91)(0.11,0.13){7}{\line(0,1){0.13}}
\multiput(86.63,49.85)(0.12,0.13){7}{\line(0,1){0.13}}
\multiput(87.45,50.76)(0.11,0.11){8}{\line(0,1){0.11}}
\multiput(88.31,51.64)(0.11,0.11){8}{\line(1,0){0.11}}
\multiput(89.20,52.49)(0.13,0.12){7}{\line(1,0){0.13}}
\multiput(90.12,53.30)(0.14,0.11){7}{\line(1,0){0.14}}
\multiput(91.07,54.07)(0.14,0.10){7}{\line(1,0){0.14}}
\multiput(92.06,54.80)(0.17,0.12){6}{\line(1,0){0.17}}
\multiput(93.07,55.49)(0.17,0.11){6}{\line(1,0){0.17}}
\multiput(94.11,56.14)(0.18,0.10){6}{\line(1,0){0.18}}
\multiput(95.18,56.75)(0.22,0.11){5}{\line(1,0){0.22}}
\multiput(96.27,57.32)(0.22,0.10){5}{\line(1,0){0.22}}
\multiput(97.38,57.84)(0.28,0.12){4}{\line(1,0){0.28}}
\multiput(98.51,58.32)(0.29,0.11){4}{\line(1,0){0.29}}
\multiput(99.65,58.75)(0.29,0.10){4}{\line(1,0){0.29}}
\multiput(100.82,59.14)(0.39,0.11){3}{\line(1,0){0.39}}
\multiput(102.00,59.48)(0.40,0.10){3}{\line(1,0){0.40}}
\multiput(103.19,59.77)(0.40,0.08){3}{\line(1,0){0.40}}
\multiput(104.39,60.01)(0.61,0.10){2}{\line(1,0){0.61}}
\multiput(105.60,60.21)(0.61,0.07){2}{\line(1,0){0.61}}
\put(106.82,60.36){\line(1,0){1.22}}
\put(108.04,60.46){\line(1,0){1.88}}
%\end
%\circle(189.58,29.75){61.53}
\put(189.58,60.52){\line(1,0){1.23}}
\put(190.81,60.49){\line(1,0){1.22}}
\multiput(192.03,60.42)(0.61,-0.06){2}{\line(1,0){0.61}}
\multiput(193.25,60.30)(0.61,-0.09){2}{\line(1,0){0.61}}
\multiput(194.47,60.12)(0.60,-0.11){2}{\line(1,0){0.60}}
\multiput(195.67,59.91)(0.40,-0.09){3}{\line(1,0){0.40}}
\multiput(196.87,59.64)(0.40,-0.10){3}{\line(1,0){0.40}}
\multiput(198.06,59.32)(0.29,-0.09){4}{\line(1,0){0.29}}
\multiput(199.23,58.96)(0.29,-0.10){4}{\line(1,0){0.29}}
\multiput(200.39,58.55)(0.28,-0.11){4}{\line(1,0){0.28}}
\multiput(201.53,58.10)(0.22,-0.10){5}{\line(1,0){0.22}}
\multiput(202.65,57.60)(0.22,-0.11){5}{\line(1,0){0.22}}
\multiput(203.75,57.06)(0.22,-0.12){5}{\line(1,0){0.22}}
\multiput(204.83,56.47)(0.18,-0.10){6}{\line(1,0){0.18}}
\multiput(205.88,55.84)(0.17,-0.11){6}{\line(1,0){0.17}}
\multiput(206.91,55.17)(0.17,-0.12){6}{\line(1,0){0.17}}
\multiput(207.91,54.46)(0.14,-0.11){7}{\line(1,0){0.14}}
\multiput(208.88,53.71)(0.13,-0.11){7}{\line(1,0){0.13}}
\multiput(209.82,52.92)(0.13,-0.12){7}{\line(1,0){0.13}}
\multiput(210.73,52.10)(0.11,-0.11){8}{\line(1,0){0.11}}
\multiput(211.60,51.23)(0.12,-0.13){7}{\line(0,-1){0.13}}
\multiput(212.44,50.34)(0.11,-0.13){7}{\line(0,-1){0.13}}
\multiput(213.24,49.41)(0.11,-0.14){7}{\line(0,-1){0.14}}
\multiput(214.01,48.45)(0.10,-0.14){7}{\line(0,-1){0.14}}
\multiput(214.73,47.46)(0.11,-0.17){6}{\line(0,-1){0.17}}
\multiput(215.42,46.45)(0.11,-0.17){6}{\line(0,-1){0.17}}
\multiput(216.07,45.40)(0.10,-0.18){6}{\line(0,-1){0.18}}
\multiput(216.67,44.33)(0.11,-0.22){5}{\line(0,-1){0.22}}
\multiput(217.23,43.24)(0.10,-0.22){5}{\line(0,-1){0.22}}
\multiput(217.74,42.13)(0.12,-0.28){4}{\line(0,-1){0.28}}
\multiput(218.22,41.00)(0.11,-0.29){4}{\line(0,-1){0.29}}
\multiput(218.64,39.85)(0.09,-0.29){4}{\line(0,-1){0.29}}
\multiput(219.02,38.68)(0.11,-0.39){3}{\line(0,-1){0.39}}
\multiput(219.35,37.50)(0.10,-0.40){3}{\line(0,-1){0.40}}
\multiput(219.64,36.30)(0.12,-0.60){2}{\line(0,-1){0.60}}
\multiput(219.88,35.10)(0.09,-0.61){2}{\line(0,-1){0.61}}
\multiput(220.07,33.89)(0.07,-0.61){2}{\line(0,-1){0.61}}
\put(220.21,32.67){\line(0,-1){1.22}}
\put(220.30,31.45){\line(0,-1){1.23}}
\put(220.34,30.22){\line(0,-1){1.23}}
\put(220.34,28.99){\line(0,-1){1.23}}
\put(220.28,27.77){\line(0,-1){1.22}}
\multiput(220.18,26.54)(-0.08,-0.61){2}{\line(0,-1){0.61}}
\multiput(220.03,25.33)(-0.10,-0.61){2}{\line(0,-1){0.61}}
\multiput(219.82,24.12)(-0.08,-0.40){3}{\line(0,-1){0.40}}
\multiput(219.58,22.91)(-0.10,-0.40){3}{\line(0,-1){0.40}}
\multiput(219.28,21.72)(-0.11,-0.39){3}{\line(0,-1){0.39}}
\multiput(218.94,20.55)(-0.10,-0.29){4}{\line(0,-1){0.29}}
\multiput(218.55,19.38)(-0.11,-0.29){4}{\line(0,-1){0.29}}
\multiput(218.11,18.24)(-0.10,-0.23){5}{\line(0,-1){0.23}}
\multiput(217.63,17.11)(-0.11,-0.22){5}{\line(0,-1){0.22}}
\multiput(217.10,16.00)(-0.11,-0.22){5}{\line(0,-1){0.22}}
\multiput(216.53,14.91)(-0.10,-0.18){6}{\line(0,-1){0.18}}
\multiput(215.92,13.85)(-0.11,-0.17){6}{\line(0,-1){0.17}}
\multiput(215.26,12.81)(-0.12,-0.17){6}{\line(0,-1){0.17}}
\multiput(214.57,11.80)(-0.11,-0.14){7}{\line(0,-1){0.14}}
\multiput(213.83,10.82)(-0.11,-0.14){7}{\line(0,-1){0.14}}
\multiput(213.06,9.87)(-0.12,-0.13){7}{\line(0,-1){0.13}}
\multiput(212.25,8.95)(-0.11,-0.11){8}{\line(0,-1){0.11}}
\multiput(211.40,8.06)(-0.11,-0.11){8}{\line(-1,0){0.11}}
\multiput(210.52,7.21)(-0.13,-0.12){7}{\line(-1,0){0.13}}
\multiput(209.60,6.39)(-0.14,-0.11){7}{\line(-1,0){0.14}}
\multiput(208.65,5.61)(-0.14,-0.11){7}{\line(-1,0){0.14}}
\multiput(207.68,4.87)(-0.17,-0.12){6}{\line(-1,0){0.17}}
\multiput(206.67,4.17)(-0.17,-0.11){6}{\line(-1,0){0.17}}
\multiput(205.63,3.51)(-0.18,-0.10){6}{\line(-1,0){0.18}}
\multiput(204.58,2.89)(-0.22,-0.12){5}{\line(-1,0){0.22}}
\multiput(203.49,2.31)(-0.22,-0.11){5}{\line(-1,0){0.22}}
\multiput(202.39,1.78)(-0.23,-0.10){5}{\line(-1,0){0.23}}
\multiput(201.26,1.29)(-0.29,-0.11){4}{\line(-1,0){0.29}}
\multiput(200.12,0.85)(-0.29,-0.10){4}{\line(-1,0){0.29}}
\multiput(198.96,0.45)(-0.39,-0.12){3}{\line(-1,0){0.39}}
\multiput(197.78,0.10)(-0.40,-0.10){3}{\line(-1,0){0.40}}
\multiput(196.59,-0.21)(-0.40,-0.09){3}{\line(-1,0){0.40}}
\multiput(195.39,-0.46)(-0.60,-0.10){2}{\line(-1,0){0.60}}
\multiput(194.18,-0.67)(-0.61,-0.08){2}{\line(-1,0){0.61}}
\put(192.97,-0.83){\line(-1,0){1.22}}
\put(191.74,-0.94){\line(-1,0){1.23}}
\put(190.52,-1.00){\line(-1,0){1.23}}
\put(189.29,-1.01){\line(-1,0){1.23}}
\put(188.07,-0.98){\line(-1,0){1.22}}
\multiput(186.84,-0.89)(-0.61,0.07){2}{\line(-1,0){0.61}}
\multiput(185.62,-0.76)(-0.61,0.09){2}{\line(-1,0){0.61}}
\multiput(184.41,-0.58)(-0.60,0.12){2}{\line(-1,0){0.60}}
\multiput(183.20,-0.35)(-0.40,0.09){3}{\line(-1,0){0.40}}
\multiput(182.01,-0.07)(-0.39,0.11){3}{\line(-1,0){0.39}}
\multiput(180.82,0.26)(-0.29,0.09){4}{\line(-1,0){0.29}}
\multiput(179.66,0.63)(-0.29,0.10){4}{\line(-1,0){0.29}}
\multiput(178.50,1.05)(-0.28,0.12){4}{\line(-1,0){0.28}}
\multiput(177.37,1.51)(-0.22,0.10){5}{\line(-1,0){0.22}}
\multiput(176.25,2.02)(-0.22,0.11){5}{\line(-1,0){0.22}}
\multiput(175.16,2.58)(-0.21,0.12){5}{\line(-1,0){0.21}}
\multiput(174.08,3.17)(-0.17,0.11){6}{\line(-1,0){0.17}}
\multiput(173.04,3.81)(-0.17,0.11){6}{\line(-1,0){0.17}}
\multiput(172.01,4.49)(-0.14,0.10){7}{\line(-1,0){0.14}}
\multiput(171.02,5.21)(-0.14,0.11){7}{\line(-1,0){0.14}}
\multiput(170.06,5.97)(-0.13,0.11){7}{\line(-1,0){0.13}}
\multiput(169.13,6.77)(-0.13,0.12){7}{\line(-1,0){0.13}}
\multiput(168.23,7.60)(-0.11,0.11){8}{\line(0,1){0.11}}
\multiput(167.36,8.47)(-0.12,0.13){7}{\line(0,1){0.13}}
\multiput(166.53,9.38)(-0.11,0.13){7}{\line(0,1){0.13}}
\multiput(165.73,10.31)(-0.11,0.14){7}{\line(0,1){0.14}}
\multiput(164.98,11.28)(-0.12,0.17){6}{\line(0,1){0.17}}
\multiput(164.26,12.27)(-0.11,0.17){6}{\line(0,1){0.17}}
\multiput(163.58,13.30)(-0.11,0.17){6}{\line(0,1){0.17}}
\multiput(162.95,14.35)(-0.12,0.21){5}{\line(0,1){0.21}}
\multiput(162.36,15.42)(-0.11,0.22){5}{\line(0,1){0.22}}
\multiput(161.81,16.52)(-0.10,0.22){5}{\line(0,1){0.22}}
\multiput(161.30,17.64)(-0.12,0.28){4}{\line(0,1){0.28}}
\multiput(160.84,18.77)(-0.10,0.29){4}{\line(0,1){0.29}}
\multiput(160.43,19.93)(-0.09,0.29){4}{\line(0,1){0.29}}
\multiput(160.06,21.10)(-0.11,0.39){3}{\line(0,1){0.39}}
\multiput(159.74,22.28)(-0.09,0.40){3}{\line(0,1){0.40}}
\multiput(159.46,23.48)(-0.11,0.60){2}{\line(0,1){0.60}}
\multiput(159.23,24.68)(-0.09,0.61){2}{\line(0,1){0.61}}
\multiput(159.06,25.90)(-0.06,0.61){2}{\line(0,1){0.61}}
\put(158.93,27.12){\line(0,1){1.22}}
\put(158.85,28.34){\line(0,1){1.23}}
\put(158.82,29.57){\line(0,1){1.23}}
\put(158.83,30.80){\line(0,1){1.23}}
\put(158.90,32.02){\line(0,1){1.22}}
\multiput(159.01,33.24)(0.08,0.61){2}{\line(0,1){0.61}}
\multiput(159.18,34.46)(0.11,0.60){2}{\line(0,1){0.60}}
\multiput(159.39,35.67)(0.09,0.40){3}{\line(0,1){0.40}}
\multiput(159.65,36.87)(0.10,0.40){3}{\line(0,1){0.40}}
\multiput(159.96,38.05)(0.12,0.39){3}{\line(0,1){0.39}}
\multiput(160.31,39.23)(0.10,0.29){4}{\line(0,1){0.29}}
\multiput(160.71,40.39)(0.11,0.29){4}{\line(0,1){0.29}}
\multiput(161.16,41.53)(0.10,0.22){5}{\line(0,1){0.22}}
\multiput(161.65,42.65)(0.11,0.22){5}{\line(0,1){0.22}}
\multiput(162.19,43.76)(0.12,0.22){5}{\line(0,1){0.22}}
\multiput(162.77,44.84)(0.10,0.18){6}{\line(0,1){0.18}}
\multiput(163.39,45.90)(0.11,0.17){6}{\line(0,1){0.17}}
\multiput(164.06,46.93)(0.12,0.17){6}{\line(0,1){0.17}}
\multiput(164.76,47.93)(0.11,0.14){7}{\line(0,1){0.14}}
\multiput(165.51,48.91)(0.11,0.13){7}{\line(0,1){0.13}}
\multiput(166.29,49.85)(0.12,0.13){7}{\line(0,1){0.13}}
\multiput(167.11,50.76)(0.11,0.11){8}{\line(0,1){0.11}}
\multiput(167.97,51.64)(0.11,0.11){8}{\line(1,0){0.11}}
\multiput(168.86,52.49)(0.13,0.12){7}{\line(1,0){0.13}}
\multiput(169.78,53.30)(0.14,0.11){7}{\line(1,0){0.14}}
\multiput(170.73,54.07)(0.14,0.10){7}{\line(1,0){0.14}}
\multiput(171.72,54.80)(0.17,0.12){6}{\line(1,0){0.17}}
\multiput(172.73,55.49)(0.17,0.11){6}{\line(1,0){0.17}}
\multiput(173.77,56.14)(0.18,0.10){6}{\line(1,0){0.18}}
\multiput(174.84,56.75)(0.22,0.11){5}{\line(1,0){0.22}}
\multiput(175.93,57.32)(0.22,0.10){5}{\line(1,0){0.22}}
\multiput(177.04,57.84)(0.28,0.12){4}{\line(1,0){0.28}}
\multiput(178.17,58.32)(0.29,0.11){4}{\line(1,0){0.29}}
\multiput(179.31,58.75)(0.29,0.10){4}{\line(1,0){0.29}}
\multiput(180.48,59.14)(0.39,0.11){3}{\line(1,0){0.39}}
\multiput(181.66,59.48)(0.40,0.10){3}{\line(1,0){0.40}}
\multiput(182.85,59.77)(0.40,0.08){3}{\line(1,0){0.40}}
\multiput(184.05,60.01)(0.61,0.10){2}{\line(1,0){0.61}}
\multiput(185.26,60.21)(0.61,0.07){2}{\line(1,0){0.61}}
\put(186.48,60.36){\line(1,0){1.22}}
\put(187.70,60.46){\line(1,0){1.88}}
%\end
\put(189.58,30.25){\line(0,1){30.50}}
\put(189.58,68.50){\makebox(0,0)[cc]{$a$}}
\put(133.61,62.94){\makebox(0,0)[cc]{$b$}}
\put(213.28,62.94){\makebox(0,0)[cc]{$b$}}
\put(182.83,13.50){\makebox(0,0)[cc]{$-$}}
\put(210.08,40.50){\makebox(0,0)[cc]{$-$}}
\put(165.08,38.00){\makebox(0,0)[cc]{$+$}}
\put(211.58,21.25){\makebox(0,0)[cc]{$+$}}
%\emline(84.33,46.33)(135.67,13.00)
\multiput(84.33,46.33)(0.18,-0.12){278}{\line(1,0){0.18}}
%\end
%\emline(164.00,46.33)(215.33,13.00)
\multiput(164.00,46.33)(0.18,-0.12){278}{\line(1,0){0.18}}
%\end
%\emline(159.33,30.00)(220.33,30.00)
\put(159.33,30.00){\line(1,0){61.00}}
%\end
%\emline(110.00,30.00)(128.33,54.00)
\multiput(110.00,30.00)(0.12,0.16){153}{\line(0,1){0.16}}
%\end
%\emline(189.44,30.00)(207.78,54.00)
\multiput(189.44,30.00)(0.12,0.16){153}{\line(0,1){0.16}}
%\end
\put(18.89,42.78){\makebox(0,0)[cc]{$+$}}
\put(29.44,12.22){\makebox(0,0)[cc]{$-$}}
\put(110.56,46.67){\makebox(0,0)[cc]{$-$}}
\put(99.44,17.22){\makebox(0,0)[cc]{$+$}}
\bezier{44}(189.44,45.00)(195.00,46.11)(198.89,42.22)
\bezier{56}(171.67,30.00)(167.78,36.11)(172.78,40.00)
\bezier{60}(206.11,30.00)(210.00,23.33)(202.78,21.11)
\end{picture}


 }




\frame[shrink=2]{
\frametitle{Quantum correlations for two-particle singlet state}

$$E (\theta ) = {3/[j(j+1)]} C(\theta )$$
with non-normalized
{\footnotesize
\begin{eqnarray}
C(\theta )&=&
\langle J= 0 ,M= 0\mid \alpha \cdot \hat{J}^A \otimes \beta \cdot
\hat{J}^B\mid
J=0,M=0\rangle
\nonumber
 \\
&=&
 \sum_{m,m'}
\langle  00 \mid jm,j-m \rangle
\langle  jm',j-m'\mid 00 \rangle \times \nonumber
\\
&&
\qquad
\qquad
\qquad
\times
^A\langle jm\mid  ^B\langle j-m\mid
\alpha \cdot \hat{J}^A \otimes \beta \cdot \hat{J}^B
\mid jm'\rangle ^A \mid j-m'\rangle ^B  \nonumber
\\
&=&
 \sum_{m,m'}
\langle  00 \mid jm,j-m \rangle
\langle  jm',j-m'\mid 00 \rangle       \times \nonumber  \\
&&
\qquad
\qquad
\qquad   \times
\langle jm\mid
\alpha \cdot \hat{J}^{A}
\mid jm'\rangle
 \langle j-m\mid
 \beta \cdot \hat{J}^B
 \mid j-m'\rangle                             \nonumber
\\
&=&
 \sum_{m,m'}
{(-1)^{j-m}(-1)^{j-m'}\over 2j+1}
\langle jm\mid
 \hat{J}^A_z
\mid jm'\rangle
\langle j-m\mid
 \beta \cdot \hat{J}^B
 \mid j-m'\rangle \nonumber
\\
&=&
 \sum_{m,m'}
{(-1)^{j-m}(-1)^{j-m'}\over 2j+1}
m \delta _{m m'}
\langle j-m\mid
 \beta \cdot \hat{J}^B
 \mid j-m'\rangle \nonumber
\\
&=&  \sum_m
m{(-1)^{2j-2m}\over 2j+1}
 \langle j-m\mid
 \beta \cdot \hat{J}^B
 \mid j-m\rangle  = {1\over 2j+1}  \sum_m
-m^2 \beta_z      = -{1 \over 2j+1}\cos \theta  \sum_{m=-j}^j
m^2
\qquad {\rm for} \; 0\le \theta \le \pi
 \nonumber
\\
&=&- {j(j+1)\over 3} \, \cos \theta
\qquad {\rm for} \; 0\le \theta \le \pi
\quad .
 \nonumber
\end{eqnarray}
}
}




\frame[shrink=1.2]{
\frametitle{Two-particle  correlations cntd.}
\begin{center}
%TexCad Options
%\grade{\off}
%\emlines{\off}
%\beziermacro{\off}
%\reduce{\on}
%\snapping{\off}
%\quality{4.00}
%\graddiff{0.01}
%\snapasp{1}
%\zoom{1.00}
\unitlength 0.500mm
\linethickness{0.4pt}
\begin{picture}(102.00,102.00)
%\emline(10.00,10.00)(10.00,100.00)
\put(10.00,10.00){\line(0,1){90.00}}
%\end
\put(10.00,55.00){\line(1,0){45.00}}
\put(55.00,55.00){\line(1,0){45.00}}
\put(10.00,10.00){\line(1,1){90.00}}
\put(100.00,100.00){\line(-1,0){45.00}}
\put(10.00,10.00){\line(1,0){45.00}}
%\bezier{284}(10.00,10.00)(30.00,10.00)(55.00,55.00)
\put(10.00,10.00){\line(1,0){1.41}}
\multiput(11.41,10.06)(0.71,0.08){2}{\line(1,0){0.71}}
\multiput(12.84,10.22)(0.48,0.09){3}{\line(1,0){0.48}}
\multiput(14.28,10.50)(0.36,0.10){4}{\line(1,0){0.36}}
\multiput(15.73,10.89)(0.29,0.10){5}{\line(1,0){0.29}}
\multiput(17.20,11.39)(0.25,0.10){6}{\line(1,0){0.25}}
\multiput(18.67,12.01)(0.21,0.10){7}{\line(1,0){0.21}}
\multiput(20.16,12.73)(0.21,0.12){7}{\line(1,0){0.21}}
\multiput(21.66,13.57)(0.19,0.12){8}{\line(1,0){0.19}}
\multiput(23.18,14.52)(0.17,0.12){9}{\line(1,0){0.17}}
\multiput(24.70,15.58)(0.15,0.12){10}{\line(1,0){0.15}}
\multiput(26.24,16.75)(0.14,0.12){11}{\line(1,0){0.14}}
\multiput(27.79,18.03)(0.13,0.12){12}{\line(1,0){0.13}}
\multiput(29.36,19.43)(0.12,0.12){13}{\line(1,0){0.12}}
\multiput(30.93,20.94)(0.11,0.12){14}{\line(0,1){0.12}}
\multiput(32.52,22.55)(0.11,0.12){14}{\line(0,1){0.12}}
\multiput(34.12,24.28)(0.12,0.13){14}{\line(0,1){0.13}}
\multiput(35.74,26.12)(0.12,0.14){14}{\line(0,1){0.14}}
\multiput(37.36,28.08)(0.12,0.15){14}{\line(0,1){0.15}}
\multiput(39.00,30.14)(0.12,0.16){14}{\line(0,1){0.16}}
\multiput(40.65,32.32)(0.12,0.16){14}{\line(0,1){0.16}}
\multiput(42.31,34.60)(0.12,0.17){14}{\line(0,1){0.17}}
\multiput(43.99,37.00)(0.11,0.17){15}{\line(0,1){0.17}}
\multiput(45.67,39.51)(0.11,0.17){15}{\line(0,1){0.17}}
\multiput(47.37,42.14)(0.11,0.18){15}{\line(0,1){0.18}}
\multiput(49.09,44.87)(0.11,0.19){15}{\line(0,1){0.19}}
\multiput(50.81,47.72)(0.12,0.20){15}{\line(0,1){0.20}}
\multiput(52.55,50.67)(0.12,0.21){21}{\line(0,1){0.21}}
%\end
%\bezier{284}(55.00,55.00)(80.00,100.00)(100.00,100.00)
\multiput(55.00,55.00)(0.12,0.21){15}{\line(0,1){0.21}}
\multiput(56.75,58.11)(0.12,0.20){15}{\line(0,1){0.20}}
\multiput(58.50,61.11)(0.12,0.19){15}{\line(0,1){0.19}}
\multiput(60.23,64.00)(0.11,0.19){15}{\line(0,1){0.19}}
\multiput(61.94,66.78)(0.11,0.18){15}{\line(0,1){0.18}}
\multiput(63.65,69.45)(0.11,0.17){15}{\line(0,1){0.17}}
\multiput(65.34,72.01)(0.12,0.17){14}{\line(0,1){0.17}}
\multiput(67.02,74.45)(0.12,0.17){14}{\line(0,1){0.17}}
\multiput(68.69,76.78)(0.12,0.16){14}{\line(0,1){0.16}}
\multiput(70.34,79.00)(0.12,0.15){14}{\line(0,1){0.15}}
\multiput(71.99,81.11)(0.12,0.14){14}{\line(0,1){0.14}}
\multiput(73.62,83.11)(0.12,0.13){14}{\line(0,1){0.13}}
\multiput(75.23,84.99)(0.11,0.13){14}{\line(0,1){0.13}}
\multiput(76.84,86.77)(0.11,0.12){14}{\line(0,1){0.12}}
\multiput(78.43,88.43)(0.12,0.12){13}{\line(1,0){0.12}}
\multiput(80.01,89.98)(0.13,0.12){12}{\line(1,0){0.13}}
\multiput(81.58,91.42)(0.13,0.11){12}{\line(1,0){0.13}}
\multiput(83.14,92.75)(0.14,0.11){11}{\line(1,0){0.14}}
\multiput(84.68,93.97)(0.15,0.11){10}{\line(1,0){0.15}}
\multiput(86.21,95.07)(0.17,0.11){9}{\line(1,0){0.17}}
\multiput(87.73,96.06)(0.19,0.11){8}{\line(1,0){0.19}}
\multiput(89.24,96.94)(0.21,0.11){7}{\line(1,0){0.21}}
\multiput(90.73,97.71)(0.25,0.11){6}{\line(1,0){0.25}}
\multiput(92.21,98.37)(0.29,0.11){5}{\line(1,0){0.29}}
\multiput(93.68,98.92)(0.36,0.11){4}{\line(1,0){0.36}}
\multiput(95.14,99.36)(0.48,0.11){3}{\line(1,0){0.48}}
\multiput(96.58,99.68)(0.72,0.11){2}{\line(1,0){0.72}}
\put(98.02,99.89){\line(1,0){1.98}}
%\end
\put(5.00,100.00){\makebox(0,0)[cc]{$+1$}}
\put(5.00,10.00){\makebox(0,0)[cc]{$-1$}}
\put(5.00,55.00){\makebox(0,0)[cc]{$0$}}
\put(60.00,50.00){\makebox(0,0)[cc]{$\pi /2$}}
\put(100.00,50.00){\makebox(0,0)[cc]{$\pi$}}
\put(102.00,59.00){\makebox(0,0)[cc]{$\theta$}}
\put(14.00,102.00){\makebox(0,0)[cc]{$E$}}
\put(22.00,38.00){\makebox(0,0)[cc]{$E_c(\theta )$}}
\put(55.00,28.00){\makebox(0,0)[cc]{$E_{qm}(\theta )$}}
\put(35.00,13.00){\makebox(0,0)[cc]{$E_s(\theta )$}}
\put(55.00,55.00){\circle*{2.00}}
\end{picture}
\end{center}
More anti-coincidences of detector clicks between $0< \theta < \pi /2$;\\
more coincidences of detector clicks between $\pi /2< \theta < \pi $;    \\
same-as-classical and quantum for $\theta = 0,\pi /2 ,\pi$.
 }




%%%%%%%%%%%%%%%%%%%%%%%%%%%%%%%%%%%%%%%%%%%%%%%%%%%%%%%%%%%%%%%%%%%%%%%%%%%%%%%%%%%%

%\section{}
\frame[shrink=1.2]{
\frametitle{Boole-Bell-type inequalities}

\begin{itemize}
\item<+->
Would you believe that
(i) it rains in Vienna with probability 80\%;
(ii) it rains in Budapest with probability 80\%;
(i)\&(ii) it rains in Vienna \& jointly in Budapest with probability 0.1\% ?
Exactly when would you start believing me?

\item<+->
Around 1860 Boole: ``conditions of possible experience'' (in ``Laws of Thought'')

\item<+->
Around 1965 Bell: similar inequalities as classical bounds for probabilities of joint events.

\item<+->
Pitowsky \& others: geometric interpretation as ``inside--outside'' conditions with regards to faces
of correlation polytopes: Take all possibilities of classical events. Take their joints.
Interpret the entries in the truth tables as vectors in a vector space.
These vectors form the vertices of a ``correlation polytope'' formed by the convex sum.
The surface of this polytope represents all classical probability distributions.
The faces of this polytope form  the inside--outside relations. They are represented by Boole-Bell inequalities.

\end{itemize}

}


\frame[shrink=1.2]{
\frametitle{Kochen-Specker theorem \& quantum ``meaning''}

\begin{itemize}
\item<+->
``It is impossible to consistently (re)construct an entire set of
quantum properties from its parts.
Therefore, a comprehensive list of `elements of physical reality'
cannot exist.''

Simon Kochen and Ernst P. Specker,
``The Problem of Hidden Variables in Quantum Mechanics,''
Journal of Mathematics and Mechanics 17(1), pp.59-87 (1967)

Review in Karl Svozil, ``Quantum Logic,''
(Springer, Singapore,1998)

\item<+->
Feynman: ``Nobody understands quantum mechanics.'' (in ``The Character of Physical Law'')

\item<+->
Is it useless to  even think about possible interpretations of the formalism;
even more so to go beyond the quantum?
Will the human mine ever transcend the quantum world?
Strong anti-rationalist tendencies (Bohr, Heisenberg,... versus Einstein, Schr{\"o}dinger, De Broglie, ...).


\end{itemize}
}


%%%%%%%%%%%%%%%%%%%%%%%%%%%%%%%%%%%%%%%%%%%%%%%%%%%%%%%%%%%%%%%%%%%%%%%%%%%%%%%%%%%%
%%%%%%%%%%%%%%%%%%%%%% PART II



\section{Quantum computation}




%\section{}
\subsection{No-cloning (no-copy) theorem}
\frame[shrink=1.2]{
\frametitle{Reversible, one-to-one computation}
%TexCad Options
%\grade{\off}
%\emlines{\off}
%\beziermacro{\off}
%\reduce{\on}
%\snapping{\off}
%\quality{0.20}
%\graddiff{0.01}
%\snapasp{1}
%\zoom{0.50}
\unitlength 0.50mm
\linethickness{0.4pt}
\begin{picture}(194.00,95.00)
\put(10.00,10.00){\circle*{2.00}}
\put(20.00,10.00){\circle*{2.00}}
\put(30.00,10.00){\circle*{2.00}}
\put(40.00,10.00){\circle*{2.00}}
\put(10.00,20.00){\circle{2.00}}
\put(10.00,11.00){\vector(0,1){8.00}}
\put(10.00,30.00){\circle{2.00}}
\put(10.00,21.00){\vector(0,1){8.00}}
\put(10.00,40.00){\circle{2.00}}
\put(10.00,31.00){\vector(0,1){8.00}}
\put(10.00,50.00){\circle{2.00}}
\put(10.00,41.00){\vector(0,1){8.00}}
\put(10.00,60.00){\circle{2.00}}
\put(10.00,51.00){\vector(0,1){8.00}}
\put(10.00,70.00){\circle{2.00}}
\put(10.00,61.00){\vector(0,1){8.00}}
\put(10.00,80.00){\circle{2.00}}
\put(10.00,71.00){\vector(0,1){8.00}}
\put(10.00,90.00){\circle{2.00}}
\put(10.00,81.00){\vector(0,1){8.00}}
\put(20.00,20.00){\circle{2.00}}
\put(20.00,11.00){\vector(0,1){8.00}}
\put(20.00,30.00){\circle{2.00}}
\put(20.00,21.00){\vector(0,1){8.00}}
\put(20.00,40.00){\circle{2.00}}
\put(20.00,31.00){\vector(0,1){8.00}}
\put(20.00,50.00){\circle{2.00}}
\put(20.00,41.00){\vector(0,1){8.00}}
\put(20.00,60.00){\circle{2.00}}
\put(20.00,51.00){\vector(0,1){8.00}}
\put(20.00,70.00){\circle{2.00}}
\put(20.00,61.00){\vector(0,1){8.00}}
\put(20.00,80.00){\circle{2.00}}
\put(20.00,71.00){\vector(0,1){8.00}}
\put(20.00,90.00){\circle{2.00}}
\put(20.00,81.00){\vector(0,1){8.00}}
\put(30.00,20.00){\circle{2.00}}
\put(30.00,11.00){\vector(0,1){8.00}}
\put(30.00,30.00){\circle{2.00}}
\put(30.00,21.00){\vector(0,1){8.00}}
\put(30.00,40.00){\circle{2.00}}
\put(30.00,31.00){\vector(0,1){8.00}}
\put(30.00,50.00){\circle{2.00}}
\put(30.00,41.00){\vector(0,1){8.00}}
\put(30.00,60.00){\circle{2.00}}
\put(30.00,51.00){\vector(0,1){8.00}}
\put(30.00,70.00){\circle{2.00}}
\put(30.00,61.00){\vector(0,1){8.00}}
\put(30.00,80.00){\circle{2.00}}
\put(30.00,71.00){\vector(0,1){8.00}}
\put(30.00,90.00){\circle{2.00}}
\put(30.00,81.00){\vector(0,1){8.00}}
\put(40.00,20.00){\circle{2.00}}
\put(40.00,11.00){\vector(0,1){8.00}}
\put(40.00,30.00){\circle{2.00}}
\put(40.00,21.00){\vector(0,1){8.00}}
\put(40.00,40.00){\circle{2.00}}
\put(40.00,31.00){\vector(0,1){8.00}}
\put(40.00,50.00){\circle{2.00}}
\put(40.00,41.00){\vector(0,1){8.00}}
\put(40.00,60.00){\circle{2.00}}
\put(40.00,51.00){\vector(0,1){8.00}}
\put(40.00,70.00){\circle{2.00}}
\put(40.00,61.00){\vector(0,1){8.00}}
\put(40.00,80.00){\circle{2.00}}
\put(40.00,71.00){\vector(0,1){8.00}}
\put(40.00,90.00){\circle{2.00}}
\put(40.00,81.00){\vector(0,1){8.00}}
\put(50.00,10.00){\circle*{2.00}}
\put(50.00,20.00){\circle{2.00}}
\put(50.00,11.00){\vector(0,1){8.00}}
\put(50.00,30.00){\circle{2.00}}
\put(50.00,21.00){\vector(0,1){8.00}}
\put(50.00,40.00){\circle{2.00}}
\put(50.00,31.00){\vector(0,1){8.00}}
\put(50.00,50.00){\circle{2.00}}
\put(50.00,41.00){\vector(0,1){8.00}}
\put(50.00,60.00){\circle{2.00}}
\put(50.00,51.00){\vector(0,1){8.00}}
\put(50.00,70.00){\circle{2.00}}
\put(50.00,61.00){\vector(0,1){8.00}}
\put(50.00,80.00){\circle{2.00}}
\put(50.00,71.00){\vector(0,1){8.00}}
\put(50.00,90.00){\circle{2.00}}
\put(50.00,81.00){\vector(0,1){8.00}}
\put(80.00,10.00){\circle*{2.00}}
\put(90.00,10.00){\circle*{2.00}}
\put(100.00,10.00){\circle*{2.00}}
\put(110.00,10.00){\circle*{2.00}}
\put(80.00,20.00){\circle{2.00}}
\put(80.00,11.00){\vector(0,1){8.00}}
\put(80.00,30.00){\circle{2.00}}
\put(80.00,21.00){\vector(0,1){8.00}}
\put(90.00,20.00){\circle{2.00}}
\put(90.00,11.00){\vector(0,1){8.00}}
\put(90.00,30.00){\circle{2.00}}
\put(90.00,21.00){\vector(0,1){8.00}}
\put(100.00,20.00){\circle{2.00}}
\put(100.00,11.00){\vector(0,1){8.00}}
\put(100.00,30.00){\circle{2.00}}
\put(100.00,21.00){\vector(0,1){8.00}}
\put(100.00,40.00){\circle{2.00}}
\put(100.00,31.00){\vector(0,1){8.00}}
\put(100.00,50.00){\circle{2.00}}
\put(100.00,41.00){\vector(0,1){8.00}}
\put(100.00,60.00){\circle{2.00}}
\put(100.00,51.00){\vector(0,1){8.00}}
\put(100.00,70.00){\circle{2.00}}
\put(100.00,61.00){\vector(0,1){8.00}}
\put(100.00,80.00){\circle{2.00}}
\put(100.00,71.00){\vector(0,1){8.00}}
\put(100.00,90.00){\circle{2.00}}
\put(100.00,81.00){\vector(0,1){8.00}}
\put(110.00,20.00){\circle{2.00}}
\put(110.00,11.00){\vector(0,1){8.00}}
\put(110.00,30.00){\circle{2.00}}
\put(110.00,21.00){\vector(0,1){8.00}}
\put(110.00,40.00){\circle{2.00}}
\put(110.00,31.00){\vector(0,1){8.00}}
\put(110.00,50.00){\circle{2.00}}
\put(110.00,41.00){\vector(0,1){8.00}}
\put(120.00,10.00){\circle*{2.00}}
\put(120.00,20.00){\circle{2.00}}
\put(120.00,11.00){\vector(0,1){8.00}}
\put(120.00,30.00){\circle{2.00}}
\put(120.00,21.00){\vector(0,1){8.00}}
\put(120.00,40.00){\circle{2.00}}
\put(120.00,31.00){\vector(0,1){8.00}}
\put(120.00,50.00){\circle{2.00}}
\put(120.00,41.00){\vector(0,1){8.00}}
\put(80.00,31.00){\vector(2,1){18.67}}
\put(90.00,31.00){\vector(1,1){8.67}}
\put(110.00,51.00){\vector(-1,1){8.67}}
\put(119.67,51.00){\vector(-2,1){18.00}}
\put(10.00,5.00){\makebox(0,0)[cc]{$p_1$}}
\put(20.00,5.00){\makebox(0,0)[cc]{$p_2$}}
\put(30.00,5.00){\makebox(0,0)[cc]{$p_3$}}
\put(40.00,5.00){\makebox(0,0)[cc]{$p_4$}}
\put(50.00,5.00){\makebox(0,0)[cc]{$p_5$}}
\put(80.00,5.00){\makebox(0,0)[cc]{$p_1$}}
\put(90.00,5.00){\makebox(0,0)[cc]{$p_2$}}
\put(100.00,5.00){\makebox(0,0)[cc]{$p_3$}}
\put(110.00,5.00){\makebox(0,0)[cc]{$p_4$}}
\put(120.00,5.00){\makebox(0,0)[cc]{$p_5$}}
\put(5.00,95.00){\makebox(0,0)[cc]{a)}}
\put(75.00,95.00){\makebox(0,0)[cc]{b)}}
\put(145.00,95.00){\makebox(0,0)[cc]{c)}}
\put(170.00,90.00){\circle{2.00}}
\put(170.00,80.00){\circle{2.00}}
\put(170.00,70.00){\circle{2.00}}
\put(170.00,60.00){\circle{2.00}}
\put(170.00,50.00){\circle{2.00}}
\put(170.00,40.00){\circle{2.00}}
\put(170.00,30.00){\circle{2.00}}
\put(170.00,20.00){\circle{2.00}}
\put(170.00,10.00){\circle*{2.00}}
\put(170.00,5.00){\makebox(0,0)[cc]{$p_1$}}
\put(170.00,11.00){\vector(0,1){8.00}}
\put(170.00,21.00){\vector(0,1){8.00}}
\put(170.00,31.00){\vector(0,1){8.00}}
\put(170.00,41.00){\vector(0,1){8.00}}
\put(170.00,51.00){\vector(0,1){8.00}}
\put(170.00,61.00){\vector(0,1){8.00}}
\put(170.00,71.00){\vector(0,1){8.00}}
\put(170.00,81.00){\vector(0,1){8.00}}
\put(180.00,70.00){\circle{2.00}}
\put(190.00,70.00){\circle{2.00}}
\put(180.00,80.00){\circle{2.00}}
\put(180.00,71.00){\vector(0,1){8.00}}
\put(180.00,90.00){\circle{2.00}}
\put(180.00,81.00){\vector(0,1){8.00}}
\put(190.00,80.00){\circle{2.00}}
\put(190.00,71.00){\vector(0,1){8.00}}
\put(190.00,90.00){\circle{2.00}}
\put(190.00,81.00){\vector(0,1){8.00}}
\put(150.00,50.00){\circle{2.00}}
\put(150.00,60.00){\circle{2.00}}
\put(150.00,51.00){\vector(0,1){8.00}}
\put(150.00,70.00){\circle{2.00}}
\put(150.00,61.00){\vector(0,1){8.00}}
\put(150.00,80.00){\circle{2.00}}
\put(150.00,71.00){\vector(0,1){8.00}}
\put(150.00,90.00){\circle{2.00}}
\put(150.00,81.00){\vector(0,1){8.00}}
\put(160.00,50.00){\circle{2.00}}
\put(160.00,60.00){\circle{2.00}}
\put(160.00,51.00){\vector(0,1){8.00}}
\put(160.00,70.00){\circle{2.00}}
\put(160.00,61.00){\vector(0,1){8.00}}
\put(160.00,80.00){\circle{2.00}}
\put(160.00,71.00){\vector(0,1){8.00}}
\put(160.00,90.00){\circle{2.00}}
\put(160.00,81.00){\vector(0,1){8.00}}
\put(168.67,39.67){\vector(-2,1){18.00}}
\put(169.00,40.33){\vector(-1,1){8.33}}
\put(170.67,60.33){\vector(1,1){8.67}}
\put(170.67,59.67){\vector(2,1){18.67}}
\put(174.00,50.00){\makebox(0,0)[cc]{$s$}}
\put(163.00,50.00){\makebox(0,0)[cc]{$s$}}
\put(153.00,50.00){\makebox(0,0)[cc]{$s$}}
\put(194.00,70.00){\makebox(0,0)[cc]{$t$}}
\put(183.00,70.00){\makebox(0,0)[cc]{$t$}}
\put(173.00,70.00){\makebox(0,0)[cc]{$t$}}
\end{picture}
\\
The lowest ``root'' represents the
initial state interpretable as program. Forward computation represents
upwards motion
through a sequence of states represented by open circles. Different
symbols $p_i$ correspond to different initial states, that is, different
programs.
a) One-to-one computation.
b) Many-to-one junction which is information discarding. Several
computational paths, moving upwards, merge into one.
c) One-to-many computation is allowed only
 if no information is
created and discarded; e.g., in copy-type operations on blank memory.
}


\frame[shrink=1.2]{
\frametitle{No-cloning (no-copy) theorem}

\begin{itemize}
\item<+->
Ideally, a perfect qcopy device $A$, acting upon an arbitrary state $\psi$
and some arbitrary blank state $b$, would do this:\\
$$
\psi
\otimes
\vert b\rangle
\otimes
\vert A_i\rangle
\longrightarrow
\psi
\otimes
\psi
\otimes
\vert A_f\rangle .
$$

\item<+->
Suppose it would copy the two ``quasi-classical'' state
``$+$''
and
``$-$'' accordingly:
$$
\vert +,b,  A_i\rangle
\longrightarrow
\vert +,+,  A_f\rangle
,\qquad
\vert -,b,  A_i\rangle
\longrightarrow
\vert -,-, A_f\rangle.
$$

\item<+->
By the linearity of quantum mechanics, the state
$
{1\over \sqrt{2}}(
\vert + \rangle +
\vert - \rangle  )
$
is copied according to
$$
{1\over \sqrt{2}}(
\vert + \rangle  +
\vert - \rangle  )
\otimes
\vert
b,
 A_i\rangle
\longrightarrow
{1\over \sqrt{2}}(
\vert + ,+, A_f\rangle +
\vert - ,-,A_f \rangle  )
$$
$$
\qquad
\qquad
\neq
{1\over \sqrt{2}}(
\vert + \rangle  +
\vert - \rangle  )
\otimes
{1\over \sqrt{2}}(
\vert + \rangle  +
\vert - \rangle  )
\otimes
 \vert
 A_i\rangle
.
$$


\end{itemize}
}


\subsection{Visions of parallelism \& interference}
\frame[shrink=2]{
\frametitle{Visions of parallelism \& interference}

\begin{itemize}

\item<+->
A single qubit ``embodies'' two classically contradictory states at once.
This is the basis of ``quantum parallelism.''

\item<+->
$n$ single qubits ``embody'' $2^n$ classically contradictory states at once.
A linear increase of quantum information
is associated with an exponential increase of embodied classically states -- ``quantum parallelism.''

\item<+->
The information in $N$ qubits can be coded in a ``distributed'' (``entangled'') manner,
such that measurement of a single qubit ``destroys'' this information and makes a readout impossible.


\item<+->
Encoding of a classical decision problem by
\begin{itemize}
\item<+->
``folding'' a quantum state as a coherent superposition of all
(contradictory) classical cases
\item<+->
processing this coherent superposition; and finally
\item<+->
``unfolding'' the processed state properly such that a readout of the unfolded state presents
the solution to the decision problem (equivalent to a state identification).
\end{itemize}
\end{itemize}


}



%\section{}
\subsection{Deutsch algorithm: parity of a function of one bit}
\frame{
\frametitle{Deutsch algorithm: parity of a function of one bit}
\begin{table}
\centerline{
\begin{tabular}{cccccccccccccc}
\hline
$f$& $ 0$ &$1$\\
\hline
$f_0$ &$0$ &$0$\\
$f_1$ &$0$ &$1$\\
$f_2$ &$1$ &$0$\\
$f_{3}$ &$1$ & $1$\\
\hline
\end{tabular}
}
\caption{The binary functions of one bit considered in Deutsch's problem. \label{2005-ko-t11dp}
}
\end{table}
 }



\frame{
\frametitle{Interlude: definition of elementary unitary operations on single bits}

\begin{itemize}

\item<+->
$\textsf{\textbf{X}}
=
\left(
\begin{array}{rr}
0&1\\
1&0\\
\end{array}
\right)$ is the {\em not}-operator


\item<+->
$\textsf{\textbf{H}}=
\frac{1}{\sqrt{2}}\left(
\begin{array}{rr}
1&1\\
1&-1\\
\end{array}
\right)$   is the normalized Hadamard matrix
\end{itemize}

}


\frame{
\frametitle{Three step strategy}

\begin{itemize}
\item<+->
First step: unfold the quantum bit
\item<+->
Second step: process the quantum bit
\item<+->
Third step: read out the quantum bit
\end{itemize}
To preserve reversibility of information processing, start with  two bits instead of one
(undorgoing an irreversible functional transformation $f$):
$$
\textsf{\textbf{U}}_f(\vert x\rangle \vert y\rangle )
=\vert x\rangle \vert y \oplus f(x) \rangle
$$
}

\frame[shrink=2]{
\frametitle{Deutsch algorithm: parity of a function of one bit cntd.}

\begin{itemize}
\item<+->
Start with $\vert 0\rangle \vert 0\rangle$ [or rather $(\textsf{\textbf{X}}\otimes\textsf{\textbf{X}})
(\vert 0\rangle \vert 0\rangle )$  for convenience];
\item<+->
then ``unfold'' with the two Hadamards
$\textsf{\textbf{H}}\otimes\textsf{\textbf{H}}$;
\item<+->
then apply $\textsf{\textbf{U}}_f$
\end{itemize}

{\scriptsize
\begin{table}
\centerline{
\begin{tabular}{cccccccccccccc}
\hline\hline
 & $\frac{1}{2}[\vert 0\rangle \vert 0 \oplus f(0)\rangle $ &-& $\vert 0\rangle \vert 1 \oplus f(0)\rangle $ &-& $\vert 1\rangle \vert 0 \oplus f(1)\rangle $ &+& $\vert 1\rangle \vert 1 \oplus f(1)\rangle ]$\\
\hline
$f_0$: $\psi_1$ & $\frac{1}{2}(\vert 0\rangle \vert 0\rangle $ &-& $\vert 0\rangle \vert 1\rangle $ &-& $\vert 1\rangle \vert 0\rangle $ &+& $\vert 1\rangle \vert 1\rangle )$\\
$f_1$: $\psi_2$ & $\frac{1}{2}(\vert 0\rangle \vert 0\rangle $ &-& $\vert 0\rangle \vert 1\rangle $ &-& $\vert 1\rangle \vert 1\rangle $ &+& $\vert 1\rangle \vert 0\rangle )$\\
$f_2$: -$\psi_2$ & $\frac{1}{2}(\vert 0\rangle \vert 1\rangle $ &-& $\vert 0\rangle \vert 0\rangle $ &-& $\vert 1\rangle \vert 0\rangle $ &+& $\vert 1\rangle \vert 1\rangle )$\\
$f_3$: -$\psi_1$ & $\frac{1}{2}(\vert 0\rangle \vert 1\rangle $ &-& $\vert 0\rangle \vert 0\rangle $ &-& $\vert 1\rangle \vert 1\rangle $ &+& $\vert 1\rangle \vert 0\rangle )$\\
\hline\hline
\end{tabular}
}
\caption{State evolution of $\textsf{\textbf{U}}_f(\textsf{\textbf{H}}\otimes\textsf{\textbf{H}})
(\textsf{\textbf{X}}\otimes\textsf{\textbf{X}})
(\vert 0\rangle \vert 0\rangle )$
for the four functions $f_0,f_1,f_2,f_3$.
$\textbf{X}$ and
$\textbf{H}$ stand for the not operator and the (normalized) Hadamard transformation.
 \label{2005-ko-t1}
}
\end{table}
}
 }




\frame[shrink=2]{
\frametitle{Third step: Readout \& state identification in Deutsch's case}

\begin{itemize}
\item<+->
The encoding Ansatz
$\textsf{\textbf{U}}_f(\textsf{\textbf{H}}\otimes\textsf{\textbf{H}})
(\textsf{\textbf{X}}\otimes\textsf{\textbf{X}})
(\vert 0\rangle \vert 0\rangle )$,
resulting in the two different states
$$
\vert \psi_1 \rangle = \pm \frac{1}{2} (
\vert 0\rangle   -
\vert 1\rangle  )(
\vert 0\rangle   -
\vert 1\rangle  )\equiv \pm \frac{1}{2} ((1,-1)\otimes (1,-1))^T=\pm \frac{1}{2} (1,-1,-1,1)^T
\label{2005-ko-e111}
$$
for $f_0$ as well as $f_3$, and
$$
\vert \psi_2 \rangle = \pm \frac{1}{2} (
\vert 0\rangle   +
\vert 1\rangle  )(
\vert 0\rangle   -
\vert 1\rangle  )\equiv \pm \frac{1}{2} ((1,1)\otimes (1,-1))^T=\pm \frac{1}{2} (1,-1,1,-1)^T
$$
for $f_1$ as well as $f_2$.



\item<+->
Finally, application of two additional Hadamard-transformations for each one of the two bits  yields a
representation in the standard computational basis; i.e.,
$$
(\textsf{\textbf{H}}\otimes\textsf{\textbf{H}})
\textsf{\textbf{U}}_f
(\textsf{\textbf{H}}\otimes\textsf{\textbf{H}})
(\textsf{\textbf{X}}\otimes\textsf{\textbf{X}})
(\vert 0\rangle \vert 0\rangle )
=
\left\{
\begin{array}{ccl}
\vert 1\rangle \vert 1\rangle \equiv (0,0,0,1)^T  &\textrm{ for } f(0)=f(1),\\
\vert 0\rangle \vert 1\rangle \equiv (0,1,0,0)^T   &\textrm{ for } f(0)\neq f(1).\\
\end{array}
\right.
$$
\end{itemize}
}



%\section{}
\frame{
\frametitle{Other speedups incl. factoring \& database search}

\begin{itemize}
\item<+->
Finding the period of a function, related to prime factorization, related to RSA encryption
``Shor's'' algorithm.

\item<+->
Finding whether or not a function acquires ``1'' on an argument space or is ``0''
everywhere (``database search'').

\item<+->
General parity cannot be substantially sped up.

\end{itemize}

}


%\section{}
%\subsection{What may and may not be possible}
\frame{
\frametitle{What may and may not be possible}

\begin{itemize}
\item<+->
Speedup for all problems translatable into state identification problems.

\item<+->
Speedup questionable for problems which are classically recursion theoretic hard,
such as the Ackermann or the Busy Beaver function.

\item<+->
Still no quantum speedup for NP-complete problems.

\end{itemize}

}


%\section{}
\subsection{Quantum cryptography \& man-in-the-middle attacks}
\frame{
\frametitle{Quantum cryptography}
\frametitle{History}

\begin{itemize}
\item<+-> [1970]
Stephen Wiesner, {\em ``Conjugate coding:''}
noisy transmission of two or more ``complementary messages'' by using single photons in
two or more complementary polarization directions/bases.

\item<+-> [1984]
BB84 Protocol: key growing via quantum channel \& additional classical bidirectional communication channel

\item<+-> [1989]
First realization by Bennett et al. at 1989 IBM Yorktown Heights, 1993 by Gisin across Lake Geneva,
2003-present DARPA Network Boston (permanent real-time).

\end{itemize}
 }

\frame{
\frametitle{BB84 Protocol}
%$\longrightarrow$ time\\
\includegraphics[height=8cm]{2005-qcrypt-pres-BBBSS92.pdf}
from [BBBSS92]
}


%\section{}
%\subsection{Man-in-the-middle attacks}
\frame{
\frametitle{Man-in-the-middle attack:  requiring classical authorization (merely ``key growing'')}
%TexCad Options
%\grade{\off}
%\emlines{\off}
%\beziermacro{\on}
%\reduce{\on}
%\snapping{\off}
%\quality{2.00}
%\graddiff{0.01}
%\snapasp{1}
%\zoom{1.00}
\unitlength 0.70mm
\linethickness{0.4pt}
\begin{picture}(155.00,45.00)
\put(24.98,9.98){\line(1,0){20.06}}
\put(34.96,12.48){\circle{4.99}}
\put(35.46,12.48){\circle{4.99}}
\put(24.98,9.68){\line(1,0){20.06}}
\put(24.98,19.96){\line(1,0){20.06}}
\put(34.96,22.46){\circle{4.99}}
\put(34.96,22.46){\makebox(0,0)[cc]{c}}
\put(35.06,12.48){\makebox(0,0)[cc]{q}}
\put(45.00,5.00){\dashbox{1.33}(20.00,20.00)[cc]{}}
\put(135.02,9.98){\line(-1,0){20.06}}
\put(125.04,12.48){\circle{4.99}}
\put(124.54,12.48){\circle{4.99}}
\put(135.02,9.68){\line(-1,0){20.06}}
\put(135.02,19.96){\line(-1,0){20.06}}
\put(125.04,22.46){\circle{4.99}}
\put(125.04,22.46){\makebox(0,0)[cc]{c}}
\put(124.94,12.48){\makebox(0,0)[cc]{q}}
\put(95.00,5.00){\dashbox{1.33}(20.00,20.00)[cc]{}}
\put(40.00,0.00){\framebox(80.00,30.00)[cc]{}}
\put(42.33,35.00){\makebox(0,0)[lc]{box-in-the-middle}}
\put(45.00,27.33){\makebox(0,0)[lc]{fake ``Bob''}}
\put(95.33,27.33){\makebox(0,0)[lc]{fake ``Alice''}}
\put(80.00,45.00){\makebox(0,0)[cc]{Eve}}
\put(135.00,5.00){\dashbox{1.33}(20.00,20.00)[cc]{}}
\put(135.33,27.33){\makebox(0,0)[lc]{Bob}}
\put(5.00,5.00){\dashbox{1.33}(20.00,20.00)[cc]{}}
\put(5.00,27.33){\makebox(0,0)[lc]{Alice}}
\put(65.00,20.00){\line(1,0){30.00}}
\put(80.00,20.00){\line(0,1){20.00}}
\put(80.00,15.00){\makebox(0,0)[ct]{copy or}}
\put(80.00,10.00){\makebox(0,0)[ct]{misinform}}
\end{picture}
\begin{center}
from http://arxiv.org/abs/quant-ph/0501062
\end{center}
}

\frame{
\frametitle{Techniques \& gadgets}
\begin{itemize}


\item<+->
Photon sources: faint laser pulses,
photon pairs generated by parametric downconversion, photon guns,
 $\ldots$

\item<+->
Quantum channels: single-mode fibers, free-space links, $\ldots$

\item<+->
Single-photon detection: photon counters,  $\ldots$

\item<+->
(Quantum) Random number generators: calcite prism,
 $\ldots$
\end{itemize}
}



\section{Resources}
%\subsection{}
\frame[shrink=2]{
\frametitle{Resources}

\begin{itemize}
\item<+->
David Mermin's qc lecture: very good, very popular:  \\
http://people.ccmr.cornell.edu/$\sim$mermin/qcomp/CS483.html

\item<+->
Up-to-May 2005 collection of findings \& open questions:\\
M. Arndt et al., ``Quantum Physics from A to Z''     \\
http://www.arxiv.org/abs/quant-ph/0505187

\item<+->
Older, very influential article by Schr{\"o}dinger (the ``cat'' papers):\\
E. Schr{\"o}dinger, ``Die gegenw{\"a}rtige Situation in der Quantenmechanik'',
Naturwissenschaften 23, pp.807-812; 823-828; 844-849 (1935).\\
http://wwwthep.physik.uni-mainz.de/$\sim$matschul/rot/schroedinger.pdf

\item<+->
Quantum ``measurement'' paradoxes:\\
L. Vaidman, Z. Naturforsch. 56 a, 100-107 (2001)  \\
http://arxiv.org/abs/quant-ph/0102049

\item<+->
Staging quantum cryptography with chocolate balls\\
http://arxiv.org/abs/physics/0510050

\item<+->
Up-to-date discussion on (subscribable abstracts):\\
http://arxiv.org/archive/quant-ph
\end{itemize}

}

\section{Conclusions}
%\subsection{}
\frame{
\frametitle{Conclusions}

\begin{itemize}
\item<+->
Single quantum concepts and their experimental realization.

\item<+->
Novel phenomena which go beyond the classical field.

\item<+->
Possible application in quantum information processing.


\item<+->
With growing integration \& miniaturization,
thechnology will ``reach the quanta'' soon.


\item<+->
Many open questions, active \& fascinating research field!

\end{itemize}

}

\frame{
\centerline{\Large Thank you for your attention!}
 }

\end{document}
