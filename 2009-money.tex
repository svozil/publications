\documentclass[%
reprint,
%superscriptaddress,
%groupedaddress,
%unsortedaddress,
%runinaddress,
%frontmatterverbose,
 %preprint,
showpacs,
preprintnumbers,
%nofootinbib,
%nobibnotes,
%bibnotes,
 amsmath,
 amssymb,
 aps,
% prl,
pra,
%prb,
%rmp,
%prstab,
%prstper,
longbibliography,
%floatfix,
%lengthcheck,%
]{revtex4-1}

%\usepackage{pslatex}
%\usepackage{hyperref}

\usepackage[left]{eurosym}

%\bibliographystyle{apsrev4-1long}

\begin{document}



\title{An Apology for Money}

\author{Karl Svozil}
\email{svozil@tuwien.ac.at}
\homepage{http://tph.tuwien.ac.at/~svozil}
\affiliation{Institute for Theoretical Physics, Vienna University of Technology,  \\
Wiedner Hauptstra\ss e 8-10/136, A-1040 Vienna, Austria}

\begin{abstract}
Because of their provable indeterminateness, certain economic values and thus prices as well as the money stock depend on subjective confidence, beliefs and fantasies loosely bound by market constraints. One may imagine such a monetary system as being ``suspended in thought.'' Monetization of ``(derivative) promises'' of any kind, as well as interest levied by banks in return for money created via monetizing future profits systematically reallocates resources toward the financial institutions, and away from industrial and manual production, farming and labor. Unfortunately, the alternatives appear to be even more troublesome than the present state of affairs. Any system based on interest-free fiat money creation, in order to avoid hyperinflation through excessive borrowing of ``free'' debt, has either to rely on unjustifiable privileges or chance. And any system based on commodity instead of fiat money is heavily depending on the quantity of commodities, and also incapable of waging or defending against war through the effective monetization of future loot or loss.
\end{abstract}

\pacs{01.75.+m,05.65.+b}
\keywords{money, monetization, value, appropriation}

\maketitle

%\tableofcontents

%\tableofcontents

{\footnotesize
\begin{quote}
\begin{flushright}
$\ldots$~a blind man eager to see who knows \\
that the night has no end, \\
he is still on the go. The rock is still rolling.\\
\mbox{[[~$\ldots$~]]}\\
One must imagine Sisyphus happy.\\
{\em Albert Camus}, in {\it Le Mythe de Sisyphe (English translation: The Myth of Sisyphus)}
\end{flushright}
\end{quote}
}

\section{Enigmatic pervasion of money}

{\em Money} appears to be one of the most amazing and mind-boggling entities:
we are conditioned to its pervasion, yet we may have merely enigmatic, uncertain ideas of
how it is created, how it evolves, and even how it is being accounted.
The epistemology of money is confusing and comprises many intertwining layers of narratives;
some so trivial they resemble well-told fairy tales of deception~\cite{1948-Samuelson,Begg,kyrer-penker-vwl},
some so ``deep'' they appear to be rooted in metaphysics~\cite{soros-alchemy}.
Can we evade~\cite{Bouchaud-08} the maze or veil created by our conditioning, and erected through
(dis)information from the media, contradictory economic theories, ideologies,
and influential groups who have a vested interest in one way or another?

In what follows,  the creation of money by monetization will be studied
in contemporary prevalent fractional reserve systems, followed by questions related to levying interest,  value and price, as well as dynamical (re)appropriation.
Finally, the differences between commodity based money and fiat money will be discussed.
The title was chosen because, despite all negative consequences, the author can think of no ``reasonable'' alternative to fiat money,
thus in a very general sense he is afraid that,
despite its apparent inherent and irreducible lack of stability and the resulting inevitable crises, there is no feasible alternative to it.



\section{Medium of exchange}


Suppose an omnipotent agent organizing a society of individuals and institutions allowing ways to co-operate.
Obviously, any such configuration should not consist of self-sufficient ``monads,''
but the parties should have scarce entities or {\em assets}  to offer to one another; i.e.,
these assets present some form of {\em value} in the mind of other agents or participants.
The recognition, negotiation and exchange of these assets take place in some {\em agora} or {\em market}.

Somewhat arbitrarily disregarding other functions of money as a {\em measure of economic values} and thus of price,
a {\em unit of account},  a {\em store of value}, as well as a {\em measure of dept},  money will be introduced as a {\em medium of exchange} and {\em transaction}.
The amount of value of an asset expressed in terms or units of money is called {\em price.}

There emerge two immediate questions:
(i) what is the value of assets, and how are the prices fixed; and
(ii) how exactly did the negotiating parties obtain their money?
Let us consider the second question first.
Quite simply, one can obtain money, say, for a bull.
That, of course, is only relegating the issue to the customer who offers this money:
 from where did he obtain the money?
Probably he has sold some hay bushels to somebody else in exchange of money.
This indirect barter could go on forever without any clue about how the money was introduced into the system in the first place, provided
the economy contains enough money to allow unimpeded exchange.


\section{Fiat monetization}

So how exactly does money enter the system in the first place?
The answer is {\em monetization,}
i.e., the process of converting some asset into some form of money that is generally accepted as a settlement of an exchange or a debt.
Obviously, in order to be ``generally accepted,''
the issuing agency has to be a publicly certified and accepted {\em authority.}
Commodity money needs no monetization, as the market value of the commodity could, at least in ideal markets, be related to the market value of the asset exchanged,
thus fixing the price in terms of the commodity (money).

Fiat money presents no intrinsic value, and thus cannot be directly related by its (nonexistent) commodity value.
As a substitute of intrinsic value, some (central or noncentral) bank authority issuing the fiat money ``guarantees'' and ``certifies'' its value.
Pointedly stated, in an almost ``magical'' manner, some agency (im)prints something on a sheet of paper or digital account,
and in that manner ``creates'' money out of ``thin air.''
Henceforth, any such agency will be called {\em bank.}
Examples of banks are central banks issuing central bank money (e.g., coins and bills),
private (investment) noncentral banks, or funds, creating computerized  deposit money accounts containing digits,
or  IOU's (abbreviated form of ``I-Owe-You'') on some substratum, mostly on paper.
``Trust,'' ``faith'' \& ``authority'' is required; else everybody would print his own money.

For the sake of demonstration, suppose you are a cashier.
Then you surely would not take a sheet of blank paper where I just wrote ``\EUR{100}''
as down payment for a bottle of wine, returning to me some central bank notes as change;
yet you would be willing to value that same sheet of paper if it is ``backed''
by some authority, such as a credit card company, you have been conditioned to trust.

Monetization facilitates the chain of exchanges, as banks pass on the money created to somebody possessing assets, thereby acquiring (rights on) these assets.
In the view of the asset holder, monetization is the act of ``turning in'' (rights on) assets, thereby obtaining money.
From the bank's perspective, the exchange looks conversely.
In this process, the bank acquires both the asset as well as liabilities (balanced by the ownership of the asset).
Note that, the bank just ``produces'' money effortlessly ``out of thin air,''
whereas the asset holder had to acquire (e.g., inherit or produce) the asset before selling it to the bank.
Surely this puts the banks in a very privileged position:
it ``effortlessly'' acquires assets and their associated utilities.
Another privilege of banks which will be discussed below is the levy of interest.

After monetization, the bank can utilize the asset, whereas the seller and previous owner of that asset can utilize the money issued by the bank.
In one extreme case, the money may remain ``dormant'' in the bank's account, possibly without even collecting (much) interest; this would be most favorable for the bank.
In the other extreme, the seller rushes to convert the bank's money immediately into nonbank assets, commodities (such as silver, gold, or oil),
or central bank money; any such exchange would be least favorable for the bank.

Why should anyone trade in fiat money issued by banks, which is intrinsically worthless,
for an asset which has some utility?
Because anyone would exchange fiat money for other utilities.

Examples of monetarization are the bank's acquisition of
(i) government bonds (based on future government income; e.g., claims of taxes),
(ii) real estate property,
(iii) commodities,
(iv) shares in a business or company,
(v) future claims of  taxes (government bonds), profits or assets, and
(vi) foreign money;
(vii) finance derivatives, e.g., over-the-counter (OTC) derivatives.
{\em Pro forma,} the insertion of money {\em via} monetization is just another exchange,
taking place between the bank and the holder of the asset without any ``intermediate'' money state;
the role of the bank's asset being played by money;
(viii) ``love letters''~\cite{Sibert10,Flannery2009,Hreinsson2009}; i.e., mutual liabilities exchanged by banks
(e.g., three subsidiaries of Icelandic banks
were posting such notes as collateral at the
{\it Central Bank of Luxembourg},
receiving {\it Eurosystem} loans in exchange, on which they subsequently defaulted~\cite{ECBank2009}).


Could monetization of assets go on forever?
Essentially yes~\cite{schneider-VWLIII}; monetization is, at least in principle, only bound by three constraints:
(i) by the asset value, as assessed by rating agencies;
(ii) by the reserve assets an economy is capable to render; and
(iii) as the seller may request to be paid in money or commodities a bank is incapable to produce (such as gold, silver or energy)
by the reserve of types of money or commodities, e.g.,
by central bank money (in case of noncentral banks) or foreign exchange (in case of central banks).
In particular, in {\em fractional reserve systems}~\cite{ModernMoneyMechanics},
the money creation by noncentral banks should, at least by this principle,
be bound by the inverse of the required fraction of central bank reserves~\cite{kyrer-penker-vwl}.
How much and what kind of asset qualifies as adequate {\em collateral} and {\em eligible asset} for reserves
is a matter of convention (see, for instance Ref.~\cite[Chapter~6,7]{EUROMonPol08} for the {\em Eurosystem}, and
Ref.~\cite{FRAC2009} for the {\em Federal Reserve} system).

It is an understatement that, in order to get rid of external limits,
a bank will particularly try to evade the reserve requirements (iii) by various methods and techniques.
(The ``explosion'' of M3 in terms of M0 and M1 (cf. below) is a very clear indicator thereof.)
In doing so, at least in principle,
any individual bank could acquire the available marketed assets by newly created money
even beyond the ``utility threshold'' --- which is bound by the interest rates --- for private investors.
Of course, acquisition may not be a bank's primary role or concern
--- which is often said to be credit creation ---
since after acquisition the bank would have to properly utilize the asset; a task which it may find notoriously incapable of performing.

Ideally, the money volume is ``counterbalanced'' by marketable assets (and future entities; see below), so that there neither is a ``lack'' nor
an ``excess'' of money, resulting in deflation and recession, or in inflation, respectively.
I leave it to the reader to imagine how hopeless such an endeavor of the creation and maintenance of a ``balanced fiat monetary equilibrium''
in economies with dynamical creation and annihilation of assets, products, expectations, and value must appear.
Indeed, the public belief in such a effective equilibrium is a necessary and inevitable delusion for the acceptance any fiat money.
As long as this delusion can be maintained for or imposed upon the majority of market participants, fiat money is relatively stable;
regardless of the disproportion between volume fiat money and marketable assets (inclusive future entities).

Note that monetization of assets depends greatly on fixing its ``intuitive'' market value in terms of a formal {\em price} (in currency units).
As assets are often not directly marketed or traded, the fixation of asset prices is consigned to {\em rating agencies}.
If the rating agencies  are co-owned by the very banks monetizing the asset, or if they are paid by either (buyer's or seller's) side of the transaction,
a {\em conflict of interest} may occur~\cite{smith-walter-09}: e.g., by overrating assets systematically,
a bank my effectively being able to generate its own almost unlimited supply of money.

Note also that the ratio of money created by the central bank versus other banks can be estimated by ratios of  currency {\em components,}
serving as empirical measures of aggregates of money stock.
Bank money is often denoted by M1,M2,M3; as compared  to the amount M0 of currency, i.e., coins and notes, in circulation.
This ratio amounts to a few percent (M3 is no longer published for the U.S. Dollar), so ``most of the money'' is not in currency stock.
Through the {\em fractional reserve  banking system} utilizing the {\em reverse multiplier}~\cite{ModernMoneyMechanics}, and through other ``less benign \& accountable''  practices of money creation,
--- e.g., by bundling and re-selling debt as investments which have a (triple-A or lower) rating from rating agencies indirectly belonging to the issuers ---
most of this noncurrency money is created by noncentral banks.

In more recent times (around 2000 aD) the money created {\it via} monetization of ``products'' of the ``financial sector'', such as
speculative derivatives of any kind, has {\em by far outnumbered} the money created by governments (bonds) in their attempt of ``deficit spending''
as a means to maintain low unemployment rates and social stability.
Therefore, it appears almost absurd when investment bankers (and some associated ``conservative'' economists),
blame governments of ``Keynsianism,'' while simultaneously
engaging in excessive fiat money creation of several magnitudes higher than any of these additional government bonds.
Furthermore, in times of crises they attempt to consolidate their assets by seducing or blackmailing governments into securing them through
future taxation; thereby effectively ``taking hostage'' entire populations, as well as the future generations,
of these states by getting them into debt.




\section{Interest}

Some nonbank agents, such as explorers, invaders, investors or inventors might require money for future profits.
Examples of such nonbank agents are homeowners expecting future salaries,
industries expecting the production of future assets,
speculators expecting a development of future markets favorable for them, or
states waging war on other states in the expectation of victory, allowing the unsolicited exploitation of the opponent's wealth.

Monetization treats the expectation of future profits quite similarly as assets:
a bank can monetize the expectation of future profits by acquiring the right to collecting paybacks from the investor in the future.
In order to make sense for the investor, these repayments should at least be counterbalanced by the expected profits.
There is a difference between a directly obtained asset and a future asset:
whereas the {\em ownership rights} of assets are immediately transferred to banks in the first, direct monetization case,
the banks obtain no immediate control over future assets.
In more concrete terms: whereas, for example, at direct monetization,
the bank can re-sell a monetized real estate property immediately after acquisition,
it could only re-sell the rights of future assets in the indirect case.
As future profits are necessarily uncertain and subject to possible failures, they are always at risk.

For a variety of reasons
---
e.g., in order to counterbalance their risk and the resulting unwillingness to donate money for uncertain future profits,
or to compensate for the temporal delay in consumption~\cite{Boehm-Bawerk} or foregone profits,
and as a reflection of the market price of ``generic'' future profit expectations
---
banks levy interest.
Debt, i.e., the obligation to repay in the future, is always associated with interest~\cite{pesek-saving_1968}.
Interest is the right to (regularly) collect money from the debtor, in addition to the principal --- or to increase the principal as
the time of lending increases --- at a certain rate.

Note that, without ``fiat'' credit and dept, the amount of money could only be sustained proportional to the growth (or decline) of marketed assets,
as at any given moment it would only be possible to invest money which has already been created,
and not also money created {\it in expectation} of future profits.
By lack of liquidity, this bound seriously cripples an economy as compared to economies allowing  ``fiat'' credit.
It also assumes that, through monetization and its inverse,
a ``reasonable'' equilibrium or balance between the money stock and marketed assets can be maintained, thereby synchronously accounting for all created and annihilated or ``stored'' assets;
an assumption which is highly questionable.


Alas, if the ``fiat'' credit and thus also the debt has no direct backing by commodities or monetized assets, the money creation is principally unbounded,
resulting in monetary crises if the future profits are overestimated.
Yet, despite these unfavorable side effects, the creation of money
through the monetization of future profits has been one of the driving forces for
the increase of production of assets~ \cite{soros-alchemy,2006-Binswanger}.
Anybody arguing against monetization of future profits and ``fiat'' credit might just as well propose going back to some kind of unrealistic ``monetary stone age.''

\subsection{Interest as tax and appropriation}

As a result the banking sector receives a certain amount of ``additional'' income on an annual base in terms of the interest paid.
Where exactly does this money required to pay the interest, in addition to the principal granted, come from?
It cannot come from any other source than the fiat money created by the banks themselves.
As the overall amount of valuable assets competing for money (and {\it vice versa}) is limited, the effect is a sort of general ``taxation'' by interest~\cite{champ-freeman-2004},
a re-appropriation of assets toward the banks.
Even under ideal conditions, this amounts to a geometric progression of the volume of money, the assets created, as well as a redistribution of wealth
in favor of the financial sector.

\subsection{Consequences of no or low interest}

In view of the possible imbalances from the accumulation of wealth by the financial sector, attempts have been made, e.g.,  by Christian~\cite{noonan-57,Weiss-Zinswucher-Christ} {\it (``usury'')}
and Islamic~\cite{Hanke-Zinswucher-Islam} {\it (``Riba'')} communities, to abandon interest altogether.
Despite the limits of sustainable growth (in terms of the monetary basis), which might seriosly cripple an economy,
the abandonment of interest causes two undesirable alternatives:
(i) either the amount of interest has to be limited ``from the outside'' by ``certain criteria'' which effectively introduce privileges:
if there is a limited supply of credit, who should receive it?
(ii) if there is no limit to the amount of credit available, any agent in the market would find it possible, at least in the extreme case,
 to ``buy up all available assets,'' because of the zero cost of borrowing, without penalty.
But then, if there are more than one agents competing in the market, prices will go up {\it ad infinitum;} effectively causing hyperinflation.

For example, the high demand for real estate properties reflects the particular importance and the relevance  of housing to individuals and families.
From the point of view of the buyer, the price of a property appears to be limited by the portion of the household income available
for the payment of dept accepted for acquiring that property; i.e., the product of  interest rate times the property  price should not exceed the
buyer's available income, and thus the price of the property is bound by  the income divided by the interest rate.
As a result, property prices tend to increase on decreasing interest rates,
as potential buyers can afford to bid higher prices.
The leverage or ratio of this price increase is determined by the inverse interest rate.
In the (absurd) limit, with ``free credit'' associated with zero interest rate,
a single buyer would be able to bid an unlimited price for any given property.
By unrealistically assuming those prices will not go up due to competing money,
the buyer could acquire all properties available on the market.


\subsection{Inflation and price}

There seems to be a common belief that it is possible to  curb the money supply by regulatory measures.
Indeed, interest rates of consumer  credits and, say, the U.S. {\it federal funds rate} appear to be correlated.
This is usually explained by money volume constraints on the noncentral banks, effectively established {\em via} some regulatory mechanisms:
in order to prevent bank runs or an unbounded lending policy,
banks usually should not be able to create more money than a certain percentage or ``fraction'' of some ``securities'' or reserves they hold;
resulting in a ``mild'' form of reserve multiplication~\cite{1948-Samuelson,Begg,ModernMoneyMechanics}.
This ``fractional reserve system,''  in view of the recent events connected to the ``packaging'' and ``reselling'' of dept by the financial industry in the U.S. and elsewhere,
appears to be a commonly told fairy tale.

On the contrary, it is in the legitimate interest of banks to avoid any reserve constraints, by any ``quasi-legal'' possibility:
In the present competitive and highly liquid financial market environment,
it is impossible for financial institutions to avoid stretching the regulatory bonds to the extreme;
otherwise they will be ``out of business'' soon, overtaken by the competitors which attract their greedy investors.

The amount of outstanding credit of a financial institution is directly proportional to the interest it levies, and consequently to its income.
There is, for instance, no immediate reason why a bank should not create money and lend it out for a lower interest rate than the central bank,
provided it is not ``too much'' bound by minimal ``fractional reserves''~\cite{ModernMoneyMechanics}:
even if the interest rate is arbitrary low, as long as it is positive, there is some obtainable gain.
Likewise, no customer needs to fail because of defaulting credit:
in the extreme case it would even be conceivable to levy no interest at all until such time when the customer can serve the interest again.
Indeed, the customer may be released from debt totally and permanently; this, however, should
be done  ``secretly,'' because otherwise all debtors would attempt to default as well.
Such gifts, of course, can only be granted because the cost of money creation for banks is negligible.



\section{Value and price}

{\em Price} is the amount of value in terms of money;  fixed in a market or {\em agora,} ideally {\it via} supply and demand.
That is, money is the unit of price and indirectly also of value.
In a less declamatory but more practical manner, value and prices are derived from fantasies people have about a scarce asset.
Suppose I possess a horse, and develop fantasies about romantic rides in the woods;
I might get so excited about this emotionally that my break-even point for selling this horse to somebody else (with similar fantasies)
settles at a multitude of the price at which I bought the horse myself.
The exchange will go through if I can communicate, establish and realize that kind of fantasy at some market,
especially also if utilities change.

Recall, for example,  past price rises of some inner city property, or of some sections close to the sea shore or to a lake.
These sections have been valued very poorly by the original farmers possessing them;
for their utilization of land was not in terms of beauty and recreation, but in terms of harvest.

As there are various markets with very different fantasies and utilities --- some of them rather isolated from each other --- many fantasies co-exist at any given time in a single economy.
The common element of the economy is the money available or created.
Since it is dependent on various asset values and prices, which itself are determined by fantasies, the amount of monetization is a dynamic, volatile quantity.
Moreover, the relative appropriation is dynamic:
it may, for instance, be possible for one group of assets --- say, for example, stocks or other financial assets ---
 to ``overtake'' other sectors or economic segments --- say, for example, labor salaries or property prices.
Thereby, a dynamic appropriation of money is obtained.
A formalization may be envisaged by constructing a linear vector space; every market segment corresponding to a dimension.
A state of the economy is then associated with a vector in this multidimensional state.
The dynamics might be modeled by (nonlinear) maps.

\subsection{Containment and ``osmosis'' of money types}

If the markets are relatively isolated, these (re)appropriations may not be perceivable for some time:
for instance, a financial {\em Wall Street} tycoon will not influence the prices of sausages sold on  {\em Wall Street}
too much, as he might not be interested in buying a sausage there;
and even if he regularly buys sausages for their good taste, he has only use of a very limited number of them.
Indeed, the stronger stratified a society, the less will fantasies in one sector will ``leak through''
and affect prices in other sectors.
Nevertheless, in the long run, the different market segments or sectors tend to connect through the monetary base.
Thus eventually the fantasies exerted in one of them will ``diffuse'' into other sectors almost like ``osmosis''
through small interconnections~\cite[Section~1(f)]{Friedman-2008}.
If, for instance, the same {\em Wall Street} tycoon attempts to ``take over'' most sausage stands of {\em Manhatten,}
the very high price he may have to pay for them may indirectly (through the rate of return on investment) affect the street price for sausages there.
In reaction, as inflation (in terms of sausage price)  goes up, labor costs will increase, contributing to a spiral of inflation.

Effectively, this systems of ``types'' of money --- starting from ``precious commodities'' such as gold, silver, oil, and continuing with the various money components with increasing ``virtuality'' ---
resembles the multitude of ``rubles'' in the late Soviet Union; there the ``gold ruble'' would hypothetically buy 0.987412 gram of gold, yet was never available to the general public.
Also today, monetarily one of the ``worst'' market transactions would be the {\em direct exchange} of the ``most abstract'' form of money,
such as, for instance, government bonds or financial (OTC) derivatives, into gold, silver or platinum.

At the time of writing, once again~\cite{ReinhartRogoff}
the monetary system is inflated (flooded) by monetized fantasies created primarily by the financial institutions,
and to a lesser extent by all sorts of government spending, in particular wars~\cite{BonnerWiggin,StiglitzBilmes}.
In order to back up the huge unsustainable debts, the financial institutions have
turned for government and central bank help~\cite{dunken-crises2}.
Governments effectively back derivatives and other monetized fantasies created by the financial institutions
by (future) tax revenues;
whereas central banks everywhere seem to monetize government or even  private debt.

It is not totally unreasonable to speculate that the
magnitude of these transactions can neither be sustained
nor contained to the market sectors in which they have
been created. Thus the massive amounts of money
volumes in M3 will eventually spill over to the consumer sectors,
and to M1 as well, thereby causing (hyper)inflation
everywhere. Moreover, the  capacity to
``back up'' the M3 volume by taxation and inflation will decrease for socio-political reasons,
as societies will grow weary of saving financial institutions through acquiring their risks and debt.
Some or all ``ultimate debtors'' such as big governments and/or their associated
central banks will then be forced to submit to
strategies to get rid of debt, mostly by inflation and by defaulting dept,
as outlined in the previous Section~\ref{stgrod}.

There may be several ways to survive this monetary ``meltdown;'' some of them rely on gold
(not gold options on paper, but on self-storing the precious metal) or other commodities, some on real estate;
some people even rely on
little bottles of alcohol and strong spirits, for which, as they claim, demand will be high in times of crises.
In principle, one could make huge profits by going into debt now,  buying commodities and real estate.
After the monetary ``meltdown'' and the following consolidation of the currency,
presently acquired debt could then be easily repaid in terms of the new money.
If that sounds strange, consider buying a real estate property today
for its price thirty years ago.


\subsection{Instability}

Finally, let me point out several reasons why the belief
that the equilibrium between supply \& demand will in general settle at a single particular price,
and the idea that there exist equilibriums in economies in general,
is an idealistic illusion:
As money and its various forms and derivatives
is itself marketed, the price of money becomes recursive, self-referential and reflexive.
Trade policies and military deployment might enforce prices.
The market participants might suffer from an overload of information,
accompanied by a lack of reliable criteria or authorities to evaluate the information, or might be fed with disinformation.
The perpetual flow of spontaneous news and opinions via the media may make impossible the formation of a ``communication equilibrium.''
As markets tend to become virtualized, it is not totally unreasonable to suspect
that those who control; i.e., possess and pay through ads, the media control the market and public policy.
Thus the modern markets are driven by whatever {\em communication} and {\em (dis)information} is fed into them.
The intra-market dynamics might not be sufficiently efficient to settle prices; or there may be no convergence toward a {\em single} price,
but rather price cycles and other more chaotic regimes.
The volume creation and annihilation of money and debt by governments, (central) banks,
corporations and individuals might not allow a stabile settlement of prices by creating (expectations of) a chaotic, or alernating, or ``spiralling'' regime~\cite{Kauffman198753}.


\section{Formal incompleteness of macroeconomics}

Related to the the impossibility of equilibria is fact that no complete macroeconomic theory (or ``maximal trading strategy'') exists.
In what follows it is proved by reduction to the halting problem~\cite{godel1,turing-36,davis-58,smullyan-92}
that every macroeconomic theory strong enough to contain {\em substitution,}
{\em self-reference} and (Peano) arithmetic is {\em incomplete} in the sense there exist provable {\em true} sentences which are {\em not derivable}
and thus {\em independent} of that macroeconomic theory.

The scheme of the proof by contradiction is as follows:
the existence of a hypothetical macroeconomic prediction model (trading strategy)
capable of solving the problem of whether or not a certain ``halting'' state in a macroeconomy (e.g., a certain price) will occur,
is {\em assumed.}
This could, for instance, be a ``winning tactics'' of some suspicious ``super-trading strategy''
which takes the code of an arbitrary macroeconomic theory as input and outputs ``yes'' or ``no,''
depending on whether or not the macroeconomic theory predicts a particular ``halting'' state.
One may also think of it as a sort of ``oracle'' or ``black box'' taking in an arbitrary
trading strategy in terms of its symbolic code, and outputting one of two symbolic states, say ``yes'' or ``no,''
referring to reaching or not reaching its desired goal, respectively.

Based on this {\em ``hypothetical macroeconomic prediction model (trading strategy)''} we construct another {\em ``diagonalization (trading) strategy''} as follows:
upon receiving some arbitrary {\em ``macroeconomic theory (trading strategy)''} code as input, the {\em ``diagonalization (trading) strategy''}
consults the {\em ``hypothetical macroeconomic prediction model (trading strategy)''} whether or not this
{\em ``macroeconomic theory (trading strategy)''} reaches a macroeconomic ``halting'' state (e.g., a certain price); and upon receiving some answer it just does the {\em opposite:}
if  the {\em ``hypothetical macroeconomic prediction model (trading strategy)''} decides that the {\em ``macroeconomic theory (trading strategy)''}
{\em reaches a particular macroeconomic state (e.g., price),}
the {\em ``diagonalization (trading) strategy''}  {\em counteracts} that prediction (it may do so easily by some kind of market intervention, such as volume leveraged buying or selling).
Alternatively, if  the {\em ``hypothetical macroeconomic prediction model (trading strategy)''} decides that the {\em ``macroeconomic theory (trading strategy)''} does {\em not reach a macroeconomic ``halting'' state in a (e.g., a certain price),}
the {\em ``diagonalization (trading) strategy''} {\em steers the economy into that state}.

The {\em ``diagonalization (trading) strategy''} can be forced to execute a paradoxical task by
receiving {\em its own program code} as {\em ``macroeconomic theory (trading strategy).''}
Because by considering the {\em ``diagonalization (trading) strategy,''}
the {\em ``hypothetical macroeconomic prediction model (trading strategy)''} steers the {\em ``diagonalization (trading) strategy''} into
{\em  a  macroeconomic ``halting'' state (e.g., a certain price)} if it discovers that it {\em does not halt;}
conversely,  the {\em ``hypothetical macroeconomic prediction model (trading strategy)''} steers the {\em ``diagonalization (trading) strategy''} into
{\em  a state different from the ``halting'' state} if it discovers that it {\em reaches a ``halting'' state.}

The contradiction obtained in applying the {\em   ``diagonalization (trading) strategy''} to its own code proves that this program,
and in particular the {\em ``hypothetical macroeconomic prediction model (trading strategy),''} cannot exist.

\section{Fiat money games}

Fiat money is unbounded by any supply constraints (it can be produced almost cost-free).
Thus, as long as the  market participants can be manipulated to believe in and trust the authorities issuing fiat money
--- an what other choice do they have most of the time, when the possession of gold is prohibited and silver is demonetized? ---
its volume can be expanded and contracted at will of those in control, at least as long as command over the money volume is possible.
It is not too unreasonable to speculate that these ``powers to be'' need not always and necessarily act in a benign intention to serve the public good,
but rather in their own interest.
Alas, even if uniform benignancy is assumed, what is good for one group of interests may not serve the interest of other groups; so there always will be ambivalent tradeoffs and
dynamical reappropriations.
In what follows a few schemes are noted which could be used for the reapproriation of the wealth of nations and within the global economy.

One may argue that commodity based money did not prevent business cycles either,
and that overheating an economy by unsustainable credit creation is not limited to fiat currencies only.
However, it may be argued that fiat currencies are more vulnerable to economic crises,
and bubbles are both greater and longer; and a subsequent correction tends to be ``more painful'' than in the commodity based case.


\subsection{Business cycles of inflation and deflation}

By alternating periods of low and high interest rates it is possible to extract assets from an economy:
(i) ``inflationary periods'' are characterized by low interest rates,
which encourage ``bubble buildups'' through investments with very low yield at ``high'' (with respect to subsequent periods of high interest) prices, as credit money is ``cheap.''
(ii) subsequent ``deflationary periods'' can be initiated by rising interest rates, which cause corrections (lowering) in prices, credit defaults, and a burst of the investment bubble created
in the previous inflationary period.

By subjecting a market to successive periods of inflation and deflation, it is possible to acquire great wealth by inside knowledge:
an inside investor would be prudent to invest into assets (e.g., stocks, derivatives and real estate)
shortly before or at the initiation of inflationary periods,
``riding high on the wave of increasing profits and prices,''
and, shortly before a deflationary period sets in,
``pulling out'' of these investments, thereby converting them into ``save havens'' like
commodities (e.g., gold and silver); remaining in these ``conservative positions'' until just before a new inflationary period begins.
In that way, by controlling the business cycle through money supply, more and more economic assets can be acquired almost risk-free.

This economic cycle strategy almost appears as if it represented a {\em perpetuum mobile;} but it is not, as it can only succeed if the investor is capable of controlling the money supply.
Alas, central banks, and the fractional reserve banking system through the reverse multiplier as a whole, have been institutionalized  and should be capable of doing just that
--- controlling the money supply --- at least
if they are independent and not constraint by politics and external monetary constraints.

One could speculate that a ``proper price for investment money'' (i.e., interest for credit) might be definable
by the {\em absence} of business cylces of the type described above.
Whether or not such an ``ideal'' interest rate exists remains highly questionable, as value and price appear highly subjective entities in particular also
if they are determined by future expectations.
Monopolistic central banks, and the fractional reserve banking system as a whole,
merely represent a particular way of handling these issues.



\subsection{Strategies to get rid of debt}
\label{stgrod}

In what follows, three modes of getting rid of debt  are mentioned:
(i)
monetization of debt accompanied by (hyper)inflation,
(ii)
cutbacks in spending, and
(iii)
bankruptcy, the latter two being accompanied by recession.
Unfortunately, these seem to be the long-term options, as there is no ``free lunch.''

\subsubsection{Monetization of debt and (hyper)inflation}

Surely, not everybody can get rid of debt by eliminating his creditor in a style executed by Philipp IV of France against the Knights Templar in 1307.
Alternatively,
{\em inflation,} and even more so {\em hyperinflation,} is one of the major processes to get rid of debt.
The basic idea is quite simple: if debtors are able to keep interest payments at sustainable levels
in the first time of the loan,
they need not bother about the principal and the interest payments
at later times: because, relative to future price and income levels, inflation will ``melt away'' both the principals and interest;
i.e., quite literally,  debtors could pay back the principal and interest from their ``pocket money.''

This, the author believes, is a strategy employed on all scales; at least subconsciously, by small investors
acquiring home loans, and up to the government level.
In general, the higher the inflation, the faster is the relative reduction of debt,
as measured in absolute debt
divided by the absolute income;
but also the more difficult it is to sustain payments of interest in the first time of the loan.


\subsubsection{Cutbacks in spending}

Another possibility to get rid of debt is by allocating resources (such as taxes) to pay off debt.
Despite the fact that these measures are politically difficult to impose,
they are also bad for business activities, as they reduce the amount of money available in the economy,
and thus foster recession.


\subsubsection{Bankruptcy and ``haircuts''}

A third possibility is to ``write off'' debt by  bankruptcy or partial remission of a debt, called ``haircut'' nowadays.  Here the difficulty lies on the borrower's side;
if borrowers are large banks, this might have negative influences on all kinds of business activities,
and may again cause recession.




\section{Commodity {\it versus} fiat money}


With respect to types of money, there appear to be at least two major options:
(i) commodity based money and
(ii) fiat money.


Despite the obvious difference that a commodity based monetary system  is tied much stronger to the almost uncontrollable availability and abundance of commodity ---
culminating in the (economically negligible) ``production'' of gold from mercury through transmutation~\cite{PhysRev.60.473}
and the undesirable dependence of the amount of exchange money on the  aggregate amount of the commodity~\cite{Wicksell-geld-engl}  --- there exist other drawbacks as well.

In a commodity based monetary system it is impossible to increase the money supply by the mere expectation of future profits.
The resulting lack of liquidity cripples commodity money based economies with respect to others, in particular with respect to economic expansion and military defense.
From a financial point of view, the amount of military expansion is dominated by the arbitrary but strict limits on the commodities (mostly silver and gold).
Thus eventually any such commodity money based economy will fall prey to an economy based on fiat money.
This has happened, for instance, due to the expansionist (monetary and military) policy of {\it NAZI} Germany before 1938, who sacked \& absorbed the Austrian gold reserves after her occupation.

Thus, for pragmatic reasons, the only remaining alternative appears to be fiat money not directly backed by any commodity.
One may argue that the supply (or increase) of fiat money should somehow be linked to the gross domestic product,
but this can be abandoned from the outright for many reasons:
there is no direct control of fiat money once the system is set ``into motion.''
Indeed, the fiat money created by the financial sector, or by the aggregate of property,
by far outnumbers any kind of economic indicator even weakly linked to the gross domestic product.
So, fiat money can only be backed by the belief in it alone.




\section{Summary and outlook}

Some very general options for monetary systems have been enumerated and compared.
The creation of present fiat money {\it via monetization,} as well as its appropriation in various setups has been  discussed.
We have identified private banks as the main source of money through monetization.
Thereby, subject to reserve constraints,
banks absorb (debt related to) assets of value and in exchange issue fiat money in the form of quantity information in deposit money  accounts.
The asset value is inevitably determined by subjective beliefs and fantasies loosely bound by market constraints.
One may imagine such a monetary system as being ``suspended in thought;'' its continuity,
floating and benign evolution being guaranteed by common faith.

Any such system is vulnerable to crises and business cycles.
For instance, as asset values are subject to disinformation, fraudulent manipulation or hype in anticipation of future profits or losses,
there may be positive and negative feedbacks resulting in price settlements pushing certain equity segments far beyond a stable equilibrium with respect to the rest of  the markets.

Inevitably, the interest levied by banks in return for money created via monetizing debt systematically reallocates
resources toward the financial institutions, and away from industrial and manual production, farming and labor.

In prosperous times the abundance of fiat money guarantees a continuation of economic growth and individual welfare.
However, some very elementary dynamical tendencies drive economic systems into crises.
One of these tendencies is the Matthews Effect observed in many other configuration~\cite{merton-68}: it is the {\em accumulation} of desirable
entities at very few locations, accompanied by a {\em  thinning out} of these entities everywhere else;
sometimes stated as ``the rich getting richer and the poor getting poorer''~\cite{PlutonomyReport2}.

Another common illusion is the belief that Adam Smith's ``invisible hand'' always works for the ``greater good'' of economies;
when in reality economic players are confronted with boundless, relentless greed and prisoner's dilemmas of all sorts.

Unfortunately, the alternatives appear to be even more troublesome
than the present state of affairs. Any system based on interest-free fiat money creation,
in order to avoid hyperinflation through excessive borrowing associated with ``free'' debt,  has either to rely on unjustifiable privileges or chance.
And any system based on commodity money is heavily dependent on the
quantity of commodities, and also incapable of waging or defending against (economic) war through the effective monetization of future loot or loss.

So, what are the political, economic and social options?
Ought we, for instance, curb banks in their possibilities to create money?
Maybe we should, but if we overdo we cripple our economies by penalizing investments.
If we do not regulate them at all, we stimulate the natural greediness of people,
and foster pyramid scheme type unsustainable business models which assume ever increasing prices (money supply) resulting in economic crises.

The regulatory fine-tuning requires criteria of performance and reliable theories to forecast market behaviors; unfortunately we do not have any such instruments.
But even if such criteria and regulatory instruments will exist in the future --- which I doubt --- there might simply be not any possibility to prevent economic crises and the resulting
business cycles.
This may be due to the inherent self-referential character of economic processes, which tend to amplify gains and losses through market hysteria,
and which --- in a diagonalization type manner~\cite{smullyan-92} --- are capable of counteracting the very regulatory procedures which are established.


Ought we thus accept occasional monetary crises and the associated business cycles?
I am afraid, yes.

Ought we  accept imbalances of appropriation and a (geometric) redistribution of wealth toward ``the rich,''
and in particular toward the banks and other financial institutions, as well as other aggregates commanding ever increasing amounts of money?
I am afraid, yes. I am unaware of any measure which could counterbalance the accumulation of wealth,
also called the Matthews Effect, in the long run.


There are quite serious political connotations to keep in mind:
As money is the representation of a particular type of asset value, those who control and create money
have equivalent capabilities to deplore economic and political power.
It is quite commonly accepted that societies and empires may be ``steered'' or even dominated by those who have money~\cite{Quigley-TaH};
to the effect that ``money'' renders entire governments; or at least corrupts or overthrows them.
At some point we might wake up and realize that, facilitated by money,  our ``democracies'' have turned into
plutonomies~\cite{PlutonomyReport1,PlutonomyReport2} and oligarchies.
(In a recent {\em Bloomberg} commentary by William Pesek, Richard Duncan~\cite{dunken-crises2} is cited by saying
``The financial industry has become a menace to society,'' Duncan said.
``Its ability to create credit has brought it undue political influence, enabling the industry to deregulate itself and to engage
in such excesses [[$\ldots$]] Credit creation is too dangerous to be left to the discretion of bankers.'')

To close this brief discussion in a positive, pragmatic mood, let me mention ways to legally get rich along the monetary lines discussed,
without relying on inherited wealth:
(i)~One of the first and foremost opportunities would be to acquire or start up a central bank if some country would allow one to do so; possibly in exchange of a credit line.
(ii)~A fallback option would be to acquire or start up a noncentral bank, or some organization issuing notes which are accepted as some form of exchange payment.
(iii)~A third option would be to wait until chance singles one out as a beneficiary of the Matthews Effect. (This may never happen.)
(iv)~A fourth option would be to dynamically increase debt levels, which must be associated with sustainable levels of interest payments.
On a more existentialist and personal level, I propose to consider money as one of Sisyphus' more absurd assignments~\cite{camus-mos}.

%\bibliography{svozil}


%merlin.mbs 2010-03-15 4.21a (PWD, AO, DPC)
%Control: key (0)
%Control: author (0) dotless jnrlst
%Control: editor formatted (1) identically to author
%Control: production of article title (0) allowed
%Control: page (1) range
%Control: year (0) verbatim
%Control: production of eprint (0) enabled
\begin{thebibliography}{40}%
\makeatletter
\providecommand \@ifxundefined [1]{%
 \@ifx{#1\undefined}
}%
\providecommand \@ifnum [1]{%
 \ifnum #1\expandafter \@firstoftwo
 \else \expandafter \@secondoftwo
 \fi
}%
\providecommand \@ifx [1]{%
 \ifx #1\expandafter \@firstoftwo
 \else \expandafter \@secondoftwo
 \fi
}%
\providecommand \natexlab [1]{#1}%
\providecommand \enquote  [1]{``#1''}%
\providecommand \bibnamefont  [1]{#1}%
\providecommand \bibfnamefont [1]{#1}%
\providecommand \citenamefont [1]{#1}%
\providecommand \href@noop [0]{\@secondoftwo}%
\providecommand \href [0]{\begingroup \@sanitize@url \@href}%
\providecommand \@href[1]{\@@startlink{#1}\@@href}%
\providecommand \@@href[1]{\endgroup#1\@@endlink}%
\providecommand \@sanitize@url [0]{\catcode `\\12\catcode `\$12\catcode
  `\&12\catcode `\#12\catcode `\^12\catcode `\_12\catcode `\%12\relax}%
\providecommand \@@startlink[1]{}%
\providecommand \@@endlink[0]{}%
\providecommand \url  [0]{\begingroup\@sanitize@url \@url }%
\providecommand \@url [1]{\endgroup\@href {#1}{\urlprefix }}%
\providecommand \urlprefix  [0]{URL }%
\providecommand \Eprint [0]{\href }%
\@ifxundefined \urlstyle {%
  \providecommand \doi  [0]{\begingroup \@sanitize@url \@doi}%
  \providecommand \@doi [1]{\endgroup \@@startlink {\doibase
  #1}doi:\discretionary {}{}{}#1\@@endlink }%
}{%
  \providecommand \doi  [0]{doi:\discretionary{}{}{}\begingroup
  \urlstyle{rm}\Url }%
}%
\providecommand \doibase [0]{http://dx.doi.org/}%
\providecommand \Doi [0]{\begingroup \@sanitize@url \@Doi }%
\providecommand \@Doi  [1]{\endgroup\@@startlink{\doibase#1}\@@Doi}%
\providecommand \@@Doi [1]{#1\@@endlink}%
\providecommand \selectlanguage [0]{\@gobble}%
\providecommand \bibinfo  [0]{\@secondoftwo}%
\providecommand \bibfield  [0]{\@secondoftwo}%
\providecommand \translation [1]{[#1]}%
\providecommand \BibitemOpen [0]{}%
\providecommand \bibitemStop [0]{}%
\providecommand \bibitemNoStop [0]{.\EOS\space}%
\providecommand \EOS [0]{\spacefactor3000\relax}%
\providecommand \BibitemShut  [1]{\csname bibitem#1\endcsname}%
%</preamble>
\bibitem [{\citenamefont {Samuelson}\ and\ \citenamefont
  {Nordhaus}(1948-2004)}]{1948-Samuelson}%
  \BibitemOpen
  \bibfield  {author} {\bibinfo {author} {\bibfnamefont {Paul~A.}\ \bibnamefont
  {Samuelson}}\ and\ \bibinfo {author} {\bibfnamefont {William~D.}\
  \bibnamefont {Nordhaus}},\ }\href@noop {} {\emph {\bibinfo {title}
  {Economics}}}\ (\bibinfo  {publisher} {McGraw-Hill},\ \bibinfo {address} {New
  York, NY},\ \bibinfo {year} {1948-2004})\BibitemShut {NoStop}%
\bibitem [{\citenamefont {Begg}\ \emph {et~al.}(2005)\citenamefont {Begg},
  \citenamefont {Fischer},\ and\ \citenamefont {Dornbusch}}]{Begg}%
  \BibitemOpen
  \bibfield  {author} {\bibinfo {author} {\bibfnamefont {David}\ \bibnamefont
  {Begg}}, \bibinfo {author} {\bibfnamefont {Stanley}\ \bibnamefont {Fischer}},
  \ and\ \bibinfo {author} {\bibfnamefont {Rudiger}\ \bibnamefont
  {Dornbusch}},\ }\href@noop {} {\emph {\bibinfo {title} {Economics (Eigth
  Edition)}}}\ (\bibinfo  {publisher} {McGraw-Hill},\ \bibinfo {address}
  {London},\ \bibinfo {year} {2005})\BibitemShut {NoStop}%
\bibitem [{\citenamefont {Kyrer}\ and\ \citenamefont
  {Penker}(2000)}]{kyrer-penker-vwl}%
  \BibitemOpen
  \bibfield  {author} {\bibinfo {author} {\bibfnamefont {Alfred}\ \bibnamefont
  {Kyrer}}\ and\ \bibinfo {author} {\bibfnamefont {Walter}\ \bibnamefont
  {Penker}},\ }\href@noop {} {\emph {\bibinfo {title} {{V}olkswirtschaftslehre.
  {G}rundz{\"{u}}ge der {W}irtschaftstheorie und -politik}}},\ \bibinfo
  {edition} {6th}\ ed.\ (\bibinfo  {publisher} {Oldenbourg},\ \bibinfo
  {address} {M{\"{u}}nchen, Wien},\ \bibinfo {year} {2000})\BibitemShut
  {NoStop}%
\bibitem [{\citenamefont {Soros}(1987, 1994, 2003)}]{soros-alchemy}%
  \BibitemOpen
  \bibfield  {author} {\bibinfo {author} {\bibfnamefont {George}\ \bibnamefont
  {Soros}},\ }\href@noop {} {\emph {\bibinfo {title} {The Alchemy of Finance:
  Reading the Mind of the Market}}}\ (\bibinfo  {publisher} {John Wiley \&
  Sons},\ \bibinfo {address} {Hoboken, New Jersey},\ \bibinfo {year} {1987,
  1994, 2003})\BibitemShut {NoStop}%
\bibitem [{\citenamefont {Bouchaud}(2008)}]{Bouchaud-08}%
  \BibitemOpen
  \bibfield  {author} {\bibinfo {author} {\bibfnamefont {Jean-Philippe}\
  \bibnamefont {Bouchaud}},\ }\bibfield  {title} {\enquote {\bibinfo {title}
  {Economics needs a scientific revolution},}\ }\Doi {10.1038/4551181a}
  {\bibfield  {journal} {\bibinfo  {journal} {Nature},\ }\textbf {\bibinfo
  {volume} {455}},\ \bibinfo {pages} {1181} (\bibinfo {year} {2008})},\ ISSN
  \bibinfo {issn} {0028-0836}\BibitemShut {NoStop}%
\bibitem [{\citenamefont {Sibert}(2010)}]{Sibert10}%
  \BibitemOpen
  \bibfield  {author} {\bibinfo {author} {\bibfnamefont {Anne}\ \bibnamefont
  {Sibert}},\ }\href {http://voxeu.org/index.php?q=node/5059} {\enquote
  {\bibinfo {title} {Love letters from {I}celand: Accountability of the
  {E}urosystem},}\ } (\bibinfo {year} {2010}),\ \bibinfo {note} {vox posting
  from 18 May 2010}\BibitemShut {NoStop}%
\bibitem [{\citenamefont {J}(2009)}]{Flannery2009}%
  \BibitemOpen
  \bibfield  {author} {\bibinfo {author} {\bibfnamefont {Mark}\ \bibnamefont
  {J}},\ }\href {http://sic.althingi.is/pdf/RNAvefVidauki3Enska.pdf} {\enquote
  {\bibinfo {title} {Iceland�s failed banks: A post-mortem},}\ } (\bibinfo
  {year} {2009}),\ \bibinfo {note} {report prepared for the Icelandic Special
  Investigation Commission, 9 March 2009}\BibitemShut {NoStop}%
\bibitem [{\citenamefont {P{\'a}ll}\ \emph {et~al.}(2009)\citenamefont
  {P{\'a}ll}, \citenamefont {Tryggvi},\ and\ \citenamefont
  {Sigr�dur}}]{Hreinsson2009}%
  \BibitemOpen
  \bibfield  {author} {\bibinfo {author} {\bibnamefont {P{\'a}ll}}, \bibinfo
  {author} {\bibfnamefont {Gunnarsson}\ \bibnamefont {Tryggvi}}, \ and\
  \bibinfo {author} {\bibfnamefont {Benediktsd{\'o}ttir}\ \bibnamefont
  {Sigr�dur}},\ }\href {http://sic.althingi.is/pdf/RNAvefurKafli21Enska.pdf}
  {\enquote {\bibinfo {title} {Causes of the collapse of the icelanldic banks �
  responsibility, mistakes and negligence},}\ } (\bibinfo {year} {2009}),\
  \bibinfo {note} {report prepared for the Icelandic Special Investigation
  Commission, 9 March 2009}\BibitemShut {NoStop}%
\bibitem [{\citenamefont {{European Central Bank.
  Eurosystem}}(2009)}]{ECBank2009}%
  \BibitemOpen
  \bibfield  {author} {\bibinfo {author} {\bibnamefont {{European Central Bank.
  Eurosystem}}},\ }\href
  {http://www.ecb.int/press/pr/date/2009/html/pr090305_2.en.html} {\enquote
  {\bibinfo {title} {5 {M}arch 2009 - {E}urosystem monetary policy operations
  in 2008},}\ } (\bibinfo {year} {2009}),\ \bibinfo {note} {press
  Release}\BibitemShut {NoStop}%
\bibitem [{\citenamefont {Schneider}(1952-1973)}]{schneider-VWLIII}%
  \BibitemOpen
  \bibfield  {author} {\bibinfo {author} {\bibfnamefont {Erich}\ \bibnamefont
  {Schneider}},\ }\href@noop {} {\emph {\bibinfo {title} {{E}inf{\"{u}}hrung in
  die {W}irtschaftstheorie III. {G}eld, {K}redit, {V}olkseinkommen und
  {B}esch{\"{a}}ftigung}}}\ (\bibinfo  {publisher} {J.C.B.Mohr (Paul
  Siebeck)},\ \bibinfo {address} {T{\"{u}}bingen},\ \bibinfo {year}
  {1952-1973})\BibitemShut {NoStop}%
\bibitem [{\citenamefont {Nichols}(1961, 1992)}]{ModernMoneyMechanics}%
  \BibitemOpen
  \bibfield  {author} {\bibinfo {author} {\bibfnamefont {Dorothy~M.}\
  \bibnamefont {Nichols}},\ }\href@noop {} {\emph {\bibinfo {title} {Modern
  Money Mechanics}}}\ (\bibinfo  {publisher} {Federal Reserve Bank of
  Chicago},\ \bibinfo {address} {Chicago},\ \bibinfo {year} {1961,
  1992})\BibitemShut {NoStop}%
\bibitem [{\citenamefont {{European Central Bank.
  Eurosystem}}(2008)}]{EUROMonPol08}%
  \BibitemOpen
  \bibfield  {author} {\bibinfo {author} {\bibnamefont {{European Central Bank.
  Eurosystem}}},\ }\href {http://www.ecb.int/pub/pdf/other/gendoc2008en.pdf}
  {\emph {\bibinfo {title} {The Implementation of Monetary Policy in the {EURO}
  Area}}}\ (\bibinfo  {publisher} {European Central Bank},\ \bibinfo {address}
  {Frankfurt},\ \bibinfo {year} {2008})\BibitemShut {NoStop}%
\bibitem [{\citenamefont {{Federal Reserve}}(2009)}]{FRAC2009}%
  \BibitemOpen
  \bibfield  {author} {\bibinfo {author} {\bibnamefont {{Federal Reserve}}},\
  }\href {http://www.frbdiscountwindow.org/acceptancecriteria.pdf} {\enquote
  {\bibinfo {title} {Discount window \& payment system risk acceptance criteria
  for commonly pledged asset types},}\ } (\bibinfo {year} {2009}),\ \bibinfo
  {note} {{F}ederal {R}eserve release}\BibitemShut {NoStop}%
\bibitem [{\citenamefont {Smith}\ and\ \citenamefont
  {Walter}(2009)}]{smith-walter-09}%
  \BibitemOpen
  \bibfield  {author} {\bibinfo {author} {\bibfnamefont {Roy~C.}\ \bibnamefont
  {Smith}}\ and\ \bibinfo {author} {\bibfnamefont {Ingo}\ \bibnamefont
  {Walter}},\ }\href {http://archive.nyu.edu/handle/2451/26532} {\enquote
  {\bibinfo {title} {Rating agencies: Is there an agency issue?}}\ } (\bibinfo
  {year} {2009}),\ \bibinfo {note} {fIN-01-003}\BibitemShut {NoStop}%
\bibitem [{\citenamefont {B{\"{o}}hm-Bawerk}(1884)}]{Boehm-Bawerk}%
  \BibitemOpen
  \bibfield  {author} {\bibinfo {author} {\bibfnamefont {Eugen}\ \bibnamefont
  {B{\"{o}}hm-Bawerk}},\ }\href
  {http://www.archive.org/details/kapitalundkapita01bh} {\emph {\bibinfo
  {title} {{K}apital und {K}apitalzins}}}\ (\bibinfo  {publisher} {Kluwer
  Academic Publishers},\ \bibinfo {year} {1884})\BibitemShut {NoStop}%
\bibitem [{\citenamefont {Pesek}\ and\ \citenamefont
  {Saving}(1968)}]{pesek-saving_1968}%
  \BibitemOpen
  \bibfield  {author} {\bibinfo {author} {\bibfnamefont {Boris~P.}\
  \bibnamefont {Pesek}}\ and\ \bibinfo {author} {\bibfnamefont {Thomas~R.}\
  \bibnamefont {Saving}},\ }\href@noop {} {\emph {\bibinfo {title} {The
  Foundations of Money and Banking}}}\ (\bibinfo  {publisher} {The Macmillan
  Company},\ \bibinfo {address} {New York},\ \bibinfo {year}
  {1968})\BibitemShut {NoStop}%
\bibitem [{\citenamefont {Binswanger}(2006)}]{2006-Binswanger}%
  \BibitemOpen
  \bibfield  {author} {\bibinfo {author} {\bibfnamefont {Hans~Christoph}\
  \bibnamefont {Binswanger}},\ }\href@noop {} {\emph {\bibinfo {title} {{D}ie
  {W}achstumsspirale}}},\ \bibinfo {edition} {2nd}\ ed.\ (\bibinfo  {publisher}
  {Metropolis-Verlag},\ \bibinfo {address} {Marburg},\ \bibinfo {year}
  {2006})\BibitemShut {NoStop}%
\bibitem [{\citenamefont {Champ}\ and\ \citenamefont
  {Freeman}(2004)}]{champ-freeman-2004}%
  \BibitemOpen
  \bibfield  {author} {\bibinfo {author} {\bibfnamefont {Bruce}\ \bibnamefont
  {Champ}}\ and\ \bibinfo {author} {\bibfnamefont {Scott}\ \bibnamefont
  {Freeman}},\ }\href@noop {} {\emph {\bibinfo {title} {Modelling Monetary
  Economies}}}\ (\bibinfo  {publisher} {Cambridge University Press},\ \bibinfo
  {address} {Cambridge, UK},\ \bibinfo {year} {2004})\BibitemShut {NoStop}%
\bibitem [{\citenamefont {{Noonan, jr.}}(1957)}]{noonan-57}%
  \BibitemOpen
  \bibfield  {author} {\bibinfo {author} {\bibfnamefont {John~Thomas}\
  \bibnamefont {{Noonan, jr.}}},\ }\href@noop {} {\emph {\bibinfo {title} {The
  scholastic analysis of usury}}}\ (\bibinfo  {publisher} {Harvard University
  Press},\ \bibinfo {address} {Cambridge, Mass.},\ \bibinfo {year}
  {1957})\BibitemShut {NoStop}%
\bibitem [{\citenamefont {Wei{\ss}}(2005)}]{Weiss-Zinswucher-Christ}%
  \BibitemOpen
  \bibfield  {author} {\bibinfo {author} {\bibfnamefont {Andreas~M.}\
  \bibnamefont {Wei{\ss}}},\ }\bibfield  {title} {\enquote {\bibinfo {title}
  {{Z}insen und {W}ucher. {D}as kirchliche {Z}insverbot und die {H}indernisse
  auf dem {W}eg zu seiner {K}orrektur},}\ }in\ \Doi {10.1007/3-211-28108-8_5}
  {\emph {\bibinfo {booktitle} {{G}eld- und {K}reditwesen im {S}piegel der
  {W}issenschaft}}},\ \bibinfo {editor} {edited by\ \bibinfo {editor}
  {\bibfnamefont {Ulrike}\ \bibnamefont {Aichhorn}}}\ (\bibinfo  {publisher}
  {Springer},\ \bibinfo {address} {Vienna},\ \bibinfo {year} {2005})\ pp.\
  \bibinfo {pages} {123--156}\BibitemShut {NoStop}%
\bibitem [{\citenamefont {Hanke}(2005)}]{Hanke-Zinswucher-Islam}%
  \BibitemOpen
  \bibfield  {author} {\bibinfo {author} {\bibfnamefont {Marcus}\ \bibnamefont
  {Hanke}},\ }\bibfield  {title} {\enquote {\bibinfo {title} {{Z}insverbot und
  islamische {B}ank. {V}on {D}atteln und {K}reditkarten},}\ }in\ \Doi
  {10.1007/3-211-28108-8_6} {\emph {\bibinfo {booktitle} {{G}eld- und
  {K}reditwesen im {S}piegel der {W}issenschaft}}},\ \bibinfo {editor} {edited
  by\ \bibinfo {editor} {\bibfnamefont {Ulrike}\ \bibnamefont {Aichhorn}}}\
  (\bibinfo  {publisher} {Springer},\ \bibinfo {address} {Vienna},\ \bibinfo
  {year} {2005})\ pp.\ \bibinfo {pages} {157--175}\BibitemShut {NoStop}%
\bibitem [{\citenamefont {Friedman}(2008)}]{Friedman-2008}%
  \BibitemOpen
  \bibfield  {author} {\bibinfo {author} {\bibfnamefont {Milton}\ \bibnamefont
  {Friedman}},\ }\bibfield  {title} {\enquote {\bibinfo {title} {Quantity
  theory of money},}\ }in\ \Doi {10.1057/9780230226203.1374} {\emph {\bibinfo
  {booktitle} {The New Palgrave Dictionary of Economics. Second Edition}}},\
  \bibinfo {editor} {edited by\ \bibinfo {editor} {\bibfnamefont {Steven~N.}\
  \bibnamefont {Durlauf}}\ and\ \bibinfo {editor} {\bibfnamefont {Lawrence~E.}\
  \bibnamefont {Blume}}}\ (\bibinfo  {publisher} {Palgrave Macmillan},\
  \bibinfo {year} {2008})\BibitemShut {NoStop}%
\bibitem [{\citenamefont {Reinhart}\ and\ \citenamefont
  {Rogoff}(2009)}]{ReinhartRogoff}%
  \BibitemOpen
  \bibfield  {author} {\bibinfo {author} {\bibfnamefont {Carmen~M.}\
  \bibnamefont {Reinhart}}\ and\ \bibinfo {author} {\bibfnamefont {Kenneth~S.}\
  \bibnamefont {Rogoff}},\ }\href@noop {} {\emph {\bibinfo {title} {This Time
  Is Different: Eight Centuries of Financial Folly}}}\ (\bibinfo  {publisher}
  {Princeton University Press},\ \bibinfo {address} {Princeton, New Jersey},\
  \bibinfo {year} {2009})\BibitemShut {NoStop}%
\bibitem [{\citenamefont {Bonner}\ and\ \citenamefont
  {Wiggin}(2006)}]{BonnerWiggin}%
  \BibitemOpen
  \bibfield  {author} {\bibinfo {author} {\bibfnamefont {William}\ \bibnamefont
  {Bonner}}\ and\ \bibinfo {author} {\bibfnamefont {Addison}\ \bibnamefont
  {Wiggin}},\ }\href@noop {} {\emph {\bibinfo {title} {Empire of Debt: The Rise
  of an Epic Financial Crisis}}}\ (\bibinfo  {publisher} {Wiley \& Sons,
  Inc.},\ \bibinfo {address} {Hoboken, New Jersey},\ \bibinfo {year}
  {2006})\BibitemShut {NoStop}%
\bibitem [{\citenamefont {Stiglitz}\ and\ \citenamefont
  {Bilmes}(2008)}]{StiglitzBilmes}%
  \BibitemOpen
  \bibfield  {author} {\bibinfo {author} {\bibfnamefont {Joseph~E}\
  \bibnamefont {Stiglitz}}\ and\ \bibinfo {author} {\bibfnamefont {Linda}\
  \bibnamefont {Bilmes}},\ }\href@noop {} {\emph {\bibinfo {title} {The Three
  Trillion Dollar War}}}\ (\bibinfo  {publisher} {W. W. Norton},\ \bibinfo
  {address} {New York},\ \bibinfo {year} {2008})\BibitemShut {NoStop}%
\bibitem [{\citenamefont {Duncan}(2005)}]{dunken-crises2}%
  \BibitemOpen
  \bibfield  {author} {\bibinfo {author} {\bibfnamefont {Richard}\ \bibnamefont
  {Duncan}},\ }\href@noop {} {\emph {\bibinfo {title} {The Dollar Crisis:
  Causes, Consequences, Cures, Revised and Updated}}},\ \bibinfo {edition}
  {2nd}\ ed.\ (\bibinfo  {publisher} {Wiley},\ \bibinfo {address} {New York},\
  \bibinfo {year} {2005})\ ISBN \bibinfo {isbn} {978-0-470-82170-1}\BibitemShut
  {NoStop}%
\bibitem [{\citenamefont {Kauffman}(1987)}]{Kauffman198753}%
  \BibitemOpen
  \bibfield  {author} {\bibinfo {author} {\bibfnamefont {Louis~H.}\
  \bibnamefont {Kauffman}},\ }\bibfield  {title} {\enquote {\bibinfo {title}
  {Self-reference and recursive forms},}\ }\Doi {10.1016/0140-1750(87)90034-0}
  {\bibfield  {journal} {\bibinfo  {journal} {Journal of Social and Biological
  Structures},\ }\textbf {\bibinfo {volume} {10}},\ \bibinfo {pages} {53--72}
  (\bibinfo {year} {1987})},\ ISSN \bibinfo {issn} {0140-1750}\BibitemShut
  {NoStop}%
\bibitem [{\citenamefont {G{\"{o}}del}(1931)}]{godel1}%
  \BibitemOpen
  \bibfield  {author} {\bibinfo {author} {\bibfnamefont {Kurt}\ \bibnamefont
  {G{\"{o}}del}},\ }\bibfield  {title} {\enquote {\bibinfo {title} {{\"{U}}ber
  formal unentscheidbare {S\"{a}}tze der {P}rincipia {M}athematica und
  verwandter {S}ysteme},}\ }\href@noop {} {\bibfield  {journal} {\bibinfo
  {journal} {Monatshefte f{\"{u}}r Mathematik und Physik},\ }\textbf {\bibinfo
  {volume} {38}},\ \bibinfo {pages} {173--198} (\bibinfo {year} {1931})},\
  \bibinfo {note} {{E}nglish translation in Ref.~\cite{godel-ges1}, and in
  Ref.~\cite{davis}}\BibitemShut {NoStop}%
\bibitem [{\citenamefont {{T}uring}(1936-7 and 1937)}]{turing-36}%
  \BibitemOpen
  \bibfield  {author} {\bibinfo {author} {\bibfnamefont {A.~M.}\ \bibnamefont
  {{T}uring}},\ }\bibfield  {title} {\enquote {\bibinfo {title} {On computable
  numbers, with an application to the {E}ntscheidungsproblem},}\ }\href@noop {}
  {\bibfield  {journal} {\bibinfo  {journal} {Proceedings of the London
  Mathematical Society, Series 2},\ }\textbf {\bibinfo {volume} {42, 43}},\
  \bibinfo {pages} {230--265, 544--546} (\bibinfo {year} {1936-7 and 1937})},\
  \bibinfo {note} {reprinted in Ref.~\cite{davis}}\BibitemShut {NoStop}%
\bibitem [{\citenamefont {Davis}(1958)}]{davis-58}%
  \BibitemOpen
  \bibfield  {author} {\bibinfo {author} {\bibfnamefont {Martin}\ \bibnamefont
  {Davis}},\ }\href@noop {} {\emph {\bibinfo {title} {Computability and
  Unsolvability}}}\ (\bibinfo  {publisher} {McGraw-Hill},\ \bibinfo {address}
  {New York},\ \bibinfo {year} {1958})\BibitemShut {NoStop}%
\bibitem [{\citenamefont {Smullyan}(1992)}]{smullyan-92}%
  \BibitemOpen
  \bibfield  {author} {\bibinfo {author} {\bibfnamefont {Raymond~M.}\
  \bibnamefont {Smullyan}},\ }\href@noop {} {\emph {\bibinfo {title}
  {{G}{\"{o}}del's Incompleteness Theorems}}}\ (\bibinfo  {publisher} {Oxford
  University Press},\ \bibinfo {address} {New York, New York},\ \bibinfo {year}
  {1992})\BibitemShut {NoStop}%
\bibitem [{\citenamefont {Sherr}\ \emph {et~al.}(1941)\citenamefont {Sherr},
  \citenamefont {Bainbridge},\ and\ \citenamefont {Anderson}}]{PhysRev.60.473}%
  \BibitemOpen
  \bibfield  {author} {\bibinfo {author} {\bibfnamefont {R.}~\bibnamefont
  {Sherr}}, \bibinfo {author} {\bibfnamefont {K.~T.}\ \bibnamefont
  {Bainbridge}}, \ and\ \bibinfo {author} {\bibfnamefont {H.~H.}\ \bibnamefont
  {Anderson}},\ }\bibfield  {title} {\enquote {\bibinfo {title} {Transmutation
  of mercury by fast neutrons},}\ }\Doi {10.1103/PhysRev.60.473} {\bibfield
  {journal} {\bibinfo  {journal} {Physical Review},\ }\textbf {\bibinfo
  {volume} {60}},\ \bibinfo {pages} {473--479} (\bibinfo {year}
  {1941})}\BibitemShut {NoStop}%
\bibitem [{\citenamefont {Wicksell}(1936)}]{Wicksell-geld-engl}%
  \BibitemOpen
  \bibfield  {author} {\bibinfo {author} {\bibfnamefont {Johan Gustav~Knut}\
  \bibnamefont {Wicksell}},\ }\href
  {http://www.archive.org/details/interestandprice033322mbp} {\emph {\bibinfo
  {title} {Interest and prices. {A} study of the causes regulating the value of
  money}}}\ (\bibinfo  {publisher} {Macmillan And Company Limited},\ \bibinfo
  {address} {London},\ \bibinfo {year} {1936})\BibitemShut {NoStop}%
\bibitem [{\citenamefont {Merton}(1968)}]{merton-68}%
  \BibitemOpen
  \bibfield  {author} {\bibinfo {author} {\bibfnamefont {Robert~K.}\
  \bibnamefont {Merton}},\ }\bibfield  {title} {\enquote {\bibinfo {title} {The
  {M}atthew effect in science},}\ }\Doi {10.1126/science.159.3810.56}
  {\bibfield  {journal} {\bibinfo  {journal} {Science},\ }\textbf {\bibinfo
  {volume} {159}},\ \bibinfo {pages} {56--63} (\bibinfo {year}
  {1968})}\BibitemShut {NoStop}%
\bibitem [{\citenamefont {Citigroup}(March 5, 2006)}]{PlutonomyReport2}%
  \BibitemOpen
  \bibfield  {author} {\bibinfo {author} {\bibnamefont {Citigroup}},\ }\href
  {http://www.scribd.com/doc/6674229/Citigroup-Mar-5-2006-Plutonomy-Report-Par%
t-2} {\enquote {\bibinfo {title} {Reviting plutonomy: The rich getting
  richer},}\ } (\bibinfo {year} {March 5, 2006})\BibitemShut {NoStop}%
\bibitem [{\citenamefont {Quigley}(1966)}]{Quigley-TaH}%
  \BibitemOpen
  \bibfield  {author} {\bibinfo {author} {\bibfnamefont {Carroll}\ \bibnamefont
  {Quigley}},\ }\href@noop {} {\emph {\bibinfo {title} {Tragedy and Hope: A
  History of the World in Our Time}}}\ (\bibinfo  {publisher} {Macmillan},\
  \bibinfo {address} {New York},\ \bibinfo {year} {1966})\BibitemShut {NoStop}%
\bibitem [{\citenamefont {Citigroup}(October 16, 2005)}]{PlutonomyReport1}%
  \BibitemOpen
  \bibfield  {author} {\bibinfo {author} {\bibnamefont {Citigroup}},\ }\href
  {http://www.scribd.com/doc/6674234/Citigroup-Oct-16-2005-Plutonomy-Report-Pa%
rt-1} {\enquote {\bibinfo {title} {Plutonomy: Buying luxury, explaining global
  imbalances},}\ } (\bibinfo {year} {October 16, 2005})\BibitemShut {NoStop}%
\bibitem [{\citenamefont {Camus}(1942)}]{camus-mos}%
  \BibitemOpen
  \bibfield  {author} {\bibinfo {author} {\bibfnamefont {Albert}\ \bibnamefont
  {Camus}},\ }\href@noop {} {\emph {\bibinfo {title} {Le Mythe de Sisyphe
  (English translation: The Myth of Sisyphus)}}}\ (\bibinfo {year}
  {1942})\BibitemShut {NoStop}%
\bibitem [{\citenamefont {G{\"{o}}del}(1986)}]{godel-ges1}%
  \BibitemOpen
  \bibfield  {author} {\bibinfo {author} {\bibfnamefont {Kurt}\ \bibnamefont
  {G{\"{o}}del}},\ }in\ \href@noop {} {\emph {\bibinfo {booktitle} {Collected
  Works. Publications 1929-1936. Volume {I}}}},\ \bibinfo {editor} {edited by\
  \bibinfo {editor} {\bibfnamefont {S.}~\bibnamefont {Feferman}}, \bibinfo
  {editor} {\bibfnamefont {J.~W.}\ \bibnamefont {Dawson}}, \bibinfo {editor}
  {\bibfnamefont {S.~C.}\ \bibnamefont {Kleene}}, \bibinfo {editor}
  {\bibfnamefont {G.~H.}\ \bibnamefont {Moore}}, \bibinfo {editor}
  {\bibfnamefont {R.~M.}\ \bibnamefont {Solovay}}, \ and\ \bibinfo {editor}
  {\bibfnamefont {J.}~\bibnamefont {van Heijenoort}}}\ (\bibinfo  {publisher}
  {Oxford University Press},\ \bibinfo {address} {Oxford},\ \bibinfo {year}
  {1986})\BibitemShut {NoStop}%
\bibitem [{\citenamefont {Davis}(1965)}]{davis}%
  \BibitemOpen
  \bibfield  {author} {\bibinfo {author} {\bibfnamefont {Martin}\ \bibnamefont
  {Davis}},\ }\href@noop {} {\emph {\bibinfo {title} {The Undecidable. Basic
  Papers on Undecidable, Unsolvable Problems and Computable Functions}}}\
  (\bibinfo  {publisher} {Raven Press},\ \bibinfo {address} {Hewlett, N.Y.},\
  \bibinfo {year} {1965})\BibitemShut {NoStop}%
\end{thebibliography}%

\end{document}

