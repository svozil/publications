%\documentclass[amsmath,table,sans,amsfonts, handout]{beamer}
\documentclass[amsmath,table,sans,amsfonts]{beamer}
\usepackage[T1]{fontenc}
%%\usepackage{beamerthemeshadow}
%%\usepackage[headheight=1pt,footheight=10pt]{beamerthemeboxes}
%%\addfootboxtemplate{\color{structure!80}}{\color{white}\tiny \hfill Karl Svozil (TU Vienna)\hfill}
%%\addfootboxtemplate{\color{structure!65}}{\color{white}\tiny \hfill mur.sat \hfill}
%%\addfootboxtemplate{\color{structure!50}}{\color{white}\tiny \hfill Graz, 2010-12-11\hfill}
%\usepackage[dark]{beamerthemesidebar}
%\usepackage[headheight=24pt,footheight=12pt]{beamerthemesplit}
%\usepackage{beamerthemesplit}
%\usepackage[bar]{beamerthemetree}
\usepackage{graphicx}
\usepackage{pgf}
%\usepackage{eepic}
%\usepackage[usenames]{color}
%\newcommand{\Red}{\color{Red}}  %(VERY-Approx.PANTONE-RED)
%\newcommand{\Green}{\color{Green}}  %(VERY-Approx.PANTONE-GREEN)

\usepackage{musixtex}

%\RequirePackage[german]{babel}
%\selectlanguage{german}
%\RequirePackage[isolatin]{inputenc}

%\pgfdeclareimage[height=0.5cm]{logo}{tu-logo}
%\logo{\pgfuseimage{logo}}
\beamertemplatetriangleitem
%\beamertemplateballitem

\beamerboxesdeclarecolorscheme{alert}{red}{red!15!averagebackgroundcolor}
%\begin{beamerboxesrounded}[scheme=alert,shadow=true]{}
%\end{beamerboxesrounded}

%\beamersetaveragebackground{yellow!10}

%\beamertemplatecircleminiframe

\newtheorem{question}{Question}
\newtheorem{conjecture}[question]{Principle}
\newtheorem{challenge}[question]{Challenge}
\usepackage{tikz}
\newcommand{\bra}[1]{\left< #1 \right|}
\newcommand{\ket}[1]{\left| #1 \right>}

\newcommand{\iprod}[2]{\langle #1 | #2 \rangle}
\newcommand{\mprod}[3]{\langle #1 | #2 | #3 \rangle}
\newcommand{\oprod}[2]{| #1 \rangle\langle #2 |}

\newcommand{\proj}[3]{\begin{smallmatrix} #1 & #2 & #3 \end{smallmatrix}}
\newcommand{\projbf}[3]{\begin{smallmatrix} \mathbf{#1} & \mathbf{#2} & \mathbf{#3} \end{smallmatrix}}

\sloppy
\parskip .7em %vskip between paragraphs

\newcommand{\seq}[1]{\mathbf{#1}}
\newcommand{\floor}[1]{\left\lfloor #1 \right\rfloor}
\newcommand{\ceil}[1]{\left\lceil #1 \right\rceil}
\newcommand{\m}[1]{\widetilde{#1}}
%\newcommand{\p}[1]{\scriptsize\textcolor{black}{$[#1]$}}

\begin{document}

\title{\bf \textcolor{blue}{Quantum versus classical physics and their possible
relationship to cognitive phenomena such as music}}
\subtitle{\textcolor{purple!60}{\url{http://tph.tuwien.ac.at/~svozil/publ/2021-CICC-DICTA-PIA-UTalca-pres.pdf}
\\
%http://arxiv.org/abs/1206.6024
}}
\author{Karl Svozil}
\institute{ITP TU Wien, Vienna Austria\\
svozil@tuwien.ac.at
%{\tiny Disclaimer: Die hier vertretenen Meinungen des Autors verstehen sich als Diskussionsbeitr�ge und decken sich nicht notwendigerweise mit den Positionen der Technischen Universit�t Wien oder deren Vertreter.}
}
\date{Talca (Chile) {\&} Vienna (Austria), Friday, January 8, 2020}
\maketitle


% \frame{
% \frametitle{Contents}
% \tableofcontents
% }

\section{Very brief introduction to quantum mechanics}


 \frame{
 \frametitle{Scientific modelling -- a caveat}

{\color{purple}
The bigger question wrt scientific modelling is that of what is an acceptance ``truth'' or ``fact''.

Often issues are multi-layered and appear to be an ambiguous, even inconsistent patchwork rather than a consistently woven carpet.

We have  in our minds a model, a narrative -- Hertz calls it an ``image'' -- of the world. This model is rather stable in many areas,
and weakly determined and sketchy in others. It is based on our own experience, on deductive reasoning relative to believes in grounding axioms, and on acceptance of ``ecclesial authority''.

Usually it is easy to corroborate or falsify certain junks of this model, in particular, if phenomena are empirically reproducible: think of ``switching on the lights''.
}}

 \frame{
 \frametitle{Scientific modelling -- a caveat cntd.}

{\color{purple}

But there exist other situations when certain phenomena are not reproducible at will; or lack what we consider ``explanation'' (telos) or appear to be confusing and contradictory.

Think of ball lighting, or certain astronomical events -- such as meteorites or gamma-ray bursts -- which occur sporadically and cannot be (re)produced, which took some time to enter science proper.

When it comes to claims of ESP or UFO/UAP/AAP the situation gets blended with personal emotions, anxieties and even evangelical aspirations.

This is also true for ``interpretations'' of the quantum mechanical formalism. In particular, beware of the quantum ``hocus pocus''!

And keep in mind that all theories are temporal, and science is a historic process far from ``completion''.
 }

}

 \frame{
 \frametitle{Quantum formalism 101 as a theory of vectors}


\begin{itemize}
\pause
\item {\color{purple}{pure states and elementary binary observables/propositions}} are (unit) vectors or the one-dimensional projection operators spanned by them;
\pause
\item {\color{purple}{context or maximal knowledge of a physical system}} are orthonormal bases representing sets of elementary binary observables/propositions wihich are
\begin{itemize}
\item {\color{blue}{mutually exclusive}} aka orthogonal (need scalar/inner product for orthogonality, thus Hilbert space) as well as
\item {\color{blue}{complete}} ie they span the entire vector space;
\end{itemize}
\pause
\item {\color{purple}{state evolution}} is a generalized (unitary) permutation/rotation of some orthonormal basis aka ``frame'' into another one;
\pause
\item {\color{purple}{probability}} is defined in terms of generalized projections (Gleason's ``derivation of the Born rule): take a state vector $\vert \psi \rangle$,
take some elementary observable $\vert \phi \rangle$, then the probability of the occurrence (frequentist)/expectation (Bayesian) of observable $\phi$ given $\psi$ is
$\vert \langle \phi \mid \psi \rangle \vert^2$, where $ \langle \cdot \mid \cdot \rangle$ stands for the scalar/inner product.
\end{itemize}

}

 \frame{
 \frametitle{Coherent superposition aka linear combination of states}

\begin{itemize}
\pause
\item {\color{purple}{Representation of state vectors in terms of different bases}} results in linear combination of basis vectors, eg
$\vert \psi \rangle = \vert {\bf b}_1 \rangle +  \vert {\bf b}_2 \rangle + \cdots  \vert {\bf b}_n \rangle$;
\pause
\item {\color{purple}{Claims of quantum parallelism}}
$\vert \psi \rangle$ ``co-represents'' $n$ mutually exclusive basis states $\vert {\bf b}_1 \rangle$,
$\vert {\bf b}_2 \rangle$, $\ldots  \vert {\bf b}_n \rangle$;
\pause
\item Wrt the basis/frame in which $\vert \psi \rangle$ is an element, the quantized system is {\color{purple}{value definite}}:
any measurement of $\vert \psi \rangle$ yields $\vert \psi \rangle$ with certainty;
\pause
\item Wrt the continuum of other bases/frames in which $\vert \psi \rangle$ is {\color{blue}{not}} an element,
the quantized system is {\color{purple}{value {\color{blue}\bf {in}}definite}}: any measurement of $\vert \psi \rangle$
yields the occurence of $\vert {\bf b}_i \rangle$ given $\vert \psi \rangle$ with the respective frequency/expectation $\vert \langle {\bf b}_i \mid \psi \rangle \vert^2$;
and because of Pythagoras \& exclusivity \& completeness $\sum_i\vert \langle {\bf b}_i \mid \psi \rangle \vert^2=1$.
\end{itemize}

}

 \frame{
 \frametitle{Entanglement as indecomposable superpositions of multi-partite states}

\begin{itemize}
\pause
\item {\color{purple}{Multi-particle configurations}}such as as the $k$-particle configuration can be
\begin{itemize}
\item {\color{purple}{either decomposable}} ie representable in product forms
$\vert \Psi \rangle = \vert \psi_1 \cdots \psi_k \rangle$;
\pause
\item {\color{purple}{or indecomposable aka entangled/German ``verschr\"ankt''}} ie {\color{blue}{not}} representable in product forms such;
\end{itemize}
\pause
\item {\color{purple}{Entangled states lack {\color{blue}{individual}} value definiteness}} of its constituent (particle) parts;
\pause
\item {\color{purple}{Entangled states show value definiteness wrt {\color{blue}{relational}} properties}}, such as,
for instance in one (the singlet) of the Bell basis states
$\vert \Psi^- \rangle = \frac{1}{\sqrt{2}}\left(\vert +_1-_2  \rangle - \vert -_1+_2  \rangle \right)$
``one particle has opposite spin/polarization wrt the other particle in all spatial directions'';
any constituent particle has a 50:50 chance to be in either state $\vert +  \rangle$ or $\vert -  \rangle$.
\end{itemize}
}

\section{Quantization of brains and cognive functionalities}

 \frame{
 \frametitle{Brains are quantized}

Just like ``classical'' computers, (human) brains are ultimately -- that is, on the ``deepest level'' (cf Anderson) of physical description  -- quantized physical systems.

Whether quantization -- and, in particular, coherent superpositions and entanglement -- play an important part in cognition is a question of huge importance.

It can be either seen as {\color{purple}{deficiancy}} (lack of value definiteness) or as an {\color{purple}{opportunity}}.

}


 \frame{
 \frametitle{Music as a quantum cognitive process}

Quantum music (cf Putz \& Svozil \url{https://doi.org/10.1007/s00500-015-1835-x} ) may present  more freedom due to multiplicity of expression; in particular,

\begin{itemize}
\item
superposition:
\begin{music}
\generalmeter{\meterfrac24}% 2/4 meter chosen
\startextract % starting real score
\Notes
{\color[rgb]{1,0.1,0.1}\zq c}{\color[rgb]{0.6,0.6,1}\zq g}  \enotes
\Notes
{\color[rgb]{1,0.6,0.6}\zq c}{\color[rgb]{0.1,0.1,1}\zq g}  \enotes
\Endpiece
\zendextract % terminate excerpt
\end{music}

\end{itemize}
}


 \frame{
 \frametitle{Music as a quantum cognitive process cntd.}

\begin{itemize}
\item
entanglement
\begin{music}
\parindent10mm
\instrumentnumber{1} % a single instrument
\setname1{Piano} % whose name is Piano
\setstaffs1{2} % with two staffs
%\generalmeter{\meterfrac44} % 4/4 meter chosen
\startextract % starting real score
\notes
\hspace{1 mm}{\color[rgb]{0,1,0}\zq e}{\color[rgb]{0.5,0,0.5}\zq o}
|
\hspace{0 mm}{\color[rgb]{0.5,0,0.5}\zq e}{\color[rgb]{0,1,0}\zq o}
\en
\bar
\notes
\hspace{1 mm}{\color[rgb]{0,1,0}\zq e}{\color[rgb]{0,1,0}\zq o}
|
\hspace{0 mm}{\color[rgb]{0.5,0,0.5}\zq e}{\color[rgb]{0.5,0,0.5}\zq o}
\en
\zendextract % terminate excerpt
\end{music}
\pause
\item complementarity
\begin{music}
\startextract % starting real score
\notes
{\color{black!100}\wh{h}}\enotes\bar
\notes {\color{black!85}\wh{h}}\enotes\bar
\notes {\color{black!70}\wh{h}}\enotes\bar
\notes {\color{black!55}\wh{h}}\enotes\bar
\notes {\color{black!40}\wh{h}}\enotes\bar
\notes {\color{black!30}\wh{h}}\enotes\bar
\notes {\color{black!20}\wh{h}}\enotes\bar
\notes {\color{black!10}\wh{h}}
\enotes
\zendextract
\end{music}

\end{itemize}
}

\frame{

\centerline{\Large {\color{magenta} Thank you for your attention!}}

\begin{center}\color{orange}
$\widetilde{\qquad \qquad }$
$\widetilde{\qquad \qquad}$
$\widetilde{\qquad \qquad }$
\end{center}
 }
 \end{document}
