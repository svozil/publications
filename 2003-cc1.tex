%%tth:\begin{html}<LINK REL=STYLESHEET HREF="http://tph.tuwien.ac.at/~svozil/ssh.css">\end{html}
%\documentclass[prl,preprint,showpacs,showkeys,amsfonts]{revtex4}
%\usepackage{graphicx}
\documentstyle[]{article}
%\RequirePackage{times}
%\RequirePackage{courier}
%\RequirePackage{mathptm}
\renewcommand{\baselinestretch}{1.3}
\begin{document}





\title{Some motivation for the Cantor expanson of random reals in
physics-continued}
\author{}
\date{ }
\maketitle

\begin{abstract}
Just a few comments \& examples
of the usefulness of Cantor expansion in physical systems.
\end{abstract}

%http://www.chemguide.co.uk/basicorg/isomerism/polarised.html
%http://www.exploratorium.edu/snacks/polarized_mosaic.html
%\pacs{03.67.-a,03.67.Hk,03.65.Ta}
%\keywords{quantum information theory,quantum measurement theory}



The genuine strength of the Cantor expansion unfolds
when varying choices and varying interactions on different scales are considered.
This is a generalisation of self-similarity with intrinsic scale dependence.
Geometric objects of this type might not scale in a self-similar manner
but might be codable by a cantor expansion.

Consider a generalized Cantor set obtained, for example,
by cutting out in the $n$'th construction step $1/(n+1)$'th
of each of the remaining segments (starting from the real interval $[0,1]$ at $n=1$)
at a random position of $n+1$ positions of equal length \cite{nn}.
In Figure \ref{20303-cc-f1}, the construction process is depicted.
\begin{figure}
\begin{center}
%TexCad Options
%\grade{\off}
%\emlines{\off}
%\beziermacro{\on}
%\reduce{\on}
%\snapping{\off}
%\quality{2.00}
%\graddiff{0.01}
%\snapasp{1}
%\zoom{1.00}
\unitlength 0.6mm
\linethickness{1.5pt}
\begin{picture}(90.00,40.00)
\put(0.00,40.00){\line(1,0){90.00}}
\put(90.00,40.00){\line(0,0){0.00}}
\put(0.00,30.00){\line(1,0){30.00}}
\put(60.00,30.00){\line(1,0){30.00}}
\put(0.00,20.00){\line(1,0){7.67}}
\put(16.00,20.00){\line(1,0){14.00}}
\put(60.00,20.00){\line(1,0){11.00}}
\put(79.00,20.00){\line(1,0){11.00}}
\put(0.00,10.00){\line(1,0){1.67}}
\put(4.33,10.00){\line(1,0){3.33}}
\put(17.67,10.00){\line(1,0){12.33}}
\put(59.67,10.00){\line(1,0){6.33}}
\put(68.33,10.00){\line(1,0){2.67}}
\put(78.67,10.00){\line(1,0){9.67}}
\put(0.00,0.00){\line(1,0){0.33}}
\put(0.67,0.00){\line(1,0){1.00}}
\put(4.33,0.00){\line(1,0){0.33}}
\put(5.00,0.00){\line(1,0){2.33}}
\put(17.67,0.00){\line(1,0){6.00}}
\put(24.67,0.00){\line(1,0){5.33}}
\put(60.00,0.00){\line(1,0){2.33}}
\put(63.00,0.00){\line(1,0){2.67}}
\put(68.33,0.00){\line(1,0){0.33}}
\put(69.00,0.00){\line(1,0){1.67}}
\put(78.67,0.00){\line(1,0){2.67}}
\put(83.00,0.00){\line(1,0){5.00}}
\end{picture}
\caption{\label{20303-cc-f1}First construction steps of a generalized Cantor set.}
\end{center}
\end{figure}



Another example is the generalized Koch curve obtained by inserting in the $n$'th
construction step $n+1$ scaled down copies of the object obtained in the $n$'s
construction step at random positions.
A different variation of the Koch curve is obtained if different objects
(as compared to previous construction steps) are inserted.
Any one of the above examples can be efficiently coded by
Cantor expansions of random reals.
For an efficient encoding,
associate with every construction step a place in the expansion.
Then, the basis chosen for this particular place in the expansion
should be identified with the number of different segments in that construction step.
For example, in the generalized Cantor set discussed above, there are $n+1$ segments
at the $n$'th construction level; therefore, the basis chosen for the $n$'th position
should be $n+1$. (This linear dependence is only an example, and much more general functions for the bases
are possible.)
It is not too speculative to assume that this might reflect the physical property
of different object formations at different (e.g., length or time) scales, which might
be caused by nonsimilar interactions at different scales.


\bibliography{svozil}
%\bibliographystyle{apsrev}
\bibliographystyle{unsrt}

\end{document}

