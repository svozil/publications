%\documentclass[pra,showpacs,showkeys,amsfonts,amsmath,twocolumn,handou]{revtex4}
\documentclass[amsmath,red,table,handout]{beamer}
%\documentclass[pra,showpacs,showkeys,amsfonts]{revtex4}
\usepackage[T1]{fontenc}
%%\usepackage{beamerthemeshadow}
\usepackage[headheight=1pt,footheight=10pt]{beamerthemeboxes}
\addfootboxtemplate{\color{structure!80}}{\color{white}\tiny \hfill Karl Svozil (TU Vienna)\hfill}
\addfootboxtemplate{\color{structure!65}}{\color{white}\tiny \hfill Emerging Technologies\hfill}
\addfootboxtemplate{\color{structure!50}}{\color{white}\tiny \hfill SS~2011\hfill}
%\usepackage[dark]{beamerthemesidebar}
%\usepackage[headheight=24pt,footheight=12pt]{beamerthemesplit}
%\usepackage{beamerthemesplit}
%\usepackage[bar]{beamerthemetree}
\usepackage{graphicx}
\usepackage{pgf}
%\usepackage[usenames]{color}
%\newcommand{\Red}{\color{Red}}  %(VERY-Approx.PANTONE-RED)
%\newcommand{\Green}{\color{Green}}  %(VERY-Approx.PANTONE-GREEN)

%\RequirePackage[german]{babel}
%\selectlanguage{german}
%\RequirePackage[isolatin]{inputenc}

\pgfdeclareimage[height=0.5cm]{logo}{tu-logo}
\logo{\pgfuseimage{logo}}
\beamertemplatetriangleitem
%\beamertemplateballitem

\beamerboxesdeclarecolorscheme{alert}{red}{red!15!averagebackgroundcolor}
\beamerboxesdeclarecolorscheme{alert2}{purple}{orange!15!averagebackgroundcolor}
\newcounter{nc}[part]
\setcounter{nc}{1}

\begin{document}

\title{\bf \textcolor{red}{Emerging Technologies}}
%\subtitle{Naturwissenschaftlich-Humanisticher Tag am BG 19\\Weltbild und Wissenschaft\\http://tph.tuwien.ac.at/\~{}svozil/publ/2005-BG18-pres.pdf}
\subtitle{\textcolor{orange!60}{\small http://tph.tuwien.ac.at/$\sim$svozil/publ/2011-EmTech-pres.pdf}}
\author{Karl Svozil}
\institute{Institut f\"ur Theoretische Physik, Vienna University of Technology, \\
Wiedner Hauptstra\ss e 8-10/136, A-1040 Vienna, Austria\\
svozil@tuwien.ac.at
%{\tiny Disclaimer: Die hier vertretenen Meinungen des Autors verstehen sich als Diskussionsbeitr�ge und decken sich nicht notwendigerweise mit den Positionen der Technischen Universit�t Wien oder deren Vertreter.}
}
\date{SS~2011}
\maketitle

\frame{\tableofcontents}


%%%%%%%%%%%%%%%%%%%%%%%%%%%%%%%%%%%%%%%%%%%%%%%%%%%%%%%%%%%%%%%%%%%%%%%%%%%%%%%%%%%%%%%%%%%%%%%%%%%%%%%%%
%%%%%%%%%%%%%%%%%%%%%%%%%%%%%%%%%%%%%%%%%%%%%%%%%%%%%%%%%%%%%%%%%%%%%%%%%%%%%%%%%%%%%%%%%%%%%%%%%%%%%%%%%
%%%%%%%%%%%%%%%%%%%%%%%%%%%%%%%%%%%%%%%%%%%%%%%%%%%%%%%%%%%%%%%%%%%%%%%%%%%%%%%%%%%%%%%%%%%%%%%%%%%%%%%%%
%%%%%%%%%%%%%%%%%%%%%%%%%%%%%%%%%%%%%%%%%%%%%%%%%%%%%%%%%%%%%%%%%%%%%%%%%%%%%%%%%%%%%%%%%%%%%%%%%%%%%%%%%



\section{Foundations of Computer Science}

\frame{
\begin{center}
{\color{purple}
\Huge Part \Roman{nc}: \\
Foundations of Computer Science }
\end{center}
\addtocounter{nc}{1}
\begin{center}{\color{lime}
$\widetilde{\qquad \qquad }$
$\widetilde{\qquad \qquad}$
$\widetilde{\qquad \qquad }$ }
\end{center}
 }

%\subsection{What is an algorithm?}

\frame{
\frametitle{What is an algorithm?}
\begin{columns}
\begin{column}{7cm}
{\em ``When, several years ago, I saw for the first time an Instrument
which, when carried, automatically records the numbers of steps
taken by a pedestrian, it occurred to me at once that the entire
arithmetic could be subjected to a similar kind of machinery so that
not only counting but also addition and subtraction, multiplication
and division could be accomplished by a suitably arranged machine
easily, promptly, and with sure results.''}
\end{column}
\begin{column}{4cm}
\begin{overprint}
\includegraphics[width=5 true cm]{2011-emtech-Leibniz-Machine.pdf}
\end{overprint}
\end{column}
\end{columns}
$\;$\\
Gottfried Wilhelm Freiherr von Leibniz (1646-1716), on his Calculating Machine about the year 1671,
quoted in David Eugene Smith, A Source Book in Mathematics [1929], pp. 180-181. \\
http://www.archive.org/details/sourcebookinmath00smit
}


\frame{
\frametitle{What is an algorithm? (Leibnitz contd.)}
\begin{columns}
\begin{column}{7cm}
{\em ``A kind of general algebra in which all truths of reason would be reduced to a kind of calculus.
At the same time, this would be a kind of universal language or writing,
though infinitely different from all such languages which have thus far been proposed;
for the characters and the words themselves would direct the mind, and the errors - excepting those of fact -
would only be calculation mistakes.''}\\
(Gottfried Wilhelm Freiherr von Leibniz,
letter to Nicolas Remond, 10 January 1714, in Loemker 1969: 654. Translation revised.)

Bits: zero and one are associated with the presence or absence of a monad.
\end{column}
\begin{column}{4cm}
\begin{overprint}
\includegraphics[width=5 true cm]{2011-emtech-Leibniz.pdf}
\end{overprint}
\end{column}
\end{columns}
}


\frame{
\frametitle{What is an algorithm? (Contd.)}

Informally, the concept of an algorithm is often illustrated by the example of a recipe for accomplishing
some task; e.g., cooking.  It involves ``{\em \color{blue} paper--\&--pencil operations.}''
\begin{center}
\includegraphics[width=3.5 true cm]{2011-emtech-PaperAndPencil.pdf}
\includegraphics[width=9 true cm]{2011-emtech-cooking-11697-large.pdf}
\end{center}
}


\frame{
\frametitle{Church-Turing Thesis ``under permanent physical attack''}

\begin{center}
\begin{tabular}{lcr}
informal&$\Leftrightarrow$&formal\\
\hline\hline
physically feasible&$\Leftrightarrow$&formally representable\\
physically feasible&$\Rightarrow$&{\em ``highly nontrivial''}\\
{\em ``trivial''}&$\Leftarrow$&formally representable\\
(imbedding)&&\\
\hline \hline
algorithm&$\Leftrightarrow$&recursive function\\
&&$\Updownarrow$\\
algorithm&$\Leftrightarrow$&universal Turing machine\\
\end{tabular}
\end{center}

}


\frame{
\frametitle{Church-Turing Thesis cntd.}
\begin{columns}
\begin{column}{8.5cm}
{\em  ``The reason why we find it possible to construct, say, electronic
 calculators, and indeed why we can perform mental arithmetic, cannot
 be found in mathematics or logic. {\em
 The reason is that the laws of physics `happen to' permit the
 existence of physical models for the operations of arithmetic}
 such as addition, subtraction and multiplication.
 If they did not, these familiar operations would be
 noncomputable functions. We might still
 know {\em of} them and invoke them in mathematical proofs
 (which would presumably be called `nonconstructive') but we could
 not perform them.''}
\end{column}
\begin{column}{3cm}
\begin{overprint}
\includegraphics[width=5 true cm]{2011-emtech-DeutschDavid.pdf}
\end{overprint}
\end{column}
\end{columns}
David Deutsch, {\it Quantum theory, the {C}hurch-{T}uring principle and the
  universal quantum computer}, Proceedings of the Royal Society London
  \textbf{A 400}, 97--119 (1985).
http://dx.doi.org/10.1098/rspa.1985.007

}


\frame{
\frametitle{Church-Turing Thesis cntd.}
{\em `` $\ldots$ how can we ever exclude the possibility of our
 presented,
 some day (perhaps by some extraterrestrial visitors), with a (perhaps
 extremely complex) device or ``oracle'' that ``computes'' a
 noncomputable function?''
 }

M.~Davis, {\it Computability and Unsolvability} (McGraw-Hill, New York, 1958).

\begin{center}
\includegraphics[width=4 true cm]{2011-emtech-alien.pdf}
\end{center}
}

\frame{
\frametitle{Turing machine computability}
\begin{columns}
\begin{column}{7.5cm}
Alan Turing enshrined that part of mathematics, which can be ``constructed''
by paper and pencil operations, into a Turing machine which possesses:

a potentially
unbounded {\em \color{blue}  one-dimensional tape divided into cells},

some {\em \color{blue} finite memory}, and

some
{\em \color{blue} read-write head}

which transfers back and forth information from the tape to this memory.
A table of transition rules figuring as the ``program'' steers the machine deterministically.
%            http://plato.stanford.edu/entries/turing-machine/  http://www.abelard.org/turpap2/tp2-ie.asp
\end{column}
\begin{column}{4cm}
\begin{overprint}
\includegraphics[width=5 true cm]{2011-emtech-Turing-machine.pdf}
\end{overprint}
\end{column}
\end{columns}
}

\frame{
\frametitle{Universality and robustness}
\begin{columns}
\begin{column}{8.5cm}
{\em \color{blue} Universality:} A universal Turing machine is capable
of simulating all other Turing machines -- including itself.

{\em \color{blue}  Robustness:} ``robust'' formalizations are essentially equivalent; i.e.,
any formalism can be somehow ``translated'' one-to-one into any other one of that ``robust'' class.
(E.g., one operating system ``on top of another'' via some virtual machine: Virtualbox, www.winehq.org, $\ldots$)
\end{column}
\begin{column}{3cm}
\begin{overprint}
\includegraphics[width=3.5 true cm]{2011-emtech-Turing.pdf}
\end{overprint}
\end{column}
\end{columns}
Alan~M. Turing, {\it On computable numbers, with an application to the
  {E}ntscheidungsproblem,} Proceedings of the London Mathematical Society,
  Series 2 \textbf{42 and 43}, 230--265 and 544--546 (1936-7 and 1937).

}





%\subsection{Uncomputability}

\frame{
\frametitle{Turing uncomputability and the halting problem}
\begin{columns}
\begin{column}{7.5cm}
Proof by contradiction using
{\em \color{blue} encoding}
as well as
{\em \color{blue}
(Cantor) diagonalization.}

\begin{center}{\color{blue}
$\widetilde{\qquad \qquad }$
$\widetilde{\qquad \qquad}$
$\widetilde{\qquad \qquad }$ }
\end{center}

Consider a universal computer $U$ and  an arbitrary algorithm
$B(X)$ whose input is a string of symbols $X$.  Assume that there exists
a ``halting algorithm'' ${\tt HALT}$ which is able to decide whether $B$
terminates on $X$ or not.
The domain of ${\tt HALT}$  is the set of legal programs.
The range of ${\tt HALT}$ are classical bits.
\end{column}
\begin{column}{3cm}
\begin{overprint}
\includegraphics[width=5.5 true cm]{2011-emtech-MouthOfTruth.pdf}
\end{overprint}
\end{column}
\end{columns}
}

\frame{
\frametitle{Turing uncomputability and the halting problem cntd.}
\begin{columns}
\begin{column}{7.5cm}
Using ${\tt HALT}(B(X))$ we shall construct another deterministic
computing agent $A$, which has as input any effective program $B$ and
which proceeds as follows:  Upon reading the program $B$ as input, $A$
makes a copy of it.  This can be readily achieved, since the program $B$
is presented to $A$ in some encoded form
$\ulcorner B\urcorner $,
i.e., as a string of
symbols.  In the next step, the agent uses the code
$\ulcorner B\urcorner $
 as input
string for $B$ itself; i.e., $A$ forms  $B(\ulcorner B\urcorner )$,
henceforth denoted by
$B(B)$.  The agent now hands $B(B)$ over to its subroutine ${\tt HALT}$.
Then, $A$ proceeds as follows:  if ${\tt HALT}(B(B))$ decides that
$B(B)$ halts, then the agent $A$ does not halt; this can for instance be
realized by an infinite {\tt DO}-loop; if ${\tt HALT}(B(B))$ decides
that $B(B)$ does {\em not} halt, then $A$ halts.
\end{column}
\begin{column}{3cm}
\begin{overprint}
\includegraphics[width=5.5 true cm]{2011-emtech-MouthOfTruth.pdf}
\end{overprint}
\end{column}
\end{columns}
}

\frame{
\frametitle{Turing uncomputability and the halting problem cntd.}
\begin{columns}
\begin{column}{7.5cm}
The agent $A$ will now be confronted with the following paradoxical
task:  take the own code as input and proceed to determine whether or not it halts.
Then, whenever $A(A)$
halts, ${\tt HALT}(A(A))$, by the definition of $A$, would force $A(A)$ not to halt.
Conversely,
whenever $A(A)$ does not halt, then ${\tt HALT}(A(A))$ would steer
$A(A)$ into the halting mode.  In both cases one arrives at a complete
contradiction.  Classically, this contradiction can only be consistently
avoided by assuming the nonexistence of $A$ and, since the only
nontrivial feature of $A$ is the use of the peculiar halting algorithm
${\tt HALT}$, the impossibility of any such halting algorithm.
\end{column}
\begin{column}{3cm}
\begin{overprint}
\includegraphics[width=5.5 true cm]{2011-emtech-MouthOfTruth.pdf}
\end{overprint}
\end{column}
\end{columns}
}


\frame{
\frametitle{Undecidability of the rule inference (induction) problem}

Induction in physics is the inference of general rules
dominating and generating physical behaviors from these behaviors.
For any deterministic system strong enough to support
universal computation, the general induction problem
is provable unsolvable.
Induction is thereby reduced to the unsolvability of
the rule inference problem,

Informally, the algorithmic idea of the proof is to take any sufficiently powerful
rule or method of induction and, in using it, define some
functional behavior which is not identified by it.
This amounts
to constructing an algorithm which
(passively!)
 ``fakes'' the ``guesser'' by simulating some particular function $\varphi $
until the guesser
pretends to guess this function correctly.
In a second,  diagonalization step, the ``faking'' algorithm then switches to a different
 function $\varphi^\ast  \neq \varphi $, such that the guesser's guesses become incorrect.



}



\frame[shrink=2]{
\frametitle{Busy Beaver Number}

The busy beaver function
addresses the following
question: given a finite system;
i.e., a system whose algorithmic description is of finite length.
What is the biggest number producible by such a system before halting?

Let $\Sigma (n)$ denote the busy beaver function of $n$.
 Originally, T. Rado
 asked how
 many $1$'s a Turing machine with $n$ possible states and an empty
 input tape
 could print on that tape before halting.

 The first values of the Turing busy beaver function $\Sigma _T(x)$
 are finite and are known:                       \\
 $\Sigma _T(1)=1$,                               \\
 $\Sigma _T(2)= 4$,                              \\
  $\Sigma _T(3)=6$,                              \\
 $\Sigma _T(4)= 13$,                             \\
 $\Sigma _T(5) \ge 1915$,                        \\
 $\Sigma_T(7)\ge 22961$,                         \\
 $\Sigma_T(8)\ge 3\cdot (7\cdot 3^{92}-1)/2$.    \\


}


\frame{
\frametitle{Omega Number}

Omega $\Omega$, the halting probability, is the sum
$$\sum_{U(p) \downarrow} 2^{-\vert p \vert}$$
of all halting, prefix-free programs of some universal computer $U$.


}


%%%%%%%%%%%%%%%%%%%%%%%%%%%%%%%%%%%%%%%%%%%%%%%%%%%%%%%%%%%%%%%%%%%%%%%%%%%%%%%%%%%%%%%%%%%%%%%%%%%%%%%%%


\frame[shrink=2]{
\frametitle{Continuum urn}

With probability 1, a real initial value ``taken from the continuum urn'' is uncomputable (indeed, algorithmically incompressible = random).

In the measure theoretic sense, ``almost all'' reals are
 uncomputable. This can be demonstrated by the following argument:
 Let $M=\{ r_i\}$ be an  infinite point set (i.e., $M$ is a
 set of
 points $r_i$) which is denumerable and which is the subset of a dense
 set. Then, for instance, every $r_i\in M$ can be enclosed in the
 interval \begin{equation}
 I(i,\delta)= [r_i-2^{-i-1}\delta
 ,
 r_i+2^{-i-1}\delta]\quad ,
 \end{equation}
 where $\delta $ may be arbitrary small (we choose $\delta$ to be
 small enough that all intervals are disjoint).
 Since $M$ is denumerable, the measure $\mu$ of these intervals can
 be summed up, yielding
  \begin{equation}
 \sum_i \mu( I(i,\delta))= \delta \sum_{i=1}^\infty 2^{-i}=\delta \quad
 . \end{equation}
 From $\delta \rightarrow 0$ follows $\mu (M)=0$.


}

%%%%%%%%%%%%%%%%%%%%%%%%%%%%%%%%%%%%%%%%%%%%%%%%%%%%%%%%%%%%%%%%%%%%%%%%%%%%%%%%%%%%%%%%%%%%%%%%%%%%%%%%%
%%%%%%%%%%%%%%%%%%%%%%%%%%%%%%%%%%%%%%%%%%%%%%%%%%%%%%%%%%%%%%%%%%%%%%%%%%%%%%%%%%%%%%%%%%%%%%%%%%%%%%%%%
%%%%%%%%%%%%%%%%%%%%%%%%%%%%%%%%%%%%%%%%%%%%%%%%%%%%%%%%%%%%%%%%%%%%%%%%%%%%%%%%%%%%%%%%%%%%%%%%%%%%%%%%%
%%%%%%%%%%%%%%%%%%%%%%%%%%%%%%%%%%%%%%%%%%%%%%%%%%%%%%%%%%%%%%%%%%%%%%%%%%%%%%%%%%%%%%%%%%%%%%%%%%%%%%%%%

%\subsection{Karp-Cook Thesis}


\frame{
\frametitle{NP complexity class}

\begin{itemize}
\item<+->
In computational complexity theory, NP (``Non-deterministic Polynomial time'')
is the set of problems solvable

\begin{itemize}
\item<+->
by non-deterministic ``oracles''
and
\item<+->
polynomial time verifiability
\end{itemize}

\item<+->
{NP completeness}

An NP-complete problem is one which is robust in the following sense:
it is in NP and
it is NP-hard, i.e. every other problem in NP is reducible to it.

Example: ``travelling salesman''

Garey, M. and D. Johnson, Computers and Intractability; A Guide to the Theory of NP-Completeness, 1979


\item<+->
{Karp-Cook Thesis}

$NP\neq P$

No proof so far.


\end{itemize}

}








%%%%%%%%%%%%%%%%%%%%%%%%%%%%%%%%%%%%%%%%%%%%%%%%%%%%%%%%%%%%%%%%%%%%%%%%%%%%%%%%%%%%%%%%%%%%%%%%%%%%%%%%%
%%%%%%%%%%%%%%%%%%%%%%%%%%%%%%%%%%%%%%%%%%%%%%%%%%%%%%%%%%%%%%%%%%%%%%%%%%%%%%%%%%%%%%%%%%%%%%%%%%%%%%%%%
%%%%%%%%%%%%%%%%%%%%%%%%%%%%%%%%%%%%%%%%%%%%%%%%%%%%%%%%%%%%%%%%%%%%%%%%%%%%%%%%%%%%%%%%%%%%%%%%%%%%%%%%%
%%%%%%%%%%%%%%%%%%%%%%%%%%%%%%%%%%%%%%%%%%%%%%%%%%%%%%%%%%%%%%%%%%%%%%%%%%%%%%%%%%%%%%%%%%%%%%%%%%%%%%%%%

\section{Physical foundations of computation}

\frame{
\begin{center}
{\color{purple}
\Huge Part \Roman{nc}: \\
Physical foundations of computation}
\end{center}
\addtocounter{nc}{1}
\begin{center}{\color{lime}
$\widetilde{\qquad \qquad }$
$\widetilde{\qquad \qquad}$
$\widetilde{\qquad \qquad }$ }
\end{center}
 }



\frame[shrink=1.01]{
\frametitle{Maxwell's Demon}
\begin{center}
\includegraphics{2011-emtech-MaxwellsDemon.pdf}
\end{center}
}


\frame[shrink=1.01]{
\frametitle{Maxwell's Demon}
\begin{center}
\includegraphics{2011-emtech-MaxwellsDemonPicture.pdf}
\end{center}
}

\frame[shrink=1.01]{
\frametitle{Maxwell's Demon}
\begin{center}
\includegraphics{2011-emtech-MaxwellsDemonPicture2.pdf}
\end{center}
}

\frame[shrink=1.01]{
\frametitle{Maxwell's Demon}
\begin{center}
\includegraphics{2011-emtech-MaxwellsDemonPicture3.pdf}
\end{center}
}

\frame[shrink=1.01]{
\frametitle{Maxwell's Demon}
\begin{center}
\includegraphics{2011-emtech-MaxwellsDemonPicture4.pdf}
\end{center}
}


\frame{
\frametitle{Information theoretic ``solution'' to Maxwell's Demon}

R. Landauer,  {Irreversibility and Heat Generation in the Computing Process},
{IBM Journal of Research and Development} {\bf 3},  {183-191} (1961)

\begin{itemize}

\item<+->
 logical irreversibility in connection with information-discarding processes ---
``cleared'' memory can be from a variety of previous states

\item<+->
Each logical step must somehow correspond to a physical state

\item<+-> (``the bad news'')
logical irreversibility is associated with physical ``heat dissipation''
and ``entropy increase''  ;-(

\item<+-> (``the good news'')
logically reversibile operations need not be associated with physical ``heat dissipation''
and ``entropy increase'' ;-)

\end{itemize}


}

\frame[shrink=1.01]{
\frametitle{Modern-day ``solution'' of Maxwell's question cntd.}
\begin{center}
\includegraphics{2011-emtech-emtech-demon-op.pdf}
\end{center}
}

\frame{
\frametitle{Reversible computation from irreversible one}

Charles H. Bennett, Logical Reversibility of Computation,
{IBM Journal of Research and Development} {\bf 17}, {525-532} (1973).
\\
Charles H. Bennett,  {The Thermodynamics of Computation---A Review},
{International Journal of Theoretical Physics} {\bf 21}, {905-940} (1982).


\begin{itemize}

\item<+->
 Every (irreversible) computer can be made logically reversible at every step

\item<+->
saving of all intermediate results, avoiding erasure

\item<+->
copy of computation ``result;''  outcome

\item<+->
reverse computation to ``get rid'' of the intermediate results,

\item<+->
one is left with the original ``input'' and one copy of the ``output''

\item<+->
splitting up the computation into many steps results in less memory requirements

\end{itemize}

}

\frame[shrink=1.01]{
\frametitle{Reversible computation from irreversible one}
\begin{center}
\includegraphics{2011-emtech-MaxwellsDemonBennett71.pdf}
\end{center}
}

\frame[shrink=1.01]{
\frametitle{Reversible computation from irreversible one}
\begin{center}
\includegraphics{2011-emtech-MaxwellsDemonBennett71-2.pdf}
\end{center}
}

\frame{
\frametitle{Information theoretic ``solution'' to Maxwell's Demon cntd.}

\begin{itemize}

\item<+->
 As Maxwell's demon acquires information while performing its task, it ``heats up.''

\item<+->
Setting Maxwell's demon into its initial configuration means
erasure of informations, which in turn means energy dissipation.

\end{itemize}

}

\frame{
\frametitle{Why all this?}


\begin{itemize}

\item<+->
 Consider this: ``information is physical;''
i.e., has a physical representation.

\item<+->
Due to the (unitary) quantum evolution, quantum computation is reversible.

\end{itemize}


}




\frame{
\frametitle{Cellular Automata}

\begin{itemize}

\item<+->
John von Neumann, {Theory of Self-Reproducing Automata},
(A. W. Burks, editor),  {University of Illinois Press}, {Urbana}, 1966


\item<+->
Konrad Zuse,
{{R}echnender {R}aum},
{Elektronische Datenverarbeitung},
{\bf 8}, {336-344}, (1967)\\
  URL: http://www.idsia.ch/$\widetilde{\;}$juergen/digitalphysics.html

\end{itemize}

}

\frame[shrink=1.01]{
\frametitle{Cellular Automata}
\begin{center}
\includegraphics{2011-emtech-CellularAutomaton.pdf}
\end{center}
}

\frame[shrink=1.01]{
\frametitle{Billiard Ball Cellular Automata}
\begin{center}
\includegraphics{2011-emtech-BBCellularAutomaton.pdf}
\end{center}
}


\frame[shrink=1.01]{
\frametitle{Computational Complementarity and generalized urn model}

Edward F. Moore,
{Gedanken-Experiments on Sequential Machines},
in {Automata Studies},
ed. by {C. E. Shannon and J. McCarthy},
(Princeton  University Press,  {Princeton} 1956)
\begin{center}
\includegraphics{2011-emtech-SubwayAutomaton.pdf}
\end{center}
}




%%%%%%%%%%%%%%%%%%%%%%%%%%%%%%%%%%%%%%%%%%%%%%%%%%%%%%%%%%%%%%%%%%%%%%%%%%%%%%%%%%%%%%%%%%%%%%%%%%%%%%%%%
%%%%%%%%%%%%%%%%%%%%%%%%%%%%%%%%%%%%%%%%%%%%%%%%%%%%%%%%%%%%%%%%%%%%%%%%%%%%%%%%%%%%%%%%%%%%%%%%%%%%%%%%%
%%%%%%%%%%%%%%%%%%%%%%%%%%%%%%%%%%%%%%%%%%%%%%%%%%%%%%%%%%%%%%%%%%%%%%%%%%%%%%%%%%%%%%%%%%%%%%%%%%%%%%%%%
%%%%%%%%%%%%%%%%%%%%%%%%%%%%%%%%%%%%%%%%%%%%%%%%%%%%%%%%%%%%%%%%%%%%%%%%%%%%%%%%%%%%%%%%%%%%%%%%%%%%%%%%%




\section{Quantum Mechanics}

\frame{
\begin{center}
{\color{purple}
\Huge Part \Roman{nc}: \\
Quantum Mechanics}
\end{center}
\addtocounter{nc}{1}
\begin{center}{\color{lime}
$\widetilde{\qquad \qquad }$
$\widetilde{\qquad \qquad}$
$\widetilde{\qquad \qquad }$ }
\end{center}
 }

\frame{
\frametitle{Hilbert space}

All quantum
mechanical entities are represented by objects
of Hilbert spaces. A {\em Hilbert space} is a linear
vector space ${\cal H}$ over the field $\Phi$ of complex numbers
(with vector addition
and scalar multiplication), together  with a complex function
$(\cdot ,\cdot
)$, the {\em scalar} or {\em inner product}, defined on ${\cal
H}\times{\cal H}$ such that
(i)
$(x,x)=0$ if and only if $x=0$;
(ii)
$(x,x)\ge 0$ for all $x \in{\cal H}$;
(iii)
$(x+y,z)=(x,z)+(y,z)$ for all $x,y,z \in {\cal H}$;
(iv)
$(\alpha x,y)=\alpha (x,y)$ for all $x,y \in {\cal H}, \alpha \in \Phi$;
(v)
$(x,y)=\overline{(y,x)}$ for all $x,y \in {\cal H}$
($\overline{\alpha }$ stands for the complex conjugate of $\alpha$);
(vi)
If $x_n\in {\cal H}$, $n=1,2,\ldots$, and if $\lim_{n,m\rightarrow
\infty} (x_n-x_m,x_n-x_m)=0$, then there exists an $x\in {\cal H}$ with
$\lim_{n\rightarrow \infty} (x_n-x,x_n-x)=0$.


}


\frame{
\frametitle{State}


 A pure {\em physical state} is represented by
a  vector of  the Hilbert space ${\cal H} $.
Therefore, if two vectors $x,y\in {\cal H}$ represent physical
states, their vector sum
$z=x+y\in{\cal H}$ represent a physical state as well.
This state $z$ is called the {\em coherent superposition} of state $x$
and
$y$. Coherent state superpositions will become most important in quantum
information theory.

}

\frame{
\frametitle{Observables}
{\em Observables} $A$ are represented by self-adjoint
operators $A$
on the Hilbert space ${\cal H}$ such that $(Ax,y)=(x,Ay)$ for all
$x,y\in {\cal H}$. (Observables and their corresponding operators are
identified.)

In what follows, unless stated differently, only
{\em finite} dimensional Hilbert spaces are considered.
 Then, the vectors
corresponding to states can be written as usual vectors in complex
Hilbert space.
Furthermore, bounded
self-adjoint operators are  equivalent to bounded Hermitean operators.
They can be represented by matrices, and the self-adjoint
conjugation
is just transposition and complex conjugation of the matrix elements.

}

\frame{
\frametitle{ }
Elements $b_i,b_j\in {\cal H}$ of the set of orthonormal base vectors
satisfy
$(b_i, b_j) =\delta_{ij}$,
where $\delta_{ij}$ is the Kronecker delta function.
Any state $x$ can be written as a linear
combination of
the set of orthonormal base vectors $\{b_1,b_2,\cdots \}$,
i.e.,
$x =\sum_{i=1}^N   \beta_i b_i$, where $N$ is the dimension of ${\cal
H}$ and
$\beta_i=(b_i,x) \in \Phi$.
In the Dirac bra-ket notation, unity is given by
${\bf 1}=\sum_{i=1}^N \vert b_i\rangle \langle b_i\vert $.
Furthermore,
any Hermitean operator has a spectral representation
$A=\sum_{i=1}^N \alpha_i P_i$,
where the $P_i$'s  are orthogonal projection operators onto the
orthonormal eigenvectors $a_i$ of $A$ (nondegenerate
case).
}




\frame{
\frametitle{Complementarity}
Observables are said to be {\em compatible} if they can be defined
simultaneously with arbitrary accuracy; i.e., if they are
``independent.'' A criterion for compatibility is the {\em commutator.}
Two observables ${A},{B}$ are compatible, if their {\em
commutator} vanishes; i.e.,
if $\left[
{A},
{B}
\right] =
{A}
{B}  -
{B}
{A}   =0$.
For example, position and momentum operators
$
\left[
{{\frak x}},
{{\frak p_x}}
\right] =
{{\frak x}}
{{\frak p_x}}-
{{\frak p_x}}
{{\frak x}} =
x
{\hbar \over i} {\partial \over \partial x}-
{\hbar \over i} {\partial \over \partial x}
x
=i\, \hbar
\neq 0
$
and thus do not commute. Therefore, position and momentum of a state
cannot be measured simultaneously with arbitrary accuracy.
It can be shown that this property gives rise to the {\em Heisenberg
uncertainty relations}
$
\Delta x
\Delta p_x \ge {\hbar \over 2}
$,
where
$
\Delta x
$
and
$
\Delta p_x
$
is given by
$
\Delta x =\sqrt{\langle x^2\rangle -\langle x\rangle ^2}
$
and
$
\Delta p_x =\sqrt{\langle p_x^2\rangle -\langle p_x\rangle ^2}
$,  respectively.
}


\frame{
\frametitle{Outcome}
The result of any single measurement of the observable $A$
on a state $x\in {\cal H}$
can only be one of the real eigenvalues of the corresponding
Hermitean operator $A$.
If $x$ is in a coherent superposition of eigenstates of $A$, the
particular outcome of any such single measurement is indeterministic;
i.e.,
it cannot be predicted with certainty. As a
result of the measurement,
the system is in the state which corresponds to the eigenvector $a_n$ of
$A$ with the associated real-valued eigenvalue
$\alpha_n$; i.e., $Ax=\alpha_n a_n$ (no summation convention here).


}


\frame{
\frametitle{Evolution}

This ``transition'' $x\rightarrow a_n$ has given rise to speculations
concerning the
``collapse
of the wave function (state).''  But, as has been argued recently,
 it is
possible to reconstruct coherence; i.e., to ``reverse the collapse of
the wave function (state)'' if the process of measurement is
reversible. After this reconstruction, no information about the
measurement must be left, not even in principle.
}

\frame[shrink=2]{
\frametitle{ }
How did Schr\"odinger, the creator of wave mechanics, perceive the
$\psi$-function? In his
1935 paper
``Die Gegenw\"artige
Situation in der Quantenmechanik'' (``The present situation in quantum
mechanics''), Schr\"odinger states,
\begin{quote}
{\em Die $\psi$-Funktion als Katalog der Erwartung:}
$\ldots$
Sie [[die $\psi$-Funktion]] ist jetzt das Instrument zur Voraussage der
Wahrscheinlichkeit von Ma\ss zahlen. In ihr ist die jeweils erreichte
Summe theoretisch begr\"undeter Zukunftserwartung verk\"orpert,
gleichsam wie in einem {\em Katalog} niedergelegt.
$\ldots$
Bei jeder Messung ist man gen\"otigt, der $\psi$-Funktion ($=$dem
Voraussagenkatalog) eine eigenartige, etwas pl\"otzliche Ver\"anderung
zuzuschreiben, die von der {\em gefundenen Ma\ss zahl} abh\"angt und
sich {\em nicht vorhersehen l\"a\ss t;} woraus allein schon deutlich
ist, da\ss~ diese zweite Art von Ver\"anderung der $\psi$-Funktion mit
ihrem regelm\"a\ss igen Abrollen {\em zwischen} zwei Messungen nicht das
mindeste zu tun hat. Die abrupte Ver\"anderung durch die Messung
$\ldots$ ist der interessanteste Punkt der ganzen Theorie. Es ist genau
{\em der} Punkt, der den Bruch mit dem naiven Realismus verlangt. Aus
{\em diesem} Grund kann man die $\psi$-Funktion {\em nicht} direkt an
die Stelle des Modells oder des Realdings setzen. Und zwar nicht etwa
weil man einem Realding oder einem Modell nicht abrupte unvorhergesehene
\"Anderungen zumuten d\"urfte, sondern weil vom realistischen Standpunkt
die Beobachtung ein Naturvorgang ist wie jeder andere und nicht per se
eine Unterbrechung des regelm\"a\ss igen Naturlaufs hervorrufen darf.
\end{quote}
It therefore seems not unreasonable to state that, epistemologically,
quantum mechanics is more a theory of knowledge of an
(intrinsic) observer rather than the platonistic physics ``God knows.''
The  wave function, i.e., the state of the physical system in a
particular
representation (base), is a representation of the observer's knowledge;
it is a representation or name or code or index of
the information or knowledge the observer has access to.
}

\frame{
\frametitle{Probability}
The probability $P_y(x)$ to find a system represented by state $x$
in some state $y$ of an orthonormalized basis is given by
$P_y(x)=\vert (x,y) \vert^2 $.
}

\frame{
\frametitle{Expectation value}
The {\em average value} or {\em expectation value} of an observable
$A$ in the state
$ x$
is given by
$\langle A\rangle_ x =
\sum_{i=1}^N \alpha_i
\vert (x,a_i) \vert^2$.
}

\frame{
\frametitle{ }
The dynamical law or equation of motion can be written in the form
$x (t) =Ux (t_0) $,
where $U^\dagger =U^{-1}$ (``$\dagger $ stands for transposition and
complex conjugation) is a
linear {\em unitary evolution operator}.

The {\em Schr\"odinger equation}
$
i\hbar {\partial \over \partial t}  \psi (t)    =
H \psi (t) $
 is obtained by identifying $U$ with
$U=e^{-iHt/\hbar }$,
where $H$ is a self-adjoint  Hamiltonian (``energy'') operator,
by differentiating the equation of motion
with respect to the time variable $t$;
i.e.,
$
 {\partial \over \partial t} \psi (t) =-\,{iH\over
\hbar
}e^{-i{H}t/\hbar}
\psi (t_0 ) = -\,{i{H}\over \hbar } \psi (t)
$.
}

\frame{
\frametitle{ }
In terms of the
set of orthonormal base vectors $\{ b_1, b_2, \ldots
\}$, the Schr\"odinger equation can be written as
$i\hbar {\partial \over \partial t} ( b_i , \psi (t) )   =
\sum_{j}
H_{ij}( b_j, \psi (t) )$.
In the case of position base states $\psi (x,t)=( x, \psi (t)
)$, the Schr\"odinger equation takes on the form
 $
i\hbar {\partial \over \partial t}  \psi (x,t) =
{H} \psi (x,t)=\left[{{{\frak p}} {{\frak p}}\over 2m}+
{V}(x)\right]\psi (x,t) = \left[-\,{\hbar^2
\over
2m}\nabla^2+V(x)\right] \psi (x,t)$.

}
\frame{
\frametitle{ }

For stationary $ \psi_n
(t)=
e^{-(i/\hbar )E_nt}  \psi_n $, the Schr\"odinger equation
can be brought into its time-independent form
$H\, \psi_n
=
E_n\, \psi_n $.
Here,
$i\hbar {\partial \over \partial t} \psi_n (t)
=
E_n \, \psi_n (t) $  has been used;
$E_n$
and $\psi_n $
stand for the $n$'th eigenvalue and eigenstate of
$H$, respectively.
}

\frame{
\frametitle{ }

Usually, a physical problem is defined by the Hamiltonian ${H}$.
The problem of finding the physically relevant states reduces to finding
a complete set of eigenvalues and eigenstates of ${H}$.
Most elegant solutions utilize the symmetries of the problem, i.e., of
${H}$. There exist two ``canonical'' examples, the $1/r$-potential
and
the harmonic oscillator potential, which can be solved wonderfully by
these methods (and they are presented over and over again in standard
courses of quantum mechanics), but not many more.
}


\section{Quantum information \& computation}

\frame{
\begin{center}
{\color{purple}
\Huge Part \Roman{nc}: \\
Universal Quantum Computation }
\end{center}
\addtocounter{nc}{1}
\begin{center}{\color{lime}
$\widetilde{\qquad \qquad }$
$\widetilde{\qquad \qquad}$
$\widetilde{\qquad \qquad }$ }
\end{center}
 }

\frame{
\frametitle{Concepts}

Qubits are the fundamental units of quantum information.
They refer to quantum states and observables which behave nonclassically;
in particular they are capable of
\begin{itemize}
\item<1->
randomness of single events,
\item<1->
complementarity (incompleteness),
\item<1->
value indefiniteness (randomness),
\item<1->
coherent superpositions (parallel co-representation of classically mutually excluding cases), and
\item<1->
entanglement
(information spread over a multitude of particles or observables),
\item<1->
``interaction-free'' counterfactual potentiality.
\end{itemize}
At the same time they are subject to {\em reversibility} in-between ``irreversible'' measurements.

All of these features can be used to efficiently and securely compute, distribute and transfer information.


}

\frame{
\frametitle{Status}

\begin{itemize}
\item<1->
Quantum cryptography is implemented
[bbn.com/DARPA (US), idquantique.com (Switzerland), magiqtech.com (US/Australia),
qinetiq.com (UK), NEC (Japan), Siemens (Austria/Germany), $\ldots$];
\item<1-> Quantum computer
hardware not (yet?) existent; e.g., problems with maintaining coherence;
still needed: a ``quantum transistor;''
\item<1->
Quantum algorithms:
\begin{itemize}
\item<1-> Deutsch-type algorithm (counterexample: parity is qcomp-hard; gain only factor 2);
\item<1-> Factoring (Shor's algorithm): speedup may or may not be exponential;
\item<1-> Grover's search algorithm (quadratic speedup).
\end{itemize}
\end{itemize}


}

\section{Universal Quantum Computation}



\frame{
\frametitle{Universal Quantum Computation}
\begin{itemize}
\item<1->
Quantum computation uses {\em arbitrary unitary transformations} $U^{-1}=U^\dagger =(U^\ast )^T$
of quantum states
in $n$--dimensional Hilbert space as the ``universal'' model of computation.

\item<1-> Two-qbit operations are sufficient to guarantee ``classical universality''
in the sense of Church-Turing computability.

\item<1->
By definition, unitary transformations are {\em ``reversible.''} If the functions $f$ they code are irreversible, more auxiliary ``tracking'' bits are necessary:
$$\textsf{\textbf{U}}_f(\vert x\rangle \vert y \rangle )= \vert x\rangle \vert y \oplus f(x)\rangle ,$$
with $\oplus$ standing for the sum modulo two, for which $z\oplus z=0$ for all $z\in \{0,1\}$ (this guarantees reversibility!).

\item<1->
Thus, during a computation, all information is transformed one-to-one.
There is no possibility to {\em copy} arbitrary qubits (one-to-many), or get rid of them (many-to-one).
The classical analogy are {\em permutations}.

\end{itemize}
 }

\subsection{No-Cloning (no-copy) theorem}
\frame{
\frametitle{No-Cloning (no-copy) theorem}

\begin{itemize}
\item<1-> Consider some state $\vert \psi \rangle$ which needs to be copied,
and some blank state $\vert b\rangle$ which serves as ``blank quantum paper'',
as well as some ``quantum copier'' represented by a unitary transformation $U$.
Then, ideally, for arbitrary states $\vert \psi \rangle$,
$$
U(\vert \psi \rangle \vert b \rangle )
=
\vert \psi \rangle  \vert \psi \rangle
.
$$


\item<1->   Unfortunately, for the coherent superposition
$\alpha_\varphi \vert \varphi \rangle +\alpha_\psi \vert \psi \rangle$
of two arbitrary states $\vert \varphi \rangle$ and $\vert \psi \rangle$, with
$
U(\vert \varphi \rangle  \vert b\rangle )
=
\vert \varphi \rangle  \vert \varphi \rangle
$, due to linearity,
$$
U(\alpha_\psi \vert \psi\rangle +\alpha_\varphi \vert \varphi \rangle )
=
\alpha_\psi \vert \psi  \rangle  \vert \psi  \rangle
+
\alpha_\varphi \vert \varphi \rangle \vert \varphi \rangle
,
$$


\item<1->
whereas a ``true copy'' would be
$$\begin{array}{l}
(\alpha_\psi \vert \psi \rangle + \alpha_\varphi \vert \varphi \rangle )
(\alpha_\psi \vert \psi \rangle + \alpha_\varphi \vert \varphi \rangle ) =  \\
\qquad
=\alpha_\psi^2  \vert \psi \rangle \vert \psi \rangle
+
\alpha_\psi \alpha_\varphi (\vert \psi \rangle \vert \varphi \rangle
+\vert \varphi \rangle \vert \psi \rangle )
+
\alpha_\varphi^2  \vert \varphi \rangle \vert \varphi \rangle    .
\end{array}
$$
Due to linearity (unitarity), we {\em miss} all the {\em interference} terms.
\end{itemize}
 }

\frame{
\frametitle{No-Cloning (no-copy) theorem cntd.}
Nevertheless, one could still attempt to copy  quantum states
$$\begin{array}{l}
U(\vert \psi \rangle  \vert b\rangle )
=
\vert \psi \rangle  \vert \psi \rangle
\text{ and }\\
\left[U(\vert \varphi \rangle  \vert b\rangle )\right]^t =
\langle \varphi \vert  \langle b \vert U^{-1} = \langle \varphi \vert  \langle  \varphi \vert .
\end{array}
$$
Since  $\langle b\vert b\rangle=1$ and
$$\begin{array}{l}
\langle \varphi \vert  \langle  \varphi \vert  \psi \rangle  \vert \psi \rangle =
\langle \varphi \vert \psi \rangle^2=     \\
\qquad =\langle \varphi \vert  \langle b \vert U^{-1}U(\vert \psi \rangle  \vert b\rangle )  =
\langle \varphi \vert \psi \rangle ,
\end{array}
$$
is only satisfied by either  $\langle \varphi \vert \psi \rangle=0$ or $1$,
two nonidentical states  $\psi$ and $\varphi$ can only be simultaneously copied if
they are {\em orthogonal};
all other states cannot be copied.
This is a generalization of the fact that  classical bits can  be copied.



 }
\subsection{Standard quantum computing transformations}
\frame{
\frametitle{Qbit versus Cbit}
Let the {\em classical bit states} (cbit) be represented by 2-dim vectors:
$$
\vert 0\rangle \equiv
\left(
\begin{array}{c}
1\\
0
\end{array}
\right)
\quad
\text{ and }
\quad
\vert 1\rangle \equiv
\left(
\begin{array}{c}
0\\
1
\end{array}
\right).
$$
$\;$\\
\begin{beamerboxesrounded}[scheme=alert2,shadow=true]{{\color{yellow}The {\em quantum bit  (qbit) states} are
the {\em coherent superposition} of cbits}}

$$
\alpha_0 \vert 0\rangle +\alpha_1 \vert 1\rangle \equiv
\left(
\begin{array}{c}
\alpha_0\\
\alpha_1
\end{array}
\right)
,
\quad \text{ with  } \quad
\vert
\alpha_0
\vert^2
+
\vert
\alpha_1
\vert^2
=1 .
$$
\end{beamerboxesrounded}

Quantum mechanics allows the co-representation of two classically contradictory bit states in one qbit.


\begin{beamerboxesrounded}[scheme=alert2,shadow=true]{{\color{yellow}Quantum parallelism}}
In general, $n$ qbits can co-represent $2^n$ cbits.
\end{beamerboxesrounded}
}

\subsection{Quantum parallelism}
\frame{
\frametitle{Quantum parallelism}
{Hadamard ``spread''},
``not'' and $\sqrt{\text{``not''}}$ operators
 $$
\textsf{\textbf{H}}= {1\over \sqrt{2}}
\left(
\begin{array}{rr}
1&1\\
1&-1
\end{array}
\right) ,
\;
\textsf{\textbf{X}}=
\left(
\begin{array}{rr}
0&1\\
1&0
\end{array}
\right) ,
\;
\sqrt{\textsf{\textbf{X}}}= {1\over 2}
\left(
\begin{array}{rr}
1+i&1-i\\
1-i&1+i
\end{array}
\right) .
$$

Effect of $
\textsf{\textbf{X}}
$
on cbits: negation; i.e.,
$$
\textsf{\textbf{X}}\vert 0\rangle
=
\vert 1\rangle,
\quad
\textsf{\textbf{H}}\vert 1\rangle
=
\vert 0\rangle .
$$

Effect of $
\textsf{\textbf{H}}
$
on cbits: they become qbits; i.e.,
$$
\textsf{\textbf{H}}\vert 0\rangle
=
{1\over \sqrt{2}}\left(\vert 0\rangle +\vert 1\rangle \right),
\quad
\textsf{\textbf{H}}\vert 1\rangle
=
{1\over \sqrt{2}}\left(\vert 0\rangle -\vert 1\rangle \right).
$$
}






\section{Spread--Fold--Detect}

\frame{
\begin{center}\Huge
{\color{purple}    Part \Roman{nc}:  \\
Spread--Fold--Detect  }
\end{center}
\addtocounter{nc}{1}
\begin{center}{\color{lime}
$\widetilde{\qquad \qquad }$
$\widetilde{\qquad \qquad}$
$\widetilde{\qquad \qquad }$ }
\end{center}
 }


\frame{
\frametitle{Solving decision problems by ``spreading'' functional behaviour over  multipartite states}
 {\small
(Classical) Information can be encoded by distributing it over different particles or quanta,
such that:

\begin{itemize}
\item<1->
measurements of {\em single} quanta are irrelevant, yield ``random'' results,
and even destroy the original information (by asking complementary questions);


\item<1->
well defined correlations exist and can be defined among different particles or quanta ---
even to the extent that a state is solely defined by propositions ($\equiv$ projectors)
about {\em collective} (or {\em relative})  properties of the particles or quanta involved;


\item<1->
identifying a given state of a quantized system can yield information about
{\em collective} (or {\em relative})  properties of the particles or quanta involved.

\end{itemize}
 }
 }



\frame{
\frametitle{Related physical concepts}
{\small
\begin{itemize}
\item<1->
Quantum entanglement (Schr�dinger's ``Verschr�nkung''):
the state of two or more ``entangled'' particles or quanta cannot be
constructed from or decomposed into (tensor) products of the states
of the ``single'' particles or quanta involved. E.g., in  {\it The essence of entanglement} [quant-ph/0106119],   Brukner, Zukowski \& Zeilinger write:
{\em ``the information in a composite system resides more
in the correlations than in properties of individuals.''}


\item<1->
Zeilinger's foundational principle: {\em ``An elementary system carries 1 bit of information.''}
$\ldots$ more generally:
$n$ elementary $d$-state systems (like particles or quanta) carry exactly $n$ dits of information.

\item<1->
Example: the (singlet) Bell state
$ {1\over \sqrt{2}}\left(
\vert \uparrow \downarrow \rangle
-
\vert \downarrow \uparrow \rangle \right)
$
of two electrons
 is defined by the properties that the two particles have opposite spin when measured along two (or more)
different (orthogonal) directions.

\end{itemize}
 }
 }


\subsection{Encoding  decision problems by state identification problems}

 \frame{
\frametitle{Quantum encoding  decision problems about ``collective'' behaviours}

Suppose one is interested in a decision problem which could be associated with some {\em ``collective''} property or behaviour
related to or involving, for instance,
\begin{itemize}
\item<1->
a function over a wide range of its arguments,
\item<1->
which is of ``comparative'' nature; that is, only the relative functional values count;
\item<1->
such that the single functional values are irrelevant;
e.g., are of no interest, ``annoying'' or are otherwise unnecessary.
\end{itemize}
Then it is not completely unreasonable to speculate that one could use the
kind of distributive  information capacity encountered in the quantum physics of multipartite states
for a more effective (encryption of the) solution.
 }


\subsection{Deutsch's problem and parity}
\frame{
\frametitle{Deutsch's problem: parity of a function of one bit}
 Find out whether or not an unknown
function $f$ that takes a single (classical) bit into a single (classical) bit
is {\color{green}constant} or {\color{red} not constant}, which is equal to finding the parity of $f:\{0,1\}\rightarrow \{0,1\}$
$\;$\\
$\;$\\
\centerline{
\begin{tabular}{cccccccccccccc}
\hline\hline
$f$& $ 0$ &$1$\\
\hline
\color{green}$f_0$ &$0$ &$0$\\
\color{red}$f_1$ &$0$ &$1$\\
\color{red}$f_2$ &$1$ &$0$\\
\color{green}$f_{3}$ &$1$ & $1$\\
\hline\hline
\end{tabular}
}
}





\frame{
\frametitle{Solution of Deutsch's problem}
With
$$\textsf{\textbf{U}}_f(\vert x\rangle \vert y \rangle )= \vert x\rangle \vert y \oplus f(x)\rangle ,$$
the solution  of Deutsch's problem is
$\;$ \\
$$  \begin{array}{ll}
(\textsf{\textbf{H}}\otimes \textsf{\textbf{1}})
\textsf{\textbf{U}}_f
(\textsf{\textbf{H}}   \otimes \textsf{\textbf{H}}   )
(\textsf{\textbf{X}}   \otimes \textsf{\textbf{X}}   )
(
\vert 0\rangle
\vert 0\rangle
)
=  \\
\; \\
\quad = \left\{
\begin{array}{ll}
\vert 1\rangle {1\over \sqrt{2}}\left(\vert f(0)\rangle -\vert 1\oplus f(0)\rangle \right) &\text{ for } f(0)=f(1), \\
\vert 0\rangle {1\over \sqrt{2}}\left(\vert f(0)\rangle -\vert 1\oplus f(0)\rangle \right) &\text{ for } f(0)\neq f(1).
\end{array}
\right.
\end{array}
$$

Answer obtainable in a {\em single} cycle. This is impossible classically.
\begin{center}{\color{lime}
$\widetilde{\qquad \qquad }$
$\widetilde{\qquad \qquad}$
$\widetilde{\qquad \qquad }$ }
\end{center}
However, the parity of a function
has been proven to be quantum computationally hard
Farhi et al., 1998:
It is only possible to go from $2^k$ classical queries down to $2^k/2$
quantum queries, thereby gaining a factor of 2.
 }




\frame{
\frametitle{Other (possible) quantum speedups}
\begin{itemize}
\item<1-> Shor's prime factoring algorism; prime factoring is {\em not} in NP-complete, may also be solvable
by classical polynomial means;
\item<1->
Grover's ``database search'' algorithm; speedup ``only'' quadratic and thus polynomial;
\item<1->
?!
\end{itemize}
 }


\section{Counterfactual quantum  computation}


\frame{
\frametitle{Counterfactual computation (Elitzur and Vaidman, 1993)}
Mach-Zehnder interferometer
\begin{center}
%TexCad Options
%\grade{\off}
%\emlines{\off}
%\beziermacro{\off}
%\reduce{\on}
%\snapping{\off}
%\quality{0.20}
%\graddiff{0.01}
%\snapasp{1}
%\zoom{2.00}
\unitlength 0.70mm
%\linethickness{0.4pt}
\thicklines
\begin{picture}(78.67,51.00)
{\color{orange}
\put(57.67,30.00){\line(0,-1){25.00}}
\put(5.00,45.00){\color{yellow}\makebox(0,0)[cc]{$L$}}
\put(5.00,45.00){\color{yellow}\circle{10.00}}
\put(10.00,45.00){\line(1,0){40.00}}
\put(39.67,45.00){\line(1,0){18.00}}
\put(20.00,45.00){\line(0,-1){25.00}}
\put(20.00,20.00){\line(1,0){13.00}}
\put(57.67,45.00){\line(0,-1){25.00}}
\put(57.67,20.00){\line(-1,0){35.00}}
\put(57.67,20.00){\line(1,0){13.00}}
\put(57.67,20.00){\line(0,-1){13.00}}
\put(75.17,20.00){\color{purple}\oval(7.00,8.00)[r]}
\put(78.67,26.00){\color{purple}\makebox(0,0)[cc]{$D_1$}}
\put(23.00,32.00){\makebox(0,0)[cc]{$c$}}
\put(20.00,51.00){\color{violet}\makebox(0,0)[cc]{$S_1$}}
\put(51.67,28.00){\color{violet}\makebox(0,0)[cc]{$S_2$}}
\put(13.00,41.00){\makebox(0,0)[cc]{$a$}}
\put(65.67,23.00){\makebox(0,0)[cc]{$d$}}
\put(57.67,3.33){\color{purple}\oval(8.67,8.00)[b]}
\put(65.00,-1.00){\color{purple}\makebox(0,0)[cc]{$D_2$}}
\put(60.33,10.33){\makebox(0,0)[cc]{$e$}}
\put(38.00,51.00){\makebox(0,0)[cc]{$b$}}
\put(57.00,51.00){\color{violet}\makebox(0,0)[cc]{$M$}}
\put(22.33,11.00){\color{violet}\makebox(0,0)[cc]{$M$}}
\put(34.10,16.70){\color{red}\dashbox{1.00}(9.57,6.83)[cc]{}}
\put(38.62,12.91){\color{red}\makebox(0,0)[cc]{$B$}}
\put(25.00,15.00){\color{violet}\line(-1,1){10.00}}
\put(62.67,40.00){\color{violet}\line(-1,1){10.00}}
\put(20.00,45.00){\color{violet}\circle{1.00}}
\put(22.00,43.00){\color{violet}\circle{1.00}}
\put(24.00,41.00){\color{violet}\circle{1.00}}
\put(16.00,49.00){\color{violet}\circle{1.00}}
\put(18.00,47.00){\color{violet}\circle{1.00}}
\put(57.67,20.00){\color{violet}\circle{1.00}}
\put(59.67,18.00){\color{violet}\circle{1.00}}
\put(61.67,16.00){\color{violet}\circle{1.00}}
\put(53.67,24.00){\color{violet}\circle{1.00}}
\put(55.67,22.00){\color{violet}\circle{1.00}}
}
\end{picture}
\end{center}

A single quantum (photon, neutron, electron {\it etc}) is emitted in $L$
and meets a lossless beam splitter (half-silvered mirror) $S_1$, after
which its wave function
is in a coherent superposition of $  b $ and $  c $.  The two beams are then recombined at a second lossless
beam splitter (half-silvered
mirror) $S_2$. The quant is detected at either $D_1$ or $D_2$,
corresponding to the states $d $ and $ e $, respectively.

}

\frame{
\frametitle{Counterfactual computation cntd.}

The computer is prepared to solve a decision problem, such that ---
{\em if} the particle takes pass $c$ towards the computer ---
activates it and lets the photon pass through ("yes"),
or blocks it ("no"); corresponding to the absence or presence of a bomb.
\begin{center}
$\widetilde{\qquad \qquad }$
$\widetilde{\qquad \qquad}$
$\widetilde{\qquad \qquad }$
\end{center}
Case 1:
Decision problem yields ``yes'' and thus path is free:
The computation proceeds by successive substitution (transition) of
states:
let ``$i$'' stand for a phase factor from the beam reflection at 90 degrees, then
\begin{eqnarray*}
S_1:\; a  &\rightarrow& ( b  +i c )/\sqrt{2}, \\
S_2:\; b  &\rightarrow& ( e  + i d )/\sqrt{2} ,\quad
S_2:\; c  \rightarrow ( d  + i e )/\sqrt{2}.
\end{eqnarray*}
The resulting transition is (normalization factors omitted)
$$
  a  \rightarrow b+ic \rightarrow id  + 0 \cdot e =id \quad .
$$
Thus, the emitted quant is always detected $D_1$, never in  $D_2$.
}

\frame{
\frametitle{Counterfactual computation cntd.}

Case 2: Decision problem yields ``no'' and thus path is blocked:
``Bomb'' $B$ presence is implemented by setting $c=0$; i.e.,
\begin{eqnarray*}
S_1:\; a  &\rightarrow& ( b  +i c )/\sqrt{2}\quad , \\
B:  \; c  &\rightarrow& 0,\\
S_2:\; b  &\rightarrow& ( e  + i d )/\sqrt{2}\quad ,
\end{eqnarray*}
The resulting transition is (normalization factors omitted)
$$
  a  \rightarrow b \rightarrow e + id  \quad .
$$
The emitted quant --- necessarily having taken path $b$ {\em without} activating the computer~(!) --- is detected with 50:50 chance in $D_1$ or $D_2$.
Thus, if $D_2$ clicks, we have certainty that the decision problem yields ``no''
without even having started the computation.

}


\section{Quantum cryptography}

\frame{
\begin{center}\Huge
{\color{purple}    Part \Roman{nc}:  \\
Quantum cryptography}
\end{center}
\addtocounter{nc}{1}
\begin{center}{\color{lime}
$\widetilde{\qquad \qquad }$
$\widetilde{\qquad \qquad}$
$\widetilde{\qquad \qquad }$ }
\end{center}
 }

\subsection{History}
\frame{
\frametitle{History}

\begin{itemize}
\item<+-> [1970]
Stephen Wiesner, {\em ``Conjugate coding:''}
noisy transmission of two or more ``complementary messages'' by using single photons in
two or more complementary polarization directions/bases.

\item<+-> [1984]
BB84 Protocol: key growing via quantum channel \& additional classical bidirectional communication channel

\item<+-> [1991]
EPR-Ekert protocol: maximally entangled state, three complementary polarization directions;
additional security confirmation by violation of Bell-type inequality
through data which cannot be directly used for coding


\end{itemize}
 }

\frame{
\frametitle{Man-in-the-middle attacks}
 {\footnotesize
\begin{itemize}
\item<+->
Not save against man--in--the--middle attacks.
\item<+->
Due to complementarity and value indefiniteness save against eavesdropping on the (quantum) channel.
\item<+->
Compare: ``Standard quantum key distribution protocols are provably secure against eavesdropping attacks, if quantum theory is correct.''
(from http://arxiv.org/abs/quant-ph/0405101).

\item<+->
 ``The need for the public (non-quantum) channel in this scheme to be immune to active eavesdropping can be
relaxed if the Alice and Bob have agreed beforehand on a small secret [[classical cryptographic]]  key,.."
(from BB84: C. H. Bennett and G. Brassard, 1984), pp. 175-179.)

\item<+->  More realistic:
``In accordance with our general philosophy that QKD forms a part of an overall cryptographic architecture, and not an
entirely novel architecture of its own, the DARPA Quantum Network currently employs the standardized authentication
mechanisms built into the Internet security architecture (IPsec), and in particular those provided by the Internet Key
Exchange (IKE) protocol.''
(from http://arxiv.org/abs/quant-ph/0503058)
\end{itemize}
}
}
\subsection{BB84 Protocol}
\frame{
\frametitle{BB84 Protocol}
%$\longrightarrow$ time\\
\begin{center}
\includegraphics[height=8cm]{2011-emtech-2005-qcrypt-pres-BBBSS92.pdf}
\end{center}
}




\frame{
\frametitle{Literature}
\begin{itemize}
\item<1->
Introductory: N. David Mermin,
 {\em ``Quantum Computer Science''}
  (Cambridge University Press, 2007)   \\
http://people.ccmr.cornell.edu/~mermin/qcomp/CS483.html
\item<1->
Extended:
M. A. Nielsen and I. L. Chuang,
{\em ``Quantum Computation and Quantum Information''}
(Cambridge University Press, 2000)

\item<1->
John Preskill's Caltech lecture notes, available at URL\\
http://www.theory.caltech.edu/people/preskill/ph229/

\end{itemize}

 }




\frame{
\centerline{\Huge {\color{purple}Thank you for your attention!}}
\begin{center}{\color{lime}
$\widetilde{\qquad \qquad }$
$\widetilde{\qquad \qquad}$
$\widetilde{\qquad \qquad }$ }
\end{center}
 }



\end{document}
