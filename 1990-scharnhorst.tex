 % Dear Peter:
 % I just received a thick parcel with your reprints and preprints--
 % thank you very much; I'll go through it in the next days.
 %
 % Our Scharnhorst Comment has been accapted by Phys. Lett. B as it is
 % (I just received the letter of acceptance); but I suggest to add
 % a very short footnote to it (please see \bibitem{footnote} in the
 % revised manuscript below).
 %
 % Please tell me as soon as possible whether the change is OK for
 % you.
 %
 % Best wishes & greetings
 % Karl
%----------------------START OF MANUSCRIPT -----------------------
\documentstyle[12pt]{article}
\begin{document}
\title{Impossibility of measuring faster--than--$c$ signaling
by the Scharnhorst effect}
\author{Peter W. Milonni\thanks{Permanent address: {\it
Theoretical Division, Los Alamos National Laboratory, Los Alamos, New
Mexico 87545, U.S.A.}} $\,$and Karl Svozil\\
{\small \sl Institute for Theoretical Physics,}\\
{\small \sl Technical University Vienna,}\\
{\small \sl Karlsplatz 13/136, A--1040 Vienna, Austria.}
}
 \date{}

\maketitle
\begin{abstract}
It is argued that the Scharnhorst effect cannot in principle give rise to
measured signal velocities larger than that of light in vacuum.
\end{abstract}
\baselineskip 4.5 ex

Scharnhorst \cite{scharnhorst} and subsequently Barton \cite{barton}
(see also \cite{barnett})
have
shown that the refractive index of the vacuum between parallel plates
can be less than unity for all frequencies $\omega < mc^2/\hbar$, where
$m$ is the electron mass \cite{f1}.
This arises from
the decreased polarizability of the vacuum between the plates {\it via}
virtual electron--positron pair production,
suggesting the possibility in principle of faster--than--$c$ signal
propagation.

Along these lines one could also conceive of a decreased polarizability
in the presence of
nuclei with charges above the critical charge ($Z\ge Z_{\rm
cr}\approx 164$--$172$),
associated with the charged vacuum as ground state \cite{greiner}. In
the vicinity of these nuclei, this would yield
a shift of the refractive index given by
 $\Delta n
(Z) = [1-\alpha \Pi
(Z)]^{1/2}- [1-\alpha \Pi (Z=0)]^{1/2} \approx -( \alpha /2)\Delta \Pi$
(``dived  states"), where $\alpha$ and $\Pi$ are the fine structure constant and
vacuum polarization, respectively, yielding a contribution to $\Delta n\approx
\Delta c/c=-{\cal O}(\alpha )$ which is one
order in the fine structure constant {\sl lower} than
the Scharnhorst effect.

The decrease of $n$ can be related to the negative
expectation value of the electromagnetic energy density $u$ between the
plates {\it via} the effective potential method.
In terms of the effective potential $\Gamma (A_\mu )$
($A_\mu$ stands for the expectation value of the electromagnetic field)
 and for vanishing
 momenta, $u$ may be written as \cite{coleman}
$\Gamma (A_\mu )=-(2\pi)^4\delta (0)u(A_\mu )$. It can be shown that $u$
is given by the
sum containing all one-particle irreducible {\sl (1PI)} Green's
functions with vanishing external momenta:
$u(A_\mu )=-\sum_{n=2}^\infty (1/n!) {\tilde \Gamma }^{(n)} (0,\ldots
,0)A_{\mu_1} \ldots
A_{\mu_n} $. The one--loop four-point Green's function has been used by
Scharnhorst to calculate the photon polarizability {\it via}
electron--positron pair production.
(Of course, this does not rule out cancellations of terms
${\tilde \Gamma}^{(n)}A_{\mu_1}\ldots A_{\mu_n}$ {\sl within} the
summation.)


The fact that $u<0$ is associated with
an attractive force between the plates (Casimir effect), and has already
been invoked in the somewhat similar context of wormholes and time
machines \cite{thorne}. The Scharnhorst effect is much too small to imagine
experimental tests, but it nevertheless raises an important point of principle
about whether it is possible for signal velocities to exceed the velocity $c$ of
light in vacuum. \cite{scharnhorst,barton,barnett} In this letter we show that
 the Scharnhorst
effect does not in principle allow one to measure signal velocities
 $v>c$.

A direct measurement of $v$ would involve a measurement of the time $t$
required for propagation over a fixed distance $L'$:
$v=L'/t$. Due to any uncertainty $\Delta t$ in the measured time, the
value of $v$ deduced in this way is uncertain by the amount $\Delta v=
L'\Delta t/t^2=c^2\Delta t/L'$. For the present discussion we take
$L'=L$, the plate separation in the Casimir vacuum considered by
Scharnhorst and Barton; this {\sl minimizes} the uncertainty in the
measured velocity.
\cite{footnote}

The uncertainty $\Delta t$ will be limited, aside from practical
considerations, by the uncertainty in time it takes to switch on a
signal. For instance, if a photon emitter is switched ``on'' at time
$t=0$, by exciting an atom, there is a fundamental
uncertainty as to when the quantum jump and photon emission will occur.
In the most optimistic case imaginable $\Delta t=1/\omega$, where $\omega$ is
the frequency of the signal. We could also choose $\Delta t$ to be the
radiative lifetime of an atom, but this would lead to a larger value of
$\Delta v$, and for the present argument we wish to examine the
situation most favorable to the possibility of faster--than--$c$
signaling.
Thus we take $\Delta v=c^2/\omega L=c\lambda /L$, where $\lambda$ is the
wavelength of the signal radiation.

The change in $c$ predicted by the Scharnhorst effect is $\Delta
c=\kappa \alpha^2 \hbar^4 /m^4c^3L^4$, where $\alpha$ is the fine
structure constant ($\cong 1/137$) and $\kappa \cong 1.3\times 10^{-2}$
\cite{barton}. Thus the ratio of the uncertainty in a measured value of
$v$ to $\Delta c$ is
\begin{equation}
{\Delta v\over \Delta c}={1\over \kappa \alpha^2}\left({\lambda \over
L}\right) \left( {mcL\over \hbar }\right)^4={1\over \kappa \alpha^2}
\left({\lambda \over
\lambda_c}\right) \left( {L\over \lambda_c }\right)^3
\quad ,
\end{equation}
where $\lambda_c=\hbar /mc$ ($\cong 3.9\times 10^{-11}$ cm) is the
electron Compton wavelength.
Barton has noted that the assumption $\lambda /L \ll 1$ is required not
only in order to permit clean thought experiments, but also to justify
the assumption of a local refractive index.
The theory also assumes $\lambda \gg \lambda_c$; in order again to be as
optimistic as possible in our analysis of the Scharnhorst effect, and to
minimize $\Delta v$, we use $\lambda =\lambda_c$ in (1):
\begin{equation}
{\Delta v\over \Delta c}\ge {1\over \kappa \alpha^2}\left({L\over
\lambda_c}\right)^3\approx 1.5\times 10^6 \left({L\over
\lambda_c}\right)^3 \quad .
\end{equation}
Morris {\it et al.}, in their discussion of wormholes and time machines
\cite{thorne}, have noted that plate separations less than the Compton
wavelength ``might well be forbidden''. In this connection note that,
if $L = \lambda_c,$ then $t = \hbar/mc^2 \equiv 1/\omega_{\rm max} <
1/\omega = \Delta t. $
(Recall that the Scharnhorst theory assumes $\omega < \omega_{\rm max}$.)
In other words, if $L = \hbar /mc$ the uncertainty in the time it takes a
signal to traverse the distance $L$ is larger than $t = L/c.$
This would imply $\Delta v > c.$ A more reasonable lower limit on $L$ might
 be the Bohr radius
$a_0=\lambda_c/\alpha$, in which case
\begin{equation}
{\Delta v\over \Delta c}\approx {1\over \kappa \alpha^5}\approx
3.7\times 10^{12}\quad .
\end{equation}
At such tiny separations the distinction between the plates as
macroscopic objects becomes blurred, of course, and repulsive forces
associated with overlapping electron wavefunctions come into play to
increase $u$ and weaken and remove the Scharnhorst effect. In any case
it is clear that the uncertainty in the measured propagation velocity
will always be enormously larger than the correction to $c$ associated
with the Scharnhorst effect.

We conclude, therefore, that no measurement of the faster--than--$c$
velocity of light predicted by the Scharnhorst effect is possible. It is
worth noting that our conclusion assumes the small value of the fine
structure constant determined by $e$, $\hbar$ and $c$.
In a universe in which $\alpha$ were large, our conclusion would not
hold. In other words, our conclusion rests on the small value of the
fine structure constant rather than the basic dynamical laws of physics.
This is not the first example where a violation of causality is ruled
out by the values of constants rather than dynamical laws \cite{davies}.

\begin{thebibliography}{999}
\bibitem{scharnhorst}
K. Scharnhorst, {\sl Phys. Lett.} {\bf B236}, 354 (1990).

\bibitem{barton}
G. Barton, {\sl Phys. Lett.} {\bf B237}, 559 (1990).

\bibitem{barnett}
S. M. Barnett, {\sl Nature} {\bf 344}, 289 (1990).

\bibitem{f1}
Note, however, that earlier claims of propagation velocities larger
than c due to
polarizability {\it via} virtual electron--positron pair production in
the presence of strong but slowly varying ($\lambda \gg \lambda_c=\hbar
/mc$) external
fields have been proven to be incorrect by Z. Bialynicka--Birula and I.
Bialynicka--Birula in {\sl Phys. Rev.} {\bf D2}, 2341 (1970).
A thorough discussion of the dielectric function and related
questions of causality and stability has been given by  O. V. Dolgov, D.
A. Kirzhnits and E. G. Maksimov in {\sl Rev. Mod. Phys.} {\bf 53}, 81
(1981). The distinction between signal and group velocity (as well as
other notions of velocity) is a longstanding concern and early work was
reviewed by L\'eon Brillouin, in {\sl Wave Propagation and Group
Velocity} (Academic Press, New York, 1960).

\bibitem{greiner}
W. Greiner, B. M\"uller and J. Rafelski, {\sl Quantum Electrodynamics of
Strong Fields} (Springer, Berlin 1985).

\bibitem{coleman}
S. Coleman and E. Weinberg, {\sl Phys. Rev.} {\bf D7}, 1888 (1973);
S. Coleman, in {\sl Laws of Hadronic Matter, Subnuclear Series volume
11, Part A}, ed. by A. Zichichi (Academic Press, New York, 1975).

\bibitem{thorne}
M. S. Morris, K. S. Thorne and U. Yurtsever, {\sl Phys. Rev. Lett. }
{\bf 61}, 1446 (1988); M. S. Morris and K. S. Thorne, {\sl Am. J. Phys.}
{\bf 56}, 395 (1988).

\bibitem{footnote}
 $L'$ could in principle be made arbitrary large by allowing a light
 pulse to bounce back and forth repeatedly between the plates. This
 could, however, not be used for faster--than--$c$ {\it signaling}.

\bibitem{davies}
See P. C. W. Davies, {\sl The Physics of Time Asymmetry} (University of
California Press, Berkeley, 1977), p. 127.

\end{thebibliography}
\end{document}
