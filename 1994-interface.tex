\documentclass[rmp,amssymb,named]{revtex4}
%\documentclass[rmp,amssymb,showpacs,showkeys,twocolumn]{revtex4}
%\documentclass[rmp,preprint,amsfonts,showpacs,showkeys]{revtex4}
%\documentclass[pra,showpacs,showkeys,amsfonts]{revtex4}
\bibpunct{[}{]}{,}{a}{}{;}
\usepackage{graphicx}
\usepackage{eepic}
\RequirePackage{times}
\RequirePackage{mathptm}
%\RequirePackage{courier}
%\renewcommand{\baselinestretch}{1.3}
\usepackage{xcolor}

\def\newpic#1{%
   \def\emline##1##2##3##4##5##6{%
      \put(##1,##2){\special{em:point #1##3}}%
      \put(##4,##5){\special{em:point #1##6}}%
      \special{em:line #1##3,#1##6}}}
%
% Standarddefinition von \emline herstellen
%
\newpic{}

\begin{document}
%\sloppy


\title{How real are virtual realities, how virtual is reality? \\The constructive re-interpretation of physical undecidability}


\author{Karl Svozil}
\email{svozil@tuwien.ac.at}
\homepage{http://tph.tuwien.ac.at/~svozil}
\affiliation{Institute for Theoretical Physics, Vienna University of Technology, \\
Wiedner Hauptstra\ss e 8-10/136, A-1040 Vienna, Austria}


\begin{abstract}
Throughout the ups and downs of scientific world conception there has
been a persistent vision of a world which is understandable by human
reasoning.
In a contemporary, recursion theoretic, understanding, the term
``reasoning''
is interpretable as ``constructive'' or, more specifically,
``mechanically
computable''. An expression of this statement is the assumption that
our universe is generated by the action of some deterministic computing
agent; or,
stated pointedly, that  we are living in a computer-generated
universe.
Physics then reduces to the investigation of the intrinsic,
``inner view'' of a particular virtual reality which happens to be our
universe.
In this interpretation, formal logic, mathematics and the computer
sciences are just the physical sciences of more general ``virtual''
realities, irrespective of whether they are ``really'' realized or not.
We shall study several aspects of this conception, among them
the conjecture that
randomness in physics  can be constructively re-interpreted to
correspond to uncomputa\-bi\-lity and
undecidability in mathematics.
We shall also attack the non-constructive feature of classical physics
by showing its inconsistency.
Another concern is
the modeling of interfaces, i.e., the means and
methods of communication between two universes.
On a speculative level,
this may give some clue
on such notorious questions such
as the ocurrence of ``miracles'' or on the ``mind-body problem.''
\end{abstract}

\pacs{02.10.-v,02.50.Cw,02.10.Ud}
\keywords{context, context translation, quantum logic, generalized urn models, automaton logic, Boolean algebra}


\maketitle



%\begin{flushright}
% PROSPERO. You do look, my son, in a mov'd sort,     \\
%   As if you were dismay'd; be cheerful, sir.        \\
%   Our revels now are ended. These our actors,       \\
%   As I foretold you, were all spirits, and          \\
%   Are melted into air, into thin air;               \\
%   And, like the baseless fabric of this vision,     \\
%   The cloud-capp'd towers, the gorgeous palaces,    \\
%   The solemn temples, the great globe itself,       \\
%   Yea, all which it inherit, shall dissolve,        \\
%   And, like this insubstantial pageant faded,       \\
%   Leave not a rack behind. We are such stuff        \\
%   As dreams are made on; and our little life        \\
%   Is rounded with a sleep. $\ldots$
%\end{flushright}

 \tableofcontents


\section{Physical constructivism}

%Ladies and Gentlemen!

Virtual physics is the study of the intrinsic
perception of computer-generated universes.
Algorithmic physics is the study of physical systems by methods
developed in formal logic and the computer sciences.
Both fields of research
may be conceived as two sides of the same
constructivistic
attempt.
In that way, virtual reality is a powerful
``intuition pump'' for algorithmic physics, and {\it vice versa.}

I shall first give an outline of virtual physics and algorithmic
physics. Then I shall propose a constructive
re-interpretation of undecidability in the context of algorithmic
physics. Finally,  I shall come back to virtual physics, in
particular to questions related to interface modeling and ethics.
Technical issues are addressed in the appendix.


\subsection{Virtual Physics}
Look at some computer. At ``face value'' it is a dull box; nothing
spectacular.
%In the early days mainframes like IBM's 360 series had an
%array of frenziedly blinking glitzy lights on their front panel and
%rotating tape wheels in the back; but that changed.
%%
%\footnote{Such light panels are the first visual interfaces to the
%computer's processes. Why do these flashes make us nervous while we
%experience
%the reflections of the low setting sun on water as joyful and
%relaxing?}
%%
%One may safely
%say that the duller the boxes looked, the more powerful they became.
%Today illuminations on
%computer equipment mostly means an attempt of the manufacturer to
%impress potential customers; mostly the military.
The real quality of a computer is something else.
It is no external place. When you ``enter'' computers with virtual
reality interfaces, they
are a medium to new universes; they become doors of perception.

I had a dream. I was in an old, possibly mideval, castle.
I walked through it. At times I had the feeling that there
was something ``out there,'' something so
inconceivable hidden that it was impossible to recognize.
Then suddenly I realized that there was something ``inside the walls:''
another, dual,
castle, quite as spacious as the one I was walking in, formed by the
inner side of what one would otherwise consider masonry.
There was a small opening, and I glanced through; the inside looked
like a threedimensional maze inhabited by dwarfs. The opening closed
again.

Computers are exactly such openings; doors of perception to hidden
universes.
One may ask \cite{krenn-private}, ``what exactly makes a reality
virtual?''
Or conversely, ``what makes a computer-generated universe so real?''
One may also ask, ``where exactly is this `undiscover'd
country?'
Is it in the circuitry? Is it on the screens, in the interfaces, in the
senses, in the mind?'' --- These are old questions. They have been
addressed
with respect to where exactly the mind is located. They can also be
applied to the characters in a book or on a movie screen.

To cope with the intrinsic phenomenology of computer-generated universes
systematically, we have to develop their ``virtual physics.''
This includes experiments, observations and theoretical models which
are intrinsically operational.
It is different from an outside description of a process of
computation.

Virtual physics is neither a classical discipline of
mathematics,
logic or the computer sciences, nor can it pretend to be a traditional
physical science.
Its scope is the intrinsic perception and interpretation of pure syntax.

Is this pure syntax independent of the hardware on which it is
implemented? --- Yes and no: at first it would seem that within the
domain of universal computation, syntactical structures are the same, no
matter whether they are implemented on a silicon-based Turing machine, a
nerve-based Cellular Automaton, or on a common billard table.
A second glance reveals that it may be possible for a program to
back-act on its hardware,
very much like a malignant computer virus destroys its host
processor by heating it up.


We may explore a countable number of universes of computation
by virtual physics, but can
we step outside of this syntactical frame formed by universal
computation?
There is reason to believe that this might be diffult for now.
If we extend the domain of universal computation, say, by
allowing some form of oracle computation, we risk inconsistency.
If we restrict our domain of computation, the resulting worlds will be
monotonous.
There is still another question related to consistency: Will all the
different universes of computation
---
the tree of mathematical perception
---
eventually collapse into
a single one?

Let me point out that virtual physics
is part of a program called endophysics
\cite{toffoli,r�ssler,svozil-93}. Endophysics, in short,
investigates the intrinsic perception of an observer who inescapably is
part of the observed system --- the Cartesian prison
\cite{r�ssler-private}.




\subsection{Algorithmic physics}

Algorithmic physics considers a physical system as a constructive
object, more specifically, as an algorithm, a
process of computation. It encompasses algorithmic information
theory, computational complexity theory,
the theory of recursive functions and  effective computability, in
particular undecidability theory in the physical context. The
latter fields are in their infancies, while algorithmic information
theory and computational
complexity theory have attracted large attention due to their
applicability
in statistical physics and chaos theory. Algorithmic physics is based on
the assumption that it is justified
to assume that the world is  or at least can be suitably modeled
by a machine; To be more precise, that the physical
universe is conceived as a computation and can
be described by a recursive function.


Of course,
at the present moment, everyone pretending that the universe is a
Turing-compatible machine is in a state of sin.
There are some features of our current physical woldview which seem to
be in total contradiction to an intuitive understanding of
``machine-like'' and ``computation.''

Take, for example, the classical and the quantum mechanical
concept of information.
Intuitively and classically, a unit of information is context-free.
It is independent of what other information is or
might be present. A classical bit remains unchanged, no matter by what
methods it is inferred. It can be copied. No doubts can be left.

By contrast,
quantum information is contextual.
A quantum bit may appear different,
depending on the method by which it is inferred
(cf. appendix \ref{appendix1} and
\cite{specker,redhead,peres,mermin}).
Quantum bits cannot be copied or ``cloned''
\cite{no-cloning-theorem}.
Classical tautologies are not necessarily satisfied in quantum
information theory
(cf. appendix \ref{appendix2} and
\cite{specker2,specker3}).

More generally,
quantum systems cannot be trivially ``extended'' and embedded into a
classical world \cite{specker}.
Quantum complementarity states that there are observables which cannot
be measured simultaneously with arbitrary accuracy.
There exist events which cannot be predicted; in the present dictum,
they
``occur at random.''

Even the classical physics of continua operates with entities which are
formally random. For instance, with probability one, i.e.,
almost all, elements of the continuum are
Martin-L\"of/Solovay/Chaitin-random,
a predicate characterizing (on
the average) the uncomputability of each individual digit in an
expansion.
No ``algorithmic compression'' of such random reals is possible,
and one would need an infinite amount of storage and time to represent
and manipulate them.
Classical physics based on classical analysis cannot be implemented on
a universal computer.

It turns out that celestial mechanics has never been in the realm of any
``reasonable''
Laplacian computing demon but was predistined to become part of chaos
theory. Of course,
one could endow Laplace's demon with oracle computing power, but
then oracle computation would just be another word for Almighty God.

Moreover,
recursive analysis
states that there are computable functions which have
their maximum at a uncomputable argument \cite{specker,kreisel}. This
may be important for physical variational principles.

All this together may persuade oneself into thinking that physics has
``finally'' come to the conclusion that the world is irrational at heart
and therefore cannot be fully modeled by any reasonable formalism.
Any statement denying this may be considered as unnecessary at best or
sheer heresy and nonsense at worst.

Contrary to these
understatements,
there is still reason to pretend that
the universe is governed by constructive laws, and that it can be
described by a finite set of symbols representing these laws.
This may be the reason for the ``unreasonable
effectiveness of mathematics in the
natural sciences'' \cite{wigner}.

With regards to physics, constructivism and classical
non-constructivism are pa\-ra\-digms, not facts.
After all, we shall never be able to proof whether the
world as we can perceive it is solely governed by constructive laws.
For it is impossible to know all constructive laws and their
consequences \cite{frank}.


\section{Three forms of undecidability}
What then is algorithmic physics good for?
I have already mentioned algorithmic information theory and complexity
theory, but I would like to concentrate on a different topic.
There is reason to believe that algorithmic physics in general and
recursion theoretic
diagonalization in particular is the royal road
to a {\em constructive re-interpretation} of two different
types of physical
undecidability: complementarity and unpredictability.

A third type of
physical undecidability, if it exists, is
randomness or, its weaker form, non-recursivity of a system evolution or
of the initial values {\it et cetera.} It cannot be
constructively re-interpreted. As will be argued further on, any formal
modeling of this third form of physical undecidability
necessitates a form of oracle computation which is too strong to be
consistent.

\subsection{Copenhagen interpretation of automaton logic}

Computational complementarity is based upon the observation
\cite{e-f-moore} that an interaction of the experimenter
with the  observed  object
---
modeled by an exchange of information between algorithmic objects
---
may induce a transition of the observed object which results in the
impossibility to measure another, complementary, observable;
the same is true {\it vice versa.}
The observer has a qualifying influence on the measurement result
insofar as a particular observable has to be chosen among a class of
non-co-measurable observables.
But the observer has no quantifying influence on the measurement result
insofar as the outcome of a particular  measurment is concerned (cf.
Zeilinger's contribution to this volume).


This can be modeled by finite automata.
An automaton (Mealy or Moore machine) is a finite deterministic system
with input and output capabilities.
At any time the automaton is one state of a finite set of states.
The state determins the future input-output behavior of the automaton.
If an input is applied, the machine assumes a new state, depending
both on the old state and on the input, emitting thereby
an output, dependig also on the old state and the input
(Mealy machine) or depending only on the new state (Moore machine).
Automata experiments are conducted by applying an input sequence and
observing the output sequence.
The automaton is thereby treated as a black box with known
description but unknown initial state.
As has already been observed by Moore, it may
occur that the automaton undergoes an irreversible state change, i.e.,
information about the automaton's initial state is lost.
A second, later experiment may therefore be affected by the first experiment,
and vice versa.
Hence, both experiments are incompatible.

Corresponding to any such automaton there is a propositional structure,
its so-called automaton logic or partition logic,
for which experimental statements from input-output analysis are
ordered;
cf. appendix \ref{appendix3} and Strnadl's contribution to this
volume. In Fig.
 \ref{f-hdxxx}, all experimental-logical structures of four-state Mealy
automata are drawn.
\begin{figure}
  \centering
\includegraphics[width=90mm]{ch-gal.ps}
\caption{The class ${\cal F}_4$ of non isomorphic Hasse diagrams
 of the intrinsic propositional calculi of generic
  Mealy  automata of up to four states.
 \label{f-hdxxx}}
\end{figure}


Similarity and difference between quantum  and computational
complementarity can been made precise.
A systematic investigation
 reveals that
 automaton logic
is mostly non-Boolean and thus non-classical.
Many but not all orthomodular lattices ocurring in
quantum logic  can be realized by the logic of some particular automaton
\cite{finkelstein-83,zapatrin},
but automaton logic is not identical with quantum logic
\cite{svozil-93,schaller-svozil}.



Since any finite state
automaton can be simulated by a universal computer,
the class of non-Boolean automaton logic --- and not classical Boolean
logic
--- corresponds to the natural order of events in (sufficiently complex)
computer generated universes.
To put it pointedly:
if
the physical
universe is conceived as the product of a universal computation,
then complementarity is an inevitable and necessary feature of its
intrinsic perception or endophysics.
It cannot be avoided.
Computational complementarity may serve as a constructive
re-interpretation quantum complementarity.



\subsection{Undecidability by diagonalization}



Unpredictable events  ``occuring at random'' may result from
the
intrinsic description of systems which are computable on a step-by-step
basis.
As G\"odel himself put it (cf. \cite{v-neumann-66}, p. 55),
 {\em
 ``I think the theorem of mine which von Neumann refers to is not
 that on
 the existence of undecidable propositions or that on the lengths of
 proofs but rather the fact that a complete epistemological description
 of a language $A$ cannot be given in the same language $A$, because
 the concept of truth of sentences of $A$ cannot be defined in $A$. It
 is this theorem which is the true reason for the existence of
 undecidable propositions in the formal systems containing
arithmetic.''}

That a system which is computable on a step-by-step basis features
uncomputability in forecasting
sounds amazing, if not self-contradictory, at first.
Yet this can be ``algorithmically proven'' quite easily
(cf. appendix \ref{appendix4}).
The method of diagonalization employed in the proof
closely resembles Cantor's  diagonalization method
(based on the ancient liar paradox \cite{bible})
which has been
applied by G\"odel, Turing and others for undecidability proofs in a
recursion theoretic setup.

To proof undecidability for a particular physical system, a universal
computer such as a universal Turing machine is usually embedded in that
system. Then, one (mostly implicitly) applies diagonalization to obtain
undecidability. Therefore, any physical realisation of a computer
(with potentially infinite memory) is an example for a physical system
for which undecidable propositions can be formulated. Rather than
consider this further, I shall concentrate on how the
method of diagonalization can be applied in quantum information theory.

Diagonalization effectively
transforms the classical bit value ``0'' into value ``1'' and
``1'' into ``0.''
Any information has a physical representation. The corresponding
bit states can
be quantum mechanically expressed by $\vert 0\rangle$ and
$\vert 1\rangle$.
The classical bit base is
$\{ \vert 0\rangle,
\vert 1\rangle \}$.
The evolution representing diagonaliation can be
expressed by the
unitary operator
$
\widehat{D}
$ as follows
$
\widehat{D} \vert 0\rangle  = \vert 1\rangle $, and
$\widehat{D} \vert 1\rangle  = \vert 0\rangle $.
In this state basis
($\tau_1$ stands for the Pauli spin operator),
\begin{equation}
\widehat{D}=
\tau_1 =
\left(
\begin{array}{cc}
0 & 1\\
1 & 0
\end{array}
\right) =\vert 1\rangle \langle 0\vert
+ \vert 0\rangle \langle 1\vert    \quad .
\end{equation}
$
\widehat{D}
$
will be called
{\em diagonalization} operator,
despite the fact that the only nonvanishing components are
off-diagonal.

Quantum information theory allows a coherent superposition of the
classical bit states; therefore the quantum bit base is
\begin{equation}
\{ \vert a,b\rangle \mid \vert a,b\rangle =a\vert 0\rangle +b\vert 1
\rangle
,\; \vert a\vert^2+\vert b\vert^2=1,\; a,b\in {\Bbb C} \}
\quad .
\end{equation}
$
\widehat{D}
$
has a
fixed point at
\begin{equation}
\vert {1\over \sqrt{2}},{1\over \sqrt{2}} \rangle
\quad ,
\end{equation}
 which is a coherent superposition of the classical bit base and
does not give rise to inconsistencies \cite{svozil-paradox}.
Classical undecidability is retained, since the probability
for
the fixed point state
$\vert {1\over \sqrt{2}},{1\over \sqrt{2}} \rangle $
to be in either
$\vert 0\rangle =\vert {1},{0} \rangle $
or
$\vert 1\rangle =\vert {0},{1} \rangle $
is equal; i.e.,
\begin{equation}
\vert \langle 0\mid {1\over \sqrt{2}},{1\over \sqrt{2}} \rangle \vert^2
=
\vert \langle 1\mid {1\over \sqrt{2}},{1\over \sqrt{2}} \rangle \vert^2
={1  \over 2}
\quad .
\end{equation}



Another question in this context is whether a complete
description of a system is possible within that system \cite{popper}.
In algorithmic physics, the term ``description'' is
identified with the concept of algorithm or recursive function.
Von Neumann was the first who proved the possibility of the existence of
a ``system blueprint'' or algorithmic description which would allow a
complete self-reproduction of the system. The model of
computation was universal cellular automata  \cite{v-neumann-66}. Yet,
in general
self-description by self-inspection must be excluded, and the only
possible
way to obtain such a self-description is by oracle. Also excluded must
be an up-to-date complete self-comprehension which leads to an infinite
series of such attempts. For, in order to describe itself completely, a
deterministic agent has to describe itself describing itself describing
itself $\cdots$ and so on. This
argument resembling Zeno's paradox of Achilles and the Turtoise has been
  called
 by Popper
``paradox of Tristram Shandy''
 \cite{popper}.
Finite observers cannot obtain complete self-comprehension.
In psychology, the above setup is referred to as the {\em observing
ego.} In experiments of this kind --- e.g., imagine a vase on a table;
now imagine you imagining a vase on a table;
now imagine you imagining you imagining a vase on a table;
now imagine you imagining you imagining you imagining a vase on a table;
now imagine you imagining you imagining you imagining you imagining a
vase on a table --- humans may concentrate on $3-5$ levels of
iteration.


\subsection{Consistency versus
strength --- the inconsistency of classical mechanics}



Quantitatively, one message of undecidability theorems
is that,
in a very particular sense, one cannot get more wisdom out of a system
than one has put in.
This can be proven within algorithmic information theory.
Related to this is the fact that it is shorter to describe
a family of objects than to describe particular objects of the family
\cite{chaitin,calude}.


There is yet another, qualitative, moral of undecidability theorems.
It seems that whenever a system
becomes too powerful, it becomes inconsistent.
Conversely, any reasonable, i.e., consistent, system must be limited.
This is true for formal systems as well as for physical ones.

Let us demonstrate this in the context of classical physics.
If one is willing to accept classical, i.e., Hilbert-style analysis,
then classical continuum
mechanics, and, in fact, any theory based on dense sets, becomes
inconsistent.

Continuum theory, and in fact any theory based on dense sets, allows the
construction of ``infinity machines,''
which could serve as oracles for the halting
problem \cite{weyl,gruenbaum,svozil-93}.
Their construction closely follows Zeno's paradox of Achilles
and the Tortoise by squeezing the time it takes for successive
steps of computation $\tau$ with geometric progression:
\unitlength=0.5mm
\special{em:linewidth 0.4pt}
\linethickness{0.4pt}
\begin{picture}(100.00,2.10)
\put(0.00,0.00){\line(1,0){100.00}}
\put(0.00,5.00){\makebox(0,0)[cc]{0}}
\put(0.00,-2.00){\line(0,1){4.00}}
\put(50.00,5.00){\makebox(0,0)[cc]{1}}
\put(50.00,-2.00){\line(0,1){4.00}}
\put(75.00,5.00){\makebox(0,0)[cc]{2}}
\put(75.00,-2.00){\line(0,1){4.00}}
\put(88.00,5.00){\makebox(0,0)[cc]{3}}
\put(88.00,-2.00){\line(0,1){4.00}}
\put(94.00,5.00){\makebox(0,0)[cc]{4}}
\put(94.00,-2.00){\line(0,1){4.00}}
\put(97.00,5.00){\makebox(0,0)[lc]{$\cdots$}}
\put(97.00,-2.00){\line(0,1){4.00}}
\put(99.00,-2.00){\line(0,1){4.00}}
\put(100.00,-2.00){\line(0,1){4.00}}
\put(99.67,-2.00){\line(0,1){4.00}}
\put(99.83,-2.00){\line(0,1){4.00}}
\put(99.50,-2.00){\line(0,1){4.00}}
\end{picture}
$\quad $
I.e.,
the time necessary for the $n$'th step becomes $\tau (n)=k^{n}$, $k<0$.
The limit of infinite computation is reached in finite
physical time
$ \lim_{N\rightarrow \infty}\sum_{n=1}^N \tau{(n)}=
  \lim_{N\rightarrow \infty}\sum_{n=1}^N  k^n=
    1/(1-k)$.


On such oracle machines it would be possible
to ``oracle-compute'' the non-recursive limit of
Specker's bounded increasing sequence of rational numbers
\cite{specker-theorem}
as well as Chaitin's halting ``probability'' $\Omega$ \cite{chaitin}
as well as to ``solve'' the halting
problem.
--- Ay, there's the rub!
As has been argued before, a ``halting algorithm'' may be used in a
diagonalization argument to construct a complete contradiction in the
classical bit base. The same argument can be used to derive the
inconsistency of classical mechanics. Here, the term ``classical''
refers both to physical continuum mechanics, as well as
to mathematical non-constructivism.

There
is no {\it a priori} reason in classical physics to exclude such
infinite processes and thus to avoid this
inconsistency.
One may nevertheless attack the argument both metamathematically and
physically.
Constructive mathematics denies the existence of objects
which cannot be obtained by finite means \cite{bishop,bridges}.
Finitism in physics excludes the existence of
natural
entities which correspond to
infinities.

A third possibility would be to accept fundamental inconsistencies.
How would inconsistencies in the intrinsic phenomenology ``show up'' and
be perceived by an observer? At face value, it is taken for granted that
phenomena
``have to be'' consistent.
Yet there is some reason to suspect that inconsistencies may be
perceived as a certain type of ``fuzzyness'' or unpredictability
\cite{svozil-paradox}.
 After all,
there exist reasonable algorithmic entities such as expert systems or
databases which may become inconsistent, yet remain of value for certain
applications.





\section{Interface design}

The  term {\em physical universe} will be used as a synonym for the
universe we live  and do physics in.
The terms {\em virtual reality} or {\em computer-generated universe}
or {\em cyberspace}
are synonyms for some reality mediated by some
computing agent.
The term {\em player} stands for a conscious observer, who for instance
could be thought of as living in the physical universe and who is
experiencing the virtual reality {\it via} some interface.


\subsection{Generic interface and notation}

An interface connects two universes.
It is a means or mode of communication and interaction between two
universes.
For example, one universe may be our physical universe, while the other
universe may be a virtual, computer-generated, reality.
In another interpretation, both universes may be identical.

An interface always characterises a distinction between two universes
\cite{spencer-brown}.
This distinction may be formed by a cut within one universe.
Take the process of observation. It can be modeled by a cut between the
observer and the object under observation.


\subsection{Symmetry of interface}
For an observer in one universe,
an interface is an indirect
means
of probing deeper into the other universe.
From a syntactic point of view, the interface enables an exchange of
symbols or information between two universes.
The interpretation of this exchange
is a question of semantics, convention and intent.
In most of the cases it will be intentionally clear on which
side of the interface the observer is located and on which side the
observed object.
An example is given in appendix
\ref{appendix5}.



Yet, while in many practical cases the arguments justify the view of the
interface
as an asymmetric device, it should in principle be perceived and
modeled as a {\em symmetric} device which allows information to
flow between two universes.

Take, for example, a summer meadow.
You are observing it.
How is the summer meadow observing you?
You have consciousness.
What consciousness has the summer meadow?

\subsection{Joy of cyberspace --- death in cyberspace}


What does it mean for an interface to some virtual reality ``to be
safe?''
Will the present safety measures applying to household equipment such as
television sets suffice? Will it for instance be enough to ground a data
glove or to restrict the light intensity of an eye-phone?

One may state that
the more powerful the interface is, the more beneficial or
malign will the influences from the virtual reality towards the actor
be, both from a physical and much more so psychical point of view.
Since the human consciousness has the tendency to construct a
``consistent, lucid reality,'' the consciousness of the player about
 a virtual environment gets lost.

But
even if the player using the interface is well aware of this fact, it
may be hard leaving a virtual reality. What if the player is not
aware? What if the player gets killed in virtual reality?


There appear to occur certain potentials for misfortune
if a virtual reality back-reacts in a destructive way. There is no
interface
design without any interaction between the virtual reality and the
senses of the observer.
What if the virtual reality and the interface mailfunctions or is
subject of a criminal attempt or of an unpredictable malignancy?
This may hurt the observer.
Take, as an example, a wrestling experience with a huge spider.
One has to make sure that, at least physically,
the interface cannot in any way harm the observer. Probably the only
general way to do this is to allow for some form of virtual emergency
exit; a super-rule push-bottom providing the observer with an exit
from the
interface and thus from the virtual reality at any particular instance,
no matter what.
Or, one may use intrinsic means of sensual limitations, very much as
endorphins limit certain pains.
As a consequence of unhappy occurrences, terrible traumata will occur,
being a challenge both for medicine and jurisdiction.

Much as computer viruses physical destroy hardware, malign virtual
realities will destroy actors physically.
They may make use of the actor's primary body to harm it via the
interface. This
virtual backflow \cite{svozil-93}
is an instance where there is a reference, indirect though,
of the virtual reality towards its meta-universe \cite{putnam}.
It is also an additional ``opening'' of one universe into the other; an
irregular interface, if you like.



\subsection{Indeterministic interface and miracles}

Since for safety reasons an interface to some virtual reality will have
to be designed to give only limited control
to the meta-world of the
person's physical existence, the actor's behavior will remain
``psychic'' with respect to the virtual reality.

For the same reason, i.e., the limited access
({\it via} the interface) to the meta-world, the intrinsically definable
(operationalisable)
physics of the virtual reality must remain necessarily
non-deterministic,
because there will always be factors and information emanating from the
interface which have no cause intrinsically.

Take, for example, a virtual reality which is accessible by eyephone
and data glove.
These interfaces --- eyephone and data glove --- have an intrinsic
representation in the virtual reality; e.g., as two eyes and a hand with
five fingers.

Consider the hand if it is not in use.
Then its evolution can be completely described by the laws dominating
the virtual reality.
One may also say that the interface is ``idle.''

Now consider the hand when some player decides to
use it.
This decision cannot be predicted from within the virtual reality,
but depends on outside criteria; e.g., the player
coming back from somewhere else {\it et cetera.}
Furthermore, with the interface ``at work,'' the evolution  can
no longer be completely described by the laws dominating the virtual
reality.
When the actor decides to leave the virtual reality, say, because the
actor has run out of time, money or is simply hungry, this decision is
again intrinsically undecidable.

The setup can be modeled by an infinite deterministic computation
($=$ the computer-generated universe) receiving data input {\it via} the
interface. Whereas one might be able to formulate a deterministic
``meta''-model of both infinite computation and data input on a
meta-level, the data input is not predictable nor controllable from
within the infinite computation.
Therefore, the data input as seen from within the
computer-generated universe appears as a miracle.
It may nevertheless be possible to completely describe the
interface
by changing the level of description to a higher ``meta-description''
level which includes the physical universe of the player.

This directly translates into
Philip Frank's considerations of so-called ``L\"ucken in den
Naturgesetzen''
(English translation ``gaps in the natural laws'') and ``Wunder''
(English translation
``miracle''); cf. \cite{frank}, sections III.12-15 \& VI.21.
It can also be rephrased into dualistic mind-body models as for
instance envisaged
by Eccles \cite{eccles}: In our terminology the brain as well as other
body organs may be interpreted as an interface to the physical
universe. The player is interpreted as an ``(intrinsically) immortal
soul.''
Take The Doors' ``no one here gets out alive,'' or Godard's ``we are the
dead on vacation.''

%Thank you.




\section*{Acknowledgements}
The author acknowledges stimulating discussions with Professor Cris
Calude, Dr. G\"unther Krenn, Professor Otto E. R\"ossler and Dr.
Christoph Strnadl.
Professor Ernst Specker made available the letter from Professor
Kurt Sch\"utte as well as the dissertation of Dr.
Erna Clavadetscher-Seeberger.
Thanks go also to Professor Rob Clifton for his help with the Sch\"utte
rays.


\appendix
\section*{Appendix}

\section{Contextuality of quantum information}
\label{appendix1}
Assume that in an EPR-type arragement
\cite{epr}
one wants to measure  the product
$$
P=
m_x^1m_x^2
m_y^1m_y^2
m_z^1m_z^2
$$
of the direction of the spin components of each one of the two
associated particles
$1$ and $2$ along the $x$, $y$ and $z$-axes.
Assume that the operators are normalized such that $\vert m_i^j\vert=1$,
$i\in \{ x,y,z\}$, $j\in \{ 1,2\}$.
One can determine $P$, for
instance,
by
measurement and by
counterfactual inference \cite{peres,mermin} and multiplication of
the three ``observables''
$
m_x^1m_y^2$, $
m_y^1m_x^2$ and $
m_z^1m_z^2$, by which way one obtains $+1$.
One can also do that
by
measurement and by
counterfactual inference and multiplication of
the three ``observables''
$
m_x^1m_x^2$, $
m_y^1m_y^2$ and $
m_z^1m_z^2$, by which way one obtains $-1$.
In that way, one has obtained
either $P=1$ or $P=-1$.
Associate with $P=1$ the bit state zero and with $P=-1$ the
bit state one.
Then the bit is either zero or one, depending on the way
or context it was inferred.
This kind of contextuality is deeply rooted in the non-Boolean
algebraic structure of quantum propositions.
Note also that the above argument relies heavily on counterfactual
reasoning, because, for instance, only two of the six observables
$m_i^j$ can actually be experimentally determined.


\section{Not all classical tautologies are quantum tautologies}
\label{appendix2}
I shall review the shortest example of a classical
tautology which is not valid in
threedimensional (real) Hilbert space that is known
up-to-date \cite{sch�tte-letter}.

Consider the propositions
\begin{eqnarray}
d_1   & \rightarrow &\neg b_2 \label{sf1}\quad ,\\
d_1   & \rightarrow &\neg b_3\quad ,\\
d_2   & \rightarrow &a_2 \vee b_2\quad ,\\
d_2   & \rightarrow &\neg b_3\quad ,\\
d_3   & \rightarrow &\neg b_2\quad ,\\
d_3   & \rightarrow &( a_1\vee a_2\rightarrow b_3)\quad ,\\
d_4   & \rightarrow &a_2 \vee b_2\quad ,\\
d_4   & \rightarrow &( a_1\vee a_2\rightarrow b_3)\quad ,\\
(a_2 & \vee & c_1) \vee (b_3 \vee d_1)\quad ,\\
(a_2 & \vee & c_2) \vee (a_1 \vee b_1 \rightarrow d_1)\quad ,\\
c_1   & \rightarrow & b_1 \vee d_2\quad ,\\
c_2   & \rightarrow & b_3 \vee d_2\quad ,\\
(a_2 & \vee & c_1) \vee [ (a_1 \vee a_2 \rightarrow b_3)\rightarrow
d_3]\quad ,\\
(a_2 & \vee & c_2) \vee (b_1 \vee d_3)\quad ,\\
c_2& \rightarrow & [ (a_1 \vee a_2 \rightarrow b_3)\rightarrow
d_4]\quad ,\\
c_1 &\rightarrow  &(a_1 \vee b_1 \rightarrow d_4)\label{sf16}\quad ,\\
(a_1 &\rightarrow &a_2)\vee b_1 \quad .  \label{sf17}
\end{eqnarray}
The proposition
formed by
$F:$ (\ref{sf1})$\wedge  \cdots
\wedge$(\ref{sf16})$\rightarrow$(\ref{sf17})  is a classical tautology.

$F$ is not valid in
threedimensional (real) Hilbert space $E^3$, provided one identifies the
$a$'s, $b$'s and $c$'s with the following onedimensional subspaces of
$E^3$:
\begin{eqnarray}
 a_1&\equiv &{\cal S}(1,0,0) \quad  ,\\
 a_2&\equiv  &{\cal S}(0,1,0) \quad , \\
 b_1&\equiv  &{\cal S}(0,1,1) \quad , \\
 b_2&\equiv  &{\cal S}(1,0,1) \quad , \\
 b_3&\equiv  &{\cal S}(1,1,0) \quad , \\
 c_1&\equiv  &{\cal S}(1,0,2) \quad , \\
 c_2&\equiv  &{\cal S}(2,0,1) \quad , \\
 d_1&\equiv  &{\cal S}(-1,1,1)\quad , \\
 d_2&\equiv  &{\cal S}(1,-1,1)\quad , \\
 d_3&\equiv  &{\cal S}(1,1,-1)\quad , \\
 d_4&\equiv  &{\cal S}(1,1,1) \quad ,
\end{eqnarray}
where
${\cal S}(v)=\{av \mid a\in {\Bbb R} \}$ is the subspace  spanned by $v$.

Let the ``or'' operation  be represented by
${\cal S}(v)\vee
{\cal S}(w)=\{av +bw\mid a,b\in {\Bbb R} \}$  the linear span of ${\cal
S}(v)$
 and ${\cal S}(w)$.

Let the ``and'' operation  be represented by
${\cal S}(v)\wedge
{\cal S}(w)=
{\cal S}(v)\cap
{\cal S}(w)
$  the set theoretic complement of ${\cal S}(v)$
and
$ {\cal S}(w)$.

Let the complement be represented by
$\neg {\cal S}(v)=\{w\mid v\cdot w=0\}$
the orthogonal subspace of
${\cal S}(v)$.

Let the ``implication'' relation be represented by
$
{\cal S}(v)
\rightarrow
 {\cal S}(w)
\equiv
(\neg
{\cal S}(v))
\vee
{\cal S}(w)
$.

Then,
(\ref{sf1}), $ \cdots $, (\ref{sf16})$=E^3 $, whereas (\ref{sf17})$=
\neg
 {\cal S}(1,0,0)\neq E^3$.
Therefore,
at least for states lying in the direction $(1,0,0)$ \cite{clifton},
$F$ is not a quantum tautology.

The  set of eleven  rays can be represented by vectors
from the center of a cube to the indicated points \cite{peres}, as drawn
in Fig. \ref{figure-ks}.
 \begin{figure}
 \begin{center}
$\;$
\unitlength=1.00mm
\special{em:linewidth 0.4pt}
\linethickness{0.4pt}
\begin{picture}(126.00,145.00)
\emline{80.00}{10.00}{1}{40.00}{50.00}{2}
\emline{80.00}{10.00}{3}{120.00}{50.00}{4}
\emline{80.00}{10.00}{5}{80.00}{60.00}{6}
\emline{80.00}{60.00}{7}{40.00}{100.00}{8}
\emline{80.00}{60.00}{9}{120.00}{100.00}{10}
\emline{40.00}{50.00}{11}{40.00}{100.00}{12}
\emline{120.00}{50.00}{13}{120.00}{100.00}{14}
\emline{40.00}{100.00}{15}{80.00}{140.00}{16}
\emline{80.00}{140.00}{17}{120.00}{100.00}{18}
\emline{100.00}{120.00}{19}{60.00}{80.00}{20}
\emline{60.00}{80.00}{21}{60.00}{30.00}{22}
\emline{60.00}{120.00}{23}{100.00}{80.00}{24}
\emline{100.00}{80.00}{25}{100.00}{30.00}{26}
\emline{120.00}{75.00}{27}{80.00}{35.00}{28}
\emline{80.00}{35.00}{29}{40.00}{75.00}{30}
\put(70.00,90.00){\circle*{4.00}}
\put(60.00,68.00){\circle*{4.00}}
\put(60.00,80.00){\circle*{4.00}}
\put(40.00,100.00){\circle*{4.00}}
\put(80.00,60.00){\circle*{4.00}}
\put(100.00,80.00){\circle*{4.00}}
\put(120.00,100.00){\circle*{4.00}}
\put(100.00,55.00){\circle*{4.00}}
\put(80.00,35.00){\circle*{4.00}}
\put(80.00,10.00){\circle*{4.00}}
\put(60.00,55.00){\circle*{4.00}}
\put(47.00,100.00){\makebox(0,0)[cc]{$d_2$}}
\put(87.00,10.00){\makebox(0,0)[cc]{$d_3$}}
\put(87.00,60.00){\makebox(0,0)[cc]{$d_4$}}
\put(126.00,100.00){\makebox(0,0)[cc]{$d_1$}}
\put(65.00,55.00){\makebox(0,0)[cc]{$a_1$}}
\put(105.00,55.00){\makebox(0,0)[cc]{$a_2$}}
\put(75.00,90.00){\makebox(0,0)[cc]{$c_1$}}
\put(65.00,68.00){\makebox(0,0)[cc]{$c_2$}}
\put(65.00,80.00){\makebox(0,0)[cc]{$b_2$}}
\put(105.00,80.00){\makebox(0,0)[cc]{$b_1$}}
\put(87.00,35.00){\makebox(0,0)[cc]{$b_3$}}
\put(30.00,100.00){\makebox(0,0)[cc]{$+1$}}
\put(30.00,75.00){\makebox(0,0)[cc]{$0$}}
\put(30.00,50.00){\makebox(0,0)[cc]{$-1$}}
\put(15.00,20.00){\vector(-2,-3){10.00}}
\put(15.00,20.00){\vector(0,1){20.00}}
\put(15.00,20.00){\vector(1,0){20.00}}
\put(8.00,3.00){\makebox(0,0)[cc]{$x$}}
\put(32.00,25.00){\makebox(0,0)[cc]{$y$}}
\put(19.00,36.00){\makebox(0,0)[cc]{$z$}}
\put(35.00,45.00){\makebox(0,0)[cc]{$-1$}}
\put(55.00,25.00){\makebox(0,0)[cc]{$0$}}
\put(75.00,5.00){\makebox(0,0)[cc]{$+1$}}
\put(35.00,105.00){\makebox(0,0)[cc]{$+1$}}
\put(55.00,125.00){\makebox(0,0)[cc]{$0$}}
\put(75.00,145.00){\makebox(0,0)[cc]{$-1$}}
\end{picture}
 \end{center}
 \caption{The eleven rays in the proof of the Kochen-Specker theorem
based on the construction of Sch\"utte are
obtained by connecting the center of the cube to the black dots on its
faces and edges.
\label{figure-ks}}
\end{figure}

\section{Computational complementarity}
\label{appendix3}
 Consider the
 transition and output tables and the graph of a (3,3,2)-Mealy type
automaton drawn in Fig.
 \ref{t-mealy-a}.
Let us further assume
that, given only one automaton copy,
the initial state is unknown to an observer.
The goal of the observer is to find this unknown initial state of this
automaton by performing input-output experiments on this single
automaton.
\begin{figure}
\begin{tabular}{||c||ccc||}
 \hline\hline
  & state  $1$ & state  $2$ & state  $3$  \\
 \hline\hline
 input function $\delta_1$ &$1$&$1$&$1$\\
 input function $\delta_2$ &$2$&$2$&$2$\\
 input function $\delta_3$ &$3$&$3$&$3$\\
 \hline\hline
 output function $o_1$ &1&0&0\\
 output function $o_2$ &0&1&0\\
 output function $o_3$ &0&0&1\\
 \hline\hline
\end{tabular}
$\;$\\
$\;$\\
  \centering
\includegraphics[width=90mm]{ch-mea.ps}
\caption{Transition and output tables and figure of a
 (3,2,2)-automaton of the Mealy type.
 \label{t-mealy-a}}
\end{figure}
%\newpage
 Input of 1, 2 or 3 steers the automaton into the states 1, 2 or
3, respectively. At
 the same time, the output of the automaton is 1 only if the guess is a
 ``hit,'' i.e., only if the automaton was in that state. Otherwise the
 output is 0.   Hence, after the measurement, the automaton is in a
 definite state, but if the guess is no ``hit,'' the information about
 the initial automaton state is lost.
 Therefore, the experimenter has to decide before the actual measurement
which one of the following
 hypotheses should be tested (in short-hand notation, ``$\{
1\}$'' stands for ``{\tt the automaton is in state} $1$'' {\it etc.}):
$
\{ 1 \}=
\neg
\{ 2,3 \},
\{ 2 \}=
\neg
\{ 1,3 \},
\{ 3 \}=
\neg
\{ 1,2 \}
$.
Measurement of either one of these three hypotheses (or their
complement) makes impossible measurement of the other two hypotheses.


No input, i.e., the empty input string $\emptyset$, identifies all three
internal automaton states. This corresponds to the trivial information
that the automaton is in {\em some} internal state.
Input of the symbol 1 (and all sequences of symbols starting with 1)
distinguishes between the hypothesis $\{1\}$ (output ``1'') and the
hypothesis
$\{2,3\}$ (output ``0'').
Input of the symbol 2 (and all sequences of symbols starting with 1)
distinguishes between the hypothesis $\{2\}$ (output ``1'') and the
hypothesis
$\{1,3\}$ (output ``0'').
Input of the symbol 3 (and all sequences of symbols starting with 1)
distinguishes between the hypothesis $\{3\}$ (output ``1'') and the
hypothesis
$\{1,2\}$ (output ``0'').
 The intrinsic
  propositional calculus  is thus
 defined by the partitions \cite{svozil-93}
 \begin{eqnarray}
 v(\emptyset )&=&\{\{1,2,3\}\} \quad ,\\
 v(1 )&=&\{ \{1\} , \{2,3\}  \}\quad ,\\
 v(2 )&=&\{ \{2\} , \{1,3\}  \}\quad ,\\
 v(3 )&=&\{ \{3\} , \{1,2\}  \}\quad .
 \end{eqnarray}
 It can be represented by the lattice drawn in Fig.
 \ref{f-mealy-a-i-p-c}.
 \begin{figure}
 \begin{center}
$\;$
\scalebox{0.8}{
\hbox{
\unitlength=1.50mm
\special{em:linewidth 0.4pt}
\linethickness{0.4pt}
\begin{picture}(113.28,55.00)
\put(62.33,10.00){\circle*{1.89}}
\put(62.33,50.00){\circle*{1.89}}
\put(52.33,30.00){\circle*{1.89}}
\put(32.66,30.00){\circle*{1.89}}
\put(12.66,30.00){\circle*{1.89}}
\put(112.33,30.00){\circle*{1.89}}
\put(92.66,30.00){\circle*{1.89}}
\put(72.66,30.00){\circle*{1.89}}
\emline{62.33}{50.00}{1}{112.33}{30.00}{2}
\emline{112.33}{30.00}{3}{62.33}{10.00}{4}
\emline{62.33}{50.00}{5}{92.33}{30.00}{6}
\emline{92.33}{30.00}{7}{62.33}{10.00}{8}
\emline{62.33}{10.00}{9}{72.33}{29.67}{10}
\emline{72.33}{29.67}{11}{62.33}{50.00}{12}
\emline{62.33}{50.00}{13}{52.33}{30.33}{14}
\emline{52.66}{29.67}{15}{62.33}{10.00}{16}
\emline{62.33}{10.00}{17}{32.66}{29.67}{18}
\emline{32.66}{29.67}{19}{62.33}{50.00}{20}
\emline{62.33}{50.00}{21}{12.33}{30.00}{22}
\emline{12.33}{30.00}{23}{62.33}{10.00}{24}
\put(62.33,5.00){\makebox(0,0)[cc]{${\bf 0}=\neg (1\vee 2\vee 3)$}}
\put(62.33,55.00){\makebox(0,0)[cc]{${\bf 1}=1\vee 2\vee 3$}}
\put(2.00,30.00){\makebox(0,0)[cc]{$1=\neg (2\vee 3)$}}
\put(23.00,30.00){\makebox(0,0)[cc]{$2=\neg (1\vee 3)$}}
\put(44.33,30.00){\makebox(0,0)[cc]{$3=\neg (1\vee 2)$}}
\put(62.66,30.00){\makebox(0,0)[cc]{$1\vee 2=\neg 3$}}
\put(82.66,30.00){\makebox(0,0)[cc]{$1\vee 3=\neg 2$}}
\put(102.00,30.00){\makebox(0,0)[cc]{$2\vee 3=\neg 1$}}
\end{picture}
}}
 \end{center}
 \caption{Lattice $MO3$ of intrinsic propositional calculus
 of  a  (3,2,2)-automaton
 of the Mealy type.
\label{f-mealy-a-i-p-c}}
\end{figure}
%\newpage
 This lattice is of the ``Chinese latern'' $MO3$ form.
 It is non--distributive,
 but modular.

 The obtained intrinsic propositional calculus
 in many ways  resembles the lattice obtained from photon polarization
 experiments
 or from other incompatible quantum measurements.
Consider an experiment measuring photon polarization.
 Three propositions  of the
form
\begin{center}
 ``{\tt the  photon  has polarization} $p_{\phi_1}$,''\\
 ``{\tt the  photon  has polarization} $p_{\phi_2}$,''  \\
 ``{\tt the  photon  has polarization} $p_{\phi_3}$''
\end{center}
 cannot be measured simultaneously for the angles $\phi_1 \neq
\phi_2\neq \phi_3 \neq \phi_1
 ({\rm mod} \pi )$. An irreversible measurement of one direction of
polarization would result in a state preparation, making impossible
measurement of the other directions of polarization, and
resulting in a propositional calculus of the
 ``Chinese latern'' form $MO3$.

\section{Simple proof of the recursive unsolvability of the halting
problem}
\label{appendix4}
Assume that there is an algorithmic way to forsee a particular aspect of
the future
of an arbitrary computation. Namely, whether or not the computation
will terminate.
As conceived by Turing \cite{turing},
this assumption yields to a contradiction; therefore it cannot be valid.
The proof follows Cantor's diagonalization argument, which was
used analogously by G\"odel to prove the incompleteness of arithmetic.

Consider an arbitrary algorithm $B(x)$ whose input is a string of
symbols
$x$.
 Assume that there exists a ``halting algorithm'' ${\tt
HALT}$
 which  is
 able to decide whether $B$ terminates on $x$ or not.

 Using  ${\tt HALT}(B(x))$ we shall construct another
 deterministic computing agent
$A$, which
 has as input any effective program $B$ and which proceeds as follows:
 Upon reading the program $B$ as input, $A$ makes a copy of it.
 This  can be readily achieved, since
 the program $B$ is presented to $A$  in some
 encoded form $\# (B)$, i.e., as a string of symbols. In the next
 step, the agent uses the
 code $\# (B)$ as input string for $B$ itself; i.e., $A$  forms
 $B(\#(B))$, henceforth denoted by $B(B)$. The agent now hands
 $B(B)$ over to its
 subroutine ${\tt HALT}$.
 Then, $A$ proceeds as follows:
  if ${\tt HALT}(B(B))$ decides that $B(B)$
 halts, then the agent
 $A$ does not halt;
this can for instance be realised by an infinite {\tt
 DO}-loop;
  if ${\tt HALT}(B(B))$ decides that $B(B)$
 does {\em not} halt, then
 $A$ halts.

 We shall now confront the agent $A$ with a paradoxical task by
 choosing $A$'s own code as input for itself.
---
Notice that $B$ is arbitrary and has not been specified yet.
The deterministic agent $A$ is representable by an algorithm with code
$\# (A)$. Therefore, we are free to substitute $A$ for $B$.

Assume that classically $A$ is restricted to classical bits of
information.
Then, whenever
 $A(A)$ halts,  ${\tt HALT}(A(A))$  forces
 $A(A)$ not to halt.
Conversely, whenever $A(A)$ does not halt, then ${\tt HALT}(A(A))$
 steers $A(A)$
 into the halting mode. In both cases one arrives at a
complete contradiction.
 In the classical computational base,
this contradiction can only be consistently avoided by
 assuming
 the nonexistence of $A$ and, since the only nontrivial feature of $A$
 is the use of the peculiar halting algorithm
 ${\tt HALT}$, the impossibility of any such
 halting algorithm.

\section{Interface modeling}
\label{appendix5}
Let us explicitly construct the decription of an asymmetric
interface between two universes $S_1$ and $S_2$ \cite{svozil-64}.
Assume that an intrinsic
(or operational or \mbox{endo-)} parameter
description $P(S)=\lbrace S;p_1,...,p_n,...\rbrace$   with parameters
$p_i$
contains parameters which could at least in principle
be measured by devices and processes available in the universe $S$.


Assume
 an interaction $I$
acting in both universes $S_1$ and $S_2$. Let us use this
interaction for measurements. In this way we get two
associated intrinsic parameter descriptions $P(S_1,I)$ and $P(S_2,I)$.
We shall define the universe $S_2$  {\em approximately closed}  with
respect to $S_1$ and to the interaction $I$ if $S_2$ responds only
``slightly'' to changes in $S_1$. Formally, this situation can be
written as
${\delta P(S_2,I)\over \delta p_i}\simeq 0\; \;
\forall p_i\in P(S_1,I)$ or just
${\delta P(S_2,I)\over \delta P(S_1,I)}\simeq 0$.
Using the language of cybernetics, this is identical to say that
a system $S_2$ is approximately autonomous with respect to
$S_1$ if the effect of its output affects its input only slightly,
such that no feedback loop via $S_1$ occurs \cite{varela}.


Assume  again two universes $S_1$ and $S_2$, and two interactions
$I_1$ and $I_2$. Assume further that $S_1$ and $S_2$ are
approximately closed with respect to one interaction, say $I_1$:
${\delta P(S_1,I_1)\over \delta P(S_2,I_1)}\simeq
{\delta P(S_2,I_1)\over \delta P(S_1,I_1)}\simeq 0$.
 We shall spoil the symmetry now by requiring that
one system, say $S_1$, is sensitive to interactions $I_2$,
whereas $S_2$ is not:
${\delta P(S_1,I_2)\over \delta P(S_2,I_2)}
\neq 0\wedge
{\delta P(S_2,I_2)\over \delta P(S_1,I_2)}\simeq 0$.
Hence, effectively one almost closed
system $S_1$ is a close realization of Archimedean
point, with the system $S_2$ and the interaction $I_1$ to
be described. $I_2$ serves merely as a reference interaction. Since
observations in $S_1$ will not affect
$S_2$ too much, an operational parameter description
$P(S_2,I_2)$ will be called {\em quasiextrinsic.}
Parameters in $P(S_2,I_2)$, which cannot be measured
by $I_1$ in $S_2$ are external, hidden, parameters in $S_2$.



In this context, the extrinsic parameter description might
be defined via a limit: a parameter description
$P(S_1,I_2)$ from $S_2$ is called {\em extrinsic} if $S_1$ and $S_2$
are totally closed with respect to both interactions
$I_1$ and $I_2$.
Clearly, this is impossible to realize, since there
cannot be any exchange between universes without altering
the states of both.


In what follows I shall give an example of such a configuration:
assume a pool filled with water, which will serve as system $S_1$.
Let us assume further an optical instrument recording
electromagnetic radiation as part of our system $S_2$, and the
interactions $I_1$ and $I_2$, being identified  with water wave
interactions and electromagnetic interaction respectively.
Since light does not affect water wave dynamics appreciably,
but changes the state of the optical instrument, a realization
of the described model is obtained, with the
optical instrument yielding a quasiextrinsic view of the pool.


Let a universe be represented by the symbols ``$()$''.
Let a  cut or interface be represented by some double line symbol such
as ``$][$'' characterizing the two universes it connects.

Then the process of distinction creates
a cut within one universe.
It creates two new
distinct universes; the interface being
along the cut; i.e., $
()\rightarrow (][)$.
In the process of condensation,
two formerly distinct universes communicate via
the interface, which is again symbolized as (symmetric)
cut; i.e., $
() ()\rightarrow (][) $.
If the  interface is not symmetric such as in the above
(quasi-) extrinsic setup, then
the process of condensation,
two formerly distinct universes communicate via
the interface, which is now symbolized as an asymmetric
cut; i.e., $() ()\rightarrow ()[) $ or $() ()\rightarrow (]() $.
Also for asymmetric interfaces, the process of distinction is symbolized
by an asymmetric cut; i.e.,
$()\rightarrow ()[)$ or
$()\rightarrow (]()$.


\begin{thebibliography}{99}

%\bibitem{vr-drugs}
%K. Gerbel and P. Weibel,
%{\sl Die Welt von Innen -- Endo \& Nano}
%{\sl The World from Within -- ENDO \& NANO}, ed. by
%K. Gerbel and P. Weibel (PVS Verleger, Linz, Austria, 1992).

\bibitem{krenn-private}
G. Krenn, {\it private communication.}

 \bibitem{toffoli}
 T. Toffoli,
{\sl The role of the observer in uniform systems}, in {\sl Applied
General Systems Research}, ed. by G. Klir (Plenum Press, New York,
London, 1978).

 \bibitem{r�ssler}
 O. E. R\"ossler, {\sl Endophysics}, in {\sl Real Brains, Artificial
 Minds}, ed. by J. L. Casti and A. Karlquist (North-Holland, New
 York, 1987), p. 25;
 {\sl Endophysics, Die Welt des inneren Beobachters},
 ed. by P. Weibel (Merwe Verlag, Berlin, 1992).

\bibitem{svozil-93}
K. Svozil,  {\sl Randomness and Undecidability in Physics}
(World Scientific, Singapore, 1993).

\bibitem{r�ssler-private}
 O. E. R\"ossler, {\it private communication.}

\bibitem{specker}
E. P. Specker, {\sl Dialectica} {\bf 14}, 175 (1960);
S. Kochen and E. P. Specker,
{\sl Journal of Mathematics and Mechanics} {\bf 17}, 59 (1967);
reprinted in \cite{specker-theorem}.

\bibitem{redhead}
M. Redhead,
{\sl Incompleteness, Nonlocality and Realism}
(Clarendon Press, Oxford, 1987).

\bibitem{peres}
A. Peres,
{\sl Quantum Theory: Concepts \& Methods}
(Kluwer Academic Publishers, Dordrecht, 1993).

\bibitem{mermin}
N. D. Mermin,
{\sl Rev. Mod. Phys.} {\bf 65}, 803 (1993).

\bibitem{no-cloning-theorem}
N. Herbert, {\sl Foundation of Physics} {\bf 12}, 1171 (1982);
W. K. Wooters and W. H. Zurek,
{\sl Nature} {\bf 299}, 802 (1982);
P. W. Milonni and M. L. Hardies,
{\sl Phys. Lett.} {\bf 92A}, 321 (1982);
L. Mandel,
{\sl Nature} {\bf 304}, 188 (1983);
 R. J. Glauber, {\sl Amplifyers, Attenuators and the Quantum Theory of
 Measurement}, in {\sl Frontiers in Quantum Optics}, ed. by E. R. Pikes
 and S. Sarkar (Adam Hilger, Bristol 1986).
reprinted in \cite{specker-theorem}.

\bibitem{specker2}
S. Kochen and E. P. Specker,
{\sl Logical Structures arising in quantum theory}, in
{\sl Symposium on the Theory of Models, Proceedings of the
1963 International Symposium at Berkeley}
(North Holland, Amsterdam, 1965), p. 177-189;
reprinted in \cite{specker-theorem}.

\bibitem{specker3}
S. Kochen and E. P. Specker,
{\sl The calculus of partial propositional functions,} in
{\sl Proceedings of the 1964 International Congress for Logic,
Methodology and Philosophy of Science, Jerusalem} (North Holland,
Amsterdam, 1965), p. 45-57;
reprinted in \cite{specker-theorem}.

 \bibitem{kreisel}
G. Kreisel, {\sl Synthese} {\bf 29}, 11 (1974).

 \bibitem{wigner}
 E. P. Wigner,
 {\sl ``The unreasonable effectiveness of mathematics in the natural
 sciences''}, Richard Courant Lecture delivered at New York University,
 May 11, 1959 and published in {\it
 Communications on Pure and Applied Mathematics} {\bf 13}, 1 (1960).

\bibitem{frank}
Ph. Frank, {\sl Das Kausalgesetz und seine Grenzen}
(Springer, Vienna 1932).


 \bibitem{e-f-moore}
 E. F. Moore, {\sl Gedanken-Experiments on Sequential Machines}, in
 {\sl Automata Studies}, ed. by C. E. Shannon \& J. McCarthy (Princeton
 University Press, Princeton, 1956).

 \bibitem{hopcroft}
 J. E. Hopcroft and J. D. Ullman,
 {\sl Introduction to Automata Theory, Languages, and Computation}
 (Addison-Wesley, Reading, MA, 1979).

 \bibitem{brauer}
 W. Brauer, {\sl Automatentheorie} (Teubner, Stuttgart, 1984).

 \bibitem{finkelstein-83}
 D. Finkelstein and S. R. Finkelstein,
 {\sl International Journal of Theoretical Physics} {\bf 22}, 753
 (1983).

\bibitem{zapatrin}
A. A. Grib and
R. R. Zapatrin,
{\sl International Journal of Theoretical Physics} {\bf 29}, 113 (1990);
{\it ibid} {\bf 31}, 1669
(1992).

\bibitem{schaller-svozil}
 M. Schaller and K. Svozil, {\sl Il Nuovo Cimento} {\bf 109 B}, 167
(1994).

 \bibitem{v-neumann-66}
 J. von Neumann,
 {\sl Theory of Self-Reproducing Automata}, ed. by A. W. Burks
 (University of Illinois Press, Urbana, 1966).

\bibitem{bible}
 The {\sl Bible} contains a passage, which
 refers to Epimenides, a Crete living in the capital city of
 Cnossus:
 {\it ``One of themselves, a prophet of their own, said, `Cretans are
 always liars, evil beasts, lazy gluttons.'  ''} --- St. Paul,
 Epistle to Titus I (12-13).
 For more details, see
 A. R. Anderson, {\sl St. Paul's epistle to Titus}, in {\sl The Paradox
 of the Liar}, ed. by R. L. Martin (Yale University Press, New Haven,
 1970).

\bibitem{svozil-paradox}
K. Svozil,
{\sl The consistent use of paradoxa}, TU Vienna preprint, May 1994.
This observation may be the starting point for the application
of quantum computers in inconsistent databases.

 \bibitem{popper}
 K. R. Popper, {\sl The British Journal for the Philosophy of Science}
 {\bf 1}, 117, 173 (1950).


\bibitem{chaitin}
G. J. Chaitin, {\sl Information, Randomness and Incompleteness, Second
edition}
(World Scientific, Singapore, 1987, 1990);
{\sl Algorithmic Information Theory}
(Cambridge University Press, Cambridge, 1987);
{\sl Information-Theoretic Incompleteness}
(World Scientific, Singapore, 1992).

\bibitem{calude}
 C. Calude,
{\sl Information and Randomness --- An Algorithmic Perspective}
(Springer, Berlin,
1994).


\bibitem{weyl}
H. Weyl,
{\sl Philosophy of Mathematics and Natural Science}
(Princeton University Press, Princeton, 1949).

\bibitem{gruenbaum}
A. Gr\"unbaum,
{\sl Modern Science and Zeno's paradoxes, Second edition}
(Allen and Unwin, London, 1968);
{\sl Philosophical Problems of Space of Time, Second, enlarged edition}
(D. Reidel, Dordrecht, 1973).

\bibitem{specker-theorem}
E. Specker, {\sl Selecta} (Birkh\"auser Verlag, Basel, 1990).

\bibitem{bishop}
 E. Bishop and D. S. Bridges, {\sl Constructive Analysis} (Springer,
 Berlin, 1985).


\bibitem{bridges}
D. Bridges and F. Richman, {\sl Varieties of Constructive Mathematics}
(Cambridge University Press, Cambridge, 1987).

\bibitem{spencer-brown}
Spencer Brown's ``laws of form'' do not alwasy apply, since
they result in classical Boolean logic, which is only a subset of
automaton as well as quantum logic.

 \bibitem{putnam}
H. Putnam,
{\sl Reason, Truth and History}
(Cambridge University Press,
Cambridge, 1981).

\bibitem{eccles}
J. C. Eccles,
{\sl The Mind-Brain Problem Revisited: The Microsite Hypothesis}, in
{\sl The Principles of Design and Operation of the Brain},
ed. by J. C. Eccles and  O. Creutzfeldt (Springer, Berlin, 1990), p.
549.

 \bibitem{epr}
 A. Einstein, B. Podolsky and N. Rosen, {\sl Phys. Rev.} {\bf 47}, 777
 (1935).

\bibitem{sch�tte-letter}
K. Sch\"utte, {\it letter to Professor Ernst P. Specker, dated from
April 22$^{{\rm nd}}$, 1965}; first published in
Erna Clavadetscher-Seeberger, {\it Eine partielle Pr\"adikatenlogik}
(Dissertation, ETH-Z\"urich, Z\"urich, 1983).

\bibitem{clifton}
R. Clifton, {\sl private communication.}


 \bibitem{turing}
 A. M. Turing, {\sl Proc. London Math. Soc.} {\bf (2), 42}, 230
 (1936-7).

\bibitem{svozil-64}
K. Svozil,
 {\sl Il Nuovo Cimento} {\bf 96B}, 127 (1986).

 \bibitem{varela}
 F. Varela, {\sl The Principles of Biological Autonomy} (North
Holland, New York, 1980).


\end{thebibliography}



%\newpage

%\begin{flushright}
%  HAMLET. To be, or not to be- that is the question:          \\
%    Whether 'tis nobler in the mind to suffer                 \\
%    The slings and arrows of outrageous fortune               \\
%    Or to take arms against a sea of troubles,                \\
%    And by opposing end them. To die- to sleep-               \\
%    No more; and by a sleep to say we end                     \\
%    The heartache, and the thousand natural shocks            \\
%    That flesh is heir to. 'Tis a consummation                \\
%    Devoutly to be wish'd. To die- to sleep.                  \\
%    To sleep- perchance to dream: ay, there's the rub!        \\
%    For in that sleep of death what dreams may come           \\
%    When we have shuffled off this mortal coil,               \\
%    Must give us pause. There's the respect                   \\
%    That makes calamity of so long life.                      \\
%    For who would bear the whips and scorns of time,          \\
%    Th' oppressor's wrong, the proud man's contumely,         \\
%    The pangs of despis'd love, the law's delay,              \\
%    The insolence of office, and the spurns                   \\
%    That patient merit of th' unworthy takes,                 \\
%    When he himself might his quietus make                    \\
%    With a bare bodkin? Who would these fardels bear,         \\
%    To grunt and sweat under a weary life,                    \\
%    But that the dread of something after death-              \\
%    The undiscover'd country, from whose bourn                \\
%    No traveller returns- puzzles the will,                   \\
%    And makes us rather bear those ills we have               \\
%    Than fly to others that we know not of?                   \\
%    Thus conscience does make cowards of us all,              \\
%    And thus the native hue of resolution                     \\
%    Is sicklied o'er with the pale cast of thought,           \\
%    And enterprises of great pith and moment                  \\
%    With this regard their currents turn awry                 \\
%    And lose the name of action.- Soft you now!               \\
%    The fair Ophelia!- Nymph, in thy orisons                  \\
%    Be all my sins rememb'red.
%\end{flushright}
%
%\newpage
%\section*{Curriculum vit\ae$\,$ Karl Svozil}
%Karl Svozil was born in Vienna on Dec. 12, 1956.
%After his studies at the universities of Vienna and Heidelberg
%from 1975--1981
%he has been a visiting scholar at the University of California at
%Berkeley and at Lawrence Berkeley Laboratory as well as
%at Moscow State University and at the
%Lebedev Institute.
%From
%{1984 to mid--1990} he worked for the
%Austrian Ministry
% for Science \& Research.
%In {1988} he has been granted the
%degree of a {\it Dozent}
%at the Technische Universit\"at Wien, and
% since
%{mid--1990}, holds a permanent position there.
%He is
%married and father of two-year old Alexander.
%
\end{document}
