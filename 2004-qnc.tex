\documentclass[prl,twocolumn,showpacs,showkeys,amsfonts]{revtex4}
\usepackage{graphicx}
\begin{document}


\title{Contextuality is unphysical}

\author{Karl Svozil}
\email{svozil@tuwien.ac.at}
\homepage{http://tph.tuwien.ac.at/~svozil}
\affiliation{Institut f\"ur Theoretische Physik, University of Technology Vienna,
Wiedner Hauptstra\ss e 8-10/136, A-1040 Vienna, Austria}


\begin{abstract}
Any testable configuration of observables is noncontextual.
In those cases where contextuality is assumed
to rescue value definiteness at least for comeasurable observables,
it cannot be experimentally tested.
\end{abstract}

\pacs{03.67.Mn,03.67.-a}
\keywords{Entanglement production, characterization, and manipulation;Quantum information,Quantum contextuality}


\maketitle



Contextuality \cite{bohr-1949,bell-66,hey-red,redhead} contends that,
in Bell's words \cite[Sect. 5]{bell-66}, the {\em ``$\ldots$
result of an observation may reasonably depend
not only on the state of the system  $\ldots$
but also on the complete disposition  of the apparatus.''}
This concept has been developed in reaction to the theorems by
Gleason, Bell, and Kochen and Specker.
The quantum probabilities are based on Born


NN


In what follows, the operational content of the concept of contextuality will be studied
in detail.
We shall first analyze specific configurations which can be operationalized,
and then turn to the general problem to empirically test contextuality
for configurations of (counterfactual) observables
which do not allow a classical interpretation.



%PART 1

Proofs of the Kochen-Specker theorem consider
systems of orthogonal tripods (bases) in real Hilbert spaces of dimension three
interlinked at one common leg.
The simplest interlinked system consists of just two orthogonal tripods, whose common leg
is located along the $x_3$-axis;
the other orthogonal legs both lie in the $x_1$-$x_2$--plane,
such as the tripods spanned by the two bases
$
\{
(1,0,0)^T,
(0,1,0)^T,
(0,0,1)^T
\}
$
and
$
\{
(\cos \varphi , 0,0)^T,
(0,\sin \varphi , 0)^T,
(0,0,1)^T
\}
$ (the superscript ``$T$'' indicates transposition).
This configuration is depicted in Fig.~\ref{2004-qnc-f1}a), together with
its representation in a Greechie (orthogonality) diagram \cite{greechie:71} in Fig.~\ref{2004-qnc-f1}b),
which represents orthogonal tripods by points symbolizing individual legs that are connected by smooth curves.
\begin{figure}
\begin{tabular}{ccccc}
%TexCad Options
%\grade{\off}
%\emlines{\off}
%\beziermacro{\on}
%\reduce{\on}
%\snapping{\off}
%\quality{2.00}
%\graddiff{0.01}
%\snapasp{1}
%\zoom{1.00}
\unitlength 0.70mm
\linethickness{0.4pt}
\begin{picture}(40.00,49.67)
%\put(60.33,15.00){\circle{0.00}}
%\put(60.33,15.00){\circle{2.00}}
%\put(45.33,10.00){\circle{2.00}}
%\put(30.33,5.00){\circle{2.00}}
%\put(15.33,10.00){\circle{2.00}}
%\put(0.33,16.00){\circle{0.00}}
%\put(0.33,15.00){\circle{2.00}}
\put(15.00,45.00){\line(0,-1){30.00}}
\put(15.00,15.00){\line(-1,-1){15.00}}
\put(15.00,15.00){\line(1,0){25.00}}
\put(15.00,15.00){\line(3,-4){11.00}}
\put(15.00,15.00){\line(5,3){16.67}}
\put(3.33,-1.67){\makebox(0,0)[cc]{$x_1$}}
\put(30.00,0.00){\makebox(0,0)[cc]{$x_1'$}}
\put(40.00,11.33){\makebox(0,0)[cc]{$x_2$}}
\put(35.00,23.67){\makebox(0,0)[cc]{$x_2'$}}
\put(19.33,49.67){\makebox(0,0)[cc]{$x_3=x_3'$}}
%\bezvec{60}(7.67,6.67)(14.33,2.67)(20.00,6.67)
\put(20.00,6.67){\vector(2,1){0.2}}
\bezier{60}(7.67,6.67)(14.33,2.67)(20.00,6.67)
%\end
%\bezvec{36}(29.67,16.00)(32.33,19.67)(30.00,23.00)
\put(30.00,23.00){\vector(-1,2){0.2}}
\bezier{36}(29.67,16.00)(32.33,19.67)(30.00,23.00)
%\end
\put(13.33,1.00){\makebox(0,0)[cc]{$\varphi$}}
\put(28.00,18.67){\makebox(0,0)[cc]{$\varphi$}}
\end{picture}
&&
%TexCad Options
%\grade{\off}
%\emlines{\off}
%\beziermacro{\on}
%\reduce{\on}
%\snapping{\off}
%\quality{2.00}
%\graddiff{0.01}
%\snapasp{1}
%\zoom{1.00}
\unitlength 0.80mm
\linethickness{0.4pt}
\begin{picture}(61.33,36.00)
%\emline(0.33,35.00)(30.33,25.00)
\multiput(0.33,35.00)(0.36,-0.12){84}{\line(1,0){0.36}}
%\end
%\emline(30.33,25.00)(60.33,35.00)
\multiput(30.33,25.00)(0.36,0.12){84}{\line(1,0){0.36}}
%\end
%\put(60.33,15.00){\circle{0.00}}
%\put(60.33,15.00){\circle{2.00}}
%\put(45.33,10.00){\circle{2.00}}
%\put(30.33,5.00){\circle{2.00}}
%\put(15.33,10.00){\circle{2.00}}
%\put(0.33,16.00){\circle{0.00}}
%\put(0.33,15.00){\circle{2.00}}
\put(30.33,25.00){\circle{2.00}}
\put(45.33,30.00){\circle{2.00}}
\put(60.33,35.00){\circle{2.00}}
\put(0.33,35.00){\circle{2.00}}
\put(15.33,30.00){\circle{2.00}}
\put(60.33,31.00){\makebox(0,0)[cc]{$x_1'$}}
\put(45.33,26.00){\makebox(0,0)[cc]{$x_2'$}}
\put(30.33,30.00){\makebox(0,0)[cc]{$x_3=x_3'$}}
\put(15.33,26.00){\makebox(0,0)[cc]{$x_2$}}
\put(0.33,31.00){\makebox(0,0)[cc]{$x_1$}}
\bezier{24}(0.00,20.00)(0.00,17.33)(3.00,17.33)
\bezier{28}(3.00,17.33)(10.00,17.00)(10.00,17.00)
\bezier{32}(10.00,17.00)(15.00,16.00)(15.00,13.33)
\bezier{24}(30.00,20.00)(30.00,17.33)(27.00,17.33)
\bezier{28}(27.00,17.33)(20.00,17.00)(20.00,17.00)
\bezier{32}(20.00,17.00)(15.00,16.00)(15.00,13.33)
\put(15.00,10.33){\makebox(0,0)[cc]{context}}
\put(15.00,5.33){\makebox(0,0)[cc]{$\{x_1,x_2,x_3\}$}}
\bezier{24}(60.00,20.00)(60.00,17.33)(57.00,17.33)
\bezier{28}(57.00,17.33)(50.00,17.00)(50.00,17.00)
\bezier{32}(50.00,17.00)(45.00,16.00)(45.00,13.33)
\bezier{24}(30.00,20.00)(30.00,17.33)(33.00,17.33)
\bezier{28}(33.00,17.33)(40.00,17.00)(40.00,17.00)
\bezier{32}(40.00,17.00)(45.00,16.00)(45.00,13.33)
\put(45.00,10.33){\makebox(0,0)[cc]{context}}
\put(45.00,5.33){\makebox(0,0)[cc]{$\{x_1',x_2',x_3'\}$}}
\end{picture}
&&
%TexCad Options
%\grade{\off}
%\emlines{\off}
%\beziermacro{\on}
%\reduce{\on}
%\snapping{\off}
%\quality{2.00}
%\graddiff{0.01}
%\snapasp{1}
%\zoom{1.00}
\unitlength 0.80mm
\linethickness{0.4pt}
\begin{picture}(91.34,36.00)
%\emline(0.33,35.00)(30.33,25.00)
\multiput(0.33,35.00)(0.36,-0.12){84}{\line(1,0){0.36}}
%\end
%\put(60.33,15.00){\circle{0.00}}
%\put(60.33,15.00){\circle{2.00}}
%\put(45.33,10.00){\circle{2.00}}
%\put(30.33,5.00){\circle{2.00}}
%\put(15.33,10.00){\circle{2.00}}
%\put(0.33,16.00){\circle{0.00}}
%\put(0.33,15.00){\circle{2.00}}
\put(30.33,25.00){\circle{2.00}}
\put(45.33,25.00){\circle{2.00}}
\put(0.33,35.00){\circle{2.00}}
\put(15.33,30.00){\circle{2.00}}
\put(45.33,21.00){\makebox(0,0)[cc]{$x_2'$}}
\put(30.33,30.00){\makebox(0,0)[cc]{$x_3=x_3'$}}
\put(15.33,26.00){\makebox(0,0)[cc]{$x_2$}}
\put(0.33,31.00){\makebox(0,0)[cc]{$x_1$}}
\bezier{24}(0.00,20.00)(0.00,17.33)(3.00,17.33)
\bezier{28}(3.00,17.33)(10.00,17.00)(10.00,17.00)
\bezier{32}(10.00,17.00)(15.00,16.00)(15.00,13.33)
\bezier{24}(30.00,20.00)(30.00,17.33)(27.00,17.33)
\bezier{28}(27.00,17.33)(20.00,17.00)(20.00,17.00)
\bezier{32}(20.00,17.00)(15.00,16.00)(15.00,13.33)
\put(15.00,10.33){\makebox(0,0)[cc]{context}}
\put(15.00,5.33){\makebox(0,0)[cc]{$\{x_1,x_2,x_3\}$}}
\bezier{24}(60.67,20.00)(60.67,17.33)(57.67,17.33)
%\bezier{24}(60.00,20.00)(60.00,17.33)(57.00,17.33)
\bezier{28}(57.00,17.33)(50.00,17.00)(50.00,17.00)
\bezier{32}(50.00,17.00)(45.00,16.00)(45.00,13.33)
\bezier{24}(30.00,20.00)(30.00,17.33)(33.00,17.33)
\bezier{28}(33.00,17.33)(40.00,17.00)(40.00,17.00)
\bezier{32}(40.00,17.00)(45.00,16.00)(45.00,13.33)
\put(45.00,10.33){\makebox(0,0)[cc]{context}}
\put(45.00,5.33){\makebox(0,0)[cc]{$\{x_1',x_2',x_3'\}$}}
\put(30.33,25.00){\line(1,0){30.00}}
%\emline(90.34,35.00)(60.34,25.00)
\multiput(90.34,35.00)(-0.36,-0.12){84}{\line(-1,0){0.36}}
%\end
\put(60.34,25.00){\circle{2.00}}
\put(90.34,35.00){\circle{2.00}}
\put(75.34,30.00){\circle{2.00}}
\put(60.34,30.00){\makebox(0,0)[cc]{$x_1'=x_1''$}}
\put(75.34,26.00){\makebox(0,0)[cc]{$x_3''$}}
\put(90.34,31.00){\makebox(0,0)[cc]{$x_2''$}}
\bezier{24}(90.67,20.00)(90.67,17.33)(87.67,17.33)
\bezier{28}(87.67,17.33)(80.67,17.00)(80.67,17.00)
\bezier{32}(80.67,17.00)(75.67,16.00)(75.67,13.33)
\bezier{24}(60.67,20.00)(60.67,17.33)(63.67,17.33)
\bezier{28}(63.67,17.33)(70.67,17.00)(70.67,17.00)
\bezier{32}(70.67,17.00)(75.67,16.00)(75.67,13.33)
\put(75.67,10.33){\makebox(0,0)[cc]{context}}
\put(75.67,5.33){\makebox(0,0)[cc]{$\{x_1'',x_2'',x_3''\}$}}
\end{picture}
\\
a)&\qquad \qquad   &b)&  \qquad \qquad &c)\\
\end{tabular}
\begin{center}
\end{center}
\caption{
a) Two tripods with a common leg spanning two measurement contexts;
b) Greechie (orthogonality) diagram: points stand for individual basis vectors, and
orthogonal tripods are drawn as smooth curves;
c) Greechie diagram of three tripods interconnected at two legs.
\label{2004-qnc-f1}}
\end{figure}
Thereby, every context can formally be identified with a single {\em maximal} nondegenerate self-adjoint operator,
of which the single projectors are functions
\cite{v-neumann-31,kochen1}.

For the above configuration, due to conservation laws, noncontextuality holds;
i.e., intuitively the outcome of a measurement of
a {\em link observable} associated with the ray and thus the projector along $x_3=x_3'$
should be indifferent to the choice of the context
$C \equiv \{x_1,x_2,x_3\}$ or $C' \equiv \{x_1',x_2',x_3'\}$.
More precisely, consider
 a rotation of $\varphi = \pi /4$ along the $x_3$-axis in $x_1-x_2$-plane,
such that
$ {x_1'} = (1/\sqrt{2})(1,1,0)^T$,
$ {x_2'} = (1/\sqrt{2})(-1,1,0)^T$,
and let the  maximal context operators be
$
C =
\alpha
[{x_1}^T, {x_1}]
+
\beta
[{x_2}^T, {x_2}]
+
\gamma
[{x_3}^T, {x_3}]
$,
$
C' =
\alpha
[{x_1'}^T, {x_1'}]
+
\beta
[{x_2'}^T, {x_2'}]
+
\gamma
[{x_3}^T, {x_3}]
$,
with $\alpha \neq \beta \neq \gamma \neq \alpha$;
$[x^T,x]\equiv \vert x\rangle \langle x\vert$ represents the dyadic
product of the vector $x$ with itself.

Consider the observables corresponding to projectors along the common direction
represented by the projector
$[{x_3}^T, {x_3}]\propto E_zC=E_zC' \propto [{x_3'}^T, {x_3'}]$.
Then, for all states $\rho$, the quantum expectation values
$
{\rm Trace} (\rho E_zC) =
{\rm Trace} (E_zC\rho )=
{\rm Trace} (\rho E_zC')=
{\rm Trace} (E_zC'\rho )
$
are the same and independent of the choice of the maximal context operators
$C$ and $C'$.
More explicitly, for arbitrary $\rho$,
\begin{equation}
\begin{array}{rcl}
E_zC\rho &=&
{\rm diag}(0 , 0 , 1 )
\cdot
{\rm diag}(\alpha , \beta , \gamma )
\cdot
\rho
= \gamma \rho_{33},
\\
E_zC'\rho &=&
\frac{1}{2}
\left(
  \begin{array}{ccc}
      0&0 &0 \\
    0&0&0\\
0&0&1
    \end{array}
\right)
\left(
  \begin{array}{ccc}
    \alpha +\beta & \alpha -\beta &0 \\
    \alpha -\beta &\alpha +\beta  &0\\
0&0&\gamma
    \end{array}
\right) \rho = \gamma \rho_{33}.
    \end{array}
\label{2004-qnc-eq1}
\end{equation}
This holds for single particles as well as for correlated particles;
in particular for the singlet state
of two spin-1 particles \cite{mermin80,peres-92}
$
\vert \Psi_2 \rangle
= ({1/ \sqrt{3}})(
\vert + -\rangle
+
\vert - +\rangle
-
\vert 0 0\rangle
)$.
When measuring the two contexts $C$ and $C'$ and thus also
the projections corresponding to $x_3=x_3'$ at two locations,
the outcomes of the latter observables should be independent of the contexts.
The two-particle, three-dimensional singlet state  $\vert \psi_2 \rangle$
can be obtained by entanglement of orbital angular momenta of photons cite{mvwz-2001,vwz-2002,tdtm-2003}.
A proof-of-principle also can be obtained from
realization in terms of multiport interferometers \cite{rzbb,zukowski-97}.
The explicit setup for preparation and measurement \cite{svozil-2004-analog}
of $C$ as well as of $C'$ is depicted in
Fig.~\ref{2004-analog-is33}
\begin{figure}
\begin{center}
%TexCad Options
%\grade{\off}
%\emlines{\off}
%\beziermacro{\on}
%\reduce{\on}
%\snapping{\off}
%\quality{2.00}
%\graddiff{0.01}
%\snapasp{1}
%\zoom{10.00}
\unitlength 10.00mm
\linethickness{0.6pt}
\begin{picture}(10.20,10.83)
%(0.0,-1.0)
\put(0.50,10.53){\line(1,0){1.00}}
\put(1.00,10.03){\line(0,1){0.70}}
\put(1.50,10.53){\line(1,0){1.00}}
\put(2.00,10.03){\line(0,1){0.70}}
\put(3.22,10.93){\makebox(0,0)[lc]{\footnotesize T=2/3}}
\put(3.00,10.53){\oval(0.14,0.14)[rt]}
\put(2.80,10.73){\line(1,-1){0.40}}
\put(2.80,10.33){\framebox(0.40,0.40)[cc]{}}
\put(2.50,10.53){\line(1,0){1.00}}
\put(3.00,10.03){\line(0,1){0.70}}
\put(3.50,10.53){\line(1,0){1.00}}
\put(4.00,10.03){\line(0,1){0.70}}
\put(5.22,10.93){\makebox(0,0)[lc]{\footnotesize T=1/2}}
\put(5.00,10.53){\oval(0.14,0.14)[rt]}
\put(4.80,10.73){\line(1,-1){0.40}}
\put(4.80,10.33){\framebox(0.40,0.40)[cc]{}}
\put(4.50,10.53){\line(1,0){1.00}}
\put(5.00,10.03){\line(0,1){0.70}}
\put(5.50,10.53){\line(1,0){1.00}}
\put(6.00,10.03){\line(0,1){0.70}}
\put(7.22,10.83){\makebox(0,0)[lc]{\footnotesize T=0}}
\put(7.00,10.53){\oval(0.14,0.14)[rt]}
\put(6.80,10.73){\line(1,-1){0.40}}
\put(6.50,10.53){\line(1,0){1.00}}
\put(7.00,10.03){\line(0,1){0.70}}
\put(7.50,10.53){\line(1,0){1.00}}
\put(8.00,10.03){\line(0,1){0.70}}
\put(9.00,10.03){\line(0,-1){0.50}}
%
\put(1.00,10.00){\vector(0,-1){0.5}}
\put(2.00,10.00){\vector(0,-1){0.5}}
\put(3.00,10.00){\vector(0,-1){0.5}}
\put(4.00,10.00){\vector(0,-1){0.5}}
\put(5.00,10.00){\vector(0,-1){0.5}}
\put(6.00,10.00){\vector(0,-1){0.5}}
\put(7.00,10.00){\vector(0,-1){0.5}}
\put(8.00,10.00){\vector(0,-1){0.5}}
\put(9.00,10.00){\vector(0,-1){0.5}}
%
%\put(9.00,9.53){\line(1,1){0.30}}
%\put(9.00,9.53){\line(-1,1){0.30}}
\put(9.00,9.27){\makebox(0,0)[cc]{1}}
\put(8.00,10.03){\line(0,-1){0.50}}
%\put(8.00,9.53){\line(1,1){0.30}}
%\put(8.00,9.53){\line(-1,1){0.30}}
\put(8.00,9.27){\makebox(0,0)[cc]{2}}
\put(7.00,10.03){\line(0,-1){0.50}}
%\put(7.00,9.53){\line(1,1){0.30}}
%\put(7.00,9.53){\line(-1,1){0.30}}
\put(7.00,9.27){\makebox(0,0)[cc]{3}}
\put(6.00,10.03){\line(0,-1){0.50}}
%\put(6.00,9.53){\line(1,1){0.30}}
%\put(6.00,9.53){\line(-1,1){0.30}}
\put(6.00,9.27){\makebox(0,0)[cc]{4}}
\put(5.00,10.03){\line(0,-1){0.50}}
%\put(5.00,9.53){\line(1,1){0.30}}
%\put(5.00,9.53){\line(-1,1){0.30}}
\put(5.00,9.27){\makebox(0,0)[cc]{5}}
\put(4.00,10.03){\line(0,-1){0.50}}
%\put(4.00,9.53){\line(1,1){0.30}}
%\put(4.00,9.53){\line(-1,1){0.30}}
\put(4.00,9.27){\makebox(0,0)[cc]{6}}
\put(3.00,10.03){\line(0,-1){0.50}}
%\put(3.00,9.53){\line(1,1){0.30}}
%\put(3.00,9.53){\line(-1,1){0.30}}
\put(3.00,9.27){\makebox(0,0)[cc]{7}}
\put(2.00,10.03){\line(0,-1){0.50}}
%\put(2.00,9.53){\line(1,1){0.30}}
%\put(2.00,9.53){\line(-1,1){0.30}}
\put(2.00,9.27){\makebox(0,0)[cc]{8}}
\put(0.00,10.53){\line(1,0){0.50}}
\put(-0.20,10.53){\makebox(0,0)[cc]{1}}
\put(0.50,10.53){\vector(1,0){0.01}}
\put(1.00,10.03){\line(0,-1){0.50}}
%\put(1.00,9.53){\line(1,1){0.30}}
%\put(1.00,9.53){\line(-1,1){0.30}}
\put(1.00,9.27){\makebox(0,0)[cc]{9}}
\put(8.50,10.53){\line(1,0){0.50}}
\put(9.00,10.53){\line(0,-1){0.50}}
%      t=
\put(0.60,8.41){\line(1,-1){8.84}}
\put(2.00,8.50){\line(0,-1){1.00}}
\put(2.50,8.00){\line(-1,0){1.00}}
\put(3.00,8.50){\line(0,-1){1.00}}
\put(3.50,8.00){\line(-1,0){1.00}}
\put(3.00,7.50){\line(0,-1){1.00}}
\put(3.50,7.00){\line(-1,0){1.00}}
\put(4.00,8.50){\line(0,-1){1.00}}
\put(4.50,8.00){\line(-1,0){1.00}}
\put(4.00,7.50){\line(0,-1){1.00}}
\put(4.50,7.00){\line(-1,0){1.00}}
\put(4.00,6.50){\line(0,-1){1.00}}
\put(4.50,6.00){\line(-1,0){1.00}}
\put(5.00,8.50){\line(0,-1){1.00}}
\put(5.50,8.00){\line(-1,0){1.00}}
\put(5.00,7.50){\line(0,-1){1.00}}
\put(5.50,7.00){\line(-1,0){1.00}}
\put(5.00,6.50){\line(0,-1){1.00}}
\put(5.50,6.00){\line(-1,0){1.00}}
\put(5.10,5.35){\makebox(0,0)[lc]{\footnotesize T=0}}
\put(5.00,5.00){\oval(0.14,0.14)[lb]}
\put(4.80,5.20){\line(1,-1){0.40}}
\put(5.00,5.50){\line(0,-1){1.00}}
\put(5.50,5.00){\line(-1,0){1.00}}
\put(6.00,8.50){\line(0,-1){1.00}}
\put(6.50,8.00){\line(-1,0){1.00}}
\put(6.00,7.50){\line(0,-1){1.00}}
\put(6.50,7.00){\line(-1,0){1.00}}
\put(6.00,6.50){\line(0,-1){1.00}}
\put(6.50,6.00){\line(-1,0){1.00}}
\put(6.10,5.45){\makebox(0,0)[lc]{\footnotesize T=1/2}}
\put(6.00,5.00){\oval(0.14,0.14)[lb]}
\put(5.80,5.20){\line(1,-1){0.40}}
\put(5.80,4.80){\framebox(0.40,0.40)[cc]{}}
\put(6.00,5.50){\line(0,-1){1.00}}
\put(6.50,5.00){\line(-1,0){1.00}}
\put(6.10,4.35){\makebox(0,0)[lc]{\footnotesize T=0}}
\put(6.00,4.00){\oval(0.14,0.14)[lb]}
\put(5.80,4.20){\line(1,-1){0.40}}
\put(6.00,4.50){\line(0,-1){1.00}}
\put(6.50,4.00){\line(-1,0){1.00}}
\put(7.00,8.50){\line(0,-1){1.00}}
\put(7.50,8.00){\line(-1,0){1.00}}
\put(7.00,7.50){\line(0,-1){1.00}}
\put(7.50,7.00){\line(-1,0){1.00}}
\put(7.10,6.35){\makebox(0,0)[lc]{\footnotesize T=0}}
\put(7.00,6.00){\oval(0.14,0.14)[lb]}
\put(6.80,6.20){\line(1,-1){0.40}}
\put(7.00,6.50){\line(0,-1){1.00}}
\put(7.50,6.00){\line(-1,0){1.00}}
\put(7.00,5.50){\line(0,-1){1.00}}
\put(7.50,5.00){\line(-1,0){1.00}}
\put(7.00,4.50){\line(0,-1){1.00}}
\put(7.50,4.00){\line(-1,0){1.00}}
\put(7.00,3.50){\line(0,-1){1.00}}
\put(7.50,3.00){\line(-1,0){1.00}}
\put(8.10,8.35){\makebox(0,0)[lc]{\footnotesize T=0}}
\put(8.00,8.00){\oval(0.14,0.14)[lb]}
\put(7.80,8.20){\line(1,-1){0.40}}
\put(8.00,8.50){\line(0,-1){1.00}}
\put(8.50,8.00){\line(-1,0){1.00}}
\put(8.00,7.50){\line(0,-1){1.00}}
\put(8.50,7.00){\line(-1,0){1.00}}
\put(8.00,6.50){\line(0,-1){1.00}}
\put(8.50,6.00){\line(-1,0){1.00}}
\put(8.00,5.50){\line(0,-1){1.00}}
\put(8.50,5.00){\line(-1,0){1.00}}
\put(8.00,4.50){\line(0,-1){1.00}}
\put(8.50,4.00){\line(-1,0){1.00}}
\put(8.00,3.50){\line(0,-1){1.00}}
\put(8.50,3.00){\line(-1,0){1.00}}
\put(8.00,2.50){\line(0,-1){1.00}}
\put(8.50,2.00){\line(-1,0){1.00}}
\put(9.10,8.45){\makebox(0,0)[lc]{\footnotesize T=1/2}}
\put(9.00,8.00){\oval(0.14,0.14)[lb]}
\put(8.80,8.20){\line(1,-1){0.40}}
\put(8.80,7.80){\framebox(0.40,0.40)[cc]{}}
\put(9.00,8.50){\line(0,-1){1.00}}
\put(9.50,8.00){\line(-1,0){1.00}}
\put(9.10,7.35){\makebox(0,0)[lc]{\footnotesize T=0}}
\put(9.00,7.00){\oval(0.14,0.14)[lb]}
\put(8.80,7.20){\line(1,-1){0.40}}
\put(9.00,7.50){\line(0,-1){1.00}}
\put(9.50,7.00){\line(-1,0){1.00}}
\put(9.00,6.50){\line(0,-1){1.00}}
\put(9.50,6.00){\line(-1,0){1.00}}
\put(9.00,5.50){\line(0,-1){1.00}}
\put(9.50,5.00){\line(-1,0){1.00}}
\put(9.00,4.50){\line(0,-1){1.00}}
\put(9.50,4.00){\line(-1,0){1.00}}
\put(9.00,3.50){\line(0,-1){1.00}}
\put(9.50,3.00){\line(-1,0){1.00}}
\put(9.10,2.45){\makebox(0,0)[lc]{\footnotesize T=1/2}}
\put(9.00,2.00){\oval(0.14,0.14)[lb]}
\put(8.80,2.20){\line(1,-1){0.40}}
\put(8.80,1.80){\framebox(0.40,0.40)[cc]{}}
\put(9.00,2.50){\line(0,-1){1.00}}
\put(9.50,2.00){\line(-1,0){1.00}}
\put(9.10,1.35){\makebox(0,0)[lc]{\footnotesize T=0}}
\put(9.00,1.00){\oval(0.14,0.14)[lb]}
\put(8.80,1.20){\line(1,-1){0.40}}
\put(9.00,1.50){\line(0,-1){1.00}}
\put(9.50,1.00){\line(-1,0){1.00}}
\put(1.00,9.00){\line(0,-1){0.50}}
\put(9.50,0.00){\line(1,0){0.50}}
%\put(10.00,0.00){\line(-1,-1){0.30}}
%\put(10.00,0.00){\line(-1,1){0.30}}
\put(10.20,0.00){\makebox(0,0)[cc]{1}}
\put(1.00,8.50){\line(0,-1){0.50}}
\put(2.00,9.00){\line(0,-1){0.50}}
\put(9.50,1.00){\line(1,0){0.50}}
%\put(10.00,1.00){\line(-1,-1){0.30}}
%\put(10.00,1.00){\line(-1,1){0.30}}
\put(10.20,1.00){\makebox(0,0)[cc]{2}}
\put(2.00,7.50){\line(0,-1){0.50}}
\put(2.00,7.00){\line(1,0){0.50}}
\put(3.00,9.00){\line(0,-1){0.50}}
\put(9.50,2.00){\line(1,0){0.50}}
%\put(10.00,2.00){\line(-1,-1){0.30}}
%\put(10.00,2.00){\line(-1,1){0.30}}
\put(10.20,2.00){\makebox(0,0)[cc]{3}}
\put(3.00,6.50){\line(0,-1){0.50}}
\put(3.00,6.00){\line(1,0){0.50}}
\put(4.00,9.00){\line(0,-1){0.50}}
\put(9.50,3.00){\line(1,0){0.50}}
%\put(10.00,3.00){\line(-1,-1){0.30}}
%\put(10.00,3.00){\line(-1,1){0.30}}
\put(10.20,3.00){\makebox(0,0)[cc]{4}}
\put(4.00,5.50){\line(0,-1){0.50}}
\put(4.00,5.00){\line(1,0){0.50}}
\put(5.00,9.00){\line(0,-1){0.50}}
\put(9.50,4.00){\line(1,0){0.50}}
%\put(10.00,4.00){\line(-1,-1){0.30}}
%\put(10.00,4.00){\line(-1,1){0.30}}
\put(10.20,4.00){\makebox(0,0)[cc]{5}}
\put(5.00,4.50){\line(0,-1){0.50}}
\put(5.00,4.00){\line(1,0){0.50}}
\put(6.00,9.00){\line(0,-1){0.50}}
\put(9.50,5.00){\line(1,0){0.50}}
%\put(10.00,5.00){\line(-1,-1){0.30}}
%\put(10.00,5.00){\line(-1,1){0.30}}
\put(10.20,5.00){\makebox(0,0)[cc]{6}}
\put(6.00,3.50){\line(0,-1){0.50}}
\put(6.00,3.00){\line(1,0){0.50}}
\put(7.00,9.00){\line(0,-1){0.50}}
\put(9.50,6.00){\line(1,0){0.50}}
%\put(10.00,6.00){\line(-1,-1){0.30}}
%\put(10.00,6.00){\line(-1,1){0.30}}
\put(10.20,6.00){\makebox(0,0)[cc]{7}}
\put(7.00,2.50){\line(0,-1){0.50}}
\put(7.00,2.00){\line(1,0){0.50}}
\put(8.00,9.00){\line(0,-1){0.50}}
\put(9.50,7.00){\line(1,0){0.50}}
%\put(10.00,7.00){\line(-1,-1){0.30}}
%\put(10.00,7.00){\line(-1,1){0.30}}
\put(10.20,7.00){\makebox(0,0)[cc]{8}}
\put(8.00,1.50){\line(0,-1){0.50}}
\put(8.00,1.00){\line(1,0){0.50}}
\put(9.00,9.00){\line(0,-1){0.50}}
\put(9.50,8.00){\line(1,0){0.50}}
%\put(10.00,8.00){\line(-1,-1){0.30}}
%\put(10.00,8.00){\line(-1,1){0.30}}
\put(10.20,8.00){\makebox(0,0)[cc]{9}}
\put(9.00,0.50){\line(0,-1){0.50}}
\put(9.00,0.00){\line(1,0){0.50}}
\put(0.59,8.40){\line(1,-1){8.84}}
\put(1.00,8.00){\line(1,0){1.00}}
\put(10.00,0.00){\vector(1,0){0.0}}
\put(10.00,1.00){\vector(1,0){0.0}}
\put(10.00,2.00){\vector(1,0){0.0}}
\put(10.00,3.00){\vector(1,0){0.0}}
\put(10.00,4.00){\vector(1,0){0.0}}
\put(10.00,5.00){\vector(1,0){0.0}}
\put(10.00,6.00){\vector(1,0){0.0}}
\put(10.00,7.00){\vector(1,0){0.0}}
\put(10.00,8.00){\vector(1,0){0.0}}
\end{picture}
\end{center}
\caption{Experimental setup for a multiport interferometric analogue of
the measurement of two three-state particles in the singlet state,
measured by $C,C'$ in Eq.~(\ref{2004-qnc-eq1}) and depicted in Fig.~\ref{2004-qnc-f1}a-b).
$T$ denotes the transmittance of the beam splitters.
The upper part represents the preparation stage, the lower part the analyzing stage.
In the analyzing stage, not all beam splitters are required for the singlet state.
 \label{2004-analog-is33}}
\end{figure}


A generalization of the above argument for arbitrary dimensions and an
arbitrary number of observables yields
the {\em principle of limited quantum noncontextuality:}
Given a set of contexts in which one or more observables $A, \ldots$
coincide.
Then, within that set of contexts, the outcomes of $A, \ldots$ do not depend
on the context; i.e., which other observables are measured alongside.



%PART 2

Note that the Kochen-Specker set of contexts has an empty set of coinciding observables.
Indeed, already a three-tripod configuration
$\{x_1,x_2,x_3=x_3'\}-\{x_3=x_3',x_2',x_1'=x_1''\}-\{x_1'=x_1'',x_2'',x_3''\}$
as depicted in Fig.~\ref{2004-qnc-f1}c),
with two interconnections has an empty set of contexts.

Early on, Kochen \cite{hey-red} has suggested
to consider entangled multi-particle systems measured at different locations.
In its extreme form, this approach may yield an ``explosion view'' of the
Kochen-Specker proof
by requiring (at least) one particle per context;
all these particles should be ``suitable'' entangled (see below)
to allow a counterfactual inference
similar to the Einstein-Podolsky-Rosen (EPR) argument.
For
instance, a physical realization,
if it existed, of the observables in Peres' form
of the Kochen-Specker proof \cite{peres,svozil-tkadlec}
would require $N=40$ different, interconnected contexts and thus
an entangled state of just as many particles.

In order to be able to reach conclusions at
one location about the elements of physical reality \cite{epr}
at all the other $N-1$ locations,
the following {\em uniqueness property} must hold:
the $N$ particles must be in a state  $\Psi$
such that
(i) $\Psi$ is invariant under the unitary transformations $u^N$
(identical transformations $u$ for every particle),
while at the same time meeting the requirement that
(ii) a partial measurement at only one location must
fix a {\em unique} term in the expansion of $\Psi$.
The uniqueness property asserts that measurement of the state of one particle in an $N$-particle
(entangled) state defines
the state of  all the other $N-1$ particles instantaneously,
although neither particle needs to possess its own well-defined state before the measurement.

As a group theoretic argument \cite{2004-kasper-svo} shows,
this is impossible
for the spin-1 multipartite ($N>2$) case.
Indeed, already for $N=3$, the only singlet state  is (see also \cite{kok-02})
\begin{equation}
\vert \Psi_3 \rangle
= {1\over \sqrt{6}}(
\vert - + 0\rangle
-
\vert - 0 +\rangle
+
\vert + 0 - \rangle
-
\vert + - 0\rangle
+
\vert 0 - + \rangle
-
\vert 0 + - \rangle
).
\label{2004-qnc-e1}
\end{equation}
From (\ref{2004-qnc-e1}) it can be inferred that
any partial measurement at only one location cannot
fix a  unique term in the expansion of $\Psi_3$.
Suppose, for instance, that the first particle is measured to be in state
``$-$.'' Then the second and third particle may either be in state ``$0$'' or in state ``$+$''
in that particular direction.
Thus, due to the nonuniqueness property, the counterfactual inference of a context $\{x_1,x_2,x_3\}$ in the three-particle spin-1
case (\ref{2004-qnc-e1}) is impossible,
since if the property corresponding to $x_3$ is fixed,
the properties corresponding to $x_2$ and $x_3$ need not be---indeed,
they are only fixed if one more particle spin is measured in the {\em same}
direction as the spin measurement of $x_1$.
Again, Ref.~\cite{svozil-2004-analog} contains a complete construction of a
realization in terms of multiport interferometers.

Note that the uniqueness property holds for $N=2$; i.e., for  the two spin-1 particle
singlet state $\Psi_2$.
For $N=4$, there are three different singlet states.
One of these states is given by
\begin{equation}
\begin{array}{l}
\vert \Psi_4^1 \rangle
= {1\over \sqrt{9}}\left(
\vert +-+- \rangle
+
\vert -++- \rangle
-
\vert 00+-\rangle
+
\vert +--+\rangle
+
\vert -+-+\rangle
-
\vert 00-+\rangle
-
\vert +-00\rangle
-
\vert -+00\rangle
+
\vert 0000\rangle
\right),
\label{2004-qnc-e12}
\end{array}
\end{equation}
for which again the uniqueness property does not hold.
For
$N=5,6,7,8,9,10,11,12$, the number of singlet states is
6, 15, 36, 91, 232, 603, 1585, 4213, 11298, 30537, 83097, 227475, respectively,
In general, despite the abundance of singlet states,
the number of coherent terms in the sum at least doubles
with any additional particle, thus contributing to the multitude of
possibilities which spoil the uniqueness required for value definiteness.
This nonuniqueness seems to be the way quantum mechanics ``avoids''
inconsistency forced upon it by the assumption  of value definiteness
in the Kochen-Specker argument.


%The nonuniqueness property discussed above renders
%an ``explosion view'' of the Kochen-Specker argument
%physically unfeasible.
%Indeed, since the proof contains a {\it reductio ad absurdum,}
%it is interesting how quantum mechanics ``conspires''
%to avoid a contradiction with the assumption that unique elements of reality
%exist.
%This situation is quite similar to arguments based on
%enumerations of experimental outcomes of the standard EPR-type
%experiments \cite{peres222}.


Suppose one insisted on measuring all observables in the explosion view of the Kochen-Specker argument.
This will result in the measurement values
$v_1,\ldots , v_N$, $v_i\in \{+,-,0\}$ for $i=1, \ldots ,N$ different measurement directions.
From these findings it cannot be inferred that every single one of the $N$ particles
has the properties $v_1,\ldots , v_N$ (only one of the properties being measured, the others
inferred counterfactually). This is due to the fact that,
because of failure of the uniqueness property, such a counterfactual inference is impossible.
Nonetheless, this does not exclude that certain (non)singlet states of multipartite systems,
for which the uniqueness property holds,
can be utilized for the sake of similar arguments than the Kochen-Specker proof.
In variants of the Greenberger-Horne-Zeilinger theorem
\cite{ghz,ghsz,mermin90b}, such three- and four-particle
states with the uniqueness property have been used to derive
complete contradictions by a counterfactual argument.
In these arguments, contextuality is neither an essential feature, nor can it be directly tested.


In summary,
despite its appeal as a metaphor, quantum contextuality defies operationalization.
Unlike complementarity, randomness and interference,
it seems to be no valuable computational resource in quantum information and computation theory.
We are thus led to the conclusion that,
as far as experimental tests are possible, quantum mechanics appears noncontextual,
and as far as contextuality is suggested as a possible quantum mechanical feature,
it does not refer to any operationalizable scheme.

\section*{Acknowledgments}
The kind permission of Michael Reck to
use a program for multiport interferometers
developed at the University of Innsbruck from 1994-1996, as well as
discussions with Peter Kasperkovitz on group theoretical questions
are gratefully acknowledged.
Some of the mentioned ideas were discussed with Rob Clifton in the summer of 1999.


\bibliography{svozil}
\bibliographystyle{apsrev}
%\bibliographystyle{unsrt}
%\bibliographystyle{plain}

\end{document}




Several scenarios to operationalize
\cite{cabello-98,simon-2002,huang-2003} as well as to avoid
\cite{pitowsky-82,pitowsky-83,meyer:99}
the Kochen-Specker theorem have been proposed.
The {\it contextual} \cite{bell-66,hey-red,redhead} interpretation suggests,
in Bell's words \cite[Sect. 5]{bell-66}, that the {\em ``$\ldots$
result of an observation may reasonably depend
not only on the state of the system  $\ldots$
but also on the complete disposition  of the apparatus.''}
In this interpretation, any quantum violation of bounds to classical probability
amounts to a violation of noncontextuality,
independent of locality; i.e., of the locatedness of the (sub)systems.


Stated pointedly, despite its appeal as a metaphor
and as a heroic attempt
to rescue the remainders of a historic, classical understanding,
contextuality seem to be physically misleading at worst
and metaphysical at best:
As far as contextuality can be operationalized and measured,
it is likely to be invalidated by experiment;
and in cases where it seems to be needed for an ``explanation,''
it cannot be utilized physically and does not seem to indicate progressive
research programs which could result in new phenomena.

TensorProduct[a_, b_] :=    Table[(*a, b are nxn and mxm - martices*) a[[Ceiling[s/Length[b]], Ceiling[t/Length[b]]]]*b[[s - Floor[(s - 1)/Length[b]]*Length[b],t - Floor[(t - 1)/Length[b]]*Length[b]]], {s, 1,Length[a]*Length[b]}, {t, 1, Length[a]*Length[b]}];


UIN=
{
{u11,u12,u13,u14,u15,u16,u17,u18,u19},
{u21,u22,u23,u24,u25,u26,u27,u28,u29},
{u31,u32,u33,u34,u35,u36,u37,u38,u39},
{u41,u42,u43,u44,u45,u46,u47,u48,u49},
{u51,u52,u53,u54,u55,u56,u57,u58,u59},
{u61,u62,u63,u64,u65,u66,u67,u68,u69},
{u71,u72,u73,u74,u75,u76,u77,u78,u79},
{u81,u82,u83,u84,u85,u86,u87,u88,u89},
{u91,u92,u93,u94,u95,u96,u97,u98,u99}
};

pin={1,0,0,0,0,0,0,0,0};
pfi={0,0,1,0,-1,0,1,0,0};

#############################################

ClebschGordan[{1, +1}, {1, -1}, {1, 0}]

#############################################


ST[0, 0] = 1;
ST[1, 0] = 0;
ST[1, 1] = 1;

NN = 2; While[NN <= 4,         (* Number of particles *)

L  = 0; While[L <= NN,        (* Total spin *)


If[L == 0 , ST[NN,L] =  ST[NN-1,L+1],

   If[L == NN , ST[NN,L] = ST[NN-1,L-1],

       If[L == NN-1  , ST[NN,L] = ST[NN-1,L-1] + ST[NN-1,L],

         ST[NN,L] =  ST[NN-1,L-1] + ST[NN-1,L] + ST[NN-1,L+1]
         ]
     ]

];


L++]
Print["NN = ",NN];
L  = 0; While[L <= NN, Print[" L = ",L," : ",ST[NN,L]];L++] ;
NN++]

########################################################
ST[0, 0] = 1;
ST[1, 0] = 0;
ST[1, 1] = 1;

a=Table[         (* Number of particles *)

L  = 0; While[L <= NN,        (* Total spin *)


If[L == 0 , ST[NN,L] =  ST[NN-1,L+1],

   If[L == NN , ST[NN,L] = ST[NN-1,L-1],

       If[L == NN-1  , ST[NN,L] = ST[NN-1,L-1] + ST[NN-1,L],

         ST[NN,L] =  ST[NN-1,L-1] + ST[NN-1,L] + ST[NN-1,L+1]
         ]
     ]

];


L++]
Print["NN = ",NN];
(*L  = 0; While[L <= NN, Print[" L = ",L," : ",ST[NN,L]];L++] ;*)
ST[NN,0],{NN,2,40}]

