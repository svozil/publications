Karl Svozil, Wien: Was erh�lt die steuerzahlende �ffentlichkeit f�r ihre Forschungsgelder?

Grundlagenforschung wird wie ehedem �ber das allgemeine Steueraufkommen finanziert.
Damit stellt sich einerseits die Frage, in welchem Umfang �ffentliche Mittel �berhaupt zur Verf�gung gestellt und demokratisch-republikanisch gerechtfertigt werden k�nnen.
Andererseits sind die Methoden, mit denen diese �ffentlichen Mittel verteilt und m�glichst effektiv genutzt werden k�nnen, nicht unumstritten.
Ein Teilaspekt hiervon ist die gegenw�rtige ``neoliberale'' Tendenz,
durch Auslagerungen und Pseudoprivatisierung der Forschungsst�tten aus der hoheitsstaatlichen Verwaltung
das ``Return of Investment'' zu erh�hen.
Ein anderer Aspekt ist die hermetische Abgrenzung des Forschungsestablishments gegen�ber Versuchen,
der �ffentlichkeit mehr Mitsprache in Finanzierungsentscheidungen zu geben.
Generell wird angenommen:
(1) je weniger Staat und �ffentlichkeit in den Gang der Forschungen Einfluss nehmen, umso besser; und
(2) die forschenden ``Peers'' w�ren am besten in der Lage, das Geld untereinander zu verteilen.
Es w�rde gen�gen, das Geld den Forschern quasi ``hinzuschieben''; diese w��ten am besten, wie die
Gebarungsgrunds�tze der Wirtschaftlichkeit, Sparsamkeit und Zweckm��igkeit umzusetzen sind.
