\documentclass{article}
\usepackage{graphicx}
\begin{document}

\title{Preface to the proceedings of Quantum Structures 2002}
\author{
Dick Greechie\\
Department of Mathematics, Louisiana TechUniversity, \\
Ruston, LA 71272, USA\\
email: greechie@coes.latech.edu\\ \\
and\\ \\
Sylvia Pulmannova \\
Mathematical Institute, Slovak Academy of Sciences,\\
Stefanikova 49,
Sk-814 73
Bratislava,
Slovakia \\
email: pulmann@mat.savba.sk\\ \\
and\\ \\
Karl Svozil\\
Institut f\"ur Theoretische Physik, University of Technology Vienna,  \\
Wiedner Hauptstra\ss e 8-10/136, A-1040 Vienna, Austria\\
email: svozil@tuwien.ac.at, homepage: http://tph.tuwien.ac.at/~svozil}

\maketitle


%list who should be thanked for the organization, for the support

The international conference Quantum Structures 2002 (QS2002) was held in the city of Vienna
from July 1st to 7th, 2002 in the framework of the
biannual meetings of the International Quantum Structures Association (IQSA).
The University of Technology Vienna served as local co-organizer and sponsor,
accompanied by the City of Vienna and the Vienna Tourist Board.

%mention the amicable time

At this time of the year,
summer was in full swing and floated into the lecture halls adjacent
to the Musikverein, home of the Vienna Philharmonic Orchestra.
As it happened, there also were occasional showers and spring-like, colder periods.
One excursion walk through the City to the {\em Heurigen}
(a traditional Viennese leisure-time hangout which badly compares to a pub) had to be cancelled;
and a hike in the Vienna Woods to the Conference Banquet at
{\it Castle Wilheminenberg} turned out to be rather a chilly and sloshy but sunny adventure.
Those participating may still remember the troubles the organizers had in recollecting all the
participants going astray in the quite romantic bush walks!
All in all it has been a fairly amicable time
full of scientific discussions in the midst of the Central European summertime.

One of the probably most remarkable developments in recent years
is the gradual absorption of many ideas and concepts developed in quantum logic
into the physical mainstream.
As quantum physicists got accustomed to the algebraic structures and to explore their empirical content,
new experimental insights and stimuli from quantum information and computation theory
influenced and excited the quantum structure community at large.
These development could be felt throughout the conference.


[[[[Then to discuss the organization of the papers, perhaps citing the main contribution of  each paper.---please insert here]]]


The local organizers attempted to keep administrative efforts low
by the intensive use of internet based technology;
in particular by an interactive web interface
to a database containing all relevant information,
as well as by using {\em arxiv.org},
the {\em de facto} standard for preprint (self-)publication of physical literature.
In particular, the editorial procedures for these Proceedings could be greatly enhanced by using
{\em arxiv.org} as a front-end.
{\em arxiv.org} provides a free, convenient medium for the dissemination of preprints in variable formats;
allowing the editors to easily communicate to the reviewers
the submitted manuscripts for evaluation by hyperlinks.
The authors in turn may post the revised manuscript versions for further review and processing.
This system worked very efficiently as long as the authors were able to cope with the schemata
provided by {\em arxiv.org}.
Unfortunately,
as time passed, new and stiffer procedures were established by the operators of the preprint archives,
and ``manuscript relaying'' is strongly discouraged now, making it very difficult for the editors
of the proceedings to facilitate the prepublication of manuscripts in cases where
the authors had insufficient access to, or lacked the willingness to cope with the
set-up of the archive.
The new endorsement system may contribute to an effective barrier, a digital divide,
which the operators of this free preprint archive have established in an attempt to
maintain a feasible quality management while keeping the costs low.
We [One of us (KS)???]
take[s] this experience as a further indication to integrate the existing manuscript archives
into a database operated by an international consortium, probably even administered under
the aegis of the United Nations Educational, Scientific and Cultural Organization (UNESCO);
with clearly defined editorial and moderation practices.

So, all in all it has been a scientifically stimulating and profitable conference.
We would like to take this opportunity to thank all contributors for their collaboration.

Dick Greechie,
Sylvia Pulmannova,
Karl Svozil

\end{document}

