%%tth:\begin{html}<LINK REL=STYLESHEET HREF="http://tph.tuwien.ac.at/~svozil/ssh.css">\end{html}
\documentstyle[12pt,amsfonts]{article}
%\renewcommand{\baselinestretch}{2}
%%\RequirePackage{garamond}
\RequirePackage{mathptm}
%\RequirePackage{bookman}
%\RequirePackage{helvetic}
\RequirePackage{times}
%%\renewcommand{\baselinestretch}{2}
\begin{document}

\newtheorem{theorem}{Theorem}[subsection]
\newtheorem{defin}[theorem]{Definition}
\newtheorem{lemma}[theorem]{Lemma}
\newtheorem{corollary}[theorem]{Corollary}

 \title{Automaton logic}
\author{M. Schaller and K. Svozil\\
 {\small Institut f\"ur Theoretische Physik,
Technische Universit\"at Wien   }     \\
  {\small Wiedner Hauptstra\ss e 8-10/136,
A-1040 Vienna, Austria   }            \\
  {\small e-mail: svozil@tph.tuwien.ac.at}
}
\date{}
\maketitle

\begin{abstract}
The experimental logic of Moore and Mealy type automata is investigated.
\end{abstract}
\begin{flushleft}
key words: automaton logic; partition logic;
comparison to quantum logic;
intrinsic measurements
\end{flushleft}



\section{Introduction}

\subsection{Motivation}

Already in 1956, Moore
\cite{moore}
presented an
explicit example of a four-state automaton featuring
an ``automaton uncertainty principle'' at a very elementary level.
The formalism introduced by Moore has been extended by Conway
\cite{conway} and Chaitin \cite{chaitin-65}, among others.
See \cite{hopcroft,brauer}
for a recent review on Moore and Mealy automata.

In an article entitled ``computational complementarity'',
D. Finkelstein and S. R. Finkelstein \cite{finkelstein} were the first
to study the {\em experimental logic} of very general
automata; i.e., the ordered structure of propositions
arising from experiments on automata, and the relationship to quantum
physics. Based on this research, Grib and Zapatrin \cite{grib90,grib92}
investigated an
automaton type, whose corresponding ``macrostatements'' (propositions
about automaton ensembles),  model
arbitrary orthomodular lattices \cite{gsz}. In another interesting
development, Crutchfield
\cite{crutchfield}
described the measurement process by
introducing a hierarchy of automata.

This article goes back to Moore's original approach and deals with
an algebraic characterization of the experimental logic of Moore and
Mealy type automata.

\subsection{Classical logic versus quantum logic versus automaton logic}

In the followinf, we shall describe, in a somewhat simplified
style, the construction of the logic calculus
of classical physical systems, quantum systems and automata.

Let ${\frak S}$ be a classical system.
We denote the set of all observables of the system by
$(A_i)_{i \in I}$.
It is characteristic for classical systems that all $(A_i)_{i \in I}$ are
simultaneously measurable.
We denote the outcome of such a measurement by $(x_i)_{i \in I}$.
The set of all possible outcomes forms the observation space $O$.
The most general form of a prediction concerning ${\frak S}$ is that
the point $(x_i)_{i \in I}$ determined by actually measuring $(A_i)_{i \in I}$, will
lie in a subset $S$ of $O$.
We may call the subsets of $O$ the ``experimental propositions''
concerning ${\frak S}$.
These subsets form a Boolean algebra (which is equal to
the power set of $O$).
Associated with the system ${\frak S}$ is the phase space $\Gamma$.
According to the concept of a phase space, the state of ${\frak S}$
is represented by a point $p \in \Gamma$, which determines
the outcome of the measurements $(A_i)_{i \in I}$ in a deterministic way.
We may assume a mapping $f: \Gamma \rightarrow O$, which describes
this correspondence.
Each experimental proposition ${\frak S}$ corresponds to a subset
$\Gamma_S$ of $\Gamma$ by $\Gamma_S = f^{-1}(S)$.
These subsets $\Gamma_S$ form the propositional calculus of the system
${\frak S}$, which is also a Boolean algebra
[using $f^{-1}(S \cup T) = f^{-1}(S) \cup f^{-1}(T)$,
$f^{-1}(S \cap T) = f^{-1}(S) \cap f^{-1}(T)$ and
$f^{-1}(S') = (f^{-1}(S))'$].

The situation in quantum mechanics is as follows.
Let ${\frak S}$ be a quantum system and let $(B_j)_{j \in J}$ be a
set of compatible measurements.
The experimental propositions concerning the measurement of
$(B_j)_{j \in J}$ are again subsets of the observation space $O_J$
of all possible outcomes $(x_j)_{j \in J}$
(now, $O_J$ depends on the set $(B_j)_{j \in J}$).
According to the quantum mechanical formalism, the subsets $\Gamma_S$
of the phase space $\Gamma$ have to be replaced by closed subspaces
of an appropriate Hilbert space $H$
(or equivalently, by projections operators $p_S$ of $H$).
The set $L(H)$ of all closed subspaces is called the propositional
calculus (quantum logic) of the system ${\frak S}$.
$L(H)$ forms a complete atomistic orthomodular lattice
(cf.~\cite{kalmbach,piziak,ptak}).
The story of quantum logic goes back to the seminal
paper of Birkhoff and von Neumann \cite{birkhoff}.
The interest in quantum logic was revived through the
investigations of Jauch \cite{jauch} and Piron \cite{piron}.
The historical development and the different approaches to quantum logic
can  be found in \cite{jammer}.

At last we turn to automata logic.
An automaton (Mealy or Moore automaton) is a finite deterministic system
with input and output capabilities.
At any time the automaton is in a state $q$ of a finite set of states $Q$.
The state determines the future input--output behavior of the automaton.
If an input is applied, the automaton assumes a new state, depending
both on the old state and on the input. An output is emitted
which depends on the old state and the input
(Mealy automaton) or only on the new state (Moore
automaton).
Automaton experiments are conducted by applying an input sequence and
observing the output sequence.
The automaton is thereby treated as a black box with known
description but unknown initial state.
Let $E$ be an automaton experiment and let $O_E$ be the observation space,
i.e., $O_E$ is the set of all possible outcomes of $E$.
Because of the deterministic nature of the automaton,
for every experiment $E$ there exists a mapping $\lambda_E: Q \rightarrow O_E$,
determining the outcome of $E$, and depending on the initial state
of the automaton.
As in the classical and quantum case, experimental propositions concerning
the experiment $E$
are subsets $S_E$ of $O_E$.
For every experiment $E$, the inverse images of the sets $S_E$
under $\lambda_E$ forms a Boolean algebra
(more exactly, a field of sets). The elements of this
Boolean algebra are subsets of the state set $Q$.
We obtain a propositional calculus, termed the automaton logic,
if we ``paste'' all Boolean algebras corresponding to all experiments
together. This calculus forms a partition logic \cite{svozil,schaller}.
Intuitively, as
has already been observed by Moore \cite{moore}, it may
occur that the automaton undergoes an irreversible state change, i.e.,
information about the automaton's initial state is lost.
A second, later experiment may therefore be affected by the first experiment,
and vice versa.
Hence, both experiments are incompatible.
In this setup, the observer has a qualifying influence on the measurement
result insofar as a particular observable has to be chosen among a class
of non-co-measurable observables.
But the observer has no quantifying influence on the measurement result
insofar as the outcome of a particular measurement is concerned.


\section{Orthomodular Posets}

The appropriate algebraic structures to describe the logic of automata
are found in the theory of orthomodular posets.
Orthomodular structures arose from lattice
theory
\cite{bhlat,graetzer,szasz}
and quantum logic \cite{birkhoff,giuntini-91}.
The basic notion of orthomodular posets will be defined first.
Then, a new type of logic, termed partition logic, will be introduced.
We shall prove a representation theorem, which identifies certain
orthomodular posets with partition logics.
Some examples of the new concepts will be given.
More detailed introductions into the theory of orthomodular structures
can be found in the book of G.~Kalmbach \cite{kalmbach}
and in the book of P.~Pt\'ak and S.~Pulmannov\'a \cite{ptak}.
The books by J.~Jauch \cite{jauch} and C~.Piron \cite{piron}, among
others, deal with
physical applications,
mainly in the context of quantum mechanics.

\subsection{Basic Definitions}

\begin{defin}
\label{OMP}
An orthomodular poset (OMP) is a set $L$ endowed with
a partial order $\le$ and a unary operation $'$,
called the orthocomplement, such that
the following conditions for all $a,b \in L$ are satisfied:

(i) $L$ possesses a least and a greatest element $0$ and $1$,
 and $0\neq1$;

(ii) $a \leq b$ implies $b'\leq a'$;

(iii) $(a')' = a$;

(iv) if $a \leq b'$, then the supremum $a\vee b$ exists;

(v) if $a\leq b$, then $b=a\vee (a' \wedge b)$
(orthomodular law).
\end{defin}

The symbols $\vee, \wedge$ denote the lattice-theoretic operations induced by
$\le$.
If an OMP is an lattice, we call it an {\em orthomodular lattice}
(OML).
An OMP $L$ does neither have to be distributive nor a lattice.
On the other hand, de Morgan's law is valid in $L$:
If $a \vee b$ exists in $L$, then $a' \wedge b'$ exists also and
$a' \wedge b' = (a \vee b)'$ [use condition (ii)].
In particular, $1' = 0$ and $0' = 1$.
Moreover, condition (v) yields $a \vee a' = 1$ for any $a \in L$
(and, dually, we also have $a \wedge a' = 0$ for any $a \in L$).
The {\em orthogonality relation} $\perp$ for elements $a,b$ of an OMP $L$ is
defined by

$ a \perp b$ ($a$ is orthogonal to b) if $a \le b'$ \\
holds. A pair $a,b \in L$ is called {\em compatible}, denoted by
$a \leftrightarrow b$, if there exist three mutually orthogonal
elements $a_1,b_1,c$ such that $a = a_1 \vee c$ and $b = b_1 \vee c$.

We now exhibit some basic examples of OMPs.
Every Boolean algebra is an OMP.
The lattice $L(H)$ of all projection operators on a (real or complex)
Hilbert space $H$ (or, equivalently, the lattice of all closed subspaces of
$H$) is an OMP, with the relation $\le$ given by the inclusion and with
the operation $'$ given by the formation of the orthocomplement in $H$.

\begin{defin}
\label{sub-OMP}
A subset $M$ of $L$ is called a sub-OMP of $L$ if the following conditions
are satisfied:

(i) $0 \in M$;

(ii) if $a \in M$, then $a' \in M$;

(iii) if $a,b \in M$ and $a \perp b$, then $a \vee b \in M$
(the supremum is taken in $L$).
\end{defin}

The sub-OMP $\Gamma A$ generated by an arbitrary subset $A$ of $L$
is the smallest subalgebra of $L$ containing $A$; it always exists.

\begin{defin}
\label{morphism}
Let $L_1,L_2$ be OMPs.
A mapping $f: L_1 \rightarrow L_2$ is called a morphism (of OMPs)
if the following conditions are satisfied:

(i) $f(0) = 0$;

(ii) $f(a') = f(a)'$;

(iii) if $a \perp b$, then $f(a \vee b) = f(a) \vee f(b)$ \\
A morphism $f: L_1 \rightarrow L_2$ is called an isomorphism (of OMPs) if $f$ is
injective, maps $L_1$ onto $L_2$ and the mapping $f^{-1}$ is also
a morphism.
\end{defin}

\begin{lemma}
\label{isomorphism}
A bijective mapping $f: L_1 \rightarrow L_2$ is an isomorphism iff
the following conditions are satisfied:

(i) $f(a') = f(a)'$;

(ii) $a \le b$ iff $f(a) \le f(b)$.
\end{lemma}

Proof.
(i) If $f$ is a morphism, $f$ preserves the order.
If $a \le b$ for $a,b \in L_1$, then the orthomodular law yields
$b = a \wedge c$ for a $c \in L_1$ such that $c \perp a$.
Then, $f(b) = f(a) \wedge f(c)$, and, therefore, $f(a) \le f(b)$.
For an isomorphism $f$, $f$ and $f^{-1}$ are morphism, hence we
get condition (ii).

(ii) The converse direction is trivial.\\

We shall prove three lemmas about the compatibility relation.
The propositions and their proofs are taken from \cite{ptak}.

\begin{lemma}
\label{comp1}
Let $L$ be an OMP and $a,b \in L$:

(i) If $a \perp b$, then $a \leftrightarrow b$;

(ii) If $a \le b$, then $a \leftrightarrow b$;

(iii) If $a \perp b$, then $b = (a \vee b) \wedge a'$.
\end{lemma}

Proof.
(i) Since $a \perp b, b \perp 0$ and $0 \perp a$, we can write
$a= a \vee 0$ and $b = b \vee 0$.

(ii) According to the orthomodular law we can write $b = a \vee c$
for $c = b \wedge a'$.
Therefore, $c \perp a$ and we have $a = a \vee 0$ and $b = b \vee c$.

(iii) $a \perp b$ implies $b' = a \vee (a' \wedge b)$ according to the
orthomodular law.
Forming the orthocomplement and using De Morgan's law, we obtain
$b = (a \vee b) \wedge a'$.

\begin{lemma}
\label{comp2}
Suppose that $a \leftrightarrow b$. Then, every pair in the
set $\{a,a',b,b'\}$ is compatible.
\end{lemma}

Proof.
It suffices to prove that the assumption $a \leftrightarrow b$
implies $a' \leftrightarrow b$.
The other assertions are not difficult to prove.
Suppose that $a \leftrightarrow b$.
Then, $a = a_1 \vee c$ and $b = b_1 \vee c$, where $a_1, b_1, c$
are mutually orthogonal in $L$.
Since $a \perp b_1$, we have $(a \vee b_1) \wedge b_1' = a$
(Lemma \ref{comp1}.iii).
Thus, $a' = (a \vee b_1)' \vee b_1$.
We need to check that the elements $c,b_1$ and $(a \vee b_1)'$
are mutually orthogonal. This is the case,
since $b_1 \le a \vee b_1 = ((a \vee b_1)')'$
and $c \le a \vee b_1 = ((a \vee b_1)')'$.

\begin{lemma}
\label{comp3}
If $a,b \in L$ and $a \leftrightarrow b$, then $a \vee b$ and $a \wedge b$
exist in $L$.
Moreover, if $a= a_1 \vee c$ and $b = b_1 \vee c$ for mutually
orthogonal elements $a_1,b_1,c$, then $a \vee b = a_1 \vee b_1 \vee c$
and $a \wedge b = c$.
Further, we have $a_1 = a \wedge b', b_1 = b \wedge a'$.
Hence, the elements $a_1,b_1,c$ are uniquely determined by $a$ and $b$.
\end{lemma}

Proof.
Since $a_1,b_1,c$ are mutually orthogonal, we find that the
supremum $a_1 \vee b_1 \vee c$ exists in $L$.
Furthermore, $a \le a_1 \vee b_1 \vee c$ and $b \le a_1 \vee b_1 \vee
c$. Let $e$ be an element of $L$ such that $a \le e$ and $b \le e$.
The inequalities $a_1 \vee c \le e, b_1 \vee c\le e$ imply
$(a_1 \vee c) \vee (b_1 \vee c) \le e$.
Hence, $a_1 \vee b_1 \vee c$ is the supremum of $a,b$.
The existence of $a \wedge b$ follows from
Lemma \ref{comp2} and from the equality
$a \wedge b = (a' \vee b')'$.
We now show that $a \wedge b = c$.
On the one hand, $c \le a \wedge b$.
On the other hand, $a \wedge b = (a_1 \vee c) \wedge b \le
(b' \vee c) \wedge b = c$ (Lemma \ref{comp1}.iii).
Finally, we have $a_1 \le a$ and $a_1 \le b'$.
Moreover, by Lemma \ref{comp2}, $a \wedge b'$ exists in $L$.
Thus, $a_1 \le a \wedge b'$.
Furthermore, $a \wedge b' = (a_1 \vee c) \wedge b' \le
(a_1 \vee c) \wedge c' = a_1$.
The proof of the equality $b_1 = b \wedge a'$ is similar.



\subsection{Ideals and States}

The definition of an ideal is similar to the definition of a
lattice ideal.
Additionally we require in condition (ii) that the elements $a$ and $b$
have to be orthogonal.

\begin{defin}
\label{ideal}
Let $L$ be an OMP. A nonvoid subset $I$ of $L$ is called an ideal if it
satisfies the following conditions:

(i) $a \in I, b \le a$ imply $b \in I$;

(ii) $a,b \in  I, a \perp b$ imply $a \vee b \in I$.
\end{defin}

\begin{defin}
\label{primeideal}
An ideal $P$ of $L$, $P \neq L$ is called prime if $a \perp b$ implies
$a \in P$ or $b \in P$.
\end{defin}

We denote by $P(L)$ the set of all prime ideals of $L$.
Let ${\frak P}(P(L))$ the power set of $P(L))$.
We define a mapping $p:L \rightarrow {\frak P}(P(L))$ by
$p(a) = \{ P \in P(L)\mid a \not\in P\}$.
$p$ is called the {\em $p$--function}.

\begin{lemma}
\label{prime1}
Let $P$, $P \neq L$, be an ideal.
The following conditions are equivalent:

(i) $P$ is a prime ideal;

(ii) $a \wedge b \in P$ and $a \leftrightarrow b$ imply $a \in P$
or $b \in P$;

(iii) $a \in P$ iff $a' \not\in P$.
\end{lemma}

Proof.
(i) implies (ii).
Since $a \leftrightarrow b$, there exist three mutually orthogonal elements
$a_1, b_1, c$ such that $a = a_1 \vee c$ and $b = b_1 \vee c$.
From Lemma \ref{comp3} we know that $c = a \wedge b$.
$a_1 \perp b_1$ implies $a_1 \in P$ or $b_1 \in P$.
Therefore, according to the definition of an ideal, $a \in P$ or $b \in P$.

(ii) implies (iii).
We remark that $0 = a \wedge a' \in P$.
We know from Lemma \ref{comp2} that $a \leftrightarrow a'$.
Hence, $a \in P$ or $a' \in P$.
If both $a$ and $a'$ are in $P$, then also $1 = a \vee a' \in P$ and
$P = L$, which contradicts our assumption $P \neq L$.

(iii) implies (i).
Let $a,b \in L$ and $a \perp b$.
We have to prove that $a \in P$ or $b \in P$.
If $a \in P$ the condition is satisfied.
Let us assume $a \not\in P$.
It follows that $a' \in P$.
Since $b \le a'$, we obtain $b \in P$ according to the definition of an ideal.\\

Remark: Compared to lattice prime ideals, the compatibility
of the elements $a$ and $b$ is required additionally.

\begin{lemma}
\label{p-function}
The $p$--function possesses the following properties:

(i) $p(0) = \emptyset$;

(ii) $p(a') = p(a)'$;

(iii) if $a \perp b$, then $p(a \vee b) = p(a) \cup p(b)$;

(iv) if $a \le b$, then $p(a) \subseteq p(b)$.
\end{lemma}

Proof.
(i) $p(0) = \{P \in P(L)\mid 0 \not\in P \} = \emptyset$;

(ii) $p(a') = \{P \in P(L)\mid a' \not\in P\} =$  \\
\hspace*{12mm}
$= \{P \in P(L)\mid a\in P\}$ (using Lemma \ref{prime1})
$= P(L) \backslash p(a)$;

(iii) $P \in p(a \vee b) \Leftrightarrow a \vee b \not\in P
\Leftrightarrow a \not\in P \mbox{ or } b \not\in P
\Leftrightarrow $\\
\hspace*{12mm}
$\Leftrightarrow P \in p(a) \mbox{ or } P \in p(b)
\Leftrightarrow P \in p(a) \cup p(b)$;

(iv) Let $a \le b$ and $P \in p(a)$.
Then $a \not\in P$, and this implies $b \not\in P$
according to the definition of an ideal.


\begin{defin}
\label{state}
A state (i.e.,.~a two-valued state) on an OMP $L$ is mapping
$s:L \rightarrow [0,1]$ (i.e.,.~a mapping $s:L \rightarrow \{0,1\}$)
such that

(i) $s(1) = 1$;

(ii) if $a \perp b$, then $s(a \vee b) = s(a) + s(b)$.
\end{defin}

States are probability measures on an OMP.
The definition is not exact when applied to infinite OMPs (cf.
\cite{ptak}), but in this work we only deal with finite OMPs.
Furthermore, in what follows we need the concept of two-valued states,
which is strongly connected to the definition of a prime ideal.
We denote by $S(L)$ the set of all two-valued states on $L$.

\begin{lemma}
\label{state2}
Suppose that $a,b \in L$ and $a \le b$.
Then, $s(a) \le s(b)$ for any $s \in S(L)$.
\end{lemma}

Proof.
Using the orthomodular law, we can write $b = a \vee (b \wedge a')$,
and therefore
$s(b) = s(a) + s(b \wedge a') \ge s(a)$.

\begin{lemma}
\label{state3}
(i) Let $P$ be a prime ideal. Define a mapping $s:L \rightarrow \{0,1\}$ by
\[ s(x) = \left\{
\begin{array}{l}
0, \mbox{ if } x \in P \\
1, \mbox { if } x \not\in P \\
\end{array}
\right. \]
Then, $s$ is a two-valued state.

(ii) Let $s$ be a two-valued state.
Set $P = \{ x \in L\mid s(x) = 0\}$.
Then, $P$ is a prime ideal.
\end{lemma}

Proof.
(i) Since $1 \not\in P$, the condition $s(1) = 1$ is satisfied.
Suppose that $a \perp b$.
We have to show that $s(a \vee b) = s(a) + s(b)$.
We first assume that $a \vee b \in P$.
Then also $a,b \in P$, and we obtain
$0 = s(a \vee b) = s(a) + s(b) = 0 + 0 = 0$.
Let us now assume that $a \vee b \not\in P$.
According to the definition of a prime ideal, one of both elements $a,b$
has to be in $P$.
If both $a$ and $b$ are in $P$, then also $a \vee b \in P$,
which contradicts our assumption.
Thus, we obtain $1 = s(a \vee b) = s(a) + s(b) = 1 + 0$.

(ii) At first we prove that $P$ is an ideal.
Let $a \in P$ and $b \le a$.
Since $b \le a$, we know by Lemma \ref{state2} that $s(b) \le s(a)$.
Together, we obtain $s(b) = 0$; hence $b \in P$.
Let us now assume that $a,b \in P$ and $a \perp b$.
We have $s(a \vee b) = s(a) + s(b) = 0 + 0 = 0$.
Therefore, we obtain $a \vee b \in P$.
Finally we have to show that $P$ is prime.
Let $a,b$ two elements of $L$ such that the relation $a \perp b$ holds.
We have $s(a \vee b) = s(a) + s(b) \le 1$.
Hence, $s(a) = 0$ or $s(b) = 0$ and therefore
$a \in P$ or $b \in P$. \\

As we shall see later, the set of all prime ideals $P(L)$
(i.e.,.~the state space $S(L)$) can be very poor
(in the extreme case it can be empty).
It seems therefore useful to distinguish the cases when $P(L)$
is relatively big.

\begin{defin}
\label{richprime}
(i) An OMP $L$ is called rich if the following implication holds:

$\{P \in P(L)\mid a \not\in P\} \subseteq \{P \in P(L)\mid b \not\in P\}$
implies $a \le b$.

(ii) An OMP $L$ is called prime if for all $a,b \in L, a \neq b$
there exist a prime ideal $P \in P(L)$ containing exactly one of both
$a$ and $b$.
\end{defin}

Using the $p$--function, we can write:

(i) $L$ is rich if $p(a) \subseteq p(b)$ implies $a \le b$; and

(ii) $L$ is prime if $a \neq b$ implies $p(a) \neq p(b)$.

\begin{lemma}
\label{richprime2}
Every rich OMP is prime.
\end{lemma}

Proof.
Let $L$ be a rich OMP. For all $x \in L$ we set
$p(x) = \{P \in P(L)\mid x \not\in P\}$.
Let $a,b$ be two arbitrary elements of $L$.
If $p(a) = p(b)$ then $a = b$ by the richness of $L$.
Therefore, for $a\neq b$ also $p(a) \neq p(b)$, and a prime ideal
containing exactly one of both $a$ and $b$ exists.

\subsection{Concrete Logics and Partition Logics}

\begin{defin}
\label{concrete}
A concrete logic is a pair $(\Omega,\Delta)$, where $\Omega$ stands for
a set and $\Delta$ stands for a collection of subsets of $\Omega$
satisfying:

(i) $\emptyset \in \Delta$;

(ii) if $A \in \Delta$ then $\Omega \backslash A \in \Delta$;

(iii) if $A,B \in \Delta $ and $A \cap B = \emptyset$, then
$A \cup B \in \Delta$.
\end{defin}

A routine check of the axioms (i)-(v) in definition \ref{concrete}
shows that a concrete logic
becomes an OMP if we take the set inclusion for the relation $\le $ and
the set complement for the orthocomplement $'$.

A simple example of a concrete logic is a pair $(\Omega, \Delta)$,
where $\Omega$ is a finite set of even cardinality and $\Delta$  is the
collection of all subsets of $\Omega$ with an even number of elements.

\begin{theorem}
\label{thgudder}
(Gudder)
An OMP $L$ is isomorphic to a concrete logic iff $L$ is rich.
\end{theorem}

Proof.
(i) Suppose first that $L$ is isomorphic to a concrete logic
$(\Omega, \Delta)$.
We may assume that $L = \Delta$.
Take $A,B \in \Delta$ such that $A \not\le B$.
We have to prove that $p(A) \not\subseteq p(B)$.
$A \not\le B$ implies $A \backslash B \neq \emptyset$ and therefore
we can choose a point $q \in A\backslash B$.
Put $P = \{C \in \Delta \mid q \not\in C$\}.
A routine check verifies that $P$ is a prime ideal.
From the definition of $P$ it follows that $P \in p(A)$,
but $P \not\in p(B)$.

(ii) Conversely, suppose that $L$ is rich.
Put $\Omega = P(L)$ and put $\Delta = \{p(a)\mid a \in L\}$, where $p$ denotes the
the $p$--function.
Let us show that $(\Omega,\Delta)$ is a concrete logic.
From Lemma \ref{p-function}.i,ii, we know that
$p(0) = \emptyset \in \Delta$ and that
$p(a)' = P(L)\backslash p(a) \in \Delta$ for any $p(a) \in \Delta$.
Now, let $p(a)$ and $p(b)$ be orthogonal elements of $\Delta$.
Since $p(a) \cap p(b) = 0$, we obtain $p(a) \subseteq P(L) \backslash
p(b) = p(b')$.
By the richness of $L$ we conclude that $a \perp b$.
Hence, $a \vee b$ exists and $p(a) \cup p(b) = p(a \vee b) \in \Delta$
(using \ref{p-function}.iii), proving that $(\Omega,\Delta)$ is
indeed a concrete logic.
From lemma \ref{p-function}.iv we know that $a \le b$
implies $p(a) \subseteq p(b)$.
Conversely, from the
richness of $L$ we know that $p(a) \subseteq p(b)$ implies $a\le b$.
Hence, by Lemma \ref{isomorphism}, the mapping $p:L \rightarrow \Delta$
is an isomorphism.\\

Remark: Theorem \ref{thgudder} and Theorem \ref{thprime} are related to
the Birkhoff-Stone representation theorem for distributive lattices
and Boolean algebras, respectively.

\begin{defin}
A relation $\approx$ on a set $M$ is called an equivalence relation if it
satisfies the following conditions for all $a,b,c \in M$:

(i) $a \approx a$ (reflexivity);

(ii) $a \approx b$ implies $b \approx a$ (symmetry);

(iii) $a \approx b$ and $b \approx c$ implies $a \approx c$ (transitivity).
\end{defin}

\begin{defin}
Let $M$ be a set. A collection ${\frak A}$ of subsets of $M$ is called
a partition of $M$ if it has the following properties:

(i) $A \cap B = \emptyset$ or $A = B$ for all $A,B \in {\frak A}$;

(ii) $\bigcup {\frak A} = \bigcup\limits_{A \in {\frak A}} A = M$.
\end{defin}

Let $\approx$ be an equivalence relation on $M$ and let $a \in M$.
The {\em equivalence class of $a$ modulo $\approx$} is the set
$[a] = \{b \in M\mid a \approx b\}$.
The set of all equivalence classes modulo $\approx$,
written $M/\approx$, is called the {\em quotient set of $M$ by $\approx$}.
It is easy to check that $M/\approx$ forms a partition of $M$,
called the {\em partition induced by $\approx$}, or the
{\em partition corresponding to $\approx$}.
Let ${\frak A}$ and ${\frak B}$ be partitions of a set $M$.
We call ${\frak A}$ {\em finer} than ${\frak B}$ or say that
${\frak A}$ is a {\em refinement} of ${\frak B}$ if for every $A \in {\frak A}$
there exists a $B \in {\frak B}$ such that $A \subseteq B$. \\

The following definition of the pasting technique is due
to Navara and Rogalewicz \cite{navara}.

\begin{defin}
Let ${\frak L}$ be a family of $OMPs$ satisfying the following condition:

For all $P,Q \in {\frak L}$, $P \cap Q$ is a sub-OMP of both $P$ and $Q$,
and the partial orderings and the orthocomplementations coincide on
$P \cap Q$. \\
Define on the set $L = \bigcup {\frak L} = \bigcup\limits_{P \in {\frak L}} P$
a relation $\le$ and a unary operation $'$ as follows:

(i) $a \le b$ iff there exists a $P \in {\frak L}$ such that
$a,b \in P$ and $a \le_P b$;

(ii) $a' = b$ iff there exists a $P \in {\frak L}$ such that
$a,b \in P$ and $a'^P = b$ \\
(the indices indicate that the operations belong to the respective OMP).
The set $L$ together with $\le$ and $'$ is called the {\em pasting} of
the family ${\frak L}$.
\end{defin}

Let $P$ be a partition of a set $M$.
The {\em Boolean algebra generated by $P$} is the set
$B_P = \{ \bigcup S = \bigcup\limits_{A \in S} A \mid S \subseteq P\}$, together
with the inclusion and the complement.

\begin{defin}[Partition logic]
Let ${\frak R}$ be a family of partitions of a set $M$.
The pasting of the Boolean algebras $B_R,R \in {\frak R}$, is called
a partition logic, denoted by $(M,{\frak R})$.
\end{defin}

It follows from the definition that the orthocomplement $A'$ of
a partition logic $(M,{\frak R})$ is identical with the set complement
$M \backslash A$.
Further $A \le B$ implies $A \subseteq B$.
The converse is in the general not true.

\begin{lemma}
A partition logic $P=(M,{\frak R})$ is an OMP iff the following conditions
are satisfied:

(i) the relation $\le$ is transitive;

(ii) if $A \perp B$ ($A \le B'$), then the supremum $A \vee B$ exists.
\end{lemma}

Proof.
(i) If $P$ is an OMP, the two conditions are satisfied.

(ii) Let $P=(M,{\frak R})$ be a partition logic satisfying the two conditions.
Let $A \in P$. Then there exists  a $R \in {\frak R}$ such that $A \in
B_R$.
In $B_R$ we have $A \le_{B_R} A$ and therefore also $A \le A$, hence,
the relation $\le$ is reflexive.
Let $A,B \in P$ and assume $A \le B,B \le A$.
We obtain $A \subseteq B$ and $B \subseteq A$, obtaining $A = B$.
Hence, $\le$ is also symmetric.
Taking also condition (i) into account, we showed that $\le$
is an order relation.
The axioms (i)-(iii) of definition \ref{OMP} follow trivially.
Axiom (iv) is identical with condition (ii).
Let $A,B \in P$ and $A \le B'$.
Then, there exist a $R \in {\frak R}$, such that
$A,B \in B_R$, and therefore also $A \cup B \in B_R$ and $A,B \le A \cup B$.
By condition (ii) the supremum $A \vee B$ exists.
We obtain $A,B \subseteq A \vee B \subseteq A \cup B$, which implies
$A \vee B = A \cup B$.
In the same way, if $A' \le B$ holds, we obtain
$A \wedge B = A \cap B$.
Let $A \le B$.
Then, $A \vee (A' \wedge B) = A \cup (A' \cap B) = B$, satisfying
the orthomodular law. \\

A {\em block} of an OMP $L$ is a maximal Boolean subalgebra of $L$.
Every element $x$ of $L$ is contained in at least one block,
since the Boolean subalgebra generated by $x$ (and consisting of
$x,x',0,1$) can be embedded into a maximal one.
We denote the set of all blocks of an OMP $L$ by ${\frak B}(L)$.\\

\begin{theorem}
Let $L$ be an OMP.
Then, $L$ is the pasting of its blocks ${\frak B}(L)$.
\end{theorem}

For a proof see Navara and Rogalewicz \cite{navara}.\\

Let $L$ be an OMP and let $x,y \in L$ with $x \le y$.
The sub-OMP $\Gamma\{x,y\}$ generated by the set $\{x,y\}$
is equivalent to the set
$\{0,1,x,x',y,y',x' \wedge y,x \vee y'\}$
(some of the elements may coincide, for instance, if $x=0$ or $x=y$).
Moreover,
$\Gamma\{x,y\}$ is a Boolean algebra.
We put  ${\frak C}(L) = \{\Gamma\{x,y\}\mid x,y \in L \mbox{ and } x \le
y\}$. $L$ is the pasting of the family ${\frak C}(L)$.

\begin{theorem}
\label{thprime}
An OMP $L$ is isomorphic to a partition logic iff $L$ is prime.
\end{theorem}

Proof.
(i) The proof is analogous to the proof of theorem \ref{thgudder}.i.
First, suppose that $L$ is isomorphic to a partition logic
$R=(M,{\frak R})$. We may assume that $L = R$.
Take $A,B \in R$ such that $A \neq B$.
$A \neq B$ implies
$(A \backslash B) \cup (B \backslash A) \neq \emptyset$
and therefore we can choose a point
$q \in (A \backslash B) \cup (B \backslash A)$.
Put $P= \{C \in R\mid q \not\in C\}$.
A routine check verifies that $P$ is a prime ideal.
From the definition of $P$ it follows that exactly one of both $A,B$ is
element of $P$.
Therefore $L$ is prime.

(ii) Conversely, suppose that $L$ is prime.
Put $M = P(L)$.
Let $\Gamma \in {\frak C}(L)$.
Define a partition $R_{\Gamma}$ of $P(L)$ by
$R_{\Gamma} = \{p(a)\mid a \mbox{ is atom of } \Gamma \}$
($p$ is the $p$--function).
It follows from Lemma \ref{p-function} that $R_{\Gamma}$ is indeed
a partition of $M = P(L)$.
Put ${\frak R} = \{R_{\Gamma}\mid \Gamma \in {\frak C}(L)\}$ and let
$R$ be the partition logic $(M,{\frak R})$.
We propose that $p: L \rightarrow R$ is an isomorphism.
$p$ is injective by the primeness of $L$, $p$ is
surjective by the construction of $R$.
For every $\Gamma \in {\frak C}(L)$, the restriction of $p$ to
$\Gamma$, $p\mid\Gamma: \Gamma \rightarrow R_{\Gamma}$, is an isomorphism.
Since  by Lemma \ref{isomorphism}, $L$ is the pasting of ${\frak C}(L)$
and $R$ is the
pasting of ${\frak R}$, $L$ and $R$ are also isomorphic.

Remark: If every element of $L$ can be written as a supremum of
a finite set of atoms, we may also use the family ${\frak B}(L)$
instead of the family ${\frak C}(L)$.

\begin{corollary}
Every concrete logic is a partition logic.
\end{corollary}

Proof. The proposition follows from Lemma \ref{richprime2},
Theorem \ref{thgudder} and Theorem \ref{thprime}. \\

We shall see later that there exist OMPs which are prime but not rich.
The class of concrete logics is therefore a proper subclass
of the class of partition logics.

\subsection{Greechie logics}

In this part we introduce a technique to design OMPs with special
properties.

\begin{defin}
Let ${\frak B}$ be a system of Boolean algebras.
We say that ${\frak B}$ is almost disjoint if for any pair $A,B \in {\frak B}$
at least one of the following conditions is satisfied:

(i) $A = B$;

(ii) $A \cap B = \{0,1\}$;

(iii) $A \cap B = \{0,1,x,x'\}$, \\
where $x$ is an atom in both $A$ and $B$. Moreover, $x' = x'_A
=x'_B$.
\end{defin}

\begin{defin}
Let ${\frak B}$ be an almost disjoint system of Boolean algebras.
A finite sequence $(B_0,B_1,\ldots,B_{n-1})$ of elements of ${\frak B}$
is called a loop of order $n$ if the following conditions are
satisfied (the computation of the $i,j,k$ is modulo $n$):

(i) $B_i \cap B_{i+1} = \{0,1,x_i,x'_i\}$ for $0 \le i \le n-1$;

(ii) $B_i \cap B_j = \{0,1\}$ for $j \neq i-1,i,i+1$;

(iii) $B_i \cap B_j \cap B_k = \{0,1\}$ for distinct indices $i,j,k$.
\end{defin}

Observe that every loop $(B_0,B_1,\ldots,B_{n-1})$ uniquely
determines a sequence of atoms $(e_0, \ldots, e_{n-1})$
such that $e_i$ is the common atom of $B_i$ and $B_{i+1}$.

\begin{lemma}
Let ${\frak B}$ be an almost disjoint system of Boolean algebras and
let $L$ be the pasting of ${\frak B}$.
Then, $\le$ is a partial order and the operation $'$ is an
orthocomplementation.
\end{lemma}

For a proof see \cite{kalmbach,ptak}.

\begin{theorem}
\label{thgreechie}
(Greechie)
Let ${\frak B}$ be an almost disjoint system of Boolean algebras and let
$L$ be the pasting of ${\frak B}$.
Then,

(i) $L$ is an OMP iff ${\frak B}$ does not contain a loop of order 3.

(ii) $L$ is an OML iff ${\frak B}$ does not contain a loop of
order 3 or 4.
\end{theorem}

For a proof see \cite{kalmbach,ptak}. \\

\begin{defin}
An OMP is called a Greechie logic if the following conditions are satisfied:

(i) every element of $L$ can be written as supremum of at most countably
many mutually orthogonal atoms in $L$;

(ii) the collection of all blocks in $L$ forms an almost disjoint system.
\end{defin}

A useful way of exhibiting the Greechie logics is the drawing of
Greechie diagrams.
Let $L$ be a logic and ${\frak B}$ be a system of blocks of it.
Then, $L = \bigcup {\frak B}$.
The Greechie diagram associated with $L$ consist of a set of
points and a set of lines.
The points are in one-to-one correspondence
with the atoms of $L$; the lines are in one-to-one correspondence
with the blocks of $L$.

\begin{figure}
\unitlength 1mm
\begin{picture}(50,20)(00,05)
\multiput(10,10)(15,0){3}{\circle*{1.5}}
\put(10,10){\line(1,0){30}}
\put(9,12){$a$}
\put(24,12){$b$}
\put(39,12){$c$}
\end{picture}
\caption{ }
\label{boo2h2}
\end{figure}
For instance, the drawing in Fig. \ref{boo2h2}
represents the Boolean algebra ${\bf 2}^3$ with the atoms $a,b$ and $c$.
If $L$ is not a Boolean algebra, then it contains several blocks which may or
may not have atoms in common.
\begin{figure}
\unitlength 1mm
\begin{picture}(110,20)(00,05)
\multiput(10,10)(15,0){3}{\circle*{1.5}}
\put(10,10){\line(1,0){30}}
\put(9,12){$a$}
\put(24,12){$b$}
\put(39,12){$c$}
\multiput(70,10)(15,0){3}{\circle*{1.5}}
\put(70,10){\line(1,0){30}}
\put(69,12){$c$}
\put(84,12){$d$}
\put(99,12){$e$}
\end{picture}
\caption{ }
\label{boo2h2d}
\end{figure}
If two distinct blocks drawn in Fig. \ref{boo2h2d}
of $L$ have exactly one atom $c$ in common, then the corresponding edges have
a corner at $c$.
\begin{figure}
\unitlength 1mm
\begin{picture}(70,40)
\multiput(30,10)(-10,10){3}{\circle*{1.5}}
\multiput(30,10)(10,10){3}{\circle*{1.5}}
\put(30,10){\line(-1,1){20}}
\put(30,10){\line(1,1){20}}
\put(10,32){$a$}
\put(20,22){$b$}
\put(29,05){$c$}
\put(37,22){$d$}
\put(47,32){$e$}
\end{picture}
\caption{ }
\label{boo2h2d1}
\end{figure}
For instance, the Greechie diagram drawn in Fig. \ref{boo2h2d1}
corresponds to the Hasse diagram drawn in Fig. \ref{boo2h2d2}.
\begin{figure}
\unitlength 1mm
\begin{picture}(80,65)(0,00)

\put(40,55){\circle*{1.5}}
\multiput(10,25)(15,0){5}{\circle*{1.5}}
\multiput(10,40)(15,0){5}{\circle*{1.5}}
\put(40,10){\circle*{1.5}}

\multiput(10,25)(15,0){4}{\line(1,1){15}}
\multiput(10,25)(30,0){2}{\line(2,1){30}}
\multiput(10,40)(15,0){4}{\line(1,-1){15}}
\multiput(10,40)(30,0){2}{\line(2,-1){30}}

\put(40,10){\line(-2,1){30}}
\put(40,10){\line(-1,1){15}}
\put(40,10){\line(0,1){15}}
\put(40,10){\line(1,1){15}}
\put(40,10){\line(2,1){30}}

\put(40,55){\line(-2,-1){30}}
\put(40,55){\line(-1,-1){15}}
\put(40,55){\line(0,-1){15}}
\put(40,55){\line(1,-1){15}}
\put(40,55){\line(2,-1){30}}

\put(6,22){$a$}
\put(22,22){$b$}
\put(42,22){$c$}
\put(57,22){$d$}
\put(72,22){$e$}

\put(6,40){$a'$}
\put(22,40){$b'$}
\put(42,40){$c'$}
\put(57,40){$d'$}
\put(72,40){$e'$}

\put(39,5){0}
\put(39,57){1}
\end{picture}
\caption{ }
\label{boo2h2d2}
\end{figure}

Note that the Greechie diagrams allow us to detect the presence of the loops
of order 3 or 4 and, therefore, indicate whether the structure in question
is or is not an OMP (OML).
A loop of order 3 shows up as a ``triangle'' and a loop of order 4
as a ``square''.
\begin{figure}
\unitlength 1mm
\begin{picture}(120,50)
\put(10,10){
a)$\qquad$
\begin{picture}(30,30)
\multiput(0,0)(7.5,12.5){3}{\circle*{1.5}}
\multiput(0,0)(15,0){3}{\circle*{1.5}}
\multiput(30,0)(-7.5,12.5){3}{\circle*{1.5}}
\put(0,0){\line(1,0){30}}
\put(0,0){\line(3,5){15}}
\put(30,0){\line(-3,5){15}}
\end{picture}
}
\put(80,10){
b)$\qquad$
\begin{picture}(30,30)
\multiput(0,0)(12.5,0){3}{\circle*{1.5}}
\multiput(0,12.5)(12.5,0){3}{\circle*{1.5}}
\multiput(0,25)(12.5,0){3}{\circle*{1.5}}
\multiput(0,0)(0,12.5){3}{\line(1,0){25}}
\multiput(0,0)(12.5,0){3}{\line(0,1){25}}
\end{picture}
}
\end{picture}
\caption{ }
\label{boo2h2d4}
\end{figure}
For instance,
the  Greechie diagram drawn in Fig. \ref{boo2h2d4}a)
does not define an OMP,
the Greechie diagram drawn in Fig. \ref{boo2h2d4}b)
defines an OMP, which is not a lattice.

\begin{defin}
Let $L$ be a Greechie logic.
Let $X$ be the set of all atoms of $L$ and let ${\frak B}$ the system of all
blocks of $L$.
A subset $W \subseteq X$ is called a weight (on the Greechie diagram) if
$|W \cap B| = 1$ for any block $B \in {\frak B}$
($|A|$ denotes the cardinal number of the set $A$).
\end{defin}

$W(X)$ will denote the set of all weights on $L$.

\begin{lemma}
\label{weight}
Let $L$ be a Greechie logic and let $X$ be the set of its atoms.
Let $\varphi: P(L) \rightarrow W(X)$ be the mapping defined by
the formula $\varphi(P) = \{x \in X\mid x \not\in P \}$.
Then, $\varphi$ is an isomorphism of sets.
\end{lemma}

Proof.
$\varphi(P)$ is a weight on $X$ for any $P \in P(L)$.
The mapping $\varphi: P(L) \rightarrow W(X)$ is injective.
To show that $\varphi$ is also surjective, take a weight $W \in W(X)$.
Put $P = \{a \in L\mid \mbox{there exists a } x \in W \mbox{ such that }
a \le x'\}$.
A routine check yields that $P$ is a prime ideal of $L$.
Since $\varphi(P) = W$, the proof is completed. \\

We may use the one-to-one correspondence between prime ideals
and weights to construct OMPs with special properties.

\begin{figure}
\unitlength=1mm
\begin{picture}(60,30)(0,10)

\multiput(20,10)(10,00){4}{\circle*{1.5}}
\multiput(20,20)(10,00){4}{\circle*{1.5}}
\multiput(20,30)(10,00){4}{\circle*{1.5}}

\put(20,10){\line(1,0){30}}
\put(20,20){\line(1,0){30}}
\put(20,30){\line(1,0){30}}

\put(20,10){\line(0,1){20}}
\put(30,10){\line(0,1){20}}
\put(40,10){\line(0,1){20}}
\put(50,10){\line(0,1){20}}

\put(21,32){$a_{1}$}
\put(31,32){$a_{2}$}
\put(41,32){$a_{3}$}
\put(51,32){$a_{4}$}
\put(21,22){$a_{5}$}
\put(31,22){$a_{6}$}
\put(41,22){$a_{7}$}
\put(51,22){$a_{8}$}
\put(21,12){$a_{9}$}
\put(31,12){$a_{10}$}
\put(41,12){$a_{11}$}
\put(51,12){$a_{12}$}

\end{picture}
\caption{\label{omp34} $W_{3,4}$}
\end{figure}

\begin{figure}
\unitlength=1mm
\begin{picture}(140,40)
\unitlength=1mm
\put(00,00){
\begin{picture}(60,30)(0,0)

\multiput(20,10)(10,00){4}{\circle*{1.5}}
\multiput(20,20)(10,00){4}{\circle*{1.5}}
\multiput(20,30)(10,00){4}{\circle*{1.5}}

\linethickness{0.5mm}
\put(20,10){\line(1,0){30}}
\put(20,20){\line(1,0){30}}
\put(20,30){\line(1,0){30}}

\thinlines
\put(20,10){\line(0,1){20}}
\put(30,10){\line(0,1){20}}
\put(40,10){\line(0,1){20}}
\put(50,10){\line(0,1){20}}

\put(13,09){$B_3$}
\put(13,19){$B_2$}
\put(13,29){$B_1$}

\put(33,00){a)}

\end{picture}
}


\put(70,00){
\unitlength=1mm
\begin{picture}(60,30)(0,00)

\multiput(20,10)(10,00){4}{\circle*{1.5}}
\multiput(20,20)(10,00){4}{\circle*{1.5}}
\multiput(20,30)(10,00){4}{\circle*{1.5}}

\thinlines
\put(20,10){\line(1,0){30}}
\put(20,20){\line(1,0){30}}
\put(20,30){\line(1,0){30}}

\linethickness{0.5mm}
\put(20,10){\line(0,1){20}}
\put(30,10){\line(0,1){20}}
\put(40,10){\line(0,1){20}}
\put(50,10){\line(0,1){20}}

\put(19,32){$B_a$}
\put(29,32){$B_b$}
\put(39,32){$B_c$}
\put(49,32){$B_d$}

\put(33,00){b)}

\end{picture}
}

\end{picture}
\caption{\label{omp34b} Two disjoint coverings of $W_{3,4}$}
\end{figure}

Consider the Greechie diagram of Fig.~\ref{omp34}.
According to Theorem \ref{thgreechie} the associated
Greechie logic is an OMP, termed $W_{3,4}$ by Greechie.
We propose that it possesses no prime ideals.
Consider Fig.~\ref{omp34b}.
In these figures the bold lines indicate a disjoint covering of $W_{3,4}$
by its blocks.
In Fig.~\ref{omp34b}.a, the covering consists of 3 blocks $B_1,B_2$ and
$B_3$. Therefore, \\
$|W| = |W \cap X| = |W \cap (B_1 \cup B_2 \cup B_3)| =
|W \cap B_1| + |W \cap B_2| + |W \cap B_3| = 3$
for any $W \in W(X)$.
In Fig.~\ref{omp34b}.b, there is a  disjoint covering consisting of 4
blocks
$B_a,B_b,B_c$ and $B_d$, and, therefore $|W| = 4$ for any $W \in W(X)$.
This is a contradiction.
Hence,
there is no weight and no
prime ideal on $W_{3,4}$.

The latter fact is also seen by the following simple reasoning.
Assume that the OMP $W_{3,4}$ is isomorphic to a partition logic $(M,{\frak R})$.
Let $x \in M$.
$x$ has to be element in one of the atoms of $B_a$.
Without loss of generality we may assume that $x \in a_1$.
$x$ has to be element in one of the atoms of $B_b$.
Since $x\in a_1$, $x \in a_2$ is not possible,
because $a_1,a_2$ are atoms of the same block $B_1$.
Without loss of generality we may assume that $x \in a_6$.
$x$ has to be element in one of the atoms of $B_c$.
The only choice left is $x \in a_{11}$.
$x$ has to be element in one of the atoms of $B_d$.
But every choice $x \in a_4,x \in a_8$ or $x \in a_{12}$ is in
contradiction to $x \in a_1,x \in a_6$ and $x \in a_{11}$, respectively.
Therefore, the OMP $W_{3,4}$ is not isomorphic to a partition logic.
Furthermore, there exist OMLs such that $P(L) = \emptyset$.
An example is the ``spider'' lattice of Fig.~\ref{spider}
(cf.~\cite{ptak}, p.37).

\begin{figure}
\begin{center}
%TexCad Options
%\grade{\off}
%\emlines{\off}
%\beziermacro{\off}
%\reduce{\on}
%\snapping{\off}
%\quality{0.20}
%\graddiff{0.01}
%\snapasp{1}
%\zoom{1.00}
\unitlength 0.80mm
\linethickness{0.4pt}
\begin{picture}(115.00,99.00)
%\emline(40.00,10.00)(90.00,10.00)
\put(40.00,10.00){\line(1,0){50.00}}
%\end
%\emline(90.00,10.00)(115.00,53.00)
\multiput(90.00,10.00)(0.12,0.21){209}{\line(0,1){0.21}}
%\end
%\emline(115.00,53.00)(90.00,98.00)
\multiput(115.00,53.00)(-0.12,0.22){209}{\line(0,1){0.22}}
%\end
%\emline(90.00,98.00)(40.00,98.00)
\put(90.00,98.00){\line(-1,0){50.00}}
%\end
%\emline(40.00,98.00)(15.00,53.00)
\multiput(40.00,98.00)(-0.12,-0.22){209}{\line(0,-1){0.22}}
%\end
%\emline(15.00,53.00)(40.00,10.00)
\multiput(15.00,53.00)(0.12,-0.21){209}{\line(0,-1){0.21}}
%\end
%\emline(46.00,20.00)(84.33,20.00)
\put(46.00,20.00){\line(1,0){38.33}}
%\end
%\emline(84.33,20.00)(103.67,53.00)
\multiput(84.33,20.00)(0.12,0.20){162}{\line(0,1){0.20}}
%\end
%\emline(103.67,53.00)(84.33,88.00)
\multiput(103.67,53.00)(-0.12,0.22){162}{\line(0,1){0.22}}
%\end
%\emline(84.33,88.00)(45.67,88.00)
\put(84.33,88.00){\line(-1,0){38.66}}
%\end
%\emline(45.67,88.00)(25.67,53.00)
\multiput(45.67,88.00)(-0.12,-0.21){167}{\line(0,-1){0.21}}
%\end
%\emline(25.67,53.00)(45.33,20.00)
\multiput(25.67,53.00)(0.12,-0.20){164}{\line(0,-1){0.20}}
%\end
%\emline(51.00,30.00)(78.67,30.00)
\put(51.00,30.00){\line(1,0){27.67}}
%\end
%\emline(78.67,30.00)(91.67,53.00)
\multiput(78.67,30.00)(0.12,0.21){109}{\line(0,1){0.21}}
%\end
%\emline(91.67,53.00)(78.33,78.00)
\multiput(91.67,53.00)(-0.12,0.22){112}{\line(0,1){0.22}}
%\end
%\emline(78.33,78.00)(51.67,78.00)
\put(78.33,78.00){\line(-1,0){26.66}}
%\end
%\emline(51.67,78.00)(37.33,53.00)
\multiput(51.67,78.00)(-0.12,-0.21){120}{\line(0,-1){0.21}}
%\end
%\emline(37.33,53.00)(51.00,30.00)
\multiput(37.33,53.00)(0.12,-0.20){114}{\line(0,-1){0.20}}
%\end
%\emline(27.00,74.67)(44.33,65.00)
\multiput(27.00,74.67)(0.21,-0.12){81}{\line(1,0){0.21}}
%\end
%\emline(103.33,74.33)(85.33,65.33)
\multiput(103.33,74.33)(-0.24,-0.12){76}{\line(-1,0){0.24}}
%\end
\put(64.83,48.83){\oval(15.67,20.33)[t]}
%\emline(28.00,30.33)(57.00,49.00)
\multiput(28.00,30.33)(0.19,0.12){156}{\line(1,0){0.19}}
%\end
%\emline(101.67,30.67)(72.67,48.67)
\multiput(101.67,30.67)(-0.19,0.12){150}{\line(-1,0){0.19}}
%\end
%\emline(65.00,98.00)(65.00,59.00)
\put(65.00,98.00){\line(0,-1){39.00}}
%\end
%\emline(65.00,10.67)(64.67,11.00)
\multiput(65.00,10.67)(-0.11,0.11){3}{\line(-1,0){0.11}}
%\end
%\emline(65.00,10.00)(65.00,30.00)
\put(65.00,10.00){\line(0,1){20.00}}
%\end
\put(40.00,98.00){\circle*{2.00}}
\put(65.00,98.00){\circle*{2.00}}
\put(90.00,98.00){\circle*{2.00}}
\put(65.00,88.00){\circle*{2.00}}
\put(84.00,88.00){\circle*{2.00}}
\put(46.00,88.00){\circle*{2.00}}
\put(52.00,78.00){\circle*{2.00}}
\put(65.00,78.00){\circle*{2.00}}
\put(78.00,78.00){\circle*{2.00}}
\put(103.00,74.00){\circle*{2.00}}
\put(94.00,70.00){\circle*{2.00}}
\put(85.00,65.00){\circle*{2.00}}
\put(44.00,65.00){\circle*{2.00}}
\put(35.00,70.00){\circle*{2.00}}
\put(27.00,74.00){\circle*{2.00}}
\put(65.00,59.00){\circle*{2.00}}
\put(57.00,49.00){\circle*{2.00}}
\put(73.00,49.00){\circle*{2.00}}
\put(85.00,41.00){\circle*{2.00}}
\put(94.00,36.00){\circle*{2.00}}
\put(102.00,31.00){\circle*{2.00}}
\put(28.00,30.00){\circle*{2.00}}
\put(36.00,35.00){\circle*{2.00}}
\put(44.00,41.00){\circle*{2.00}}
\put(65.00,30.00){\circle*{2.00}}
\put(65.00,20.00){\circle*{2.00}}
\put(65.00,10.00){\circle*{2.00}}
\put(91.00,53.00){\circle*{2.00}}
\put(103.00,53.00){\circle*{2.00}}
\put(114.00,53.00){\circle*{2.00}}
\put(15.00,53.00){\circle*{2.00}}
\put(26.00,53.00){\circle*{2.00}}
\put(38.00,53.00){\circle*{2.00}}
\put(40.00,10.00){\circle*{2.00}}
\put(45.00,20.00){\circle*{2.00}}
\put(51.00,30.00){\circle*{2.00}}
\put(79.00,30.00){\circle*{2.00}}
\put(84.00,20.00){\circle*{2.00}}
\put(90.00,10.00){\circle*{2.00}}
\end{picture}
\end{center}
\caption{\label{spider} The spider}
\end{figure}


\begin{figure}
\unitlength=1mm
\begin{picture}(50,50)
%square

\multiput(10,40)(15,0){3}{\circle*{1.5}}
\multiput(10,25)(15,0){3}{\circle*{1.5}}
\multiput(10,10)(15,0){3}{\circle*{1.5}}

\multiput(10,10)(00,15){3}{\line(1,0){30}}
\multiput(10,10)(15,00){3}{\line(0,1){30}}

\put(11,42){$a_1$}
\put(26,42){$a_2$}
\put(41,42){$a_3$}
\put(11,27){$a_4$}
\put(26,27){$a_5$}
\put(41,27){$a_6$}
\put(11,12){$a_7$}
\put(26,12){$a_8$}
\put(41,12){$a_9$}

\end{picture}
\caption{\label{square1} Example 1}
\end{figure}

\begin{figure}
\unitlength=1mm
\begin{picture}(120,60)
%minisquares

\put(00,30){
\unitlength=1mm
\begin{picture}(30,30)
%minisquare1

\multiput(05,10)(0,7.5){3}{\line(1,0){15}}
\multiput(05,10)(7.5,0){3}{\line(0,1){15}}

\put(05.0,25.0){\circle*{1.5}} %1
\put(12.5,17.5){\circle*{1.5}} %5
\put(20.0,10.0){\circle*{1.5}} %9

\small
\put(00,03){$W_1 = \{a_1,a_5,a_9\}$}

\end{picture}
}

\put(40,30){
\unitlength=1mm
\begin{picture}(30,30)
%minisquare2

\multiput(05,10)(0,7.5){3}{\line(1,0){15}}
\multiput(05,10)(7.5,0){3}{\line(0,1){15}}

\put(05.0,25.0){\circle*{1.5}} %1
\put(20.0,17.5){\circle*{1.5}} %6
\put(12.5,10.0){\circle*{1.5}} %8

\small
\put(00,03){$W_2 = \{a_1,a_6,a_8\}$}

\end{picture}
}

\put(80,30){
\unitlength=1mm
\begin{picture}(30,30)
%minisquare3

\multiput(05,10)(0,7.5){3}{\line(1,0){15}}
\multiput(05,10)(7.5,0){3}{\line(0,1){15}}

\put(12.5,25.0){\circle*{1.5}} %2
\put(05.0,17.5){\circle*{1.5}} %4
\put(20.0,10.0){\circle*{1.5}} %9

\small
\put(00,03){$W_3 = \{a_2,a_4,a_9\}$}

\end{picture}
}

\put(00,00){
\unitlength=1mm
\begin{picture}(30,30)
%minisquare4

\multiput(05,10)(0,7.5){3}{\line(1,0){15}}
\multiput(05,10)(7.5,0){3}{\line(0,1){15}}

\put(20.0,25.0){\circle*{1.5}} %3
\put(12.5,17.5){\circle*{1.5}} %5
\put(05.0,10.0){\circle*{1.5}} %7

\small
\put(00,03){$W_4 = \{a_3,a_5,a_7\}$}

\end{picture}
}

\put(40,00){
\unitlength=1mm
\begin{picture}(30,30)
%minisquare5

\multiput(05,10)(0,7.5){3}{\line(1,0){15}}
\multiput(05,10)(7.5,0){3}{\line(0,1){15}}

\put(20.0,25.0){\circle*{1.5}} %3
\put(05.0,17.5){\circle*{1.5}} %4
\put(12.5,10.0){\circle*{1.5}} %8

\small
\put(00,03){$W_5 = \{a_3,a_4,a_8\}$}

\end{picture}
}

\put(80,00){
\unitlength=1mm
\begin{picture}(30,30)
%minisquare6

\multiput(05,10)(0,7.5){3}{\line(1,0){15}}
\multiput(05,10)(7.5,0){3}{\line(0,1){15}}

\put(12.5,25.0){\circle*{1.5}} %2
\put(20.0,17.5){\circle*{1.5}} %6
\put(05.0,10.0){\circle*{1.5}} %7

\small
\put(00,03){$W_6 = \{a_2,a_6,a_7\}$}

\end{picture}
}

\end{picture}
\caption{\label{square2} The weights of Example 1}
\end{figure}

\begin{figure}
\unitlength=1mm
\begin{picture}(50,50)
%square-partition logic

\multiput(10,40)(15,0){3}{\circle*{1.5}}
\multiput(10,25)(15,0){3}{\circle*{1.5}}
\multiput(10,10)(15,0){3}{\circle*{1.5}}

\multiput(10,10)(00,15){3}{\line(1,0){30}}
\multiput(10,10)(15,00){3}{\line(0,1){30}}

\small
\put(11,42){$\{1,2\}$}
\put(26,42){$\{3,4\}$}
\put(41,42){$\{5,6\}$}
\put(11,27){$\{3,6\}$}
\put(26,27){$\{1,5\}$}
\put(41,27){$\{2,4\}$}
\put(11,12){$\{4,5\}$}
\put(26,12){$\{2,6\}$}
\put(41,12){$\{1,3\}$}

\end{picture}
\caption{\label{square3} The isomorphic partition logic to example 1}
\end{figure}

If a Greechie logic is prime (rich), we may also use the
one-to-one correspondence between prime ideals and weights to
construct the isomorphic partition logic
(the isomorphic concrete logic).
We give two examples.
Consider the Greechie diagram of the OMP $L$,
drawn in Fig.~\ref{square1}.
Let $X$ be the set of its atoms $\{a_1, \ldots,a_9\}$.
$L$ possesses 6 weights, $W(X) = \{W_1, \ldots,W_6\}$; see Fig.~\ref{square2}.
According to Lemma \ref{weight},
instead the mapping $p: L \rightarrow {\frak P}(P(L))$
the mapping
$q: X \rightarrow {\frak P}(W(X))$ is used.
$q$ is defined by
$q(a) = \{ W \in W(X) \mid a \in W\}$ for all $a \in X$.
For instance, $q(a_1) = \{W_1,W_2\}$.
We obtain the partition logic of Fig.~\ref{square3}.
(The numbers denote the corresponding weights.)
A check of the axioms in Definition \ref{concrete} shows
that $L$ is a concrete logic.

\begin{figure}

\unitlength=1mm
\begin{picture}(160,50)(00,00)
\small

\multiput(65,10)(00,15){3}{\circle*{2}}
\multiput(35,10)(-7.5,7.5){3}{\circle*{2}}
\multiput(20,25)(7.5,7.5){3}{\circle*{2}}
\multiput(95,10)(7.5,7.5){3}{\circle*{2}}
\multiput(110,25)(-7.5,7.5){3}{\circle*{2}}

\put(35,10){\line(1,0){60}}
\put(35,40){\line(1,0){60}}
\put(65,10){\line(0,1){30}}
\put(35,10){\line(-1,1){15}}
\put(20,25){\line(1,1){15}}
\put(95,10){\line(1,1){15}}
\put(110,25){\line(-1,1){15}}

\put(15,24){1}
\put(22.5,31.5){2}
\put(30,39){3}
\put(65,42){4}
\put(100,39){5}
\put(107.5,31.5){6}
\put(115,24){7}
\put(107.5,16.5){8}
\put(100,9){9}
\put(65,5){10}
\put(29,9){11}
\put(21.5,16.5){12}
\put(67,24){13}

\end{picture}

\caption{\label{ex21} Example 2}

\end{figure}

\begin{figure}

\unitlength=1mm
\begin{picture}(135,175)

\put(00,140){
\unitlength=1mm
\begin{picture}(40,30)
%W1
\put(07.5,17.5){\line(1,1){7.5}}
\put(07.5,17.5){\line(1,-1){7.5}}
\put(15.0,10.0){\line(1,0){15}}
\put(15.0,25.0){\line(1,0){15}}
\put(22.5,10.0){\line(0,1){15}}
\put(37.5,17.5){\line(-1,1){7.5}}
\put(37.5,17.5){\line(-1,-1){7.5}}

\put(07.5,17.5){\circle*{1.5}} %1
\put(30.0,25.0){\circle*{1.5}} %5
\put(30,10){\circle*{1.5}} %9
\put(22.5,17.5){\circle*{1.5}} %13

\put(5,3){$W_1=\{1,5,9,13\}$}
\end{picture}
}

\put(45,140){
\unitlength=1mm
\begin{picture}(40,30)
%W2
\put(07.5,17.5){\line(1,1){7.5}}
\put(07.5,17.5){\line(1,-1){7.5}}
\put(15.0,10.0){\line(1,0){15}}
\put(15.0,25.0){\line(1,0){15}}
\put(22.5,10.0){\line(0,1){15}}
\put(37.5,17.5){\line(-1,1){7.5}}
\put(37.5,17.5){\line(-1,-1){7.5}}

\put(07.5,17.5){\circle*{1.5}} %1
\put(22.5,25.0){\circle*{1.5}} %4
\put(33.75,21.25){\circle*{1.5}} %6
\put(30,10){\circle*{1.5}} %9

\put(5,3){$W_2=\{1,4,6,9\}$}
\end{picture}
}

\put(90,140){
\unitlength=1mm
\begin{picture}(40,30)
%W3
\put(07.5,17.5){\line(1,1){7.5}}
\put(07.5,17.5){\line(1,-1){7.5}}
\put(15.0,10.0){\line(1,0){15}}
\put(15.0,25.0){\line(1,0){15}}
\put(22.5,10.0){\line(0,1){15}}
\put(37.5,17.5){\line(-1,1){7.5}}
\put(37.5,17.5){\line(-1,-1){7.5}}

\put(07.5,17.5){\circle*{1.5}} %1
\put(30.0,25.0){\circle*{1.5}} %5
\put(33.75,13.75){\circle*{1.5}} %8
\put(22.5,10){\circle*{1.5}} %10

\put(5,3){$W_3=\{1,5,8,10\}$}
\end{picture}
}

\put(00,105){
\unitlength=1mm
\begin{picture}(40,30)
%W4
\put(07.5,17.5){\line(1,1){7.5}}
\put(07.5,17.5){\line(1,-1){7.5}}
\put(15.0,10.0){\line(1,0){15}}
\put(15.0,25.0){\line(1,0){15}}
\put(22.5,10.0){\line(0,1){15}}
\put(37.5,17.5){\line(-1,1){7.5}}
\put(37.5,17.5){\line(-1,-1){7.5}}

\put(11.25,21.25){\circle*{1.5}} %2
\put(30.0,25.0){\circle*{1.5}} %5
\put(30,10){\circle*{1.5}} %9
\put(11.25,13.75){\circle*{1.5}} %12
\put(22.5,17.5){\circle*{1.5}} %13

\put(5,3){$W_4=\{2,5,9,12,13\}$}
\end{picture}
}
\put(45,105){
\unitlength=1mm
\begin{picture}(40,30)
%W5
\put(07.5,17.5){\line(1,1){7.5}}
\put(07.5,17.5){\line(1,-1){7.5}}
\put(15.0,10.0){\line(1,0){15}}
\put(15.0,25.0){\line(1,0){15}}
\put(22.5,10.0){\line(0,1){15}}
\put(37.5,17.5){\line(-1,1){7.5}}
\put(37.5,17.5){\line(-1,-1){7.5}}

\put(11.25,21.25){\circle*{1.5}} %2
\put(30.0,25.0){\circle*{1.5}} %5
\put(33.75,13.75){\circle*{1.5}} %8
\put(15,10){\circle*{1.5}} %11
\put(22.5,17.5){\circle*{1.5}} %13

\put(5,3){$W_5=\{2,5,8,11,13\}$}
\end{picture}
}
\put(90,105){
\unitlength=1mm
\begin{picture}(40,30)
%W6
\put(07.5,17.5){\line(1,1){7.5}}
\put(07.5,17.5){\line(1,-1){7.5}}
\put(15.0,10.0){\line(1,0){15}}
\put(15.0,25.0){\line(1,0){15}}
\put(22.5,10.0){\line(0,1){15}}
\put(37.5,17.5){\line(-1,1){7.5}}
\put(37.5,17.5){\line(-1,-1){7.5}}

\put(11.25,21.25){\circle*{1.5}} %2
\put(22.5,25.0){\circle*{1.5}} %4
\put(33.75,21.25){\circle*{1.5}} %6
\put(30,10){\circle*{1.5}} %9
\put(11.25,13.75){\circle*{1.5}} %12

\put(5,3){$W_6=\{2,4,6,9,12\}$}
\end{picture}
}
\put(00,70){
\unitlength=1mm
\begin{picture}(40,30)
%W7
\put(07.5,17.5){\line(1,1){7.5}}
\put(07.5,17.5){\line(1,-1){7.5}}
\put(15.0,10.0){\line(1,0){15}}
\put(15.0,25.0){\line(1,0){15}}
\put(22.5,10.0){\line(0,1){15}}
\put(37.5,17.5){\line(-1,1){7.5}}
\put(37.5,17.5){\line(-1,-1){7.5}}

\put(11.25,21.25){\circle*{1.5}} %2
\put(22.5,25.0){\circle*{1.5}} %4
\put(37.5,17.5){\circle*{1.5}} %7
\put(15,10){\circle*{1.5}} %11

\put(5,3){$W_7=\{2,4,7,11\}$}
\end{picture}
}

\put(45,70){
\unitlength=1mm
\begin{picture}(40,30)
%W8
\put(07.5,17.5){\line(1,1){7.5}}
\put(07.5,17.5){\line(1,-1){7.5}}
\put(15.0,10.0){\line(1,0){15}}
\put(15.0,25.0){\line(1,0){15}}
\put(22.5,10.0){\line(0,1){15}}
\put(37.5,17.5){\line(-1,1){7.5}}
\put(37.5,17.5){\line(-1,-1){7.5}}

\put(11.25,21.25){\circle*{1.5}} %2
\put(22.5,25.0){\circle*{1.5}} %4
\put(33.75,21.25){\circle*{1.5}} %6
\put(33.75,13.75){\circle*{1.5}} %8
\put(15,10){\circle*{1.5}} %11

\put(5,3){$W_8=\{2,4,6,8,11\}$}
\end{picture}
}

\put(90,70){
\unitlength=1mm
\begin{picture}(40,30)
%W9
\put(07.5,17.5){\line(1,1){7.5}}
\put(07.5,17.5){\line(1,-1){7.5}}
\put(15.0,10.0){\line(1,0){15}}
\put(15.0,25.0){\line(1,0){15}}
\put(22.5,10.0){\line(0,1){15}}
\put(37.5,17.5){\line(-1,1){7.5}}
\put(37.5,17.5){\line(-1,-1){7.5}}

\put(11.25,21.25){\circle*{1.5}} %2
\put(30.0,25.0){\circle*{1.5}} %5
\put(33.75,13.75){\circle*{1.5}} %8
\put(22.5,10){\circle*{1.5}} %10
\put(11.25,13.75){\circle*{1.5}} %12

\put(5,3){$W_9=\{2,5,8,10,12\}$}
\end{picture}
}

\put(00,35){
\unitlength=1mm
\begin{picture}(40,30)
%W10
\put(07.5,17.5){\line(1,1){7.5}}
\put(07.5,17.5){\line(1,-1){7.5}}
\put(15.0,10.0){\line(1,0){15}}
\put(15.0,25.0){\line(1,0){15}}
\put(22.5,10.0){\line(0,1){15}}
\put(37.5,17.5){\line(-1,1){7.5}}
\put(37.5,17.5){\line(-1,-1){7.5}}

\put(15.0,25.0){\circle*{1.5}} %3
\put(37.5,17.5){\circle*{1.5}} %7
\put(15,10){\circle*{1.5}} %11
\put(22.5,17.5){\circle*{1.5}} %13

\put(5,3){$W_{10}=\{3,7,11,13\}$}
\end{picture}
}

\put(45,35){
\unitlength=1mm
\begin{picture}(40,30)
%W11
\put(07.5,17.5){\line(1,1){7.5}}
\put(07.5,17.5){\line(1,-1){7.5}}
\put(15.0,10.0){\line(1,0){15}}
\put(15.0,25.0){\line(1,0){15}}
\put(22.5,10.0){\line(0,1){15}}
\put(37.5,17.5){\line(-1,1){7.5}}
\put(37.5,17.5){\line(-1,-1){7.5}}

\put(15.0,25.0){\circle*{1.5}} %3
\put(33.75,21.25){\circle*{1.5}} %6
\put(33.75,13.75){\circle*{1.5}} %8
\put(15,10){\circle*{1.5}} %11
\put(22.5,17.5){\circle*{1.5}} %13

\put(5,3){$W_{11}=\{3,6,8,11,13\}$}
\end{picture}
}

\put(90,35){
\unitlength=1mm
\begin{picture}(40,30)
%W12
\put(07.5,17.5){\line(1,1){7.5}}
\put(07.5,17.5){\line(1,-1){7.5}}
\put(15.0,10.0){\line(1,0){15}}
\put(15.0,25.0){\line(1,0){15}}
\put(22.5,10.0){\line(0,1){15}}
\put(37.5,17.5){\line(-1,1){7.5}}
\put(37.5,17.5){\line(-1,-1){7.5}}

\put(15.0,25.0){\circle*{1.5}} %3
\put(33.75,21.25){\circle*{1.5}} %6
\put(30,10){\circle*{1.5}} %9
\put(11.25,13.75){\circle*{1.5}} %12
\put(22.5,17.5){\circle*{1.5}} %13

\put(5,3){$W_{12}=\{3,6,9,12,13\}$}
\end{picture}
}

\put(00,00){
\unitlength=1mm
\begin{picture}(40,30)
%W13
\put(07.5,17.5){\line(1,1){7.5}}
\put(07.5,17.5){\line(1,-1){7.5}}
\put(15.0,10.0){\line(1,0){15}}
\put(15.0,25.0){\line(1,0){15}}
\put(22.5,10.0){\line(0,1){15}}
\put(37.5,17.5){\line(-1,1){7.5}}
\put(37.5,17.5){\line(-1,-1){7.5}}

\put(15.0,25.0){\circle*{1.5}} %3
\put(37.5,17.5){\circle*{1.5}} %7
\put(22.5,10){\circle*{1.5}} %10
\put(11.25,13.75){\circle*{1.5}} %12

\put(5,3){$W_{13}=\{3,7,10,12\}$}
\end{picture}
}

\put(45,00){
\unitlength=1mm
\begin{picture}(40,30)
%W14
\put(07.5,17.5){\line(1,1){7.5}}
\put(07.5,17.5){\line(1,-1){7.5}}
\put(15.0,10.0){\line(1,0){15}}
\put(15.0,25.0){\line(1,0){15}}
\put(22.5,10.0){\line(0,1){15}}
\put(37.5,17.5){\line(-1,1){7.5}}
\put(37.5,17.5){\line(-1,-1){7.5}}

\put(15.0,25.0){\circle*{1.5}} %3
\put(33.75,21.25){\circle*{1.5}} %6
\put(33.75,13.75){\circle*{1.5}} %8
\put(22.5,10){\circle*{1.5}} %10
\put(11.25,13.75){\circle*{1.5}} %12

\put(5,3){$W_{14}=\{3,6,8,10,12\}$}
\end{picture}
}

\end{picture}
\caption{\label{ex22} The weights of Example 2}
\end{figure}


\begin{figure}

\unitlength=1mm
\begin{picture}(160,70)(10,10)
\small

\multiput(80,20)(00,20){3}{\circle*{2}}
\multiput(50,20)(-10,10){3}{\circle*{2}}
\multiput(30,40)(10,10){3}{\circle*{2}}
\multiput(110,20)(10,10){3}{\circle*{2}}
\multiput(130,40)(-10,10){3}{\circle*{2}}

\put(50,20){\line(1,0){60}}
\put(50,60){\line(1,0){60}}
\put(80,20){\line(0,1){40}}
\put(50,20){\line(-1,1){20}}
\put(30,40){\line(1,1){20}}
\put(110,20){\line(1,1){20}}
\put(130,40){\line(-1,1){20}}

\put(35,62){\{10,11,12,13,14\}}
\put(70,62){\{2,6,7,8\}}
\put(100,62){\{1,3,4,5,9\}}
\put(15,49){\{4,5,6,7,8,9\}}
\put(125,49){\{2,6,8,11,12,14\}}
\put(15,39){\{1,2,3\}}
\put(133,39){\{7,10,13\}}
\put(83,39){\{1,4,5,10,11,12\}}
\put(10,29){\{4,6,9,12,13,14\}}
\put(125,29){\{3,5,8,9,11,14\}}
\put(35,15){\{5,7,8,10,11\}}
\put(70,15){\{3,9,13,14\}}
\put(100,15){\{1,2,4,6,12\}}

\end{picture}

\caption{\label{ex23} The isomorphic partition logic to example 2}

\end{figure}

The second example describes a partition logic which is not a concrete logic.
The example is taken from \cite{ptak}, p.~39.
Consider the Greechie diagram of the OMP $L$, drawn in
drawn in Fig.~\ref{ex21},.
Let $X$ be the set if its atoms $\{a_1, \ldots, a_{13}\}$.
$L$ possesses 14 weights, $W(X) = \{W_1, \ldots, W_{14}\}$,
drawn in Fig.~\ref{ex22} (the numbers denote the atoms in a weight).
We obtain the partition logic of Fig.~\ref{ex23}
(the numbers denote the corresponding weights).
We propose that $L$ is not a concrete logic.
The disjoint sets $\{1,2,3\}$ and $\{7,10,13\}$ are both in $L$,
but not their union $\{1,2,3,7,10,13\}$
(we identify $L$ with its isomorphic partition logic).
Therefore, condition (iii) of Definition \ref{concrete}
is not satisfied, and, $L$ is not a concrete logic (cf.~\cite{ptak}, p.~39).

\section{Automata Theory}

More detailed introductions to automata theory can be found in
\cite{booth,brauer,conway,hopcroft}.

\subsection{Basic Definitions}

An {\em alphabet} is a finite nonvoid set.
The elements of an alphabet are called {\em symbols}.
A {\em word} (or {\em string}) is a finite (possibly empty) sequence of
symbols.
The {\em length} of a word $w$, denoted by $|w|$, is the number of symbols
composing the string.
The {\em empty word} is denoted by $\epsilon$.
$\Sigma^*$ denotes the set of all words over an alphabet $\Sigma$.
The {\em concatenation} of two words is the word formed by writing
the first, followed by the second, with no intervening space.
Let $\Sigma$ be an alphabet.
$\Sigma^*$ with the concatenation as operation forms a monoid,
where the empty word $\epsilon$ is the identity.
A ({\em formal}) {\em language} over an alphabet $\Sigma$ is a
subset of $\Sigma^*$.

\begin{defin}
\label{moore}
A Moore automaton $M$ is a five-tuple $M=(Q,\Sigma,\Delta,\delta,\lambda)$,
where

(i) $Q$ is a finite set, called the set of states;

(ii) $\Sigma$ is an alphabet, called the input alphabet;

(iii) $\Delta$ is an alphabet, called the output alphabet;

(iv) $\delta$ is a mapping $Q \times \Sigma$ to $Q$, called
the transition function;

(v) $\lambda$ is a mapping $Q$ to $\Delta$, called the output function.
\end{defin}

Let us sketch the appropriate picture informally.
At any time the automaton is in a state $q \in Q$,
emitting the output $\lambda(q) \in \Delta$.
If an input $a \in \Sigma$ is applied to the automaton,
in the next discrete time step
the automaton instantly assumes the state $p = \delta(q,a)$ and emits
the output $\lambda(p)$.

\begin{defin}
A Mealy automaton is a five-tuple $M=(Q,\Sigma,\Delta,\delta,\lambda)$
where $Q,\Sigma,\Delta,\delta$ are as in the Moore automaton and
$\lambda$ is a mapping from $Q \times \Sigma$ to $\Delta$.
\end{defin}

A Mealy automaton emits the output at the instant of the transition
from one state to another. The output depends both on the previous
state and on the input.

We use directed graphs, called {\em transition diagrams}, to describe
Moore and Mealy automata.
The vertices of the graph correspond to the states of the automaton.
For a Moore automaton, every vertex is labeled by a pair
$(q/x),q \in Q,x \in \Delta$, where $q$ is the corresponding state of
the automaton and $x=\lambda(q)$ is the associated output with this state.
If there is a transition from state $q$ to state $p$ on input $a$,
then there is an arc labeled $a$ from state $q$ to state $p$ in the
transition diagram.
For Mealy automata, the vertices are labeled with the corresponding
state.
If there is a transition from state $q$ to state $p$ on input $a$,
then there is an arc from state $q$ to state $p$
labeled $(a,\lambda(p,a))$.
For example, Fig.~\ref{moorea} represents a Moore automaton and
Fig.~\ref{a2} represents a Mealy automaton.

To formally describe the behavior of a automaton, it is desirable
to extend the transition function $\delta$ to apply to a state and
an input {\em word}, rather than to a state and to a single symbol.
We define a mapping $\hat{\delta}$ from $Q \times \Sigma^*$ to $Q$.
We shall denote by
$\hat{\delta}(q,w)$ the state in which the automaton is
after reading $w$, starting from state $q$.
Formally, we define

(i) $\hat{\delta}(q,\epsilon) = q$, and

(ii) $\hat{\delta}(q,wa) = \delta(\hat{\delta}(q,w),a)$
for $w \in \Sigma^*$ and $a \in \Sigma$. \\
We also extend the output function $\lambda$ to a mapping
$\hat{\lambda}: Q \times \Sigma^* \rightarrow \Delta^*$.
Let $a_1, \ldots ,a_n \in \Sigma$. We define

$\hat{\lambda}(q,a_1 \cdots a_n) = \lambda(q) \lambda(\delta(q,a_1))
\lambda(\delta(q,a_1a_2)) \cdots \lambda(\delta(q,a_1 \cdots a_n))$ \\
for Moore automata and

$\hat{\lambda}(q,a_1 \cdots a_n) = \lambda(q,a_1)\lambda(\delta(q,a_1),a_2)
\cdots \lambda(\delta(q,a_1 \cdots a_{n-1}), a_n)$ \\
for Mealy automata.
$\hat{\lambda}(q,w)$ is the output sequence
obtained by applying an input sequence $a_1 \cdots a_n$.
Since $\hat{\delta}(q,a) = \delta(q,a)$ and
$\hat{\lambda}(q,a) = \lambda(q,a)$ for any input symbol $a$
(i.e.,. $\hat{\lambda}(q) = \lambda(q)$), we may
again use $\delta$ (i.e.,.~$\lambda$) in place of $\hat{\delta}$
(i.e.,.~$\hat{\lambda}$).
Note that for a word $w$ with $|w| = n$,  the length of the output
sequence is $n +1 $ for a Moore automaton and $n$ for a Mealy automaton.


Let $p,q$ be any two states belonging to the state set $Q$.
Then, $p$ is {\em equivalent} to ({\em indistinguishable} from) $q$,
written as $p \equiv q$ iff $\lambda(p,w) = \lambda(q,w)$
for all possible words $w \in \Sigma^*$. Otherwise the states
are said to be {\em distinguishable}.
We call an automaton {\em minimal} if any two states of the automaton
are distinguishable.
We say that a word $w \in \Sigma^*$ distinguishes the two states
$p$ and $q$ if $\lambda(q,w) \neq \lambda(p,w)$.
A somewhat weaker equivalence property is that of {\em k-equivalence}.
For each positive integer $k$ we say that $p$ is $k$-equivalent
to state $q$, written as $p \stackrel{k}{\equiv} q$,
iff $\lambda(p,w) = \lambda(q,w)$ for all input sequences $w \in \Sigma^*$
of length $k$.
Both, equivalence $\equiv$ and $k$-equivalence $\stackrel{k}{\equiv}$
are equivalence relations that obey the reflexive, symmetric and transitive
laws.
We denote the partition corresponding to $\equiv$ by $Q/\equiv$
and the partition corresponding to $\stackrel{k}{\equiv}$ by
$Q/\stackrel{k}{\equiv}$.

\begin{theorem}
(i) (Moore)
Let $M=(Q,\Delta,\Sigma,\delta,\lambda)$ be a Moore automaton with $n$
states and $m$ outputs.
Further, let $\lambda$ be onto.
Then, two distinguishable states can be distinguished by some word
of length at most $n-m$.

(ii) (Huffman/Mealy)
Two distinguishable states of a Mealy automaton  with $n$ states
can be distinguished by some word of length at most $n-1$.
\end{theorem}

Proof.
(i) We denote by $f(k)$ the number of equivalence classes of
$\stackrel{k}{\equiv}$ and by $f(\infty)$ the number of
equivalence classes of $\equiv$.
Then, plainly \\
$m = f(0) \le f(1) \le f(2) \le \ldots \le f(\infty) \le n$ \\
and so we can define $N$ as the least $k$ with $f(k) = f(k+1)$.
We propose $f(N) = f(N+1) = f(N+2) = \ldots = f(\infty)$.
$p \stackrel{N+1}{\equiv} q$ implies
$\delta(p,a) \stackrel{N}{\equiv}
\delta(q,a)$
for all $a \in \Sigma$ and therefore also
$\delta(p,a) \stackrel{N+1}{\equiv}
\delta(q,a)$ (using $f(N) = f(N+1)$).
Together, with $\lambda(p) = \lambda(q)$ we obtain
$p \stackrel{N+2}{\equiv}
q$, proving the equality chain above. \\
$m = f(0) < f(1) < \ldots < f(N) = f(\infty) \le n$
implies $m + N \le n$ and any two distinguishable states are
distinguishable by a word of length at most $N \le n-m$.

(ii) The proof is analogous to (i). \\


Let $M_1=(Q_1,\Sigma,\Delta,\delta_1,\lambda_1)$,
$M_2=(Q_2,\Sigma,\Delta,\delta_2,\lambda_2)$ be two automata of the
same type (both are either Moore or Mealy automata).
A state $q_1 \in Q_1$ is said to be {\em equivalent} to
a state $q_2 \in Q_2$ iff $\lambda_1(q_1,w) = \lambda_2(q_2,w)$
for all $w \in \Sigma^*$.
The two automata $M_1$ and $M_2$ are said to be {\em equivalent},
if for each state $q_1 \in Q_1$ there exists an equivalent state $q_2 \in Q_2$,
and, conversely, for each state $q_2 \in Q_2$ there exists an equivalent state
$q_1 \in Q_1$.

\begin{theorem}
Let $M = (Q, \Sigma, \Delta ,\delta, \lambda)$ be a Moore or
Mealy automaton.
Then, there exists a minimal automaton equivalent to $M$.
\end{theorem}

Proof.
Put $M^m = (Q/\equiv, \Sigma, \Delta, \delta^m, \lambda^m)$.
Define $\delta^m([q],a) = [\delta(q,a)]$ for all
$[q] \in Q/\equiv$ and all $a \in \Sigma$.
If $M$ is a Moore automaton, define
$\lambda^m([q]) = \lambda(q)$.
If $M$ is a Mealy automaton define
$\lambda^m([q],a) = \lambda(q,a)$.
According to the construction, $M^m$ is minimal.
Every state $q \in Q$ is equivalent
to the state $[q] \in Q/\equiv$.
Therefore, also $M$ and $M^m$ are equivalent.\\

Now, let $M_1$ be a Moore automaton and $M_2$ be a Mealy automaton.
There can never be equivalence in the above sense between
these automata because the output of a Moore automaton to the input
$w \in \Sigma^*$ contains one more symbol than the output of the
Mealy automaton.
However, we may neglect the first output symbol of a Moore automaton
$M=(Q,\Sigma,\Delta,\delta,\lambda)$
by using a reduced output function
$\lambda':Q\times \Sigma^* \rightarrow \Delta^*$
defined by \\
$\lambda'(q,a_1 \cdots a_n) = \lambda(\delta(q,a_1)) \cdots
\lambda(\delta(q,a_1 \cdots a_n))$. \\
Note that $\lambda(q,w) = \lambda(q)\lambda'(q,w)$.
We may prove the following equivalence theorems, equating the
Mealy and Moore models.

\begin{theorem}
If $M_1=(Q,\Sigma,\Delta,\delta,\lambda_1)$ is a Moore automaton, then there
exists a Mealy automaton $M_2$ equivalent to $M_1$.
\end{theorem}

Proof.
Put $M_2=(Q,\Sigma,\Delta,\delta,\lambda_2)$, where
$\lambda_2(q,a) = \lambda(\delta(q,a))$ for any $q \in Q$ and any
$a \in \Sigma$.
The two automata are equivalent.

\begin{theorem}
Let $M_1=(Q,\Sigma,\Delta,\delta_1,\lambda_1)$ be a Mealy automaton.
Then, there exists a Moore automaton $M_2$ equivalent to $M_1$.
\end{theorem}

Proof.
Put $M_2=(Q \times \Delta,\Sigma,\Delta,\delta_2,\lambda_2)$.
Define $\delta_2((q,x),a) = (\delta_1(q,a),\lambda(q,a))$ and
$\lambda_2((q,x))=x$ for any $(q,x) \in Q \times \Delta$
and $a \in \Sigma$.
Then, the states $q \in Q$ of $M_1$ and $(q,x) \in Q \times \Delta$, $x$ arbitrary,
of $M_2$ are equivalent. Therefore, also $M_1$ and $M_2$ are equivalent.

\subsection{Automata experiments}

In what follows we assume that we are dealing with a Moore or Mealy automaton,
which is contained in a black box
with input-output interface.
Thus, we are only allowed to observe
the input and output sequences associated with the box.
To conduct an experiment, the experimenter applies an input sequence
and notes the resulting output sequence.
Using this output sequence, the experimenter tries to interpret the information
contained in the sequence to determine the values of the
unknown parameters.
If there is enough information in the output sequence, the experimenter
will state
conclusions about the unknown parameters.
If, however, the results are inconclusive, the experimenter can decide to
extend the experiment by applying another input sequence to obtain
more information.
Alternatively the experimenter may terminate the experiment with the
conclusion that the desired parameter cannot be measured.

Two general types of problems have to be distinguished.
The first one deals with a situation in which very little
about the device is known
except that it is a Moore or Mealy automaton   with a given input set
and that it is one
particular automaton from a general class of automaton.
In this case, we are dealing with an {\em automaton identification
problem}.
To solve this problem we must determine the model that can be used to
describe the automaton's input-output behavior.

The second general class includes measurement and control problems.
In this case, we conduct experiments on an automaton with a known
transition table (i.e the five-tuple $(Q,\Sigma,\Delta,\delta,\lambda)$).
Here, we are interested in measuring and/or controlling
various parameters of the automaton.

The types of experiments that we can perform are limited by the number
of identical  copies of the automaton we have available for
investigation, the amount
of flexibility that we allow the experimenter, and the amount of a priori
information available about the automaton's internal behavior.
Usually, when we are carrying out an experiment, we assume that only a
{\em single} copy of the automaton is available.
Such an experiment is called a {\em simple experiment}.
On occasion, however, we have several identical copies of the automaton or
a single automaton with a ``reset'' button.
Experiments that take advantage of the availability of effectively more than
one copy
of an automaton are called {\em multiple experiments}.

The amount of flexibility that we allow the experimenter in selecting the
input sequences is an important consideration.
If the input sequence is fixed in advance, we say that the experimenter is
required to perform a {\em preset experiment}.
If the experimenter can modify the input sequence in response to information
gained from the output sequences, we call this an
{\em adaptive (branch) experiment} in which the input consists of a succession of
subsequences, each corresponding to a decision on the experimenter's part.

We shall describe two important measurement problems.
In the first, the {\em terminal state identification (homing) problem},
we are dealing with an automaton with an unknown initial state $q$.
The goal is to identify the final state of the automaton.
We apply an appropriate input sequence $w \in \Sigma^*$ and observe the
resulting output $\lambda(q,w)$.
On the basis of this observation we are able to specify the
terminal state $p= \delta(q,w)$.
The terminal-state identification problem is always solvable.

The {\em initial-state identification (diagnosing) problem}
deals with the problem of trying to determine the unknown initial
state of the automaton.
To solve this problem we apply an appropriate input word to the automaton
or we carry out an adaptive experiment.
From the observation of the corresponding output, we are able to
make propositions of the initial state.
Not all initial-state identification problems have unique solutions.
More exactly, there exist automata such that the initial state of the
automaton is not determinable.
The first automaton of this kind was invented to demonstrate that
particular feature by Moore
\cite{moore}.
It is quite remarkable that Moore's original motivation for the
introduction of Moore automata was the modeling of the Heisenberg
uncertainty principle.


\begin{figure}

\unitlength=1mm
\begin{picture}(70,60)(0,7)

\put(10,10){\circle*{2}}
\put(58,10){\circle*{2}}
\put(34,22){\circle*{2}}
\put(34,46){\circle*{2}}

\thicklines

\put(12,13){\vector(2,3){20}}
\put(56,10){\vector(-1,0){44}}
\put(56,11){\vector(-2,1){20}}
\put(34,24){\vector(0,1){20}}
\put(32,21){\vector(-2,-1){20}}
\put(36,43){\vector(2,-3){20}}

\small

\put(37,23){$q_1/0$}
\put(59,5){$q_2/0$}
\put(7,5){$q_3/0$}
\put(33,49){$q_4/1$}

\put(17,30){0,1}
\put(34,5){1}
\put(24,14){1}
\put(40,14){0}
\put(35,30){0}
\put(46,30){0,1}

\end{picture}
\caption{\label{moorea} Moore's uncertainty automaton}
\end{figure}

Consider the Moore automaton of Fig.~\ref{moorea}.
All four states are mutually distinguishable:
The first free output symbol distinguishes $q_4$, which has output 1,
from all other states, which have output 0. \\
To distinguish between $q_1$ and $q_2$ we apply the input 0
($\lambda(q_1,0) = 01$, $\lambda(q_2,0) =00$). \\
To distinguish between $q_1$ and $q_3$ we apply the input 1
($\lambda(q_1,1) = 00$, $\lambda(q_3,1) =01$). \\
To distinguish between $q_2$ and $q_3$ we apply the input 0
($\lambda(q_2,0) = 00$, $\lambda(q_2,0) =01$).

Nevertheless, the initial state is not determinable.
Any experiment which distinguishes between $q_1$ and $q_2$ cannot
distinguish between $q_1$ and $q_3$.
Conversely, any experiment  which distinguishes between $q_1$ and $q_3$
cannot distinguish between $q_1$ and $q_3$.
Note that any experiment which begins with the input 1
does not permit $q_1$ to be distinguished from $q_2$
(since in either case the first input is 0 and the second state is $q_3$,
so that no future inputs can produce different outputs).
Similarly, any experiment which begins with the input 0
does not permit $q_1$ to be distinguished from $q_3$.
Moore \cite{moore} speaks of an ``analogue of the
Heisenberg uncertainty principle,'' which was termed
``Moore's uncertainty Principle'' by Conway \cite{conway}.
D. Finkelstein and S. R. Finkelstein
have called this feature ``computational complementarity.''

Note that, as has already been pointed out by Moore,
if an arbitrary number of identical automaton copies
in the same initial state were available,
the initial-state problem would be solvable by multiple
experiments
for any minimal automaton.
In this setup,
for every pair $\{p,q\}$  of states, one could take a ``fresh''
automaton copy and apply an input word which distinguishes the two
states $p$ and $q$. From the observed outputs one could then determine
the initial state.

A preset experiment is completely specified by an input word $w \in \Sigma^*$.
Formally, an adaptive experiment can be defined by a mapping
$E: \Delta^* \rightarrow \Sigma \cup \{\epsilon\}$.
The experiment $E$ is carried out in the following way:

(i) If the automaton is a Mealy automaton, $E(\epsilon)$ denotes the
first input symbol.
For a Moore automaton, $E(x)$ denotes the first input symbol, where $x$ is the
first observed output symbol, which comes free.

(ii) Let us assume the input $w \in \Sigma^*$ was applied and the
output $W \in \Delta^*$ was observed.
Then, we apply the input $E(W)$ to the automaton.
The experiment terminates if $E(W) = \epsilon$.

The class of preset experiments is a subclass of the class
of adaptive experiments.
For every experiment $E$ we denote the obtained output of an initial
state $q$ by $\lambda_E(q)$.
$\lambda_E$ defines a mapping $Q$ to $\Delta^*$.

\subsection{Propositional Calculus of Automata}

In the following, we shall investigate the logic of the initial-state
identification problem.
We call a proposition regarding the initial state of the automaton
{\em experimentally decidable} if there is an experiment which
determines the truth value of the proposition.
The most general form of a prediction concerning the initial state $q$
of the automaton is that the initial state $q$ is contained in a subset $P$ of the
state set $Q$.
Therefore, we may identify propositions concerning the initial state
with subsets of $Q$.
A subset $P$ of $Q$ is then  identified with the proposition that the
initial state is contained in $P$.
More explicitly, we are dealing with propositions of the form,
{\em ``the initial  state of the automaton is in $P$''},
where $P$ is a subset of the set of automaton states $Q$.

We are now dealing with the problem of which subsets of the state set
are experimentally decidable.
Note, for instance,
that the proposition $\{q_1\}$ (i.e.~the proposition ``the initial
state of the automaton is $q_1$'') regarding Moore's uncertainty
automaton (cf.~Fig.~\ref{moorea}) is not decidable.

\begin{defin}[Automaton Propositional Calculus]
Let $E$ be an experiment (a preset or adaptive one).
We define an equivalence relation on the state set $Q$ by \\
$q \stackrel{E}{\equiv} p$ iff $\lambda_E(q) = \lambda_E(p)$ \\
for any $q,p \in Q$.
We denote the partition of $Q$ corresponding to $\stackrel{E}{\equiv}$
by $Q/\stackrel{E}{\equiv}$.
The propositions decidable by the experiment $E$ are
the elements of the Boolean algebra generated by $Q/\stackrel{E}{\equiv}$,
denoted by $B_E$.
There is also another way to construct the experimentally decidable
propositions of an experiment $E$.
Let $\lambda_E(P)  = \bigcup\limits_{q \in P}\lambda_E(q)$ be the direct
image of $P$ under $\lambda_E$ for any $P \subseteq Q$.
We denote the direct image of $Q$ by $O_E$, $O_E = \lambda_E(Q)$.
It follows that the most general form of a prediction concerning
the outcome $W$ of the experiment $E$ is that $W$ lays in a subset of
$O_E$.
Therefore, the experimentally decidable propositions consist of all
inverse images $\lambda_E^{-1}(S)$ of subsets $S$ of $O_E$,
a procedure which can be constructively formulated (e.g.; as an
effectively computable algorithm), and which also
leads to the Boolean algebra $B_E$.
Let ${\frak B}$ be the set of all Boolean algebras $B_E$.
We call the partition logic $R= (Q,{\frak B})$ an automaton propositional
calculus.
\end{defin}

This calculus possesses the following properties:

(i) $R$ contains two special propositions:
the proposition $\emptyset$, that the automaton is in no initial state,
which is always false, and,
the proposition $Q$ , that the automaton is in an arbitrary state,
which is always true.
${\bf 0} \equiv \emptyset$ is the least element and
${\bf 1} \equiv Q$ is the
greatest element in $R$.

(ii) Let $A \in R$. Any experiment which decides $A$ decides also $A' = Q
\backslash A$.
Moreover, $A$ is true iff $A'$ is false.

(iii) Let $A,B \in R$. $A \leq B$ holds iff

(a) there is an experiment which decides both propositions $A$ and $B$.

(b) $A$ implies $B$ (whenever $A$ is true, then also $B$ is true), which is
also expressed by $A \subseteq B$.\\
The use of a nontransitive implication relation is not new (cf.
\cite{specker1,kochen1,kochen2}).

\begin{figure}

\unitlength=1mm
\begin{picture}(100,80)(0,0)

\put(50,70){\circle*{1.5}}
\multiput(10,30)(20,0){5}{\circle*{1.5}}
\multiput(10,50)(20,0){5}{\circle*{1.5}}
\put(50,10){\circle*{1.5}}

\multiput(10,30)(20,0){4}{\line(1,1){20}}
\multiput(10,30)(40,0){2}{\line(2,1){40}}
\multiput(10,50)(20,0){4}{\line(1,-1){20}}
\multiput(10,50)(40,0){2}{\line(2,-1){40}}

\put(50,10){\line(-2,1){40}}
\put(50,10){\line(-1,1){20}}
\put(50,10){\line(0,1){20}}
\put(50,10){\line(1,1){20}}
\put(50,10){\line(2,1){40}}

\put(50,70){\line(-2,-1){40}}
\put(50,70){\line(-1,-1){20}}
\put(50,70){\line(0,-1){20}}
\put(50,70){\line(1,-1){20}}
\put(50,70){\line(2,-1){40}}

\small

\put(4,25){\{1,2\}}
\put(25,25){\{3\}}
\put(51,25){\{4\}}
\put(71,25){\{2\}}
\put(91,25){\{1,3\}}

\put(4,52){\{3,4\}}
\put(20,52){\{1,2,4\}}
\put(51,52){\{1,2,3\}}
\put(71,52){\{1,3,4\}}
\put(91,52){\{2,4\}}

\put(49,5){$\emptyset$}
\put(44,72){\{1,2,3,4\}}
\end{picture}

\caption{\label{mooreh} Hasse diagram of Moore's uncertainty automaton}

\end{figure}

\begin{figure}
%TexCad Options
%\grade{\off}
%\emlines{\off}
%\beziermacro{\off}
%\reduce{\on}
%\snapping{\off}
%\quality{0.20}
%\graddiff{0.01}
%\snapasp{1}
%\zoom{1.00}
\unitlength 1.00mm
\linethickness{0.4pt}
\begin{picture}(68.00,60.00)
(0,10)
\put(20.00,20.00){\circle*{2.00}}
\put(60.00,20.00){\circle*{2.00}}
\put(40.00,50.00){\circle*{2.00}}
\thicklines\put(22,19){\vector(1,0){36}}
\put(58.00,21.00){\vector(-1,0){36.00}}
\put(59.00,23.00){\vector(-2,3){16.00}}
\put(41.00,47.00){\vector(2,-3){16.00}}
\put(23.00,23.00){\vector(2,3){16.00}}
\put(37.00,47.00){\vector(-2,-3){16.00}}
\put(22.00,15.00){$q_1$}
\put(55.00,15.00){$q_2$}
\put(45.00,49.00){$q_3$}
\put(40.00,15.00){2/0}
\put(35.00,22.00){1/0}
\put(50.00,38.00){3/0}
\put(42.00,30.00){2/0}
\put(18.00,30.00){1/0}
\put(35.00,38.00){3/0}
\put(8.00,10.00){1/1}
\put(68.00,10.00){2/1}
\put(38.00,60.00){3/1}
\put(40.00,53.50){\circle{7.33}}
\put(36.33,54.33){\vector(0,-1){1.33}}
\put(63.00,17.67){\circle{7.33}}
\put(17.33,17.33){\circle{7.33}}
\put(66.67,16.33){\vector(0,1){1.00}}
\put(13.67,18.33){\vector(0,-1){1.00}}
\end{picture}
\caption{\label{a2} Quantumlike Mealy automaton}
\end{figure}

\begin{figure}
\unitlength=1mm
\begin{picture}(140,60)(0,00)

\multiput(10,30)(20,0){3}{\circle*{1.5}}
\multiput(90,30)(20,0){3}{\circle*{1.5}}
\put(70,10){\circle*{1.5}}
\put(70,50){\circle*{1.5}}

\put(10,30){\line(3,1){60}}
\put(30,30){\line(2,1){40}}
\put(50,30){\line(1,1){20}}

\put(10,30){\line(3,-1){60}}
\put(30,30){\line(2,-1){40}}
\put(50,30){\line(1,-1){20}}

\put(90,30){\line(-1,1){20}}
\put(110,30){\line(-2,1){40}}
\put(130,30){\line(-3,1){60}}

\put(90,30){\line(-1,-1){20}}
\put(110,30){\line(-2,-1){40}}
\put(130,30){\line(-3,-1){60}}

\small

\put(3,29){\{1\}}
\put(23,29){\{2\}}
\put(43,29){\{3\}}

\put(92,29){\{1,2\}}
\put(112,29){\{1,3\}}
\put(132,29){\{2,3\}}

\put(69,5){$\emptyset$}
\put(64,52){\{1,2,3\}}

\end{picture}

\caption{\label{h2} Hasse diagram of the automaton logic of the quantumlike
Mealy automaton}

\end{figure}

\begin{figure}
%TexCad Options
%\grade{\off}
%\emlines{\off}
%\beziermacro{\off}
%\reduce{\on}
%\snapping{\off}
%\quality{0.20}
%\graddiff{0.01}
%\snapasp{1}
%\zoom{1.00}
\unitlength 1.00mm
\linethickness{0.4pt}
\begin{picture}(59.00,49.01)(0,0)
\multiput(10,10)(25,0){3}{\circle*{2}}
\put(35.00,35.00){\circle*{2.00}}
\thicklines
\put(12,12){\vector(1,1){21}}
\put(35.00,12.00){\vector(0,1){21.00}}
\put(58.00,12.00){\vector(-1,1){21.00}}
\put(28,49.70){R/1}
\put(36.00,49.70){S/1}
\put(39.00,33.00){1}
\put(9.00,4.00){2}
\put(34.00,4.00){3}
\put(59.00,4.00){4}
\put(18.00,24.00){R/1}
\put(13.00,20.00){S/2}
\put(36.00,24.00){R/2}
\put(36.00,20.00){S/2}
\put(46.00,24.00){R/3}
\put(51.00,20.00){S/3}
\put(35.00,42.00){\circle{14.02}}
\put(28.00,42.67){\vector(0,-1){0.33}}
\end{picture}
\caption{\label{m3} Mealy automaton yielding the partition logic of
Moore's uncertainty automaton}
\end{figure}

\begin{figure}
%TexCad Options
%\grade{\off}
%\emlines{\off}
%\beziermacro{\off}
%\reduce{\on}
%\snapping{\off}
%\quality{0.20}
%\graddiff{0.01}
%\snapasp{1}
%\zoom{1.00}
\unitlength 1.00mm
\linethickness{0.4pt}
\begin{picture}(57.80,57.13)
(0,0)
\multiput(20,20)(30,0){2}{\circle*{2}}
\multiput(20,50)(30,0){2}{\circle*{2}}
\thicklines\put(22,20){\vector(1,0){26}}
\put(50.00,22.00){\vector(0,1){26.00}}
\put(48.00,50.00){\vector(-1,0){26.00}}
\put(20.00,48.00){\vector(0,-1){26.00}}
\put(53.00,18.00){$q_1$}
\put(45.00,46.00){$q_2$}
\put(14.00,48.00){$q_3$}
\put(24.00,23.00){$q_4$}
\put(52.00,37.00){1/0}
\put(52.00,41.00){0/0}
\put(25.00,51.00){0/1}
\put(12.00,29.00){0/0}
\put(12.00,33.00){1/1}
\put(40.00,15.00){1/0}
\put(57.00,57.00){1/0}
\put(8.00,9.00){0/0}
\put(53.37,52.70){\circle{9.00}}
\put(17.00,17.33){\circle{9.00}}
\put(51.00,56.00){\vector(-1,-1){0.67}}
\put(20.33,14.67){\vector(1,4){0.33}}
\end{picture}
\caption{\label{a3}  Mealy automaton yielding a nontransitive
propositional calculus}

\end{figure}

We shall give some examples.
First, we shall construct the propositional calculus of Moore's original
uncertainty automaton (cf.~Fig.~\ref{moorea}).
There are 3 different partitions accessible by experiments.
The preset experiment $\epsilon$ corresponds to observing
only the first free output
of the Moore automaton without any input.
Therefore it yields the partition
$Q/(\epsilon) = \{\{q_1,q_2,q_3\},\{q_4\}\}$. \\
The preset experiment 0, i.e., input of 0, yields the
partition\\
$Q/(0) = \{\{q_1,q_3\},\{q_2\},\{q_4\}\}$. \\
The preset experiment 1, i.e., the input of 1, yields the
partition  \\
$Q/(1) = \{\{q_1,q_2\},\{q_3\},\{q_4\}\}$. \\
$Q/(0)$ and $Q/(1)$ are finer partitions than $Q/(\epsilon)$
and we may neglect $Q/(\epsilon)$ by forming the propositional calculus.
We obtain the partition logic drawn in Fig.~\ref{mooreh}
(the numbers denote the corresponding states).
A Hilbert space representation of the partition logic is drawn in
Fig.~\ref{identi}.
\begin{figure}
\unitlength=0.7mm
\special{em:linewidth 0.4pt}
\linethickness{0.4pt}
\begin{picture}(75.00,75.00)
\put(40.00,40.00){\line(0,1){30.00}}
\put(40.00,40.00){\line(1,0){30.00}}
\put(40.00,40.00){\line(-4,-3){20.00}}
\put(40.00,40.00){\line(2,-3){10.00}}
\put(40.00,40.00){\line(5,2){25.00}}
\put(17.00,21.00){\makebox(0,0)[cc]{12}}
\put(52.00,21.00){\makebox(0,0)[cc]{2}}
\put(75.00,40.00){\makebox(0,0)[cc]{3}}
\put(69.00,51.00){\makebox(0,0)[cc]{13}}
\put(40.00,75.00){\makebox(0,0)[cc]{4}}
\end{picture}
\caption{\label{identi} Identification of atoms with rays in
three-dimensional real Hilbert space.
If $v(a)$ is the subspace spanned by $a$,
$v(12) \perp v(3)$,
 $v(2)\perp v(13)$,
$v(12) \perp v(4)$,
$v(2) \perp v(4)$,
$v(3) \perp v(4)$,
$v(13) \perp v(4)$, $v(12)\neq v(2)$
}
\end{figure}

The automaton defined by Fig.~\ref{a2} yields a propositional calculus
drawn in Fig.~\ref{h2}, which is also found in the quantum logic
of two-dimensional Hilbert space.

Every automaton proposition calculus is by definition a partition logic.
Conversely, to every partition logic, a Mealy automaton can be
effectively constructed
which possesses that partition logic as propositional calculus
(cf.~\cite{svozil}).
Let $R = (Q,{\frak R})$ be a partition logic.
We rewrite every $P \in {\frak R}$ as  an indexed family $P = (P_i)_{i
\in I_n}$, where the index set $I_n$ denotes the set $\{1,\ldots,n\}$ of
natural numbers.
We assume that $P_i \neq P_j$ for $i \neq j$.
$N$ denotes the greatest number of elements in a partition $P \in {\frak R}$.
We put $M = (Q,{\frak R},I_N,\delta,\lambda)$.
Next, the transition and output functions $\delta$
and
$\lambda$ have to be properly defined. Let $p$ be an arbitrary element
of
$Q$. For all $q \in Q$ and all $P \in  {\frak R}$ we define \\
(i) $\delta(q,P) = p$ and \\
(ii) $\lambda(q,P) = i$ iff $q \in P_i$.\\
In doing so, we obtain as the automaton propositional calculus the
partition logic $(M,{\frak R})$.
Instead of ${\frak R}$, we could also use the decomposition ${\frak
C}(R)$, yielding an automaton with at most three outputs.

We illustrate this construction by an example:
Consider the partition logic of Moore's original automaton.
It is given by
$(Q, {\frak P}) = (\{1,2,3,4\},\{R=\{R_1=\{1\},R_2=\{2,3\},R_3=\{4\}\},
S=\{S_1=\{1,2\},S_2=\{3\},S_3=\{4\}\}\}$.
We obtain the Mealy automaton $M = \{Q,\{R,S\},\{1,2,3\},\delta,\lambda\}$
where $\delta$ and $\lambda$ are represented by the transition diagram
Fig.~\ref{m3}.


We have already remarked that not every partition logic is  an
orthomodular poset.
An automaton example for this case is given in Fig. \ref{a3}.
The finest partition accessible by experiments are
$Q/(00) = \{\{1\},\{2\},\{3,4\}\}$ and
$Q/(10) = \{\{1,2\},\{3\},\{4\}\}$
(the numbers denote the corresponding states).
Here,
$\{1\} \leq \{1,2\}$ and $\{1,2\} \leq \{1,2,3\}$ holds,
but $\{1\} \leq \{1,2,3\}$ does not hold.


\begin{thebibliography}{bib}
\bibitem{birkhoff}
Birkhoff, G. and von Neumann, J. (1936). The logic of
quantum mechanics,
{\em Annals of Mathematics} {\bf 37}, 823-843.

\bibitem{bhlat}
Birkhoff, G. (1948). {\sl Lattice Theory, Second Edition}, American
 Mathematical Society, New York.

\bibitem{booth}
Booth, T.~L. (1967). {\em Sequential automata and Automata Theory},
Wiley and Sons, New York, London, Sydney.


\bibitem{brauer}
Brauer, W. (1984). {\em Automatentheorie},
Teubner, Stuttgart.


\bibitem{chaitin-65}
Chaitin, G. J. (1965). {\sl IEEE Transactions on Electronic
Computers},
{\bf EC-14},
466.


\bibitem{conway}
Conway, J.~H. (1971). {\em Regular Algebras and Finite automata},
Clowes and Sons, London.

\bibitem{crutchfield}
Crutchfield, J.~P. (1993). Observing Complexity and the Complexity of
Observation,
{\em Inside versus Outside}, ed. by H. Atmanspacher and G. J.
Dalenoort, Springer, Berlin, pp.
235-272.

\bibitem{finkelstein}
Finkelstein, D. and Finkelstein, S.~R. (1983). Computational
Complementarity,
{\em International Journal of Theoretical Physics},
{\bf 22}, 753-779.

 \bibitem{giuntini-91}
Giuntini, R. (1991). {\sl Quantum Logic and Hidden Variables},
BI Wissenschaftsverlag, Mannheim.

\bibitem{graetzer}
Gr\"atzer, G. (1971). {\em Lattice Theory}, Freeman, San Francisco.

\bibitem{grib90}
Grib, A.~A. and Zapatrin, R.~R. (1990).
 Automata simulating
quantum logics,
{\em International Journal of Theoretical Physics},
{\bf 29}, 113-123.

\bibitem{grib92}
Grib, A.~A. and Zapatrin, R.~R. (1992).
Macroscopic realizations of
quantum logics,
{\em International Journal of Theoretical Physics},
{\bf 31}, 1669-1687.

\bibitem{gsz}
Grib, A.~A., Svozil, K. and Zapatrin, R.~R. (1995).
Empirical logic of finite automata: microstatements versus
macrostatements,
{\em preprint}.


\bibitem{hopcroft}
Hopcroft, J.~E. and Ullman, J.~D. (1979).
 {\em Introduction to Automata
Theory, Languages and Computation},
Addison-Wesley, Reading-Mass.

\bibitem{jammer}
Jammer, M. (1974). {\em The Philosophy of Quantum Mechanics},
Wiley and Sons, New York, Sydney, Toronto.

\bibitem{jauch}
Jauch, J. (1968). {\em Foundations of Quantum Mechanics},
Addison-Wesley, Reading, Mass.

\bibitem{kalmbach}
Kalmbach, G. (1983). {\em Orthomodular Lattices},
Academic Press, London, New York.

\bibitem{kochen1}
Kochen S., Specker E. (1965).
 Logical structures arising in quantum
theory,
{\sl Symposium on the Theory of Models, Proceedings of the 1963
International Symposium at Berkely}, North-Holland, Amsterdam, pp.
177-189.

\bibitem{kochen2}
Kochen S., Specker E. (1965). The calculus of partial propositional
functions,
{\sl Proceedings of the 1964 International Congress for Logic, Methodology
and Philosophy of Science, Jerusalem},
North-Holland, Amsterdam,pp. 45-57.

\bibitem{moore} Moore, E.F. (1956).
 {\em Gedankenexperiments on sequential
automata}, in {\em Automata Studies}, ed. by C.E.Shannon and J.McCarthy,
Princeton Univ. Press, Princeton, pp.  129-153.

\bibitem{navara}
Navara, M. and Rogalewicz, V. (1991). The pasting construction for
orthomodular posets, {\em Mathematische Nachrichten}, {\bf 154},
157-168.

\bibitem{piron}
Piron, C. (1976). {\em Foundations of Quantum Physics},
Benjamin, Reading, Mass.

\bibitem{piziak}
Piziak, R. (1991). Orthomodular Lattices and Quadratic Spaces: A Survey,
{\em Rocky Mountain Journal of Mathematics}, {\bf 21}, 951-992.

\bibitem{ptak}
Pt\'ak, P. and Pulmannov\'a, S. (1991). {\em Orthomodualar Structures as
Quantum Logics}, Kluwer, Dordrecht.

\bibitem{schaller}
Schaller, M. and Svozil, K. (1994). Partition Logics of Automata,
{\em Il Nuovo Cimento}, {\bf 109 B},  167-176.

\bibitem{specker1}
Specker E. (1960). Die Logik nicht gleichzeitig entscheidbarer Aussagen,
{\sl Dialectica}, {\bf 14}, 239-246.



\bibitem{svozil}
Svozil, K. (1993). {\em Randomness and Undecidability in Physics},
World Scientific, Singapore.

\bibitem{szasz}
Sz\'asz, G. (1963). {\em Introduction to Lattice Theory},
Academic Press, New York.

\end{thebibliography}

\end{document}
