\documentclass{article}
\usepackage{graphicx}
\bibliographystyle{plain}

\begin{document}

\title{Is randomness guaranteed by quantum value indefiniteness?}
\author{
Alastair Abbott
and
Cristian S. Calude\footnote{Department of Computer Science, University of Auckland, Private Bag 92019, Auckland, New Zealand.}
and
Karl Svozil\footnote{Institute for Theoretical Physics, University of Technology Vienna,
Wiedner Hauptstrasse 8-10/136, 1040 Vienna,  Austria.}
}

\maketitle


A recent article~\cite{10.1038/nature09008}, heralded by a {\em News and Views} commentary~\cite{10.1038/464988a},
reports the realization of random numbers certified by {B}ell's theorem.
Another recent paper relates quantum randomness to logical independence~\cite{1367-2630-12-1-013019}.
Both papers also expose some common misconceptions about logical independence and randomness.

An essential ingredient of the papers mentioned is quantum {\em value indefiniteness}, as exposed by Bell-,
Kochen-Specker-, and Greenberger-Horne-Zeilinger-type theorems,
stating that there cannot exist any consistent, context independent, omniscience for quantum systems with three or more mutually exclusive outcomes.
Indeed, as quantum {\em complementarity} is a necessary but no sufficient criterion for quantum value indefiniteness,
this goes beyond the various quantum random number generators based on beam splitters with two output ports.

Schemes involving quantum {\em value indefiniteness} have been previously proposed~\cite{2008-cal-svo,svozil-2009-howto}
(for an earlier realization see ref.~\cite{fiorentino:032334})
by arguing that, essentially, if there is no definite pre-existing element of physical reality
associated with a measurement outcome,
and there is also no causal law describing the acquisition of the measurement result,
then the outcome of a measurement can only be indeterministic and incomputable.
Any such proof is based upon the assumption that quantum mechanics is a maximally complete theory,
and no ``completion'' by, for example, nonlocal, contextual hidden parameters, or by dynamical context translation during the measurement process, is possible.
However, even though these arguments prove incomputability under the aforementioned assumptions,
they do not certify the generated strings to be {\em random} in a more formal sense~\cite{2008-cal-svo}.

With regards to linking logical independence to quantum randomness, it seems rather trivial that, say, if one starts out with one axiom, say A1: ``$1+1=2$''
one ends up with ``incompleteness'' or ``independence'' of, say, the proposition A2: ``$1+2=3$'' with respect to A1.
Alas, again, in no way certifies this kind of ``independence'' randomness in the formal sense.

\bibliography{svozil}

\end{document}
