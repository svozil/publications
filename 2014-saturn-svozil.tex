\documentclass[%
 %reprint,
  twocolumn,
 %superscriptaddress,
 %groupedaddress,
 %unsortedaddress,
 %runinaddress,
 %frontmatterverbose,
 % preprint,
 showpacs,
 showkeys,
 preprintnumbers,
 %nofootinbib,
 %nobibnotes,
 %bibnotes,
 amsmath,amssymb,
 aps,
 % prl,
  pra,
 % prb,
 % rmp,
 %prstab,
 %prstper,
  longbibliography,
 %floatfix,
 %lengthcheck,%
 ]{revtex4-1}

%\usepackage{cdmtcs-pdf}

\usepackage{mathptmx}% http://ctan.org/pkg/mathptmx

\usepackage{amssymb,amsthm,amsmath}

\usepackage{tikz}
\usepackage[breaklinks=true,colorlinks=true,anchorcolor=blue,citecolor=blue,filecolor=blue,menucolor=blue,pagecolor=blue,urlcolor=blue,linkcolor=blue]{hyperref}
\usepackage{graphicx}% Include figure files
\usepackage{url}

\usepackage{xcolor}

\begin{document}


\title{Physical aspects of providence, dualistic free will, miracles, and oracles}

%\cdmtcsauthor{Karl Svozil}
%\cdmtcsaffiliation{Vienna University of Technology}
%\cdmtcstrnumber{407}
%\cdmtcsdate{September 2011}
%\coverpage

\author{Karl Svozil}
\affiliation{Institute for Theoretical Physics, Vienna
    University of Technology, Wiedner Hauptstra\ss e 8-10/136, A-1040
    Vienna, Austria}
\email{svozil@tuwien.ac.at} \homepage[]{http://tph.tuwien.ac.at/~svozil}


\pacs{01.70.+w}
\keywords{providence, free will, determinism, indeterminism, choice, agent, oracle, miracle}
%\preprint{CDMTCS preprint nr. 407/2011}

\begin{abstract}
Some physical aspects and scenarios for providence and free will are discussed; mostly in a dualistic context which avoids the problems for strictly (in)deterministic universes. These schemes require a fine-tuned mixture of determinism and indeterminism allowing ontological gaps in the natural laws. These gaps facilitate and permit the insertion and communication of intentions and choices via interfaces serving as Cartesian cuts. Algorithmic models are computer games, taking signals as input from some agent acting as an oracle.
\end{abstract}

\maketitle

\section{Introduction}

We shall present some nomenclature,
followed by subjective motivation to pursue metaphysical questions of existence pertinent to the topics discussed,
and present a very brief historic account on physical unknowables~\cite{svozil-07-physical_unknowables}.


\subsection{Nomenclature}

Since theological nomenclature hardly belongs to the standard repertoire of physicists, some {\it termini technici} will be mentioned upfront.
Thereby we will mainly follow Philipp Frank's (informal) definitions of {\em gaps} and {\em miracles}~\cite{frank,franke},
as well as Robert Russell's notions allowing him
to formulate the doctrine of {\em non-interventionist objective divine action (NIODA)}~\cite[Part~II,Chapter~4]{Russel-nioda-1}.

In the theological context,  {\it creatio ex nihilo} often refers to the `initial boot up of the universe;'
whereas {\it creatio continua} stands for the permanent intervention of the divine throughout past, present, and future.
Alas, as we will be mainly interested with physical events, we shall refer to
{\it creatio ex nihilo}, or just {\it ex nihilo,} as something coming from nothing; in particular, from no intrinsic~\cite{svozil-94} causation
(and thus rather consider the theological {\it creatio continua}; apologies for this potential confusion).
{\it Ex nihilo} denies, and is in contradiction, to the {\em principle of sufficient reason}, stating that nothing is without intrinsic causation, and {\it vice versa}.

According to Frank~\cite[Sect.~II,~12]{frank,franke}, a {\em gap} stands for the {\em incompleteness} of the laws of nature,
which allow for the occurrence of events without any unique natural (immanent, intrinsic) cause, and for the possible intervention of higher powers~\cite[Sect.~II,~9]{frank,franke}: {\em ``Under
certain circumstances they do not say what definitely has to happen
but allow for several possibilities; which of these possibilities comes
about depends on that higher power which therefore can intervene
without violating laws of nature.''}

This is different from a direct breach or `rapture' of the laws of nature~\cite[Sect.~II,~10]{frank,franke};
also referred to as {\em ontological gap} by a forced {\em intervention} in the otherwise uniformly causal connection of events~\cite[Sect.~3.C.3, Type~II]{Russel-nioda-1}.
An example for an ontological gap would be the sudden {\it ad hoc} turn of a celestial object which
would otherwise have proceeded along a trajectory governed by the laws of inertia and gravitation.

Often, the resulting correlations are subjectively and semantically experienced as {\em synchronicity},
that is, with a {\em purpose} -- the events are not causally connected but {\em ``stand to one
another in a meaningful relationship of simultaneity''}~\cite{jung1,jung1e}.
A more personal example is Jung's experience of a solid oak table suddenly split right across,
soon followed by a strong steel knife breaking in pieces for no apparent reason~\cite[pp.~111-2, 104-5]{jung-memories,jung-memories-e}.

In what follows we shall adopt Frank's conceptualization of a {\em miracle}~\cite[Sect.~II,~15]{frank,franke}
as a {\em gap} (in Frank's sense cited above) which is exploited according to a {\em plan}  or purpose;
so a `higher power' interferes {\it via} the incompleteness (lack of determinacy) of the laws of nature to pursue an intention.

Note that this notion of miracle is different from the common acceptation quoted by Voltaire,
according to which a miracle is the violation of divine and eternal laws~\cite[Sect.~330]{voltaire-dict}.
Russel refers to the latter as `miracle in the Humean sense'~\cite[Sect.~3.C.7 ]{Russel-nioda-1}: {\em ``a miracle is an event which violates the laws of nature and which contradicts science.''}

Finally, an {\em oracle} is an agent capable of a {\em decision} or an {\em emanation} (such as a random number) which cannot be produced by a universal computer.
Again, we take up Frank's conception of a gap, or of Russel's NIODA, to conceptualize physical oracles.

\subsection{The mind-boggling fact of existence}

Our first and foremost existential problem appears to be that of {\em existence}~\cite{holt-existence}: why is there something rather than nothing?
In particular, why does the universe exist? What are we here for? What is our origin and our destiny?
(I would therefore disagree with Camus~\cite{camus-mos} that the only serious philosophical problem is whether to commit suicide.)
It might not be too speculative to suggest that the resulting mind-boggling amazement is the root of both religion and science,
which share a single goal, and thus might be seen as two sides of the same endeavor: the pursuit of truth -- how ephemeral this may appear.

Reactions to existence can turn into feelings of terror, anxiety, panic or dread;
a {\it  pavor nocturnus} (night terror) in full waking life.
These are mostly stipulated by the suspicion that there are no grounds on which to anchor possible answers;
the issue of existence rises before us like an barrier insurmountable for rational thought.
Indeed, contemplation on existence, if pursuit honestly and consequentially,
may result in madness and total destructive dissolution of the self; without any hope of return to normality or redemption.

Various strategies have been developed to cope with the individual human experience of existence.
We may group these strategies into three branches: (i) religion, (ii) philosophy (e.g. existentialism, materialism), and (iii) science.
As all human endeavors, these three branches share a common ground: all of them are narratives.

Religion is a very powerful narrative, and it is a very big grace to be able to believe.
Because all of a sudden, when viewed with belief, the Universe becomes light and bright, and full of deep meaning.

Natural sciences is a less gratifying narrative, but the resulting recommendations may have, on the average, more practical usefulness.
Alas, as philosophers of science (e.g. Lakatos, Popper \& Feyerabend) clearly have stated,
there is no absolute truth in ``scientific understanding.''
Even mathematics, such as Ramsey theory, cannot help, because it itself is a formalism.

Many so-called `philosophical' approaches such as atheism and materialism appear to be mere ideologies in disguise.
(An exception is a Camus-type existentialism.)
Indeed, these ideologies may become utterly dangerous in times of political revolution and unrest (cf. Robspierre, Lenin, Stalin, to name but a few leaders)
-- because they need not refer to any type of immutable soul, and no ultimate responsibility.

Also logical `proofs' of the existence of God, from  Anselm of Canterbury onwards to G\"odel and beyond~\cite{benzm-Paleo-2014}
are based on assumption which make them {\em means relative} with respect to these assumptions
and the type of logic used.


What can we learn from this? I suggest to keep in mind that all narratives -- as useful as they may appear in tool-building and technology --
represent no absolute truth but are of preliminary nature, and are subject to constant changes, as time passes by  and history shows.
Presently physics claims that the world consists almost entirely of a vacuum void  which fluctuates by creating and annihilating all kinds of
quanta -- the quanta themselves are point particles with no spatial extension at all; some of them mediate various forces,
some of them carry various charges and capacities to interact.
According to this understanding, a `solid' table is a very special emptiness containing these quanta.
Gravity has been translated into the geometry of space-time.
Many observations in deep space appear unexplained; the metaphorical hypothesis for these puzzling phenomena being {\em dark matter}
(of unknown, probably yet undiscovered, type),
which appears to be invisible, yet makes up most of the universe.
%  http://home.web.cern.ch/about/physics/dark-matter
These fragmentary hypotheses and conjectures are hardly a solid basis for a scientific ontology; less a complete, temporally stable body of knowledge!

I would personally recommend to adapt a contemplative strategy of {\em evenly-suspended attention}
outlined by  Freud~\cite{Freud-1912}.
It is a manner in which the individual should listen to the universe; without any too strong (mostly unconscious) emotional bias.
(For instance, fear creates tendencies to accept the opposite: fear of determinism yields longing for indeterminism, and {\it vice versa}~\cite{2002-cross}.)
We need to be open for new approaches, scenarios and phenomena; as well as of being aware
of the vastness of the domains of physical existence we know very little about.
Augustinus' {\it dictum} {\it ``Ei mihi, qui nescio saltem quid nesciam!''}
{ (Alas for me, that I do not at least know the extent of my own ignorance!)}
%(in {\it ``Confessiones''}, Book XI, chapter 25)
guides us more than ever.

So, by having this in mind, we are finally in the position to cautiously, and, with hopefully evenly-suspended attention,
engage questions regarding a `room for divinity' in the sciences,
or, conversely, a `room for science' in religion.


\subsection{A rise of indeterminism}

Almost unnoticed at first,  the tide of indeterminism started to build
toward the end of the nineteenth century~\cite{purrington,Kragh-qg}.
At that time, the prevalent mechanistic theories faced an increasing number of anomalies:
to name but a few, there was
Poincar\'e's discovery of  instabilities  of trajectories of celestial bodies
(which made them extremely sensible to initial conditions),
radioactivity~\cite{Kragh-1997AHESradioact,Kragh-2009_RePoss5},
X-rays,
specific heats of gases and solids,
emission and absorption of light, in particular, blackbody radiation,
and
the irreversibility dilemma of statistical physics based on reversible mechanics and electrodynamics.

{\it Fin de si\'ecle} 1900 followed a short period of revolutionary new physics, in particular,
quantum theory and relativity theory,
without any strong metaphysical preference toward either determinism or indeterminism.
Then indeterminism erupted boldly with Born's claim that quantum mechanics has it both ways:
the quantum state evolves strictly deterministically,
whereas the individual event or measurement outcome occurs indeterministically.
Born made it clear that he was\cite[866]{born-26-1}  {\em ``inclined to give up determinism in the world of atoms;''}
that there is no cause for certain individual quantum events;
that is, such outcomes occur irreducibly at random.

Another indeterministic feature of quantum mechanics is {\em complementarity:} there exist collections of observables (such as position and momentum)
which cannot be simultaneously operationalized (i.e. prepared and measured) with arbitrary precision.
Still another indeterministic quantum feature is the {\em value indefiniteness} of at least all but one complementary observables~\cite{specker-60,PhysRevA.89.032109}.

There followed a fierce controversy, with many researchers such as Born, Bohr, Heisenberg, and Pauli
taking the indeterministic stance,
whereas others,
like Planck~\cite{born-55}, Einstein~\cite{epr,ein-reply}, Schr\"odinger, and De Brogli, leaning toward determinism.
This latter position was pointedly put forward by Einstein's {\it dictum} in a letter to Born,
dated December~12, 1926~\cite[113]{born-69}:
{``In any case I am convinced that he [the Old One] does not throw dice.''}

At present, indeterminism is preached by the orthodoxy to the extend that it is declared {\em ``the message of the quantum~\cite{zeil-05_nature_ofQuantum}.''}
This is motivated from formal theorems about predictions of general deterministic theories
(relative to some supposedly reasonable assumptions such as omniexistence and contextuality~\cite{svozil-2013-omelette} as well as locality) --
such as Bell's theorem~\cite{Pit-94} and the Kochen-Specker theorem~\cite{specker-60,pitowsky:218,cabello:210401,PhysRevA.89.032109}.

The last quarter of the twentieth century saw the rise of yet another form of physical indeterminism,
originating in Poincar\'e's aforementioned
discovery of instabilities of the motion of classical bodies
against variations of initial conditions~\cite{Campbell-1882,poincare14,Diacu96-ce}.
This scenario of {\em deterministic chaos} resulted in a plethora of claims regarding indeterminism
that resonated with a general public susceptible to fables and fairy tales~\cite{bricmont}.

In parallel, G\"odel's incompleteness theorems~\cite{godel1,tarski:32,davis-58,davis,smullyan-92},
as well as related findings in the computer sciences~\cite{turing-36,chaitin3,calude:02,gruenwald-vitanyi},
put an end to Hilbert's program of finding a finite axiom system for all mathematics.
G\"odel's incompleteness theorems also established formal
bounds on provability, predictability, and induction.
(The incompleteness theorems also put an end to philosophical contentions
expressed by~\cite[101]{schlick-35} that, beyond epistemic unknowables and
the ``essential incompetence  of human knowledge,'' there is ``not a single real
question for which it would be {\em logically} impossible to find a solution.'')

Alas, just like determinism, physical indeterminism cannot be proved, nor can there be given any reasonable criterion for its falsification.
After all, how can one check against all laws and find none applicable?
Unless one is willing to denote any system whose laws are currently unknown
or whose behavior is hard to predict with present techniques as indeterministic,
there is no scientific substance to such absolute claims,
especially  if one takes into account the bounds imposed by the theory of recursive functions.
So both positions --  determinism as well as indeterminism -- must be considered conjectural.


\section{{\it Weltanschauung} (worldview) between {\it Scylla} and {\it Charybdis}}

Like {\it Odysseus} trapped between {\it Scylla} and {\it Charybdis},
our physical worldview, as well as providence and free will, appears to be severely restricted by
physical determinism as well as complete indeterminism.
Ontologically a clockwork universe, as well as one pushing uncontrollable chance,
leaves no room for divine interaction and willable alternatives.

\subsection{Deterministic {\it Scylla}}
Determinism blocks free will by the principle of sufficient reason.
Determinism might be beautiful and
``rich'' in the sense of allowing ornamentation,
but it lacks any kind of {\em steering mechanism}, or {\em freedom of choice.}

Formally, one of the most extreme forms of  determinism is expressed by the unitary quantum mechanical state evolution,
amounting to mere permutations, that is, one-to-one transitions, among states and orthonormal bases~\cite{Schwinger.60}.

\subsection{Indeterministic {\it Charybdis}}
Indeterminism,
at least in the form of the {\it creatio ex nihilo}(or rather {\it creatio continua}) of events without any cause,
leaves no room for choice either: because if events emerge {\it ex nihilo} and uncontrollably,
there is no freedom of choice between alternatives either.
Indeed, it is very difficult even to imagine how `primordial chaos' could be characterized and perceived --
due to its very nature, mathematical randomness is lacks any
constructive definability; it makes sense only for infinite strings~\cite{calude:02}.
As already pointed out by von Neumann~\cite{von-neumann1}, {\em ``any one who considers arithmetical methods of
producing random digits is, of course, in a state of sin.''}
(Nevertheless, modern recursion theory allows for the `quasi-construction' of random reals,
such as Chaitin's Omega number~\cite{calude-dinneen06}; alas without any computable rate of convergence.
In this sense, these computations are not dissimilar to `drawing a random real' from the continuum urn; facilitated by the Axiom of Choice.)

How could we imagine the apparent lawfulness of the universe despite primordial chaos?
Maybe Ramsey theory could give us a clue -- stating that for all kinds of sequences  -- experimental as well as in symbolic ones --
correlations are unavoidable~\cite{GS-90}.
An example is the following theorem~\cite[Sect.~1.1]{Ramsey-GRS-90}:
{\em ``In any collection of six people either three of them mutually know each
other or three of them mutually do not know each other.''}
So maybe what we call `causality' is just `correlations'?
Of course, it is a far way from this kind of speculation to a proof that the natural laws are derivable from Ramsey theory.

\section{Dualistic interfaces as path through {\it Scylla} and {\it Charybdis}}

Despite this gloomy perspective, there is a third alternative which we shall discuss here:
the possibility that {\em transcendent agents} interact with a(n) (in)deterministic universe via suitable {\em interfaces.}
Immanence refers to all operational, intrinsic means available to embedded observers~\cite{toffoli:79,svozil-94} from within some universe;
whereas transcendence goes beyond these means.
In what follows we shall refer to the transcendental universe as the beyond.
We shall also adopt the terminology of Calude and Poznanovi\'{c}~\cite{CaludePoznanovic}
which identifies four components for a formal discussion of free will:
agents, objects, contexts and choices.
Informally speaking, an agent chooses an object in some context or environment.


\subsection{Computer game metaphor}

For the sake of  metaphorical models,
take Eccles' mind-brain model~\cite{eccles:papal},
or consider a virtual reality, and, more particular, {\em a computer game.} In such a gaming universe, various human players are represented
by avatars.
There, the universe is identified with the game world created by an algorithm (essentially, some computer program),
and the transcendental agent is identified with the human gamer.
The interface essentially consists of any kind of device and method connecting the human body with the avatar.
Like the god {\em Janus} in the Roman mythology, an interface possesses two faces or handles: one into the universe, and a second one into the beyond.

Human players constantly input or inject choices through the interface, and {\it vice versa.}
In this {\em hierarchical, dualistic} scenario, such choices need not solely (or even entirely) be determined
by any conditions of the game world:
human players are transcendental with respect to the context of the game world,
and are subject to their own universe they live in (including the interface).
Alas the game world itself is totally deterministic in a very specific way:
it allows the player's input from beyond; but other than that it is created by a computation.
One may think of a player as a specific sort of indeterministic (with respect to intrinsic means)
{\em oracle}, or subprogram, or functional library.

Another algorithmic metaphor is an {\em operating system},
or a {\em real-time computer system}, serving as context.
(This is different from a classical Turing machine, whose emphasis is not on interaction with some user-agent.
The user is identified with the agent.
Any user not embedded within the context is thus transcendent with respect to this computation context.
In all these cases the  real-time computer system acts deterministically on any input received from the agent.
It observes and obeys commands of the agent handed over to it {\em via} some interface.
An interface could be anything allowing communication between the real-time computer system and the (human) agent;
say a touch screen, a typewriter(/display), or any brain-computer interface.
One may also say that without any such intervention the operating system remains dormant or idle.
The ``meaning'' of the real-time computer system is the interaction with, and response to, the agent.
The agent here has the function of an oracle which is constantly monitored.

\subsection{How to acknowledge intentionality?}

The mere existence of gaps in the causal fabric are no sufficient condition for the existence of providence or free will,
because these gaps may be completely filled up by {\em creatio ex nihilo:}
in that way, an indeterministic universe is not steered by an agent,
but erratically pushed around by chance.
(This latter scenario seems to be the foundational,
metaphysical basis of deterministic chaos, spontaneous symmetry breaking, instabilities due to discontinuities,
as well as of the irreducible indeterminism characterizing certain individual quantum outcomes.)


As has already been observed by Frank~\cite[Kapitel~{III}, Sects.~14,~15]{frank},
in order for any {\em miracle} or free will to manifest itself
through any such gap in the natural laws, it needs to be {\em systematic,}
{\em according to a plan}
and
{\em intentional} (German {\it planm\"a\ss ig}).
Because if there were no possibilities to inject information or other matter or content
into the universe from beyond, there would be no possibility to manipulate the universe,
and therefore no substantial choice.

In its purest form, any dualistic choice manifests itself in a single bit.
Such a dichotomic signal may be communicated through a noisy channel requiring more than one bit,
or directly by the communication of a classical bit.
Again, it is important to stress that the occurrence of a single bit (or any finite concatenation thereof)
cannot differentiate between chance or choice.
(Due to Ramsey theory, absolute randomness is vacuous even for infinite sequences~\cite{2014-nobit,CaludePoznanovic}.)



Alas, intentionality may turn out to be difficult or even impossible to prove.
How can one intrinsically decide between chance on the one hand, and providence, or some agent executing free will through the gap interface, on the other hand?
The interface must in both cases employ gaps in the intrinsic laws of the universe,
thereby allowing steering and communicating with it in a feasible, consistent manner.
That excludes any kind of absolute predictability of the signals emanating from it.
(Otherwise, the behavior across the interface would be predictable and deterministic.)
Hence, for an embedded observer~\cite{toffoli:79} employing intrinsic  means
which are operationally available in his universe~\cite{svozil-94},
no definite criterion can exist to either prove or falsify claims regarding mere
chance (by {\it creatio ex nihilo}) {\it versus} the free choice of an agent.
Both cases
--
free will of some agent as well as complete chance
--
express themselves by irreducible intrinsic indeterminism.

For the sake of an example, suppose for a moment that
we would possess a sort of {\em `Ark of the Covenant,'}
an oracle which communicates to us the will of the beyond, and, in particular, of divinity.
How could we be sure of that?  (Sarfatti, in order to investigate the paranormal, attempted to built what he called an  {\em  Eccles telegraph} by connecting a
radioactive source to a typewriter.)
This situation is not dissimilar to problems in recognizing hypercomputation, that is, computational capacities beyond universal computation~\cite{2007-hc}.



\section{Physical gaps}

If we translate the algorithmic metaphors mentioned earlier into the context of our own universe,
we have to observe whether all the respective components are physically feasible.
In particular, we need to ask the following questions:
(i) What might serve as a context; that is, do there exist natural laws which could be identified with the game universe/operating/real-time computer system?
(ii) Do there exist potential interfaces in our universe allowing communication with some (supposedly transcendental) agent?
(iii) Are there constraints on such interventions~\cite{maryland,greenberger-svozil,svozil-07-physical_unknowables}?
We may also speculate about the transcendental nature of any agent communicating with our universe {\em via} such interface.

The first question, in particular, the existence of suitable  {\em gaps in the natural laws}
and the causal fabric of the universe, has been investigated by Frank~\cite[Chapter~{III}, Sec.~12]{frank,franke},
as well as by more recent research~\cite{Russel-nioda-1}.
Several physical gap constructions will be critically reviewed next.





\subsection{Deterministic chaos and spontaneous symmetry breaking}

Already in 1873, Maxwell identified a certain kind of {\em instability} at {\em singular points}
as rendering a gap in the natural laws~\cite[211-212]{Campbell-1882}:
{\em ``$\ldots$~when an infinitely small variation in the present state may bring about a finite difference in the state of the
system in a finite time, the condition of the system is said to be unstable.
It is manifest that the existence of unstable conditions renders impossible the prediction of future events, if our
knowledge of the present state is only approximate, and not accurate.
$\ldots$~the system has a quantity of potential energy, which is
capable of being transformed into motion, but which cannot begin to be so transformed till the system has reached
a certain configuration, to attain which requires an expenditure of work, which in certain cases may be
infinitesimally small, and in general bears no definite proportion to the energy developed in consequence thereof.''}

Fig.~\ref{fig:2014-fw-instability} (see also Frank's figure 1 in Chapter~{III}, Section~13) depicts a one dimensional gap configuration envisioned by Maxwell: a
{\em ``rock loosed by frost and balanced on a singular point of the mountain-side, the little spark which
kindles the great forest,~$\ldots$''}
On top, the rock is in perfect balanced symmetry.
A small perturbation or fluctuation causes this symmetry to be broken,
thereby pushing the rock either to the left or to the right hand side of the potential divide.
This dichotomic alternative can be coded by $0$ and by $1$, respectively.
        \begin{figure}
                \begin{centering}
\unitlength 3mm % = 2.845pt
\linethickness{0.4pt}
\ifx\plotpoint\undefined\newsavebox{\plotpoint}\fi % GNUPLOT compatibility
\begin{picture}(9,7)(0,0)
\thicklines
\put(0,0){\color{blue}\line(1,0){3.0}}
\put(9,0){\color{blue}\line(1,0){3.0}}
\put(3,0){\color{orange}\line(1,2){3.0}}
\put(9,0){\color{orange}\line(-1,2){3.0}}
\put(6,6.9){\color{black}\circle*{2}}
\put(1.5,1){\color{gray}\circle*{2}}
\put(10.5,1){\color{gray}\circle*{2}}
\put(1.5,1){\color{white}\makebox(0,0)[cc]{$0$}}
\put(10.5,1){\color{white}\makebox(0,0)[cc]{$1$}}
\end{picture}
                \end{centering}
                \caption{(Color online) A gap created by a black particle sitting on top of a potential well.
The two final states are indicated by grey circles. Their positions can be coded by $0$ and $1$, respectively.}
                \label{fig:2014-fw-instability}
        \end{figure}

One may object to this scenario of {\em spontaneous symmetry breaking}
by maintaining that, if indeed the symmetry is perfect, there is no movement,
and the particle or rock stays on top of the tip (potential).
Any slightest movement might either result from a microscopic asymmetry of the initial state of the particle,
or from fluctuations of any form, either in the particle's position due top quantum zero point fluctuations,
or by the surrounding environment of the particle.
For instance, any collision of gas molecules with the rock may push the latter over the edge
by thermal fluctuations.

Moreover, {\em deterministic chaos} is not indeterministic at all:
the randomness resides in the {\em assumption} of the continuum from which the initial value is `drawn' (just like an urn).
In this case, almost all (of measure one) initial values
are not representable by any algorithmically compressible number;
that is,  they are random~\cite{MartinL�f1966602,calude:02}.
Deterministic chaos
unfolds the information contained therein by a recursively enumerable (computable),
deterministic evolution function.
If the continuum assumption is dropped, then what remains is Maxwell's
and Poincar{\'e}'s observation of the {\em epistemic} unpredictability
of the behavior of a deterministic system
due to instabilities and diverging evolutions from almost identical initial state.


\subsection{Quantum oracles}

A quantum mechanical gap can be realized by a {\em half-silvered mirror}~\cite{svozil-qct,stefanov-2000,zeilinger:qct},
with a 50:50 chance of transmission and reflection,
as depicted in Fig.~\ref{fig:2014-fw-qcointoss}.
A gap certified by quantum value indefiniteness necessarily has to operate with more than two exclusive outcomes~\cite{PhysRevA.89.032109}.
Ref.~\cite{2012-incomput-proofsCJ} presents such a qtrit configuration.
        \begin{figure}
                \begin{centering}
%\grade{\on}
%\emlines{\off}
%\epic{\off}
%\beziermacro{\on}
%\reduce{\on}
%\snapping{\off}
%\pvinsert{% Your \input, \def, etc. here}
%\quality{8.000}
%\graddiff{0.005}
%\snapasp{1}
%\zoom{8.0000}
\unitlength 0.7mm % = 2.845pt
\linethickness{0.4pt}
\ifx\plotpoint\undefined\newsavebox{\plotpoint}\fi % GNUPLOT compatibility
\begin{picture}(94.75,44.875)(0,0)

\put(10,10){\circle{10}}
\thinlines
%\emline(6.452,6.452)(13.568,13.523)
\multiput(6.452,6.452)(.033885714,.033671429){210}{\line(1,0){.033885714}}
%\end
%\emline(13.523,6.452)(6.408,13.523)
\multiput(13.523,6.452)(-.033880952,.033671429){210}{\line(-1,0){.033880952}}
%\end

\thicklines
{\color{blue}
\put(15,10){\line(1,0){44.5}}
%\dottedline(59.75,10)(94.75,10)
\multiput(59.68,9.93)(.972222,0){37}{{\rule{.8pt}{.8pt}}}
%\end
%\dottedline(59.875,9.875)(59.875,44.875)
\multiput(59.805,9.805)(0,.972222){37}{{\rule{.8pt}{.8pt}}}
%\end
}
\put(30,10){\color{black}\circle*{8}}
\put(59.68,39.93){\color{gray}\circle*{8}}
\put(89.68,9.93){\color{gray}\circle*{8}}
\put(59.68,39.93){\color{white}\makebox(0,0)[cc]{$0$}}
\put(89.68,9.93){\color{white}\makebox(0,0)[cc]{$1$}}

\thicklines
{\color{orange}
%\dashline{1}(50,0)(70,20)
\multiput(49.93,-.07)(.0322581,.0322581){20}{\line(1,0){.0322581}}
\multiput(51.22,1.22)(.0322581,.0322581){20}{\line(1,0){.0322581}}
\multiput(52.51,2.51)(.0322581,.0322581){20}{\line(1,0){.0322581}}
\multiput(53.801,3.801)(.0322581,.0322581){20}{\line(0,1){.0322581}}
\multiput(55.091,5.091)(.0322581,.0322581){20}{\line(0,1){.0322581}}
\multiput(56.381,6.381)(.0322581,.0322581){20}{\line(1,0){.0322581}}
\multiput(57.672,7.672)(.0322581,.0322581){20}{\line(1,0){.0322581}}
\multiput(58.962,8.962)(.0322581,.0322581){20}{\line(0,1){.0322581}}
\multiput(60.252,10.252)(.0322581,.0322581){20}{\line(0,1){.0322581}}
\multiput(61.543,11.543)(.0322581,.0322581){20}{\line(0,1){.0322581}}
\multiput(62.833,12.833)(.0322581,.0322581){20}{\line(1,0){.0322581}}
\multiput(64.123,14.123)(.0322581,.0322581){20}{\line(1,0){.0322581}}
\multiput(65.414,15.414)(.0322581,.0322581){20}{\line(0,1){.0322581}}
\multiput(66.704,16.704)(.0322581,.0322581){20}{\line(0,1){.0322581}}
\multiput(67.994,17.994)(.0322581,.0322581){20}{\line(0,1){.0322581}}
\multiput(69.285,19.285)(.0322581,.0322581){20}{\line(0,1){.0322581}}
%\end
}
\end{picture}
                \end{centering}
                \caption{(Color online) A gap created by a quantum coin toss. A single quantum (symbolized by a black circle
from a source (left crossed circle)
impinges on a semi-transparent mirror (dashed line), where it is reflected and transmitted with a 50:50 chance.
The two final states are indicated by grey circles. The exit ports of the mirror can be coded by $0$ and $1$, respectively.}
                \label{fig:2014-fw-qcointoss}
        \end{figure}


One may object to the orthodox view~\cite{zeil-05_nature_ofQuantum} of {\em quantum indeterminism} by pointing out
that it is merely based on a believe
--
actually, Born's {\em inclinations
``to give up determinism in the world of atoms''}~\cite[p.~866]{born-26-1}
(English translation in~\cite[p.~54]{wheeler-Zurek:83}).
Strictly speaking, any such claims (as well as the converse claims of absolute determinism) are provable improvable~\cite{svozil-2013-omelette}.

Furthermore, it is not at all clear where exactly the randomness generated by a half-siver mirror resides; that is,
where the stochasticity comes from, and what are its origin.
(Often vacuum fluctuations originating from the second, empty, input port are mentioned,
but, pointedly stated~\cite[p.~249]{chau}, these {\em ``mysterious vacuum fluctuations $\ldots$ may be regarded as sugar coating for the bitter pill of
quantum theory.''})

More generally, any irreversible measurement process,
and, in particular,
any associated `collapse,' or, by another denomination, `reduction' of the quantum state (or the wave function) to the post-measurement state
is a postulate which appears to be {\em means relative} in the following sense.

The beam splitter setup (without which way detector) is not irreversible at all
because a 50:50 mirror has a quantum mechanical representation as a permutation of the state,
such as a unitary Hadamard transformation;
that is, it is totally deterministic, and one-to-one.
(Experimentally, this can be demonstrated by serially concatenating two such 50:50 mirrors so that the output ports of the first mirror
are the input ports of the second mirror. The result (modulo an overall phase) is a Mach-Zehnder interferometer reconstructing the original
quantum state of the particle.)

Formally -- that is, within quantum theory proper, augmented by the prevalent orthodox `Kopenhagen-type' interpretation --
it is not too difficult to locate the origin of randomness in the beam splitter configuration:
it is (i) the possibility that a quantum state can be in a {\em coherent superposition} of classically distinct and mutually exclusive states;
and (ii) the possibility that an {\em irreversible measurement} {\it ad hoc} and {\it ex nihilo} stochastically `chooses' or `selects' one of these
classically mutually exclusive properties, associated with a measurement outcome. This, according to the orthodox interpretation
of quantum mechanics, is an irreducible indeterministic many-to-one process --
it transforms the coherent superposition of a multitude of (classically distinct) properties into a single, classical property.
This latter assumption (ii) is sometimes referred to as the {\em reduction postulate.}
Note that the multitude of states are all associated with distinct potential measurement outcomes of the measurement;
another type of measurement or observable would have other distinct potential measurement outcomes; yet all such potential observables --
with their respective potential measurement outcomes -- are assumed to ontologically exist simultaneously.

Already
Schr\"odinger has expressed his dissatisfaction with both assumptions (i) and (ii), and, in particular, with the quantum mechanical concept of
ontological existence of
{\em coherent superposition}, in various forms at various stages of his life:
he polemicized against (i) by quoting the burlesque situation of a cat which is supposed to be in a coherent superposition between death and life~\cite{schrodinger}.
He also noted the curious fact that, as a consequence of (i) and
in the absence of measurement and state reduction (ii), according to quantum mechanics we all (as well as the physical universe in general),
would become quantum jelly~\cite{schroedinger-interpretation}.

Alas, what in the orthodox scriptures of quantum mechanics often is referred to as `irreversible measurement' remains conceptually unclear,
and is inconsistent with other parts of quantum theory.
Indeed, it is not even clear that, ontologically, an irreversible measurement exists!
Wigner~\cite{wigner:mb} and, in particular, Everett~\cite{everett,everett-collw} put forward ontologic arguments against irreversible measurements
by extending the cut between a quantum object and the classical measurement apparatus to include both object
{\em as well as} the measurement apparatus in a uniform quantum description.
As this latter situation is described by a permutation (i.e. by a unitary transformation),
irreversibility, and what constitutes `measurement' is lost.
Indeed, the reduction postulate (ii) and the uniform unitarity of the quantum evolution cannot both be true, because the former
essentially yields a many-to-one mapping of states, whereas uniform unitarity merely amounts to a one-to-one mapping, that is, a permutation, of states.
In no way can a many-to-one mapping `emerge' from any sort of concatenation of one-to-one mappings!
Stated differently, according to the reduction postulate (ii), information is lost;
whereas, according to the unitary state evolution, no information is ever lost.
So, either one of these postulates must be ontologically wrong (they may be epistemically justified {\em for all practical purposes}~\cite{bell:a1}, though).
In view of this situation, I am (to use Born's dictum~\cite[p.~866]{born-26-1})
inclined to give up the reduction postulate disrupting permutativity, and, in particular,
unitarity, in the world of single quantum phenomena, in favor of the latter; that is, in favor of  permutativity, and, in particular,
unitarity.



The effort to do so may be high, as detailed beam recombination analysis of  a Stern-Gerlach device (the spin analogue of a beam splitter in the Mach-Zehnder interferometer)
shows~\cite{engrt-sg-I,engrt-sg-II}.
Nonetheless, experiments (and proposals) to `undo' quantum measures
abound~\cite{PhysRevD.22.879,PhysRevA.25.2208,greenberger2,Nature351,Zajonc-91,PhysRevA.45.7729,PhysRevLett.73.1223,PhysRevLett.75.3783,hkwz}.
Thus we could say that {\em for all practical purposes}~\cite{bell-a},
that is, {relative to the physical means}~\cite{Myrvold2011237} available to resolve the huge number of degrees of freedom involving a
macroscopic measurement apparatus, measurements {\em appear to be} irreversible, but a close enough look reveals that they are not.
So, irreversibility of quantum measurements appears to be epistemic and means relative, subjective and conventional; but not ontic.
(As already argued by Maxwell, this is just the same for the second law of thermodynamics~\cite{Myrvold2011237}.)



\subsection{Vacuum fluctuations}

As stated by Milonni~\cite[p.~xiii]{milonni-book} and others~\cite{einstein-aether,dirac-aether}, {\em ``$\ldots$~there is no vacuum in the ordinary sense of
tranquil nothingness. There is instead a fluctuating quantum vacuum.''}
One of the observable vacuum effects is the {\em spontaneous emission of radiation}~\cite{Weinberg-search}:
{\em ``$\ldots$~the process of spontaneous emission of radiation is one in which ``particles'' are actually created.
Before the event, it consists of an excited atom, whereas after the event, it consists of an atom in a state of lower energy, plus a photon.''}
Recent experiment achieve single photon production by spontaneous emission~\cite{PhysRevLett.39.691,PhysRevLett.85.290,Buckley-12,Stevenson-spontemi,Sanguinetti},
for instance by electroluminescence.
Indeed, most of the visible light emitted by the sun or other sources of blackbody radiation, including incandescent bulbs,
is due to spontaneous emission~\cite[p.~78]{milonni-book} and thus subject to {\em creatio ex nihilo}.


A gap based on vacuum fluctuations is schematically depicted in Fig.~\ref{fig:2014-fw-vacuumfluctuation}.
It consists of an atom in an excited state, which transits into a state of lower energy, thereby producing a photon.
The photon (non-)creation can be coded by the symbols $0$ and $1$, respectively.
        \begin{figure}
                \begin{centering}
% This is a LaTeX picture output by TeXCAD.
% File name: [1.pic].
% Version of TeXCAD: 4.3
% Reference / build: 30-Jun-2012 (rev. 105)
% For new versions, check: http://texcad.sf.net/
% Options on the following lines.
%\grade{\on}
%\emlines{\off}
%\epic{\off}
%\beziermacro{\on}
%\reduce{\on}
%\snapping{\off}
%\pvinsert{% Your \input, \def, etc. here}
%\quality{8.000}
%\graddiff{0.005}
%\snapasp{1}
%\zoom{4.0000}
\unitlength 0.6mm % = 2.845pt
\linethickness{0.4pt}
\ifx\plotpoint\undefined\newsavebox{\plotpoint}\fi % GNUPLOT compatibility
\begin{picture}(79.526,40)(0,0)
\thicklines
\put(0,0){\color{blue}\line(1,0){50}}
\put(0,40){\color{orange}\line(1,0){50}}
\put(25,40){\color{gray}\vector(0,-1){39}}
\thinlines
{\color{gray}
\qbezier(30,20)(35,13)(40,20)
\qbezier(70,20)(65,27)(60,20)
\qbezier(50,20)(45,27)(40,20)
\qbezier(50,20)(55,13)(60,20)
\put(70,20){\vector(1,-1){2}}
}
\put(77,20){\color{gray}\circle*{8}}
\put(77,20){\color{white}\makebox(0,0)[cc]{$1$}}
\end{picture}
                \end{centering}
                \caption{(Color online) A gap created by the spontaneous creation of a photon.}
                \label{fig:2014-fw-vacuumfluctuation}
        \end{figure}


It might not be too unreasonable to speculate that all gap scenarios, including spontaneous symmetry breaking and quantum oracles, are ultimately based on vacuum fluctuations.

%\subsection{Miscellaneous proposals}


\section{Provable undecidabilities}

One may ask if there is some {\em emergence of undecidability}
if one considers systems of high complexities.
Indeed, as recursion theory shows, this is the case: once a system is capable of expressing self-substitution (reflexion),
Peano arithmetic, universal computation, or recursive functions,
certain true statements (in particular, its consistency) cannot be proven by intrinsic methods and means alone.
As G\"odel stated~\cite[55]{v-neumann-66} (see also~\cite[p.~554]{fef-84}),
 {\em
``a complete epistemological description
 of a language $A$ cannot be given in the same language $A$, because
 the concept of truth of sentences of $A$ cannot be defined in $A$. It
 is this theorem which is the true reason for the existence of
 undecidable propositions in the formal systems containing arithmetic.''}

Alas, as fascinating as theses epistemic issues appear,
they may not be relevant for ontologic gap constructions or NIODA
-- except, perhaps, the speculation that the universe has been created to solve undecidable problems by actually simulating the computation.
But then, why should divinity not just be able to {\em know} the solution of, say, any halting problem?
Divinity certainly is not bounded by any intrinsic formal means!
(This is related to proof theoretical questions of why is it better to be able to prove something rather than to know it.
Certain proofs are nonconstructive or totally unusable for any practical purposes.)

In any case, there seems no room for divine interference/interaction
because everything in these scenarios is deeply deterministic; its just that we provably cannot know what the outcome of a `complex' deterministic process is.


\subsection{Unpredictability}

For any deterministic system strong enough to support
universal computation,  the general forecast or prediction
problem is provable unsolvable.
This proposition will be argued by reduction to the halting problem, which is provable unsolvable.

A clear distinction should be made between {\em determinism}
(such as {\em computable evolution laws}) and {\em predictability}~\cite{suppes-1993}.
Determinism does not exclude unpredictability in the long run.
The local (temporal), step-by-step evolution of the system can be perfectly deterministic and computable,
whereas recursion-theoretic unknowables correspond to global observables at unbounded time scales.
Indeed, (nontrivial) provable unpredictability requires determinism,
because formalized proofs require formal systems or algorithmic behavior.

\subsection{Undecidability of the induction problem}

Induction, in physics, is the inference of general rules
dominating and generating physical behaviors from these behaviors alone.
For any deterministic system strong enough to support
universal computation, the general induction problem
is provable unsolvable.
Induction is thereby reduced to the insolvability of
the rule inference problem~\cite{go-67,blum75blum,angluin:83,ad-91,li:92}
of identifying a rule or law reproducing the behavior of a deterministic system
by observing its input-output performance by purely algorithmic means
(not by intuition).

Informally, the algorithmic idea of the proof is to take any sufficiently powerful
rule or method of induction and, by using it, to define some
functional behavior that is not identified by it.
This amounts
to constructing an algorithm which
(passively)
fakes the guesser by simulating some particular function
until the guesser
pretends to be able to guess the function correctly.
In a second,  diagonalization step, the faking algorithm then switches to a different
 function to invalidate the guesser's guess.

\subsection{Results in classical recursion theory with implications for theoretical physics}


The following theorems of recursive  analysis~\cite{aberth-80,Weihrauch} have some
implications for theoretical physics~\cite{kreisel}:
(1)
There exist recursive monotone bounded sequences of rational numbers
whose limit is no computable number
\citep{Specker49}.
A concrete example of such a number is Chaitin's Omega number~\cite{chaitin3,calude:02,calude-dinneen06},
the halting probability for a computer (using prefix-free code),
which can be defined by a sequence of rational numbers
with no computable rate of convergence.
(2)
There exist a recursive real function which has its maximum in the unit interval
at no recursive real number~\cite{Specker57}.
This has implications for the principle of least action.
(3)
There exists a real number $r$ such that $G(r) = 0$ is recursively undecidable for $G(x)$
in a class of functions which involves polynomials and the sine function
\citep{wang}.
This, again, has some bearing on  the principle of least action.
(4)
There exist incomputable solutions of the wave equations for computable initial values
\citep{pr1,bridges1}.
(5)
On the basis of theorems of recursive analysis~\cite{Scarpellini-63,richardson68},
many questions in dynamical systems theory are provable undecidable~\cite{1985cfd..book.....F,dc-d93,Stewart-91,calude:037103}.


\section{Reprogramming the universe}

So far, all that physics has attempted is preparing physical states and devices capable to manipulate such states in certain ways so that,
by causality, a desirable physical state evolution follows.
In algorithmic terms, this is like feeding the appropriate input into some pre-defined computer,
and processing this input by a pre-defined algorithm to obtain some desired output.
Pointedly stated, so far physics has employed merely a pocket calculator, initially provided -- through {\it creatio ex nihilo} -- by divinity.

A next step would be to {\em change the laws of nature} themselves; that is, in algorithmic language, by {\em reprogramming the universe.}
I suggest to call this type of manipulation {\em ontologic magic} (in contrast to the {\em epistemic magic} performed by professional magicians),
or just {\em magic}.
Of course, magic requires the universe to be programmable; and the natural laws to be mutable.
This, I speculate, might achieve an explanation for the Resurrection of Jesus in modern terms, which, as has been pointed out by Russel,
should not be related to NIODA~\cite[Sect.~3.C.7]{Russel-nioda-1}.




\section{Afterthoughts}


A possibility for providence, free will and oracles will based on dualism has been proposed.
This scenario avoids the problems encountered in totally (in)deterministic universes
by a fine-tuned mixture of determinism and indeterminism, thereby
allowing gaps in the natural laws
in which intentions and choices can be communicated via interfaces serving as Cartesian cuts.
By intrinsic means alone, any such signals might not be differentiable from irreducible chance.


These considerations are related to other, logic constraints on free will and omniscience
discussed elsewhere~\cite{maryland,greenberger-svozil,svozil-07-physical_unknowables}
which can be derived by diagonalization and reduction to the halting problem.
Because the simultaneous enactment of omniscience, omnipotence on the one hand,
as well as free will on the other hand, are inconsistent:
any agent commanding the omniscience and omnipotence may freely choose to counteract its own predictions.
Thus consistency demands the absence of at least one of these features.

We do not suggest that the existence of the aforementioned gaps are either necessary or sufficient for the occurrence
of providence, free will, miracles or oracles. At the moment there does not seem to exist any consolidated gap mechanism, say, for providence
or free will. Even its location is unclear, although most authors seem to agree that, for free will of human nature, it must be the human brain.
To mention an anecdote: connecting a typewriter to a radioactive source has turned out to be not very helpful.

Nevertheless, if we remain within the orthodox interpretation of quantum mechanics, irreversible state reductions present a viable gap mechanism.
Irreversible state reductions are based on postulates that during an irreversible measurement of
any quantum state which is in a coherent superposition of the possible properties measured, any individual quant
(together with the disposition of the classical measurement device)  {\it ex nihilo} `chooses' exactly one such property; and this `choice' is irreducible
(that is ontologically and not merely epistemically) indeterministic.

Both these gap mechanisms -- quantum fluctuations as well as irreversible measurements -- could, at least in principle and highly speculatively, serve as an interface facilitating
free will as well other as non-interventionist objective transcendental agent interactions with our physical universe.
I personally would tend to favor quantum field theoretic fluctuations as gap mechanism, but I have no strong opinion on any such gap mechanism. In view of the importance of these subjects, I strongly encourage experimental research into these options. Maybe someday we will be in the position to  either falsify or corroborate the hypotheses.

Whether a human individual (i) either takes the stance of the current quantum orthodoxy that certain individual quantum events occur irreducible at random,
(ii) or, alternatively, that, at least sometimes, they signify  non-interventionist objective transcendental agent action,
(iii) or, still alternatively, one insists on total determinism,
remains a very personal choice.
Alas, it is my strong inclination that, in view of the enigma of existence, nothing should be outrightly rejected,
and all options carefully examined, in particular, also by scientific methods.


\begin{acknowledgments}
This research has been partly supported by FP7-PEOPLE-2010-IRSES-269151-RANPHYS.
The manuscript has been prepared for a research grant {\em  SATURN: Scientific and Theological Understandings of Randomness in Nature,}
offered through  Jim Bradley at Calvin College and  Bob Russell at the Graduate Theological Union and  the Center for Theology and the Natural Sciences, Berkeley, California.
All misconception and errors are mine -- {\it Confiteor $\ldots$ quia peccavi nimis cogitatione, verbo, opere et omissione:
mea culpa, mea culpa, mea maxima culpa $\ldots$}
\end{acknowledgments}

\bibliography{svozil}

\end{document}
