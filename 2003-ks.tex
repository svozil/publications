%%tth:\begin{html}<LINK REL=STYLESHEET HREF="/~svozil/ssh.css">\end{html}
\documentclass[pra,preprint,amsfonts]{revtex4}
%\documentclass[pra,showpacs,showkeys,amsfonts]{revtex4}
\usepackage{graphicx}
%\documentstyle[amsfonts]{article}
\RequirePackage{times}
\RequirePackage{courier}
\RequirePackage{mathptm}
%\renewcommand{\baselinestretch}{1.3}
\textheight 21 true cm
\begin{document}

%\def\frak{\cal }
%\sloppy



\title{The Kochen-Specker Theorem\\
{\tt http://tph.tuwien.ac.at/$\;\widetilde{ }$svozil/publ/2003-ks.\{pdf,ps,dvi,tex\}}}

\author{Karl Svozil}
\email{svozil@tuwien.ac.at}
\homepage{http://tph.tuwien.ac.at/~svozil}
\affiliation{Institut f\"ur Theoretische Physik, University of Technology Vienna,
Wiedner Hauptstra\ss e 8-10/136, A-1040 Vienna, Austria}

\maketitle

\tableofcontents

\clearpage

\section{Brief history, context \& motivation}


\subsection{Kolmogorov's axioms of classical probability theory (1933) \cite{kolmogorov2}}

Let $e_i$ be an event in a set $M $
 and let $h$ stand for some
  hypothesis, then $P(e_i)$ is the probability of obtaining
 the event $e_i$.

For all $e_i\subset M $ and a certain event
 ${\bf 1}$,
 \begin{eqnarray*}
 0\le P(e_i)&\le & 1\\
 P({\bf 1})&=&1\\
 P(e_1,e_2,\ldots )&=& P(e_1)+P(e_2)+\cdots \\
 & & \qquad {\rm for\;pairwise\;disjoint\;} e_i\quad  .
 \end{eqnarray*}

\clearpage

\subsection{Von Neumann's axioms of quantum probabilities and expectation values (1932) \cite{v-neumann-49}}

The quantum expectation value of a self-adjoint operator $A$ is
$$\langle A\rangle ={\rm Trace}(\rho A)$$
with $\rho$ being a quantum state operator which must be
(i)  $\rho^\dagger =\rho$,
(ii) of trace class ${\rm tr} (\rho )=1$, and
(iii) positive semidefinite
$\langle u | \rho | u \rangle \ge 0$
(in another notation, $u^T\rho u\ge 0$)
for all vectors $u$ in the Hilbert space.
For the state to be pure, it must be a projector $\rho^2=\rho$,
or equivalently, ${\rm tr}(\rho^2)=1$.

Note 1: If $A$ is a (pure) state operator $\psi$,
then  $\langle \psi \rangle ={\rm Trace}(\rho \psi)$  is
the probability to find $\psi$ if the system is in state $\rho$.

Note 2: If $\rho$ is {\em pure,} then it can be written as a projection
along a vector ${\vec x}$; i.e.,
$\vert {\vec x} \rangle \langle {\vec x} \vert $, and
$$\langle A\rangle
={\rm Trace}\Large(\; \vert {\vec x} \rangle \langle {\vec x} \vert  A \; \Large)
=\langle {\vec x} \vert  A \vert {\vec x} \rangle .
$$

If $A$ is a pure state itself; i.e., $A=\psi = \vert {\vec y} \rangle \langle {\vec y} \vert $,
then
$\langle A\rangle =
\langle {\vec x} \vert  {\vec y} \rangle \langle {\vec y} \vert {\vec x} \rangle
=
\vert \langle {\vec x} \vert  {\vec y} \rangle  \vert^2$
is the probability to find the system in state $\psi$ if it  was prepared in state $\rho$.

Note 3: Quantum probabilities are based on Hilbert space entities.
This is different from classical probability theory,
whose axioms are based on Boolean algebra.


\clearpage

\subsection{Gleason's theorem (1957) \cite{Gleason}}

{\em Heuristically speaking, Gleason's theorem states that,
as long as Kolmogorov's axioms hold for comeasurable events (associatiated with
commuting observables),
then Von Neumann's  quantum probabilities and expectation values hold.}

Let there be  projectors in a Hilbert space standing for elementary (yes-no) tests;
and suppose that compatible tests correspond to commuting operators.

If, for all orthogonal projectors
(associated with comeasurable observables)
$E$ and $F$
in Hilbert spaces of dimension higher than two,
the sum $G=E+F$ (which is also a projector)
has expectation value
$$
\langle G\rangle
=
\langle E\rangle
+
\langle F\rangle
.
$$
then $\langle E\rangle   ={\rm Trace}(\rho E)$.

Note 1: No trivial result, the proof is intricate; Gleason has been one of the most outstanding mathematicians at the time.

Note 2: Inspired many mathematicians \& physicists; among them Mackey \cite{mackey:63}, Specker, Zierler \& Schlessinger,  and Bell.

Note 3: Was recently constructively proven by Bridge \& Richman \cite{rich-bridge},
as well as Hrushovski \& Pitowsky \cite{hru-pit-2003}.


\clearpage

\subsection{The Kochen-Specker theorem}

\subsubsection{The quantum jigsaw puzzle}

Consider a particular finite set of orthogonal tripods.
The task is to colorize the tripods with two colors
``red'' \& ``green'' according to the following rules:\\
(i) every tripod has one red and two green legs;\\
(ii) a leg belonging to more than one tripod has the same color everywhere.
\begin{figure}[hb]
\begin{center}
%TexCad Options
%\grade{\off}
%\emlines{\off}
%\beziermacro{\off}
%\reduce{\on}
%\snapping{\off}
%\quality{2.00}
%\graddiff{0.01}
%\snapasp{1}
%\zoom{7.45}
\unitlength 0.80mm
\linethickness{0.4pt}
\begin{picture}(107.62,83.81)
\put(15.48,83.81){\line(1,-1){10.00}}
\put(25.48,73.81){\line(-1,-3){5.00}}
\put(20.48,58.81){\line(-3,-2){15.00}}
\put(5.48,48.81){\line(-1,3){5.00}}
\put(0.48,63.81){\line(1,3){5.00}}
\put(5.48,78.81){\line(2,1){10.00}}
\put(15.48,73.81){\line(-1,-2){5.00}}
\put(10.48,63.81){\line(1,-1){5.00}}
\put(10.48,63.81){\line(-1,0){5.00}}
\put(84.05,15.24){\line(-1,1){10.00}}
\put(74.05,25.24){\line(-3,-1){15.00}}
\put(59.05,20.24){\line(-2,-3){10.00}}
\put(49.05,5.24){\line(3,-1){15.00}}
\put(64.05,0.24){\line(3,1){15.00}}
\put(79.05,5.24){\line(1,2){5.00}}
\put(74.05,15.24){\line(-2,-1){10.00}}
\put(64.05,10.24){\line(-1,1){5.00}}
\put(64.05,10.24){\line(0,-1){5.00}}
\put(50.48,73.81){\line(-1,-2){5.00}}
\put(45.48,63.81){\line(2,-1){20.00}}
\put(65.48,53.81){\line(-1,3){5.00}}
\put(60.48,68.81){\line(3,1){15.00}}
\put(75.48,73.81){\line(-3,2){15.00}}
\put(60.48,83.81){\line(-1,0){10.00}}
\put(50.48,83.81){\line(0,-1){10.00}}
\put(55.48,78.81){\line(0,-1){10.00}}
\put(55.48,68.81){\line(1,-1){5.00}}
\put(55.48,68.81){\line(-1,-1){5.00}}
\put(33.81,14.29){\line(1,2){5.00}}
\put(38.81,24.29){\line(-2,1){20.00}}
\put(18.81,34.29){\line(1,-3){5.00}}
\put(23.81,19.29){\line(-3,-1){15.00}}
\put(8.81,14.29){\line(3,-2){15.00}}
\put(23.81,4.29){\line(1,0){10.00}}
\put(33.81,4.29){\line(0,1){10.00}}
\put(28.81,9.29){\line(0,1){10.00}}
\put(28.81,19.29){\line(-1,1){5.00}}
\put(28.81,19.29){\line(1,1){5.00}}
\put(30.48,53.81){\line(-1,-3){5.00}}
\put(25.48,38.81){\line(5,-1){25.00}}
\put(50.48,33.81){\line(1,1){10.00}}
\put(60.48,43.81){\line(-4,3){20.00}}
\put(40.48,58.81){\line(-2,-1){10.00}}
\put(35.48,48.81){\line(2,-1){10.00}}
\put(45.48,43.81){\line(1,1){5.00}}
\put(45.48,43.81){\line(0,-1){5.00}}
\put(10.71,63.81){\line(1,2){5.00}}
\put(10.24,63.81){\line(1,2){5.00}}
\put(45.48,43.57){\line(-2,1){10.00}}
\put(45.48,44.05){\line(-2,1){10.00}}
\put(55.00,68.57){\line(-1,-1){4.76}}
\put(50.71,63.81){\line(1,1){4.76}}
\put(64.20,10.04){\line(2,1){10.24}}
\put(64.05,10.48){\line(2,1){10.00}}
\put(28.57,19.52){\line(0,-1){10.71}}
\put(29.05,8.81){\line(0,1){10.71}}
\put(100.00,50.00){\line(0,-1){10.00}}
\put(100.00,40.00){\line(5,-2){7.62}}
\put(100.00,40.02){\line(-5,-3){5.95}}
\put(98.32,33.33){\line(1,4){1.67}}
\put(99.98,40.00){\line(-2,1){5.71}}
\put(97.38,45.24){\line(1,-2){2.62}}
\put(100.00,40.00){\line(2,1){5.24}}
\put(75.00,40.00){\vector(1,0){10.00}}
\put(80.00,45.00){\makebox(0,0)[cc]{?}}
\put(37.14,72.86){\makebox(0,0)[cc]{$\cdots$}}
\put(50.00,23.81){\makebox(0,0)[cc]{$\cdots$}}
\put(74.05,61.90){\makebox(0,0)[cc]{$\cdots$}}
\put(6.67,30.71){\makebox(0,0)[cc]{$\cdots$}}
\put(102.86,35.24){\makebox(0,0)[cc]{$\cdots$}}
\put(103.33,46.19){\makebox(0,0)[cc]{$\ddots$}}
\end{picture}
\end{center}
\caption{\label{f-jigsaw}
The quantum jigsaw puzzle in three dimensions: is it possible to
consistently arrange into a whole certain (tripod)
pieces, only one such part
being actually measurable? Every tripod has a thick ``red'' leg and two thin ``green'' legs.}
\end{figure}

\clearpage

\subsubsection{Formulation of the theorem}
In Hilbert spaces of dimension higher than two,
it is impossible to associate definite numerical values, 0 or 1 (associated
with truth values of elements of physical existence {\em false} and {\em true}, respectively),
with every projection operator $E_i$ in such a way that, every if every set of commuting
$E_i$ satisfies $\sum_i E_i ={\Bbb I}$, the corresponding values, amely $P(E_i)\in\{0,1\}$,
also satisfy  $\sum_i P(E_i) =1$.

Specker 1960  \cite{specker-60}:

{\em ``In einem gewissen Sinne geh\"oren aber auch die scholastischen
Spekulationen \"uber die $\langle \langle$Infuturabilien$\rangle \rangle$ hieher, das heisst die Frage,
ob sich die g\"ottliche Allwissenheit auch auf Ereignisse erstrecke,
die eingetreten w\"aren, falls etwas geschehen w\"are, was nicht geschehen ist. $\ldots$

$\ldots$
Ein elementargeometrisches Argument zeigt,
dass eine solche Zuordnung [[von Wahrheitswerten]] unm\"oglich ist,
und dass daher \"uber ein quantenmechanisches System (von Ausnahmef\"allen abgesehen)
keine konsistenten Prophezeiungen m\"oglich sind.''}

Formulation of proofs by Kamber \cite{kamber64,kamber65},
Zierler \& Schlessinger \cite{ZirlSchl-65},
Bell
\cite{bell-66},
Kochen\& Specker \cite{kochen1},
Alda \cite{Alda,Alda2},
Peres \cite{peres},
Mermin \cite{mermin-93},
$\ldots$
Tkadlec \&  KS  \cite{svozil-tkadlec},
Calude, Hertling \& KS \cite{CalHerSvo},
KS \cite[Chapter 7 \& 8]{svozil-ql}.




\clearpage


\section{Some derivations}

\subsection{Greechie ortogonality diagrams}

%TexCad Options
%\grade{\off}
%\emlines{\off}
%\beziermacro{\on}
%\reduce{\on}
%\snapping{\off}
%\quality{2.00}
%\graddiff{0.01}
%\snapasp{1}
%\zoom{1.00}
\unitlength 1mm
\linethickness{0.4pt}
\begin{picture}(131.05,50.00)
\put(30.00,20.00){\line(0,1){30.00}}
\put(30.00,20.00){\line(5,-4){16.67}}
\put(30.00,20.00){\line(-2,-1){19.33}}
\put(16.33,9.33){\makebox(0,0)[cc]{$x$}}
\put(47.00,12.00){\makebox(0,0)[cc]{$y$}}
\put(34.67,47.00){\makebox(0,0)[cc]{$z$}}
\put(74.33,31.67){\makebox(0,0)[cc]{{\Huge $\equiv$}}}
\put(90.00,30.00){\line(1,0){20.00}}
\put(110.00,30.00){\line(1,0){20.00}}
\put(130.00,30.00){\circle{2.11}}
\put(110.00,30.00){\circle{2.11}}
\put(90.00,30.00){\circle{2.11}}
\put(90.00,22.00){\makebox(0,0)[cc]{$x$}}
\put(110.00,21.67){\makebox(0,0)[cc]{$y$}}
\put(130.00,21.67){\makebox(0,0)[cc]{$z$}}
\end{picture}


\clearpage
\subsection{Peres' proof \cite{peres-91}}

%%%%%%%%%%%%%%%%%%%%%%%%%%
\let\charmi=A\let\charmii=B      % abbr. of coord.: A=-1, B=-2
\newcommand{\ex}[1]%
  {\let\char#1\ifx\char\charmi-1\else\ifx\char\charmii-2\else#1\fi\fi}

\unitlength .5mm
\newcommand{\emline}[4]%
  {\put(#1,#2){\special{em:moveto}}\put(#3,#4){\special{em:lineto}}}

\newsavebox{\vertex}\savebox{\vertex}{%      % vertex in diagrams
  {\unitlength1mm\begin{picture}(0,0)\put(0,0){\circle*{1}}\end{picture}}}
  \newcommand{\disc}{\usebox{\vertex}}
  \newcommand{\point}[2]{\put(#1,#2){\disc}}
\newsavebox{\subdiagram}\savebox{\subdiagram}{%   % circle in diagrams
  {\unitlength1mm\begin{picture}(0,0)\put(0,0){\circle{2}}\end{picture}}}
  \newcommand{\discbig}{\usebox{\subdiagram}}
\newcommand{\place}[6]%
  {\put(#1,#2){\hspace{#3pt}\raisebox{#4pt}{\makebox(0,0)[#5]{$#6$}}}}
\newsavebox{\shortdiagram}           % abbr. for the main logic
\savebox{\shortdiagram}{\raisebox{.25\baselineskip}{\begin{picture}(12,0)(-1,0)
  \put(0,0){\disc}\put(10,0){\disc}\put(5,0){\discbig}\put(0,0){\line(1,0){10}}
  \end{picture}}}
\newcommand{\edges}{       % of the main logic
  \put (-20,  0){\line( 1, 1){10}}  \put (-10,-10){\line(1,0){20}}
  \put (-20,  0){\line( 1,-1){10}}  \put (-10, 10){\line(1,0){20}}
  \put ( 20,  0){\line(-1,-1){10}}  \put (  0,-10){\line(0,1){20}}
  \put ( 20,  0){\line(-1, 1){10}}
  }
\newcommand{\vertices}{    % of the main logic
  \put (-15,-5){\disc}  \multiput (-10, 10)(10,0){3}{\disc}
  \put (-15, 5){\disc}  \multiput (-10,-10)(10,0){3}{\disc}
  \put ( 15, 5){\disc}  \multiput (-20,  0)(20,0){3}{\disc}
  \put ( 15,-5){\disc}
  }
%%%%%%%%%%%%%%%%%%%%%%%%%%


\begin{figure}[hb]
\unitlength .45\textwidth
\newcommand{\onethird}[3]{\axis#1\side#1#2\side#1#3\cross#2#3}
\newcommand{\axis}[4]%
  {\point#1\point#2\point#3\point#4\emline#1#3\emline#3{0}{0}}
\newcommand{\side}[8]%
  {\point#5\point#6\point#7\point#8\emline#1#5\emline#5#6\emline#2#8\emline#8#7}
\newcommand{\cross}[8]{\emline#1#7\emline#7#3\emline#3#5\emline#2{0}{0}}
\catcode`\!=\active  \def!{\bar1}
\begin {center}
\begin {picture}(2,2)(-1,-1)
% \input svtk.dat
\onethird
{{{ 0.100}{-0.995}}{{ 0.050}{-0.747}}{{ 0.000}{-0.500}}{{ 0.000}{-0.250}}}%
{{{ 0.643}{-0.766}}{{ 0.087}{-0.050}}{{ 0.100}{-0.250}}{{ 0.536}{-0.112}}}%
{{{-0.643}{-0.766}}{{-0.087}{-0.050}}{{-0.100}{-0.250}}{{-0.536}{-0.112}}}%
\onethird
{{{ 0.812}{ 0.584}}{{ 0.622}{ 0.417}}{{ 0.433}{ 0.250}}{{ 0.217}{ 0.125}}}%
{{{ 0.342}{ 0.940}}{{-0.000}{ 0.100}}{{ 0.167}{ 0.212}}{{-0.171}{ 0.520}}}%
{{{ 0.985}{-0.174}}{{ 0.087}{-0.050}}{{ 0.267}{ 0.038}}{{ 0.365}{-0.408}}}%
\onethird
{{{-0.912}{ 0.411}}{{-0.672}{ 0.330}}{{-0.433}{ 0.250}}{{-0.217}{ 0.125}}}%
{{{-0.985}{-0.174}}{{-0.087}{-0.050}}{{-0.267}{ 0.038}}{{-0.365}{-0.408}}}%
{{{-0.342}{ 0.940}}{{-0.000}{ 0.100}}{{-0.167}{ 0.212}}{{ 0.171}{ 0.520}}}%
\put (0,0){\circle{0.2}}
\put ( 0.000, 0.100){\discbig}
\put ( 0.050,-0.747){\discbig}
\put ( 0.217, 0.125){\discbig}
\footnotesize
\place { 0.100}{-0.995}{ 0}{-6}{ t}{2!!}
\place { 0.050}{-0.747}{ 6}{ 0}{l }{211}
\place { 0.000}{-0.500}{ 6}{ 0}{l }{01!}
\place { 0.000}{-0.250}{ 2}{ 3}{lb}{011}
\place { 0.000}{ 0.100}{ 0}{-4}{ t}{100}
\place { 0.100}{-0.250}{ 4}{-4}{lt}{2!1}
\place {-0.100}{-0.250}{-4}{-4}{rt}{21!}
\place { 0.643}{-0.766}{ 6}{ 0}{l }{102}
\place { 0.365}{-0.408}{ 6}{ 0}{l }{20!}
\place { 0.087}{-0.050}{ 4}{ 0}{lb}{010}
\place {-0.643}{-0.766}{-6}{ 0}{r }{120}
\place {-0.365}{-0.408}{-6}{ 0}{r }{2!0}
\place {-0.087}{-0.050}{-4}{ 0}{rb}{001}
\place { 0.812}{ 0.584}{ 6}{ 0}{lb}{!!2}
\place { 0.622}{ 0.417}{ 6}{ 0}{lt}{112}
\place {-0.912}{ 0.411}{-6}{ 0}{rb}{!2!}
\place {-0.672}{ 0.330}{-6}{ 0}{rt}{121}
\place { 0.433}{ 0.250}{ 3}{-3}{lt}{1!0}
\place { 0.217}{ 0.125}{-6}{ 1}{r }{110}
\place {-0.433}{ 0.250}{-3}{-3}{rt}{10!}
\place {-0.217}{ 0.125}{ 6}{ 1}{l }{101}
\place { 0.267}{ 0.038}{ 6}{ 0}{l }{1!2}
\place { 0.167}{ 0.212}{ 0}{ 6}{lb}{!12}
\place {-0.267}{ 0.038}{-6}{ 0}{r }{12!}
\place {-0.167}{ 0.212}{ 0}{ 6}{rb}{!21}
\place { 0.985}{-0.174}{ 4}{-6}{ t}{201}
\place { 0.536}{-0.112}{ 0}{-6}{lt}{!02}
\place {-0.985}{-0.174}{-4}{-6}{ t}{210}
\place {-0.536}{-0.112}{ 0}{-6}{rt}{!20}
\place { 0.342}{ 0.940}{ 0}{ 6}{ b}{021}
\place { 0.171}{ 0.520}{-4}{ 4}{rb}{0!2}
\place {-0.342}{ 0.940}{ 0}{ 6}{ b}{012}
\place {-0.171}{ 0.520}{ 4}{ 4}{lb}{02!}
\end {picture}
\end {center}
\caption{Greechie diagram of a set of propositions
embeddable in ${\Bbb R}^3$
 without any two-valued probability measure
  \protect\cite[Figure 9]{svozil-tkadlec}.
( 1$!$2 denotes ${\rm Span}(1,-1,\protect\sqrt2)$.)
\label{sk33}}
\end{figure}

\clearpage

Let us prove that there is no two-valued probability measure
\cite{svozil-tkadlec,tkadlec-96}.
Due to the symmetry of the problem, we can  choose a particular
coordinate axis such that, without loss of generality,
$P(100)=1$.
Furthermore, we may assume (case 1) that
$P(21\bar{1}) = 1$.
It immediately follows that $P(001) = P(010) =
P(102) = P(\bar{1}20) = 0$.
A second glance shows that $P(20\bar{1}) = 1$, $P(1\bar{1}2) = P(112) = 0$.

Let us now suppose (case 1a)
 that $P(201) = 1$. Then  we obtain $P(\bar{1}12) = P(\bar{1}\bar{1}2)
= 0$. We are forced to accept $P(110)
= P(1\bar{1}0)  = 1$ --- a contradiction, since $(110)$ and
$(1\bar{1}0)$ are
orthogonal to each other and lie on one edge.

Hence we have to assume (case 1b) that $P(201) = 0$.
This gives immediately
$P(\bar{1}02)=1$ and
$P(211) =0$.  Since $P(01\bar{1})=0$, we obtain $P(2\bar{1}\bar{1})=1$ and thus
$P(120)=0$.
This requires $P(2\bar{1}0)=1$ and therefore $P(12\bar{1})=P(121)=0$.
Observe that $P(210) = 1$, and thus $P(\bar{1}2\bar{1}) = P(\bar{1}21) = 0$.
In the following step, we notice that $P(10\bar{1}) =
P(101) = 1$ --- a contradiction,
since $(101)$ and $(10\bar{1})$ are
orthogonal to each other and lie on one edge.


Thus we are forced to assume (case 2) that
$P(2\bar{1}1) = 1$. There is no third alternative, since $P(011)=0$ due to the
orthogonality with $(100)$. Now we can repeat the argument for case 1 in
its mirrored form.


\clearpage

\subsection{Kochen \& Specker's ``elementargeometrisches Argument''}

\subsubsection{Nonfull set of two-valued probability measures}


``Full''  means that, for all $p,q$ with
$p \not\perp q$, there is a probability measure
$P$ such that $P(p) = P(q) =1$.

\begin{figure}[hb]
\begin{center}
%TexCad Options
%\grade{\on}
%\emlines{\off}
%\beziermacro{\off}
%\reduce{\on}
%\snapping{\off}
%\quality{2.00}
%\graddiff{0.01}
%\snapasp{1}
%\zoom{0.80}
\unitlength 0.50mm
\linethickness{0.4pt}
\begin{picture}(150.33,50.00)
%\emline(10.00,25.00)(30.00,5.00)
\multiput(10.00,25.00)(0.12,-0.12){167}{\line(0,-1){0.12}}
%\end
%\emline(30.00,5.00)(70.00,5.00)
\put(30.00,5.00){\line(1,0){40.00}}
%\end
%\emline(70.00,5.00)(90.00,25.00)
\multiput(70.00,5.00)(0.12,0.12){167}{\line(0,1){0.12}}
%\end
%\emline(90.00,25.00)(70.00,45.00)
\multiput(90.00,25.00)(-0.12,0.12){167}{\line(0,1){0.12}}
%\end
%\emline(70.00,45.00)(30.00,45.00)
\put(70.00,45.00){\line(-1,0){40.00}}
%\end
%\emline(30.00,45.00)(10.00,25.00)
\multiput(30.00,45.00)(-0.12,-0.12){167}{\line(0,-1){0.12}}
%\end
%\emline(50.00,44.67)(50.00,5.00)
\put(50.00,44.67){\line(0,-1){39.67}}
%\end
\put(30.00,45.00){\circle{2.00}}
\put(50.00,45.00){\circle{2.00}}
\put(70.00,45.00){\circle{2.00}}
\put(30.00,5.00){\circle{2.00}}
\put(50.00,5.00){\circle{2.00}}
\put(70.00,5.00){\circle{2.00}}
\put(10.00,25.00){\circle{2.00}}
\put(90.00,25.00){\circle{2.00}}
\put(90.00,15.00){\makebox(0,0)[cc]{$a_0$}}
\put(70.00,0.00){\makebox(0,0)[cc]{$a_2$}}
\put(50.00,0.00){\makebox(0,0)[cc]{$a_6$}}
\put(30.00,0.00){\makebox(0,0)[cc]{$a_4$}}
\put(30.00,50.00){\makebox(0,0)[cc]{$a_3$}}
\put(50.00,50.00){\makebox(0,0)[cc]{$a_5$}}
\put(70.33,50.00){\makebox(0,0)[cc]{$a_1$}}
\put(10.00,15.00){\makebox(0,0)[cc]{$a_7$}}
\put(105.00,25.00){\makebox(0,0)[cc]{$\equiv$}}
%\emline(120.00,25.00)(149.33,25.00)
\put(120.00,25.00){\line(1,0){29.33}}
%\end
\put(120.00,25.00){\circle{2.00}}
\put(149.33,25.00){\circle{2.00}}
\put(135.00,25.00){\circle{10.00}}
\put(120.00,14.67){\makebox(0,0)[cc]{$a_7$}}
\put(149.33,15.00){\makebox(0,0)[cc]{$a_0$}}
\end{picture}
\end{center}
\caption{\label{f-ksg1a}
Greechie diagram
of a Hilbert lattice
 with a nonfull set of probability measures.
From \protect\cite[Figures 2.2 and 4]{svozil-tkadlec} and
based upon
 \protect\cite[$\Gamma_1$]{kochen1},
 \protect\cite[Figure 2.4.6]{pulmannova-91} and
\protect\cite[Figures 12--14]{schaller-96}.
The logo stands for
``1$\rightarrow$0''.}
\end{figure}


\begin{figure}[hb]
\begin{center}
%TexCad Options
%\grade{\on}
%\emlines{\off}
%\beziermacro{\off}
%\reduce{\on}
%\snapping{\off}
%\quality{2.00}
%\graddiff{0.01}
%\snapasp{1}
%\zoom{0.84}
\unitlength 0.50mm
\linethickness{0.4pt}
\begin{picture}(95.84,100.01)
%\emline(24.84,10.01)(44.84,30.01)
\multiput(24.84,10.01)(0.12,0.12){167}{\line(0,1){0.12}}
%\end
%\emline(44.84,30.01)(44.84,70.01)
\put(44.84,30.01){\line(0,1){40.00}}
%\end
%\emline(44.84,70.01)(24.84,90.01)
\multiput(44.84,70.01)(-0.12,0.12){167}{\line(0,1){0.12}}
%\end
%\emline(24.84,90.01)(4.84,70.01)
\multiput(24.84,90.01)(-0.12,-0.12){167}{\line(0,-1){0.12}}
%\end
%\emline(4.84,70.01)(4.84,30.01)
\put(4.84,70.01){\line(0,-1){40.00}}
%\end
%\emline(4.84,30.01)(24.84,10.01)
\multiput(4.84,30.01)(0.12,-0.12){167}{\line(0,-1){0.12}}
%\end
%\emline(5.17,50.01)(44.84,50.01)
\put(5.17,50.01){\line(1,0){39.67}}
%\end
\put(4.84,30.01){\circle{2.00}}
\put(4.84,50.01){\circle{2.00}}
\put(4.84,70.01){\circle{2.00}}
\put(44.84,30.01){\circle{2.00}}
\put(44.84,50.01){\circle{2.00}}
\put(44.84,70.01){\circle{2.00}}
\put(24.84,10.01){\circle{2.00}}
\put(24.84,90.01){\circle{2.00}}
%\emline(24.84,10.01)(94.84,50.01)
\multiput(24.84,10.01)(0.21,0.12){334}{\line(1,0){0.21}}
%\end
%\emline(94.84,50.01)(24.84,90.01)
\multiput(94.84,50.01)(-0.21,0.12){334}{\line(-1,0){0.21}}
%\end
\put(94.84,50.01){\circle{2.00}}
\put(59.84,30.01){\circle{2.00}}
\put(59.84,20.01){\makebox(0,0)[cc]{$a_9$}}
\put(94.84,60.01){\makebox(0,0)[cc]{$a_8$}}
\put(24.84,100.01){\makebox(0,0)[cc]{$a_0$}}
\put(49.84,70.01){\makebox(0,0)[cc]{$a_2$}}
\put(49.84,50.01){\makebox(0,0)[cc]{$a_6$}}
\put(49.84,30.01){\makebox(0,0)[cc]{$a_4$}}
\put(-0.16,30.01){\makebox(0,0)[cc]{$a_3$}}
\put(-0.16,50.01){\makebox(0,0)[cc]{$a_5$}}
\put(-0.16,70.34){\makebox(0,0)[cc]{$a_1$}}
\put(24.84,0.01){\makebox(0,0)[cc]{$a_7$}}
\end{picture}
\end{center}
\caption{\label{f-ksg1}
Greechie diagram of
$\Gamma_1$ in
 \protect\cite{kochen1}.
}
\end{figure}

\clearpage

\subsubsection{Nonseparating set of two-valued probability measures}
``Separating''  means that, for all $p,q$ with
$p \neq q$, there is a probability measure
$P$ such that $P(p) \neq P(q)$.

Scarcity of two-valued (dispersionless) states;
too few to construct embeddings in a classical structure.

\begin{figure}[hb]
\begin{center}
%TexCad Options
%\grade{\on}
%\emlines{\off}
%\beziermacro{\off}
%\reduce{\on}
%\snapping{\off}
%\quality{2.00}
%\graddiff{0.01}
%\snapasp{1}
%\zoom{0.50}
\unitlength 0.50mm
\linethickness{0.4pt}
\begin{picture}(190.67,109.67)
%\emline(165.67,19.67)(145.67,39.67)
\multiput(165.67,19.67)(-0.12,0.12){167}{\line(0,1){0.12}}
%\end
%\emline(145.67,39.67)(145.67,79.67)
\put(145.67,39.67){\line(0,1){40.00}}
%\end
%\emline(145.67,79.67)(165.67,99.67)
\multiput(145.67,79.67)(0.12,0.12){167}{\line(0,1){0.12}}
%\end
%\emline(165.67,99.67)(185.67,79.67)
\multiput(165.67,99.67)(0.12,-0.12){167}{\line(1,0){0.12}}
%\end
%\emline(185.67,79.67)(185.67,39.67)
\put(185.67,79.67){\line(0,-1){40.00}}
%\end
%\emline(185.67,39.67)(165.67,19.67)
\multiput(185.67,39.67)(-0.12,-0.12){167}{\line(-1,0){0.12}}
%\end
%\emline(185.34,59.67)(145.67,59.67)
\put(185.34,59.67){\line(-1,0){39.67}}
%\end
\put(185.67,39.67){\circle{2.00}}
\put(185.67,59.67){\circle{2.00}}
\put(185.67,79.67){\circle{2.00}}
\put(145.67,39.67){\circle{2.00}}
\put(145.67,59.67){\circle{2.00}}
\put(145.67,79.67){\circle{2.00}}
\put(165.67,19.67){\circle{2.00}}
\put(165.67,99.67){\circle{2.00}}
%\emline(165.67,19.67)(95.67,59.67)
\multiput(165.67,19.67)(-0.21,0.12){334}{\line(-1,0){0.21}}
%\end
%\emline(95.67,59.67)(165.67,99.67)
\multiput(95.67,59.67)(0.21,0.12){334}{\line(1,0){0.21}}
%\end
\put(95.67,59.67){\circle{2.00}}
\put(95.67,74.67){\makebox(0,0)[cc]{$a_8=a_8'$}}
\put(165.67,109.67){\makebox(0,0)[cc]{$b=a_9=a_0'$}}
\put(140.67,79.67){\makebox(0,0)[cc]{$a_2'$}}
\put(140.67,59.67){\makebox(0,0)[cc]{$a_6'$}}
\put(140.67,39.67){\makebox(0,0)[cc]{$a_4'$}}
\put(190.67,39.67){\makebox(0,0)[cc]{$a_3'$}}
\put(190.67,59.67){\makebox(0,0)[cc]{$a_5'$}}
\put(190.67,80.00){\makebox(0,0)[cc]{$a_1'$}}
\put(165.67,9.67){\makebox(0,0)[cc]{$a_7'$}}
%\emline(25.00,19.67)(45.00,39.67)
\multiput(25.00,19.67)(0.12,0.12){167}{\line(0,1){0.12}}
%\end
%\emline(45.00,39.67)(45.00,79.67)
\put(45.00,39.67){\line(0,1){40.00}}
%\end
%\emline(45.00,79.67)(25.00,99.67)
\multiput(45.00,79.67)(-0.12,0.12){167}{\line(0,1){0.12}}
%\end
%\emline(25.00,99.67)(5.00,79.67)
\multiput(25.00,99.67)(-0.12,-0.12){167}{\line(-1,0){0.12}}
%\end
%\emline(5.00,79.67)(5.00,39.67)
\put(5.00,79.67){\line(0,-1){40.00}}
%\end
%\emline(5.00,39.67)(25.00,19.67)
\multiput(5.00,39.67)(0.12,-0.12){167}{\line(1,0){0.12}}
%\end
%\emline(5.33,59.67)(45.00,59.67)
\put(5.33,59.67){\line(1,0){39.67}}
%\end
\put(5.00,39.67){\circle{2.00}}
\put(5.00,59.67){\circle{2.00}}
\put(5.00,79.67){\circle{2.00}}
\put(45.00,39.67){\circle{2.00}}
\put(45.00,59.67){\circle{2.00}}
\put(45.00,79.67){\circle{2.00}}
\put(25.00,19.67){\circle{2.00}}
\put(25.00,99.67){\circle{2.00}}
%\emline(25.00,19.67)(95.00,59.67)
\multiput(25.00,19.67)(0.21,0.12){334}{\line(1,0){0.21}}
%\end
%\emline(95.00,59.67)(25.00,99.67)
\multiput(95.00,59.67)(-0.21,0.12){334}{\line(-1,0){0.21}}
%\end
\put(25.00,109.67){\makebox(0,0)[cc]{$a=a_0=a_9'$}}
\put(50.00,79.67){\makebox(0,0)[cc]{$a_2$}}
\put(50.00,59.67){\makebox(0,0)[cc]{$a_6$}}
\put(50.00,39.67){\makebox(0,0)[cc]{$a_4$}}
\put(-0.00,39.67){\makebox(0,0)[cc]{$a_3$}}
\put(-0.00,59.67){\makebox(0,0)[cc]{$a_5$}}
\put(-0.00,80.00){\makebox(0,0)[cc]{$a_1$}}
\put(25.00,9.67){\makebox(0,0)[cc]{$a_7$}}
\end{picture}
\end{center}
\caption{\label{f-ksg3}
Greechie diagram
of a Hilbert lattice
 with a nonseparating set of probability measures
 \protect\cite[$\Gamma_3$]{kochen1}.
}
\end{figure}


\clearpage

\subsubsection{Nonexistence  of two-valued probability measures}

\begin{figure}[hb]
\begin{center}
%TexCad Options
%\grade{\off}
%\emlines{\off}
%\beziermacro{\off}
%\reduce{\on}
%\snapping{\off}
%\quality{2.00}
%\graddiff{0.01}
%\snapasp{1}
%\zoom{1.25}
\unitlength 0.80mm
\linethickness{0.4pt}
\begin{picture}(117.68,117.45)
%\circle(60.00,60.00){20.00}
\multiput(60.00,70.00)(1.07,-0.12){2}{\line(1,0){1.07}}
\multiput(62.15,69.77)(0.34,-0.12){6}{\line(1,0){0.34}}
\multiput(64.20,69.08)(0.19,-0.11){10}{\line(1,0){0.19}}
\multiput(66.05,67.96)(0.12,-0.11){13}{\line(1,0){0.12}}
\multiput(67.62,66.47)(0.11,-0.16){11}{\line(0,-1){0.16}}
\multiput(68.84,64.68)(0.11,-0.29){7}{\line(0,-1){0.29}}
\multiput(69.64,62.68)(0.12,-0.71){3}{\line(0,-1){0.71}}
\put(69.99,60.54){\line(0,-1){2.16}}
\multiput(69.87,58.38)(-0.12,-0.42){5}{\line(0,-1){0.42}}
\multiput(69.29,56.30)(-0.11,-0.21){9}{\line(0,-1){0.21}}
\multiput(68.28,54.39)(-0.12,-0.14){12}{\line(0,-1){0.14}}
\multiput(66.88,52.74)(-0.16,-0.12){11}{\line(-1,0){0.16}}
\multiput(65.16,51.43)(-0.25,-0.11){8}{\line(-1,0){0.25}}
\multiput(63.19,50.52)(-0.53,-0.12){4}{\line(-1,0){0.53}}
\put(61.08,50.06){\line(-1,0){2.16}}
\multiput(58.92,50.06)(-0.53,0.12){4}{\line(-1,0){0.53}}
\multiput(56.81,50.52)(-0.25,0.11){8}{\line(-1,0){0.25}}
\multiput(54.84,51.43)(-0.16,0.12){11}{\line(-1,0){0.16}}
\multiput(53.12,52.74)(-0.12,0.14){12}{\line(0,1){0.14}}
\multiput(51.72,54.39)(-0.11,0.21){9}{\line(0,1){0.21}}
\multiput(50.71,56.30)(-0.12,0.42){5}{\line(0,1){0.42}}
\put(50.13,58.38){\line(0,1){2.16}}
\multiput(50.01,60.54)(0.12,0.71){3}{\line(0,1){0.71}}
\multiput(50.36,62.68)(0.11,0.29){7}{\line(0,1){0.29}}
\multiput(51.16,64.68)(0.11,0.16){11}{\line(0,1){0.16}}
\multiput(52.38,66.47)(0.12,0.11){13}{\line(1,0){0.12}}
\multiput(53.95,67.96)(0.19,0.11){10}{\line(1,0){0.19}}
\multiput(55.80,69.08)(0.52,0.12){8}{\line(1,0){0.52}}
%\end
\put(60.00,70.00){\line(0,1){40.00}}
\put(60.00,70.00){\line(2,3){23.11}}
\put(60.00,70.00){\line(1,0){49.33}}
\put(60.00,70.00){\line(5,2){42.44}}
\put(109.33,70.00){\line(-3,1){26.22}}
\put(83.11,104.44){\line(-5,-3){23.11}}
\put(69.11,55.33){\line(-1,-2){21.11}}
\put(51.11,55.78){\line(-6,-5){30.67}}
\put(51.11,55.56){\line(-1,1){34.22}}
\put(50.89,55.56){\line(-5,-1){40.22}}
\put(51.11,55.78){\line(-5,2){41.56}}
\put(16.67,90.00){\line(2,-5){9.78}}
\put(9.56,72.44){\line(1,-1){21.11}}
\put(10.89,47.56){\line(6,-1){25.33}}
\put(69.11,55.33){\line(4,-3){30.00}}
\put(69.11,55.33){\line(1,-2){18.22}}
\put(69.33,55.11){\line(0,-1){43.78}}
\put(87.56,18.67){\line(0,1){22.89}}
\put(69.33,11.11){\line(2,5){9.87}}
\put(48.22,13.11){\line(6,5){21.11}}
\put(69.33,55.56){\line(5,3){23.56}}
\put(50.89,55.33){\line(1,-5){5.16}}
\put(60.00,70.22){\line(-5,1){29.33}}
\put(102.38,86.96){\line(-1,0){30.97}}
\put(59.99,69.95){\circle{1.27}}
\put(59.99,90.77){\circle{1.27}}
\put(59.99,109.81){\circle{1.27}}
\put(82.96,104.60){\circle{1.27}}
\put(71.16,86.96){\circle{1.27}}
\put(102.51,86.96){\circle{1.27}}
\put(82.71,78.96){\circle{1.27}}
\put(109.37,69.95){\circle{1.27}}
\put(92.86,69.95){\circle{1.27}}
\put(85.88,86.96){\circle{7.62}}
\put(70.14,96.73){\circle{7.62}}
\put(93.88,75.28){\circle{7.62}}
\put(98.32,106.26){\circle{7.62}}
\put(80.28,62.24){\circle{7.62}}
\put(69.42,55.24){\circle{1.27}}
\put(87.46,41.64){\circle{1.27}}
\put(99.12,32.76){\circle{1.27}}
\put(87.74,18.05){\circle{1.27}}
\put(78.86,35.54){\circle{1.27}}
\put(69.42,10.83){\circle{1.27}}
\put(69.42,30.26){\circle{1.27}}
\put(48.05,13.05){\circle{1.27}}
\put(56.10,29.71){\circle{1.27}}
\put(50.83,55.52){\circle{1.27}}
\put(20.02,29.98){\circle{1.27}}
\put(36.40,43.31){\circle{1.27}}
\put(10.31,47.47){\circle{1.27}}
\put(30.57,51.63){\circle{1.27}}
\put(9.47,72.45){\circle{1.27}}
\put(30.29,76.33){\circle{1.27}}
\put(16.41,90.21){\circle{1.27}}
\put(26.40,65.51){\circle{1.27}}
\put(87.50,31.99){\circle{7.62}}
\put(75.56,26.71){\circle{7.62}}
\put(60.57,23.10){\circle{7.62}}
\put(53.36,43.36){\circle{7.62}}
\put(25.33,45.03){\circle{7.62}}
\put(21.44,60.85){\circle{7.62}}
\put(21.72,77.50){\circle{7.62}}
\put(44.48,73.34){\circle{7.62}}
\put(-0.76,58.35){\circle{7.62}}
\put(83.61,7.84){\circle{7.62}}
\put(59.94,65.00){\makebox(0,0)[cc]{$q_0$}}
\put(65.49,58.61){\makebox(0,0)[cc]{$r_0$}}
\put(54.95,58.61){\makebox(0,0)[cc]{$p_0$}}
\put(43.89,14.72){\makebox(0,0)[cc]{$p_1$}}
\put(64.72,11.11){\makebox(0,0)[cc]{$p_2$}}
\put(83.33,16.94){\makebox(0,0)[cc]{$p_3$}}
\put(105.00,32.78){\makebox(0,0)[cc]{$p_4$}}
\put(109.72,64.72){\makebox(0,0)[cc]{$r_1$}}
\put(105.00,82.50){\makebox(0,0)[cc]{$r_2$}}
\put(86.67,100.56){\makebox(0,0)[cc]{$r_3$}}
\put(65.83,106.11){\makebox(0,0)[cc]{$r_4$}}
\put(16.39,96.11){\makebox(0,0)[cc]{$q_1$}}
\put(12.22,76.67){\makebox(0,0)[cc]{$q_2$}}
\put(10.00,53.61){\makebox(0,0)[cc]{$q_3$}}
\put(26.11,27.22){\makebox(0,0)[cc]{$q_4$}}
%\bezier{384}(20.03,29.97)(-18.76,60.46)(10.13,97.12)
\multiput(20.03,29.97)(-0.14,0.12){19}{\line(-1,0){0.14}}
\multiput(17.31,32.18)(-0.13,0.12){19}{\line(-1,0){0.13}}
\multiput(14.77,34.41)(-0.12,0.12){19}{\line(-1,0){0.12}}
\multiput(12.41,36.66)(-0.12,0.12){19}{\line(0,1){0.12}}
\multiput(10.22,38.92)(-0.12,0.13){17}{\line(0,1){0.13}}
\multiput(8.21,41.20)(-0.11,0.14){16}{\line(0,1){0.14}}
\multiput(6.38,43.50)(-0.12,0.17){14}{\line(0,1){0.17}}
\multiput(4.72,45.81)(-0.11,0.18){13}{\line(0,1){0.18}}
\multiput(3.25,48.14)(-0.12,0.21){11}{\line(0,1){0.21}}
\multiput(1.94,50.48)(-0.11,0.24){10}{\line(0,1){0.24}}
\multiput(0.82,52.84)(-0.12,0.30){8}{\line(0,1){0.30}}
\multiput(-0.13,55.21)(-0.11,0.34){7}{\line(0,1){0.34}}
\multiput(-0.89,57.61)(-0.12,0.48){5}{\line(0,1){0.48}}
\multiput(-1.49,60.01)(-0.10,0.61){4}{\line(0,1){0.61}}
\multiput(-1.90,62.44)(-0.12,1.22){2}{\line(0,1){1.22}}
\put(-2.14,64.88){\line(0,1){2.46}}
\put(-2.20,67.34){\line(0,1){2.47}}
\multiput(-2.08,69.81)(0.10,0.83){3}{\line(0,1){0.83}}
\multiput(-1.79,72.30)(0.12,0.63){4}{\line(0,1){0.63}}
\multiput(-1.32,74.80)(0.11,0.42){6}{\line(0,1){0.42}}
\multiput(-0.67,77.32)(0.12,0.36){7}{\line(0,1){0.36}}
\multiput(0.15,79.86)(0.11,0.28){9}{\line(0,1){0.28}}
\multiput(1.16,82.41)(0.12,0.26){10}{\line(0,1){0.26}}
\multiput(2.34,84.98)(0.11,0.22){12}{\line(0,1){0.22}}
\multiput(3.69,87.57)(0.12,0.20){13}{\line(0,1){0.20}}
\multiput(5.23,90.17)(0.11,0.17){15}{\line(0,1){0.17}}
\multiput(6.94,92.79)(0.12,0.16){27}{\line(0,1){0.16}}
%\end
%\bezier{332}(10.13,97.12)(27.52,116.64)(59.89,69.83)
\multiput(10.13,97.12)(0.11,0.12){12}{\line(0,1){0.12}}
\multiput(11.51,98.54)(0.13,0.11){11}{\line(1,0){0.13}}
\multiput(12.93,99.76)(0.16,0.11){9}{\line(1,0){0.16}}
\multiput(14.40,100.77)(0.22,0.12){7}{\line(1,0){0.22}}
\multiput(15.91,101.59)(0.26,0.10){6}{\line(1,0){0.26}}
\multiput(17.47,102.21)(0.40,0.10){4}{\line(1,0){0.40}}
\multiput(19.07,102.62)(0.82,0.11){2}{\line(1,0){0.82}}
\put(20.72,102.83){\line(1,0){1.69}}
\multiput(22.41,102.85)(0.87,-0.09){2}{\line(1,0){0.87}}
\multiput(24.15,102.66)(0.45,-0.10){4}{\line(1,0){0.45}}
\multiput(25.94,102.27)(0.37,-0.12){5}{\line(1,0){0.37}}
\multiput(27.77,101.68)(0.27,-0.11){7}{\line(1,0){0.27}}
\multiput(29.64,100.88)(0.21,-0.11){9}{\line(1,0){0.21}}
\multiput(31.56,99.89)(0.20,-0.12){10}{\line(1,0){0.20}}
\multiput(33.53,98.70)(0.17,-0.12){12}{\line(1,0){0.17}}
\multiput(35.54,97.30)(0.15,-0.11){14}{\line(1,0){0.15}}
\multiput(37.60,95.71)(0.14,-0.12){15}{\line(1,0){0.14}}
\multiput(39.70,93.91)(0.13,-0.12){17}{\line(1,0){0.13}}
\multiput(41.85,91.91)(0.12,-0.12){19}{\line(0,-1){0.12}}
\multiput(44.05,89.71)(0.12,-0.13){19}{\line(0,-1){0.13}}
\multiput(46.28,87.31)(0.11,-0.13){20}{\line(0,-1){0.13}}
\multiput(48.57,84.71)(0.12,-0.14){20}{\line(0,-1){0.14}}
\multiput(50.90,81.91)(0.12,-0.15){20}{\line(0,-1){0.15}}
\multiput(53.27,78.91)(0.12,-0.15){21}{\line(0,-1){0.15}}
\multiput(55.69,75.70)(0.12,-0.17){35}{\line(0,-1){0.17}}
%\end
%\bezier{424}(59.89,109.96)(117.68,117.45)(116.61,70.10)
\multiput(59.89,109.96)(0.98,0.11){4}{\line(1,0){0.98}}
\multiput(63.80,110.41)(1.26,0.11){3}{\line(1,0){1.26}}
\multiput(67.57,110.73)(1.81,0.10){2}{\line(1,0){1.81}}
\put(71.20,110.92){\line(1,0){6.84}}
\multiput(78.04,110.91)(1.61,-0.10){2}{\line(1,0){1.61}}
\multiput(81.25,110.71)(1.02,-0.11){3}{\line(1,0){1.02}}
\multiput(84.32,110.39)(0.73,-0.11){4}{\line(1,0){0.73}}
\multiput(87.25,109.93)(0.56,-0.12){5}{\line(1,0){0.56}}
\multiput(90.05,109.34)(0.44,-0.12){6}{\line(1,0){0.44}}
\multiput(92.70,108.62)(0.31,-0.11){8}{\line(1,0){0.31}}
\multiput(95.21,107.77)(0.26,-0.11){9}{\line(1,0){0.26}}
\multiput(97.59,106.79)(0.22,-0.11){10}{\line(1,0){0.22}}
\multiput(99.82,105.68)(0.19,-0.11){11}{\line(1,0){0.19}}
\multiput(101.92,104.44)(0.16,-0.11){12}{\line(1,0){0.16}}
\multiput(103.87,103.08)(0.14,-0.12){13}{\line(1,0){0.14}}
\multiput(105.69,101.58)(0.12,-0.12){14}{\line(1,0){0.12}}
\multiput(107.37,99.95)(0.12,-0.14){13}{\line(0,-1){0.14}}
\multiput(108.91,98.19)(0.12,-0.16){12}{\line(0,-1){0.16}}
\multiput(110.30,96.30)(0.11,-0.18){11}{\line(0,-1){0.18}}
\multiput(111.56,94.28)(0.11,-0.21){10}{\line(0,-1){0.21}}
\multiput(112.68,92.13)(0.11,-0.25){9}{\line(0,-1){0.25}}
\multiput(113.66,89.86)(0.11,-0.30){8}{\line(0,-1){0.30}}
\multiput(114.50,87.45)(0.12,-0.42){6}{\line(0,-1){0.42}}
\multiput(115.20,84.91)(0.11,-0.53){5}{\line(0,-1){0.53}}
\multiput(115.76,82.24)(0.11,-0.70){4}{\line(0,-1){0.70}}
\multiput(116.19,79.44)(0.09,-0.98){3}{\line(0,-1){0.98}}
\multiput(116.47,76.52)(0.07,-1.53){2}{\line(0,-1){1.53}}
\put(116.61,73.46){\line(0,-1){3.36}}
%\end
%\bezier{296}(116.61,70.10)(116.88,44.68)(69.26,55.38)
\put(116.61,70.10){\line(0,-1){2.04}}
\multiput(116.55,68.06)(-0.11,-0.96){2}{\line(0,-1){0.96}}
\multiput(116.33,66.15)(-0.10,-0.45){4}{\line(0,-1){0.45}}
\multiput(115.94,64.36)(-0.11,-0.33){5}{\line(0,-1){0.33}}
\multiput(115.40,62.70)(-0.12,-0.26){6}{\line(0,-1){0.26}}
\multiput(114.69,61.15)(-0.11,-0.18){8}{\line(0,-1){0.18}}
\multiput(113.81,59.73)(-0.12,-0.14){9}{\line(0,-1){0.14}}
\multiput(112.78,58.43)(-0.12,-0.12){10}{\line(-1,0){0.12}}
\multiput(111.58,57.26)(-0.15,-0.12){9}{\line(-1,0){0.15}}
\multiput(110.21,56.21)(-0.19,-0.12){8}{\line(-1,0){0.19}}
\multiput(108.69,55.28)(-0.24,-0.12){7}{\line(-1,0){0.24}}
\multiput(107.00,54.47)(-0.31,-0.11){6}{\line(-1,0){0.31}}
\multiput(105.15,53.79)(-0.40,-0.11){5}{\line(-1,0){0.40}}
\multiput(103.14,53.23)(-0.54,-0.11){4}{\line(-1,0){0.54}}
\multiput(100.96,52.79)(-0.78,-0.11){3}{\line(-1,0){0.78}}
\multiput(98.62,52.47)(-1.25,-0.10){2}{\line(-1,0){1.25}}
\put(96.12,52.28){\line(-1,0){2.67}}
\put(93.45,52.21){\line(-1,0){2.83}}
\multiput(90.62,52.26)(-1.50,0.09){2}{\line(-1,0){1.50}}
\multiput(87.63,52.44)(-1.05,0.10){3}{\line(-1,0){1.05}}
\multiput(84.48,52.74)(-0.83,0.11){4}{\line(-1,0){0.83}}
\multiput(81.16,53.16)(-0.70,0.11){5}{\line(-1,0){0.70}}
\multiput(77.68,53.71)(-0.61,0.11){6}{\line(-1,0){0.61}}
\multiput(74.04,54.37)(-0.53,0.11){9}{\line(-1,0){0.53}}
%\end
%\bezier{356}(99.49,32.64)(96.01,-3.47)(44.64,7.23)
\multiput(99.49,32.64)(-0.11,-0.88){3}{\line(0,-1){0.88}}
\multiput(99.16,29.99)(-0.12,-0.63){4}{\line(0,-1){0.63}}
\multiput(98.70,27.48)(-0.12,-0.48){5}{\line(0,-1){0.48}}
\multiput(98.10,25.09)(-0.11,-0.32){7}{\line(0,-1){0.32}}
\multiput(97.36,22.84)(-0.11,-0.26){8}{\line(0,-1){0.26}}
\multiput(96.49,20.72)(-0.11,-0.22){9}{\line(0,-1){0.22}}
\multiput(95.49,18.74)(-0.11,-0.19){10}{\line(0,-1){0.19}}
\multiput(94.34,16.88)(-0.12,-0.16){11}{\line(0,-1){0.16}}
\multiput(93.07,15.16)(-0.12,-0.13){12}{\line(0,-1){0.13}}
\multiput(91.66,13.57)(-0.12,-0.11){13}{\line(-1,0){0.12}}
\multiput(90.11,12.11)(-0.14,-0.11){12}{\line(-1,0){0.14}}
\multiput(88.43,10.78)(-0.18,-0.12){10}{\line(-1,0){0.18}}
\multiput(86.61,9.59)(-0.22,-0.12){9}{\line(-1,0){0.22}}
\multiput(84.66,8.53)(-0.26,-0.12){8}{\line(-1,0){0.26}}
\multiput(82.57,7.60)(-0.32,-0.11){7}{\line(-1,0){0.32}}
\multiput(80.35,6.80)(-0.39,-0.11){6}{\line(-1,0){0.39}}
\multiput(77.99,6.14)(-0.50,-0.11){5}{\line(-1,0){0.50}}
\multiput(75.50,5.61)(-0.66,-0.10){4}{\line(-1,0){0.66}}
\multiput(72.87,5.21)(-0.92,-0.09){3}{\line(-1,0){0.92}}
\multiput(70.10,4.94)(-1.45,-0.07){2}{\line(-1,0){1.45}}
\put(67.20,4.80){\line(-1,0){3.03}}
\multiput(64.17,4.80)(-1.59,0.06){2}{\line(-1,0){1.59}}
\multiput(61.00,4.93)(-1.10,0.09){3}{\line(-1,0){1.10}}
\multiput(57.69,5.19)(-0.86,0.10){4}{\line(-1,0){0.86}}
\multiput(54.25,5.58)(-0.72,0.11){5}{\line(-1,0){0.72}}
\multiput(50.68,6.11)(-0.60,0.11){10}{\line(-1,0){0.60}}
%\end
%\bezier{304}(44.64,7.23)(20.30,15.52)(50.80,55.65)
\multiput(44.64,7.23)(-0.27,0.10){7}{\line(-1,0){0.27}}
\multiput(42.75,7.96)(-0.24,0.12){7}{\line(-1,0){0.24}}
\multiput(41.04,8.79)(-0.19,0.12){8}{\line(-1,0){0.19}}
\multiput(39.51,9.73)(-0.15,0.12){9}{\line(-1,0){0.15}}
\multiput(38.17,10.77)(-0.12,0.11){10}{\line(-1,0){0.12}}
\multiput(37.01,11.92)(-0.11,0.14){9}{\line(0,1){0.14}}
\multiput(36.02,13.18)(-0.11,0.19){7}{\line(0,1){0.19}}
\multiput(35.22,14.54)(-0.10,0.24){6}{\line(0,1){0.24}}
\multiput(34.60,16.00)(-0.11,0.39){4}{\line(0,1){0.39}}
\multiput(34.17,17.57)(-0.09,0.56){3}{\line(0,1){0.56}}
\put(33.91,19.25){\line(0,1){1.78}}
\put(33.84,21.03){\line(0,1){1.89}}
\multiput(33.95,22.92)(0.10,0.66){3}{\line(0,1){0.66}}
\multiput(34.23,24.91)(0.12,0.52){4}{\line(0,1){0.52}}
\multiput(34.70,27.01)(0.11,0.37){6}{\line(0,1){0.37}}
\multiput(35.36,29.21)(0.12,0.33){7}{\line(0,1){0.33}}
\multiput(36.19,31.52)(0.11,0.27){9}{\line(0,1){0.27}}
\multiput(37.21,33.93)(0.12,0.25){10}{\line(0,1){0.25}}
\multiput(38.40,36.45)(0.11,0.22){12}{\line(0,1){0.22}}
\multiput(39.78,39.08)(0.11,0.20){14}{\line(0,1){0.20}}
\multiput(41.34,41.81)(0.12,0.19){15}{\line(0,1){0.19}}
\multiput(43.08,44.64)(0.11,0.17){17}{\line(0,1){0.17}}
\multiput(45.01,47.58)(0.12,0.17){18}{\line(0,1){0.17}}
\multiput(47.11,50.63)(0.12,0.16){31}{\line(0,1){0.16}}
%\end
\end{picture}
\end{center}
\caption{\label{f-ksg2}
Greechie diagram
of a Hilbert lattice
 with no two-valued probability measure
 \protect\cite[$\Gamma_2$]{kochen1}.
$P(p_0)=P(p_1)=P(p_2)=P(p_3)=P(p_4)=1$ yields a contradiction, since
only one of the subspaces $p_0$ and $p_4$ can acquire probability 1.
}
\end{figure}



\clearpage

\section{Strategies to cope with or avoid the theorem}

\subsection{Contextuality and ``Ur''-operators}

{\em ``The value of certain observables depends on the experimental context; i.e.,
which other comeasurable observables are measured alongside this observable.''}
Or, alternatively:
{\em ``No (EPR-like) element of physical reality
corresponds to certain observables unless their measurement context is fully specified.''}

``Ur''-operator: two self-adjoint operators $A$ and $B$ commute if and only if there
exists a self-adjoint ``Ur''-operator $U$ and two real-valued functions
\index{Ur-operator}
$f$ and
$g$ such that $A=f(U)$ and $B=g(U)$
(Varadarajan
\cite[p. 119-120, Theorem 6.9]{varadarajanI} and
  Pt{\'{a}}k and Pulmannov{\'{a}}
\cite[p. 89, Theorem 4.1.7]{pulmannova-91}).
 A generalization to an arbitrary
number of
mutually commuting operators is straightforward. Stated pointedly: every
set of mutually commuting observables
can be represented by just one ``Ur''-operator, such that all the
operators are functions thereof.  (Dual Greechie Diagram, cf. Tkadlec  \cite{tkadlec-96}.)

Yuji Hasegawa, Rudolf Loidl, Gerald Badurek, Matthias Baron, Helmut Rauch,
{\it Violation of a Bell-like inequality in single-neutron interferometry},
http://www.arxiv.org/abs/quant-ph/0311121;
Nature 425, 45 - 48 (04 September 2003); doi:10.1038/nature01881
\begin{figure}[b]
\begin{center}
%TexCad Options
%\grade{\off}
%\emlines{\off}
%\beziermacro{\on}
%\reduce{\on}
%\snapping{\off}
%\quality{2.00}
%\graddiff{0.01}
%\snapasp{1}
%\zoom{3.34}
\unitlength 1.00mm
\linethickness{0.4pt}
\begin{picture}(61.33,36.00)
%\emline(0.33,35.00)(30.33,25.00)
\multiput(0.33,35.00)(0.36,-0.12){84}{\line(1,0){0.36}}
%\end
%\emline(30.33,25.00)(60.33,35.00)
\multiput(30.33,25.00)(0.36,0.12){84}{\line(1,0){0.36}}
%\end
%\put(60.33,15.00){\circle{0.00}}
%\put(60.33,15.00){\circle{2.00}}
%\put(45.33,10.00){\circle{2.00}}
%\put(30.33,5.00){\circle{2.00}}
%\put(15.33,10.00){\circle{2.00}}
%\put(0.33,16.00){\circle{0.00}}
%\put(0.33,15.00){\circle{2.00}}
\put(30.33,25.00){\circle{2.00}}
\put(45.33,30.00){\circle{2.00}}
\put(60.33,35.00){\circle{2.00}}
\put(0.33,35.00){\circle{2.00}}
\put(15.33,30.00){\circle{2.00}}
\put(60.33,31.00){\makebox(0,0)[cc]{$E$}}
\put(45.33,26.00){\makebox(0,0)[cc]{$D$}}
\put(30.33,30.00){\makebox(0,0)[cc]{$A???$}}
\put(15.33,26.00){\makebox(0,0)[cc]{$C$}}
\put(0.33,31.00){\makebox(0,0)[cc]{$B$}}
\bezier{24}(0.00,20.00)(0.00,17.33)(3.00,17.33)
\bezier{28}(3.00,17.33)(10.00,17.00)(10.00,17.00)
\bezier{32}(10.00,17.00)(15.00,16.00)(15.00,13.33)
\bezier{24}(30.00,20.00)(30.00,17.33)(27.00,17.33)
\bezier{28}(27.00,17.33)(20.00,17.00)(20.00,17.00)
\bezier{32}(20.00,17.00)(15.00,16.00)(15.00,13.33)
\put(15.00,7.33){\makebox(0,0)[cc]{context $A-B-C$}}
\bezier{24}(60.00,20.00)(60.00,17.33)(57.00,17.33)
\bezier{28}(57.00,17.33)(50.00,17.00)(50.00,17.00)
\bezier{32}(50.00,17.00)(45.00,16.00)(45.00,13.33)
\bezier{24}(30.00,20.00)(30.00,17.33)(33.00,17.33)
\bezier{28}(33.00,17.33)(40.00,17.00)(40.00,17.00)
\bezier{32}(40.00,17.00)(45.00,16.00)(45.00,13.33)
\put(45.00,7.33){\makebox(0,0)[cc]{context $A-D-E$}}
\end{picture}
\end{center}
\caption{Contextuality: The result of a measurement of certain observables $A$
depends on other measurements on observables $B, C, D, E, \ldots $,
even though the latter observables commute with $A$, but not with one another.
Is there an ``ideal, nonprobabilistic'' test of contextuality? Anything like this:
A particle or system (Hilbert space dimension $\ge 3$) is prepared in a certain state; then
the outcome of observable $A$ depends on whether $B,C$ or $D,E$ is measured alongside with $A$, respectively.
\label{f-ffiab2} }
\end{figure}



\clearpage

\subsection{Principle of limited quantum noncontextuality (SK, work in progress)}
{\em Given a set of contexts in which one or more observables $A, \ldots$
coincide.
Then, within that set of contexts, the outcomes of $A, \ldots$ do not depend
on the context; i.e., which other observables are measured alongside.}

Note: The Kochen-Specker set of contexts has an empty set of coinciding observables.

Proof for $D=3$:
%TexCad Options
%\grade{\off}
%\emlines{\off}
%\beziermacro{\on}
%\reduce{\on}
%\snapping{\off}
%\quality{2.00}
%\graddiff{0.01}
%\snapasp{1}
%\zoom{1.00}
\unitlength 0.600mm
\linethickness{0.4pt}
\begin{picture}(48.00,50.00)
\put(30.00,20.00){\line(0,1){30.00}}
\put(30.00,20.00){\line(5,-4){16.67}}
\put(30.00,20.00){\line(-2,-1){19.33}}
\put(35.33,6.33){\makebox(0,0)[cc]{$x'$}}
\put(47.00,12.00){\makebox(0,0)[cc]{$y$}}
\put(34.67,47.00){\makebox(0,0)[cc]{$z$}}
\put(30.00,20.00){\line(1,-5){3.13}}
\put(30.00,20.00){\line(6,1){18.00}}
\put(15.33,8.33){\makebox(0,0)[cc]{$x$}}
\put(47.67,27.33){\makebox(0,0)[cc]{$y'$}}
\end{picture}

Let
$$A=
\alpha
{\vec x}\otimes {\vec x}
+
\beta
{\vec y}\otimes {\vec y}
+
\gamma
{\vec z}\otimes {\vec z}
$$
and (rotation of $\pi /4$ around the origin in $x-y$-plane)
$$A'=
\alpha
{\vec x'}\otimes {\vec x'}
+
\beta
{\vec y'}\otimes {\vec y'}
+
\gamma
{\vec z}\otimes {\vec z}
$$
with
$ {\vec x'} = (1/\sqrt{2})(1,1)$,
$ {\vec y'} = (1/\sqrt{2})(-1,1)$.

Consider observables $E_zA=E_zA'$.
Then, for all states $\rho$,
$$
{\rm Trace} (\rho E_zA)
{\rm Trace} (E_zA\rho )=
{\rm Trace} (\rho E_zA')=
{\rm Trace} (E_zA'\rho )
$$
more explicitly, for arbitrary $\rho$,
$$
E_zA\rho =
{\rm diag}(0 , 0 , 1 )
\cdot
{\rm diag}(\alpha , \beta , \gamma )
\cdot
\rho
=
\frac{1}{2}
\left(
  \begin{array}{ccc}
      & & \\
    &&\\
&&1
    \end{array}
\right)
\left(
  \begin{array}{ccc}
    \alpha  & & \\
    &\beta&\\
&&\gamma
    \end{array}
\right) \rho = \gamma \rho_{33},
$$
$$
E_zA'\rho =
\frac{1}{2}
\left(
  \begin{array}{ccc}
      & & \\
    &&\\
&&1
    \end{array}
\right)
\left(
  \begin{array}{ccc}
    \alpha +\beta & \alpha -\beta & \\
    \alpha -\beta &\alpha +\beta  &\\
&&\gamma
    \end{array}
\right) \rho = \gamma \rho_{33},
$$




\clearpage



\subsection{Pitowsky (1983) \cite{pitowsky-82,pitowsky-83}}

Paradoxical set decompositions...\\
http://link.aps.org/abstract/PRD/v27/p2316

\subsection{Nonexistence of property; context ranslation principle}

{\em ``Ask your fridge about the oil level in your car;)''}\\
http://tph.tuwien.ac.at/~svozil/publ/2003-garda.pdf


\clearpage

\subsection{Meyer (1999) \cite{meyer:99} based on Godsil and Zaks \cite{godsil-zaks}}

Set of dense rays with Pothagorean property $x^2+y^2+z^2 = n^2,\;
n\in {\Bbb N}$ is consistently three-colorable; i.e., has chromatic number three:
\begin{description}
\item{color \#1} if $x$ is odd, $y$ and $z$ are even,
\item{color \#2} if $y$ is odd, $z$ and $x$ are even,
\item{color \#3} if $z$ is odd, $x$ and $y$ are even.
\end{description}


But: (i) all three colors are dense everywhere; (ii) certain elementary geometric entities such as triangles could not exist.


Cf. also \cite{havlicek-2000} and many others, including

Appleby,\\
http://www.arxiv.org/abs/quant-ph/0308114

and Peres,\\
http://www.arxiv.org/abs/quant-ph/0310035\\
{\em ``Finite precision measurement nullifies Euclid's postulates''}
\\
in particular the line $x=y$ does not intersect with the unit circle $x^2+y^2+z^2 = 1$.

\clearpage

\section{How can you measure a contradiction? (Rob Clifton, around 1994) }


The Proof of Kochen \& Specker is based on a {\em reductio ad adsurdum.}
Thus the argument is not directly operationalizable.


\subsection{Redhead-Kochen multiparticle configuration (before 1993) \cite{redhead}}

Construct a generalized multiparticle singlet state,
and measure inconsistent observables separately.

\subsection{Greenberger-Horne-Zeilinger setup \cite{ghz,ghsz}}

Concrete three-particle configuration. Just as in the original Kochen-Specker proof,
the quantum mechanical computation,
combined with value definiteness, yields a complete contradiction.

Note: The difference between the Kochen-Specker and the Greenberger-Horne-Zeilinger setup
is (contextual) the single particle versus the (nonlocal) separated three particle  configuration.

\subsection{Other proposals}

\begin{enumerate}
\item
Proposed Experimental Tests of the Bell-Kochen-Specker Theorem \\
A. Cabello and G. Garc�a-Alcaine                                 \\
Phys. Rev. Lett. 80, 1797-1799 (1998

\item
Feasible "Kochen-Specker" Experiment with Single Particles         \\
C. Simon, M. Zukowski, H. Weinfurter, and A. Zeilinger               \\
Phys. Rev. Lett. 85, 1783-1786 (2000)

\item
Experimental Test of the Kochen-Specker Theorem with Single Photons    \\
Y.-F. Huang, C.-F. Li, Y.-S. Zhang, J.-W. Pan, and G.-C. Guo             \\
Phys. Rev. Lett. 90, 250401 (2003)
\end{enumerate}


\clearpage

\bibliography{svozil}
\bibliographystyle{apsrev}
%\bibliographystyle{unsrt}
%\bibliographystyle{plain}


\end{document}

