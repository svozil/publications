\PassOptionsToPackage{usenames,dvipsnames}{xcolor}
%\documentclass[amsmath,table,sans,amsfonts, handout]{beamer}
\documentclass[amsmath,table,sans,amsfonts,hyperref={colorlinks,citecolor=blue,linkcolor=blue,urlcolor=purple}]{beamer}
\usepackage[T1]{fontenc}
%%\usepackage{beamerthemeshadow}
%%\usepackage[headheight=1pt,footheight=10pt]{beamerthemeboxes}
%%\addfootboxtemplate{\color{structure!80}}{\color{white}\tiny \hfill Karl Svozil (TU Vienna)\hfill}
%%\addfootboxtemplate{\color{structure!65}}{\color{white}\tiny \hfill mur.sat \hfill}
%%\addfootboxtemplate{\color{structure!50}}{\color{white}\tiny \hfill Graz, 2010-12-11\hfill}
%\usepackage[dark]{beamerthemesidebar}
%\usepackage[headheight=24pt,footheight=12pt]{beamerthemesplit}
%\usepackage{beamerthemesplit}
%\usepackage[bar]{beamerthemetree}
\usepackage{graphicx}
\usepackage{pgf}
%\usepackage{eepic}
%\newcommand{\Red}{\color{Red}}  %(VERY-Approx.PANTONE-RED)
%\newcommand{\Green}{\color{Green}}  %(VERY-Approx.PANTONE-GREEN)

\definecolor{applegreen}{rgb}{0.55, 0.71, 0.0}

\usepackage{fourier-orns}  %fancy symbols https://mirror.easyname.at/ctan/fonts/fourier-GUT/doc/fourier-orns-doc.pdf


%%%%%%%%%%%%%%%%%%%%%%%%%%%%%
\usepackage{iftex}
\ifxetex
\usepackage{fontspec}% Schriftumschaltung mit den nativen XeTeX-Anweisungen
                     % vornehmen. Voreinstellung: Latin Modern
\usepackage[ngerman]{babel}% Sprachumschaltung: Deutsch nach neuer Rechtschreibung

%
% XeLaTeX
%
\XeTeXinputencoding cp1252
\usepackage{fontspec}
%%\setmainfont{Times New Roman}
%\setmainfont{Garamond Libre}
%\setsansfont{Garamond Libre}
\setmainfont{EB Garamond}
\setsansfont{EB Garamond}
%
\else
\usepackage[latin1]{inputenc}
\usepackage[T1]{fontenc}
\fi
%%%%%%%%%%%%%%%%%%%%%%%%%%%%%

%\RequirePackage[german]{babel}
%\selectlanguage{german}
%\RequirePackage[isolatin]{inputenc}

%\pgfdeclareimage[height=0.5cm]{logo}{tu-logo}
%\logo{\pgfuseimage{logo}}
\beamertemplatetriangleitem
%\beamertemplateballitem

\beamerboxesdeclarecolorscheme{alert}{red}{red!15!averagebackgroundcolor}
%\begin{beamerboxesrounded}[scheme=alert,shadow=true]{}
%\end{beamerboxesrounded}

%\beamersetaveragebackground{yellow!10}

%\beamertemplatecircleminiframe

\newtheorem{question}{Question}
\newtheorem{conjecture}[question]{Principle}
\newtheorem{challenge}[question]{Challenge}
\usepackage{tikz}
\newcommand{\bra}[1]{\left< #1 \right|}
\newcommand{\ket}[1]{\left| #1 \right>}

\newcommand{\iprod}[2]{\langle #1 | #2 \rangle}
\newcommand{\mprod}[3]{\langle #1 | #2 | #3 \rangle}
\newcommand{\oprod}[2]{| #1 \rangle\langle #2 |}

\newcommand{\proj}[3]{\begin{smallmatrix} #1 & #2 & #3 \end{smallmatrix}}
\newcommand{\projbf}[3]{\begin{smallmatrix} \mathbf{#1} & \mathbf{#2} & \mathbf{#3} \end{smallmatrix}}

\sloppy
\parskip .7em %vskip between paragraphs

\newcommand{\seq}[1]{\mathbf{#1}}
\newcommand{\floor}[1]{\left\lfloor #1 \right\rfloor}
\newcommand{\ceil}[1]{\left\lceil #1 \right\rceil}
\newcommand{\m}[1]{\widetilde{#1}}
%\newcommand{\p}[1]{\scriptsize\textcolor{black}{$[#1]$}}

\usepackage[most]{tcolorbox}
\begin{document}

\title{\bf \textcolor{blue}{Varieties of logic as algebra \\ and some of their empirical realisations}}
\subtitle{WS From Organizations to Goal-Directedness: Systemic and Interdisciplinary Modeling.\\
\url{http://tph.tuwien.ac.at/~svozil/publ/2022-Granada-pres.pdf}}
\author{Karl Svozil}
\institute{ITP TU Wien, Vienna Austria\\
svozil@tuwien.ac.at
%{\tiny Disclaimer: Die hier vertretenen Meinungen des Autors verstehen sich als Diskussionsbeitr�ge und decken sich nicht notwendigerweise mit den Positionen der Technischen Universit�t Wien oder deren Vertreter.}
}
\date{Granada, Spain, Friday, February 4, 2022}
\maketitle


% \frame{
% \frametitle{Contents}
% \tableofcontents
% }

\section{Empirical nonclassical logic}


 \frame{
 \frametitle{Empirical nonclassical logic}


\begin{itemize}
\item[1.] {\color{purple} Birkhoff \& von Neumann ``The Logic of Quantum Mechanics'' (1936) DOI
\href{https://doi.org/10.2307/1968621}{10.2307/1968621}}
``One of the aspects of quantum theory which has attracted
 the most general attention, is the novelty of the logical notions which it presupposes.
It asserts that even a complete mathematical description of a physi-
 cal system $\Sigma$ does not in general enable one to predict with certainty the result
 of an experiment on $\Sigma$, and that in particular one can never predict with cer-
 tainty both the position and the momentum of $\Sigma$ (Heisenberg's Uncertainty
 Principle). It further asserts that most pairs of observations are incompatible,
 and cannot be made on $\Sigma$ simultaneously (Principle of Non-commutativity of
 Observations).''
\end{itemize}
}


 \frame{
 \frametitle{Empirical nonclassical logic cntd.}
\begin{itemize}
\item[2.]  {\color{purple} Foulis \& Randall ``Operational statistics'' (1972) DOI
\href{https://doi.org/10.1063/1.1665890}{10.1063/1.1665890} }
``The purpose of the series of papers here begun is to
erect a new mathematical foundation for an operational theory of probability and statistics based upon
a generalization of the conventional notion of a sample
space. In subsequent papers, we shall formally establish on this foundation the notion of a
``physical system'' and an affiliated ``theory of measurement.'' This
latter generalized theory of measurement should
prove to be particularly useful in the developing behavioral sciences and in addition shed some light on
the difficulties that surround the measuring process
in quantum mechanics. $\ldots$  We are prepared,
for instance, to regard test procedures on an assembly line, data gathering processes (such as opinion
polling), pencil and paper operations (such as executing computational algorithms), and even procedures
involving subjective approvals or disapprovals as
bona fide physical operations.''
\end{itemize}
}


 \section{Decay \& reconstruction of empirical logics by the pasting of contexts}

 \frame{
 \frametitle{Decay \& reconstruction of empirical logics by the pasting of contexts}

\begin{itemize}
\item[{\color{purple}Context}]
A
\colorbox{yellow!40}{\color{blue}\bf context}
or
\colorbox{yellow!40}{\color{blue}\bf maximal observable}
is a collection of observables that is {\color{blue}complete} and {\color{blue}mutually exclusive}.
It has a hypergraph representation as smooth curve.

\begin{minipage}{0.25\textwidth}
\resizebox{1\textwidth}{!}{
\begin{tikzpicture}  [scale=1]

\tikzstyle{every path}=[line width=1pt]

\newdimen\ms
\ms=0.1cm
\tikzstyle{s1}=[color=red,rectangle,inner sep=3.5]
\tikzstyle{c3}=[circle,inner sep={\ms/8},minimum size=5*\ms]
\tikzstyle{c2}=[circle,inner sep={\ms/8},minimum size=3*\ms]
\tikzstyle{c1}=[circle,inner sep={\ms/8},minimum size=2*\ms]

% Define positions of all observables

\coordinate (a1) at (0,2);
\coordinate (a2) at (0,1);
\coordinate (a3) at (0,0);
\coordinate (a4) at (1,0);
\coordinate (a5) at (2,0);
\coordinate (a6) at (2,1);
\coordinate (a7) at (2,2);
\coordinate (a8) at (1.5,{2+(3.5-2)/2});
\coordinate (a9) at (1,3.5);
\coordinate (a10) at (0.5,{2+(3.5-2)/2});

% draw contexts

\draw [color=orange] (a3) -- (a5);

% draw atoms

\draw (a3) coordinate[c2,fill=orange,label=below:$a_1$];

\draw (a4) coordinate[c2,fill=orange,label=below:$a_2$];

\draw (a5) coordinate[c2,fill=orange,label=below:$a_3$];


\end{tikzpicture}
}
\end{minipage}

\item[{\color{purple}Pasting}]
A \colorbox{yellow!40}{\color{blue}\bf pasting construction} is a {\color{blue}collection of contexts} with possible intertwining contexts.
It has a respective hypergraph representation.


\begin{minipage}{0.25\textwidth}
\resizebox{1\textwidth}{!}{
\begin{tikzpicture}  [scale=1]

\tikzstyle{every path}=[line width=1pt]

\newdimen\ms
\ms=0.1cm
\tikzstyle{s1}=[color=red,rectangle,inner sep=3.5]
\tikzstyle{c3}=[circle,inner sep={\ms/8},minimum size=5*\ms]
\tikzstyle{c2}=[circle,inner sep={\ms/8},minimum size=3*\ms]
\tikzstyle{c1}=[circle,inner sep={\ms/8},minimum size=2*\ms]

% Define positions of all observables

\coordinate (a1) at (0,2);
\coordinate (a2) at (0,1);
\coordinate (a3) at (0,0);
\coordinate (a4) at (1,0);
\coordinate (a5) at (2,0);
\coordinate (a6) at (2,1);
\coordinate (a7) at (2,2);
\coordinate (a8) at (1.5,{2+(3.5-2)/2});
\coordinate (a9) at (1,3.5);
\coordinate (a10) at (0.5,{2+(3.5-2)/2});

% draw contexts

\draw [color=orange] (a1) -- (a3);
\draw [color=blue] (a3) -- (a5);
\draw [color=red] (a5) -- (a7);
\draw [color=green] (a7) -- (a9);
\draw [color=gray] (a9) -- (a1);

% draw atoms

\draw (a1) coordinate[c2,fill=orange,label=left:$a_1$];
\draw (a1) coordinate[c1,fill=gray];

\draw (a2) coordinate[c1,fill=orange,label=left:$a_2$];

\draw (a3) coordinate[c2,fill=blue,label=below:$a_3$];
\draw (a3) coordinate[c1,fill=orange];

\draw (a4) coordinate[c1,fill=blue,label=below:$a_4$];

\draw (a5) coordinate[c2,fill=red,label=below:$a_5$];
\draw (a5) coordinate[c1,fill=blue];

\draw (a6) coordinate[c1,fill=red,label=right:$a_6$];

\draw (a7) coordinate[c2,fill=green,label=right:$a_7$];
\draw (a7) coordinate[c1,fill=red];

\draw (a8) coordinate[c1,fill=green,label=above right:$a_8$];

\draw (a9) coordinate[c2,fill=gray,label=above:$a_9$];
\draw (a9) coordinate[c1,fill=green];

\draw (a10) coordinate[c1,fill=gray,label=above left:$a_{10}$];

\end{tikzpicture}
}
\end{minipage}

\end{itemize}
}


 \frame{
 \frametitle{Anecdotal example: Firefly in a box}
Two intertwining contexts with three mutually exclusive observables per context; e.g.,\\
David W. Cohen (1989) DOI \href{https://doi.org/10.1007/978-1-4613-8841-8}{10.1007/978-1-4613-8841-8},  \\
Dvure{\v{c}}enskij, Pulmannov{\'{a}}, KS DOI \href{https://doi.org/10.5169/seals-116747}{10.5169/seals-116747}.

\begin{center}
\begin{tabular}{cc}
\begin{minipage}{0.35\textwidth}
\resizebox{1\textwidth}{!}{
%TexCad Options
%\grade{\off}
%\emlines{\off}
%\beziermacro{\off}
%\reduce{\on}
%\snapping{\off}
%\quality{0.20}
%\graddiff{0.01}
%\snapasp{1}
%\zoom{1.00}
\unitlength 0.80mm
\linethickness{1pt}
\begin{picture}(56.00,47.00)
%\emline(0.00,17.00)(50.00,17.00)
\put(0.00,17.00){\color{orange}\line(1,0){50.00}}
%\end
%\emline(50.00,17.00)(50.00,47.00)
\put(50.00,17.00){\color{blue}\line(0,1){30.00}}
%\end
%\emline(50.00,47.00)(0.00,47.00)
\put(50.00,47.00){\line(-1,0){50.00}}
%\end
%\emline(0.00,47.00)(0.00,17.00)
\put(0.00,47.00){\line(0,-1){30.00}}
%\end
%\emline(25.00,17.00)(25.00,15.00)
\put(25.00,17.00){\line(0,-1){2.00}}
%\end
%\emline(50.00,32.00)(52.00,32.00)
\put(50.00,32.00){\line(1,0){2.00}}
%\end
%\emline(52.00,32.00)(52.00,32.00)
\put(52.00,32.00){\line(0,1){0.00}}
%\end
\put(56.00,23.00){\makebox(0,0)[cc]{f}}
\put(56.00,38.00){\makebox(0,0)[cc]{b}}
\put(13.00,11.00){\makebox(0,0)[cc]{l}}
\put(37.00,11.00){\makebox(0,0)[cc]{r}}
\end{picture}
}
\end{minipage}
&
\begin{minipage}{0.65\textwidth}
\resizebox{1\textwidth}{!}{
\begin{tikzpicture}  [scale=0.30]

\newdimen\ms
\ms=0.05cm

\tikzstyle{every path}=[line width=2pt]

\tikzstyle{s1}=[color=red,rectangle,inner sep=3.5]
\tikzstyle{c3}=[circle,inner sep={\ms/8},minimum size=5*\ms]
\tikzstyle{c2}=[circle,inner sep={\ms/8},minimum size=3*\ms]
\tikzstyle{c1}=[circle,inner sep={\ms/8},minimum size=2*\ms]

% Radius of regular polygons
\newdimen\R
\R=6cm     % outer circle

%\r= { \R * sqrt(3) }     % inner circle
%\newdimen\r
%\r=    {\R * sqrt(3)/2}       % inner circle

%\newdimen\K
%\K=3cm

% Define positions of all observables
\path
  (0,6 ) coordinate(1)
  (3,3    ) coordinate(2)
  (6,0 ) coordinate(3)
  (9,3) coordinate(4)
  (12,6  ) coordinate(5)
;

% draw contexts

\draw [color=orange] (1) -- (2) -- (3);
\draw [color=blue] (3) -- (4) -- (5);


%
%%
%% draw atoms
%%
%
\draw (1) coordinate[c3,fill=orange,label={above left:$l \equiv \{1,2\} \equiv \begin{pmatrix}1,0,0\end{pmatrix}$}];   %
%
\draw (2) coordinate[c3,fill=orange,label={left: $r \equiv \{3,4\}  \equiv  \frac{1}{\sqrt{2}}\begin{pmatrix}0,1,1\end{pmatrix}$}];    %
%
\draw (3) coordinate[c3,fill=blue,label={below: $u \equiv \{5\} \equiv  \frac{1}{\sqrt{2}}\begin{pmatrix}0,1,-1\end{pmatrix}$}]; %
\draw (3) coordinate[c2,fill=orange];  %
%
\draw (4) coordinate[c3,fill=blue,label={right: $b \equiv \{1,3\}  \equiv  \begin{pmatrix} -\frac{1}{\sqrt{2}},\frac{1}{2},\frac{1}{2}\end{pmatrix}$}];  %
%
\draw (5) coordinate[c3,fill=blue,label={above right: $f \equiv \{2,4\} \equiv  \begin{pmatrix}   \frac{1}{\sqrt{2}},\frac{1}{2},\frac{1}{2} \end{pmatrix}$}];  %
%
%

\end{tikzpicture}
}
\end{minipage}
\end{tabular}
\end{center}

}

\frame{
\begin{center}
{\large {\color{purple}$\ldots$~ as if that is not enough~$\ldots$}}
\end{center}

                                                    \vspace{1.15cm}

\centerline{\Large {\color{applegreen}\decofourleft \hspace{.15cm} {\it interlude} \hspace{.15cm} \decofourright}}

                                                    \vspace{1.15cm}
\begin{center}
{\large {\color{blue} $\ldots$~in another (part of?) the talk enter probability theory~$\ldots$}}
\end{center}

 }


\section{Note on nonclassical versus quantum probability distributions}


\frame{
\frametitle{A brief note on nonclassical versus quantum probability distributions}
\begin{itemize}
\item[Tactics] {\color{purple} what ``we do'' tactically:}
\pause
\begin{itemize}
\item[{\color{purple}$BOO$}] {\color{blue} take some suitable bag / collection of (maybe quantum or partition logic) observables which are in different (intertwined) contexts;}
\pause
\item[{\color{purple}$CL$}] {\color{blue} see how a classical interpretation (aka two-valued states) performs on them---classical predictions;}
\pause
\item[{\color{purple}$QU$}] {\color{blue} see how a quantum interpretation (eg, vertex labeling by vectors) performs on them---quantum predictions;}
\pause
\item[{\color{purple}$CL/QU$@$BOO$}] {\color{blue} hopefully establish a discrepancy between classical \& quantum predictions ---bingo!}
\end{itemize}
\pause
\item[Note]  {\color{purple}  There are three important issues to consider:}
\pause
\begin{itemize}
\item[{\color{purple}Fact}] {\color{blue} in general the logic / algebra does not uniquely determine the probability distribution aka the predictions; }
\pause
\item[{\color{purple}Question}] {\color{blue} ``given some logic or some observables, what possible probability distributions are allowed relative to which axioms of probability?''}
\pause
\item[{\color{purple}Choice}] {\color{blue} of the distribution depends on the physical / psychological {\it etc} realization of the $BOO$.}
\end{itemize}
\end{itemize}
}

\frame{
 \frametitle{Anecdotal example: probabilities on a cyclic logic whose respective hypergraph is a pentagon aka pentagram aka house}
$\;$\\
\begin{tabular}{ c c }
\begin{minipage}{0.45\textwidth}
\resizebox{1\textwidth}{!}{
\begin{tikzpicture}  [scale=1]

\tikzstyle{every path}=[line width=1pt]

\newdimen\ms
\ms=0.1cm
\tikzstyle{s1}=[color=red,rectangle,inner sep=3.5]
\tikzstyle{c3}=[circle,inner sep={\ms/8},minimum size=5*\ms]
\tikzstyle{c2}=[circle,inner sep={\ms/8},minimum size=3*\ms]
\tikzstyle{c1}=[circle,inner sep={\ms/8},minimum size=2*\ms]

% Define positions of all observables

\coordinate (a1) at (0,2);
\coordinate (a2) at (0,1);
\coordinate (a3) at (0,0);
\coordinate (a4) at (1,0);
\coordinate (a5) at (2,0);
\coordinate (a6) at (2,1);
\coordinate (a7) at (2,2);
\coordinate (a8) at (1.5,{2+(3.5-2)/2});
\coordinate (a9) at (1,3.5);
\coordinate (a10) at (0.5,{2+(3.5-2)/2});

% draw contexts

\draw [color=orange] (a1) -- (a3);
\draw [color=blue] (a3) -- (a5);
\draw [color=red] (a5) -- (a7);
\draw [color=green] (a7) -- (a9);
\draw [color=gray] (a9) -- (a1);

% draw atoms

\draw (a1) coordinate[c2,fill=orange,label=left:$a_1$];
\draw (a1) coordinate[c1,fill=gray];

\draw (a2) coordinate[c1,fill=orange,label=left:$a_2$];

\draw (a3) coordinate[c2,fill=blue,label=below:$a_3$];
\draw (a3) coordinate[c1,fill=orange];

\draw (a4) coordinate[c1,fill=blue,label=below:$a_4$];

\draw (a5) coordinate[c2,fill=red,label=below:$a_5$];
\draw (a5) coordinate[c1,fill=blue];

\draw (a6) coordinate[c1,fill=red,label=right:$a_6$];

\draw (a7) coordinate[c2,fill=green,label=right:$a_7$];
\draw (a7) coordinate[c1,fill=red];

\draw (a8) coordinate[c1,fill=green,label=above right:$a_8$];

\draw (a9) coordinate[c2,fill=gray,label=above:$a_9$];
\draw (a9) coordinate[c1,fill=green];

\draw (a10) coordinate[c1,fill=gray,label=above left:$a_{10}$];

\end{tikzpicture}
}
\end{minipage} &
\begin{minipage}{0.45\textwidth}
1) {\color{purple}classical} probability distributions in terms of
convex combinations  of the 11 two-valued states thereon;
\\
$\;$\\
2) {\color{purple}quantum} probability distributions according to Born, Gleason, and Lov{\'a}sz;
\\
$\;$\\
3) {\color{purple}exotic} probability  according to Gerelle \& Greechie \& Miller (1974) and Wright (1978)
\\
$\;$\\
4) --- ... ?  \\
\end{minipage}
\end{tabular}

}

\frame{
\begin{center}
{\large {\color{purple}So far we only spoke about comparing \\
different probability distributions on fixed collections of (interwined)observables~$\ldots$}}
\end{center}

                                                    \vspace{1.15cm}

\centerline{\Large {\color{applegreen}\decofourleft \hspace{.15cm} {\it interlude} \hspace{.15cm} \decofourright}}

                                                    \vspace{1.15cm}
\begin{center}
{\large {\color{blue}  $\ldots$~now we shall be talking about \\``weird'' nonclassical collections of (interwined)observables~$\ldots$}}
\end{center}

 }

\section{Inseparability}


\frame{
 \frametitle{Inseparability 101:
Kochen \& Specker's demarcation criterion  \href{https://doi.org/10.1512/iumj.1968.17.17004}{1967, Theorem~0 of DOI: 10.1512/iumj.1968.17.17004}
}
\begin{center}
\includegraphics[width=1\textwidth]{2021-QCQMB2021-pres-KSTH0.png}
\\
$\;$
\\
\includegraphics[width=1\textwidth]{2021-QCQMB2021-pres-Gamma3.png}
\\
Graph of $\Gamma_3$
\end{center}

}


\frame{
 \frametitle{Hypergraphs with nonseparable set of two-valued states \\third column is Kochen \& Specker (1967, $\Gamma_3$)}

\begin{center}
\resizebox{.88\textwidth}{!}{%
\begin{tabular}{ c c c c }
\begin{tikzpicture}  [scale=0.5]

\tikzstyle{every path}=[line width=1pt]

\newdimen\ms
\ms=0.1cm
\tikzstyle{s1}=[color=red,rectangle,inner sep=3.5]
\tikzstyle{c4}=[circle,inner sep={\ms/8},minimum size=5*\ms]
\tikzstyle{c3}=[circle,inner sep={\ms/8},minimum size=4*\ms]
\tikzstyle{c2}=[circle,inner sep={\ms/8},minimum size=3*\ms]
\tikzstyle{c1}=[circle,inner sep={\ms/8},minimum size=2*\ms]
\tikzstyle{cs1}=[circle,inner sep={\ms/8},minimum size=1*\ms]

% Define positions of all observables


\coordinate (psi) at (1,0);
\coordinate (vd) at (0,2);
\coordinate (dv) at (4,2);
\coordinate (uu) at (2,4);
\coordinate (vu) at (1,5);
\coordinate (uv) at (3,5);
\coordinate (cv) at (0,6);
\coordinate (vc) at (4,6);
\coordinate (dd) at (2.5,8);

\coordinate (aux1) at (4.5,1);
\coordinate (aux2) at (9,1);
\coordinate (aux3) at (4.5,7.8);
\coordinate (aux4) at (8.5,7.8);

\coordinate (P) at (10.5,8);

\coordinate (M) at (6.5,6);
\coordinate (N) at (12,0);

\coordinate (psi2) at ({3+9},0);
\coordinate (vd2) at ({0+9},2);
\coordinate (dv2) at ({4+9},2);
\coordinate (uu2) at ({2+9},4);
\coordinate (vu2) at ({1+9},5);
\coordinate (uv2) at ({3+9},5);
\coordinate (cv2) at ({0+9},6);
\coordinate (vc2) at ({4+9},6);
\coordinate (dd2) at ({1.5+9},8);


% draw contexts

\draw [color=orange] (psi) -- (vd)  coordinate[cs1,fill=white,draw=gray,pos=0.33,label=below left:{\scriptsize \color{gray}2}] (2)  coordinate[cs1,fill=white,draw=gray,pos=0.66,label=below left:{\scriptsize \color{gray}3}] (3);
\draw [color=blue] (psi) -- (uu)  coordinate[cs1,fill=white,draw=gray,pos=0.33,label={[left,xshift=0.2mm]:{\scriptsize \color{gray}20}}] (20)  coordinate[cs1,fill=white,draw=gray,pos=0.66,label={[left,xshift=0.2mm]:{\scriptsize \color{gray}21}}] (21);
\draw [color=red] (psi) -- (dv) coordinate[cs1,fill=white,draw=gray,pos=0.33,label=above:{\scriptsize \color{gray}17}]  (17) coordinate[cs1,fill=white,draw=gray,pos=0.66,label=above:{\scriptsize \color{gray}16}] (16);
\draw [color=green] (vd) -- (vc);
\draw [color=gray] (dv) -- (cv);
\draw [color=magenta] (vu) -- (uv) coordinate[cs1,fill=white,draw=gray,pos=0.33,label=above:{\scriptsize \color{gray}18}] (18)  coordinate[cs1,fill=white,draw=gray,pos=0.66,label=above:{\scriptsize \color{gray}19}] (19);
\draw [color=cyan] (cv) -- (dd) coordinate[cs1,fill=white,draw=gray,pos=0.33,label=above left:{\scriptsize \color{gray}8}]  (8) coordinate[cs1,fill=white,draw=gray,pos=0.66,label=above left:{\scriptsize \color{gray}9}] (9);
\draw [color=olive] (vc) -- (dd) coordinate[cs1,fill=white,draw=gray,pos=0.33,label=right:{\scriptsize \color{gray}12}] (12)  coordinate[cs1,fill=white,draw=gray,pos=0.66,label=right:{\scriptsize \color{gray}11}] (11);

\draw [color=orange] (psi2) -- (vd2)  coordinate[cs1,fill=white,draw=gray,pos=0.33,label=above:{\scriptsize \color{gray}2'}] (22)  coordinate[cs1,fill=white,draw=gray,pos=0.66,label=above:{\scriptsize \color{gray}3'}] (32);
\draw [color=blue] (psi2) -- (uu2)  coordinate[cs1,fill=white,draw=gray,pos=0.33,label={[above right,xshift=0.2mm]:{\scriptsize \color{gray}20'}}] (202)  coordinate[cs1,fill=white,draw=gray,pos=0.66,label={[left,xshift=0.2mm]:{\scriptsize \color{gray}21'}}] (212);
\draw [color=red] (psi2) -- (dv2) coordinate[cs1,fill=white,draw=gray,pos=0.33,label=right:{\scriptsize \color{gray}17'}]  (172) coordinate[cs1,fill=white,draw=gray,pos=0.66,label=right:{\scriptsize \color{gray}16'}] (162);
\draw [color=green] (vd2) -- (vc2);
\draw [color=gray] (dv2) -- (cv2);
\draw [color=magenta] (vu2) -- (uv2) coordinate[cs1,fill=white,draw=gray,pos=0.33,label=above:{\scriptsize \color{gray}18'}] (182)  coordinate[cs1,fill=white,draw=gray,pos=0.66,label=above:{\scriptsize \color{gray}19'}] (192);
\draw [color=cyan] (cv2) -- (dd2) coordinate[cs1,fill=white,draw=gray,pos=0.33,label=left:{\scriptsize \color{gray}8'}]  (82) coordinate[cs1,fill=white,draw=gray,pos=0.66,label=below:{\scriptsize \color{gray}9'}] (92);
\draw [color=olive] (vc2) -- (dd2) coordinate[cs1,fill=white,draw=gray,pos=0.33,label=right:{\scriptsize \color{gray}12'}] (122)  coordinate[cs1,fill=white,draw=gray,pos=0.66,label=right:{\scriptsize \color{gray}11'}] (112);


\draw [ForestGreen] plot [smooth] coordinates {(psi)  (aux1) (M) (aux4) (P) (uu2) };
\draw [RubineRed] plot [smooth] coordinates {(uu) (dd) (aux3) (M) (aux2) (N)};

% draw atoms

\draw (psi) coordinate[c4,fill=orange,label=below:$\Psi$];
\draw (psi) coordinate[c3,fill=blue];
\draw (psi) coordinate[c2,fill=red];
\draw (psi) coordinate[c1,fill=ForestGreen];

\draw (vd) coordinate[c2,fill=orange,label=above:$vd$];
\draw (vd) coordinate[c1,fill=green];

\draw (dv) coordinate[c2,fill=red,label=above:$dv$];
\draw (dv) coordinate[c1,fill=gray];

\draw (uu) coordinate[c4,fill=gray,label=right:$uu$];
\draw (uu) coordinate[c3,fill=green];
\draw (uu) coordinate[c2,fill=RubineRed];
\draw (uu) coordinate[c1,fill=blue];

\draw (vu) coordinate[c2,fill=gray,label=left:$vu$];
\draw (vu) coordinate[c1,fill=magenta];

\draw (uv) coordinate[c2,fill=green,label=right:$uv$];
\draw (uv) coordinate[c1,fill=magenta];

\draw (cv) coordinate[c2,fill=gray,label=above:$cv$];
\draw (cv) coordinate[c1,fill=cyan];

\draw (vc) coordinate[c2,fill=green,label=right:$vc$];
\draw (vc) coordinate[c1,fill=olive];

\draw (dd) coordinate[c3,fill=olive,label=above:$dd$];
\draw (dd) coordinate[c2,fill=cyan];
\draw (dd) coordinate[c1,fill=RubineRed];

\draw (psi2) coordinate[c4,fill=orange,label=below:$\Psi'$];
\draw (psi2) coordinate[c3,fill=blue];
\draw (psi2) coordinate[c2,fill=red];
\draw (psi2) coordinate[c1,fill=RubineRed];

\draw (vd2) coordinate[c2,fill=orange,label=above:$vd'$];
\draw (vd2) coordinate[c1,fill=green];

\draw (dv2) coordinate[c2,fill=red,label=right:$dv'$];
\draw (dv2) coordinate[c1,fill=gray];

\draw (uu2) coordinate[c4,fill=gray,label=right:$uu'$];
\draw (uu2) coordinate[c3,fill=green];
\draw (uu2) coordinate[c2,fill=RubineRed];
\draw (uu2) coordinate[c1,fill=blue];

\draw (vu2) coordinate[c2,fill=gray,label=left:$vu'$];
\draw (vu2) coordinate[c1,fill=magenta];

\draw (uv2) coordinate[c2,fill=green,label=right:$uv'$];
\draw (uv2) coordinate[c1,fill=magenta];

\draw (cv2) coordinate[c2,fill=gray,label=right:$cv'$];
\draw (cv2) coordinate[c1,fill=cyan];

\draw (vc2) coordinate[c2,fill=green,label=right:$vc'$];
\draw (vc2) coordinate[c1,fill=olive];

\draw (dd2) coordinate[c3,fill=olive,label=above:$dd'$];
\draw (dd2) coordinate[c2,fill=cyan];
\draw (dd2) coordinate[c1,fill=ForestGreen];

%\draw (N) coordinate[c2,fill=RubineRed,label=right:$N$];

\draw (M) coordinate[c2,fill=RubineRed,label=right:$M$];
\draw (M) coordinate[c1,fill=ForestGreen];

%\draw (aux1) coordinate[c2,fill=white,draw=gray,label=right:{\scriptsize \color{gray}$O$}];

%\draw (P) coordinate[c2,fill=white,draw=gray,label=right:{\scriptsize \color{gray}$P$}];

\end{tikzpicture}
&
\begin{tikzpicture}  [scale=0.5]

\tikzstyle{every path}=[line width=1pt]

\newdimen\ms
\ms=0.1cm
\tikzstyle{s1}=[color=red,rectangle,inner sep=3.5]
\tikzstyle{c4}=[circle,inner sep={\ms/8},minimum size=5*\ms]
\tikzstyle{c3}=[circle,inner sep={\ms/8},minimum size=4*\ms]
\tikzstyle{c2}=[circle,inner sep={\ms/8},minimum size=3*\ms]
\tikzstyle{c1}=[circle,inner sep={\ms/8},minimum size=2*\ms]
\tikzstyle{cs1}=[circle,inner sep={\ms/8},minimum size=1*\ms]

% Define positions of all observables


\coordinate (psi) at (1,0);
\coordinate (vd) at (0,2);
\coordinate (dv) at (4,2);
\coordinate (uu) at (2,4);
\coordinate (vu) at (1,5);
\coordinate (uv) at (3,5);
\coordinate (cv) at (0,6);
\coordinate (vc) at (4,6);
\coordinate (dd) at (2.5,8);

\coordinate (aux1) at (4.5,1);
\coordinate (aux2) at (9,1);
\coordinate (aux3) at (4.5,7.8);
\coordinate (aux4) at (8.5,7.8);

\coordinate (P) at (10.5,8);

\coordinate (M) at (6.5,6);
\coordinate (N) at (12,0);

\coordinate (psi2) at ({3+9},0);
\coordinate (vd2) at ({0+9},2);
\coordinate (dv2) at ({4+9},2);
\coordinate (uu2) at ({2+9},4);
\coordinate (vu2) at ({1+9},5);
\coordinate (uv2) at ({3+9},5);
\coordinate (cv2) at ({0+9},6);
\coordinate (vc2) at ({4+9},6);
\coordinate (dd2) at ({1.5+9},8);


% draw contexts

\draw [color=orange] (psi) -- (vd)  coordinate[cs1,fill=white,draw=gray,pos=0.33,label=below left:{\scriptsize \color{gray}2}] (2)  coordinate[cs1,fill=white,draw=gray,pos=0.66,label=below left:{\scriptsize \color{gray}3}] (3);
%\draw [color=blue] (psi) -- (uu)  coordinate[cs1,fill=white,draw=gray,pos=0.33,label={[left,xshift=0.2mm]:{\scriptsize \color{gray}20}}] (20)  coordinate[cs1,fill=white,draw=gray,pos=0.66,label={[left,xshift=0.2mm]:{\scriptsize \color{gray}21}}] (21);
\draw [color=red] (psi) -- (dv) coordinate[cs1,fill=white,draw=gray,pos=0.33,label=above:{\scriptsize \color{gray}17}]  (17) coordinate[cs1,fill=white,draw=gray,pos=0.66,label=above:{\scriptsize \color{gray}16}] (16);
\draw [color=green] (vd) -- (vc);
\draw [color=gray] (dv) -- (cv);
\draw [color=magenta] (vu) -- (uv) coordinate[cs1,fill=white,draw=gray,pos=0.33,label=above:{\scriptsize \color{gray}18}] (18)  coordinate[cs1,fill=white,draw=gray,pos=0.66,label=above:{\scriptsize \color{gray}19}] (19);
\draw [color=cyan] (cv) -- (dd) coordinate[cs1,fill=white,draw=gray,pos=0.33,label=above left:{\scriptsize \color{gray}8}]  (8) coordinate[cs1,fill=white,draw=gray,pos=0.66,label=above left:{\scriptsize \color{gray}9}] (9);
\draw [color=olive] (vc) -- (dd) coordinate[cs1,fill=white,draw=gray,pos=0.33,label=right:{\scriptsize \color{gray}12}] (12)  coordinate[cs1,fill=white,draw=gray,pos=0.66,label=right:{\scriptsize \color{gray}11}] (11);

\draw [color=orange] (psi2) -- (vd2)  coordinate[cs1,fill=white,draw=gray,pos=0.33,label=above:{\scriptsize \color{gray}2'}] (22)  coordinate[cs1,fill=white,draw=gray,pos=0.66,label=above:{\scriptsize \color{gray}3'}] (32);
%\draw [color=blue] (psi2) -- (uu2)  coordinate[cs1,fill=white,draw=gray,pos=0.33,label={[above right,xshift=0.2mm]:{\scriptsize \color{gray}20'}}] (202)  coordinate[cs1,fill=white,draw=gray,pos=0.66,label={[left,xshift=0.2mm]:{\scriptsize \color{gray}21'}}] (212);
\draw [color=red] (psi2) -- (dv2) coordinate[cs1,fill=white,draw=gray,pos=0.33,label=right:{\scriptsize \color{gray}17'}]  (172) coordinate[cs1,fill=white,draw=gray,pos=0.66,label=right:{\scriptsize \color{gray}16'}] (162);
\draw [color=green] (vd2) -- (vc2);
\draw [color=gray] (dv2) -- (cv2);
\draw [color=magenta] (vu2) -- (uv2) coordinate[cs1,fill=white,draw=gray,pos=0.33,label=above:{\scriptsize \color{gray}18'}] (182)  coordinate[cs1,fill=white,draw=gray,pos=0.66,label=above:{\scriptsize \color{gray}19'}] (192);
\draw [color=cyan] (cv2) -- (dd2) coordinate[cs1,fill=white,draw=gray,pos=0.33,label=left:{\scriptsize \color{gray}8'}]  (82) coordinate[cs1,fill=white,draw=gray,pos=0.66,label=below:{\scriptsize \color{gray}9'}] (92);
\draw [color=olive] (vc2) -- (dd2) coordinate[cs1,fill=white,draw=gray,pos=0.33,label=right:{\scriptsize \color{gray}12'}] (122)  coordinate[cs1,fill=white,draw=gray,pos=0.66,label=right:{\scriptsize \color{gray}11'}] (112);


\draw [ForestGreen] plot [smooth] coordinates {(psi)  (aux1) (M) (aux4) (P) (uu2) };
\draw [RubineRed] plot [smooth] coordinates {(uu) (dd) (aux3) (M) (aux2) (N)};

% draw atoms

\draw (psi) coordinate[c3,fill=orange,label=below:$\Psi$];
%\draw (psi) coordinate[c3,fill=blue];
\draw (psi) coordinate[c2,fill=red];
\draw (psi) coordinate[c1,fill=ForestGreen];

\draw (vd) coordinate[c2,fill=orange,label=above:$vd$];
\draw (vd) coordinate[c1,fill=green];

\draw (dv) coordinate[c2,fill=red,label=above:$dv$];
\draw (dv) coordinate[c1,fill=gray];

\draw (uu) coordinate[c4,fill=gray,label=right:$uu$];
\draw (uu) coordinate[c3,fill=green];
\draw (uu) coordinate[c2,fill=RubineRed];
%\draw (uu) coordinate[c1,fill=blue];

\draw (vu) coordinate[c2,fill=gray,label=left:$vu$];
\draw (vu) coordinate[c1,fill=magenta];

\draw (uv) coordinate[c2,fill=green,label=right:$uv$];
\draw (uv) coordinate[c1,fill=magenta];

\draw (cv) coordinate[c2,fill=gray,label=above:$cv$];
\draw (cv) coordinate[c1,fill=cyan];

\draw (vc) coordinate[c2,fill=green,label=right:$vc$];
\draw (vc) coordinate[c1,fill=olive];

\draw (dd) coordinate[c3,fill=olive,label=above:$dd$];
\draw (dd) coordinate[c2,fill=cyan];
\draw (dd) coordinate[c1,fill=RubineRed];

\draw (psi2) coordinate[c3,fill=orange,label=below:$\Psi'$];
%\draw (psi2) coordinate[c3,fill=blue];
\draw (psi2) coordinate[c2,fill=red];
\draw (psi2) coordinate[c1,fill=RubineRed];

\draw (vd2) coordinate[c2,fill=orange,label=above:$vd'$];
\draw (vd2) coordinate[c1,fill=green];

\draw (dv2) coordinate[c2,fill=red,label=right:$dv'$];
\draw (dv2) coordinate[c1,fill=gray];

\draw (uu2) coordinate[c4,fill=gray,label=right:$uu'$];
\draw (uu2) coordinate[c3,fill=green];
\draw (uu2) coordinate[c2,fill=RubineRed];
%\draw (uu2) coordinate[c1,fill=blue];

\draw (vu2) coordinate[c2,fill=gray,label=left:$vu'$];
\draw (vu2) coordinate[c1,fill=magenta];

\draw (uv2) coordinate[c2,fill=green,label=right:$uv'$];
\draw (uv2) coordinate[c1,fill=magenta];

\draw (cv2) coordinate[c2,fill=gray,label=right:$cv'$];
\draw (cv2) coordinate[c1,fill=cyan];

\draw (vc2) coordinate[c2,fill=green,label=right:$vc'$];
\draw (vc2) coordinate[c1,fill=olive];

\draw (dd2) coordinate[c3,fill=olive,label=above:$dd'$];
\draw (dd2) coordinate[c2,fill=cyan];
\draw (dd2) coordinate[c1,fill=ForestGreen];

%\draw (N) coordinate[c2,fill=RubineRed,label=right:$N$];

\draw (M) coordinate[c2,fill=RubineRed,label=right:$M$];
\draw (M) coordinate[c1,fill=ForestGreen];

%\draw (aux1) coordinate[c2,fill=white,draw=gray,label=right:{\scriptsize \color{gray}$O$}];

%\draw (P) coordinate[c2,fill=white,draw=gray,label=right:{\scriptsize \color{gray}$P$}];

\end{tikzpicture}
&
\begin{tikzpicture}  [scale=0.5]

\tikzstyle{every path}=[line width=1pt]

\newdimen\ms
\ms=0.1cm
\tikzstyle{s1}=[color=red,rectangle,inner sep=3.5]
\tikzstyle{c3}=[circle,inner sep={\ms/8},minimum size=4*\ms]
\tikzstyle{c2}=[circle,inner sep={\ms/8},minimum size=3*\ms]
\tikzstyle{c1}=[circle,inner sep={\ms/8},minimum size=2*\ms]
\tikzstyle{cs1}=[circle,inner sep={\ms/8},minimum size=1*\ms]

% Define positions of all observables


\coordinate (a1) at (2,0);
\coordinate (a2) at (0,2);
\coordinate (a3) at (4,2);
\coordinate (uu) at (2,4);
\coordinate (a4) at (1,5);
\coordinate (a5) at (3,5);
\coordinate (a6) at (0,6);
\coordinate (a7) at (4,6);
\coordinate (a8) at (2,8);

\coordinate (a12) at ({2+8},0);
\coordinate (a22) at ({0+8},2);
\coordinate (a32) at ({4+8},2);
\coordinate (uu2) at ({2+8},4);
\coordinate (a42) at ({1+8},5);
\coordinate (a52) at ({3+8},5);
\coordinate (a62) at ({0+8},6);
\coordinate (a72) at ({4+8},6);
\coordinate (a82) at ({2+8},8);

\coordinate (O) at (10,8);
\coordinate (M) at (6,4);
\coordinate (N) at (10,0);

\coordinate (aux1) at (4.5,0.5);
\coordinate (aux2) at (4.5,7.5);

\coordinate (aux12) at (7.5,0.5);
\coordinate (aux22) at (7.5,7.5);
% draw contexts

\draw [color=orange] (a1) -- (a2)  coordinate[cs1,fill=white,draw=gray,pos=0.5] (2) ;
\draw [color=red] (a1) -- (a3) coordinate[cs1,fill=white,draw=gray,pos=0.5]  (17);
\draw [color=green] (a2) -- (a7);
\draw [color=gray] (a3) -- (a6);
\draw [color=magenta] (a4) -- (a5) coordinate[cs1,fill=white,draw=gray,pos=0.5] (18);
\draw [color=cyan] (a6) -- (a8) coordinate[cs1,fill=white,draw=gray,pos=0.5]  (8);
\draw [color=olive] (a7) -- (a8) coordinate[cs1,fill=white,draw=gray,pos=0.5] (12);


\draw [color=orange!100!white] (a12) -- (a22)  coordinate[cs1,fill=white,draw=gray,pos=0.5] (2) ;
\draw [color=red!100!white] (a12) -- (a32) coordinate[cs1,fill=white,draw=gray,pos=0.5]  (17);
\draw [color=green!100!white] (a22) -- (a72);
\draw [color=gray!100!white] (a32) -- (a62);
\draw [color=magenta!100!white] (a42) -- (a52) coordinate[cs1,fill=white,draw=gray,pos=0.5] (18);
\draw [color=cyan!100!white] (a62) -- (a82) coordinate[cs1,fill=white,draw=gray,pos=0.5]  (8);
\draw [color=olive!100!white] (a72) -- (a82) coordinate[cs1,fill=white,draw=gray,pos=0.5] (12);

\draw [ForestGreen] plot [smooth] coordinates {(a1) (aux1) (M) (aux22) (O)};
\draw [RubineRed] plot [smooth] coordinates {(a8) (aux2) (M) (aux12) (N)};

% draw atoms

\draw (a1) coordinate[c3,fill=orange,label=below:$a_1$];
\draw (a1) coordinate[c2,fill=ForestGreen];
\draw (a1) coordinate[c1,fill=red];

\draw (a2) coordinate[c2,fill=orange,label=above:$a_2$];
\draw (a2) coordinate[c1,fill=green];

\draw (a3) coordinate[c2,fill=red,label=above:$a_3$];
\draw (a3) coordinate[c1,fill=gray];


\draw (a4) coordinate[c2,fill=gray,label=left:$a_4$];
\draw (a4) coordinate[c1,fill=magenta];

\draw (a5) coordinate[c2,fill=green,label=right:$a_5$];
\draw (a5) coordinate[c1,fill=magenta];

\draw (a6) coordinate[c2,fill=gray,label=above:$a_6$];
\draw (a6) coordinate[c1,fill=cyan];

\draw (a7) coordinate[c2,fill=green,label=above:$a_7$];
\draw (a7) coordinate[c1,fill=olive];

\draw (a8) coordinate[c3,fill=olive,label=above:$a_8$];
\draw (a8) coordinate[c2,fill=cyan];
\draw (a8) coordinate[c1,fill=RubineRed];

\draw (a12) coordinate[c3,fill=orange,label=below:$a_1'$];
\draw (a12) coordinate[c2,fill=RubineRed];
\draw (a12) coordinate[c1,fill=red];

\draw (a22) coordinate[c2,fill=orange,label=above:$a_2'$];
\draw (a22) coordinate[c1,fill=green];

\draw (a32) coordinate[c2,fill=red,label=above:$a_3'$];
\draw (a32) coordinate[c1,fill=gray];


\draw (a42) coordinate[c2,fill=gray,label=left:$a_4'$];
\draw (a42) coordinate[c1,fill=magenta];

\draw (a52) coordinate[c2,fill=green,label=right:$a_5'$];
\draw (a52) coordinate[c1,fill=magenta];

\draw (a62) coordinate[c2,fill=gray,label=above:$a_6'$];
\draw (a62) coordinate[c1,fill=cyan];

\draw (a72) coordinate[c2,fill=green,label=above:$a_7'$];
\draw (a72) coordinate[c1,fill=olive];

\draw (a82) coordinate[c3,fill=cyan,label=above:$a_8'$];
\draw (a82) coordinate[c2,fill=olive];
\draw (a82) coordinate[c1,fill=ForestGreen];

%\draw (N) coordinate[c2,fill=RubineRed,label=right:$N$];

\draw (M) coordinate[c2,fill=RubineRed,label=right:$M$];
\draw (M) coordinate[c1,fill=ForestGreen];


%\draw (O) coordinate[c2,fill=white,draw=gray,label=right:{\scriptsize \color{gray}$O$}];

\end{tikzpicture}
\\
\begin{tikzpicture}  [scale=0.5]

\tikzstyle{every path}=[line width=1pt]

\newdimen\ms
\ms=0.1cm
\tikzstyle{s1}=[color=red,rectangle,inner sep=3.5]
\tikzstyle{c4}=[circle,inner sep={\ms/8},minimum size=5*\ms]
\tikzstyle{c3}=[circle,inner sep={\ms/8},minimum size=4*\ms]
\tikzstyle{c2}=[circle,inner sep={\ms/8},minimum size=3*\ms]
\tikzstyle{c1}=[circle,inner sep={\ms/8},minimum size=2*\ms]
\tikzstyle{cs1}=[circle,inner sep={\ms/8},minimum size=1*\ms]

% Define positions of all observables


\coordinate (psi) at (1,0);
\coordinate (vd) at (0,2);
\coordinate (dv) at (4,2);
\coordinate (uu) at (2,4);
\coordinate (vu) at (1,5);
\coordinate (uv) at (3,5);
\coordinate (cv) at (0,6);
\coordinate (vc) at (4,6);
\coordinate (dd) at (2.5,8);

\coordinate (aux1) at (4.5,1);
\coordinate (aux2) at (9,1);
\coordinate (aux3) at (4.5,7.8);
\coordinate (aux4) at (8.5,7.8);

\coordinate (P) at (10.5,8);

\coordinate (M) at (6.5,6);
\coordinate (N) at (12,0);

\coordinate (psi2) at ({3+9},0);
\coordinate (vd2) at ({0+9},2);
\coordinate (dv2) at ({4+9},2);
\coordinate (uu2) at ({2+9},4);
\coordinate (vu2) at ({1+9},5);
\coordinate (uv2) at ({3+9},5);
\coordinate (cv2) at ({0+9},6);
\coordinate (vc2) at ({4+9},6);
\coordinate (dd2) at ({1.5+9},8);


% draw contexts

\draw [color=lightgray] (psi) -- (vd)  coordinate[c1,fill=lightgray,draw=lightgray,pos=0.33,label=below left:{\scriptsize \color{lightgray}2}] (2)  coordinate[c1,fill=lightgray,draw=lightgray,pos=0.66,label=below left:{\scriptsize \color{lightgray}3}] (3);
\draw [color=lightgray] (psi) -- (uu)  coordinate[c1,fill=lightgray,draw=lightgray,pos=0.33,label={[left,xshift=0.2mm]:{\scriptsize \color{lightgray}20}}] (20)  coordinate[c1,fill=lightgray,draw=lightgray,pos=0.66,label={[left,xshift=0.2mm]:{\scriptsize \color{lightgray}21}}] (21);
\draw [color=lightgray] (psi) -- (dv) coordinate[c1,fill=lightgray,draw=lightgray,pos=0.33,label=above:{\scriptsize \color{lightgray}17}]  (17) coordinate[c1,fill=lightgray,draw=lightgray,pos=0.66,label=above:{\scriptsize \color{lightgray}16}] (16);
\draw [color=lightgray] (vd) -- (vc);
\draw [color=lightgray] (dv) -- (cv);
\draw [color=lightgray] (vu) -- (uv) coordinate[c1,fill=lightgray,draw=lightgray,pos=0.33,label=above:{\scriptsize \color{lightgray}18}] (18)  coordinate[c1,fill=lightgray,draw=lightgray,pos=0.66,label=above:{\scriptsize \color{lightgray}19}] (19);
\draw [color=lightgray] (cv) -- (dd) coordinate[c1,fill=lightgray,draw=lightgray,pos=0.33,label=above left:{\scriptsize \color{lightgray}8}]  (8) coordinate[c1,fill=lightgray,draw=lightgray,pos=0.66,label=above left:{\scriptsize \color{lightgray}9}] (9);
\draw [color=lightgray] (vc) -- (dd) coordinate[c1,fill=lightgray,draw=lightgray,pos=0.33,label=right:{\scriptsize \color{lightgray}12}] (12)  coordinate[c1,fill=lightgray,draw=lightgray,pos=0.66,label=right:{\scriptsize \color{lightgray}11}] (11);

\draw [color=lightgray] (psi2) -- (vd2)  coordinate[c1,fill=lightgray,draw=lightgray,pos=0.33,label=above:{\scriptsize \color{lightgray}2'}] (22)  coordinate[c1,fill=lightgray,draw=lightgray,pos=0.66,label=above:{\scriptsize \color{lightgray}3'}] (32);
\draw [color=lightgray] (psi2) -- (uu2)  coordinate[c1,fill=lightgray,draw=lightgray,pos=0.33,label={[above right,xshift=0.2mm]:{\scriptsize \color{lightgray}20'}}] (202)  coordinate[c1,fill=lightgray,draw=lightgray,pos=0.66,label={[left,xshift=0.2mm]:{\scriptsize \color{lightgray}21'}}] (212);
\draw [color=lightgray] (psi2) -- (dv2) coordinate[c1,fill=lightgray,draw=lightgray,pos=0.33,label=right:{\scriptsize \color{lightgray}17'}]  (172) coordinate[c1,fill=lightgray,draw=lightgray,pos=0.66,label=right:{\scriptsize \color{lightgray}16'}] (162);
\draw [color=lightgray] (vd2) -- (vc2);
\draw [color=lightgray] (dv2) -- (cv2);
\draw [color=lightgray] (vu2) -- (uv2) coordinate[c1,fill=lightgray,draw=lightgray,pos=0.33,label=above:{\scriptsize \color{lightgray}18'}] (182)  coordinate[c1,fill=lightgray,draw=lightgray,pos=0.66,label=above:{\scriptsize \color{lightgray}19'}] (192);
\draw [color=lightgray] (cv2) -- (dd2) coordinate[c1,fill=lightgray,draw=lightgray,pos=0.33,label=left:{\scriptsize \color{lightgray}8'}]  (82) coordinate[c1,fill=lightgray,draw=lightgray,pos=0.66,label=below:{\scriptsize \color{lightgray}9'}] (92);
\draw [color=lightgray] (vc2) -- (dd2) coordinate[c1,fill=lightgray,draw=lightgray,pos=0.33,label=right:{\scriptsize \color{lightgray}12'}] (122)  coordinate[c1,fill=lightgray,draw=lightgray,pos=0.66,label=right:{\scriptsize \color{lightgray}11'}] (112);


\draw [lightgray] plot [smooth] coordinates {(psi)  (aux1) (M) (aux4) (P) (uu2) };
\draw [lightgray] plot [smooth] coordinates {(uu) (dd) (aux3) (M) (aux2) (N)};

% draw atoms

\draw (psi) coordinate[c2,fill=red,draw=red,label=below:$\Psi$];

\draw (vd) coordinate[c1,fill=lightgray,draw=lightgray,label=above:$vd$];

\draw (dv) coordinate[c1,fill=lightgray,draw=lightgray,label=above:$dv$];

\draw (uu) coordinate[c1,fill=lightgray,draw=lightgray,label=right:$uu$];

\draw (vu) coordinate[c1,fill=lightgray,draw=lightgray,label=left:$vu$];

\draw (uv) coordinate[c1,fill=lightgray,draw=lightgray,label=right:$uv$];

\draw (cv) coordinate[c1,fill=lightgray,draw=lightgray,label=above:$cv$];

\draw (vc) coordinate[c1,fill=lightgray,draw=lightgray,label=right:$vc$];

\draw (dd) coordinate[c2,fill=white,draw=green,label=above:$dd$];

\draw (psi2) coordinate[c2,fill=red,draw=red,label=below:$\Psi'$];

\draw (vd2) coordinate[c1,fill=lightgray,draw=lightgray,label=above:$vd'$];

\draw (dv2) coordinate[c1,fill=lightgray,draw=lightgray,label=right:$dv'$];

\draw (uu2) coordinate[c1,fill=lightgray,draw=lightgray,label=right:$uu'$];

\draw (vu2) coordinate[c1,fill=lightgray,draw=lightgray,label=left:$vu'$];

\draw (uv2) coordinate[c1,fill=lightgray,draw=lightgray,label=right:$uv'$];

\draw (cv2) coordinate[c1,fill=lightgray,draw=lightgray,label=right:$cv'$];

\draw (vc2) coordinate[c1,fill=lightgray,draw=lightgray,label=right:$vc'$];

\draw (dd2) coordinate[c2,fill=white,draw=green,label=above:$dd'$];

%\draw (N) coordinate[c1,fill=lightgray,draw=lightgray,label=right:$N$];

\draw (M) coordinate[c2,fill=white,draw=green,label=right:$M$];

%\draw (aux1) coordinate[c1,fill=lightgray,draw=lightgray,label=right:{\scriptsize \color{lightgray}$O$}];

%\draw (P) coordinate[c1,fill=lightgray,draw=lightgray,label=right:{\scriptsize \color{lightgray}$P$}];

\end{tikzpicture}
&
\begin{tikzpicture}  [scale=0.5]

\tikzstyle{every path}=[line width=1pt]

\newdimen\ms
\ms=0.1cm
\tikzstyle{s1}=[color=lightgray,rectangle,inner sep=3.5]
\tikzstyle{c4}=[circle,inner sep={\ms/8},minimum size=5*\ms]
\tikzstyle{c3}=[circle,inner sep={\ms/8},minimum size=4*\ms]
\tikzstyle{c2}=[circle,inner sep={\ms/8},minimum size=3*\ms]
\tikzstyle{c1}=[circle,inner sep={\ms/8},minimum size=2*\ms]
\tikzstyle{cs1}=[circle,inner sep={\ms/8},minimum size=1*\ms]

% Define positions of all observables


\coordinate (psi) at (1,0);
\coordinate (vd) at (0,2);
\coordinate (dv) at (4,2);
\coordinate (uu) at (2,4);
\coordinate (vu) at (1,5);
\coordinate (uv) at (3,5);
\coordinate (cv) at (0,6);
\coordinate (vc) at (4,6);
\coordinate (dd) at (2.5,8);

\coordinate (aux1) at (4.5,1);
\coordinate (aux2) at (9,1);
\coordinate (aux3) at (4.5,7.8);
\coordinate (aux4) at (8.5,7.8);

\coordinate (P) at (10.5,8);

\coordinate (M) at (6.5,6);
\coordinate (N) at (12,0);

\coordinate (psi2) at ({3+9},0);
\coordinate (vd2) at ({0+9},2);
\coordinate (dv2) at ({4+9},2);
\coordinate (uu2) at ({2+9},4);
\coordinate (vu2) at ({1+9},5);
\coordinate (uv2) at ({3+9},5);
\coordinate (cv2) at ({0+9},6);
\coordinate (vc2) at ({4+9},6);
\coordinate (dd2) at ({1.5+9},8);


% draw contexts

\draw [color=lightgray] (psi) -- (vd)  coordinate[c1,fill=lightgray,draw=lightgray,pos=0.33,label=below left:{\scriptsize \color{lightgray}2}] (2)  coordinate[c1,fill=lightgray,draw=lightgray,pos=0.66,label=below left:{\scriptsize \color{lightgray}3}] (3);
%\draw [color=lightgray] (psi) -- (uu)  coordinate[c1,fill=lightgray,draw=lightgray,pos=0.33,label={[left,xshift=0.2mm]:{\scriptsize \color{lightgray}20}}] (20)  coordinate[c1,fill=lightgray,draw=lightgray,pos=0.66,label={[left,xshift=0.2mm]:{\scriptsize \color{lightgray}21}}] (21);
\draw [color=lightgray] (psi) -- (dv) coordinate[c1,fill=lightgray,draw=lightgray,pos=0.33,label=above:{\scriptsize \color{lightgray}17}]  (17) coordinate[c1,fill=lightgray,draw=lightgray,pos=0.66,label=above:{\scriptsize \color{lightgray}16}] (16);
\draw [color=lightgray] (vd) -- (vc);
\draw [color=lightgray] (dv) -- (cv);
\draw [color=lightgray] (vu) -- (uv) coordinate[c1,fill=lightgray,draw=lightgray,pos=0.33,label=above:{\scriptsize \color{lightgray}18}] (18)  coordinate[c1,fill=lightgray,draw=lightgray,pos=0.66,label=above:{\scriptsize \color{lightgray}19}] (19);
\draw [color=lightgray] (cv) -- (dd) coordinate[c1,fill=lightgray,draw=lightgray,pos=0.33,label=above left:{\scriptsize \color{lightgray}8}]  (8) coordinate[c1,fill=lightgray,draw=lightgray,pos=0.66,label=above left:{\scriptsize \color{lightgray}9}] (9);
\draw [color=lightgray] (vc) -- (dd) coordinate[c1,fill=lightgray,draw=lightgray,pos=0.33,label=right:{\scriptsize \color{lightgray}12}] (12)  coordinate[c1,fill=lightgray,draw=lightgray,pos=0.66,label=right:{\scriptsize \color{lightgray}11}] (11);

\draw [color=lightgray] (psi2) -- (vd2)  coordinate[c1,fill=lightgray,draw=lightgray,pos=0.33,label=above:{\scriptsize \color{lightgray}2'}] (22)  coordinate[c1,fill=lightgray,draw=lightgray,pos=0.66,label=above:{\scriptsize \color{lightgray}3'}] (32);
%\draw [color=lightgray] (psi2) -- (uu2)  coordinate[c1,fill=lightgray,draw=lightgray,pos=0.33,label={[above right,xshift=0.2mm]:{\scriptsize \color{lightgray}20'}}] (202)  coordinate[c1,fill=lightgray,draw=lightgray,pos=0.66,label={[left,xshift=0.2mm]:{\scriptsize \color{lightgray}21'}}] (212);
\draw [color=lightgray] (psi2) -- (dv2) coordinate[c1,fill=lightgray,draw=lightgray,pos=0.33,label=right:{\scriptsize \color{lightgray}17'}]  (172) coordinate[c1,fill=lightgray,draw=lightgray,pos=0.66,label=right:{\scriptsize \color{lightgray}16'}] (162);
\draw [color=lightgray] (vd2) -- (vc2);
\draw [color=lightgray] (dv2) -- (cv2);
\draw [color=lightgray] (vu2) -- (uv2) coordinate[c1,fill=lightgray,draw=lightgray,pos=0.33,label=above:{\scriptsize \color{lightgray}18'}] (182)  coordinate[c1,fill=lightgray,draw=lightgray,pos=0.66,label=above:{\scriptsize \color{lightgray}19'}] (192);
\draw [color=lightgray] (cv2) -- (dd2) coordinate[c1,fill=lightgray,draw=lightgray,pos=0.33,label=left:{\scriptsize \color{lightgray}8'}]  (82) coordinate[c1,fill=lightgray,draw=lightgray,pos=0.66,label=below:{\scriptsize \color{lightgray}9'}] (92);
\draw [color=lightgray] (vc2) -- (dd2) coordinate[c1,fill=lightgray,draw=lightgray,pos=0.33,label=right:{\scriptsize \color{lightgray}12'}] (122)  coordinate[c1,fill=lightgray,draw=lightgray,pos=0.66,label=right:{\scriptsize \color{lightgray}11'}] (112);


\draw [lightgray] plot [smooth] coordinates {(psi)  (aux1) (M) (aux4) (P) (uu2) };
\draw [lightgray] plot [smooth] coordinates {(uu) (dd) (aux3) (M) (aux2) (N)};

% draw atoms

\draw (psi) coordinate[c2,fill=red,draw=red,label=below:$\Psi$];

\draw (vd) coordinate[c1,fill=lightgray,draw=lightgray,label=above:$vd$];

\draw (dv) coordinate[c1,fill=lightgray,draw=lightgray,label=above:$dv$];

\draw (uu) coordinate[c1,fill=lightgray,draw=lightgray,label=right:$uu$];

\draw (vu) coordinate[c1,fill=lightgray,draw=lightgray,label=left:$vu$];

\draw (uv) coordinate[c1,fill=lightgray,draw=lightgray,label=right:$uv$];

\draw (cv) coordinate[c1,fill=lightgray,draw=lightgray,label=above:$cv$];

\draw (vc) coordinate[c1,fill=lightgray,draw=lightgray,label=right:$vc$];

\draw (dd) coordinate[c2,fill=white,draw=green,label=above:$dd$];

\draw (psi2) coordinate[c2,fill=red,draw=red,label=below:$\Psi'$];

\draw (vd2) coordinate[c1,fill=lightgray,draw=lightgray,label=above:$vd'$];

\draw (dv2) coordinate[c1,fill=lightgray,draw=lightgray,label=right:$dv'$];

\draw (uu2) coordinate[c1,fill=lightgray,draw=lightgray,label=right:$uu'$];

\draw (vu2) coordinate[c1,fill=lightgray,draw=lightgray,label=left:$vu'$];

\draw (uv2) coordinate[c1,fill=lightgray,draw=lightgray,label=right:$uv'$];

\draw (cv2) coordinate[c1,fill=lightgray,draw=lightgray,label=right:$cv'$];

\draw (vc2) coordinate[c1,fill=lightgray,draw=lightgray,label=right:$vc'$];

\draw (dd2) coordinate[c2,fill=white,draw=green,label=above:$dd'$];

%\draw (N) coordinate[c1,fill=lightgray,draw=lightgray,label=right:$N$];

\draw (M) coordinate[c2,fill=white,draw=green,label=right:$M$];

%\draw (aux1) coordinate[c1,fill=lightgray,draw=lightgray,label=right:{\scriptsize \color{lightgray}$O$}];

%\draw (P) coordinate[c1,fill=lightgray,draw=lightgray,label=right:{\scriptsize \color{lightgray}$P$}];

\end{tikzpicture}
&
\begin{tikzpicture}  [scale=0.5]

\tikzstyle{every path}=[line width=1pt]

\newdimen\ms
\ms=0.1cm
\tikzstyle{s1}=[color=lightgray,rectangle,inner sep=3.5]
\tikzstyle{c3}=[circle,inner sep={\ms/8},minimum size=4*\ms]
\tikzstyle{c2}=[circle,inner sep={\ms/8},minimum size=3*\ms]
\tikzstyle{c1}=[circle,inner sep={\ms/8},minimum size=2*\ms]
\tikzstyle{cs1}=[circle,inner sep={\ms/8},minimum size=1*\ms]

% Define positions of all observables


\coordinate (a1) at (2,0);
\coordinate (a2) at (0,2);
\coordinate (a3) at (4,2);
\coordinate (uu) at (2,4);
\coordinate (a4) at (1,5);
\coordinate (a5) at (3,5);
\coordinate (a6) at (0,6);
\coordinate (a7) at (4,6);
\coordinate (a8) at (2,8);

\coordinate (a12) at ({2+8},0);
\coordinate (a22) at ({0+8},2);
\coordinate (a32) at ({4+8},2);
\coordinate (uu2) at ({2+8},4);
\coordinate (a42) at ({1+8},5);
\coordinate (a52) at ({3+8},5);
\coordinate (a62) at ({0+8},6);
\coordinate (a72) at ({4+8},6);
\coordinate (a82) at ({2+8},8);

\coordinate (O) at (10,8);
\coordinate (M) at (6,4);
\coordinate (N) at (10,0);

\coordinate (aux1) at (4.5,0.5);
\coordinate (aux2) at (4.5,7.5);

\coordinate (aux12) at (7.5,0.5);
\coordinate (aux22) at (7.5,7.5);
% draw contexts

\draw [color=lightgray] (a1) -- (a2)  coordinate[c1,fill=lightgray,draw=lightgray,pos=0.5] (2) ;
\draw [color=lightgray] (a1) -- (a3) coordinate[c1,fill=lightgray,draw=lightgray,pos=0.5]  (17);
\draw [color=lightgray] (a2) -- (a7);
\draw [color=lightgray] (a3) -- (a6);
\draw [color=lightgray] (a4) -- (a5) coordinate[c1,fill=lightgray,draw=lightgray,pos=0.5] (18);
\draw [color=lightgray] (a6) -- (a8) coordinate[c1,fill=lightgray,draw=lightgray,pos=0.5]  (8);
\draw [color=lightgray] (a7) -- (a8) coordinate[c1,fill=lightgray,draw=lightgray,pos=0.5] (12);


\draw [color=lightgray!100!white] (a12) -- (a22)  coordinate[c1,fill=lightgray,draw=lightgray,pos=0.5] (2) ;
\draw [color=lightgray!100!white] (a12) -- (a32) coordinate[c1,fill=lightgray,draw=lightgray,pos=0.5]  (17);
\draw [color=lightgray!100!white] (a22) -- (a72);
\draw [color=lightgray!100!white] (a32) -- (a62);
\draw [color=lightgray!100!white] (a42) -- (a52) coordinate[c1,fill=lightgray,draw=lightgray,pos=0.5] (18);
\draw [color=lightgray!100!white] (a62) -- (a82) coordinate[c1,fill=lightgray,draw=lightgray,pos=0.5]  (8);
\draw [color=lightgray!100!white] (a72) -- (a82) coordinate[c1,fill=lightgray,draw=lightgray,pos=0.5] (12);

\draw [lightgray] plot [smooth] coordinates {(a1) (aux1) (M) (aux22) (O)};
\draw [lightgray] plot [smooth] coordinates {(a8) (aux2) (M) (aux12) (N)};

% draw atoms

\draw (a1) coordinate[c2,fill=red,draw=red,label=below:$a_1$];

\draw (a2) coordinate[c1,fill=lightgray,draw=lightgray,label=above:$a_2$];

\draw (a3) coordinate[c1,fill=lightgray,draw=lightgray,label=above:$a_3$];


\draw (a4) coordinate[c1,fill=lightgray,draw=lightgray,label=left:$a_4$];

\draw (a5) coordinate[c1,fill=lightgray,draw=lightgray,label=right:$a_5$];

\draw (a6) coordinate[c1,fill=lightgray,draw=lightgray,label=above:$a_6$];

\draw (a7) coordinate[c1,fill=lightgray,draw=lightgray,label=above:$a_7$];

\draw (a8) coordinate[c2,fill=white,draw=green,label=above:$a_8$];

\draw (a12) coordinate[c2,fill=red,draw=red,label=below:$a_1'$];

\draw (a22) coordinate[c1,fill=lightgray,draw=lightgray,label=above:$a_2'$];

\draw (a32) coordinate[c1,fill=lightgray,draw=lightgray,label=above:$a_3'$];


\draw (a42) coordinate[c1,fill=lightgray,draw=lightgray,label=left:$a_4'$];

\draw (a52) coordinate[c1,fill=lightgray,draw=lightgray,label=right:$a_5'$];

\draw (a62) coordinate[c1,fill=lightgray,draw=lightgray,label=above:$a_6'$];

\draw (a72) coordinate[c1,fill=lightgray,draw=lightgray,label=above:$a_7'$];

\draw (a82) coordinate[c2,fill=white,draw=green,label=above:$a_8'$];

%\draw (N) coordinate[c1,fill=lightgray,draw=lightgray,label=right:$N$];

\draw (M) coordinate[c2,fill=white,draw=green,label=right:$M$];

%\draw (O) coordinate[c1,fill=lightgray,draw=lightgray,label=right:{\scriptsize \color{lightgray}$O$}];

\end{tikzpicture}
\\
\begin{tikzpicture}  [scale=0.5]

\tikzstyle{every path}=[line width=1pt]

\newdimen\ms
\ms=0.1cm
\tikzstyle{s1}=[color=red,rectangle,inner sep=3.5]
\tikzstyle{c4}=[circle,inner sep={\ms/8},minimum size=5*\ms]
\tikzstyle{c3}=[circle,inner sep={\ms/8},minimum size=4*\ms]
\tikzstyle{c2}=[circle,inner sep={\ms/8},minimum size=3*\ms]
\tikzstyle{c1}=[circle,inner sep={\ms/8},minimum size=2*\ms]
\tikzstyle{cs1}=[circle,inner sep={\ms/8},minimum size=1*\ms]

% Define positions of all observables


\coordinate (psi) at (1,0);
\coordinate (vd) at (0,2);
\coordinate (dv) at (4,2);
\coordinate (uu) at (2,4);
\coordinate (vu) at (1,5);
\coordinate (uv) at (3,5);
\coordinate (cv) at (0,6);
\coordinate (vc) at (4,6);
\coordinate (dd) at (2.5,8);

\coordinate (aux1) at (4.5,1);
\coordinate (aux2) at (9,1);
\coordinate (aux3) at (4.5,7.8);
\coordinate (aux4) at (8.5,7.8);

\coordinate (P) at (10.5,8);

\coordinate (M) at (6.5,6);
\coordinate (N) at (12,0);

\coordinate (psi2) at ({3+9},0);
\coordinate (vd2) at ({0+9},2);
\coordinate (dv2) at ({4+9},2);
\coordinate (uu2) at ({2+9},4);
\coordinate (vu2) at ({1+9},5);
\coordinate (uv2) at ({3+9},5);
\coordinate (cv2) at ({0+9},6);
\coordinate (vc2) at ({4+9},6);
\coordinate (dd2) at ({1.5+9},8);


% draw contexts

\draw [color=lightgray] (psi) -- (vd)  coordinate[c1,fill=lightgray,draw=lightgray,pos=0.33,label=below left:{\scriptsize \color{lightgray}2}] (2)  coordinate[c1,fill=lightgray,draw=lightgray,pos=0.66,label=below left:{\scriptsize \color{lightgray}3}] (3);
\draw [color=lightgray] (psi) -- (uu)  coordinate[c1,fill=lightgray,draw=lightgray,pos=0.33,label={[left,xshift=0.2mm]:{\scriptsize \color{lightgray}20}}] (20)  coordinate[c1,fill=lightgray,draw=lightgray,pos=0.66,label={[left,xshift=0.2mm]:{\scriptsize \color{lightgray}21}}] (21);
\draw [color=lightgray] (psi) -- (dv) coordinate[c1,fill=lightgray,draw=lightgray,pos=0.33,label=above:{\scriptsize \color{lightgray}17}]  (17) coordinate[c1,fill=lightgray,draw=lightgray,pos=0.66,label=above:{\scriptsize \color{lightgray}16}] (16);
\draw [color=lightgray] (vd) -- (vc);
\draw [color=lightgray] (dv) -- (cv);
\draw [color=lightgray] (vu) -- (uv) coordinate[c1,fill=lightgray,draw=lightgray,pos=0.33,label=above:{\scriptsize \color{lightgray}18}] (18)  coordinate[c1,fill=lightgray,draw=lightgray,pos=0.66,label=above:{\scriptsize \color{lightgray}19}] (19);
\draw [color=lightgray] (cv) -- (dd) coordinate[c1,fill=lightgray,draw=lightgray,pos=0.33,label=above left:{\scriptsize \color{lightgray}8}]  (8) coordinate[c1,fill=lightgray,draw=lightgray,pos=0.66,label=above left:{\scriptsize \color{lightgray}9}] (9);
\draw [color=lightgray] (vc) -- (dd) coordinate[c1,fill=lightgray,draw=lightgray,pos=0.33,label=right:{\scriptsize \color{lightgray}12}] (12)  coordinate[c1,fill=lightgray,draw=lightgray,pos=0.66,label=right:{\scriptsize \color{lightgray}11}] (11);

\draw [color=lightgray] (psi2) -- (vd2)  coordinate[c1,fill=lightgray,draw=lightgray,pos=0.33,label=above:{\scriptsize \color{lightgray}2'}] (22)  coordinate[c1,fill=lightgray,draw=lightgray,pos=0.66,label=above:{\scriptsize \color{lightgray}3'}] (32);
\draw [color=lightgray] (psi2) -- (uu2)  coordinate[c1,fill=lightgray,draw=lightgray,pos=0.33,label={[above right,xshift=0.2mm]:{\scriptsize \color{lightgray}20'}}] (202)  coordinate[c1,fill=lightgray,draw=lightgray,pos=0.66,label={[left,xshift=0.2mm]:{\scriptsize \color{lightgray}21'}}] (212);
\draw [color=lightgray] (psi2) -- (dv2) coordinate[c1,fill=lightgray,draw=lightgray,pos=0.33,label=right:{\scriptsize \color{lightgray}17'}]  (172) coordinate[c1,fill=lightgray,draw=lightgray,pos=0.66,label=right:{\scriptsize \color{lightgray}16'}] (162);
\draw [color=lightgray] (vd2) -- (vc2);
\draw [color=lightgray] (dv2) -- (cv2);
\draw [color=lightgray] (vu2) -- (uv2) coordinate[c1,fill=lightgray,draw=lightgray,pos=0.33,label=above:{\scriptsize \color{lightgray}18'}] (182)  coordinate[c1,fill=lightgray,draw=lightgray,pos=0.66,label=above:{\scriptsize \color{lightgray}19'}] (192);
\draw [color=lightgray] (cv2) -- (dd2) coordinate[c1,fill=lightgray,draw=lightgray,pos=0.33,label=left:{\scriptsize \color{lightgray}8'}]  (82) coordinate[c1,fill=lightgray,draw=lightgray,pos=0.66,label=below:{\scriptsize \color{lightgray}9'}] (92);
\draw [color=lightgray] (vc2) -- (dd2) coordinate[c1,fill=lightgray,draw=lightgray,pos=0.33,label=right:{\scriptsize \color{lightgray}12'}] (122)  coordinate[c1,fill=lightgray,draw=lightgray,pos=0.66,label=right:{\scriptsize \color{lightgray}11'}] (112);


\draw [lightgray] plot [smooth] coordinates {(psi)  (aux1) (M) (aux4) (P) (uu2) };
\draw [lightgray] plot [smooth] coordinates {(uu) (dd) (aux3) (M) (aux2) (N)};

% draw atoms

\draw (psi) coordinate[c2,fill=white,draw=green,label=below:$\Psi$];

\draw (vd) coordinate[c1,fill=lightgray,draw=lightgray,label=above:$vd$];

\draw (dv) coordinate[c1,fill=lightgray,draw=lightgray,label=above:$dv$];

\draw (uu) coordinate[c1,fill=lightgray,draw=lightgray,label=right:$uu$];

\draw (vu) coordinate[c1,fill=lightgray,draw=lightgray,label=left:$vu$];

\draw (uv) coordinate[c1,fill=lightgray,draw=lightgray,label=right:$uv$];

\draw (cv) coordinate[c1,fill=lightgray,draw=lightgray,label=above:$cv$];

\draw (vc) coordinate[c1,fill=lightgray,draw=lightgray,label=right:$vc$];

\draw (dd) coordinate[c2,fill=red,draw=red,label=above:$dd$];

\draw (psi2) coordinate[c2,fill=white,draw=green,label=below:$\Psi'$];

\draw (vd2) coordinate[c1,fill=lightgray,draw=lightgray,label=above:$vd'$];

\draw (dv2) coordinate[c1,fill=lightgray,draw=lightgray,label=right:$dv'$];

\draw (uu2) coordinate[c1,fill=lightgray,draw=lightgray,label=right:$uu'$];

\draw (vu2) coordinate[c1,fill=lightgray,draw=lightgray,label=left:$vu'$];

\draw (uv2) coordinate[c1,fill=lightgray,draw=lightgray,label=right:$uv'$];

\draw (cv2) coordinate[c1,fill=lightgray,draw=lightgray,label=right:$cv'$];

\draw (vc2) coordinate[c1,fill=lightgray,draw=lightgray,label=right:$vc'$];

\draw (dd2) coordinate[c2,fill=red,draw=red,label=above:$dd'$];

%\draw (N) coordinate[c1,fill=lightgray,draw=lightgray,label=right:$N$];

\draw (M) coordinate[c2,fill=white,draw=green,label=right:$M$];

%\draw (aux1) coordinate[c1,fill=lightgray,draw=lightgray,label=right:{\scriptsize \color{lightgray}$O$}];

%\draw (P) coordinate[c1,fill=lightgray,draw=lightgray,label=right:{\scriptsize \color{lightgray}$P$}];

\end{tikzpicture}
&
\begin{tikzpicture}  [scale=0.5]

\tikzstyle{every path}=[line width=1pt]

\newdimen\ms
\ms=0.1cm
\tikzstyle{s1}=[color=lightgray,rectangle,inner sep=3.5]
\tikzstyle{c4}=[circle,inner sep={\ms/8},minimum size=5*\ms]
\tikzstyle{c3}=[circle,inner sep={\ms/8},minimum size=4*\ms]
\tikzstyle{c2}=[circle,inner sep={\ms/8},minimum size=3*\ms]
\tikzstyle{c1}=[circle,inner sep={\ms/8},minimum size=2*\ms]
\tikzstyle{cs1}=[circle,inner sep={\ms/8},minimum size=1*\ms]

% Define positions of all observables


\coordinate (psi) at (1,0);
\coordinate (vd) at (0,2);
\coordinate (dv) at (4,2);
\coordinate (uu) at (2,4);
\coordinate (vu) at (1,5);
\coordinate (uv) at (3,5);
\coordinate (cv) at (0,6);
\coordinate (vc) at (4,6);
\coordinate (dd) at (2.5,8);

\coordinate (aux1) at (4.5,1);
\coordinate (aux2) at (9,1);
\coordinate (aux3) at (4.5,7.8);
\coordinate (aux4) at (8.5,7.8);

\coordinate (P) at (10.5,8);

\coordinate (M) at (6.5,6);
\coordinate (N) at (12,0);

\coordinate (psi2) at ({3+9},0);
\coordinate (vd2) at ({0+9},2);
\coordinate (dv2) at ({4+9},2);
\coordinate (uu2) at ({2+9},4);
\coordinate (vu2) at ({1+9},5);
\coordinate (uv2) at ({3+9},5);
\coordinate (cv2) at ({0+9},6);
\coordinate (vc2) at ({4+9},6);
\coordinate (dd2) at ({1.5+9},8);


% draw contexts

\draw [color=lightgray] (psi) -- (vd)  coordinate[c1,fill=lightgray,draw=lightgray,pos=0.33,label=below left:{\scriptsize \color{lightgray}2}] (2)  coordinate[c1,fill=lightgray,draw=lightgray,pos=0.66,label=below left:{\scriptsize \color{lightgray}3}] (3);
%\draw [color=lightgray] (psi) -- (uu)  coordinate[c1,fill=lightgray,draw=lightgray,pos=0.33,label={[left,xshift=0.2mm]:{\scriptsize \color{lightgray}20}}] (20)  coordinate[c1,fill=lightgray,draw=lightgray,pos=0.66,label={[left,xshift=0.2mm]:{\scriptsize \color{lightgray}21}}] (21);
\draw [color=lightgray] (psi) -- (dv) coordinate[c1,fill=lightgray,draw=lightgray,pos=0.33,label=above:{\scriptsize \color{lightgray}17}]  (17) coordinate[c1,fill=lightgray,draw=lightgray,pos=0.66,label=above:{\scriptsize \color{lightgray}16}] (16);
\draw [color=lightgray] (vd) -- (vc);
\draw [color=lightgray] (dv) -- (cv);
\draw [color=lightgray] (vu) -- (uv) coordinate[c1,fill=lightgray,draw=lightgray,pos=0.33,label=above:{\scriptsize \color{lightgray}18}] (18)  coordinate[c1,fill=lightgray,draw=lightgray,pos=0.66,label=above:{\scriptsize \color{lightgray}19}] (19);
\draw [color=lightgray] (cv) -- (dd) coordinate[c1,fill=lightgray,draw=lightgray,pos=0.33,label=above left:{\scriptsize \color{lightgray}8}]  (8) coordinate[c1,fill=lightgray,draw=lightgray,pos=0.66,label=above left:{\scriptsize \color{lightgray}9}] (9);
\draw [color=lightgray] (vc) -- (dd) coordinate[c1,fill=lightgray,draw=lightgray,pos=0.33,label=right:{\scriptsize \color{lightgray}12}] (12)  coordinate[c1,fill=lightgray,draw=lightgray,pos=0.66,label=right:{\scriptsize \color{lightgray}11}] (11);

\draw [color=lightgray] (psi2) -- (vd2)  coordinate[c1,fill=lightgray,draw=lightgray,pos=0.33,label=above:{\scriptsize \color{lightgray}2'}] (22)  coordinate[c1,fill=lightgray,draw=lightgray,pos=0.66,label=above:{\scriptsize \color{lightgray}3'}] (32);
%\draw [color=lightgray] (psi2) -- (uu2)  coordinate[c1,fill=lightgray,draw=lightgray,pos=0.33,label={[above right,xshift=0.2mm]:{\scriptsize \color{lightgray}20'}}] (202)  coordinate[c1,fill=lightgray,draw=lightgray,pos=0.66,label={[left,xshift=0.2mm]:{\scriptsize \color{lightgray}21'}}] (212);
\draw [color=lightgray] (psi2) -- (dv2) coordinate[c1,fill=lightgray,draw=lightgray,pos=0.33,label=right:{\scriptsize \color{lightgray}17'}]  (172) coordinate[c1,fill=lightgray,draw=lightgray,pos=0.66,label=right:{\scriptsize \color{lightgray}16'}] (162);
\draw [color=lightgray] (vd2) -- (vc2);
\draw [color=lightgray] (dv2) -- (cv2);
\draw [color=lightgray] (vu2) -- (uv2) coordinate[c1,fill=lightgray,draw=lightgray,pos=0.33,label=above:{\scriptsize \color{lightgray}18'}] (182)  coordinate[c1,fill=lightgray,draw=lightgray,pos=0.66,label=above:{\scriptsize \color{lightgray}19'}] (192);
\draw [color=lightgray] (cv2) -- (dd2) coordinate[c1,fill=lightgray,draw=lightgray,pos=0.33,label=left:{\scriptsize \color{lightgray}8'}]  (82) coordinate[c1,fill=lightgray,draw=lightgray,pos=0.66,label=below:{\scriptsize \color{lightgray}9'}] (92);
\draw [color=lightgray] (vc2) -- (dd2) coordinate[c1,fill=lightgray,draw=lightgray,pos=0.33,label=right:{\scriptsize \color{lightgray}12'}] (122)  coordinate[c1,fill=lightgray,draw=lightgray,pos=0.66,label=right:{\scriptsize \color{lightgray}11'}] (112);


\draw [lightgray] plot [smooth] coordinates {(psi)  (aux1) (M) (aux4) (P) (uu2) };
\draw [lightgray] plot [smooth] coordinates {(uu) (dd) (aux3) (M) (aux2) (N)};

% draw atoms

\draw (psi) coordinate[c2,fill=white,draw=green,label=below:$\Psi$];

\draw (vd) coordinate[c1,fill=lightgray,draw=lightgray,label=above:$vd$];

\draw (dv) coordinate[c1,fill=lightgray,draw=lightgray,label=above:$dv$];

\draw (uu) coordinate[c1,fill=lightgray,draw=lightgray,label=right:$uu$];

\draw (vu) coordinate[c1,fill=lightgray,draw=lightgray,label=left:$vu$];

\draw (uv) coordinate[c1,fill=lightgray,draw=lightgray,label=right:$uv$];

\draw (cv) coordinate[c1,fill=lightgray,draw=lightgray,label=above:$cv$];

\draw (vc) coordinate[c1,fill=lightgray,draw=lightgray,label=right:$vc$];

\draw (dd) coordinate[c2,fill=red,draw=red,label=above:$dd$];

\draw (psi2) coordinate[c2,fill=white,draw=green,label=below:$\Psi'$];

\draw (vd2) coordinate[c1,fill=lightgray,draw=lightgray,label=above:$vd'$];

\draw (dv2) coordinate[c1,fill=lightgray,draw=lightgray,label=right:$dv'$];

\draw (uu2) coordinate[c1,fill=lightgray,draw=lightgray,label=right:$uu'$];

\draw (vu2) coordinate[c1,fill=lightgray,draw=lightgray,label=left:$vu'$];

\draw (uv2) coordinate[c1,fill=lightgray,draw=lightgray,label=right:$uv'$];

\draw (cv2) coordinate[c1,fill=lightgray,draw=lightgray,label=right:$cv'$];

\draw (vc2) coordinate[c1,fill=lightgray,draw=lightgray,label=right:$vc'$];

\draw (dd2) coordinate[c2,fill=red,draw=red,label=above:$dd'$];

%\draw (N) coordinate[c1,fill=lightgray,draw=lightgray,label=right:$N$];

\draw (M) coordinate[c2,fill=white,draw=green,label=right:$M$];

%\draw (aux1) coordinate[c1,fill=lightgray,draw=lightgray,label=right:{\scriptsize \color{lightgray}$O$}];

%\draw (P) coordinate[c1,fill=lightgray,draw=lightgray,label=right:{\scriptsize \color{lightgray}$P$}];

\end{tikzpicture}
&
\begin{tikzpicture}  [scale=0.5]

\tikzstyle{every path}=[line width=1pt]

\newdimen\ms
\ms=0.1cm
\tikzstyle{s1}=[color=lightgray,rectangle,inner sep=3.5]
\tikzstyle{c3}=[circle,inner sep={\ms/8},minimum size=4*\ms]
\tikzstyle{c2}=[circle,inner sep={\ms/8},minimum size=3*\ms]
\tikzstyle{c1}=[circle,inner sep={\ms/8},minimum size=2*\ms]
\tikzstyle{cs1}=[circle,inner sep={\ms/8},minimum size=1*\ms]

% Define positions of all observables


\coordinate (a1) at (2,0);
\coordinate (a2) at (0,2);
\coordinate (a3) at (4,2);
\coordinate (uu) at (2,4);
\coordinate (a4) at (1,5);
\coordinate (a5) at (3,5);
\coordinate (a6) at (0,6);
\coordinate (a7) at (4,6);
\coordinate (a8) at (2,8);

\coordinate (a12) at ({2+8},0);
\coordinate (a22) at ({0+8},2);
\coordinate (a32) at ({4+8},2);
\coordinate (uu2) at ({2+8},4);
\coordinate (a42) at ({1+8},5);
\coordinate (a52) at ({3+8},5);
\coordinate (a62) at ({0+8},6);
\coordinate (a72) at ({4+8},6);
\coordinate (a82) at ({2+8},8);

\coordinate (O) at (10,8);
\coordinate (M) at (6,4);
\coordinate (N) at (10,0);

\coordinate (aux1) at (4.5,0.5);
\coordinate (aux2) at (4.5,7.5);

\coordinate (aux12) at (7.5,0.5);
\coordinate (aux22) at (7.5,7.5);
% draw contexts

\draw [color=lightgray] (a1) -- (a2)  coordinate[c1,fill=lightgray,draw=lightgray,pos=0.5] (2) ;
\draw [color=lightgray] (a1) -- (a3) coordinate[c1,fill=lightgray,draw=lightgray,pos=0.5]  (17);
\draw [color=lightgray] (a2) -- (a7);
\draw [color=lightgray] (a3) -- (a6);
\draw [color=lightgray] (a4) -- (a5) coordinate[c1,fill=lightgray,draw=lightgray,pos=0.5] (18);
\draw [color=lightgray] (a6) -- (a8) coordinate[c1,fill=lightgray,draw=lightgray,pos=0.5]  (8);
\draw [color=lightgray] (a7) -- (a8) coordinate[c1,fill=lightgray,draw=lightgray,pos=0.5] (12);


\draw [color=lightgray!100!white] (a12) -- (a22)  coordinate[c1,fill=lightgray,draw=lightgray,pos=0.5] (2) ;
\draw [color=lightgray!100!white] (a12) -- (a32) coordinate[c1,fill=lightgray,draw=lightgray,pos=0.5]  (17);
\draw [color=lightgray!100!white] (a22) -- (a72);
\draw [color=lightgray!100!white] (a32) -- (a62);
\draw [color=lightgray!100!white] (a42) -- (a52) coordinate[c1,fill=lightgray,draw=lightgray,pos=0.5] (18);
\draw [color=lightgray!100!white] (a62) -- (a82) coordinate[c1,fill=lightgray,draw=lightgray,pos=0.5]  (8);
\draw [color=lightgray!100!white] (a72) -- (a82) coordinate[c1,fill=lightgray,draw=lightgray,pos=0.5] (12);

\draw [lightgray] plot [smooth] coordinates {(a1) (aux1) (M) (aux22) (O)};
\draw [lightgray] plot [smooth] coordinates {(a8) (aux2) (M) (aux12) (N)};

% draw atoms

\draw (a1) coordinate[c2,fill=white,draw=green,label=below:$a_1$];

\draw (a2) coordinate[c1,fill=lightgray,draw=lightgray,label=above:$a_2$];

\draw (a3) coordinate[c1,fill=lightgray,draw=lightgray,label=above:$a_3$];


\draw (a4) coordinate[c1,fill=lightgray,draw=lightgray,label=left:$a_4$];

\draw (a5) coordinate[c1,fill=lightgray,draw=lightgray,label=right:$a_5$];

\draw (a6) coordinate[c1,fill=lightgray,draw=lightgray,label=above:$a_6$];

\draw (a7) coordinate[c1,fill=lightgray,draw=lightgray,label=above:$a_7$];

\draw (a8) coordinate[c2,fill=red,draw=red,label=above:$a_8$];

\draw (a12) coordinate[c2,fill=white,draw=green,label=below:$a_1'$];

\draw (a22) coordinate[c1,fill=lightgray,draw=lightgray,label=above:$a_2'$];

\draw (a32) coordinate[c1,fill=lightgray,draw=lightgray,label=above:$a_3'$];


\draw (a42) coordinate[c1,fill=lightgray,draw=lightgray,label=left:$a_4'$];

\draw (a52) coordinate[c1,fill=lightgray,draw=lightgray,label=right:$a_5'$];

\draw (a62) coordinate[c1,fill=lightgray,draw=lightgray,label=above:$a_6'$];

\draw (a72) coordinate[c1,fill=lightgray,draw=lightgray,label=above:$a_7'$];

\draw (a82) coordinate[c2,fill=red,draw=red,label=above:$a_8'$];

%\draw (N) coordinate[c1,fill=lightgray,draw=lightgray,label=right:$N$];

\draw (M) coordinate[c2,fill=white,draw=green,label=right:$M$];

%\draw (O) coordinate[c1,fill=lightgray,draw=lightgray,label=right:{\scriptsize \color{lightgray}$O$}];

\end{tikzpicture}
\end{tabular}
}
\end{center}

KS, \href{https://doi.org/10.1103/PhysRevA.103.022204}{DOI:10.1103/PhysRevA.103.022204}
}

\section{Nonunitality}

\frame{
 \frametitle{Hypergraph with nonunital set of 6 value assignments}

\begin{center}
  \includegraphics[height=0.5\textwidth]{2021-QCQMB2021-pres-Tkadlec-nonunital.png}
\end{center}
{\footnotesize
Josef Tkadlec,
\href{https://doi.org/10.1023/A:1026646229896}{DOI:10.1023/A:1026646229896}
based on Erna Clavadetscher-Seeberger, Diss. ETH Z\"�rich (Specker)
\href{https://www.research-collection.ethz.ch/handle/20.500.11850/138142}{handle ETH: 20.500.11850/138142}
based on Sch\"utte's letters to Specker, April 22nd, 1965 \& November 3rd, 1983 (communicated to KS by Specker).}

}

\section{Exotic contextuality from coloring}

\frame{
 \frametitle{Hypergraph with exotic contextuality derived from coloring}

Hypergraph of biconnected intertwined contexts representing complete graphs with a separating set of 6 two-valued states which is non-partitionable: $G_{32}$, cf.  Figure~6, p.~121 Greechie (1971) DOI:~10.1016/0097-3165(71)90015-X

\begin{center}
\resizebox{.9\textwidth}{!}{%
                        \begin{tabular}{ c c c }
                                \begin{tikzpicture}  [scale=0.8]

                                        \newdimen\ms
                                        \ms=0.05cm

                                        \tikzstyle{every path}=[line width=1pt]

                                        \tikzstyle{c3}=[circle,inner sep={\ms/8},minimum size=6*\ms]
                                        \tikzstyle{c2}=[circle,inner sep={\ms/8},minimum size=4*\ms]
                                        \tikzstyle{c1}=[circle,inner sep={\ms/8},minimum size=0.8*\ms]

                                        % Radius of regular polygons
                                        \newdimen\R
                                        \R=30mm     % outer circle

                                        %\r= { \R * sqrt(3) }     % inner circle
                                        %\newdimen\r
                                        %\r=    {\R * sqrt(3)/2}       % inner circle

                                        %\newdimen\K
                                        %\K=3cm

                                        % Define positions of all observables
                                        \path
                                        ({ 180 - 0 * 360 /6}:\R      ) coordinate(1)
                                        ({ 180 - 30 - 0 * 360 /6}:{\R * sqrt(3)/2}      ) coordinate(2)
                                        ({ 180 - 1 * 360 /6}:\R   ) coordinate(3)
                                        ({ 180 - 30 - 1 * 360 /6}:{\R * sqrt(3)/2}   ) coordinate(4)
                                        ({ 180 - 2 * 360 /6}:\R  ) coordinate(5)
                                        ({ 180 - 30 - 2 * 360 /6}:{\R * sqrt(3)/2}  ) coordinate(6)
                                        ({ 180 - 3 * 360 /6}:\R  ) coordinate(7)
                                        ({ 180 - 30 - 3 * 360 /6}:{\R * sqrt(3)/2}  ) coordinate(8)
                                        ({ 180 - 4 * 360 /6}:\R     ) coordinate(9)
                                        ({ 180 - 30 - 4 * 360 /6}:{\R * sqrt(3)/2}     ) coordinate(10)
                                        ({ 180 - 5 * 360 /6}:\R     ) coordinate(11)
                                        ({ 180 - 30 - 5 * 360 /6}:{\R * sqrt(3)/2}     ) coordinate(12)
                                        ;

                                        % draw contexts

                                        \draw [color=cyan] (1) -- (2) -- (3);
                                        \draw [color=red] (3) -- (4) -- (5);
                                        \draw [color=green] (5) -- (6) -- (7);
                                        \draw [color=blue] (7) -- (8) -- (9);
                                        \draw [color=magenta] (9) -- (10) -- (11);    %
                                        \draw [color=olive] (11) -- (12) -- (1);    %
                                        \draw [color=orange] (2) -- (8)  coordinate[pos=0.85]  (15);
                                        \draw [color=teal] (4) -- (10)  coordinate[pos=0.15]  (13);
                                        \draw [color=MidnightBlue] (6) -- (12)  coordinate[pos=0.325]  (14);
                                        \draw [color=gray] (13) --(15);

                                        %
                                        %%
                                        %% draw atoms
                                        %%
                                        %
                                        \draw (1) coordinate[c3,fill=cyan,label={left: $\{ 1,2\} $}];   %
                                        \draw (1) coordinate[c2,fill=olive];  %
                                        %
                                        \draw (2) coordinate[c3,fill=cyan,label={above left: $\{ 3,4\}$}];    %
                                        \draw (2) coordinate[c2,fill=orange];    %
                                        %
                                        \draw (3) coordinate[c3,fill=red,label={above: $\{ 5,6\} $}]; %
                                        \draw (3) coordinate[c2,fill=cyan];  %
                                        %
                                        \draw (4) coordinate[c3,fill=red,label={above: $\{ 1,3\}$}];  %
                                        \draw (4) coordinate[c2,fill=teal];  %
                                        %
                                        \draw (5) coordinate[c3,fill=green,label={above: $\{ 2,4\} $}];  %
                                        \draw (5) coordinate[c2,fill=red];  %
                                        %
                                        \draw (6) coordinate[c3,fill=green,label={above right: $\{ 1,5\} $}];
                                        \draw (6) coordinate[c2,fill=MidnightBlue];
                                        %
                                        \draw (7) coordinate[c3,fill=blue,label={right: $\{ 3,6\}$}];  %
                                        \draw (7) coordinate[c2,fill=green];  %
                                        %
                                        \draw (8) coordinate[c3,fill=blue,label={below right: $\{ 2,5\}$}];  %
                                        \draw (8) coordinate[c2,fill=orange];  %
                                        %
                                        \draw (9) coordinate[c3,fill=magenta,label={below: $\{ 1,4\}$}];
                                        \draw (9) coordinate[c2,fill=blue];  %
                                        %
                                        \draw (10) coordinate[c3,fill=magenta,label={below: $\{ 2,6\}$}];  %
                                        \draw (10) coordinate[c2,fill=teal];  %
                                        %
                                        \draw (11) coordinate[c3,fill=olive,label={below: $\{ 3,5\}$}];  %
                                        \draw (11) coordinate[c2,fill=magenta];  %
                                        %
                                        \draw (12) coordinate[c3,fill=olive,label={below left: $\{ 4,6\}$}];
                                        \draw (12) coordinate[c2,fill=MidnightBlue];
                                        %
                                        \draw (13) coordinate[c3,fill=MidnightBlue,label={right: $\{ 4,5\}$}];  %
                                        \draw (13) coordinate[c2,fill=gray];  %
                                        %
                                        \draw (14) coordinate[c3,fill=teal,label=0:{$\{ 2,3\}$}];  %
                                        \draw (14) coordinate[c2,fill=gray];  %
                                        %
                                        \draw (15) coordinate[c3,fill=orange,label={below left: $\{1,6\}$}];  %
                                        \draw (15) coordinate[c2,fill=gray];  %
                                        %
                                \end{tikzpicture}
                                &&
                                \begin{tikzpicture}  [scale=0.8]

                                        \newdimen\ms
                                        \ms=0.05cm

                                        \tikzstyle{every path}=[line width=1pt]

                                        \tikzstyle{c3}=[circle,inner sep={\ms/8},minimum size=6*\ms]
                                        \tikzstyle{c2}=[circle,inner sep={\ms/8},minimum size=4*\ms]
                                        \tikzstyle{c1}=[circle,inner sep={\ms/8},minimum size=0.8*\ms]

                                        % Radius of regular polygons
                                        \newdimen\R
                                        \R=30mm     % outer circle

                                        %\r= { \R * sqrt(3) }     % inner circle
                                        %\newdimen\r
                                        %\r=    {\R * sqrt(3)/2}       % inner circle

                                        %\newdimen\K
                                        %\K=3cm

                                        % Define positions of all observables
                                        \path
                                        ({ 180 - 0 * 360 /6}:\R      ) coordinate(1)
                                        ({ 180 - 30 - 0 * 360 /6}:{\R * sqrt(3)/2}      ) coordinate(2)
                                        ({ 180 - 1 * 360 /6}:\R   ) coordinate(3)
                                        ({ 180 - 30 - 1 * 360 /6}:{\R * sqrt(3)/2}   ) coordinate(4)
                                        ({ 180 - 2 * 360 /6}:\R  ) coordinate(5)
                                        ({ 180 - 30 - 2 * 360 /6}:{\R * sqrt(3)/2}  ) coordinate(6)
                                        ({ 180 - 3 * 360 /6}:\R  ) coordinate(7)
                                        ({ 180 - 30 - 3 * 360 /6}:{\R * sqrt(3)/2}  ) coordinate(8)
                                        ({ 180 - 4 * 360 /6}:\R     ) coordinate(9)
                                        ({ 180 - 30 - 4 * 360 /6}:{\R * sqrt(3)/2}     ) coordinate(10)
                                        ({ 180 - 5 * 360 /6}:\R     ) coordinate(11)
                                        ({ 180 - 30 - 5 * 360 /6}:{\R * sqrt(3)/2}     ) coordinate(12)
                                        ;

                                        % draw contexts

                                        \draw [color=cyan] (1) -- (2) -- (3);
                                        \draw [color=red] (3) -- (4) -- (5);
                                        \draw [color=green] (5) -- (6) -- (7);
                                        \draw [color=blue] (7) -- (8) -- (9);
                                        \draw [color=magenta] (9) -- (10) -- (11);    %
                                        \draw [color=olive] (11) -- (12) -- (1);    %
                                        \draw [color=orange] (2) -- (8)  coordinate[pos=0.85]  (15);
                                        \draw [color=teal] (4) -- (10)  coordinate[pos=0.15]  (13);
                                        \draw [color=MidnightBlue] (6) -- (12)  coordinate[pos=0.325]  (14);
                                        \draw [color=gray] (13) --(15);

                                        %
                                        %%
                                        %% draw atoms
                                        %%
                                        %
                                        \draw (1) coordinate[c3,fill=red,label={left: $\{ 1,2\} $}];   %
                                        %
                                        \draw (2) coordinate[c3,fill=green,label={above left: $\{ 3,4\}$}];    %
                                        %
                                        \draw (3) coordinate[c3,fill=blue,label={above: $\{ 5,6\} $}]; %
                                        %
                                        \draw (4) coordinate[c3,fill=red,label={above: $\{ 1,3\}$}];  %
                                        %
                                        \draw (5) coordinate[c3,fill=cyan,label={above: $\{ 2,4\} $}];  %
                                        %
                                        \draw (6) coordinate[c3,fill=red,label={above right: $\{ 1,5\} $}];
                                        %
                                        \draw (7) coordinate[c3,fill=green,label={right: $\{ 3,6\}$}];  %
                                        %
                                        \draw (8) coordinate[c3,fill=blue,label={below right: $\{ 2,5\}$}];  %
                                        %
                                        \draw (9) coordinate[c3,fill=red,label={below: $\{ 1,4\}$}];
                                        %
                                        \draw (10) coordinate[c3,fill=cyan,label={below: $\{ 2,6\}$}];  %
                                        %
                                        \draw (11) coordinate[c3,fill=green,label={below: $\{ 3,5\}$}];  %
                                        %
                                        \draw (12) coordinate[c3,fill=cyan,label={below left: $\{ 4,6\}$}];
                                        %
                                        \draw (13) coordinate[c3,fill=blue,label={right: $\{ 4,5\}$}];  %
                                        %
                                        \draw (14) coordinate[c3,fill=green,label=0:{$\{ 2,3\}$}];  %
                                        %
                                        \draw (15) coordinate[c3,fill=red,label={below left: $\{1,6\}$}];  %
                                        %
                                \end{tikzpicture}
                        \end{tabular}
}
                \end{center}
Mohammad H. Shekarriz \& KS, vertex labeling by partitions of $\{1,2,3,4,5,6\}$ with no faithful orthogonal representation  \href{http://arxiv.org/abs/2105.08520}{arXiv:2105.08520}.
}

\frame{

\centerline{\Large {\color{magenta} Thank you for your attention!}}

\begin{center}\color{orange}
$\widetilde{\qquad \qquad }$
$\widetilde{\qquad \qquad}$
$\widetilde{\qquad \qquad }$
\end{center}
 }
 \end{document}


A
\colorbox{yellow!40}{\begin{minipage}{.8\textwidth} \color{blue}
\bf \large context\end{minipage}}
or
\colorbox{yellow!40}{\begin{minipage}{.8\textwidth} \color{blue}
\bf \large maximal observable\end{minipage}}
