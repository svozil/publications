\PassOptionsToPackage{usenames,dvipsnames}{xcolor}
%\documentclass[amsmath,table,sans,amsfonts, handout]{beamer}
\documentclass[amsmath,table,sans,amsfonts,hyperref={colorlinks,citecolor=blue,linkcolor=blue,urlcolor=purple}]{beamer}
\usepackage[T1]{fontenc}
%%\usepackage{beamerthemeshadow}
%%\usepackage[headheight=1pt,footheight=10pt]{beamerthemeboxes}
%%\addfootboxtemplate{\color{structure!80}}{\color{white}\tiny \hfill Karl Svozil (TU Vienna)\hfill}
%%\addfootboxtemplate{\color{structure!65}}{\color{white}\tiny \hfill mur.sat \hfill}
%%\addfootboxtemplate{\color{structure!50}}{\color{white}\tiny \hfill Graz, 2010-12-11\hfill}
%\usepackage[dark]{beamerthemesidebar}
%\usepackage[headheight=24pt,footheight=12pt]{beamerthemesplit}
%\usepackage{beamerthemesplit}
%\usepackage[bar]{beamerthemetree}
\usepackage{graphicx}
\usepackage{pgf}
%\usepackage{eepic}
%\newcommand{\Red}{\color{Red}}  %(VERY-Approx.PANTONE-RED)
%\newcommand{\Green}{\color{Green}}  %(VERY-Approx.PANTONE-GREEN)

\usepackage{mathbbol}

\usepackage{multirow}

\definecolor{applegreen}{rgb}{0.55, 0.71, 0.0}

\usepackage{fourier-orns}  %fancy symbols https://mirror.easyname.at/ctan/fonts/fourier-GUT/doc/fourier-orns-doc.pdf

%\usepackage{musixtex}

\newcommand{\Abschnitt}[1]{{\section #1}}

%%%%%%%%%%%%%%%%%%%%%%%%%%%%%
\usepackage{iftex}
\ifxetex
\usepackage{fontspec}% Schriftumschaltung mit den nativen XeTeX-Anweisungen
                     % vornehmen. Voreinstellung: Latin Modern
%\usepackage[ngerman]{babel}% Sprachumschaltung: Deutsch nach neuer Rechtschreibung



%
% XeLaTeX
%
\XeTeXinputencoding cp1252
\usepackage{fontspec}
%%\setmainfont{Times New Roman}
\setmainfont{Garamond}
\setsansfont{Garamond}
%\setmainfont{EB Garamond}
%\setsansfont{EB Garamond}
%
\else
\usepackage[latin1]{inputenc}
\usepackage[T1]{fontenc}
\fi
%%%%%%%%%%%%%%%%%%%%%%%%%%%%%

%\RequirePackage[german]{babel}
%\selectlanguage{german}
%\RequirePackage[isolatin]{inputenc}

%\pgfdeclareimage[height=0.5cm]{logo}{tu-logo}
%\logo{\pgfuseimage{logo}}
\beamertemplatetriangleitem
%\beamertemplateballitem

\beamerboxesdeclarecolorscheme{alert}{red}{red!15!averagebackgroundcolor}
%\begin{beamerboxesrounded}[scheme=alert,shadow=true]{}
%\end{beamerboxesrounded}

%\beamersetaveragebackground{yellow!10}

%\beamertemplatecircleminiframe


\usepackage{mathbbol}

\usepackage{multirow}



\newcommand\myotimes{ }

\newtheorem{question}{Question}
\newtheorem{conjecture}[question]{Principle}
\newtheorem{challenge}[question]{Challenge}

\usepackage{tikz}
\usetikzlibrary{calc,shapes.geometric}

\newcommand{\bra}[1]{\left< #1 \right|}
\newcommand{\ket}[1]{\left| #1 \right>}

\newcommand{\iprod}[2]{\langle #1 | #2 \rangle}
\newcommand{\mprod}[3]{\langle #1 | #2 | #3 \rangle}
\newcommand{\oprod}[2]{| #1 \rangle\langle #2 |}

\newcommand{\proj}[3]{\begin{smallmatrix} #1 & #2 & #3 \end{smallmatrix}}
\newcommand{\projbf}[3]{\begin{smallmatrix} \mathbf{#1} & \mathbf{#2} & \mathbf{#3} \end{smallmatrix}}

\sloppy
\parskip .7em %vskip between paragraphs

\newcommand{\seq}[1]{\mathbf{#1}}
\newcommand{\floor}[1]{\left\lfloor #1 \right\rfloor}
\newcommand{\ceil}[1]{\left\lceil #1 \right\rceil}
\newcommand{\m}[1]{\widetilde{#1}}
%\newcommand{\p}[1]{\scriptsize\textcolor{black}{$[#1]$}}

\usepackage[most]{tcolorbox}
\begin{document}

\title{\textcolor{black}{\bf (FAPP) Infinity (FAPP) Does It}}
\subtitle{\footnotesize \url{http://tph.tuwien.ac.at/~svozil/publ/2024-Erice-pres.pdf}
\\
\footnotesize based on \href{https://arxiv.org/abs/2409.06470}{arXiv:2409.06470}
}
\author{\textcolor{black}{Karl Svozil}}
\institute{\normalsize \textcolor{black}{Institute for Theoretical Physics, TU Wien}\\
\textcolor{black}{karl.svozil@tuwien.ac.at}
%{\tiny Disclaimer: Die hier vertretenen Meinungen des Autors verstehen sich als Diskussionsbeitr�ge und decken sich nicht notwendigerweise mit den Positionen der Technischen Universit�t Wien oder deren Vertreter.}
}
\date{{\color{purple}Saturday, November 2nd, 2024,\\
Mathematical Aspects in Non Equilibrium Systems --- From Micro to Macro, Erice, Sicily, Italy}}
\maketitle


% \frame{
% \frametitle{Contents}
% \tableofcontents
% }

 \frame{
\frametitle{Erice, 1981: Dirac, Teller, Wigner, Zichichi, (Andreotti, with wife), $\ldots$, KS}

``SEMINARIO INTERNAZIONALE SULLE IMPLICAZIONI MONDIALI DI UN CONFLITTO NUCLEARE Erice -- 14/19 Agosto 1981''  ---
The World-wide Implications of a Nuclear War. 1st International Seminar on Nuclear War, Erice, Italy, 14 -- 19 August 1981


Eg, Paul Dirac, ``The Futility of War'', etc, ``Hidden'' Proceedings:
DOI 10.1142/9789814536608

\begin{center}
\includegraphics[width=0.4\textwidth]{2024-Erice-Foto3-Wigner-Dirac-Zichichi.jpg}\\
{\tiny (\url{https://ettoremajoranafoundation.it/the-history/} misdated photo 1982 $\longrightarrow$ 1981?)}
\end{center}
 }

\section{The Quantum Measurement Problem: Two types of quantum processes}

\begin{frame}%[shrink=4]
 \frametitle{The Quantum Measurement Problem: Two types of quantum processes}

Von Neumann (1932, DOI 10.1007/978-3-642-61409-5, as quoted by Everett, 1957, DOI 10.1103/RevModPhys.29.454)
differentiates between two distinct types of processes (``Eingriff, Prozess''):


\begin{itemize}

\pause
\item   \color{magenta}
Process 1: The discontinuous change brought about
by the observation of a quantity with eigenstates
$\phi_1, \phi_2, \cdots ,$
in which the state $\psi$ will be changed to
the state $\phi_j$ with probability $\vert \langle \psi \vert \phi_j \rangle \vert^2$.

\pause
\item  \color{magenta}
Process 2: The continuous, deterministic change of
state of an isolated system with time according to
a wave equation $\partial \psi / \partial t= U \psi$, where U is a linear [[unitary]]
operator.

\end{itemize}


\pause
\color{blue}
The Quantum Measurement Problem:
Processes of type 2 can never ``produce'' any process of type 1. Or, can they?

\end{frame}


\section{One possible solution: Infinite tensor products from an infinite nesting of Wigner's friends}

\begin{frame}%[shrink=4]
 \frametitle{One possible solution: Infinite tensor products from an infinite nesting of Wigner's friends}

{\Large
\begin{itemize}

\pause
\item  \color{magenta}
$\ldots$ through infinite tensor products interpreted as nested, chained, or iterated Wigner's friend scenarios.

\pause
\item  \color{magenta}
Infinite tensor products can disrupt unitary equivalence through sectorization and factorization.

\pause
\item  \color{magenta}
parallels to concepts from real analysis, recursive mathematics, and statistical physics.

\end{itemize}
}

\end{frame}


\begin{frame}[shrink=4]
 \frametitle{Von Neumann-Landau scheme of measurement}

The `object' is prepared in an initial state $\vert \psi \rangle$ which,
with respect to a `mismatching' context (relative to that preparation)---or,
 equivalently, orthonormal basis or maximal operator, is in a coherent superposition (linear combination)
$\vert \psi \rangle = \sum_{i=1}^n a_i \vert \psi_i \rangle$ of (basis) elements $\vert \psi_i\rangle$ of that context.

The `measurement ancilla'---along with synonymous terms such as `provision', `component', or `arrangement'---should be represented by another state,
denoted as $\vert \varphi \rangle$.
This state, in relation to a suitable basis ${\vert \varphi_1 \rangle, \ldots, \vert \varphi_n \rangle}$,
can also be expressed as a coherent superposition: $\vert \varphi \rangle = \sum_{j=1}^n b_j \vert \varphi_j \rangle$.
When an interaction occurs between the `object' and the `measurement ancilla',
the combined state
\begin{equation}
\vert \Psi \rangle = \sum_{i,j=1}^n c_{ij} \vert \psi_i \rangle \otimes \vert \varphi_j \rangle =
\sum_{i,j=1}^n c_{ij} \vert \psi_i \varphi_j \rangle
\label{2024-u-vNsiqm}
\end{equation}
becomes a non-factorizable tensor product, meaning that the coefficients $c_{ij}$
cannot be written as products $a_i b_j$.
\end{frame}




\begin{frame}%[shrink]
 \frametitle{Infinite chaining of Von Neumann-Landau scheme of measurement}

{\footnotesize
The finite chaining of the von Neumann-Landau scheme of measurement is straightforward,
infinite chaining requires infinite tensor products which are highly nontrivial
%(von Neumann  1939, \url{http://eudml.org/doc/88704}, with Murray 1936-43 DOI 10.2307/1968693 -- 10.2307/1969107)
(von Neumann  1939, with Murray 1936-43).
\begin{itemize}

\item[(i)]
We start with elementary tensors \(\bigotimes_{n=1}^\infty \vert k_n \rangle \).

\item[(ii)]
We define the inner product on elementary tensors by the product of the individual inner products:
\[ \left\langle \bigotimes_{n=1}^\infty \vert k_n\rangle \middle| \bigotimes_{n=1}^\infty \vert j_n\rangle  \right\rangle
= \begin{cases}
    \prod_{n=1}^\infty \langle k_n \vert j_n \rangle_{\mathcal{H}_n} &  \text{converging,}\\
    0 & \text{otherwise.}
\end{cases}
\]

\item[(iii)]
We then consider finite linear combinations of these elementary tensors:
   \[ \sum_{i} c_i \left( \bigotimes_{n=1}^\infty \vert {k_n^{(i)}} \rangle \right), \]
   where \(c_i\) are complex coefficients and \(\vert k_n^{(i)} \rangle\)
are basis vectors in the elementary tensor product labeled by a \colorbox{yellow!30}{\color{red}countable (enumerable) index~$i$}.

\item[(iv)]
We finally obtain the Hilbert space \(\bigotimes_{n=1}^\infty \mathcal{H}_n\)
by taking the completion of the space of finite linear combinations of elementary tensors.

\end{itemize}

}

\end{frame}

\section{Problems and `resolution' by sectorization}

\begin{frame}
\frametitle{Problems with infinite products: cardinality}

However, this construction does not directly address the uncountable infinity of elementary products. To illustrate this,
we can draw an analogy with the representation of real numbers as expansions in an $n$-ary system, where they are encoded using a (finite) set of basis elements.
Just as Cantor's diagonal argument shows that the reals cannot be enumerated by any countable set of indices,
so too can we not enumerate the uncountable infinity of elementary products in the infinite tensor product.
In von Neumann's own words (1939, \url{http://eudml.org/doc/88704}):

\colorbox{yellow!30}{
\begin{minipage}{\textwidth}
\color{red}\textit{``generalisations of the direct product lead to higher set-theoretical powers
(G. Cantor's ``Alephs'').''}
\end{minipage}
}

\end{frame}


 \frame{
 \frametitle{Problems with infinite products: inner product and orthogonality}

{\footnotesize


When we extend to an infinite tensor product, say
\( | \Psi \rangle = \bigotimes_{i=1}^{\infty}| \psi_i \rangle \) and \( |  \Phi \rangle = \bigotimes_{i=1}^{\infty} | \phi_i \rangle \),
the inner product would apparently be
\(
\langle \Psi \vert \Phi \rangle = \prod_{i=1}^{\infty} \langle \psi_i \vert \phi_i \rangle
\).

Suppose, for the sake of demonstration, that each \( \langle \psi_i \vert \phi_i \rangle \)
is very slightly less than 1. As a consequence, the infinite product can converge to zero, and thus those
vectors which are only `slightly apart' appear orthogonal.
Formally, suppose that
\(
\langle \psi_i \vert \phi_i \rangle = 1 - \epsilon_i  = \delta_i
\),
or
\(
  \epsilon_i = 1 - \langle \psi_i \vert \phi_i \rangle
\),
where
\(
0 < \epsilon_i \ll 1
\).
For a large number of factors, the infinite product behaves approximately as
\begin{equation}
\langle \Psi \vert \Phi \rangle = \prod_{i=1}^{\infty} (1 - \epsilon_i) \approx \exp\left(-\sum_{i=1}^{\infty} \epsilon_i\right)
.
\label{2024-u-dwsp}
\end{equation}
If the series \( \sum_{i=1}^{\infty} \epsilon_i \) diverges (even if slowly), this product will converge to zero, that is,
\(
\prod_{i=1}^{\infty} (1 - \epsilon_i) \rightarrow 0.
\)

% ### Implications for Inner Products


Furthermore, the inner product would also become zero for states $\vert \Psi \rangle$ and $\vert \Phi \rangle$
that differ in only a single or a finite number of the infinitely many subfactor components, where $\langle \psi_i \vert \phi_i \rangle = 0$, with all the rest being identical.

}

}

 \frame{
 \frametitle{Problems with infinite products: inner product and orthogonality II}

{\huge
\color{magenta}
Take-home message: ``Problems with/loss of orthogonality (and other issues) indicate breakdown of unitary equivalence.''
}
}

 \frame{
 \frametitle{Sectorization}
{\footnotesize

Already von Neumann 1939, \url{http://eudml.org/doc/88704} suggested a
`solution' by partitioning the product space into equivalence classes `sectors' containing
only those infinite tensor products that are located `close to' each other,
such that their deviations from each other are `small'. Two vectors $ \vert \Psi \rangle $ and $ \vert \Phi \rangle $
are in the same sector if they are equivalent, denoted by $ \vert \Psi \rangle \sim \vert \Phi \rangle $,
when all but finitely many subfactors are either equal or unitary equivalent and to or
`close to' one another (only a unitary transformation apart); that is,

\begin{equation}
\sum_{i=1}^{\infty} \left( 1 - \langle \psi_i \vert \phi_i \rangle \right)
=\sum_{i=1}^{\infty} \epsilon_i
\le
\sum_{i=1}^{\infty} \left| 1 - \langle \psi_i \vert \phi_i \rangle \right|
< \infty
.
\end{equation}

Grangier and Van Den Bossche 2023, DOI {10.1007/s10701-023-00678-x}, 10.3390/e25121600, 10.1088/1742-6596/2533/1/012008
that those `sectors' (of infinite-dimensional equivalent states) correspond to `classical' counter readings.

Previous similar proposals by Hepp 1972, DOI 10.5169/seals-114381 have been criticized by Bell 1975, DOI {10.5169/seals-114661}
on the basis of the requirement of transfinite means. (Cf. also Bridgman 1934 ``A Physicist's Second Reaction to {M}engenlehre''.)

}

 }


\section{Further problems with unitary equivalence from factorization}
\frame{
 \frametitle{Further problems with unitary equivalence from factorization}
{\Large
\color{blue}
Another issue is related to entanglement of infinite components, resulting in different
 `factors' of type I, II, III that are not unitary equivalent:

$\;$\\
\huge
\color{magenta}
``Problems with factorization indicate breakdown of unitary equivalence.''
}

}

 \section{Is  (FAPP, Bell 1990 DOI 10.1088/2058-7058/3/8/26) (ir)reversibility means relative?}

\begin{frame}%[shrink=8]
 \frametitle{Is  (FAPP, Bell 1990 DOI 10.1088/2058-7058/3/8/26) (ir)reversibility means relative?
Some related questions.}


\begin{itemize}

\pause
\item   \color{magenta}
Loschmidt's \textit{Umkehreinwand} in classical statistical physics: according to Maxwell
\url{https://makingscience.royalsociety.org/items/rr_8_188} one needs
\textit{``avoiding all personal inquiries [[about individual molecules]] which would only get me into trouble.''}

\pause
\item  \color{magenta}
Specker sequences of computable numbers converge to an uncomputable limit.
One example is Chaitin's constant, the halting probability of prefix-free program codes on a universal computer.
whose rate of convergence is tied to the halting time, and therefore, `grows faster' than any computable function.

\pause
\item  \color{magenta}
Convergence of sequences of rational numbers to an irrational number in the real numbers.
For instance, consider the continued fraction or the binomial series expansions
of \(\sqrt{2}\), truncated at various points.
In the infinite limit there is no trace-back.

\end{itemize}


\pause
\color{blue}
The Quantum Measurement Problem:
Processes of type 2 can never ``produce'' any process of type 1. Or, can they?

\end{frame}

\begin{frame}%[shrink=4]
 \frametitle{A remark on For All Practical Purposes (FAPP)-ness}


{\Large
\begin{itemize}

\pause
\item  \color{magenta}
If you are comfortable with the second law of thermodynamics, you might consider that the quantum measurement problem can be resolved by means relativity.

\pause
\item  \color{magenta}
If you still have issues with the irreversibility of quantum measurements, you might consider rethinking the classical \textit{Umkehreinwand} and consider it unresolved.

\end{itemize}
}
\end{frame}

\begin{frame}%[shrink=4]
 \frametitle{Summary of one possible solution: Infinite tensor products from an infinite nesting of Wigner's friends}


{\Large
\begin{itemize}

\pause
\item  \color{magenta}
$\ldots$ through infinite tensor products interpreted as nested, chained, or iterated Wigner's friend scenarios.

\pause
\item  \color{magenta}
Infinite tensor products can disrupt unitary equivalence through sectorization and factorization.

\pause
\item  \color{magenta}
parallels to concepts from real analysis, recursive mathematics, and statistical physics.

\end{itemize}
}
\end{frame}

\frame{

\centerline{\Large {\color{magenta} Thank you for your attention!}}

\begin{center}\color{magenta}
$\widetilde{\qquad \qquad }$
$\widetilde{\qquad \qquad}$
$\widetilde{\qquad \qquad }$
\end{center}
 }
 \end{document}


















\section{ }

\frame{
 \frametitle{ }

\begin{itemize}
\item[$\bullet$] {
%\color{purple}
}
\pause
\item[$\bullet$] {
%\color{purple}
}
\end{itemize}
}

\section{ }

\frame{
 \frametitle{ }

\begin{itemize}
\item[$\bullet$] {
%\color{purple}
}
\pause
\item[$\bullet$] {
%\color{purple}
}
\end{itemize}
}

\section{ }

\frame{
 \frametitle{ }

\begin{itemize}
\item[$\bullet$] {
%\color{purple}
}
\pause
\item[$\bullet$] {
%\color{purple}
}
\end{itemize}
}

\section{ }

\frame{
 \frametitle{ }

\begin{itemize}
\item[$\bullet$] {
%\color{purple}
}
\pause
\item[$\bullet$] {
%\color{purple}
}
\end{itemize}
}

