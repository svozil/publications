\documentclass[aps,rmp,preprint,amsfonts,showpacs,showkeys]{revtex4}
%\documentclass[pra,showpacs,showkeys,amsfonts]{revtex4}
\usepackage{graphicx}
%\documentstyle[amsfonts]{article}
\RequirePackage{times}
%\RequirePackage{courier}
\RequirePackage{mathptm}
%\renewcommand{\baselinestretch}{1.3}
\bibpunct{[}{]}{,}{a}{}{;}
\begin{document}
%\sloppy



\title{Dynamical Theory of Price and Money in Volatile Markets \\ A Physicist's Reaction to Economics\footnote{This paper was stimulated by Gerhard Adam's question ``how is money created?'' in the cafeteria of the Vienna University of Technology approximately ten years ago. I am still striving to answer it.}
}
%\title{Dynamical Theory of Virtual Money in Volatile Markets}


\author{Karl Svozil}
\email{svozil@tuwien.ac.at}
\homepage{http://tph.tuwien.ac.at/~svozil}
\affiliation{Institute for Theoretical Physics, Vienna University of Technology,
Wiedner Hauptstra\ss e 8-10/136, A-1040 Vienna, Austria}

\begin{abstract}
The creation and annihilation of money and its economic effects are reviewed. Economic values appear ``in the mind'' of the market participants; e.g., by pretending, maintaining and achieving a particular price for a certain asset. Upon its creation by banks, this kind of ``value phantasy'' is converted into ``real money'' often in terms of buyer's debt accompanied by a simultaneous payment of fiat money to the seller. This money is then multiplied on the money market and is competing against other money supplies for the traded assets, goods and services, where it may cause dilution, inflation and reallocation of resources.
\end{abstract}

\pacs{01.60.+q}
\keywords{Money, Markets}

\maketitle


\tableofcontents

{\footnotesize
\begin{quote}
\begin{flushright}
You do look, my son, in a moved sort,              \\
As if you were dismay'd: be cheerful, sir.         \\
Our revels now are ended. These our actors,        \\
As I foretold you, were all spirits, and           \\
Are melted into air, into thin air:                \\
And, like the baseless fabric of this vision, $\ldots$     \\
Leave not a rack behind. We are such stuff         \\
As dreams are made on; and our little life         \\
Is rounded with a sleep.\\
{\em William Shakespeare}, in {\it The Tempest}
\end{flushright}
\end{quote}
}

\section{Introduction}

Physics and economics appear to be similar in certain aspects:
A national (global) economy is just another dynamical physical system; with an influx of energy and resources and a (sometimes) self-conscious habitat.
Alike other dynamical systems, economic variables  such as {\em price} or {\em gross domestic product} can be measured and enumerated in time series
which may follow some deterministic \cite{frank} or stochastic ``economic'' laws.
Intuitively, prices might be to the markets what physical theory is to Nature:
it represents some kind of ``value'' and ``truth.''
Also, one might pretend that, apart from its creation and annihilation via monetarization (see below) of assets and debt \cite{Wicksell-geld,Wicksell-1907,schneider-VWLIII}
money is conserved: if not burned as paper money it cannot vanish into thin air, and remains in the market forever:
in this respect it behaves like energy.
The transition of money into various forms of investment resembles the continuing transformation of one form of energy into another.
Accounting takes care of the bookkeeping.
From this perspective, the economy could be envisioned as a statistical system, not unlike the thermodynamic ensembles,
in particular the grand canonical ensemble; and price as entropy.

While such views and analogies may be formally justified, they concentrate on static equilibria and not on
dynamical processes of creation, annihilation and (re-)distribution of wealth and marketed assets.
Indeed, economics seems to be preoccupied with unique equilibria --- probably to be able to write down equations --- the most famous being the equality of supply and demand at a (single proper) market price.
Alas, for various reasons, economic equilibrium theories are inappropriate \cite{soros-alchemy,2006-Binswanger}:
\renewcommand{\labelenumi}{(\roman{enumi})}
\begin{enumerate}

\item
The market participant might suffer from an overload of information,
accompanied by a lack of reliable criteria or authorities to evaluate the information, or are fed with disinformation.
The perpetual flow of spontaneous news and opinions via the media make impossible the formation of a ``communication equilibrium.''
As markets tend to become virtualized, it is not totally unreasonable to suspect
that those who control; i.e., possess and pay through ads, the media control the market and public policy.
Thus the modern markets are driven by whatever {\em communication} and {\em (dis)information} is fed into them.
(Hayek used this argument to argue for an open market as opposed to a (centrally) planned economy \cite{Hayek-45}.)

\item
The intra-market dynamics might not be sufficiently efficient to settle prices; or there may be no convergence towards a {\em single} price,
but rather price cycles and other more chaotic regimes.

\item
Market participants do not always act rationally.
For instance, they tend to hold true what they would like to believe, are overconfident, optimistic and copy other people's actions.
The phantasies,  greed, emotions and expectations of market participants as well as
monopolies might introduce random and irrational bias.
%http://en.wikipedia.org/wiki/Behavioral_economics
The same illusory, enthusiastic expectations which push prices up to record levels
drag prices down in times of fear and crises.
(This behavior is not dissimilar to somebody suffering from bipolar disorder.)

\item
Trade policies and military deployment might enforce prices.

\item
The volume creation and annihilation of money and debt by governments, (central) banks,
corporations and individuals might not allow a stabile settlement of prices by creating (expectations of) a chaotic regime.

\item
As money and its various forms and derivatives
is itself marketed, the price of money becomes recursive, self-referential and reflexive; with all consequences known from
classical recursion theory \cite{rogers1,odi:89}; in particular diagonalization.
\end{enumerate}

Traditional economics also seems to overemphasize the rationality and loyalty of market participants
and their willingness to Gentlemen-like ``stick to the rules.''
Indeed, experience demonstrates the contrary:
investment money is anonymous and stripped from the identity of the person or institution investing.
Profits and the increase of wealth are almost inevitable the sole measure of financial conduct.
Hence, those participants winning over alternative modes of conduct by circumventing and
breaking the rules will be the heroes of the markets.
The trading floor of every stock or commodity exchange resembles more a war room then an {\it agora};
this reflects the ongoing {\it ``bellum omnium contra omnes.''}
So, whenever there is an opportunity and a loophole available, there eventually will be someone exploiting it.
Only in times of crisis and bust their investors blame their investment bankers for taking too much risk and loosing,
whereas in times of bubble the latter ones are hailed the prophets of the day.


Another apparent tendency of economics (e.g., Refs.~\cite{1948-Samuelson,Begg}) is its sometimes apologetic use of formalism.
This is occasionally counterbalanced by uninspired attempts to apply techniques known from theoretical physics
such as the Feynman path integral (e.g., Ref.~\cite{Baaquie}) to the markets.
In what follows,
we shall investigate the consequences of certain dynamical economic behaviors
in a rather straightforward manner in the analytic tradition of
Hume~\cite{Hume-1742},
Wicksell~\cite{Wicksell-geld,Wicksell-1907},
Schneider~\cite{schneider-VWLIII},
Friedman~\cite{Friedman-2008}
and Binswanger~\cite{2006-Binswanger} which presents
 a form of ``dynamic accountancy'' and the quantity theory of money.

\section{Money}

Money is a unit or {\em measure of economic values} and thus of price,
a {\em unit of account}, as well as a {\em store of value}~\cite{1948-Samuelson,Begg}.
In a transaction, money serves as a {\em medium of exchange} and trade.
Money is also a unit or {\em measure of dept} used in a settlement of a debt~\cite{Wicksell-geld,schneider-VWLIII}
requiring ownership property which could serve as a collateral~\cite{2002-Heinsohn-Steiger}.

Money is created and destroyed in several ways.
In what follows, we shall consider
a bounded market domain such as a country, e.g., the United States of America,
or a collection of countries with a common market policy, such as the Euro-zone within the European Union.
We shall use the following market participants:
\begin{enumerate}
\item
The {\em Central Bank} is capable of creating central bank money in the associated currency,
such as for instance ``cash'' (i.e., central bank notes  or coins), or central bank accounts.
\item
The (commercial) {\em banks} are capable of creating bank money such as a transactional checking account.
Banks earn money by lending money and receiving interest.
\item
The {\em creditors and investors} (previously \cite{Wicksell-geld,Wicksell-1907} called {\em capitalists})
present capital or other assets.  In return they earn interest.
\item
The  {\em recipients} or {\em debtors} borrow money in the expectation of future profits. In return they pay interest.
\end{enumerate}

\subsection{Primary money creation by banks}

It is not totally unreasonable to state
that modern central and commercial banks are ``clearing houses''
transforming ``thin air'' or assets into money (and back).
Just how this is done will be reviewed below~\cite{schneider-VWLIII}.
As the ratio between the amounts of money created
by the commercial banks as compared to the amount of money
created by the central bank tends to increase in time, the commercial banks, and to a much lesser degree the central banks
are responsible for the amount of money created.
Let us consider a few cases how this comes about.


\subsubsection{Monetarization}

One form of money creation by a (central) bank is by monetarization of non-monetary assets.
Thereby, the (central) bank acquires (and then owns) assets which do not serve as means of payments.
The banks pays for these assets by creating accounts which hold money secured by itself; i.e., by the (central) bank.

Monetarization represents a straightforward flow of money from someone selling the asset to the bank; i.e.,
$$
\mbox{(seller \& holder of asset)}
\rightarrow
\mbox{[bank]}
\rightarrow
\mbox{(seller \& recipient of money)}
.
$$
In this process, the bank acquires both asset and dept (balanced by the ownership of that asset).
Examples of monetarization are the acquisition of
(i) real estate, (ii) shares in a business (iii) claims of (future) taxes and (iv) foreign money.


\subsubsection{Checks and ``I Owe You''~'s}

In a very similar way banks create money in transactional checking accounts by buying from somebody owning
checks or I Owe You's (IOUs) drawn from a third party.  In this case, the bank then holds the title.
This can be schematically represented by
$$
\mbox{(issuer of check)}
\rightarrow
\mbox{(seller \& recipient of check)}
\rightarrow
\mbox{[bank]}
\rightarrow
\mbox{(seller \& recipient of money)}
.
$$

\subsubsection{Money creation via credit and debt}

The bank grants money to an entrepreneur or investor for acquiring some asset.
In this process, the bank serves as an intermediary between the investor who is the recipient of money (in the form of a bank account).
At the same time, the investor becomes debtor; i.e.,
$$
\mbox{(investor)}
\rightarrow
\mbox{[bank]}
\rightarrow
\mbox{(investor \& debtor \& recipient of credit money)}
.
$$
The bank receives (future) securities (such as mortgages and other rights)
on the properties of the investor (borrower) and at the same time transfers credit money to the investor.
For example, credit money is lent to someone (i) buying real estate, or (ii) investing  into a business, or
(iii) a client or state preparing and conducting a (legal) war in hope of (future) taxes and bait.


For investments, interest rates always compete with the rate of profit of the investment:
the lower the interest rate, the more it may be worthwhile to pursue and maintain also low-profit business.

The price of the credit is the interest,
which at the same time is the gain and award of the bank for taking the risk of lending out money.
Indeed, the interest paid on  credit is a {\em major source of income} of a commercial bank.
Thus the creation of credit  money presents a business opportunity for commercial banks:
banks love to create credit, despite the fact that in times of crises they suffer from bad loans and default of payment.
Conversely, they accept money as deposits only insofar as this money contributes to their reserve,
thereby enlarging their capacity to generate credit.
In order to generate more income, banks are ingenious in creating an ever increasing amount of credit,
thereby increasing their income proportionately.

Indeed, the generation of credit money without sound bases has been the origin of several monetary and economic crises.
In the Great Depression, credit was used as a speculative investment into stocks,
and the lending of credit to borrowers deemed ``subprime'' initiated the recent ``credit crunch.''


\subsection{Money annihilation}

In the presented regime, money is annihilated essentially in reverse order as it is created.
This is expressible by inverting the arrows in the previous schema; i.e., by (i) the bank selling off assets and
(ii) by repayment of debt to the banks.

Any money contraction process is painful for the market because money is withdrawn and extracted from it,
resulting in a loss of liquidity.
Therefore, as absurdly as ist may sound, it might be better for market stability and growth that this shock never happens;
that is, if the principal of credit is never paid back but is constantly refinanced.
This, effectively, is the argument of some economists claiming that the perpetual maintenance of the United States trade deficit
is essential and beneficial for her trade partners and the rest of the world.

\subsection{Time scales}

In the absurd limit, one piece of money would be sufficient for all transactional purposes.
The bigger the economy and the market, the faster would have to be the circulation of this piece of money.
Thus, at least to a certain extent, the scarcity of money could be compensated by the rapidity of its turnaround.
In the real economy, turnaround cycles might be limited by natural processes.
These limitations are less important for virtual money in virtual markets.


\subsection{Money supply multiplication}

In principle,  commercial banks should be interested to grant as many loans as possible,
since the more loans are issued, the more interest (and provisions) is paid to them.
These interests and provisions contribute to the profits of the bank.
An unlimited possibility of banks to create credit money via debt would thus flood the markets with credit money.
This credit money would then compete on the market for the scarce assets
against money from other sources such as salaries or investments or profits from businesses.

Hence, money created by banks via dept travels in an ever increasing shock wave forward in time.
This may cause inflation and reallocations of money to some sectors (such as real estate)
favored by bank money creation.
On the positive side, this money may create incentives for investment and growth.


The money creation by banks is no linear process when it is iterated and occurs recursively.
Consider, for the sake of demonstration, somebody selling off a section of land to some buyer.
This seller is left with money which he could reinvest, for instance by buying a much bigger section of land
as previously owned, with the money earned by selling the previous section, plus the mortgage money (created by
the bank) on the new section.
The seller of this latter section may do the same, and so on, in principle {\it ad infinitum.}
This may amounts in an explosive growth of money from a finite seed.

Or, somebody owning money could go to a bank and put the money into an account.
The bank could then use some part or all of the money obtained as a {\em reserve} against sudden withdrawal
and issue new bank money --- exceeding the original amount of money --- in terms of credit against debt.
These new loans cound then  be reinvested and fed back into the banking system, which would treat this new money
in essentially the same way as reserve and issue new bank money in terms of credit against debt.
This process could go on forever, resulting in a limited increase of money supply at best
(if the ratio between the principal of the credit and the money put into the bank is strictly smaller than one)
and into an unlimited avalanche or shock of money causing hyperinflation at worst.
Schematically, one may write
$$
\mbox{(investor)}
\rightarrow
\mbox{[bank]}
\rightarrow
\mbox{(investor \& debtor \& recipient of credit money)}\rightarrow
\mbox{[bank]}
\rightarrow  \cdots
$$

The individual act of every such money multiplication can be formalized by the {\em money multiplication ratio}
defined by the ratio between the bank money created via credit and the money deposited in the bank; i.e.,
$$
\mbox{(money multiplication ratio)}
=
{
\mbox{(bank money created via credit for debtor)}
\over
\mbox{(money deposited by investor)}
}
.
$$
The amount of money created in this way is the sum of a geometric progression,
with the money multiplication ratio as common factor.
The behavior of this geometric series in terms of iterations is discussed in Appendix~\ref{Geometric progression and series}.

A somewhat similar situation occurs through refinancing: as long as the value or potential price
of an asset goes up (and some rating agency is willing to certify this), a debtor may
pay the interest with new debt {\it ad infinitum.}

\subsubsection{Money in the ``golden days''}

It is quite natural to ask
whether and how it is possible in this scenario to curb this lending activities by the banks
and thus limit the money supply from money creation processes.
In the ``golden days'' when money was tied to gold, the ``goldsmiths'' used to issue
anonymous notes based on their gold reserves,
and the amount of credit was limited by the reserves in gold they possessed or were trusted with.

In the next step towards ``fiat money,'' in order to increase their profits (via interest) from lending ``gold'' in terms of ``gold-backed notes,''
the ``goldsmiths'' issued and distributed more gold-backed notes than they actually had gold in store,
relying on the fact that people would not redeem all the notes in gold
(at least not suddenly and at once)\footnote{A class-action lawsuit, filed in August 2005, alleged that Morgan Stanley, a global financial services provider headquartered in New York City,
told clients it was selling them precious metals that they would own in full and that the company would store; cf. URL
http://www.reuters.com/article/fundsFundsNews/idUSN1228014520070612?sp=true.}.
Of course, they had to set aside some ``cash gold'' for people redeeming their notes.
Statistically, the laws of probability require that,
in the case of a ``large'' number of customers (in ``normal'' times of no bank runs),
the variations (variance) of the {\em aggregate holdings} set aside as reserves for instant withdrawal of these customers
would only vary with the {\em square root} of this ``large'' number times  the variation
of an {\em individual holding}~\cite[p.~67]{Wicksell-geld-engl}; i.e.,
$$
\mbox{(aggregate reserves)}
= \sqrt{\mbox{(number of customers)}}
\times
\mbox{(individual reserves)}
.
$$
 Thus the {\em relative} ``cash'' holdings of this bank (with respect to all ``cash'' deposits) needs
not increase linearly but only with the square root of all deposits. For large number of customers and deposits,
this relative rate goes to zero.

Nevertheless, if the banks are not cheating and if they prepare for the worst (of all customers wanting their gold ``cash'' back),
they can only lend out as much money as there is gold.
Thus the amount of money is limited by the amount of gold.
This presents several drawbacks for an expanding economy:
\begin{enumerate}
\item
In a ``golden'' scenario,
the amount of available money would depend on the volume of gold, which is strongly dependent on factors which are external to the economy,
among them technology, the chance to find new gold deposits, the industrial usage of gold, and the occupation of gold-rich civilizations.
This is hardly useful for a dynamic economy in need of liquidity not limited to the amount of some precious commodity.


\item
The resulting volatility in price --- the ups and downs on the market ---
would seriously hamper the realization of some preliminary price frame, the expectations of future income, and ultimately trade and investment activities.


\item
If the production of goods and services, as for instance counted in the
accumulated {\em gross domestic product}
exceeds the amount of gold --- as has been the case in the
late nineteenth century --- there will be deflation, since there will be more and more goods per unit of money.
In the limit of higher and higher production of goods relative to the available gold reserves, money becomes infinitely valuable,
which means that, effectively, nobody can afford it.

\item
International conflicts such as wars and other excessively expensive investment activities will be strongly penalized if not made impossible
for countries maintaining the gold standard (thus favoring those countries abandoning it),
since the amount of money and thus the ability to finance a conflict or an investment is strictly limited
by the gold reserves that happen to be available.
\end{enumerate}

\subsubsection{Fiat money}

To circumvent the volume limits of the gold standard,
influential groups, financial authorities and governments have abandoned the gold standard
in favor of {\em fiat money} which is
not directly (without any market and pricing) exchanged by any precious commodity such as gold.
Alternatively, it is
backed by some ``trustworthy'' authority; in most cases the central bank or the government itself.
A piece of money is fiat money if its nominal or ``face value''  (e.g., the value it has in payment of taxes)
is higher than its market value as a commodity.

In the times of the gold standard, one just had to make sure that all
creators of gold-backed money really had a sufficient backing of gold at their disposal;
otherwise they would have produced counterfeits flooding the market with unbacked notes.
Indeed, historically the printing of unbacked notes by central banks and governments --- a legal quasi--counterfeit --- has been the reason
for the total abandonment of the gold standard.
From 1944 to 1968, the Bretton Woods Agreements still guaranteed the U.S. dollar as a  gold standard based
reserve currency at a fixed price of 35 dollar to an ounce of gold.
(Due to World War II, the European countries were highly in debt and transferred large amounts of gold into the United States.)
In 1967 and 1968  there was a run on gold in the British ``Pound Sterling area'' and the USA,
and the fixed redemption of gold to the dollar had to be given up in favor of liquidity and US federal deficits.
Afterwards, the U.S. dollar celebrated its comeback as the only currency in which crude oil, the ``black gold'' has to be traded (although not at a fixed price).


In times of fiat money not directly (without market) redeemable in gold or other precious commodities, there has to be another
way to curb inflation and thus the money supply.
Note that, with fiat money, the emphasis is on the {\em authority} to create money.
Because in principle, without authority anybody could take a piece of paper and
create a ``money note'' by writing an amount of currency on it.

\subsection{Money increase through interest paid}

The increase in the monetary base through credit makes necessary not only the generation of credit money by the banks
but also the creation of money ``compensating'' the interest paid for these credit,
because at the end of the credit cycle, when a credit is paid back,
the debtor has not only paid back the principal, but also the (compound) interest.
Thus  in order to avoid deflation, the aggregate money supply must at least grow as strong as the aggregate interest; i.e.,
$$
\mbox{(aggregate money supply)}
\ge
\mbox{(aggregate interest)}
.
$$
From exactly where this ``new'' money should come from --- despite new credit to somebody else buying these products ---
remains unclear; one possibility would be the creation of new money by the (central) banks via a reassessment and revaluation of assets in possession.
The latter possibility would require inflation.


\subsection{Limits on and control over money creation}

In principle, there is no limit on the creation of fiat money by banks.
Thus, if price stability is the goal, in order to prevent the effect of too much competing money for the marketed goods (resulting in
inflation), the rate of money creation and destruction should be fine-tuned to the growth and decline of the marketed assets.
If such a delicate balance~\cite{2006-Binswanger}
between  the available assets and services (e.g., measured in terms of of the gross domestic product)
on the one hand and the available money on the other hand
can be controlled and maintained over long periods of time
by the standard measures of central banks (influencing the credit activity and the multiplicity of money supply via the interest rates)
remains highly questionable.
The Author tends to agree with Soros' rather pessimistic views \cite{soros-alchemy} that occasional bust phases are inevitable.

\subsubsection{Reserve policies}
One of the methods invented for issuing ``too much'' fiat money is to require the commercial banks to keep a certain ratio of the
generated credit money as central bank capital reserve.
Also, it is claimed that the money multiplication ratio is maintained strictly smaller that one, thus preventing
explosive growth of the money supply.

This method has proven to be notoriously difficult to establish and control, as commercial banks and
investment firms, in collaboration with rating agencies and accountants, have invented various ingenious ways
to shift and hide credit and debt from, or ``sideways around,'' their books.
One possibility which has been widely discussed, practiced and criticized lately is the packaging and bundling of debt
into (mortgage) funds.

\subsubsection{Dependence between interest rate, volume of money and price}


Another, seemingly much more efficient, though paradoxical,
instrument of money volume control is the fixing of interest rates by the central bank,
which is then translated into adjusting interest rates in the commercial bank sector.
There are two conflicting views about the connection between interest rates and the resulting volume of money
resulting in inflation:

\subsubsection*{Indirect dependence}
In the first view, the higher the interest, the less business projects appear to be profitable enough to support that interest.
Also, the consumption of commodities will decline.
As a result, only those projects will remain whose expected profitability is higher than the interest rate;
all others will be either not started or be discontinued.
This curbs the amount of credit money by reducing the amount of investments.

The unfavorable side effect of this measure is the resulting contraction of the economy,
because economic growth is linked to investment and credit.
As less and less money is available and more and more consumers have to curb their spending
(for instance by getting unemployed and living on some kind of ``social welfare''),
a downward spiral of activity may result, which is the reverse of the upward spiral  \cite{2006-Binswanger} of growth.

The observation that {\em increasing} interest rates could result in a contraction of (credit) money and thus
in a {\em decrease} of prices and inflation was already observed around 1900~\cite{Wicksell-1907},
despite some empirical evidence to the contrary.

\subsubsection*{Direct dependence}
In the second view, one may argue that, confronted with high interest rates,
a business has to surcharge its customers with a higher product price resulting
from the higher interest paid for financing and maintaining the production.
In a very similar way, increasing cost factors such as increasing commodity (i.e., energy) prices contribute to inflation,
because investment capital is just one cost factor among several.
This contributes to inflation and has and adverse effect to the deflatory tendency just discussed.

The Author knows of no recent systematic empirical study investigating which one of the effect prevails under certain circumstances.\footnote{
In observing  economic arguments, one is sometimes reminded of an apocryphal story about the Vienna-born MIT-Physicist Victor Weisskopf.
One day a student entered his office, telling Weisskopf that he had just been to Vienna and rided an old tram there. On the ceiling of the tram the
student identified
two adjacent ventilators running in {\em opposite} directions. Weisskopf thought for a while and explained the student why this has to be so.
A year later the same student entered Weisskopf's office again, declaring that he has visited Vienna a second time, and that,
on closer inspection, had found out that
the ventilators on the tram ceiling circulated in the {\em same} direction. Upon hearing this, Weisskopf allegedly jumped up and cried out loudly:
``oh, but that can be explained even more easily!''}
More empirical data seem to be needed in order to be able to state the effects
of changes of the interest rates.
A first glance at the inflation rate and interest rate data for the
U.S. economy depicted in Fig.~\ref{2008-mm-in_inf} shows some correlations.
(Note that these correlations might also be established by the believe of the market participants in the capability of the
Federal Reserve System to be able to stabilize markets and prices via variations of interest rates.)
\begin{figure}
\begin{center}
 \includegraphics[width=120mm]{2008-mm-in_inf}
\end{center}
\caption{The rates of interest (Federal funds overnight) and annual inflation (seasonally adjusted Consumer Price Index)
over a period of 50 years in the United States of America. Source:  {\em Federal Reserve Statistical Release, H.15, Selected Interest Rates}
and
{\it U.S. Department of Labor: Bureau of Labor Statistics}.}
\label{2008-mm-in_inf}
\end{figure}

Why, for instance, it should be in the powers of the European Central Bank
to stop the strongly increasing oil  and
and other commodity costs (originating from higher demand from China, India and elsewhere)
 by raising the interest rate for the Euro remains a mystery to the Author.
Of course, if the European Central Bank would effectively ruin the European economy by imposing too high prices for interest
(money investments), then the demand for oil and other commodities would drop, which would very likely result in
lower prices of these goods temorarily.
Surely one should not presume that the European Central Bank would sacrifice the well-doing of the
European economy for price stability in the rest of the world!
Even granted that being correct, the collaterals of these measures in terms of a lowering in economic growth or even recession, unemployment, poverty
and the related human misery and social instability are so severe that they should be administered with extreme care.


%\section{Dilemma between regulation and growth}


%\section{Recursivity of the creation of money and dept}




\section{Price}

Since the days of Adam Smith \cite{1776-smith-wealthofnations},
Economics \cite{1948-Samuelson,Begg} defines {\em price} in a market as the value of a good in terms of money;
that is, {\em money} (per asset) is the unit of price.
Stated differently, in the economic context the informal notion of {\em value} is formalized by a concrete,
definite fixation of price,
which thus reflects the beliefs and behaviors of all market participants.

Ideally, the price (as well as the appropriated quantities of variable supply)
is settled in a {\em market} subject to {\em supply} and {\em demand}
by an equilibrium of the latter two~\cite{1874-walras,1890-marshall}.
We have already discussed some of the factors contributing to the fifficulties in establishing a unique
equilibrium on the market resulting in price volatility.


\subsection{Price formation by interest rates}
For scarce assets which are considered important and central in life,
potential sellers tend to bet on ever increasing prices of their assets.
Potential buyers tend to go to their ``financial limits'' (and sometimes beyond) to acquire these assets.
Let us consider housing as an example for scarce and highly demanded resources.
The high demand for real estate properties reflects the particular importance and the relevance  of proper accommodation to individuals and families.
The price of a property
is not directly related to any potential future income
but seems to be solely determined by the portion of the household income available
for the payment of dept accepted for acquiring that property; i.e.,
$$
\mbox{(price of property)}
\le
{
\mbox{(available houshold income)}
\over
\mbox{(interest rate)}
}       .
$$

As a result, property prices tend to increase on decreasing interest rates.
The leverage or ratio of this price increase is determined by the inverse interest rate.
In the (absurd) limit, with ``free credit'' associated with zero interest rate,
a single buyer would be able to bid an unlimited price for any given property.
By unrealistically assuming those prices will not go up due to competing money,
the buyer could acquire all properties available on the market.


\subsection{Role of rating agencies and assessors}
This modes of money creation require that the value; i.e., the price, of the asset is determined.
The role of rating agencies, reviewers and assessors in this respect is of great importance.
Corruption, e.g., direct or indirect influence
and manipulation by the customers (and payers\footnote{In German,
the following saying exist: {\em ``$\ldots$ wessen Brot du isst, dessen Lied du singst.''
(``$\ldots$ you sing the song of those who give you bread.'')}}),
 bad expertise or just laissez-faire (let do) whatever the market seems
to dictate is detrimental.
Nevertheless, it has to be accepted that assessors and rating agencies cannot evaluate ``against
the market for a very long time,'' simply because of the fact that they
would ``rate themselves out of business.''
Ultimately, only those rating agencies will be successful which
conform to the expectations of their customers.


\subsection{Pyramid schemes and vicious cycles}
A pyramid scheme is a non-sustainable business model which assumes ever increasing prices (money supply)
or a constant multiplicity of participation.
One formalization of such schemes is a geometric progression with a common ratio of strictly greater than one.
Although no market participant might intend such a pursuit, the boom phases on stock or real estate markets
tend to behave not in a dissimilar manner. In such a scenario, the last investor buying the shares or property before the market slump is the one suffering most.


Another behavior is the (vicious) cycle  \cite{soros-alchemy,2006-Binswanger} in which an economy spirals up
and down.
It is extremely difficult to break a {\em psychological barrier} originating in negative expectations of investors for future demand and profit.
In times of crises, it appears prudent for an individual to take not too many risks and not to invest too much; but common prudency
sometimes amounts to a general market downturn and in a decline of the economy, or in a quasi-stable state of low employment and living standard.


\subsection{Price and trade volumes}
Under ``normal conditions,'' in the real estate and other financial, stock and commodity markets,
prices are formed under the assumption of the market participants that at any given time only a tiny fraction of commodities
is for sale --- like the tip of an iceberg.
If for instance all (or at least large parts of) the world's gold reserves,
or all the properties in Europe, or all the stocks traded in the New York stock exchange
would become available for sale at once, prices would plunge dramatically, and would presumably tend to zero,
regardless of the soundness of such investments.

Thus it makes no operational sense to contemplate on the ``price of all the European land,''
or all the ``available gold,'' or ``all the stocks traded,'' or the ``price in Euro of all New Zealand Dollars.''
Likewise, the absolute value of a company cannot be directly related to the number of its stocks times the price per stock.
Rather, they reflect a price of the share on the condition of the available volumes of supply and demand.
Nonetheless it makes operational sense to speak of the value of some assets in any given finite portfolio which
is small enough not to effect prices by selling it off.


\subsection{Not all money contributes to price formation}

As long as some money is ``locked'' into some investment,
it does not compete on the market with  money from other sources and therefore
is irrelevant for price formation.
Already John Stuart Mill \cite[p.~589]{mill-1844} stated that,
{\em ``The issues of a Government paper, even when not permanent, will raise prices;
because Governments usually issue their paper in purchases for consumption.
If issued to pay off a portion of the [[already existing]] national debt, we believe they would have no effect.''}


For the sake of demonstration, let us assume the seller of a real estate property uses his money
(i)
to buy a new property;
(ii)
to buy stocks in a stock exchange;
(iii)
to put the savings aside in a bank account for later use;
(iv)
to buy a boat;
(v) or
gives the money away to charity to buy food for the poor.
Only in the latter case (v) the money obtained from the sale of a real estate property will compete with other money to buy food.
In case (i) it is re-invested into the real estate sector of the economy.
In case (ii) it is re-invested into the shareholding sector of the economy.
In case (iii) it is re-invested into the banking sector.
In case (iv) it is used with other money in the boat market.

To give another example, suppose a person has just made a huge profit on the money market.
As long as this person re-invests this profit into stocks, the price of,
say, real estate in New Zealand's Northland will remain unaffected. However, if the same person
phantasizes to settle in Northland and starts buying up land close to, say, Whale Bay in Matapouri, prices will go up there.

In general, while it is true that money can be converted into any form of marketed commodity,
its effect in the market and thus its relevance with respect to price formation depends on
the particular economic segment in which it is invested.
Therefore it makes no operational sense to state that ``suchandsuch money contributes to general inflation.''
Because if, for instance, it remains within a narrow sector of the economy, it could have zero effect on other sectors.
It is only when these sectors communicate --- by transforming value in terms of a flow of money from one sector into another ---
that one sector will influence another.

In principle, every influx of money affects the price in the sector invested.
Some segments of an economy may thus ``lift off'' or get ``dephased'' or ``withdrawn'' from other
sectors; in particular from ``Main street,'' where people sell their work and their life's time for money.
On a much larger scale, this issue comes up in the pricing of certain raw materials and commodities as compared to
the production of goods, as well as in regional pricing.

Friedmann \cite[Section~1(f)]{Friedman-2008} emphasizes the importance of how much preliminary locked money
will ultimately be released and shows up on the markets, creating inflation there.
Empirically, not much seems to be known about the time behavior of these money flows, containment and  release into different sectors.\footnote{
Theoretically, it could be expected that, by pursuing similar re-investment strategies,
second- or third- round effects are just iterations of the first round of investments.
Thus,  the effective {\em (non-)consumption ratio} of the quantities of money that go into commodities and consumption,
versus money that goes into store or other non-consumption purposes in terms of the available aggregate of  money can be defined by
$$
\begin{array}{rcl}
r_c&=&\mbox{(consumption ratio)}
 =
{
\mbox{(money for consumption)}
\over
\mbox{(available aggregate of money)}
}
,\\
r_n&=&
\mbox{(non-consumption ratio)}
=
{
\mbox{(money for non-consumption)}
\over
\mbox{(available aggregate of money)}
}
.
\end{array}
$$
Note that $r_c+r_n=1$. In what follows, we shall abbreviate $A= \mbox{(available aggregate of money)}$.
Suppose the quantities of money in different sectors are grouped into a {\em vector}
defined in a linear vector space whose dimension is the number of such sectors in the model.
For the sake of demonstration, consider a model with only two sectors, namely a consumption sector
and a non-consumption sector; i.e.,
${\bf v}_0=A(r_c,r_n)$.
Then, after $i$ iterations, the original amount $A$ gets dispersed as
${\bf v}_i=A(r_c r_n^i,r_n^{i+1})
$,
which, in the infinite time limit (i.e., $i\rightarrow \infty$), for $r_n<1$, dies out as
${\bf v}_\infty = (0,0)$.
}


\section{Reallocation of assets via monetary measures}

The issue of holding the ``proper'' relative levels of pricing remains highly nontrivial.
If the price levels of various sectors in the (national and world) economies are not balanced,
this could give rise to economic, social and political unrest.
Exactly how such a balance should be defined seems very difficult to define.
An advantags for one market participant seems to be a disadvantage for another and {\em vice versa}.

The accumulation of attention and profit obeys Matthew principle (named from St Matthew,
with reference to Matthew 25:29): {\em `For unto every one that hath shall be given, and he shall have abundance:
but from him that hath not shall be taken away even that which he hath.'}
Let me mention a pathetic statement of Paul Dirac, the ``father of the electron equation,'' which on the occasion of an 1982 Erice meeting
on {\em ``the worldwide consequences of nuclear war''} stated that (from my personal recollection) it would be ruinous and unwise
to wage (nuclear) war on the issue of whether or not parents could pass on the acquired wealth to their children {\em versus} giving every child the opportunity
to make the best out of life, independent of the previous achievements by the parents.



\section{Necessities for growth and prosperity}

It has been argued by economists (e.g., Keynes \cite{keynes-GTEIM}) concerned with possible unfavorable quasi-stabile equilibria
% http://news-service.stanford.edu/pr/97/970910greenspan.html
characterized by high unemployment, declining profits and slow economic growth
that it is important to keep the money supply and the investment levels  high; sometimes even at the price of
excessive government deficit spending.
Others (e.g., Hayek \cite{Hayek-45}) argue that, in order to avoid (hyper-)inflation and the associated lack of control of the economy by the (central) banks,
these instruments should be applied with great care.

\subsection{Profit incentives}
Note that growth can only happen if some new assets or services are produced and presented which did not exist previously.
In order for that to happen, the manufacturers could either use money set aside for investments previously, or take credit money (and the associated debt).
For the latter, presumably more often occurring situation, the {\em expected profit rate} (per invested unit of money) should exceed the
{\em expected interest rate}; i.e.,
$$
\mbox{(expected interest rate)} < \mbox{(expectated profit rate)} .
$$
Otherwise, no capitalist would finance a money consuming business.
Thus a necessary condition for an economy to grow  is that
the rate of interest on money is kept sufficiently low to allow profits.


\subsection{Bootstrap theory of economic cycles}

There are at least two main arguments supporting the ``creation of new money:''
(i) the increase of assets and services available for money purchases, as well as (ii) the necessity to repay interest on the bank credits.
Why should an additional amount of money not produce a higher amount of available, competing money per asset or service, thereby causing inflation?

There exists a third, temporal factor in money increase which is associated with a {\em time lag}
between the consumption and investments made possible with the additionally created money, and its inflationary effect.
In the extreme, new money might be so efficiently invested that through these investments the volume of goods or services  (over-)compensate for
the additional money influx into the markets, thereby stabilizing the prices.
Already
Hume~\cite[II.III.7]{Hume-1742}, in the ``golden days,'' stated
{\em ``$\ldots$~that though the high price of commodities be a necessary consequence of the increase of gold and silver [[the money of the ``golden days'']],
yet it follows not immediately upon that increase; but some time is required before the money circulates through the whole state,
and makes its effect be felt on all ranks of people.
At first, no alteration is perceived; by degrees the price rises, first of one commodity, then of another;
till the whole at last reaches a just proportion with the new quantity of specie which is in the kingdom.
In my opinion, it is only in this interval or intermediate situation, between the acquisition of money and rise of prices,
that the increasing quantity of gold and silver is favorable to industry.''}



\section{What could be done}


With regards to the supply of money and its required volume,
apparently there seem to be two kinds of strategies:
(i) to supply the markets with liquidity and thus to ``print money,''
or
(ii) to attempt to contain the money volume by ``printing no money.''
This, of course, has very little to do with really printing cash notes, but mainly with encouraging,
allowing or penalizing --- through lower or higher interest rates --- the creation of money via credit and debt.
In terms of central banks, this translates into (i) ``to lower the interest rate'' and ``lower the criteria of eligible security and reserves'' for loans versus
(ii) ``to keep interest rates `high''' and to ``maintain strict minimal reserve standards.''

As it turns out, monetary policy makers at the Federal Reserve have been favoring the high liquidity strategy (i),
arguing that the containment strategy (ii) has build up or even caused the Great Depression of the 1920/30's.
From an economics point of view, it is exciting to observe the effects of this conduct;
in particular whether or not all this liquidity will inflate the markets in the long run, thereby causing extraordinary price rises.

Alas, it might not be unfair to state that --- beyond some very localized effects on the economy --- central banks seem to have only limited influence
on the money supply and on pricing.
In times of bang and bust, market participants don't care very much what the central bankers say or do.
Despite advises to behave prudently, they try (even by borrowing investment money) to make as much profit as possible; and they try to save what is left in declining price rallies.
Overall, they expect the central banks to behave according to their expectations.\footnote{
In this respect, one central banker told the Author off records his impression that ``the tail wags the dog.''}

There is only one certainty:  a central bank could easily ruin the economy by fixing the interest at either too high or too levels,
thereby creating a liquidity crisis, either resulting in a money volume collapse, or in (hyper-)inflation, respectively.
Indeed, Friedman suggested  to ``abolish the Federal Reserve and replace it with a computer,'' effectively
replacing central banks with a mechanical system that would keep the quantity of money going up at a steady rate;
e.g.,  proportional to the gross domestic product~\cite{Friedman-2008}.
% http://www.youtube.com/watch?v=9V5OP-VmXgE
% http://www.econlib.org/library/Enc/bios/Friedman.html


The markets' behavior seem to be strongly determined and biased by the {\em expectations} of the participants.
Almost everything (good or bad) seems to be possible if
a sufficient number of market participants believe in it for a sufficiently long time.
Therefore, it might not be unjustified to state that probably the most
important criterion for a prospering market and economy is the creation and maintenance  of a {\em belief} in prosperity and growth.
Thus sometimes marketing is more important than monetary measures.


This leaves individuals and policy makers with a very uneasy feeling towards economic issues; probably the most disturbing being the lack of control over the markets.
Whatever measures are taken --- presently, the most important is the setting of central bank interest rates,
as well as setting the standards for reserves and securities backing credits --- their effects on the economy remain obscure and
very difficult to predict.
Thus, from the point of view of the natural sciences, economics still requires a lot of attention, both theoretically as well as empirically.





\appendix

\section{Geometric progression and series}
\label{Geometric progression and series}


A {\em geometric progression (geometric sequence)}
is a sequence of numbers
where each term after the first $a_0$ is  the previous term
times a fixed non-zero number called the {\em (common) ratio} $r$; i.e., $a_{k+1}= a_k r$ and thus
$$a_0, a_1, a_2,  \ldots ,a_k, \ldots ,a_n= a_0, a_0r, a_0r^2, \ldots ,a_0r^k, \ldots ,a_0r^n\quad .$$

An example of a geometric expression is the {\em compound interest} under the assumption of a constant interest rate $p$.
If $a_0$ is the principal, then $a_0(1+p)^k$ is the amount due after $k$ years.
The growth of this function for various values of $p$  over time $n$ is depicted in Fig.~\ref{2008-mm-zz}.
\begin{figure}
\begin{center}
 \includegraphics[width=120mm]{2008-mm-zz}
\end{center}
\caption{The time (in units $n$) behavior of the compound interest.}
\label{2008-mm-zz}
\end{figure}

A {\em geometric series} is the sum of all the numbers in a geometric progression; i.e.,
$$\sum_{k=0}^n a_k = a_0 \sum_{k=0}^n r^k=a_0(1+r+r^2+ \cdots +r^k+ \cdots +r^n)\quad .$$
By multiplying all sides with $1-r$,
the sum can be rewritten as
$$(1-r) \sum_{k=0}^n r^k=1+ r+r^2+ \cdots +r^k+ \cdots +r^n - (r+r^2+ \cdots +r^k+ \cdots +r^n +r^{n+1}) = 1-r^{n+1}$$
and, since the middle terms cancel out,
$$\sum_{k=0}^n a_k =a_0 {1-r^{n+1}\over 1-r}  \quad \mathrm{ and }\quad  \sum_{k=1}^n a_k =a_0 r{1-r^{n}\over 1-r}\quad .$$
Depending on the value of the common ration, there are seven regimes (depicted in Fig.~\ref{2008-mm-gs}),
such that in the limit of ``large'' $k \rightarrow \infty $,
\renewcommand{\labelenumi}{(\roman{enumi})}
\begin{enumerate}
\item
for $r < -1$,
$r^{n}\rightarrow \pm \infty$, and thus the sum diverges, with ever increasing zigzag values.

\item
for $r = -1$,
$r^{n}\rightarrow (-1)^n$, and thus the sum diverges in a constant zigzag movement between $1$ and $0$.

\item
for $-1 < r < 0$,
$r^{n}\rightarrow 0$, and thus, in an ever diminishing zigzag movement, the sum converges
$$\sum_{k=0}^\infty a_k = a_0 {1\over 1-r} \quad \mathrm{ and }\quad   \sum_{k=1}^\infty a_k = a_0 {r\over 1-r}\quad .$$

\item
for $r = 0$,
the sum remains constant, the only term contributing remaining (per definition) $0^0\in \{0,1\}$.

\item
for $0 < r < 1$,
$r^{n}\rightarrow 0$ and thus
$$\sum_{k=0}^\infty a_k = a_0 {1\over 1-r} \quad \mathrm{ and }\quad   \sum_{k=1}^\infty a_k = a_0 {r\over 1-r}\quad .$$

\item
for $r = 1$,
$r^{n}= 1$ and thus there is a constant slope or increase; i.e.;
$$\sum_{k=0}^n a_k = a_0 + \sum_{k=1}^n a_k = a_0 r (1+n) \rightarrow \infty .$$

\item
for $r > 1$ ,
$r^{n}\rightarrow \infty$ and thus
$$\sum_{k=0}^\infty a_k = a_0 +\sum_{k=1}^\infty a_k \rightarrow \infty \quad .$$

\end{enumerate}

\begin{figure}
\begin{center}
 \includegraphics[width=120mm]{2008-mm-gs}
\end{center}
\caption{The time (in units $n$)  behavior of a geometric series measured in units of $n$ depends on the value of the common ratio $r$.}
\label{2008-mm-gs}
\end{figure}





%\bibliography{svozil}
%\bibliographystyle{apalike}
%\bibliographystyle{named}
%\bibliographystyle{osa}


\begin{thebibliography}{10}
\newcommand{\enquote}[1]{``#1''}
\expandafter\ifx\csname url\endcsname\relax
  \def\url#1{{#1}}\fi
\expandafter\ifx\csname urlprefix\endcsname\relax\def\urlprefix{}\fi

\bibitem{frank}
P.~Frank, {\em Das Kausalgesetz und seine Grenzen\/} (Springer, Vienna, 1932),
  english translation in Ref.~\cite{franke}.

\bibitem{Wicksell-geld}
J.~G.~K. Wicksell, {\em {G}eldzins und {G}{\"{u}}terpreise: {E}ine {S}tudie
  {\"{u}}ber die den {T}auschwert des {G}eldes bestimmenden {U}rsachen\/} (G.
  Fischer, Jena, 1898), {E}nglish translation in \cite{Wicksell-geld-engl}.

\bibitem{Wicksell-1907}
J.~G.~K. Wicksell, \enquote{The Influence of the Rate of Interest on Prices,}
  Economic Journal {\bf XVII}, 213--220 (1907), read before the Economic
  Section of the British Association, 1906.

\bibitem{schneider-VWLIII}
E.~Schneider, {\em {E}inf{\"{u}}hrung in die {W}irtschaftstheorie III. {G}eld,
  {K}redit, {V}olkseinkommen und {B}esch{\"{a}}ftigung\/} (J.C.B.Mohr (Paul
  Siebeck), T{\"{u}}bingen, 1952-1973).

\bibitem{soros-alchemy}
G.~Soros, {\em The Alchemy of Finance: Reading the Mind of the Market\/} (John
  Wiley \& Sons, Hoboken, New Jersey, 1987, 1994, 2003).

\bibitem{2006-Binswanger}
H.~C. Binswanger, {\em {D}ie {W}achstumsspirale\/} (Metropolis-Verlag, Marburg,
  2006), second edn.

\bibitem{Hayek-45}
F.~Hayek, \enquote{The Use of Knowledge in Society,} American Economic Review
  {\bf XXXV}, 519--530 (1945).
\newline http://www.virtualschool.edu/mon/Economics/HayekUseOfKnowledge.html

\bibitem{rogers1}
H.~{Rogers, Jr.}, {\em Theory of Recursive Functions and Effective
  Computability\/} (MacGraw-Hill, New York, 1967).

\bibitem{odi:89}
P.~Odifreddi, {\em Classical Recursion Theory, Vol. 1\/} (North-Holland,
  Amsterdam, 1989).

\bibitem{1948-Samuelson}
P.~A. Samuelson and W.~D. Nordhaus, {\em Economics\/} (McGraw-Hill, New York,
  NY, 1948-2004).

\bibitem{Begg}
D.~Begg, S.~Fischer, and R.~Dornbusch, {\em Economics (Eigth Edition)\/}
  (McGraw-Hill, London, 2005).

\bibitem{Baaquie}
B.~E. Baaquie, {\em Quantum Finance: Path Integrals and Hamiltonians for
  Options and Interest Rates\/} (Cambridge University Press, Cambridge, 2004).

\bibitem{Hume-1742}
D.~Hume, {\em Essays, Moral, Political, and Literary\/} (1742).
\newline http://www.econlib.org/library/LFBooks/Hume/hmMPL.html

\bibitem{Friedman-2008}
M.~Friedman, \enquote{Quantity Theory of Money,} in {\em The New Palgrave
  Dictionary of Economics. Second Edition\/}, S.~N. Durlauf and L.~E. Blume,
  eds.,   (2008).
\newline http://dx.doi.org/10.1057/9780230226203.1374

\bibitem{2002-Heinsohn-Steiger}
G.~Heinsohn and O.~Steiger, {\em {E}igentum, {Z}ins und {G}eld.
  {U}ngel{\"{o}}ste {R}{\"{a}}tsel der {W}irtschaftswissenschaft\/}
  (Metropolis-Verlag, Marburg, 2002).

\bibitem{Wicksell-geld-engl}
J.~G.~K. Wicksell, {\em Interest and prices. {A} study of the causes regulating
  the value of money\/} (Macmillan And Company Limited, London, 1936).
\newline http://www.archive.org/details/interestandprice033322mbp

\bibitem{1776-smith-wealthofnations}
A.~Smith, {\em An Inquiry into the Nature and Causes of the Wealth of
  Nations\/} (1776).
\newline http://www.econlib.org/library/Smith/smWN.html

\bibitem{1874-walras}
L.~Walras, {\em El{\'{e}}ments d'{\'{e}}conomie pure ou th{\'{e}}orie de la
  richesse sociale\/} (1874,1877).

\bibitem{1890-marshall}
A.~Marshall, {\em Principles of Economics\/} (1890).
\newline http://www.econlib.org/library/Marshall/marP29.html

\bibitem{mill-1844}
J.~S. Mill, \enquote{Review of books by Thomas Tooke and R. Torrens,}
  Westminster Review {\bf 41} (1844).

\bibitem{keynes-GTEIM}
J.~M. Keynes, {\em The General Theory of Employment, Interest, and Money\/}
  (Palgrave Macmillan, London, 1936).
\newline http://ebooks.adelaide.edu.au/k/keynes/john\_maynard/k44g/

\bibitem{franke}
P.~Frank and R.~C. (Editor), {\em The Law of Causality and its Limits (Vienna
  Circle Collection)\/} (Springer, Vienna, 1997).

\end{thebibliography}

\end{document}



The status of the economy, and in particular the preferences, emphasis, appropriations and prices therein, is ``in the mind'' of market participants.
Ultimately, the market participants make true what they believe is true, or what they want to be true.



\section{Credit rates and price}

\section{Loan versus payment}


gp[r_, n_] := (1 - r^(n + 1))/(1 - r) // FullSimplify

Show[{ListPlot[{Table[gp[-1.5, n], {n, 0, 10}],
    Table[gp[-1, n], {n, 0, 10}], Table[gp[0.8, n], {n, 0, 10}],
    Table[1 + n, {n, 0, 10}], Table[gp[1.5, n], {n, 0, 4}],
    Table[gp[-0.8, n], {n, 0, 10}], Table[gp[0, n], {n, 0, 10}]},
   Joined -> True, Frame -> True,
   FrameLabel -> {Style[n, Bold, FontSize -> 34],
     Style[HoldForm[Row[{(1 - r^n), "/", (1 - r)}]], Bold,
      FontSize -> 34]}, AspectRatio -> 0.8,
   FrameTicksStyle -> Directive[Black, 24, Bold], Axes -> False,
   PlotRange -> {{1, 10}, {-1, 10}},(*FrameTicks->{{{1,1996},{5,
   2000},{11,2006}},{0,10,20,30,40,50,60,70},None,None},*)
   Joined -> True, Mesh -> All,
   PlotStyle -> {{Blue}, {Red, Dashed, Thick}, {Black,
      Thick}, {Orange, DotDashed, Thick}, {Green, Dotted,
      Thick}, {Brown, Thick}, {Gray, Dotted, Thick}}],
  Graphics[{Text[
     Style[HoldForm[r > 1], Large, Bold, Green], {3, 7.5}],
    Text[Style[HoldForm[r < -1], Large, Bold, Blue], {8.2, 9.5}],
    Text[Style[HoldForm[0 < r < 1], Large, Bold, Black], {5.6, 4}],
    Text[Style[HoldForm[r = -1], Large, Bold, Red], {7.95, -0.5}],
    Text[Style[HoldForm[r = 1], Large, Bold, Orange], {5.2, 6}],
    Text[Style[HoldForm[r = 0], Large, Bold, Gray], {3.9, 1.4}],
    Text[Style[HoldForm[-1 < r < 0], Large, Bold,
      Brown], {2.85, -0.3}]}]}]

Export["/mytex/2008-mm-gs.ps", Out[ ], "EPS", ImageSize -> 900]


(* compound interest *)
gpt[r_, n_] := (1 + r)^(n);

Show[{ListPlot[{Table[gpt[0.000001, n], {n, 0, 30}],
    Table[gpt[0.02, n], {n, 0, 30}], Table[gpt[0.04, n], {n, 0, 30}],
    Table[gpt[0.06, n], {n, 0, 30}], Table[gpt[0.08, n], {n, 0, 30}],
    Table[gpt[0.1, n], {n, 0, 30}], Table[gpt[0.12, n], {n, 0, 30}],
    Table[gpt[0.14, n], {n, 0, 30}]}, Joined -> True, Frame -> True,
   FrameLabel -> {Style[n, Bold, FontSize -> 34],
     Style[HoldForm[(1 + p)^n], Bold, FontSize -> 34]},
   AspectRatio -> 0.8, FrameTicksStyle -> Directive[Black, 24, Bold],
   Axes -> False, PlotRange -> {{1, 20}, {0.8, 5}},(*FrameTicks->{{{1,
   1996},{5,2000},{11,2006}},{0,10,20,30,40,50,60,70},None,None},*)
   Joined -> True, Mesh -> All, PlotStyle -> {{Black, Thick}}],
  Graphics[{Text[
     Style[HoldForm[p = 0], Large, Bold, Black], {18, 1.2}],
    Text[Style[HoldForm[2 %], Large, Bold, Black], {18, 1.6}],
    Text[Style[HoldForm[4 %], Large, Bold, Black], {18, 2.2}],
    Text[Style[HoldForm[6 %], Large, Bold, Black], {18, 3}],
    Text[Style[HoldForm[8 %], Large, Bold, Black], {18, 4}],
    Text[Style[HoldForm[10 %], Large, Bold, Black], {16.5, 4.8}],
    Text[Style[HoldForm[12 %], Large, Bold, Black], {14, 4.8}],
    Text[Style[HoldForm[14 %], Large, Bold, Black], {12, 4.8}]}]}]


Export["/mytex/2008-mm-zz.ps", Out[ ], "EPS", ImageSize -> 900]

FEDFUNDS = {0.80, 1.22, 1.06, 0.85, 0.83, 1.28, 1.39, 1.29, 1.35,
   1.43, 1.43, 1.64, 1.68, 1.96, 2.18, 2.24, 2.35, 2.48, 2.45, 2.50,
   2.50, 2.62, 2.75, 2.71, 2.75, 2.73, 2.95, 2.96, 2.88, 2.94, 2.84,
   3.00, 2.96, 3.00, 3.00, 3.00, 2.99, 3.24, 3.47, 3.50, 3.28, 2.98,
   2.72, 1.67, 1.20, 1.26, 0.63, 0.93, 0.68, 1.53, 1.76, 1.80, 2.27,
   2.42, 2.48, 2.43, 2.80, 2.96, 2.90, 3.39, 3.47, 3.50, 3.76, 3.98,
   4.00, 3.99, 3.99, 3.97, 3.84, 3.92, 3.85, 3.32, 3.23, 2.98, 2.60,
   2.47, 2.44, 1.98, 1.45, 2.54, 2.02, 1.49, 1.98, 1.73, 1.17, 2.00,
   1.88, 2.26, 2.61, 2.33, 2.15, 2.37, 2.85, 2.78, 2.36, 2.68, 2.71,
   2.93, 2.90, 2.90, 2.94, 2.93, 2.92, 3.00, 2.98, 2.90, 3.00, 2.99,
   3.02, 3.49, 3.48, 3.50, 3.48, 3.38, 3.48, 3.48, 3.43, 3.47, 3.50,
   3.50, 3.42, 3.50, 3.45, 3.36, 3.52, 3.85, 3.90, 3.98, 4.04, 4.09,
   4.10, 4.04, 4.09, 4.12, 4.01, 4.08, 4.10, 4.32, 4.42, 4.60, 4.65,
   4.67, 4.90, 5.17, 5.30, 5.53, 5.40, 5.53, 5.76, 5.40, 4.94, 5.00,
   4.53, 4.05, 3.94, 3.98, 3.79, 3.90, 3.99, 3.88, 4.13, 4.51, 4.60,
   4.71, 5.05, 5.76, 6.11, 6.07, 6.02, 6.03, 5.78, 5.91, 5.82, 6.02,
   6.30, 6.61, 6.79, 7.41, 8.67, 8.90, 8.61, 9.19, 9.15, 9.00, 8.85,
   8.97, 8.98, 8.98, 7.76, 8.10, 7.94, 7.60, 7.21, 6.61, 6.29, 6.20,
   5.60, 4.90, 4.14, 3.72, 3.71, 4.15, 4.63, 4.91, 5.31, 5.56, 5.55,
   5.20, 4.91, 4.14, 3.50, 3.29, 3.83, 4.17, 4.27, 4.46, 4.55, 4.80,
   4.87, 5.04, 5.06, 5.33, 5.94, 6.58, 7.09, 7.12, 7.84, 8.49, 0.40,
   0.50, 0.78, 0.01, 0.03, 9.95, 9.65, 8.97, 9.35, 0.51, 1.31, 1.93,
   2.92, 2.01, 1.34, 0.06, 9.45, 8.53, 7.13, 6.24, 5.54, 5.49, 5.22,
   5.55, 6.10, 6.14, 6.24, 5.82, 5.22, 5.20, 4.87, 4.77, 4.84, 4.82,
   5.29, 5.48, 5.31, 5.29, 5.25, 5.02, 4.95, 4.65, 4.61, 4.68, 4.69,
   4.73, 5.35, 5.39, 5.42, 5.90, 6.14, 6.47, 6.51, 6.56, 6.70, 6.78,
   6.79, 6.89, 7.36, 7.60, 7.81, 8.04, 8.45, 8.96, 9.76, 0.03, 0.07,
   0.06, 0.09, 0.01, 0.24, 0.29, 0.47, 0.94, 1.43, 3.77, 3.18, 3.78,
   3.82, 4.13, 7.19, 7.61, 0.98, 9.47, 9.03, 9.61, 0.87, 2.81, 5.85,
   8.90, 9.08, 5.93, 4.70, 5.72, 8.52, 9.10, 9.04, 7.82, 5.87, 5.08,
   3.31, 2.37, 3.22, 4.78, 4.68, 4.94, 4.45, 4.15, 2.59, 0.12, 0.31,
   9.71, 9.20, 8.95, 8.68, 8.51, 8.77, 8.80, 8.63, 8.98, 9.37, 9.56,
   9.45, 9.48, 9.34, 9.47, 9.56, 9.59, 9.91, 0.29, 0.32, 1.06, 1.23,
   1.64, 1.30, 9.99, 9.43, 8.38, 8.35, 8.50, 8.58, 8.27, 7.97, 7.53,
   7.88, 7.90, 7.92, 7.99, 8.05, 8.27, 8.14, 7.86, 7.48, 6.99, 6.85,
   6.92, 6.56, 6.17, 5.89, 5.85, 6.04, 6.91, 6.43, 6.10, 6.13, 6.37,
   6.85, 6.73, 6.58, 6.73, 7.22, 7.29, 6.69, 6.77, 6.83, 6.58, 6.58,
   6.87, 7.09, 7.51, 7.75, 8.01, 8.19, 8.30, 8.35, 8.76, 9.12, 9.36,
   9.85, 9.84, 9.81, 9.53, 9.24, 8.99, 9.02, 8.84, 8.55, 8.45, 8.23,
   8.24, 8.28, 8.26, 8.18, 8.29, 8.15, 8.13, 8.20, 8.11, 7.81, 7.31,
   6.91, 6.25, 6.12, 5.91, 5.78, 5.90, 5.82, 5.66, 5.45, 5.21, 4.81,
   4.43, 4.03, 4.06, 3.98, 3.73, 3.82, 3.76, 3.25, 3.30, 3.22, 3.10,
   3.09, 2.92, 3.02, 3.03, 3.07, 2.96, 3.00, 3.04, 3.06, 3.03, 3.09,
   2.99, 3.02, 2.96, 3.05, 3.25, 3.34, 3.56, 4.01, 4.25, 4.26, 4.47,
   4.73, 4.76, 5.29, 5.45, 5.53, 5.92, 5.98, 6.05, 6.01, 6.00, 5.85,
   5.74, 5.80, 5.76, 5.80, 5.60, 5.56, 5.22, 5.31, 5.22, 5.24, 5.27,
   5.40, 5.22, 5.30, 5.24, 5.31, 5.29, 5.25, 5.19, 5.39, 5.51, 5.50,
   5.56, 5.52, 5.54, 5.54, 5.50, 5.52, 5.50, 5.56, 5.51, 5.49, 5.45,
   5.49, 5.56, 5.54, 5.55, 5.51, 5.07, 4.83, 4.68, 4.63, 4.76, 4.81,
   4.74, 4.74, 4.76, 4.99, 5.07, 5.22, 5.20, 5.42, 5.30, 5.45, 5.73,
   5.85, 6.02, 6.27, 6.53, 6.54, 6.50, 6.52, 6.51, 6.51, 6.40, 5.98,
   5.49, 5.31, 4.80, 4.21, 3.97, 3.77, 3.65, 3.07, 2.49, 2.09, 1.82,
   1.73, 1.74, 1.73, 1.75, 1.75, 1.75, 1.73, 1.74, 1.75, 1.75, 1.34,
   1.24, 1.24, 1.26, 1.25, 1.26, 1.26, 1.22, 1.01, 1.03, 1.01, 1.01,
   1.00, 0.98, 1.00, 1.01, 1.00, 1.00, 1.00, 1.03, 1.26, 1.43, 1.61,
   1.76, 1.93, 2.16, 2.28, 2.50, 2.63, 2.79, 3.00, 3.04, 3.26, 3.50,
   3.62, 3.78, 4.00, 4.16, 4.29, 4.49, 4.59, 4.79, 4.94, 4.99, 5.24,
   5.25, 5.25, 5.25, 5.25, 5.24, 5.25, 5.26, 5.26, 5.25, 5.25, 5.25,
   5.26, 5.02, 4.94, 4.76, 4.49, 4.24, 3.94, 2.98, 2.61, 2.28, 1.98,
   2.00, 2.01, 2.00, 1.81};

CPIAUCSL = {26.800, 26.900, 26.900, 27.000, 26.900, 26.900, 26.900,
   27.000, 26.900, 26.900, 26.900, 26.900, 26.900, 26.900, 26.800,
   26.700, 26.800, 26.800, 26.800, 26.800, 26.800, 26.800, 26.800,
   26.700, 26.800, 26.700, 26.900, 26.800, 26.900, 26.900, 26.800,
   26.900, 26.900, 26.900, 27.000, 27.200, 27.300, 27.300, 27.400,
   27.500, 27.500, 27.600, 27.700, 27.800, 27.900, 27.900, 28.000,
   28.100, 28.200, 28.300, 28.300, 28.300, 28.400, 28.500, 28.600,
   28.700, 28.900, 28.900, 28.900, 28.900, 28.900, 28.900, 28.900,
   28.900, 29.000, 29.000, 29.000, 29.000, 29.000, 29.000, 29.000,
   29.100, 29.200, 29.200, 29.300, 29.400, 29.400, 29.400, 29.400,
   29.400, 29.400, 29.500, 29.600, 29.600, 29.600, 29.600, 29.600,
   29.800, 29.800, 29.800, 29.800, 29.800, 29.800, 29.800, 29.800,
   29.800, 29.900, 29.900, 30.000, 30.000, 30.000, 30.000, 30.000,
   30.100, 30.200, 30.200, 30.200, 30.200, 30.200, 30.300, 30.400,
   30.400, 30.400, 30.400, 30.400, 30.500, 30.500, 30.500, 30.500,
   30.600, 30.700, 30.800, 30.700, 30.800, 30.800, 30.900, 30.900,
   30.900, 30.900, 31.000, 31.000, 31.000, 31.000, 31.100, 31.100,
   31.100, 31.200, 31.300, 31.300, 31.300, 31.300, 31.400, 31.500,
   31.600, 31.600, 31.600, 31.600, 31.700, 31.800, 31.900, 31.900,
   32.100, 32.200, 32.300, 32.400, 32.400, 32.500, 32.700, 32.800,
   32.900, 32.900, 32.900, 32.900, 33.000, 33.000, 33.100, 33.100,
   33.300, 33.400, 33.500, 33.600, 33.700, 33.900, 34.000, 34.100,
   34.200, 34.300, 34.400, 34.500, 34.700, 34.900, 35.000, 35.100,
   35.300, 35.400, 35.600, 35.700, 35.800, 36.100, 36.300, 36.400,
   36.600, 36.800, 36.900, 37.100, 37.300, 37.500, 37.700, 37.900,
   38.100, 38.300, 38.500, 38.600, 38.800, 38.900, 39.000, 39.200,
   39.400, 39.600, 39.800, 39.900, 39.900, 40.000, 40.100, 40.300,
   40.500, 40.600, 40.700, 40.800, 40.900, 41.000, 41.100, 41.200,
   41.400, 41.400, 41.500, 41.600, 41.700, 41.800, 41.900, 42.100,
   42.200, 42.400, 42.500, 42.700, 43.000, 43.400, 43.700, 43.900,
   44.200, 44.200, 45.000, 45.200, 45.600, 45.900, 46.300, 46.800,
   47.300, 47.800, 48.100, 48.600, 49.000, 49.300, 49.900, 50.600,
   51.000, 51.500, 51.900, 52.300, 52.600, 52.800, 53.000, 53.100,
   53.500, 54.000, 54.200, 54.600, 54.900, 55.300, 55.600, 55.800,
   55.900, 56.000, 56.100, 56.400, 56.700, 57.000, 57.300, 57.600,
   57.900, 58.100, 58.400, 58.700, 59.300, 59.600, 60.000, 60.200,
   60.500, 60.800, 61.100, 61.300, 61.600, 62.000, 62.300, 62.700,
   63.000, 63.400, 63.900, 64.500, 65.000, 65.500, 65.900, 66.500,
   67.100, 67.500, 67.900, 68.500, 69.200, 69.900, 70.600, 71.400,
   72.200, 73.000, 73.700, 74.400, 75.200, 76.000, 76.900, 78.000,
   79.000, 80.100, 80.900, 81.700, 82.500, 82.600, 83.200, 83.900,
   84.700, 85.600, 86.400, 87.200, 88.000, 88.600, 89.100, 89.700,
   90.500, 91.500, 92.200, 93.100, 93.400, 93.800, 94.100, 94.400,
   94.700, 94.700, 95.000, 95.900, 97.000, 97.500, 97.700, 97.700,
   98.100, 98.000, 97.700, 97.900, 98.000, 98.100, 98.800, 99.200,
   99.400, 99.800, 100.100, 100.400, 100.800, 101.100, 101.400,
   102.100, 102.600, 102.900, 103.300, 103.500, 103.700, 104.100,
   104.400, 104.700, 105.100, 105.300, 105.500, 105.700, 106.300,
   106.800, 107.000, 107.200, 107.500, 107.700, 107.900, 108.100,
   108.500, 109.000, 109.500, 109.900, 109.700, 109.100, 108.700,
   109.000, 109.400, 109.500, 109.600, 110.000, 110.200, 110.400,
   110.800, 111.400, 111.800, 112.200, 112.700, 113.000, 113.500,
   113.800, 114.300, 114.700, 115.000, 115.400, 115.600, 116.000,
   116.200, 116.500, 117.200, 117.500, 118.000, 118.500, 119.000,
   119.500, 119.900, 120.300, 120.700, 121.200, 121.600, 122.200,
   123.100, 123.700, 124.100, 124.500, 124.500, 124.800, 125.400,
   125.900, 126.300, 127.500, 128.000, 128.600, 128.900, 129.100,
   129.900, 130.500, 131.600, 132.500, 133.400, 133.700, 134.200,
   134.700, 134.800, 134.800, 135.100, 135.600, 136.000, 136.200,
   136.600, 137.000, 137.200, 137.800, 138.200, 138.300, 138.600,
   139.100, 139.400, 139.700, 140.100, 140.500, 140.800, 141.100,
   141.700, 142.100, 142.300, 142.800, 143.100, 143.300, 143.800,
   144.200, 144.300, 144.500, 144.800, 145.000, 145.600, 146.000,
   146.300, 146.300, 146.700, 147.100, 147.200, 147.500, 147.900,
   148.400, 149.000, 149.300, 149.400, 149.800, 150.100, 150.500,
   150.900, 151.200, 151.800, 152.100, 152.400, 152.600, 152.900,
   153.100, 153.500, 153.700, 153.900, 154.700, 155.000, 155.500,
   156.100, 156.400, 156.700, 157.000, 157.200, 157.700, 158.200,
   158.700, 159.100, 159.400, 159.700, 159.800, 159.900, 159.900,
   160.200, 160.400, 160.800, 161.200, 161.500, 161.700, 161.800,
   162.000, 162.000, 162.000, 162.200, 162.600, 162.800, 163.200,
   163.400, 163.500, 163.900, 164.100, 164.400, 164.700, 164.700,
   164.800, 165.900, 166.000, 166.000, 166.700, 167.100, 167.800,
   168.100, 168.400, 168.800, 169.300, 170.000, 171.000, 170.900,
   171.200, 172.200, 172.700, 172.700, 173.600, 173.900, 174.200,
   174.600, 175.600, 176.000, 176.100, 176.400, 177.300, 177.700,
   177.400, 177.400, 178.100, 177.600, 177.500, 177.400, 177.700,
   178.000, 178.500, 179.300, 179.500, 179.600, 180.000, 180.500,
   180.800, 181.200, 181.500, 181.800, 182.600, 183.600, 183.900,
   183.200, 182.900, 183.100, 183.700, 184.500, 185.100, 184.900,
   185.000, 185.500, 186.200, 186.700, 187.100, 187.400, 188.200,
   188.900, 189.100, 189.400, 189.800, 190.800, 191.600, 191.700,
   191.700, 192.300, 192.900, 193.800, 193.500, 193.600, 194.900,
   196.200, 198.800, 199.200, 198.300, 198.200, 199.400, 199.300,
   199.600, 200.600, 201.300, 201.900, 202.900, 203.800, 202.900,
   201.900, 202.100, 203.300, 203.552, 204.158, 205.098, 205.751,
   206.700, 207.246, 207.708, 207.749, 208.509, 209.055, 210.930,
   211.680, 212.516, 212.571, 213.301, 213.743, 215.132, 217.403,
   219.181, 218.880, 218.813};

DeltaCPIAUCSL =
  Table[CPIAUCSL[[i + 12]] - CPIAUCSL[[i]], {i, 1,
    Length[CPIAUCSL] - 12}];

ListPlot[{FEDFUNDS, DeltaCPIAUCSL}, Joined -> True, Frame -> True,
 FrameLabel -> {Style[time, Bold, FontSize -> 34],
   Style[HoldForm[%], Bold, FontSize -> 34]}, AspectRatio -> 0.8,
 FrameTicksStyle -> Directive[Black, 24, Bold], Axes -> False,
 PlotRange -> All,
 FrameTicks -> {{{1, HoldForm[1954]}, {(Length[CPIAUCSL] - 12)/4,
     1968}, {3*(Length[CPIAUCSL] - 12)/4,
     1995}, {(Length[CPIAUCSL] - 12)/2,
     1954 + (2008 - 1954)/2}, {Length[CPIAUCSL] - 12,
     HoldForm[2008]}}, {0, 1, 2, 3, 4, 5, 6, 7, 8, 9, 10, 11}, None,
   None}(*,Filling->{1->{{2},{Gray,Orange}}}*),
 PlotStyle -> {{Red, Thick}, {Blue, Thick}},
 Filling -> {1 -> {{2}, {Orange, Yellow}}}]

Export["/mytex/2008-mm-in_inf.ps", Out[ ], "EPS", ImageSize -> 900]

ListPlot[{FEDFUNDS-DeltaCPIAUCSL}, Joined -> True, Frame -> True,
 FrameLabel -> {Style[time, Bold, FontSize -> 34],
   Style[HoldForm[%], Bold, FontSize -> 34]}, AspectRatio -> 0.8,
 FrameTicksStyle -> Directive[Black, 24, Bold], Axes -> False,
 PlotRange -> All,
 FrameTicks -> {{{1, HoldForm[1954]}, {(Length[CPIAUCSL] - 12)/4,
     1968}, {3*(Length[CPIAUCSL] - 12)/4,
     1995}, {(Length[CPIAUCSL] - 12)/2,
     1954 + (2008 - 1954)/2}, {Length[CPIAUCSL] - 12,
     HoldForm[2008]}}, {0, 1, 2, 3, 4, 5, 6, 7, 8, 9, 10, 11}, None,
   None}(*,Filling->{1->{{2},{Gray,Orange}}}*),
 PlotStyle -> {{Red, Thick}, {Blue, Thick}}]

