\documentstyle[12pt]{article}
%\renewcommand{\baselinestretch}{2}
 \def\Bbb{\bf}
\begin{document}
%  \special{!userdict begin /bop-hook{gsave 200 30 translate
%  65 rotate /Times-Roman findfont 216 scalefont setfont
%  0 0 moveto 0.95 setgray (ROHFASSUNG) show grestore}def end}

% \def\Bbb{\bf }
% \def\frak{\cal }

\title{Set theory and physics}
\author{K. Svozil\\
 {\small Institut f\"ur Theoretische Physik}  \\
  {\small University of Technology Vienna }     \\
  {\small Wiedner Hauptstra\ss e 8-10/136}    \\
  {\small A-1040 Vienna, Austria   }            \\
  {\small e-mail: svozil@tph.tuwien.ac.at}\\
  {\small www: http://tph.tuwien.ac.at/$\widetilde{\;\;}\,$svozil}}
\date{ }
\maketitle

\begin{flushright}
{\scriptsize 1995-set.tex}
\end{flushright}
\newpage

\begin{abstract}
{\em
In as much as physical theories are formalizable, set theory provides a
framework for theoretical physics.
Four speculations about the relevance of set theoretical modeling for
physics are presented: the r\^{o}le of
transcendental set theory (i) in chaos theory, (ii) for paradoxical
decompositions of solid threedimensional objects,
 (iii) in
the theory of effective computability (Church-Turing thesis) related to
the possible
``solution of supertasks'', and (iv) for weak solutions.
Several approaches to set theory and their advantages and disadvantages
for physical applications are discussed: Cantorian
``naive'' (i.e., non-axiomatic) set theory, constructivism
and operationalism.
In the author's opinion, an attitude of ``suspended attention'' (a
term borrowed from psychoanalysis) seems most promising for progress.
Physical and set theoretical entities
must be operationalized
wherever possible. At the same time, physicists should be open to
``bizarre'' or ``mindboggling'' new formalisms,
which need not be operationalizable or testable at the time of their
creation, but which may successfully lead to novel fields of
phenomenology and technology.
}
\end{abstract}


\newpage

\section{A short history of set theory, with emphasis on operationalism}
Physicists usually do not pay much attention to the particulars of set
theory. They tend to have
a pragmatic attitude
towards the foundations of the formal sciences, combined with the
suspicion that, as has been stated by Einstein
(\cite{einstein-math}, translated from
German)\footnote{
{\em
``Insofern sich die S\"atze der Mathematik auf die Wirklichkeit
beziehen, sind sie nicht sicher, und insofern sie sicher sind,
beziehen sie sich nicht auf die Wirklichkeit.''}}
{\em  ``insofar mathematical theorems refer to reality,
 they are not sure,
 and insofar they are sure,
 they do not refer  to reality.''}

Yet there are instances when foundational issues {\em do} play a
r\^{o}le. It is due to a lack of expertise and experience, that the
empiric researcher is then particularly vulnerable to misconceptions.
Below we shall give examples where set theoretic specifications are
essential to the argument.
But we have to first briefly review set theory in general.


In Cantorian  (i.e., non-axiomatic) set theory, the ``definition''
of the concept of a set reads
(\cite{cantor-set}, translated from German
\cite{zer-fr}),\footnote{
{\em ``Unter einer ``Menge'' verstehen wir jede Zusammenfassung $M$ von
bestimmten wohlunterschiedenen Objekten $m$ unsrer Anschauung oder
unseres Denkens (welche die ``Elemente'' von $M$ genannt werden) zu
einem Ganzen.''}}
{\em
``A set is a collection into a whole of definite distinct objects of
our intuition or of our thought. The objects are called the elements
(members) of the set.''}
As general as it is conceived, Cantorian set theory
would
provide a powerful
mathematical framework for theoretical physics.
Per definition,  hardly any conceivable object does {\em not} fall
within its domain.
Indeed, how gratifying and ambitious, but also how challenging this
conception, one can imagine from Hilbert's emphatic declaration
(cf. \cite{hil-26}, p. 170,
translated from German),\footnote{
{\em ``Aus dem Paradies, das Cantor uns geschaffen, soll uns niemand
vertreiben k\"onnen.''}}
{\em ``From the paradise which  Cantor created, no one shall
be able to expel us.''}


Alas, Cantorian set theory, at least its uncritical development, proved
inconsistent.
Both Cantor
(cf.
\cite{zer-fr}, p. 7 and \cite{cantor}, p. IV)
and Hilbert were fully aware of the set theoretical antinomies such as
Russell's paradox,  {\em ``The set of all
sets that are not members of themselves.''}
Cantor himself  discovered one of the first antinomies around 1895, even
before the Burali-Forti antinomy. In 1899, Cantor wrote in a letter to
Dedekind
(\cite{cantor}, p. 443, translated from German
\cite{boos}),\footnote{
{\em ``Eine Vielheit kann n\"amlich so beschaffen sein, da\ss $\,$ die
Annahme eines ``Zusammenseins'' {\em aller} ihrer Elemente auf einen
Widerspruch f\"uhrt, so da\ss $\,$ es unm\"oglich ist, die Vielheit als
eine Einheit, als ``ein fertiges Ding'' aufzufassen. Solche Vielheiten
nenne ich {\em absolut unendliche} oder {\em inkonsistente Vielheiten.}
[Absatz]
Wie man sich leicht \"uberzeugt, ist z. B.  der ``Inbegriff alles
Denkbaren'' eine solche Vielheit; $\ldots$''
}}
{\em For a multiplicity can be so constituted that the assumption of a
``being together'' of {\em all} its elements leads to a contradiction,
so that it is impossible to consider the multiplicity a a unit[y], thus
``a complete thing.'' I call such multiplicities {\em absolutely
infinite} or inconsistent multiplicities.
[paragraph]
As one readily convinces oneself, the ``aggregate of everything
thinkable'' is, for example, such a multiplicity; $\ldots$''}

In Hilbert's formalist view of the infinite,
all proofs
using non-terminating sequences of operations should be
substituted by finite processes and proof methods
(cf. \cite{hil-26}, p. 162).\footnote{
{\em ``Und so wie das Operieren mit dem Unendlichkleinen durch Prozesse
im Endlichen ersetzt wurde, welche ganz dasselbe leisten und zu
ganz denselben eleganten Beziehungen f\"uhren, so m\"ussen \"uberhaupt
die Schlu\ss weisen mit dem Unendlichen durch endliche Prozesse
ersetzt werden, die gerade dasselbe leisten, d.h. dieselben
Beweisg\"ange und dieselben Methoden der Gewinnung von Formeln und
S\"atzen erm\"oglichen.''}}
For Hilbert, a typical example for this program
was Weierstra\ss 's approach to analysis.



Others, among them Zermelo and Fraenkel \cite{zer-fr}, were less secure
in the
``Cantorian heaven of set theory'' and attempted to block the paradoxes
by axiomatically
restricting the rules of set generation.
The necessary price was a restriction of the mathematical Universe.

Rather unexpected for Hilbert, but obvious for Brouwer
(cf. \cite{g-brower}, p. 88),
 G\"odel \cite{godel1} showed that
in any
reasonably strong axiomatic theory (rich enough to allow for
arithmetic),
consistency cannot be proven. (One may quite justifiable ask whether a
``proof'' of consistency would really be of any value; after all, if a
theory were inconsistent, then consistency could also be ``proved''
therein!)

Further restrictions to set generation were imposed by
constructive mathematics, anticipated by the radical
``Verbotsdiktator'' Kronecker. The varieties of constructive mathematics
\cite{bridges-richman} comprises
the intuitionistic
school around Brouwer and  Heyting, the Russian school, and Bishop's
constructive mathematics. For  more recent developments, see Bridges
\cite{bishop-bridges,bridges-richman}.
Essentially, the existence of  mathematical objects is accepted only if
the objects can be constructed by an
algorithm.
An algorithm is a finite procedure. That is, its algorithmic information
(minimal description length), as well as its execution time is finite.
It can be perceived
as the step-by-step execution of a deterministic computer program.
Constructive mathematics is not particularly concerned with the actual
size of algorithmic information
and dynamic complexity (time and space), as long as they are
finite (although it is acknowledged that such considerations are
important for practical applications).
In Russian constructive mathematics, the term ``algorithm'' is
a synonym for a finite sequence of symbols in a fixed programming
language.



Enter physics. Not long after Hilbert's bold statement concerning the
Cantorian paradise (which was directed against the uncritical use of the
{\em ``actual infinity''} in mathematics and the natural sciences)
appeared a critical essay on the methods of set theory by Bridgman
\cite{bridgman}. (Landauer has referred to Bridgman's
article
at several occasions \cite{landauer-67,landauer-87,landauer-95}.)
Bridgman's {\em operationalism} was directed against the uncritical use
of theoretical concepts
\cite{bridgman-logic,bridgman-theory}.
 In particular, he demanded that the meaning
of theoretical concepts should ultimately be based on concrete
physical operations. That is,
(cf. \cite{bridgman-reflextions}, p. V),
{\em
``the meaning of one's terms are to be found
by an analysis of the operations which one performs in applying the
term in concrete situations or in verifying the truth of statements or
in finding the answers to questions.''}
In his later writings, Bridgman clarified (and somewhat weakened)
operationalism
by differentiating between ``instrumental'' and ``paper-and-pencil''
operations
(cf. \cite{bridgman-nature}, p. 8-10),
{\em
``It is often supposed that the operational criterion of meaning demands
that the operations which give meaning to a physical concept {\em must}
be instrumental operations. This is, I believe, palpably a mistaken
point of view, for simple observation shows that physicists do
profitably employ concepts the meaning of which is not to be found in
the instrumental operations of the laboratory, and which cannot be
reduced to such operations without residue. Nearly all the concepts of
theoretical and mathematical physics are of this character, such for
example as the stress inside an elastic body subject to surface forces,
or the $\psi$ function of wave mechanics. $\ldots$ we may single out
$\ldots$ the sort of operations performed by the theoretical physicist
in his mathematical manipulations and characterize these as
`paper-and-pencil' operations. Among paper-and-pencil operations are to
be included all manipulations with symbols, whether or not the symbols
are the conventional symbols of mathematics.
$\ldots$
a great latitude is allowed to the verbal and the
paper-and-pencil operation. I think, however, that physicists are agreed
in imposing one restriction on the freedom of such operations, namely
that such operations must be capable of eventually, although perhaps
indirectly, making connection with instrumental operations.''}

Bridgman pointed out that in Cantorian set theory there is one
particularly vicious method of specifying operations. In his own words
(cf.
\cite{bridgman}, p. 106),
{\em ``It is possible to set up rules which determine a non-terminating
sequence of operations, as for instance, the rules by which the sequence
of the natural number is engendered. But it is obviously not legitimate
to specify in this way a non-terminating operation, and then to treat
this non-terminating complex as itself a simpler operation which may be
used as an intuitive ultimate in the specification of another operation.
Such a non-terminating complex can be treated in this way only when it
can be proved equivalent to some other procedure specifiable in finite
terms, and which can, therefore, be actually executed. Otherwise, the
non-terminating complex must be treated as the end, and no other
operations be demanded after it; our ordinary experience of the order of
operations as performed in time evidently requires this.''}
In present-day, recursion theoretic, terminology, a ``complex
operation'' would be called a
``sub-program'' or ``(sub-)algorithm,'' and the term ``non-terminating''
would be translated as
``diverging'' in the sense of ``non-recursively bound.'' In terms of
recursion theory, Bridgman's claim can be re-interpreted such that no
diverging algorithm should  be allowed as legal input of any other
(terminating) algorithm.

One may go even further than Bridgman and assume that, since infinite
entities are not operational, infinities have to be
abandoned altogether.
The elimination of even potential
infinities leaves us with merely finite objects.
Finitistic arguments and physical limits have been put forward
by Gandi
\cite{gandy1,gandy2},
 Mundici \cite{mundici}, Landauer \cite{landauer-onlim} and Casti
\cite{casti-onlim}.



\section{The ``Go-Go'' principle}

The Cantorian ``permissive'' approach to the foundations of
mathematics stimulated the
invention, creation and investigation of the weirdest ``monsters'' of
thought.
Per definition, no construction or speculation could be crazy
enough to be
excluded from the formal sciences.
While inconsistent, this attitude brought forth an undeniable
advancement in the formal sciences insofar as objects were discovered
which had novel, sometimes bizarre and even ``mindboggling'' features.


Take, for instance, Cantor's map of the unit line onto the unit square
which is one-to-one, or Peano's continuous map from a line onto the unit
square. Another example is the Cantor set (nowadays called a
``fractal'')
${\cal C}= \{ \sum_{n=1}^\infty c_n3^{-n}\mid c_n\in \{0,2\} \mbox{ for
each }n\}$, which has vanishing measure $\mu ({\cal
C})=\lim_{n\rightarrow 0}(2/3)^n=0$ but which can be brought into a
one-to-one correspondence to the unit interval of the binary reals.
Another ``mindboggling'' result concerning measure-theoretic
non-preservation is the Banach-Tarski paradox discussed below.

The spirit behind all these findings seems to be
 that
``everything goes.'' Stated pointedly:
\begin{quote}
{\em Every method and object should be permitted as long
as it is not excluded by the rules.} Or:
{\em Anything that is not forbidden is allowed.}
\end{quote}
In the following, this attitude will be called the ``Go-Go'' principle.
It may be applied both to the formal and to the natural
sciences.\footnote{
The author wants to make it quite clear that he neither rejects nor
supports the ``Go-Go'' principle for reasons which are discussed below.}

A few remarks are in order.
As has been pointed out before, consistency cannot be
proven from within the rules, at least not if the rules are strong
enough to allow for arithmetic or universal computation.


The
``Go-Go''
principle  collides with the axiomatic method using recursive
rules of inference. It can be expressed as follows:
\begin{quote}
{\em Every method or
object is excluded which is not derivable by the rules.}
Or:
{\em Anything that is not allowed is forbidden.}
\end{quote}
One might jokingly call this
the ``No-Go'' principle.\footnote{
My first denomination of this style was ``No-No.'' The present term
``No-Go'' is due to a Freudian slip
by Professor Joseph F. Traub.}
 Despite its rather restrictive attitude,
the axiomatic method seems good enough to include analysis
\cite{bishop-bridges} (and, at least good enough to re-derive many
important results first discovered by
``Go-Go.'')
Formalist like Hilbert have even claimed that it should turn out to be
all the same, finally.


One should also be aware that the ``Go-Go'' principle allows a
pragmatic
point of view, which most researchers practice anyway: since it is
difficult to develop progressive and innovative ideas, the real problem
in the sciences
might not be to eliminate ill-conceived concepts and methods but to
introduce novel features at all.
Otherwise, one might argue,  mathematicians would have just to evoke
an automated proof machine, a
``perfect publicator,'' which makes its creators superfluous.

Therefore, judged from a pragmatic point of view, the ``Go-Go''
principle might prove progressive but unreliable.
To put it pointedly: the ``Go-Go'' principle might be essential for
producing novel results, for the discovery of undiscovered land
(Hilbert's paradise),
even if it is
known that it yields antinomies.

Despite all positive aspects which have been mentioned so far, as
liberal as it is conceived, the
``Go-Go'' principle is unable to cope with its own limitations, in
particular with respect to applications to physics.
Therefore, physicists are occasionally confronted with ``effects'' or
``predictions'' of physical theory which have their origin in
non-constructive, non-operational features of the set theory underlying
that physical theory.
But even if
such ``effects'' from theoretical artifacts might prove elusive most of
the time, seldom enough (cf. non-Euclidean geometry) they might lead us
to totally unexpected classes of phenomena.

In what follows, some speculative examples inspired
by
``Go-Go'' are discussed.
They correspond to paper-and-pencil operations.
If they will eventually be capable of making connection, perhaps
indirectly, with instrumental operations, remains to be seen.


\subsection{``Chaos'' theory}
The emergence of ``chaos
theory'' has highlighted the use of classical continua
\cite{calude-sv}.\footnote{
I would also like to point the reader's attention to the question of the
preservation of computability in classical analysis; in particular
 to older attempts by Specker \cite{specker}, Wang \cite{wang},
Kreisel
\cite{kreisel}
and    Stef\u anescu  \cite{ds},
as well as to the more recent
ones by Pour-El and Richards \cite{pour-el}
(cf. objections raised by Penrose \cite{penrose} and Bridges
\cite{bridges1})
and
Calude,
Campbell,
Svozil and
 \c{S}tef\u anescu \cite{calude-sv}.}
There, the scenario is that the equation of motion
seems to ``reveal'' the algorithmic information
\cite{chaitin,calude,vitani} of the initial value
\cite{shaw,smale,ford}.\footnote{
In a very recent book on finite precision computations by
Chaitin-Chatelin and  Fraysse\'{e} \cite{chaitin-chatelin}
point out that, in a certain, well-defined way, exact absolute
information is too unstable and does not give rise to the full richness
of physical solutions. In particular, finite-precision arithmetic is
more suitable to model physical systems that fluctuate.}


Consider, for example, the
logistic equation of motion
$f:x_n\rightarrow x_{n+1}=f(x_n)=\alpha x_n(1-x_n)$ for variable $x_n$
at discrete times
$n\in {\Bbb N_0}$. It can, for $\alpha =4$
and after the variable transformation $x_n=\sin^2(\pi
X_n)$, be rewritten as $f:X_n\rightarrow X_{n+1}=2X_n \mbox{(mod
1)}$,
where $\mbox{(mod 1)}$ means that one has to drop the integer part of
$2X_n$. By assuming a starting value $X_0$, the formal solution after
$n$ iterations is $f^{(n)}(X_0)=X_n=2^nX_0$ (mod 1).
Note that,
if $X_0$ is in binary representation, $f^{(n)}$ is just $n$ times a left
 shift
of the digits of $X_0$, followed by a left truncation before the decimal
point.

Assume now that the measurement precision is the first $m$ bits of
$X_n$,
in the binary expansion of
$X_n$. In a single time step, the evolution function $f$ effectively
reveals the next digit of $X_0$, which was unobservable before.
That is, in order to be able to measure the initial value for an
arbitrary but finite precision $m'$, one has to wait and measure $X_0$
until time
$\max (m'-m,0)$.

The only possible  ``chaotic'' feature in this
scenario resides in the initial
value: the theoretician has to {\em assume} that $X_0\in (0,1)$ is
uncomputable or even Martin-L\"of/Solovay/Chaitin
random. Then the computable function $f^{(n)}(X_0)$ yields
a measurable bit stream which reconstructs the binary expansion of
$X_0$, which is uncomputable or even
Mar\-tin-L\"of\-/So\-lo\-vay/\-Chai\-tin  random.
To put it pointedly: if the input is a random real, then the output
approximates a random real; in more physical terms: if
unpredictability is assumed, then chaotic motion follows. (More
ironically:
garbage in,
garbage out.)
That is all there is.

It is amazing how susceptible the general public as well as
many physicists
are to contemplate this form of ``chaotic'' motion as a fundamental fact
about the nature of (physical) reality rather than as a theoretical
assumption.
In the author's opinion, one of the reasons\footnote{
There seem to be powerful counter-rationalistic
forces, not to mention wishful thinking, which seduce people into
believe systems that physics has ``finally re-discovered''
ever-to-remain obscure phenomena, that we are even on the verge of the
``end of the age of the natural sciences;''
forces which seem to be directed against the scientific research program
of the Enlightenment
put forward by Descartes, Hume, Humboldt and others.}
 for this willingness of physicists to
accept Mar\-tin-L\"of\-/So\-lo\-vay/\-Chai\-tin randomness as a matter
of natural fact is that physicists have been trained in
the domain of classical continuum mechanics \cite{goldstein}.
The term ``classical''
here refers to both non-quantum mechanics, as well as to Cantorian set
theory.

To be more precise, recall that
Cantor's  famous diagonalization argument \cite{cantor-set}
asserts that the set of reals in the interval $[0,1]$ are nondenumerable:
Assume that there exists an effectively
 computable
 enumeration of the decimal reals in the interval $[0,1]$ of the form
$$ \begin{array}{c}
 r_1=0.r_{11}r_{12}r_{13}r_{14} \cdots \\
 r_2=0.r_{21}r_{22}r_{23}r_{24} \cdots \\
 r_3=0.r_{31}r_{32}r_{33}r_{34} \cdots \\
 r_4=0.r_{41}r_{42}r_{43}r_{44} \cdots \\
 \vdots
 \end{array}\qquad .$$
 Consider the real number formed by the diagonal elements
 $0.r_{11}r_{22}r_{33}\cdots $.
 Now change each of these digits, avoiding zero and nine.
 (This is necessary because reals with different
 digit sequences are identified if one of them ends with an
 infinite sequence of nines and the other with zeros, for example
 $0.0999\ldots =0.1000\ldots $.)
 The result is a real
  $r'=0.r_{1}'r_{2}'r_{3}'\cdots $
  with $r_n'\neq  r_{nn}$
 which thus differs from each of the original numbers in at least
 one (i.e., the ``diagonal'') position.
 Therefore there exists at least one real which is not contained in the
 original enumeration.

Indeed, any denumerable set of numbers is of Lebesgue measure zero.
 Let $M=\{ r_i\}$ be an  infinite point set (i.e., $M$ is a
 set of
 points $r_i$) which is denumerable and which is the subset of a dense
 set. Then, for instance, every $r_i\in M$ can be enclosed in the
 interval $$
 I(i,\delta)= [r_i-2^{-i-1}\delta
 ,
 r_i+2^{-i-1}\delta]\quad ,
 $$
 where $\delta $ may be arbitrary small (we choose $\delta$ to be
 small enough that all intervals are disjoint).
 Since $M$ is denumerable, the measure $\mu$ of these intervals can
 be summed up, yielding
$$
 \sum_i \mu( I(i,\delta))= \delta \sum_{i=1}^\infty 2^{-i}=\delta \quad
 . $$
 From $\delta \rightarrow 0$ follows $\mu (M)=0$.
 Example for denumerable point sets of reals are the {\em rationals}
${\Bbb Q}$ and
 the {\em algebraic reals.} (Algebraic reals $x$  satisfy some
 equation $ a_0x^n +a_1x^{n-1}+ \cdots  +a_n=0$,
 where $a_i\in {\Bbb N} $ and not all $a_i$'s vanish.) Consequently,
their
 measures vanish. The complement
 sets
 of {\em irrationals} ${\Bbb R} - {\Bbb Q}$ and {\em transcendentals}
 (non-algebraic reals) are thus of measure one \cite{hardy-54}.

It is easy to algorithmically prove that the computable reals
are denumerable.\footnote{
For the remainder of this paper we fix a finite
alphabet $A$ and denote by $A^*$ the set of all   strings over $A$;
$|x|$ is the length of the string $x$.
A {\it (Chaitin) computer} $C$ is  a partial recursive function carrying
strings  (on  $A$) into strings such that the domain of
$C$ is prefix-free, i.e. no admissible program can be a prefix of another
admissible program. If $C$ is a  computer, then $C(x)=y$ denotes that
$C$ terminates on program $x$ and outputs $y$. $\emptyset$ denotes
empty input or output.
$T_C$ denotes the
time complexity, i.e.  $T_C(x)$ is the running time of $C$ on the entry
$x$, if
$x$ is in the domain of $C$; $T_C(x)$ is undefined in the opposite
case.}
The range of the partial recursive function $\varphi_C$
corresponding to
an arbitrary computer $C$ can be
explicitly enumerated as follows.
Begin at step zero with an empty enumeration.
In the $n$'th step, take all legal programs
(i.e., programs which are in the domain of $C$)
of code length
$n$ and run $C$ up to time $n$; add
in
quasi-lexicographical order
all output numbers which have not
yet occurred (up to time $n-1$) in the enumeration.\footnote{
Notice that
this scenario
remains true for any (infinite) {\em dense} set such as the rationals or
the
computable numbers (cf. recursive unsolvability of the rule inference
problem \cite{gould}). The
time necessary to exactly specify an arbitrary initial value can only be
finitely bounded for discrete, finite models such as ones involving a
fundamental cut-off parameter which would essentially truncate the reals
at some final decimal place $M$ after the comma (or, equivalently, an
equivalence relation identifying all reals in the interval
$[\sum_{i=1}^M r_i,\sum_{i=1}^{M} r_i+10^{-M})$.
}


That means that if the continuum is treated as an ``urn,'' from
which the initial values are drawn, then ``almost all,'' i.e., with
probability one, such initial values are not effectively computable.
One can even prove that the stronger statement ``almost all'' elements
of the
continuum have incompressible algorithmic information; i.e., they are
Mar\-tin-L\"of\-/So\-lo\-vay/\-Chai\-tin random
\cite{chaitin,calude,vitani}.

But what does it mean to ``prove'' that ``almost all'' of them are
non-recursive; stronger: random reals?
It is obviously impossible to give just a single constructive example of
such a non-recursive real.

What does it mean ``to pull a real number --- the initial value {\it
in spe} --- out of the continuum urn?''
How could we conceive the process of selecting one real symbolizing
the initial value over the other? We need the Axiom of Choice for that.
The Axiom of Choice asserts that for any set $x$, there is a choice
function $c$ of $x$, such that
 $c(y)\in y$ for all $y\in \mbox{dom}(c)=x-\emptyset$ in the domain of
$c$. The Axiom of Choice is non-constructive,
at least for arbitrary non-constructive subsets of ${\Bbb R}$. That is,
there does
not exist any effectively computable, i.e.,  recursive, choice function
which would ``sort out'' the initial value $X_0$. Therefore, chaos
theory presupposes not only
Mar\-tin-L\"of\-/So\-lo\-vay/\-Chai\-tin random reals,
but nonconstructive choice functions.

Moreover, what type of computation is necessary to implement the
innocent-looking evolution function $f$ of the logistic equation above?
Recall that, since the initial value $X_0$ is
Mar\-tin-L\"of\-/So\-lo\-vay/\-Chai\-tin random with probability one,
its description is algorithmically incompressible and infinite.
Therefore, any ``computation'' rigorously implementing $f$ should be
capable of handling infinite input. In Bridgman's terms, this
requirement is non-operational. (Cf. Landauer \cite{landauer-onlim}
and the author \cite{svozil-93}).

The above mentioned problems of handling
Mar\-tin-L\"of\-/So\-lo\-vay/\-Chai\-tin random
objects become even more pressing if one realizes that, from the point
of view of coding theory, an algorithm and its input are
interchangeable, the difference between them being a matter of
convention:
consider a particular algorithm $p$ implemented on a computer
$C(p,s)$ with a particular input
$s$; and a second algorithm $p'$ with the empty input $\emptyset$.
Assume that the only difference between $p$ and $p'$ is that the latter
algorithm encodes the input $s$ as a constant, whereas the former reads
in (the code of) the object $s$.
Hence, $C(p,s) =C(p',\emptyset )$. Notice that, for
Mar\-tin-L\"of\-/So\-lo\-vay/\-Chai\-tin random objects $s$,
the algorithmic information content $H(p)$ remains finite, whereas
$H(p')=\infty $. In this sense, recursive functions of non-recursively
enumerable variables are equivalent to non-recursive functions.




 \subsection{Isometric miracles}

In what follows I shall shortly review non-measure preserving isometric
functions; often referred to as the ``Banach-Tarski
paradox.''
The ``mindboggling'' feature here is that an arbitrary solid
object of
${\Bbb
R}^{n\ge 3}$ can be partitioned into a {\em finite} number of pieces,
which are then rearranged by isometries, i.e., {\em distance preserving}
maps such as rotations and translations, to yield other arbitrary
solid objects.
This procedure could be the ideal basis of a perfect production belt:
produce a single prototype and ``Banach-Tarski-clone'' an arbitrary
number thereof. Or, produce an elephant from a mosquito!\footnote{
In German, ``aus einer M\"ucke einen Elefanten machen.''}

Let us briefly review another application in chaos theory. Consider all
bijections of a set
$A$.
The most systematic way of doing this is to work in the context of group
actions.
Recall that a group $G$ is said to act on $A$ if to each $g\in G$
there corresponds a bijective function from $A$ to $A$,
also denoted by $g$,
such that for any $g,h\in G$ and $x\in A$, $g(h(x))=(gh)(x)$ and
$1(x)=x$.

An {\em isometry} of a metric space is a distance-preserving bijection
of the metric space onto itself.
 A bijection $a:{\Bbb R}^n\rightarrow {\Bbb R}^n$  is called {\em
affine} if for all
$x,y\in {\Bbb R}^n$ and reals $\alpha ,\beta $ with $\alpha +\beta =1$,
$a(\alpha x+\beta y)=\alpha a(x)+\beta a(y)$.
(Note that every isometry is affine, with $a=1$.)

Let $G$ be a group acting on $A\subset X$.
$A$ is {\em $G$-paradoxical} (or, paradoxical with respect to $G$) if
there are
 $(n+m)$
pairwise disjoint subsets
$E_1,\ldots ,E_n , F_1,\ldots ,F_m $ of $A$, and
 $(n+m)$
group actions $g_1,\ldots ,g_n,h_1,\ldots ,h_m \in G$
such that $A=\bigcup_{i=1}^ng_i(E_i)=\bigcup_{j=1}^mh_j(F_j)$.
In other words, $A$ is $G$-paradoxical if it has two disjoint subsets
$\bigcup_iE_i$ and $\bigcup_jF_j$, each of which can be taken apart and
rearranged {\it via} $G$ to cover all of $A$.

Suppose $G$ acts on $X$ and $E,F\subset X$.
Then $E$ and $F$ are $G$-equidecomposible if $E$ and $F$ can each be
partitioned into the same number of $G$-congruent pieces.
Formally, $E=\bigcup_{i=1}^nE_i$ and $F=\bigcup_{i=1}^nF_i$, with
$E_i\cap E_j=F_i\cap F_j=\emptyset $ if $i<j$
and there are $g_1,\ldots ,g_n\in G$ such that for each $i$,
$g_i(E_i)=F_i$.
There is a remarkable result, usually called Banach-Tarski
paradox:\footnote{
Suppose a group $G$ acts on $A\subset X$.
Then Tarski's theorem  states that there exists a finitely-additive,
$G$-invariant
measure $\mu :{\cal P}(x)\rightarrow [0,\infty )$ with $\mu (A)=1$ if and
only if $A$ is {\em not} $G$-paradoxical.
For more results and questions see Wagon's book \cite{wagon1}.}
 {\em If $A$ and $B$ are two bounded subsets of ${\Bbb R}^n,\;n\ge 3$,
each
having
nonempty interior, then $A$ and $B$ are equidecomposible with respect to
the group of isometries.}

It can be proven that only five pieces are needed to perform ball
doubling in ${\Bbb R}^3$.
One is confronted with the ``mindboggling'' result that an arbitrary
solid body of ${\Bbb R}^n,\; n\ge 3$ can be ``cut'' into finitely many
parts,
which then may be reassembled {\it via} distance-preserving procedures
to
give an arbitrarily shaped other solid body. Pointedly stated, one could
``produce'' the sun out of a marble; or a an arbitrary number of perfect
copies from a single original (the perfect production belt!).

Obviously, the pieces needed for such types of paradoxical constructions
are not measurable. They are also not recursively enumerable and
non-constructive and thus non-operational in Bridgman's terminology.
But does this imply that ``paradoxical''
equidecompositions are physically forbidden?


 Augenstein \cite{augenstein} and Pitowsky
\cite{pitowsky} have given two possible applications of ``paradoxical''
equidecomposibility in physics.
In what follows, another, speculative, application is proposed.
It is assumed that the reader has
 a heuristic comprehension of the concept of ``attractors'' (see also
 ref. \cite{shaw,schuster1}).
An attempt towards a formal definition of an attractor can be found in
\cite{eckmann1}.
For the time being, it suffices to keep in mind that an attractor $A$ is
a point set embedded in a manifold $X$ (e.g. ${\Bbb R}^n$), with the
following essentials.

{\it R1)}
all points $x\in A$ are {\em cumulation points} of $f$;

{\it R2)}
{\em topological undecomposibility:}
for arbitrary $x,y\in A$ and arbitrary $\; {\rm diam}(A)\ge \epsilon >0$
there must be
chains $x=x_0,x_1,\ldots ,x_n=y$ and $y=y_0,y_1,\ldots ,y_m=x$ such that
$\; {\rm dist}(x_i,f^{(g(i))}(x_{i-1}))<\epsilon \;$ and
$\;{\rm dist}(y_i,f^{(g'(i))}(y_{i-1}))<\epsilon \;$ with $\;g(i),
g'(i)\ge 1\;$ for all $i=1,2,\ldots ,n$.
This formal condition boils down to the requirement that with respect to
the function $f$, $A$ cannot be decomposed into more ``elementary''
attractors which are subsets of $A$.

The following condition of strangeness shall be imposed upon attractors.

{\it S)}
$A_S$ is {\em strange} if to every $\delta \le \mbox{diam} (A_S)$
and $\epsilon < \delta $ there exists a $N(\epsilon ,\delta )$ such that
for arbitrary two points $x,y\in A_S$, $\mbox{dist}(x,y)<\epsilon $,
$\mbox{dist} (f^{(N)}(x),f^{(N)}(y))\ge \delta $.

The above condition guarantees that, heuristically speaking, arbitrarily
close points become arbitrarily separated in time.
I shall restrict further considerations to dynamical systems $(f,X)$ for
which
the basin of attraction (i.e., the set of initial points from which the
flow is attracted)
is the entire embedding space $X$.


There are strong relationships between the property of strangeness
and Tarski's theorem, which shall be presented next.
Consider the group of automorphisms $S$ of $A(X,f)$; i.e., the
bijections under which $A(X,f)$ is invariant.
Automorphisms can be interpreted
 as  {\em symmetries} of $(X,f)$.
 For attractors, the flow is a symmetry, i.e., $f^{(i)}\in S$.
Any subset $A_1$ of a strange attractor $A_S$
with nonzero diameter $\mbox{diam}(A_1)>0$ can be completed to $A_S$ by
application of some $f^{(i)}\in S$ such that $f^{(i)}(A_1)=A_S$. In this
 sense,
$A_1$ is physically equivalent to $A_S$.
Conversely, if $A$ is not strange, this property does not hold.
 In terms of paradoxical decompositions,
 the property of strangeness can then be alternatively defined {\it
 via} paradoxical equidecompositions.

 {\it S')}
 $A_S$ is strange if it is paradoxical with respect to the time flow
 $f$.

It then follows from
Tarski's theorem \cite{f1} that
there is no finitely additive measure on strange attractors
which is invariant with respect to the symmetries (invariants)
of motion. For regular attractors such a measure exists.


      The apparent question is which type of attractors are
 equidecomposible with respect to which kind of group actions?  To put
 it in more physical terms: Suppose there exist two dynamical systems,
 represented by $(X,f_1)$ and $(X,f_2)$, with associated attractors
 $A(f_1)$ and $A(f_2)$, respectively (the embedding space $X$ is
 unchanged, therefore we drop it as argument).  Do there exist physical
 (parameter or other) changes corresponding to group actions
 $G:A(f_1)\mapsto A(f_2)$ ?  Indeed, this is the case for period
 doubling solutions.  There, $f_1$ and $f_2$ are nonlinear functions,
 which are in general not distance preserving.  Along these lines, the
 notion of equidecomposibility of attractors could become a powerful
 tool for a systematic investigation of parameter and symmetry changes.



According to the
 Banach-Tarski paradox,
this would allow the occurrence of strange attractors (``chaotic
motion'') even for
 distance preserving, linear time flows.
 This kind of paradoxical decomposition requires the application of the
Axiom of Choice (cf. the brief discussion above).

Thus it is not completely speculative to suggest testing the Axiom of
Choice {\it via} the reconstruction of strange (chaotic) attractors by
physical timeseries from distance preserving flows in ${\Bbb R}^n$,
$n\ge 3$.


 \subsection{Oracle computing}

Zeno's paradoxes \cite{zeno}, formulated around fifth century B.C., will
probably remain with us forever;
very much like an eternal Zen {\it koan} presented to us by this great
greek mathematical master(s) at the beginning of scientific thought.
It is the author's believe that neither Weierstra\ss 's ``Epsilontik''
nor modern approaches such as nonstandard analysis \cite{ns-an} have
contributed substantially to the ``mindboggling'' feature that
(in Simplicius' interpretation of Zenos's paradox of Achilles and the
Tortoise, quoted from
 \cite{zeno}, p. 45) if space is infinitely divisible,
 and if
 {\em
 ``$\cdots$ there is motion, it is possible in a finite time to
 traverse an infinite number of positions, making an infinite
 number of contacts one by one.'' }


I shall review here a recursion theoretic version of Zeno's paradox,
which has been discussed by Weyl
 \cite{weyl:49},
Gr\"unbaum
(\cite{gruenbaum:74}, p. 630), Thomson \cite{thomson}, Benacerraf
\cite{benacerraf}, and more recently by Pitowsky \cite{pitowsky-90},
Hogarth
\cite{hogarth}, Earman \& Norton \cite{earman} and the author
\cite{svozil-93,svozil-95}.

Continuum theory, in fact any dense set, in principle allows the
construction of ``infinity machines,''
which could serve as oracles for the halting
problem.
Their construction closely follows Zeno's paradox of Achilles
and the Tortoise by squeezing the time it takes for successive
steps of computation $\tau$ with geometric progression:
\unitlength=0.5mm
\special{em:linewidth 0.4pt}
\linethickness{0.4pt}
\begin{picture}(100.00,2.10)
\put(0.00,0.00){\line(1,0){100.00}}
\put(0.00,5.00){\makebox(0,0)[cc]{0}}
\put(0.00,-2.00){\line(0,1){4.00}}
\put(50.00,5.00){\makebox(0,0)[cc]{1}}
\put(50.00,-2.00){\line(0,1){4.00}}
\put(75.00,5.00){\makebox(0,0)[cc]{2}}
\put(75.00,-2.00){\line(0,1){4.00}}
\put(88.00,5.00){\makebox(0,0)[cc]{3}}
\put(88.00,-2.00){\line(0,1){4.00}}
\put(94.00,5.00){\makebox(0,0)[cc]{4}}
\put(94.00,-2.00){\line(0,1){4.00}}
\put(97.00,5.00){\makebox(0,0)[lc]{$\cdots$}}
\put(97.00,-2.00){\line(0,1){4.00}}
\put(99.00,-2.00){\line(0,1){4.00}}
\put(100.00,-2.00){\line(0,1){4.00}}
\put(99.67,-2.00){\line(0,1){4.00}}
\put(99.83,-2.00){\line(0,1){4.00}}
\put(99.50,-2.00){\line(0,1){4.00}}
\end{picture}
$\quad $
That is,
the time necessary for the $n$'th step becomes $\tau (n)=k^{n}$,
$0<k<1$. The limit of infinite computation is then reached in finite
physical time
$ \lim_{N\rightarrow \infty}\sum_{n=1}^N \tau{(n)}=
  \lim_{N\rightarrow \infty}\sum_{n=1}^N  k^n=
    1/(1-k)$.

It can be shown by a
diagonalization argument that the
application of
such oracle subroutines
would result in a paradox in classical physics
(cf. \cite{svozil-93}, p. 24, 114).
The paradox is constructed in the context of the halting problem.
It is formed in a similar manner as Cantor's diagonalization
argument.
Consider an arbitrary algorithm $B(x)$ whose input is a string of
symbols
$x$.
 Assume that there exists (wrong) a ``halting algorithm'' ${\tt HALT}$
 which  is
 able to decide whether $B$ terminates on $x$ or not.

 Using  ${\tt HALT}(B(x))$ we shall construct another
 deterministic computing agent
$A$, which
 has as input any effective program $B$ and which proceeds as follows:
 Upon reading the program $B$ as input, $A$ makes a copy of it.
 This  can be readily achieved, since
 the program $B$ is presented to $A$  in some
 encoded form $\# (B)$, i.e., as a string of symbols. In the next
 step, the agent uses the
 code $\# (B)$ as input string for $B$ itself; i.e., $A$  forms
 $B(\#(B))$, henceforth denoted by $B(B)$. The agent now hands
 $B(B)$ over to its
 subroutine ${\tt HALT}$.
 Then, $A$ proceeds as follows:
  if ${\tt HALT}(B(B))$ decides that $B(B)$
 halts, then the agent
 $A$ does not halt;
this can for instance be realized by an infinite {\tt
 DO}-loop;
  if ${\tt HALT}(B(B))$ decides that $B(B)$
 does {\em not} halt, then
 $A$ halts.

 We shall now confront the agent $A$ with a paradoxical task by
 choosing $A$'s own code as input for itself.
---
Notice that $B$ is arbitrary and has not yet been specified and we are
totally justified to do that:
The deterministic agent $A$ is representable by an algorithm with code
$\# (A)$. Therefore, we are free to substitute $A$ for $B$.

Assume that classically $A$ is restricted to classical bits of
information.
Then, whenever
 $A(A)$ halts,  ${\tt HALT}(A(A))$  forces
 $A(A)$ not to halt.
Conversely, whenever $A(A)$ does not halt, then ${\tt HALT}(A(A))$
 steers $A(A)$
 into the halting mode. In both cases one arrives at a
complete contradiction.

Therefore, at least in this example, too powerful physical models (of
computation) are inconsistent.
It almost goes without saying that the concept of infinity
machines is neither constructive nor operational in the current physical
framework.

\subsection{Weak solutions}

Consider an ordinary differential equation (of one variable $t$) of the
form
$
Lx= \sum_{n=0}^\infty c_n(t) {d^n x/dt^n}= \tau (t)
 $, where $\tau (t)$ is an arbitrary known distribution [e.g., $\tau
(t)=\delta (t)$].
$x$ is a {\em weak solution}
if $Lx= \tau (t)$ is satisfied as a distribution, yet
$x$ is not sufficiently smooth so that
the operations in $L$ (i.e., differentiations) cannot be performed.
How relevant are weak solutions for physical applications?

In
electrodynamics, for instance, point charges are modeled by Dirac delta
functions $\delta$. The wave equation can therefore give rise to weak,
discontinuous solutions. Are discontinuities mere
theoretical abstractions, which indicate ``sharp'' changes of the
physical parameter, or should they be taken more seriously? These
questions connect to the quantum field theoretic program of
renormalization and regularization.

\section{The alternatives}

The above speculations suggest that the theoretical physicist is
occasionally confronted with set theoretical consequences which cannot
be straightforwardly abandoned as ``artificial'' or ``irrelevant.'' They
bear important, even technological, consequences. In what follows, two
extreme alternatives will be discussed to cope with them. (No claim of
completeness is made.)


\subsection{Abandon non-operational entities altogether}

In view of the problems of Cantorian, transfinite set theory, one may
take the radical step and abolish non-constructive and non-operational
objects altogether.
This was Bridgman's goal.  Related epistemological approaches had been
anticipated by
Boskovich, and have more recently been put forward by
Zeilinger and Svozil
\cite{zeilinger-priv,zeilinger-svozil}, among others.
R\"ossler's \cite{roessler} endo-/exophysics approach and
the
author's \cite{svozil} intrinsic-extrinsic distinction
differ only insofar, as the operational mode of perception is
contrasted with a hierarchical mode of perception of an observer outside
of the system.

It should be noted that
operationalism is not directed primarily towards the elimination of
antinomies.
The
elimination of metaphysical concepts, such as absolute space and time,
and their substitution by physically
operationalizable concepts, is at the core of operationalism, and more
specifically, of Einstein's theory of relativity (cf. \cite{bridgman},
p. 103),
{\em ``$\ldots$ the
meaning of length is to be sought in those operations by which the
length of physical objects is determined, and the meaning of
simultaneity is sought in those physical operations by which it is
determined whether two physical events are simultaneous or not.''}
More recently, it has been applied for a definition of the dimension of
space-time
\cite{zeilinger-svozil,svozil86}, for
complementarity  \cite{svozil-93,schaller-svozil}
and undecidability
\cite{svozil-93}.

The elimination of set theoretical antinomies, as discussed by Bridgman,
is a bonus of, and a clear argument for the approach.
Indeed, it is quite justifiable to consider operationalism
as the consequential persuasion of Descartes' \cite{descartes} sketch of
the scientific method.
Its goal is the substitution of metaphysical concepts by purely physical
correspondents.

The drawback of operationalism might lie in its too rigid, dogmatic
interpretation. Whatever is operational depends on the particular period
of scientific investigation. Therefore, the entities allowed by
operationalism constantly change with time and are no fixed kanon. To
kanonize them means to cripple scientific progress.


To give an example:  in ancient Greece, supersonic air travel or
radio-wave transmission were not feasible; therefore, any methods
employing these operations to test whether the earth is ball-shaped were
not allowed. But that, of course, does not imply that
supersonic air travel or
radio-wave transmission is impossible in principle!\footnote{
Every time claims that the means at hand are final.
Nowadays, for example, faster-than-light travel or superluminal
signalling is not feasible. But does that mean that
faster-than-light travel or superluminal signalling is impossible?}

Nevertheless, one may quite justifiable argue that, if executed
carefully,  the necessity to operationalize physics will push science
forward.


\subsection{``Go-Go'' science}

Another possibility would be not to care about set theory at all and
pursue a ``Go-Go'' strategy. The advantage of such a method of
progression
would be its open-mindedness. A disadvantage would be the vulnerability
to unreliable conclusions and claims, which are either incorrect or
have no counterpart in physics. \footnote{
See Jaffe and Quinn \cite{jaffe-quinn} for a discussion of a related
aspect.}

\subsection{Synthesis}

In view of the advantages and drawbacks of the two extreme positions
outlined above, an attitude of
``suspended attention''
(a term borrowed from psychoanalysis) seems most promising.

This means that the theorist should be ``on the lookout'' for
innovative, new formal objects, while
not loosing sight of operational tests and
practical implementations of such findings.

\section{Epilogue: mathematical {\it versus} physical universe}

From the time of ancient
civilizations until today,
the development of mathematics seems to be strongly
connected to the advancements in the physical sciences.
Mathematical concepts were introduced
on the demand to explain natural phenomena. Conversely, physical
theories were created with whatever mathematical formalism was
available.
This observation might suggest a rather obvious explanation for
 {\em ``the unreasonable effectiveness of mathematics in the natural
 sciences''} (cf. Wigner \cite{wigner} and Einstein
\cite{einstein-math}, among others). Yet, there remains
an amazement that the
 mathematical belief system can be implemented at all! There
seems no {\it a priori} reason for this remarkable coincidence.

One of the most radical metaphysical speculations concerning the
interrelation between mathematics and physics is that they are the same,
that they are equivalent. In other words: the only
``reasonable'' mathematical universe is the physical universe we are
living in!
As a consequence, every mathematical statement would translate into
physics and
{\em vice versa.}

As is suggested by their allegedly esoteric, almost ``occult,''
practice of mathematical knowledge, the
Pythagoreans might have been the first to believe in this equivalence
(cf. Aristotle's {\sl Metaphysics}, Book I, 5; Book XIII, 6; translated
into English
\cite{aristotle-met}): {\em ``
$\ldots$--since, then, all other things seemed in their whole nature
to be modeled on numbers, and numbers seemed to be the first things in
the whole of nature, they [[the Pythagoreans]] supposed the elements of
numbers to be the
elements of all things, and the whole heaven to be a musical scale and
a number.''
``And the Pythagoreans, also, believe
in one kind of number--the mathematical; only they say it is not
separate but sensible substances are formed out of it. For they
construct the whole universe out of numbers $ \ldots$''}\footnote{
Aristotle proceeds, {\em ``$\ldots$--only not numbers
consisting of abstract units; they suppose the units to have spatial
magnitude. But how the first 1 was constructed so as to have
magnitude, they seem unable to say.''}}

It has to be admitted that, from a contemporary point of view, such an
equivalence between mathematics and physics appears implausible and
excessively speculative.
Even in the framework of axiomatic set theory, there seem to be many
(possibly an
infinite number of) conceivable mathematical universes but only one
physical universe.\footnote{
No attempt is made here to review the many-worlds interpretation of
quantum mechanic, or other exotic speculations such as
parallel universes in cosmology.}
For example, Zermelo-Fraenkel
set theory can be developed with or without the axiom of choice, with or
without the continuum hypothesis. Axioms for Euclidean as well as for
non-Euclidean geometries have been given.

Are there criteria such as ``reasonableness'' which may single out one
mathematical universe from the others? That turns out to be
difficult. Let
us for instance agree that the least requirement one should impose upon
a
``reasonable'' mathematical formalism is its {\em consistency.} As
appealing as this identification sounds, it is of no practical help. As
has been pointed out by G\"odel
\cite{godel1}, for strong enough mathematical formalisms\footnote{
Here, only strong enough formalisms, in which arithmetic and
universal computation can be implemented, will be considered. Weaker
mathematical universes would
be monotonous.}
consistency is no constructive notion. For this reason, mathematicians
do not know whether axiomatic Zermelo-Fraenkel set theory is
consistent.\footnote{
As has been noticed before, naive (i.e., non-axiomatic) approaches
are unreliable and plagued by inconsistencies.}

Let us finally take the opposite standpoint and reject the assumption of
an equivalence
between mathematical and physical entities.
Even then, there appears to be a straightforward coincidence between
mathematics and ``virtual'' physics
\cite{svo-95}:
 Any
axiomatizable mathematical formalism is
constructive {\it per definition}, since any derivation within a formal
system is equivalent to an effective computation. Therefore, any such
mathematical model can be implemented on a universal computer. The
resulting universe can then be investigated by means and methods which
are operational from within that universe. --- A metaphysical
speculation which brings us back to Bridgman's perception of Cantorian
set theory, the greatest attempt so far to reach out and encompass all
of (meta-)physics into the domain of the formal sciences.


\begin{thebibliography}{99}

 \bibitem{einstein-math}
 A. Einstein, {\sl Sitzungsberichte der Preu\ss ischen Akademie der
 Wissenschaften} {\bf 1}, 123 (1921); reprinted in A. Einstein, {\sl
Mein Weltbild} (Ullstein, 1988); p. 119-120.


 \bibitem{cantor-set}
 G. Cantor,
{\sl Beitr\"age zur Begr\"undung der transfiniten Mengenlehre},
{\sl Math. Annalen} {\bf 46}, 481-512 (1895);
{\it ibid.} {\bf 49}, 207-246 (1897); reprinted in
 \cite{cantor}.

\bibitem{hil-26}
D. Hilbert,
{\sl \"Uber das Unendliche,}
{\sl Math. Annalen} {\bf 95}, 161-190 (1926).

 \bibitem{cantor}
 G. Cantor, {\sl Gesammelte Abhandlungen} , eds. A. Fraenkel and E.
 Zermelo (Springer, Berlin, 1932).

\bibitem{boos}
W. Boos, {\it Consistency and Konsistenz},
{\sl Erkenntnis} {\bf 26}, 1-43 (1987).

 \bibitem{zer-fr}
A. A. Fraenkel, Y. Bar-Hillel and A. Levy,
{\sl Foundations of Set Theory, Second Revised Edition}
(North Holland, Amsterdam, 1984).

\bibitem{g-brower}
H. Wang,
{\sl Reflections on Kurt G\"odel}
(MIT Press, Cambridge, MA, 1991).

\bibitem{godel1}
K. G\"odel, {\sl Monatshefte f\"ur Mathematik und Physik}
{\bf 38}, 173 (1931).

\bibitem{bridges-richman}
D. Bridges and F. Richman,
{\sl Varieties of Constructive Mathematics}
(Cambridge University Press, Cambridge, 1987).

 \bibitem{bishop-bridges}
 E. Bishop and D. S. Bridges, {\sl Constructive Analysis} (Springer,
 Berlin, 1985).


\bibitem{bridgman}
P. W. Bridgman,
{\sl A Physicists Second Reaction to
Mengenlehre,}
{\sl Scripta Mathematica} {\bf 2}, 101-117; 224-234 (1934).


\bibitem{landauer-67}
R. Landauer,
{\sl Wanted: a physically possible theory of physics,} in
{\sl IEEE Spectrum} {\bf 4}, 105-109 (1967).

\bibitem{landauer-87}
R. Landauer,
{\sl Fundamental Physical Limitations of the Computational Process; an
Informal Commentary,} in
{\sl Cybernetics Machine Group Newssheet} 1/1/87.

\bibitem{landauer-95}
R. Landauer,
{\sl Advertisement For a Paper I Like,}
in {\sl On Limits,} ed. by J. L. Casti and J. F. Traub
(Santa Fe Institute Report 94-10-056, Santa Fe, NM, 1994), p.39.

\bibitem{bridgman-logic}
P. W. Bridgman,
{\sl The Logic of Modern Physics}
(New York, 1927).

\bibitem{bridgman-theory}
P. W. Bridgman,
{\sl The Nature of Physical Theory}
(Princeton, 1936).

\bibitem{bridgman-reflextions}
P. W. Bridgman,
{\sl Reflections of a Physicist}
(Philosophical Library, New York, 1950).

\bibitem{bridgman-nature}
P. W. Bridgman,
{\sl The Nature of Some of Our Physical Concepts}
(Philosophical Library, New York, 1952).


\bibitem{gandy1}
R. O. Gandy.  Limitations to mathematical knowledge,
in  D. van Dalen, D. Lascar, and
J. Smiley (eds.). {\it Logic Colloquium '82},  North Holland, Amsterdam,
1982, 129-146.

\bibitem{gandy2}
R. O. Gandy.  Church's Thesis and principles for mechanics,
in J. Barwise, H. J. Kreisler and
K. Kunen  (eds.). {\it  The Kleene Symposium}, North Holland, Amsterdam,
1980, 123-148.


  \bibitem{mundici}
D. Mundici, {\sl Irreversibility, uncertainty,
relativity and
computer limitations,} {\it Il Nuovo Cimento} {\bf 61}, 297-305 (1981).

\bibitem{landauer-onlim}
R. Landauer,
{\sl John Casti's page on ``Finiteness and Real-World Limits''},
in {\sl On Limits,} ed. by J. L. Casti and J. F. Traub
(Santa Fe Institute Report 94-10-056, Santa Fe, NM, 1994), p.34.

\bibitem{casti-onlim}
J. L. Casti,
{\sl Finiteness and Real-World Limits},
in {\sl On Limits,} ed. by J. L. Casti and J. F. Traub
(Santa Fe Institute Report 94-10-056, Santa Fe, NM, 1994), p.34.

\bibitem{specker}
E. Specker, {\sl Selecta} (Birkh\"auser Verlag, Basel, 1990).

\bibitem{wang}  P. S. Wang, The undecidability of the existence of
zeros of real elementary functions, {\it J. Assoc. Comput. Mach.}
{\bf 21}
(1974), 586-589.

 \bibitem{kreisel}
G. Kreisel, A notion of mechanistic theory, {\it Synthese} 29(1974),
11-26.

\bibitem{ds}  D. \c Stef\u anescu, {\it Mathematical Models in Physics},
 University of Bucharest Press, 1984. (in Romanian)

\bibitem{pour-el}
 M. Pour-El, I. Richards, {\em
Computability in Analysis and Physics},
 Springer-Verlag, Berlin,  1989.

 \bibitem{penrose}
 R. Penrose. {\em The Emperor's New Mind. Concerning Computers, Minds, and
 the Laws of Physics}, Vintage,  London, 1990. (First published by
Oxford University Press, Oxford, 1989.)

\bibitem{bridges1}
D. S. Bridges, Constructive mathematics and unbounded operators---a
reply to Hellman,
{\it J. Phil. Logic}. (in press)

\bibitem{calude-sv}
C. Calude,
D. I. Campbell,
 K. Svozil and
D. \c{S}tef\u anescu,
{\sl Strong Determinism vs.  Computability},
{\it  e-print quant-ph/9412004}.

\bibitem{chaitin}
G. J. Chaitin, {\sl Information, Randomness and Incompleteness, Second
edition}
(World Scientific, Singapore, 1987, 1990);
{\sl Algorithmic Information Theory}
(Cambridge University Press, Cambridge, 1987);
{\sl Information-Theoretic Incompleteness}
(World Scientific, Singapore, 1992).

\bibitem{calude}
 C. Calude,
{\sl Information and Randomness --- An Algorithmic Perspective}
(Springer, Berlin,
1994).

 \bibitem{vitani}
 M. Li and P. M. B. Vit\'{a}nyi, {\sl Kolmogorov Complexity and its
 Applications}, in {\sl Handbook of Theoretical Computer Sciences,
  Algorithms and Complexity, Volume A}
 (Elsevier, Amsterdam and MIT Press, Cambridge, MA., 1990).


 \bibitem{shaw}
 R. Shaw, {\sl Z. Naturforsch.} {\bf 36a}, 80 (1981).

\bibitem{smale}
St. Smale, {\sl The Mathematics of Time} (Springer, New York, 1980).

 \bibitem{ford}
 J. Ford, {\sl Physics Today} {\bf 40} (4), 1 (April 1983).

\bibitem{chaitin-chatelin}
F. Chaitin-Chatelin and V. Fraysse\'{e},
{\sl Lectures on Finite Precision Computations}
(SIAM, 1955).

\bibitem{goldstein}
H. Goldstein,
{\sl Classical Mechanics, Second Edition}
(Addison-Wesley Publishing Company, Reading, MA, 1980).

 \bibitem{hardy-54}
 G. H. Hardy and E. M. Wright, {\sl An Introduction to the Theory of
 Numbers} (3rd edition) (Cambridge University Press, London, 1954).

 \bibitem{gould}
 E. M. Gold, {\sl Information and Control} {\bf 10}, 447 (1967).

\bibitem{wagon1}
St. Wagon, {\sl The Banach--Tarski paradox (2nd printing)}
(Cambridge University Press, Cambridge 1986)

 \bibitem{augenstein}
 B. W. Augenstein, {\sl International Journal of Theoretical Physics}
 {\bf 23}, 1197 (1984); Conceiving Nature---Discovering Reality, {\sl
Journal of Scientific Exploration} {\bf 8}, 279-282 (1994).

 \bibitem{pitowsky}
 I. Pitowsky, {\sl Phys.
 Rev. Lett.} {\bf 48}, 1299 (1982); {\sl Phys. Rev.} {\bf D27}, 2316
 (1983); N. D. Mermin, {\sl Phys. Rev. Lett.}
 {\bf 49}, 1214 (1982); A. L. Macdonald, {\it Ibid.}, 1215 (1982); I.
 Pitowsky, {\it Ibid.}, 1216 (1982);
 {\sl Quantum Probability --- Quantum Logic} (Springer,
 Berlin, 1989).

\bibitem{schuster1}
H. G. Schuster,
{\sl Deterministic Chaos}
(Physik Verlag, Weinheim 1984).

\bibitem{eckmann1}
J.--P. Eckmann and D. Ruelle,
{\sl Rev. Mod. Phys.} {\bf 57}, 617 (1985)


 \bibitem{f1}
{\sl Tarski's theorem} can be stated as follows \cite{wagon1}:
Suppose a group $G$ acts on $A\subset X$.
Then there exists a finitely--additive, $G$--invariant
measure $\mu :{\cal P}(x)\mapsto [0,\infty )$ with $\mu (A)=1$ if and
only if $A$ is {\em not} $G$--paradoxical.

 \bibitem{zeno}
 H. D. P. Lee, {\sl Zeno of Elea} (Cambridge University Press,
 Cambridge, 1936; reprinted by Adolf M. Hakkert, Amsterdam, 1967).

\bibitem{ns-an}
W. McLaughlin and S. L. Miller, {\it An epistemological use of
nonstandard analysis to answer Zeno's objections against motion},
{\sl Synthese} {\bf 92}, 371-384 (1992).

\bibitem{weyl:49}
H. Weyl,
{\sl Philosophy of Mathematics and Natural Science}
(Princeton University Press, Princeton, 1949).

\bibitem{gruenbaum:74}
A. Gr\"unbaum,
{\sl Modern Science and Zeno's paradoxes, Second edition}
(Allen and Unwin, London, 1968);
{\sl Philosophical Problems of Space of Time, Second, enlarged edition}
(D. Reidel, Dordrecht, 1973).

\bibitem{thomson}
J. F. Thomson, {\sl Tasks and super-tasks},
{\sl Analysis} {\bf 15}, 1-13 (1954).

\bibitem{benacerraf}
P. Benacerraf, {\sl Tasks, super-tasks, and the modern eleatics},
{\sl The Journal of Philosophy} {\bf 59}, 765-784 (1962).

\bibitem{pitowsky-90}
I. Pitowsky,
{\sl The physical Church-Turing thesis and physical complexity theory},
{\sl Iyyun, A Jerusalem Philosophical Quarterly} {\bf 39}, 81-99 (1990).


\bibitem{hogarth}
M. Hogarth,
{\sl Non-Turing computers and non-turing computability},
{\sl PSA 1994}, {\bf 1}, 126-138 (1994).

\bibitem{earman}
J. Earman and J. D. Norton,
{\sl Forever is a day: supertasks in Pitowsky and Malament-Hogarth
spacetimes},
{\sl Philosophy of Science} {\bf 60}, 22-42 (1993).


\bibitem{svozil-93}
K. Svozil,  {\sl Randomness and Undecidability in Physics}
(World Scientific, Singapore, 1993).

\bibitem{svozil-95}
 K. Svozil,
{\sl On the computational power of physical systems,
undecidability, the consistency of phenomena and
 the practical uses of paradoxes},
in {\sl Fundamental Problems in
Quantum Theory: A Conference Held in Honor of Professor John A.
Wheeler}, ed. by D. M. Greenberger and A. Zeilinger,  {\sl Annals of the
New York Academy of Sciences} {\bf 755}, 834-842 (1995).

\bibitem{zeilinger-priv}
A. Zeilinger, {\it private communication}.

 \bibitem{roessler}
 O. E. R\"ossler, {\sl Endophysics}, in {\sl Real Brains, Artificial
 Minds}, ed. by J. L. Casti and A. Karlquist (North-Holland, New
 York, 1987), p. 25;
{\sl Endophysics, Die Welt des inneren Beobachters},
 ed. by P. Weibel (Merwe Verlag, Berlin, 1992).

\bibitem{svozil}
 K. Svozil,
 {\sl On the setting of scales for space and time in arbitrary
 quantized media}  (Lawrence Berkeley Laboratory preprint
 LBL-16097, May 1983);
{\sl Europhysics Letters} {\bf 2}, 83 (1986);
 {\sl Il Nuovo Cimento} {\bf 96B}, 127 (1986); see also
\cite{svozil-93}.

\bibitem{zeilinger-svozil}
A. Zeilinger and K. Svozil, {\sl Phys. Rev. Lett.}
{\bf 54}, 2553 (1985);
 K. Svozil  and A. Zeilinger,
{\sl International Journal of Modern Physics} {\bf A1}, 971 (1986);
K. Svozil and A. Zeilinger,
{\sl Physica Scripta} {\bf T21}, 122 (1988).

\bibitem{svozil86}
 K. Svozil,
{\sl J. Phys.} {\bf A19}, L1125 (1986);
K. Svozil, {\sl J. Phys.} {\bf A20}, 3861 (1987).

\bibitem{schaller-svozil}
 M. Schaller and K. Svozil, {\sl Il Nuovo Cimento} {\bf 109 B}, 167
(1994); {\sl International Journal of Theoretical Physics}, {\it in
press.}

\bibitem{descartes}
R. Descartes, {\sl Discours de la m\`{e}thode pour bien conduire sa
raison et chercher la v\'{e}rit\'{e} dans les sciences}
[English translation: Discourse on the method of rightly conducting the
reason, and seeking truth in the sciences] (1637).

\bibitem{jaffe-quinn}
A. Jaffe and F. Quinn,
{\sl ``Theoretical mathematics'': toward a cultural synthesis of
mathematics and theoretical physics},
{\sl Bulletin (New Series) of the American Mathematical Society}
{\bf 29}, 1-13 (1993).

 \bibitem{wigner}
 E. P. Wigner,
 {\sl ``The unreasonable effectiveness of mathematics in the natural
 sciences''}, Richard Courant Lecture delivered at New York University,
 May 11, 1959 and published in {\it
 Communications on Pure and Applied Mathematics} {\bf 13}, 1 (1960).

\bibitem{aristotle-met}
Aristotle, {\sl Metaphysics}, around 350 B.C.,
translated by W. D. Ross, e-print
{\tt http://the-tech.mit.edu/Classics/Aristotle/metaphysics.txt}.

\bibitem{svo-95}
K. Svozil,
{\sl How real are virtual realities,
how virtual is reality?
The constructive re-interpretation of physical undecidability}, in {\sl
Complexity}, {\it in press}.

\end{thebibliography}
\newpage
 \tableofcontents

\end{document}
