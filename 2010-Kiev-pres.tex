%\documentclass[pra,showpacs,showkeys,amsfonts,amsmath,twocolumn,handou]{revtex4}
\documentclass[amsmath,red,table,sans,handout]{beamer}
%\documentclass[pra,showpacs,showkeys,amsfonts]{revtex4}
%\documentclass[pra,showpacs,showkeys,amsfonts]{revtex4}
\usepackage[T1]{fontenc}
%%\usepackage{beamerthemeshadow}
\usepackage[headheight=1pt,footheight=10pt]{beamerthemeboxes}
\addfootboxtemplate{\color{structure!80}}{\color{white}\tiny \hfill Karl Svozil (TU Vienna)\hfill}
\addfootboxtemplate{\color{structure!65}}{\color{white}\tiny \hfill Refractive Index in Exotic Vacua\hfill}
\addfootboxtemplate{\color{structure!50}}{\color{white}\tiny \hfill May 27, 2010, Kyiv, Ukraine\hfill}
%\usepackage[dark]{beamerthemesidebar}
%\usepackage[headheight=24pt,footheight=12pt]{beamerthemesplit}
%\usepackage{beamerthemesplit}
%\usepackage[bar]{beamerthemetree}
\usepackage{graphicx}
\usepackage{pgf}
%\usepackage[usenames]{color}
%\newcommand{\Red}{\color{Red}}  %(VERY-Approx.PANTONE-RED)
%\newcommand{\Green}{\color{Green}}  %(VERY-Approx.PANTONE-GREEN)

%\RequirePackage[german]{babel}
%\selectlanguage{german}
%\RequirePackage[isolatin]{inputenc}

\pgfdeclareimage[height=0.5cm]{logo}{tu-logo}
\logo{\pgfuseimage{logo}}
\beamertemplatetriangleitem
%\beamertemplateballitem

\beamerboxesdeclarecolorscheme{alert}{red}{red!15!averagebackgroundcolor}
\beamerboxesdeclarecolorscheme{alert2}{orange}{orange!15!averagebackgroundcolor}
%\begin{beamerboxesrounded}[scheme=alert,shadow=true]{}
%\end{beamerboxesrounded}

%\beamersetaveragebackground{green!10}

%\beamertemplatecircleminiframe

\usepackage{feynmf}             %Package for feynman diagrams.

\begin{document}

\title{\bf \textcolor{red}{Refractive Index in Exotic Vacua}}
%\subtitle{Naturwissenschaftlich-Humanisticher Tag am BG 19\\Weltbild und Wissenschaft\\http://tph.tuwien.ac.at/\~{}svozil/publ/2005-BG18-pres.pdf}
\subtitle{\textcolor{orange!60}{\small http://tph.tuwien.ac.at/$\sim$svozil/publ/2010-Kiev-pres.pdf}\\
\textcolor{gray!60}{\footnotesize http://arxiv.org/abs/1003.1238 \& http://arxiv.org/abs/physics/0210091}
}
\author{Karl Svozil}
\institute{Institut f\"ur Theoretische Physik, Vienna University of Technology, \\
Wiedner Hauptstra\ss e 8-10/136, A-1040 Vienna, Austria\\
svozil@tuwien.ac.at
%{\tiny Disclaimer: Die hier vertretenen Meinungen des Autors verstehen sich als Diskussionsbeitr�ge und decken sich nicht notwendigerweise mit den Positionen der Technischen Universit�t Wien oder deren Vertreter.}
}
\date{May 27, 2010, Kyiv, Ukraine}
\maketitle



%\frame{
%\frametitle{Contents}
%\tableofcontents
%}


\section{My research interests}


\frame{
\frametitle{My research interests}
\begin{itemize}
\item<1->
Quantum foundations, in particular quantum contextuality
\item<1->
Quantum information and communication theory
\item<1->
Quantum random number generators
\item<1->
Physical unknowables (G{\"o}del-Turing types)
\item<1->
Quantum Field Theory, in particular quantum vacua, and questions related to convergence of perturbation series
\end{itemize}
}



\section{Quantum vacua}
\frame{
\frametitle{Some speculations about quantum vacua}
{
\begin{itemize}

\item<1->
Prof. Urban (Graz) once stated
{\it ``The quantum vacuum feels like a supermarket (Kastner\&\"Ohler in Graz, Magelan(?) in Kyiv) on Saturdays!''}

\item<1->
Speculation 0: Different vacua result in different (renormalized) radiative corrections to masses, $g-2$, $n$.
\item<1->
Speculation 1: Changes of index of refraction bring about (inverse) changes of the velocity of light in vacua.
\item<1->
Super-Speculation 2: ``Supercavitation'' in the ``quantum ether'' [cf. Paul A. M. Dirac. Is there an aether? Nature {\bf 168}, 906--907 (1951)],
as a possibility for faster-than-light propagation?
\end{itemize}
}
}



\section{Mass, $g-2$ between parallel conducting plates}
\frame{
\frametitle{Mass, $g-2$ between parallel conducting plates}
K. Svozil, ``Mass and anomalous magnetic moment of an electron between two conducting parallel plates'', Physical Review Letters 54, 742-744 (1985) [DOI:10.1103/PhysRevLett.54.742].\\
M. Kreuzer and K. Svozil, ``QED between plates: mass and anomalous magnetic moment of an electron'', Physical Review D 34, 1429-1437 (1986) [DOI:10.1103/PhysRevD.34.1429].
$\;$\\
$\;$\\
Discretization of electromagnetic field modes between parallel plates

$${\Delta m} = -{\alpha \over 2a}\left[ \log (4am)+1\right]$$
$$\Delta (g-2) = -{\alpha \over am}\left[ \log (4am)-2\right]$$


}


\section{Scharnhorst effect: $n$ between parallel conducting plates}
\frame{
\frametitle{Scharnhorst effect: $n$ between parallel conducting plates}
K. Scharnhorst, On propagation of light in the vacuum between plates. Physics
Letters B 236 (1990) 354-359. \\
G. Barton and K. Scharnhorst, QED between parallel mirrors: light signals faster
than c, or amplifed by the vacuum. Journal of Physics A: Mathematical and
General 26 (1993) 2037-2046.\\
P. Milonni and K. Svozil, Impossibility of measuring faster-than-c signaling by the
Scharnhorst effect. Physics Letters B 248 (1990) 437-438

$$c(a)= {c\over n(a)} > c$$

}


\section{Refractive index of ``vacuum'' with fermions}
\frame{
\frametitle{Refractive index of ``vacuum'' with fermions}

Volkmar Putz, Karl Svozil, Can a computer be ``pushed'' to perform faster-than-light? http://arxiv.org/abs/1003.1238
$\;$\\   $\;$\\
$\;$\\   $\;$\\
\begin{center}
\begin{fmffile}{QED_vacuum_polarization}
\begin{fmfgraph*}(120,40)
\fmfleft{i}
\fmfright{o}
\fmf{photon,label=$k$}{i,v1}
\fmf{photon,label=$k$}{v2,o}
\fmf{fermion,left,tension=0.4}{v1,v2,v1}
%\fmf{photon}{v1,v2}
\fmfdot{v1,v2}
\put(30.00,-10.00){\framebox(60,60)[cc]{$\varepsilon_F$}}
\end{fmfgraph*}
\end{fmffile}
\end{center}
$\;$\\
$${\Delta }{\Pi}_{\mu \nu}(k^2)=-\left(g_{\mu \nu }k^2 - k_\mu k_\nu \right) \frac{2\alpha}{3\pi}  \log \frac{\varepsilon_F}{m},
$$
}



\frame{
\frametitle{Refractive index of ``vacuum'' with fermions cntd.}

Effective mass term:
$$
M(k)=\epsilon^\mu { \Pi}_{\mu \nu}(k)\epsilon^\nu
$$
such that the eigenvalue equation is
$$
{ k}^2+ M(k)=(k^0)^2,
$$
where $k^\mu=({\bf k},k^0=\omega)$; and
$$
\vert  {\bf k} \vert \approx \omega - \frac{1}{2 \omega} M(k).
$$
Thus the index of refraction can be defined by
$$
n(\omega )=\frac{\vert {\bf k} \vert}{\omega}\approx 1 - \frac{1}{2 \omega^2} M(k).
$$
Hence the change of the refractive index is given by
$$
{\Delta }n(\omega )\approx -\frac{\alpha}{3\pi \omega^2} (\epsilon^\mu k_\mu)^2  \log \frac{\varepsilon_F}{m}.
$$
 }



\frame[squeeze]{
\frametitle{Refractive index of ``vacuum'' with fermions cntd.}

The group velocity is given by~[Equ.~(2)]{Scharnhorst-1998}
$$v_{gr}={c\over n_{gr}}$$ with
$$n_{gr}(\omega )= n (\omega )+ \omega \left[\partial n (\omega )/\partial \omega \right]$$
which, for transversal waves, turns out to be $n (\omega )$.
As a result, the speed of light $c/(1-{ \Delta }n)\approx c+{ \Delta }c$  is changed by
$${  \Delta }c = c {  \Delta }n.$$

}


\section{Convergence of perturbation series}
\frame{
\frametitle{Convergence of perturbation series---Advertisement for a paper I like}

Christiane Rousseau, Divergent series: past, present, future ..., http://www.dms.umontreal.ca/$~\widetilde~$rousseac/divergent.pdf
$\;$\\
$\;$\\
Euler equation $$x^2y'+y=x$$
on the one hand has a formal series solution
$$\hat{f}(x) =\lim_{k\rightarrow \infty}f_k(x)=\lim_{k\rightarrow \infty}\sum_{n= 0}^{k-1} (-1)^n n!x^{n+1}$$
which is divergent for all nonzero values of $x$.

On the other hand,  the solution can be written as a convergent integral  for $x\ge 0$:
$$f(x)= \int_0^\infty {e^{-t/x}\over 1+t} dt.$$

}
\frame{
\frametitle{Convergence of perturbation series---Advertisement for a paper I like cntd.}

For any $x\ge 0$,
$$\vert f(x)-f_k \vert \le k!x^{k+1}.$$
So, for fixed $x$, the difference diverges as $k$ reaches its limit, but for small $k$,
the difference remains finite, and is minimum when $k\approx 1/x$.

}


\frame{
\frametitle{Convergence of perturbation series---Advertisement for a paper I like cntd.}

Christiane Rousseau states in her summary:
$\;$\\
$\;$\\
{\it
``Divergent series occur generically in many situations with differential equations and dynamical
systems. Their divergence carries a lot of geometric information on the solutions of their
equations. For instance, if the formal power series solution of the Euler equation ... had
been convergent, its sum could not have been ramified. The space of sums of convergent
power series is not sufficiently rich to encode the rich dynamics of the solutions of differential
equations, hence the divergence.''}

}



\frame{



\centerline{\Large Thank you for your attention!}

\begin{center}
$\widetilde{\qquad \qquad }$
$\widetilde{\qquad \qquad}$
$\widetilde{\qquad \qquad }$
\end{center}
 }

\end{document}
