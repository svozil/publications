%%tth:\begin{html}<LINK REL=STYLESHEET HREF="/~svozil/ssh.css">\end{html}
\documentclass[prl,preprint,showpacs,showkeys]{revtex4}
\usepackage{graphicx}
%\documentstyle[amsfonts]{article}
%\RequirePackage{times}
%\RequirePackage{courier}
%\RequirePackage{mathptm}
%\renewcommand{\baselinestretch}{1.3}
\begin{document}

\title{Parameter cheats}

\begin{abstract}
By definition, any random variable is isomorphically equivalent
to any other one
if the variables can be transformed into one another by a one-to-one (reversible)
transformation.
\end{abstract}
\pacs{03.65.Ud,03.65.Ta}
\keywords{Bell's inequalities; correlation polytopes, probability theory}

\maketitle

\section{Setup}



Let us consider a singlet state,
for which thew sum of all angular momenta and spins is zero.
In the quantum mechanical case, let us assume two particles of spin 1/2
in an EPR-Bohm configuration.
Then the probability
$P^{=}(\theta )$
to find the angular momentum or spin of
both particles
measured along two axis which are an angle $\theta $ apart
in the same direction is given by

\begin{eqnarray}
P^{=}_{qm} (\theta )&=&\sin^2 (\theta /2)\\
P^{=}_{cl} (\theta ) &=& \theta /\pi       \\
P^{=}_{s} (\theta ) &=&
{1\over 2}
+
{2\over \pi}
\sum_{k=0}^{n>1}
{
\sin \left[(2k+1)\left({2\Delta/ \pi}-1\right) \right]
\over 2k+1}
\nonumber \\
&\stackrel{n\rightarrow \infty}{\longrightarrow}& H(2\theta /\pi -1)=(1/2)(1+{\rm sgn} (2\theta /\pi
-1))
\end{eqnarray}
for $0\le \theta \le \pi$.
Figure   \ref{2001-cheat-fprob} represents different joint probability functions of the  parameter $\theta$.
\begin{figure}
 \includegraphics[width=10cm]{2001-cheat-fprob.eps}
 \caption{
Different joint probability functions of the  parameter $\theta$.
For the stronger-than-quantum case, $n=11$.
}
\label{2001-cheat-fprob}
\end{figure}



\section{Quantum cheat for classical system}

Then, in order to be able to fake
a quantum form of the classical expression,
we introduce a ``cheat parameter'' $\delta $,
which is obtained from the angular parameter $\theta $
by a nonlinear transformation
$T:\theta \mapsto \delta$
from the Ansatz
\begin{eqnarray}
P^{=}_{cl}(\theta (\delta ))
&=&
\nonumber \\
P^{=}_{cl}(\delta )
&=&
{\theta (\delta )\over \pi}=
\sin^2 \left({\delta \over 2}\right)
.
\label{2001-e1}
\end{eqnarray}
The right hand side of Eq. (\ref{2001-e1})
yields
\begin{eqnarray}
\theta &=& \pi \sin^2 \left({\delta \over 2} \right)\\
\delta &=& 2\arcsin \sqrt{{\theta \over \pi}}
\end{eqnarray}
where $0\le \delta  \le \pi$.
Figure   \ref{2001-cheat-f1} represents a numerical evaluation
of the deformed parameter scale $\delta$ in terms $\theta$.
\begin{figure}
 \includegraphics[width=10cm]{2001-cheat-f1.eps}
\begin{center}a)\end{center}
 \includegraphics[width=10cm]{2001-cheat-fbrav.eps}
\begin{center}b)\end{center}
 \caption{
a) Evaluation of the deformed parameter scale $\delta$ versus $\theta$.
b) Evaluation of the linear reference parameter $\delta$.
}
\label{2001-cheat-f1}
\end{figure}



Of course, we cannot expect from the cheat parameter to inherit the linear behavior of the
old parameter; in particular
$
\delta_3(\theta_3)
=
\delta_1(\theta_1)+
\delta_2(\theta_2)$ does not imply $\theta_3=\theta_1+\theta_2$,
and
$\delta(\theta_1)+\delta(\theta_2)=\delta(\theta_3)$




\section{Classical cheat for quantum system}

In order to be able to fake
a classical form of the quantum expression,
we introduce a ``cheat parameter'' $\phi $,
which is obtained from the angular parameter $\theta $
by a nonlinear transformation
$T:\theta \mapsto \phi$
from the Ansatz
\begin{eqnarray}
P^{=}_{qm}(\theta (\phi ))
&=&
\nonumber \\
P^{=}_{qm}(\phi )
&=&
{\phi \over \pi}=
\sin^2 \left({\theta (\phi ) \over 2}\right).
\label{2001-e2}
\end{eqnarray}
The right hand side of Eq. (\ref{2001-e2})
yields
\begin{eqnarray}
\theta &=& 2\arcsin \sqrt{{\phi \over \pi}}\\
\phi &=& \pi \sin^2 \left({\theta \over 2} \right)
\end{eqnarray}
where $0\le \phi  \le \pi$.
Figure   \ref{2001-cheat-f2} represents a numerical evaluation
of the deformed parameter scale $\phi$ in terms $\theta$.
\begin{figure}
 \includegraphics[width=10cm]{2001-cheat-f2.eps}
 \caption{Evaluation of the deformed parameter scale $\phi$ versus $\theta$.}
\label{2001-cheat-f2}
\end{figure}



\section{Stronger-than-quantum (STQ) cheat for classical system}

In order to be able to fake
a STR form of the classical expression,
we introduce a ``cheat parameter'' $\Delta $,
which is obtained from the angular parameter $\theta $
by a nonlinear transformation
$T:\theta \mapsto \Delta$
from the Ansatz
\begin{eqnarray}
P^{=}_{cl}(\theta (\Delta ))
&=&
\nonumber \\
P^{=}_{cl}(\Delta )
&=&
{1\over 2}
+
{2\over \pi}
\sum_{k=0}^n
{
\sin \left[(2k+1)\left({2\Delta/ \pi}-1\right) \right]
\over 2k+1}
=
{\delta (\Delta )\over \pi }
,
\label{2001-e3}
\end{eqnarray}
where $n\ge 1$.
In the limit,
$$\lim_{n\rightarrow \infty}
{4\over \pi}
\sum_{k=0}^n
{
\sin \left[(2k+1)\left({2\Delta \over \pi}-1\right) \right]
\over 2k+1}
=
{\rm sgn}\left({2\Delta \over \pi}-1\right).
$$

The right hand side of Eq. (\ref{2001-e3})
yields
\begin{eqnarray}
\theta &=&
{\pi \over 2}
+
{2}
\sum_{k=0}^n
{
\sin \left[(2k+1)\left({2\Delta/ \pi}-1\right) \right]
\over 2k+1}
\end{eqnarray}

\section{How do the cheats perform?}

Let us consider the Clauser-Horne (CH) inequality
\begin{equation}
-1\leq P(A_{1}B_{1})+P(A_{1}B_{2})+P(A_{2}B_{2})-P(A_{2}B_{1})-P(A_{1})-P(B_{2}) \leq 0
\label{2001-cheat-ech}
\end{equation}
and a classical system on which a quantum cheat has been applied.
Let the angles be
\begin{eqnarray}
A_{1}&:&\delta_1 = 0,
\nonumber \\
B_{1}&:&\delta_2 = \pi /4,
\nonumber \\
A_{2}&:&\delta_3 = \pi /2,
\nonumber \\
B_{2}&:&\delta_4 = 3\pi /4.
\nonumber
\end{eqnarray}

Identify
$P(A_{i})=P(B_{i})= 1/2$
and
\begin{eqnarray}
P(A_1B_1)&=& P^{=}_{cl}((\delta_2 -\delta_1)/2 = \pi /8),
\nonumber \\
P(A_2B_2)&=& P^{=}_{cl}((\delta_4 -\delta_3)/2 = \pi /8),
\nonumber \\
P(A_1B_2)&=& P^{=}_{cl}((\delta_4 -\delta_1)/2 =3\pi /8),
\nonumber \\
P(A_2B_1)&=& P^{=}_{cl}((\delta_3 -\delta_2)/2 = \pi /8).
\nonumber
\end{eqnarray}
With a choice of these angles, the right hand side of Eq. (\ref{2001-cheat-ech}) is violated.

\end{document}
