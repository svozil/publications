%%%%%%%%%%%%%%%%%%%%% chapter.tex %%%%%%%%%%%%%%%%%%%%%%%%%%%%%%%%%
%
% sample chapter
%
% Use this file as a template for your own input.
%
%%%%%%%%%%%%%%%%%%%%%%%% Springer-Verlag %%%%%%%%%%%%%%%%%%%%%%%%%%
%\motto{Use the template \emph{chapter.tex} to style the various elements of your chapter content.}
\chapter{UFO sagas and legends after 2003}
\label{2023-UFO-part-History-chapter-2004-2023} % Always give a unique label
% use \chaptermark{}
% to alter or adjust the chapter heading in the running head


\abstract*{The Nimitz encounter was an event that took place in November 2004 and involved a strange encounter between US Navy pilots and an unknown aerial object. The USS Princeton, part of the Nimitz Carrier Strike Group, had been tracking strange aircraft for two weeks prior to the incident and reported that the objects would appear at high altitudes, then rapidly descend toward the ocean and hover. Commander David Fravor and another pilot, Alex Dietrich, were directed to investigate the unknown contact and observed two objects. One was just below the surface of the water, and the second was an oval-shaped object hovering erratically above the water. Fravor reported that as he spiraled down to get closer to the object, it turned toward him and started to mirror his movements, but eventually disappeared in a sudden and rapid manner, only to reappear at the secret Combat Air Patrol (CAP) point.
Despite the presence of eyewitnesses, there is no official documentation or radar data that covers the incident. However, photos associated with the event were authenticated by the Department of Defense in 2020.
}


\abstract{The Nimitz encounter was an event that took place in November 2004 and involved a strange encounter between US Navy pilots and an unknown aerial object. The USS Princeton, part of the Nimitz Carrier Strike Group, had been tracking strange aircraft for two weeks prior to the incident and reported that the objects would appear at high altitudes, then rapidly descend toward the ocean and hover. Commander David Fravor and another pilot, Alex Dietrich, were directed to investigate the unknown contact and observed two objects. One was just below the surface of the water, and the second was an oval-shaped object hovering erratically above the water. Fravor reported that as he spiraled down to get closer to the object, it turned toward him and started to mirror his movements, but eventually disappeared in a sudden and rapid manner, only to reappear at the secret Combat Air Patrol (CAP) point.
Despite the presence of eyewitnesses, there is no official documentation or radar data that covers the incident. However, photos associated with the event were authenticated by the Department of Defense in 2020.
}


\section{Nimitz encounter on November 10--14, 2004}
\label{2023-UFO-part-History-chapter-2004-2023-Nimitz}

\subsection{Context and sources}

The following narrative is a mesh from several recollections of the event~\cite{NimitzSCURep2019Mar,Chierici2015Mar,Mizokami2017Dec,Powerfuljre2019Oct}. No official document or radar data exist that covers them. It became famous because of the release of photos---authenticated by the DoD on April 27, 2020~\cite{DOD2020}---associated with (parts of) it, as well as reports in legacy media~\cite{CooperNYT2017,BryanBender2017}.

In December 2008, almost ten years before the December 2017 New York Times article,
a participant in the events contacted Colm A. Kelleher.
At the time, the participant was retired from the US Navy and was interviewed for a job at the
Bigelow Aerospace Advanced Space Studies (BAASS)~\cite[p.~41]{Lacatski-2021}.\index{BAASS}
BAASS was hiring personnel, as it had won a contract for
Advanced Aerospace Weapon System Applications Program (AAWSAP)~\cite{AAWSAP2008}.\index{AAWSAP}

AAWSAP was originally stimulated by George Knapp, a journalist specializing in ``strange'' encounters and UFOs,
as well as by Robert Bigelow. Bigelow had bought the
Sherman aka Skinwalker Ranch in Utah (now owned by Brandon Fugal, another property tycoon),
allegedly known for some ``strange phenomena'' encountered there~\cite{ShaefferSkeptic2022May}.

After reading a book about these alleged occurrences~\cite{ColmA.Kelleher2009Jul} and traveling to
Skinwalker Ranch, James Lacatski, an employee of the Defense Intelligence Agency (DIA),
 had a minor ``vision.''

As far as I can tell, Bigelow was the friend and campaign contributor of the late
Nevada Senator Harry Reid, a former Majority Leader of the United States Senate.
With the support of Reid and two other Senators,
Lacatski developed a government contract to investigate UFOs,
strange phenomena, and the Skinwalker Ranch, all under the guise of researching future aerospace technologies and threats.
BAASS was founded as a contractor for this small project---AAWSAP---under the aegis of the Defense Intelligence Agency (DIA).

After listening to Kurth's story, Kelleher passed it on to Jim Lacatski at the Defense Intelligence Agency (DIA). Lacatski, in turn, put Jonathan Axelrod on the case, and within months, Axelrod and his team had comprehensively interviewed and obtained detailed testimony from many witnesses: pilots, radar operators, and others from the Nimitz Aircraft Carrier Strike Group~\cite[p. 41]{Lacatski-2021}.
Allegedly part of it has been released~\cite{ExexSummaryNimitz2019Jun}, but its origin and validity are unclear at this point.

\subsection{USS Princeton observes groups of objects raining out of the sky toward the sea}

The USS Princeton, part of a strike group,
had been tracking strange groups of objects for several days---from November 4 to November 14 over a 10 day period,
there were probably 100 different contacts in groups of five to ten, at seemingly random times~\cite{WestDayKevin2021Feb}.
Ballistic missile defense had observed objects beyond the upper atmosphere.
These groups of five to ten   emerged    at an altitude of 80\,000 feet.
The group of objects would then rapidly descend toward the ocean and end at 35\,000 to 20\,000 feet close to Catalina Island,
where they would hover, almost stationary, or move southbound towards Guadalupe Island, which is approximately 500 kilometres (300 miles) away.
They kept a tight formation in the same relative positions to each other over the entire time,
at a speed of approximately 100 knots or 200~km/hour.
Afterwards, they would either shoot straight back upward~\cite{Cooper2017Dec} or drop down to the ocean
at incredible speeds of approximately 24\,000 miles/hour or 40\,000 km/hour,
before vanishing from the radar close to Guadalupe Island~\cite{WestDayKevin2021Feb}.


The operators first thought that their new AN/SPY-1B phased array radar
system---at the time, the most sophisticated and powerful tactical radar on the planet---was malfunctioning.
Therefore, the air control systems were taken down and recalibrated in an effort to clarify what were assumed
to be false radar returns. However, after this recalibration, the strange unexplainable tracks were even sharper and clearer.

At some point, when the objects appeared on radar,
Gary Voorhis, a system technician aboard the USS Princeton,
went up to the bridge to observe them through the ship's ``Big Eyes,''
which were heavily magnified binoculars~\cite{PMWitnesses2021Mar}.
The objects were too far away to discern any specific details, but Voorhis could clearly see something whitish moving unpredictably in the distance. He
recalls that, at night, the objects had a uniform phosphorescent glow~\cite{WestVoorhis2020Jan}.
They could be seen more easily in the dark than in the daylight.

An airborne (E-2 Hawkeye) radar system, after focusing their radar on the coordinates USS Princeton directed them toward,
observed a faint signal of that target.
However, the radar returns from the contact were not good enough to generate a target track.

At this point, the operators on the USS Princeton seemed determined to find out what was behind this phenomenon. They approached fighter pilots on their training missions to obtain a closer look.

\subsection{Whitewater}

On November 14, 2004, Lieutenant Colonel Douglas ``Cheeks'' Kurth had just finished a routine post-maintenance check flight---so he was already flying in the air---when
he received a strange and intriguing communication from USS Princeton.
Despite being in close proximity to San Diego's homeport, Kurth was asked to investigate an unidentified airborne contact,
which was an uncommon request given the location.
What made the communication even more peculiar was the fact that Kurth was questioned about the ordnance he had on board.
He had none.

As Kurth neared the designated location, USS Princeton cautioned him to maintain an altitude of at least 10\,000 feet,
as Commander David Fravor and Lieutenant Commander Alex Dietrich's group of Super Hornets
were on their way to the target~\cite{Dietrich-TicTac}.
Kurth's radar picked up the two Super Hornets but no other contacts.
At that point, USS Princeton instructed him to abandon the mission and head back to the ship.
Since he was close, Kurth decided to stay for a while and watch what would happen~\cite{Chierici2015Mar}.

It was a late morning on a gorgeous day in Southern California, with the sea almost as still as glass, presenting perfect flying conditions. As Cheeks flew over the area, he noticed a disturbance on the ocean surface, a circular section of turbulent water spanning 50--100 meters in diameter. He termed it ``whitewater,'' since it appeared as although something was lurking beneath the surface, such as a shoal or the aftermath of a sinking ship.

Kurth circled back to get a closer look but could not ascertain what was causing the frothing effect on the water.
As he turned away, coinciding with the arrival of the Super Hornets at the location,
the whitewater subsided, and the ocean surface reverted to its serene state.
The area of the previous commotion was now completely unrecognizable.

He never saw anything other than whitewater.


\subsection{Encounter with an oval object}

On the same day, November 14, 2004, Commander David Fravor and Lieutenant Commander Alex Dietrich
were flying Super Hornets of the USS Nimitz Carrier Strike Group~\cite{Fravor_2023-HOC}.
Just as Kurth before,
Fravor was first asked if their jets were carrying any ordnance. He responded that all they had were two captive-carry training missiles.

Fravor and Dietrich were directed by the USS Princeton to investigate an unknown aerial contact or object at a particular location
where one of these strangely moving objects, which had been observed by the USS Princeton for days, appeared on the radar screens.
Upon arriving at the supposed location, which was the merge spot (or radar merge cell) on the AN/SPY-1B phased array radar where
Fravor and Dietrich's Super Hornets could no longer be distinguished from the object,
Fravor realized that there were two objects rather than one.

He saw a large cross-shaped object just below the surface of the water,
about the size of a large passenger aircraft, causing a ``whitewater'' effect, which Kurth also observed.
Then, the weapons systems officer in the rear seat, and shortly thereafter Fravor, spotted a second object hovering erratically in a zig-zag style just 50 feet above the water. Fravor estimated that the second, smaller object, which was whitish and
oval-shaped without wings or a propulsion system, was approximately 40 feet long.

Four people in total, including the two pilots Fravor and Dietrich, as well as their respective weapons systems officers,
observed a smaller object for approximately five minutes. Fravor reported that, as he spiraled down to get closer to the object,
it suddenly seemed to recognize his plane. The object turned toward him and started to mirror his movements; as he went down,
the object was coming up. Fravor's plane and the object circled around each other, but while the object was ascending, Fravor's plane was descending.

At some point, Fravor wanted to get closer to the object and join it.
He dropped the nose of his plane aggressively, thereby cutting across the circle.
Then, in Fravor's own words, the object reacted and ``starts to accelerate, and within about less than a second as
I start to pull nose onto it [[$\ldots$]] it crosses right in front of me [[$\ldots$]] it just goes `poof' and it's gone.''
The object had disappeared, and not only from Fravor's view. Nobody could locate it any longer.
Additionally, the object under the whitewater was also gone.

In Day's alternative recollection~\cite{WestDayKevin2021Feb} of the engagement, Fravor told him that, as Fravor's plane
approached the formation of five objects, the lead object of the formation did  a barrel roll around him,
and headed straight down to the water. Fravor followed and chased it down towards the ocean floor.

\subsection{Object waiting at the Combat Air Patrol point}

The Super Hornets then communicated with the USS Princeton.
At that point, the operator at the USS Princeton told Fravor: ``You are not going to believe this but that thing is back at your CAP point.''
This computer-generated Combat Air Patrol (CAP) point was 60 miles away---not only was
this  location highly protected and encrypted, but the object needed at most 30--40 seconds to reach it, amounting to more than
3700~miles per hour, or 6000~km per hour.
The radar did not track it---it just vanished and reappeared there.

However, by the time they arrived, the object had already disappeared.

\subsection{Another wave of fighters recording a movie}

Another wave of fighters, including Lieutenant Commander Chad Underwood~\cite{Phelan2019Dec},
was dispatched for investigation. Underwood's fighter was equipped with a targeting pod including an advanced FLIR (infrared) camera.
Some (at least similar) object appeared on the onboard radar and started jamming the radar.
Underwood switched to passive recording FLIR (infrared) and video
camera mode.
This is the video that was later authenticated by the DoD~\cite{DOD2020}, which is often referred to as the ``Tic Tac'' video, although he himself did not see any unusual object.




\subsection{Estimate of eye resolution}

Let us estimate Fravor's conceivable eye resolution from a distance of 20\,000 feet \cite[time = 646\,s]{Powerfuljre2019Oct}. A US National Bureau of Standards publication reviewing visual acuity,\index{visual acuity}\index{acuity} states that~\cite[p.~10]{Howett1983Jul} ``it is traditionally assumed that the finest detail that can just be made out by an eye with normal visual acuity, viewing black lines on a white background, with moderate levels of illumination, subtends a visual angle of 1 minute of arc.'' This means that by approximating
$\sin(x) \approx x$ for $x\ll 1$, the (size of the) smallest object discernible with human eyes at a distance $d$ (in SI units)
is approximately $d \times 2 \times \pi /(360\times 60)$~m. That is, for 20\,000~feet~$\approx 20\,000 \times 0.3048$~m~$\approx 6096$~m,
the smallest object discernible for an average person is approximately $6096 \times 2 \times \pi /(360\times 60)$~m~$\approx 1.7$~m or approximately 6~feet.
Therefore, Fravor's estimate of the size of the craft of 40~feet~\cite[time\,=\,652\,s]{Powerfuljre2019Oct} is reasonable, in particular,
assuming that Fravor's visual acuity may be better than average, and the illumination was very good.

The events described suggest the presence of advanced aircraft technology beyond the current capabilities
of the US military and aerospace industry.

\subsection{Analysis of the flir, gofast, and gimbal videos}

In what follows are the critical analyses by Mick West and the group at Metabunk.
These in turn have been met with criticisms from other UFO investigators on two grounds~\cite{WestElizondo2021Mar}:
\begin{itemize}
\item Firstly, in line with a strategy employed by the late Edward Teller
(an experience I had the privilege of witnessing personally~\cite{Etim1992Aug}),
whenever one tried to ``nail Teller down'' with some argument,
he would take evasive action by stating that he could not disclose certain information due to national security concerns,
even though he was privy to classified information that supported his positions.
\item Secondly, similar to the first reason, there is an insinuation that anyone criticizing the ``strangeness'' of these videos
lacks knowledge of important contextual factors,
such as additional sensor data, better and longer videos,
and the opinions of highly competent experts who have already analyzed these videos and excluded prosaic explanations.
\end{itemize}

\subsubsection*{Flir video}

As mentioned earlier, one of the
three videos~\cite{NAVAIRFOIAFLIR2020Apr,NAVAIRFOIAGOFAST2020Apr,NAVAIRFOIAGIMBAL2020Apr}
that have been authenticated by the Department of Defense on April 27, 2020~\cite{DOD2020}---the ``FLIR'' video---is related to the previous narrative.
It was taken from a different airplane after Fravor's encounter with a ``tic-tac'' shaped UAP.

A detailed analysis~\cite{West2022Nov,WestFlir2020May,WestFravorWest2020Sep} by the ``debunker'' Mick West~\cite{Metabunk2023Mar}  \index{West, Mick}
attempted to demonstrate that the flir video actually shows an ordinary aircraft moving with constant velocity.
The video is overlaid with a time code and relevant information from the display to
show that all apparent movements are consistent with the object moving in a straight line to the left.
The camera was tracking the object visually, and every time there occurred a physical lens change or a camera rotation,
the lock was briefly lost, causing the object to appear as if it had suddenly changed speed.
The analysis suggests that the object was a distant aircraft flying along normally.

\subsubsection*{Mirroring movement}

West also speculates that Fravor might have mistaken~\cite{WestFravorWest2020Sep,WestAlexDietrich2021Jun} the ``mirroring'' movement of an object that in reality remained almost
static in the center of the circle he was flying:
Fravor thought that he was  flying in a circle with some object mirroring him on the other side,
but in reality this might have been an illusion caused by some object that was actually in the middle (rather than on the opposite end) of the circle.
When Fravor decided to cut across the circle, he saw something coming towards him unexpectedly fast and zipping by,
which was actually a loss of situational awareness from a misrepresentation of the airplane--object configuration.

\subsubsection*{Gofast video}

West argues~\cite{West2022Nov,West2019Jun} that the object in the ``GOFAST'' video~\cite{NAVAIRFOIAGOFAST2020Apr}
is not actually flying low and fast over the water as it appears, but rather is flying high and slow.
He uses information from the video, such as the altitude of the jet, calibrated airspeed,
and distance to the target, to calculate the height and speed of the object.

West suggests that the object is actually around 13\,000 feet above the water and traveling at a speed of 20 to 40 knots,
which is consistent with wind speed at that altitude. He also estimates that the object is about 6 to 7 feet in size,
which is similar to the size of a weather balloon.
This assumption would explain why it appears cold on the infrared camera.

\subsubsection*{Gimbal video}

The Metabunk team including West argue~\cite{markus2022Sep,West2022Nov,West2022Mar,WestFravorWest2020Sep} that the object in the ``GIMBAL'' video~\cite{NAVAIRFOIAGIMBAL2020Apr}
likely shows a glare~\cite{Talvala_2007} that hides the actual object, and that the shape of this glare only appears to be rotating because
the camera mounted on a gimbal rotates, and in order to de-rotate the horizon, any glare---which does not rotate relative to the camera like the horizon does---appears to rotate.
Therefore, the rotation of the UAP axes appears to be an illusion, created by the visual system of the aircraft.

In more detail, the rotation of the object is an artifact originating from the optical and post-processing system used.
The camera, which is mounted on a two-axis gimbal system, can only rotate along two axes (roll and pitch), when actually three axes of rotation would
be needed to ``faithfully'' track an object.
In order to keep the horizon at constant (zero, horizontal) angle, a post-processing step called ``de-ro,'' short for ``de-rotation,'' is applied to the video:
the de-ro device (or algorithm) takes the entire image and rotates it to obtain an unchanging horizon.

However, at the same time, the orientation of the glare from some thermal radiation (possibly from a jet plane) remains fixed relative to the camera (sensor) frame~\cite[time = 632~s]{West2022Mar}:
as the ``glare orientation is [[fixed]] relative to the camera [[sensor frame]]---rotate the camera and
the background will rotate but the glare angle will remain unchanged [[relative to the camera sensor frame]].''
The reason for this is that glare is an artifact of the camera's optics.  Therefore, under camera rotations, the glare remains
invariant with respect to the frame due to the co-varying nature of the optics/(sensor) frame system.


Moreover, if the de-rotation is applied through post-processing to obtain a non-varying horizon orientation,
and as the relative angle of the horizon and the camera glare/(sensor) frame system changes (due to camera rotation), the glare's angle relative to the horizon changes.
In summary, the alleged rotation of the object in the ``GIMBAL'' video is an illusory artifact of the camera/post-processing system used to track the object.

\subsubsection*{Videos authenticated by the government}


Suppose that West's aforementioned analysis is correct,
and that other videos released by the Pentagon,
such as a pyramid-shaped object seen flying through the sky in night vision footage~\cite{Cnn2021Apr},
are just bokeh~\cite{West2023Mar};
or that the Aguadilla infrared footage of UFOs, taken from a Bombardier DHC-8 airplane operated by the
U.S. Customs \& Border Protection, probably shows a pair of hot air wedding lanterns~\cite{Lianza2023Apr}.
This leaves us with little of value ever released by government players.


It appears not totally unreasonable to speculate that some of these supposed ``leaks''
could be a deliberate attempt to divert public attention and create frustration
by releasing unremarkable sightings, while allowing UFO enthusiasts
and even mainstream media to believe that they hold significant value.
By doing so, these leaks could serve as a smokescreen to obscure potentially more critical information,
or even to discredit the legitimacy of the entire UFO phenomenon.


% https://youtu.be/iTvm_xDVaDI

%https://www.metabunk.org/threads/aguadilla-infrared-footage-of-ufos-probably-hot-air-wedding-lanterns.8952/
%http://www.ipaco.fr/EN_IFO_B_heart_130425.pdf




\section{The woo according to Chris Bledsoe's alleged abduction on January 8, 2007, and by ``The Lady'' in 2012}
\index{Bledsoe, Chris}

According to his own unconfirmed accounts, after losing everything in the 2007 financial crisis and suffering from a debilitating chronic disease, Chris Bledsoe, a deeply religious family man and successful business owner from North Carolina, was abducted by a UFO in 2007 and five years later witnessed an apparition of "The Lady"~\cite{BledsoeChris}.

On January 8, 2007, Chris Bledsoe, his son, and three of his work friends went fishing on the Cape Fear River in North Carolina.
This fishing expedition turned out to be an incredible and terrifying experience that changed their lives forever.
They allegedly saw a fleet of UFOs with orange balls of light descending from the sky over the river.
Some feared the world was ending. Chris Bledsoe's son claimed to have seen
``little creatures come walking out of the woods''~\cite[time=699 s]{UOE2008OctS1E1}.
As he claimed to have seen a football-shaped spike-covered craft,
Chris had the most unique experience of the group.
Many hours had passed by that he could not account for when he finally returned to the group.

After he returned home, he allegedly followed his barking dogs and suddenly saw, very close (no more than three or four feet away), a creature staring at him, with the proportions of a child, red eyes, and a very shiny glowing surface, as though some small person had been dipped in glass and it had molded to the body~\cite{UOE2008OctS1E1}.

Bledsoe's experience was investigated by the Mutual UFO Network (MUFON). When their investigators arrived, they found no physical evidence and therefore had to rely on anecdotal witness testimonies, and in particular, Chris Bledsoe's. Chris agreed to undergo a regression hypnosis to access his blocked (or false?) memories. During hypnosis, he started talking about being taken onboard a craft by beings who called themselves his Guardians.

Bledsoe underwent two psychological tests with Dr. Deborah Gioia~\cite{GioiaMaryland2023Feb}, an associate professor at the University of Maryland, Baltimore: a SKID (now SCID) diagnostic interview, as well as the Minnesota Multiphasic Personality Inventory (MMPI). These tests did not reveal a psychotic personality, but rather a ``proneness to unusual ideas and beliefs which probably made him more susceptible but certainly did not create any kind of mental illness''~\cite{UOE2008OctS1E1}. During the interview, he mentioned a time of stress: in 2001, he had lost two million dollars because he had 27 houses set on the market for a year and a half. This caused a lot of depression, but he never told his family about this loss.

He also agreed to a polygraph ``lie detector'' examination, which was performed by Robert J. Drdak~\cite[time=2099s]{Drdak2023Feb}, who had worked as a Special Agent of the Federal Bureau of Investigation (FBI). Based on the test results, Drdak concluded that Bledsoe ``failed the test~[[$\ldots$]] his deceptive responses were focused around things that he actually saw that night''\cite{UOE2008OctS1E1}.


For the next five years, shadow people, orbs, and unexplained noises were reportedly commonplace for the Bledsoes. Despite this, Chris held onto the hope that the Guardians were watching over his family, and after reflecting on his hypnotic regression, he came to view extraterrestrial beings as angelic in nature. In 2012, he had another extraordinary experience when he reported being abducted once again, this time by a beautiful ``Lady'' who identified herself as the Mother Goddess.

Chris Bledsoe has been prominent in the American UFO scene and has met many individuals, among them
John B. Alexander,\index{Alexander, John B.}
Tom DeLonge,\index{DeLonge, Tom}
Luis Elizondo,\index{Elizondo, Luis}
Diana Walsh Pasulka,\index{Pasulka, Diana Walsh} and
Jim Semivan.\index{Semivan, Jim}




\section{Hazard aircrew encounter of 2014}

This is a quotation from a US Navy hazard report caused by several unidentified aerial vehicles operating in some operation area (W-72)
without coordinating with the controlling agency. The report was released after a Freedom of Information Act (FOIA) request~\cite{Hazard-2014-navi-foia}.
This passage describes a situation where a hazard aircrew (HAC) operating in exclusive-use airspace noticed
two radar trackfiles that were not communicating with the controlling agency or other aircraft.
The HAC identified these trackfiles as Hazard Unidentified Aerial Devices (HUADs)
and used on-board sensors to verify that they were small, infrared-significant objects,
rather than false radar track files. While tracking one of the HUADs,
the HAC also observed two additional, non-radar-significant objects flying at a high speed at a distance of approximately five nautical miles away.


\begin{svgraybox}
\noindent F. Event:
Operating in exclusive-use W-72 Air 2A and 2B, the Hazard aircrew (HAC) noticed two radar trackfiles
in 2A that were not communicating with the controlling agency or other aircraft. Radar indicated Hazard
Unidentified Aerial Device (HUAD) 1 at 0.0 Mach and 15 thousand feet. HUAD2 indicated 12 thousand feet
traveling at 0.0 Mach. HAC used multiple on-board sensors to verify that HUAD1 and HUAD2 were small IR
significant objects and not false radar track files. While tracking HUAD2 on radar and forward looking
infrared (FLIR,) two additional, non-radar-significant objects, HUAD3 and HUAD4, were seen flying through
the FLIR field-of-view at a high speed at a distance between the HAC and HUAD2, approximately 5 NM
away.

\noindent G. Hazard Date, Local: April 23, 2014
\end{svgraybox}

\section{Sentient ``tic tac'' shaped object highlight on  May 6, 2021}

The following slide entitled ``(U)BLUF'' is from a FOIA request~\cite{NRO2021May} of a presentation of the National Reconnaissance Office (NRO)
entitled ``Recent Sentient Highlights.''

\begin{svgraybox}
At 0038:17Z 6 May 2021. Sentient
[[likely that the program or project is related to Sentient or its applications, like ``NIMBUS'', ``ORION'', or ``SAGE'']]
image processing detected a possible airborne object  $\approx 78$~km southeast of
[[likely that the location is near a coast or an ocean like ``San Diego,'' ``Japan,'' ``Mediterranean Sea,'' or ``Indian Ocean'']]
The object was small ($< 10$~m), and did not match the visual signature of typical aircraft detections.

\begin{itemize}
\item The object did, however, vaguely resemble similar detections of airborne objects by US Navy  aircraft and surface vessels in the
[[likely that the location is near a coast or an ocean like ``San Diego,'' ``Japan,'' ``Mediterranean Sea,'' or ``Indian Ocean'']]
Operating Areas (``Unidentified Aerial Phenomena'').

\item There is a rough similarity to the previously reported ``tic tac'' shape.

\item The object was likely not a sensor artifact or focal plane anomaly (although more in-depth imagery analysis is warranted).
\end{itemize}
There were no correlating tracks present in
[[likely that these sources or systems are related to satellite imagery or communications like ``SBIRS,'' ``NRO,'' or ``NSA'']]
reporting, nor was there any correlating
ELINT/SIGINT in
[[likely that these sources or systems are related to satellite imagery or communications like ``SBIRS,'' ``NRO,'' or ``NSA'']]
reporting, despite time-coincident
[[likely that this vessel is related to a foreign adversary or competitor that has capabilities or interests in space like ``Shiyan,'' ``Yuan Wang,'' or ``Kosmos'']]
access/collection.


\begin{itemize}
\item Sentient detections did, however, detect the presence of the prob
[[likely that this vessel is related to a foreign adversary or competitor that has capabilities or interests in space like ``Shiyan,'' ``Yuan Wang,'' or ``Kosmos'']]
in the same imagery $\approx 25$~km to the west.

\item In recent reporting  the
[[possible that the area is related to the Nimitz Carrier Strike Group encounters in 2004, which occurred off the coast of Southern California, like ``SOCAL,'' ``W-291,'' or ``SCA'']]
has been associated with command-and-control (C2) activities, as well as more traditional telemetry and space functions --
the simultaneous presence of this high-interest vessel is notable, although possibly merely coincidental.
\end{itemize}
Confidence is relatively low in this detection, but the potential linkage to similar phenomenon off of
[[possible that the area is related to the Nimitz Carrier Strike Group encounters in 2004, which occurred off the coast of Southern California, like ``SOCAL,'' ``W-291,'' or ``SCA'']]
may warrant further investigation.
\end{svgraybox}


\section{Spheres with a cube inside}

An interview~\cite{ExoMagazinTV2022Oct} of Robert Fleischer with Christopher Karl Mellon~\index{Mellon, Christopher Karl} contains a very nice summary of one type of strangeness encountered by the US Navy. Asked about the shapes of UFOs/UAPs, Mellon, a private equity investor, and former Deputy Assistant Secretary of Defense for Intelligence in the Clinton and George W. Bush administrations, responded that the military had recently been observing mysterious objects in the sky. These objects were described as spheres that were six feet wide and had a cube inside, with the points of the cube touching the inside of the sphere. These objects did not float like balloons or lighter-than-air objects,
but instead were maneuverable, flew at hundreds of miles per hour, and flew against the wind without any exhaust~\cite{Graves_2023_OS_HOC}.
The source of energy and force behind these objects remains unknown, and they did not have any air intake. They had a very low radar cross-section, making them difficult to detect, and they were well concealed. Pilots reported seeing these objects on radar, but were unable to see them with the naked eye unless they flew close to them. These objects appeared in military training areas and seemed to linger indefinitely, which is unusual, as drones usually have a limited loiter time due to their running out of electric battery power.
