\documentstyle[12pt,pslatex,html]{article}
\begin{document}

\section{Publications}

1.\\
 K. Svozil\\
``On the computational power of physical systems,
undecidability, the consistency of phenomena and
 the practical uses of paradoxa''\\
in {\sl Fundamental Problems in
Quantum Theory: A Conference Held in Honor of Professor John A.
Wheeler}, ed. by D. M. Greenberger and A. Zeilinger,  {\sl Annals of the
New York Academy of Sciences} {\bf 755}, 834-841 (1995)


2.\\
Cristian Calude,
Douglas I. Campbell,
 Karl Svozil and
Doru \c{S}tef\u anescu \\
``Strong Determinism vs.  Computability''\\
in {\it The Foundational Debate, Complexity and Constructivity in
Mathematics and Physics}, Werner DePauli Schimanovich, Eckehart K\"ohler
and Friedrich Stadler, eds. (Kluwer, Dordrecht, Boston, London, 1995),
p. 115-131


3.\\
 K. Svozil\\
``A constructivist manifesto for the physical  sciences''\\
 in
{\it The Foundational Debate, Complexity and Constructivity in
Mathematics and Physics}, Werner DePauli Schimanovich, Eckehart K\"ohler
and Friedrich Stadler, eds. (Kluwer, Dordrecht, Boston, London, 1995),
p. 65-88

4.\\
K. Svozil\\
``Time paradoxa reviewed''\\
 Phys. Lett. {\bf A 199}, 323-326 (1995)


5.\\
 K. Svozil, \\
``Quantum computation and complexity theory I''\\
 Bulletin of the European Association of Theoretical Computer
Sciences {\bf 55}, 170-207 (1995)

6.\\
K. Svozil,\\
``Quantum computation and complexity theory II''\\
{\sl Bulletin of the European Association of Theoretical Computer
Sciences} {\bf 56}, 116-145 (1995)

7.\\
K. Svozil\\
``Halting probability amplitude of quantum
computers''\\
 Journal of Universal Computer Science
{\bf 1}, nr. 3, 1-4 (March 1995)


8.\\
 M. Schaller and K. Svozil,
``Automaton partition logic versus quantum logic''
{\sl International Journal of Theoretical Physics}, {\bf 34}, 1741-1750
(1995).
%partition.tex
% LITERAR MECHANA


9.\\
 M. Schaller and K. Svozil,
``Automaton logic''
{http://tph.tuwien.ac.at/\~{}svozil/publ/al.ps},
{\sl International Journal of Theoretical Physics}, {\it in print}.
%al.tex

10.\\
K. Svozil,
\htmladdnormallink{``Set Theory and Physics''}
{http://tph.tuwien.ac.at/\~{}svozil/publ/set.ps},
{\sl Foundations of Physics}, {\it in print}.


11.\\
K. Ehrenberger, D. Felix and K. Svozil,
``Origin of Auditory Fractal Random Signals in Guinea Pigs'',
{\sl Neuroreport} {\bf 6} (1995), {\it in print}.
(Bestaetigung bei Prof. Ehrenberger/AKH-HNO)


12.\\
A. Dvure\v{c}enskij, S. Pulmannov\'a and K. Svozil,
\htmladdnormallink{``Partition Logics, Orthoalgebras and Automata''}
{http://tph.tuwien.ac.at/\~{}svozil/publ/dvur.ps},
{\sl Helvetica Physica Acta}, {\bf 68}, 407-428 (1995).
%dvur.tex
% LITERAR MECHANA

13.\\     K. Svozil,
\htmladdnormallink{``Consistent use of paradoxes in   deriving
contraints on the dynamics of physical systems and of no-go-theorems''}
{http://tph.tuwien.ac.at/\~{}svozil/publ/paradox.ps},
{\sl Foundations of Physics Letters}, {\it in print}.
 %paradox.tex

14.\\
N. Brunner, K. Svozil and M. Baaz,
``Effective quantum observables''
{\sl Il Nuovo Cimento}, {\bf B110} (1995) {\it in print}.
(Bestaetigung bei Brunner/BOKU)


15.\\
N. Brunner, K. Svozil and M. Baaz,
``The axiom of choice in quantum theory''
{\sl Mathematical Logic Quarterly}, {\bf 46} (1996)  {\it in print}.
(Bestaetigung bei Brunner/BOKU)


16.\\
G. Krenn,  J. Summhammer and K.Svozil,
``Interaction-Free Preparation''
{\sl Phys. Rev.} {\bf A}, {\it in print}.
(Bestaetigung bei J. Summhammer/AI)



\section{Lectures \& posters}






1.\\
K. Svozil\\
``Varieties of Physical Undecidability''\\
Limits to Scientific Knowledge\\
Abisko (Schweden)\\
15. 5.95\\
((1-svoz3))

2.\\
K. Svozil\\
``Extrinsic versus intrinsic representability''
Limits to Scientific Knowledge\\
Abisko (Schweden)\\
17. 5.95  \\
((2-svoz4))

3.\\
K. Svozil\\
``Bridgman's operationalism and other set theoretical issues related to
physics''\\
Kurt G\"odel Seminar, Technische Universit\"at Wien\\
Wien (\"Osterreich)\\
19. 6.95    \\
((3-svoz5))

4.\\
K. Svozil\\
``Quantum algorithmic information theory''\\
Chaitin Complexity and its Applications\\
Mangalia (Rum\"anien) \\
2. 7.95       \\
((1-svoz6))

5.\\
K. Svozil\\
``Operational relevance of set theoretic terms''     \\
Chaitin Complexity and its Applications\\
Mangalia (Rum\"anien) \\
2. 7.95         \\
((2-svoz7))


6.\\
K. Svozil\\
``Quantum recursion theory''     \\
10th International Congress of Logic, Methodology and Philosophy of
Science\\
Florenz (Italien) \\
23. 8.95          \\
((2-svoz8))



7.\\
K. Krenn, J. Sumhammer and K. Svozil\\
``Interaction-free preparation''\\
\"OPG-Tagung\\
Leoben (\"Osterreich)\\
20. 9. 1995         \\
((2-svoz9))


8.\\
K. Svozil\\
Consistent use of paradoxes in deriving constraints on the
dynamics of physical systems and of no-go-theorems\\
\"OPG-Tagung\\
Leoben (\"Osterreich)\\
21. 9. 1995           \\
((2-svoz10))


9.\\
K. Svozil\\
``The diagonalization method in quantum recursion theory''\\
\"OPG-Tagung\\
Leoben (\"Osterreich)\\
21. 9. 1995             \\
((2-svoz11))

10.\\
K. Svozil, D. Felix and K. Ehrenberger\\
``Amplification by Stochastic Interference''\\
\"OPG-Tagung\\
Leoben (\"Osterreich)\\
21. 9. 1995               \\
((2-svoz12))


11.\\
Cristian Calude,
Douglas I. Campbell,
 Karl Svozil and
Doru \c{S}tef\u anescu \\
``Strong Determinism vs.  Computability''\\
\"OPG-Tagung\\
Leoben (\"Osterreich)\\
21. 9. 1995                 \\
((2-svoz13))


12.\\
K. Svozil\\
``Halting probability amplitude of quantum
computers''\\
\"OPG-Tagung\\
Leoben (\"Osterreich)\\
21. 9. 1995                   \\
((2-svoz14))


13.\\
K. Svozil\\
``Quantenlogik''\\
Atominstitut\\
22. 11. 1995


14.\\
K. Svozil\\
``Some physical aspects of future communication technology''\\
Universisad Federal do Rio De Janairo/Brasilien \\
28. 11. 1995



\end{document}
