\PassOptionsToPackage{usenames,dvipsnames}{xcolor}
%\documentclass[amsmath,table,sans,amsfonts, handout]{beamer}
\documentclass[amsmath,table,sans,amsfonts,hyperref={colorlinks,citecolor=blue,linkcolor=blue,urlcolor=purple}]{beamer}
\usepackage[T1]{fontenc}
%%\usepackage{beamerthemeshadow}
%%\usepackage[headheight=1pt,footheight=10pt]{beamerthemeboxes}
%%\addfootboxtemplate{\color{structure!80}}{\color{white}\tiny \hfill Karl Svozil (TU Vienna)\hfill}
%%\addfootboxtemplate{\color{structure!65}}{\color{white}\tiny \hfill mur.sat \hfill}
%%\addfootboxtemplate{\color{structure!50}}{\color{white}\tiny \hfill Graz, 2010-12-11\hfill}
%\usepackage[dark]{beamerthemesidebar}
%\usepackage[headheight=24pt,footheight=12pt]{beamerthemesplit}
%\usepackage{beamerthemesplit}
%\usepackage[bar]{beamerthemetree}
\usepackage{graphicx}
\usepackage{pgf}
%\usepackage{eepic}
%\newcommand{\Red}{\color{Red}}  %(VERY-Approx.PANTONE-RED)
%\newcommand{\Green}{\color{Green}}  %(VERY-Approx.PANTONE-GREEN)

\definecolor{applegreen}{rgb}{0.55, 0.71, 0.0}

\usepackage{fourier-orns}  %fancy symbols https://mirror.easyname.at/ctan/fonts/fourier-GUT/doc/fourier-orns-doc.pdf

%\usepackage{musixtex}

\newcommand{\Abschnitt}[1]{{\section #1}}

%%%%%%%%%%%%%%%%%%%%%%%%%%%%%
\usepackage{iftex}
\ifxetex
\usepackage{fontspec}% Schriftumschaltung mit den nativen XeTeX-Anweisungen
                     % vornehmen. Voreinstellung: Latin Modern
%\usepackage[ngerman]{babel}% Sprachumschaltung: Deutsch nach neuer Rechtschreibung



%
% XeLaTeX
%
\XeTeXinputencoding cp1252
\usepackage{fontspec}
%%\setmainfont{Times New Roman}
\setmainfont{Garamond}
\setsansfont{Garamond}
%\setmainfont{EB Garamond}
%\setsansfont{EB Garamond}
%
\else
\usepackage[latin1]{inputenc}
\usepackage[T1]{fontenc}
\fi
%%%%%%%%%%%%%%%%%%%%%%%%%%%%%

%\RequirePackage[german]{babel}
%\selectlanguage{german}
%\RequirePackage[isolatin]{inputenc}

%\pgfdeclareimage[height=0.5cm]{logo}{tu-logo}
%\logo{\pgfuseimage{logo}}
\beamertemplatetriangleitem
%\beamertemplateballitem

\beamerboxesdeclarecolorscheme{alert}{red}{red!15!averagebackgroundcolor}
%\begin{beamerboxesrounded}[scheme=alert,shadow=true]{}
%\end{beamerboxesrounded}

%\beamersetaveragebackground{yellow!10}

%\beamertemplatecircleminiframe

\newtheorem{question}{Question}
\newtheorem{conjecture}[question]{Principle}
\newtheorem{challenge}[question]{Challenge}
\usepackage{tikz}
\newcommand{\bra}[1]{\left< #1 \right|}
\newcommand{\ket}[1]{\left| #1 \right>}

\newcommand{\iprod}[2]{\langle #1 | #2 \rangle}
\newcommand{\mprod}[3]{\langle #1 | #2 | #3 \rangle}
\newcommand{\oprod}[2]{| #1 \rangle\langle #2 |}

\newcommand{\proj}[3]{\begin{smallmatrix} #1 & #2 & #3 \end{smallmatrix}}
\newcommand{\projbf}[3]{\begin{smallmatrix} \mathbf{#1} & \mathbf{#2} & \mathbf{#3} \end{smallmatrix}}

\sloppy
\parskip .7em %vskip between paragraphs

\newcommand{\seq}[1]{\mathbf{#1}}
\newcommand{\floor}[1]{\left\lfloor #1 \right\rfloor}
\newcommand{\ceil}[1]{\left\lceil #1 \right\rceil}
\newcommand{\m}[1]{\widetilde{#1}}
%\newcommand{\p}[1]{\scriptsize\textcolor{black}{$[#1]$}}

\usepackage[most]{tcolorbox}
\begin{document}

\title{\textcolor{blue}{\bf The Einstein-Podolski-Rosen argument revisited (again)}}
\subtitle{\footnotesize \url{http://tph.tuwien.ac.at/~svozil/publ/2022-Cordoba-pres.pdf}
\\
\footnotesize based on \\
\href{https://doi.org/10.48550/arXiv.2209.09590}{arXiv:2209.09590, DOI: 10.48550/arXiv.2209.09590} and\\
\href{https://doi.org/10.3390/e24121724}{Entropy 2022, 24(12), 1724, DOI: 10.3390/e24121724}
}
\author{\textcolor{blue}{Karl Svozil}}
\institute{\normalsize \textcolor{blue}{Institute for theoretical Physics, TU Wien}\\
\textcolor{blue}{svozil@tuwien.ac.at}
%{\tiny Disclaimer: Die hier vertretenen Meinungen des Autors verstehen sich als Diskussionsbeitr�ge und decken sich nicht notwendigerweise mit den Positionen der Technischen Universit�t Wien oder deren Vertreter.}
}
\date{{\color{purple}Tuesday, November 29, 2022, XI Conference on Quantum Foundations, C\'ordoba, Argentina}}
\maketitle


% \frame{
% \frametitle{Contents}
% \tableofcontents
% }

\section{What is this year's Nobel Prize in physics is about? (my guess)}

\frame{
 \frametitle{What is is what this year's Nobel Prize in physics is about? (my guess)}


\begin{itemize}
\item[$\bullet$]
Einstein, Podolsky, Rosen (EPR) according to Einstein (letter to Schr\"odinger, 1935): for entangled (non-factorizable states), the measurement outcome ``here'' affects state of the particle constituent (and its measurement outcome) ``there''.


\item[$\bullet$]
Boole,  $\ldots$, Bell, Specker, Wigner, $\ldots$, Clauser, Horne, Shimony \& Holt (CHSH), $\ldots$, Froissart, Pitowsky, $\ldots$:
experimentally testable violations of classical probabilities and expectations in certain configurations of quantum observables due to the Born Rule, Gleason's theorem.

\item[$\bullet$]
$\ldots$ under strict Einstein locality conditions (aka space-like separations).

\end{itemize}

}

\section{What can still be realized classically (potential misrepresentations of this year's Nobel prize): Relational encoding}

\frame{
 \frametitle{What can still be realized classically (potential misrepresentations of this year's Nobel prize): Relational encoding}

\begin{itemize}
\item[$\bullet$]
Peres' bomb example(cf \href{https://doi.org/10.1119/1.11393}{DOI: 10.1119/1.11393}):
 a bomb with initial angular momentum zero explodes into two fragments, both carrying opposite angular momenta (due to conservation).
In such a case, the fragments represent shares that are relationally encoded.

\item[$\bullet$]
Since quantum mechanics is vector based, quantum shares (eg entangled singlet states) perform differently except for three singular relative orientations of measuremts:
there are regions with stronger-than-classical correlations: either more $++/--$ or $+-/-+$ outcomes.
(This is responsible for the violations of the aforementioned (in)equalities.)

\item[$\bullet$]
Unlike quantum entanlement, and
although relationally encoded, the individual bomb fragments/constituents still maintain a (supposedly unknown thus epistemic uncertain) value definiteness.

\end{itemize}

}

\frame{
 \frametitle{``Derivation'' of classical conditions of possible experience}

$2^4=16$ possible valuation cases for four binary variables; assuming independence and non-contextuality:
\begin{center}
\resizebox{.7\textwidth}{!}{
\begin{tabular}{c|cccc|cccc||r}
valuation \# & $A$ &  $A'$ & $B$ & $B'$ & $AB$ & $AB'$ & $A'B$ & $A'B'$  & $CHSH=AB+AB'+A'B-A'B'$ \\
\hline
1  & $+1$  &$+1$ &  $+1$ &  $+1$ &  $+1$ &  $+1$ &  $+1$ &  $+1$ &   2 \\
2  & $+1$  &$+1$ &  $+1$ &  $-1$ &  $+1$ &  $-1$ &  $+1$ &  $-1$ &   2 \\
3  & $+1$  &$+1$ &  $-1$ &  $+1$ &  $-1$ &  $+1$ &  $-1$ &  $+1$ &  -2 \\
4  & $+1$  &$+1$ &  $-1$ &  $-1$ &  $-1$ &  $-1$ &  $-1$ &  $-1$ &  -2 \\
5  & $+1$  &$-1$ &  $+1$ &  $+1$ &  $+1$ &  $+1$ &  $-1$ &  $-1$ &   2 \\
6  & $+1$  &$-1$ &  $+1$ &  $-1$ &  $+1$ &  $-1$ &  $-1$ &  $+1$ &  -2 \\
7  & $+1$  &$-1$ &  $-1$ &  $+1$ &  $-1$ &  $+1$ &  $+1$ &  $-1$ &   2 \\
8  & $+1$  &$-1$ &  $-1$ &  $-1$ &  $-1$ &  $-1$ &  $+1$ &  $+1$ &  -2 \\
9  & $-1$  &$+1$ &  $+1$ &  $+1$ &  $-1$ &  $-1$ &  $+1$ &  $+1$ &  -2 \\
10  & $-1$  &$+1$ &  $+1$ &  $-1$ &  $-1$ &  $+1$ &  $+1$ &  $-1$ &   2 \\
11  & $-1$  &$+1$ &  $-1$ &  $+1$ &  $+1$ &  $-1$ &  $-1$ &  $+1$ &  -2 \\
12  & $-1$  &$+1$ &  $-1$ &  $-1$ &  $+1$ &  $+1$ &  $-1$ &  $-1$ &   2 \\
13 & $-1$  &$-1$ &  $+1$ &  $+1$ &  $-1$ &  $-1$ &  $-1$ &  $-1$ &  -2 \\
14  & $-1$  &$-1$ &  $+1$ &  $-1$ &  $-1$ &  $+1$ &  $-1$ &  $+1$ &  -2 \\
15  & $-1$  &$-1$ &  $-1$ &  $+1$ &  $+1$ &  $-1$ &  $+1$ &  $-1$ &   2 \\
16  & $-1$  &$-1$ &  $-1$ &  $-1$ &  $+1$ &  $+1$ &  $+1$ &  $+1$ &   2 \\
\hline
\end{tabular}
}
\end{center}

All classical expectations are convex combinations of these ``extreme cases'', so $\big\vert \langle  AB+AB'+A'B-A'B' \rangle \big\vert \le 2$.
However, the quantum bound is $2 \sqrt{2} > 2$, so violations may occur.


}

\section{The meaning of EPR/CHSH}


\frame{
 \frametitle{What do these violations mean? Possible ``evasion'' strategies (something has to be given up $\ldots$)}

\begin{itemize}
\item[$\bullet$]
evocation of a bit exchange, either signifying/communicating one outcome, or one bit of nonlocality, or the context;

\item[$\bullet$]
denying the co-existence of complementary counterfactual observables/shares; but then: how come the outcomes are correlated at all?

\item[$\bullet$]
Abandonment of  space-time frameworks like kantian Einstein relativity.

\end{itemize}

}

\frame{
 \frametitle{Possible ``evasion'' strategies to keep a classical framework}

\begin{itemize}
\item[$\bullet$]
In such ``patched'' relativities ``locality'' is defined in terms of entanglement: two events are ``close'' if they share entangled quantum states.

\item[$\bullet$]
There is no absolute notion of ``proximity'' and ``space-time frames''.

\item[$\bullet$]
Effectively, space-time emerges from quantum theory; it is an epiphenomenon, a secondary concept.

\end{itemize}

}



\section{Speculative summarizing questions and outlook}

\frame{
 \frametitle{Speculative summarizing questions and outlook}

\begin{itemize}

\item[(i)]  When can two outcomes be considered independent and separated? I suggest to interpret ``spatial separation''---two distinct points
in space-time---by decomposability of the quantum state; that is, whether the states of the constituents factorize.

\item[(ii)] Can it occur that, for two (or more) of the same constituents, some of their observables factorize (aka separable, not entangled) and are therefore
categorized as ``spatially separate or apart'', and other observables are inseparable (aka not factorable, entangled)?

\item[(iii)] Can it happen that all observables of two quanta are disentangled?


\item[(iv)] Is it possible to generate emerging space-time categories such as frames or coordinatizations, by purely quantum mechanical means?

\end{itemize}
}



\frame{

\centerline{\Large {\color{magenta} Thank you for your attention!}}

\begin{center}\color{orange}
$\widetilde{\qquad \qquad }$
$\widetilde{\qquad \qquad}$
$\widetilde{\qquad \qquad }$
\end{center}
 }
 \end{document}


















\section{ }

\frame{
 \frametitle{ }

\begin{itemize}
\item[$\bullet$] {
%\color{purple}
}
\pause
\item[$\bullet$] {
%\color{purple}
}
\end{itemize}
}

\section{ }

\frame{
 \frametitle{ }

\begin{itemize}
\item[$\bullet$] {
%\color{purple}
}
\pause
\item[$\bullet$] {
%\color{purple}
}
\end{itemize}
}

\section{ }

\frame{
 \frametitle{ }

\begin{itemize}
\item[$\bullet$] {
%\color{purple}
}
\pause
\item[$\bullet$] {
%\color{purple}
}
\end{itemize}
}

\section{ }

\frame{
 \frametitle{ }

\begin{itemize}
\item[$\bullet$] {
%\color{purple}
}
\pause
\item[$\bullet$] {
%\color{purple}
}
\end{itemize}
}

