%\documentclass[pra,showpacs,showkeys,amsfonts,amsmath,twocolumn]{revtex4}
\documentclass[amsmath,table,sans]{beamer}
%\documentclass[pra,showpacs,showkeys,amsfonts]{revtex4}
\usepackage[T1]{fontenc}
%%\usepackage{beamerthemeshadow}
%%\usepackage[headheight=1pt,footheight=10pt]{beamerthemeboxes}
%%\addfootboxtemplate{\color{structure!80}}{\color{white}\tiny \hfill Karl Svozil (TU Vienna)\hfill}
%%\addfootboxtemplate{\color{structure!65}}{\color{white}\tiny \hfill mur.sat \hfill}
%%\addfootboxtemplate{\color{structure!50}}{\color{white}\tiny \hfill Graz, 2010-12-11\hfill}
%\usepackage[dark]{beamerthemesidebar}
%\usepackage[headheight=24pt,footheight=12pt]{beamerthemesplit}
%\usepackage{beamerthemesplit}
%\usepackage[bar]{beamerthemetree}
\usepackage{graphicx}
\usepackage{pgf}
%\usepackage[usenames]{color}
%\newcommand{\Red}{\color{Red}}  %(VERY-Approx.PANTONE-RED)
%\newcommand{\Green}{\color{Green}}  %(VERY-Approx.PANTONE-GREEN)

%\RequirePackage[german]{babel}
%\selectlanguage{german}
%\RequirePackage[isolatin]{inputenc}

\pgfdeclareimage[height=0.5cm]{logo}{tu-logo}
\logo{\pgfuseimage{logo}}
\beamertemplatetriangleitem
%\beamertemplateballitem

\beamerboxesdeclarecolorscheme{alert}{red}{red!15!averagebackgroundcolor}
%\begin{beamerboxesrounded}[scheme=alert,shadow=true]{}
%\end{beamerboxesrounded}

%\beamersetaveragebackground{yellow!10}

%\beamertemplatecircleminiframe

\begin{document}

\title{\bf \textcolor{blue}{Some Unrelated Thoughts About Aesthetics \& Quantum Physics}}
%\subtitle{Naturwissenschaftlich-Humanisticher Tag am BG 19\\Weltbild und Wissenschaft\\http://tph.tuwien.ac.at/\~{}svozil/publ/2005-BG18-pres.pdf}
\subtitle{\textcolor{orange!60}{\small http://tph.tuwien.ac.at/$\sim$svozil/publ/2010-mursat-pres.pdf\\
http://arxiv.org/abs/physics/0505088}}
\author{Karl Svozil}
\institute{Institut f\"ur Theoretische Physik, University of Technology Vienna, \\
Wiedner Hauptstra\ss e 8-10/136, A-1040 Vienna, Austria\\
svozil@tuwien.ac.at
%{\tiny Disclaimer: Die hier vertretenen Meinungen des Autors verstehen sich als Diskussionsbeitr�ge und decken sich nicht notwendigerweise mit den Positionen der Technischen Universit�t Wien oder deren Vertreter.}
}
\date{mur.sat Graz, December 11, 2010}
\maketitle

\frame{

\centerline{\Large Part I: }
\centerline{ }
\centerline{\Large  \color{blue} Three principles of aesthetic complexity}

\begin{center}
{\color{orange}
$\widetilde{\qquad \qquad }$
$\widetilde{\qquad \qquad}$
$\widetilde{\qquad \qquad }$
}
\end{center}
}

\section{Three principles of aesthetic complexity}
\frame{
\frametitle{Three principles of aesthetic complexity}

\begin{itemize}
\item<1->
A necessary condition for an artistic form or design to appear appealing is its
complexity to lie within a bracket between monotony and chaos.

\begin{itemize}
\item<1->
Too condensed encoding makes a decryption of a work of art impossible and is perceived as chaotic by the untrained mind,
whereas
\item<1->
too regular structures are perceived as monotonous, too orderly and not very stimulating
\end{itemize}


\item<1->
Due to human predisposition, this bracket is invariably based on natural forms; with rather limited plasticity.

\item<1->
Aesthetic complexity trends are dominated by the available resources, and thus also by cost and scarcity.


\end{itemize}

}


\frame{
\frametitle{First law of aesthetic complexity}
\begin{center}
\includegraphics<1>[height=6cm]{2007-alpbach-patt.jpg}\\
{\bf Too low-complex patterns appear monotonous and dull; too high-complex patterns appear irritating and chaotic.}
\end{center}
}

\frame{
\frametitle{Second law of aesthetic complexity}
\begin{center}
\includegraphics<1>[height=8cm]{2005-ae-foliage.jpg}\\
{\tiny ``Nature Beauty:'' Autumn foliage near Baden, Lower Austria, Oct. 15, 2000}
\end{center}
}

\frame{
\begin{center}
\includegraphics<1>[height=8cm]{2005-ae-parkett-l.jpg}\\
{\tiny ``Art Beauty:'' Parquet flooring in the gallery rooms of the Garden Palais
 Liechtenstein, late 18th century, Vienna, Austria}
\end{center}
}


\frame{
\begin{center}
\includegraphics<1>[height=8cm]{2005-ae-bospiral.jpg}\\
{\tiny ``Art Beauty:'' Santino Bussi (1664-1736) Stucco detail in the Sala Terrena of the Garden Palais
 Liechtenstein, after 1700, Vienna, Austria}
\end{center}
}


\frame{
\begin{center}
\includegraphics<1>[height=5cm]{2005-ae-greekornament2.jpg}\\
{\tiny ``Art Beauty:'' Greek ornament from left to right: upper part of a stele,
termination of the marble tiles of the Pantheon;
the upper part of a stele;
by Lewis Vulliamy and
reprinted by Owen Jones, Grammar of Ornament}
\end{center}
}

\frame{
\begin{center}
\includegraphics<1>[height=8cm]{2005-ae-JanVanHuysum_Blumenstrauss.jpg}\\
{\tiny Jan Van Huysum, Flowers}
\end{center}
}

\frame{
\frametitle{Third law of aesthetic complexity}
\begin{itemize}
\item<1->
``Why build one pretty house if you can have two ugly ones for the same price?''   (Loos' principle, 1908)

\item<1->
After two years it became clear
        to both of us [[Sch\"onberg and Cage]] that I [[Cage]] had no feeling
        for harmony. For Schoenberg, harmony was not just coloristic: it
        was structural. It was the means one used to distinguish one part
        of a composition from another. Therefore he said I'd never be able
        to write music. ``Why not?'' ``You'll come to a wall and won't be
        able to get through.'' ``Then I'll spend my life knocking my head
        against that wall.''
(John Milton Cage, An Autobiographical Statement,  1989)\\
  http://www.newalbion.com/artists/cagej/autobiog.html

\end{itemize}
}

\section{The Conundrum of Quantum Jellification}

\frame{

\centerline{\Large Part II: }
\centerline{ }
\centerline{\Large  \color{blue} The Conundrum of Quantum Jellification}

\begin{center}
{\color{orange}
$\widetilde{\qquad \qquad }$
$\widetilde{\qquad \qquad}$
$\widetilde{\qquad \qquad }$
}
\end{center}
}

\frame{
\frametitle{The Conundrum of Quantum Jellification}

\begin{quote}
{
The idea that  [the alternate measurement outcomes] be not alternatives but {\em all} really happening simultaneously
seems lunatic to [the quantum theorist], just {\em impossible.}
He thinks that if the laws of nature took {\em this} form for,
let me say,
a quarter of an hour, we should find our surroundings rapidly turning into a quagmire, a sort of a featureless jelly or plasma,
all contours becoming blurred, we ourselves probably becoming jelly fish.
It is strange that he should believe this.
For I understand he grants that unobserved nature does behave this way -- namely according to the wave equation.
$\ldots$ according to the quantum theorist, nature is prevented from rapid
jellification only by our perceiving or observing it.
}
\end{quote}

Schr�dinger, E. (1995). The Interpretation of Quantum Mechanics.
Dublin seminars (1949-1955) and other unpublished assays. Woodbridge,
Connecticut: Ox Bow Press, pp. 19--20.

}


\frame{

\centerline{\Large \color{orange} Thank you for your attention!}

\begin{center}
{ \color{blue}
$\widetilde{\qquad \qquad }$
$\widetilde{\qquad \qquad}$
$\widetilde{\qquad \qquad }$
}
\end{center}


 }

\end{document}
