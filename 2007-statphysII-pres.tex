%\documentclass[pra,showpacs,showkeys,amsfonts,amsmath,twocolumn]{revtex4}
\documentclass[amsmath,blue,handout,table]{beamer}
%\documentclass[pra,showpacs,showkeys,amsfonts]{revtex4}
\usepackage[T1]{fontenc}
\usepackage{beamerthemeshadow}
%\usepackage[dark]{beamerthemesidebar}
%\usepackage[headheight=24pt,footheight=12pt]{beamerthemesplit}
%\usepackage{beamerthemesplit}
%\usepackage[bar]{beamerthemetree}
\usepackage{graphicx}
\usepackage{pgf}
%\usepackage[usenames]{color}
%\newcommand{\Red}{\color{Red}}  %(VERY-Approx.PANTONE-RED)
%\newcommand{\Green}{\color{Green}}  %(VERY-Approx.PANTONE-GREEN)

%\RequirePackage[german]{babel}
%\selectlanguage{german}
%\RequirePackage[isolatin]{inputenc}

\pgfdeclareimage[height=0.5cm]{logo}{tu-logo}
\logo{\pgfuseimage{logo}}
\beamertemplatetriangleitem
\begin{document}

\title{\bf \textcolor{yellow}{STATPHYS II Erg�nzung Deterministisches Chaos und Quantenkorrelationen}}
%\subtitle{Naturwissenschaftlich-Humanisticher Tag am BG 19\\Weltbild und Wissenschaft\\http://tph.tuwien.ac.at/\~{}svozil/publ/2005-BG18-pres.pdf}
\subtitle{\textcolor{yellow!60}{http://tph.tuwien.ac.at/$\sim$svozil/publ/2007-statphysII-pres.pdf}}
\author{Karl Svozil}
\institute{Institut f\"ur Theoretische Physik, University of Technology Vienna, \\
Wiedner Hauptstra\ss e 8-10/136, A-1040 Vienna, Austria\\
svozil@tuwien.ac.at
%{\tiny Disclaimer: Die hier vertretenen Meinungen des Autors verstehen sich als Diskussionsbeitr�ge und decken sich nicht notwendigerweise mit den Positionen der Technischen Universit�t Wien oder deren Vertreter.}
}
\date{Dec. 9, 2005}
\maketitle

\frame{\tableofcontents}


%%%%%%%%%%%%%%%%%%%%%%%%%%%%%%%%%%%%%%%%%%%%%%%%%%%%%%%%%%%%%%%%%%%%%%%%%%%%%%%%%%%%%%%%%%%%%%%%%%%%%%%%%
%%%%%%%%%%%%%%%%%%%%%%%%%%%%%%%%%%%%%%%%%%%%%%%%%%%%%%%%%%%%%%%%%%%%%%%%%%%%%%%%%%%%%%%%%%%%%%%%%%%%%%%%%
%%%%%%%%%%%%%%%%%%%%%%%%%%%%%%%%%%%%%%%%%%%%%%%%%%%%%%%%%%%%%%%%%%%%%%%%%%%%%%%%%%%%%%%%%%%%%%%%%%%%%%%%%
%%%%%%%%%%%%%%%%%%%%%%%%%%%%%%%%%%%%%%%%%%%%%%%%%%%%%%%%%%%%%%%%%%%%%%%%%%%%%%%%%%%%%%%%%%%%%%%%%%%%%%%%%



\section{Deterministic chaos}


\subsection{``Solution'' of the n-body problem}

\frame{
\frametitle{Laplace}



Present events are connected with preceding ones
by a tie based upon the evident principle that a thing
cannot occur without a cause which produces it. This
axiom, known by the name of the principle of sufficient
reason, extends even to actions which are considered
indifferent $\ldots$


We ought then to regard the present state of the
universe as the effect of its anterior state and as the
cause of the one which is to follow. Given for one
instant an intelligence which could comprehend all the
forces by which nature is animated and the respective
situation of the beings who compose it an intelligence
sufficiently vast to submit these data to analysis it
would embrace in the same formula the movements of
the greatest bodies of the universe and those of the
lightest atom; for it, nothing would be uncertain and
the future, as the past, would be present to its eyes.

}


\frame{
\frametitle{Oscar II/Leffler/Weierstrass}

Mittag-Leffler, G.: The n-body problem (Price Announcement), Acta Matematica, 1985/1986,7

Given a system of arbitrarily many mass points that attract each
according to Newton's law, under the assumption that no two points ever collide,
try to find a representation of the coordinates of each point
as a series in a variable that is some known function of time and for
all of whose values the series converges uniformly.

}


\frame{
\frametitle{Poincar{\'e}}

Wissenschaft und Methode, 1914

W\"urden wir die Gesetze der Natur und den Zustand des Universums
f\"ur einen gewissen Zeitpunkt genau kennen, so
k\"onnten wir den Zustand dieses Universums f\"ur
irgendeinen sp\"ateren Zeitpunkt genau voraussagen.
Aber
[[~$\ldots$~]]
 es kann der Fall eintreten,
da\ss \. kleine Unterschiede in den Anfangsbedingungen
gro\ss e Unterschiede in den sp\"ateren Erscheinungen bedingen;
ein kleiner Irrtum in den ersteren kann einen au\ss erordentlich gro\ss en
Irrtum f\"ur den letzteren nach sich ziehen
Die Vorhersage wird unm\"oglich und wir haben eine
``zuf\"allige Erscheinung''.

}


\frame{
\frametitle{``Solutions''}

3-K�rperproblem: Sundman, K. E.: Memoire sur le probleme de trois corps, Acta Mathematica 36 (1912): 105--179

n-K�rperproblem: Wang, Qui Dong. The global solution of the n-body problem,Celestial Mechanics, 1991, 50,73-88

Siehe auch: Diacu, F.: The solution of the n-body Problem, The Mathematical Intelligencer,1996,18,p.66-70

Konvergenz ``langsam'' ``unberechenbarer Konvergenzradius''?

}





\subsection{Lyapunov exponent}
\frame[shrink=2]{
\frametitle{Lyapunov exponent}

 The {\em Lyapunov exponent} $\lambda$
\index{Lyapunov exponent}
 can be introduced as
 a measure
 \label{lyap}
 of the separation of
 two distinct initial values.
 Consider a discrete
 time evolution of the form $x_{n+1}=f(x_n)$ and an uncertainty
 interval $(x_0,x_0+\epsilon )$ of measure $\epsilon$ which,
 after $n$ iterations, becomes
 $(f^{(n)}(x_0),f^{(n)}(x_0+\epsilon
 ))$, which is of measure $\epsilon \exp \{n\lambda (x_0)\}$.
 $f^{(n)}$ stands for
 the $n$-fold iteration of $f$.
 The {\em Lyapunov
 exponent}
 $\lambda (x_0)$
 is defined for $\epsilon \rightarrow 0$ and $n\rightarrow \infty$ as
 \begin{eqnarray}
 \lambda (x_0)&=&\lim_{n\rightarrow \infty} \lim_{\epsilon \rightarrow
 0}   {1\over n}\log \left\vert {f^{(n)}(x_0+\epsilon
 )-f^{(n)}(x_0) \over \epsilon }\right\vert \nonumber \\
            & &\quad = \lim_{n\rightarrow \infty}{1\over n} \log
 \left\vert \left. {d\over dx}   f^{(n)}(x)\right\vert_{x=x_0}
\right\vert   \nonumber     \\
            & &\quad \quad = \lim_{n\rightarrow \infty}{1\over n} \log
 \left\vert \left. \prod_{i=0}^{n-1}{f'(x)}
\right\vert_{x=x_i}
\right\vert
\nonumber  \\
            & &\quad \quad \quad = \lim_{n\rightarrow \infty}{1\over n}
  \sum_{i=0}^{n-1}\log \left\vert \left. {f'(x)}
\right\vert_{x=x_i}
  \right\vert
\quad ,
 \label{ly-ex}
 \end{eqnarray}
 where the chain rule
 $$\left.{d\over d x}f^{(n)}(x)\right\vert_{x=x_0}=
 \left.{d\over d x}f(f(\cdots f(x)\cdots ))\right\vert_{x=x_0}=
 f'(x_{n-1})f'(x_{n-2})\cdots f'(x_1)f'(x_0)$$
and $\vert a_1 a_2 a_3 \cdots a_n\vert =
 \vert a_1\vert \vert  a_2\vert  \vert a_3\vert  \cdots \vert a_n\vert $
for $a_i\in {\Bbb R}$, $i=1,\ldots n$,
 has been used.
}

%%%%%%%%%%%%%%%%%%%%%%%%%%%%%%%%%%%%%%%%%%%%%%%%%%%%%%%%%%%%%%%%%%%%%%%%%%%%%%%%%%%%%%%%%%%%%%%%%%%%%%%%%

\frame{

A criterion for {\em ``deterministic chaos''}
 is a ``suitable'' evolution function capable of ``unfolding'' the
 information of a random real associated with the ``true'' but unknown
 initial value $x_0$. I.e.,
 either the uncertainty $\delta x_0$ of the
 initial value  or a corresponding variation of the initial value
 increases with time.

For positive Lyaponov exponent, an initial uncertainty increases
exponentially with
 time.




}


\frame{
For continuous maps $G(x)=dx/dt$, one obtains a change of uncertainty
 $\delta x$ by \begin{equation}
 \label{lya}
{d(\delta x)\over dt} \approx {\partial G(x)\over \partial x}
\delta x  \quad .
 \end{equation}

With
 $\lambda (x_0)
 = \partial
G/\partial x\mid_{x_0}$, then
for constant
$\lambda $ the
solution of (\ref{lya}) can be obtained by integration:  $\delta
x(t-t_0)=\delta x_0\exp [\lambda (t-t_0)]$.

 For $\lambda > 0$, a {\sl linear} increase in the precision
 of the initial value $\delta x_0$ renders merely a {\sl logarithmic}
 increase in the accuracy of the prediction.

}


%%%%%%%%%%%%%%%%%%%%%%%%%%%%%%%%%%%%%%%%%%%%%%%%%%%%%%%%%%%%%%%%%%%%%%%%%%%%%%%%%%%%%%%%%%%%%%%%%%%%%%%%%

\subsection{Continuum urn}

\frame[shrink=2]{
\frametitle{Continuum urn}

With probability 1, a real initial value ``taken from the continuum urn'' is uncomputable.
Deterministic chaos ``unfolds'' the initial value.

In the measure theoretic sense, ``almost all'' reals are
 uncomputable. This can be demonstrated by the following argument:
 Let $M=\{ r_i\}$ be an  infinite point set (i.e., $M$ is a
 set of
 points $r_i$) which is denumerable and which is the subset of a dense
 set. Then, for instance, every $r_i\in M$ can be enclosed in the
 interval \begin{equation}
 I(i,\delta)= [r_i-2^{-i-1}\delta
 ,
 r_i+2^{-i-1}\delta]\quad ,
 \end{equation}
 where $\delta $ may be arbitrary small (we choose $\delta$ to be
 small enough that all intervals are disjoint).
 Since $M$ is denumerable, the measure $\mu$ of these intervals can
 be summed up, yielding
  \begin{equation}
 \sum_i \mu( I(i,\delta))= \delta \sum_{i=1}^\infty 2^{-i}=\delta \quad
 . \end{equation}
 From $\delta \rightarrow 0$ follows $\mu (M)=0$.


}

%%%%%%%%%%%%%%%%%%%%%%%%%%%%%%%%%%%%%%%%%%%%%%%%%%%%%%%%%%%%%%%%%%%%%%%%%%%%%%%%%%%%%%%%%%%%%%%%%%%%%%%%%
%%%%%%%%%%%%%%%%%%%%%%%%%%%%%%%%%%%%%%%%%%%%%%%%%%%%%%%%%%%%%%%%%%%%%%%%%%%%%%%%%%%%%%%%%%%%%%%%%%%%%%%%%
%%%%%%%%%%%%%%%%%%%%%%%%%%%%%%%%%%%%%%%%%%%%%%%%%%%%%%%%%%%%%%%%%%%%%%%%%%%%%%%%%%%%%%%%%%%%%%%%%%%%%%%%%
%%%%%%%%%%%%%%%%%%%%%%%%%%%%%%%%%%%%%%%%%%%%%%%%%%%%%%%%%%%%%%%%%%%%%%%%%%%%%%%%%%%%%%%%%%%%%%%%%%%%%%%%%

\subsection{Verhulst/Feigenbaum scenario}


\frame[shrink=1.2]{
\frametitle{$x_{n+1}=f(x_n)=\alpha x_n(1-x_n)$}

\includegraphics{ch-big}
Study of the iterates of the logistic map
 $x_{n+1}=f(x_n)=\alpha x_n(1-x_n)$  and the
 associated Lyapunov exponents as a function of the parameter
 $\alpha$.



}


\frame{

for $\alpha \in [0,a_1] $, there exists one stable fixed point
 $x^\ast_1
=f(x^\ast_1 )$ and a system converges and remains at $x^\ast_1$;
}
\frame{
\frametitle{Periodic regime}
for $\alpha \in (a_1, a_\infty )$ there is, depending on the parameter
 $\alpha$, a hierarchy
 of fixed points and associated periodic trajectories. By varying
 $\alpha$ one notices a succession of
fixed point instabilities accompanied by bifurcations at $a_N$: if an
$N$'th
order fixed point $x^\ast_N$ is defined by its recurrence after $N$
computing steps (and not before), that is after $N$ iterations of $f$,
$$x_N^\ast = f^{(N)}(x_N^\ast )=\underbrace{f(f(\cdots f}_{N\; {\rm
times}} (x_N^\ast )\cdots ))\qquad ,$$
then $x_N^\ast $ characterises a periodic evolution in state space
(for the logistic equation, $a_1\approx 3.00$ and $a_\infty \approx
3.57$). for large $N$, the following scaling laws are
universal: for adjacent fixed points,
$$F_1 =\lim_{N\rightarrow \infty}{a_N-a_{N-1}\over
 a_{N+1}-a_N}=4.6692\cdots ,\quad
F_2 =\lim_{N\rightarrow \infty}{x_N^\ast-x_{N-1}^\ast \over
 x_{N+1}^\ast -x_N^\ast } = 2.5029\cdots
\qquad ;$$

}


\frame{
\frametitle{``Chaotic'' regime}

for $\alpha \in [a_\infty ,4)$ aperiodicity sets in, followed by a
fine-structure which is not explained here. In this regime the Lyapunov
exponent is mostly positive, the unfolding of the algorithmic
 information of the initial value.


}


\frame{
\frametitle{``Chaotic'' regime}
for $\alpha =4$ and after the variable transformation $x_n=\sin^2(\pi
X_n)$ one obtains a map $f:X_n\rightarrow X_{n+1}=2X_n$ (mod 1),
where (mod 1) means that one has to drop the integer part of $2X_n$.
By assuming a starting value $X_0$, the formal solution to $n$
iterations is $f^{(n)}(X_0)=X_n=2^nX_0$ (mod 1).
$f$ is easily computable:
if $X_0$ is in binary representation, $f^{(n)}$ is just $n$ times a left
 shift
of the digits of $X_0$, followed by a left truncation before the decimal
point (see Fig. \ref{fig-shift}).
Now assume $X_0\in (0,1)$ is Martin-L\"of/Solovay/Chaitin random.
Then the computable function $f^{(n)}(X_0)$ yields
 a Mar\-tin-L\"of\-/So\-lo\-vay/\-Chai\-tin  random evolution.
It should be stressed again that in {\em ``deterministic chaos,''} the
 evolution function $f$ itself is
computable / recursive, $X_0$ is random, and $f$ ``unfolds'' the
 ``information''
contained in $X_0$ in time.
(For $\alpha \in (4, \infty ) $ the evolution for most  points $x_0\in
(0,1)$ diverges.)

}






%%%%%%%%%%%%%%%%%%%%%%%%%%%%%%%%%%%%%%%%%%%%%%%%%%%%%%%%%%%%%%%%%%%%%%%%%%%%%%%%%%%%%%%%%%%%%%%%%%%%%%%%%
%%%%%%%%%%%%%%%%%%%%%%%%%%%%%%%%%%%%%%%%%%%%%%%%%%%%%%%%%%%%%%%%%%%%%%%%%%%%%%%%%%%%%%%%%%%%%%%%%%%%%%%%%
%%%%%%%%%%%%%%%%%%%%%%%%%%%%%%%%%%%%%%%%%%%%%%%%%%%%%%%%%%%%%%%%%%%%%%%%%%%%%%%%%%%%%%%%%%%%%%%%%%%%%%%%%
%%%%%%%%%%%%%%%%%%%%%%%%%%%%%%%%%%%%%%%%%%%%%%%%%%%%%%%%%%%%%%%%%%%%%%%%%%%%%%%%%%%%%%%%%%%%%%%%%%%%%%%%%







\section{Zwei-Teichenkorrelationen}

\frame{

{\Huge Zwei-Teichenkorrelationen}

}

\subsection{Experimentelle Konfiguration}
\frame{
\frametitle{Experimentelle Konfiguration}

\begin{itemize}
\item<+->
Zwei Me\ss richtungen ${ a(\theta_1,\varphi_1)}$ and ${ b(\theta_2,\varphi_2)}$
von zwei
dichotomischen Observablen mit den Werten ``-1'' und ``1''
an zwei verschiedenen Orten.
\item<+->
Die Me\ss richtung ${a}$ wird am Standort von ``Alice'' gemessen;
Die Me\ss richtung ${b}$ wird am Standort von ``Bob'' gemessen;
(``Alice'' und ``Bob'' kennen die jeweils andere Me\ss richtung nicht.)
\item<+->
Zwei-Teichenkorrelationsfunktion  $E(\theta_1,\varphi_1,\theta_2,\varphi_2 )$
wird definiert �ber die Mittelung der Ausg�nge  $O({ a})_i, O({ b} )_i\in {-1,1}$  des $i$ten
Experimentes; d.h.,
$$E(\theta )={1\over N}\sum_{i=1}^N O({ a})_i O({ b})_i.$$
\end{itemize}
}

\subsection{Klassische Korrelationsfunktion}

\frame{
\frametitle{Klassische Korrelationsfunktion}

Es werden zwei Teilchen angenommen, deren Drehmomente uniform verteilt und gegengerichtet sind
``Singlett''-Zustand.
$$
\begin{array}{l}
E(a,b) = {A_+(a,b)-A_-(a,b)\over 2\pi}= {2A_+(a,b) -2\pi \over 2\pi}=
\\
\qquad \qquad \qquad \qquad  ={2\over \pi}\vert a-b\vert - 1={2\over \pi}\theta - 1
 \end{array}
$$

%TexCad Options
%\grade{\on}
%\emlines{\off}
%\beziermacro{\on}
%\reduce{\on}
%\snapping{\off}
%\quality{8.00}
%\graddiff{0.01}
%\snapasp{1}
%\zoom{0.60}
\unitlength 0.40mm
\linethickness{0.4pt}
\begin{picture}(220.35,68.50)
(0,0)
%\circle(30.25,29.75){61.53}
\put(30.25,60.52){\line(1,0){1.23}}
\put(31.48,60.49){\line(1,0){1.22}}
\multiput(32.70,60.42)(0.61,-0.06){2}{\line(1,0){0.61}}
\multiput(33.92,60.30)(0.61,-0.09){2}{\line(1,0){0.61}}
\multiput(35.14,60.12)(0.60,-0.11){2}{\line(1,0){0.60}}
\multiput(36.34,59.91)(0.40,-0.09){3}{\line(1,0){0.40}}
\multiput(37.54,59.64)(0.40,-0.10){3}{\line(1,0){0.40}}
\multiput(38.73,59.32)(0.29,-0.09){4}{\line(1,0){0.29}}
\multiput(39.90,58.96)(0.29,-0.10){4}{\line(1,0){0.29}}
\multiput(41.06,58.55)(0.28,-0.11){4}{\line(1,0){0.28}}
\multiput(42.20,58.10)(0.22,-0.10){5}{\line(1,0){0.22}}
\multiput(43.32,57.60)(0.22,-0.11){5}{\line(1,0){0.22}}
\multiput(44.42,57.06)(0.22,-0.12){5}{\line(1,0){0.22}}
\multiput(45.50,56.47)(0.18,-0.10){6}{\line(1,0){0.18}}
\multiput(46.55,55.84)(0.17,-0.11){6}{\line(1,0){0.17}}
\multiput(47.58,55.17)(0.17,-0.12){6}{\line(1,0){0.17}}
\multiput(48.58,54.46)(0.14,-0.11){7}{\line(1,0){0.14}}
\multiput(49.55,53.71)(0.13,-0.11){7}{\line(1,0){0.13}}
\multiput(50.49,52.92)(0.13,-0.12){7}{\line(1,0){0.13}}
\multiput(51.40,52.10)(0.11,-0.11){8}{\line(1,0){0.11}}
\multiput(52.27,51.23)(0.12,-0.13){7}{\line(0,-1){0.13}}
\multiput(53.11,50.34)(0.11,-0.13){7}{\line(0,-1){0.13}}
\multiput(53.91,49.41)(0.11,-0.14){7}{\line(0,-1){0.14}}
\multiput(54.68,48.45)(0.10,-0.14){7}{\line(0,-1){0.14}}
\multiput(55.40,47.46)(0.11,-0.17){6}{\line(0,-1){0.17}}
\multiput(56.09,46.45)(0.11,-0.17){6}{\line(0,-1){0.17}}
\multiput(56.74,45.40)(0.10,-0.18){6}{\line(0,-1){0.18}}
\multiput(57.34,44.33)(0.11,-0.22){5}{\line(0,-1){0.22}}
\multiput(57.90,43.24)(0.10,-0.22){5}{\line(0,-1){0.22}}
\multiput(58.41,42.13)(0.12,-0.28){4}{\line(0,-1){0.28}}
\multiput(58.89,41.00)(0.11,-0.29){4}{\line(0,-1){0.29}}
\multiput(59.31,39.85)(0.09,-0.29){4}{\line(0,-1){0.29}}
\multiput(59.69,38.68)(0.11,-0.39){3}{\line(0,-1){0.39}}
\multiput(60.02,37.50)(0.10,-0.40){3}{\line(0,-1){0.40}}
\multiput(60.31,36.30)(0.12,-0.60){2}{\line(0,-1){0.60}}
\multiput(60.55,35.10)(0.09,-0.61){2}{\line(0,-1){0.61}}
\multiput(60.74,33.89)(0.07,-0.61){2}{\line(0,-1){0.61}}
\put(60.88,32.67){\line(0,-1){1.22}}
\put(60.97,31.45){\line(0,-1){1.23}}
\put(61.01,30.22){\line(0,-1){1.23}}
\put(61.01,28.99){\line(0,-1){1.23}}
\put(60.95,27.77){\line(0,-1){1.22}}
\multiput(60.85,26.54)(-0.08,-0.61){2}{\line(0,-1){0.61}}
\multiput(60.70,25.33)(-0.10,-0.61){2}{\line(0,-1){0.61}}
\multiput(60.49,24.12)(-0.08,-0.40){3}{\line(0,-1){0.40}}
\multiput(60.25,22.91)(-0.10,-0.40){3}{\line(0,-1){0.40}}
\multiput(59.95,21.72)(-0.11,-0.39){3}{\line(0,-1){0.39}}
\multiput(59.61,20.55)(-0.10,-0.29){4}{\line(0,-1){0.29}}
\multiput(59.22,19.38)(-0.11,-0.29){4}{\line(0,-1){0.29}}
\multiput(58.78,18.24)(-0.10,-0.23){5}{\line(0,-1){0.23}}
\multiput(58.30,17.11)(-0.11,-0.22){5}{\line(0,-1){0.22}}
\multiput(57.77,16.00)(-0.11,-0.22){5}{\line(0,-1){0.22}}
\multiput(57.20,14.91)(-0.10,-0.18){6}{\line(0,-1){0.18}}
\multiput(56.59,13.85)(-0.11,-0.17){6}{\line(0,-1){0.17}}
\multiput(55.93,12.81)(-0.12,-0.17){6}{\line(0,-1){0.17}}
\multiput(55.24,11.80)(-0.11,-0.14){7}{\line(0,-1){0.14}}
\multiput(54.50,10.82)(-0.11,-0.14){7}{\line(0,-1){0.14}}
\multiput(53.73,9.87)(-0.12,-0.13){7}{\line(0,-1){0.13}}
\multiput(52.92,8.95)(-0.11,-0.11){8}{\line(0,-1){0.11}}
\multiput(52.07,8.06)(-0.11,-0.11){8}{\line(-1,0){0.11}}
\multiput(51.19,7.21)(-0.13,-0.12){7}{\line(-1,0){0.13}}
\multiput(50.27,6.39)(-0.14,-0.11){7}{\line(-1,0){0.14}}
\multiput(49.32,5.61)(-0.14,-0.11){7}{\line(-1,0){0.14}}
\multiput(48.35,4.87)(-0.17,-0.12){6}{\line(-1,0){0.17}}
\multiput(47.34,4.17)(-0.17,-0.11){6}{\line(-1,0){0.17}}
\multiput(46.30,3.51)(-0.18,-0.10){6}{\line(-1,0){0.18}}
\multiput(45.25,2.89)(-0.22,-0.12){5}{\line(-1,0){0.22}}
\multiput(44.16,2.31)(-0.22,-0.11){5}{\line(-1,0){0.22}}
\multiput(43.06,1.78)(-0.23,-0.10){5}{\line(-1,0){0.23}}
\multiput(41.93,1.29)(-0.29,-0.11){4}{\line(-1,0){0.29}}
\multiput(40.79,0.85)(-0.29,-0.10){4}{\line(-1,0){0.29}}
\multiput(39.63,0.45)(-0.39,-0.12){3}{\line(-1,0){0.39}}
\multiput(38.45,0.10)(-0.40,-0.10){3}{\line(-1,0){0.40}}
\multiput(37.26,-0.21)(-0.40,-0.09){3}{\line(-1,0){0.40}}
\multiput(36.06,-0.46)(-0.60,-0.10){2}{\line(-1,0){0.60}}
\multiput(34.85,-0.67)(-0.61,-0.08){2}{\line(-1,0){0.61}}
\put(33.64,-0.83){\line(-1,0){1.22}}
\put(32.41,-0.94){\line(-1,0){1.23}}
\put(31.19,-1.00){\line(-1,0){1.23}}
\put(29.96,-1.01){\line(-1,0){1.23}}
\put(28.74,-0.98){\line(-1,0){1.22}}
\multiput(27.51,-0.89)(-0.61,0.07){2}{\line(-1,0){0.61}}
\multiput(26.29,-0.76)(-0.61,0.09){2}{\line(-1,0){0.61}}
\multiput(25.08,-0.58)(-0.60,0.12){2}{\line(-1,0){0.60}}
\multiput(23.87,-0.35)(-0.40,0.09){3}{\line(-1,0){0.40}}
\multiput(22.68,-0.07)(-0.39,0.11){3}{\line(-1,0){0.39}}
\multiput(21.49,0.26)(-0.29,0.09){4}{\line(-1,0){0.29}}
\multiput(20.33,0.63)(-0.29,0.10){4}{\line(-1,0){0.29}}
\multiput(19.17,1.05)(-0.28,0.12){4}{\line(-1,0){0.28}}
\multiput(18.04,1.51)(-0.22,0.10){5}{\line(-1,0){0.22}}
\multiput(16.92,2.02)(-0.22,0.11){5}{\line(-1,0){0.22}}
\multiput(15.83,2.58)(-0.21,0.12){5}{\line(-1,0){0.21}}
\multiput(14.75,3.17)(-0.17,0.11){6}{\line(-1,0){0.17}}
\multiput(13.71,3.81)(-0.17,0.11){6}{\line(-1,0){0.17}}
\multiput(12.68,4.49)(-0.14,0.10){7}{\line(-1,0){0.14}}
\multiput(11.69,5.21)(-0.14,0.11){7}{\line(-1,0){0.14}}
\multiput(10.73,5.97)(-0.13,0.11){7}{\line(-1,0){0.13}}
\multiput(9.80,6.77)(-0.13,0.12){7}{\line(-1,0){0.13}}
\multiput(8.90,7.60)(-0.11,0.11){8}{\line(0,1){0.11}}
\multiput(8.03,8.47)(-0.12,0.13){7}{\line(0,1){0.13}}
\multiput(7.20,9.38)(-0.11,0.13){7}{\line(0,1){0.13}}
\multiput(6.40,10.31)(-0.11,0.14){7}{\line(0,1){0.14}}
\multiput(5.65,11.28)(-0.12,0.17){6}{\line(0,1){0.17}}
\multiput(4.93,12.27)(-0.11,0.17){6}{\line(0,1){0.17}}
\multiput(4.25,13.30)(-0.11,0.17){6}{\line(0,1){0.17}}
\multiput(3.62,14.35)(-0.12,0.21){5}{\line(0,1){0.21}}
\multiput(3.03,15.42)(-0.11,0.22){5}{\line(0,1){0.22}}
\multiput(2.48,16.52)(-0.10,0.22){5}{\line(0,1){0.22}}
\multiput(1.97,17.64)(-0.12,0.28){4}{\line(0,1){0.28}}
\multiput(1.51,18.77)(-0.10,0.29){4}{\line(0,1){0.29}}
\multiput(1.10,19.93)(-0.09,0.29){4}{\line(0,1){0.29}}
\multiput(0.73,21.10)(-0.11,0.39){3}{\line(0,1){0.39}}
\multiput(0.41,22.28)(-0.09,0.40){3}{\line(0,1){0.40}}
\multiput(0.13,23.48)(-0.11,0.60){2}{\line(0,1){0.60}}
\multiput(-0.10,24.68)(-0.09,0.61){2}{\line(0,1){0.61}}
\multiput(-0.27,25.90)(-0.06,0.61){2}{\line(0,1){0.61}}
\put(-0.40,27.12){\line(0,1){1.22}}
\put(-0.48,28.34){\line(0,1){1.23}}
\put(-0.51,29.57){\line(0,1){1.23}}
\put(-0.50,30.80){\line(0,1){1.23}}
\put(-0.43,32.02){\line(0,1){1.22}}
\multiput(-0.32,33.24)(0.08,0.61){2}{\line(0,1){0.61}}
\multiput(-0.15,34.46)(0.11,0.60){2}{\line(0,1){0.60}}
\multiput(0.06,35.67)(0.09,0.40){3}{\line(0,1){0.40}}
\multiput(0.32,36.87)(0.10,0.40){3}{\line(0,1){0.40}}
\multiput(0.63,38.05)(0.12,0.39){3}{\line(0,1){0.39}}
\multiput(0.98,39.23)(0.10,0.29){4}{\line(0,1){0.29}}
\multiput(1.38,40.39)(0.11,0.29){4}{\line(0,1){0.29}}
\multiput(1.83,41.53)(0.10,0.22){5}{\line(0,1){0.22}}
\multiput(2.32,42.65)(0.11,0.22){5}{\line(0,1){0.22}}
\multiput(2.86,43.76)(0.12,0.22){5}{\line(0,1){0.22}}
\multiput(3.44,44.84)(0.10,0.18){6}{\line(0,1){0.18}}
\multiput(4.06,45.90)(0.11,0.17){6}{\line(0,1){0.17}}
\multiput(4.73,46.93)(0.12,0.17){6}{\line(0,1){0.17}}
\multiput(5.43,47.93)(0.11,0.14){7}{\line(0,1){0.14}}
\multiput(6.18,48.91)(0.11,0.13){7}{\line(0,1){0.13}}
\multiput(6.96,49.85)(0.12,0.13){7}{\line(0,1){0.13}}
\multiput(7.78,50.76)(0.11,0.11){8}{\line(0,1){0.11}}
\multiput(8.64,51.64)(0.11,0.11){8}{\line(1,0){0.11}}
\multiput(9.53,52.49)(0.13,0.12){7}{\line(1,0){0.13}}
\multiput(10.45,53.30)(0.14,0.11){7}{\line(1,0){0.14}}
\multiput(11.40,54.07)(0.14,0.10){7}{\line(1,0){0.14}}
\multiput(12.39,54.80)(0.17,0.12){6}{\line(1,0){0.17}}
\multiput(13.40,55.49)(0.17,0.11){6}{\line(1,0){0.17}}
\multiput(14.44,56.14)(0.18,0.10){6}{\line(1,0){0.18}}
\multiput(15.51,56.75)(0.22,0.11){5}{\line(1,0){0.22}}
\multiput(16.60,57.32)(0.22,0.10){5}{\line(1,0){0.22}}
\multiput(17.71,57.84)(0.28,0.12){4}{\line(1,0){0.28}}
\multiput(18.84,58.32)(0.29,0.11){4}{\line(1,0){0.29}}
\multiput(19.98,58.75)(0.29,0.10){4}{\line(1,0){0.29}}
\multiput(21.15,59.14)(0.39,0.11){3}{\line(1,0){0.39}}
\multiput(22.33,59.48)(0.40,0.10){3}{\line(1,0){0.40}}
\multiput(23.52,59.77)(0.40,0.08){3}{\line(1,0){0.40}}
\multiput(24.72,60.01)(0.61,0.10){2}{\line(1,0){0.61}}
\multiput(25.93,60.21)(0.61,0.07){2}{\line(1,0){0.61}}
\put(27.15,60.36){\line(1,0){1.22}}
\put(28.37,60.46){\line(1,0){1.88}}
%\end
\put(30.25,30.25){\line(0,1){30.50}}
%\dottedline(1.75,235.75)(2,235.25)
\multiput(1.68,235.68)(.125,-.25){3}{{\rule{.4pt}{.4pt}}}
\put(0.00,0.00){}
%\dottedline(2,235.25)(63.5,234.75)
\multiput(1.93,235.18)(.99194,-.00806){63}{{\rule{.4pt}{.4pt}}}
\put(0.00,0.00){}
%\dottedline(9.5,255.5)(55.75,214.25)
\multiput(9.43,255.43)(.72266,-.64453){65}{{\rule{.4pt}{.4pt}}}
\put(0.00,0.00){}
\put(30.25,68.50){\makebox(0,0)[cc]{$a$}}
%\emline(0.00,30.00)(61.00,30.00)
\put(0.00,30.00){\line(1,0){61.00}}
%\end
%\circle(109.92,29.75){61.53}
\put(109.92,60.52){\line(1,0){1.23}}
\put(111.15,60.49){\line(1,0){1.22}}
\multiput(112.37,60.42)(0.61,-0.06){2}{\line(1,0){0.61}}
\multiput(113.59,60.30)(0.61,-0.09){2}{\line(1,0){0.61}}
\multiput(114.81,60.12)(0.60,-0.11){2}{\line(1,0){0.60}}
\multiput(116.01,59.91)(0.40,-0.09){3}{\line(1,0){0.40}}
\multiput(117.21,59.64)(0.40,-0.10){3}{\line(1,0){0.40}}
\multiput(118.40,59.32)(0.29,-0.09){4}{\line(1,0){0.29}}
\multiput(119.57,58.96)(0.29,-0.10){4}{\line(1,0){0.29}}
\multiput(120.73,58.55)(0.28,-0.11){4}{\line(1,0){0.28}}
\multiput(121.87,58.10)(0.22,-0.10){5}{\line(1,0){0.22}}
\multiput(122.99,57.60)(0.22,-0.11){5}{\line(1,0){0.22}}
\multiput(124.09,57.06)(0.22,-0.12){5}{\line(1,0){0.22}}
\multiput(125.17,56.47)(0.18,-0.10){6}{\line(1,0){0.18}}
\multiput(126.22,55.84)(0.17,-0.11){6}{\line(1,0){0.17}}
\multiput(127.25,55.17)(0.17,-0.12){6}{\line(1,0){0.17}}
\multiput(128.25,54.46)(0.14,-0.11){7}{\line(1,0){0.14}}
\multiput(129.22,53.71)(0.13,-0.11){7}{\line(1,0){0.13}}
\multiput(130.16,52.92)(0.13,-0.12){7}{\line(1,0){0.13}}
\multiput(131.07,52.10)(0.11,-0.11){8}{\line(1,0){0.11}}
\multiput(131.94,51.23)(0.12,-0.13){7}{\line(0,-1){0.13}}
\multiput(132.78,50.34)(0.11,-0.13){7}{\line(0,-1){0.13}}
\multiput(133.58,49.41)(0.11,-0.14){7}{\line(0,-1){0.14}}
\multiput(134.35,48.45)(0.10,-0.14){7}{\line(0,-1){0.14}}
\multiput(135.07,47.46)(0.11,-0.17){6}{\line(0,-1){0.17}}
\multiput(135.76,46.45)(0.11,-0.17){6}{\line(0,-1){0.17}}
\multiput(136.41,45.40)(0.10,-0.18){6}{\line(0,-1){0.18}}
\multiput(137.01,44.33)(0.11,-0.22){5}{\line(0,-1){0.22}}
\multiput(137.57,43.24)(0.10,-0.22){5}{\line(0,-1){0.22}}
\multiput(138.08,42.13)(0.12,-0.28){4}{\line(0,-1){0.28}}
\multiput(138.56,41.00)(0.11,-0.29){4}{\line(0,-1){0.29}}
\multiput(138.98,39.85)(0.09,-0.29){4}{\line(0,-1){0.29}}
\multiput(139.36,38.68)(0.11,-0.39){3}{\line(0,-1){0.39}}
\multiput(139.69,37.50)(0.10,-0.40){3}{\line(0,-1){0.40}}
\multiput(139.98,36.30)(0.12,-0.60){2}{\line(0,-1){0.60}}
\multiput(140.22,35.10)(0.09,-0.61){2}{\line(0,-1){0.61}}
\multiput(140.41,33.89)(0.07,-0.61){2}{\line(0,-1){0.61}}
\put(140.55,32.67){\line(0,-1){1.22}}
\put(140.64,31.45){\line(0,-1){1.23}}
\put(140.68,30.22){\line(0,-1){1.23}}
\put(140.68,28.99){\line(0,-1){1.23}}
\put(140.62,27.77){\line(0,-1){1.22}}
\multiput(140.52,26.54)(-0.08,-0.61){2}{\line(0,-1){0.61}}
\multiput(140.37,25.33)(-0.10,-0.61){2}{\line(0,-1){0.61}}
\multiput(140.16,24.12)(-0.08,-0.40){3}{\line(0,-1){0.40}}
\multiput(139.92,22.91)(-0.10,-0.40){3}{\line(0,-1){0.40}}
\multiput(139.62,21.72)(-0.11,-0.39){3}{\line(0,-1){0.39}}
\multiput(139.28,20.55)(-0.10,-0.29){4}{\line(0,-1){0.29}}
\multiput(138.89,19.38)(-0.11,-0.29){4}{\line(0,-1){0.29}}
\multiput(138.45,18.24)(-0.10,-0.23){5}{\line(0,-1){0.23}}
\multiput(137.97,17.11)(-0.11,-0.22){5}{\line(0,-1){0.22}}
\multiput(137.44,16.00)(-0.11,-0.22){5}{\line(0,-1){0.22}}
\multiput(136.87,14.91)(-0.10,-0.18){6}{\line(0,-1){0.18}}
\multiput(136.26,13.85)(-0.11,-0.17){6}{\line(0,-1){0.17}}
\multiput(135.60,12.81)(-0.12,-0.17){6}{\line(0,-1){0.17}}
\multiput(134.91,11.80)(-0.11,-0.14){7}{\line(0,-1){0.14}}
\multiput(134.17,10.82)(-0.11,-0.14){7}{\line(0,-1){0.14}}
\multiput(133.40,9.87)(-0.12,-0.13){7}{\line(0,-1){0.13}}
\multiput(132.59,8.95)(-0.11,-0.11){8}{\line(0,-1){0.11}}
\multiput(131.74,8.06)(-0.11,-0.11){8}{\line(-1,0){0.11}}
\multiput(130.86,7.21)(-0.13,-0.12){7}{\line(-1,0){0.13}}
\multiput(129.94,6.39)(-0.14,-0.11){7}{\line(-1,0){0.14}}
\multiput(128.99,5.61)(-0.14,-0.11){7}{\line(-1,0){0.14}}
\multiput(128.02,4.87)(-0.17,-0.12){6}{\line(-1,0){0.17}}
\multiput(127.01,4.17)(-0.17,-0.11){6}{\line(-1,0){0.17}}
\multiput(125.97,3.51)(-0.18,-0.10){6}{\line(-1,0){0.18}}
\multiput(124.92,2.89)(-0.22,-0.12){5}{\line(-1,0){0.22}}
\multiput(123.83,2.31)(-0.22,-0.11){5}{\line(-1,0){0.22}}
\multiput(122.73,1.78)(-0.23,-0.10){5}{\line(-1,0){0.23}}
\multiput(121.60,1.29)(-0.29,-0.11){4}{\line(-1,0){0.29}}
\multiput(120.46,0.85)(-0.29,-0.10){4}{\line(-1,0){0.29}}
\multiput(119.30,0.45)(-0.39,-0.12){3}{\line(-1,0){0.39}}
\multiput(118.12,0.10)(-0.40,-0.10){3}{\line(-1,0){0.40}}
\multiput(116.93,-0.21)(-0.40,-0.09){3}{\line(-1,0){0.40}}
\multiput(115.73,-0.46)(-0.60,-0.10){2}{\line(-1,0){0.60}}
\multiput(114.52,-0.67)(-0.61,-0.08){2}{\line(-1,0){0.61}}
\put(113.31,-0.83){\line(-1,0){1.22}}
\put(112.08,-0.94){\line(-1,0){1.23}}
\put(110.86,-1.00){\line(-1,0){1.23}}
\put(109.63,-1.01){\line(-1,0){1.23}}
\put(108.41,-0.98){\line(-1,0){1.22}}
\multiput(107.18,-0.89)(-0.61,0.07){2}{\line(-1,0){0.61}}
\multiput(105.96,-0.76)(-0.61,0.09){2}{\line(-1,0){0.61}}
\multiput(104.75,-0.58)(-0.60,0.12){2}{\line(-1,0){0.60}}
\multiput(103.54,-0.35)(-0.40,0.09){3}{\line(-1,0){0.40}}
\multiput(102.35,-0.07)(-0.39,0.11){3}{\line(-1,0){0.39}}
\multiput(101.16,0.26)(-0.29,0.09){4}{\line(-1,0){0.29}}
\multiput(100.00,0.63)(-0.29,0.10){4}{\line(-1,0){0.29}}
\multiput(98.84,1.05)(-0.28,0.12){4}{\line(-1,0){0.28}}
\multiput(97.71,1.51)(-0.22,0.10){5}{\line(-1,0){0.22}}
\multiput(96.59,2.02)(-0.22,0.11){5}{\line(-1,0){0.22}}
\multiput(95.50,2.58)(-0.21,0.12){5}{\line(-1,0){0.21}}
\multiput(94.42,3.17)(-0.17,0.11){6}{\line(-1,0){0.17}}
\multiput(93.38,3.81)(-0.17,0.11){6}{\line(-1,0){0.17}}
\multiput(92.35,4.49)(-0.14,0.10){7}{\line(-1,0){0.14}}
\multiput(91.36,5.21)(-0.14,0.11){7}{\line(-1,0){0.14}}
\multiput(90.40,5.97)(-0.13,0.11){7}{\line(-1,0){0.13}}
\multiput(89.47,6.77)(-0.13,0.12){7}{\line(-1,0){0.13}}
\multiput(88.57,7.60)(-0.11,0.11){8}{\line(0,1){0.11}}
\multiput(87.70,8.47)(-0.12,0.13){7}{\line(0,1){0.13}}
\multiput(86.87,9.38)(-0.11,0.13){7}{\line(0,1){0.13}}
\multiput(86.07,10.31)(-0.11,0.14){7}{\line(0,1){0.14}}
\multiput(85.32,11.28)(-0.12,0.17){6}{\line(0,1){0.17}}
\multiput(84.60,12.27)(-0.11,0.17){6}{\line(0,1){0.17}}
\multiput(83.92,13.30)(-0.11,0.17){6}{\line(0,1){0.17}}
\multiput(83.29,14.35)(-0.12,0.21){5}{\line(0,1){0.21}}
\multiput(82.70,15.42)(-0.11,0.22){5}{\line(0,1){0.22}}
\multiput(82.15,16.52)(-0.10,0.22){5}{\line(0,1){0.22}}
\multiput(81.64,17.64)(-0.12,0.28){4}{\line(0,1){0.28}}
\multiput(81.18,18.77)(-0.10,0.29){4}{\line(0,1){0.29}}
\multiput(80.77,19.93)(-0.09,0.29){4}{\line(0,1){0.29}}
\multiput(80.40,21.10)(-0.11,0.39){3}{\line(0,1){0.39}}
\multiput(80.08,22.28)(-0.09,0.40){3}{\line(0,1){0.40}}
\multiput(79.80,23.48)(-0.11,0.60){2}{\line(0,1){0.60}}
\multiput(79.57,24.68)(-0.09,0.61){2}{\line(0,1){0.61}}
\multiput(79.40,25.90)(-0.06,0.61){2}{\line(0,1){0.61}}
\put(79.27,27.12){\line(0,1){1.22}}
\put(79.19,28.34){\line(0,1){1.23}}
\put(79.16,29.57){\line(0,1){1.23}}
\put(79.17,30.80){\line(0,1){1.23}}
\put(79.24,32.02){\line(0,1){1.22}}
\multiput(79.35,33.24)(0.08,0.61){2}{\line(0,1){0.61}}
\multiput(79.52,34.46)(0.11,0.60){2}{\line(0,1){0.60}}
\multiput(79.73,35.67)(0.09,0.40){3}{\line(0,1){0.40}}
\multiput(79.99,36.87)(0.10,0.40){3}{\line(0,1){0.40}}
\multiput(80.30,38.05)(0.12,0.39){3}{\line(0,1){0.39}}
\multiput(80.65,39.23)(0.10,0.29){4}{\line(0,1){0.29}}
\multiput(81.05,40.39)(0.11,0.29){4}{\line(0,1){0.29}}
\multiput(81.50,41.53)(0.10,0.22){5}{\line(0,1){0.22}}
\multiput(81.99,42.65)(0.11,0.22){5}{\line(0,1){0.22}}
\multiput(82.53,43.76)(0.12,0.22){5}{\line(0,1){0.22}}
\multiput(83.11,44.84)(0.10,0.18){6}{\line(0,1){0.18}}
\multiput(83.73,45.90)(0.11,0.17){6}{\line(0,1){0.17}}
\multiput(84.40,46.93)(0.12,0.17){6}{\line(0,1){0.17}}
\multiput(85.10,47.93)(0.11,0.14){7}{\line(0,1){0.14}}
\multiput(85.85,48.91)(0.11,0.13){7}{\line(0,1){0.13}}
\multiput(86.63,49.85)(0.12,0.13){7}{\line(0,1){0.13}}
\multiput(87.45,50.76)(0.11,0.11){8}{\line(0,1){0.11}}
\multiput(88.31,51.64)(0.11,0.11){8}{\line(1,0){0.11}}
\multiput(89.20,52.49)(0.13,0.12){7}{\line(1,0){0.13}}
\multiput(90.12,53.30)(0.14,0.11){7}{\line(1,0){0.14}}
\multiput(91.07,54.07)(0.14,0.10){7}{\line(1,0){0.14}}
\multiput(92.06,54.80)(0.17,0.12){6}{\line(1,0){0.17}}
\multiput(93.07,55.49)(0.17,0.11){6}{\line(1,0){0.17}}
\multiput(94.11,56.14)(0.18,0.10){6}{\line(1,0){0.18}}
\multiput(95.18,56.75)(0.22,0.11){5}{\line(1,0){0.22}}
\multiput(96.27,57.32)(0.22,0.10){5}{\line(1,0){0.22}}
\multiput(97.38,57.84)(0.28,0.12){4}{\line(1,0){0.28}}
\multiput(98.51,58.32)(0.29,0.11){4}{\line(1,0){0.29}}
\multiput(99.65,58.75)(0.29,0.10){4}{\line(1,0){0.29}}
\multiput(100.82,59.14)(0.39,0.11){3}{\line(1,0){0.39}}
\multiput(102.00,59.48)(0.40,0.10){3}{\line(1,0){0.40}}
\multiput(103.19,59.77)(0.40,0.08){3}{\line(1,0){0.40}}
\multiput(104.39,60.01)(0.61,0.10){2}{\line(1,0){0.61}}
\multiput(105.60,60.21)(0.61,0.07){2}{\line(1,0){0.61}}
\put(106.82,60.36){\line(1,0){1.22}}
\put(108.04,60.46){\line(1,0){1.88}}
%\end
%\circle(189.58,29.75){61.53}
\put(189.58,60.52){\line(1,0){1.23}}
\put(190.81,60.49){\line(1,0){1.22}}
\multiput(192.03,60.42)(0.61,-0.06){2}{\line(1,0){0.61}}
\multiput(193.25,60.30)(0.61,-0.09){2}{\line(1,0){0.61}}
\multiput(194.47,60.12)(0.60,-0.11){2}{\line(1,0){0.60}}
\multiput(195.67,59.91)(0.40,-0.09){3}{\line(1,0){0.40}}
\multiput(196.87,59.64)(0.40,-0.10){3}{\line(1,0){0.40}}
\multiput(198.06,59.32)(0.29,-0.09){4}{\line(1,0){0.29}}
\multiput(199.23,58.96)(0.29,-0.10){4}{\line(1,0){0.29}}
\multiput(200.39,58.55)(0.28,-0.11){4}{\line(1,0){0.28}}
\multiput(201.53,58.10)(0.22,-0.10){5}{\line(1,0){0.22}}
\multiput(202.65,57.60)(0.22,-0.11){5}{\line(1,0){0.22}}
\multiput(203.75,57.06)(0.22,-0.12){5}{\line(1,0){0.22}}
\multiput(204.83,56.47)(0.18,-0.10){6}{\line(1,0){0.18}}
\multiput(205.88,55.84)(0.17,-0.11){6}{\line(1,0){0.17}}
\multiput(206.91,55.17)(0.17,-0.12){6}{\line(1,0){0.17}}
\multiput(207.91,54.46)(0.14,-0.11){7}{\line(1,0){0.14}}
\multiput(208.88,53.71)(0.13,-0.11){7}{\line(1,0){0.13}}
\multiput(209.82,52.92)(0.13,-0.12){7}{\line(1,0){0.13}}
\multiput(210.73,52.10)(0.11,-0.11){8}{\line(1,0){0.11}}
\multiput(211.60,51.23)(0.12,-0.13){7}{\line(0,-1){0.13}}
\multiput(212.44,50.34)(0.11,-0.13){7}{\line(0,-1){0.13}}
\multiput(213.24,49.41)(0.11,-0.14){7}{\line(0,-1){0.14}}
\multiput(214.01,48.45)(0.10,-0.14){7}{\line(0,-1){0.14}}
\multiput(214.73,47.46)(0.11,-0.17){6}{\line(0,-1){0.17}}
\multiput(215.42,46.45)(0.11,-0.17){6}{\line(0,-1){0.17}}
\multiput(216.07,45.40)(0.10,-0.18){6}{\line(0,-1){0.18}}
\multiput(216.67,44.33)(0.11,-0.22){5}{\line(0,-1){0.22}}
\multiput(217.23,43.24)(0.10,-0.22){5}{\line(0,-1){0.22}}
\multiput(217.74,42.13)(0.12,-0.28){4}{\line(0,-1){0.28}}
\multiput(218.22,41.00)(0.11,-0.29){4}{\line(0,-1){0.29}}
\multiput(218.64,39.85)(0.09,-0.29){4}{\line(0,-1){0.29}}
\multiput(219.02,38.68)(0.11,-0.39){3}{\line(0,-1){0.39}}
\multiput(219.35,37.50)(0.10,-0.40){3}{\line(0,-1){0.40}}
\multiput(219.64,36.30)(0.12,-0.60){2}{\line(0,-1){0.60}}
\multiput(219.88,35.10)(0.09,-0.61){2}{\line(0,-1){0.61}}
\multiput(220.07,33.89)(0.07,-0.61){2}{\line(0,-1){0.61}}
\put(220.21,32.67){\line(0,-1){1.22}}
\put(220.30,31.45){\line(0,-1){1.23}}
\put(220.34,30.22){\line(0,-1){1.23}}
\put(220.34,28.99){\line(0,-1){1.23}}
\put(220.28,27.77){\line(0,-1){1.22}}
\multiput(220.18,26.54)(-0.08,-0.61){2}{\line(0,-1){0.61}}
\multiput(220.03,25.33)(-0.10,-0.61){2}{\line(0,-1){0.61}}
\multiput(219.82,24.12)(-0.08,-0.40){3}{\line(0,-1){0.40}}
\multiput(219.58,22.91)(-0.10,-0.40){3}{\line(0,-1){0.40}}
\multiput(219.28,21.72)(-0.11,-0.39){3}{\line(0,-1){0.39}}
\multiput(218.94,20.55)(-0.10,-0.29){4}{\line(0,-1){0.29}}
\multiput(218.55,19.38)(-0.11,-0.29){4}{\line(0,-1){0.29}}
\multiput(218.11,18.24)(-0.10,-0.23){5}{\line(0,-1){0.23}}
\multiput(217.63,17.11)(-0.11,-0.22){5}{\line(0,-1){0.22}}
\multiput(217.10,16.00)(-0.11,-0.22){5}{\line(0,-1){0.22}}
\multiput(216.53,14.91)(-0.10,-0.18){6}{\line(0,-1){0.18}}
\multiput(215.92,13.85)(-0.11,-0.17){6}{\line(0,-1){0.17}}
\multiput(215.26,12.81)(-0.12,-0.17){6}{\line(0,-1){0.17}}
\multiput(214.57,11.80)(-0.11,-0.14){7}{\line(0,-1){0.14}}
\multiput(213.83,10.82)(-0.11,-0.14){7}{\line(0,-1){0.14}}
\multiput(213.06,9.87)(-0.12,-0.13){7}{\line(0,-1){0.13}}
\multiput(212.25,8.95)(-0.11,-0.11){8}{\line(0,-1){0.11}}
\multiput(211.40,8.06)(-0.11,-0.11){8}{\line(-1,0){0.11}}
\multiput(210.52,7.21)(-0.13,-0.12){7}{\line(-1,0){0.13}}
\multiput(209.60,6.39)(-0.14,-0.11){7}{\line(-1,0){0.14}}
\multiput(208.65,5.61)(-0.14,-0.11){7}{\line(-1,0){0.14}}
\multiput(207.68,4.87)(-0.17,-0.12){6}{\line(-1,0){0.17}}
\multiput(206.67,4.17)(-0.17,-0.11){6}{\line(-1,0){0.17}}
\multiput(205.63,3.51)(-0.18,-0.10){6}{\line(-1,0){0.18}}
\multiput(204.58,2.89)(-0.22,-0.12){5}{\line(-1,0){0.22}}
\multiput(203.49,2.31)(-0.22,-0.11){5}{\line(-1,0){0.22}}
\multiput(202.39,1.78)(-0.23,-0.10){5}{\line(-1,0){0.23}}
\multiput(201.26,1.29)(-0.29,-0.11){4}{\line(-1,0){0.29}}
\multiput(200.12,0.85)(-0.29,-0.10){4}{\line(-1,0){0.29}}
\multiput(198.96,0.45)(-0.39,-0.12){3}{\line(-1,0){0.39}}
\multiput(197.78,0.10)(-0.40,-0.10){3}{\line(-1,0){0.40}}
\multiput(196.59,-0.21)(-0.40,-0.09){3}{\line(-1,0){0.40}}
\multiput(195.39,-0.46)(-0.60,-0.10){2}{\line(-1,0){0.60}}
\multiput(194.18,-0.67)(-0.61,-0.08){2}{\line(-1,0){0.61}}
\put(192.97,-0.83){\line(-1,0){1.22}}
\put(191.74,-0.94){\line(-1,0){1.23}}
\put(190.52,-1.00){\line(-1,0){1.23}}
\put(189.29,-1.01){\line(-1,0){1.23}}
\put(188.07,-0.98){\line(-1,0){1.22}}
\multiput(186.84,-0.89)(-0.61,0.07){2}{\line(-1,0){0.61}}
\multiput(185.62,-0.76)(-0.61,0.09){2}{\line(-1,0){0.61}}
\multiput(184.41,-0.58)(-0.60,0.12){2}{\line(-1,0){0.60}}
\multiput(183.20,-0.35)(-0.40,0.09){3}{\line(-1,0){0.40}}
\multiput(182.01,-0.07)(-0.39,0.11){3}{\line(-1,0){0.39}}
\multiput(180.82,0.26)(-0.29,0.09){4}{\line(-1,0){0.29}}
\multiput(179.66,0.63)(-0.29,0.10){4}{\line(-1,0){0.29}}
\multiput(178.50,1.05)(-0.28,0.12){4}{\line(-1,0){0.28}}
\multiput(177.37,1.51)(-0.22,0.10){5}{\line(-1,0){0.22}}
\multiput(176.25,2.02)(-0.22,0.11){5}{\line(-1,0){0.22}}
\multiput(175.16,2.58)(-0.21,0.12){5}{\line(-1,0){0.21}}
\multiput(174.08,3.17)(-0.17,0.11){6}{\line(-1,0){0.17}}
\multiput(173.04,3.81)(-0.17,0.11){6}{\line(-1,0){0.17}}
\multiput(172.01,4.49)(-0.14,0.10){7}{\line(-1,0){0.14}}
\multiput(171.02,5.21)(-0.14,0.11){7}{\line(-1,0){0.14}}
\multiput(170.06,5.97)(-0.13,0.11){7}{\line(-1,0){0.13}}
\multiput(169.13,6.77)(-0.13,0.12){7}{\line(-1,0){0.13}}
\multiput(168.23,7.60)(-0.11,0.11){8}{\line(0,1){0.11}}
\multiput(167.36,8.47)(-0.12,0.13){7}{\line(0,1){0.13}}
\multiput(166.53,9.38)(-0.11,0.13){7}{\line(0,1){0.13}}
\multiput(165.73,10.31)(-0.11,0.14){7}{\line(0,1){0.14}}
\multiput(164.98,11.28)(-0.12,0.17){6}{\line(0,1){0.17}}
\multiput(164.26,12.27)(-0.11,0.17){6}{\line(0,1){0.17}}
\multiput(163.58,13.30)(-0.11,0.17){6}{\line(0,1){0.17}}
\multiput(162.95,14.35)(-0.12,0.21){5}{\line(0,1){0.21}}
\multiput(162.36,15.42)(-0.11,0.22){5}{\line(0,1){0.22}}
\multiput(161.81,16.52)(-0.10,0.22){5}{\line(0,1){0.22}}
\multiput(161.30,17.64)(-0.12,0.28){4}{\line(0,1){0.28}}
\multiput(160.84,18.77)(-0.10,0.29){4}{\line(0,1){0.29}}
\multiput(160.43,19.93)(-0.09,0.29){4}{\line(0,1){0.29}}
\multiput(160.06,21.10)(-0.11,0.39){3}{\line(0,1){0.39}}
\multiput(159.74,22.28)(-0.09,0.40){3}{\line(0,1){0.40}}
\multiput(159.46,23.48)(-0.11,0.60){2}{\line(0,1){0.60}}
\multiput(159.23,24.68)(-0.09,0.61){2}{\line(0,1){0.61}}
\multiput(159.06,25.90)(-0.06,0.61){2}{\line(0,1){0.61}}
\put(158.93,27.12){\line(0,1){1.22}}
\put(158.85,28.34){\line(0,1){1.23}}
\put(158.82,29.57){\line(0,1){1.23}}
\put(158.83,30.80){\line(0,1){1.23}}
\put(158.90,32.02){\line(0,1){1.22}}
\multiput(159.01,33.24)(0.08,0.61){2}{\line(0,1){0.61}}
\multiput(159.18,34.46)(0.11,0.60){2}{\line(0,1){0.60}}
\multiput(159.39,35.67)(0.09,0.40){3}{\line(0,1){0.40}}
\multiput(159.65,36.87)(0.10,0.40){3}{\line(0,1){0.40}}
\multiput(159.96,38.05)(0.12,0.39){3}{\line(0,1){0.39}}
\multiput(160.31,39.23)(0.10,0.29){4}{\line(0,1){0.29}}
\multiput(160.71,40.39)(0.11,0.29){4}{\line(0,1){0.29}}
\multiput(161.16,41.53)(0.10,0.22){5}{\line(0,1){0.22}}
\multiput(161.65,42.65)(0.11,0.22){5}{\line(0,1){0.22}}
\multiput(162.19,43.76)(0.12,0.22){5}{\line(0,1){0.22}}
\multiput(162.77,44.84)(0.10,0.18){6}{\line(0,1){0.18}}
\multiput(163.39,45.90)(0.11,0.17){6}{\line(0,1){0.17}}
\multiput(164.06,46.93)(0.12,0.17){6}{\line(0,1){0.17}}
\multiput(164.76,47.93)(0.11,0.14){7}{\line(0,1){0.14}}
\multiput(165.51,48.91)(0.11,0.13){7}{\line(0,1){0.13}}
\multiput(166.29,49.85)(0.12,0.13){7}{\line(0,1){0.13}}
\multiput(167.11,50.76)(0.11,0.11){8}{\line(0,1){0.11}}
\multiput(167.97,51.64)(0.11,0.11){8}{\line(1,0){0.11}}
\multiput(168.86,52.49)(0.13,0.12){7}{\line(1,0){0.13}}
\multiput(169.78,53.30)(0.14,0.11){7}{\line(1,0){0.14}}
\multiput(170.73,54.07)(0.14,0.10){7}{\line(1,0){0.14}}
\multiput(171.72,54.80)(0.17,0.12){6}{\line(1,0){0.17}}
\multiput(172.73,55.49)(0.17,0.11){6}{\line(1,0){0.17}}
\multiput(173.77,56.14)(0.18,0.10){6}{\line(1,0){0.18}}
\multiput(174.84,56.75)(0.22,0.11){5}{\line(1,0){0.22}}
\multiput(175.93,57.32)(0.22,0.10){5}{\line(1,0){0.22}}
\multiput(177.04,57.84)(0.28,0.12){4}{\line(1,0){0.28}}
\multiput(178.17,58.32)(0.29,0.11){4}{\line(1,0){0.29}}
\multiput(179.31,58.75)(0.29,0.10){4}{\line(1,0){0.29}}
\multiput(180.48,59.14)(0.39,0.11){3}{\line(1,0){0.39}}
\multiput(181.66,59.48)(0.40,0.10){3}{\line(1,0){0.40}}
\multiput(182.85,59.77)(0.40,0.08){3}{\line(1,0){0.40}}
\multiput(184.05,60.01)(0.61,0.10){2}{\line(1,0){0.61}}
\multiput(185.26,60.21)(0.61,0.07){2}{\line(1,0){0.61}}
\put(186.48,60.36){\line(1,0){1.22}}
\put(187.70,60.46){\line(1,0){1.88}}
%\end
\put(189.58,30.25){\line(0,1){30.50}}
\put(189.58,68.50){\makebox(0,0)[cc]{$a$}}
\put(133.61,62.94){\makebox(0,0)[cc]{$b$}}
\put(213.28,62.94){\makebox(0,0)[cc]{$b$}}
\put(182.83,13.50){\makebox(0,0)[cc]{$-$}}
\put(210.08,40.50){\makebox(0,0)[cc]{$-$}}
\put(165.08,38.00){\makebox(0,0)[cc]{$+$}}
\put(211.58,21.25){\makebox(0,0)[cc]{$+$}}
%\emline(84.33,46.33)(135.67,13.00)
\multiput(84.33,46.33)(0.18,-0.12){278}{\line(1,0){0.18}}
%\end
%\emline(164.00,46.33)(215.33,13.00)
\multiput(164.00,46.33)(0.18,-0.12){278}{\line(1,0){0.18}}
%\end
%\emline(159.33,30.00)(220.33,30.00)
\put(159.33,30.00){\line(1,0){61.00}}
%\end
%\emline(110.00,30.00)(128.33,54.00)
\multiput(110.00,30.00)(0.12,0.16){153}{\line(0,1){0.16}}
%\end
%\emline(189.44,30.00)(207.78,54.00)
\multiput(189.44,30.00)(0.12,0.16){153}{\line(0,1){0.16}}
%\end
\put(18.89,42.78){\makebox(0,0)[cc]{$+$}}
\put(29.44,12.22){\makebox(0,0)[cc]{$-$}}
\put(110.56,46.67){\makebox(0,0)[cc]{$-$}}
\put(99.44,17.22){\makebox(0,0)[cc]{$+$}}
\bezier{44}(189.44,45.00)(195.00,46.11)(198.89,42.22)
\bezier{56}(171.67,30.00)(167.78,36.11)(172.78,40.00)
\bezier{60}(206.11,30.00)(210.00,23.33)(202.78,21.11)
\end{picture}


 }



\subsection{Quantenkorrelationen}


\frame[containsverbatim]{
\frametitle{Definition des Tensorprodukts}


$$(a\otimes b)_{ij} \equiv
\left(
\begin{array}{cccc}
a_{11}b & a_{12}b&\cdots \\
a_{21}b & a_{22}b&\cdots \\
\cdots &\cdots & \ddots
\end{array}
\right)
=
\left(
\begin{array}{cccc}
a_{11}b_{11} & a_{11}b_{12}&\cdots \\
a_{11}b_{21} & a_{11}b_{22}&\cdots \\
\cdots &\cdots & \ddots
\end{array}
\right)
$$


{\tiny
\begin{verbatim}
(* Definition des Tensorproducts *)

TensorProduct[a_, b_] :=    Table[(*a, b are nxn and mxm - matrices*)
a[[Ceiling[s/Length[b]], Ceiling[t/Length[b]]]]*b[[s - Floor[(s -
1)/Length[b]]*Length[b],t - Floor[(t - 1)/Length[b]]*Length[b]]], {s,
  1,Length[a]*Length[b]}, {t, 1, Length[a]*Length[b]}];

(* Definition des Tensorproducts zwischen Vektoren *)

TensorProductVec[x_, y_] := Flatten[Table[ x[[i]] y[[j]] , {i, 1, Length[x]}, {j, 1, Length[y]}]];
\end{verbatim}
}

}


\frame[containsverbatim]{
\frametitle{Definition des Projektors durch dyadisches Produkts eines Vektors}


$$(a\otimes a)_{ij} \equiv
\left(
\begin{array}{cccc}
a_1a_1& a_1a_2&\cdots \\
a_2a_1& a_2a_2&\cdots \\
\cdots &\cdots & \ddots
\end{array}
\right)
$$


{\tiny

\begin{verbatim}
(* Definition des dyadischen Produkts *)

DyadicProductVec[x_] :=  Table[x[[i]] x[[j]], {i, 1, Length[x]}, {j, 1, Length[x]}];
\end{verbatim}
}
}



\frame[containsverbatim]{
\frametitle{Definition Sigma--Matrizen}


$$ \sigma (r,\theta,\varphi ) =r
\left(
\begin{array}{cccc}
\cos \theta & e^{-i\varphi}\sin \theta \\
e^{i\varphi}\sin \theta& \cos \theta
\end{array}
\right)
$$


{\tiny
\begin{verbatim}
(* Definition of the sigma matrices *)


vecsig[r_, tt_, p_ ]:= r * { {Cos[tt], Sin[tt] Exp[-I p]}, {Sin[tt] Exp[I p], -Cos[tt]}}
\end{verbatim}
}

}


\frame[containsverbatim]{
\frametitle{Basen}

{\tiny
\begin{verbatim}
BellBasis = (1/Sqrt[2]) {{1, 0, 0, 1}, {0, 1, 1, 0}, {0, 1, -1, 0}, {1, 0, 0,-1}};

Basis =  {{1, 0, 0, 0}, {0, 1, 0, 0}, {0, 0, 1, 0}, {0, 0, 0,1}};

vp = {0,1};
vm = {1,0};
\end{verbatim}
}

}


\frame[containsverbatim]{
\frametitle{Definition des zwei--Teilchen Singletzustandes}


$$\vert \Psi \langle ={1\over \sqrt{2}}[(0,1)\otimes (1,0)-(1,0)\otimes (0,1)]$$

{\tiny
\begin{verbatim}
(* Definition of singlet state *)

psi2s =(1/Sqrt[2])* (TensorProductVec[{0, 1}, {1, 0}] - TensorProductVec[{1, 0}, {0, 1}]);
\end{verbatim}
}

}


\frame[containsverbatim]{
\frametitle{Definition der ein--Teilchen Operatoren}

$$F_1^\pm (\theta ,\varphi ) ={1\over 2}[{\Bbb I} \pm \sigma(   1,\theta,\varphi )]\otimes {\Bbb I} $$
$$F_2^\pm (\theta ,\varphi ) ={\Bbb I} \otimes {1\over 2}[{\Bbb I} \pm \sigma(   1,\theta,\varphi )]$$

{\tiny
\begin{verbatim}
(* Definition of single operators for occurrence of spin up *)

SingleParticleProjector2first[x_,p_,pm_] :=
           TensorProduct[1/2(IdentityMatrix[2] + pm * vecsig[1, x, p]), IdentityMatrix[2]]

SingleParticleProjector2second[x_,p_,pm_] :=
           TensorProduct[IdentityMatrix[2], 1/2(IdentityMatrix[2] + pm * vecsig[1, x, p])]


\end{verbatim}
}

}


\frame[containsverbatim]{
\frametitle{Definition der zwei--Teilchen Operatoren}


$$F_{12}^{\pm_1 \pm_2} ={1\over 2}[{\Bbb I} \pm_1 \sigma(   1,\theta_1,\varphi_1 )]\otimes
{1\over 2}[{\Bbb I} \pm_2 \sigma(   1,\theta_2,\varphi_2 )]$$


{\tiny
\begin{verbatim}
(* Definition of two-particle joint operator for occurrence of spin up and down  *)

JointProjector2[x1_, x2_,p1_,p2_,pm1_,pm2_] :=       TensorProduct[
                 1/2(IdentityMatrix[2] + pm1 * vecsig[1, x1, p1]),
                 1/2(IdentityMatrix[2] + pm2 * vecsig[1, x2, p2]) ]
\end{verbatim}
}

}


\frame[containsverbatim]{
\frametitle{Born-Regel}

$$P(\theta_1,,\theta_2, \varphi_1,\varphi_2, \pm_1, \pm_2)  =
{\rm Tr}(E_\Psi F_{12}^{\pm_1 \pm_2})$$

{\tiny
\begin{verbatim}
(* Probability of concurrence of two equal events for two - particle probability in singlet Bell state
   for occurrence of spin up
 *)

JointProb2s[x1_, x2_,p1_,p2_,pm1_,pm2_] := FullSimplify[
      Tr[DyadicProductVec[psi2s].JointProjector2[x1,x2,p1,p2,pm1,pm2]] ]

(* sum of joint probabilities add up to one *)

FullSimplify[
  Sum[JointProb2s[x1, x2, p1, p2, pm1, pm2], {pm1, -1, 1, 2}, {pm2, -1, 1, 2}]]
\end{verbatim}
}

}


\frame[containsverbatim]{
\frametitle{Quantenkorrelationen cntd. }

$$E(\theta_1,\varphi_1,\theta_2,\varphi_2 )= -\cos \varphi_1 \cos \varphi_2-\cos
   (\theta_1-\theta_2) \sin \varphi_1 \sin \varphi_2$$

{\tiny
\begin{verbatim}
(* Probability of concurrence of two equal events
 *)

P2Es[x1_, x2_,p1_,p2_] = FullSimplify[
       Sum[UnitStep[pm1*pm2]*JointProb2s[x1, x2, p1, p2, pm1, pm2],
           {pm1, -1, 1, 2}, {pm2, -1, 1, 2}]];

(* Probability of concurrence of two non-equal events
 *)

P2NEs[x1_, x2_,p1_,p2_] = FullSimplify[
       Sum[UnitStep[-pm1*pm2]*JointProb2s[x1, x2, p1, p2, pm1, pm2],
           {pm1, -1, 1, 2}, {pm2, -1, 1, 2}]];


(* Expectation function
 *)

Expectation2s[x1_, x2_,p1_,p2_] = FullSimplify[
         P2Es[x1,x2,p1,p2] - P2NEs[x1,x2,p1,p2] ]

\end{verbatim}
}

}


\frame[shrink=2]{
\frametitle{Quantenkorrelationen cntd.}
\begin{center}
%TexCad Options
%\grade{\off}
%\emlines{\off}
%\beziermacro{\off}
%\reduce{\on}
%\snapping{\off}
%\quality{4.00}
%\graddiff{0.01}
%\snapasp{1}
%\zoom{1.00}
\unitlength 1.00mm
\linethickness{0.4pt}
\begin{picture}(102.00,102.00)
%\emline(10.00,10.00)(10.00,100.00)
\put(10.00,10.00){\line(0,1){90.00}}
%\end
\put(10.00,55.00){\line(1,0){45.00}}
\put(55.00,55.00){\line(1,0){45.00}}
\put(10.00,10.00){\line(1,1){90.00}}
\put(100.00,100.00){\line(-1,0){45.00}}
\put(10.00,10.00){\line(1,0){45.00}}
%\bezier{284}(10.00,10.00)(30.00,10.00)(55.00,55.00)
\put(10.00,10.00){\line(1,0){1.41}}
\multiput(11.41,10.06)(0.71,0.08){2}{\line(1,0){0.71}}
\multiput(12.84,10.22)(0.48,0.09){3}{\line(1,0){0.48}}
\multiput(14.28,10.50)(0.36,0.10){4}{\line(1,0){0.36}}
\multiput(15.73,10.89)(0.29,0.10){5}{\line(1,0){0.29}}
\multiput(17.20,11.39)(0.25,0.10){6}{\line(1,0){0.25}}
\multiput(18.67,12.01)(0.21,0.10){7}{\line(1,0){0.21}}
\multiput(20.16,12.73)(0.21,0.12){7}{\line(1,0){0.21}}
\multiput(21.66,13.57)(0.19,0.12){8}{\line(1,0){0.19}}
\multiput(23.18,14.52)(0.17,0.12){9}{\line(1,0){0.17}}
\multiput(24.70,15.58)(0.15,0.12){10}{\line(1,0){0.15}}
\multiput(26.24,16.75)(0.14,0.12){11}{\line(1,0){0.14}}
\multiput(27.79,18.03)(0.13,0.12){12}{\line(1,0){0.13}}
\multiput(29.36,19.43)(0.12,0.12){13}{\line(1,0){0.12}}
\multiput(30.93,20.94)(0.11,0.12){14}{\line(0,1){0.12}}
\multiput(32.52,22.55)(0.11,0.12){14}{\line(0,1){0.12}}
\multiput(34.12,24.28)(0.12,0.13){14}{\line(0,1){0.13}}
\multiput(35.74,26.12)(0.12,0.14){14}{\line(0,1){0.14}}
\multiput(37.36,28.08)(0.12,0.15){14}{\line(0,1){0.15}}
\multiput(39.00,30.14)(0.12,0.16){14}{\line(0,1){0.16}}
\multiput(40.65,32.32)(0.12,0.16){14}{\line(0,1){0.16}}
\multiput(42.31,34.60)(0.12,0.17){14}{\line(0,1){0.17}}
\multiput(43.99,37.00)(0.11,0.17){15}{\line(0,1){0.17}}
\multiput(45.67,39.51)(0.11,0.17){15}{\line(0,1){0.17}}
\multiput(47.37,42.14)(0.11,0.18){15}{\line(0,1){0.18}}
\multiput(49.09,44.87)(0.11,0.19){15}{\line(0,1){0.19}}
\multiput(50.81,47.72)(0.12,0.20){15}{\line(0,1){0.20}}
\multiput(52.55,50.67)(0.12,0.21){21}{\line(0,1){0.21}}
%\end
%\bezier{284}(55.00,55.00)(80.00,100.00)(100.00,100.00)
\multiput(55.00,55.00)(0.12,0.21){15}{\line(0,1){0.21}}
\multiput(56.75,58.11)(0.12,0.20){15}{\line(0,1){0.20}}
\multiput(58.50,61.11)(0.12,0.19){15}{\line(0,1){0.19}}
\multiput(60.23,64.00)(0.11,0.19){15}{\line(0,1){0.19}}
\multiput(61.94,66.78)(0.11,0.18){15}{\line(0,1){0.18}}
\multiput(63.65,69.45)(0.11,0.17){15}{\line(0,1){0.17}}
\multiput(65.34,72.01)(0.12,0.17){14}{\line(0,1){0.17}}
\multiput(67.02,74.45)(0.12,0.17){14}{\line(0,1){0.17}}
\multiput(68.69,76.78)(0.12,0.16){14}{\line(0,1){0.16}}
\multiput(70.34,79.00)(0.12,0.15){14}{\line(0,1){0.15}}
\multiput(71.99,81.11)(0.12,0.14){14}{\line(0,1){0.14}}
\multiput(73.62,83.11)(0.12,0.13){14}{\line(0,1){0.13}}
\multiput(75.23,84.99)(0.11,0.13){14}{\line(0,1){0.13}}
\multiput(76.84,86.77)(0.11,0.12){14}{\line(0,1){0.12}}
\multiput(78.43,88.43)(0.12,0.12){13}{\line(1,0){0.12}}
\multiput(80.01,89.98)(0.13,0.12){12}{\line(1,0){0.13}}
\multiput(81.58,91.42)(0.13,0.11){12}{\line(1,0){0.13}}
\multiput(83.14,92.75)(0.14,0.11){11}{\line(1,0){0.14}}
\multiput(84.68,93.97)(0.15,0.11){10}{\line(1,0){0.15}}
\multiput(86.21,95.07)(0.17,0.11){9}{\line(1,0){0.17}}
\multiput(87.73,96.06)(0.19,0.11){8}{\line(1,0){0.19}}
\multiput(89.24,96.94)(0.21,0.11){7}{\line(1,0){0.21}}
\multiput(90.73,97.71)(0.25,0.11){6}{\line(1,0){0.25}}
\multiput(92.21,98.37)(0.29,0.11){5}{\line(1,0){0.29}}
\multiput(93.68,98.92)(0.36,0.11){4}{\line(1,0){0.36}}
\multiput(95.14,99.36)(0.48,0.11){3}{\line(1,0){0.48}}
\multiput(96.58,99.68)(0.72,0.11){2}{\line(1,0){0.72}}
\put(98.02,99.89){\line(1,0){1.98}}
%\end
\put(5.00,100.00){\makebox(0,0)[cc]{$+1$}}
\put(5.00,10.00){\makebox(0,0)[cc]{$-1$}}
\put(5.00,55.00){\makebox(0,0)[cc]{$0$}}
\put(58.00,50.00){\makebox(0,0)[cc]{$\pi /2$}}
\put(100.00,50.00){\makebox(0,0)[cc]{$\pi$}}
\put(102.00,59.00){\makebox(0,0)[cc]{$\theta$}}
\put(14.00,102.00){\makebox(0,0)[cc]{$E$}}
\put(30.00,38.00){\makebox(0,0)[cc]{$E_c(\theta )$}}
\put(46.00,28.00){\makebox(0,0)[cc]{$E_{qm}(\theta )$}}
\put(35.00,13.00){\makebox(0,0)[cc]{$E_s(\theta )$}}
\put(55.00,55.00){\circle*{2.00}}
\end{picture}
\end{center}

 }




\frame{
\frametitle{Konsequenzen f�r korrelierte Ereignisse}

Bereich $0 \le \theta = \theta_1-\theta_2  \le \pi/2$:

Mehr ungleiche Quantenereignisse als klassisch zu erwarten w�re.

Bereich $\pi/2 < \theta = \theta_1-\theta_2  \le \pi$:

Mehr gleiche Quantenereignisse als klassisch zu erwarten w�re.


}


\frame{
\frametitle{Verletzung der Bell'schen Ungleichungen}

Clauser-Horne Ungleichung
$$-1 \leq p_{13} + p_{14} + p_{23} - p_{24}- p_{1} -p_{3} \leq 0$$

Clauser-Horne-Shimony-Holt Ungleichung
$$CHSH(a,b,a',b'): = \vert E(a,b)+E(a,b')+E(a',b)-E(a',b') \vert \le 2$$

QM: Tsirelson Bound $2\sqrt{2}$ (Minmax-Prinzip)

}


\frame[shrink=2]{
\frametitle{Kochen-Specker-Theorem}
%TeXCAD Picture [4.pic]. Options:
%\grade{\on}
%\emlines{\off}
%\epic{\off}
%\beziermacro{\on}
%\reduce{\on}
%\snapping{\off}
%\quality{8.00}
%\graddiff{0.01}
%\snapasp{1}
%\zoom{5.6569}
\unitlength 1mm % = 2.85pt
\linethickness{0.8pt}
\ifx\plotpoint\undefined\newsavebox{\plotpoint}\fi % GNUPLOT compatibility
\begin{picture}(134.09,125.99)(0,0)
%\emline(86.39,101.96)(111.39,58.46)
\multiput(86.39,101.96)(.067385445,-.117250674){371}{\line(0,-1){.117250674}}
%\end
%\emline(86.39,14.96)(111.39,58.46)
\multiput(86.39,14.96)(.067385445,.117250674){371}{\line(0,1){.117250674}}
%\end
%\emline(36.47,101.96)(11.47,58.46)
\multiput(36.47,101.96)(-.067385445,-.117250674){371}{\line(0,-1){.117250674}}
%\end
%\emline(36.47,14.96)(11.47,58.46)
\multiput(36.47,14.96)(-.067385445,.117250674){371}{\line(0,1){.117250674}}
%\end
\put(86.39,101.71){\line(-1,0){50}}
\put(86.39,15.21){\line(-1,0){50}}
\put(86.28,101.76){\circle{2.97}}
\put(86.28,15.16){\circle{2.97}}
\put(93.53,89.21){\circle{2.97}}
\put(93.53,27.71){\circle{2.97}}
\put(29.24,89.21){\circle{2.97}}
\put(29.24,27.71){\circle{2.97}}
\put(102.37,73.47){\circle{2.97}}
\put(102.37,43.44){\circle{2.97}}
\put(20.4,73.47){\circle{2.97}}
\put(20.4,43.44){\circle{2.97}}
\put(111.21,58.45){\circle{2.97}}
\put(11.56,58.45){\circle{2.97}}
\put(36.34,101.76){\circle{2.97}}
\put(36.34,15.16){\circle{2.97}}
\put(52.99,101.76){\circle{2.97}}
\put(52.99,15.16){\circle{2.97}}
\put(69.68,101.76){\circle{2.97}}
\put(69.68,15.16){\circle{2.97}}
\qbezier(29.2,27.73)(23.55,-5.86)(52.99,15.24)
\qbezier(93.57,27.73)(99.22,-5.86)(69.78,15.24)
\qbezier(29.2,27.88)(36.93,75)(69.63,101.91)
\qbezier(93.57,27.88)(85.84,75)(53.13,101.91)
\qbezier(52.69,15.24)(87.47,40.96)(93.72,89.27)
\qbezier(70.08,15.24)(35.3,40.96)(29.05,89.27)
\qbezier(93.72,89.27)(98.4,125.99)(69.49,102.06)
\qbezier(29.05,89.27)(24.37,125.99)(53.28,102.06)
\qbezier(20.15,73.72)(-11.67,58.52)(20.15,43.31)
\qbezier(20.33,73.72)(61.34,93.16)(102.36,73.72)
\qbezier(102.36,73.72)(134.09,58.52)(102.53,43.31)
\qbezier(102.53,43.31)(60.99,23.43)(20.15,43.49)
\put(30.41,114.02){\makebox(0,0)[cc]{$(0,1,-1,0)$}}
\put(30.41,2.65){\makebox(0,0)[cc]{$(0,0,1,-1)$}}
\put(52.68,114.38){\makebox(0,0)[cc]{$(1,0,0,1)$}}
\put(52.68,2.3){\makebox(0,0)[cc]{$(1,-1,0,0)$}}
\put(91.93,114.2){\makebox(0,0)[cc]{$(-1,1,1,1)$}}
\put(91.93,2.48){\makebox(0,0)[cc]{$(1,1,1,1)$}}
\put(69.65,114.38){\makebox(0,0)[cc]{$(1,1,1,-1)$}}
\put(73.65,2.3){\makebox(0,0)[cc]{$(1,1,-1,-1)$}}
\put(103.24,94.22){\makebox(0,0)[cc]{$(1,1,-1,1)$}}
\put(19.45,94.22){\makebox(0,0)[cc]{$(0,1,1,0)$}}
\put(106.24,22.45){\makebox(0,0)[cc]{$(1,-1,1,-1)$}}
\put(19.45,22.45){\makebox(0,0)[cc]{$(0,0,1,1)$}}
\put(110.13,77.96){\makebox(0,0)[cc]{$(1,0,1,0)$}}
\put(12.55,77.96){\makebox(0,0)[cc]{$(0,0,0,1)$}}
\put(110.13,38.72){\makebox(0,0)[cc]{$(1,0,-1,0)$}}
\put(12.55,38.72){\makebox(0,0)[cc]{$(0,1,0,0)$}}
\put(120.92,57.98){\makebox(0,0)[l]{$(1,1,0,-1)$}}
\put(1.77,57.98){\makebox(0,0)[rc]{$(1,0,0,0)$}}
\end{picture}
}


\end{document}


<< Algebra`ReIm`

(* Definition of the Tensor Product *)

TensorProduct[a_, b_] :=    Table[(*a, b are nxn and mxm - matrices*)
a[[Ceiling[s/Length[b]], Ceiling[t/Length[b]]]]*b[[s - Floor[(s -
1)/Length[b]]*Length[b],t - Floor[(t - 1)/Length[b]]*Length[b]]], {s,
  1,Length[a]*Length[b]}, {t, 1, Length[a]*Length[b]}];


(* Definition of the Tensor Product between two vectors *)

TensorProductVec[x_, y_] := Flatten[Table[ x[[i]] y[[j]] , {i, 1, Length[x]}, {j, 1, Length[y]}]];


(* Definition of the Dyadic Product *)

DyadicProductVec[x_] :=  Table[x[[i]] x[[j]], {i, 1, Length[x]}, {j, 1, Length[x]}];

(* Definition of the sigma matrices *)


vecsig[r_, tt_, p_ ]:= r * { {Cos[tt], Sin[tt] Exp[-I p]}, {Sin[tt] Exp[I p], -Cos[tt]}}

(* Definition of some vectors *)

BellBasis = (1/Sqrt[2]) {{1, 0, 0, 1}, {0, 1, 1, 0}, {0, 1, -1, 0}, {1, 0, 0,-1}};

Basis =  {{1, 0, 0, 0}, {0, 1, 0, 0}, {0, 0, 1, 0}, {0, 0, 0,1}};

vp = {0,1};
vm = {1,0};

(* 2 PARTICLES *)


(* Definition of singlet state *)

psi2s =(1/Sqrt[2])* (TensorProductVec[{0, 1}, {1, 0}] - TensorProductVec[{1, 0}, {0, 1}]);

(* Definition of operators *)

(* Definition of single operators for occurrence of spin up *)

SingleParticleProjector2first[x_,p_,pm_] :=   TensorProduct[1/2(IdentityMatrix[2] + pm * vecsig[1, x, p]), IdentityMatrix[2]]

SingleParticleProjector2second[x_,p_,pm_] :=   TensorProduct[IdentityMatrix[2], 1/2(IdentityMatrix[2] + pm * vecsig[1, x, p])]



(* Definition of two-particle joint operator for occurrence of spin up and down  *)

JointProjector2[x1_, x2_,p1_,p2_,pm1_,pm2_] :=       TensorProduct[
                 1/2(IdentityMatrix[2] + pm1 * vecsig[1, x1, p1]),
                 1/2(IdentityMatrix[2] + pm2 * vecsig[1, x2, p2]) ]


(* Definition of probabilities *)


(* Probability of concurrence of two equal events for two - particle probability in singlet Bell state
   for occurrence of spin up
 *)

JointProb2s[x1_, x2_,p1_,p2_,pm1_,pm2_] := FullSimplify[
      Tr[DyadicProductVec[psi2s].JointProjector2[x1,x2,p1,p2,pm1,pm2]] ]

(* sum of joint probabilities add up to one *)

FullSimplify[
  Sum[JointProb2s[x1, x2, p1, p2, pm1, pm2], {pm1, -1, 1, 2}, {pm2, -1, 1, 2}]]

(* Probability of concurrence of two equal events
 *)

P2Es[x1_, x2_,p1_,p2_] = FullSimplify[
       Sum[UnitStep[pm1*pm2]*JointProb2s[x1, x2, p1, p2, pm1, pm2],
           {pm1, -1, 1, 2}, {pm2, -1, 1, 2}]];

(* Probability of concurrence of two non-equal events
 *)

P2NEs[x1_, x2_,p1_,p2_] = FullSimplify[
       Sum[UnitStep[-pm1*pm2]*JointProb2s[x1, x2, p1, p2, pm1, pm2],
           {pm1, -1, 1, 2}, {pm2, -1, 1, 2}]];


(* Expectation function
 *)

Expectation2s[x1_, x2_,p1_,p2_] = FullSimplify[
         P2Es[x1,x2,p1,p2] - P2NEs[x1,x2,p1,p2] ]
