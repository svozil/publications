%%%%%%%%%%%%%%%%%%%%% chapter.tex %%%%%%%%%%%%%%%%%%%%%%%%%%%%%%%%%
%
% sample chapter
%
% Use this file as a template for your own input.
%
%%%%%%%%%%%%%%%%%%%%%%%% Springer-Verlag %%%%%%%%%%%%%%%%%%%%%%%%%%
%\motto{Use the template \emph{chapter.tex} to style the various elements of your chapter content.}
\chapter{Are we an unaware participant of some galactic hypercivilization?}
\label{2023-UFO-part-Speculation-among-hypercivilization-zoo-hypothesis} % Always give a unique label
% use \chaptermark{}
% to alter or adjust the chapter heading in the running head


\abstract*{Fermi's well-known inquiry, ``Where is everybody?,'' can be explored in the context of the distribution of civilizations within a galactic framework. Some people speculate that, even though it is not officially recognized or proven by science, ``they'' (i.e., extraterrestrial life) may already be present on Earth. The Zoo Hypothesis suggests that these ``Others'' may choose to avoid interaction with us and instead monitor our activities from a distance. It is possible that we are not aware that Earth is part of a greater hypercivilization within the galaxy.}


\abstract{Fermi's well-known inquiry, ``Where is everybody?,'' can be explored in the context of the distribution of civilizations within a galactic framework. Some people speculate that, even though it is not officially recognized or proven by science, ``they'' (i.e., extraterrestrial life) may already be present on Earth. The Zoo Hypothesis suggests that these ``Others'' may choose to avoid interaction with us and instead monitor our activities from a distance. It is possible that we are not aware that Earth is part of a greater hypercivilization within the galaxy.}

\section{The overall context}

I encourage the reader to ``step out of contemporary timeframe'' and consider this universe in a more holistic context as a container for evolving consciousness,
and life in very general terms.
In that perspective it is almost certain that some form of sentient life form has developed long before us,
and will develop long after our current civilization has faltered~\cite[Chapter~1]{schopenhauer-dwawuv-VII,Nietzsche-WahrheitLuege}.
The apparent lapse of time acts like a malstream, swallowing up the worlds.
All civilizations, alien and earthbound, seem to be subject to these very same conditions.

\section{Galactic expansion}
\label{2023-UFO-part-Speculation-among-hypercivilization-zoo-hypothesis-ex}

A significant aspect of the ongoing discussion surrounding the existence and prevalence of extraterrestrial life relates to interstellar travel
and colonization.
Already in 1963 Carl Sagan considered direct contact among galactic civilizations
by relativistic interstellar spaceflight, that is, limited and relative to our contemporary means of space travel~\cite{Sagan_1963}.
In 1975, Michael Hart proposed~\cite{1975QJRAS..16..128H,hart_1995} a ``Bostrom-type'' argument:
that interstellar travel would be possible for a technologically advanced civilization,
and that migration would spread through the galaxy in a matter of a few million years.
Given that this time frame is short in comparison to the age of the galaxy, he argued that the
lack of settlers or evidence of their engineering projects in the solar system indicates the absence of extraterrestrial life~\cite{Jones-1985}.
Therefore, and for the current name possibly misrepresenting Fermi's position toward the existence of extraterrestrians,
it has been suggested to call it the Hart-Tipler argument~\cite{Gray-2015}.
A more nuanced analysis by Newman and Sagan~\cite{Newman_1981} has come to a different conclusion.
For a recent study, I refer to a paper by Beatriz Gato-Rivera~\cite{Gato-Rivera2005Dec}.


\section{Where are they?}
\label{2023-UFO-part-Speculation-among-hypercivilization-zoo-hypothesis-fp}


\subsection{Origin}

In a letter dated October 17, 1984, and published (as a preprint) by Eric M. Jones of Los Alamos National Laboratory~\cite{Jones-1985}, Emil Konopinski\index{Konopinski, Emil} recalled the context of Fermi's famous question, ``where are they?''---or rather ``but where is everybody?'' It seems quite probable that the incident of Fermi's question occurred most likely in Los Alamos---at what since January 1, 1947, was called Los Alamos Scientific Laboratory (LASL)---in the summer of 1950.

Emil Konopinski had a fairly clear memory of how the discussion of extraterrestrials began while Enrico Fermi,\index{Fermi, Enrico} Edward Teller,\index{Teller, Edward} Herbert F. York, and he were walking to lunch at Fuller Lodge in Los Alamos. When he joined the party, they were discussing evidence about flying saucers. This immediately brought to his mind a cartoon he had recently seen in the New Yorker, explaining why public trash cans were disappearing from the streets of New York City. The cartoon showed what was evidently a flying saucer sitting in the background and streaming toward it, ``aliens'' (endowed with antennas) carrying the trash cans to the flying saucer.

More amusing was Fermi's comment that it was a very reasonable theory since it accounted for two separate phenomena: first, the reports of flying saucers and second, the disappearance of the trash cans.

There ensued a discussion as to whether the saucers could somehow exceed the speed of light. It was after they were at the luncheon table that Fermi surprised them with the question: ``But where is everybody?'' It was his way of putting it that drew laughs from them.

He then turned to Teller, who recalls Fermi asking, ``Edward, what is your opinion on the likelihood of obtaining clear evidence of a material object moving faster than light within the next ten years?'' Teller's response (from his memory) was ``$10^{-6}$.'' Fermi responded, ``This estimate is far too low. The probability is more like ten percent.'' According to York, Fermi then proceeded to perform a series of calculations on the likelihood of the existence of planets similar to Earth, the probability of life existing on an Earth-like planet, the probability of humans arising given the presence of life, and the expected emergence and duration of advanced technology. Based on these estimates, he concluded that extraterrestrial beings should have already visited Earth numerous times in the distant past. As Fermi and Teller continued to discuss these chances further, the probability changed rapidly as Edward and Fermi bounced arguments off each other.

From my own lunch experience with Teller and Dirac in Erice, Sicily, in August 1992,
I can imagine that this recollection might well be authentic~\cite{dirac-81}.
In mid-1950, according to Turner~\cite{TurnerAustralia1971}, Teller might have been
already exposed to materials or investigations related to UFOs, but his rather secretive attitude would not have revealed much to Fermi and the other luncheon participants.


\subsection{They may already be watching us or are among us}


As has already been mentioned those Others might already ``walk among us'' but prefer not to be recognized as aliens.
This has already been put forward in the context of abductions.
However, given the circumstances, this standpoint cannot be deemed scientific, at least not in the immediate sense:
As Popper~\cite{Popper-Kreuzer-1079} stated, it should be considered ``blablabla'' since it does not satisfy the demarcation criterion for science, namely, falsifiability~\cite{popper,popper-en}.


\subsection{Rough estimate for the probability of watching a UFO}

Here is an attempt at a very crude estimate of how probable it might be to observe a flying saucer ``live'' in action.
Earlier in Section~\ref{2023-UFO-part-History-chapter-post-1945-pre-1947-KA},
we already quoted a US National Bureau of Standards publication that reviews visual acuity\index{visual acuity}\index{acuity}.
It states that ``it
is traditionally assumed that the finest detail that can just be made out by an eye with normal visual acuity,
viewing black lines on a white background, with moderate levels of illumination, subtends a visual angle of
1 minute of arc''~\cite[p.~10]{Howett1983Jul}.
From this, we conclude that the smallest object discernable with human eyes at a distance $d$ (in SI units meters) is approximately
$d \times 2 \times \pi /(360\times 60)$ meters.

Suppose now, in a sort of rough back-of-the-envelope calculation, that the average UFO is the size of a ball with a diameter of 30 meters. That is, the maximal distance $D$ from which we can see this UFO is given by:

$$30\text{ m} = \frac{D \times 2 \times \pi}{360 \times 60}\text{ m}$$

which can be rearranged to give:

$$D = \frac{30 \times 360 \times 60}{2 \times \pi} \approx \frac{30 \times 360 \times 60}{2 \times 3} \approx 30 \times 360 \times 10 =
100,000\text{ m},$$ or 100 km.

Assuming that the UFO is above the horizon and standing on the ground in a flat surrounding domain, we could ``cover'' an area of approximately $\pi \times 100^2$ km$^2 \approx 30,000 = 3 \times 10^4$ km$^2$.

On the other hand, Earth's average radius is approximately 6400~km, and consequently, Earth's surface is approximately $4\times \pi \times 6400^2 \approx 4\times 3.14 \times 6400^2 \approx 5 \times 10^8$~km$^2$.

Therefore, assuming that the density of UFOs on Earth is equidistributed and not extremely high---let us say, one craft in the atmosphere.
Then,
your chances of staring up and seeing a flying saucer are approximately 1 in 10,000, or $10^{-4}$,
which is one percent of one percent.

Of course, chances may increase if the density of saucers increases,
if more people look up,
if the saucer size is larger or if binoculars are used.

However, it is important to keep in mind that correctly categorizing an object or entity as a ``flying saucer''
is challenging when it is up in the air. From a great distance, and without sophisticated sensors,
almost anything can appear like a flying saucer to an untrained and hopeful and expecting observer.
I suspect that labeling an object as a ``flying saucer'' or ``UFO from outer space'' reflects a Rorschach test---a
projective psychological test that reveals more about the observer's perspective or desires than the object itself.





\section{Nonfraternization}
\label{2023-UFO-part-Speculation-among-hypercivilization-zoo-hypothesis-nfp}

From historic examples of European expansion and colonization, it may be inferred that eventually, the centers of empires suffer from a kind of ``imperial backflow:'' the former colonies tend to affect those centers in unfavorable ways relative to the autonomy of the centers. In particular, there tends to be a kind of ``osmosis'' that is characterized by an influx of individuals from colonies gravitating to richer and technologically more advanced civilization hubs.

To prevent this from happening, a policy of strict nonfraternization could be imposed~\cite{GulfOfSilencey2020}.
This refers to a policy that prohibits or strictly limits social interactions between individuals or groups who are in a position of power,
 autonomy or authority, and those who are not.

This can include limiting or prohibiting social interactions, relationships of any informal kind,
or any other type of personal connection that could create a potential connection. The goal of strict nonfraternization is to maintain an impartial environment and to avoid any actions that could create potential drawbacks and a loss of autonomy for the group in power and undermine their integrity or effectiveness.

As noted by Fort~\cite{FortBotD}, Bramley~\cite{Bramley1993Mar}, and others,
and as discussed in Section~\ref{2023-UFO-part-Perception-abductions-wao},
we may have little to offer advanced extraterrestrial civilizations visiting us,
and we should consider the reasons why they might choose to contact us.
If we put ourselves in their position, it becomes apparent that the asymmetry between their advanced technology
and our own would make any potential gains from contact relatively marginal compared to the potential risks and losses.

This consideration is likely to dominate future principles of compliance with less technologically advanced civilizations,
particularly as we approach a saturation point in space exploration and colonization.
There are vast material resources that could be extracted from non-inhabited rocks or planets without causing harm to sentient life forms.
However, exploring and engaging with sentient life forms on distant exoplanets may not be an attractive prospect,
especially if it involves dealing with primitive or unpredictable tribes.
We may choose nonfraternization simply because of the possible inconveniences of making ourselves known to them.
And because they have no need to know us.

Nevertheless, as already mentioned in the preface, we might hope for some ``alien Prometheus'' who,
similar to the allegorical figure in Greek mythology, disobeys nonfraternization and gifts some of their scientific and technological achievements to humans.
However, then, why should they?
And might the consequences---in terms of ``us'' competing in space and resources against ``them''---not be discouraging,
let alone their legal penalties that may be associated with such an intercultural knowledge breach?

\section{Zoo hypothesis}
\label{2023-UFO-part-Speculation-among-hypercivilization-zoo-hypothesis-zh}

The zoo hypothesis~\cite{Ball1973347} is a concept in the field of astrobiology that
proposes that extraterrestrial civilizations may be observing humanity, but they refrain from making contact
to preserve humanity's natural development.
The hypothesis suggests that advanced civilizations may be aware of our existence,
but they choose to keep their distance to avoid interference with our natural development,
much like a zoo would observe animals without interfering with their natural behavior,
or how anthropologist would study some indigenous tribe.
This idea implies that the extraterrestrial civilization is aware of us but is avoiding interference with us because it considers us a ``primitive'' species in terms of technological and scientific advancement. The hypothesis suggests that extraterrestrial civilizations may be observing humanity in the same way that humans observe animals in a zoo without interfering with their natural development.
