\documentclass[10pt,a4paper]{amsart}
\textwidth=16cm
\textheight=24.7cm
\oddsidemargin= 0mm
\topmargin -17mm

%----------------------------------------------------- ENVIRONMENTS

\newtheorem{theorem}{Theorem}
\newtheorem{definition}{Definition}

%----------------------------------------------------- DEFINITIONS

\title{Pseudocontexts}
%\title{On Reichenbach's common cause principle}
\author{Mirko Navara, Karl Svozil}
\thanks{Department of Cybernetics, Faculty of Electrical Engineering, Czech Technical University in Prague,
%Technick\'{a}~2, 166~27 Prague,
Czech Republic;
e-mail: navara@fel.cvut.cz}
\thanks{Institute for Theoretical Physics,
TU Wien,
%Wiedner Hauptstrasse 8-10/136,
%1040 Vienna,
Austria;
e-mail: karl.svozil@tuwien.ac.at}
\date{}                                           % no date
\keywords{contextuality, two-valued states, quantum states}

\begin{document}
\maketitle
\thispagestyle{empty}
\pagestyle{empty}

Quantum systems are described using \emph{contexts}, i.e., systems $x_1,\ldots,x_n$ of mutually orthogonal events forming a partition of unity.
For each state~$p$, they satisfy
$$\sum_{i=1}^n p(x_i)=1\,.$$
We introduced  \emph{pseudocontexts of order~$n$}, i.e., systems $x_1,\ldots,x_n, y_1,\ldots,y_n$ of mutually \emph{non-commuting} events, which satisfy some equalities at all states, namely
\begin{equation}
\sum_{i=1}^n p(x_i)=\sum_{i=1}^n p(y_i)\,.
\label{eq}
\end{equation}
%Except for (\ref{eq}), the values of states on a pseudocontext should be arbitrary.

It is obvious that a pseudocontext of order~$1$ does not exist.
It is less straightforward that  a pseudocontext of order~$2$ does not exist.
The existence of a pseudocontext of order~$3$ was raised as an open problem,
even for the more general structures of orthomodular lattices~\cite{Rog:correspondentce}.
Subsequently, it was applied to the uniqueness problem for bounded observables, formulated in~\cite{Gudder:Uniq} and solved in~\cite{N:Uniqueness}.
It has been extensively generalized in~\cite{MayetNR}.
However, these constructions used blocks with different numbers of atoms, thus they did not admit an embedding into Hilbert lattices.

A question remained open for a longer time whether a pseudocontext of order~$3$
can exist in Hilbert lattices. This problem can be reformulated as a \emph{coordinatization} of the (already known) orthomodular
lattice---meaning the labeling of its atoms with vectors in a (3D real) Hilbert space so that each block represents an orthogonal basis,
while no other orthogonality relations occur. We solved this problem and found such coordinatizations, leaving one degree of freedom.
Depending on its choice, the sum (\ref{eq}) satisfies different inequalities, but it never allows the whole interval $[0,3]$,
possible in general orthomodular lattices.
The proof was done in~\cite{NSvozil:pseudocontext}, using a combination of manual decomposition of the task and computer algebra solving the subtasks.


%\section*{Acknowledgement}
%This work was supported by FWF: Project I 4579-N and GA\v{C}R: Project 20-09869L.

\begin{thebibliography}{9}
\bibitem{Rog:correspondentce} Rogalewicz, V.: Personal letter to Ren\'e Mayet (around 1990).

\bibitem{Gudder:Uniq} Gudder, S.P.:
Uniqueness and existence properties of bounded observables. \textit{Pacific J. Math.} {\bf 19} (1966), 81--93.


\bibitem{N:Uniqueness} Navara, M.:
Uniqueness of bounded observables.
{\it Ann. Inst. H. Poincar\'e --- Theor. Phys.\/}
{\bf 63} (1995), no.~2, 155--176.

\bibitem{MayetNR} Mayet, R., Navara, M., Rogalewicz, V.:
Orthomodular lattices with rich state spaces.
{\it Algebra Universalis\/} {\bf 43} (2000), 1--30.


\bibitem{NSvozil:pseudocontext} Navara, M., Svozil, K.:
A novel form of contextuality predicting probabilistic equivalence between two sets of three mutually noncommuting observables. \textit{Physical Review A} \textbf{109} (2024), 022222.
DOI: 10.1103/PhysRevA.109.022222

\end{thebibliography}
\end{document}
%-----------------------------------------------------
