\PassOptionsToPackage{usenames,dvipsnames}{xcolor}
%\documentclass[amsmath,table,sans,amsfonts, handout]{beamer}
\documentclass[amsmath,table,sans,amsfonts,hyperref={colorlinks,citecolor=blue,linkcolor=blue,urlcolor=purple}]{beamer}
\usepackage[T1]{fontenc}
%%\usepackage{beamerthemeshadow}
%%\usepackage[headheight=1pt,footheight=10pt]{beamerthemeboxes}
%%\addfootboxtemplate{\color{structure!80}}{\color{white}\tiny \hfill Karl Svozil (TU Vienna)\hfill}
%%\addfootboxtemplate{\color{structure!65}}{\color{white}\tiny \hfill mur.sat \hfill}
%%\addfootboxtemplate{\color{structure!50}}{\color{white}\tiny \hfill Graz, 2010-12-11\hfill}
%\usepackage[dark]{beamerthemesidebar}
%\usepackage[headheight=24pt,footheight=12pt]{beamerthemesplit}
%\usepackage{beamerthemesplit}
%\usepackage[bar]{beamerthemetree}
\usepackage{graphicx}
\usepackage{pgf}
%\usepackage{eepic}
%\newcommand{\Red}{\color{Red}}  %(VERY-Approx.PANTONE-RED)
%\newcommand{\Green}{\color{Green}}  %(VERY-Approx.PANTONE-GREEN)

\definecolor{applegreen}{rgb}{0.55, 0.71, 0.0}

\usepackage{fourier-orns}  %fancy symbols https://mirror.easyname.at/ctan/fonts/fourier-GUT/doc/fourier-orns-doc.pdf

\usepackage{musixtex}

%%%%%%%%%%%%%%%%%%%%%%%%%%%%%
\usepackage{iftex}
\ifxetex
\usepackage{fontspec}% Schriftumschaltung mit den nativen XeTeX-Anweisungen
                     % vornehmen. Voreinstellung: Latin Modern
\usepackage[ngerman]{babel}% Sprachumschaltung: Deutsch nach neuer Rechtschreibung

%
% XeLaTeX
%
\XeTeXinputencoding cp1252
\usepackage{fontspec}
%%\setmainfont{Times New Roman}
%\setmainfont{Garamond Libre}
%\setsansfont{Garamond Libre}
\setmainfont{EB Garamond}
\setsansfont{EB Garamond}
%
\else
\usepackage[latin1]{inputenc}
\usepackage[T1]{fontenc}
\fi
%%%%%%%%%%%%%%%%%%%%%%%%%%%%%

%\RequirePackage[german]{babel}
%\selectlanguage{german}
%\RequirePackage[isolatin]{inputenc}

%\pgfdeclareimage[height=0.5cm]{logo}{tu-logo}
%\logo{\pgfuseimage{logo}}
\beamertemplatetriangleitem
%\beamertemplateballitem

\beamerboxesdeclarecolorscheme{alert}{red}{red!15!averagebackgroundcolor}
%\begin{beamerboxesrounded}[scheme=alert,shadow=true]{}
%\end{beamerboxesrounded}

%\beamersetaveragebackground{yellow!10}

%\beamertemplatecircleminiframe

\newtheorem{question}{Question}
\newtheorem{conjecture}[question]{Principle}
\newtheorem{challenge}[question]{Challenge}
\usepackage{tikz}
\newcommand{\bra}[1]{\left< #1 \right|}
\newcommand{\ket}[1]{\left| #1 \right>}

\newcommand{\iprod}[2]{\langle #1 | #2 \rangle}
\newcommand{\mprod}[3]{\langle #1 | #2 | #3 \rangle}
\newcommand{\oprod}[2]{| #1 \rangle\langle #2 |}

\newcommand{\proj}[3]{\begin{smallmatrix} #1 & #2 & #3 \end{smallmatrix}}
\newcommand{\projbf}[3]{\begin{smallmatrix} \mathbf{#1} & \mathbf{#2} & \mathbf{#3} \end{smallmatrix}}

\sloppy
\parskip .7em %vskip between paragraphs

\newcommand{\seq}[1]{\mathbf{#1}}
\newcommand{\floor}[1]{\left\lfloor #1 \right\rfloor}
\newcommand{\ceil}[1]{\left\lceil #1 \right\rceil}
\newcommand{\m}[1]{\widetilde{#1}}
%\newcommand{\p}[1]{\scriptsize\textcolor{black}{$[#1]$}}

\usepackage[most]{tcolorbox}
\begin{document}

\title{\bf \textcolor{blue}{Exploring the Possibilities of Quantum Music: Navigating a New Musical Frontier in Quantum Musical Space}}
\subtitle{\url{http://tph.tuwien.ac.at/~svozil/publ/2023-QMusic-pres.pdf}}
\author{Karl Svozil}
\institute{ITP TU Wien, Vienna, Austria\\
svozil@tuwien.ac.at
%{\tiny Disclaimer: Die hier vertretenen Meinungen des Autors verstehen sich als Diskussionsbeitr�ge und decken sich nicht notwendigerweise mit den Positionen der Technischen Universit�t Wien oder deren Vertreter.}
}
\date{``Constructive Disturbances of Art in Science'', Vienna, Monday, May 15, 2023}
\maketitle


% \frame{
% \frametitle{Contents}
% \tableofcontents
% }

\section{Introduction}


 \frame{
 \frametitle{Collaboration \& Publications}


\begin{itemize}
\item[$\bullet$] {\color{blue}   Volkmar Putz and Karl Svozil, ``Quantum Music'', Soft Computing 21(6), 1467-1471 (2017), DOI: 10.1007/s00500-015-1835-x}
\item[$\bullet$] {\color{blue}   Volkmar Putz and Karl Svozil, ``Quantum Music, Quantum Arts and Their Perception'', in ``Quantum Computing in the Arts and Humanities
An Introduction to Core Concepts, Theory and Applications'', Hg. Eduardo Reck Miranda, (Springer, 2022), arXiv:2108.05207 (2022), DOI: 10.1007/978-3-030-95538-0\_5}
\end{itemize}
}

\section{Realm of quantum expressibility}

\frame{
 \frametitle{Realm of quantum expressibility I: Boolean algebras \textit{versus} geometric, vector based, means}

\begin{itemize}
\item[$\bullet$] {\color{purple}
Classical music is in terms of classical physical states based on Boolean algebras, power sets, set theoretic unions, intersections, complements,~$\ldots$}
\pause
\item[$\bullet$] {\color{purple}
Quantum music is vector based; pure states are vectors, temporal evolution is a
generalized form of permutation (aka unitary one-to-one modulation) of that vector}
\end{itemize}
}

 \frame{
 \frametitle{Realm of quantum expressibility II: Parallelism \& Entanglement}

\begin{itemize}
\item[$\bullet$] {\color{purple}
parallelization through coherent superposition (aka simultaneous linear combination) of classically mutually exclusive
tones or signals that are acoustic,  optic, touch, taste,  or otherwise sensory}
\pause
\item[$\bullet$] {\color{purple}
entanglement not merely by classical correlation
but by relational encoding
of multi-partite states such that}
\begin{itemize}
\item[$\ast$] {\color{blue}
any classical information is ``scrambled'' into relational, joint multi-partite/tonal properties }
\item[$\ast$] {\color{blue}
while at the same time losing value definiteness about the single constituents of such multi-partite states   }
\end{itemize}
{\color{purple}
This
can be seen as a sort of zero-sum game, a tradeoff between individual and collective properties}
\end{itemize}
}

\frame{
 \frametitle{Realm of quantum expressibility III: Complementarity \& Contextuality}

\begin{itemize}
\item[$\bullet$] {\color{purple}
Complementarity associated with value (in)definiteness of certain tones or signals
that is acoustic,  optic, touch, taste,  or otherwise:
if one such observable is definite, another is not, and \textit{vice versa}}
\pause
\item[$\bullet$] {\color{purple}
Contextuality is an ``enhanced'' form of complementarity and value indefiniteness that can be defined in various
ways,
in particular, emphasizing homomorphic, structure-preserving nonembeddability into classical schemes  }
\end{itemize}
}

\section{Quantum musical realization}

\frame{
 \frametitle{Quantum musical realization I: Quantum musical tones}
\begin{center}
\begin{music}
\startextract
\NOtes\zsong{$\vert \Psi_c \rangle$}{\color[rgb]{1,0,0}\ca{c}}\en
\NOtes\zsong{$\vert \Psi_d \rangle$}{\color[rgb]{0.5,0.5,0}\ca{d}}\en
\NOtes\zsong{$\vert \Psi_e \rangle$}{\color[rgb]{0,1,0}\ca{e}}\en
\NOtes\zsong{$\vert \Psi_f \rangle$}{\color[rgb]{0,0.5,0.5}\ca{f}}\en
\NOtes\zsong{$\vert \Psi_g \rangle$}{\color[rgb]{0,0,1}\ca{g}}\en
\NOtes\zsong{$\vert \Psi_a \rangle$}{\color[rgb]{0.5,0,0.5}\ca{h}}\en
\NOtes\zsong{$\vert \Psi_b \rangle$}{\color[rgb]{0.5,0.25,0.25}\ca{i}}\en
\zendextract
\end{music}
\end{center}
Temporal succession of quantum tones
$\vert \Psi_c \rangle$,
$\vert \Psi_d \rangle$,~$\ldots$,
$\vert \Psi_b \rangle$
in the C major scale
forming the (reduced because it contains only seven tones) octave basis ${\mathfrak B}$ of $\mathbb{C}^7$:
the basis elements are formalized by the Cartesian basis tuples
\begin{align*}
\vert \Psi_c \rangle &=\begin{pmatrix}0,0,0,0,0,0,1\end{pmatrix}, \\
\vert \Psi_d \rangle &=\begin{pmatrix}0,0,0,0,0,1,0\end{pmatrix},\\
&\ldots                                                           \\
\vert \Psi_b \rangle &=\begin{pmatrix}1,0,0,0,0,0,0\end{pmatrix}
\end{align*}
}

\frame{
 \frametitle{Quantum musical realization II: Quantum musical compositions}
A musical ``composition''---indeed, any
 succession of quantized tones forming
a ``melody''---would be obtained by successive unitary permutations of the state $\vert \psi \rangle$.
The realm of such compositions would be spanned by the succession of all
unitary transformations $\textsf{\textbf{U}}: {\mathfrak B} \mapsto {\mathfrak B}'$
mapping some orthonormal basis ${\mathfrak B}$ into another orthonormal basis ${\mathfrak B}'$
}


\section{Realm of quantum perception}

\frame{
 \frametitle{Realm of quantum perception I: Can single quanta be perceived by human observers?}

This is an open question, however there are indications that at least the human eye can ``watch'' single quanta.

Human rod cells respond to individual photons:

Selig Hecht and Simon Shlaer and Maurice Henri Pirenne, 1942, DOI: 10.1085/jgp.25.6.819 (Review: Gerald Westheimer, 2016, DOI: 10.1201/9781315373034-2 ).


Moreover, recent reports suggest that humans might be capable of subjectively ``being aware'' of the detection of a
 single-photon incident on the cornea with a probability
significantly above chance: Jonathan N. Tinsley and Maxim I. Molodtsov and Robert Prevedel and David Wartmann and Jofre Espigul\'e-Pons and Mattias Lauwers and Alipasha Vaziri, 2016, ``Direct detection of a single photon by humans'', DOI: 10.1038/ncomms12172

}

\frame{
 \frametitle{Realm of quantum perception II: Classical perception of quantum musical parallelism}

If a classical auditorium listens to the quantum musical state $\vert \psi \rangle $ which is a coherent superposition of classical tones
the individual classical listeners  may perceive $\vert \psi \rangle $ very differently;
that is, they will hear only a {\em single one} of the different classical tones contained in $\vert \psi \rangle $.
}

\begin{frame}[shrink=3]
 \frametitle{Example of the classical perception of the quantum musical parallelism}

For the sake of a demonstration,
let us try a two-note quantum composition.
We start with a pure quantum mechanical state
in the two-dimensional subspace spanned by  $\vert \Psi_c \rangle $ and $\vert \Psi_g \rangle$,
specified by
\begin{equation}
\vert \psi_1\rangle =
\frac45 \vert \Psi_c \rangle
+ \frac35 \vert \Psi_g \rangle = \frac15 \begin{pmatrix} 4 \\ 3  \end{pmatrix}
.
\end{equation}
$\vert \psi_1 \rangle$ would be detected by the listener as $c$ in 64\%
of all measurements (listenings), and as $g$ in 36\%
of all listenings.
Using the unitary transformation $\textsf{\textbf{X}}= \begin{pmatrix} 0 & 1 \\ 1 & 0 \end{pmatrix}$, the next quantum tone would be
\begin{equation}
\vert \psi_2 \rangle = \textsf{\textbf{X}}  \vert \psi_1 \rangle =
\frac35 \vert \Psi_c \rangle
+ \frac45 \vert \Psi_g \rangle = \frac15 \begin{pmatrix} 3 \\ 4  \end{pmatrix}.
\end{equation}
This means for the quantum melody of both quantum tones $\vert \psi_1\rangle$ and $\vert  \psi_2 \rangle$ in
succession in repeated measurements,
in $0.64^2 = 40.96\%$
of all cases $c-g$ is heard,
in $0.36^2 = 12.96\%$
of all cases $g-c$,
in $0.64\cdot0.36 = 23.04\%$
of all cases $c-c$ or $g-g$, respectively.
\begin{center}
\begin{music}
\generalmeter{\meterfrac24}% 2/4 meter chosen
\startextract % starting real score
\Notes
{\color[rgb]{1,0.1,0.1}\zq c}{\color[rgb]{0.6,0.6,1}\zq g}  \enotes
\Notes
{\color[rgb]{1,0.6,0.6}\zq c}{\color[rgb]{0.1,0.1,1}\zq g}  \enotes
\Endpiece
\zendextract % terminate excerpt
\end{music}
\end{center}
\end{frame}


\frame{
\frametitle{Tradeoff quantum versus classical music, and how to experiencing it?}

\begin{itemize}
\item[$\bullet$] {\color{purple}
Quantum music presents a novel form of musical expressibility and tonal forms}

\item[$\bullet$] {\color{purple}
Quantum music lacks some classical forms of musical expressibility---all that are not one-to-one; eg, ``getting rid'' of tones is only possible by transformation into other tones; no ``silenzio''}

\item[$\bullet$] {\color{purple}
Quantum music may be ``difficult'' to perceive; and may sometimes involve paradoxical experiences---cf Schr\"odinger's cat or quantum jellyfish (late Dublin seminars) metaphors
}
\end{itemize}
}

\frame{
\frametitle{Hochschule D�sseldorf  // Prof. Christian  Jendreiko}

\begin{center}
\resizebox{.90\textwidth}{!}{
\includegraphics{2023-QMusic-pres-Jendreiko-HSD-IMG_1320.jpeg}
}
\end{center}
 }

\frame{

\centerline{\Large {\color{magenta} Thank you for your attention!}}

\begin{center}\color{orange}
$\widetilde{\qquad \qquad }$
$\widetilde{\qquad \qquad}$
$\widetilde{\qquad \qquad }$
\end{center}
 }
 \end{document}
