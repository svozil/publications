%\documentclass[pra,showpacs,showkeys,amsfonts,amsmath,twocolumn,handou]{revtex4}
\documentclass[amsmath,blue,table,sans,handout]{beamer}
%\documentclass[pra,showpacs,showkeys,amsfonts]{revtex4}
\usepackage[T1]{fontenc}
%%\usepackage{beamerthemeshadow}
\usepackage[headheight=1pt,footheight=10pt]{beamerthemeboxes}
\addfootboxtemplate{\color{structure!80}}{\color{white}\tiny \hfill Karl Svozil (TU Vienna)\hfill}
\addfootboxtemplate{\color{structure!65}}{\color{white}\tiny \hfill Trivalent decision problems $\ldots$\hfill}
\addfootboxtemplate{\color{structure!50}}{\color{white}\tiny \hfill UC2008/PC2008, August  28th, 2008\hfill}
%\usepackage[dark]{beamerthemesidebar}
%\usepackage[headheight=24pt,footheight=12pt]{beamerthemesplit}
%\usepackage{beamerthemesplit}
%\usepackage[bar]{beamerthemetree}
\usepackage{graphicx}
\usepackage{pgf}
%\usepackage[usenames]{color}
%\newcommand{\Red}{\color{Red}}  %(VERY-Approx.PANTONE-RED)
%\newcommand{\Green}{\color{Green}}  %(VERY-Approx.PANTONE-GREEN)

%\RequirePackage[german]{babel}
%\selectlanguage{german}
%\RequirePackage[isolatin]{inputenc}

\pgfdeclareimage[height=0.5cm]{logo}{tu-logo}
\logo{\pgfuseimage{logo}}
\beamertemplatetriangleitem
%\beamertemplateballitem

\beamerboxesdeclarecolorscheme{alert}{red}{red!15!averagebackgroundcolor}
%\begin{beamerboxesrounded}[scheme=alert,shadow=true]{}
%\end{beamerboxesrounded}

%\beamersetaveragebackground{green!10}

%\beamertemplatecircleminiframe

\begin{document}

\title{\bf \textcolor{blue}{On the Solution of Trivalent Decision Problems by Quantum State Identification}}
%\subtitle{Naturwissenschaftlich-Humanisticher Tag am BG 19\\Weltbild und Wissenschaft\\http://tph.tuwien.ac.at/\~{}svozil/publ/2005-BG18-pres.pdf}
\subtitle{\textcolor{orange!60}{\small http://tph.tuwien.ac.at/$\sim$svozil/publ/2008-UC2008-pres.pdf}}
\author{Karl Svozil}
\institute{Institut f\"ur Theoretische Physik, Vienna University of Technology, \\
Wiedner Hauptstra\ss e 8-10/136, A-1040 Vienna, Austria\\
svozil@tuwien.ac.at
%{\tiny Disclaimer: Die hier vertretenen Meinungen des Autors verstehen sich als Diskussionsbeitr�ge und decken sich nicht notwendigerweise mit den Positionen der Technischen Universit�t Wien oder deren Vertreter.}
}
\date{Vienna, Austria, August  28th, 2008}
\maketitle



%\frame{
%\frametitle{Contents}
%\tableofcontents
%}



\section{Trivalent decision problems}

\frame{
\frametitle{Main results}

\begin{itemize}
\item<1->
Trivalent decision problems via state identification. (KS \& Josef Tkadlec):

\begin{itemize}
\item<1->
There does not exist a trivalent decision problem
encodable into three dimensional Hilbert space.

\item<1->
There exists trivalent decision problems
encodable into 27--dimensional Hilbert space.
\end{itemize}
\item<1->
How to detect hypercomputation (Alexander Leitsch \& G{\"{u}}nter Schachner \& KS):

\begin{itemize}
\item<1->
Because of the actual infinities involved,
there cannot exist any ``operational'' proof of hypercomputation, but:

\item<1->
``black box'' model of hypercomputation with input/output interfaces;
\item<1->
find  highly ``asymmetric'' problems which are computationally ``easy''
to generate and ``hard'' to solve;


\item<1->
two or more hypercomputers  are used to ``compete'' against or ``check'' themselves.


\end{itemize}
\end{itemize}

}

\subsection{Why quantum computation?}
\frame{
\frametitle{Why quantum computation?}

Why should quantum computation  outperform classical computation?

\begin{itemize}
\item<1->

(Classical) Physics might ``harvest'' the power of dense sets or maybe even the continuum (e.g., Zeno squeezed time cycles,
Banach-Tarski set decomposition, $\ldots$)
[[Issue: there is no ``actual infinity;'' even ``potential''
infinity is only ``in our minds,'' and
not operational]]

\item<1->
Quantum parallelism:
$n$ qbits (qdits) can co-represent (via ``superposition'') exponentially many; i.e.,
$2^n$ ($d^n$) classically mutually exclusive bit (dit) states.
[[Issue: how to extract suitable information from the quantum state?]]

\item<1->
interference; but also possible classical (Cristian Calude);
\item<1->
Quantum randomness, complementarity (quantum cryptography) \& value indefiniteness $\ldots$
\item<1->
$\ldots$
$\ldots$
$\ldots$
\end{itemize}
[[Poll: (i) does quantum computation
outperform classical computation? --- (ii) and if so: why?]]
}


\subsection{Solving decision problems by quantum state identification}

\frame{
\frametitle{Distributing classically useful information among several quanta}

(Classical) Information can be encoded by distributing it over different particles or quanta,
such that:

\begin{itemize}
\item<1->
measurements of {\em single} quanta are irrelevant, yield ``random'' results,
and even destroy the original information (by asking complementary questions);


\item<1->
well defined correlations exist and can be defined among different particles or quanta ---
even to the extend that a state is solely defined by propositions ($\equiv$ projectors)
about {\em collective} (or {\em relative})  properties of the particles or quanta involved;


\item<1->
identifying a given state of a quantized system can yield information about
{\em collective} (or {\em relative})  properties of the particles or quanta involved.

\end{itemize}
 }



\frame{
\frametitle{Related physical concepts}



\begin{itemize}
\item<1->
Quantum entanglement (Schr�dinger's ``Verschr�nkung''):
the state of two or more ``entangled'' particles or quanta cannot be
constructed from or decomposed into (tensor) products of the states
of the ``single'' particles or quanta involved.

E.g., in  {\it The essence of entanglement} [quant-ph/0106119],   Brukner, Zukowski \& Zeilinger write:
{\em ``the information in a composite system resides more
in the correlations than in properties of individuals.''}


\item<1->
Zeilinger's foundational principle: {\em ``An elementary system carries 1 bit of information.''}
$\ldots$
$\ldots$
$\ldots$ more generally:
$n$ elementary $d$-state systems (like particles or quanta) carry exactly $n$ dits of information.

\item<1->
Example: the (singlet) Bell state
$
\vert \uparrow \downarrow \rangle
-
\vert \downarrow \uparrow \rangle
$
of two electrons
 is defined by the properties that the two particles have opposite spin when measured along two
different (orthogonal) directions.

\end{itemize}
 }


\subsection{Encoding  decision problems by state identification problems}

 \frame{
\frametitle{Quantum encoding  decision problems about ``collective'' behaviours}

Suppose one is interested in a decision problem which could be associated with some {\em ``collective''} property or behaviour
related to or involving, for instance,
\begin{itemize}
\item<1->
a function over a wide range of its arguments,
\item<1->
which is of ``comparative'' nature; that is, only the relative functional values count;
\item<1->
such that the single functional values are irrelevant;
e.g., are of no interest, ``annoying'' or are otherwise unnecessary.
\end{itemize}
Then it is not completely unreasonable to speculate that one could use the
kind of distributive  information capacity encountered in the quantum physics of multipartite states
for a more effective (encryption of the) solution.
 }

 \frame{
\frametitle{Encoding  decision problems by state identification problems}

\begin{itemize}
\item<1->
Re-encode the behaviour of the algorithm or function involved in the decision problem
into an orthogonal set of states, such that every distinct function results in
a {\em single} distinct state orthogonal to all the other ones.
Suppose that this is impossible because the number of functions exceeds the number of orthogonal states, then
\begin{itemize}
\item<1->
one could attempt to find a suitable representation of the functions in terms of the base states.
\item<1->
Alternatively, the dimension of the Hilbert space could be increased by the addition of auxiliary Qbits.
The latter method is hardly feasible for general $q$-ary functions of $n$ dits,
since the number of possible functions increases with $q^{d^n}$, as compared to the dimension $d^n$ of the Hilbert space
of the input states.
In our case of trivalent ($q=3$) functions of a single ($n=1$) trit ($d=3$), and there are 27 such functions on three-dimensional Hilbert space.
[For the original Deutsch algorithm computing the parity (constancy or nonconstancy) of the four binary functions of one bit, there are  $2^{2^1}=4$ such functions.]
\end{itemize}
\end{itemize}
}

 \frame{
\frametitle{Encoding  decision problems by state identification problems cntd.}

\begin{itemize}
\item<1->
For a one-to-one correspondence between functions and orthogonal states,
trivalent decision problems among the 27 trivalent functions of a single trit  require
three three-state quanta associated with
the set of $3^3=27$ states corresponding to some orthogonal base in ${\mathbb{C}}^3\otimes {\mathbb{C}}^3\otimes {\mathbb{C}}^3$.
Then, create three {\em equipartitions} containing  three elements per partition --- thus, every such partition element contains nine orthogonal states ---
such that
\begin{itemize}
\item<1->
{\em one of the partitions} corresponds to the solution of the decision problem.
\item<1->
The other two partitions ``complete'' the system of partitions such that
the set theoretic intersection of any three arbitrarily chosen elements of the three partition
with {\em one element per partition} always yields a {\em single} base state.
\end{itemize}
\end{itemize}
}

 \frame{
\frametitle{Encoding  decision problems by state identification problems cntd.}

\begin{itemize}
\item<1->
Formally, the three partitions correspond to a system of three co-measurable
{\em filter operators} $\textsf{\textbf{F}}_i$, $i=1,2,3$
with the following properties:
\begin{itemize}
\item[(F1)]
Every filter $\textsf{\textbf{F}}_i$
corresponds to an operator (or a set of operators)
which generates one of the three
equipartitions of the $27$-dimensional state space into
three slices (i.e., partition elements) containing $27/3=9$ states per slice.
A filter is said to separate two eigenstates if the eigenvalues are different.
\item[(F2)]
From each one of the three partitions of (F1), take an arbitrary element.
The intersection of these elements of all different partitions (one element per partition)
results in a {\it single} one of the $27$ different states.
\item[(F3)]
The union of all those single states generated by the intersections of (F2)
is the entire set of states.
\end{itemize}

\item<1->
As the first partition corresponds to the solution of the decision problem,
the corresponding first filter operator corresponds to the ``quantum oracle'' operator solving the decision problem.

\end{itemize}

 }


%%%%%%%%%%%%%%%%%%%%%%%%%%%%%%%%%%%%%%%%%%%%%%%%%%%%%%%%%%%%%%%%%%%%%%%%%%%%%%%%%%%%

\subsection{Example of trivalent functions of a single trit}

 \frame{
\frametitle{Example of trivalent functions of a single trit}
Formally, we shall consider the functions $$f \,{:}\linebreak[0]\,\;
\{-,0,+\} \to \{-,0,+\}$$ which will be denoted as triples
$$\bigl( f(-), f(0), f(+) \bigr)\; .$$
There are $3^{3^1} = 27$ such functions.
They can be enumerated in lexicographic order ``$-<0<+$'' as follows:
{\tiny
$$
\begin{array}{lllll}
\begin{array}{ll}
f_{0}: & (---)\\  f_{1}: & (--0)\\  f_{2}: & (--+)\\ f_{3}: & (-0-)\\  f_{4}: & (-00)\\  f_{5}: & (-0+)\\  f_{6}: & (-+-)\\  f_{7}: & (-+0)\\  f_{8}: & (-++)\\
\end{array}
&   \qquad \qquad \qquad &
\begin{array}{ll}
f_{9}: & (0--)\\  f_{10}: & (0-0)\\  f_{11}: & (0-+)\\ f_{12}: & (00-)\\  f_{13}: & (000)\\  f_{14}: & (00+)\\ f_{15}: & (0+-)\\  f_{16}: & (0+0)\\  f_{17}: & (0++)\\
\end{array}
&   \qquad \qquad \qquad &
\begin{array}{ll}
f_{18}: & (+--)\\  f_{19}: & (+-0)\\  f_{20}: & (+-+)\\ f_{21}: & (+0-)\\  f_{22}: & (+00)\\  f_{23}: & (+0+)\\ f_{24}: & (++-)\\  f_{25}: & (++0)\\  f_{26}: & (+++)
\end{array}
\end{array}
$$
}
 The trits will be coded by
elements of some  orthogonal base in ${\mathbb{C}}^3$. Without loss of
generality we may take
$(1,0,0) = |-\rangle$, $(0,1,0) = |0\rangle$,
$(0,0,1) = |+\rangle $.


 }

 \frame{
\frametitle{Example of trivalent functions of a single trit cntd.}
For a given ``quantum oracle'' function
$$g \,{:}\linebreak[0]\,\;
\{-,0,+\} \to {\mathbb{C}}$$
we represent a function
$f
\,{:}\linebreak[0]\,\; \{-,0,+\} \to \{-,0,+\}$ by a linear subspace of
${\mathbb{C}}^3$ generated by the vector
  $$g\bigl(f(-)\bigr)\,|-\rangle +
    g\bigl(f(0)\bigr)\,|0\rangle +
    g\bigl(f(+)\bigr)\,|+\rangle\,,$$
i.e., by the vector
$$\bigl(g(f(-)), g(f(0)), g(f(+))\bigr).$$


In order to be able to implement the first, re-encoding, step of the above strategy,
we will be searching for a function $g$ such that the subspaces representing functions
$\{-,0,+\} \to \{-,0,+\}$ are nonzero and form the smallest possible number --- ideally only one ---
of orthogonal triples.
}

 \frame{
\frametitle{Example of trivalent functions of a single trit cntd.}

Consider a function $g$ such that we obtain three orthogonal triples of orthogonal vectors,
each one of the three triples containing nine triples of the form  $\bigl( f(-), f(0), f(+) \bigr)$ and associated with cases of the functions $f$,
which can grouped into three partitions of three triples of the form $\bigl( f(-), f(0), f(+) \bigr)$.
Let the values of $g$ be the $\sqrt[3]{1}$ (in the set
of complex numbers). Let us, for the sake of simplicity and briefness of notation,
denote $\alpha ={\rm e}^{2\pi {\rm i}/3}=-{1\over 2}(1-i\sqrt{3})$. Then the values of $g$ are
$\alpha$, $\alpha^2 = \alpha^\ast  = {\rm e}^{-2\pi {\rm i}/3}=-{1\over 2}(1+i\sqrt{3})$ and $\alpha^3=1$. Moreover, $\alpha
\alpha^\ast  = 1$ and $\alpha + \alpha^\ast = -1$. Then, the ``quantum
oracle'' function $g$ might be given by the following table:
  $$
  \begin {array}{c||c|c|c}

  x    & -       & 0 & +\\\hline
  g(x) & \alpha^\ast  & 1 & \alpha    \\

  \end {array}
  $$
and (if we identify `$-$' with `$-1$' and `$+$' with `$+1$') might be
expressed by
  $$g(x) = \alpha^x = {\rm e}^{2\pi{\rm i}x/3}\,.$$

}

 \frame{
\frametitle{Example of trivalent functions of a single trit cntd.}
$g$ maps the 27 triples of functions $\bigl( f(-), f(0), f(+) \bigr)$ into nine groups of three triples of functions,
such that triples within the nine groups are assigned the same vector (except a nonzero multiple)
by the following scheme:
{\tiny
 $$
  \begin {array}{ccc}

\;\\
\left.
  \begin {array}{c}
  (-,-,-) \\
  (0,0,0) \\
  (+,+,+) \\
  \end {array}
\right\}
\mapsto  (1,1,1)
&
\left.
  \begin {array}{ccc}
 (-,-,0) \\
(0,0,+)\\
(+,+,-)\\
\end {array}
\right\}
 \mapsto (1,1,\alpha)
&
\left.
  \begin {array}{ccc}
(-,-,+) \\
(0,0,-)\\
(+,+,0) \\
\end {array}
\right\}
 \mapsto (1,1,\alpha^\ast ) \\
\;\\
%%%
\left.
  \begin {array}{ccc}
 (-,0,+)\\
 (0,+,-) \\
 (+,-,0)\\
\end {array}
\right\}
\mapsto  (1,\alpha,\alpha^\ast )
&
\left.
  \begin {array}{ccc}
&(-,0,-)\\
&(0,+,0) \\
&(+,-,+)\\
\end {array}
\right\}
\mapsto  (1,\alpha,1)
&
\left.
  \begin {array}{ccc}
&(-,+,-)\\
&(0,-,0)   \\
&(+,0,+)\\
\end {array}
\right\}
 \mapsto (1,\alpha^\ast ,1)\\
\;\\
%%%
 \left.
  \begin {array}{ccc}
 (-,+,0)\\
 (+,0,-) \\
 (0,-,+)\\
\end {array}
\right\}
\mapsto (1,\alpha^\ast ,\alpha)
&
\left.
  \begin {array}{ccc}
&(0,-,-)\\
&(+,0,0) \\
&(-,+,+)\\
\end {array}
\right\}
 \mapsto (\alpha,1,1)
&
\left.
  \begin {array}{ccc}
&(+,-,-)\\
&(-,0,0)\\
&(0,+,+)\\
\end {array}
\right\}
 \mapsto  (\alpha^\ast ,1,1)\\
\;\\

  \end {array}
  $$
}

More generally, one can prove by contradiction that in general the function $g$ cannot be defined
in such a way that we obtain at most two orthogonal triples of subspaces.
}

 \frame{
\frametitle{Example of trivalent functions of a single trit cntd.}
The geometric constraints in threedimensional Hilbert space can be interpreted
as the impossibility to ``fold'' a decision problem
into an appropriate quantum state identification in low-dimensional Hilbert space.

This can be circumvented by the introduction of additional quanta,
thereby increasing the dimension of Hilbert space.
In that way, the functions of a small number of bits can be mapped one-to-one onto orthogonal quantum states.
However, this strategy fails for a large number of arguments, since the ratio
of the number of $q$-ary functions of $n$ dits to the dimension of the Hilbert space of $n$ dits
$d^{-n} q^{d^n}$ increases fast with growing $n$.



}


%%%%%%%%%%%%%%%%%%%%%%%%%%
\section{How to acknowledge hypercomputation?}

 \frame{
\frametitle{How to acknowledge hypercomputation?}
Already in 1958, Martin Davis,
in {\it Computability and Unsolvability} (p. 11) asks:

 {\em `` $\ldots$ how can we ever exclude the possibility of our being presented,
 some day (perhaps by some extraterrestrial visitors), with a (perhaps
 extremely complex) device or ``oracle'' that ``computes'' a
 non-computable function?''}
}

 \frame{
\frametitle{How to acknowledge hypercomputation cntd.?}

Some concepts and questions:
\begin{itemize}
\item<1->
Black box model;

\item<1->
Are there there any ``operational verifiability'' beyond the capacity to solve low-polynomial (in terms of time \& memory space)
problems?

\item<1->
Consider asymmetric problems which are ``easy'' to generate but ``difficult'' to solve; e.g., graph isomorphism.


\item<1->
Consider two or more hypercomputers  which are used to ``compete'' against or ``check'' themselves.

\end{itemize}
}

%%%%%%%%%%%%%%%%%%%%%%%%%%

\frame{
\centerline{\Large Thank you for your attention!}
 }


\end{document}
