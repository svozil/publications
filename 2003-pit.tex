
\documentstyle[amstex,amsfonts,preprint,aps]{revtex}
%%%%%%%%%%%%%%%%%%%%%%%%%%%%%%%%%%%%%%%%%%%%%%%%%%%%%%%%%%%%%%%%%%%%%%%%%%%%%%%%%%%%%%%%%%%%%%%%%%%%%%%%%%%%%%%%%%%%%%%%%%%%
%TCIDATA{OutputFilter=LATEX.DLL}
%TCIDATA{Created=Tue Sep 30 16:24:53 2003}
%TCIDATA{LastRevised=Thu Nov 13 16:51:11 2003}
%TCIDATA{<META NAME="GraphicsSave" CONTENT="32">}
%TCIDATA{<META NAME="DocumentShell" CONTENT="Journal Articles\REVTeX - APS and AIP Article">}
%TCIDATA{Language=American English}
%TCIDATA{CSTFile=revtxtci.cst}
%TCIDATA{PageSetup=72,72,72,72,0}
%TCIDATA{Counters=arabic,1}
%TCIDATA{AllPages=
%H=36
%F=36
%}


\newtheorem{theorem}{Theorem}
\newtheorem{acknowledgement}[theorem]{Acknowledgement}
\newtheorem{algorithm}[theorem]{Algorithm}
\newtheorem{axiom}[theorem]{Axiom}
\newtheorem{claim}[theorem]{Claim}
\newtheorem{conclusion}[theorem]{Conclusion}
\newtheorem{condition}[theorem]{Condition}
\newtheorem{conjecture}[theorem]{Conjecture}
\newtheorem{corollary}[theorem]{Corollary}
\newtheorem{criterion}[theorem]{Criterion}
\newtheorem{definition}[theorem]{Definition}
\newtheorem{example}[theorem]{Example}
\newtheorem{exercise}[theorem]{Exercise}
\newtheorem{lemma}[theorem]{Lemma}
\newtheorem{notation}[theorem]{Notation}
\newtheorem{problem}[theorem]{Problem}
\newtheorem{proposition}[theorem]{Proposition}
\newtheorem{remark}[theorem]{Remark}
\newtheorem{solution}[theorem]{Solution}
\newtheorem{summary}[theorem]{Summary}

\begin{document}
\title{New Bell Inequalities}
\author{I. Pitowsky}
\address{The Hebrew University}
\date{\today}
\maketitle
\pacs{}

\begin{abstract}
A new set if Bell type inequalities is proposed
\end{abstract}

\section{The CUT polytope and the BELL polytope}

Let ${\bf a}=(a_{1},a_{2},...,a_{n})\in \{-1,1\}^{n}$, for $1\leq i<j\leq n$
denote $\sigma _{ij}({\bf a})=a_{i}a_{j}$ and consider it as a vector in $%
{\Bbb R}^{\frac{1}{2}n(n-1)}$ with lexicographic order on the indices. The
convex hull, in ${\Bbb R}^{\frac{1}{2}n(n-1)}$, of $\{\sigma _{ij}({\bf a}%
);\ {\bf a}\in \{-1,1\}^{n}\}$ is a polytope, call it $BELL(n)$. Similarly,
Let ${\bf x}=(x_{1},x_{2},...,x_{n})\in \{0,1\}^{n}$ for $1\leq i<j\leq n$
denote $\delta _{ij}({\bf x})=x_{i}\oplus x_{j}=x_{i}+x_{j}-2x_{i}x_{j}$
and, again, consider it as a vector in ${\Bbb R}^{\frac{1}{2}n(n-1)}$. The
cut polytope $CUT(n)$ is the convex hull of $\{\delta _{ij}({\bf x});\ {\bf x%
}\in \{0,1\}^{n}\}$. The relations between these two polytopes is not hard
to derive. Since for all $1\leq i<j\leq n$
\begin{equation}
\delta _{ij}(x)=x_{i}\oplus x_{j}=\frac{1-s_{i}s_{j}}{2}=\frac{1-\sigma
_{ij}(s)}{2}\quad where\ s_{i}=2x_{i}-1
\end{equation}
we conclude that $q_{ij}\in $ $BELL(n)$ if and only if $%
%TCIMACRO{\UNICODE[m]{0xbd}}%
%BeginExpansion
{\frac12}%
%EndExpansion
(1-q_{ij})\in CUT(n)$.

The face inequalities for $BELL(n)$ have the form $\sum_{1\leq i<j\leq
n}\alpha _{ij}q_{ij}\leq \alpha $, where $q_{ij}$ is any element of $BELL(n)$
and $\alpha _{ij}$ and $\alpha $ are real numbers. The inequality is valid
if and only if it is satisfied by all the vertices $\sigma _{ij}({\bf a}%
)=a_{i}a_{j}$ of $BELL(n)$. It represents a {\em facet} if, in addition,
equality holds for a subset of the $\sigma _{ij}$'s which spans an affine
subspace of co-dimension one.

There is a great deal of interest in deriving facet inequalities for $BELL(n)
$ because Bell inequalities, and their generalizations are among them. From
the point of view of computational complexity deriving all the facet
inequalities for all $n$ is an impossible task [1] (see also the appendix).
This, however, does not mean that some inequalities, or even infinite
families of inequalities (for growing $n$) are beyond discovery. In
particular, for applications to physics, we are interested in finding valid
inequalities with some vanishing coefficients $\alpha _{ij}$, so they can be
written in the bipartite form:
\begin{equation}
\sum_{i=1}^{k}\sum_{j=1}^{k}\alpha _{ij}a_{i}b_{j}\leq \alpha
\end{equation}
Or the more general bipartite form

\begin{equation}
\sum_{i=1}^{k}\alpha _{i}a_{i}+\sum_{i=1}^{k}\beta
_{i}b_{i}+\sum_{i=1}^{k}\sum_{j=1}^{k}\alpha _{ij}a_{i}b_{j}\leq \alpha
\end{equation}
Here ${\bf a}=(a_{1},a_{2},...,a_{k})$, and ${\bf b}=(b_{1},b_{2},...,b_{k})$
are elements of $\{-1,1\}^{k}$, and $n=2k$. In other words, the coefficients
of $a_{i}a_{j}$ and $b_{i}b_{j}$ vanish. Such inequalities give us a bound
on the predictions of local hidden variable theories. Suppose that Alice and
Bob are sharing a sequence of identically prepared two part systems.
Suppose, moreover, that on her side Alice is measuring on each system one of
$k$ possible observables, whose range of values is $a_{i}=\pm 1$. Likewise,
Bob on his side is measuring one of $k$ observables with values $b_{j}=\pm 1$%
, $j=1,2,...,k$. Denote the average result of $a_{i}b_{j}$ by $q_{ij}$.
Classically, if (2) is a valid inequality for $BELL(n)$ we would expect the
averages $q_{ij}$ to satisfy $\sum_{i=1}^{m}\sum_{j=1}^{m}a_{ij}q_{ij}\leq
\alpha $. As is well known, the quantum mechanical expectation values
violate some such inequalities. However, there are only a handful of known
two particle inequalities, and all infinite families of the form (2), (3)
(with variable $m$) were found to be derivatives of the CH inequality.

In the present paper we shall use known inequalities of $CUT(n)$ to derive
two particle inequalities with any number of measurements on each side. The
inequalities have the general form (2), (3). On the way we also derive three
particle inequalities which involve only {\em pair} correlations and have
the form

\begin{equation}
\sum_{i=1}^{k}\sum_{j=1}^{k}\alpha
_{ij}a_{i}b_{j}+\sum_{i=1}^{k}\sum_{j=1}^{k}\beta
_{ij}a_{i}c_{j}+\sum_{i=1}^{k}\sum_{j=1}^{k}\gamma _{ij}b_{i}c_{j}\leq
\alpha \quad a_{i},b_{i},c_{i}\in \{-1,1\}
\end{equation}
And also analogues of (3) with marginals.

\section{The Clique-Web inequalities of the CUT polytope}

A graph $G=(V,E)$ consists of a set of vertices $V_{n}=\{1,2,...,n\}$ and a
set of edges $E$, which are just pairs of vertices. We shall always assume
that if $\{i,j\}\in E$ then $i\neq j$. (Note, we do {\em not} assume
directionality, so that $\{i,j\}=\{j,i\}$ is the same edge). If the set of
vertices $V_{n}$ has been fixed we shall often speak loosely on `the graph $%
E $' mentioning only the edges. The set of all pairs on $n$ vertices is
called the {\em complete} graph and denoted by $K_{n}$.

Let $S\subset V_{n}=\{1,2,...,n\}$ be a non empty subset of vertices. Denote
by $\kappa (S)=\{\{i,j\};\ i\neq j,\;i,j\in S\}$. If $\kappa (S)\subset E$
then $\kappa (S)$ is called a clique in the graph $G=(V_{n},E)$. Also,
define a graph $\delta (S)=\{\{i,j\};\ i\in S,j\notin S\ or\ i\notin S,j\in
S\}$. The graph $\delta (S)$ is called {\em a} {\em cut} (or a cut in $K_{n}$%
). Denote by $(x_{1},x_{2},...,x_{n})\in \{0,1\}^{n}$ the indicator function
of $S$, so that $x_{i}=1$ for $i\in S$ and $x_{i}=0$ otherwise. Then $%
\{i,j\}\in $ $\delta (S)$ if and only if $x_{i}\oplus
x_{j}=x_{i}+x_{j}-2x_{i}x=1$. Hence, the vertices of the cut polytope $%
CUT(n) $, that is $\delta _{ij}(x)=x_{i}\oplus x_{j}$, are the indicator
functions of the cuts $\delta (S)$.

The Inequalities that we shall consider follow from a precise
characterization of $\delta (S)\cap E$ for a particular type of graphs $E$
called a web:

\begin{definition}
Let $p,q$, and, $r$ ne three integers such that $q\geq 2$ and $p-q=2r+1$.
The {\em antiweb }$AW_{p}^{r}$ is the graph whose set of vertices is $%
V_{p}=\{1,2,...,p\}$ and set of edges is $\{\{i,i+1\},...,\{i,i+r\}\ ;\
i=1,2,...,p,\ addition\ is\
%TCIMACRO{\func{mod}}%
%BeginExpansion
\mathop{\rm mod}%
%EndExpansion
p\}$ The {\em web} $W_{p}^{r}$ is the complement in $K_{p}$ of the antiweb $%
AW_{p}^{r}$.
\end{definition}

In figure 1 the graph $K_{p}$ is depicted and the elements of $W_{p}^{r}$
have a gray background.

\begin{center}
{\large Insert Figure 1}
\end{center}

The graph $W_{p}^{r}$ is called a ``web'' because we obtain a spider web
picture when we draw the vertices $\{1,2,...,p\}$on the circumference of a
circle and the edges of $W_{p}^{r}$ as chords. For these graphs Alon [2]
proved the following

\begin{theorem}
Let $p$, $r$ be integers such that $p\geq 2r+3$, $r\geq 1$. Let $S\subset
\{1,2,...,p\}$ and assume that $\left| S\right| =s$.

1. If $s\leq r$, then $\left| \delta S\cap AW_{p}^{r}\right| $ $\geq
s(2r+1-s)$, with equality if and only if $\kappa (S)$ is a clique in $%
AW_{p}^{r}$.

2. If $r+1\leq s\leq \frac{p}{2}$, then $\left| \delta S\cap
AW_{p}^{r}\right| \geq r(r+1)$ with equality if and only if $S$ is an
interval in $\{1,2,...,p\}$, that is, it has the form $S=%
\{i,i+1,i+2,...,i+s-1\}$ for some $1\leq i\leq p$ (addition is $%
%TCIMACRO{\func{mod}}%
%BeginExpansion
\mathop{\rm mod}%
%EndExpansion
p$.)
\end{theorem}

As an outcome of Alon's theorem Deza and Laurent proved the following
inequality [3] ($q=p-2r-1$):

\begin{equation}
\sum_{\{i,j\}\in W_{p}^{r}}x_{i}\oplus x_{j}+\sum_{1\leq i<j\leq
q}y_{i}\oplus y_{j}-\sum_{i=1}^{p}\sum_{j=1}^{q}x_{i}\oplus y_{j}\leq 0
\end{equation}
Here $x_{1},...,x_{p},y_{1},...,y_{q}\in \{0,1\}$ are arbitrary. This is, in
fact, a facet inequality of $CUT(n)$, for $n=p+q$. Equality holds in the
following cases

{\bf 1}. The graph $\{\{i,j\};\ x_{i}x_{j}=1\}$ is a clique in $AW_{p}^{r}$
and $y_{1}=...=y_{q}=0$.

{\bf 2}. The set $\{i;\ x_{i}=1\}$ is an interval in $\{1,2,...,p\}$, and $%
r+1\leq \sum_{i=1}^{p}x_{i}\leq p-r-2$, and $\sum_{i=1}^{p}x_{i}-%
\sum_{i=1}^{q}y_{i}=r,r+1$.

Using (1) with $a_{i}=2x_{i}-1$, $d_{i}=2y_{i}-1$and the fact that $\left|
W_{p}^{r}\right| =\frac{pq}{2}$ we get for $BELL(n)$ the following \
inequality:

\begin{equation}
\sum_{i=1}^{p}\sum_{j=1}^{q}a_{i}d_{j}-\sum_{\{i,j\}\in
W_{p}^{r}}a_{i}a_{j}-\sum_{1\leq i<j\leq q}d_{i}d_{j}\leq q(r+1)
\end{equation}
Where $a_{1},...,a_{p},d_{1},...,d_{q}\in \{-1,1\}$ are arbitrary.

\section{New Bell Inequalities}

We can get from (6) new two or three particles, multi measurements
inequalities using the fact that the $d_{i}$ can be given an arbitrary value
$\pm 1$, and that $W_{p}^{r}$ breaks naturally into blocks. To see that let $%
l\geq 1$, $m\geq 2$ be natural numbers and choose $q=m+1$, $r+1=lm$ and $%
p=(2l+1)m$, and divide the variables $a_{1},...,a_{p}$ into three groups:
The first $lm$ variables denote by $a_{1},...,a_{lm}$ the next $m$ variables
denote by $c_{1},...,c_{m}$ and the last $lm$ variables by $b_{1},...,b_{lm}$%
. The $a$ variables are broken into $l$ groups of $m$ variables each ${\bf a}%
_{1}=(a_{1},...,a_{m})$, ..., ${\bf a}_{l}=(a_{(l-1)m+1},...,a_{lm})$.
Similarly the $b$ variables are broken into $l$ blocks of $m$ variables: $%
{\bf b}_{1}=(b_{1},...,b_{m})$, ..., ${\bf b}_{l}=(b_{(l-1)m+1},...,b_{lm})$%
. Also, put ${\bf c}=(c_{1},...,c_{m})$ then the sum over the edges of $%
W_{p}^{r}$ in (6) involves only a mix of $ab$ variables, $ac$ variables or $%
bc$ variables, as can be seen from figure 2.

\begin{center}
{\large Insert Figure\ 2}
\end{center}

In order to write the inequalities in a concise form introduce the following
notation: If ${\bf u=}(u_{1},...,u_{m})$ and ${\bf v=}(v_{1},...,v_{m})$ are
two $m$-dimensional vectors put ${\bf u\ast v=}\sum_{i=1}^{m}%
\sum_{j=i}^{m}u_{i}v_{j}$. Note that in general ${\bf u\ast v\neq v\ast u}$.
Also, if ${\bf w=}(w_{1},...,w_{k})$ put ${\bf u\times w=}%
\sum_{i=1}^{m}\sum_{j=1}^{k}u_{i}w_{j}$. With these notations and with ${\bf %
d}=(d_{1},...,d_{m+1})$ the inequality (6) becomes:

\begin{gather}
\sum_{i=1}^{l}{\bf a}_{i}{\bf \times d}+\sum_{i=1}^{l}{\bf b}_{i}{\bf \times
d\ +\ c}\times {\bf d}-{\bf a}_{1}\ast {\bf c-}\sum_{i=1}^{l}{\bf b}_{i}\ast
{\bf a}_{i}\smallskip  \nonumber \\
-\sum_{i=1}^{l-1}{\bf a}_{i+1}\ast {\bf b}_{i}-{\bf c\ast b}_{l}\leq
\sum_{1\leq i<j\leq m+1}d_{i}d_{j}+lm(m+1)
\end{gather}
To derive three particles inequality we can choose any $\pm 1$ values for
the $d_{i}$'s. However, a choice which is compatible with Alon's theorem
gives the strongest inequality. Thus, if we choose $d_{i}=1$, $1\leq i\leq
m+1$, then equality will hold in (7) whenever the set $S$ of indices on
which the $a$'s, $b$'s and $c$'s have value $+1$ determines a clique $\kappa
(S)\subset AW_{p}^{r}$. A two particle inequality can be obtained when we
choose the elements $c_{i}$ to have a fixed value $\pm 1$. Then equality
will hold when the above mentioned set $S$ includes the indices on which $%
c_{i}=1$.

Similarly we can choose some of the $d_{i}^{\prime }$'s to be $-1$, say on a
set of indices whose size is $k$, $k\leq \frac{m+1}{2}$. Then we will get a
three particle inequality, with equality holding each time we assign the $a$%
's, $b$'s and $c$'s the value $+1$ on an interval of length $(l+1)m-k$ or
length $(l+1)m-k+1$. Fixing the value of the $c_{i}$'s will give us two
particles inequality, with equality holding whenever $\{i\ ;c_{i}=1\}$ is a
part of that interval.

All these inequalities can be subjected to the usual symmetry operations of
the polytope $BELL(n)$ to obtain new inequalities. The symmetries are the
permutations of the variables, and the reflection $a_{i}\rightarrow -a_{i}$,
$b_{i}\rightarrow -b_{i}$, $c_{i}\rightarrow -c_{i}$ of any or all the
variables.

\section{An example}

Consider the case $m=l=3$, then $p=(2l+1)m=21$, $q=m+1=4$. We have nine $a$%
-variables $a_{1},...,a_{9}$, nine $b$-variables $b_{1},...,b_{9}$, and
three $c$-variables $c_{1},c_{2},c_{3}$. If we choose $d_{i}=1$ $i=1,2,3,4$
then (7) becomes:

\begin{gather}
4\sum_{i=1}^{9}a_{i}+4\sum_{i=1}^{9}b_{i}+4%
\sum_{i=1}^{3}c_{i}-(a_{1}c_{1}+a_{1}c_{2}+a_{1}c_{3}+a_{2}c_{2}+a_{2}c_{3}+a_{3}c_{3})-\smallskip
\\
-(a_{1}b_{1}+a_{2}b_{1}+a_{3}b_{1}+a_{2}b_{2}+a_{3}b_{2}+a_{3}b_{3})-(a_{4}b_{1}+a_{4}b_{2}+a_{4}b_{3}+a_{5}b_{2}+a_{5}b_{3}+a_{6}b_{3})-\smallskip
\nonumber \\
-(a_{4}b_{4}+a_{5}b_{4}+a_{6}b_{4}+a_{5}b_{5}+a_{6}b_{5}+a_{6}b_{6})-(a_{7}b_{4}+a_{7}b_{5}+a_{7}b_{6}+a_{8}b_{5}+a_{8}b_{6}+a_{9}b_{6})-\smallskip
\nonumber \\
-(a_{7}b_{7}+a_{8}b_{7}+a_{9}b_{7}+a_{8}b_{8}+a_{9}b_{8}+a_{9}b_{9})-(b_{7}c_{1}+b_{8}c_{1}+b_{9}c_{1}+b_{8}c_{2}+b_{9}c_{2}+b_{9}c_{3})\leq 42
\nonumber
\end{gather}

\bigskip If we take $c_{i}=1$ we get the following two particle inequality

\begin{gather}
a_{1}+2a_{2}+3a_{3}+4\sum_{i=4}^{9}a_{i}+4%
\sum_{i=1}^{6}b_{i}+3b_{7}+2b_{8}+b_{9}-\smallskip \\
-(a_{1}b_{1}+a_{2}b_{1}+a_{3}b_{1}+a_{2}b_{2}+a_{3}b_{2}+a_{3}b_{3})-(a_{4}b_{1}+a_{4}b_{2}+a_{4}b_{3}+a_{5}b_{2}+a_{5}b_{3}+a_{6}b_{3})-\smallskip
\nonumber \\
-(a_{4}b_{4}+a_{5}b_{4}+a_{6}b_{4}+a_{5}b_{5}+a_{6}b_{5}+a_{6}b_{6})-(a_{7}b_{4}+a_{7}b_{5}+a_{7}b_{6}+a_{8}b_{5}+a_{8}b_{6}+a_{9}b_{6})-\smallskip
\nonumber \\
-(a_{7}b_{7}+a_{8}b_{7}+a_{9}b_{7}+a_{8}b_{8}+a_{9}b_{8}+a_{9}b_{9})\leq 30
\nonumber
\end{gather}

If we keep the same parameters and choose $d_{1}=d_{2}=1$ and $%
d_{3}=d_{4}=-1 $ we get the marginal free three particle inequality

\begin{gather}
-(a_{1}c_{1}+a_{1}c_{2}+a_{1}c_{3}+a_{2}c_{2}+a_{2}c_{3}+a_{3}c_{3})- \\
-(a_{1}b_{1}+a_{2}b_{1}+a_{3}b_{1}+a_{2}b_{2}+a_{3}b_{2}+a_{3}b_{3})-(a_{4}b_{1}+a_{4}b_{2}+a_{4}b_{3}+a_{5}b_{2}+a_{5}b_{3}+a_{6}b_{3})-\smallskip
\nonumber \\
-(a_{4}b_{4}+a_{5}b_{4}+a_{6}b_{4}+a_{5}b_{5}+a_{6}b_{5}+a_{6}b_{6})-(a_{7}b_{4}+a_{7}b_{5}+a_{7}b_{6}+a_{8}b_{5}+a_{8}b_{6}+a_{9}b_{6})-\smallskip
\nonumber \\
-(a_{7}b_{7}+a_{8}b_{7}+a_{9}b_{7}+a_{8}b_{8}+a_{9}b_{8}+a_{9}b_{9})-(b_{7}c_{1}+b_{8}c_{1}+b_{9}c_{1}+b_{8}c_{2}+b_{9}c_{2}+b_{9}c_{3})\leq 34
\nonumber
\end{gather}

and any choice of values $c_{i}=\pm 1$ gives a valid two particle inequality.

\section{Appendix: Relations with probability theory and complexity theory.}

Let $x=(x_{1},x_{2},...,x_{n})\in \{0,1\}^{n}$, for $1\leq i\leq j\leq n$
denote $\pi _{ij}(x)=x_{i}x_{j}$ and consider it as a vector in ${\Bbb R}^{%
\frac{1}{2}n(n+1)}$ with lexicographic order on the indices. The convex
hull, in ${\Bbb R}^{\frac{1}{2}n(n+1)}$, of \ $\{\pi _{ij}(x);\ x\in
\{0,1\}^{n}\}$ is called correlation polytope, and denoted by $COR(n)$.
Similarly, we have denoted by $CUT(n)$ the convex hull in ${\Bbb R}^{\frac{1%
}{2}n(n-1)}$of $\sigma _{ij}(x)=x_{i}\oplus x_{j}=x_{i}+x_{j}-2x_{i}x_{j}$
for $1\leq i<j\leq n$. (Note that $COR(n)$ has dimension $\frac{1}{2}n(n+1)$
while $CUT(n)$ has dimension $\frac{1}{2}n(n-1)$). For these polytope we
have: [3],[4]

\begin{theorem}
(a) Let $p_{ij}\in {\Bbb R}^{\frac{1}{2}n(n+1)}$ then $p_{ij}\in COR(n)$ if
and only if there is a probability space $(X,\Sigma ,\mu )$ and events $%
E_{1},E_{2},...,E_{n}\in \Sigma $ such that $p_{ij}=\mu (E_{i}E_{j})$ for $%
1\leq i\leq j\leq n$ . (b) Let $c_{ij}\in {\Bbb R}^{\frac{1}{2}n(n-1)}$ then
$c_{ij}\in CUT(n)$ if and only if there is a probability space $(X,\Sigma
,\mu )$ and events $E_{1},E_{2},...,E_{n}\in \Sigma $ such that $c_{ij}=\mu
(E_{i}\bigtriangleup E_{j})=\mu \lbrack (E_{i}\setminus E_{j})\cup
(E_{j}\setminus E_{i})]=$ $\mu (E_{i})+\mu (E_{j})-2\mu (E_{i}E_{j})$ for $%
1\leq i<j\leq n$.
\end{theorem}

For the correlation polytope we have the following complexity results which
can be immediately transferred to the cut polytope and the Bell polytope.

{\large 1. }Deciding whether a given rational $p_{ij}\in {\Bbb R}^{\frac{1}{2%
}n(n+1)}$ is an element of $COR(n)$ is an NP-complete problem [1].

{\large 2.}Deciding whether a given inequality is not valid for $COR(n)$ is
an NP-complete problem [1].

All this means that unless NP=P (or, at least NP=coNP) deriving all the
inequalities for any of these polytopes is a computationally impossible task
for large $n$. A few infinite families of inequalities are known [3] (and
any inequality for one of these polytopes is easily translated to an
inequality of the others). In most cases the inequalities do not yield a
bipartite or tripartite form, which are potentially applicable to quantum
mechanics. I hope the above Clique-Web inequalities do.\newpage

{\large References}

[1] Pitowsky, I. Correlation Polytopes: Their Geometry and Complexity, {\em %
Mathematical Programming A50}, 395-414 (1991).

[2] Alon, N. The CW-inequalities for vectors in $l_{1}$ {\em European
Journal of Combinatorics 1}, 1-6 (1990).

[3] Deza, M., M., and Laurent, M. {\em Geometry of Cuts and Metrics}
Heidelberg, Springer, Algorithms and Combinatorics, Vol. 15, 587 pp.(1997)

[4] Pitowsky, I. {\em Quantum Probability, Quantum Logic}, Heidelberg,
Springer, Lecture Notes in Physics 321, (1989).

\end{document}
