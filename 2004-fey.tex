\documentclass{article}
\begin{document}

\title{Feyerabend and physics}
\author{Karl Svozil\footnote{email: svozil@tuwien.ac.at, homepage: http://tph.tuwien.ac.at/$\sim$svozil}\\
Institut f\"ur Theoretische Physik, University of Technology Vienna, \\
Wiedner Hauptstra\ss e 8-10/136, A-1040 Vienna, Austria}

\maketitle

\begin{abstract}
Feyerabend frequently discussed physics.
He also referred to the history of the subject when motivating his philosophy of science.
Alas, as some examples show, his understanding of physics remained superficial.
Partly due to the complexity of the formalism which has left many philosophers at a loss,
physicists have attempted to develop their own meaning of the quanta.
This has stimulated a new kind of empiricism, an {\em experimental philosophy,}
which is plagued by the inevitable interpretation of the raw data, in particular incommensurability.
Feyerabend has expressed profound insights into methodological issues related to the progress of physics, a legacy which remains to be implemented in the times to come:
the conquest of abundance, the richness of reality,
the many worlds which still await discovery,
and the vast openness of the physical universe.
\end{abstract}


%%%%%%%%%%%%%%%%%%%%%%%%%%%%%%%%%%%%%%%%%%%%%%%%%%%%%%%%%%%%%%%%%%%%%%%%%%%%%%%%%%%%%%%%%%%%%%%%%%%%%%%%%%%%%%%
%\documentclass[rmp,twocolumn,showpacs,showkeys,amsfonts]{revtex4}
%\usepackage{graphicx}
%\RequirePackage{times}
%%\RequirePackage{courier}
%\RequirePackage{mathptm}
%%\renewcommand{\baselinestretch}{1.3}
%\begin{document}
%
%\title{Feyerabend and physics\footnote{Presented at the International Symposium Paul Feyerabend 1924-1994.
%A philosopher from Vienna, University of Vienna, June 18-19, 2004. Also available from URL {\tt http://www.arxiv.org/abs/physics/0406079}}}
%\author{Karl Svozil}
% \email{svozil@tuwien.ac.at}
%\homepage{http://tph.tuwien.ac.at/~svozil}
%\affiliation{Institut f\"ur Theoretische Physik, University of Technology Vienna,
%Wiedner Hauptstra\ss e 8-10/136, A-1040 Vienna, Austria}
%
%\begin{abstract}
%Feyerabend frequently discussed physics.
%He also referred to the history of the subject when motivating his philosophy of science.
%Alas, as some examples show, his understanding of physics remained superficial.
%Partly due to the complexity of the formalism which has left many philosophers at a loss,
%physicists have attempted to develop their own meaning of the quanta.
%This has stimulated a new kind of empiricism, an {\em experimental philosophy,}
%which is plagued by the inevitable interpretation of the raw data, in particular incommensurability.
%Feyerabend has also expressed profound insights into the possibilities for the progress of physics,
%a legacy which remains to be implemented in the times to come:
%the conquest of abundance, the richness of reality,
%the many worlds which still await discovery,
%and the vast openness of the physical universe.
%\end{abstract}
%
%
%\pacs{01.60.+q,01.70.+w,01.65.+g}
%\keywords{Biographies, tributes, personal notes, and obituaries; Philosophy of science; History of science}
%
%\maketitle
%
%\tableofcontents
%%%%%%%%%%%%%%%%%%%%%%%%%%%%%%%%%%%%%%%%%%%%%%%%%%%%%%%%%%%%%%%%%%%%%%%%%%%%%%%%%%%%%%%%%%%%%%%%%%%%%%%%%%%%%%%


\section*{Preamble}

In the early morning hours before this talk, I had a horrifying dream.
I found
myself in the position of being expelled from the
physics department.
I enter it lately, coming home to my institute,
%which I really like working in,
either from some mushroom picking or from this conference.
The atmosphere is hostile.
I walk to my room.
The room is occupied with some post-doc students
of the department head.
The windows which usually overlook the city center are blinded.
I am told that the head of the department was trying to reach me the entire day,
and that he summons me up on a very grave and serious affair.
When I enter his gigantic office, he sits at a huge table.
Other very serious members of the institute are gathered as well.
They immediately tell me to get seated and listen to
the indictments.
When I try to recall which scientific crimes I could have possibly committed,
I wake up.

In retrospect, I know what crimes I have committed:
Long time ago, in almost another live with other persons and other institutions,
I have told them ``the truth,''
at least my visions of ``the truth.''
These visions were in many ways totally off mainstream,
and I suffered from the disguise of my colleagues.
In Berkeley, I had to appeal to the head of the {\em Lawrence Berkeley Laboratory}'s
physics department to get a paper on relativity theory published as an {\em LBL} preprint
\cite{svo-83,svo5,svozil-relrel}
which was rejected by Chew on the basis of Stapp's judgment
that, if my recollection is correct {\em ``this is not the way to proceed.''}
Not that I did not also pursue ``normal'' science,
publishable in {\em Physical Review Letters,}
in {\em Physical Review} or in the {\em Journal of Mathematical Physics.}
But I did also crazy stuff, for which I suffered in my early days.
Even later on, when I spoke in a conference organized by the {\em Institut Wiener Kreis}
about the {\em physics of virtual realities}
 \cite{svozil-nat-acad},
 some of the material contained in my book on {\em Randomness and Undecidability in Physics} \cite{svozil-93},
I still remember Professor Flamm shaking his head in disguise, saying,
{\em ``Dieser Svozil ist total \"ubergeschnappt''}
(in English {\em ``Svozil has gone totally crazy''}),
and one of the organizers, Jimmy Shimanovich,  later tried to propitiate me with the words,
{\em ``but at least you have one advantage over most of the other speakers: you have already
prepared your manuscript before your talk!''}


I am deeply thankful to Paul Feyerabend for emphasizing so unequivocally the necessity and the value
of original research and the pursuit of ``crazy'' and unfashionable ideas \cite{dyson-unfash}
and methods.
I know that many people, including Lakatos, Kuhn, Dyson and others before and after him have expressed this
necessity, but never were they so outspoken  as Feyerabend.
He gave all those talented original undergraduates and young scientists in the wild
a clear message which could help to set them free, thereby giving science yet another unexpected turn.
In the words of the {\em Bhagavad Gita,}
{\em ``go out and conquer yourself a prosperous kingdom!''}


\section{General attitude}

In his autobiography Feyerabend admitted
that  for his Ph.D. thesis supervised by Hans Thirring \footnote{
Hlavka reports that Hans Thirring and Hahn conducted parapsychological
experiments in an apartment at the {\em Ringstrasse}.}
he had started working on a problem of classical electrodynamics
which he could not solve (p.~85 of Ref.~\cite{feyerabend-auto}).
He then turned to Kraft and to Thirring who accepted a thesis not in physics proper,
but in the philosophy of science.
Later, Feyerabend wrote several papers
\cite{Oberheim-97,Oberheim-99}
on physics-related topics, in particular on the interpretation on quantum mechanics,
on classical  and  on statistical physics, reprinted
mainly in the first volume of his {\em Philosophical Papers} \cite{fey-philpapers1,fey-philpapers2,fey-philpapers3}.
%A brief and incomplete bibliography is contained in the Appendix.
Unlike Popper's attempts to ``falsify the Copenhagen interpretation''
\cite{2002-peres} and argue against the quantum logic introduced by Birkhoff and von Neumann
\cite{dalla-2002},
Feyerabend pursued these investigations in a cautious, considerate and  self-critical style.

Professor Fischer recalls \cite{fischer} that the physicists in Berkeley,
in particular Karplus \cite{fischer-04},
generously evaluated Feyerabend to be ``merely'' two decades behind current research,
the average philosopher being at least half-a-century behind.
Also, Fischer recalls, Feyerabend was happy with the evaluation, and told him
that there was no essential difference between a physicist and a good philosopher,
and that Feyerabend considered himself to be too stupid to be
a good physicist: { ``Apart from his stupidity --- he assured me --- nothing separated him from
being a physicist.''}

I am inclined to agree with this self-evaluation only partially.
Feyerabend certainly was very intelligent and a person full of resources.
It may well be that he did not want to be bothered with the sometimes tedious
task to work out theories formally, or to setup and run experiments.

For whatever reasons, Feyerabend's contributions to physics were minor.
In contrast to physics, Feyerabend's contributions and insights into methodological issues are,
at least in my opinion,  remarkable.
The style in which his statements were expressed was provocative;
sometimes even bordering to the offensive; always gathering attention and raising eyebrows.

Getting attention was certainly one of his biggest intentions.
He did neither succeed as opera singer, nor at the theater, but certainly at the academic stage.
Often the reactions were harsh.
In an article published in {\em Nature}
(p.~596 of Ref.~\cite{theo-psi-1987}), Feyerabend was referred to as
{ ``the Salvador Dali of academic philosophy, and currently the worst enemy of science;''}
a denunciation which deeply saddened him (Chapter 12 of Ref.~\cite{feyerabend-auto}).
I do not think that such a term is justified.
Popper with his naive viewpoints and his talk about {\em ``blablabla''}
certainly did more harm to science \cite{svozil-2002-popper}
than any other dilettante claiming to know the proceeds of science before;
but not Feyerabend.
 On the contrary I believe that Feyerabend was right in suggesting
that input from the outside does science proper good;
even if one is not willing to grant that
{ ``science has now become as oppressive as the ideologies it had once to fight''}
\cite{feyerabend-defense,feyer-81}.


Besides his methodological openness,
Feyerabend's lasting message, in my opinion, is the ``conquest of abundance,''
the ``richness'' of the phenomena around us, and
the ``vastness'' of the territories still awaiting to be discovered.
Of course, this message, as many things Feyerabend said, is not entirely new.
One finds similarities with Bergson, Broad, in Huxley's {\em Doors of Perception,}
as well as in modern neurophysiologic investigations.
But it is still worth stressing that
the restricted view of the world in the present scientific perspective
is rather a consequence of  tradeoffs between comprehensibility and exhaustiveness
than a property of nature.
We are just at the beginning
of the scientific revolutions, and there are numerous challenging
and worthwhile tasks out there for the generations to come.
The pursuit of science is one of the greatest passions of life,
and our capabilities to recognize and manipulate the physical world
may only be limited by our phantasy.
Maybe one hopefully happy day we will be able to {\em tune} the world according to our will alone.


\section{Tower of Pisa example in ``Against Method''}

One of the things which Feyerabend discussed in {\em Against Method} \cite{feyerabend} in greater detail
is the Tower of Pisa example.
It is about an old argument against earth rotation which has been already put forward by Aristotle:
A stone from a high tower arrives at the foot of the tower without
any shift relative to the horizontal position of the release point on top of the tower.


Admittedly, Feyerabend had other objectives in mind,
in particular some supposed   ``deceptions'' by Galileo,
who allegedly ``brushed aside'' topics seemingly in conflict with his heliocentric approach by maintaining that the phenomena
could be correctly described while at the same time
``hiding'' new ``absurd'' theoretical assumptions.
Feyerabend completely omitted the contemporary physics of the Tower of Pisa example.

Indeed, Galileo seems to have committed himself to the attitude that
there should be no shift whatsoever,
a wrong conjecture which also seemed to have been accepted  by Copernicus.
Newton and Hook investigated this topic more carefully.
Indeed, this may have been the starting point of Newton's theory of gravity.
Incidentally, also Gauss  and  Laplace held wrong theoretical opinions on the phenomenon.

After a succession of inconclusive measurements by different researchers,
Hall performed  experiments in Harvard in 1902
 \cite{hall-1903a,hall-1903b}.
Due to the admirable effort of the {\em American Physical Society}
to retroscan their entire collection of scholarly articles published in the {\em Physical Reviews},
Hall's superbly written contributions are easily obtainable.
A later review by Armitage \cite{armitage} which is also cited in  {\em Against Method}
states,
\begin{quote}
{ ``$\ldots$ Thus Newton's experimental test for the diurnal rotation
of the Earth may be said to have given positive results of the expected
order of magnitude, though the persistent occurrence of an unaccountable southward deviation
has continued to be a matter for inconclusive speculation.''}
\end{quote}

Despite our present conception of a ferocious earth rotation,
which reaches its peak of  464 m/sec or  1670 km/hour at the equator,
and which may give rise to measurable effects even if the relative motions are assumed to be small,
in his writings Feyerabend never mentioned the contemporary physical situation,
in particular the Coriolis force  and  the Kepler problem.
This seems to be characteristic for the attitude of many philosophers of science,
as Feyerabend himself polemically notes \cite{feyerabend-defense,feyer-81},
\begin{quote}
{ ``$\ldots$ Kuhn encourages people who have no idea why a stone falls to the ground
to talk with assurance about scientific method.
Now I have no objection to incompetence but I do object when incompetence is accompanied by boredom
and self-righteousness.
And this is exactly what happens. $\ldots$''}
\end{quote}
When one reads these strong words,
written in an intellectual climate of the seventies of the past century,
one has little doubt that the boldness and self-esteem of such statements
provoked antagonism.

Coming back to Tower of Pisa example,
some model calculation were done by Martina Jedinger  and   Iva Brezinova here in Vienna,
yielding a latitudinal shift of 9.6 cm towards South and a longitudinal shift of
0.6 cm towards East. Intuitively, the large latitudinal shift could be understood
by considering that (air resistance left aside), the falling body remains in a plane
spanned by the direction of the gravity pull towards the center of the earth,
and by the direction of velocity at its release point.
At the same time, the earth, and with it the foot of the tower, revolves around an axis
which is currently tilted at 23.5$^{\circ}$ with respect to the ecliptic axis,
the line drawn from the center of the earth and perpendicular to the ecliptic plane;
a configuration depicted in  Fig.~\ref{f-2004-f1}.
\begin{figure}
\begin{center}
%TexCad Options
%\grade{\off}
%\emlines{\off}
%\beziermacro{\on}
%\reduce{\on}
%\snapping{\off}
%\quality{2.00}
%\graddiff{0.01}
%\snapasp{1}
%\zoom{1.00}
\unitlength 0.30mm
\linethickness{0.4pt}
\begin{picture}(100.00,111.67)
%\circle(50.00,50.00){100.00}
\multiput(50.00,100.00)(1.60,-0.10){4}{\line(1,0){1.60}}
\multiput(56.39,99.59)(0.57,-0.11){11}{\line(1,0){0.57}}
\multiput(62.68,98.36)(0.36,-0.12){17}{\line(1,0){0.36}}
\multiput(68.76,96.35)(0.24,-0.12){24}{\line(1,0){0.24}}
\multiput(74.54,93.57)(0.18,-0.12){30}{\line(1,0){0.18}}
\multiput(79.91,90.07)(0.14,-0.12){35}{\line(1,0){0.14}}
\multiput(84.78,85.92)(0.12,-0.13){36}{\line(0,-1){0.13}}
\multiput(89.09,81.17)(0.12,-0.17){31}{\line(0,-1){0.17}}
\multiput(92.76,75.92)(0.12,-0.23){25}{\line(0,-1){0.23}}
\multiput(95.72,70.24)(0.12,-0.32){19}{\line(0,-1){0.32}}
\multiput(97.93,64.23)(0.12,-0.52){12}{\line(0,-1){0.52}}
\multiput(99.36,57.98)(0.10,-1.06){6}{\line(0,-1){1.06}}
\multiput(99.97,51.60)(-0.10,-3.20){2}{\line(0,-1){3.20}}
\multiput(99.77,45.20)(-0.11,-0.70){9}{\line(0,-1){0.70}}
\multiput(98.75,38.87)(-0.11,-0.38){16}{\line(0,-1){0.38}}
\multiput(96.92,32.73)(-0.12,-0.27){22}{\line(0,-1){0.27}}
\multiput(94.33,26.87)(-0.12,-0.20){28}{\line(0,-1){0.20}}
\multiput(91.01,21.39)(-0.12,-0.15){34}{\line(0,-1){0.15}}
\multiput(87.01,16.38)(-0.12,-0.12){38}{\line(-1,0){0.12}}
\multiput(82.41,11.93)(-0.16,-0.12){32}{\line(-1,0){0.16}}
\multiput(77.28,8.10)(-0.21,-0.12){27}{\line(-1,0){0.21}}
\multiput(71.69,4.95)(-0.28,-0.11){21}{\line(-1,0){0.28}}
\multiput(65.76,2.55)(-0.44,-0.12){14}{\line(-1,0){0.44}}
\multiput(59.56,0.92)(-0.91,-0.12){7}{\line(-1,0){0.91}}
\put(53.20,0.10){\line(-1,0){6.41}}
\multiput(46.80,0.10)(-0.91,0.12){7}{\line(-1,0){0.91}}
\multiput(40.44,0.92)(-0.44,0.12){14}{\line(-1,0){0.44}}
\multiput(34.24,2.55)(-0.28,0.11){21}{\line(-1,0){0.28}}
\multiput(28.31,4.95)(-0.21,0.12){27}{\line(-1,0){0.21}}
\multiput(22.72,8.10)(-0.16,0.12){32}{\line(-1,0){0.16}}
\multiput(17.59,11.93)(-0.12,0.12){38}{\line(-1,0){0.12}}
\multiput(12.99,16.38)(-0.12,0.15){34}{\line(0,1){0.15}}
\multiput(8.99,21.39)(-0.12,0.20){28}{\line(0,1){0.20}}
\multiput(5.67,26.87)(-0.12,0.27){22}{\line(0,1){0.27}}
\multiput(3.08,32.73)(-0.11,0.38){16}{\line(0,1){0.38}}
\multiput(1.25,38.87)(-0.11,0.70){9}{\line(0,1){0.70}}
\multiput(0.23,45.20)(-0.10,3.20){2}{\line(0,1){3.20}}
\multiput(0.03,51.60)(0.10,1.06){6}{\line(0,1){1.06}}
\multiput(0.64,57.98)(0.12,0.52){12}{\line(0,1){0.52}}
\multiput(2.07,64.23)(0.12,0.32){19}{\line(0,1){0.32}}
\multiput(4.28,70.24)(0.12,0.23){25}{\line(0,1){0.23}}
\multiput(7.24,75.92)(0.12,0.17){31}{\line(0,1){0.17}}
\multiput(10.91,81.17)(0.12,0.13){36}{\line(0,1){0.13}}
\multiput(15.22,85.92)(0.14,0.12){35}{\line(1,0){0.14}}
\multiput(20.09,90.07)(0.18,0.12){30}{\line(1,0){0.18}}
\multiput(25.46,93.57)(0.24,0.12){24}{\line(1,0){0.24}}
\multiput(31.24,96.35)(0.36,0.12){17}{\line(1,0){0.36}}
\multiput(37.32,98.36)(0.91,0.12){14}{\line(1,0){0.91}}
%\end
\bezier{32}(31.00,106.66)(28.33,103.33)(32.33,104.00)
\bezier{20}(32.33,104.00)(35.00,104.33)(36.66,105.66)
\bezier{20}(36.66,105.66)(38.33,107.66)(36.33,108.33)
\put(5.33,2.33){}
\bezier{188}(1.33,61.33)(-4.67,52.67)(32.00,51.67)
\bezier{140}(32.00,51.67)(53.00,53.00)(65.00,60.67)
\bezier{168}(65.00,60.67)(92.67,77.33)(86.00,84.67)
\put(60.33,58.00){\vector(2,1){31.67}}
\put(31.67,111.67){\line(1,-3){37.22}}
\put(36.00,108.67){\vector(-4,1){1.33}}
\end{picture}
\end{center}
\caption{Direction of inertial motion of an object released from a point close to the earth's surface.
\label{f-2004-f1}}
\end{figure}



In principle, such a setup could even measure the configuration of distant masses by Mach's principle.
Recall that, according to Einstein's perception of Mach,
the inertial motion of a body should be determined in relation to all
other bodies in the universe; in short, ``matter there governs inertia here.''
%[e.g., http://www.bun.kyoto-u.ac.jp/$\widetilde$suchii/mach.pr.html]
As the earth's gravity pull is known and the shift of falling bodies is measurable,
a reverse computation could yield the inertial motion the distant masses measurable by falling bodies.
But this is beyond the scope of this little review.



\section{Quantum mechanics}

Feyerabend wrote several contributions to the foundational debate
in quantum mechanics.
They are quite detailed and reflect the ongoing debate at the time they were written,
but I failed to find new aspects in them which had a lasting impact on the community.
At least Feyerabend was cautious enough not to state any erroneous claims as Popper.


\subsection{Feyerabend's writings on quantum mechanics}


The first volume of the {\em Philosophical Papers} \cite{fey-philpapers1}
contains the following five manuscripts on quantum mechanics in consecutive order:
{\em On the quantum theory of measurement} \cite{fey-papers1-measure},
{\em Professor {B}ohm's philosophy of nature} \cite{fey-papers1-bohm},
{\em {R}eichenbach's interpretation of quantum mechanics} \cite{fey-papers1-Reichen},
{\em {N}iels {B}ohr's world view} \cite{fey-papers1-Bohr},
and
{\em Hidden variables and the argument of {E}instein, {P}odolsky and {R}osen} \cite{fey-papers1-EPR}.


In {\em On the quantum theory of measurement} \cite{fey-papers1-measure},
Feyerabend attempted a reconciliation between the two types of time evolutions
in quantum mechanics:
the unitary, reversible evolution of the state
in-between measurements,
and the irreversible ``reduction of the wave--packet,''
or ``collapse of the wave function'' --- if such notions are appropriate ---
in a classical measurement device, producing for instance a click in a particle detector.
Presently, the situation regarding this issue seems as unsettled as ever,
despite some dramatic empirical developments through single quantum experiments;
in particular the reconstruction of quantum states after (reversible) ``measurements'' such as
the quantum ``eraser'' experiments (e.g., Refs.~\cite{hkwz,greenberger2}).

{\em Professor {B}ohm's philosophy of nature} \cite{fey-papers1-bohm}
is a critical evaluation of Bohm's theory of hidden parameters.
Feyerabend expresses his mixed feeling of the book: on the one hand,
its approach is fresh and original, on the other hand Feyerabend is reluctant to
abandon the traditional Copenhagen interpretation of quantum mechanics.

{\em {N}iels {B}ohr's world view} \cite{fey-papers1-Bohr} starts with
a refutation of an erroneous claim by Popper to have falsified the Copenhagen interpretation
(see also Ref.~\cite{2002-peres}).
It is also an almost heroic monumental effort to understand that interpretation and its
alleged creator, Bohr.
The paper contains 101 references.

In {\em Hidden variables and the argument of {E}instein, {P}odolsky and {R}osen} \cite{fey-papers1-EPR},
Feyerabend reviewes the paper by these three authors \cite{epr}.
Its first sentence appears to be slightly misleading, at least to me:
\begin{quote}
{ ``Opponents of Bohr's interpretation often refer to an argument by
{E}instein, {P}odolsky and {R}osen, (EPR) according to which,
the formalism of wave mechanics is such that it demands the existence of
exact simultaneous values of non-commuting observables.''}
\end{quote}
It is the particular physical setup using two correlated particles
which allows the measurement of two non-commuting observables on two particles,
one observable per particle.
Through counterfactual inference,
this property is then ascribed to the partner particle as well, and {\em vice versa}.
In that counterfactual way, one may maintain to ``measure'' two observables
which are non-commuting and thus non-co-measurable quantum mechanically.
So, {E}instein, {P}odolsky and {R}osen claim, quantum mechanics is incomplete, since one
can measure more than this theory is able to predict.
Feyerabend then proceeds to derive consequences of such a hypothetical more complete
theory, allowing ``superstates'' by hidden parameters.

Feyerabend's general approach in this debate seems to be dominated by an antagonism
against Popper.
As Popper favors realism and argues against Bohr's Copenhagen interpretation,
Feyerabend objects and argues in Bohr's favor;
although cautiously and with many reservations.
There seemed to have been even a mini-foundational debate between philosophers
of science going on, which developed in parallel to
the physical debate, and which was almost totally neglected by the physicists.
At least for me, this debate seems to have lead nowhere.
But Feyerabend is here in good company with very many physicists and laymen alike.



\subsection{Philosophers at a loss to understand the new physics}


Recall Feyerabend's statement cited above on people who have no idea why a stone falls to the ground
talking  with assurance about scientific method; where incompetence is accompanied by boredom
and self-righteousness.
These are very harsh,
critical words which in my opinion characterize Feyerabends (self-) provoking
stile.
They are, I think, not entirely unjust.
Indeed, their the main premise in my opinion {\em is} correct:
most philosophers nowadays are at a complete loss
of understanding the more recent developments in physics.
With {\em philosophers} I mean everybody with an academic degree after
a study mainly concentrating on philosophy, as compared to the natural sciences.

There are great exceptions to the rule, but these are rare and sparse.
I certainly do not want to contribute to the ridiculous debate of the natural sciences with the rest of the faculties,
sparked by Sokal \cite{sokal-aff},
as I certainly do not want to argue that the natural sciences are immune to fraud,
misbehavior, stupidity and deception.
All I want to say is that any philosophy of science will be misleading
without a proper education in and knowledge of the  subject.
This is particularly true for the philosophy of science.
So, I am afraid, I have to urge philosophers and students of philosophy of science
to study mathematics, physics, logic, chemistry, biology and computer science proper.
At least the mastering of one of these subjects is necessary
in order to be able to comprehend, more so to contribute, to the ongoing debates in these areas.


In the meantime, physicists like myself will go wild and usurp territories
which would be better covered by
the philosophers, as they have much more background in the historical debates and are less inclined
to state ridiculously naive claims on foundational questions such as reality and metaphysics.
We desperately need philosophy after all, as we desperately need philosophers!
But we need to educate them better in the sciences, if they wish to consider science.
And please do not confuse attempts to brainwash people into science proper with concerns of competence.

From these very general remarks, let me now come back to quantum physics, which still remains
a very active research area.
Almost since its introduction in 1900 it has been the subject of intense philosophical debates,
both within the physics community ---
at that time in central Europe, the physicists, due to the good old Humboldt type curriculum,
were much better trained  in classical philosophy ---  and among philosophers of science.
Also Feyerabend contributed to this debate, as already mentioned.
If one is not willing to digest the volumes of Jammer
\cite{jammer:66,jammer1,jammer-92},
or the collection of original articles by Wheeler and Zurek \cite{wheeler-Zurek:83},
one gets a good glimpse of what was and still is going on from Schr\"odinger's
series of three articles on {\em ``Die gegenw{\"{a}}rtige {S}ituation in der {Q}uantenmechanik''} \cite{schrodinger}
(English translation {\em ``The Present Situation In Quantum Mechanics''} \cite[pp. 152-167]{wheeler-Zurek:83}).
I think that I can safely say that, although { ``nobody understands quantum mechanics''}
(cf. Richard Feynman in Ref.~\cite{feynman-law}, p. 129),
nobody not able to comprehend these Schr\"odinger articles should make a public appearance
on related topics.


\subsection{Old topics in a new terminology: scholasticism, empiricism, realism--idealism}

The debate on the foundations of quantum mechanics has taken a somewhat unexpected turn,
in particular in recent years, when, due to new experimental techniques, single quanta
could be investigated:
it turned into controversies with associated long-lasting debates and  huge philosophical records.
At the same time, some of the old concepts became formalized.
In what follows, I take up the task of reviewing some of these issues,
being well aware of the risk of being dilettantish.

One of the big issues is related to the realism {\it versus} idealism debate.
Stace \cite{stace} characterized realism by the
supposition that
{ ``some entities sometimes exist without being experienced by any finite mind.''}
He was an outspoken idealist, claiming that
\begin{quote}
{ ``$\ldots$
we have not the faintest reason for believing in the existence of
inexperienced entities
$\ldots$
[[Realism]] has been adopted
$\ldots$
solely because it simplifies our view of the universe.''}
\end{quote}

In quantum mechanics, this debate may probably be summed up by the term
{\em Bohr--Einstein debate,}
with Einstein firmly positioned as a realist.
One of the questions concerns the existence of physical properties
even in the absence of their direct physical observation.
Einstein \cite{epr}
suggested to do just that, and effectively accept indirectly inferred counterfactuals
als {\em ``elements of physical reality.''}






A further step was taken by the Swiss mathematician Specker,
who, stimulated by the quantum logic developed by Birkhoff and von Neumann
\cite{birkhoff-36},
pondered about the logic of propositions which are not co-measurable; i.e., not simultaneously
measurable \cite{specker-60}.
Specker related such structures in quantum physics to
  Scholasticism,
in particular to scholastic speculations  about the existence of ``infuturabilities'' or
``counterfactuals.''
The question is whether or not
the omniscience (comprehensive knowledge) of God extends to events which
would have occurred if something  had happened which did not
happen.
If so, could all such events be pasted together to form a consistent whole?


Concerns about co-measurability were inevitable because  quantum mechanics has introduced new features
hitherto hardly heard of in classical physics.
Complementarity is a system property first discovered in quantum theory,
making it impossible to jointly measure two complementary observables; or at least prohibiting
their joint measurement with arbitrary precision.
The associated non-commutative operators and the resulting non-distributive propositional structure
became facts of everyday professional life for theoreticians and experimentalists alike.
Yet, what is hardly noticed even by the specialists is the fact
that complementarity and non-distributivity
not necessarily implies total abandonment of non-classicality:
quasi-classical models such as generalized urn models
\cite{wright:pent,wright} and finite automata (e.g., Chapter 10 of Ref.~\cite{svozil-ql})
can be isomorphically embedded into Boolean algebras
with the help of two-valued probability measures,
which abound for such models  \cite{svozil-2001-eua}.

That such non-co-measurable propositions exist might have come as no surprise
even to the classical mind in retrospect.
Indeed, one could take this as a good example for the fact that our phantasy is not good and wild enough to conceive of
the many available alternative options we have for almost any given situation.



Bell \cite{bell-87} and others took up an idea expressed
in the influential article of Einstein, Podolsky and Rosen \cite{epr}, to which also Feyerabend referred.
The Bell-type inequalities are particular instances of Boole's
{\em ``conditions of possible experience''} \cite{Boole,Boole-62},
which are consistency conditions on joint probabilities,
for specific physical setups.
The   Einstein-Podolsky-Rosen argument
suggests that, although two non-co-measurable properties
(associated with complementary observables)
cannot be directly measured at a single quantum,
one may infer from certain two-quanta states
(satisfying a uniqueness property \cite{svozil-2004-vax})
one property per quantum and subsequently counterfactually
infer the other property from its twin quantum.

In doing so one implicitly assumes that counterfactuals exist:
it is assumed that, if the counterfactually inferred property would have been measured
---
which was not the case
---
it would have come out in the expected way.

On top of that,
the Bell inequalities contain sums of terms (e.g., joint probabilities or expectation values)
which
could only be measured subsequently (or at least in different experimental setups; one at a time),
since they correspond to different parameter settings which, according
to quantum complementarity,
cannot be measured simultaneously.
To the realist assuming that
entities exist without being experienced by any finite mind,
this is no big deal.
Even collecting terms associated with measuring different non-co-measurable setups
and summing them up as if they referred to a single quantum is hardly disturbing.
(This makes possible a criticism put forward recently \cite{Hess&Philipp2002}.)

But the assumption of an ``all out'' omni-realism may be improper in the quantum domain.
Quanta prepared in a specific state in a given experimental context
might simply not be capable to ``know''
their precise states in different contexts \cite{svozil-2003-garda}.

As an analogy, no finite agent such as a computer program can be set up to answer
all conceivable questions  --- it may be at a complete loss at answering some or even most of them.
These kind of restricted capabilities appears quite natural and is not very exciting; certainly not
as ``mind-boggling'' or ``mystical'' as the quantum tales of Bohr.
It is a new form of realism,
one which is based on the assumption that certain things do not have all conceivable properties
we would wish them to have; just a finite number of properties, that is it.



Nevertheless, let me emphasize that the non-classicality of quantum mechanics goes well beyond complementarity.
There is a finite constructive proof of the impossibility
of value definiteness for quantized systems whose description require Hilbert spaces
of dimension higher than two.
It turned out that formally there are
``not enough'' two-valued states to allow a faithful embedding of certain tightly interconnected
finite propositional structures
into any Boolean algebra.
This can be characterized by
non-separable or non-unital sets of two-valued states;
the strongest result being the non-existence of two-valued states today known as the ``Kochen-Specker theorem''
\cite{kochen1}.
In fact, once the propositional structure has been enumerated explicitly, a proof is technically not very demanding
and amounts to a coloring theorem (chromaticity) on these sets.

To get a taste of the type of argument, Fig.~\ref{f-2004-fey2} depicts a quantum propositional structure; i.e., a logic,
with a  non-separable set of two-valued states, such that $P(a) = P(b)$.
It is the graph $\Gamma_3$ of Ref.~\cite{kochen1}
represented by the Greechie orthogonality diagram in Ref.~\cite{svozil-tkadlec}.
\begin{figure}
\begin{center}
%TexCad Options
%\grade{\on}
%\emlines{\off}
%\beziermacro{\off}
%\reduce{\on}
%\snapping{\off}
%\quality{2.00}
%\graddiff{0.01}
%\snapasp{1}
%\zoom{0.50}
\unitlength 0.450mm
\linethickness{0.4pt}
\begin{picture}(190.67,109.67)
%\emline(165.67,19.67)(145.67,39.67)
\multiput(165.67,19.67)(-0.12,0.12){167}{\line(0,1){0.12}}
%\end
%\emline(145.67,39.67)(145.67,79.67)
\put(145.67,39.67){\line(0,1){40.00}}
%\end
%\emline(145.67,79.67)(165.67,99.67)
\multiput(145.67,79.67)(0.12,0.12){167}{\line(0,1){0.12}}
%\end
%\emline(165.67,99.67)(185.67,79.67)
\multiput(165.67,99.67)(0.12,-0.12){167}{\line(1,0){0.12}}
%\end
%\emline(185.67,79.67)(185.67,39.67)
\put(185.67,79.67){\line(0,-1){40.00}}
%\end
%\emline(185.67,39.67)(165.67,19.67)
\multiput(185.67,39.67)(-0.12,-0.12){167}{\line(-1,0){0.12}}
%\end
%\emline(185.34,59.67)(145.67,59.67)
\put(185.34,59.67){\line(-1,0){39.67}}
%\end
\put(185.67,39.67){\circle{2.00}}
\put(185.67,59.67){\circle{2.00}}
\put(185.67,79.67){\circle{2.00}}
\put(145.67,39.67){\circle{2.00}}
\put(145.67,59.67){\circle{2.00}}
\put(145.67,79.67){\circle{2.00}}
\put(165.67,19.67){\circle{2.00}}
\put(165.67,99.67){\circle{2.00}}
%\emline(165.67,19.67)(95.67,59.67)
\multiput(165.67,19.67)(-0.21,0.12){334}{\line(-1,0){0.21}}
%\end
%\emline(95.67,59.67)(165.67,99.67)
\multiput(95.67,59.67)(0.21,0.12){334}{\line(1,0){0.21}}
%\end
\put(95.67,59.67){\circle{2.00}}
\put(95.67,74.67){\makebox(0,0)[cc]{$a_8=a_8'$}}
\put(165.67,109.67){\makebox(0,0)[cc]{$b=a_9=a_0'$}}
\put(140.67,79.67){\makebox(0,0)[cc]{$a_2'$}}
\put(140.67,59.67){\makebox(0,0)[cc]{$a_6'$}}
\put(140.67,39.67){\makebox(0,0)[cc]{$a_4'$}}
\put(190.67,39.67){\makebox(0,0)[cc]{$a_3'$}}
\put(190.67,59.67){\makebox(0,0)[cc]{$a_5'$}}
\put(190.67,80.00){\makebox(0,0)[cc]{$a_1'$}}
\put(165.67,9.67){\makebox(0,0)[cc]{$a_7'$}}
%\emline(25.00,19.67)(45.00,39.67)
\multiput(25.00,19.67)(0.12,0.12){167}{\line(0,1){0.12}}
%\end
%\emline(45.00,39.67)(45.00,79.67)
\put(45.00,39.67){\line(0,1){40.00}}
%\end
%\emline(45.00,79.67)(25.00,99.67)
\multiput(45.00,79.67)(-0.12,0.12){167}{\line(0,1){0.12}}
%\end
%\emline(25.00,99.67)(5.00,79.67)
\multiput(25.00,99.67)(-0.12,-0.12){167}{\line(-1,0){0.12}}
%\end
%\emline(5.00,79.67)(5.00,39.67)
\put(5.00,79.67){\line(0,-1){40.00}}
%\end
%\emline(5.00,39.67)(25.00,19.67)
\multiput(5.00,39.67)(0.12,-0.12){167}{\line(1,0){0.12}}
%\end
%\emline(5.33,59.67)(45.00,59.67)
\put(5.33,59.67){\line(1,0){39.67}}
%\end
\put(5.00,39.67){\circle{2.00}}
\put(5.00,59.67){\circle{2.00}}
\put(5.00,79.67){\circle{2.00}}
\put(45.00,39.67){\circle{2.00}}
\put(45.00,59.67){\circle{2.00}}
\put(45.00,79.67){\circle{2.00}}
\put(25.00,19.67){\circle{2.00}}
\put(25.00,99.67){\circle{2.00}}
%\emline(25.00,19.67)(95.00,59.67)
\multiput(25.00,19.67)(0.21,0.12){334}{\line(1,0){0.21}}
%\end
%\emline(95.00,59.67)(25.00,99.67)
\multiput(95.00,59.67)(-0.21,0.12){334}{\line(-1,0){0.21}}
%\end
\put(25.00,109.67){\makebox(0,0)[cc]{$a=a_0=a_9'$}}
\put(50.00,79.67){\makebox(0,0)[cc]{$a_2$}}
\put(50.00,59.67){\makebox(0,0)[cc]{$a_6$}}
\put(50.00,39.67){\makebox(0,0)[cc]{$a_4$}}
\put(-0.00,39.67){\makebox(0,0)[cc]{$a_3$}}
\put(-0.00,59.67){\makebox(0,0)[cc]{$a_5$}}
\put(-0.00,80.00){\makebox(0,0)[cc]{$a_1$}}
\put(25.00,9.67){\makebox(0,0)[cc]{$a_7$}}
\end{picture}
\end{center}
\caption{
Greechie (orthogonality) diagram
\cite{greechie:71},  consisting of {\em points} which
symbolize observables (representable by the spans of vectors
in $n$-dimensional Hilbert space).
All points belonging to a context; i.e., to a maximal set of co-measurable observables
(representable as some orthonormal basis of  $n$-dimensional Hilbert space),
are connected by {\em smooth curves}.
Two smooth curves may be crossing in  common {\em link observables}.
In three dimensions, smooth curves and the associated points stand for tripods.
Two valued measures (interpretable as truth assignments)
assign exactly one point per context the value 1; the other elements are 0.
If $P(a=a_0=a_9') = 1$ for any
two-valued probability measure $P$, then $P(a_8) =0$.
Furthermore,
$P(a_7)=0$, since by a similar argument $P(a)=1$ implies $P(a_7)=0$:
for, if $P(a_7)=1$, then $P(a_1)=P(a_2)=P(a_3)=P(a_4)=0$,
resulting in the necessity for $P(a_5)=P(a_6)=1$,
which is contradicting the assumption that there can only be one element
per context which is 1.
Therefore, $P(b=a_9=a_0') = 1$. Symmetry requires that the
reverse implication is also fulfilled, and therefore  $P(b) =
P(a)$ for every two-valued probability measure $P$.
\label{f-2004-fey2}
}
\end{figure}



\section{Physicists at their own: incommensurabilities in experimental philosophy}

\subsection{A ``meaning'' of the quantum?}

So what have the physicists produced when they
were left alone to interpret the ``meaning'' of the formalism?
They have developed a variety of interpretations,
the number of which is probably as great as there are physicists
and almost as great as the number of philosophical concepts of reality.
Let us just mention a couple of these interpretations and some of their creators and devotees:
{
the Copenhagen interpretation (Bohr),
the many-worlds interpretation  (Everett),
Bohm's interpretation,
the consistent histories approach (Griffiths) and finally a
``realistic'' interpretation  (Einstein, De Broglie, Schr\"odinger).}
All of these make no empirical difference.
Yet they serve as a kind of scaffolding \cite{svozil-2002-noiq};
without it science would be reduced to an automated proof technique,
straying without guidance,
devoid of any
idea of how to proceed with (hopefully) progressive research programs
\cite{lakatosch}.

\subsection{Attempts toward a new kind of {\em experimental philosophy}}

Interpretations aside,
it is tempting to speak of a kind of ``experimental philosophy,''
by which ancient questions of philosophy can be decided by empirical techniques;
i.e., in the laboratory.
This, it seems, is just another kind of extreme empiricism.
But beware of incommensurabilities when interpreting the raw data:
terms in different theories do not share the same meaning and thus cannot be directly related and compared.
Stated differently, incommensurability asserts
that there is an inevitable lack of  common theoretical as well as of operational
terms due to conceptual differences.

Already Hertz \cite{hertz-94} spoke of different illusory ``images'' or ``models''
(German ``Bilder'')
of the mechanical laws, being all in a certain correspondence with the sense data.
As examples, he mentioned the concepts of {\em force}, as expressed in Newtonian physics,
opposed to {\em energy}, expressed in the Hamiltonian formalism.
These images are illusions, which can be applied to the particular purpose they were created for,
but otherwise cannot be taken for reality.
In  Hertz's own words (cf. page 1 of  Ref.~\cite{hertz-94}),
\begin{quote}
{ ``we create illusory inner  images, or symbols of the external objects,
such that the consequences of these images are always
the images of  the consequences of the objects represented.
$\ldots$
Besides this, we have no certainty that our conceptions of the objects
have anything else in common than this {\em single} fundamental relation''}
\footnote{``Wir machen uns innere Scheinbilder oder Symbole der \"ausseren Gegenst\"ande,
und zwar machen wir sie von solcher Art, dass die denknotwendigen Folgen der Bilder stets wieder die Bilder
seien von den notwendigen Folgen der abgebildeten Gegenst\"ande.
$\ldots$
In der That wissen wir auch nicht,
ob unsere Vorstellungen von den Dingen mit jenen in irgend etwas anderem \"ubereinstimmen,
als allein in eben jener {\em einen} fundamentalen
Beziehung.''}.
\end{quote}

\subsection{A philosophy from detector clicks?}

The raw data {\it per se} cannot decide ancient questions of philosophy --- how could they?
Dichotonic events such as clicks in a counter decide just one thing:
whether there is or there is not such click.
Thus any interpretation of the raw data inevitably introduces theoretical constructions which
are in no direct correspondence to the empirical basis,
and may change as times and fashions go by.
All we can ever observe as sense data are ultimately discrete events such as detector clicks.
There is no other empirical basis than such clicks; they are all we have in
constructing what we call the world.
During that world construction, conventions and theories enter.
But conventions and theories are neither evident nor eternal.
They just reflect reasonable assumptions.
For example, the operationalist Bridgman,
in a heroic attempt to base physics on empirical grounds alone,
had to introduce more and more theory to make sense of the raw data
\cite{bridgman27,bridgman,bridgman36,bridgman50,bridgman52}.


The foundational debate in quantum mechanics in general,
and the interpretation of the experiments first performed by Aspect, Grangier and Roger
\cite{aspect-81,aspect-82a,aspect-82b} on the violation of Bell-type inequalities in particular,
demonstrate this quite clearly.
The basis of claims of the so-called ``quantum non-locality''
is the occurrence of pairs of clicks of a specific type in some detectors.
Occasionally, a click here and a click there, that is all, isn't it?
For special parameter configurations, there are more or less joint clicks
than can be expected classically.

Quantum theory gives ``meaning'' to these (in) frequent occurrences of pairs of clicks;
but besides quantum theory, several alternative theories could also explain the clicks.
You may not like these theories, since they do not give any new insights
and look suspiciously artificial, but that may not be a good criterion to favor one physical theory
over the other.
Certainly we cannot rule out that,
as time goes by and physics progresses, other theories might give
even better interpretations of the raw data than quantum mechanics.
So, beware of ``experimental philosophies'' which only superficially seem to be corroborated
by the empirical records.

Another example of  incommensurability in physics is the interpretation of
the Michelson Morley interference experiment,
which gave a ``null'' result
for a model of the luminiferous ether drift \cite{mi-mo-1887,Shankland}.
At first, they were interpreted by  FitzGerald \cite{FitzGerald2,bell-92}, Larmor, Lorentz and others
as an indication that the ether may effectively ``shrink'' objects in very much the same way
as we perceive Lorentz contractions nowadays.
Later on, Einstein's theory of special relativity \cite{ein-05} was interpreted as proving that
an ether ``does not exist.''
Einstein himself \cite{einstein-aether}, as well as for instance Dirac \cite{dirac-aether}
and Bell \cite{bell-sr1,bell-92} held much more differentiated opinions on this subject.
So, again, there does not seem to be a unique way of interpreting the raw data of
the Michelson Morley interference experiment \cite{sterret98}.


Let me also mention a somewhat unrelated issue.
Some physicists ``go wild'' and pretend that the transient status of their science reflects
final truth of the world; they
tell fairy tales about the first three minutes of the Universe, short histories of time and what not.
This is good for marketing purposes and sells well.
What they do not seem to acknowledge and the public simply does not want to hear
is the historic aspect of our findings which makes our present knowledge transient
and preliminary.

In this way of thinking, which emphasizes transitions and the continuation of research programs,
I tend to agree with Lakatosch \cite{lakatosch}.
Feyerabend's critique of Lakatosch is that the latter does not offer a methodology.
Yet the same could be said of Feyerabend's methodology
\cite{feyerabend-defense,feyer-81}.
And if openness, or suspended attention as Freud put it, is no methodology, then so be it.



\section{Some personal remarks}

\subsection{Personal encounters}
In the spring semester 1983, I  attended a course of Feyerabend on the philosophy
of science in Berkeley during my stay as a visiting scholar at the Lawrence Berkeley
Laboratory and the University.
Feyerabend made a sad but rebellious impression,
best described by the German word ``unerf\"ullt,'' whose English translation
is ``unrealized, unfulfilled.''
His spirits were strong and he gave a quite good performance.

His audience consisted of about twenty people; maybe half of them students,
the other part devotees and curious listeners.
It was rather obvious from his reactions that he despised the fan club gathered
to listen to the master's voice, but somehow he longed for them, too---very ambivalent,
rather narcissistic
and probably not unusual for prominent people.

After one of his lectures, I approached him and asked him if we could meet.
He responded friendly but not very enthusiastically.
I guess he was not really interested in a very young, naive physicist from Vienna.
What cold I offer him despite boredom?

Alas, I had the impression that he was after the girls.
I guess that if I would have been a pretty girl,
I would have had very good chances of meeting him and have a chat or two.
But again this is one of those counterfactuals I was speaking of before.

Feyerabend himself
\cite{feyerabend-auto}
and also
Fischer \cite{fischer}
spoke about Feyerabend's sex life.
Fischer recalls that Feyerabend told him he was beaten in his face (in German {\em ``Ohrfeige''})
by his parents for putting his hands in his trouser pockets
because they thought that he would masturbate \cite{fischer-04}.
For me it is quite remarkable that he was a womanizer on one side
while on the other hand seemed not to have had a single coitus during his entire life span.
I take this as an indication that there was deep dissatisfaction with this situation,
a malady which was possibly not only caused by his war time injuries,
but by psychic traumata which may have been deeply hidden and never showed up during
his conscious phases in-between dreams and deep sleep.
This may also be the ultimate reason for the kind of ``Unerf\"ulltheit''
I observed, but of course I am wildly speculating here.


\subsection{Science policy almost unimpressed}
There was another issue which Feyerabend seemed to have taken quite light-heartedly;
at least that was my impression:
nobody took him seriously.

Paul Feyerabend had become almost a shooting star of philosophy of science,
an icon of freedom and heresy
to a generation coming of age in the late period of the twentieth century.
Yet he never quite managed to obtain influence
and convince scientific peers, governments and electorates to implement
his recommendations regarding the selection of science funding and the implementation
of science in general.
In the lectures I attended during 1983,
he strongly supported a system of  lay judges for science assessment  and  financing;
in closest analogy to the procedures established in the judicial system.
I did never find so strong commitments to  lay judges in his writings
as I heard them in these lectures.
I do not know and also did not dare to ask whether he was saddened by his lack of influence
in practical terms.
I would suspect that he had such a bad opinion about science mandarins
and politicians that he expected not too much.

There may be evident and straightforward reasons why various funding agencies never seriously
considered to adopt Feyerabend's proposals:
Feyerabend's proposal would introduce an uncontrollable element in the distribution of money.
For not only might quacks receive public funding;
An even more disturbing consequence could be that, by a fairly independent
selection of committee members and lay evaluators,
powerful groups within the scientific community might loose their carefully
crafted and delicately executed influence over the smooth flow of money towards them and their clients.

To some readers this may sound like a silly conspiracy theory.
To these I respond that the matter is not obvious but
quite serious and deserves careful attention of the general tax paying
public, to which this article is not addressed.
Let me just mention
a large-scale study \cite{1981-cole}, in which 150 research projects of
physics, chemistry and economic science were re-examined by the {\it
National Science Foundation.} The results were devastating.
This study showed how strongly the acceptance or refusal of
a research project depends on the choice of the particular reviewer
evaluating that proposal:
\begin{quote}
{ ``An experiment in which 150
proposals submitted to the National Science Foundation were evaluated
independently by a new set of reviewers indicates that getting a
research grant depends to a significant extend on chance.
$\ldots$
the degree of disagreement within the
population of eligible reviewers is such that whether or not a proposal
is funded depends in a large proportion of cases upon which reviewers
happen to be selected for it.''}
\end{quote}
This finding is rather disturbing, as
in many funding agencies, the ``referents'' in charge
of selection of the peer reviewers
are nominated in a rather unaccountable and
certainly not very transparent  manner.

As an attempt to implement Feyerabend's approach to methodology,
I would like to suggest to distribute the funding for research projects in the following manner:
First, make a very rough plausibility check to eliminate
applications which are obviously fraudulent, inconsistent or otherwise impossible,
unlawful or catastrophic.
This could be done by lay judges.
Then, in the first round of money distribution,
choose 10\%
projects for funding
totally at random.
Choose the next 20\%
by a system of lay judges, as suggested by Feyerabend.
Finally, distribute the remaining 70\%
via the conventional peer review process.
After five years, publish the outcome of the projects funded by all three selection categories
and adjust the relative magnitude of these categories accordingly.
The outcomes might be quite amazing.


\subsection{Bureaucratic dangers to science}
There is one development which Feyerabend did not foresee: the growing detrimental dominance
of the administrative bureaucracies over the scientists.
This is administered through an ever increasing net of what appears to be checks and balances,
of scientometric factors and numbers, of frequent evaluations and proposals
and various certification and standardization procedures.
This makes perfect sense for the administration,
whose major task it is to distribute public money smoothly, accountably in terms of records,  and
free of risks.
But this is not seldom opposed to science, and also opposed to the methodological
openness Feyerabend had in mind.
It also is quite frustrating for the researchers which are captives of this treadmill.
The behemoth created by the Sixth Framework Program (FP6)
and the establishment of the European Research Area (ERA)
are such examples, but there are numerous others on local and institutional scales.


\subsection{For creativity and abundance}
Let me finish more conciliatory and stress the heritage of Paul Feyerabend.
To me, of the many wise and weird things he said and wrote, two messages
stand out.
The first one is in the spirit of the
Enlightenment and gets close to what also Kant had in mind:
try on your own, let not others decide what you think;
do not stop where other people, authorities  and  mandarins tell you to halt.
And finally, in his last manuscript, Feyerabend calls upon us
to reach out and conquer the abundance.

\section*{Acknowledgements and disclaimer}
This paper grew out of a seminar held at the Institute of Philosophy of the University of Vienna jointly with Kurt Fischer and Anja Weiberg in the fall semester 2003/04.
I am deeply indebted  Professor Fischer for the discussions and insights he shared with me.

The views expressed here are my personal impressions and do not
in any way reflect the official position of any of the institutions involved.
Misinterpretations and errors are in the sole responsibility of the author.

%\bibliography{svozil}
%\bibliographystyle{osa}
%\bibliographystyle{apsrmp}
%%\bibliographystyle{alpha}
%\bibliographystyle{amsplain}
%%\bibliographystyle{apsrev}



\providecommand{\bysame}{\leavevmode\hbox to3em{\hrulefill}\thinspace}
\providecommand{\MR}{\relax\ifhmode\unskip\space\fi MR }
% \MRhref is called by the amsart/book/proc definition of \MR.
\providecommand{\MRhref}[2]{%
  \href{http://www.ams.org/mathscinet-getitem?mr=#1}{#2}
}
\providecommand{\href}[2]{#2}
\begin{thebibliography}{10}

\bibitem{armitage}
Angus Armitage, \emph{The deviation of falling bodies}, Annals of Science
  \textbf{5} (1941-47), 342--351.

\bibitem{aspect-81}
Alain Aspect, Philippe Grangier, and G{\'{e}}rard Roger, \emph{Experimental
  tests of realistic local theories via {B}ell's theorem}, Physical Review
  Letters \textbf{47} (1981), 460--463.

\bibitem{aspect-82a}
\bysame, \emph{Experimental realization of Einstein-Podolsky-Rosen-Bohm {\it
  gedankenexperiment}: A new violation of {B}ell's inequalities}, Physical
  Review Letters \textbf{49} (1982), 1804--1807.

\bibitem{aspect-82b}
\bysame, \emph{Experimental test of {B}ell's inequalities using time- varying
  analyzers}, Physical Review Letters \textbf{49} (1982), 1804--1807.

\bibitem{bell-sr1}
John~S. Bell, \emph{How to teach special relativity}, Progress in Scientific
  Culture \textbf{1} (1976), no.~2, Reprinted in \cite[pp. 67-80]{bell-87}.

\bibitem{bell-87}
\bysame, \emph{Speakable and unspeakable in quantum mechanics}, Cambridge
  University Press, Cambridge, 1987.

\bibitem{bell-92}
\bysame, \emph{{G}eorge {F}rancis {F}itz{G}erald}, Physics World \textbf{5}
  (1992), no.~9, 31--35, Abridged version by Denis Weaire.

\bibitem{birkhoff-36}
Garrett Birkhoff and John von Neumann, \emph{The logic of quantum mechanics},
  Annals of Mathematics \textbf{37} (1936), no.~4, 823--843.

\bibitem{Boole-62}
George Boole, \emph{On the theory of probabilities}, Philosophical Transactions
  of the Royal Society of London \textbf{152} (1862), 225--252.

\bibitem{Boole}
\bysame, \emph{An investigation of the laws of thought}, Dover edition, New
  York, 1958.

\bibitem{bridgman27}
Percy~W. Bridgman, \emph{The logic of modern physics}, New York, 1927.

\bibitem{bridgman}
\bysame, \emph{A physicist's second reaction to {M}engenlehre}, Scripta
  Mathematica \textbf{2} (1934), 101--117, 224--234, Cf. R. Landauer
  \cite{landauer-95}.

\bibitem{bridgman36}
\bysame, \emph{The nature of physical theory}, Princeton, 1936.

\bibitem{bridgman50}
\bysame, \emph{Reflections of a physicist}, Philosophical Library, New York,
  1950.

\bibitem{bridgman52}
\bysame, \emph{The nature of some of our physical concepts}, Philosophical
  Library, New York, 1952.

\bibitem{dalla-2002}
Maria Luisa~Dalla Chiara and Roberto Giuntini, \emph{Popper and the logic of
  quantum mechanics}, Invited talk at the Karl Popper 2002 Centenary Congress,
  Vienna, July 5th, 2002.

\bibitem{1981-cole}
Stephen Cole, Jonathan~R. Cole, and Gary~A. Simon, \emph{Chance and consensus
  in peer review}, Science \textbf{214} (1981), 881--885.

\bibitem{dirac-aether}
Paul A.~M. Dirac, \emph{Is there an aether?}, Nature \textbf{168} (1951),
  906--907.

\bibitem{dyson-unfash}
Freeman Dyson, \emph{Unfashionable pursuits}, Mathematical Intelligencer
  (1983), 47--54, {D}yson gave related talks; e.g., at {Y}ale, Minnesota and
  {B}onn.

\bibitem{ein-05}
Albert Einstein, \emph{{Z}ur {E}lektrodynamik bewegter {K}{\"{o}}rper}, Annalen
  der Physik \textbf{17} (1905), 891--921, {E}nglish translation in the
  Appendix of \cite{miller-1998}.

\bibitem{einstein-aether}
\bysame, \emph{{\"{A}}ther und {R}elativit{\"{a}}tstheorie. {R}ede gehalten am
  5. {M}ai 1920 an der {R}eichs-{U}niversit{\"{a}}t {L}eiden}, Springer,
  Berlin, 1920.

\bibitem{epr}
Albert Einstein, Boris Podolsky, and Nathan Rosen, \emph{Can quantum-mechanical
  description of physical reality be considered complete?}, Physical Review
  \textbf{47} (1935), 777--780.

\bibitem{feyerabend}
Paul~K. Feyerabend, \emph{Against method}, New Left Books, London, 1974.

\bibitem{feyerabend-defense}
\bysame, \emph{How to defend society against science}, Radical Philosophy
  \textbf{11} (1975), reprinted in \cite{feyer-81}.

\bibitem{fey-papers1-EPR}
\bysame, \emph{Hidden variables and the argument of {E}instein, {P}odolsky and
  {R}osen}, Problems of Empiricism (Philosophical Papers, Vol 2), Cambridge
  University Press, Cambridge, 1981, pp.~298--342.

\bibitem{feyer-81}
\bysame, \emph{How to defend society against science}, Scientific Revolutions
  (Ian Hacking, ed.), Oxford University Press, Oxford, 1981, pp.~156--167.

\bibitem{fey-papers1-Bohr}
\bysame, \emph{{N}iels {B}ohr's world view}, Problems of Empiricism
  (Philosophical Papers, Vol 2), Cambridge University Press, Cambridge, 1981,
  pp.~247--297.

\bibitem{fey-papers1-measure}
\bysame, \emph{On the quantum theory of measurement}, Problems of Empiricism
  (Philosophical Papers, Vol 2), Cambridge University Press, Cambridge, 1981,
  pp.~207--218.

\bibitem{fey-philpapers2}
\bysame, \emph{Problems of empiricism. {P}hilosophical papers, Volume 2},
  Cambridge University Press, Cambridge, 1981.

\bibitem{fey-papers1-bohm}
\bysame, \emph{Professor {B}ohm's philosophy of nature}, Problems of Empiricism
  (Philosophical Papers, Vol 2), Cambridge University Press, Cambridge, 1981,
  pp.~219--235.

\bibitem{fey-philpapers1}
\bysame, \emph{Realism, rationalism and scientific method. {P}hilosophical
  papers, Volume 1}, Cambridge University Press, Cambridge, 1981.

\bibitem{fey-papers1-Reichen}
\bysame, \emph{{R}eichenbach's interpretation of quantum mechanics}, Problems
  of Empiricism (Philosophical Papers, Vol 2), Cambridge University Press,
  Cambridge, 1981, pp.~236--246.

\bibitem{feyerabend-auto}
\bysame, \emph{Killing time}, The University of Chicago Press, Chicago and
  London, 1995.

\bibitem{fey-philpapers3}
Paul~K. Feyerabend and John Preston, \emph{Knowledge, science and relativism.
  {P}hilosophical papers, Volume 3}, Cambridge University Press, Cambridge,
  1999.

\bibitem{feynman-law}
Richard~P. Feynman, \emph{The character of physical law}, MIT Press, Cambridge,
  MA, 1965.

\bibitem{fischer}
Kurt~Rudolf Fischer, \emph{{P}aul {F}eyerabend: A personal reminiscence},
  Aufs{\"{a}}tze zur angloamerikanischen und {\"{o}}sterreichischen Philosophie
  (Kurt~Rudolf Fischer, ed.), Lang, Frankfurt am Main, Wien, 1999, pp.~77--89.

\bibitem{fischer-04}
\bysame, \emph{{F}eyerabend's {W}eltanschauung}, Springer, Vienna, 2004.

\bibitem{FitzGerald2}
George~Francis FitzGerald, The Scientific Writings of the Late {G}eorge
  {F}rancis {F}itzGerald (J.~Larmor, ed.), Dublin, 1902.

\bibitem{greechie:71}
J.~R. Greechie, \emph{Orthomodular lattices admitting no states}, Journal of
  Combinatorial Theory \textbf{10} (1971), 119--132.

\bibitem{greenberger2}
Daniel~B. Greenberger and A.~YaSin, \emph{``{H}aunted'' measurements in quantum
  theory}, Foundation of Physics \textbf{19} (1989), no.~6, 679--704.

\bibitem{hall-1903a}
Edwin~H. Hall, \emph{Do falling bodies move south?}, Phys. Rev. (Series I)
  \textbf{17} (1903), 179--190.

\bibitem{hall-1903b}
\bysame, \emph{Do falling bodies move south?}, Phys. Rev. (Series I)
  \textbf{17} (1903), 245--254.

\bibitem{hertz-94}
Heinrich Hertz, \emph{{P}rinzipien der {M}echanik}, Barth, Leipzig, 1894.

\bibitem{hkwz}
Thomas~J. Herzog, Paul~G. Kwiat, Harald Weinfurter, and Anton Zeilinger,
  \emph{Complementarity and the quantum eraser}, Physical Review Letters
  \textbf{75} (1995), no.~17, 3034--3037.

\bibitem{Hess&Philipp2002}
K.~Hess and W.~Philipp, \emph{Exclusion of time in the theorem of {B}ell},
  Europhysics Letters \textbf{57} (2002), no.~6, 775--781.

\bibitem{hooker}
Clifford~Alan Hooker, The Logico-Algebraic Approach to Quantum Mechanics.
  Volume I: Historical Evolution, Reidel, Dordrecht, 1975.

\bibitem{jammer:66}
Max Jammer, \emph{The conceptual development of quantum mechanics}, McGraw-Hill
  Book Company, New York, 1966.

\bibitem{jammer1}
\bysame, \emph{The philosophy of quantum mechanics}, John Wiley \& Sons, New
  York, 1974.

\bibitem{jammer-92}
\bysame, \emph{John {S}teward {B}ell and the debate on the significance of his
  contributions to the foundations of quantum mechanics}, Bell's Theorem and
  the Foundations of Modern Physics (A.~van~der Merwe, F.~Selleri, and
  G.~Tarozzi, eds.), World Scientific, Singapore, 1992, pp.~1--23.

\bibitem{kochen1}
Simon Kochen and Ernst~P. Specker, \emph{The problem of hidden variables in
  quantum mechanics}, Journal of Mathematics and Mechanics \textbf{17} (1967),
  no.~1, 59--87, Reprinted in \cite[pp. 235--263]{specker-ges}.

\bibitem{lakatosch}
Imre Lakatos, \emph{Philosophical papers. 1.~the methodology of scientific
  research programmes}, Cambridge University Press, Cambridge, 1978.

\bibitem{landauer-95}
Rolf Landauer, \emph{Advertisement for a paper {I} like}, On Limits (John~L.
  Casti and J.~F. Traub, eds.), Santa Fe Institute Report 94-10-056, Santa Fe,
  NM, 1994, p.~39.

\bibitem{mi-mo-1887}
A.~A. Michelson and E.~W. Morley, \emph{On the relative motion of the earth and
  the luminiferous ether}, Amer. J. Sci. \textbf{34} (1887), 333--345.

\bibitem{miller-1998}
Arthur~I. Miller, \emph{{A}lbert {E}instein's special theory of relativity},
  Springer, New York, 1998.

\bibitem{Oberheim-97}
Eric Oberheim, \emph{Bibliographie {P}aul {F}eyerabends}, Journal for General
  Philosophy of Science \textbf{28} (1997), 211--234, An earlier german version
  of The Works of Paul Feyerabend.

\bibitem{Oberheim-99}
\bysame, \emph{The works of {P}aul {F}eyerabend}, Knowledge, Science and
  Relativism. {P}hilosophical Papers, Volume 3 (John Preston, ed.), Cambridge
  University Press, Cambridge, 1999, An improved and updated bibliography of
  the works of Paul Feyerabend., pp.~227--251.

\bibitem{2002-peres}
Asher Peres, \emph{{K}arl {P}opper and the {C}openhagen interpretation}, Stud.
  History and Philos. of Modern Physics \textbf{33} (2002), 23--34.

\bibitem{schrodinger}
Erwin Schr{\"{o}}dinger, \emph{Die gegenw{\"{a}}rtige {S}ituation in der
  {Q}uantenmechanik}, Naturwissenschaften \textbf{23} (1935), 807--812,
  823--828, 844--849, {E}nglish translation in \cite{trimmer} and \cite[pp.
  152-167]{wheeler-Zurek:83}; http://www.emr.hibu.no/lars/eng/cat/.

\bibitem{Shankland}
R.~S. Shankland, S.~W. McCuskey, F.~C. Leone, and G.~Kuerti, \emph{New analysis
  of the interferometer observations of {D}ayton {C}. {M}iller}, Rev. Mod.
  Phys. \textbf{27} (1955), 167--178.

\bibitem{sokal-aff}
Alan Sokal, http://www.physics.nyu.edu/faculty/sokal/.

\bibitem{specker-60}
Ernst Specker, \emph{{D}ie {L}ogik nicht gleichzeitig entscheidbarer
  {A}ussagen}, Dialectica \textbf{14} (1960), 175--182, Reprinted in \cite[pp.
  175--182]{specker-ges}; {E}nglish translation: {\it The logic of propositions
  which are not simultaneously decidable}, reprinted in \cite[pp.
  135-140]{hooker}.

\bibitem{specker-ges}
\bysame, \emph{Selecta}, Birkh{\"{a}}user Verlag, Basel, 1990.

\bibitem{stace}
Walter~Terence Stace, \emph{The refutation of realism}, Readings in
  philosophical analysis (Herbert Feigl and Wilfrid Sellars, eds.),
  Appleton--Century--Crofts, New York, 1949.

\bibitem{sterret98}
Susan~G. Sterrett, \emph{Sounds like light: {E}instein's special theory of
  relativity and {M}ach's work in acoustics and aerodynamics}, Studies In
  History and Philosophy of Science Part B: Studies In History and Philosophy
  of Modern Physics \textbf{29} (1998), 1--35.

\bibitem{svo-83}
Karl Svozil, \emph{On the setting of scales for space and time in arbitrary
  quantized media}, Lawrence Berkeley Laboratory preprint \textbf{LBL-16097}
  (1983), http://heplibw3.slac.stanford.edu/spires/find/hep?key=1089510 a pdf
  scan is at URL http://ccdb3fs.kek.jp/cgi-bin/img/allpdf?198309187.

\bibitem{svo5}
\bysame, \emph{Connections between deviations from {L}orentz transformation and
  relativistic energy-momentum relation}, Europhysics Letters \textbf{2}
  (1986), 83--85, excerpts from \cite{svo-83}.

\bibitem{svozil-93}
\bysame, \emph{Randomness \& undecidability in physics}, World Scientific,
  Singapore, 1993.

\bibitem{svozil-nat-acad}
\bysame, \emph{A constructivist manifesto for the physical sciences}, The
  Foundational Debate, Complexity and Constructivity in Mathematics and Physics
  (Dordrecht, Boston, London) (Werner~DePauli Schimanovich, Eckehart
  K{\"{o}}hler, and Friedrich Stadler, eds.), Kluwer, 1995, Cf.
  \cite{svozil-complexity:95}, pp.~65--88.

\bibitem{svozil-complexity:95}
\bysame, \emph{How real are virtual realities, how virtual is reality? {T}he
  constructive re-interpretation of physical undecidability}, Complexity
  \textbf{1} (1996), no.~4, 43--54.

\bibitem{svozil-ql}
\bysame, \emph{Quantum logic}, Springer, Singapore, 1998.

\bibitem{svozil-relrel}
\bysame, \emph{Relativizing relativity}, Foundations of Physics \textbf{30}
  (2000), no.~7, 1001--1016, e-print {\tt arXiv:quant-ph/0001064}.

\bibitem{svozil-2002-popper}
\bysame, \emph{The dangerous misconceptions of {S}ir {K}arl {R}aimund
  {P}opper}, 2002.

\bibitem{svozil-2001-eua}
\bysame, \emph{Logical equivalence between generalized urn models and finite
  automata}, 2002.

\bibitem{svozil-2002-noiq}
\bysame, \emph{What could be more practical than a good interpretation?}, 2002.

\bibitem{svozil-2004-vax}
\bysame, \emph{On counterfactuals and contextuality}, 2004.

\bibitem{svozil-2003-garda}
\bysame, \emph{Quantum information via state partitions and the context
  translation principle}, Journal of Modern Optics \textbf{51} (2004),
  811--819.

\bibitem{svozil-tkadlec}
Karl Svozil and Josef Tkadlec, \emph{Greechie diagrams, nonexistence of
  measures in quantum logics and {K}ochen--{S}pecker type constructions},
  Journal of Mathematical Physics \textbf{37} (1996), no.~11, 5380--5401.

\bibitem{theo-psi-1987}
T.~Theocharis and M.~Psimoloulos, \emph{Where science has gone wrong}, Nature
  \textbf{329} (1987), 595--598.

\bibitem{trimmer}
J.~D. Trimmer, \emph{The present situation in quantum mechanics: a translation
  of {S}chr{\"{o}}dinger's ``cat paradox''}, Proc. Am. Phil. Soc. \textbf{124}
  (1980), 323--338, Reprinted in \cite[pp. 152-167]{wheeler-Zurek:83}.

\bibitem{wheeler-Zurek:83}
John~Archibald Wheeler and Wojciech~Hubert Zurek, \emph{Quantum theory and
  measurement}, Princeton University Press, Princeton, 1983.

\bibitem{wright:pent}
Ron Wright, \emph{The state of the pentagon. {A} nonclassical example},
  Mathematical Foundations of Quantum Theory (A.~R. Marlow, ed.), Academic
  Press, New York, 1978, pp.~255--274.

\bibitem{wright}
\bysame, \emph{Generalized urn models}, Foundations of Physics \textbf{20}
  (1990), 881--903.

\end{thebibliography}


\end{document}


