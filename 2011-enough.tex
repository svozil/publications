\documentclass[%
 reprint,
 %superscriptaddress,
 %groupedaddress,
 %unsortedaddress,
 %runinaddress,
 %frontmatterverbose,
 %preprint,
 showpacs,
 showkeys,
 preprintnumbers,
 %nofootinbib,
 %nobibnotes,
 %bibnotes,
 amsmath,amssymb,
 aps,
 prl,
 % pra,
 % prb,
 % rmp,
 %prstab,
 %prstper,
  longbibliography,
 %floatfix,
 %lengthcheck,%
 ]{revtex4-1}

\usepackage[breaklinks=true,colorlinks=true,anchorcolor=blue,citecolor=blue,filecolor=blue,menucolor=blue,pagecolor=blue,urlcolor=blue,linkcolor=blue]{hyperref}
\usepackage{graphicx}% Include figure files
\usepackage{xcolor}

 \begin{document}

\title{How much contextuality?}


\author{Karl Svozil}
\affiliation{Institute of Theoretical Physics, Vienna
    University of Technology, Wiedner Hauptstra\ss e 8-10/136, A-1040
    Vienna, Austria}
\email{svozil@tuwien.ac.at} \homepage[]{http://tph.tuwien.ac.at/~svozil}


\date{\today}

\begin{abstract}
The amount of contextuality is quantified in terms of the probability of the necessary violations of noncontextual assignments to counterfactual elements of physical reality.
\end{abstract}

\pacs{03.65.Ta, 03.65.Ud}
\keywords{quantum  measurement theory, quantum contextuality, counterfactual observables}
%\preprint{CDMTCS preprint nr. 372/2009}
\maketitle

Some of the mind boggling features attributed to quantized systems are their alleged ability
to counterfactually~\citep{svozil-2006-omni,vaidman:2009} respond to complementary queries~\cite{epr,clauser},
as well as their capacity to experimentally render outcomes
which have not been encoded into them prior to measurement~\cite{zeil-99}.
Moreover, under certain ``reasonable'' assumptions, and by excluding various exotic quasi-classical possibilities~\cite{pitowsky-82,meyer:99},
quantum mechanics appears to ``outperform'' classical correlations by allowing
higher-that-classical coincidences of certain events,
reflected by violations of Boole-Bell type constraints on classical probabilities~\cite{Boole-62,froissart-81,pitowsky}.
One of the unresolved issues is the reason (beyond geometric and formal arguments)
for the quantitative form of these violations~\cite{cirelson,filipp-svo-04-qpoly-prl};
in particular, why Nature should not allow higher-than-quantum or maximal
violations~\cite{pop-rohr,svozil-krenn} of
Boole's conditions of possible experience~\cite[p.~229]{Boole-62}.

The Kochen-Specker theorem~\cite{kochen1}, stating the impossibility of a
consistent truth assignment to  potential outcomes of  even a finite number of certain
interlinked complementary observables,
gave further indication for the absence of classical simultaneous omniscience in the quantum domain.
From a purely operational point of view, the quantitative
predictions that result from
Bell- as well as Kochen-Specker-type
theorems present a advancement over quantum complementarity.
But they do not explicitly indicate the conceivable interpretation of these findings;
at least not on the phenomenologic level.
Thus the resulting explanations, although sufficient and conceptually desirable and gratifying,
lack the necessity.


One possibility to interpret these findings, and the prevalent one among physicists, is in terms of contextuality.
Contextuality can be motivated by the benefits of a quasi-classical analysis.
In particular, omniscience appears to be corroborated by
the feasability of the potential measurements involved:
it is thereby implicitly assumed that all potentially observable elements of physical reality~\cite{epr} exist prior to any measurement; albeit
any such (potential) measurement outcome (the entirety of which could thus consistently pre-exist before the actual measurement)
depends on whatever other observables (the context) are co-measured alongside~\cite{bohr-1949,bell-66}.
As, contrary to a very general interpretation of that assumption, the quantum mechanical observables are represented context independently,
any such contextual behavior should be restricted to {\em single} quanta and outcomes within the quantum statistical bounds.
This, in essence, is quantum realism in disguise.
Nevertheless, it requires very little modifications --
indeed, none on the statistical level, and some on the level of individual outcomes as described below --
both of the quantum as well as of the classical representations.

Einstein-Podolsky-Rosen type experiments~\cite{epr}
for entangled higher than two-dimensional quantized systems seem to indicate that
contextuality, if viable, will remain hidden to any direct physical operationalization
(and thus might be criticized to be metaphysical)
even if counterfactual measurements are allowed~\cite{svozil:040102}.
Because ``the
immense majority of the experimental violations of Bell
inequalities does not prove quantum nonlocality, but just
quantum contextuality''~\cite{cabello:210401},
current claims of proofs of noncontextuality are solely based on violations of classical constraints in Boole-Bell-type,
Kochen-Specker-type, or Greenbergerger-Horne-Zeilinger-type  configurations.

Nevertheless, insistence on the simultaneous physical contextual coexistence
of certain finite sets of counterfactual observables
necessarily results in ``ambivalent'' truth assignments which could be explicitly illustrated by
a {\em forced} tabulation \cite{peres222,svozil_2010-pc09}
of contextual truth values for Boole-Bell-type or Kochen-Specker-type configurations.
Here contextual means that the truth value of a particular quantum observable depends
on whatever other observables are measured alongside this particular observable.
Any forced tabulation of truth values would render occurrences of mutually contradicting,
potential, counterfactual
outcomes of one and the same observable, depending on the measurement context~\cite{svozil-2008-ql}.
The amount of this violation of noncontextuality can be quantified by the frequency of occurrence of contextuality.
In what follows these frequencies will be calculated for a number of experimental configurations suggested in the literature.

First, consider the generalized Clauser-Horne-Shimony-Holt (CHSH) inequality
\begin{equation}
-\lambda \le E(a,b)+E(a,b')+E(a',b)-E(a',b') \le \lambda
\label{2011-enough-e1}
\end{equation}
which, for $\lambda =2$ and $\lambda =2\sqrt{2}$,
represents bounds for classical~\cite{chsh,clauser} and quantum~\cite{cirelson:80}
expectations of dichotomic observables with outcomes ``$-1$'' and ``$+1$,'' respectively.
The algebraic maximal violation associated with $\lambda =4$
is attainable only for hypothetical ``nonlocal boxes''~\cite{pop-rohr,svozil-krenn,popescu-97,PhysRevA.71.022101}
or by bit exchange~\cite{svozil-2004-brainteaser}.

Eq.~(\ref{2011-enough-e1}) can be rewritten in an explicitly contextual form by the substitution
\begin{equation}
E(x,y)
\mapsto
E(x_y,y_x)
,
\label{2011-enough-e2}
\end{equation}
where $x_y$ stands for ``observable $x$ measured alongside observable $y$''
\cite{svozil_2010-pc09}.
Contextuality manifests itself through $x_y\neq x_{y'}$.
Because in the particular CHSH configuration there are no other observables measured alongside the ones
that appear already in Eq.~(\ref{2011-enough-e1}), this form is without ambiguity.

Eq.~(\ref{2011-enough-e1}) refers to the expectation values for four complementary measurement configurations on the same particles (two particles
and two measurement configurations per particle).
These expectation values can in principle be computed from the statistical average of
the individual two-particle contributions.
This requires that all of them exist counterfactually
-- a requirement that, at least according to the contextuality assumption, is satisfied --
because only one of the four  configurations can actually be simultaneosly measurable;
the other three have to be assigned in a consistent manner
and contribute to the expectation values  $E(a,b)=(1/N)\sum_{i=1}^N a_ib_i$.
Here, $a_i$ and $b_i$  stand for the outcomes of the dichotomic observables $a$ and $b$
in the $i$th experiment; $N$ is the number of individual experiments.
Suppose we are interested in individual outcomes contributing to a violation of  Eq.~(\ref{2011-enough-e1}).
For the sake of simplicity, suppose further that one would like to force the algebraic maximum of $\lambda = 4$
upon  Eq.~(\ref{2011-enough-e1}), and suppose that only one observable, say $b'$,
is contextual (a highly counterintuitive assumption).
Then one obtains, for individual outcomes, say, in the $i$th experiment,
\begin{equation}
(\pm 1) (\pm 1)
+
(\pm 1) x
+
(\pm 1) (\pm 1)
-
(\pm 1) (-x)
=
4,
\label{2011-enough-e3}
\end{equation}
and thus $x=\pm 1$.
That is, the algebaic maximum of $\lambda =4$ can be reached by a single instance
of contextual assignment $b'_a = -b'_{a'}$ per quantum.
Table~\ref{2011-enough-t1} enumerates the two possible truth value assignments
associated with this configuration.
\begin{table}
\begin{center}
\begin{tabular}{ccccccccccccccccccccccccccccccccc}
\hline\hline
$a_b$&$a_{b'}$&$a'_b$&$a'_{b'}$&$b_a$&$b_{a'}$&$b'_a$&$b'_{a'}$\\
\hline
$+1$&$+1$&$+1$&$+1$&$+1$&$+1$&$+1$&$-1 $\\
$-1$&$-1$&$-1$&$-1$&$-1$&$-1$&$-1$&$+1 $\\
\hline
$+1$&$+1$&$+1$&$+1$&$+1$&$+1$&$+1$&$+1 $\\
$-1$&$-1$&$-1$&$-1$&$-1$&$-1$&$-1$&$-1 $\\
\hline\hline
\end{tabular}
\end{center}
\caption{The first two rows represent contextual assignments associated with an algebraic maximal rendition ($\lambda =4$)
of the CHSH inequality.
The third and the fourth assignments are noncontextual.}
\label{2011-enough-t1}
\end{table}

It should be stressed that  there is no unique correspondence between the proportionality of contextuality
and amount of CHSH violation.
Indeed, it can be expected that
there are several possible sets of truth assignments with relative frequencies with differing amounts of contextuality
yielding the same violation.
This plasticity is particularly true for more than one instance of contextuality,
where two or more violations of noncontextuality may compensate each other.
Take, for example,
the four-touple
 $(E(a,b),E(a,b'),E(a',b),E(a',b'))$
of expectation values contained in Eq.~(\ref{2011-enough-e1}),
and its transition
$(+1,+1,+1,-1) \rightarrow (+1,+1,-1,-1)$,
which, for example, can be achieved by changing one instance  of contextuality at $b'$
to two instances of contextuality at $b'$ and $b$,
resulting in $E(a,b)+E(a,b')+E(a',b)-E(a',b') = 4 \rightarrow 2$.


That contextuality could accommodate any bound $0< \lambda <4$
can be demonstrated by  interpreting
all possible noncontextual and contextual assignments, as well as the resulting
corresponding joint expectations enumerated in Table~\ref{2011-enough-t2}
as vertices of a convex correlation polytope.
\begin{table*}
\begin{center}
\begin{tabular}{cccccccc|ccccccccccccccccccccccccc}
\hline\hline
$a_b$&$a_{b'}$&$a'_b$&$a'_{b'}$&$b_a$&$b_{a'}$&$b'_a$&$b'_{a'}$&$a_b b_a$&$a_{b'} b'_a$&$a'_b b_{a'}$&$a'_{b'} b'_{a'}$     \\
\hline
$  -1$&$   -1$&$   -1$&$   -1$&$   -1$&$   -1$&$   -1$&$   -1$&$   +1$&$   +1$&$   +1$&$   +1$\\
$  -1$&$   -1$&$   -1$&$   -1$&$   -1$&$   -1$&{\color{green}$\bf{}-1$}&{\color{red}$\bf{}+1$}&$   +1$&$   +1$&$   +1$&$   -1$\\
$  -1$&$   -1$&$   -1$&$   -1$&$   -1$&$   -1$&{\color{red}$\bf{}+1$}&{\color{green}$\bf{}-1$}&$   +1$&$   -1$&$   +1$&$   +1$\\
$  -1$&$   -1$&$   -1$&$   -1$&$   -1$&$   -1$&$   +1$&$   +1$&$   +1$&$   -1$&$   +1$&$   -1$\\
$  -1$&$   -1$&$   -1$&$   -1$&{\color{green}$\bf{}-1$}&{\color{red}$\bf{}+1$}&$   -1$&$   -1$&$   +1$&$   +1$&$   -1$&$   +1$\\
$  -1$&$   -1$&$   -1$&$   -1$&{\color{green}$\bf{}-1$}&{\color{red}$\bf{}+1$}&{\color{green}$\bf{}-1$}&{\color{red}$\bf{}+1$}&$   +1$&$   +1$&$   -1$&$   -1$\\
$  -1$&$   -1$&$   -1$&$   -1$&{\color{green}$\bf{}-1$}&{\color{red}$\bf{}+1$}&{\color{red}$\bf{}+1$}&{\color{green}$\bf{}-1$}&$   +1$&$   -1$&$   -1$&$   +1$\\
$  -1$&$   -1$&$   -1$&$   -1$&{\color{green}$\bf{}-1$}&{\color{red}$\bf{}+1$}&$   +1$&$   +1$&$   +1$&$   -1$&$   -1$&$   -1$\\
%$  -1$&$   -1$&$   -1$&$   -1$&$   +1$&$   -1$&$   -1$&$   -1$&$   -1$&$   +1$&$   +1$&$   +1$\\
%$  -1$&$   -1$&$   -1$&$   -1$&$   +1$&$   -1$&$   -1$&$   +1$&$   -1$&$   +1$&$   +1$&$   -1$\\
%$  -1$&$   -1$&$   -1$&$   -1$&$   +1$&$   -1$&$   +1$&$   -1$&$   -1$&$   -1$&$   +1$&$   +1$\\
%$  -1$&$   -1$&$   -1$&$   -1$&$   +1$&$   -1$&$   +1$&$   +1$&$   -1$&$   -1$&$   +1$&$   -1$\\
%$  -1$&$   -1$&$   -1$&$   -1$&$   +1$&$   +1$&$   -1$&$   -1$&$   -1$&$   +1$&$   -1$&$   +1$\\
%$  -1$&$   -1$&$   -1$&$   -1$&$   +1$&$   +1$&$   -1$&$   +1$&$   -1$&$   +1$&$   -1$&$   -1$\\
%$  -1$&$   -1$&$   -1$&$   -1$&$   +1$&$   +1$&$   +1$&$   -1$&$   -1$&$   -1$&$   -1$&$   +1$\\
%$  -1$&$   -1$&$   -1$&$   -1$&$   +1$&$   +1$&$   +1$&$   +1$&$   -1$&$   -1$&$   -1$&$   -1$\\
%$  -1$&$   -1$&$   -1$&$   +1$&$   -1$&$   -1$&$   -1$&$   -1$&$   +1$&$   +1$&$   +1$&$   -1$\\
%$  -1$&$   -1$&$   -1$&$   +1$&$   -1$&$   -1$&$   -1$&$   +1$&$   +1$&$   +1$&$   +1$&$   +1$\\
%$  -1$&$   -1$&$   -1$&$   +1$&$   -1$&$   -1$&$   +1$&$   -1$&$   +1$&$   -1$&$   +1$&$   -1$\\
%$  -1$&$   -1$&$   -1$&$   +1$&$   -1$&$   -1$&$   +1$&$   +1$&$   +1$&$   -1$&$   +1$&$   +1$\\
%$  -1$&$   -1$&$   -1$&$   +1$&$   -1$&$   +1$&$   -1$&$   -1$&$   +1$&$   +1$&$   -1$&$   -1$\\
%$  -1$&$   -1$&$   -1$&$   +1$&$   -1$&$   +1$&$   -1$&$   +1$&$   +1$&$   +1$&$   -1$&$   +1$\\
%$  -1$&$   -1$&$   -1$&$   +1$&$   -1$&$   +1$&$   +1$&$   -1$&$   +1$&$   -1$&$   -1$&$   -1$\\
%$  -1$&$   -1$&$   -1$&$   +1$&$   -1$&$   +1$&$   +1$&$   +1$&$   +1$&$   -1$&$   -1$&$   +1$\\
%$  -1$&$   -1$&$   -1$&$   +1$&$   +1$&$   -1$&$   -1$&$   -1$&$   -1$&$   +1$&$   +1$&$   -1$\\
%$  -1$&$   -1$&$   -1$&$   +1$&$   +1$&$   -1$&$   -1$&$   +1$&$   -1$&$   +1$&$   +1$&$   +1$\\
%$  -1$&$   -1$&$   -1$&$   +1$&$   +1$&$   -1$&$   +1$&$   -1$&$   -1$&$   -1$&$   +1$&$   -1$\\
%$  -1$&$   -1$&$   -1$&$   +1$&$   +1$&$   -1$&$   +1$&$   +1$&$   -1$&$   -1$&$   +1$&$   +1$\\
%$  -1$&$   -1$&$   -1$&$   +1$&$   +1$&$   +1$&$   -1$&$   -1$&$   -1$&$   +1$&$   -1$&$   -1$\\
%$  -1$&$   -1$&$   -1$&$   +1$&$   +1$&$   +1$&$   -1$&$   +1$&$   -1$&$   +1$&$   -1$&$   +1$\\
%$  -1$&$   -1$&$   -1$&$   +1$&$   +1$&$   +1$&$   +1$&$   -1$&$   -1$&$   -1$&$   -1$&$   -1$\\
%$  -1$&$   -1$&$   -1$&$   +1$&$   +1$&$   +1$&$   +1$&$   +1$&$   -1$&$   -1$&$   -1$&$   +1$\\
%$  -1$&$   -1$&$   +1$&$   -1$&$   -1$&$   -1$&$   -1$&$   -1$&$   +1$&$   +1$&$   -1$&$   +1$\\
%$  -1$&$   -1$&$   +1$&$   -1$&$   -1$&$   -1$&$   -1$&$   +1$&$   +1$&$   +1$&$   -1$&$   -1$\\
%$  -1$&$   -1$&$   +1$&$   -1$&$   -1$&$   -1$&$   +1$&$   -1$&$   +1$&$   -1$&$   -1$&$   +1$\\
%$  -1$&$   -1$&$   +1$&$   -1$&$   -1$&$   -1$&$   +1$&$   +1$&$   +1$&$   -1$&$   -1$&$   -1$\\
%$  -1$&$   -1$&$   +1$&$   -1$&$   -1$&$   +1$&$   -1$&$   -1$&$   +1$&$   +1$&$   +1$&$   +1$\\
%$  -1$&$   -1$&$   +1$&$   -1$&$   -1$&$   +1$&$   -1$&$   +1$&$   +1$&$   +1$&$   +1$&$   -1$\\
%$  -1$&$   -1$&$   +1$&$   -1$&$   -1$&$   +1$&$   +1$&$   -1$&$   +1$&$   -1$&$   +1$&$   +1$\\
%$  -1$&$   -1$&$   +1$&$   -1$&$   -1$&$   +1$&$   +1$&$   +1$&$   +1$&$   -1$&$   +1$&$   -1$\\
%$  -1$&$   -1$&$   +1$&$   -1$&$   +1$&$   -1$&$   -1$&$   -1$&$   -1$&$   +1$&$   -1$&$   +1$\\
%$  -1$&$   -1$&$   +1$&$   -1$&$   +1$&$   -1$&$   -1$&$   +1$&$   -1$&$   +1$&$   -1$&$   -1$\\
%$  -1$&$   -1$&$   +1$&$   -1$&$   +1$&$   -1$&$   +1$&$   -1$&$   -1$&$   -1$&$   -1$&$   +1$\\
%$  -1$&$   -1$&$   +1$&$   -1$&$   +1$&$   -1$&$   +1$&$   +1$&$   -1$&$   -1$&$   -1$&$   -1$\\
%$  -1$&$   -1$&$   +1$&$   -1$&$   +1$&$   +1$&$   -1$&$   -1$&$   -1$&$   +1$&$   +1$&$   +1$\\
%$  -1$&$   -1$&$   +1$&$   -1$&$   +1$&$   +1$&$   -1$&$   +1$&$   -1$&$   +1$&$   +1$&$   -1$\\
%$  -1$&$   -1$&$   +1$&$   -1$&$   +1$&$   +1$&$   +1$&$   -1$&$   -1$&$   -1$&$   +1$&$   +1$\\
%$  -1$&$   -1$&$   +1$&$   -1$&$   +1$&$   +1$&$   +1$&$   +1$&$   -1$&$   -1$&$   +1$&$   -1$\\
%$  -1$&$   -1$&$   +1$&$   +1$&$   -1$&$   -1$&$   -1$&$   -1$&$   +1$&$   +1$&$   -1$&$   -1$\\
%$  -1$&$   -1$&$   +1$&$   +1$&$   -1$&$   -1$&$   -1$&$   +1$&$   +1$&$   +1$&$   -1$&$   +1$\\
%$  -1$&$   -1$&$   +1$&$   +1$&$   -1$&$   -1$&$   +1$&$   -1$&$   +1$&$   -1$&$   -1$&$   -1$\\
%$  -1$&$   -1$&$   +1$&$   +1$&$   -1$&$   -1$&$   +1$&$   +1$&$   +1$&$   -1$&$   -1$&$   +1$\\
%$  -1$&$   -1$&$   +1$&$   +1$&$   -1$&$   +1$&$   -1$&$   -1$&$   +1$&$   +1$&$   +1$&$   -1$\\
%$  -1$&$   -1$&$   +1$&$   +1$&$   -1$&$   +1$&$   -1$&$   +1$&$   +1$&$   +1$&$   +1$&$   +1$\\
%$  -1$&$   -1$&$   +1$&$   +1$&$   -1$&$   +1$&$   +1$&$   -1$&$   +1$&$   -1$&$   +1$&$   -1$\\
%$  -1$&$   -1$&$   +1$&$   +1$&$   -1$&$   +1$&$   +1$&$   +1$&$   +1$&$   -1$&$   +1$&$   +1$\\
%$  -1$&$   -1$&$   +1$&$   +1$&$   +1$&$   -1$&$   -1$&$   -1$&$   -1$&$   +1$&$   -1$&$   -1$\\
%$  -1$&$   -1$&$   +1$&$   +1$&$   +1$&$   -1$&$   -1$&$   +1$&$   -1$&$   +1$&$   -1$&$   +1$\\
%$  -1$&$   -1$&$   +1$&$   +1$&$   +1$&$   -1$&$   +1$&$   -1$&$   -1$&$   -1$&$   -1$&$   -1$\\
%$  -1$&$   -1$&$   +1$&$   +1$&$   +1$&$   -1$&$   +1$&$   +1$&$   -1$&$   -1$&$   -1$&$   +1$\\
%$  -1$&$   -1$&$   +1$&$   +1$&$   +1$&$   +1$&$   -1$&$   -1$&$   -1$&$   +1$&$   +1$&$   -1$\\
%$  -1$&$   -1$&$   +1$&$   +1$&$   +1$&$   +1$&$   -1$&$   +1$&$   -1$&$   +1$&$   +1$&$   +1$\\
%$  -1$&$   -1$&$   +1$&$   +1$&$   +1$&$   +1$&$   +1$&$   -1$&$   -1$&$   -1$&$   +1$&$   -1$\\
%$  -1$&$   -1$&$   +1$&$   +1$&$   +1$&$   +1$&$   +1$&$   +1$&$   -1$&$   -1$&$   +1$&$   +1$\\
%$  -1$&$   +1$&$   -1$&$   -1$&$   -1$&$   -1$&$   -1$&$   -1$&$   +1$&$   -1$&$   +1$&$   +1$\\
%$  -1$&$   +1$&$   -1$&$   -1$&$   -1$&$   -1$&$   -1$&$   +1$&$   +1$&$   -1$&$   +1$&$   -1$\\
%$  -1$&$   +1$&$   -1$&$   -1$&$   -1$&$   -1$&$   +1$&$   -1$&$   +1$&$   +1$&$   +1$&$   +1$\\
%$  -1$&$   +1$&$   -1$&$   -1$&$   -1$&$   -1$&$   +1$&$   +1$&$   +1$&$   +1$&$   +1$&$   -1$\\
%$  -1$&$   +1$&$   -1$&$   -1$&$   -1$&$   +1$&$   -1$&$   -1$&$   +1$&$   -1$&$   -1$&$   +1$\\
%$  -1$&$   +1$&$   -1$&$   -1$&$   -1$&$   +1$&$   -1$&$   +1$&$   +1$&$   -1$&$   -1$&$   -1$\\
%$  -1$&$   +1$&$   -1$&$   -1$&$   -1$&$   +1$&$   +1$&$   -1$&$   +1$&$   +1$&$   -1$&$   +1$\\
%$  -1$&$   +1$&$   -1$&$   -1$&$   -1$&$   +1$&$   +1$&$   +1$&$   +1$&$   +1$&$   -1$&$   -1$\\
%$  -1$&$   +1$&$   -1$&$   -1$&$   +1$&$   -1$&$   -1$&$   -1$&$   -1$&$   -1$&$   +1$&$   +1$\\
%$  -1$&$   +1$&$   -1$&$   -1$&$   +1$&$   -1$&$   -1$&$   +1$&$   -1$&$   -1$&$   +1$&$   -1$\\
%$  -1$&$   +1$&$   -1$&$   -1$&$   +1$&$   -1$&$   +1$&$   -1$&$   -1$&$   +1$&$   +1$&$   +1$\\
%$  -1$&$   +1$&$   -1$&$   -1$&$   +1$&$   -1$&$   +1$&$   +1$&$   -1$&$   +1$&$   +1$&$   -1$\\
%$  -1$&$   +1$&$   -1$&$   -1$&$   +1$&$   +1$&$   -1$&$   -1$&$   -1$&$   -1$&$   -1$&$   +1$\\
%$  -1$&$   +1$&$   -1$&$   -1$&$   +1$&$   +1$&$   -1$&$   +1$&$   -1$&$   -1$&$   -1$&$   -1$\\
%$  -1$&$   +1$&$   -1$&$   -1$&$   +1$&$   +1$&$   +1$&$   -1$&$   -1$&$   +1$&$   -1$&$   +1$\\
%$  -1$&$   +1$&$   -1$&$   -1$&$   +1$&$   +1$&$   +1$&$   +1$&$   -1$&$   +1$&$   -1$&$   -1$\\
%$  -1$&$   +1$&$   -1$&$   +1$&$   -1$&$   -1$&$   -1$&$   -1$&$   +1$&$   -1$&$   +1$&$   -1$\\
%$  -1$&$   +1$&$   -1$&$   +1$&$   -1$&$   -1$&$   -1$&$   +1$&$   +1$&$   -1$&$   +1$&$   +1$\\
%$  -1$&$   +1$&$   -1$&$   +1$&$   -1$&$   -1$&$   +1$&$   -1$&$   +1$&$   +1$&$   +1$&$   -1$\\
%$  -1$&$   +1$&$   -1$&$   +1$&$   -1$&$   -1$&$   +1$&$   +1$&$   +1$&$   +1$&$   +1$&$   +1$\\
 $\vdots$&$\vdots$&$\vdots$&$\vdots$&$\vdots$&$\vdots$&$\vdots$&$\vdots$&$\vdots$&$\vdots$&$\vdots$&$\vdots$\\
%$  -1$&$   +1$&$   -1$&$   +1$&$   -1$&$   +1$&$   -1$&$   -1$&$   +1$&$   -1$&$   -1$&$   -1$\\
 {\color{green}$\bf{}-1$}&{\color{red}$\bf{}+1$}&{\color{green}$\bf{}-1$}&{\color{red}$\bf{}+1$}&{\color{green}$\bf{}-1$}&{\color{red}$\bf{}+1$}&{\color{green}$\bf{}-1$}&{\color{red}$\bf{}+1$}&$   +1$&$   -1$&$   -1$&$   +1$\\
%$  -1$&$   +1$&$   -1$&$   +1$&$   -1$&$   +1$&$   +1$&$   -1$&$   +1$&$   +1$&$   -1$&$   -1$\\
%$  -1$&$   +1$&$   -1$&$   +1$&$   -1$&$   +1$&$   +1$&$   +1$&$   +1$&$   +1$&$   -1$&$   +1$\\
%$  -1$&$   +1$&$   -1$&$   +1$&$   +1$&$   -1$&$   -1$&$   -1$&$   -1$&$   -1$&$   +1$&$   -1$\\
%$  -1$&$   +1$&$   -1$&$   +1$&$   +1$&$   -1$&$   -1$&$   +1$&$   -1$&$   -1$&$   +1$&$   +1$\\
%$  -1$&$   +1$&$   -1$&$   +1$&$   +1$&$   -1$&$   +1$&$   -1$&$   -1$&$   +1$&$   +1$&$   -1$\\
%$  -1$&$   +1$&$   -1$&$   +1$&$   +1$&$   -1$&$   +1$&$   +1$&$   -1$&$   +1$&$   +1$&$   +1$\\
%$  -1$&$   +1$&$   -1$&$   +1$&$   +1$&$   +1$&$   -1$&$   -1$&$   -1$&$   -1$&$   -1$&$   -1$\\
%$  -1$&$   +1$&$   -1$&$   +1$&$   +1$&$   +1$&$   -1$&$   +1$&$   -1$&$   -1$&$   -1$&$   +1$\\
%$  -1$&$   +1$&$   -1$&$   +1$&$   +1$&$   +1$&$   +1$&$   -1$&$   -1$&$   +1$&$   -1$&$   -1$\\
%$  -1$&$   +1$&$   -1$&$   +1$&$   +1$&$   +1$&$   +1$&$   +1$&$   -1$&$   +1$&$   -1$&$   +1$\\
%$  -1$&$   +1$&$   +1$&$   -1$&$   -1$&$   -1$&$   -1$&$   -1$&$   +1$&$   -1$&$   -1$&$   +1$\\
%$  -1$&$   +1$&$   +1$&$   -1$&$   -1$&$   -1$&$   -1$&$   +1$&$   +1$&$   -1$&$   -1$&$   -1$\\
%$  -1$&$   +1$&$   +1$&$   -1$&$   -1$&$   -1$&$   +1$&$   -1$&$   +1$&$   +1$&$   -1$&$   +1$\\
%$  -1$&$   +1$&$   +1$&$   -1$&$   -1$&$   -1$&$   +1$&$   +1$&$   +1$&$   +1$&$   -1$&$   -1$\\
%$  -1$&$   +1$&$   +1$&$   -1$&$   -1$&$   +1$&$   -1$&$   -1$&$   +1$&$   -1$&$   +1$&$   +1$\\
%$  -1$&$   +1$&$   +1$&$   -1$&$   -1$&$   +1$&$   -1$&$   +1$&$   +1$&$   -1$&$   +1$&$   -1$\\
%$  -1$&$   +1$&$   +1$&$   -1$&$   -1$&$   +1$&$   +1$&$   -1$&$   +1$&$   +1$&$   +1$&$   +1$\\
%$  -1$&$   +1$&$   +1$&$   -1$&$   -1$&$   +1$&$   +1$&$   +1$&$   +1$&$   +1$&$   +1$&$   -1$\\
%$  -1$&$   +1$&$   +1$&$   -1$&$   +1$&$   -1$&$   -1$&$   -1$&$   -1$&$   -1$&$   -1$&$   +1$\\
%$  -1$&$   +1$&$   +1$&$   -1$&$   +1$&$   -1$&$   -1$&$   +1$&$   -1$&$   -1$&$   -1$&$   -1$\\
%$  -1$&$   +1$&$   +1$&$   -1$&$   +1$&$   -1$&$   +1$&$   -1$&$   -1$&$   +1$&$   -1$&$   +1$\\
%$  -1$&$   +1$&$   +1$&$   -1$&$   +1$&$   -1$&$   +1$&$   +1$&$   -1$&$   +1$&$   -1$&$   -1$\\
%$  -1$&$   +1$&$   +1$&$   -1$&$   +1$&$   +1$&$   -1$&$   -1$&$   -1$&$   -1$&$   +1$&$   +1$\\
%$  -1$&$   +1$&$   +1$&$   -1$&$   +1$&$   +1$&$   -1$&$   +1$&$   -1$&$   -1$&$   +1$&$   -1$\\
%$  -1$&$   +1$&$   +1$&$   -1$&$   +1$&$   +1$&$   +1$&$   -1$&$   -1$&$   +1$&$   +1$&$   +1$\\
%$  -1$&$   +1$&$   +1$&$   -1$&$   +1$&$   +1$&$   +1$&$   +1$&$   -1$&$   +1$&$   +1$&$   -1$\\
%$  -1$&$   +1$&$   +1$&$   +1$&$   -1$&$   -1$&$   -1$&$   -1$&$   +1$&$   -1$&$   -1$&$   -1$\\
%$  -1$&$   +1$&$   +1$&$   +1$&$   -1$&$   -1$&$   -1$&$   +1$&$   +1$&$   -1$&$   -1$&$   +1$\\
%$  -1$&$   +1$&$   +1$&$   +1$&$   -1$&$   -1$&$   +1$&$   -1$&$   +1$&$   +1$&$   -1$&$   -1$\\
%$  -1$&$   +1$&$   +1$&$   +1$&$   -1$&$   -1$&$   +1$&$   +1$&$   +1$&$   +1$&$   -1$&$   +1$\\
%$  -1$&$   +1$&$   +1$&$   +1$&$   -1$&$   +1$&$   -1$&$   -1$&$   +1$&$   -1$&$   +1$&$   -1$\\
%$  -1$&$   +1$&$   +1$&$   +1$&$   -1$&$   +1$&$   -1$&$   +1$&$   +1$&$   -1$&$   +1$&$   +1$\\
%$  -1$&$   +1$&$   +1$&$   +1$&$   -1$&$   +1$&$   +1$&$   -1$&$   +1$&$   +1$&$   +1$&$   -1$\\
%$  -1$&$   +1$&$   +1$&$   +1$&$   -1$&$   +1$&$   +1$&$   +1$&$   +1$&$   +1$&$   +1$&$   +1$\\
%$  -1$&$   +1$&$   +1$&$   +1$&$   +1$&$   -1$&$   -1$&$   -1$&$   -1$&$   -1$&$   -1$&$   -1$\\
%$  -1$&$   +1$&$   +1$&$   +1$&$   +1$&$   -1$&$   -1$&$   +1$&$   -1$&$   -1$&$   -1$&$   +1$\\
%$  -1$&$   +1$&$   +1$&$   +1$&$   +1$&$   -1$&$   +1$&$   -1$&$   -1$&$   +1$&$   -1$&$   -1$\\
%$  -1$&$   +1$&$   +1$&$   +1$&$   +1$&$   -1$&$   +1$&$   +1$&$   -1$&$   +1$&$   -1$&$   +1$\\
%$  -1$&$   +1$&$   +1$&$   +1$&$   +1$&$   +1$&$   -1$&$   -1$&$   -1$&$   -1$&$   +1$&$   -1$\\
%$  -1$&$   +1$&$   +1$&$   +1$&$   +1$&$   +1$&$   -1$&$   +1$&$   -1$&$   -1$&$   +1$&$   +1$\\
%$  -1$&$   +1$&$   +1$&$   +1$&$   +1$&$   +1$&$   +1$&$   -1$&$   -1$&$   +1$&$   +1$&$   -1$\\
%$  -1$&$   +1$&$   +1$&$   +1$&$   +1$&$   +1$&$   +1$&$   +1$&$   -1$&$   +1$&$   +1$&$   +1$\\
%$  +1$&$   -1$&$   -1$&$   -1$&$   -1$&$   -1$&$   -1$&$   -1$&$   -1$&$   +1$&$   +1$&$   +1$\\
%$  +1$&$   -1$&$   -1$&$   -1$&$   -1$&$   -1$&$   -1$&$   +1$&$   -1$&$   +1$&$   +1$&$   -1$\\
%$  +1$&$   -1$&$   -1$&$   -1$&$   -1$&$   -1$&$   +1$&$   -1$&$   -1$&$   -1$&$   +1$&$   +1$\\
%$  +1$&$   -1$&$   -1$&$   -1$&$   -1$&$   -1$&$   +1$&$   +1$&$   -1$&$   -1$&$   +1$&$   -1$\\
%$  +1$&$   -1$&$   -1$&$   -1$&$   -1$&$   +1$&$   -1$&$   -1$&$   -1$&$   +1$&$   -1$&$   +1$\\
%$  +1$&$   -1$&$   -1$&$   -1$&$   -1$&$   +1$&$   -1$&$   +1$&$   -1$&$   +1$&$   -1$&$   -1$\\
%$  +1$&$   -1$&$   -1$&$   -1$&$   -1$&$   +1$&$   +1$&$   -1$&$   -1$&$   -1$&$   -1$&$   +1$\\
%$  +1$&$   -1$&$   -1$&$   -1$&$   -1$&$   +1$&$   +1$&$   +1$&$   -1$&$   -1$&$   -1$&$   -1$\\
%$  +1$&$   -1$&$   -1$&$   -1$&$   +1$&$   -1$&$   -1$&$   -1$&$   +1$&$   +1$&$   +1$&$   +1$\\
%$  +1$&$   -1$&$   -1$&$   -1$&$   +1$&$   -1$&$   -1$&$   +1$&$   +1$&$   +1$&$   +1$&$   -1$\\
%$  +1$&$   -1$&$   -1$&$   -1$&$   +1$&$   -1$&$   +1$&$   -1$&$   +1$&$   -1$&$   +1$&$   +1$\\
%$  +1$&$   -1$&$   -1$&$   -1$&$   +1$&$   -1$&$   +1$&$   +1$&$   +1$&$   -1$&$   +1$&$   -1$\\
%$  +1$&$   -1$&$   -1$&$   -1$&$   +1$&$   +1$&$   -1$&$   -1$&$   +1$&$   +1$&$   -1$&$   +1$\\
%$  +1$&$   -1$&$   -1$&$   -1$&$   +1$&$   +1$&$   -1$&$   +1$&$   +1$&$   +1$&$   -1$&$   -1$\\
%$  +1$&$   -1$&$   -1$&$   -1$&$   +1$&$   +1$&$   +1$&$   -1$&$   +1$&$   -1$&$   -1$&$   +1$\\
%$  +1$&$   -1$&$   -1$&$   -1$&$   +1$&$   +1$&$   +1$&$   +1$&$   +1$&$   -1$&$   -1$&$   -1$\\
%$  +1$&$   -1$&$   -1$&$   +1$&$   -1$&$   -1$&$   -1$&$   -1$&$   -1$&$   +1$&$   +1$&$   -1$\\
%$  +1$&$   -1$&$   -1$&$   +1$&$   -1$&$   -1$&$   -1$&$   +1$&$   -1$&$   +1$&$   +1$&$   +1$\\
%$  +1$&$   -1$&$   -1$&$   +1$&$   -1$&$   -1$&$   +1$&$   -1$&$   -1$&$   -1$&$   +1$&$   -1$\\
%$  +1$&$   -1$&$   -1$&$   +1$&$   -1$&$   -1$&$   +1$&$   +1$&$   -1$&$   -1$&$   +1$&$   +1$\\
%$  +1$&$   -1$&$   -1$&$   +1$&$   -1$&$   +1$&$   -1$&$   -1$&$   -1$&$   +1$&$   -1$&$   -1$\\
%$  +1$&$   -1$&$   -1$&$   +1$&$   -1$&$   +1$&$   -1$&$   +1$&$   -1$&$   +1$&$   -1$&$   +1$\\
%$  +1$&$   -1$&$   -1$&$   +1$&$   -1$&$   +1$&$   +1$&$   -1$&$   -1$&$   -1$&$   -1$&$   -1$\\
%$  +1$&$   -1$&$   -1$&$   +1$&$   -1$&$   +1$&$   +1$&$   +1$&$   -1$&$   -1$&$   -1$&$   +1$\\
%$  +1$&$   -1$&$   -1$&$   +1$&$   +1$&$   -1$&$   -1$&$   -1$&$   +1$&$   +1$&$   +1$&$   -1$\\
%$  +1$&$   -1$&$   -1$&$   +1$&$   +1$&$   -1$&$   -1$&$   +1$&$   +1$&$   +1$&$   +1$&$   +1$\\
%$  +1$&$   -1$&$   -1$&$   +1$&$   +1$&$   -1$&$   +1$&$   -1$&$   +1$&$   -1$&$   +1$&$   -1$\\
%$  +1$&$   -1$&$   -1$&$   +1$&$   +1$&$   -1$&$   +1$&$   +1$&$   +1$&$   -1$&$   +1$&$   +1$\\
%$  +1$&$   -1$&$   -1$&$   +1$&$   +1$&$   +1$&$   -1$&$   -1$&$   +1$&$   +1$&$   -1$&$   -1$\\
%$  +1$&$   -1$&$   -1$&$   +1$&$   +1$&$   +1$&$   -1$&$   +1$&$   +1$&$   +1$&$   -1$&$   +1$\\
%$  +1$&$   -1$&$   -1$&$   +1$&$   +1$&$   +1$&$   +1$&$   -1$&$   +1$&$   -1$&$   -1$&$   -1$\\
%$  +1$&$   -1$&$   -1$&$   +1$&$   +1$&$   +1$&$   +1$&$   +1$&$   +1$&$   -1$&$   -1$&$   +1$\\
%$  +1$&$   -1$&$   +1$&$   -1$&$   -1$&$   -1$&$   -1$&$   -1$&$   -1$&$   +1$&$   -1$&$   +1$\\
%$  +1$&$   -1$&$   +1$&$   -1$&$   -1$&$   -1$&$   -1$&$   +1$&$   -1$&$   +1$&$   -1$&$   -1$\\
%$  +1$&$   -1$&$   +1$&$   -1$&$   -1$&$   -1$&$   +1$&$   -1$&$   -1$&$   -1$&$   -1$&$   +1$\\
%$  +1$&$   -1$&$   +1$&$   -1$&$   -1$&$   -1$&$   +1$&$   +1$&$   -1$&$   -1$&$   -1$&$   -1$\\
%$  +1$&$   -1$&$   +1$&$   -1$&$   -1$&$   +1$&$   -1$&$   -1$&$   -1$&$   +1$&$   +1$&$   +1$\\
%$  +1$&$   -1$&$   +1$&$   -1$&$   -1$&$   +1$&$   -1$&$   +1$&$   -1$&$   +1$&$   +1$&$   -1$\\
%$  +1$&$   -1$&$   +1$&$   -1$&$   -1$&$   +1$&$   +1$&$   -1$&$   -1$&$   -1$&$   +1$&$   +1$\\
%$  +1$&$   -1$&$   +1$&$   -1$&$   -1$&$   +1$&$   +1$&$   +1$&$   -1$&$   -1$&$   +1$&$   -1$\\
%$  +1$&$   -1$&$   +1$&$   -1$&$   +1$&$   -1$&$   -1$&$   -1$&$   +1$&$   +1$&$   -1$&$   +1$\\
%$  +1$&$   -1$&$   +1$&$   -1$&$   +1$&$   -1$&$   -1$&$   +1$&$   +1$&$   +1$&$   -1$&$   -1$\\
 $\vdots$&$\vdots$&$\vdots$&$\vdots$&$\vdots$&$\vdots$&$\vdots$&$\vdots$&$\vdots$&$\vdots$&$\vdots$&$\vdots$\\
 {\color{red}$\bf{}+1$}&{\color{green}$\bf{}-1$}&{\color{red}$\bf{}+1$}&{\color{green}$\bf{}-1$}&{\color{red}$\bf{}+1$}&{\color{green}$\bf{}-1$}&{\color{red}$\bf{}+1$}&{\color{green}$\bf{}-1$}&$   +1$&$   -1$&$   -1$&$   +1$\\
%$  +1$&$   -1$&$   +1$&$   -1$&$   +1$&$   -1$&$   +1$&$   +1$&$   +1$&$   -1$&$   -1$&$   -1$\\
%$  +1$&$   -1$&$   +1$&$   -1$&$   +1$&$   +1$&$   -1$&$   -1$&$   +1$&$   +1$&$   +1$&$   +1$\\
%$  +1$&$   -1$&$   +1$&$   -1$&$   +1$&$   +1$&$   -1$&$   +1$&$   +1$&$   +1$&$   +1$&$   -1$\\
%$  +1$&$   -1$&$   +1$&$   -1$&$   +1$&$   +1$&$   +1$&$   -1$&$   +1$&$   -1$&$   +1$&$   +1$\\
%$  +1$&$   -1$&$   +1$&$   -1$&$   +1$&$   +1$&$   +1$&$   +1$&$   +1$&$   -1$&$   +1$&$   -1$\\
%$  +1$&$   -1$&$   +1$&$   +1$&$   -1$&$   -1$&$   -1$&$   -1$&$   -1$&$   +1$&$   -1$&$   -1$\\
%$  +1$&$   -1$&$   +1$&$   +1$&$   -1$&$   -1$&$   -1$&$   +1$&$   -1$&$   +1$&$   -1$&$   +1$\\
%$  +1$&$   -1$&$   +1$&$   +1$&$   -1$&$   -1$&$   +1$&$   -1$&$   -1$&$   -1$&$   -1$&$   -1$\\
%$  +1$&$   -1$&$   +1$&$   +1$&$   -1$&$   -1$&$   +1$&$   +1$&$   -1$&$   -1$&$   -1$&$   +1$\\
%$  +1$&$   -1$&$   +1$&$   +1$&$   -1$&$   +1$&$   -1$&$   -1$&$   -1$&$   +1$&$   +1$&$   -1$\\
%$  +1$&$   -1$&$   +1$&$   +1$&$   -1$&$   +1$&$   -1$&$   +1$&$   -1$&$   +1$&$   +1$&$   +1$\\
%$  +1$&$   -1$&$   +1$&$   +1$&$   -1$&$   +1$&$   +1$&$   -1$&$   -1$&$   -1$&$   +1$&$   -1$\\
%$  +1$&$   -1$&$   +1$&$   +1$&$   -1$&$   +1$&$   +1$&$   +1$&$   -1$&$   -1$&$   +1$&$   +1$\\
%$  +1$&$   -1$&$   +1$&$   +1$&$   +1$&$   -1$&$   -1$&$   -1$&$   +1$&$   +1$&$   -1$&$   -1$\\
%$  +1$&$   -1$&$   +1$&$   +1$&$   +1$&$   -1$&$   -1$&$   +1$&$   +1$&$   +1$&$   -1$&$   +1$\\
%$  +1$&$   -1$&$   +1$&$   +1$&$   +1$&$   -1$&$   +1$&$   -1$&$   +1$&$   -1$&$   -1$&$   -1$\\
%$  +1$&$   -1$&$   +1$&$   +1$&$   +1$&$   -1$&$   +1$&$   +1$&$   +1$&$   -1$&$   -1$&$   +1$\\
%$  +1$&$   -1$&$   +1$&$   +1$&$   +1$&$   +1$&$   -1$&$   -1$&$   +1$&$   +1$&$   +1$&$   -1$\\
%$  +1$&$   -1$&$   +1$&$   +1$&$   +1$&$   +1$&$   -1$&$   +1$&$   +1$&$   +1$&$   +1$&$   +1$\\
%$  +1$&$   -1$&$   +1$&$   +1$&$   +1$&$   +1$&$   +1$&$   -1$&$   +1$&$   -1$&$   +1$&$   -1$\\
%$  +1$&$   -1$&$   +1$&$   +1$&$   +1$&$   +1$&$   +1$&$   +1$&$   +1$&$   -1$&$   +1$&$   +1$\\
%$  +1$&$   +1$&$   -1$&$   -1$&$   -1$&$   -1$&$   -1$&$   -1$&$   -1$&$   -1$&$   +1$&$   +1$\\
%$  +1$&$   +1$&$   -1$&$   -1$&$   -1$&$   -1$&$   -1$&$   +1$&$   -1$&$   -1$&$   +1$&$   -1$\\
%$  +1$&$   +1$&$   -1$&$   -1$&$   -1$&$   -1$&$   +1$&$   -1$&$   -1$&$   +1$&$   +1$&$   +1$\\
%$  +1$&$   +1$&$   -1$&$   -1$&$   -1$&$   -1$&$   +1$&$   +1$&$   -1$&$   +1$&$   +1$&$   -1$\\
%$  +1$&$   +1$&$   -1$&$   -1$&$   -1$&$   +1$&$   -1$&$   -1$&$   -1$&$   -1$&$   -1$&$   +1$\\
%$  +1$&$   +1$&$   -1$&$   -1$&$   -1$&$   +1$&$   -1$&$   +1$&$   -1$&$   -1$&$   -1$&$   -1$\\
%$  +1$&$   +1$&$   -1$&$   -1$&$   -1$&$   +1$&$   +1$&$   -1$&$   -1$&$   +1$&$   -1$&$   +1$\\
%$  +1$&$   +1$&$   -1$&$   -1$&$   -1$&$   +1$&$   +1$&$   +1$&$   -1$&$   +1$&$   -1$&$   -1$\\
%$  +1$&$   +1$&$   -1$&$   -1$&$   +1$&$   -1$&$   -1$&$   -1$&$   +1$&$   -1$&$   +1$&$   +1$\\
%$  +1$&$   +1$&$   -1$&$   -1$&$   +1$&$   -1$&$   -1$&$   +1$&$   +1$&$   -1$&$   +1$&$   -1$\\
%$  +1$&$   +1$&$   -1$&$   -1$&$   +1$&$   -1$&$   +1$&$   -1$&$   +1$&$   +1$&$   +1$&$   +1$\\
%$  +1$&$   +1$&$   -1$&$   -1$&$   +1$&$   -1$&$   +1$&$   +1$&$   +1$&$   +1$&$   +1$&$   -1$\\
%$  +1$&$   +1$&$   -1$&$   -1$&$   +1$&$   +1$&$   -1$&$   -1$&$   +1$&$   -1$&$   -1$&$   +1$\\
%$  +1$&$   +1$&$   -1$&$   -1$&$   +1$&$   +1$&$   -1$&$   +1$&$   +1$&$   -1$&$   -1$&$   -1$\\
%$  +1$&$   +1$&$   -1$&$   -1$&$   +1$&$   +1$&$   +1$&$   -1$&$   +1$&$   +1$&$   -1$&$   +1$\\
%$  +1$&$   +1$&$   -1$&$   -1$&$   +1$&$   +1$&$   +1$&$   +1$&$   +1$&$   +1$&$   -1$&$   -1$\\
%$  +1$&$   +1$&$   -1$&$   +1$&$   -1$&$   -1$&$   -1$&$   -1$&$   -1$&$   -1$&$   +1$&$   -1$\\
%$  +1$&$   +1$&$   -1$&$   +1$&$   -1$&$   -1$&$   -1$&$   +1$&$   -1$&$   -1$&$   +1$&$   +1$\\
%$  +1$&$   +1$&$   -1$&$   +1$&$   -1$&$   -1$&$   +1$&$   -1$&$   -1$&$   +1$&$   +1$&$   -1$\\
%$  +1$&$   +1$&$   -1$&$   +1$&$   -1$&$   -1$&$   +1$&$   +1$&$   -1$&$   +1$&$   +1$&$   +1$\\
%$  +1$&$   +1$&$   -1$&$   +1$&$   -1$&$   +1$&$   -1$&$   -1$&$   -1$&$   -1$&$   -1$&$   -1$\\
%$  +1$&$   +1$&$   -1$&$   +1$&$   -1$&$   +1$&$   -1$&$   +1$&$   -1$&$   -1$&$   -1$&$   +1$\\
%$  +1$&$   +1$&$   -1$&$   +1$&$   -1$&$   +1$&$   +1$&$   -1$&$   -1$&$   +1$&$   -1$&$   -1$\\
%$  +1$&$   +1$&$   -1$&$   +1$&$   -1$&$   +1$&$   +1$&$   +1$&$   -1$&$   +1$&$   -1$&$   +1$\\
%$  +1$&$   +1$&$   -1$&$   +1$&$   +1$&$   -1$&$   -1$&$   -1$&$   +1$&$   -1$&$   +1$&$   -1$\\
%$  +1$&$   +1$&$   -1$&$   +1$&$   +1$&$   -1$&$   -1$&$   +1$&$   +1$&$   -1$&$   +1$&$   +1$\\
%$  +1$&$   +1$&$   -1$&$   +1$&$   +1$&$   -1$&$   +1$&$   -1$&$   +1$&$   +1$&$   +1$&$   -1$\\
%$  +1$&$   +1$&$   -1$&$   +1$&$   +1$&$   -1$&$   +1$&$   +1$&$   +1$&$   +1$&$   +1$&$   +1$\\
%$  +1$&$   +1$&$   -1$&$   +1$&$   +1$&$   +1$&$   -1$&$   -1$&$   +1$&$   -1$&$   -1$&$   -1$\\
%$  +1$&$   +1$&$   -1$&$   +1$&$   +1$&$   +1$&$   -1$&$   +1$&$   +1$&$   -1$&$   -1$&$   +1$\\
%$  +1$&$   +1$&$   -1$&$   +1$&$   +1$&$   +1$&$   +1$&$   -1$&$   +1$&$   +1$&$   -1$&$   -1$\\
%$  +1$&$   +1$&$   -1$&$   +1$&$   +1$&$   +1$&$   +1$&$   +1$&$   +1$&$   +1$&$   -1$&$   +1$\\
%$  +1$&$   +1$&$   +1$&$   -1$&$   -1$&$   -1$&$   -1$&$   -1$&$   -1$&$   -1$&$   -1$&$   +1$\\
%$  +1$&$   +1$&$   +1$&$   -1$&$   -1$&$   -1$&$   -1$&$   +1$&$   -1$&$   -1$&$   -1$&$   -1$\\
%$  +1$&$   +1$&$   +1$&$   -1$&$   -1$&$   -1$&$   +1$&$   -1$&$   -1$&$   +1$&$   -1$&$   +1$\\
%$  +1$&$   +1$&$   +1$&$   -1$&$   -1$&$   -1$&$   +1$&$   +1$&$   -1$&$   +1$&$   -1$&$   -1$\\
%$  +1$&$   +1$&$   +1$&$   -1$&$   -1$&$   +1$&$   -1$&$   -1$&$   -1$&$   -1$&$   +1$&$   +1$\\
%$  +1$&$   +1$&$   +1$&$   -1$&$   -1$&$   +1$&$   -1$&$   +1$&$   -1$&$   -1$&$   +1$&$   -1$\\
%$  +1$&$   +1$&$   +1$&$   -1$&$   -1$&$   +1$&$   +1$&$   -1$&$   -1$&$   +1$&$   +1$&$   +1$\\
%$  +1$&$   +1$&$   +1$&$   -1$&$   -1$&$   +1$&$   +1$&$   +1$&$   -1$&$   +1$&$   +1$&$   -1$\\
%$  +1$&$   +1$&$   +1$&$   -1$&$   +1$&$   -1$&$   -1$&$   -1$&$   +1$&$   -1$&$   -1$&$   +1$\\
%$  +1$&$   +1$&$   +1$&$   -1$&$   +1$&$   -1$&$   -1$&$   +1$&$   +1$&$   -1$&$   -1$&$   -1$\\
%$  +1$&$   +1$&$   +1$&$   -1$&$   +1$&$   -1$&$   +1$&$   -1$&$   +1$&$   +1$&$   -1$&$   +1$\\
%$  +1$&$   +1$&$   +1$&$   -1$&$   +1$&$   -1$&$   +1$&$   +1$&$   +1$&$   +1$&$   -1$&$   -1$\\
%$  +1$&$   +1$&$   +1$&$   -1$&$   +1$&$   +1$&$   -1$&$   -1$&$   +1$&$   -1$&$   +1$&$   +1$\\
%$  +1$&$   +1$&$   +1$&$   -1$&$   +1$&$   +1$&$   -1$&$   +1$&$   +1$&$   -1$&$   +1$&$   -1$\\
%$  +1$&$   +1$&$   +1$&$   -1$&$   +1$&$   +1$&$   +1$&$   -1$&$   +1$&$   +1$&$   +1$&$   +1$\\
%$  +1$&$   +1$&$   +1$&$   -1$&$   +1$&$   +1$&$   +1$&$   +1$&$   +1$&$   +1$&$   +1$&$   -1$\\
%$  +1$&$   +1$&$   +1$&$   +1$&$   -1$&$   -1$&$   -1$&$   -1$&$   -1$&$   -1$&$   -1$&$   -1$\\
%$  +1$&$   +1$&$   +1$&$   +1$&$   -1$&$   -1$&$   -1$&$   +1$&$   -1$&$   -1$&$   -1$&$   +1$\\
%$  +1$&$   +1$&$   +1$&$   +1$&$   -1$&$   -1$&$   +1$&$   -1$&$   -1$&$   +1$&$   -1$&$   -1$\\
%$  +1$&$   +1$&$   +1$&$   +1$&$   -1$&$   -1$&$   +1$&$   +1$&$   -1$&$   +1$&$   -1$&$   +1$\\
%$  +1$&$   +1$&$   +1$&$   +1$&$   -1$&$   +1$&$   -1$&$   -1$&$   -1$&$   -1$&$   +1$&$   -1$\\
%$  +1$&$   +1$&$   +1$&$   +1$&$   -1$&$   +1$&$   -1$&$   +1$&$   -1$&$   -1$&$   +1$&$   +1$\\
%$  +1$&$   +1$&$   +1$&$   +1$&$   -1$&$   +1$&$   +1$&$   -1$&$   -1$&$   +1$&$   +1$&$   -1$\\
%$  +1$&$   +1$&$   +1$&$   +1$&$   -1$&$   +1$&$   +1$&$   +1$&$   -1$&$   +1$&$   +1$&$   +1$\\
%$  +1$&$   +1$&$   +1$&$   +1$&$   +1$&$   -1$&$   -1$&$   -1$&$   +1$&$   -1$&$   -1$&$   -1$\\
%$  +1$&$   +1$&$   +1$&$   +1$&$   +1$&$   -1$&$   -1$&$   +1$&$   +1$&$   -1$&$   -1$&$   +1$\\
%$  +1$&$   +1$&$   +1$&$   +1$&$   +1$&$   -1$&$   +1$&$   -1$&$   +1$&$   +1$&$   -1$&$   -1$\\
%$  +1$&$   +1$&$   +1$&$   +1$&$   +1$&$   -1$&$   +1$&$   +1$&$   +1$&$   +1$&$   -1$&$   +1$\\
%$  +1$&$   +1$&$   +1$&$   +1$&$   +1$&$   +1$&$   -1$&$   -1$&$   +1$&$   -1$&$   +1$&$   -1$\\
 $\vdots$&$\vdots$&$\vdots$&$\vdots$&$\vdots$&$\vdots$&$\vdots$&$\vdots$&$\vdots$&$\vdots$&$\vdots$&$\vdots$\\
 $  +1$&$   +1$&$   +1$&$   +1$&$   +1$&$   +1$&{\color{green}$\bf{}-1$}&{\color{red}$\bf{}+1$}&$   +1$&$   -1$&$   +1$&$   +1$\\
 $  +1$&$   +1$&$   +1$&$   +1$&$   +1$&$   +1$&{\color{red}$\bf{}+1$}&{\color{green}$\bf{}-1$}&$  +1$&$   +1$&$   +1$&$   -1$\\
 $  +1$&$   +1$&$   +1$&$   +1$&$   +1$&$   +1$&$   +1$&$   +1$&$   +1$&$   +1$&$   +1$&$   +1$\\
\hline\hline
\end{tabular}
\end{center}
\caption{(Color online) Contextual (bold) and noncontextual value assignments, and the associated joint values.}
\label{2011-enough-t2}
\end{table*}
According to the  Minkoswki-Weyl representation theorem~\cite[p~29]{ziegler}, an equivalent (hull) representation of the
associated convex polyhedron is in terms of the halfspaces defined by Boole-Bell type inequalities of the form
\begin{equation}
\begin{array}{l}
-1\le E(a_b)+E(b_a)+E(a_b b_a),  \\
-1\le  E(a_b)-E(b_a)-E(a_b b_a),  \\
-1\le  -E(a_b)+E(b_a)-E(a_b b_a),  \\
-1\le  -E(a_b)-E(b_a)+E(a_b b_a),
\end{array}
\label{2011-enough-e4}
\end{equation}
(and the inequalities resulting from permuting
$a \leftrightarrow a'$,
$b \leftrightarrow b'$)
which, for $E(a_b)=E(b_a)=0$, reduce to $-1\le E(a_b b_a)\le 1$.
Note that, by taking only the 16 context-independent ($x_y=x_{y'}$) from all the 256 assignments,
the CHSH inequality~(\ref{2011-enough-e1}) with $\lambda =2$ is recovered.

Next, for the sake of demonstration, an example configuration will be given that conforms to
Tsirel'son's maximal quantum bound of $\lambda = 2\sqrt{2}$~\cite{cirelson}.
Substituting this for $2\sqrt{2}$ in Eq.~(\ref{2011-enough-e3})
yields $x=\pm (\sqrt{2}-1)$; that is,
the (limit) frequency for the occurrence of contextual assignments
$
b'_{a} = - b'_{a'}
$
as enumerated in
Table~\ref{2011-enough-t1} with respect to the associated noncontextual assignments
$
b'_{a} = b'_{a'}
$
(rendering 2 to the sum of terms in the CHSH expression) should be
$(\sqrt{2}-1) : (2-\sqrt{2})$.
More explicitly,
if there are four different assignments,
enumerated in Table~\ref{2011-enough-t1},
which may contribute quantum mechanically by the correct (limiting) frequency,
then Table~\ref{2011-enough-t3} is a simulation of 20 assignments rendering the maximal quantum bound
for the CHSH inequalities.
\begin{table}
\begin{center}
\begin{tabular}{cccccccc} % |ccccccccccccccccccccccccc}
\hline\hline
$a_b$&$a_{b'}$&$a'_b$&$a'_{b'}$&$b_a$&$b_{a'}$&$b'_a$&$b'_{a'}$\\% &$a_b b_a$&$a_{b'} b'_a$&$a'_b b_{a'}$&$a'_{b'} b'_{a'}$     \\
\hline
$+1$&$+1$&$+1$&$+1$&$+1$&$+1$&$+1$&$+1$\\ % &$+1$&$+1$&$+1$&$+1$\\
$-1$&$-1$&$-1$&$-1$&$-1$&$-1$&$-1$&$-1$\\ % &$+1$&$+1$&$+1$&$+1$\\
$+1$&$+1$&$+1$&$+1$&$+1$&$+1$&{\color{green}$\bf{}+1$}&{\color{red}$\bf{}-1$}\\ % &$+1$&$+1$&$+1$&$-1$\\
$-1$&$-1$&$-1$&$-1$&$-1$&$-1$&{\color{green}$\bf{}-1$}&{\color{red}$\bf{}+1$}\\ % &$+1$&$+1$&$+1$&$-1$\\
$-1$&$-1$&$-1$&$-1$&$-1$&$-1$&$-1$&$-1$\\ % &$+1$&$+1$&$+1$&$+1$\\
$+1$&$+1$&$+1$&$+1$&$+1$&$+1$&{\color{green}$\bf{}+1$}&{\color{red}$\bf{}-1$}\\ % &$+1$&$+1$&$+1$&$-1$\\
$+1$&$+1$&$+1$&$+1$&$+1$&$+1$&{\color{green}$\bf{}+1$}&{\color{red}$\bf{}-1$}\\ % &$+1$&$+1$&$+1$&$-1$\\
$+1$&$+1$&$+1$&$+1$&$+1$&$+1$&$+1$&$+1$\\ % &$+1$&$+1$&$+1$&$+1$\\
$-1$&$-1$&$-1$&$-1$&$-1$&$-1$&{\color{green}$\bf{}-1$}&{\color{red}$\bf{}+1$}\\ % &$+1$&$+1$&$+1$&$-1$\\
$+1$&$+1$&$+1$&$+1$&$+1$&$+1$&$+1$&$+1$\\ % &$+1$&$+1$&$+1$&$+1$\\
$-1$&$-1$&$-1$&$-1$&$-1$&$-1$&$-1$&$-1$\\ % &$+1$&$+1$&$+1$&$+1$\\
$+1$&$+1$&$+1$&$+1$&$+1$&$+1$&$+1$&$+1$\\ % &$+1$&$+1$&$+1$&$+1$\\
$+1$&$+1$&$+1$&$+1$&$+1$&$+1$&$+1$&$+1$\\ % &$+1$&$+1$&$+1$&$+1$\\
$+1$&$+1$&$+1$&$+1$&$+1$&$+1$&{\color{green}$\bf{}+1$}&{\color{red}$\bf{}-1$}\\ % &$+1$&$+1$&$+1$&$-1$\\
$+1$&$+1$&$+1$&$+1$&$+1$&$+1$&{\color{green}$\bf{}+1$}&{\color{red}$\bf{}-1$}\\ % &$+1$&$+1$&$+1$&$-1$\\
$+1$&$+1$&$+1$&$+1$&$+1$&$+1$&$+1$&$+1$\\ % &$+1$&$+1$&$+1$&$+1$\\
$-1$&$-1$&$-1$&$-1$&$-1$&$-1$&{\color{green}$\bf{}-1$}&{\color{red}$\bf{}+1$}\\ % &$+1$&$+1$&$+1$&$-1$\\
$-1$&$-1$&$-1$&$-1$&$-1$&$-1$&$-1$&$-1$\\ % &$+1$&$+1$&$+1$&$+1$\\
$+1$&$+1$&$+1$&$+1$&$+1$&$+1$&{\color{green}$\bf{}+1$}&{\color{red}$\bf{}-1$}\\ % &$+1$&$+1$&$+1$&$-1$\\
$+1$&$+1$&$+1$&$+1$&$+1$&$+1$&$+1$&$+1$\\ % &$+1$&$+1$&$+1$&$+1$\\
$+1$&$+1$&$+1$&$+1$&$+1$&$+1$&{\color{green}$\bf{}+1$}&{\color{red}$\bf{}-1$}\\ % &$+1$&$+1$&$+1$&$-1$\\
$-1$&$-1$&$-1$&$-1$&$-1$&$-1$&$-1$&$-1$\\ % &$+1$&$+1$&$+1$&$+1$\\
$-1$&$-1$&$-1$&$-1$&$-1$&$-1$&{\color{green}$\bf{}-1$}&{\color{red}$\bf{}+1$}\\ % &$+1$&$+1$&$+1$&$-1$\\
$+1$&$+1$&$+1$&$+1$&$+1$&$+1$&$+1$&$+1$\\ % &$+1$&$+1$&$+1$&$+1$\\
$+1$&$+1$&$+1$&$+1$&$+1$&$+1$&{\color{green}$\bf{}+1$}&{\color{red}$\bf{}-1$}\\ % &$+1$&$+1$&$+1$&$-1$\\
$-1$&$-1$&$-1$&$-1$&$-1$&$-1$&$-1$&$-1$\\ % &$+1$&$+1$&$+1$&$+1$\\
$+1$&$+1$&$+1$&$+1$&$+1$&$+1$&$+1$&$+1$\\ % &$+1$&$+1$&$+1$&$+1$\\
$-1$&$-1$&$-1$&$-1$&$-1$&$-1$&{\color{green}$\bf{}-1$}&{\color{red}$\bf{}+1$}\\ % &$+1$&$+1$&$+1$&$-1$\\
$-1$&$-1$&$-1$&$-1$&$-1$&$-1$&$-1$&$-1$\\ % &$+1$&$+1$&$+1$&$+1$\\
$-1$&$-1$&$-1$&$-1$&$-1$&$-1$&{\color{green}$\bf{}-1$}&{\color{red}$\bf{}+1$}\\ % &$+1$&$+1$&$+1$&$-1$\\
$-1$&$-1$&$-1$&$-1$&$-1$&$-1$&$-1$&$-1$\\ % &$+1$&$+1$&$+1$&$+1$\\
$-1$&$-1$&$-1$&$-1$&$-1$&$-1$&$-1$&$-1$\\ % &$+1$&$+1$&$+1$&$+1$\\
$-1$&$-1$&$-1$&$-1$&$-1$&$-1$&$-1$&$-1$\\ % &$+1$&$+1$&$+1$&$+1$\\
$+1$&$+1$&$+1$&$+1$&$+1$&$+1$&{\color{green}$\bf{}+1$}&{\color{red}$\bf{}-1$}\\ % &$+1$&$+1$&$+1$&$-1$\\
$-1$&$-1$&$-1$&$-1$&$-1$&$-1$&{\color{green}$\bf{}-1$}&{\color{red}$\bf{}+1$}\\ % &$+1$&$+1$&$+1$&$-1$\\
$-1$&$-1$&$-1$&$-1$&$-1$&$-1$&$-1$&$-1$\\ % &$+1$&$+1$&$+1$&$+1$\\
$-1$&$-1$&$-1$&$-1$&$-1$&$-1$&{\color{green}$\bf{}-1$}&{\color{red}$\bf{}+1$}\\ % &$+1$&$+1$&$+1$&$-1$\\
$-1$&$-1$&$-1$&$-1$&$-1$&$-1$&$-1$&$-1$\\ % &$+1$&$+1$&$+1$&$+1$\\
$+1$&$+1$&$+1$&$+1$&$+1$&$+1$&{\color{green}$\bf{}+1$}&{\color{red}$\bf{}-1$}\\ % &$+1$&$+1$&$+1$&$-1$\\
$-1$&$-1$&$-1$&$-1$&$-1$&$-1$&{\color{green}$\bf{}-1$}&{\color{red}$\bf{}+1$}\\ % &$+1$&$+1$&$+1$&$-1$\\
\hline\hline
\end{tabular}
\end{center}
\caption{(Color online) 20 Counterfactual assignments of contextual (bold) and noncontextual values,
and the associated joint values, rendering an approximation $2.95$ for Tsirel'son's maximal quantum bound
$2\sqrt{2}$ for the CHSH sum.}
\label{2011-enough-t3}
\end{table}

With regards to Kochen-Specker type configurations~\cite{kochen1,cabello-96}
with no two-valued state,
any co-existing set of observables (associated with the configuration)
must breach noncontextuality at least once.
Other  Kochen-Specker type configurations~\cite{kochen1,svozil-ql,CalHerSvo}
still allowing two-valued states,
albeit an insufficient number for a homeomorphic embedding into Boolean algebras,
might require contextual value assignments for quantum statistical reasons;
but this question remains unsolved at present.

In summary, several concrete, quantitative examples of contextual assignments
for co-existing complementary -- and thus strictly counterfactual -- observables
have been given.
The amount of noncontextuality can be characterized quantitatively by
the required relative amount of contextual assignments versus noncontextual ones
reproducing quantum mechanical predictions;
or, alternatively,  by
the required relative amount of contextual assignment versus all assignments.
One may thus consider the average number of contextual assignments per quantum as a criterion.

With regard to the above criteria, as could be expected, Kochen-Specker type configurations
require assignments which violate noncontextuality for every single quantum, whereas Boole-Bell-type
configurations, such as CHSH, would still allow occasional noncontextual assignments.
In this sense, Kochen-Specker-type arguments violate noncontextuality stronger than Boole-Bell-type ones.

These considerations are relevant under the assumption that contextuality is a viable concept
for explaining the
experiments~\cite{hasegawa:230401,cabello:210401,Bartosik-09,PhysRevLett.103.160405,kirch-09}.
As I have argued elsewhere~\cite{svozil-2004-analog,svozil-2006-omni,svozil:040102,svozil_2010-pc09},
this might not be the case; at least contextuality might not be a necessary quantum feature.
In particular the abandonment of quantum omniscience, in the sense that a quantum system can carry information
about its state with regard to only a {\em single} context~\cite{zeil-99},
in conjunction with a {\em context translation principle}~\cite{svozil-2003-garda,svozil-2008-ql}
might yield an alternative approach to the quantum formalism.
Thereby  the many degrees of freedom of the ``quasi-classical'' measurement apparatus
effectively introduce stochasticity in the case
of a mismatch between preparation and measurement context.

Clearly, these considerations have large consequences for the type of randomness that could be rendered by
quantum random number generators based on beam splitters, and on quantum oracles in
general~\cite{svozil-2006-ran,PhysRevA.82.022102},
as context translation schemes may still be deterministic and even computable,
whereas irreducible indeterminism can be postulated only from a complete lawlessness~\cite{zeil-05_nature_ofQuantum}
of the underlying processes.



%\bibliography{svozil}

%merlin.mbs apsrev4-1.bst 2010-07-25 4.21a (PWD, AO, DPC) hacked
%Control: key (0)
%Control: author (0) dotless jnrlst
%Control: editor formatted (1) identically to author
%Control: production of article title (0) allowed
%Control: page (1) range
%Control: year (0) verbatim
%Control: production of eprint (0) enabled
\begin{thebibliography}{40}%
\makeatletter
\providecommand \@ifxundefined [1]{%
 \@ifx{#1\undefined}
}%
\providecommand \@ifnum [1]{%
 \ifnum #1\expandafter \@firstoftwo
 \else \expandafter \@secondoftwo
 \fi
}%
\providecommand \@ifx [1]{%
 \ifx #1\expandafter \@firstoftwo
 \else \expandafter \@secondoftwo
 \fi
}%
\providecommand \natexlab [1]{#1}%
\providecommand \enquote  [1]{``#1''}%
\providecommand \bibnamefont  [1]{#1}%
\providecommand \bibfnamefont [1]{#1}%
\providecommand \citenamefont [1]{#1}%
\providecommand \href@noop [0]{\@secondoftwo}%
\providecommand \href [0]{\begingroup \@sanitize@url \@href}%
\providecommand \@href[1]{\@@startlink{#1}\@@href}%
\providecommand \@@href[1]{\endgroup#1\@@endlink}%
\providecommand \@sanitize@url [0]{\catcode `\\12\catcode `\$12\catcode
  `\&12\catcode `\#12\catcode `\^12\catcode `\_12\catcode `\%12\relax}%
\providecommand \@@startlink[1]{}%
\providecommand \@@endlink[0]{}%
\providecommand \url  [0]{\begingroup\@sanitize@url \@url }%
\providecommand \@url [1]{\endgroup\@href {#1}{\urlprefix }}%
\providecommand \urlprefix  [0]{URL }%
\providecommand \Eprint [0]{\href }%
\providecommand \doibase [0]{http://dx.doi.org/}%
\providecommand \selectlanguage [0]{\@gobble}%
\providecommand \bibinfo  [0]{\@secondoftwo}%
\providecommand \bibfield  [0]{\@secondoftwo}%
\providecommand \translation [1]{[#1]}%
\providecommand \BibitemOpen [0]{}%
\providecommand \bibitemStop [0]{}%
\providecommand \bibitemNoStop [0]{.\EOS\space}%
\providecommand \EOS [0]{\spacefactor3000\relax}%
\providecommand \BibitemShut  [1]{\csname bibitem#1\endcsname}%
\let\auto@bib@innerbib\@empty
%</preamble>
\bibitem [{\citenamefont {Svozil}(2009{\natexlab{a}})}]{svozil-2006-omni}%
  \BibitemOpen
  \bibfield  {author} {\bibinfo {author} {\bibfnamefont {Karl}\ \bibnamefont
  {Svozil}},\ }\bibfield  {title} {\enquote {\bibinfo {title} {Quantum
  scholasticism: On quantum contexts, counterfactuals, and the absurdities of
  quantum omniscience},}\ }\href {\doibase 10.1016/j.ins.2008.06.012}
  {\bibfield  {journal} {\bibinfo  {journal} {Information Sciences}\ }\textbf
  {\bibinfo {volume} {179}},\ \bibinfo {pages} {535--541} (\bibinfo {year}
  {2009}{\natexlab{a}})}\BibitemShut {NoStop}%
\bibitem [{\citenamefont {Vaidman}(2007)}]{vaidman:2009}%
  \BibitemOpen
  \bibfield  {author} {\bibinfo {author} {\bibfnamefont {Lev}\ \bibnamefont
  {Vaidman}},\ }\bibfield  {title} {\enquote {\bibinfo {title} {Counterfactuals
  in quantum mechanics},}\ }in\ \href {\doibase 10.1007/978-3-540-70626-7_40}
  {\emph {\bibinfo {booktitle} {Compendium of Quantum Physics}}},\ \bibinfo
  {editor} {edited by\ \bibinfo {editor} {\bibfnamefont {Daniel}\ \bibnamefont
  {Greenberger}}, \bibinfo {editor} {\bibfnamefont {Klaus}\ \bibnamefont
  {Hentschel}}, \ and\ \bibinfo {editor} {\bibfnamefont {Friedel}\ \bibnamefont
  {Weinert}}}\ (\bibinfo  {publisher} {Springer},\ \bibinfo {address} {Berlin,
  Heidelberg},\ \bibinfo {year} {2007})\ pp.\ \bibinfo {pages} {132--136},\
  \Eprint {http://arxiv.org/abs/arXiv:0709.0340} {arXiv:0709.0340} \BibitemShut
  {NoStop}%
\bibitem [{\citenamefont {Einstein}\ \emph {et~al.}(1935)\citenamefont
  {Einstein}, \citenamefont {Podolsky},\ and\ \citenamefont {Rosen}}]{epr}%
  \BibitemOpen
  \bibfield  {author} {\bibinfo {author} {\bibfnamefont {Albert}\ \bibnamefont
  {Einstein}}, \bibinfo {author} {\bibfnamefont {Boris}\ \bibnamefont
  {Podolsky}}, \ and\ \bibinfo {author} {\bibfnamefont {Nathan}\ \bibnamefont
  {Rosen}},\ }\bibfield  {title} {\enquote {\bibinfo {title} {Can
  quantum-mechanical description of physical reality be considered complete?}}\
  }\href {\doibase 10.1103/PhysRev.47.777} {\bibfield  {journal} {\bibinfo
  {journal} {Physical Review}\ }\textbf {\bibinfo {volume} {47}},\ \bibinfo
  {pages} {777--780} (\bibinfo {year} {1935})}\BibitemShut {NoStop}%
\bibitem [{\citenamefont {Clauser}\ and\ \citenamefont
  {Shimony}(1978)}]{clauser}%
  \BibitemOpen
  \bibfield  {author} {\bibinfo {author} {\bibfnamefont {J.~F.}\ \bibnamefont
  {Clauser}}\ and\ \bibinfo {author} {\bibfnamefont {A.}~\bibnamefont
  {Shimony}},\ }\bibfield  {title} {\enquote {\bibinfo {title} {{B}ell's
  theorem: experimental tests and implications},}\ }\href {\doibase
  10.1088/0034-4885/41/12/002} {\bibfield  {journal} {\bibinfo  {journal}
  {Reports on Progress in Physics}\ }\textbf {\bibinfo {volume} {41}},\
  \bibinfo {pages} {1881--1926} (\bibinfo {year} {1978})}\BibitemShut {NoStop}%
\bibitem [{\citenamefont {Zeilinger}(1999)}]{zeil-99}%
  \BibitemOpen
  \bibfield  {author} {\bibinfo {author} {\bibfnamefont {Anton}\ \bibnamefont
  {Zeilinger}},\ }\bibfield  {title} {\enquote {\bibinfo {title} {A
  foundational principle for quantum mechanics},}\ }\href {\doibase
  10.1023/A:1018820410908} {\bibfield  {journal} {\bibinfo  {journal}
  {Foundations of Physics}\ }\textbf {\bibinfo {volume} {29}},\ \bibinfo
  {pages} {631--643} (\bibinfo {year} {1999})}\BibitemShut {NoStop}%
\bibitem [{\citenamefont {Pitowsky}(1982)}]{pitowsky-82}%
  \BibitemOpen
  \bibfield  {author} {\bibinfo {author} {\bibfnamefont {Itamar}\ \bibnamefont
  {Pitowsky}},\ }\bibfield  {title} {\enquote {\bibinfo {title} {Resolution of
  the {E}instein-{P}odolsky-{R}osen and {B}ell paradoxes},}\ }\href {\doibase
  10.1103/PhysRevLett.48.1299} {\bibfield  {journal} {\bibinfo  {journal}
  {Physical Review Letters}\ }\textbf {\bibinfo {volume} {48}},\ \bibinfo
  {pages} {1299--1302} (\bibinfo {year} {1982})}\BibitemShut {NoStop}%
\bibitem [{\citenamefont {Meyer}(1999)}]{meyer:99}%
  \BibitemOpen
  \bibfield  {author} {\bibinfo {author} {\bibfnamefont {David~A.}\
  \bibnamefont {Meyer}},\ }\bibfield  {title} {\enquote {\bibinfo {title}
  {Finite precision measurement nullifies the {K}ochen-{S}pecker theorem},}\
  }\href {\doibase 10.1103/PhysRevLett.83.3751} {\bibfield  {journal} {\bibinfo
   {journal} {Physical Review Letters}\ }\textbf {\bibinfo {volume} {83}},\
  \bibinfo {pages} {3751--3754} (\bibinfo {year} {1999})},\ \Eprint
  {http://arxiv.org/abs/quant-ph/9905080} {quant-ph/9905080} \BibitemShut
  {NoStop}%
\bibitem [{\citenamefont {Boole}(1862)}]{Boole-62}%
  \BibitemOpen
  \bibfield  {author} {\bibinfo {author} {\bibfnamefont {George}\ \bibnamefont
  {Boole}},\ }\bibfield  {title} {\enquote {\bibinfo {title} {On the theory of
  probabilities},}\ }\href {http://www.jstor.org/stable/108830} {\bibfield
  {journal} {\bibinfo  {journal} {Philosophical Transactions of the Royal
  Society of London}\ }\textbf {\bibinfo {volume} {152}},\ \bibinfo {pages}
  {225--252} (\bibinfo {year} {1862})}\BibitemShut {NoStop}%
\bibitem [{\citenamefont {Froissart}(1981)}]{froissart-81}%
  \BibitemOpen
  \bibfield  {author} {\bibinfo {author} {\bibfnamefont {M.}~\bibnamefont
  {Froissart}},\ }\bibfield  {title} {\enquote {\bibinfo {title} {Constructive
  generalization of {B}ell's inequalities},}\ }\href
  {http://dx.doi.org/10.1007/BF02903286} {\bibfield  {journal} {\bibinfo
  {journal} {Il Nuovo Cimento B (1971-1996)}\ }\textbf {\bibinfo {volume}
  {64}},\ \bibinfo {pages} {241--251} (\bibinfo {year} {1981})},\ \bibinfo
  {note} {10.1007/BF02903286}\BibitemShut {NoStop}%
\bibitem [{\citenamefont {Pitowsky}(1989)}]{pitowsky}%
  \BibitemOpen
  \bibfield  {author} {\bibinfo {author} {\bibfnamefont {Itamar}\ \bibnamefont
  {Pitowsky}},\ }\href@noop {} {\emph {\bibinfo {title} {Quantum
  Probability---Quantum Logic}}}\ (\bibinfo  {publisher} {Springer},\ \bibinfo
  {address} {Berlin},\ \bibinfo {year} {1989})\BibitemShut {NoStop}%
\bibitem [{\citenamefont {{Cirel'son (=Tsirel'son)}}(1993)}]{cirelson}%
  \BibitemOpen
  \bibfield  {author} {\bibinfo {author} {\bibfnamefont {Boris~S.}\
  \bibnamefont {{Cirel'son (=Tsirel'son)}}},\ }\bibfield  {title} {\enquote
  {\bibinfo {title} {Some results and problems on quantum {B}ell-type
  inequalities},}\ }\href {http://www.tau.ac.il/~tsirel/download/hadron.pdf}
  {\bibfield  {journal} {\bibinfo  {journal} {Hadronic Journal Supplement}\
  }\textbf {\bibinfo {volume} {8}},\ \bibinfo {pages} {329--345} (\bibinfo
  {year} {1993})}\BibitemShut {NoStop}%
\bibitem [{\citenamefont {Filipp}\ and\ \citenamefont
  {Svozil}(2004)}]{filipp-svo-04-qpoly-prl}%
  \BibitemOpen
  \bibfield  {author} {\bibinfo {author} {\bibfnamefont {Stefan}\ \bibnamefont
  {Filipp}}\ and\ \bibinfo {author} {\bibfnamefont {Karl}\ \bibnamefont
  {Svozil}},\ }\bibfield  {title} {\enquote {\bibinfo {title} {Generalizing
  {T}sirelson's bound on {B}ell inequalities using a min-max principle},}\
  }\href {\doibase 10.1103/PhysRevLett.93.130407} {\bibfield  {journal}
  {\bibinfo  {journal} {Physical Review Letters}\ }\textbf {\bibinfo {volume}
  {93}},\ \bibinfo {pages} {130407} (\bibinfo {year} {2004})},\ \Eprint
  {http://arxiv.org/abs/quant-ph/0403175} {quant-ph/0403175} \BibitemShut
  {NoStop}%
\bibitem [{\citenamefont {Popescu}\ and\ \citenamefont
  {Rohrlich}(1994)}]{pop-rohr}%
  \BibitemOpen
  \bibfield  {author} {\bibinfo {author} {\bibfnamefont {S.}~\bibnamefont
  {Popescu}}\ and\ \bibinfo {author} {\bibfnamefont {D.}~\bibnamefont
  {Rohrlich}},\ }\bibfield  {title} {\enquote {\bibinfo {title} {Quantum
  nonlocality as an axiom},}\ }\href {\doibase 10.1007/BF02058098} {\bibfield
  {journal} {\bibinfo  {journal} {Foundations of Physics}\ }\textbf {\bibinfo
  {volume} {24}},\ \bibinfo {pages} {379--358} (\bibinfo {year}
  {1994})}\BibitemShut {NoStop}%
\bibitem [{\citenamefont {Krenn}\ and\ \citenamefont
  {Svozil}(1998)}]{svozil-krenn}%
  \BibitemOpen
  \bibfield  {author} {\bibinfo {author} {\bibfnamefont {G{\"{u}}nther}\
  \bibnamefont {Krenn}}\ and\ \bibinfo {author} {\bibfnamefont {Karl}\
  \bibnamefont {Svozil}},\ }\bibfield  {title} {\enquote {\bibinfo {title}
  {Stronger-than-quantum correlations},}\ }\href {\doibase
  10.1023/A:1018821314465} {\bibfield  {journal} {\bibinfo  {journal}
  {Foundations of Physics}\ }\textbf {\bibinfo {volume} {28}},\ \bibinfo
  {pages} {971--984} (\bibinfo {year} {1998})}\BibitemShut {NoStop}%
\bibitem [{\citenamefont {Kochen}\ and\ \citenamefont
  {Specker}(1967)}]{kochen1}%
  \BibitemOpen
  \bibfield  {author} {\bibinfo {author} {\bibfnamefont {Simon}\ \bibnamefont
  {Kochen}}\ and\ \bibinfo {author} {\bibfnamefont {Ernst~P.}\ \bibnamefont
  {Specker}},\ }\bibfield  {title} {\enquote {\bibinfo {title} {The problem of
  hidden variables in quantum mechanics},}\ }\href {\doibase
  10.1512/iumj.1968.17.17004} {\bibfield  {journal} {\bibinfo  {journal}
  {Journal of Mathematics and Mechanics (now Indiana University Mathematics
  Journal)}\ }\textbf {\bibinfo {volume} {17}},\ \bibinfo {pages} {59--87}
  (\bibinfo {year} {1967})}\BibitemShut {NoStop}%
\bibitem [{\citenamefont {Bohr}(1949)}]{bohr-1949}%
  \BibitemOpen
  \bibfield  {author} {\bibinfo {author} {\bibfnamefont {Niels}\ \bibnamefont
  {Bohr}},\ }\bibfield  {title} {\enquote {\bibinfo {title} {Discussion with
  {E}instein on epistemological problems in atomic physics},}\ }in\ \href
  {\doibase 10.1016/S1876-0503(08)70379-7} {\emph {\bibinfo {booktitle}
  {{A}lbert {E}instein: Philosopher-Scientist}}},\ \bibinfo {editor} {edited
  by\ \bibinfo {editor} {\bibfnamefont {P.~A.}\ \bibnamefont {Schilpp}}}\
  (\bibinfo  {publisher} {The Library of Living Philosophers},\ \bibinfo
  {address} {Evanston, Ill.},\ \bibinfo {year} {1949})\ pp.\ \bibinfo {pages}
  {200--241}\BibitemShut {NoStop}%
\bibitem [{\citenamefont {Bell}(1966)}]{bell-66}%
  \BibitemOpen
  \bibfield  {author} {\bibinfo {author} {\bibfnamefont {John~S.}\ \bibnamefont
  {Bell}},\ }\bibfield  {title} {\enquote {\bibinfo {title} {On the problem of
  hidden variables in quantum mechanics},}\ }\href {\doibase
  10.1103/RevModPhys.38.447} {\bibfield  {journal} {\bibinfo  {journal}
  {Reviews of Modern Physics}\ }\textbf {\bibinfo {volume} {38}},\ \bibinfo
  {pages} {447--452} (\bibinfo {year} {1966})}\BibitemShut {NoStop}%
\bibitem [{\citenamefont {Svozil}(2009{\natexlab{b}})}]{svozil:040102}%
  \BibitemOpen
  \bibfield  {author} {\bibinfo {author} {\bibfnamefont {Karl}\ \bibnamefont
  {Svozil}},\ }\bibfield  {title} {\enquote {\bibinfo {title} {Proposed direct
  test of a certain type of noncontextuality in quantum mechanics},}\ }\href
  {\doibase 10.1103/PhysRevA.80.040102} {\bibfield  {journal} {\bibinfo
  {journal} {Physical Review A}\ }\textbf {\bibinfo {volume} {80}},\ \bibinfo
  {eid} {040102} (\bibinfo {year} {2009}{\natexlab{b}})}\BibitemShut {NoStop}%
\bibitem [{\citenamefont {Cabello}(2008)}]{cabello:210401}%
  \BibitemOpen
  \bibfield  {author} {\bibinfo {author} {\bibfnamefont {Ad\'an}\ \bibnamefont
  {Cabello}},\ }\bibfield  {title} {\enquote {\bibinfo {title} {Experimentally
  testable state-independent quantum contextuality},}\ }\href {\doibase
  10.1103/PhysRevLett.101.210401} {\bibfield  {journal} {\bibinfo  {journal}
  {Physical Review Letters}\ }\textbf {\bibinfo {volume} {101}},\ \bibinfo
  {eid} {210401} (\bibinfo {year} {2008})}\BibitemShut {NoStop}%
\bibitem [{\citenamefont {Peres}(1978)}]{peres222}%
  \BibitemOpen
  \bibfield  {author} {\bibinfo {author} {\bibfnamefont {Asher}\ \bibnamefont
  {Peres}},\ }\bibfield  {title} {\enquote {\bibinfo {title} {Unperformed
  experiments have no results},}\ }\href {\doibase 10.1119/1.11393} {\bibfield
  {journal} {\bibinfo  {journal} {American Journal of Physics}\ }\textbf
  {\bibinfo {volume} {46}},\ \bibinfo {pages} {745--747} (\bibinfo {year}
  {1978})}\BibitemShut {NoStop}%
\bibitem [{\citenamefont {Svozil}(2010)}]{svozil_2010-pc09}%
  \BibitemOpen
  \bibfield  {author} {\bibinfo {author} {\bibfnamefont {Karl}\ \bibnamefont
  {Svozil}},\ }\bibfield  {title} {\enquote {\bibinfo {title} {Quantum value
  indefiniteness},}\ }\href {\doibase 10.1007/s11047-010-9241-x} {\bibfield
  {journal} {\bibinfo  {journal} {Natural Computing}\ }\textbf {\bibinfo
  {volume} {online first}},\ \bibinfo {pages} {1--12} (\bibinfo {year}
  {2010})},\ \Eprint {http://arxiv.org/abs/arXiv:1001.1436} {arXiv:1001.1436}
  \BibitemShut {NoStop}%
\bibitem [{\citenamefont {Svozil}(2009{\natexlab{c}})}]{svozil-2008-ql}%
  \BibitemOpen
  \bibfield  {author} {\bibinfo {author} {\bibfnamefont {Karl}\ \bibnamefont
  {Svozil}},\ }\bibfield  {title} {\enquote {\bibinfo {title} {Contexts in
  quantum, classical and partition logic},}\ }in\ \href
  {http://arxiv.org/abs/quant-ph/0609209} {\emph {\bibinfo {booktitle}
  {Handbook of Quantum Logic and Quantum Structures}}},\ \bibinfo {editor}
  {edited by\ \bibinfo {editor} {\bibfnamefont {Kurt}\ \bibnamefont
  {Engesser}}, \bibinfo {editor} {\bibfnamefont {Dov~M.}\ \bibnamefont
  {Gabbay}}, \ and\ \bibinfo {editor} {\bibfnamefont {Daniel}\ \bibnamefont
  {Lehmann}}}\ (\bibinfo  {publisher} {Elsevier},\ \bibinfo {address}
  {Amsterdam},\ \bibinfo {year} {2009})\ pp.\ \bibinfo {pages} {551--586},\
  \Eprint {http://arxiv.org/abs/arXiv:quant-ph/0609209}
  {arXiv:quant-ph/0609209} \BibitemShut {NoStop}%
\bibitem [{\citenamefont {Clauser}\ \emph {et~al.}(1969)\citenamefont
  {Clauser}, \citenamefont {Horne}, \citenamefont {Shimony},\ and\
  \citenamefont {Holt}}]{chsh}%
  \BibitemOpen
  \bibfield  {author} {\bibinfo {author} {\bibfnamefont {John~F.}\ \bibnamefont
  {Clauser}}, \bibinfo {author} {\bibfnamefont {Michael~A.}\ \bibnamefont
  {Horne}}, \bibinfo {author} {\bibfnamefont {Abner}\ \bibnamefont {Shimony}},
  \ and\ \bibinfo {author} {\bibfnamefont {Richard~A.}\ \bibnamefont {Holt}},\
  }\bibfield  {title} {\enquote {\bibinfo {title} {Proposed experiment to test
  local hidden-variable theories},}\ }\href {\doibase
  10.1103/PhysRevLett.23.880} {\bibfield  {journal} {\bibinfo  {journal}
  {Physical Review Letters}\ }\textbf {\bibinfo {volume} {23}},\ \bibinfo
  {pages} {880--884} (\bibinfo {year} {1969})}\BibitemShut {NoStop}%
\bibitem [{\citenamefont {{Cirel'son (=Tsirel'son)}}(1980)}]{cirelson:80}%
  \BibitemOpen
  \bibfield  {author} {\bibinfo {author} {\bibfnamefont {Boris~S.}\
  \bibnamefont {{Cirel'son (=Tsirel'son)}}},\ }\bibfield  {title} {\enquote
  {\bibinfo {title} {Quantum generalizations of {B}ell's inequality},}\ }\href
  {http://www.tau.ac.il/~tsirel/download/qbell80.pdf} {\bibfield  {journal}
  {\bibinfo  {journal} {Letters in Mathematical Physics}\ }\textbf {\bibinfo
  {volume} {4}},\ \bibinfo {pages} {93--100} (\bibinfo {year}
  {1980})}\BibitemShut {NoStop}%
\bibitem [{\citenamefont {Popescu}\ and\ \citenamefont
  {Rohrlich}(1997)}]{popescu-97}%
  \BibitemOpen
  \bibfield  {author} {\bibinfo {author} {\bibfnamefont {Sandu}\ \bibnamefont
  {Popescu}}\ and\ \bibinfo {author} {\bibfnamefont {Daniel}\ \bibnamefont
  {Rohrlich}},\ }\bibfield  {title} {\enquote {\bibinfo {title} {Action and
  passion at a distance},}\ }in\ \href {http://arxiv.org/abs/quant-ph/9605004}
  {\emph {\bibinfo {booktitle} {Potentiality, Entanglement and
  Passion-at-a-Distance: Quantum Mechanical Studies for Abner Shimony, Volume
  Two (Boston Studies in the Philosophy of Science)}}},\ \bibinfo {editor}
  {edited by\ \bibinfo {editor} {\bibfnamefont {R.~S.}\ \bibnamefont {Cohen}},
  \bibinfo {editor} {\bibfnamefont {M.~A.}\ \bibnamefont {Horne}}, \ and\
  \bibinfo {editor} {\bibfnamefont {J.}~\bibnamefont {Stachel}}}\ (\bibinfo
  {publisher} {Kluwer Academic publishers},\ \bibinfo {address} {Dordrecht},\
  \bibinfo {year} {1997})\ pp.\ \bibinfo {pages} {197--206},\ \Eprint
  {http://arxiv.org/abs/quant-ph/9605004} {quant-ph/9605004} \BibitemShut
  {NoStop}%
\bibitem [{\citenamefont {Barrett}\ \emph {et~al.}(2005)\citenamefont
  {Barrett}, \citenamefont {Linden}, \citenamefont {Massar}, \citenamefont
  {Pironio}, \citenamefont {Popescu},\ and\ \citenamefont
  {Roberts}}]{PhysRevA.71.022101}%
  \BibitemOpen
  \bibfield  {author} {\bibinfo {author} {\bibfnamefont {Jonathan}\
  \bibnamefont {Barrett}}, \bibinfo {author} {\bibfnamefont {Noah}\
  \bibnamefont {Linden}}, \bibinfo {author} {\bibfnamefont {Serge}\
  \bibnamefont {Massar}}, \bibinfo {author} {\bibfnamefont {Stefano}\
  \bibnamefont {Pironio}}, \bibinfo {author} {\bibfnamefont {Sandu}\
  \bibnamefont {Popescu}}, \ and\ \bibinfo {author} {\bibfnamefont {David}\
  \bibnamefont {Roberts}},\ }\bibfield  {title} {\enquote {\bibinfo {title}
  {Nonlocal correlations as an information-theoretic resource},}\ }\href
  {\doibase 10.1103/PhysRevA.71.022101} {\bibfield  {journal} {\bibinfo
  {journal} {Physical Review A}\ }\textbf {\bibinfo {volume} {71}},\ \bibinfo
  {pages} {022101} (\bibinfo {year} {2005})}\BibitemShut {NoStop}%
\bibitem [{\citenamefont
  {Svozil}(2005{\natexlab{a}})}]{svozil-2004-brainteaser}%
  \BibitemOpen
  \bibfield  {author} {\bibinfo {author} {\bibfnamefont {Karl}\ \bibnamefont
  {Svozil}},\ }\bibfield  {title} {\enquote {\bibinfo {title} {Communication
  cost of breaking the {B}ell barrier},}\ }\href {\doibase
  10.1103/PhysRevA.72.050302} {\bibfield  {journal} {\bibinfo  {journal}
  {Physical Review A}\ }\textbf {\bibinfo {volume} {72}},\ \bibinfo {pages}
  {050302(R)} (\bibinfo {year} {2005}{\natexlab{a}})},\ \Eprint
  {http://arxiv.org/abs/physics/0510050} {physics/0510050} \BibitemShut
  {NoStop}%
\bibitem [{\citenamefont {Ziegler}(1994)}]{ziegler}%
  \BibitemOpen
  \bibfield  {author} {\bibinfo {author} {\bibfnamefont {G{\"{u}}nter~M.}\
  \bibnamefont {Ziegler}},\ }\href@noop {} {\emph {\bibinfo {title} {Lectures
  on Polytopes}}}\ (\bibinfo  {publisher} {Springer},\ \bibinfo {address} {New
  York},\ \bibinfo {year} {1994})\BibitemShut {NoStop}%
\bibitem [{\citenamefont {Cabello}\ \emph {et~al.}(1996)\citenamefont
  {Cabello}, \citenamefont {Estebaranz},\ and\ \citenamefont
  {Garc{\'{i}}a-Alcaine}}]{cabello-96}%
  \BibitemOpen
  \bibfield  {author} {\bibinfo {author} {\bibfnamefont {Ad{\'{a}}n}\
  \bibnamefont {Cabello}}, \bibinfo {author} {\bibfnamefont {Jos{\'{e}}~M.}\
  \bibnamefont {Estebaranz}}, \ and\ \bibinfo {author} {\bibfnamefont
  {G.}~\bibnamefont {Garc{\'{i}}a-Alcaine}},\ }\bibfield  {title} {\enquote
  {\bibinfo {title} {{B}ell-{K}ochen-{S}pecker theorem: A proof with 18
  vectors},}\ }\href {\doibase 10.1016/0375-9601(96)00134-X} {\bibfield
  {journal} {\bibinfo  {journal} {Physics Letters A}\ }\textbf {\bibinfo
  {volume} {212}},\ \bibinfo {pages} {183--187} (\bibinfo {year}
  {1996})}\BibitemShut {NoStop}%
\bibitem [{\citenamefont {Svozil}(1998)}]{svozil-ql}%
  \BibitemOpen
  \bibfield  {author} {\bibinfo {author} {\bibfnamefont {Karl}\ \bibnamefont
  {Svozil}},\ }\href@noop {} {\emph {\bibinfo {title} {Quantum Logic}}}\
  (\bibinfo  {publisher} {Springer},\ \bibinfo {address} {Singapore},\ \bibinfo
  {year} {1998})\BibitemShut {NoStop}%
\bibitem [{\citenamefont {Calude}\ \emph {et~al.}(1999)\citenamefont {Calude},
  \citenamefont {Hertling},\ and\ \citenamefont {Svozil}}]{CalHerSvo}%
  \BibitemOpen
  \bibfield  {author} {\bibinfo {author} {\bibfnamefont {Cristian}\
  \bibnamefont {Calude}}, \bibinfo {author} {\bibfnamefont {Peter}\
  \bibnamefont {Hertling}}, \ and\ \bibinfo {author} {\bibfnamefont {Karl}\
  \bibnamefont {Svozil}},\ }\bibfield  {title} {\enquote {\bibinfo {title}
  {Embedding quantum universes into classical ones},}\ }\href {\doibase
  10.1023/A:1018862730956} {\bibfield  {journal} {\bibinfo  {journal}
  {Foundations of Physics}\ }\textbf {\bibinfo {volume} {29}},\ \bibinfo
  {pages} {349--379} (\bibinfo {year} {1999})}\BibitemShut {NoStop}%
\bibitem [{\citenamefont {Hasegawa}\ \emph {et~al.}(2006)\citenamefont
  {Hasegawa}, \citenamefont {Loidl}, \citenamefont {Badurek}, \citenamefont
  {Baron},\ and\ \citenamefont {Rauch}}]{hasegawa:230401}%
  \BibitemOpen
  \bibfield  {author} {\bibinfo {author} {\bibfnamefont {Yuji}\ \bibnamefont
  {Hasegawa}}, \bibinfo {author} {\bibfnamefont {Rudolf}\ \bibnamefont
  {Loidl}}, \bibinfo {author} {\bibfnamefont {Gerald}\ \bibnamefont {Badurek}},
  \bibinfo {author} {\bibfnamefont {Matthias}\ \bibnamefont {Baron}}, \ and\
  \bibinfo {author} {\bibfnamefont {Helmut}\ \bibnamefont {Rauch}},\ }\bibfield
   {title} {\enquote {\bibinfo {title} {Quantum contextuality in a
  single-neutron optical experiment},}\ }\href {\doibase
  10.1103/PhysRevLett.97.230401} {\bibfield  {journal} {\bibinfo  {journal}
  {Physical Review Letters}\ }\textbf {\bibinfo {volume} {97}},\ \bibinfo {eid}
  {230401} (\bibinfo {year} {2006})}\BibitemShut {NoStop}%
\bibitem [{\citenamefont {Bartosik}\ \emph {et~al.}(2009)\citenamefont
  {Bartosik}, \citenamefont {Klepp}, \citenamefont {Schmitzer}, \citenamefont
  {Sponar}, \citenamefont {Cabello}, \citenamefont {Rauch},\ and\ \citenamefont
  {Hasegawa}}]{Bartosik-09}%
  \BibitemOpen
  \bibfield  {author} {\bibinfo {author} {\bibfnamefont {H.}~\bibnamefont
  {Bartosik}}, \bibinfo {author} {\bibfnamefont {J.}~\bibnamefont {Klepp}},
  \bibinfo {author} {\bibfnamefont {C.}~\bibnamefont {Schmitzer}}, \bibinfo
  {author} {\bibfnamefont {S.}~\bibnamefont {Sponar}}, \bibinfo {author}
  {\bibfnamefont {A.}~\bibnamefont {Cabello}}, \bibinfo {author} {\bibfnamefont
  {H.}~\bibnamefont {Rauch}}, \ and\ \bibinfo {author} {\bibfnamefont
  {Y.}~\bibnamefont {Hasegawa}},\ }\bibfield  {title} {\enquote {\bibinfo
  {title} {Experimental test of quantum contextuality in neutron
  interferometry},}\ }\href {\doibase 10.1103/PhysRevLett.103.040403}
  {\bibfield  {journal} {\bibinfo  {journal} {Physical Review Letters}\
  }\textbf {\bibinfo {volume} {103}},\ \bibinfo {pages} {040403} (\bibinfo
  {year} {2009})},\ \Eprint {http://arxiv.org/abs/arXiv:0904.4576}
  {arXiv:0904.4576} \BibitemShut {NoStop}%
\bibitem [{\citenamefont {Amselem}\ \emph {et~al.}(2009)\citenamefont
  {Amselem}, \citenamefont {R\aa{}dmark}, \citenamefont {Bourennane},\ and\
  \citenamefont {Cabello}}]{PhysRevLett.103.160405}%
  \BibitemOpen
  \bibfield  {author} {\bibinfo {author} {\bibfnamefont {Elias}\ \bibnamefont
  {Amselem}}, \bibinfo {author} {\bibfnamefont {Magnus}\ \bibnamefont
  {R\aa{}dmark}}, \bibinfo {author} {\bibfnamefont {Mohamed}\ \bibnamefont
  {Bourennane}}, \ and\ \bibinfo {author} {\bibfnamefont {Ad\'an}\ \bibnamefont
  {Cabello}},\ }\bibfield  {title} {\enquote {\bibinfo {title}
  {State-independent quantum contextuality with single photons},}\ }\href
  {\doibase 10.1103/PhysRevLett.103.160405} {\bibfield  {journal} {\bibinfo
  {journal} {Physical Review Letters}\ }\textbf {\bibinfo {volume} {103}},\
  \bibinfo {pages} {160405} (\bibinfo {year} {2009})}\BibitemShut {NoStop}%
\bibitem [{\citenamefont {Kirchmair}\ \emph {et~al.}(2009)\citenamefont
  {Kirchmair}, \citenamefont {Z{\"{a}}hringer}, \citenamefont {Gerritsma},
  \citenamefont {Kleinmann}, \citenamefont {G{\"{u}}hne}, \citenamefont
  {Cabello}, \citenamefont {Blatt},\ and\ \citenamefont {Roos}}]{kirch-09}%
  \BibitemOpen
  \bibfield  {author} {\bibinfo {author} {\bibfnamefont {G.}~\bibnamefont
  {Kirchmair}}, \bibinfo {author} {\bibfnamefont {F.}~\bibnamefont
  {Z{\"{a}}hringer}}, \bibinfo {author} {\bibfnamefont {R.}~\bibnamefont
  {Gerritsma}}, \bibinfo {author} {\bibfnamefont {M.}~\bibnamefont
  {Kleinmann}}, \bibinfo {author} {\bibfnamefont {O.}~\bibnamefont
  {G{\"{u}}hne}}, \bibinfo {author} {\bibfnamefont {A.}~\bibnamefont
  {Cabello}}, \bibinfo {author} {\bibfnamefont {R.}~\bibnamefont {Blatt}}, \
  and\ \bibinfo {author} {\bibfnamefont {C.~F.}\ \bibnamefont {Roos}},\
  }\bibfield  {title} {\enquote {\bibinfo {title} {State-independent
  experimental test of quantum contextuality},}\ }\href {\doibase
  10.1038/nature08172} {\bibfield  {journal} {\bibinfo  {journal} {Nature}\
  }\textbf {\bibinfo {volume} {460}},\ \bibinfo {pages} {494--497} (\bibinfo
  {year} {2009})},\ \Eprint {http://arxiv.org/abs/arXiv:0904.1655}
  {arXiv:0904.1655} \BibitemShut {NoStop}%
\bibitem [{\citenamefont {Svozil}(2005{\natexlab{b}})}]{svozil-2004-analog}%
  \BibitemOpen
  \bibfield  {author} {\bibinfo {author} {\bibfnamefont {Karl}\ \bibnamefont
  {Svozil}},\ }\bibfield  {title} {\enquote {\bibinfo {title} {Noncontextuality
  in multipartite entanglement},}\ }\href {\doibase
  10.1088/0305-4470/38/25/013} {\bibfield  {journal} {\bibinfo  {journal} {J.
  Phys. A: Math. Gen.}\ }\textbf {\bibinfo {volume} {38}},\ \bibinfo {pages}
  {5781--5798} (\bibinfo {year} {2005}{\natexlab{b}})},\ \Eprint
  {http://arxiv.org/abs/quant-ph/0401113} {quant-ph/0401113} \BibitemShut
  {NoStop}%
\bibitem [{\citenamefont {Svozil}(2004)}]{svozil-2003-garda}%
  \BibitemOpen
  \bibfield  {author} {\bibinfo {author} {\bibfnamefont {Karl}\ \bibnamefont
  {Svozil}},\ }\bibfield  {title} {\enquote {\bibinfo {title} {Quantum
  information via state partitions and the context translation principle},}\
  }\href {\doibase 10.1080/09500340410001664179} {\bibfield  {journal}
  {\bibinfo  {journal} {Journal of Modern Optics}\ }\textbf {\bibinfo {volume}
  {51}},\ \bibinfo {pages} {811--819} (\bibinfo {year} {2004})},\ \Eprint
  {http://arxiv.org/abs/quant-ph/0308110} {quant-ph/0308110} \BibitemShut
  {NoStop}%
\bibitem [{\citenamefont {Calude}\ and\ \citenamefont
  {Svozil}(2008)}]{svozil-2006-ran}%
  \BibitemOpen
  \bibfield  {author} {\bibinfo {author} {\bibfnamefont {Cristian~S.}\
  \bibnamefont {Calude}}\ and\ \bibinfo {author} {\bibfnamefont {Karl}\
  \bibnamefont {Svozil}},\ }\bibfield  {title} {\enquote {\bibinfo {title}
  {Quantum randomness and value indefiniteness},}\ }\href {\doibase
  10.1166/asl.2008.016} {\bibfield  {journal} {\bibinfo  {journal} {Advanced
  Science Letters}\ }\textbf {\bibinfo {volume} {1}},\ \bibinfo {pages}
  {165--168} (\bibinfo {year} {2008})},\ \bibinfo {note} {eprint
  arXiv:quant-ph/0611029},\ \Eprint
  {http://arxiv.org/abs/arXiv:quant-ph/0611029} {arXiv:quant-ph/0611029}
  \BibitemShut {NoStop}%
\bibitem [{\citenamefont {Calude}\ \emph {et~al.}(2010)\citenamefont {Calude},
  \citenamefont {Dinneen}, \citenamefont {Dumitrescu},\ and\ \citenamefont
  {Svozil}}]{PhysRevA.82.022102}%
  \BibitemOpen
  \bibfield  {author} {\bibinfo {author} {\bibfnamefont {Cristian~S.}\
  \bibnamefont {Calude}}, \bibinfo {author} {\bibfnamefont {Michael~J.}\
  \bibnamefont {Dinneen}}, \bibinfo {author} {\bibfnamefont {Monica}\
  \bibnamefont {Dumitrescu}}, \ and\ \bibinfo {author} {\bibfnamefont {Karl}\
  \bibnamefont {Svozil}},\ }\bibfield  {title} {\enquote {\bibinfo {title}
  {Experimental evidence of quantum randomness incomputability},}\ }\href
  {\doibase 10.1103/PhysRevA.82.022102} {\bibfield  {journal} {\bibinfo
  {journal} {Phys. Rev. A}\ }\textbf {\bibinfo {volume} {82}},\ \bibinfo
  {pages} {022102} (\bibinfo {year} {2010})}\BibitemShut {NoStop}%
\bibitem [{\citenamefont {Zeilinger}(2005)}]{zeil-05_nature_ofQuantum}%
  \BibitemOpen
  \bibfield  {author} {\bibinfo {author} {\bibfnamefont {Anton}\ \bibnamefont
  {Zeilinger}},\ }\bibfield  {title} {\enquote {\bibinfo {title} {The message
  of the quantum},}\ }\href {\doibase 10.1038/438743a} {\bibfield  {journal}
  {\bibinfo  {journal} {Nature}\ }\textbf {\bibinfo {volume} {438}},\ \bibinfo
  {pages} {743} (\bibinfo {year} {2005})}\BibitemShut {NoStop}%
\end{thebibliography}%

\end{document}

~~~~~~~~~~~~~~~~~~~~~~~~~~~~~~~~~~~~~~~~~~~~~~~~~~~~~~~~~~~~~~~~~~~~~~~~~~~~~~~~~~~~~~~~~~~~~~~

MatrixForm[ Flatten[Table[{ab, abb, aab, aabb, ba, baa, bba, bbaa, ab*ba, abb*bba, aab*baa, aabb*bbaa}, {ab, -1, 1, 2}, {abb, -1, 1, 2}, {aab, -1, 1, 2}, {aabb, -1, 1, 2}, {ba, -1, 1, 2}, {baa, -1, 1, 2}, {bba, -1, 1, 2}, {bbaa, -1, 1, 2}], 7]]

~~~~~~~~~~~~~~~~~~~~~~~~~~~~~~~~~~~~~~~~~~~~~~~~~~~~~~~~~~~~~~~~~~~~~~~~~~~~~~~~~~~~~~~~~~~~~~~

http://www.ifor.math.ethz.ch/~fukuda/cdd_home/index.html

V-representation
begin
   256  13  integer
1  -1   -1   -1   -1   -1   -1   -1   -1   1   1   1   1
1  -1   -1   -1   -1   -1   -1   -1   1   1   1   1   -1
1  -1   -1   -1   -1   -1   -1   1   -1   1   -1   1   1
1  -1   -1   -1   -1   -1   -1   1   1   1   -1   1   -1
1  -1   -1   -1   -1   -1   1   -1   -1   1   1   -1   1
1  -1   -1   -1   -1   -1   1   -1   1   1   1   -1   -1
1  -1   -1   -1   -1   -1   1   1   -1   1   -1   -1   1
1  -1   -1   -1   -1   -1   1   1   1   1   -1   -1   -1
1  -1   -1   -1   -1   1   -1   -1   -1   -1   1   1   1
1  -1   -1   -1   -1   1   -1   -1   1   -1   1   1   -1
1  -1   -1   -1   -1   1   -1   1   -1   -1   -1   1   1
1  -1   -1   -1   -1   1   -1   1   1   -1   -1   1   -1
1  -1   -1   -1   -1   1   1   -1   -1   -1   1   -1   1
1  -1   -1   -1   -1   1   1   -1   1   -1   1   -1   -1
1  -1   -1   -1   -1   1   1   1   -1   -1   -1   -1   1
1  -1   -1   -1   -1   1   1   1   1   -1   -1   -1   -1
1  -1   -1   -1   1   -1   -1   -1   -1   1   1   1   -1
1  -1   -1   -1   1   -1   -1   -1   1   1   1   1   1
1  -1   -1   -1   1   -1   -1   1   -1   1   -1   1   -1
1  -1   -1   -1   1   -1   -1   1   1   1   -1   1   1
1  -1   -1   -1   1   -1   1   -1   -1   1   1   -1   -1
1  -1   -1   -1   1   -1   1   -1   1   1   1   -1   1
1  -1   -1   -1   1   -1   1   1   -1   1   -1   -1   -1
1  -1   -1   -1   1   -1   1   1   1   1   -1   -1   1
1  -1   -1   -1   1   1   -1   -1   -1   -1   1   1   -1
1  -1   -1   -1   1   1   -1   -1   1   -1   1   1   1
1  -1   -1   -1   1   1   -1   1   -1   -1   -1   1   -1
1  -1   -1   -1   1   1   -1   1   1   -1   -1   1   1
1  -1   -1   -1   1   1   1   -1   -1   -1   1   -1   -1
1  -1   -1   -1   1   1   1   -1   1   -1   1   -1   1
1  -1   -1   -1   1   1   1   1   -1   -1   -1   -1   -1
1  -1   -1   -1   1   1   1   1   1   -1   -1   -1   1
1  -1   -1   1   -1   -1   -1   -1   -1   1   1   -1   1
1  -1   -1   1   -1   -1   -1   -1   1   1   1   -1   -1
1  -1   -1   1   -1   -1   -1   1   -1   1   -1   -1   1
1  -1   -1   1   -1   -1   -1   1   1   1   -1   -1   -1
1  -1   -1   1   -1   -1   1   -1   -1   1   1   1   1
1  -1   -1   1   -1   -1   1   -1   1   1   1   1   -1
1  -1   -1   1   -1   -1   1   1   -1   1   -1   1   1
1  -1   -1   1   -1   -1   1   1   1   1   -1   1   -1
1  -1   -1   1   -1   1   -1   -1   -1   -1   1   -1   1
1  -1   -1   1   -1   1   -1   -1   1   -1   1   -1   -1
1  -1   -1   1   -1   1   -1   1   -1   -1   -1   -1   1
1  -1   -1   1   -1   1   -1   1   1   -1   -1   -1   -1
1  -1   -1   1   -1   1   1   -1   -1   -1   1   1   1
1  -1   -1   1   -1   1   1   -1   1   -1   1   1   -1
1  -1   -1   1   -1   1   1   1   -1   -1   -1   1   1
1  -1   -1   1   -1   1   1   1   1   -1   -1   1   -1
1  -1   -1   1   1   -1   -1   -1   -1   1   1   -1   -1
1  -1   -1   1   1   -1   -1   -1   1   1   1   -1   1
1  -1   -1   1   1   -1   -1   1   -1   1   -1   -1   -1
1  -1   -1   1   1   -1   -1   1   1   1   -1   -1   1
1  -1   -1   1   1   -1   1   -1   -1   1   1   1   -1
1  -1   -1   1   1   -1   1   -1   1   1   1   1   1
1  -1   -1   1   1   -1   1   1   -1   1   -1   1   -1
1  -1   -1   1   1   -1   1   1   1   1   -1   1   1
1  -1   -1   1   1   1   -1   -1   -1   -1   1   -1   -1
1  -1   -1   1   1   1   -1   -1   1   -1   1   -1   1
1  -1   -1   1   1   1   -1   1   -1   -1   -1   -1   -1
1  -1   -1   1   1   1   -1   1   1   -1   -1   -1   1
1  -1   -1   1   1   1   1   -1   -1   -1   1   1   -1
1  -1   -1   1   1   1   1   -1   1   -1   1   1   1
1  -1   -1   1   1   1   1   1   -1   -1   -1   1   -1
1  -1   -1   1   1   1   1   1   1   -1   -1   1   1
1  -1   1   -1   -1   -1   -1   -1   -1   1   -1   1   1
1  -1   1   -1   -1   -1   -1   -1   1   1   -1   1   -1
1  -1   1   -1   -1   -1   -1   1   -1   1   1   1   1
1  -1   1   -1   -1   -1   -1   1   1   1   1   1   -1
1  -1   1   -1   -1   -1   1   -1   -1   1   -1   -1   1
1  -1   1   -1   -1   -1   1   -1   1   1   -1   -1   -1
1  -1   1   -1   -1   -1   1   1   -1   1   1   -1   1
1  -1   1   -1   -1   -1   1   1   1   1   1   -1   -1
1  -1   1   -1   -1   1   -1   -1   -1   -1   -1   1   1
1  -1   1   -1   -1   1   -1   -1   1   -1   -1   1   -1
1  -1   1   -1   -1   1   -1   1   -1   -1   1   1   1
1  -1   1   -1   -1   1   -1   1   1   -1   1   1   -1
1  -1   1   -1   -1   1   1   -1   -1   -1   -1   -1   1
1  -1   1   -1   -1   1   1   -1   1   -1   -1   -1   -1
1  -1   1   -1   -1   1   1   1   -1   -1   1   -1   1
1  -1   1   -1   -1   1   1   1   1   -1   1   -1   -1
1  -1   1   -1   1   -1   -1   -1   -1   1   -1   1   -1
1  -1   1   -1   1   -1   -1   -1   1   1   -1   1   1
1  -1   1   -1   1   -1   -1   1   -1   1   1   1   -1
1  -1   1   -1   1   -1   -1   1   1   1   1   1   1
1  -1   1   -1   1   -1   1   -1   -1   1   -1   -1   -1
1  -1   1   -1   1   -1   1   -1   1   1   -1   -1   1
1  -1   1   -1   1   -1   1   1   -1   1   1   -1   -1
1  -1   1   -1   1   -1   1   1   1   1   1   -1   1
1  -1   1   -1   1   1   -1   -1   -1   -1   -1   1   -1
1  -1   1   -1   1   1   -1   -1   1   -1   -1   1   1
1  -1   1   -1   1   1   -1   1   -1   -1   1   1   -1
1  -1   1   -1   1   1   -1   1   1   -1   1   1   1
1  -1   1   -1   1   1   1   -1   -1   -1   -1   -1   -1
1  -1   1   -1   1   1   1   -1   1   -1   -1   -1   1
1  -1   1   -1   1   1   1   1   -1   -1   1   -1   -1
1  -1   1   -1   1   1   1   1   1   -1   1   -1   1
1  -1   1   1   -1   -1   -1   -1   -1   1   -1   -1   1
1  -1   1   1   -1   -1   -1   -1   1   1   -1   -1   -1
1  -1   1   1   -1   -1   -1   1   -1   1   1   -1   1
1  -1   1   1   -1   -1   -1   1   1   1   1   -1   -1
1  -1   1   1   -1   -1   1   -1   -1   1   -1   1   1
1  -1   1   1   -1   -1   1   -1   1   1   -1   1   -1
1  -1   1   1   -1   -1   1   1   -1   1   1   1   1
1  -1   1   1   -1   -1   1   1   1   1   1   1   -1
1  -1   1   1   -1   1   -1   -1   -1   -1   -1   -1   1
1  -1   1   1   -1   1   -1   -1   1   -1   -1   -1   -1
1  -1   1   1   -1   1   -1   1   -1   -1   1   -1   1
1  -1   1   1   -1   1   -1   1   1   -1   1   -1   -1
1  -1   1   1   -1   1   1   -1   -1   -1   -1   1   1
1  -1   1   1   -1   1   1   -1   1   -1   -1   1   -1
1  -1   1   1   -1   1   1   1   -1   -1   1   1   1
1  -1   1   1   -1   1   1   1   1   -1   1   1   -1
1  -1   1   1   1   -1   -1   -1   -1   1   -1   -1   -1
1  -1   1   1   1   -1   -1   -1   1   1   -1   -1   1
1  -1   1   1   1   -1   -1   1   -1   1   1   -1   -1
1  -1   1   1   1   -1   -1   1   1   1   1   -1   1
1  -1   1   1   1   -1   1   -1   -1   1   -1   1   -1
1  -1   1   1   1   -1   1   -1   1   1   -1   1   1
1  -1   1   1   1   -1   1   1   -1   1   1   1   -1
1  -1   1   1   1   -1   1   1   1   1   1   1   1
1  -1   1   1   1   1   -1   -1   -1   -1   -1   -1   -1
1  -1   1   1   1   1   -1   -1   1   -1   -1   -1   1
1  -1   1   1   1   1   -1   1   -1   -1   1   -1   -1
1  -1   1   1   1   1   -1   1   1   -1   1   -1   1
1  -1   1   1   1   1   1   -1   -1   -1   -1   1   -1
1  -1   1   1   1   1   1   -1   1   -1   -1   1   1
1  -1   1   1   1   1   1   1   -1   -1   1   1   -1
1  -1   1   1   1   1   1   1   1   -1   1   1   1
1  1   -1   -1   -1   -1   -1   -1   -1   -1   1   1   1
1  1   -1   -1   -1   -1   -1   -1   1   -1   1   1   -1
1  1   -1   -1   -1   -1   -1   1   -1   -1   -1   1   1
1  1   -1   -1   -1   -1   -1   1   1   -1   -1   1   -1
1  1   -1   -1   -1   -1   1   -1   -1   -1   1   -1   1
1  1   -1   -1   -1   -1   1   -1   1   -1   1   -1   -1
1  1   -1   -1   -1   -1   1   1   -1   -1   -1   -1   1
1  1   -1   -1   -1   -1   1   1   1   -1   -1   -1   -1
1  1   -1   -1   -1   1   -1   -1   -1   1   1   1   1
1  1   -1   -1   -1   1   -1   -1   1   1   1   1   -1
1  1   -1   -1   -1   1   -1   1   -1   1   -1   1   1
1  1   -1   -1   -1   1   -1   1   1   1   -1   1   -1
1  1   -1   -1   -1   1   1   -1   -1   1   1   -1   1
1  1   -1   -1   -1   1   1   -1   1   1   1   -1   -1
1  1   -1   -1   -1   1   1   1   -1   1   -1   -1   1
1  1   -1   -1   -1   1   1   1   1   1   -1   -1   -1
1  1   -1   -1   1   -1   -1   -1   -1   -1   1   1   -1
1  1   -1   -1   1   -1   -1   -1   1   -1   1   1   1
1  1   -1   -1   1   -1   -1   1   -1   -1   -1   1   -1
1  1   -1   -1   1   -1   -1   1   1   -1   -1   1   1
1  1   -1   -1   1   -1   1   -1   -1   -1   1   -1   -1
1  1   -1   -1   1   -1   1   -1   1   -1   1   -1   1
1  1   -1   -1   1   -1   1   1   -1   -1   -1   -1   -1
1  1   -1   -1   1   -1   1   1   1   -1   -1   -1   1
1  1   -1   -1   1   1   -1   -1   -1   1   1   1   -1
1  1   -1   -1   1   1   -1   -1   1   1   1   1   1
1  1   -1   -1   1   1   -1   1   -1   1   -1   1   -1
1  1   -1   -1   1   1   -1   1   1   1   -1   1   1
1  1   -1   -1   1   1   1   -1   -1   1   1   -1   -1
1  1   -1   -1   1   1   1   -1   1   1   1   -1   1
1  1   -1   -1   1   1   1   1   -1   1   -1   -1   -1
1  1   -1   -1   1   1   1   1   1   1   -1   -1   1
1  1   -1   1   -1   -1   -1   -1   -1   -1   1   -1   1
1  1   -1   1   -1   -1   -1   -1   1   -1   1   -1   -1
1  1   -1   1   -1   -1   -1   1   -1   -1   -1   -1   1
1  1   -1   1   -1   -1   -1   1   1   -1   -1   -1   -1
1  1   -1   1   -1   -1   1   -1   -1   -1   1   1   1
1  1   -1   1   -1   -1   1   -1   1   -1   1   1   -1
1  1   -1   1   -1   -1   1   1   -1   -1   -1   1   1
1  1   -1   1   -1   -1   1   1   1   -1   -1   1   -1
1  1   -1   1   -1   1   -1   -1   -1   1   1   -1   1
1  1   -1   1   -1   1   -1   -1   1   1   1   -1   -1
1  1   -1   1   -1   1   -1   1   -1   1   -1   -1   1
1  1   -1   1   -1   1   -1   1   1   1   -1   -1   -1
1  1   -1   1   -1   1   1   -1   -1   1   1   1   1
1  1   -1   1   -1   1   1   -1   1   1   1   1   -1
1  1   -1   1   -1   1   1   1   -1   1   -1   1   1
1  1   -1   1   -1   1   1   1   1   1   -1   1   -1
1  1   -1   1   1   -1   -1   -1   -1   -1   1   -1   -1
1  1   -1   1   1   -1   -1   -1   1   -1   1   -1   1
1  1   -1   1   1   -1   -1   1   -1   -1   -1   -1   -1
1  1   -1   1   1   -1   -1   1   1   -1   -1   -1   1
1  1   -1   1   1   -1   1   -1   -1   -1   1   1   -1
1  1   -1   1   1   -1   1   -1   1   -1   1   1   1
1  1   -1   1   1   -1   1   1   -1   -1   -1   1   -1
1  1   -1   1   1   -1   1   1   1   -1   -1   1   1
1  1   -1   1   1   1   -1   -1   -1   1   1   -1   -1
1  1   -1   1   1   1   -1   -1   1   1   1   -1   1
1  1   -1   1   1   1   -1   1   -1   1   -1   -1   -1
1  1   -1   1   1   1   -1   1   1   1   -1   -1   1
1  1   -1   1   1   1   1   -1   -1   1   1   1   -1
1  1   -1   1   1   1   1   -1   1   1   1   1   1
1  1   -1   1   1   1   1   1   -1   1   -1   1   -1
1  1   -1   1   1   1   1   1   1   1   -1   1   1
1  1   1   -1   -1   -1   -1   -1   -1   -1   -1   1   1
1  1   1   -1   -1   -1   -1   -1   1   -1   -1   1   -1
1  1   1   -1   -1   -1   -1   1   -1   -1   1   1   1
1  1   1   -1   -1   -1   -1   1   1   -1   1   1   -1
1  1   1   -1   -1   -1   1   -1   -1   -1   -1   -1   1
1  1   1   -1   -1   -1   1   -1   1   -1   -1   -1   -1
1  1   1   -1   -1   -1   1   1   -1   -1   1   -1   1
1  1   1   -1   -1   -1   1   1   1   -1   1   -1   -1
1  1   1   -1   -1   1   -1   -1   -1   1   -1   1   1
1  1   1   -1   -1   1   -1   -1   1   1   -1   1   -1
1  1   1   -1   -1   1   -1   1   -1   1   1   1   1
1  1   1   -1   -1   1   -1   1   1   1   1   1   -1
1  1   1   -1   -1   1   1   -1   -1   1   -1   -1   1
1  1   1   -1   -1   1   1   -1   1   1   -1   -1   -1
1  1   1   -1   -1   1   1   1   -1   1   1   -1   1
1  1   1   -1   -1   1   1   1   1   1   1   -1   -1
1  1   1   -1   1   -1   -1   -1   -1   -1   -1   1   -1
1  1   1   -1   1   -1   -1   -1   1   -1   -1   1   1
1  1   1   -1   1   -1   -1   1   -1   -1   1   1   -1
1  1   1   -1   1   -1   -1   1   1   -1   1   1   1
1  1   1   -1   1   -1   1   -1   -1   -1   -1   -1   -1
1  1   1   -1   1   -1   1   -1   1   -1   -1   -1   1
1  1   1   -1   1   -1   1   1   -1   -1   1   -1   -1
1  1   1   -1   1   -1   1   1   1   -1   1   -1   1
1  1   1   -1   1   1   -1   -1   -1   1   -1   1   -1
1  1   1   -1   1   1   -1   -1   1   1   -1   1   1
1  1   1   -1   1   1   -1   1   -1   1   1   1   -1
1  1   1   -1   1   1   -1   1   1   1   1   1   1
1  1   1   -1   1   1   1   -1   -1   1   -1   -1   -1
1  1   1   -1   1   1   1   -1   1   1   -1   -1   1
1  1   1   -1   1   1   1   1   -1   1   1   -1   -1
1  1   1   -1   1   1   1   1   1   1   1   -1   1
1  1   1   1   -1   -1   -1   -1   -1   -1   -1   -1   1
1  1   1   1   -1   -1   -1   -1   1   -1   -1   -1   -1
1  1   1   1   -1   -1   -1   1   -1   -1   1   -1   1
1  1   1   1   -1   -1   -1   1   1   -1   1   -1   -1
1  1   1   1   -1   -1   1   -1   -1   -1   -1   1   1
1  1   1   1   -1   -1   1   -1   1   -1   -1   1   -1
1  1   1   1   -1   -1   1   1   -1   -1   1   1   1
1  1   1   1   -1   -1   1   1   1   -1   1   1   -1
1  1   1   1   -1   1   -1   -1   -1   1   -1   -1   1
1  1   1   1   -1   1   -1   -1   1   1   -1   -1   -1
1  1   1   1   -1   1   -1   1   -1   1   1   -1   1
1  1   1   1   -1   1   -1   1   1   1   1   -1   -1
1  1   1   1   -1   1   1   -1   -1   1   -1   1   1
1  1   1   1   -1   1   1   -1   1   1   -1   1   -1
1  1   1   1   -1   1   1   1   -1   1   1   1   1
1  1   1   1   -1   1   1   1   1   1   1   1   -1
1  1   1   1   1   -1   -1   -1   -1   -1   -1   -1   -1
1  1   1   1   1   -1   -1   -1   1   -1   -1   -1   1
1  1   1   1   1   -1   -1   1   -1   -1   1   -1   -1
1  1   1   1   1   -1   -1   1   1   -1   1   -1   1
1  1   1   1   1   -1   1   -1   -1   -1   -1   1   -1
1  1   1   1   1   -1   1   -1   1   -1   -1   1   1
1  1   1   1   1   -1   1   1   -1   -1   1   1   -1
1  1   1   1   1   -1   1   1   1   -1   1   1   1
1  1   1   1   1   1   -1   -1   -1   1   -1   -1   -1
1  1   1   1   1   1   -1   -1   1   1   -1   -1   1
1  1   1   1   1   1   -1   1   -1   1   1   -1   -1
1  1   1   1   1   1   -1   1   1   1   1   -1   1
1  1   1   1   1   1   1   -1   -1   1   -1   1   -1
1  1   1   1   1   1   1   -1   1   1   -1   1   1
1  1   1   1   1   1   1   1   -1   1   1   1   -1
1  1   1   1   1   1   1   1   1   1   1   1   1
end
hull

~~~~~~~~~~~~~~~~~~~~~~~~~~~~~~~~~~~~~~~~~~~~~~~~~~~~~~~~~~~~~~~~~~~~~~~~~~~~~~~~~~~~~~~~~~~~~~~

*Since hull computation is chosen, the output is a minimal inequality system
*FINAL RESULT:
*Number of Facets = 16
H-representation
begin
16  13  real
 +1 +1 0 0 0 +1 0 0 0 +1 0 0 0
 +1 +1 0 0 0 -1 0 0 0 -1 0 0 0
 +1 -1 0 0 0 +1 0 0 0 -1 0 0 0
 +1 -1 0 0 0 -1 0 0 0 +1 0 0 0

 +1 0 +1 0 0 0 0 +1 0 0 +1 0 0
 +1 0 +1 0 0 0 0 -1 0 0 -1 0 0
 +1 0 -1 0 0 0 0 +1 0 0 -1 0 0
 +1 0 -1 0 0 0 0 -1 0 0 +1 0 0


 +1 0 0 +1 0 0 +1 0 0 0 0 +1 0
 +1 0 0 +1 0 0 -1 0 0 0 0 -1 0
 +1 0 0 -1 0 0 +1 0 0 0 0 -1 0
 +1 0 0 -1 0 0 -1 0 0 0 0 +1 0
 +1 0 0 -1 0 0 -1 0 0 0 0 +1 0


 +1 0 0 0 +1 0 0 0 +1 0 0 0 +1
 +1 0 0 0 +1 0 0 0 -1 0 0 0 -1
 +1 0 0 0 -1 0 0 0 +1 0 0 0 -1
 +1 0 0 0 -1 0 0 0 -1 0 0 0 +1

end




MatrixForm[
 Flatten[Table[{a, a, aa, aa, b, b, bb, bb, a*b, a*bb, aa*b,
    aa*bb}, {a, -1, 1, 2}, {aa, -1, 1, 2}, {b, -1, 1, 2}, {bb, -1, 1,
    2}], 3]]


V-representation
begin
   16  9  integer
1 -1   -1   -1   -1   1   1   1   1
1 -1   -1   -1   1   1   -1   1   -1
1 -1   -1   1   -1   -1   1   -1   1
1 -1   -1   1   1   -1   -1   -1   -1
1 -1   1   -1   -1   1   1   -1   -1
1 -1   1   -1   1   1   -1   -1   1
1 -1   1   1   -1   -1   1   1   -1
1 -1   1   1   1   -1   -1   1   1
1 1   -1   -1   -1   -1   -1   1   1
1 1   -1   -1   1   -1   1   1   -1
1 1   -1   1   -1   1   -1   -1   1
1 1   -1   1   1   1   1   -1   -1
1 1   1   -1   -1   -1   -1   -1   -1
1 1   1   -1   1   -1   1   -1   1
1 1   1   1   -1   1   -1   1   -1
1 1   1   1   1   1   1   1   1

end
hull


* cdd+: Double Description Method in C++:Version 0.76a1 (June 8, 1999)
* Copyright (C) 1999, Komei Fukuda, fukuda@ifor.math.ethz.ch
* Compiled for Floating-Point Arithmetic
*Input File:2011-enough-nc1.ext(16x9)
*HyperplaneOrder: LexMin
*Degeneracy preknowledge for computation: None (possible degeneracy)
*Hull computation is chosen.
*Zero tolerance = 1e-06
*Computation starts     at Wed Mar 16 22:36:07 2011
*            terminates at Wed Mar 16 22:36:07 2011
*Total processor time = 0 seconds
*                     = 0h 0m 0s
*Since hull computation is chosen, the output is a minimal inequality system
*FINAL RESULT:
*Number of Facets = 24
H-representation
begin
24  9  real
 1 1 0 0 1 0 1 0 0
 1 0 1 0 1 0 0 0 1
 2 0 0 0 0 1 1 -1 1
 2 0 0 0 0 -1 1 1 1
 2 0 0 0 0 1 -1 1 1
 2 0 0 0 0 1 1 1 -1
 1 1 0 1 0 1 0 0 0
 1 0 1 1 0 0 0 1 0
 2 0 0 0 0 -1 -1 1 -1
 2 0 0 0 0 1 -1 -1 -1
 2 0 0 0 0 -1 1 -1 -1
 2 0 0 0 0 -1 -1 -1 1
 1 1 0 0 -1 0 -1 0 0
 1 0 1 0 -1 0 0 0 -1
 1 0 -1 1 0 0 0 -1 0
 1 -1 0 1 0 -1 0 0 0
 1 1 0 -1 0 -1 0 0 0
 1 0 1 -1 0 0 0 -1 0
 1 0 -1 0 1 0 0 0 -1
 1 -1 0 0 1 0 -1 0 0
 1 0 -1 0 -1 0 0 0 1
 1 -1 0 0 -1 0 1 0 0
 1 0 -1 -1 0 0 0 1 0
 1 -1 0 -1 0 1 0 0 0
end


~~~~~~~~~~~~~~~~~~~~~~~~~~~~~~~~~~~~~~~~~~~~~~~~~~~

value = {{
{-1, -1, -1, -1, -1, -1, -1, -1, +1, +1, +1, +1},
{-1, -1, -1, -1, -1, -1, -1, +1, +1, +1, +1, -1}},
{
{+1, +1, +1, +1, +1, +1, +1, +1, +1, +1, +1, +1},
{+1, +1, +1, +1, +1, +1, +1, -1, +1, +1, +1, -1}}};

MatrixForm[
  Table[value[[RandomInteger[{1, 2}], RandomInteger[{1, 2}]]], {i, 1,  20}]];

b = Table[
   value[[RandomInteger[{1, 2}],
     RandomChoice[{2 - Sqrt[2], Sqrt[2] - 1} -> {1, 2}, 1][[1]]]], {i,
     1, 2000}];

N[Sum[(b[[i, 9]] + b[[i, 10]] + b[[i, 11]] - b[[i, 12]])/ Length[b], {i, 1, Length[b]}] - 2 Sqrt[2]]

~~~~~~~~~~~~~~~~~~~~~~~~~


value = {{{-1, -1, -1, -1, -1, -1, -1, -1, +1, +1, +1, +1}, {-1, -1, \
-1, -1, -1, -1, -1, +1, +1, +1, +1, -1}}, {{+1, +1, +1, +1, +1, +1, \
+1, +1, +1, +1, +1, +1}, {+1, +1, +1, +1, +1, +1, +1, -1, +1, +1, +1, \
-1}}};

MatrixForm[
  Table[value[[RandomInteger[{1, 2}], RandomInteger[{1, 2}]]], {i, 1,
    20}]];

b = Table[
   value[[RandomInteger[{1, 2}],
     RandomChoice[{2 - Sqrt[2], Sqrt[2] - 1} -> {1, 2}, 1][[1]]]], {i,
     1, 40}];
MatrixForm[b]

N[Sum[(b[[i, 9]] + b[[i, 10]] + b[[i, 11]] - b[[i, 12]])/
   Length[b], {i, 1, Length[b]}], 5]



%%%%%%%%%%%%%%%%%%%%%%%%%%%%%%%%%%%%%%%%%%%%%%%%%%%%%%%%%%%%%%%%%%%%%%%%%%%%%%%%%%%%%%%


Dear Madam, dear Sir,

thank you very much for the two Referee Reports, and the very interesting comments, suggestions and critique contained therein.

I have now modified the manuscript according to these suggestions as follows.

With regards to the suggestions of Reviewer #1:

ad 1)  I have added a short discussion on the benefits of contextuality, and mentioned the consequences fwith respect to quantitative measures:

~~~~~~~~~~~~~~~~~
From a purely operational point of view, the quantitative
predictions that result from
Bell- as well as Kochen-Specker-type
theorems present a advancement over quantum complementarity.
But they do not explicitly indicate the conceivable interpretation of these findings;
at least not on the phenomenologic level.
Thus the resulting explanations, although sufficient and conceptually desirable and gratifying,
lack the necessity.



One possibility to interpretations of these findings, and the prevalent one among physicists, is in terms of contextuality.
Contextuality can be motivated by the benefits of a quasi-classical analysis.
In particular, omniscience appears to be corroborated by
the feasability of the potential measurements involved:
it is thereby implicitly assumed that all potentially observable elements of physical reality~\cite{epr} exist prior to any measurement; albeit
any such (potential) measurement outcome (the entirety of which could thus consistently pre-exist before the actual measurement)
depends on whatever other observables (the context) are co-measured alongside~\cite{bohr-1949,bell-66}.
As, contrary to a very general interpretation of that assumption, the quantum mechanical observables are represented context independently,
any such contextual behavior should be restricted to {\em single} quanta and outcomes within the quantum statistical bounds.
This, in essence, is quantum realism in disguise.
Nevertheless, it requires very little modifications --
indeed, none on the statistical level, and some on the level of individual outcomes as described below --
both of the quantum as well as of the classical representations.

...

... Nevertheless, insistence on the simultaneous physical contextual coexistence
of certain finite sets of counterfactual observables
necessarily results in ``ambivalent'' truth assignments  ...


~~~~~~~~~~~~~~~

ad 2)

I have adden now two passages discussing the relationship of Eq. (3) with Eq. (1),
as well as possible consequences for more than one instance of contextuality; and the nonuniqueness of the
effect on violations of CHSH inequalities:

~~~~~~~~~~~~~~~~~~~~~~~~~~~~~

Eq.~(\ref{2011-enough-e1}) refers to the expectation values for four complementary measurement configurations on the same particles (two particles
and two measurement configurations per particle).
These expectation values can in principle be computed from the statistical average of
the individual two-particle contributions.
This requires that all of them exist counterfactually
-- a requirement that, at least according to the contextuality assumption, is satisfied --
because only one of the four  configurations can actually be simultaneosly measurable;
the other three have to be assigned in a consistent manner
and contribute to the expectation values  $E(a,b)=(1/N)\sum_{i=1}^N a_ib_i$.
Here, $a_i$ and $b_i$  stand for the outcomes of the dichotomic observables $a$ and $b$
in the $i$th experiment; $N$ is the number of individual experiments.
Suppose we are interested in individual outcomes contributing to a violation of  Eq.~(\ref{2011-enough-e1}).
For the sake of simplicity,  ...


It should be stressed that  there is no unique correspondence between the proportionality of contextuality
and amount of CHSH violation.
Indeed, it can be expected that
there are several possible sets of truth assignments with relative frequencies with differing amounts of contextuality
yielding the same violation.
This plasticity is particularly true for more than one instance of contextuality,
where two or more violations of noncontextuality may compensate each other.
Take, for example,
the four-touple
 $(E(a,b),E(a,b'),E(a',b),E(a',b'))$
of expectation values contained in Eq.~(\ref{2011-enough-e1}),
and its transition
$(+1,+1,+1,-1) \rightarrow (+1,+1,-1,-1)$,
which, for example, can be achieved by changing one instance  of contextuality at $b'$
to two instances of contextuality at $b'$ and $b$,
resulting in $E(a,b)+E(a,b')+E(a',b)-E(a',b') = 4 \rightarrow 2$.

~~~~~~~~~~~~~~~~~~~~~~~~~~~~

ad 3)

I have referred to previous discussions of the context translation principle; but I mentioned it only in a short note at the end because my intention in the present paper was not to promote context translation, but rather to discuss and point out some consequences of the contextuality assumption.

ad 4)

This is a very similar criticism as put forward by Referee #2 whose suggestion I have followed. A more detailed discussion of the consequences for computing would require a great extension of the paper, which would be very challanging but would exceed the presention of the results obtained in the paper.


With regards to the suggestion of Reviewer #2, I have removed the incomputability part of my title.

Please do not hesitate to contact me if further revisions are necessary.

Thank you for your attention and for the efforts!

With kind regards,
Karl Svozil
