\documentstyle [12pt,pslatex]{article}

\hyphenation{or-tho-atom-istic or-tho-rep-re-sent-able}

\addtolength{\topmargin}{-20mm}
\addtolength{\textheight}{ 30mm}

\newcommand{\N}{{\bf N}}       % natural numbers
\newcommand{\R}{{\bf R}}       % real numbers
\newcommand{\HL}{H_3}          % Hilbert logic
\newcommand{\dist}{{\rm d}}    % distance function
\newcommand{\Sp}{{\rm Sp}}     % span

\let\charmi=A\let\charmii=B      % abbr. of coord.: A=-1, B=-2
\newcommand{\ex}[1]%
  {\let\char#1\ifx\char\charmi-1\else\ifx\char\charmii-2\else#1\fi\fi}

\unitlength .5mm
\newcommand{\emline}[4]%
  {\put(#1,#2){\special{em:moveto}}\put(#3,#4){\special{em:lineto}}}

\newsavebox{\vertex}\savebox{\vertex}{%      % vertex in diagrams
  {\unitlength1mm\begin{picture}(0,0)\put(0,0){\circle*{1}}\end{picture}}}
  \newcommand{\disc}{\usebox{\vertex}}
  \newcommand{\point}[2]{\put(#1,#2){\disc}}
\newsavebox{\subdiagram}\savebox{\subdiagram}{%   % circle in diagrams
  {\unitlength1mm\begin{picture}(0,0)\put(0,0){\circle{2}}\end{picture}}}
  \newcommand{\discbig}{\usebox{\subdiagram}}
\newcommand{\place}[6]%
  {\put(#1,#2){\hspace{#3pt}\raisebox{#4pt}{\makebox(0,0)[#5]{$#6$}}}}
\newsavebox{\shortdiagram}           % abbr. for the main logic
\savebox{\shortdiagram}{\raisebox{.25\baselineskip}{\begin{picture}(12,0)(-1,0)
  \put(0,0){\disc}\put(10,0){\disc}\put(5,0){\discbig}\put(0,0){\line(1,0){10}}
  \end{picture}}}
\newcommand{\edges}{       % of the main logic
  \put (-20,  0){\line( 1, 1){10}}  \put (-10,-10){\line(1,0){20}}
  \put (-20,  0){\line( 1,-1){10}}  \put (-10, 10){\line(1,0){20}}
  \put ( 20,  0){\line(-1,-1){10}}  \put (  0,-10){\line(0,1){20}}
  \put ( 20,  0){\line(-1, 1){10}}
  }
\newcommand{\vertices}{    % of the main logic
  \put (-15,-5){\disc}  \multiput (-10, 10)(10,0){3}{\disc}
  \put (-15, 5){\disc}  \multiput (-10,-10)(10,0){3}{\disc}
  \put ( 15, 5){\disc}  \multiput (-20,  0)(20,0){3}{\disc}
  \put ( 15,-5){\disc}
  }

\newcounter{cislo}
\newenvironment{conditions}%
  {\begin{list}{\rm(\arabic{cislo})}{\usecounter{cislo}
    \itemsep0pt\topsep5pt\parsep\parskip
    \settowidth{\labelwidth}{(1)}\labelsep 1ex
    \leftmargin\labelwidth\addtolength{\leftmargin}{\labelsep}
    \addtolength{\leftmargin}{\parindent}}}%
  {\end{list}}

\newcounter{oftheorem}[section]
\newenvironment{mytheorem}[1]%
{\begin{trivlist}
  \renewcommand{\theoftheorem}{#1~\arabic{section}.\arabic{oftheorem}}
  \refstepcounter{oftheorem}
  \item[\hspace{\labelsep}\bf\thesection.\arabic{oftheorem}.\,#1.]}%
{\end{trivlist}}

\newenvironment{definition}{\begin{mytheorem}{Definition}}{\end{mytheorem}}
\newenvironment{lemma}{\begin{mytheorem}{Lemma}\sl}{\end{mytheorem}}
\newenvironment{proposition}{\begin{mytheorem}{Proposition}\sl}{\end{mytheorem}}
\newenvironment{theorem}{\begin{mytheorem}{Theorem}\sl}{\end{mytheorem}}
\newenvironment{corollary}{\begin{mytheorem}{Corollary}\sl}{\end{mytheorem}}
\newenvironment{remark}{\begin{mytheorem}{Remark}}{\end{mytheorem}}
\newenvironment{problem}{\begin{mytheorem}{Problem}}{\end{mytheorem}}
\newenvironment{example}{\begin{mytheorem}{Example}}{\end{mytheorem}}
\newenvironment{question}{\begin{mytheorem}{Question}}{\end{mytheorem}}

\newenvironment{proof}%
  {\begin{trivlist}
  \item[\hspace{\labelsep}\bf Proof.]}%
  {\end{trivlist}}



\begin{document}

\title {Greechie diagrams, nonexistence of measures in quantum logics and
Kochen--Specker type constructions}

\author {K. Svozil%
\thanks {Institut f\"ur Theoretische Physik, University of Technology
Vienna, Wiedner Hauptstra\ss{}e 8-10/136, A-1040 Vienna, Austria; e-mail:
svozil@tph.tuwien.ac.at}
\and
J. Tkadlec%
\thanks{Department of Mathematics, Faculty of Electrical Engineering,
Czech Technical University, CZ-166~27 Prague, Czech Republic;
e-mail: tkadlec@math.feld.cvut.cz}}

\maketitle

\begin {abstract}
We use Greechie diagrams to construct finite orthomodular lattices
`realizable' in the orthomodular lattice of subspaces in a threedimensional
Hilbert space such that the set of two-valued states is not `large' (i.e.,
full, separating, unital, nonempty, resp.). We discuss the number of
elements
of such orthomodular lattices, of their sets of (ortho)generators and of their
subsets which do not admit `large' set of two-valued states. We show
connections with other results of this type.
\end {abstract}




\section {Introduction}


Quantum logic, as it has been pioneered by Birkhoff and von Neumann
\cite{birk-vonneu}, is usually derived from Hilbert space. There, the
logical primitives, such as propositions and the logical operators ``and'',
``or'' and ``not'' are defined by Hilbert space entities. For instance,
consider the threedimensional, real Hilbert space $\R^3$ with the usual
scalar product $(v,w):=\sum_{i=1}^3v_iw_i$, $v,w\in \R^3$. There, any
proposition is identified with a subspace of $\R^3$. For instance, the zero
vector corresponds to a false statement. Any line spanned by a nonzero
vector corresponds to the statement that the physical system is in the pure
state associated with the vector. Any plain formed by the linear combination
of two (non-colinear) vectors $v,w$ corresponds to the statement that the
physical system is either in the pure state $v$ {\em or\/} in the pure state
$w$. The whole Hilbert space $\R^3$ corresponds to the tautology (true
propositions). The logical ``and''-operation is identified with the set
theoretical intersection of two propositions; e.g., with the intersection of
two lines. The logical ``not''-operation, or the ``complement'', is
identified with taking the orthogonal subspace; e.g., the complement of a
line is the plain orthogonal to that line.

In this top-down approach, one arrives at a propositional calculus which
resembles the classical one, but differs from it in several important
aspects. They are non-Boolean, i.e., non-distributive, algebraic structures.
Furthermore, as has first been pointed out by Kochen and Specker in the
context of partial algebras \cite{kochen-specker-65}, there exist certain
{\em finite\/} sets of lines, such that the associated propositional
structure cannot be classically embedded. That is, there does not exist any
classical, i.e., two-valued, measure which could be interpreted as the fact
that propositions are either ``true'' ($\equiv$ measure value~1) or
``false'' ($\equiv$ measure value~0). Kochen and Specker's original
construction used 117 lines. The number of lines has been subsequently
reduced \cite{Peres1,Peres2,Mermin,Schuette}. These constructions are
examples of propositional structures without any two-valued measures.


This paper deals with the following questions: which orthomodular
structure---finite or infinite---underlies the Kochen--Specker construction.
The question can be approached from two different viewpoints: (i)~Which {\em
minimal\/} set of propositions generates some Kochen--Specker type
configurations? By ``generate'' we mean the construction of the
propositional structure containing it. (ii)~What is the {\em minimal
propositional structure\/} containing some sort of Kochen--Specker type
configuration? In particular, is it finite or infinite?




\section {Basic notions}


The following definition gives two main concepts of a propositional
structure.


\begin {definition}
An {\em orthomodular poset\/} is a structure $(P,\le,',0,1)$ fulfilling the
following conditions:
  \begin {conditions}
  \item
  $(P,\le)$ is a partial ordered set such that $0\le a \le 1$ for every $a
  \in P$.
  \item
  $'\colon P \to P$ is an orthocomplementation, i.e., for every $a,b \in P$:
     (a)~$a''=a$,
     (b)~$a\le b$ implies $b' \le a'$,
     (c)~$a \lor a' =1$.
  \item
  If $a \le b'$
  then the supremum $a\lor b$ exists in~$P$.
  \item
  If $a\le b$ then there is an element $c\in L$ such that $c\le a'$ and
  $b=a\lor c$ (the orthomodular law).
  \end {conditions}
An {\em orthomodular lattice\/} is an orthomodular poset which is a lattice.

Elements $a,b$ of an orthomodular poset are called {\em orthogonal\/}
(denoted by $a\perp b$) if $a\le b'$. A subset $O$ of an orthomodular poset is
called {\em orthogonal\/} if every pair of its elements is orthogonal.
\end {definition}


\begin {definition}
Let $P_1,P_2$ be orthomodular posets. $P_1$ is {\em orthorepresentable\/} in
$P_2$ if there is a mapping (called {\em orthoembedding\/}) $h\colon\, P_1\to P_2$
such that for every $a,b \in P_1$:
  \begin {conditions}
  \item $h(0)=0$,
  \item $h(a')=h(a)'$,
  \item $a \le b$ if and only if $h(a) \le h(b)$,
  \item $h(a \lor b) = h(a) \lor h(b)$ whenever $a \perp b$.
  \end {conditions}
$P_1$ is {\em representable\/} in~$P_2$ if there is a mapping (called {\em
embedding\/}) $h\colon\, P_1\to P_2$ such that $h$ is orthoembedding and for
every $a,b \in P_1$:
  \begin {conditions}
  \item [(4')] $h(a \lor b) = h(a) \lor h(b)$.
  \end {conditions}
The set $h(P_1)$ is then called an {\em {\rm(}ortho{\/\rm)}representation\/}
of $P_1$ in $P_2$.

A {\em suborthoposet\/} ({\em subortholattice}, resp.) is a subset such that
the identity mapping is orthoembedding (embedding, resp.).

{\em Boolean subalgebra\/} of an orthomodular poset is a suborthoposet which
is a Boolean algebra. {\em Block\/} is a maximal Boolean subalgebra.
\end {definition}


As we will see later, there are lattices $L_1,L_2$ such that $L_1$ is a
suborthoposet but not a subortholattice of $L_2$. On the other hand, a
suborthoposet of an orthomodular lattice need not be a lattice.


\begin {definition}
Let $L$ be an orthomodular lattice, $G,\bar L \subseteq P$ and let us denote
by $L(G)$ ($P(G)$, resp.) the least subortholattice (suborthoposet, resp.)
of~$L$ containing~$G$. We say that $G$ {\em generates\/} ({\em
orthogenerates}, resp.) $\bar L$ if $\bar L \subseteq L(G)$ ($\bar L
\subseteq P(G)$, resp.).
\end {definition}


$P(G)$ and $L(G)$ can be explicitly defined by the following process:
$P(G)=\bigcup_{n=0}^{\infty} P_n(G)$, $L(G)=\bigcup_{n=0}^{\infty} L_n(G)$,
where $P_0(G) = L_0(G) = G$ and, for every natural number~$n$:
  \begin {eqnarray*}
  L_{n+1} (G) &=& \big\{ \bigvee O;\; O \mbox{ is a finite subset of
                        $L_n(G) \cup L_n(G)'$}
                  \big\}, \\
  P_{n+1} (G) &=& \big\{ \bigvee O;\; O \mbox{ is a finite orthogonal subset of
                        $P_n(G) \cup P_n(G)'$}
                  \big\}
  \end {eqnarray*}
($M'$ denotes the set $\{a';\; a \in M\}$). Hence, every countable set $G$
generates a countable subortholattice and orthogenerates a countable
suborthoposet.


A very useful tool for constructing and representing some orthomodular
posets is the so-called Greechie diagram.


\begin {definition} \label{diagram}
A {\em diagram\/} is a pair $(V,E)$, where $V \neq \emptyset$ is a set of
{\em vertices\/} (usually drawn as points) and $E \subseteq \exp V \setminus
\{\emptyset\}$ is a set of {\em edges\/} (usually drawn as line segments
connecting corresponding points).

Let $n \ge 2$ be a natural number. A {\em loop\/} of order $n$ in a diagram
$(V,E)$ is a sequence $(e_1,\ldots,e_n)\in E^n$ of mutually different edges
such that there are mutually different vertices $v_1,\ldots,v_n$ with
$v_i \in e_i \cap e_{i+1}$ ($i=1,\ldots,n$, $e_{n+1}=e_1$).

A {\em Greechie diagram\/} is a diagram fulfilling the following conditions:
  \begin {conditions}
  \item Every vertex belongs to at least one edge.
  \item If there are at least two vertices then every edge is at least
        2-element.
  \item Every edge which intersects with another edge is at least 3-element.
  \item Every pair of different edges intersects in at most one vertex.
  \item There is no loop of order~3.
  \end {conditions}
\end {definition}


\begin {figure}[ht]
\hfill
%   2 points
\begin {picture}(10,20)(0,-10)
\point{0}{0} \point{10}{0}
\put(5,-10){\makebox(0,0)[b]{1}}
\end {picture}
%
\hfill
%  2-element edge
\begin {picture}(30,20)(0,-10)
\multiput (0,0)(10,0){3}{\disc} \point {20}{10}
\put ( 0,0){\line(1,0){20}}
\put (20,0){\line(0,1){10}}
\put (10,-10){\makebox(0,0)[b]{2}}
\end {picture}
%
\hfill
%  2-element intersection of edges
\begin {picture}(50,10)(0,-10)
\multiput (0,0)(10,0){6}{\disc}
\multiput (0,1)(20,-2){2}{\line(1,0){30}}
\put (25,-10){\makebox(0,0)[b]{3}}
\end {picture}
%
\hfill
%  triangle
\begin {picture}(20,30)(0,-10)
\multiput (0, 0)(10,0){3}{\disc}
\multiput (0,10)(10,0){2}{\disc}
\point {0}{20}
\put (0, 0){\line(1, 0){20}}
\put (0, 0){\line(0, 1){20}}
\put (0,20){\line(1,-1){20}}
\put (10,-10){\makebox(0,0)[b]{4}}
\end {picture}
%
\hfill\mbox{}
%
\caption {Examples of diagrams which are not Greechie diagrams.}
\label {Fdiagrams}
\end {figure}


Some examples of diagrams which are not Greechie diagrams are given in
Fig.~\ref{Fdiagrams}---these examples violates exactly one of conditions
(2)--(5) in the above definition. (We usually do not denote 1-element
edges.) The condition~(4) states that in Greechie diagrams there is no loop
of order~2.

Before we present the representation theorem let us recall that an {\em
atom\/} in an orthomodular poset $P$ is a minimal element of $P \setminus
\{0\}$.


\begin {theorem}
For every Greechie diagram with only finite edges there is exactly one (up
to an isomorphism) orthomodular poset such that there are one-to-one
correspondences between vertices and atoms and between edges and blocks
which preserve incidence relations. A Greechie diagram does not contain any
loop of order 4 if and only if the corresponding orthomodular poset is a
lattice.
\end {theorem}


The proof can be found e.g.\ in~\cite{NR}. Let us reserve the notion {\em
Greechie logic\/} for an orthomodular poset which can be represented by a
Greechie diagram with only finite edges. It is easy to see that such an
orthomodular poset does not contain any infinite chain, hence every its
element is a supremum of a finite orthogonal set of atoms.

Let us remark that there are finite orthomodular posets not representable by
Greechie diagrams---intersections of blocks might be greater than a
4-element Boolean subalgebra and hence the condition~(4) of \ref{diagram}
cannot be fulfilled. On the other hand, every orthomodular poset with only
finite and at most 3-atomic blocks (the case we are interested about) is a
Greechie logic.

We will have a special interest about the following example.


\begin {definition}
The 3-dimensional {\em Hilbert logic\/} $\HL$ is the orthomodular lattice
of linear subspaces of $\R^3$. The ordering is given by inclusion and the
orthocomplementation is given by $a'= \{v\in\R^3;\; v\perp a \}$ for every
$a \in \HL$.
\end {definition}


The least element of $\HL$ is $0 = \{(0,0,0)\}$, the greatest element of $\HL$
is $1 = \R^3$. Moreover $a \land b = a \cap b$ and $a \lor b = \Sp(a\cup b)$
for every $a,b \in \HL$, where $\Sp (G)$ is the {\em span\/} of~$G$ in
$\R^3$. (We will usually omit unnecessary parenthesis, e.g., $\Sp (1,0,0)$
denotes $\Sp (\{(1,0,0)\})$.)

Every element of $\HL \setminus \{0,1\}$ is either an atom or a coatom,
every block in $\HL$ is finite and at most 3-element, every suborthoposet
$P$ of~$\HL$ is a Greechie logic and is uniquely determined by the set
$A_1(P)$ of its 1-dimensional atoms (lines):
  $$
  P = \{0,1\} \cup A_1(P) \cup A_1(P)'.
  $$
(There might be also 2-dimensional atoms in $P$, e.g., if $P$ is 4-element.)
Moreover, for every set $G$ of lines in $\HL$ the set of lines of the
orthomodular lattice $L(G)$ (orthomodular poset $P(G)$, resp.) generated
(orthogenerated, resp.) by~$G$ can be expressed as follows: $A_1(P(G)) =
\bigcup_{n=0}^{\infty} P_n$, $A_1(L(G)) = \bigcup_{n=0}^{\infty} L_n$, where
$P_0 =L_0 = G$ and, for every natural number~$n$:
  \begin {eqnarray*}
  L_{n+1} &=& L_n \cup \big\{(a\lor b)';\; a,b \in L_n \big\}, \\
  P_{n+1} &=& P_n \cup \big\{(a\lor b)';\; a,b \in P_n \mbox{ such that }
                  a\perp b \big\}.
  \end {eqnarray*}




\section{Two-valued states and Greechie diagrams}


Let us present the main definition.


\begin {definition} \label{state}
Let $P$ be an orthomodular poset and let $G \subset P$. A {\em state\/} $s$
on~$G$ is a mapping $s\colon\, P \to [0,1]$ such that:
  \begin {conditions}
  \item $s(0)=0$,
  \item $s(a) \le s(b)$ whenever $a,b \in G$ with $a \le b$,
  \item $\sum_{a \in O} s(a) \le 1$ for every orthogonal set $O \subset G$,
  \item $\sum_{a \in O} s(a) = 1$ for every orthogonal set $O \subset G$
        with $\bigvee O =1$.
  \end {conditions}
A {\em two-valued state\/} is a state with values in $\{0,1\}$.
\end {definition}


If $G=P$ then conditions (1)--(2) follows from conditions (3)-(4) and from
the orthomodular law and, moreover, $s(a')=1-s(a)$ for every $a \in P$.

The Kochen--Specker construction gives an example of a propositional
structure without any two-valued state. We will use more general attempt and
will ask whether there is a propositional structure without `enough'
two-valued states. Originally, `enough' meant `at least one'. We will use
also the following properties of state space, which are important in quantum
logic theories.


\begin {definition}
Let $P$ be an orthomodular poset and let $G \subseteq P$. A set $S$ of
states on~$G$ is called:

{\em unital\/} if for every $a\in G \setminus \{0\}$ there is a state $s\in
S$ such that $s(a)=1$,

{\em separating\/} if for every $a,b\in G$ with $a \neq b$ there is a state
$s\in S$ such that $s(a) \neq s(b)$,

{\em full\/} if for every $a,b \in G$ with $a \not\le b$ there is a state $s
\in S$ such that $s(a) > s(b)$.
\end {definition}


Existence of a unital set of states means that every proposition which is
not a tautology is sometimes false. Existence of a separating set of states
means that a different propositions are distinguishable. Existence of a
full set of two-valued states means that if some proposition does not imply
another, then there is such a state that the first is true while the second
is not. These properties are largely studied. An orthomodular poset with a
full set of two-valued states is called a {\em concrete logic\/} (see
e.g.~\cite{Ptak-Pulmannova}), an orthomodular poset with a separating set of
two-valued states is called a {\em partition logic\/}---this notion is
within orthomodular posets equivalent to the notion of {\em automaton
logic\/} (see e.g.~\cite{svosh,svosh2,svosh3}).

It is easy to see that a full set of states is separating and that a
separating set of two-valued states is unital. Before we give examples
demonstrating differences in the above defined notions let us give some
criteria, how we can verify whether an orthomodular poset given by a
Greechie diagram has `enough' two-valued states.


\begin {definition}
Let $P$ be an orthomodular poset and let $A$ be the set of atoms in~$P$. A
{\em weight\/} $w$ on~$A$ is a mapping $w\colon\, A \to [0,1]$ such that
$\sum_{a \in O} w(a) = 1$ for every maximal orthogonal set $O \subseteq P$.
A {\em two-valued\/} weight is a weight with values in $\{0,1\}$.
\end {definition}


\begin {lemma} \label {s-w}
Let $P$ be a Greechie logic and let $A$ be the set of atoms in~$P$.
Then there is a one-to-one correspondence between two-valued states $s$
on~$P$ and two-valued weights $w$ on~$A$ given by $w=s|A$.
\end {lemma}


\begin {proof}
Obvious.
\end {proof}


Due to this correspondence we may (and will) identify states and weights and
study only the values of states on the set of atoms. Since every maximal
orthogonal set of atoms corresponds uniquely to a block, we need only to
check that the sum of values of a state on every edge in a Greechie diagram
is equal to~1.


\begin {proposition} \label {full}
Let $P$ be a Greechie logic and let $A$ be the set of atoms in~$P$. Then $P$
has a full set of two-valued states (i.e., $P$ is a concrete logic) if and
only if for every pair $a_1,a_2 \in P$ of different nonorthogonal atoms
there is a two-valued weight $w$ on~$A$ such that $w(a_1) = w(a_2) = 1$.
\end {proposition}


\begin {proof}
$\Rightarrow$: Let $a_1,a_2 \in A$ such that $a_1 \not\perp a_2$. Then $a_1
\not\le a_2'$ and there is a two-valued state $s$ on~$P$ such that $1 =
s(a_1) > s(a_2') = 0$. Hence, $s(a_2)=1$ and, according to \ref{s-w}, it
suffices to take $w=s|A$.

$\Leftarrow$: Let $b_1,b_2 \in P$ such that $b_1 \not\le b_2$, i.e., $b_1
\not\perp b_2'$. There are orthogonal sets $A_1,A_2 \neq \emptyset$ of atoms
in~$P$ such that $b_1 = \bigvee A_1$, $b_2' = \bigvee A_2$. According to
\ref{s-w}, it suffices to prove that there are atoms $a_1 \in A_1$, $a_2\in
A_2'$ and a weight $w$ on~$A$ such that $w(a_1) = w(a_2) = 1$. Let us
suppose first that $A_1 \cap A_2 = \emptyset$. Then there are atoms $a_1 \in
A_1$ and $a_2 \in A_2$ such that $a_1 \neq a_2$ and $a_1 \not\perp a_2$ and,
due to our assumption, a weight $w$ on~$A$ such that $w(a_1) = w(a_2) = 1$.
Let us suppose now that $A_1 \cap A_2 \neq \emptyset$. Then there is an atom
$a_1 \le b_1,b_2'$ and either there is an atom $a_2 \neq a_1$ such that $a_1
\not\perp a_2$, or $a_1 \perp a$ for every atom $a \neq a_1$. In both cases
there is a two-valued weight $w$ on~$A$ such that $w(a_1)=1$; in the first
case due to our assumption and in the second case we can put $w(a)=1$ iff
$a=a_1$.
\end {proof}


The situation for a separating set of states is much more complicated and we
will state a criterion in a special case (which is in our interest here).


\begin {proposition} \label {separating}
Let $P$ be a Greechie logic with at most 3-atomic blocks and let $A$ be the
set of atoms in~$P$. Then the set of two-valued states on~$P$ is separating
(i.e., $P$ is a partition logic) if and only if the following conditions
hold:

{\rm(1)} For every atom $a \in P$ there is a two-valued weight $w$ on~$A$
such that $w(a)=1$.

{\rm(2)} For every pair $a_1,a_2 \in P$ of different nonorthogonal atoms
there are two-valued weights $w_+,w_-$ on~$A$ such that $w_+(a_1) = w_+(a_2)$
and $w_-(a_1) \neq w_-(a_2)$.
\end {proposition}


\begin {proof}
$\Rightarrow$: Let $a \in A$. Then $a \neq 0$ and there is a two-valued state
$s$ on~$P$ such that $1=s(a) > s(0)=0$. Let $a_1,a_2 \in A$ such that $a_1
\neq a_2$ and $a_1 \not\perp a_2$. Then also $a_1 \neq a_2'$ and there are
two-valued states $s_-,s_+$ on~$P$ such that and $1=s_-(a_1) > s_-(a_2)=0$,
$1=s_+(a_1) > s_+(a_2')=0$, i.e., $s_+(a_1)=s_+(a_2)$. The rest follows from
\ref{s-w}.

$\Leftarrow$: Let $b_1,b_2 \in P$ such that $b_1 \neq b_2$. Since every
element of $P\setminus \{0,1\}$ is either an atom or a coatom, there are
atoms $a_1,a_2 \in P$ such that $b_1 \in \{0,a_1,a_1',1\}$ and $b_2 \in
\{0,a_2,a_2',1\}$. If $a_1=a_2$ then there are two-valued weights $w_+,w_-$
on~$A$ such that $w_+(a_1)=1$ and $w_-(a_1)=0$. If $a_1 \neq a_2$ then there
are two-valued weights $w_+,w_-$ on~$A$ such that $w_+(a_1) = w_+(a_2)$ and
$w_-(a_1) \neq w_-(a_2)$. In both cases there are, according to
\ref{s-w}, two-valued states $s_+,s_-$ on~$P$ such that either
$s_+(b_1) \neq s_+(b_2)$ or $s_-(b_1) \neq s_-(b_2)$.
\end {proof}


Let us present a lemma, which might simplify to verify criteria in
\ref{separating}.


\begin {lemma} \label {partition}
Let $P$ be a Greechie logic and let $A$ be the set of atoms in~$P$. If $W$
is an at least 3-element set of two-valued weights on~$A$ such that
$\{w^{-1}(1);\; w \in W\}$ is a partition of~$A$ then
  \begin {conditions}
  \item
  For every atom $a \in A$ there is a weight $w \in W$ such that $w(a)=1$.
  \item
  For every pair $a_1,a_2 \in A$ there is a weight $w \in W$ such that
  $w(a_1) = w(a_2)$.
  \end {conditions}
\end {lemma}


\begin {proof}
Obvious.
\end {proof}


Let us remark that in Greechie diagrams it suffices to use the above
conditions for every connected subdiagram separately (weights behave
independently on nonconnected subgraphs). In terms of orthomodular posets we
can use the following important notion.


\begin {definition}
Let ${\cal P}$ be a set of orthomodular posets such that $P_1\cap P_2 =
\{0,1\}$ for every $P_1,P_2 \in {\cal P}$ with $P_1 \neq P_2$. The
{\em horizontal sum\/} $\sum_{P \in {\cal P}} P$ is defined as $(\bigcup_{P
\in {\cal P}} P,\, \bigcup_{P \in {\cal P}} \le_P,\, \bigcup_{P \in {\cal P}}
\mbox{}^{\prime_P},\, 0, 1)$.
\end {definition}


More generally we speak about the horizontal sum of $P_i$, $i \in I$. It is
an abbreviation for saying that we take disjoint representations $\bar P_i$
of $P_i$ (e.g., $\{i\} \times P_i$), identify all $\bar 0_i$ ($i \in I$) and
all $\bar 1_i$ ($i \in I$) and take $\sum_{i \in I} P_i$. It is easy to see
that a horizontal sum of orthomodular posets (orthomodular lattices, resp.)
is an orthomodular poset (orthomodular lattice, resp.) and that a set of
states is nonempty (unital, separating, full, resp.) on a horizontal sum if
and only if it is nonempty (unital, separating, full, resp.) on every
horizontal summand.

In a Greechie diagram every connected subdiagram corresponds to a horizontal
summand. (In particular, every finite 2-atomic block is a horizontal
summand.) On the other hand, horizontal sum of Greechie logics is a Greechie
logic with the Greechie diagram, which is a (disjoint) union of summands
with only one exception---we loose isolated vertices (these correspond to
the trivial orthomodular poset $\{0,1\}$).

The notion of a horizontal sum is a special kind of the notion of {\em
pasting}. We are not interested here in a general setting (see
e.g.~\cite{NR}), thus we describe only special cases how we can obtain a new
Greechie logic using this process. Greechie diagram of the {\em pasting of
Greechie logics\/} $P_i$ ($i\in I$) {\em for atoms\/} $a_i \in P_i$ ($i\in
I$) we obtain as follows: we take disjoint union of Greechie diagrams
of~$P_i$ ($i\in I$), identify vertices corresponding to $a_i$ ($i\in I$)
and, if some $a_i$ ($i \in I$) belong to a 2-atomic block, we delete
necessary vertices corresponding to such $a_i'$ such that the condition~(3)
of \ref{diagram} is fulfilled. Greechie diagram of the {\em pasting of
Greechie logics\/} $P_i$ ($i \in I$) {\em for blocks\/} $B_i\subseteq P_i$
($i\in I$) with the same number of atoms we obtain as follows: we take
disjoint union of Greechie diagrams of~$P_i$ ($i \in I$) and identify edges
corresponding to~$B_i$ ($i \in I$) (I.e., we identify also atoms in these
blocks.) It is easy to see that such pastings of (lattice) Greechie logics
are (lattice) Greechie logics.

The notion of a horizontal sum is related also to the following notion.


\begin {definition}
Let $P$ be an orthomodular poset. The {\em distance\/} $\dist$ on~$P$ is a
mapping $\dist\colon\, P \times P \to \N \cup \{\infty\}$ defined by:
  \begin {eqnarray*}
  \dist(a,b) &=& \inf \big\{ n \in \N;\;
    \mbox{there are blocks $B_1,\ldots,B_n$ in $P$ such that}\\
    &&
    \mbox {$B_i \cap B_{i+1} \neq \{0,1\}$ for $i=0,\ldots,n$,
           $B_0 = \{a\}$, $B_{n+1} = \{b\}$} \big\}.
  \end {eqnarray*}
\end {definition}


The distance function defines the largest decomposition of~$P$ into
horizontal summands---the least summands are maximal subsets of $P \setminus
\{0,1\}$ of elements with finite distances joined with $\{0,1\}$.

The following result we will use in the sequel.


\begin {proposition} \label{noloop-concrete}
Every Greechie logic without any loop has a full set of two-valued states.
\end {proposition}


\begin {proof}
The distance function on~$P$ decompose $P$ into the horizontal sum $\sum_{i
\in I} P_i$ such that the distance of every pair of elements in every
summand is finite. It suffices to prove fullness for every summand. According
to \ref{full}, it suffices, for every $i\in I$ and for every pair $a_1,a_2$
of different nonorthogonal atoms in $P_i$, to find a weight $w$ on the set
$A$ of atoms in~$P_i$ such that $w(a_1) = w(a_2) = 1$. Let us put $A_n= \{a
\in A;\; \dist(a,a_1)=n \}$ for every natural number $n$ and let us define
$w$ by induction:
  \begin {enumerate}
  \item [I.] $w(a_1)=1$.
  \item [II.]
  Let us suppose that there is a natural number $n\ge 0$ such that $w$ is
  defined on $A_0 \cup \cdots \cup A_n$. Every element of~$A_{n+1}$ belongs
  to some block $B$ in~$P_i$ such that $B \cap A_n \neq \emptyset$. For
  every such block $B$ we have $B \cap A_n = \{a_B\}$. If $w(a_B)$=1, we
  put $w|B \cap A \setminus A_n = 0$. If $w(a_B)=0$, we can choose ($B$
  has at least three atoms) properly a $b_B \in B \cap A \setminus A_n$ and
  put $w(b_B)=1$, $w|B \cap A \setminus \{b_B\} = 0$. Properly means that if
  $n = \dist(a_2,a_1)-2$ then $b_B$ is chosen such that it does not belong to
  the same block as $a_2$ and if $n = \dist(a_2,a_1)-1$ then $b_B=a_2$.
  \end {enumerate}
\end {proof}


Let us present examples demonstrating differences in properties of state
space.


\begin {proposition}  \label{statespaces}
Let us consider the following conditions:
  \begin {conditions}
  \item The set of two-valued states is full.
  \item The set of two-valued states is separating but not full.
  \item The set of two-valued states is unital but not separating.
  \item The set of two-valued states is nonempty but not unital.
  \item The set of two-valued states is empty.
  \end {conditions}
For each of the above conditions there is an orthomodular lattice with only
finite 3-atomic blocks which fulfills it.
\end {proposition}


\begin {figure}[ht]
%
\hfill
%  3-atomic B.algebra
\begin {picture}(20,20)(0,-20)
\multiput (0,0)(10,0){3}{\disc}
\put (0,0){\line(1,0){20}}
\put (10,-20){\makebox(0,0)[b]{1}}
\end {picture}
%
\hfill
%  separating, not full set of 2-v. states
\begin {picture}(60,50)(-30,-30)
\edges\vertices
\put (-25,  0){\makebox(0,0){$a$}}
\put (-20, -7){\makebox(0,0){$a_c$}}
\put (-15,-15){\makebox(0,0){$c_a$}}
\put (  0,-15){\makebox(0,0){$c$}}
\put ( 15,-15){\makebox(0,0){$c_b$}}
\put ( 20, -7){\makebox(0,0){$b_c$}}
\put ( 25,  0){\makebox(0,0){$b$}}
\put ( 20,  7){\makebox(0,0){$b_d$}}
\put ( 15, 15){\makebox(0,0){$d_b$}}
\put (  0, 15){\makebox(0,0){$d$}}
\put (-15, 15){\makebox(0,0){$d_a$}}
\put (-20,  7){\makebox(0,0){$a_d$}}
\put (  5,  0){\makebox(0,0){$e$}}
\put (  0,-30){\makebox(0,0)[b]{2}}
\end {picture}
%
\hfill
%    unital, not separating set of 2-v. states
\begin {picture}(40,45)(-20,-30)
\multiput(-10,-10)(0,20){2}{\line(1,0){20}}  \multiput(0,-10)(0,20){2}{\discbig}
\put (-10,-10){\line(1, 1){20}}  \multiput (-10,-10)(20,0){2}{\disc}
\put (-10, 10){\line(1,-1){20}}  \multiput (-10, 10)(20,0){2}{\disc}
\put (0,0){\disc}
\put (-15,-15){\makebox(0,0){$a_1$}}
\put ( 15,-15){\makebox(0,0){$a_2$}}
\put ( 15, 15){\makebox(0,0){$a_3$}}
\put (-15, 15){\makebox(0,0){$a_4$}}
\put ( -5,  0){\makebox(0,0){$b$}}
\put (0,-30){\makebox(0,0)[b]{3}}
\end {picture}
%
\hfill
%    not unital set of 2-v. states
\begin {picture}(30,50)(-15,-20)
\multiput (-10,0)(10,0){3}{\disc}
\put (0,10){\disc}
\put (0,20){\disc}
\put (-5,10){\discbig}
\put ( 5,10){\discbig}
\put (-10,0){\line(1,0){20}}
\put (  0,0){\line(0,1){20}}
\put (-10,0){\line( 1,2){10}}
\put ( 10,0){\line(-1,2){10}}
\put (  0,25){\makebox(0,0){$b$}}
\put (-15,-5){\makebox(0,0){$a_1$}}
\put (  0,-5){\makebox(0,0){$a_2$}}
\put ( 15,-5){\makebox(0,0){$a_3$}}
\put (  0,-20){\makebox(0,0)[b]{4}}
\end {picture}
%
\hfill
%    no unital 2.
\begin {picture}(40,45)(-20,-30)
\multiput(-10,-10)(0,20){2}{\line(1,0){20}}  \multiput(0,-10)(0,20){2}{\discbig}
\multiput(-10,-10)(20,0){2}{\line(0,1){20}}  \multiput(-10,0)(20,0){2}{\discbig}
\put (-10,-10){\line(1, 1){20}}  \multiput (-10,-10)(20,0){2}{\disc}
\put (-10, 10){\line(1,-1){20}}  \multiput (-10, 10)(20,0){2}{\disc}
\put (0,0){\disc}
\put (-15,-15){\makebox(0,0){$a_1$}}
\put ( 15,-15){\makebox(0,0){$a_2$}}
\put ( 15, 15){\makebox(0,0){$a_3$}}
\put (-15, 15){\makebox(0,0){$a_4$}}
\put ( -5,  0){\makebox(0,0){$b$}}
\put (0,-30){\makebox(0,0)[b]{5}}
\end {picture}
%
\hfill\mbox{}
%
\caption {Greechie diagrams of orthomodular posets with finite 3-atomic
blocks demonstrating differences of state spaces (a~\usebox{\shortdiagram}~b
denotes the diagram~2).}
\label {Fstatespaces}
\end {figure}


\begin {proof}
(1) See Fig.~\ref{Fstatespaces}.1. It is a Boolean algebra, which obviously
has a full set of two-valued states.

(2) See Fig.~\ref{Fstatespaces}.2. For every two-valued state $s$ we have
$s(a) + s(b) \le \big(1-s(c_a) + 1-s(d_a) + 1-s(c_b) + 1-s(d_b)\big)/2 =
\big( 2 - s(c) - s(d) \big)/2 \le 3/2$. Hence $s(a)+s(b) \le 1$ and,
according to \ref{full}, this orthomodular lattice has not a full set of
two-valued states. The set $S_1 = \{s_1,s_2,s_3\}$ of states given in
Fig.~\ref{Fstates} fulfills conditions of \ref{partition}. It can be checked
that the set of all two-valued states `symmetric' to some state from~$S$
distinguish different nonorthogonal atoms. Hence the set of two-valued
states fulfills conditions of \ref{separating}. A smaller example of a
separating set of states is given in Fig.~\ref{Fstates}. We can express this
orthomodular lattice as a partition logic on a 6-element set of these
states---see Fig.~\ref{Frepresentations}.1. (Compare with the representation
on the 14-element set of states in~\cite{svosh3}.)

(3) See Fig.~\ref{Fstatespaces}.3. Let us use the previous result. For every
two-valued state $s$ with $s(a_1)=1$ we obtain $s(a_2)=s(b)=0$, hence
$s(a_4)=1$. Using the symmetry we obtain $s(a_1)=s(a_4)$ for every
two-valued state, hence the set of two-valued states is not separating. The
unitality can be verified routinely.

(4) See Fig.~\ref{Fstatespaces}.4. For every two-valued state $s$ there is an
$i \in \{1,2,3\}$ such that $s(a_i)=1$ and therefore $s(b)=0$. Hence, the
set of two-valued states is not unital. Existence of a two-valued state can
be verified routinely. (Let us note that if we paste `sides of the triangle'
not only for $b$ but for the whole block we obtain a smaller example with
25~atoms.)

(5) See Fig.~\ref{Fstatespaces}.5. According to part~(3) of this proof,
$s(a_1) = s(a_2) = s(a_3) = s(a_4)$ for every two-valued state~$s$. Hence
all these values are equal to~0 and $s(b) = 1$. The desired example we
obtain by pasting this orthomodular lattice with the orthomodular lattice
from Fig.~\ref{Fstatespaces}.4 for~$b$'s or, more effectively, by pasting
for blocks containing $b$'s and $a_2$'s.

\end {proof}

%
\newcommand{\smalldiag}[2]{\begin{picture}(40,30)(-20,-20)
  \edges#2\put(0,-20){\makebox(0,0)[b]{$s_#1$}}\end{picture}}
%
\begin {figure}[ht]
%
\smalldiag{1}{
  \point{-20}{0}\point{0}{0}\point{10}{10}\point{10}{-10}}
\hfill\smalldiag{2}{
  \point{-15}{5}\point{-5}{-10}\point{0}{10}\point{10}{-10}}
\hfill\smalldiag{3}{
  \point{-15}{-5}\point{-10}{10}\point{0}{-10}\point{15}{-5}\point{15}{5}}
\hfill\smalldiag{4}{
  \point{-15}{5}\point{-10}{10}\point{0}{0}\point{10}{10}\point{15}{-5}}
\hfill\smalldiag{5}{
  \point{-15}{-5}\point{-15}{5}\point{0}{10}\point{10}{-10}\point{15}{5}}
\hfill\smalldiag{6}{
  \point{-10}{10}\point{-10}{-10}\point{0}{0}\point{15}{-5}\point{15}{5}}
%
\caption {Separating set of two-valued states on an orthomodular lattice
from Fig.\protect\ref{Fstatespaces}.2. (only atoms in which the corresponding
state is equal to 1 are marked).}
\label {Fstates}
\end {figure}
%

%
{
\unitlength 1mm
\begin {figure}[ht]
%
\begin {picture}(60,45)(-30,-25)
\edges\vertices\footnotesize
\put (-24,  0){\makebox(0,0)[r]{$\{1\}$}}
\put (-17, -7){\makebox(0,0)[r]{$\{3,5\}$}}
\put ( -9,-15){\makebox(0,0)[r]{$\{2,4,6\}$}}
\put (  0,-15){\makebox(0,0)[c]{$\{3\}$}}
\put (  9,-15){\makebox(0,0)[l]{$\{1,5\}$}}
\put ( 17, -7){\makebox(0,0)[l]{$\{3,4,6\}$}}
\put ( 24,  0){\makebox(0,0)[l]{$\{2\}$}}
\put ( 17,  7){\makebox(0,0)[l]{$\{3,5,6\}$}}
\put (  9, 15){\makebox(0,0)[l]{$\{1,4\}$}}
\put (  0, 15){\makebox(0,0)[c]{$\{2,5\}$}}
\put ( -9, 15){\makebox(0,0)[r]{$\{3,6\}$}}
\put (-17,  7){\makebox(0,0)[r]{$\{2,4,5\}$}}
\put (  2,  0){\makebox(0,0)[l]{$\{1,4,6\}$}}
\put(0,-25){\makebox(0,0)[b]{\normalsize1}}
\end {picture}
%
\hfill
%
\begin {picture}(70,45)(-35,-25)
\catcode`\S=\active\defS{\sqrt2} % abbr
\edges\vertices\footnotesize
\put (-24,  0){\makebox(0,0)[r]{$( S,-1, 0)$}}
\put (-17, -7){\makebox(0,0)[r]{$( 1, S, 3)$}}
\put ( -9,-15){\makebox(0,0)[r]{$( 1, S,-1)$}}
\put (  0,-15){\makebox(0,0)[c]{$( 1, 0, 1)$}}
\put (  9,-15){\makebox(0,0)[l]{$(-1,-S, 1)$}}
\put ( 17, -7){\makebox(0,0)[l]{$(-1, S,-3)$}}
\put ( 24,  0){\makebox(0,0)[l]{$( S, 1, 0)$}}
\put ( 17,  7){\makebox(0,0)[l]{$(-1, S, 3)$}}
\put (  9, 15){\makebox(0,0)[l]{$(-1, S,-1)$}}
\put (  0, 15){\makebox(0,0)[c]{$( 1, 0,-1)$}}
\put ( -9, 15){\makebox(0,0)[r]{$( 1, S, 1)$}}
\put (-17,  7){\makebox(0,0)[r]{$( 1, S,-3)$}}
\put (  2,  0){\makebox(0,0)[l]{$( 0, 1, 0)$}}
\put(0,-25){\makebox(0,0)[b]{\normalsize2}}
\end {picture}
%
\caption {Various representations of an orthomodular lattice from
Fig.~\protect\ref{Fstatespaces}.2.}
\label {Frepresentations}
\end {figure}
}




\section {Subortholattices of $\HL$}


There are only several types of finite subortholattices of~$\HL$. The
following characterization of finite subortholattices of~$\HL$ seems to be in
a common knowledge (see e.g.~\cite[Example~1.5.3]{Kalmbach-83}), but we do
not know a proper reference for its proof.

\begin {lemma} \label {3lines}
Let $L$ be a subortholattice of~$\HL$ and let lines $a_1,a_2,a_3,b \in L$ be
such that $a_1,a_2,a_3$ are mutually orthogonal and $b \not\perp
a_1,a_2,a_3$. Then there is a line $c \in L$ such that $c \not\perp a_3$ and
the angle $\angle (c,a_3)$ is greater than $\angle (b,a_3)$.
\end {lemma}


\begin {proof}
Let us choose the system of coordinates such that $a_1 = \Sp (1,0,0)$, $a_2
= \Sp (0,1,0)$, $a_3 = \Sp (0,0,1)$, $b = \Sp (x,y,z)$ such that $x,y,z >0$.
Since $L$ is a subortholattice of~$\HL$, the following elements belong
to~$L$:
  \begin {eqnarray*}
  \bar b &=& (a_1 \lor a_2) \land b'              = \Sp (y,-x,0) \\
       c &=& (a_1 \lor a_3) \land (b \lor \bar b) = \Sp (x+\frac{y^2}{x},0,z).
  \end {eqnarray*}
Hence,
  \begin {eqnarray*}
  0 < \cos \angle (c,a_3) &=& \frac {z}{\sqrt { (x+\frac {y^2}{x})^2 + z^2} }
                          <  \frac {z}{\sqrt { x^2 + y^2 + z^2} }
                          = \cos \angle (b,a_3)
  \end {eqnarray*}
\end {proof}


\begin {theorem} \label {finitesubortholattices}
Let $L \subset \HL$ be a finite orthomodular lattice. Then $L$ is a
subortholattice of~$\HL$ if and only if exactly one of the following
possibilities is fulfilled:

{\rm (1)} $L = \{0,1\}$, i.e., $L$ is a 1-atomic Boolean algebra.

{\rm (2)} $L = \{0,a,a',1\}$ for some line $a \in \HL$, i.e., $L$ is a
2-atomic Boolean algebra.

{\rm (3)} $L = \{0,a_1,a_2,a_3,a_1',a_2',a_3',1\}$ for some orthogonal set
$\{a_1,a_2,a_3\}$ of lines in~$\HL$, i.e., $L$ is a 3-atomic Boolean
algebra.

{\rm (4)} $L = \{0,a,a',1\} \cup G \cup G' \cup \{a \lor b;\; b \in G\} \cup
\{a' \land b';\; b \in G\}$ for some line $a \in \HL$ and some at least
2-element set $G$ of mutually nonorthogonal atoms orthogonal to~$a$, i.e.,
$L$ is a finite pasting of at least two 3-atomic Boolean algebras for a
given atom.
\end {theorem}


\begin {proof}
It is easy to see that each of these conditions excludes the others and
gives a subortholattice of~$\HL$. Let us suppose that there is a finite
subortholattice $L$ of~$\HL$ which fulfills no condition (1)--(4) and
seek a contradiction. There are three mutually nonorthogonal lines $a,b,c
\in L$. Let $d_3 = (a \lor b)' \in L$. Since $L$ is finite, there is a line
$e \in L$ such that $\angle (e,d_3)$ is the greatest among all lines
from~$L$ nonorthogonal to~$d_3$. Since $a \not\perp b$ there is a $d_1 \in
\{a,b\}$ such that $d_1 \not\perp e, e' \land d_3'$. Let us put $d_2 = d_1'
\land d_2' \in L$. Hence, lines $d_1,d_2,d_3$ are mutually orthogonal and $e
\not\perp d_1,d_2,d_3$. According to \ref{3lines}, there is an element $f
\in L$ such that $f \not\perp d_3$ and $\angle (f,d_3) < \angle
(e,d_3)$---this contradicts to the selection of~$e$.
\end {proof}


Greechie diagrams of finite subortholattices of~$\HL$ are given in
Fig.~\ref{Fsubortholattices}.


\begin {figure}[ht]
%
\hfill
%
\begin {picture}(0,10)(0,-10)
\point {0}{0}
\put (0,-10){\makebox(0,0)[b]{1}}
\end {picture}
%
\hfill
%
\begin {picture}(10,10)(0,-10)
\point { 0}{0}
\point {10}{0}
\put (0,0){\line(1,0){10}}
\put (5,-10){\makebox(0,0)[b]{2}}
\end {picture}
%
\hfill
%
\begin {picture}(20,10)(0,-10)
\multiput (0,0)(10,0){3}{\disc}
\put (0,0){\line(1,0){20}}
\put (10,-10){\makebox(0,0)[b]{3}}
\end {picture}
%
\hfill
%
\begin {picture}(20,20)(-20,-10)
\multiput (0,0)(-10,0){3}{\disc}
\put (0,0){\line(-1,0){20}}
\multiput (0,0)(-9,4.5){3}{\disc}
\put (0,0){\line(-2,1){18}}
\put (-10,-10){\makebox(0,0)[b]{4.2}}
\end {picture}
%
\hfill
%
\begin {picture}(20,30)(-20,-10)
\multiput (0,0)(-10,0){3}{\disc}
\put (0,0){\line(-1,0){20}}
\multiput (0,0)(-9,4.5){3}{\disc}
\put (0,0){\line(-2,1){18}}
\multiput (0,0)(-4.5,9){3}{\disc}
\put (0,0){\line(-1,2){9}}
\put (-10,-10){\makebox(0,0)[b]{4.3}}
\end {picture}
%
\hfill
%
\begin {picture}(0,10)(0,-10)
\put (0,0){\makebox(0,0){\dots}}
\end {picture}
%
\hfill
%
\begin {picture}(40,20)(-20,-10)
\multiput (0,0)(-10,0){3}{\disc}
\put (0,0){\line(-1,0){20}}
\multiput (0,0)(-9,4.5){3}{\disc}
\put (0,0){\line(-2,1){18}}
\multiput (0,0)(9,4.5){3}{\disc}
\put (0,0){\line(2,1){18}}
\put (0,10){\makebox(0,0){\dots}}
\put (0,-10){\makebox(0,0)[b]{4.n}}
\end {picture}
%
\hfill
%
\begin {picture}(0,10)(0,-10)
\put (0,0){\makebox(0,0){\dots}}
\end {picture}
%
\hfill\mbox{}
\caption {Greechie diagrams of finite subortholattices of~$\HL$.}
\label {Fsubortholattices}
\end {figure}


\begin {corollary}  \label{finitesubOL-full}
Every finite subortholattice of~$\HL$ has a full set of two-valued states.
\end {corollary}


\begin {proof}
It follows from \ref{finitesubortholattices} and \ref{noloop-concrete}.
\end {proof}


As concerns infinite subortholattices of~$\HL$, there is a countable
subortholattice of~$\HL$ without any two-valued states (e.g., generated by
finite sets without any two-valued state---see \ref{3subset}). On the other
hand, there are infinite subortholattices with a full set of two-valued
states, e.g.\ infinite pastings of 3-atomic Boolean algebras for a given atom
(compare condition~(4) of \ref{finitesubortholattices}). It seems to be an
open problem whether there is an infinite subortholattice of~$\HL$ which is
not of this type and which has a two-valued state. Moreover, there might be
an interesting connection between the nonexistence of a two-valued state and
density in $\R^3$. This might give better insight into the nature of
subortholattices of~$\HL$ and the connection with famous Gleason
theorem~\cite{Gleason,Ptak-Pulmannova}, which (among other things) states
that there is no two-valued state on~$\HL$.

It should be noted that Greechie diagrams of subortholattices of~$\HL$ are
relatively `complex'---the distance of every pair of elements is at most 2
(every pair of different lines has a common orthogonal line). Hence, it is
usually difficult to give a Greechie diagram of an infinite subortholattice
of~$\HL$.




\section {Realizability in $\HL$}


The study of finite suborthoposets of~$\HL$ is more complicated. We would
like to know whether a Greechie logic is orthorepresentable in~$\HL$. The
first problem erases with the intrinsic geometrical structure of~$\HL$.


\begin {definition}
Let $P$ be an orthomodular poset. We say that $P$ is {\em weakly
realizable\/} in~$\HL$ if there is a mapping $h\colon\, P \to \HL$ such
that, for every $a,b \in P$:
  \begin {conditions}
  \item $h(0)=0$,
  \item $h(a')=h(a)'$,
  \item $h(a) \le h(b)$ whenever $a \le b$,
  \item $h(a) \neq 0$ whenever $a \neq 0$.
  \end {conditions}
If, moreover, the mapping $h$ fulfills for every $a,b \in P$:
  \begin {conditions}
  \item [(4')] $h(a) \neq h(b)$ whenever $a \neq b$
  \end {conditions}
we say that $P$ is {\em realizable}. The set $h(P)$ is called a ({\em
weak\/}) {\em realization\/} of~$P$ in~$\HL$.
\end {definition}

Weak realizability means that all orthogonality relations remains true in
the images and, since every nonzero element has a nonzero image, if the set
of two-valued states on $G \subseteq P$ is empty (not unital, resp.) then
the set of two-valued states on~$h(G)$ is empty (not unital. resp.), too.
Realizability means that, moreover, the mapping is one-to-one. Hence, if the
set of two-valued states on $G \subseteq P$ is not separating (full, resp.),
then the set of two-valued states on~$h(G)$ is not separating (full, resp.),
too. A realization need not be a suborthoposet because a new orthogonal
pairs might appear in the images.

Let us give a characterization of orthomodular posets weakly realizable
in~$\HL$.


\begin {lemma}
Let $P_{\cal P}$ be the pasting of a set ${\cal P}$ of orthomodular posets
and let there is a mapping $h\colon\, P_{\cal P} \to \HL$ such that $h(P)$
is a weak realization of~$P$ for every $P\in {\cal P}$. Then $h(P_{\cal P})$
is a weak realization of~$P_{\cal P}$ in~$\HL$. In particular, every
horizontal sum of orthomodular posets weakly realizable in~$\HL$ is weakly
realizable in~$\HL$.
\end {lemma}


\begin {proof}
Obvious.
\end {proof}

\begin {proposition}
An orthomodular poset is weakly realizable in~$\HL$ if and only if every its
block is finite and at most 3-atomic.
\end {proposition}


\begin {proof}
$\Rightarrow$: Every orthogonal set of nonzero elements in an orthomodular
poset~$P$ corresponds to an orthogonal set of nonzero elements in~$\HL$.
Since such a set in~$\HL$ is at most 3-element, every block of~$P$ is finite
with at most three atoms.

$\Leftarrow$: Let $P$ be an orthomodular poset with only finite at most
3-atomic blocks. Let us decompose $P$ into the horizontal sum $\sum_{i \in
I} P_i$ of minimal horizontal summands. Let us choose a line $l \in \HL$ and
let us define a mapping $h_i$ for every $i \in I$ as follows: $h(0)=0$,
$h(1)=1$; if $P_i$ is 4-element, then let us take an atom $a_i \in P_i$ and
put $h(a_i)=l$, $h(a_i')=l'$; if $P_i$ has more than four elements then
every its block has three atoms and we put $h(a)=l$, $h(a')=l'$ for every
atom $a \in P_i$. It is easy to see that $h_i(P_i)$ is a weak realization
of~$P_i$ in~$\HL$ and that $\bigcup_{i \in I} h_i(P_i)$ is a weak realization
of~$P$ in~$\HL$.
\end {proof}


The situation with realizability is more difficult and we do not know a
characterization of it. Some results we will present in the next section.
Let us present now another necessary condition.


\begin {proposition}
Every orthomodular poset realizable in~$\HL$ is a lattice.
\end {proposition}


\begin {proof}
Let us suppose that $P$ is an orthomodular poset with a loop of order 4
realizable in~$\HL$ and seek a contradiction. There are nonzero mutually
different elements $a_1 \perp a_2 \perp a_3 \perp a_4 \perp a_1$ in~$P$ (see
Fig.~\ref{Fnonrealizable}.2). Since for every pair of different nonzero
elements there is only one nonzero element in~$\HL$ orthogonal to them, $a_1
= a_3$---a contradiction.
\end {proof}


\begin {figure}[ht]
%
\hfill
%
\begin {picture}(30,20)(0,-20)
\multiput (0,0)(10,0){4}{\disc}
\put (0,0){\line(1,0){30}}
\put (15,-20){\makebox(0,0)[b]{1}}
\end {picture}
%
\hfill
%
\begin {picture}(40,50)(-20,-30)
\multiput(-10,-10)(0,20){2}{\line(1,0){20}}  \multiput (-10,-10)(10,0){3}{\disc}
\multiput(-10,-10)(20,0){2}{\line(0,1){20}}  \multiput (-10, 10)(10,0){3}{\disc}
\multiput(-10,  0)(20,0){2}{\disc}
\put (-15,-15){\makebox(0,0){$a_1$}}
\put ( 15,-15){\makebox(0,0){$a_2$}}
\put ( 15, 15){\makebox(0,0){$a_3$}}
\put (-15, 15){\makebox(0,0){$a_4$}}
\put (0,-30){\makebox(0,0)[b]{2}}
\end {picture}
%
\hfill
%
\begin {picture}(40,50)(-20,-20)
\put (-10, 0){\line(1,0){20}}    \multiput (-10,0)(10, 0){3}{\disc}
\put (  0, 0){\line(0,1){20}}    \multiput (  0,0)( 0,10){3}{\disc}
\put (  0,20){\line(-1,-2){10}}  \put ( -5,10){\discbig}
\put (  0,20){\line( 1,-2){10}}  \put (  5,10){\discbig}
\put (-15, -5){\makebox(0,0){$a_1$}}
\put (  0, -5){\makebox(0,0){$a_2$}}
\put ( 15, -5){\makebox(0,0){$a_3$}}
\put (  0, 25){\makebox(0,0){$b$}}
\put (  0,-20){\makebox(0,0)[b]{3}}
\end {picture}
%
\hfill\mbox{}
%
\caption {Greechie diagrams of some orthomodular posets nonrealizable in~$\HL$
(a~\usebox{\shortdiagram}~b is an abbreviation of the Greechie diagram in
Fig.~\protect\ref{Fstatespaces}.2.).}
\label {Fnonrealizable}
\end {figure}


Examples of orthomodular posets nonrealizable in~$\HL$ are given in
Fig.~\ref{Fnonrealizable}. The first has a 4-atomic block, the second is not
a lattice. The third example is much more subtle an depends on the following
intrinsic property of~$\HL$.


\begin {lemma}  \label {realizabilityoflogic}
Let $L$ be a realization of an orthomodular lattice given in
Fig.~\ref{Fstatespaces}.2. Then $\angle (a,b) \in \langle \arccos \frac 13,
\frac \pi 2)$. On the other hand, for every $\alpha \in \langle \arccos
\frac 13, \frac \pi 2)$ there is a realization of~$L$ such that $\angle
(a,b) = \alpha$.
\end {lemma}


\begin {proof}
(See also~\cite{Kochen-Specker}). Let us choose a coordinate system such
that $c = \Sp (1,0,0)$, $d = \Sp (0,1,0)$. Hence $e = \Sp (0,0,1)$. Since
$c_a \perp c$ and $d_b \perp d$, there are $x,y \in \R \setminus \{0\}$ such that
  $$
  c_a = \Sp (0,y,1), \qquad
  d_b = \Sp (x,0,1).
  $$
Since $c_b \perp c,c_a$ and $d_a \perp d,d_b$, $a \perp c_a,d_a$ and $b \perp
c_b,d_b$, we obtain
  \begin {eqnarray*}
  c_b = \Sp (0,-1,y), &\enspace&
  d_a = \Sp (-1,0,x), \\
  a   = \Sp (xy,-1,y), &\enspace&
  b   = \Sp (-1,xy,x).
  \end {eqnarray*}
Thus, using an elementary calculus,
  $$
  \cos \angle (a,b) = \frac {|xy|}{\sqrt {(1+x^2+x^2y^2)(1+y^2+x^2y^2)}}
    \in (0, 1/3 \rangle.
  $$
For an arbitrary $\alpha \in \langle \arccos \frac 13, \frac \pi2)$ we can
solve this equation and obtain, e.g.,
  $$
  x = y = \sqrt { \frac {1/\cos\alpha-1}{2} -
             \sqrt {\left( \frac {1/\cos\alpha}{2} \right)^2-1}}.
  $$
\end {proof}



For $\alpha = \arccos \frac 13$ we have exactly one realization (two
different solutions given by the symmetry of the Greechie diagram). In
Fig.~\ref{Frepresentations}.2 there is an example such that symmetries of
the realization are easily seen (with respect to the axis $o$ of $a$ and $b$
and to planes $\Sp\{a,b\}$, $\Sp\{o,a\times b\}$). For $\alpha \in
(\arccos \frac 13, \frac \pi2)$ we have two different realizations (each
symmetric with respect to the axis of $a$ and $b$).

The orthomodular lattice given in Fig.~\ref{Fnonrealizable}.3 is not
realizable, because for every triple $a_1,a_2,a_3 \in \HL$ of mutually
orthogonal nonzero elements and for every $b \in \HL$ there is an $i \in
\{1,2,3\}$ such that $\angle (b,a_i) \le \arccos \frac 1{\sqrt 3}$.

Let us note that in~\cite{Kochen-Specker} the above lemma is stated also for
$\alpha = \frac {\pi}{2}$. This is not true, because then either $x=0$ or
$y=0$ and we obtain only a weak realization.




\section {Suborthoposets of~$\HL$}


We would like to present examples of orthomodular lattices
orthorepresentable in~$\HL$. To ensure that an orthomodular lattice is
orthorepresentable in~$\HL$ it suffices to find its realization in~$\HL$
such that there are not ordered (orthogonal, resp.) pairs other than it was
intended. E.g., it can be easily verified that an orthomodular lattice given
in Fig.~\ref{Fstatespaces}.2 is orthorepresentable in~$\HL$ (see
Fig.~\ref{Frepresentations}.2). We present partial results which
orthomodular lattices are orthorepresentable (realizable, resp.) in~$\HL$.
The idea of their proofs is that we can find uncountable many (continuum)
weak realizations while only for a countable many of them some images
coincide or, in case of orthorepresentability, give a new ordered
(orthogonal, resp.) pair.

We show that there is a large class of infinite suborthoposets of~$\HL$ with
a full set of two-valued states.


\begin {proposition} \label {representablesum}
Every horizontal sum of countable many countable orthomodular lattices
orthorepresentable (realizable, resp.) in~$\HL$ is orthorepresentable
(realizable, resp.) in~$\HL$.
\end {proposition}


\begin {proof}
It suffices to prove this proposition for two summands (we can proceed by
induction). Let $L_1,L_2$ be their orthorepresentations (realizations,
resp.) in~$\HL$. It suffices to prove that we can rotate $L_2$ to~$\bar L_2$
such that $a_1 \not\subseteq a_2$ and $a_2 \not\subseteq a_1$ for every $a_1
\in L_1 \setminus \{0,1\}$ and for every $a_2 \in \bar L_2 \setminus
\{0,1\}$, i.e. such that $l \not\subseteq \bigcup (L_1 \setminus \{1\})$ for
every line $l \in \bar L_2$. If $L_2 = \{0,1\}$ then the proof is complete.
Let us suppose that $L_2 \neq \{0,1\}$. Then there is a line $l_0 \in L_2$.
Since $\bigcup (L_1 \setminus \{1\}) \neq \R^3$ there is a line $\bar l_0
\not\subseteq \bigcup (L_1 \setminus \{1\})$ and we can rotate $L_2$ such
that $l_0$ goes to $\bar l_0$. Rotating now the image of $L_2$ around $\bar
l_0$ we obtain an uncountable many possibilities while for only a countable
many of them there is a line $\bar l \in \bar L_2$ such that $\bar l
\subseteq \bigcup (L_1 \setminus \{1\})$. Indeed, for every $l \in L_2$ all
possible positions of~$\bar l$ in a unit sphere $S(0,1)$ in~$\R^3$ form a
circle $C$ with the center on $\bar l_0$ while, for every $a \in L_1
\setminus \{1\}$, $a \cap S(0,1)$ is either a 2-element set ($a$ is a line)
or a circle not identical to~$C$; hence $a \cap S(0,1) \cap C$ is at most
2-element.
\end {proof}


\begin {proposition} \label{representablepasting}
Every pasting for an atom of a pair of countable orthomodular lattices
orthorepresentable (realizable, resp.) in~$\HL$ is orthorepresentable
(realizable, resp.) in~$\HL$.
\end {proposition}


\begin {proof}
If we paste for an atom in a 2-atomic block then we obtain a horizontal sum
and the proof follows from~\ref{representablesum}. Let us suppose that we
paste for atoms in 3-atomic blocks. Let $L_1,L_2$ be orthorepresentations
(realizations, resp.) in~$\HL$ of given orthomodular lattices such that $L_1
\cap L_2 \ni l_0$ where $l_0$ represents the atom in both $L_1,L_2$ for
which we paste. It suffices to prove that there is a rotation $\bar L_2$
of~$L_2$ around the line~$l_0$ such that $a_1 \not\subseteq a_2$ and $a_2
\not\subseteq a_1$ for every $a_1 \in L_1 \setminus \{0,1,l_0,l_0'\}$ and
for every $a_2 \in L_2 \setminus \{0,1,l_0,l_0'\}$, i.e., such that $l
\not\subseteq \bigcup (L_1 \setminus \{1,l_0'\})$ for every line $l \in \bar
L_2$. This gives only countable many restrictions to uncountable possible
positions of~$\bar L_2$, hence the proof is complete.
\end {proof}


\begin {corollary}
Every countable Greechie logic with at most 3-atomic blocks and without
any loop is orthorepresentable in~$\HL$.
\end {corollary}


\begin {proof}
Every countable Greechie logic with only finite at most 3-atomic blocks is a
horizontal sum of subsequent countable pastings of finite 3-atomic Boolean
algebras for an atom. The rest follows from \ref{finitesubortholattices},
\ref{representablepasting} (using the induction) and \ref{representablesum}.
\end {proof}


According to \ref{noloop-concrete}, Greechie logics from the above Corollary
have a full set of two-valued states.


\begin {lemma} \label{representablepasting2}
Let $L_1$ be a countable orthomodular lattice orthorepresentable
(realizable, resp.) in~$\HL$ and $L_2$ be an orthomodular lattice given in
Fig.~\ref{Frealizable}.1 such that $L_1 \cap L_2 = \{0,a,b,a',b',1\}$ and $a
\neq b$ are nonorthogonal atoms in~$L_1$ (in its realization, resp.). Then
the pasting of $L_1$ and~$L_2$ is orthorepresentable (realizable, resp.)
in~$\HL$.
\end {lemma}


\begin {figure}[ht]
%
\hfill
%   3 chain
\begin {picture}(40,50)(-20,-20)
\put (-10, 0){\line(0,1){20}}  \multiput (-10,0)(0,10){3}{\disc}
\put ( 10, 0){\line(0,1){20}}  \multiput ( 10,0)(0,10){3}{\disc}
\put (-10,20){\line(1,0){20}}  \put (0,20){\disc}
\put (-15,-5){\makebox(0,0){$a$}}
\put (-15,10){\makebox(0,0){$a_c$}}
\put (-15,25){\makebox(0,0){$c_a$}}
\put (  0,25){\makebox(0,0){$c$}}
\put ( 15,25){\makebox(0,0){$c_b$}}
\put ( 15,10){\makebox(0,0){$b_c$}}
\put ( 15,-5){\makebox(0,0){$b$}}
\put (  0,-20){\makebox(0,0)[b]{1}}
\end {picture}
%
\hfill
%   no separating
\begin {picture}(40,45)(-20,-30)
\multiput(-10,-10)(0,20){2}{\line(1,0){20}}  \multiput(0,-10)(0,20){2}{\discbig}
\put (-10,-10){\line(1, 1){20}}  \multiput (-10,-10)(10,0){3}{\disc}
\put (-10, 10){\line(1,-1){20}}  \multiput (-10, 10)(10,0){3}{\disc}
\put (  0,-10){\line(0, 1){20}}  \put (0,0){\disc}
\put (-15,-15){\makebox(0,0){$a_1$}}
\put ( 15,-15){\makebox(0,0){$b_1$}}
\put ( 15, 15){\makebox(0,0){$a_2$}}
\put (-15, 15){\makebox(0,0){$b_2$}}
\put (  5,  0){\makebox(0,0){$f$}}
\put (0,-30){\makebox(0,0)[b]{2}}
\end {picture}
%
\hfill
%    no unital
\newsavebox{\web}% base of web in 2 pictures
\savebox{\web}{%
\begin{picture}(40,35)(-20,0)
\put (0,15){\line( 0, 1){20}}  \multiput (0,15)(  0,  10){3}{\disc}
\put (0,15){\line( 4, 1){20}}  \multiput (0,15)( 10, 2.5){3}{\disc}
\put (0,15){\line(-4, 1){20}}  \multiput (0,15)(-10, 2.5){3}{\disc}
\put (0,15){\line( 2,-3){10}}  \multiput (0,15)(  5,-7.5){3}{\disc}
\put (0,15){\line(-2,-3){10}}  \multiput (0,15)( -5,-7.5){3}{\disc}
\put (  0,35  ){\line( 4,-3){20}}  \put ( 10 ,27.5){\discbig}
\put (  0,35  ){\line(-4,-3){20}}  \put (-10 ,27.5){\discbig}
\put (  5, 7.5){\line( 1, 2){ 5}}  \put ( 7.5,12.5){\discbig}
\put ( -5, 7.5){\line(-1, 2){ 5}}  \put (-7.5,12.5){\discbig}
\put (-10, 0  ){\line( 1, 0){20}}  \put (  0 , 0  ){\discbig}
\end{picture}}%
\begin {picture}(60, 65)(-30,-20)
\put (0,0){\makebox(0,0)[b]{\usebox{\web}}}
\put (  0,40){\makebox(0,0){$a_1$}}
\put ( 27,20){\makebox(0,0){$a_5$}}
\put (-27,20){\makebox(0,0){$a_2$}}
\put ( 15,-5){\makebox(0,0){$a_4$}}
\put (-15,-5){\makebox(0,0){$a_3$}}
\put (  7,22){\makebox(0,0){$\bar a_5$}}
\put ( -7,21){\makebox(0,0){$\bar a_2$}}
\put ( 11, 7){\makebox(0,0){$\bar a_4$}}
\put (-11, 7){\makebox(0,0){$\bar a_3$}}
\put (0,-20){\makebox(0,0)[b]{3}}
\end {picture}
%
\hfill
%
\begin {picture}(60,65)(-30,-20)
\put (0,0){\makebox(0,0)[b]{\usebox{\web}}}
\put (  0,25  ){\line( 4,-3){10}}  \put (  5,21.2){\discbig}
\put (  0,25  ){\line(-4,-3){10}}  \put ( -5,21.2){\discbig}
\put ( 10, 0  ){\line( 1, 2){10}}  \put ( 15,10  ){\discbig}
\put (-10, 0  ){\line(-1, 2){10}}  \put (-15,10  ){\discbig}
\put ( -5, 7.5){\line( 1, 0){10}}  \put (  0, 7.5){\discbig}
\put (  0,40){\makebox(0,0){$a_1$}}
\put ( 27,20){\makebox(0,0){$a_5$}}
\put (-27,20){\makebox(0,0){$a_2$}}
\put ( 15,-5){\makebox(0,0){$a_4$}}
\put (-15,-5){\makebox(0,0){$a_3$}}
\put (0,-20){\makebox(0,0)[b]{4}}
\end {picture}
%
\hfill
%
\begin {picture}(50,60)(-30,-40)
\put (-10, 20){\line( 1, 0){20}}  \multiput (-10, 20)( 10, 0){3}{\disc}
\put (-10,-20){\line( 1, 0){20}}  \multiput (-10,-20)( 10, 0){3}{\disc}
\put (  0,-10){\line(-2, 1){20}}  \multiput (  0,-10)(-10, 5){3}{\disc}
\put (  0,-10){\line( 2, 1){20}}  \multiput (  0,-10)( 10, 5){3}{\disc}
\put (-10,-10){\line( 0, 1){20}}  \multiput (-10,-10)(  0,20){2}{\disc}
\put (-20,  0){\line( 1, 2){10}}  \put (-15, 10){\disc}
\put (-20,  0){\line( 1,-2){10}}  \put (-15,-10){\disc}
\put ( 20,  0){\line(-1, 2){10}}  \put ( 15, 10){\disc}
\put ( 20,  0){\line(-1,-2){10}}  \put ( 15,-10){\disc}
\put (-10, 10){\line( 1, 1){10}}  \put ( -5, 15){\disc}
\put (-10,-10){\line( 1,-1){10}}  \put ( -5,-15){\disc}
\put (-25, 0){\makebox(0,0){$a$}}
\put ( 10, 0){\makebox(0,0){$b$}}
\put (  0,-40){\makebox(0,0)[b]{5}}
\end {picture}
%
\hfill\mbox{}
%
\caption {Greechie diagrams of orthomodular lattices weakly realizable
in~$\HL$.}
\label {Frealizable}
\end {figure}



\begin {proof}
Let us suppose that $L_1$ is an orthorepresentation (realization, resp.)
in~$\HL$ of a given orthomodular lattice. If $a$ ($b$, resp.) is a
2-dimensional subspace of~$\HL$ then $a$ ($b$, resp.) is a part of a
4-element horizontal summand and this summand might be considered as a part
of~$L_2$. The proof then follows from \ref{representablepasting}. Let
us suppose that $a,b$ are lines. Let us consider all atoms $c_a \le a'$. We
have uncountable many possibilities which fill in the unit sphere $S(0,1)$ a
circle~$C_a$. Of course, $c_a \le a'$ and $a_c = a' \land c_a' \le a'$ but
all other ordering of $c_a$ and $a_c$ with elements of~$L_1 \setminus
\{0,1\}$ can be excluded if we exclude a countable many possibilities.
Similarly, if positions of~$c_a$ fill a circle~$C_a$ then positions of $c_b
\perp c_a,b$ fill a circle $C_b \subset b'$ ($a \not\perp b$). Again, there
is only a countable many positions of~$c_a$ for which either $c_b$ or $b_c =
b' \land c_b'$ is ordered with some element of $L_1 \setminus \{0,1,b'\}$.
Finally, it can be shown that positions of~$c$ fill a smooth curve on
$S(0,1)$ which is not a circle. Hence, there is a possibility to choose
$c_a$ such that we obtain the desired orthorepresentation (realization,
resp.).
\end {proof}


\begin {proposition}
Let $n \ge 5$ be a natural number and let $B_1,\ldots,B_n$ be finite
3-atomic Boolean algebras such that $B_i \cap B_{i+1} = \{0,a_i,a_i',1\}$
for every $i\in \{1,\ldots,n\}$, where $B_{n+1} = B_1$ and $a_1,\ldots,a_n$
are mutually different atoms. Then the pasting of $\{B_1,\ldots,B_n\}$
(so-called {\em $n$-cycle\/}) is orthorepresentable in~$\HL$.
\end {proposition}


\begin {proof}
It follows from \ref{representablepasting} and from
\ref{representablepasting2}.
\end {proof}




\section {Kochen--Specker type configurations}


We will give several examples of Kochen--Specker type configurations which
arise from Greechie diagrams. Some of these examples has been already used
in the literature in the attempt to find a subset of~$\HL$ without a
two-valued state. We present the connection to Greechie diagrams (this gives
a better geometric insight), show a nonexistence of a `large' set of
two-valued states for various concepts, and, moreover, we do not stop in
proving weak realizability but we discuss the real number of elements.


\begin {proposition} \label{nofull}
There is a finite suborthoposet of~$\HL$ such that the set of two-valued
states on it is not full.
\end {proposition}


\begin {proof}
Let us consider a suborthoposet $L$ of~$\HL$ given in
Fig.~\ref{Frepresentations}.2. It is an orthorepresentation of an
orthomodular lattice given in Fig.~\ref{Fstatespaces}.2, it is 28-element
(13-atomic) and the set of two-valued states on~$L$ is not full (see the
proof of \ref{statespaces}.(1)). In fact, in the proof of
\ref{statespaces}.(1) it was shown that there is no two-valued state on the
8-element set $\{a,c_a,d_a,c,d,c_b,d_b,b\}$ such that $s(a)=s(b)=1$ (a
reformulation of fullness---see~\ref{full}). This orthomodular lattice
can be orthogenerated e.g.\ by the 6-element set $\{a,c_a,c_b,b,d_b,d_a\}$ and
generated e.g.\ by the 3-element set $\{a,c_b,d_b\}$.
\end {proof}


\begin {proposition} \label{noseparating}
There is a finite suborthoposet of~$\HL$ such that the set of two-valued
states on it is not separating.
\end {proposition}

\begin {proof}
Let us consider an orthomodular lattice given in Fig.~\ref{Fstatespaces}.3.
It is an orthomodular lattice without a separating set of two-valued states
(see the proof of \ref{statespaces}.(2)). It has 56~elements (27~atoms) and
a 17-element subset without a separating set of states (5 marked and 6
`hidden' in every circle). It can be checked that it has the following
realization (which forms a suborthoposet of~$\HL$ given in
Fig.~\ref{Frealizable}.2---points in circles denotes the middle elements of
the diagram from Fig.~\ref{Fstatespaces}.2): $f=\Sp(0,0,1)$,
$a_1$~\usebox{\shortdiagram}~$b_1$ given by Fig.~\ref{Frepresentations}.2;
$a_2$~\usebox{\shortdiagram}~$b_2$ we obtain from the representation on
Fig.~\ref{Frepresentations}.2 rotating by $\pi/2$ around~$f$. There is a
10-element set of orthogenerators (e.g.\
$\{a_1,b_1,c_{a1},c_{b1},d_{a1},d_{b1},f,c_2,c_{b2},d_{b2}\}$) and a
4-element set of generators (e.g.\ $\{a_1,c_{b1},d_{b1},c_{b2}\}$).
\end {proof}


Let us note that we can take a realization of an orthomodular lattice given
in Fig.~\ref{Fstatespaces}.2 such that we obtain an orthorepresentation of
the orthomodular lattice given in Fig.~\ref{Fstatespaces}.3, but the set of
(ortho)generators is larger in this case.


\begin {proposition} \label{nounital}
There is a finite suborthoposet of~$\HL$ such that the set of two-valued
states on it is not unital.
\end {proposition}


\begin {proof}
Let us consider an orthomodular lattice $L$ given in
Fig.~\ref{Frealizable}.3. It is an orthomodular lattice without a unital set
of two-valued states. Indeed, for every two-valued state $s$ on~$L$ with
$s(a_1)=1$ we have $s(f)=s(a_2)=s(a_5)=0$, $s(\bar a_2)=s(\bar a_5)=1$,
$s(\bar a_3)=s(\bar a_4)=0$, $s(a_3)=s(a_4)=1$---a contradiction. It has 132
elements (65 atoms) and a 40-element subset without a unital set of states
(6 hidden in every circle and all marked $a_i$'s and~$\bar a_i$'s). Let us
find a weak realization of~$L$. It can be done as follows: Put
$f=\Sp(0,0,1)$, $a_1=\Sp(1,0,0)$, $\bar a_1=\Sp(0,1,0)$ and let $a_k,\bar
a_k$ ($k=2,\ldots,5$) be images of $a_1,\bar a_1$ in rotations around~$f$
about $k\cdot 72^\circ$. Find a realization of the orthomodular lattice
given in Fig.~\ref{Fstatespaces}.2 such that the angle of images of $a,b$
is~72$^\circ$ (see the proof of \ref{realizabilityoflogic}) and rotate this
realization to the following pairs of lines: $(a_1,a_2)$, $(\bar a_2,\bar
a_3)$, $(a_3,a_4)$, $(\bar a_4,\bar a_5)$, $(a_5,a_1)$ (i.e., $a$ goes to
the first and $b$ to the second line for every pair). It can be checked that
an orthomodular poset orthogenerated by this weak realization is finite. (In
fact, it is a weak realization of an orthomodular lattice given in
Fig.~\ref{Frealizable}.4 by the same way.)
\end {proof}


It can be shown that if we take the realization of the orthomodular lattice
given in Fig.~\ref{Fstatespaces}.2 such that the angle between $a$ and $b$ is
equal to $72^\circ$ by the expression given in the proof of
\ref{realizabilityoflogic} as the first copy and if the second and the third
copy arise by rotations around the axis of the plane given by $a$ and $b$
such that $b$ coincides with $a$ of the next copy, then some elements
coincide:
  \begin {eqnarray*}
  (c_a, c  , c_b, b_c, b, b_d, d_b)_1 &=&
  (d  , d_b, d_a, a_d, a, a_c, c_a)_2 ,\\
  (c  , d_b, d  , d_a, e  )_1 &=&
  (c_a, d  , c  , e  , c_b)_3.
  \end {eqnarray*}
(Index denotes the number of the copy.) Hence, the weak realization of the
orthomodular lattice from the above proof gives a 29-element subset of~$\HL$
without a unital set of two-valued states and the suborthoposet
orthogenerated by it has 104~elements (51~atoms), is orthogenerated by a
16-element set and generated by a 4-element set (e.g., elements $a,c_b,d_b$
of some a~\usebox{\shortdiagram}~b and some element from the inner
`pentagon'). The `almost' Greechie diagram (20 points which belong to
exactly one edge are for simplicity omitted) of this suborthoposet of~$\HL$
(realization of the orthomodular lattice given in Fig.~\ref{Frealizable}.4)
is given in Fig.~\ref{FGD2pentagons}. Elements of the 29-element subset
without a unital set of two-valued states are all marked points which are
not crossed, a set of orthogenerators is e.g.\ the set of vertices of both
pentagons with $a_i$'s and with the middle point, a set of generators
is marked by circles.

It should be noted that in~\cite{Schuette,svozil-nat-acad} there is an
example of an 11-element set of lines orthogenerating a 25-element set of
lines and a 76-element (37-atomic) suborthoposet of~$\HL$ without a unital
set of two-valued states. This suborthoposet is generated by a 3-element
set. The Greechie diagram of this example does not seem to provide an easy
survey, hence we omit it. A more detailed description of this example is
given in Section~\ref{Sdiscussion}.


\begin {figure}[p]
\unitlength 0.40\textwidth
\newcommand{\inner}[3]{\point#1\point#2\point#3\emline#1#2\emline#2#3
  \emline#2{0}{0}  % for 2 pentagons only
  }
\newcommand{\side}[6]{\emline#1#4}
\newcommand{\pentagon}[5]{\inner#1\inner#2\inner#3\inner#4\inner#5
  \side#1#2\side#2#3\side#3#4\side#4#5\side#5#1}
\newcommand{\pentagondiagram}{\pentagon
  {{{ 0    }{ 1    }}{{-0.5  }{ 0    }}{{ 0    }{-0.809}}}%
  {{{-0.951}{ 0.309}}{{-0.155}{-0.476}}{{ 0.769}{-0.250}}}%
  {{{-0.588}{-0.809}}{{ 0.405}{-0.294}}{{ 0.476}{ 0.655}}}%
  {{{ 0.588}{-0.809}}{{ 0.405}{ 0.294}}{{-0.476}{ 0.655}}}%
  {{{ 0.951}{ 0.309}}{{-0.155}{ 0.476}}{{-0.769}{-0.250}}}%
  }
\def\contr{0.333}
\newsavebox{\smallpentagon}
\savebox{\smallpentagon}{{\unitlength\contr\unitlength%
  \begin{picture}(2,2)(-1,-1)\pentagondiagram\end{picture}}}
\begin {center}
\begin {picture}(2,2)(-1,-0.9)
\pentagondiagram
\point {0}{0}
\put (0,0){\makebox(0,0){\usebox{\smallpentagon}}}
\put ( 0    , 1    ){\discbig}        % generators
\put ( 0    ,\contr){\discbig}
\put ( 0    ,-0.809){\discbig}
\put ( 0.405, 0.294){\discbig}
\put ( 0.135, 0.098){\makebox(0,0){$\times$}}    % unnecessary for a set
\put (-0.159, 0.218){\makebox(0,0){$\times$}}
\footnotesize
\place { 0    }{ 0    }{ 2}{ -9}{}{f}
\place {-0.5  }{ 0    }{15}{ -9}{}{a_2}
\place {-0.155}{-0.476}{12}{ 12}{}{a_5}
\place { 0.405}{-0.294}{-6}{ 15}{}{a_3}
\place { 0.405}{ 0.294}{-6}{-14}{}{a_1}
\place {-0.155}{ 0.476}{-3}{-15}{}{a_4}
\place { 0.000}{ 1.000}{ 0}{ 6}{ b}{d_{b1}=c_{a2}=d_{ 3}=c_{ 5}}
\place {-0.476}{ 0.655}{-4}{ 4}{rb}{d_{a1}=e_{ 3}=c_{b5}}
\place {-0.951}{ 0.309}{-6}{ 0}{r }{d_{ 1}=c_{3}=d_{b4}=c_{a5}}
\place {-0.769}{-0.250}{-6}{ 0}{r }{e_{ 1}=c_{b3}=d_{a4}}
\place {-0.588}{-0.809}{ 0}{-6}{ t}{c_{ 1}=d_{b2}=c_{a3}=d_{ 4}}
\place { 0    }{-0.809}{ 0}{-6}{ t}{c_{b1}=d_{a2}=e_{ 4}}
\place { 0.588}{-0.809}{ 0}{-6}{ t}{c_{a1}=d_{ 2}=c_{ 4}=d_{b5}}
\place { 0.769}{-0.250}{ 6}{ 0}{l }{e_{ 2}=c_{b4}=d_{a5}}
\place { 0.951}{ 0.309}{ 6}{ 0}{l }{c_{ 2}=d_{b3}=c_{a4}=d_{ 5}}
\place { 0.476}{ 0.655}{ 4}{ 4}{lb}{c_{b2}=d_{a3}=e_{ 5}}
\end {picture}
\end {center}
\caption {`Almost' Greechie diagram of a suborthoposet of~$\HL$ without a
unital set of two-valued states.}
\label {FGD2pentagons}
\footnotesize
$$
\begin {array}{l@{\,=\,\Sp\,(\,}r@{\,,\,}r@{\,,\,}r@{\,)}}
a_{ 1} & 1               & 0                & 0                \\
a_{ 2} & \sqrt{3-\sqrt5} & \sqrt{ 5+\sqrt5} & 0                \\
a_{ 3} &-\sqrt{3+\sqrt5} & \sqrt{ 5-\sqrt5} & 0                \\
a_{ 4} &-\sqrt{3+\sqrt5} &-\sqrt{ 5-\sqrt5} & 0                \\
a_{ 5} & \sqrt{3-\sqrt5} &-\sqrt{ 5+\sqrt5} & 0                \\
c_{a1} & 0               &-\sqrt{-1+\sqrt5} & 1                \\
d_{a1} & 0               & \sqrt 2          & \sqrt{-2+\sqrt5} \\
c_{ 1} & \sqrt{  \sqrt5} & \sqrt{ 2+\sqrt5} & \sqrt{ 3+\sqrt5} \\
d_{ 1} &-\sqrt{  \sqrt5} &-\sqrt{-2+\sqrt5} & \sqrt  2         \\
c_{b1} &-\sqrt{5+\sqrt5} & \sqrt{ 3-\sqrt5} &2\sqrt{-2+\sqrt5} \\
d_{b1} & \sqrt{  \sqrt5} &-\sqrt{-2+\sqrt5} & \sqrt  2         \\
e_{ 1} & \sqrt{  \sqrt5} &-\sqrt{ 2+\sqrt5} & \sqrt{ 3-\sqrt5} \\
c_{ 2} &-\sqrt{  \sqrt5} & \sqrt{ 2+\sqrt5} & \sqrt{ 3+\sqrt5} \\
c_{b2} &-\sqrt{  \sqrt5} &-\sqrt{ 2+\sqrt5} & \sqrt{ 3-\sqrt5} \\
e_{ 2} & \sqrt{5+\sqrt5} & \sqrt{ 3-\sqrt5} &2\sqrt{-2+\sqrt5} \\
f      & 0               & 0                & 1
\end {array}
$$
\end {figure}


\begin {proposition} \label {empty}
There is a finite suborthoposet of~$\HL$ such that the set of two-valued
states on it is empty.
\end {proposition}


\begin {proof}
Let us consider an orthomodular lattice $L$ which is the pasting of the
orthomodular lattice given in Fig.~\ref{Frealizable}.3 for $a_1$ and of the
orthomodular lattice given in Fig.~\ref{Frealizable}.4 for its middle point.
It is an orthomodular lattice without any two-valued state. Indeed, if $s$
is a two-valued state on~$L$ then $s(a_1)=0$ (see above). Analogously from
the other diagram, $s(a_1)=1$---a contradiction. It has 374~elements
(186~atoms) and a 110-element subset without any two-valued state (6
`hidden' in every circle and all marked except two of them---$a_1$ and $\bar
a_1$). According to \ref {representablepasting}, this orthomodular poset is
weakly realizable in~$\HL$.
\end {proof}


It can be shown that we can paste for the whole block and obtain a weak
realization which is a union of weak realizations of two copies of an
orthomodular lattice given in Fig.~\ref{Frealizable}.4. Hence, this
suborthoposet has 200~elements (99~atoms) and a 58-element subset without
any two-valued state.


It should be noted that in~\cite{Peres2} there is an example of a 33-element
set of lines without any two-valued state. Direction vectors of these lines
arise by all permutations of coordinates from $(0,0,1)$, $(0,\pm1,1)$
$(0,\pm1,\sqrt2)$, $(\pm1,\pm1,\sqrt2)$. This set of lines orthogenerates a
suborthoposet of~$\HL$ with 116~elements (57~atoms). Direction vectors of
remaining lines arise by all permutations of coordinates from
$(\pm1,\pm3,\sqrt2)$. This suborthoposet of~$\HL$ has a 17-element set of
orthogenerators (e.g.\ lines with direction vectors $(0,0,1)$, $(0,1,0)$ and
all coordinate permutations from $(0,1,\sqrt2)$, $(1,\pm1,\sqrt2)$) and a
3-element set of generators (e.g.\ lines with direction vectors $(1,0,0)$,
$(1,1,0)$, $(\sqrt2,1,1)$). The `almost' Greechie diagram (24~points which
belong to exactly one edge are for simplicity omitted) of this example is
given in Fig.~\ref{FGD33lines} (one edge is denoted by a circle). The above
mentioned 3-element set of generators is marked by circles.


\begin {figure}[p]
\unitlength .45\textwidth
\newcommand{\onethird}[3]{\axis#1\side#1#2\side#1#3\cross#2#3}
\newcommand{\axis}[4]%
  {\point#1\point#2\point#3\point#4\emline#1#3\emline#3{0}{0}}
\newcommand{\side}[8]%
  {\point#5\point#6\point#7\point#8\emline#1#5\emline#5#6\emline#2#8\emline#8#7}
\newcommand{\cross}[8]{\emline#1#7\emline#7#3\emline#3#5\emline#2{0}{0}}
\catcode`\!=\active  \def!{\bar1}
\begin {center}
\begin {picture}(2,2)(-1,-1)
% \input svtk.dat
\onethird
{{{ 0.100}{-0.995}}{{ 0.050}{-0.747}}{{ 0.000}{-0.500}}{{ 0.000}{-0.250}}}%
{{{ 0.643}{-0.766}}{{ 0.087}{-0.050}}{{ 0.100}{-0.250}}{{ 0.536}{-0.112}}}%
{{{-0.643}{-0.766}}{{-0.087}{-0.050}}{{-0.100}{-0.250}}{{-0.536}{-0.112}}}%
\onethird
{{{ 0.812}{ 0.584}}{{ 0.622}{ 0.417}}{{ 0.433}{ 0.250}}{{ 0.217}{ 0.125}}}%
{{{ 0.342}{ 0.940}}{{-0.000}{ 0.100}}{{ 0.167}{ 0.212}}{{-0.171}{ 0.520}}}%
{{{ 0.985}{-0.174}}{{ 0.087}{-0.050}}{{ 0.267}{ 0.038}}{{ 0.365}{-0.408}}}%
\onethird
{{{-0.912}{ 0.411}}{{-0.672}{ 0.330}}{{-0.433}{ 0.250}}{{-0.217}{ 0.125}}}%
{{{-0.985}{-0.174}}{{-0.087}{-0.050}}{{-0.267}{ 0.038}}{{-0.365}{-0.408}}}%
{{{-0.342}{ 0.940}}{{-0.000}{ 0.100}}{{-0.167}{ 0.212}}{{ 0.171}{ 0.520}}}%
\put (0,0){\circle{0.2}}
\put ( 0.000, 0.100){\discbig}
\put ( 0.050,-0.747){\discbig}
\put ( 0.217, 0.125){\discbig}
\footnotesize
\place { 0.100}{-0.995}{ 0}{-6}{ t}{2!!}
\place { 0.050}{-0.747}{ 6}{ 0}{l }{211}
\place { 0.000}{-0.500}{ 6}{ 0}{l }{01!}
\place { 0.000}{-0.250}{ 2}{ 3}{lb}{011}
\place { 0.000}{ 0.100}{ 0}{-4}{ t}{100}
\place { 0.100}{-0.250}{ 4}{-4}{lt}{2!1}
\place {-0.100}{-0.250}{-4}{-4}{rt}{21!}
\place { 0.643}{-0.766}{ 6}{ 0}{l }{102}
\place { 0.365}{-0.408}{ 6}{ 0}{l }{20!}
\place { 0.087}{-0.050}{ 4}{ 0}{lb}{010}
\place {-0.643}{-0.766}{-6}{ 0}{r }{120}
\place {-0.365}{-0.408}{-6}{ 0}{r }{2!0}
\place {-0.087}{-0.050}{-4}{ 0}{rb}{001}
\place { 0.812}{ 0.584}{ 6}{ 0}{lb}{!!2}
\place { 0.622}{ 0.417}{ 6}{ 0}{lt}{112}
\place {-0.912}{ 0.411}{-6}{ 0}{rb}{!2!}
\place {-0.672}{ 0.330}{-6}{ 0}{rt}{121}
\place { 0.433}{ 0.250}{ 3}{-3}{lt}{1!0}
\place { 0.217}{ 0.125}{-6}{ 1}{r }{110}
\place {-0.433}{ 0.250}{-3}{-3}{rt}{10!}
\place {-0.217}{ 0.125}{ 6}{ 1}{l }{101}
\place { 0.267}{ 0.038}{ 6}{ 0}{l }{1!2}
\place { 0.167}{ 0.212}{ 0}{ 6}{lb}{!12}
\place {-0.267}{ 0.038}{-6}{ 0}{r }{12!}
\place {-0.167}{ 0.212}{ 0}{ 6}{rb}{!21}
\place { 0.985}{-0.174}{ 4}{-6}{ t}{201}
\place { 0.536}{-0.112}{ 0}{-6}{lt}{!02}
\place {-0.985}{-0.174}{-4}{-6}{ t}{210}
\place {-0.536}{-0.112}{ 0}{-6}{rt}{!20}
\place { 0.342}{ 0.940}{ 0}{ 6}{ b}{021}
\place { 0.171}{ 0.520}{-4}{ 4}{rb}{0!2}
\place {-0.342}{ 0.940}{ 0}{ 6}{ b}{012}
\place {-0.171}{ 0.520}{ 4}{ 4}{lb}{02!}
\end {picture}
\end {center}
\caption {`Almost' Greechie diagram of a suborthoposet of~$\HL$ without any
two-valued state (e.g.\ 1$!$2 denotes $\Sp(1,-1,\protect\sqrt2)$).}
\label {FGD33lines}
\end {figure}


\begin {corollary}  \label {3subset}
There is a 3-element set of lines in~$\HL$ such that no subortholattice
of~$\HL$ containing it has a two-valued state.
\end {corollary}


It seems to be an open question whether every 3-element set of mutually
nonorthogonal lines in~$\HL$ generates a subortholattice without any
two-valued state. The least numbers in constructions are given in
Tab.~\ref{Tnumbers}.


\begin {table}[ht]
\begin {center}
\begin {tabular}{|l|*5{r|}}
\hline
`large': & full & separating & \multicolumn{2}{c|}{unital} & nonempty \\\hline
example (figure) &
  \ref{Frepresentations}.2 &
  \ref{Frealizable}.2      &
  \cite{Schuette,svozil-nat-acad}&
  \ref{FGD2pentagons}      &
  \ref{FGD33lines}         \\\hline\hline
elements of a suborthoposet & 28 & 56 & 76 & 104 & 116 \\\hline
atoms of a suborthoposet    & 13 & 27 & 37 &  51 &  57 \\\hline
lines                       &  8 & 17 & 25 &  29 &  33 \\\hline
orthogenerators             &  6 &  9 & 11 &  16 &  17 \\\hline
generators                  &  3 &  4 &  3 &   4 &   3 \\\hline
\end {tabular}
\end {center}
\caption {Numbers of elements of constructed propositional structures
in~$\HL$ without a `large' set of two-valued states.}
\label {Tnumbers}
\end {table}


Let us note that the examples in \ref{nofull} and in \ref{noseparating}
appeared in~\cite{Kochen-Specker}, the example in Fig.~\ref{Frealizable}.4
appeared (not explicitly) in~\cite{Kochen-Specker,Peres1} as a part of
their construction. In~\cite{Alda} the author uses (not explicitly) the
orthomodular lattice given in Fig.~\ref{Frealizable}.3 and paste three
copies to distinct atoms of a block obtaining thus an orthomodular lattice
without any two-valued state (however, his estimation of lines does not seem
to be correct).

In~\cite{Mermin} the author uses weak realizability of an orthomodular
lattice in Fig.~\ref{Frealizable}.5 whenever we represent elements $a,b$ by
lines in~$\HL$ such that their angle is less than~45$^{\circ}$. This leads
to the construction of an orthomodular lattice with 392~elements (146~atoms)
weakly realizable in~$\HL$ and (at most) 130-element set of lines
without any two-valued state.




\section {Discussion of physical relevance}
\label {Sdiscussion}

%%%%%%%%%%%%%%%%%%%%%%%%   mention Penrose
%%%%%%%%%%%%%%%%%%%%%%%%   mention Schuette's letter
%%%%%%%%%%%%%%%%%%%%%%%%%% eliminite blanks at bibitems etc.
%%%%%%%%%%%%%%%%%%%%%%%%%% mention Svozil book


In this final section we shall give a brief review of the physical relevance
of the above findings. The nonexistence of two-valued measures on certain
finite propositional structures in threedimensional Hilbert spaces has first
been explicitly demonstrated by Kochen and Specker \cite{Kochen-Specker}. It
is strongly recommended to read this original account. Their result has
given rise to a number of interpretations, by Kochen and Specker and others.
A detailed overview of the history of the subject can, for instance, be
found in the reviews by Mermin \cite{Mermin} and Brown \cite{brown}.


What does it physically mean that {\em three\/} nonorthogonal rays in
threedimensional Hilbert space are sufficient to generate a finite system of
rays which have no two-valued state? To state the associated Kochen--Specker
paradox explicitly, let us associate any onedimensional subspace $\Sp (v)$
spanned by a nonzero vector $v$ with the proposition that the physical
system is in a pure state associated with that subspace. That is,
  $$
  \Sp (1,0,0) = a, \quad
  \Sp (1,1,0) = b, \quad
  \Sp (\sqrt{2},1,1) = c,
  $$
where $a,b$ and $c$ are propositions. If $a$ (similar for $b$ and $c$) is
measured, then we associate the logical value ``true''  or ``false'' with
the two-valued state function $s(a)=1$ and $s(a)=0$, respectively. $a,b,c$
generate the propositional structure derived by Peres \cite[pp.\
186--190]{Peres2,peres}. That is, if $v$ and $w$ are two vectors in
threedimensional Hilbert space corresponding to the propositions $p_v$ and
$p_w$, respectively, then the vector product $v \times w$ corresponds to the
proposition $(p_v \vee p_w)'$. In particular,

{
\newcommand{\gen}[9]{\Sp(\ex#1,\ex#2,\ex#3)&=&
  (\Sp(\ex#4,\ex#5,\ex#6)\lor\Sp(\ex#7,\ex#8,\ex#9))'=\\*&&}
\def\sqrtii{\sqrt2}\catcode`\2=\active\def2{\sqrtii}
\catcode`\+=\active\def+{\lor}\footnotesize
\begin {eqnarray*}
\Sp(1,0,0) &=& a ,\\
\Sp(1,1,0) &=& b ,\\
\Sp(2,1,1) &=& c ,\\
\gen 001100110 (a+b)',\\
\gen 01A100211 (a+c)',\\
\gen 010100001 (a+(a+b)')',\\
\gen 01110001A (a+(a+c)')',\\
\gen 1A0110001 (b+(a+b)')',\\
\gen A20211001 (c+(a+b)')',\\
\gen 2AA21101A (c+(a+c)')',\\
\gen A02211010 (c+(a+(a+b)')')',\\
\gen 210001A20 ((a+b)'+(c+(a+b)')')',\\
\gen 1200012AA ((a+b)'+(c+(a+c)')')',\\
\gen 1020102AA ((a+(a+b)')'+(c+(a+c)')')',\\
\gen 21A011A20 ((a+(a+c)')'+(c+(a+b)')')',\\
\gen 201010A02 ((a+(a+b)')'+(c+(a+(a+b)')')')',\\
\gen 2A0001120 ((a+b)'+((a+b)'+(c+(a+c)')')')',\\
\gen 2A1011A02 ((a+(a+c)')'+(c+(a+(a+b)')')')',\\
\gen A12110201 (b+((a+(a+b)')'+(c+(a+(a+b)')')')')',\\
\gen 02A100A12 (a+(b+((a+(a+b)')'+(c+(a+(a+b)')')')')')',\\
\gen 20A010102 ((a+(a+b)')'+((a+(a+b)')'+(c+(a+c)')')')',\\
\gen 1A2110A12 (b+(b+((a+(a+b)')'+(c+(a+(a+b)')')')')')',\\
\gen 01210002A (a+(a+(b+((a+(a+b)')'+(c+(a+(a+b)')')')')')')',\\
\gen 0211001A2 (a+(b+(b+((a+(a+b)')'+(c+(a+(a+b)')')')')')')',\\
\gen AA21A0201 ((b+(a+b)')'+((a+(a+b)')'+(c+(a+(a+b)')')')')',\\
\gen 0A2100021 (a+(a+(b+(b+((a+(a+b)')'+(c+(a+(a+b)')')')')')')')',\\
\gen 1121A002A ((b+(a+b)')'+(a+(b+((a+(a+b)')'+(c+(a+(a+b)')')')')')')',\\
\gen A2A210012 (((a+b)'+(c+(a+b)')')'+\\*%
               &&(a+(a+(b+((a+(a+b)')'+(c+(a+(a+b)')')')')')')')',\\
\gen A212100A2 (((a+b)'+(c+(a+b)')')'+\\*%
               &&(a+(a+(b+(b+((a+(a+b)')'+(c+(a+(a+b)')')')')')')')')',\\
\gen 12A2A0012 (((a+b)'+((a+b)'+(c+(a+c)')')')'+\\*%
               &&(a+(a+(b+((a+(a+b)')'+(c+(a+(a+b)')')')')')')')',\\
\gen A01010A2A ((a+(a+b)')'+(((a+b)'+(c+(a+b)')')'+\\*%
               &&(a+(a+(b+((a+(a+b)')'+(c+(a+(a+b)')')')')')')')')',\\
\gen 1212A00A2 (((a+b)'+((a+b)'+(c+(a+c)')')')'+\\*%
               &&(a+(a+(b+(b+((a+(a+b)')'+(c+(a+(a+b)')')')')')')')')',\\
\gen 101010A21 ((a+(a+b)')'+(((a+b)'+(c+(a+b)')')'+\\*%
               &&(a+(a+(b+(b+((a+(a+b)')'+(c+(a+(a+b)')')')')')')')')')'.
\end {eqnarray*}
}

Suppose, for the sake of contradiction, that each one of the above 33
propositions corresponds to an ``element of physical reality'' \cite{epr}.
That is, suppose that its value is either ``true'' (exclusive) or ``false,''
irrespective of whether it has been actually measured or just
counterfactually inferred. Let us further assume with Peres \cite[pp.
186-190]{Peres2,peres} that---provided these ``elements of reality''
exist---$
  \Sp (0,0,1) =
  \Sp (1,0,1) =
  \Sp (0,1,1) =
  \Sp (1,-1,\sqrt{2}) =
  \Sp (1,0,\sqrt{2}) =
  \Sp (\sqrt{2},1,1) =
  \Sp (\sqrt{2},0,1) =
  \Sp (1,1,\sqrt{2}) =
  \Sp (0,1,\sqrt{2}) =
  \Sp (1,\sqrt{2},1) =$
``true.''
One can follow Peres' arguments to show that---provided these ``elements of
reality'' exist---all  other rays belong to triads which are orthogonal to
the above rays. Therefore, these latter rays must correspond to propositions
whose value is ``false.'' In particular, $\Sp (1,0,0)=\Sp (0,\sqrt{2},1)=\Sp
(0,-1,\sqrt{2})=$``false,'' associate with $s(\Sp (1,0,0))=s(\Sp
(0,\sqrt{2},1))=s(\Sp (0,-1,\sqrt{2}))=0$. Thus, $s(\Sp (1,0,0))+s(\Sp
(0,\sqrt{2},1))+s(\Sp (0,-1,\sqrt{2})) =0+0+0=0$. But $\Sp (1,0,0)$, $\Sp
(0,\sqrt{2},1)$ and $\Sp (0,-1,\sqrt{2})$ are mutually orthogonal. This is
in contradiction to the assumption that for any orthogonal triad spanning
the entire Hilbert space, the sum of the measures should be one (cf.\
\ref{state}.(4)). Notice that in order to arrive at this Kochen--Specker
paradox, we had to explicitly assume the existence of the ``elements of
reality,'' irrespective of whether they have (or could have) actually been
measured or not.

What physical use can be a paradox? How can one measure a contradiction?
Indeed, what can actually be measured is merely {\em one\/} triplet of
propositions corresponding to some of the triads of mutually orthogonal
rays. Such a measurement can be performed with the operator discussed by
Peres, or with an arrangement of beam splitters discussed by Reck,
Zeilinger, Bernstein and Bertani \cite{rzbb}.

For instance, after $c$ is found to be ``true'' (corresponding to $s(c)=1$),
then measurement of the original values of $a$ or $b$ is no longer possible.
However, suppose one would be willing to believe in the existence of
``elements of reality'' \cite{epr,peres-no-results}, which could merely be
{\em counterfactually\/} inferred. Then one could for instance---at least in
principle---``measure'' all $16$ orthogonal triads by the production of a
state with $16$ entangled subsystems. On each one of the $16$ different
entangled subsystems one could measure one of the $16$ different orthogonal
triads. This is similar to a proposal by Greenberger, Horne and Zeilinger
\cite{ghz}, which use three particles and eight-dimensional Hilbert space.
Indeed, only in such a way---namely by (counterfactually) inferring
non-comeasurable propositions---one would encounter a complete
Kochen--Specker contradiction.

As has been already proven in Kochen and Specker's original work \cite[pp.\
82--85, Theorem 4]{Kochen-Specker}, the notion of tautology is connected to a
classical (Boolean) imbedding of a partial Boolean algebra. Indeed, there
exist propositions which are tautologies in the classical (Boolean) algebra
but which are not tautologies in the partial Boolean algebra if and only if
the partial Boolean algebra does not have a unital set of two-valued states
and thus cannot be imbedded into a classical (Boolean) algebra.

This is true for all partial Boolean algebras, in particular for orthomodular
posets. Notice that the above result does not imply that every propositional
structure giving rise to a (classical) Boolean tautology which is no quantum
tautology also has no two-valued measure (cf.\ below).

Until now, the lowest number of rays necessary to produce a classical
tautology which is not always true quantum mechanically is due to
Sch\"utte~\cite {Schuette,svozil-nat-acad}. The eleven rays used by
Sch\"utte can also be generated by the three vectors $(1,0,0)$, $(1,1,0)$
and $(\sqrt{2},1,1)$ (corresponding to $a$, $b$ and~$c$) used before.
Indeed, $d = \Sp (0,1,-1) = (\Sp (1,1,0) \lor \Sp (\sqrt2,1,1))' = (a \lor
c)'$ and

  {
  \catcode`\+=\active\def+{\lor}\footnotesize
  \newcommand{\gen}[9]{\Sp(\ex#1,\ex#2,\ex#3)=
    (\Sp(\ex#4,\ex#5,\ex#6)\lor\Sp(\ex#7,\ex#8,\ex#9))'=\\*&&}
  \begin {eqnarray*}
  a_1&=&\Sp(1,0,0) = a ,\\
  a_2&=&\gen 010100001 (a+(a+b)')',\\
  b_1&=&\gen 01110001A (a+d)',\\
  b_2&=&\gen 101010A11 ((a+(a+b)')'+(b+d)')',\\
  b_3&=&\Sp(1,1,0) = b ,\\
  c_1&=&\gen 10201021A ((a+(a+b)')'+((a+d)'+(b+(a+d)')')')',\\
  c_2&=&\gen 201010A02
             ((a+(a+b)')'+((a+(a+b)')'+((a+d)'+((a+d)'+(b+(a+b)')')')')')',\\
  d_1&=&\gen A1111001A (b+d)',\\
  d_2&=&\gen 1A1110011 (b+(a+d)')',\\
  d_3&=&\gen 11A0111A0 ((a+d)'+(b+(a+b)')')',\\
  d_4&=&\gen 11101A1A0 (d+(b+(a+b)')')',
  \end {eqnarray*}
  }
where
  {
  \newcommand{\gen}[9]{\Sp(\ex#1,\ex#2,\ex#3)&=&
    (\Sp(\ex#4,\ex#5,\ex#6)\lor\Sp(\ex#7,\ex#8,\ex#9))'=\\*&&}
  \catcode`\+=\active\def+{\lor}\footnotesize
  \begin {eqnarray*}
  \gen 21A0111A1 ((a+d)'+(b+(a+d)')')',\\
  \gen A02010B1A ((a+(a+b)')'+((a+d)'+((a+d)'+(b+(a+b)')')')')',\\
  \gen 2A101111A ((a+d)'+((a+d)'+(b+(a+b)')')')'.
  \end {eqnarray*}
  }

\noindent\sloppy
As we have mentioned above, there is not a unital set of two-valued states
on a suborthoposet orthogenerated by these rays (e.g., there is no
two-valued state $s$ with $s(\Sp (1,0,0)) = 1$). On the other hand, a
two-valued can be defined by
   $s(\Sp( 0, 1, 0)) =
    s(\Sp( 0, 1, 1)) =
    s(\Sp( 1, 1, 0)) =
    s(\Sp( 1, 1, 1)) =
    s(\Sp( 1, 1, 2)) =
    s(\Sp( 1, 2, 1)) =
    s(\Sp( 2, 1, 1)) =
    s(\Sp( 1, 2,-1)) =
    s(\Sp(-1, 2, 1)) =
    s(\Sp( 1, 5, 2)) =
    s(\Sp( 2, 5, 1)) =
    s(\Sp(-1, 5, 2)) =
    s(\Sp( 2, 5,-1)) =
    s(\Sp( 1, 5,-2)) =
    s(\Sp(-2, 5, 1)) = 1$
and
   $s(\Sp( 1, 0, 0)) =
    s(\Sp( 0, 0, 1)) =
    s(\Sp( 1, 0, 1)) =
    s(\Sp( 0, 1,-1)) =
    s(\Sp( 1, 0,-1)) =
    s(\Sp( 1,-1, 0)) =
    s(\Sp( 1, 1,-1)) =
    s(\Sp( 1,-1, 1)) =
    s(\Sp(-1, 1, 1)) =
    s(\Sp(-1,-1, 2)) =
    s(\Sp(-1, 2,-1)) =
    s(\Sp( 2,-1,-1)) =
    s(\Sp( 1,-1, 2)) =
    s(\Sp(-1, 1, 2)) =
    s(\Sp( 2, 1,-1)) =
    s(\Sp( 2,-1, 1)) =
    s(\Sp( 1, 0, 2)) =
    s(\Sp( 2, 0, 1)) =
    s(\Sp(-1, 0, 2)) =
    s(\Sp( 2, 0,-1)) =
    s(\Sp( 1,-5, 2)) =
    s(\Sp( 2,-5, 1)) = 0$.

Consider now the following propositions (notice that any binary operation is
either performed by orthogonal rays or by a ray and an orthocomplement of
another ray such that these rays are orthogonal):
{
\catcode`\+=\active\def+{\lor}\catcode`\-=\active\def-{\land}
\begin{eqnarray*}
f_1   &=& d_1   \to  b_2'\\*
      &=& (d_1-b_2)'\\
f_2   &=& d_1   \to  b_3'\\*
      &=& (d_1-b_3)'\\
f_3   &=& d_2   \to a_2 \vee b_2\\*
      &=& (d_2-(a_2+b_2)')'\\
f_4   &=& d_2   \to  b_3'\\*
      &=& (d_2-b_3)'\\
f_5   &=& d_3   \to  b_2'\\*
      &=& (d_3-b_2)'\\
f_6   &=& d_3   \to ( a_1\vee a_2\to b_3)\\*
      &=& (d_3-((a_1+a_2)'+b_3)')'\\
f_7   &=& d_4   \to a_2 \vee b_2\\*
      &=& (d_4-(a_2+b_2)')'\\
f_8   &=& d_4   \to ( a_1\vee a_2\to b_3)\\*
      &=& (d_4-((a_1+a_2)'+b_3)')'\\
f_9   &=& (a_2  \vee  c_1) \vee (b_3 \vee d_1)\\*
      &=& ((a_2+c_1)'-(b_3+d_1)')'\\
f_{10}&=& (a_2  \vee  c_2) \vee (a_1 \vee b_1 \to d_1)\\*
      &=& ((a_2+c_2)'-((a_1+b_1)'+d_1)')'\\
f_{11}&=& c_1   \to  b_1 \vee d_2\\*
      &=& (c_1-(b_1+d_2)')'\\
f_{12}&=& c_2   \to  b_3 \vee d_2\\*
      &=& (c_2-(b_3+d_2)')'\\
f_{13}&=& (a_2 \vee  c_1)\vee [ (a_1\vee a_2\to b_3)\to d_3]\\*
      &=& ((a_2+c_1)'-(((a_1+a_2)'+b_3)'+d_3)')'\\
f_{14}&=& (a_2  \vee  c_2) \vee (b_1 \vee d_3) \\*
      &=& ((a_2+c_2)'-(b_1+d_3)')'\\
f_{15}&=& c_2\to  [ (a_1 \vee a_2 \to b_3)\to d_4]\\*
      &=& (c_2-(((a_1+a_2)'+b_3)'+d_4)')'\\
f_{16}&=& c_1 \to  (a_1 \vee b_1 \to d_4)\\*
      &=& (c_1-((a_1+b_1)'+d_4)')'\\
f_{17}&=& (a_1 \to a_2)\vee b_1\\*
      &=& (a_1'+a_2)+b_1  .
\end{eqnarray*}
}
The ``implication'' relation has been expressed as $ x \to y \equiv x' \vee
y \equiv (x \land y')'$.

As can be straightforwardly checked, the proposition formed by
  $$
  F\colon\; f_1 \wedge f_2 \wedge  \cdots \wedge f_{16}\rightarrow f_{17}
  $$
is a classical tautology. Nevertheless, $F$ is not valid in threedimensional
(real) Hilbert space $\HL$, since $f_1,f_2, \ldots, f_{16}=\HL$, whereas
$f_{17}= ( \Sp (1,0,0))'=\Sp (0,1,0)\vee \Sp (0,0,1) \neq \HL$.


The three vectors $(1,0,0)$, $(1,1,0)$ and $(\sqrt{2},1,1)$ generating the
Sch\"utte rays are not mutually orthogonal. Therefore, the corresponding
propositions $a$, $b$ and $c$ are not co-measurable. In the sense of partial
algebras, they cannot be combined by logical operations ``or'' ($\vee$),
``and'' ($\land$), ``not'' ($'$) to form new expressions. Thus, it would be
incorrect to state that there exists a classical tautology in the three
variables $a$, $b$ and $c$ which is no quantum tautology. Indeed, Coray
proved~\cite{coray} that all classical tautologies in three variables are
tautologies in all partial algebras, in particular in the one associated
with the logic of quantum observables.

However, also Sch\"utte's example is counterfactual in nature.
Although every  operation or relation is solely defined on
co-measurable propositions, the entire formula $F$ contains 11
nonco-measurable variables (nonorthogonal rays). In order to be able to
evaluate this formula, one would have to know the truth value of all
these 11 variables. Since they are not co-measurable, this is possible
only by counterfactual inference; in very much the same way as discussed
before in the case of the original Kochen-Specker paradox.
Indeed, Corey's result shows that {\em any\/} classical (Boolean)
tautology which is no quantum tautology will have to rely on at least
four variables which cannot be mutually orthogonal (in $\HL$), and
therefore must be based upon counterfactual inference.

Finally, let us shortly mention the relevance of these findings to
the partition logic of automata. \ref{finitesubOL-full} states that
every finite subortholattice of $\HL$ has a full (and thus separating)
set of two-valued states.  Thus, any finite subortholattice of $\HL$ can
be expressed as an automaton logic.  The subortholattices of $\HL$ which
have no two-valued state are infinite.




\section *{Acknowledgement}


This research has been partially supported by the Austrian--Czech program
AKTION.




\begin{thebibliography}{99}

\bibitem {Alda} V. Alda:
{\em On\/ {\rm 0-1} measure for projectors}.
Apl. Mat. {\bf 25} (1980), 373--374.

\bibitem {birk-vonneu} G. Birkhof, J. von Neumann:
{\em The logic of quantum mechanics}.
Ann. of Math. {\bf 37} (1936), 823--834.

\bibitem{brown} H. R. Brown:
{\em Bell's other theorem and its connection with nonlocality, part I},
in
{\em Bell's theorem and the foundations of modern physics}, A. Van der
Merwe, F. Selleri and G. Tarozzi (World Scientific, Singapore, 1992),
104--116.

\bibitem {Schuette} % Schuette rays
K. Sch\"uette, {\em letter to Professor E. P. Specker,} dated April
22nd, 1965; published in
E. Clavadetscher-Seeberger: {\it Eine partielle Pr\"adikatenlogik}.
(Dissertation, ETH-Z\"urich, Z\"urich, 1983).

\bibitem{coray} G. Coray:
{\em Validit\'e dans les alg\'ebres de Boole partielles}.
Comm. Math. Helv. {\bf 45} (1970), 49--82.

\bibitem {Gleason} A. Gleason:
{\em Measures on a closed subspaces of a Hilbert space}.
J. Math. Mech. {\bf 6} (1957), 883--894.

\bibitem{ghz} D. M. Greenberger, M. Horne and A. Zeilinger:
in {\sl Bell's Theorem, Quantum Theory, and Conceptions of the Universe},
ed. by M. Kafatos (Kluwer, Dordrecht, 1989);
D. M. Greenberger, M. A. Horne, A. Shimony and A. Zeilinger,
Am. J. Phys. {\bf 58} (1990), 1131.

\bibitem {epr} A. Einstein, B. Podolsky and N. Rosen:
{\em Can quantum-mechanical description of physical reality be considered
complete?}. Phys. Rev. {\bf 47}, 777 (1935); reprinted in J. A. Wheeler and
W. H. Zurek, eds., {\em Quantum Theory and Measurement\/} (Princeton
University Press, Princeton, 1983), 145--151.

\bibitem {Kalmbach-83} G. Kalmbach:
{\em Orthomodular Lattices}. Academic Press, New York, 1983.

\bibitem {kalmbach-86} G. Kalmbach:
{\em Measures and Hilbert Lattices}. World Scientific, Singapore, 1986.

\bibitem {kochen-specker-64} %partial algebras
S. Kochen and E. P. Specker:
{\em The calculus of partial propositional functions}, in
{\em Proceedings of the 1964 International Congress for Logic,
Methodology and Philosophy of Science, Jerusalem\/} (North Holland,
Amsterdam, 1965), 45--57.

\bibitem {kochen-specker-65} %partial algebras
S. Kochen and E. P. Specker:
{\em Logical Structures arising in quantum theory}, in
{\em Symposium on the Theory of Models, Proceedings of the
1963 International Symposium at Berkeley\/}
(North Holland, Amsterdam, 1965), 177-189.

\bibitem {Kochen-Specker} %Kochen-specker type construction
S. Kochen and E. P. Specker:
{\em The problem of hidden varibles in quantum mechanics}.
Journal of Mathematics and Mechanics {\bf 17} (1967), 59--87.
% reprinted in
% E. Specker, {\em Selecta} (Birkh\"auser Verlag, Basel, 1990)

\bibitem {Mermin}
N. D. Mermin:
{\em Hidden variables and the two theorems of John Bell}.
Rev. Mod. Phys. {\bf 65} (1993), 803--815.

\bibitem {NR} M. Navara, V. Rogalewicz:
{\it The pasting constructions for orthomodular posets}. Math. Nachrichten
{\bf 154} (1991), 157--168.

\bibitem{peres-no-results} A. Peres:
{\em Unperformed experiments have no results}. Am. J. Phys. {\bf 46} (1978),
745.

\bibitem {Peres1} A. Peres:
{\em Cryptodeterminism and quantum theory}. In Microphysical Reality and
Quantum Formalism (Kluwer Academic Publishers, Dordrecht, 1988), 115--123.

\bibitem {Peres2} A. Peres:
{\em Two simple proofs of the Kochen--Specker theorem}. J. Phys. A: Math.
Gen. {\bf 24} (1991), L175--L178.

\bibitem {peres} A. Peres:
{\em Quantum Theory: Concepts \& Methods}.
(Kluwer Academic Publishers, Dordrecht, 1993).

\bibitem {Ptak-Pulmannova} P. Pt\'ak and S. Pulmannov\'a:
{\em Orthomodular Structures as Quantum Logics}.
Kluwer Academic Publishers, Dordrecht, 1991.

\bibitem{rzbb} M. Reck, A. Zeilinger, H. J. Bernstein and P. Bertani:
{\em Experimental realization of any discrete unitary operator}.
Phys. Rev. Lett. {\bf 73}, 58 (1894); see also
F. D. Murnaghan,
{\em The Unitary and Rotation Groups}
(Spartan Books, Washington, 1962).

\bibitem {svosh} M. Schaller, K. Svozil:
{\it Partition logics of automata.} Il Nuovo Cimento {\bf 109 B} (1994),
167--176.

\bibitem {svosh2} M. Schaller, K. Svozil:
{\it Automaton partition logic versus quantum logic}.
Inter. J. Theor. Phys {\bf 35} (1996), 911--940.

\bibitem{svosh3} M. Schaller, K. Svozil:
{\it Automaton logic}. Inter. J. Theor. Phys
{\bf 34} (1995), 1741--1749.


\bibitem {kochen-specker-60}  %first mentioning
E. P. Specker: {\em Die Logik nicht gleichzeitig entscheidbarer
Aussagen}. Dialectica
{\bf
14}
(1960),
175--182.

\bibitem {svozil-93} K. Svozil:
{\em Randomness and Undecidability in Physics}. World Scientific, Singapore,
1993.

\bibitem{svozil-nat-acad} K. Svozil:
{\em A constructivist manifesto for the physical  sciences}, in
{\em The Foundational Debate, Complexity and Constructivity in
Mathematics and Physics}, Werner DePauli Schimanovich, Eckehart K\"ohler
and Friedrich Stadler, eds. (Kluwer, Dordrecht, Boston, London, 1995),
65--88.
\end{thebibliography}


\end{document}



List of labels:

Sdiscussion - the last section

Fdiagrams - non-Greechie diagrams
Fstatespaces - Greechie diagrams demonstr. differ. in state spaces
Fstates - states in a special OML
Frepresentations - representations of a special OML
Fsubortholattices - GD of all subortholattices of $\HL$
Fnonrealizable - GD of nonrealizable OMP
Frealizable - GD of realizable OMP
FGD2pentagons - GD of a suborthoposet without a unital set of 2v. states
FGD33lines - GD of a suborthoposet of $\HL$ without any 2v. state

Tnumbers - numbers of elements

diagram  - definition of GD
state    - definition of a state
s-w      - correspondence of states and weights
full     - criterion for a full set of states
separating - criterion for a separating set of states
noloop-concrete - GD without a loop is concrete
partition - partial criterion for a separating set of states
statespaces - there are 3-atomic OML demonstr. differences
3lines - 3 nonorthogonal lines generate infinite OML
finitesubortholattices - characterization
finitesubOL-full - finite subortholattice has a full set of 2-states
realizabilityoflogic - for a special logic
representablesum - horizontal sum of orthorep. (w. real.) is orthorep. (w.r.)
representablepasting - pasting for an atom of orthorep. is orthorep.
representablepasting2 -pasting of 3-chain for nonorthog. atoms
nofull - example
noseparating
nounital
empty
3subset - 3-element set generates subOL without 2v. states

