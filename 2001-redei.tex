\documentstyle[12pt]{article}
%\renewcommand{\baselinestretch}{2}
\begin{document}

 \def\Bbb{\bf }
% \def\frak{\cal }

\title{BOOK REVIEW:\\
       Quantum Logic in Algebraic Approach\\
       by Mikl{\'{o}}s R{\'{e}}dei\\}
\author{K. Svozil\\
 {\small Institut f\"ur Theoretische Physik,}
  {\small Technische Universit\"at Wien }     \\
  {\small Wiedner Hauptstra\ss e 8-10/136,}
  {\small A-1040 Vienna, Austria   }            \\
  {\small e-mail: svozil@tuwien.ac.at}\\
  {\small www: http://tph.tuwien.ac.at/$\widetilde{\;\;}\,$svozil}}
\date{ }
\maketitle

Since the invention of quantum logic by Birkhoff \&
von Neumann there has been a steady interest in the
topic. Quantum logic owes much of its current attention to the rise of
experimental foundational physics. Many researchers interested in the
foundation of
physics begin to view this field as a way to understand the quantum
better; or at least from a new, hitherto unknown viewpoint.

Yet  much
is still to be done to establish quantum logic
as an indispensable tool, an area
of research which not only offers new insights into established
quantum
physics but which progressively suggests totally novel phenomena
out of its own body of formalism and
knowledge.
The  physical community need to be convinced
that the new algebraic formalisms and techniques which have to
be learned are well worth the effort of digestion.
R{\'{e}}dei's book
{\it ``Quantum Logic in Algebraic Approach''} \cite{redei} is
an important contribution in this direction, as it represents a
comprehensive introduction to the field.

The book starts out with concise expositions of the two main ingredients of
quantum logic: the introductory chapters are devoted to  quantum
mechanics  based on the Hilbert space formalism as well as to lattice
theory. The spectral theorem as well as Gleason's theorem are reviewed
as key theorems in connection with the physical concepts of observables
and states.

In what follows, quantum logic is fully developed in the canonical
manner
as the lattice of of all closed linear subspaces of a Hilbert space,
henceforth called Hilbert lattice. Already at this stage a very
important theorem with respect to the ``nonclassicality'' of quantum
mechanics is discussed: for Hilbert spaces of dimension greater than two
no embedding exists which maps the associated Hilbert lattice into a
``classical'' Boolean algebra such that the logical (lattice) operations
are preserved among comeasurable propositions. This fact, which is often
referred to as the ``Kochen-Specker'' theorem, relates to the
nonexistence of (certain properties of) two-valued measures which in
term can be interpreted as the possibility to give values to elements
of physical reality, irrespective of whether at all or in which context
they have been measured (cf. below). Here a warning might be in order:
Although there is little
doubt that such an important result
deserves a great share of attention,   the novice
reader might not be able to appreciate it at such an early state.

The book goes on with an outline of a semantic approach to
physical theories, which is then
applied to classical mechanics as well as to quantum mechanics.
In this framework, ``concrete
quantum logic'',  is developed as the atomic, non-distributive,
non-modular, orthomodular Hilbert lattice of projections on an infinite
dimensional Hilbert space.

A theory of von Neumann algebras and von Neumann lattices is
developed. This should enable the reader to comprehend and appreciate
Birkhoff's and von Neumann's conception of quantum logic.

In the seventh chapter
R{\'{e}}dei discusses, among other issues, a topic which seems to be not
widely known
in the history of quantum mechanics: von Neumann's loss of belief in the
Hilbert space formalism of quantum mechanics. This is
arguably one of the most important contributions to the present
research, both
historically and probabilistically, for it puts quantum logic into a
new perspective.
Some connections between possible connections between quantum logic and
Cantorean set theory are also discussed.

After a chapter on the elementary theory of the quantum conditional
connective, the hidden variable problem is reviewed. This problem is
studied seriously and beyond the listing of historic events in
an operator algebraic framework.

What neither R{\'{e}}dei nor the author of this review have foreseen
is the possibility that a {\it dense subset} of the Hilbert lattice can
be consistently assigned classical
truth values independent of whether or not, or in which context, these
"elements of physical
reality" have been measured. This possibility, which has recently been
argued by Mayer \cite{meyer:99} and
extended by
Kent \cite{kent:99} and Clifton and Kent \cite{clifton:99},
makes conceivable non-contextual
hidden parameter theories
simulating essential operational aspects of quantum mechanics.
Meyer's  proof merely requires elementary combinatorics; it
constructively demonstrates how to `color' the rational unit sphere in
threedimensional real Hilbert space such that for each orthogonal triad
spanned by three vectors from the origin to the points of the sphere,
one vector is `colored' with `yes' and the other two vectors are
`colored' with `no'.

From a purely operational point of view, a ``rationalized''
quantum mechanics is indistinct from the usual real-valued formalism:
given any nonzero measurement uncertainty $\varepsilon$ and any
`non-colorable' Kochen-Specker graph $\Gamma (0)$,  there
exists another Graph (in fact, a denumerable infinity thereof)
$\Gamma (\delta)$
which lies inside the range of measurement uncertainty
$\delta \le \varepsilon$
[and thus cannot be discriminated from the `non-colorable'
$\Gamma (0)$]
which {\it can} be `colored.'
This is one of the rare cases where set theoretic
assumptions---rationals versus reals---do make a
difference: the first choice implies possible non-contextuality, the
second choice context dependence.
It remains to be seen whether these results can be extended to the Bell
inequalities and the GHZ-theorem as well.

The next chapter of R{\'{e}}dei's book is devoted to the basic ingredients of the theory
of local, algebraic relativistic quantum field theory. Bell-type
correlations are introduced in operator algebraic terms. These
considerations are developed further in a discussion of the problem of
entanglement --- the logical independence of two sub-quantum logics of a
quantum logic. The last chapter is devoted to the analysis of whether or
not superluminal correlations have common causes.

R{\'{e}}dei's book is very original and well written. It requires
familiarity with functional analysis and Hilbert space quantum mechanics
and thus might be suitable for mathematicians and theoretical physicists
alike.
But it also addresses issues of interest to philosophers of science
with a background in the formalism,
in particular with respect to the Birkhoff-von Neumann conception of
quantum logic. To put it in R{\'{e}}dei's own words: {\it ``The ideal
reader I had in mind  $\ldots$ was a somewhat philosophically minded
physicist with a strong respect and interest in mathematics.''}


%\bibliography{svozil}
%\bibliographystyle{unsrt}


\begin{thebibliography}{1}

\bibitem{redei}
Mikl{\'{o}}s R{\'{e}}dei.
\newblock {\em Quantum Logic in Algebraic Approach}.
\newblock Kluwer Academic Publishers, Dordrecht, Boston, London, 1998.

\bibitem{meyer:99}
David~A. Meyer.
\newblock Finite precision measurement nullifies the {K}ochen-{S}pecker
  theorem.
\newblock http://xxx.lanl.gov/abs/quant-ph/9905080, 1999.

\bibitem{kent:99}
Adrian Kent.
\newblock Non-contextual hidden variables and physical measurements.
\newblock http://xxx.lanl.gov/abs/quant-ph/9906006, 1999.

\bibitem{clifton:99}
Rob Clifton and Adrian Kent.
\newblock Simulating quantum mechanics by non-contextual hidden variables.
\newblock http://xxx.lanl.gov/abs/quant-ph/9908031, 1999.

\end{thebibliography}

\end{document}
