\PassOptionsToPackage{dvipsnames}{xcolor}
\documentclass[
 %reprint,
  twocolumn,
 %superscriptaddress,
 %groupedaddress,
 %unsortedaddress,
 %runinaddress,
 %frontmatterverbose,
 % preprint,
 showpacs,
 showkeys,
 preprintnumbers,
 %nofootinbib,
 %nobibnotes,
 %bibnotes,
 amsmath,amssymb,
 aps,
 %  prl,
 pra,
 % prb,
 % rmp,
 %prstab,
 %prstper,
  longbibliography,
 floatfix,
 %lengthcheck,
 ]{revtex4-2}

%\usepackage{cdmtcs-pdf}

\usepackage{mathptmx}% http://ctan.org/pkg/mathptmx

\usepackage{amssymb,amsthm,amsmath}

\usepackage{tikz}
\usetikzlibrary{calc,shapes.geometric}
\usepackage {pgfplots}
\pgfplotsset {compat=1.8}
\usepackage{graphicx}% Include figure files
%\usepackage{url}

\usepackage{xcolor}


\usepackage{hyperref}
\hypersetup{
    colorlinks,
    linkcolor={blue},
    citecolor={red!75!black},
    urlcolor={blue}
}

\usepackage{mathbbol}

\usepackage{multirow}


\newcommand\myotimes{ }

%\usetikzlibrary{external}
%\tikzexternalize

\begin{document}


\title{Converting nonlocality into contextuality}



\author{Karl Svozil}
\email{karl.svozil@tuwien.ac.at}
\homepage{http://tph.tuwien.ac.at/~svozil}

\affiliation{Institute for Theoretical Physics,
TU Wien,
Wiedner Hauptstrasse 8-10/136,
1040 Vienna,  Austria}



\date{\today}

\begin{abstract}
%Matrix pencils offer a powerful method for identifying simultaneous eigensystems of mutually commuting degenerate operators. In this study, we apply these techniques to analyze the quantum logical structures inherent in the Peres-Mermin square and the Greenberger-Horne-Zeilinger (GHZ) configuration. Our analysis reveals similar complete contradictions between classical and quantum predictions in a four-dimensional system comprising two spin-1/2 particles.
Matrix pencils provide a robust method for finding simultaneous eigensystems of mutually commuting degenerate operators. In this paper, we utilize these techniques to investigate the quantum logical structures of the Peres-Mermin square and the Greenberger-Horne-Zeilinger-Mermin configuration. Our analysis uncovers analogous complete contradictions between classical and quantum predictions in a four-dimensional system involving two spin-1/2 particles.
\end{abstract}

%\pacs{03.65.Aa, 03.65.Ta, 03.65.Ud, 03.67.-a}
\keywords{contextuality, two-valued states, quantum states, matrix pencil}
%\preprint{CDMTCS preprint nr. x}

\maketitle




\section{Introduction}

%\subsection{From physical to logical quantities}

Heuristically, quantum contextuality encompasses any aspect that contradicts classical predictions, with strong types of contextuality entailing complete contradictions relative to classical expectations. In what follows, we shall concentrate on `strong' quantum contextuality rendered by operator-valued arguments exhibiting nonlocality. While the inverse problem---converting contextuality into nonlocality~\cite{cabello2020converting}---can be of empirical importance, solving the former task can identify the particular type of contextuality exhibited, as well as suggest further experiments.

From a structural standpoint---in terms of the quantum logical algebraic relations of the associated propositions---operator-valued arguments may be closely related, although they may formally appear to be very different. For instance, as observed by Cabello~\cite{cabello-96,Cabello-2013-Hardylike}, Hardy's theorem~\cite{Hardy-92,Hardy-93} can, in quantum logical terms, be transcribed as a true-implies-false arrangement (in graph theoretical terms, a gadget) of observables~\cite{2018-minimalYIYS,svozil-2020-hardy}.
However, as we shall see in comparing Kochen-Specker (KS) and Greenberger-Horne-Zeilinger arguments, there need not be such a close relationship.


Operators in quantum mechanics can have varied spectra, but through their spectral decomposition,
their fundamental components are orthogonal projection operators formed from orthonormal bases of Hilbert spaces.
In this way, every normal operator `masks' or `contains' within itself an orthonormal basis, which
can be identified with a measurement context. From dimension three onwards, these bases can intertwine.

Due to their idempotent nature, projection operators admit only two real solutions (0 and 1) and can thus be associated with elementary logical propositions. The Hilbert space operations of the linear span and the intersection can then be identified with the logical disjunction `or' and conjunction `and' operations, respectively~\cite{birkhoff-36,vonNeumann2018Feb}.

Hence, general quantum observables, which are not elementary binary propositions,
mask the underlying contexts, as they do not directly reflect propositions but rather functions thereof~\cite[\S~82]{halmos-vs}.
To analyze and utilize these observables and their associated operators effectively,
it is necessary to delineate the undelying contexts by extracting their associated idempotent self-adjoint projection operators.
In von Neumann's words~\cite[pp.~241, 242]{vonNeumann2001},
``from the mathematical point of view the more desirable system to treat is not operator theory,
but that part of it which deals with idempotents, because that corresponds to logics,
whereas the whole system corresponds to a somewhat unpleasant extension
of logics, namely where you deal with quantities which can have any number of numerical values,
in other words, physical quantities.''

If the operator is nondegenerate and thus of maximal resolution, this extraction of projection operators is straightforward: It requires computing the eigensystem.

%For instance, consider a nondegenerate Hamiltonian operator in a quantum system. By determining its eigensystem, we can uncover the orthonormal basis that corresponds to its eigenstates, thereby revealing the measurement context. This process allows us to understand the fundamental propositions embedded within the operator, facilitating a clearer analysis of the quantum system's behavior.

%By systematically extracting these contexts, we gain deeper insights into the quantum observables, enabling a more precise interpretation and application in various quantum mechanics scenarios.


However, if the operators are degenerate, they have multiplicities in the eigenvalues,
which makes their spectral decomposition---and thus their contexts---somewhat arbitrary.
This arbitrariness of the contexts of quantum observables can be overcome
by enlisting other observables that mutually commute with each other and with the original observable.
In this way, a proper `complete' or `maximal resolution' system of observables could uniquely define a context.


A technical problem arises if the mutually commuting operators of the observables are all degenerate.
For the sake of an example take, for instance, the two Hermitian matrices
\[
\begin{pmatrix}
0 &  1 &  0 &  0 \\  1 &  0 &  0 &  0 \\  0 &  0 &  0 &  -1 \\  0 &  0 &  -1 &  0
\end{pmatrix}
\text{ and }
\begin{pmatrix}
0 &  0 &  0 &  -1 \\  0 &  0 &  1 &  0 \\  0 &  1 &  0 &  0 \\  -1 &  0 &  0 &  0
\end{pmatrix}^\intercal,
\]
which commute, and yet, none of their respective eigenvalues coincide: Indeed, the eigensystem of the first matrix consist of separable vectors
$
\begin{pmatrix}
1 ,  \pm 1 ,  0 , 0
\end{pmatrix}^\intercal
$
and
$
\begin{pmatrix}
 0 , 0  , 1 ,  \pm 1
\end{pmatrix}^\intercal
$
while
the eigenvectors of the second matrix
$
\begin{pmatrix}
1  ,  0 , 0 ,  \pm 1
\end{pmatrix}^\intercal
$
and
$
\begin{pmatrix}
 0  , 1 ,  \pm 1 , 0
\end{pmatrix}^\intercal
$
(the symbol $\intercal$ stands for transposition)
 are all nonseparable.
In such cases, finding their respective unique context can be rather tedious, although constructively feasible,
as it involves finding simultaneous eigenvectors for all the commuting operators~\cite[Section 1.3]{Horn-Johnson-MatrixAnalysis}.

%\subsection{Nonlocality and contextuality}


In the first part of this paper, we propose applying generalized matrix pencils to the problem of finding a maximal resolution operator for a set of mutually commuting degenerate operators. With this approach, determining the associated unique context becomes straightforward: it only requires computing the eigensystem of a nondegenerate and thus maximal resolution operator that has no eigenvalue multiplicities.


Next, we apply the matrix pencil method to two interesting cases: the Peres-Mermin (PM) square and Mermin's form of the Greenberger-Horne-Zeilinger (GHZM) argument. We find that the quantum logical structure of the PM square constitutes a KS argument, as it does not allow a classical counterfactual truth assignment that is both noncontextual aas well as allows global value assignments of (potential) elements of physical reality. The GHZM argument, on the other hand, is fundamentally different: it operates within a single three-partite context that permits as many classical truth assignments (eight) as its elementary propositions. However, through a parity argument, the associated eigensystem contradicts these classical predictions. This is because a serial concatenation of mutually commuting operators includes only even factors of counterfactual observables, yet their product yields a negative value quantum mechanically.

In the third part, we use the matrix pencil method to construct a two-partite GHZM-type argument. This configuration is based on the observables in the PM square that have a nonlocal eigensystem corresponding to nonseparable vectors associated with entangled quantum states.


\section{Matrix pencils}



The  algorithmic  and thus constructive
analysis of a transcription process for cases of operator-value arguments demonstrating nonclassical behavior is based on the proper spectral decomposition of the operators involved.
Mutually commuting normal operators (such as Hermitian or unitary operators that commute with their respective adjoints)
 $A_1, \ldots, A_l$ share common projection operators.
However, if their spectra are degenerate we need to find an orthonormal basis in which every single one of this collection of mutually commuting operators is diagonal.
The conventional approach to this task can be quite complex~\cite{Nordgren2020Jun}.
Alternatively, we can diagonalize the generalized matrix pencil that is a linear combination of the
operator matrices~\cite[Chapter~12]{Gantmacher2}:
%https://community.wolfram.com/groups/-/m/t/243431
\begin{equation}
P = \sum_{i=1}^{l} a_i A_i,
\label{2024-convert-matrixpencil}
\end{equation}
where $a_i$ are scalars (for our purposes, real numbers).
As $P$ commutes with $A_1, \ldots, A_l$, they share a common set of projection operators.
Moreover, since the scalar parameters $a_i$ can be adjusted, and in particular, can be identified with Kronecker delta functions $\delta_{ij}$,
and as $P$ commutes with each operator $A_j$ for $1 \le j \le l$, $P$ and $A_j$ share a common set of projection operators.

Equipped with these techniques, any collection of commeasurable multipartite observables corresponding
to mutually commuting operators can be transcribed into
%vectors $|e_i\rangle$ corresponding to the
projection operators
%$E_i = |e_i\rangle \langle e_i|$
in the spectrum of the operators of these observables.
If these operators render a maximal resolution, the respective vectors correspond to an orthonormal basis called a context with respect to $A_1, \ldots, A_l$.
The merging or pasting of possibly intertwining contexts then generates a quantum logic which can be
analyzed to identify and characterize the contextual (nonclassical) predictions and features.

%The matrix pencil of $A_1, \ldots, A_l$ yields a maximal resolution (which not necessarily is unique if $A_1, \ldots, A_l$ is not sufficiently resolving)
%in terms of common eigenvectors.
%The orthogonal projection operators corresponding to these eigenvectors
%can be bundled into a single maximal nondegenerate operator $M = \sum_{i=1}^{n} \mu_i E_i$ with $n$ mutually different eigenvalues $\mu_i$ (of multiplicity one),
%which can be conceived as containing the entirety of possible information available through $A_1, \ldots, A_l$,
%which is all that can be known quantum mechanically~\cite{zeil-99}.
%In this formal sense, the collection of observables $A_1, \ldots, A_l$ may be considered to `correspond to a context'.
%This bridges the gap between general operator-valued arguments on the one hand, and quantum logics on the other hand,
%as the latter is based on the identification of propositions with projection operators~\cite{birkhoff-36}.




\begin{table*}[ht]
\caption{\label{2024-convert-matrixpencil-peres} Eigensystems of the matrix pencils of the rows and columns of the PM square~\eqref{2024-convert-PeresSquare}
with normalization factors omitted.
The eigenvectors corresponding to the last row and column are nonseparable and thus entangled, while all others are separable.
This set of 24 vectors includes the 18 vectors of Cabello, Estebaranz and Garc{\'{i}}a-Alcaine~\cite{cabello-96}.
As already noted by Peres~\cite{peres-91},
these six `primary' contexts associated with orthogonal tetrads are disjoint (not intertwined).
In the hypergraph representation depicted in Figure~\ref{2024-convert-f-24-24}(a) they are represented as the `small ovals' on the six edges of the hypergraph.
%This arrangement of contexts is thus not isomorphic to the (biconnected) 18-9 configuration of Cabello, Estebaranz and Garc{\'{i}}a-Alcaine~\cite{cabello-96}.
%The latter matrix pencil yields the Bell basis $\big\{ \vert \Psi_+ \rangle , \vert \Psi_- \rangle , \vert \Phi_+ \rangle  , \vert \Phi_- \rangle \big\}$.
}
\begin{ruledtabular}
\begin{tabular}{ccccccc}
matrix pencils&\multicolumn{4}{c}{eigenvalues}\\
%\hline
&$a - b - c$& $-a + b - c$& $-a - b + c$&   $a + b + c$\\
\hline
$
a \sigma_z  \myotimes   \mathbb{1}_2 + b \mathbb{1}_2 \myotimes   \sigma_z + c \sigma_z  \myotimes   \sigma_z
$ & $
\vert 7 \rangle = \begin{pmatrix}  0, 1, 0, 0\end{pmatrix}^\intercal $ & $
\vert 3 \rangle = \begin{pmatrix}  0, 0, 1, 0\end{pmatrix}^\intercal $ & $
\vert 1 \rangle = \begin{pmatrix}  0, 0, 0, 1\end{pmatrix}^\intercal $ & $
\vert 17 \rangle = \begin{pmatrix}  1, 0, 0, 0\end{pmatrix}^\intercal
$ \\ $
a \mathbb{1}_2 \myotimes   \sigma_x + b \sigma_x  \myotimes   \mathbb{1}_2 + c \sigma_x  \myotimes   \sigma_x
$ & $
\vert 20 \rangle = \begin{pmatrix}  -1, -1, 1, 1\end{pmatrix}^\intercal $ & $
\vert 13 \rangle = \begin{pmatrix}  -1, 1, -1, 1\end{pmatrix}^\intercal $ & $
\vert 11 \rangle = \begin{pmatrix}  1, -1, -1, 1\end{pmatrix}^\intercal $ & $
\vert 24 \rangle = \begin{pmatrix}  1, 1,  1, 1\end{pmatrix}^\intercal
$ \\ $
a \sigma_z  \myotimes   \sigma_x + b \sigma_x  \myotimes   \sigma_z + c \sigma_y  \myotimes   \sigma_y
$ & $
\vert 21 \rangle = \begin{pmatrix}  1, 1, -1, 1\end{pmatrix}^\intercal $ & $
\vert 14 \rangle = \begin{pmatrix}  1, -1, 1, 1\end{pmatrix}^\intercal $ & $
\vert 23 \rangle = \begin{pmatrix}  -1, 1, 1,  1\end{pmatrix}^\intercal $ & $
\vert 10 \rangle = \begin{pmatrix}  -1, -1, -1, 1\end{pmatrix}^\intercal
$ \\ $
a \sigma_z  \myotimes   \mathbb{1}_2 + b \mathbb{1}_2 \myotimes   \sigma_x + c \sigma_z  \myotimes   \sigma_x
$ & $
\vert 12 \rangle = \begin{pmatrix}  -1, 1, 0, 0\end{pmatrix}^\intercal $ & $
\vert 4 \rangle = \begin{pmatrix}  0, 0, 1, 1\end{pmatrix}^\intercal $ & $
\vert 2 \rangle = \begin{pmatrix}  0, 0, -1, 1\end{pmatrix}^\intercal $ & $
\vert 22 \rangle =  \begin{pmatrix}  1, 1, 0, 0\end{pmatrix}^\intercal
$ \\ $
a \mathbb{1}_2 \myotimes   \sigma_z + b \sigma_x  \myotimes   \mathbb{1}_2 + c \sigma_x  \myotimes   \sigma_z
$ & $
\vert 15 \rangle = \begin{pmatrix}  -1, 0, 1, 0\end{pmatrix}^\intercal $ & $
\vert 8 \rangle = \begin{pmatrix}  0, 1, 0, 1\end{pmatrix}^\intercal $ & $
\vert 6 \rangle = \begin{pmatrix}  0, -1, 0, 1\end{pmatrix}^\intercal $ & $
\vert 19 \rangle = \begin{pmatrix}  1, 0, 1, 0\end{pmatrix}^\intercal
$ \\
\hline
&$-a - b - c$& $a + b - c$& $a - b + c$&   $-a + b + c$\\
\hline
$
a \sigma_z  \myotimes   \sigma_z + b \sigma_x  \myotimes   \sigma_x + c \sigma_y  \myotimes   \sigma_y
$ & $
\vert 5 \rangle = \vert \Psi_- \rangle = \begin{pmatrix}  0, 1, -1, 0\end{pmatrix}^\intercal $ & $
\vert 18 \rangle = \vert \Phi_+ \rangle = \begin{pmatrix}  1, 0, 0, 1\end{pmatrix}^\intercal $ & $
\vert 16 \rangle = \vert \Phi_- \rangle = \begin{pmatrix}  1, 0, 0, -1\end{pmatrix}^\intercal $ & $
\vert 9 \rangle = \vert \Psi_+ \rangle = \begin{pmatrix}  0, 1, 1, 0\end{pmatrix}^\intercal
$
\end{tabular}
\end{ruledtabular}
\end{table*}


\section{Peres-Mermin square}

Applying these techniques to the Peres-Mermin (PM) square~\cite{peres111,mermin90b,peres-91,cabello2021contextuality} renders
24 propositions and 24 contexts, henceforth called the 24-24 configuration,
that is the `completion' of the (minimal in four dimensions~\cite{Pavicic-2005,pavicic-2005csvcorri}) 18-9 KS configuration comprising 18 vectors in 9 contexts~\cite{cabello-96}.
In more detail, this configuration involves nine dichotomic observables with eigenvalues $\pm 1$ arranged in a $3 \times 3$ PM matrix~\eqref{2024-convert-PeresSquare}.
Its rows and columns are masking six four-element contexts, one per row and column ($\sigma_i \myotimes \sigma_j$ stands for the tensor product of Pauli spin matrices $\sigma_i \otimes \sigma_j$, with similar notation for $\mathbb{1}_2$)
\begin{equation}
\begin{pmatrix}
\sigma_z \myotimes  \mathbb{1}_2 & \mathbb{1}_2 \myotimes  \sigma_z & \sigma_z \myotimes  \sigma_z \\
\mathbb{1}_2 \myotimes  \sigma_x & \sigma_x \myotimes  \mathbb{1}_2 & \sigma_x \myotimes  \sigma_x \\
\sigma_z \myotimes  \sigma_x & \sigma_x \myotimes  \sigma_z & \sigma_y \myotimes  \sigma_y
\end{pmatrix}
.
\label{2024-convert-PeresSquare}
\end{equation}


To explicitly demonstrate the difficulties involved co-diagonalization of commuting degenerate matrices
consider the last row of  the PM square~(\ref{2024-convert-PeresSquare}).
Its operators
$\sigma_z  \myotimes   \sigma_x$, $\sigma_x  \myotimes   \sigma_z$, and $\sigma_y  \myotimes   \sigma_y $ mutually
commute---for instance, $[\sigma_z  \myotimes   \sigma_x,\sigma_y  \myotimes   \sigma_y]=0$.
However, a straightforward calculation of the eigenvectors of $\sigma_z \myotimes  \sigma_x$ yields:
$\begin{pmatrix}
{0, 1, 0, 1}
\end{pmatrix}^\intercal $,
$\begin{pmatrix}
{-1, 0, 1, 0}
\end{pmatrix}^\intercal $,
$\begin{pmatrix}
{0, -1, 0, 1}
\end{pmatrix}^\intercal $, and
$\begin{pmatrix}
{1, 0, 1, 0}
\end{pmatrix}^\intercal $.
None of these eigenvectors are eigenvectors of $\sigma_y \myotimes  \sigma_y$, and vice versa.
This demonstrates the difficulties involved in co-diagonalizing  commuting degenerate matrices.

Nonetheless, the `joint' PM square contexts are revealed as the normalized eigenvectors of the respective matrix pencils~\eqref{2024-convert-matrixpencil}.
Table~\ref{2024-convert-matrixpencil-peres} enumerates those contexts,
provided that the $\sigma$-matrices are encoded in the standard form.
%$
%\sigma ( \theta ,\varphi ) = \sigma_x \sin\theta \cos\varphi  + \sigma_y \sin\theta \sin\varphi  + \sigma_z \cos\theta
%$,
%where $0 \le \theta \le \pi$ is the polar angle in the $x$--$z$-plane
%from the $z$-axis,
%and $0 \le \varphi < 2 \pi$ is the azimuthal angle in the $x$--$y$-plane
%from the $x$-axis,
%with
%$
%\sigma_x =  {\sigma} \left(\frac{\pi}{2},0\right)= \mathrm{antidiag}\begin{pmatrix} 1 , 1  \end{pmatrix}
%$,
%$
%\sigma_y =  {\sigma} \left(\frac{\pi}{2},\frac{\pi}{2}\right)= \mathrm{antidiag}\begin{pmatrix} -i ,   i  \end{pmatrix}
%$,
%and
%$
%\sigma_z =  {\sigma} \left(0,0\right)= \mathrm{diag} \begin{pmatrix} 1 ,  -1   \end{pmatrix}
%$.
%Then the resulting set of 24 vectors contains all 18 vectors enumerated by Cabello, Estebaranz and Garc{\'{i}}a-Alcaine~\cite{cabello-96},
%but the contexts are enumerated differently and need to be reconstructed through orthogonality relations.
%Both the PM square--through the hypothetical existence of  uniform dichotomic, operator based values,
%as well as the KS set of 18 elements in 9 contexts---through the hypothetical existence of  uniform binary two-valued states,
%support parity proofs of nonclassicality.








%`Directing' or `orienting' the 24 different eigenvectors of
%Table~\ref{2024-convert-matrixpencil-peres}  such that their first non-negative entry is positive, and
%subsequently lexicographically ordering them, results in a list enumerated in Table~\ref{2024-convert-matrixpencil-peres-DirectedLexOrdered}.
%
%\begin{table*}[ht]
%\caption{\label{2024-convert-matrixpencil-peres-DirectedLexOrdered}The 24 vectors specifying the binary observables (propositions) of the PM square,
%with normalization factors omitted.}
%\begin{ruledtabular}
%%\begin{center}
%%\resizebox{0.95\textwidth}{!}{
%\begin{tabular}{lllllllll}
%$ \begin{pmatrix}   0, 0, 0, 1 \end{pmatrix}^\intercal  $, &
%$\begin{pmatrix}   0, 0, 1, -1 \end{pmatrix}^\intercal  $, &
%$ \begin{pmatrix}   0, 0, 1,  0 \end{pmatrix}^\intercal  $, &
%$ \begin{pmatrix}   0, 0, 1, 1 \end{pmatrix}^\intercal  $, &
%$ \begin{pmatrix}   0, 1, -1, 0 \end{pmatrix}^\intercal  $,
%\\
%$ \begin{pmatrix}   0, 1, 0, -1 \end{pmatrix}^\intercal  $, &
%$ \begin{pmatrix}   0, 1, 0, 0 \end{pmatrix}^\intercal  $, &
%$ \begin{pmatrix}   0, 1, 0, 1 \end{pmatrix}^\intercal  $, &
%$ \begin{pmatrix}   0, 1, 1,  0 \end{pmatrix}^\intercal  $, &
%$ \begin{pmatrix}   1, -1, -1, -1 \end{pmatrix}^\intercal  $,
%\\
%$ \begin{pmatrix}   1, -1, -1, 1 \end{pmatrix}^\intercal  $, &
%$ \begin{pmatrix}   1, -1, 0, 0 \end{pmatrix}^\intercal  $, &
%$ \begin{pmatrix}   1, -1,  1, -1 \end{pmatrix}^\intercal  $, &
%$ \begin{pmatrix}   1, -1, 1, 1 \end{pmatrix}^\intercal  $, &
%$ \begin{pmatrix}   1, 0, -1, 0 \end{pmatrix}^\intercal  $,
%\\
%$ \begin{pmatrix}   1, 0, 0, -1 \end{pmatrix}^\intercal  $, &
%$ \begin{pmatrix}   1, 0, 0,  0 \end{pmatrix}^\intercal  $, &
%$ \begin{pmatrix}   1, 0, 0, 1 \end{pmatrix}^\intercal  $, &
%$ \begin{pmatrix}   1, 0, 1, 0 \end{pmatrix}^\intercal  $, &
%$ \begin{pmatrix}   1, 1, -1, -1 \end{pmatrix}^\intercal  $,
%\\
%$ \begin{pmatrix}   1, 1, -1, 1 \end{pmatrix}^\intercal  $, &
%$ \begin{pmatrix}  1,   1, 0, 0 \end{pmatrix}^\intercal  $, &
%$ \begin{pmatrix}   1, 1, 1, -1 \end{pmatrix}^\intercal  $, &
%$ \begin{pmatrix}   1, 1, 1, 1 \end{pmatrix}^\intercal  $
%\end{tabular}
%%}
%%\end{center}
%\end{ruledtabular}
%\end{table*}

Analysis of their orthogonality relations yields an adjacency matrix that, in turn,
can be used to construct the respective  (hyper)graph through the intertwining 24 cliques and thus contexts thereof.
As can be expected, there are only four-cliques corresponding to orthonormal bases in four dimensional Hilbert space.
%They are enumerated in Table~\ref{2024-convert-matrixpencil-peres-DirectedLexOrdered-cliques}.
%Indeed, they represent the completed and extended elements and orthogonality relationships of the Cabello, Estebaranz and Garc{\'{i}}a-Alcaine~\cite{cabello-96} set of 18 vector labels in 9 contexts (also depicted in~\cite[Figure 4(c)]{pavicic-2004ksafq}
%and mentioned in~\cite{pavicic-2010nkss}, as well as~\cite[Figure 2]{2010-qchocolate}).
Figure~\ref{2024-convert-f-24-24}(a) depicts the hypergraph representing these intertwining contexts as unbroken smooth lines,
 and the vector labels as elements of these contexts,
as enumerated in Table~\ref{2024-convert-matrixpencil-peres}.
% and~\ref{2024-convert-matrixpencil-peres-DirectedLexOrdered-cliques}.

The 24 rays were already discussed by Peres~\cite{peres-91} as permutations of the vector components of
$\begin{pmatrix}
{1, 0, 0, 0}
\end{pmatrix}^\intercal $,
$\begin{pmatrix}
{1, 1, 0, 0}
\end{pmatrix}^\intercal $,
$\begin{pmatrix}
{1, -1, 0, 0}
\end{pmatrix}^\intercal $,
$\begin{pmatrix}
{1, 1, 1, 1}
\end{pmatrix}^\intercal $,
$\begin{pmatrix}
{1, 1, 1, -1}
\end{pmatrix}^\intercal $, and
$\begin{pmatrix}
{1, 1, -1, -1}
\end{pmatrix}^\intercal $.
The `full' 24-24 configuration was obtained by Pavi\v{c}i\'{c}~\cite{pavicic-2004ksafq} who
reconstructed additional 18 contexts not provided in the original Peres paper~\cite{peres-91} by hand~\cite{pavicic-private-commun-2024}.
Peres' 24-24 configuration is arranged in four-element contexts  associated with four-dimensional Hilbert space, with vector
components drawn from the set $\{-1,0,1\}$, that do not support any two-valued state.
%An early incomplete representation of the Peres configuration~\cite{peres-91} was presented by Tkadlec~\cite[Figure~1]{tkadlec-00}  who mentions  24 elements in 22 contexts.
Pavi\v{c}i\'{c}, Megill and Merlet~\cite[Table~1]{pavicic-2010nkss}
have demonstrated that Peres' 24-24 configuration contains 1,233 sets that do not support any two-valued states.
Among these 1,233 sets are six `irreducible' or `critical' configurations which do not contain any proper subset that does not support two-valued states.
Notably, among these configurations is the previously mentioned 18-9 configuration proposed by Cabello, Estebaranz and Garc{\'{i}}a-Alcaine~\cite{cabello-96}.
Previously, Pavi\v{c}i\'{c}, Merlet, McKay, and Megill~\cite[Section~5(viii)]{Pavicic-2005,pavicic-2005csvcorri} had shown that,
among all sets with 24 rays and vector components from the set $\{-1,0,1\}$, and
24 contexts, only one configuration does not
allow any two valued state---and that one is isomorphic to Peres' `full' (including 18 additional contexts)
24-24 configuration enumerated by Pavi\v{c}i\'{c}~\cite{pavicic-2004ksafq}.
This computation had taken one year on a single CPU of a supercomputer~\cite{pavicic-private-commun-2024}.
More recently, Pavi\v{c}i\'{c} and Megill~\cite[Table~1]{Pavii2018} have demonstrated that the vector components from the set $\{-1,0,1\}$
vector-generate a  24-24 set, which contains all smaller KS sets  and is simultaneously isomorphic to the `completed'  24-24 configuration configuration.


We conjecture that if a `larger' collection of contexts (such as 24-24) contains a `smaller' collection of contexts (such as 18-9),
then it inherits the scarcity or total absence of two-valued states of the latter: if the `smaller' set cannot support features related to two-valued states, such as separability of propositions~\cite[Theorem 0]{kochen1},
then intertwining or adding contexts can only impose further constraints, thereby exacerbating the situation by introducing new conditions.



%\begin{table*}[ht]
%\caption{\label{2024-convert-matrixpencil-peres-DirectedLexOrdered-cliques}The 24 cliques or contexts or orthonormal bases or maximal operators
%of the PM square.
%Numbers $1 \le i \le 24$ refer to  the vectors $ \vert i \rangle $ defined in Table~\ref{2024-convert-matrixpencil-peres-DirectedLexOrdered},
%or, equivalently, to the onedimensional orthogonal projection operator $ E_i = (\vert i \rangle \langle i \vert)/ \langle  i \vert i \rangle   $.}
%\begin{ruledtabular}
%\begin{tabular}{cccccc}
%$ C_{ 1 } =       \big\{ 5, 9, 16, 18 \big\}  $, &
%$ C_{ 2 } =       \big\{ 3, 7, 16, 18 \big\}  $, &
%$ C_{ 3 } =       \big\{ 9, 14, 16, 21 \big\}  $, &
%$ C_{ 4 } =       \big\{ 9, 13, 18, 20 \big\}  $, &
%$ C_{ 5 } =       \big\{ 11, 13, 20, 24 \big\}  $, &
%$ C_{ 6 } =       \big\{ 5, 11, 16, 24 \big\}  $,
%\\
%$ C_{ 7 } =       \big\{ 10, 14, 21, 23 \big\}  $, &
%$ C_{ 8 } =       \big\{ 5, 10, 18, 23 \big\}  $, &
%$ C_{ 9 } =       \big\{ 8, 14, 15, 23 \big\}  $, &
%$ C_{ 10 } =      \big\{ 8, 11, 19, 20 \big\}  $, &
%$ C_{ 11 } =      \big\{ 6, 13, 15, 24 \big\}  $, &
%$ C_{ 12 } =      \big\{ 6, 10, 19, 21 \big\}  $,
%\\
%$ C_{ 13 } =      \big\{ 6, 8, 15, 19 \big\}  $, &
%$ C_{ 14 } =      \big\{ 3, 6, 8, 17 \big\}  $, &
%$ C_{ 15 } =      \big\{ 4, 12, 21, 23 \big\}  $, &
%$ C_{ 16 } =      \big\{ 4, 11, 13, 22 \big\}  $, &
%$ C_{ 17 } =      \big\{ 2, 12, 20, 24 \big\}  $, &
%$ C_{ 18 } =      \big\{ 2, 10, 14, 22 \big\}  $,
%\\
%$ C_{ 19 } =      \big\{ 2, 4, 12, 22 \big\}  $, &
%$ C_{ 20 } =      \big\{ 2, 4, 7, 17 \big\}  $, &
%$ C_{ 21 } =     \big\{ 1, 7, 15, 19 \big\}  $, &
%$ C_{ 22 } =      \big\{ 1, 5, 9, 17 \big\}  $, &
%$ C_{ 23 } =      \big\{ 1, 3, 12, 22 \big\}  $, &
%$ C_{ 24 } =      \big\{ 1, 3, 7, 17 \big\}  $
%\end{tabular}
%\end{ruledtabular}
%\end{table*}



\begin{figure*}[ht]
\centering
\begin{tabular}{ccc}
\begin{minipage}{.43\textwidth}
\resizebox{1\textwidth}{!}{
\includegraphics{2024-convert-pra-externalfigures-figure0.eps}
}
    \end{minipage}%
&
    \begin{minipage}{0.27\textwidth}
\resizebox{1\textwidth}{!}{
\includegraphics{2024-convert-pra-externalfigures-figure1.eps}
}
    \end{minipage}%
&
    \begin{minipage}{0.27\textwidth}
\begin{tabular}{c}
\resizebox{1\textwidth}{!}{
\includegraphics{2024-convert-pra-externalfigures-figure2.eps}
}
\\
\resizebox{1\textwidth}{!}{
\includegraphics{2024-convert-pra-externalfigures-figure3.eps}
}
    \end{tabular}
    \end{minipage}%
\\
(a)&(b)&(c)
    \end{tabular}
\caption{\label{2024-convert-f-24-24}
(a)
Hypergraph representing contexts (or cliques or orthonormal bases or maximal operators) %enumerated in Table~\ref{2024-convert-matrixpencil-peres-DirectedLexOrdered-cliques}
as unbroken smooth lines. This is an `orthogonal completion'~\cite{peres-91,pavicic-2004ksafq} of the
KS set comprising 18 vectors in 9 contexts introduced by Cabello, Estebaranz and Garc{\'{i}}a-Alcaine~\cite{cabello-96}.
The filled shaded small ovals on the edges correspond to the `primary' isolated (nonintertwined) contexts from the matrix pencil calculations enumerated
in Table~\ref{2024-convert-matrixpencil-peres}.
(b)
Hypergraph representing a 16-12 configuration: 16 elements in 12 contexts enumerated in the first, second, fourth, and fifth row of Table~\ref{2024-convert-matrixpencil-peres}.
These vectors are separable and thus correspond to factorizable, nonentangled states.
(c)
Two equivalent hypergraph representations of a 8-4 configuration---8 elements in 4 contexts
enumerated in the third and sixth row of
Table~\ref{2024-convert-matrixpencil-peres}.
These vectors are nonseparable and thus correspond to entangled states.}
\end{figure*}




\section{Greenberger-Horne-Zeilinger argument}


Based on the GHZ argument Mermin has suggested~\cite{mermin,mermin90b} a `simple unified form for the major no-hidden-variables theorems' in which he identified four commuting three-partite operators:
$\sigma_x \myotimes  \sigma_x \myotimes  \sigma_x$, $\sigma_x \myotimes  \sigma_y \myotimes  \sigma_y$, $\sigma_y \myotimes  \sigma_x \myotimes  \sigma_y$, and $\sigma_y \myotimes  \sigma_y \myotimes  \sigma_x$.
A parity argument reveals a state-independent quantum contradiction to the classical existence of (local, noncontextual) elements of physical reality:
The quantum mechanical expectation of the product of these four commuting three-partite operators for any quantum state is
$
-1= \langle
-\mathbb{1}_8
 \rangle
=
\langle
\mathbb{1}_2
\myotimes
(
-\mathbb{1}_2
)
\myotimes
\mathbb{1}_2
 \rangle
=
\langle
(
%\underbrace{
\sigma_x  \cdot \sigma_x  \cdot \sigma_y   \cdot \sigma_y
%}_{\mathbb{1}_2}
)
\myotimes
(
%\underbrace{
\sigma_x   \cdot \sigma_y   \cdot \sigma_x   \cdot  \sigma_y
%}_{-\mathbb{1}_2}
)
\myotimes
(
%\underbrace{
\sigma_x   \cdot \sigma_y    \cdot \sigma_y    \cdot \sigma_x
%}_{\mathbb{1}_2}
) \rangle
=
\langle  (\sigma_x \myotimes  \sigma_x \myotimes  \sigma_x) \cdot (\sigma_x \myotimes  \sigma_y \myotimes  \sigma_y) \cdot (\sigma_y \myotimes  \sigma_x \myotimes  \sigma_y) \cdot (\sigma_y \myotimes  \sigma_y \myotimes  \sigma_x) \rangle
=
\langle  \sigma_x \myotimes  \sigma_x \myotimes  \sigma_x  \rangle \langle \sigma_x \myotimes  \sigma_y \myotimes  \sigma_y  \rangle \langle
\sigma_y \myotimes  \sigma_x \myotimes  \sigma_y  \rangle \langle   \sigma_y \myotimes  \sigma_y \myotimes  \sigma_x \rangle
$.
In this formulation, every operator $\sigma_x$ and $\sigma_y$ for each of the three particles occurs twice. Therefore, if classically all such single-particle observables would coexist as elements of physical reality and independent of what other measurements are made alongside,
then their respective product must be $1$, the exact negative of the quantum expectation.


Mermin's configuration can be analyzed in terms of its matrix pencil
$
a \sigma_x \myotimes  \sigma_x \myotimes  \sigma_x + b \sigma_x \myotimes  \sigma_y \myotimes  \sigma_y + c \sigma_y \myotimes  \sigma_x \myotimes  \sigma_y + d \sigma_y \myotimes  \sigma_y \myotimes  \sigma_x
%\label{2024-convert-mpghzm}
$,
thereby revealing the underlying, hidden context in terms of
the simultaneous eigensystem of the four mutually commuting operators.
%
%enumerated in Table~\ref{2024-convert-mpghzm-es}.
%
%\begin{table*}[ht]
%\caption{\label{2024-convert-mpghzm-es}Eigensystem of the matrix pencil~\eqref{2024-convert-mpghzm}
%associated with the Mermin configuration~\cite{mermin,mermin90b}, constituting an orthogonal basis in an eight-dimensional Hilbert space.
%The values `$\boldsymbol{+}$' and `$\boldsymbol{-}$' represent the measured vales $+1$ and $-1$ of the respective operators.
%}
%\centering
%\begin{ruledtabular}
%\begin{tabular}{rlcccccccc}
% \multicolumn{1}{c}{eigenvalue} & \multicolumn{1}{c}{eigenvector} & $\sigma_x \myotimes  \sigma_x \myotimes  \sigma_x$ & $\sigma_x \myotimes  \sigma_y \myotimes  \sigma_y$ & $\sigma_x \myotimes  \sigma_y \myotimes  \sigma_y$ & $\sigma_y \myotimes  \sigma_y \myotimes  \sigma_x$ \\
%\hline
%   $-a + b - c - d $ &          $ \vert \Gamma_1 \rangle = \frac{1}{\sqrt{2}}\left( \vert z_+z_-z_- \rangle  - \vert z_-z_+z_+ \rangle \right) \equiv \begin{pmatrix}          0, 0, 0, 1, -1, 0, 0, 0            \end{pmatrix}^\intercal  $  &   $\boldsymbol{-}$ &  $\boldsymbol{+}$  & $\boldsymbol{-}$   & $\boldsymbol{-}$        \\
%   $a - b + c + d  $&           $ \vert \Gamma_2 \rangle = \frac{1}{\sqrt{2}}\left( \vert z_+z_-z_- \rangle  + \vert z_-z_+z_+ \rangle \right) \equiv \begin{pmatrix}          0, 0, 0, 1, 1, 0, 0, 0            \end{pmatrix}^\intercal   $  &   $\boldsymbol{+}$ &  $\boldsymbol{-}$  & $\boldsymbol{+}$   & $\boldsymbol{+}$       \\
%   $-a - b + c - d $ &          $ \vert \Gamma_3 \rangle = \frac{1}{\sqrt{2}}\left( \vert z_+z_-z_+ \rangle  - \vert z_-z_+z_- \rangle \right) \equiv \begin{pmatrix}          0, 0, 1, 0, 0, -1, 0, 0            \end{pmatrix}^\intercal  $  &   $\boldsymbol{-}$ &  $\boldsymbol{-}$  & $\boldsymbol{+}$   & $\boldsymbol{-}$        \\
%   $a + b - c + d  $&           $ \vert \Gamma_4 \rangle = \frac{1}{\sqrt{2}}\left( \vert z_+z_-z_+ \rangle  + \vert z_-z_+z_- \rangle \right) \equiv \begin{pmatrix}          0, 0, 1, 0, 0, 1, 0, 0            \end{pmatrix}^\intercal   $  &   $\boldsymbol{+}$ &  $\boldsymbol{+}$  & $\boldsymbol{-}$   & $\boldsymbol{+}$       \\
%   $-a - b - c + d $ &          $ \vert \Gamma_5 \rangle = \frac{1}{\sqrt{2}}\left( \vert z_+z_+z_- \rangle  - \vert z_-z_-z_+ \rangle \right) \equiv \begin{pmatrix}          0, 1, 0, 0, 0, 0, -1, 0            \end{pmatrix}^\intercal  $  &   $\boldsymbol{-}$ &  $\boldsymbol{-}$  & $\boldsymbol{-}$   & $\boldsymbol{+}$        \\
%   $  a + b + c - d$  &         $ \vert \Gamma_6 \rangle = \frac{1}{\sqrt{2}}\left( \vert z_+z_+z_- \rangle  + \vert z_-z_-z_+ \rangle \right) \equiv \begin{pmatrix}          0, 1, 0, 0, 0, 0,   1, 0            \end{pmatrix}^\intercal $  &   $\boldsymbol{+}$ &  $\boldsymbol{+}$  & $\boldsymbol{+}$   & $\boldsymbol{-}$         \\
%   $-a + b + c + d $&           $ \vert \Gamma_7 \rangle = \frac{1}{\sqrt{2}}\left( \vert z_+z_+z_+ \rangle  - \vert z_-z_-z_- \rangle \right) \equiv \begin{pmatrix}          1, 0, 0, 0, 0, 0, 0, -1            \end{pmatrix}^\intercal  $  &   $\boldsymbol{-}$ &  $\boldsymbol{+}$  & $\boldsymbol{+}$   & $\boldsymbol{+}$     \\
%   $a - b - c - d  $&           $ \vert \Gamma_8 \rangle = \frac{1}{\sqrt{2}}\left( \vert z_+z_+z_+ \rangle  + \vert z_-z_-z_- \rangle \right) \equiv \begin{pmatrix}          1, 0, 0, 0, 0, 0, 0, 1            \end{pmatrix}^\intercal   $  &   $\boldsymbol{+}$ &  $\boldsymbol{-}$  & $\boldsymbol{-}$   & $\boldsymbol{-}$       \\
%\end{tabular}
%\end{ruledtabular}
%\end{table*}
%
These eight nonseparable vectors form an orthonormal basis of an eight-dimensional Hilbert space corresponding to an isolated single context~\cite[Table~1]{svozil-2020-ghz} of  entangled states.
Therefore, Mermin's configuration does not constitute a KS proof, as it still permits a separating set of eight two-valued states.

In view of this, how does one arrive at a complete GHZ contradiction with classical elements of physical reality, as outlined above?
The criterion employed in an experimental corroboration~\cite{panbdwz} is to select any one of the eigenstates forming the orthonormal basis, such as
$
(1/\sqrt{2})
\big(
\vert z_+z_+z_+ \rangle  + \vert z_-z_-z_- \rangle
\big)
%= \begin{pmatrix}          1, 0, 0, 0, 0, 0, 0, 1            \end{pmatrix}^\intercal
$.
Since this is an eigenstate of all four terms of the matrix pencil, four separate measurements can be performed (possibly temporally separated) yielding the eigenvalues
$+1$ for
$\sigma_x \myotimes  \sigma_x \myotimes  \sigma_x$
as well as $-1$ for the three others. These three factors  $-1$ and one factor $+1$ contribute to their product value $-1$, in total contradiction to the classical expectation $+1$.
Note that similar contradictions arise if the seven other eigenstates of the matrix pencil are considered~\cite[Table~1]{svozil-2020-ghz}.
%These quantum-versus-classical discrepancies are independent of the specific state.
%We chose the particular state $\vert \Gamma_8 \rangle$ because it is---among the eight eigenstates enumerated in Table~\ref{2024-convert-mpghzm-es}---an eigenstate of the individual terms involved in the matrix pencil.


%As we have seen, the quantum logic and hypergraph representation of the GHZ-Mermin argument is basic---indeed, it amounts to a single context hypergraph representable by a line~\cite[Figure~2(a)]{svozil-2020-ghz}.
%The `strength' of its conviction lies in its operational realizability as a collection of four, preferably nonlocal
%(under strict Einstein separability) measurements of the operators mentioned, and applied to one of the eigenstates~\cite{panbdwz}.
%Since all the operators involved mutually commute, they could theoretically be serially composed--applied one after the other in any order.
%%However, such a scenario would undermine the nonlocality, as it would require the respective input and output ports to be aligned.

\section{Bipartite Greenberger-Horne-Zeilinger argument}

Can an equally convincing argument be made involving just two particles?
Natural candidates would be the `nonclassical' elements of the PM square~\eqref{2024-convert-PeresSquare}.
Note that its `masked' or `hidden' contexts, revealed by the matrix pencils, can be partitioned into four `separable' type contexts
depicted in Figure~\ref{2024-convert-f-24-24}(b)
containing only separable vectors---corresponding to the first and second rows and columns---and two `nonclassical' contexts consisting of nonseparable
vectors---corresponding to the last row and column, as depicted in
Figure~\ref{2024-convert-f-24-24}(c).



Concentrating on these two latter contexts consisting of nonseparable vectors, we make the following observations:
%(i)
Since the observables from the last row and last column (with the exception of  $\sigma_y \myotimes \sigma_y$) do not commute, they cannot be simultaneously measured.
%(ii)
Nevertheless, by forming products within the last row and column, we may create two commuting operators
%in classical terms, the same `elements of physical reality', namely
%$(\sigma_z \myotimes \sigma_x) \cdot (\sigma_x \myotimes \sigma_z) \cdot (\sigma_y \myotimes \sigma_y)$
%and
%$(\sigma_z \myotimes \sigma_z) \cdot (\sigma_x \myotimes \sigma_x) \cdot (\sigma_y \myotimes \sigma_y)$.
%Within the contexts, that is, if they belong to the same product, any of these factors can be simultaneously measured.
%And yet, quantum mechanically, the first product yields $\mathbb{1}_4$ and the second $-\mathbb{1}_4$,
%and thus their respective expectations (on any state) are $+1$ and $-1$ This latter prediction is in total contradiction with classical expectations.
%
%The product of these two products contains an even number of classical supposedly `elements of physical reality' for both particles,
%similarly to the GHZ-Mermin argument for three particles.
%Dissimilar to the latter  it involves two contexts rather than just one.
%One may object that the two-partite `analog' is trivial in the sense that one effectively `merely measures identities' and not the individual expectations.
%This is due to complementarity, as, say,  $\sigma_x \myotimes \sigma_x$ is not in the same context as $\sigma_x \myotimes \sigma_z$.
%But it should be kept in mind that the basic presumptions remain the same in both arguments.
%
%The argument could be made shorter and possibly more convincing by realizing that the following operators commute because of equality:
$
(\sigma_z \myotimes \sigma_x) \cdot (\sigma_x \myotimes \sigma_z) = -(\sigma_x \myotimes \sigma_x) \cdot (\sigma_z \myotimes \sigma_z)
=
(\sigma_z \cdot \sigma_x ) \myotimes (\sigma_x \cdot \sigma_z)
=   \sigma_y  \myotimes \sigma_y
= \text{antidiag}
\begin{pmatrix} -1 , 1 ,1, -1
\end{pmatrix}
$.
%They share the Bell basis as common system of eigenvectors.
Their matrix pencil
\begin{equation}
%\begin{split}
a(\sigma_z \myotimes \sigma_x) \cdot (\sigma_x \myotimes \sigma_z) + b (\sigma_x \myotimes \sigma_x) \cdot (\sigma_z \myotimes \sigma_z)
% + c (\sigma_y \myotimes \sigma_y) \cdot (\sigma_y \myotimes \sigma_y)
%\end{split}
\label{2024-convert-mppm}
\end{equation}
has a degenerate spectrum with the Bell basis as eigenvectors---the
same as the eigenvectors of the matrix pencil of the last column of the PM square.
(Alternatively, we could have used
the pencil
\(
a(\sigma_z \myotimes \sigma_x) \cdot (\sigma_x \myotimes \sigma_z) + b \sigma_x \myotimes \sigma_x + c \sigma_z \myotimes \sigma_z
\) to avoid multiplicities.)
It is enumerated in Table~\ref{2024-convert-pm-es}.


Hence, preparing a state in one Bell basis state and measuring (successively or separately)
$
(\sigma_z \myotimes \sigma_x) \cdot (\sigma_x \myotimes \sigma_z)
$,
and either
$
(\sigma_x \myotimes \sigma_x) \cdot (\sigma_z \myotimes \sigma_z)
$
or
$
\sigma_x \myotimes \sigma_x
$
as well as
$
\sigma_z \myotimes \sigma_z
$ separately,
yields
\begin{equation}
\begin{split}  -1 =
\langle
-\mathbb{1}_4
 \rangle
=
\langle
\mathbb{1}_2 \myotimes (-\mathbb{1}_2)
 \rangle
\\
=
 \langle
(\sigma_z  \cdot \sigma_x \cdot \sigma_x \cdot \sigma_z ) \myotimes (\sigma_x \cdot \sigma_z \cdot  \sigma_x \cdot \sigma_z)
 \rangle
\\
=
\langle
(\sigma_z \myotimes \sigma_x) \cdot (\sigma_x \myotimes \sigma_z)   \cdot
(\sigma_x \myotimes \sigma_x) \cdot (\sigma_z \myotimes \sigma_z)
\rangle
\\
=
\langle
(\sigma_z \myotimes \sigma_x) \cdot (\sigma_x \myotimes \sigma_z)\rangle
\langle  (\sigma_x \myotimes \sigma_x) \cdot (\sigma_z \myotimes \sigma_z) \rangle
.
\end{split}
\end{equation}

In contrast, and in analogy to Mermin's version of the GHZ argument, the classical prediction is that the product of these terms always needs to be positive, as every alleged `element of reality', in particular corresponding to $\sigma_x$ and $\sigma_z$, enters an even number of times (indeed, twice per particle).



\begin{table}[t]
\caption{\label{2024-convert-pm-es}Eigensystem of the matrix pencil~\eqref{2024-convert-mppm}
associated with the commuting  products of operators in the last (third) row and the last (third) column of the PM square,
constituting the Bell basis.
Inclusion of  $(\sigma_y \myotimes \sigma_y) \cdot (\sigma_y \myotimes \sigma_y) = \mathbb{1}_4$
does not change the calculation and is therefore omitted.
The values $+1$ and $-1$ represent the (co)measured values of the respective commuting operators.
}
\centering
\begin{ruledtabular}
\begin{tabular}{rccccccccc}
\multicolumn{1}{c}{value} &
\multicolumn{1}{c}{vector} &
$(\sigma_z \myotimes  \sigma_x) \cdot (\sigma_x\myotimes  \sigma_z)$ &
$\sigma_x \myotimes  \sigma_x$ &
$\sigma_z\myotimes  \sigma_z$ &
$(\sigma_x \myotimes  \sigma_x) \cdot  (\sigma_z\myotimes  \sigma_z)$
\\
\hline
   $a - b$ &          $ \vert \Psi_+ \rangle $   &   $+1$  &  $+1$    &  $-1$   &  $-1$      \\
   $a - b$ &          $ \vert \Phi_- \rangle $   &   $+1$  &  $-1$    &  $+1$   &  $-1$      \\
   $-a + b$ &         $ \vert \Psi_- \rangle $   &   $-1$  &  $-1$    &  $-1$   &  $+1$      \\
   $-a + b$ &         $ \vert \Phi_+ \rangle $   &   $-1$  &  $+1$    &  $+1$   &  $+1$      \\
\end{tabular}
\end{ruledtabular}
\end{table}


%I conclude with some comments and an outlook.
%A nonlocal measurement in quantum mechanics refers to the simultaneous measurement of properties of entangled particles that are---at least in principle---located in space-like separated regions (Einstein locality).
%We therefore suggest calling an operator, or a collection of mutually commuting operators, `nonlocal' if their eigensystem of their matrix pencil---allow entangled, that is, nonseparable, eigenstates after projective measurements.
%This is the case for the last row and column of the PM square, and also for the four three-partite operators suggested by Mermin in the context of the GHZ argument.
% I shall motivate and discuss these issues further in a later publication.

%We have concentrated on transcriptions of operator-based arguments against the classical performance of quantized systems with dichotomic arguments.
%These methods allow the generalization to non-dichotomic outcomes.
%A converse transcription in terms of a hypergraph representing (possibly intertwining) contexts into nonlocal measurements can be achieved if this hypergraph allows a faithful orthogonal representation~\cite{lovasz-79,Portillo-2015}
%(or, equivalently, a coordinatization~\cite{Pavii2018}) in terms of vector labels that are entangled (or, equivalently, nonfactorizable).
%
%The sum of noncommuting operators such as the Clauser-Horne-Shimony-Holt operator
%$
%\sigma_1 \myotimes \sigma_3
%+
%\sigma_1 \myotimes \sigma_4
%+
%\sigma_2 \myotimes \sigma_3
%-
%\sigma_2 \myotimes \sigma_4
%$
%with
%$\sigma_i= \sigma ( \theta_i ,\varphi_i )$
%can be analized by acknowledging that the sum of products of scalars with normal operators is also a normal operator.
%Therefore, this sum has a (not necessarily unique if the sum is degenerate)
%spectral decomposition in terms of projection operators corresponding to some orthonormal basis.
%This orthonormal basis forms an isolated context whose elements may be simultaneously measured
%and optimized~\cite{filipp-svo-04-qpoly-prl}.
%Alternatively, the contexts formed by the spectral decompositions of the individual summands can be considered as the basis of the analysis.



\section{Conclusions}

The matrix pencil method provides an elegant solution for simultaneously diagonalizing commuting operators with degenerate spectra.
It offers a systematic approach for identifying `contextual' nonclassical performance in quantized systems,
particularly in delineating operator-valued arguments.

The Peres-Mermin (PM) square demonstrates a fundamental contradiction (quantum -1 versus classical +1) compared to classical expectations in a dichotomic operator-valued formulation. By employing matrix pencils, this contradiction can be transcribed into a Kochen-Specker (KS) type argument involving 24 vectors. This configuration, which does not support any binary (two-valued) state, consists of 6 `original' isolated contexts from the matrix pencils associated with every row and column of the PM square, as well as 18 `secondary' intertwining contexts obtained by studying orthogonalities.

Mermin's rendition of the GHZ operator-valued argument is fundamentally different. When transcribed into quantum logic, it reveals a single isolated context that is perfectly set-representable, for instance, by partition logic. The quantum state becomes crucial for any experimental corroboration: if one takes any eigenstate of the matrix pencil, it leads to a complete contradiction (again quantum -1 versus classical +1) when multiplying all the results and comparing the squares of operators in a parity argument.

In analyzing the `entangled contexts' corresponding to the last row and column of the PM square and constructing
mutually commuting products thereof, one arrives at a similar argument to Mermin's rendition of the GHZ argument.
It is also state-independent and operates within a single context. The operators involved are:
$
(\sigma_z \myotimes \sigma_x) \cdot (\sigma_x \myotimes \sigma_z)
$  and alternatively, either
$
(\sigma_x\myotimes \sigma_x) \cdot (\sigma_z\myotimes \sigma_z)
$
or
$
\sigma_x\myotimes \sigma_x
$
and
$\sigma_z\myotimes \sigma_z
$
and, although not needed for the constraction,
$(\sigma_y \myotimes \sigma_y) \cdot (\sigma_y\myotimes \sigma_y)$.
These operators commute, and for the Bell basis yield a complete contradiction (quantum -1 versus classical +1)
contingent on the assumption of noncontextual classical existence of those elements of physical reality.
This reduces the eight-dimensional argument to a four-dimensional one.
%It might be interesting to probe the factors
%$
%\sigma_z \cdot \sigma_x
%$
%and
%$\sigma_x\cdot \sigma_z
%$
%of the tensor product
%$
%(\sigma_z \myotimes \sigma_x) \cdot (\sigma_x \myotimes \sigma_z)
%= (\sigma_z \cdot \sigma_x) \myotimes (\sigma_x \cdot \sigma_z)
%$
%by the Bell states
%$|\Psi_-\rangle$
%and
%$|\Phi_+\rangle$
%in an Einstein-Podolsky-Rosen configuration,
%because this alone could `isolate' the `rub', as the quantum prediction of the observed value would be
%$-1$.
%Similarly, application of the Bell states
%$|\Psi_+\rangle$
%and
%$|\Phi_-\rangle$
%on
%$
%(\sigma_z \myotimes \sigma_z) \cdot (\sigma_x \myotimes \sigma_x)
%= (\sigma_z \cdot \sigma_x) \myotimes (\sigma_z \cdot \sigma_x)
%$
%would result in an observed value $-1$.


Why or how can operator-valued contradictions arise within a single isolated context?
This occurs because measurements such as \(\sigma_x \myotimes \sigma_z \),
which partly define a context derived from a matrix pencil,
should not be considered `local' and cannot be conducted as independent single-qubit local measurements~\cite{cabello2021contextuality}.
Such operator-valued arguments are traditionally rooted in the classical assumption that any multi-particle state
can be decomposed into single-particle states while preserving the properties of the original multi-particle state.
However, this assumption fails in the case of entangled states,
which encodes relational information at the expense of local properties~\cite{zeil-99}.
From this perspective, both dichotomic operator-valued GHZM
arguments and binary two-valued state KS arguments against noncontextuality share a nonoperational and therefore
(meta)physical presumption: the contingent use of counterfactuals.

In summary, this paper proposed using matrix pencils to find a maximal resolution operator for mutually commuting degenerate operators, simplifying the determination of the associated unique context. This method was applied to the Peres-Mermin (PM) square and Mermin's form of the Greenberger-Horne-Zeilinger (GHZM) argument. The PM square's quantum logical structure presented a Kochen-Specker (KS) argument, while the GHZM argument involved a single three-partite context, with quantum mechanical results contradicting classical predictions through a parity argument. Finally, the matrix pencil method was used to construct a two-partite GHZM-type argument based on nonlocal observables in the PM square.

\begin{acknowledgments}

This research was funded in whole or in part by the Austrian Science Fund (FWF), Grant-DOI: 10.55776/I4579. For open access purposes, the author has applied a CC BY public copyright license to any author accepted manuscript version arising from this submission.

A question by Bruno Mittnik stimulated this research.
Philippe Grangier drew my attention to von Neumann's address at the International Congress of Mathematicians in Amsterdam, delivered in September 1954~\cite{vonNeumann2001}.
I acknowledge explanations from Alastair Abbott, Costantino Budroni, Ad\'an Cabello and Jan-\AA{}ke Larsson regarding aspects of Reference~\cite{cabello2021contextuality},
as well as explanations, discussions, and suggestions from Mladen Pavicic regarding the properties of the 24-24 configuration.
(Any remaining confusion remains solely with the author.)
%The author declares no conflict of interest.
\end{acknowledgments}



\bibliography{svozil}
%\bibliographystyle{apsrev}



\end{document}




%%%%%%%%%%%%%%%%%%%%%%%%%%%%%%%%%%%%%%%%%%%%%%%%%%%%%%%%%%%%%%%%%%%%%%%%%%%%%%%%%%%%%%%%%%%%%%%%%%%%%%%%%%%%
%%%%%%%%%%%%%%%%%%%%%%%%%%%%%%%%%%%%%%%%%%%%%%%%%%%%%%%%%%%%%%%%%%%%%%%%%%%%%%%%%%%%%%%%%%%%%%%%%%%%%%%%%%%%
%%%%%%%%%%%%%%%%%%%%%%%%%%%%%%%%%%%%%%%%%%%%%%%%%%%%%%%%%%%%%%%%%%%%%%%%%%%%%%%%%%%%%%%%%%%%%%%%%%%%%%%%%%%%
%%%%%%%%%%%%%%%%%%%%%%%%%%%%%%%%%%%%%%%%%%%%%%%%%%%%%%%%%%%%%%%%%%%%%%%%%%%%%%%%%%%%%%%%%%%%%%%%%%%%%%%%%%%%
%%%%%%%%%%%%%%%%%%%%%%%%%%%%%%%%%%%%%%%%%%%%%%%%%%%%%%%%%%%%%%%%%%%%%%%%%%%%%%%%%%%%%%%%%%%%%%%%%%%%%%%%%%%%
%%%%%%%%%%%%%%%%%%%%%%%%%%%%%%%%%%%%%%%%%%%%%%%%%%%%%%%%%%%%%%%%%%%%%%%%%%%%%%%%%%%%%%%%%%%%%%%%%%%%%%%%%%%%
%%%%%%%%%%%%%%%%%%%%%%%%%%%%%%%%%%%%%%%%%%%%%%%%%%%%%%%%%%%%%%%%%%%%%%%%%%%%%%%%%%%%%%%%%%%%%%%%%%%%%%%%%%%%


From the matrix pencil operations it can be deduced that
measurements of each observable in the
last row and last column of the PM square~(\ref{2024-convert-PeresSquare}) cannot be performed as two single-qubit
local measurements: compare, for instance, the system of eigenvectors of the matrix pencil
$
a \sigma_z  \myotimes   \sigma_x + b \sigma_x  \myotimes   \sigma_z + c \sigma_y  \myotimes   \sigma_y
$.
These eigenvectors
(cf. third matrix pencil of Table~\ref{2024-convert-matrixpencil-peres})
belong to entangled (indecomposable)  states, as the product of their outer coordinates is the negative of (and thus unqual to)
the products of their inner coordinates.
A more detailed analysis shows that the plane, say, spanned by the eigenvectors
with degenerate eigenvalues for $a=1$ and $b=c=0$,
$ \begin{pmatrix}  1, 1, -1, 1\end{pmatrix}^\intercal $ and  $ \begin{pmatrix}  1, 1, 1, -1\end{pmatrix}^\intercal $ carries a continuum of pure entangled (indecomposable) states,
with the exception of two onedimensional subspaces spanned by
$ \begin{pmatrix}  1, 1, -1, 1\end{pmatrix}^\intercal  \pm   \begin{pmatrix}  1, 1, 1, -1\end{pmatrix}^\intercal $~\cite{havlicek-svozil-2020-dec}.
In any case, a projection operator formed by the dyadic product of (normalized) indecomposable eigenvectors cannot correspond to a two single-qubit local measurement.


Indeed, except for singular parameter choices of measure zero,
even measurements of the general form   $\sigma ( \theta_1 ,\varphi_1 ) \myotimes  \sigma ( \theta_2 ,\varphi_2 )$, have eigenvalues
$  \{  -1 , -1 , 1 , 1\} $ and associated eigenvectors
\begin{equation}
\begin{aligned}
&  \begin{pmatrix}  -e^{-i (\phi_1+\phi_2)}   \sin \theta_2 ,e^{-i \phi_1}  (\cos \theta_1+\cos \theta_2) ,0,\sin \theta_1 \end{pmatrix}^\intercal         ,
\\
&  \begin{pmatrix}  e^{-i   \phi_1}  (\cos \theta_1-\cos   \theta_2) ,-e^{-i (\phi_1-\phi_2)}    \sin \theta_2 ,\sin \theta_1,0  \end{pmatrix}^\intercal    ,
\\
&   \begin{pmatrix}  e^{-i  (\phi_1+\phi_2)}  \sin \theta_2 ,e^{-i  \phi_1} (\cos \theta_1-\cos  \theta_2) ,0,\sin \theta_1 \end{pmatrix}^\intercal       ,
\\
&  \begin{pmatrix}  e^{-i \phi_1}  (\cos \theta_1+\cos \theta_2) ,e^{-i   (\phi_1-\phi_2)}  \sin   \theta_2 ,\sin \theta_1,0  \end{pmatrix}^\intercal     ,
\end{aligned}
\label{2024-convert-esgtqlm}
\end{equation}
with a vanishing coordinate equal to $0$, and thus correspond to entangled states
(with the exception of $\theta_1=0$ or $\theta_2=0$ and $\phi_1=\phi_2=0$).


%%%%%%%%%%%%%%%%%%%%%%%%%%%%%%%%%%%%%%%%%%%%%%%%%%%%%%%%%%%%%%%%%%%%%%%%%%%%%%%%%%%%%%%%%%%%%%%%%%%%%%%%%%%%
%%%%%%%%%%%%%%%%%%%%%%%%%%%%%%%%%%%%%%%%%%%%%%%%%%%%%%%%%%%%%%%%%%%%%%%%%%%%%%%%%%%%%%%%%%%%%%%%%%%%%%%%%%%%
%%%%%%%%%%%%%%%%%%%%%%%%%%%%%%%%%%%%%%%%%%%%%%%%%%%%%%%%%%%%%%%%%%%%%%%%%%%%%%%%%%%%%%%%%%%%%%%%%%%%%%%%%%%%

WRONG:

In this line of argument we may employ the matrix pencil of the last (third) column of the PM square~(\ref{2024-convert-PeresSquare}), and confront it with
the Bell basis vector $\vert \Phi_+ \rangle =   (1/\sqrt{2})\big( \vert HH \rangle +  \vert VV \rangle \big) = \begin{pmatrix}          1, 0, 0, 1            \end{pmatrix}^\intercal $.
Because by the parity argument, classically, every observable in this column occurs twice, so multiplication results in squares of the observed vales.
As those are real, this product should classically result in a nonnegative number.
however, the eigenvalues of $\vert \Psi_+ \rangle$ are $+1$ for
$\sigma_x \myotimes  \sigma_x$   and $\sigma_z \myotimes  \sigma_z$ but $-1$ for $\sigma_y \myotimes  \sigma_y$.
When measuring them in the same way (possibly successively and temporally separated one after the other), this product
$
\langle (\sigma_x \myotimes  \sigma_x) \cdot (\sigma_z \myotimes  \sigma_z) \cdot (\sigma_y \myotimes  \sigma_y) \rangle
=\langle (\sigma_x \myotimes  \sigma_x) \rangle \langle (\sigma_z \myotimes  \sigma_z)  \rangle \langle (\sigma_y \myotimes  \sigma_y) \rangle
= -1$
is negative.
Thus, Mermin's two-partite transcription of the KS paradox appears to be as `nonclassical' as the GHZ argument, relative to the assumptions and means of the latter~\cite{panbdwz}.
Note that also in this case any one of the four eigenvectors---the vectors in the Bell basis, in particular, also the singlet state---resulting from the matrix pencil calculation, suffices for an non-probabilistic argument against classicality.



In this line of argument, we may utilize the matrix pencil of the last (third) column of the PM square~(\ref{2024-convert-PeresSquare}),
and compare it with the Bell basis vector $\vert \Phi_+ \rangle = (1/\sqrt{2})\left( \vert HH \rangle + \vert VV \rangle \right) = \begin{pmatrix} 1 , 0 , 0 , 1 \end{pmatrix}^\intercal $.
By the parity argument, classically, every observable in this column occurs twice, resulting in squares of the observed values.
Since these values are real, this product should classically yield a nonnegative number.
However, the eigenvalues of $\vert \Psi_+ \rangle$ are $+1$ for $\sigma_x \myotimes  \sigma_x$ and $\sigma_z \myotimes  \sigma_z$, but $-1$ for $\sigma_y \myotimes  \sigma_y$.
When measuring them in the same way (possibly successively and temporally separated),
the product $\langle (\sigma_x \myotimes  \sigma_x) \cdot (\sigma_z \myotimes  \sigma_z) \cdot (\sigma_y \myotimes  \sigma_y) \rangle = -1$ is negative.
From this point of view, Mermin's two-partite transcription of the KS paradox appears to be as `nonclassical' as the GHZ argument,
relative to the assumptions and means of the latter~\cite{panbdwz}.

Note that in this case, any one of the four eigenvectors resulting from the matrix pencil calculation---the vectors in the Bell basis, in particular, also the singlet state---suffices for a non-probabilistic
GHZ type argument against classicality.
This, of course, begs the immediate question: if the same can be achieved with classical shares such as Peres' exploding bomb~\cite{peres222}---in what way is this form of quantum contextuality different from, say, KS type configurations?
Because whereas the quantum performance of the full PM square cannot be classically emulated, a singly isolated context can.

%%%%%%%%%%%%%%%%%%%%%%%%%%%%%%%%%%%%%%%%%%%%%%%%%%%%%%%%%%%%%%%%%%%%%%%%%%%%%%%%%%%%%%%%%%%%%%%%%%%%%%%%%%%%
%%%%%%%%%%%%%%%%%%%%%%%%%%%%%%%%%%%%%%%%%%%%%%%%%%%%%%%%%%%%%%%%%%%%%%%%%%%%%%%%%%%%%%%%%%%%%%%%%%%%%%%%%%%%
%%%%%%%%%%%%%%%%%%%%%%%%%%%%%%%%%%%%%%%%%%%%%%%%%%%%%%%%%%%%%%%%%%%%%%%%%%%%%%%%%%%%%%%%%%%%%%%%%%%%%%%%%%%%
%%%%%%%%%%%%%%%%%%%%%%%%%%%%%%%%%%%%%%%%%%%%%%%%%%%%%%%%%%%%%%%%%%%%%%%%%%%%%%%%%%%%%%%%%%%%%%%%%%%%%%%%%%%%
%%%%%%%%%%%%%%%%%%%%%%%%%%%%%%%%%%%%%%%%%%%%%%%%%%%%%%%%%%%%%%%%%%%%%%%%%%%%%%%%%%%%%%%%%%%%%%%%%%%%%%%%%%%%
%%%%%%%%%%%%%%%%%%%%%%%%%%%%%%%%%%%%%%%%%%%%%%%%%%%%%%%%%%%%%%%%%%%%%%%%%%%%%%%%%%%%%%%%%%%%%%%%%%%%%%%%%%%%
%%%%%%%%%%%%%%%%%%%%%%%%%%%%%%%%%%%%%%%%%%%%%%%%%%%%%%%%%%%%%%%%%%%%%%%%%%%%%%%%%%%%%%%%%%%%%%%%%%%%%%%%%%%%
%%%%%%%%%%%%%%%%%%%%%%%%%%%%%%%%%%%%%%%%%%%%%%%%%%%%%%%%%%%%%%%%%%%%%%%%%%%%%%%%%%%%%%%%%%%%%%%%%%%%%%%%%%%%



(*Question:

What contexts are `hiding beneath' the 3rd row of Equation \
(5) of

@article{cabello2021contextuality,author={Costantino Budroni \
and Ad\'an Cabello and Otfried G\'uhne and Matthias Kleinmann and \
Jan-\AA{}ke \
Larsson},eprint={arXiv:2102.13036},url={https://https://doi.org/10.\
1103/RevModPhys.94.045007},doi={10.1103/revmodphys.94.045007},year=\
2022,month={dec},publisher={American Physical Society \
({APS})},volume={94},number={4},title={{K}ochen-{S}pecker \
contextuality},journal={Reviews of Modern \
Physics},pages={045007}}?*)


(*Definition of `my' Tensor Product*)(*a,b are nxn and mxm-matrices*)

MyTensorProduct[a_, b_] :=
  Table[a[[Ceiling[s/Length[b]], Ceiling[t/Length[b]]]]*
    b[[s - Floor[(s - 1)/Length[b]]*Length[b],
      t - Floor[(t - 1)/Length[b]]*Length[b]]], {s, 1,
    Length[a]*Length[b]}, {t, 1, Length[a]*Length[b]}];

(*Definition of the Dyadic Product*)

DyadicProductVec[x_] :=
  Table[x[[i]]  Conjugate[x[[j]]], {i, 1, Length[x]}, {j, 1,
    Length[x]}];

(*Commutator*)

Commutator[a_, b_] := a . b - b . a;


(*************************************************************************************)

yJ = MyTensorProduct[PauliMatrix[2], IdentityMatrix[2]];
Jy = MyTensorProduct[IdentityMatrix[2], PauliMatrix[2]];
yx = MyTensorProduct[PauliMatrix[2], PauliMatrix[1]];
xy = MyTensorProduct[PauliMatrix[1], PauliMatrix[2]];


zJ = MyTensorProduct[PauliMatrix[3], IdentityMatrix[2]];
Jz = MyTensorProduct[IdentityMatrix[2], PauliMatrix[3]];
zz = MyTensorProduct[PauliMatrix[3], PauliMatrix[3]];

Jx = MyTensorProduct[IdentityMatrix[2], PauliMatrix[1]];
xJ = MyTensorProduct[PauliMatrix[1], IdentityMatrix[2]];
xx = MyTensorProduct[PauliMatrix[1], PauliMatrix[1]];

zx = MyTensorProduct[PauliMatrix[3], PauliMatrix[1]];
xz = MyTensorProduct[PauliMatrix[1], PauliMatrix[3]];
yy = MyTensorProduct[PauliMatrix[2], PauliMatrix[2]];


(*

Print[`Commutators']


Commutator[zx, xz]
Commutator[zx, yy]
Commutator[xz, yy]


MatrixForm[Commutator[zJ, zx]]

*)

Print[`Eigensystem by matrix pencil calculation']

Eigensystem[a   zJ + b   Jz + c   zz]

Eigensystem[a   Jx + b   xJ + c   xx]

Eigensystem[a   zx + b   xz + c   yy]

Eigensystem[a   zJ + b   Jx + c   zx]

Eigensystem[a   Jz + b   xJ + c   xz]

Eigensystem[a   zz + b   xx + c   yy]



union = Union[Eigenvectors[a    zJ + b    Jz + c    zz],
  Eigenvectors[a    Jx + b    xJ + c    xx],
  Eigenvectors[a    zx + b    xz + c    yy],
  Eigenvectors[a    zJ + b    Jx + c    zx],
  Eigenvectors[a    Jz + b    xJ + c    xz],
  Eigenvectors[a    zz + b    xx + c    yy]]


(*

uniondirected =
  Table[If[union[[i, 1]] < 0, -union[[i]], union[[i]]], {i, 1,
    Length[union]}];

udir = Union[uniondirected]

Length[udir]

*)

ulexsorted=
{
{0, 0, 0, 1},
{0, 0, 1, -1},
{0, 0, 1,  0},
{0, 0, 1, 1},
{0, 1, -1, 0},
{0, 1, 0, -1},
{0, 1, 0, 0},
{0, 1, 0, 1},
{0, 1, 1,  0},
{1, -1, -1, -1},
{1, -1, -1, 1},
{1, -1, 0, 0},
{1, -1,  1, -1},
{1, -1, 1, 1},
{1, 0, -1, 0},
{1, 0, 0, -1},
{1, 0, 0,  0},
{1, 0, 0, 1},
{1, 0, 1, 0},
{1, 1, -1, -1},
{1, 1, -1, 1},
{1,   1, 0, 0},
{1, 1, 1, -1},
{1, 1, 1, 1}
};

LexicographicSort[{
{0, 0, 0, 1},
{0, 0, 1, -1},
{0, 0, 1,  0},
{0, 0, 1, 1},
{0, 1, -1, 0},
{0, 1, 0, -1},
{0, 1, 0, 0},
{0, 1, 0, 1},
{0, 1, 1,  0},
{1, -1, -1, -1},
{1, -1, -1, 1},
{1, -1, 0, 0},
{1, -1,  1, -1},
{1, -1, 1, 1},
{1, 0, -1, 0},
{1, 0, 0, -1},
{1, 0, 0,  0},
{1, 0, 0, 1},
{1, 0, 1, 0},
{1, 1, -1, -1},
{1, 1, -1, 1},
{1,   1, 0, 0},
{1, 1, 1, -1},
{1, 1, 1, 1}
}  ] === {
{0, 0, 0, 1},
{0, 0, 1, -1},
{0, 0, 1,  0},
{0, 0, 1, 1},
{0, 1, -1, 0},
{0, 1, 0, -1},
{0, 1, 0, 0},
{0, 1, 0, 1},
{0, 1, 1,  0},
{1, -1, -1, -1},
{1, -1, -1, 1},
{1, -1, 0, 0},
{1, -1,  1, -1},
{1, -1, 1, 1},
{1, 0, -1, 0},
{1, 0, 0, -1},
{1, 0, 0,  0},
{1, 0, 0, 1},
{1, 0, 1, 0},
{1, 1, -1, -1},
{1, 1, -1, 1},
{1,   1, 0, 0},
{1, 1, 1, -1},
{1, 1, 1, 1}
}

OrthogonalityMatrix =
  Table[If[FullSimplify[ulexsorted[[i]] . ulexsorted[[j]]] == 0, 1, 0]
        , {i,Length[ulexsorted]}, {j, Length[ulexsorted]}];

MatrixForm[OrthogonalityMatrix];

g = AdjacencyGraph[OrthogonalityMatrix];

FindClique[g, {1}, 100]

FindClique[g, {2}, 100]

FindClique[g, {3}, 100]

FindClique[g, {4}, 100]

FindClique[g, {5}, 100]


%%%%%%%%%%%%%%%%%%%%%%%%%%%%%%%%%%%%%%%%%%%%%%%%%%%%%%%%%%%%%%%%%%%%%%%%%%%%%%%%%%%%%%%%%%%%%%%%%%%%%%%%%%

Extended 18-9 set
24 atoms
24 blocks
 0 proper subsets of blocks
 4   5  9 16 18
 4   3  7 16 18
 4   9 14 16 21
 4   9 13 18 20
 4  11 13 20 24
 4   5 11 16 24
 4  10 14 21 23
 4   5 10 18 23
 4   8 14 15 23
 4   8 11 19 20
 4   6 13 15 24
 4   6 10 19 21
 4   6  8 15 19
 4   3  6  8 17
 4   4 12 21 23
 4   4 11 13 22
 4   2 12 20 24
 4   2 10 14 22
 4   2  4 12 22
 4   2  4  7 17
 4   1  7 15 19
 4   1  5  9 17
 4   1  3 12 22
 4   1  3  7 17
0 2-valued evaluations of atoms:

set of 2-valued evaluations of atoms:
nonempty: no



%%%%%%%%%%%%%%%%%%%%%%%%%%%%%%%%%%%%%%%%%%%%%%%%%%%%%%%%%%%%%%%%%%%%%%%%%%%%%%%%%%%%%%%%%%%%%%%%%%%%%%


(* GHZ-Stairs-Mermin configuration *)




(*Definition of `my' Tensor Product*)(*a,b are nxn and mxm-matrices*)

MyTensorProduct[a_, b_] :=
  Table[a[[Ceiling[s/Length[b]], Ceiling[t/Length[b]]]]*
    b[[s - Floor[(s - 1)/Length[b]]*Length[b],
      t - Floor[(t - 1)/Length[b]]*Length[b]]], {s, 1,
    Length[a]*Length[b]}, {t, 1, Length[a]*Length[b]}];

(*Definition of the Dyadic Product*)

DyadicProductVec[x_] :=
  Table[x[[i]]  Conjugate[x[[j]]], {i, 1, Length[x]}, {j, 1,
    Length[x]}];

(*Commutator*)

Commutator[a_, b_] := a . b - b . a;


(* *********************************




a xxx + b yyx + c yxy + d xyy
a y11 + b yyx + c 11x + d 1y1
a y11 + b yxy + c 11y + d 1x1
a xxx + b 11x + c x11 + d 1x1
a xyy + b 11y + c x11 + d 1y1


*)






yJJ = MyTensorProduct[MyTensorProduct[PauliMatrix[2], IdentityMatrix[2]],IdentityMatrix[2]];
xxx = MyTensorProduct[MyTensorProduct[PauliMatrix[1], PauliMatrix[1]],PauliMatrix[1]];
yyx = MyTensorProduct[MyTensorProduct[PauliMatrix[2], PauliMatrix[2]],PauliMatrix[1]];
yxy = MyTensorProduct[MyTensorProduct[PauliMatrix[2], PauliMatrix[1]],PauliMatrix[2]];
xyy = MyTensorProduct[MyTensorProduct[PauliMatrix[1], PauliMatrix[2]],PauliMatrix[2]];
JJx = MyTensorProduct[MyTensorProduct[IdentityMatrix[2], IdentityMatrix[2]],PauliMatrix[1]];
JJy = MyTensorProduct[MyTensorProduct[IdentityMatrix[2], IdentityMatrix[2]], PauliMatrix[2]];
xJJ = MyTensorProduct[MyTensorProduct[PauliMatrix[1], IdentityMatrix[2]], IdentityMatrix[2]];
JyJ = MyTensorProduct[MyTensorProduct[IdentityMatrix[2], PauliMatrix[2]], IdentityMatrix[2]];
JxJ = MyTensorProduct[MyTensorProduct[IdentityMatrix[2],PauliMatrix[1]], IdentityMatrix[2]];

Eigensystem[a xxx + b yyx + c yxy + d xyy  ]

Eigensystem[a yJJ + b yyx + c JJx + d JyJ  ]

Eigensystem[a yJJ + b yxy + c JJy + d JxJ  ]

Eigensystem[a xxx + b JJx + c xJJ + d JxJ  ]

Eigensystem[a xyy + b JJy + c xJJ + d JyJ  ]



union = Union[Eigenvectors[a xxx + b yyx + c yxy + d xyy  ],
      Eigenvectors[a yJJ + b yyx + c JJx + d JyJ  ],
      Eigenvectors[a yJJ + b yxy + c JJy + d JxJ  ],
      Eigenvectors[a xxx + b JJx + c xJJ + d JxJ  ],
      Eigenvectors[a xyy + b JJy + c xJJ + d JyJ  ]
];

uniondirected = Table[ If[ union[[i,1]] < 0, -union[[i]], union[[i]] ]
,{i, 1,Length[union]}];

udir= Union[uniondirected]

Length[udir]

OrthogonalityMatrix =
 Table[ If[ FullSimplify[udir[[i]] . udir[[j]]] == 0, 1, 0]
  , {i, Length[udir]}, {j, Length[udir]}];

MatrixForm[ OrthogonalityMatrix ]

g = AdjacencyGraph[OrthogonalityMatrix]

FindClique[g, {2}, 100]

FindClique[g, {3}, 100]

FindClique[g, {4}, 100]

FindClique[g, {5}, 100]

FindClique[g, {6}, 100]

FindClique[g, {7}, 100]

FindClique[g, {8}, 100]

FindClique[g, {9}, 100]

FindClique[g, {10}, 100]


%%%%%%%%%%%%%%%%%%%%%%%%%%%%%%%%%%%%%%%%%%%%%%%%%%%%%%%%%%%%%%%%%%%%%%%%%%%%%%%%%%%%%%%%%%%%%%%%%%%%%

(* GHZ--Shimony-Mermin configuration *)




(*Definition of `my' Tensor Product*)(*a,b are nxn and mxm-matrices*)

MyTensorProduct[a_, b_] :=
  Table[a[[Ceiling[s/Length[b]], Ceiling[t/Length[b]]]]*
    b[[s - Floor[(s - 1)/Length[b]]*Length[b],
      t - Floor[(t - 1)/Length[b]]*Length[b]]], {s, 1,
    Length[a]*Length[b]}, {t, 1, Length[a]*Length[b]}];

(*Definition of the Dyadic Product*)

DyadicProductVec[x_] :=
  Table[x[[i]]  Conjugate[x[[j]]], {i, 1, Length[x]}, {j, 1,
    Length[x]}];

(*Commutator*)

Commutator[a_, b_] := a . b - b . a;


(* *********************************

 Mermin GHZ


a xxx + b xyy + c yxy + d yyx



*)






yJJ = MyTensorProduct[MyTensorProduct[PauliMatrix[2], IdentityMatrix[2]],IdentityMatrix[2]];
xxx = MyTensorProduct[MyTensorProduct[PauliMatrix[1], PauliMatrix[1]],PauliMatrix[1]];
yyx = MyTensorProduct[MyTensorProduct[PauliMatrix[2], PauliMatrix[2]],PauliMatrix[1]];
yxy = MyTensorProduct[MyTensorProduct[PauliMatrix[2], PauliMatrix[1]],PauliMatrix[2]];
xyy = MyTensorProduct[MyTensorProduct[PauliMatrix[1], PauliMatrix[2]],PauliMatrix[2]];
JJx = MyTensorProduct[MyTensorProduct[IdentityMatrix[2], IdentityMatrix[2]],PauliMatrix[1]];
JJy = MyTensorProduct[MyTensorProduct[IdentityMatrix[2], IdentityMatrix[2]], PauliMatrix[2]];
xJJ = MyTensorProduct[MyTensorProduct[PauliMatrix[1], IdentityMatrix[2]], IdentityMatrix[2]];
JyJ = MyTensorProduct[MyTensorProduct[IdentityMatrix[2], PauliMatrix[2]], IdentityMatrix[2]];
JxJ = MyTensorProduct[MyTensorProduct[IdentityMatrix[2],PauliMatrix[1]], IdentityMatrix[2]];


MatrixForm[FullSimplify[xxx . xyy . yxy . yyx]]


xyy . yxy . yyx == -xxx


Eigensystem[a xxx + b xyy + c yxy + d yyx  ]




union = Union[Eigenvectors[a xxx + b xyy + c yxy + d yyx  ]
];

uniondirected = Table[ If[ union[[i,1]] < 0, -union[[i]], union[[i]] ]
,{i, 1,Length[union]}];

udir= Union[uniondirected]

Length[udir]

OrthogonalityMatrix =
 Table[ If[ FullSimplify[udir[[i]] . udir[[j]]] == 0, 1, 0]
  , {i, Length[udir]}, {j, Length[udir]}];

MatrixForm[ OrthogonalityMatrix ]

g = AdjacencyGraph[OrthogonalityMatrix]

FindClique[g, {8}, 100]



%%%%%%%%%%%%%%%%%%%%%%%%%%%%%%%%%%%%%%%%%%%%%%%%%%%%%%%%%%%%%%%%%%%%%%%%%%%%%%%%%%%%%%%%%%%%%%%%%%%%%%%%%
union={{0,0,0,1},
{0,0,1,0},
{1,1,0,0},
{1,-1,0,0},
{1,0,0,0},
{0,1,0,0},
{0,0,1,-1},
{0,0,1,1},
{0,1,-1,0},
{1,0,0,1},
{-1,1,1,1},
{1,1,1,-1},
{0,1,1,0},
{1,0,0,-1},
{1,-1,1,1},
{1,1,-1,1}
};

uniondirected = Table[ If[ union[[i,1]] < 0, -union[[i]], union[[i]] ]
,{i, 1,Length[union]}];

udir= uniondirected

Length[udir]

OrthogonalityMatrix =
 Table[ If[ FullSimplify[udir[[i]] . udir[[j]]] == 0, 1, 0]
  , {i, Length[udir]}, {j, Length[udir]}];

MatrixForm[ OrthogonalityMatrix ]

g = AdjacencyGraph[OrthogonalityMatrix]

FindClique[g, {2}, 100]

FindClique[g, {3}, 100]

FindClique[g, {4}, 100]

FindClique[g, {5}, 100]

FindClique[g, {6}, 100]

FindClique[g, {7}, 100]

FindClique[g, {8}, 100]

FindClique[g, {9}, 100]

FindClique[g, {10}, 100]

%%%%%%%%%%%%%%%%%%%%%%%%%%%%%%%%%%%%%%%%%%%%%%%%%%%%%%%%%%%%%%%%%%%%%%%%%%%%%%%%%%%%%%%%%%%%%%%%%%%%%%%%%%
%%%%%%%%%%%%%%%%%%%%%%%%%%%%%%%%%%%%%%%%%%%%%%%%%%%%%%%%%%%%%%%%%%%%%%%%%%%%%%%%%%%%%%%%%%%%%%%%%%%%%%%%%%

unitaryma =  (1/(2*Sqrt[2])) FullSimplify[
TensorProduct[   Transpose[{1, 1, -1, 1     }], {0, 1, 0, 1 }  ]+
TensorProduct[   Transpose[{1, -1, 1, 1     }], {-1, 0, 1, 0}  ]+
TensorProduct[   Transpose[{-1, 1, 1,  1    }], {0, -1, 0, 1}  ]+
TensorProduct[   Transpose[{-1, -1, -1, 1   }], {1, 0, 1,  0}  ] ]

MatrixForm[unitaryma]


(* check unitary? *)

unitaryma . Transpose[unitaryma] ==   IdentityMatrix[4]

(* check rotation? *)

unitaryma. Transpose[(1/Sqrt[2]){0, 1, 0, 1 }] == (1/2) {1, 1, -1, 1     }



Eigensystem[ PauliMatrix[3] ]         (* {{-1, 1}, {{0, 1}, {1, 0}}} *)

Eigensystem[ PauliMatrix[1] ]         (* {{-1, 1}, {{-1, 1}, {1, 1}}} *)


zzxx =  Flatten[TensorProduct[   {0, 1}, {-1, 1} ]] +  Flatten[TensorProduct[   {1, 0}, Transpose [{1, 1}] ]] -
           (Flatten[TensorProduct[   {0, 1}, {1, 1} ]] +  Flatten[TensorProduct[   {1, 0}, Transpose [{-1, 1}] ]])


zx = MyTensorProduct[PauliMatrix[3], PauliMatrix[1]];

unitaryma . zx . Transpose[unitaryma]

Eigensystem[unitaryma . zx . Transpose[unitaryma]]


%%%%%%%%%%%%%%%%%%%%%%%%%%%%%%%%%%%%%%%%%%%%%%%%%%%%%%%%%%%%%%%%%%%%%%%%

(*Definition of `my' Tensor Product*)(*a,b are nxn and mxm-matrices*)

MyTensorProduct[a_, b_] :=
  Table[a[[Ceiling[s/Length[b]], Ceiling[t/Length[b]]]]*
    b[[s - Floor[(s - 1)/Length[b]]*Length[b],
      t - Floor[(t - 1)/Length[b]]*Length[b]]], {s, 1,
    Length[a]*Length[b]}, {t, 1, Length[a]*Length[b]}];

(*Definition of the Dyadic Product*)

DyadicProductVec[x_] :=
  Table[x[[i]]  Conjugate[x[[j]]], {i, 1, Length[x]}, {j, 1,
    Length[x]}];

(*Commutator*)

Commutator[a_, b_] := a . b - b . a;


(*************************************************************************************)

yJ = MyTensorProduct[PauliMatrix[2], IdentityMatrix[2]];
Jy = MyTensorProduct[IdentityMatrix[2], PauliMatrix[2]];
yx = MyTensorProduct[PauliMatrix[2], PauliMatrix[1]];
xy = MyTensorProduct[PauliMatrix[1], PauliMatrix[2]];


zJ = MyTensorProduct[PauliMatrix[3], IdentityMatrix[2]];
Jz = MyTensorProduct[IdentityMatrix[2], PauliMatrix[3]];
zz = MyTensorProduct[PauliMatrix[3], PauliMatrix[3]];

Jx = MyTensorProduct[IdentityMatrix[2], PauliMatrix[1]];
xJ = MyTensorProduct[PauliMatrix[1], IdentityMatrix[2]];
xx = MyTensorProduct[PauliMatrix[1], PauliMatrix[1]];

zx = MyTensorProduct[PauliMatrix[3], PauliMatrix[1]];
xz = MyTensorProduct[PauliMatrix[1], PauliMatrix[3]];
yy = MyTensorProduct[PauliMatrix[2], PauliMatrix[2]];




Print[`Eigensystem by matrix pencil calculation']



avzxxzyy = Eigenvectors[a   zx + b   xz + c   yy]

sigma[t_,p_] := {{ Cos[t], E^(-I * p) Sin[t] },{ E^(I * p) Sin[t], -Cos[ t] }};


aevg =FullSimplify[ Eigenvectors[ MyTensorProduct[sigma[t1,p1], sigma[t2,p2]] ]]


##########################################################################

upo = Union[Permutations[{1, 1, 0, 0, 0, 0, 0, 0}],
 Permutations[{1, -1, 0, 0, 0, 0, 0, 0}],
 Permutations[{1, 1, 1, 1, 0, 0, 0, 0}],
 Permutations[{1, 1, -1, -1, 0, 0, 0, 0}],
 Permutations[{1, 1, 1, -1, 0, 0, 0, 0}],
 Permutations[{1, 1, 1, 1, 1, 1, 0, 0}],
 Permutations[{1, 1, 1, -1, -1, -1, 0, 0}],
 Permutations[{1, 1, 1, 1, -1, -1, 0, 0}],
 Permutations[{1, 1, 1, 1, 1, -1, 0, 0}],
 Permutations[{1, 1, 1, 1, 1, 1, 1, 1}],
 Permutations[{1, 1, 1, 1, -1, -1, -1, -1}],
 Permutations[{1, 1, 1, 1, 1, -1, -1, -1}],
 Permutations[{1, 1, 1, 1, 1, 1, -1, -1}],
 Permutations[{1, 1, 1, 1, 1, 1, 1, -1}]
 ]

Length[upo]



uniondirected = Table[
If[ upo[[i,1]] < 0, -upo[[i]], upo[[i]] ]
,{i, 1,Length[upo]}];

uniondirected1 = Table[
If[ (upo[[i,2]] < 0) && (upo[[i,1]] == 0), -uniondirected[[i]], uniondirected[[i]] ]
,{i, 1,Length[upo]}];

uniondirected =  uniondirected1;

uniondirected1 = Table[
If[ (upo[[i,3]] < 0) && (upo[[i,1]] == 0) && (upo[[i,2]] == 0), -uniondirected[[i]], uniondirected[[i]] ]
,{i, 1,Length[upo]}];

uniondirected =  uniondirected1;

uniondirected1 = Table[
If[ (upo[[i,4]] < 0) && (upo[[i,1]] == 0) && (upo[[i,2]] == 0)&& (upo[[i,3]] == 0), -uniondirected[[i]], uniondirected[[i]] ]
,{i, 1,Length[upo]}];

uniondirected =  uniondirected1;

uniondirected1 = Table[
If[ (upo[[i,5]] < 0) && (upo[[i,1]] == 0) && (upo[[i,2]] == 0)&& (upo[[i,3]] == 0) && (upo[[i,4]] == 0), -uniondirected[[i]], uniondirected[[i]] ]
,{i, 1,Length[upo]}];


uniondirected =  uniondirected1;

uniondirected1 = Table[
If[ (upo[[i,6]] < 0) && (upo[[i,1]] == 0) && (upo[[i,2]] == 0)&& (upo[[i,3]] == 0) && (upo[[i,4]] == 0) && (upo[[i,5]]== 0),-uniondirected[[i]], uniondirected[[i]] ]
,{i, 1,Length[upo]}];


uniondirected =  uniondirected1;

uniondirected1 = Table[
If[ (upo[[i,7]] < 0) && (upo[[i,1]] == 0) && (upo[[i,2]] == 0)&& (upo[[i,3]] == 0) && (upo[[i,4]] == 0) && (upo[[i,5]]== 0) && (upo[[i,6]]== 0),-uniondirected[[i]], uniondirected[[i]] ]
,{i, 1,Length[upo]}];


up= Union[uniondirected1];

Length[up]

OrthogonalityMatrix =
 Table[ If[ FullSimplify[up[[i]] . up[[j]]] == 0, 1, 0]
  , {i, Length[up]}, {j, Length[up]}];


g = AdjacencyGraph[OrthogonalityMatrix]

(* FindClique[g, {2}, 100]

FindClique[g, {3}, 100]

FindClique[g, {4}, 100]

FindClique[g, {5}, 100]

FindClique[g, {6}, 100]

FindClique[g, {7}, 100]
*)

FindClique[g, {8}, 100]



%%%%%%%%%%%%%%%%%%%%%%%%%%%%%%%%%%%%%%%%%%%%%%%%%%%%%%%%%%%%%%%%%%%%%%%%%%%%%%%%%%%%%%%%%%%


(*Definition of `my' Tensor Product*)(*a,b are nxn and mxm-matrices*)

MyTensorProduct[a_, b_] :=
  Table[a[[Ceiling[s/Length[b]], Ceiling[t/Length[b]]]]*
    b[[s - Floor[(s - 1)/Length[b]]*Length[b],
      t - Floor[(t - 1)/Length[b]]*Length[b]]], {s, 1,
    Length[a]*Length[b]}, {t, 1, Length[a]*Length[b]}];

(*Definition of the Dyadic Product*)

DyadicProductVec[x_] :=
  Table[x[[i]]  Conjugate[x[[j]]], {i, 1, Length[x]}, {j, 1,
    Length[x]}];

(*Commutator*)

Commutator[a_, b_] := a . b - b . a;


(* *********************************

 Mermin GHZ


a xxx + b xyy + c yxy + d yyx



*)






yJJ = MyTensorProduct[MyTensorProduct[PauliMatrix[2], IdentityMatrix[2]],IdentityMatrix[2]];
xxx = MyTensorProduct[MyTensorProduct[PauliMatrix[1], PauliMatrix[1]],PauliMatrix[1]];
yyx = MyTensorProduct[MyTensorProduct[PauliMatrix[2], PauliMatrix[2]],PauliMatrix[1]];
yxy = MyTensorProduct[MyTensorProduct[PauliMatrix[2], PauliMatrix[1]],PauliMatrix[2]];
xyy = MyTensorProduct[MyTensorProduct[PauliMatrix[1], PauliMatrix[2]],PauliMatrix[2]];
JJx = MyTensorProduct[MyTensorProduct[IdentityMatrix[2], IdentityMatrix[2]],PauliMatrix[1]];
JJy = MyTensorProduct[MyTensorProduct[IdentityMatrix[2], IdentityMatrix[2]], PauliMatrix[2]];
xJJ = MyTensorProduct[MyTensorProduct[PauliMatrix[1], IdentityMatrix[2]], IdentityMatrix[2]];
JyJ = MyTensorProduct[MyTensorProduct[IdentityMatrix[2], PauliMatrix[2]], IdentityMatrix[2]];
JxJ = MyTensorProduct[MyTensorProduct[IdentityMatrix[2],PauliMatrix[1]], IdentityMatrix[2]];


MatrixForm[FullSimplify[xxx . xyy . yxy . yyx]]


xyy . yxy . yyx == -xxx


Commutator[xxx,xyy]

Commutator[xxx,yxy]

Commutator[xxx,yyx]

Commutator[yyx,yxy]

Commutator[yyx,xyy]

Commutator[yxy,xyy]

%%%%%%%%%%%%%%%%%%%%%%%%%%%%%%%%%%%%%%%%%%%%%%%%%%%%%%%%%%%%%%%%%%%%%%%%%%%%%%%%%%%%%%%%%%



(*Definition of `my' Tensor Product*)(*a,b are nxn and mxm-matrices*)

MyTensorProduct[a_, b_] :=
  Table[a[[Ceiling[s/Length[b]], Ceiling[t/Length[b]]]]*
    b[[s - Floor[(s - 1)/Length[b]]*Length[b],
      t - Floor[(t - 1)/Length[b]]*Length[b]]], {s, 1,
    Length[a]*Length[b]}, {t, 1, Length[a]*Length[b]}];

(*Definition of the Dyadic Product*)

DyadicProductVec[x_] :=
  Table[x[[i]]  Conjugate[x[[j]]], {i, 1, Length[x]}, {j, 1,
    Length[x]}];

(*Commutator*)

Commutator[a_, b_] := a . b - b . a;


(*************************************************************************************)

yJ = MyTensorProduct[PauliMatrix[2], IdentityMatrix[2]];
Jy = MyTensorProduct[IdentityMatrix[2], PauliMatrix[2]];
yx = MyTensorProduct[PauliMatrix[2], PauliMatrix[1]];
xy = MyTensorProduct[PauliMatrix[1], PauliMatrix[2]];


zJ = MyTensorProduct[PauliMatrix[3], IdentityMatrix[2]];
Jz = MyTensorProduct[IdentityMatrix[2], PauliMatrix[3]];
zz = MyTensorProduct[PauliMatrix[3], PauliMatrix[3]];

Jx = MyTensorProduct[IdentityMatrix[2], PauliMatrix[1]];
xJ = MyTensorProduct[PauliMatrix[1], IdentityMatrix[2]];
xx = MyTensorProduct[PauliMatrix[1], PauliMatrix[1]];

zx = MyTensorProduct[PauliMatrix[3], PauliMatrix[1]];
xz = MyTensorProduct[PauliMatrix[1], PauliMatrix[3]];
yy = MyTensorProduct[PauliMatrix[2], PauliMatrix[2]];



Print[`Eigensystem by matrix pencil calculation']

Eigensystem[a   zx + b   xz + c   yy]


Eigensystem[a   zz + b   xx + c   yy]



union = Union[
  Eigenvectors[a    zx + b    xz + c    yy],
  Eigenvectors[a    zz + b    xx + c    yy]]




uniondirected =
  Table[If[union[[i, 1]] < 0, -union[[i]], union[[i]] ], {i, 1,
    Length[union]}];

udir = Union[uniondirected]

Length[udir]


OrthogonalityMatrix =
  Table[If[udir[[i]] . udir[[j]] == 0, 1, 0]
        , {i,Length[udir]}, {j, Length[udir]}];

MatrixForm[OrthogonalityMatrix];

g = AdjacencyGraph[OrthogonalityMatrix];

FindClique[g, {1}, 100]

FindClique[g, {2}, 100]

FindClique[g, {3}, 100]

FindClique[g, {4}, 100]

FindClique[g, {5}, 100]

%%%%%%%%%%%%%%%%%%%%%%%%%%%%%%%%%%%%%%%%%%%%%%%%%%%%%%%%%%%%%%%%%%%%%%%%%%%%%%%%%%%%
%%%%%%%%%%%%%%%%%%%%%%%%%%%%%%%%%%%%%%%%%%%%%%%%%%%%%%%%%%%%%%%%%%%%%%%%%%%%%%%%%%%%
%%%%%%%%%%%%%%%%%%%%%%%%%%%%%%%%%%%%%%%%%%%%%%%%%%%%%%%%%%%%%%%%%%%%%%%%%%%%%%%%%%%%
%%%%%%%%%%%%%%%%%%%%%%%%%%%%%%%%%%%%%%%%%%%%%%%%%%%%%%%%%%%%%%%%%%%%%%%%%%%%%%%%%%%%
%%%%%%%%%%%%%%%%%%%%%%%%%%%%%%%%%%%%%%%%%%%%%%%%%%%%%%%%%%%%%%%%%%%%%%%%%%%%%%%%%%%%


(*Definition of `my' Tensor Product*)(*a,b are nxn and mxm-matrices*)

MyTensorProduct[a_, b_] :=
  Table[a[[Ceiling[s/Length[b]], Ceiling[t/Length[b]]]]*
    b[[s - Floor[(s - 1)/Length[b]]*Length[b],
      t - Floor[(t - 1)/Length[b]]*Length[b]]], {s, 1,
    Length[a]*Length[b]}, {t, 1, Length[a]*Length[b]}];

(*Definition of the Dyadic Product*)

DyadicProductVec[x_] :=
  Table[x[[i]]  Conjugate[x[[j]]], {i, 1, Length[x]}, {j, 1,
    Length[x]}];

(*Commutator*)

Commutator[a_, b_] := a . b - b . a;


(*************************************************************************************)

yJ = MyTensorProduct[PauliMatrix[2], IdentityMatrix[2]];
Jy = MyTensorProduct[IdentityMatrix[2], PauliMatrix[2]];
yx = MyTensorProduct[PauliMatrix[2], PauliMatrix[1]];
xy = MyTensorProduct[PauliMatrix[1], PauliMatrix[2]];


zJ = MyTensorProduct[PauliMatrix[3], IdentityMatrix[2]];
Jz = MyTensorProduct[IdentityMatrix[2], PauliMatrix[3]];
zz = MyTensorProduct[PauliMatrix[3], PauliMatrix[3]];

Jx = MyTensorProduct[IdentityMatrix[2], PauliMatrix[1]];
xJ = MyTensorProduct[PauliMatrix[1], IdentityMatrix[2]];
xx = MyTensorProduct[PauliMatrix[1], PauliMatrix[1]];

zx = MyTensorProduct[PauliMatrix[3], PauliMatrix[1]];
xz = MyTensorProduct[PauliMatrix[1], PauliMatrix[3]];
yy = MyTensorProduct[PauliMatrix[2], PauliMatrix[2]];




bbpsip ={0,1,1,0};
bbpsim ={0,1,-1,0};
bbphip ={1,0,0,1};
bbphim ={1,0,0,-1};

ta=
{
{
zx.xz .    bbpsip,
xx.zz .    bbpsip,
yy.yy .    bbpsip
},
{
zx.xz .    bbpsim,
xx.zz .    bbpsim ,
yy.yy .    bbpsim
},
{
zx.xz .    bbphip ,
xx.zz .    bbphip   ,
yy.yy .    bbphip
},
{
zx.xz .    bbphim    ,
xx.zz .    bbphim     ,
yy.yy .    bbphim
}
}

MatrixForm[ta]

Eigensystem[a zx.xz +b xx.zz +c yy.yy]

##################################################################################################


(*Definition of `my' Tensor Product*)(*a,b are nxn and mxm-matrices*)
MyTensorProduct[a_, b_] :=
  Table[a[[Ceiling[s/Length[b]], Ceiling[t/Length[b]]]]*
    b[[s - Floor[(s - 1)/Length[b]]*Length[b],
      t - Floor[(t - 1)/Length[b]]*Length[b]]], {s, 1,
    Length[a]*Length[b]}, {t, 1, Length[a]*Length[b]}];

(*Definition of the Dyadic Product*)

DyadicProductVec[x_] :=
  Table[x[[i]]   Conjugate[x[[j]]], {i, 1, Length[x]}, {j, 1,
    Length[x]}];

(*Commutator*)

Commutator[a_, b_] := a . b - b . a;


(*************************************************************************************)

yJ = MyTensorProduct[PauliMatrix[2], IdentityMatrix[2]];
Jy = MyTensorProduct[IdentityMatrix[2], PauliMatrix[2]];
yx = MyTensorProduct[PauliMatrix[2], PauliMatrix[1]];
xy = MyTensorProduct[PauliMatrix[1], PauliMatrix[2]];


zJ = MyTensorProduct[PauliMatrix[3], IdentityMatrix[2]];
Jz = MyTensorProduct[IdentityMatrix[2], PauliMatrix[3]];
zz = MyTensorProduct[PauliMatrix[3], PauliMatrix[3]];

Jx = MyTensorProduct[IdentityMatrix[2], PauliMatrix[1]];
xJ = MyTensorProduct[PauliMatrix[1], IdentityMatrix[2]];
xx = MyTensorProduct[PauliMatrix[1], PauliMatrix[1]];

zx = MyTensorProduct[PauliMatrix[3], PauliMatrix[1]];
xz = MyTensorProduct[PauliMatrix[1], PauliMatrix[3]];
yy = MyTensorProduct[PauliMatrix[2], PauliMatrix[2]];



psim = {0, 1, -1, 0};
psip = {0, 1, +1, 0};
phim = {1, 0, 0, -1};
phip = {1, 0, 0, +1};


MatrixForm[ {
{xx.zz.psim ,  xx.psim ,  zz.psim ,  yy.psim },
{xx.zz.psip ,  xx.psip ,  zz.psip ,  yy.psip },
{xx.zz.phim ,  xx.phim ,  zz.phim ,  yy.phim },
{xx.zz.phip ,  xx.phip ,  zz.phip ,  yy.phip }}]













####################################################################

----------------------------------------------------------------------
Report of the Referee -- LS18738A/Svozil
----------------------------------------------------------------------

The technical contents of the manuscript seem, at least to me, beyond
reproach. The only issue lies in the presentation.

For starters, the manuscript should be reorganized into a more
standard article format; that is, separate sections for Introduction,
Preliminaries, Results, and Conclusions/Discussion. Moreover, I
believe it would benefit from laying out possible applications of the
main result (in the Introduction section), preferably in a manner that
does not require as much background knowledge of the subject matter on
the part of the reader as is presently the case. This "selling point"
of the manuscript should be reiterated in the abstract as well, as the
latter is currently a bit too bare-bones.

Overall, the Introduction section needs to act as a less technical
overview; this should make the manuscript more accessible to a wider
audience. Similar efforts need be directed towards the final part of
the manuscript: The ideas introduced here must be clearly stated. Some
parts can perhaps be trimmed entirely. As the author says, "I shall
motivate and discuss these issues further in a later publication." It
might be worth considering exactly which parts should be part of the
present manuscript, and which should go into a later publication.

All in all, the scientific content seems of good quality. Once the
manuscript has had its presentation reworked, I would be happy to
review it.



----------------------------------------------------------------------
Response to Report of the Referee -- LS18738A/Svozil
----------------------------------------------------------------------

Dear Editor,

I would kindly like to express my sincere gratitude to the Referee for the careful consideration and valuable suggestions made for improving the manuscript. I have endeavored to incorporate all recommendations appropriately.

In more detail:

* The article is now structured in a format suitable for PRA, as suggested by the Referee. Previously, the "prl" letter option was active, which had eliminated all existing section headings.

* I have completely rewritten the introductory section. As recommended by the Referee, the new introduction should allow readers to follow the subject and comprehend the paper's main results without requiring extensive background knowledge.

* Similarly, I have almost entirely rewritten the final section of the manuscript.

* As advised by the Referee, I have trimmed (eliminated) the respective text referred to by the Referee. This should improve the smoothness of the presentation and avoid unnecessary distractions and hints about future content (which I will pursue separately).

Once again, I would like to thank the Referee for the attention and efforts dedicated to improving the manuscript.

Sincerely,
Karl Svozil

