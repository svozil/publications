%\documentclass[pra,showpacs,showkeys,amsfonts,amsmath,twocolumn]{revtex4}
\documentclass[amsmath,blue,handout,table]{beamer}
%\documentclass[pra,showpacs,showkeys,amsfonts]{revtex4}
\usepackage[T1]{fontenc}
\usepackage{beamerthemeshadow}
%\usepackage[dark]{beamerthemesidebar}
%\usepackage[headheight=24pt,footheight=12pt]{beamerthemesplit}
%\usepackage{beamerthemesplit}
%\usepackage[bar]{beamerthemetree}
\usepackage{graphicx}
\usepackage{pgf}
%\usepackage[usenames]{color}
%\newcommand{\Red}{\color{Red}}  %(VERY-Approx.PANTONE-RED)
%\newcommand{\Green}{\color{Green}}  %(VERY-Approx.PANTONE-GREEN)

%\RequirePackage[german]{babel}
%\selectlanguage{german}
%\RequirePackage[isolatin]{inputenc}

\pgfdeclareimage[height=0.5cm]{logo}{tu-logo}
\logo{\pgfuseimage{logo}}
\beamertemplatetriangleitem
\begin{document}

\title{\bf \textcolor{yellow}{Randomness \& undecidability in physics}}
%\subtitle{Naturwissenschaftlich-Humanisticher Tag am BG 19\\Weltbild und Wissenschaft\\http://tph.tuwien.ac.at/\~{}svozil/publ/2005-BG18-pres.pdf}
\subtitle{\textcolor{yellow!60}{http://tph.tuwien.ac.at/$\sim$svozil/publ/2005-perimeter-pres.pdf}}
\author{Karl Svozil}
\institute{Institut f\"ur Theoretische Physik, University of Technology Vienna, \\
Wiedner Hauptstra\ss e 8-10/136, A-1040 Vienna, Austria\\
svozil@tuwien.ac.at
%{\tiny Disclaimer: Die hier vertretenen Meinungen des Autors verstehen sich als Diskussionsbeitr�ge und decken sich nicht notwendigerweise mit den Positionen der Technischen Universit�t Wien oder deren Vertreter.}
}
\date{Dec. 9, 2005}
\maketitle

\frame{\tableofcontents}


%%%%%%%%%%%%%%%%%%%%%%%%%%%%%%%%%%%%%%%%%%%%%%%%%%%%%%%%%%%%%%%%%%%%%%%%%%%%%%%%%%%%%%%%%%%%%%%%%%%%%%%%%
%%%%%%%%%%%%%%%%%%%%%%%%%%%%%%%%%%%%%%%%%%%%%%%%%%%%%%%%%%%%%%%%%%%%%%%%%%%%%%%%%%%%%%%%%%%%%%%%%%%%%%%%%
%%%%%%%%%%%%%%%%%%%%%%%%%%%%%%%%%%%%%%%%%%%%%%%%%%%%%%%%%%%%%%%%%%%%%%%%%%%%%%%%%%%%%%%%%%%%%%%%%%%%%%%%%
%%%%%%%%%%%%%%%%%%%%%%%%%%%%%%%%%%%%%%%%%%%%%%%%%%%%%%%%%%%%%%%%%%%%%%%%%%%%%%%%%%%%%%%%%%%%%%%%%%%%%%%%%



\section{``Quantum'' randomness}

\subsection{Quantum Coin tosses}
\frame{
\frametitle{Quantum Coin tosses}

\begin{itemize}
\item<+->
KS, ``The quantum coin toss'', Phys. Lett. A 143, 433-437 (1990):
``perfect context mismatchbetween preparation and measurement,'' using
jonjugate bases, such as ...
light from a linear polarized source is split into two beams of equal
 intensity, each having a polarization direction of $\pm 45^\circ $
 with respect to the original direction of polarization. Incoming light
 quanta are then detected. Subsequent countings in detectors 0 and 1
 correspond to subsequent bits of a sequence $\psi (n)$
(after Von Neumann-type normalization).

\item<+->
Thomas Jennewein and Ulrich Achleitner and Gregor Weihs
and Harald Weinfurter and Anton Zeilinger,
{``A Fast and Compact Quantum Random Number Generator''},
{Review of Scientific Instruments},
{\bf 71} (44), {1675-1680}, 2000  [quant-ph/9912118]


\item<+-> Quantis - Quantum Random Number Generators http://www.idquantique.com/products/quantis.htm [N. Gisin]

\end{itemize}
}

%%%%%%%%%%%%%%%%%%%%%%%%%%%%%%%%%%%%%%%%%%%%%%%%%%%%%%%%%%%%%%%%%%%%%%%%%%%%%%%%%%%%%%%%%%%%%%%%%%%%%%%%%

\subsection{Complementarity}
\frame{
\frametitle{Classical analogy I: generalized urn models}

R.~Wright, {\em The state of the pentagon. {A} nonclassical example,} in
  \emph{Mathematical Foundations of Quantum Theory}, A.~R. Marlow, ed., pp.
  255--274 (Academic Press, New York, 1978).

R.~Wright, {\em Generalized urn models,} Foundations of Physics
  \textbf{20}, 881--903 (1990).

\begin{center}
\rowcolors{2}{gray!20}{lightgray!20}
\arrayrulecolor{blue}
\begin{tabular}{ccc}
%\hline\hline
\rowcolor{gray!55}
ball type&\textcolor{red}{red}&\textcolor{green}{green}\\
%\hline
1&\textcolor{red}{0}&\textcolor{green}{0}\\
2&\textcolor{red}{0}&\textcolor{green}{1}\\
3&\textcolor{red}{1}&\textcolor{green}{0}\\
4&\textcolor{red}{1}&\textcolor{green}{1}\\
%\hline\hline
\end{tabular}
\end{center}

Schema of imprinting of the (chocolate ;-)  balls.

}


\frame[shrink=2]{
\frametitle{Classical analogy II: finite automaton models}

Edward F. Moore,
{\em Gedanken-Experiments on Sequential Machines},
in \emph{Automata Studies},
{C. E. Shannon and J. McCarthy}, ed.,
(Princeton  University Press, {Princeton},  1956).

...

KS: automaton partition logic

\begin{center}
%TexCad Options
%\grade{\off}
%\emlines{\off}
%\beziermacro{\off}
%\reduce{\on}
%\snapping{\off}
%\quality{0.20}
%\graddiff{0.01}
%\snapasp{1}
%\zoom{1.00}
\unitlength 1mm
\linethickness{0.4pt}
\begin{picture}(68.00,60.00)
(0,10)
\put(00.00,60.00){Mealy automaton}
\put(20.00,20.00){\circle*{2.00}}
\put(60.00,20.00){\circle*{2.00}}
\put(40.00,50.00){\circle*{2.00}}
\thicklines\put(22,19){\vector(1,0){36}}
\put(58.00,21.00){\vector(-1,0){36.00}}
\put(59.00,23.00){\vector(-2,3){16.00}}
\put(41.00,47.00){\vector(2,-3){16.00}}
\put(23.00,23.00){\vector(2,3){16.00}}
\put(37.00,47.00){\vector(-2,-3){16.00}}
\put(22.00,15.00){$1$}
\put(55.00,15.00){$2$}
\put(45.00,49.00){$3$}
\put(40.00,15.00){\textcolor{red}{2},0}
\put(35.00,22.00){\textcolor{green}{1},0}
\put(50.00,38.00){\textcolor{blue}{3},0}
\put(42.00,30.00){\textcolor{red}{2},0}
\put(18.00,30.00){\textcolor{green}{1},0}
\put(35.00,38.00){\textcolor{blue}{3},0}
\put(8.00,10.00){\textcolor{green}{1},1}
\put(68.00,10.00){\textcolor{red}{2},1}
\put(38.00,60.00){\textcolor{blue}{3},1}
\put(40.00,53.50){\circle{7.33}}
\put(36.33,54.33){\vector(0,-1){1.33}}
\put(63.00,17.67){\circle{7.33}}
\put(17.33,17.33){\circle{7.33}}
\put(66.67,16.33){\vector(0,1){1.00}}
\put(13.67,18.33){\vector(0,-1){1.00}}
\end{picture}
\unitlength=1mm
\begin{picture}(140,60)(0,00)
\put(3.00,50.00){Hasse diagram}

\multiput(10,30)(20,0){3}{\circle*{1.5}}
\multiput(90,30)(20,0){3}{\circle*{1.5}}
\put(70,10){\circle*{1.5}}
\put(70,50){\circle*{1.5}}

\put(10,30){\line(3,1){60}}
\put(30,30){\line(2,1){40}}
\put(50,30){\line(1,1){20}}

\put(10,30){\line(3,-1){60}}
\put(30,30){\line(2,-1){40}}
\put(50,30){\line(1,-1){20}}

\put(90,30){\line(-1,1){20}}
\put(110,30){\line(-2,1){40}}
\put(130,30){\line(-3,1){60}}

\put(90,30){\line(-1,-1){20}}
\put(110,30){\line(-2,-1){40}}
\put(130,30){\line(-3,-1){60}}

\small

\put(3,29){\textcolor{green}{\{1\}}}
\put(23,29){\textcolor{red}{\{2\}}}
\put(43,29){\textcolor{blue}{\{3\}}}

\put(92,29){\textcolor{blue}{\{1,2\}}}
\put(112,29){\textcolor{red}{\{1,3\}}}
\put(132,29){\textcolor{green}{\{2,3\}}}

\put(69,5){$\emptyset$}
\put(64,52){\{1,2,3\}}

\end{picture}
\end{center}



}


%%%%%%%%%%%%%%%%%%%%%%%%%%%%%%%%%%%%%%%%%%%%%%%%%%%%%%%%%%%%%%%%%%%%%%%%%%%%%%%%%%%%%%%%%%%%%%%%%%%%%%%%%

\subsection{Value indefiniteness \& contextuality}

\frame{
\frametitle{Boole-Bell type theorems}

\begin{itemize}
\item<+->
G. Boole, around 1860: ``Elements of possible experience''

\item<+->
Froissart, Pitowsky: Inequalities are interpretable in terms of faces of a convex correlation polytope,
formed by edges corresponding to all classical truth assignments.


\item<+->
QBounds by a minmax-theorem for Hermitean operators

\end{itemize}
}

\frame{
\frametitle{Kochen-Specker type theorems}

\begin{itemize}
\item<+->
Specker: scholastic infuturabilities (counterfactuals) -- are they possible in a quantum world?


\item<+->
Kochen\&Specker, Zierler\&Schlesinger, Alda, Kamber:
\textcolor{blue}{No  universal truth assignment exists on interconnected quantum contexts.}
(A context is the ``maximal'' set of comeasurable quantum propositions.)
Nonexistence of two-valued (dispersionless) states on quantum logics of Hilbert spaces of dim $> 2$.

\end{itemize}
}

%%%%%%%%%%%%%%%%%%%%%%%%%%%%%%%%%%%%%%%%%%%%%%%%%%%%%%%%%%%%%%%%%%%%%%%%%%%%%%%%%%%%%%%%%%%%%%%%%%%%%%%%%
%%%%%%%%%%%%%%%%%%%%%%%%%%%%%%%%%%%%%%%%%%%%%%%%%%%%%%%%%%%%%%%%%%%%%%%%%%%%%%%%%%%%%%%%%%%%%%%%%%%%%%%%%
%%%%%%%%%%%%%%%%%%%%%%%%%%%%%%%%%%%%%%%%%%%%%%%%%%%%%%%%%%%%%%%%%%%%%%%%%%%%%%%%%%%%%%%%%%%%%%%%%%%%%%%%%
%%%%%%%%%%%%%%%%%%%%%%%%%%%%%%%%%%%%%%%%%%%%%%%%%%%%%%%%%%%%%%%%%%%%%%%%%%%%%%%%%%%%%%%%%%%%%%%%%%%%%%%%%

\section{Formal incompleteness \& independence}


\subsection{Extrinsic versus intrinsic, internal observers}
\frame{
\frametitle{Extrinsic versus intrinsic, internal observers}

\begin{itemize}
\item<+->
(H. Everett), Tommaso Toffoli, KS, Otto R\"ossler: operational observables ``within'' a system.

\item<+->
G\"odel  in a reply to a letter by A. W. Burks, reprinted in [J. von Neumann, Theory of Self-Reproducing Automata, University of Illinois
Press, Urbana, 1966, a. W. Burks, editor.]: \textcolor{blue}{{\em ``I think the
theorem of mine which von Neumann refers to is ... the fact that a complete epistemological description of a language A cannot be given in the same language A,
because the concept of truth of sentences of A cannot be defined in A. It is this
theorem which is the true reason for the existence of undecidable propositions in the
formal systems containing arithmetic.''} }

\end{itemize}
}



\subsection{Reduction to G\"odel-Turing type incompleteness \& independence}
\frame{
\frametitle{Reduction to G\"odel-Turing type incompleteness \& independence}

\begin{itemize}
\item<+->
K. Popper,
{Indeterminism in Quantum Physics and in Classical Physics I \& II},
{The British Journal for the Philosophy of Science},
{bf 1} (2 \& 3),  {117-133} \& {173-195}, 1950

\item<+->
Embedding a Turing machine into a physical system: e.g.,
Christopher D. Moore, {``Unpredictability and undecidability in dynamical systems''},
{Physical Review Letters} {\bf 64}, {2354-2357}, 1990

\item<+->
G\"odel himself does not believe in relevance for physics (recollection of J.A.Wheeler)

\item<+->
Recursive unsolvability of the Rule Inference Problem.

\item<+-> J. Casti, two conferences on the topic (Santa Fe 1994 \& Abisko 1996): finally decides that
\textcolor{red}{``undecidability in physics is a Red Herring.''}

\end{itemize}
}

%%%%%%%%%%%%%%%%%%%%%%%%%%%%%%%%%%%%%%%%%%%%%%%%%%%%%%%%%%%%%%%%%%%%%%%%%%%%%%%%%%%%%%%%%%%%%%%%%%%%%%%%%


\section{``Classical'' indeterminsm \& randomness}

\subsection{Chaitin-Kolmogorov type randomness}
\frame{
\frametitle{Chaitin-Kolmogorov type randomness}

\begin{itemize}
\item<+->
Algorithmic incompressibility: ``in the limit''
the minimal algorithmic description length
(``Algorithmic Information'')
of a random string increases linearly with the string length.

\item<+->
Equivalent to all statistical tests of randomness (Martin-L\"of--Solovay randomness).

\item<+->
Algorithmic Information is generally undecidable (``reduction to the halting problem''; see also
Busy Beaver function)


\item<+->
Example: take \fcolorbox{yellow}{yellow}{7687675545646} positions in the decimal expansion of $\pi$, starting from the
\fcolorbox{yellow}{yellow}{143464668788978979797767655454475676879879787900}th position. Fapp random (eg, Borel normal), but not random against the hypothesis mentioned.

\end{itemize}

}


%%%%%%%%%%%%%%%%%%%%%%%%%%%%%%%%%%%%%%%%%%%%%%%%%%%%%%%%%%%%%%%%%%%%%%%%%%%%%%%%%%%%%%%%%%%%%%%%%%%%%%%%%

\subsection{Classical continua}
\frame{
\frametitle{Classical continua}

\begin{itemize}
\item<+->
Classical physical indeterminism (eg ``chaos'') $=$ uncomputability

\item<+->
``With probability'' one, elements of the continuum are uncomputable.


\item<+->
How does one ``grab'' such elements? Necessity of the Axiom of Choice?


\item<+->
How come we cannot exploit these trans-Turing capabilities of classical continua?

\end{itemize}
}


%%%%%%%%%%%%%%%%%%%%%%%%%%%%%%%%%%%%%%%%%%%%%%%%%%%%%%%%%%%%%%%%%%%%%%%%%%%%%%%%%%%%%%%%%%%%%%%%%%%%%%%%%

\frame{
\frametitle{My vision}

\begin{itemize}
\item<+->
Total discretization of QPhysics (Einstein's vision),
but not in configuration space. Space-time
is an idealistic concept generated by conventions (cf. Alexandrov's theorem).


\item<+->
Value definiteness of a QSystem is defined in a single context;
this information may be imprinted and distributed over several ``entangled quanta;''
other contexts remain undefined.


\item<+->
Evoution ist stricly a permutation; i.e.,  one-to-one.
``Measurements'' are purely conventional; i.e., a matter of where and how one
conceptualizes the ``cut'' or interfache between ``observer'' \& object.

\item<+->
In case of context mismatch between preparation and measurement,
a context translation occurs \& quasi-randomness is introduced (Malus law).

\end{itemize}
}

%%%%%%%%%%%%%%%%%%%%%%%%%%%%%%%%%%%%%%%%%%%%%%%%%%%%%%%%%%%%%%%%%%%%%%%%%%%%%%%%%%%%%%%%%%%%%%%%%%%%%%%%%

\frame{
\frametitle{References}

\begin{itemize}
\item<+->
Philipp Frank, {em Das Kausalgesetz und seine Grenzen} (Springer, Vienna, 1932); Engl. translation in
Philipp Frank and R.S. Cohen (Editor),  {\em The Law of Causality and its Limits (Vienna Circle Collection)},
(Springer, Vienna, 1997).

\item<+-> Conference proceedings:
     \begin{itemize}
     \item<+->John L. Casti and J. F. Traub, eds., {On Limits},
  {Why Undecidability is Too Important to be Left to {G}{\"{o}}del},
  (Santa Fe Institute Report 94-10-056, 1994).
  URL http://www.santafe.edu/sfi/publications/Working-Papers/94-10-056.pdf

      \item<+->J. L. Casti and A. Karlquist, eds.,
{\em Boundaries and Barriers. On the Limits to Scientific Knowledge}  (Addison-Wesley, Reading, MA), 1996.

      \end{itemize}

\item<+-> KS,  ``Randomness and Undecidability in Physics'' (World Scientific, Singapore, 1993).

\end{itemize}
}

%%%%%%%%%%%%%%%%%%%%%%%%%%%%%%%%%%%%%%%%%%%%%%%%%%%%%%%%%%%%%%%%%%%%%%%%%%%%%%%%%%%%%%%%%%%%%%%%%%%%%%%%%


\frame{
\centerline{\Large \textcolor{blue}{Thank you for your attention!}}
 }

\end{document}

