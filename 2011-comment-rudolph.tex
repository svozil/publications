\documentclass[%
 %reprint,
 %superscriptaddress,
 %groupedaddress,
 %unsortedaddress,
 %runinaddress,
 %frontmatterverbose,
 preprint,
 showpacs,
 showkeys,
 preprintnumbers,
 %nofootinbib,
 %nobibnotes,
 %bibnotes,
 amsmath,amssymb,
 aps,
 % prl,
 pra,
 % prb,
 % rmp,
 %prstab,
 %prstper,
  longbibliography,
 %floatfix,
 %lengthcheck,%
 ]{revtex4-1}

\usepackage[breaklinks=true,colorlinks=true,anchorcolor=blue,citecolor=blue,filecolor=blue,menucolor=blue,pagecolor=blue,urlcolor=blue,linkcolor=blue]{hyperref}
\usepackage{graphicx}% Include figure files
\usepackage{xcolor}
\usepackage[left]{eurosym}
\usepackage{url}


\begin{document}


\title{Comment on ``the quantum state cannot be interpreted statistically''}


\author{Karl Svozil}
\affiliation{Institute of Theoretical Physics, Vienna
    University of Technology, Wiedner Hauptstra\ss e 8-10/136, A-1040
    Vienna, Austria}
\email{svozil@tuwien.ac.at} \homepage[]{http://tph.tuwien.ac.at/~svozil}


\date{\today}

\begin{abstract}
This is critical review of a recent claim of proof that the quantum state cannot be interpreted statistically.
\end{abstract}

\pacs{03.65.Aa,02.50.Ey,03.67.Hk}
\keywords{statistical interpretation of quantum mechanics}
%\preprint{CDMTCS preprint nr. 372/2009}
\maketitle

A recent preprint~\cite{Pusey-arXiv:1111.3328} has generated much attention~\cite{Reich-11}
and is considered {\em `seismic'} by renown researchers.
However, in rederiving these results, no immediate contradiction
with the standard interpretation of the
quantum state, as claimed by the authors of the preprint, arises.

Let us be very explicit and suppose that a quantum is subject to a measurement or a preparation
procedure
(we will need the latter, preparation case, first) with
a dichotomic outcome, say ``$0$'' and ``$1$;''
hence the quantum state can be associated with a two-dimensional Hilbert space, say ${\Bbb R}^2$
(in what follows we do not need any interference or other terms requiring complex Hilbert space).
If these outcomes are mutually exclusive,
we can encode the states associated with ``$0$'' and ``$1$'' by  orthogonal vectors; say
\begin{equation}
\begin{array}{l}
\vert 0 \rangle = (1,0) \textrm{, and}     \\
\vert 1 \rangle = (0,1),
\end{array}
\end{equation}
respectively.
Instead of choosing the standard orthogonal cartesian measurement basis
and prepare outgoing quanta to be in one of two orthogonal states,
we can alternatively, for instance by
employing generalized beam splitters~\cite{rzbb}, also choose to encode another
pairs of states, say  $\vert 0 \rangle$ or $\vert 1 \rangle$, together with
\begin{equation}
\begin{array}{l}
\vert + \rangle =\frac{1}{\sqrt{2} }
\left( \vert 0 \rangle + \vert 1 \rangle \right) = \frac{1}{\sqrt{2}} (1,1)
\textrm{, or}  \\
\vert - \rangle =\frac{1}{\sqrt{2} }
\left( \vert 0 \rangle - \vert 1 \rangle \right) = \frac{1}{\sqrt{2}} (1,-1).
\end{array}
\end{equation}

The authors of Ref.~\cite{Pusey-arXiv:1111.3328}
do not specify what they mean by ``compatibility.''
Usually, in the quantum context, compatibility means
simultaneous co-measurability; that is, two compatible observables
$\textsf{\textbf{A}}$
and
$\textsf{\textbf{B}}$
commute
$[\textsf{\textbf{A}},\textsf{\textbf{B}}]
=
\textsf{\textbf{A}},\textsf{\textbf{B}}
-
\textsf{\textbf{B}},\textsf{\textbf{A}} = 0$.
From this perspective, pure states can also be compatible by requiring that
their representation as vectors, or
the linear subspaces which are the spans of the vectors,
or the associated projectors,
are {\em orthogonal}.
This is due to the fact that
the projectors
$\textsf{\textbf{E}}_{\psi_1} =\vert \psi_1 \rangle \langle \psi_1 \vert $
and
$\textsf{\textbf{E}}_{\psi_2} =\vert \psi_2 \rangle \langle \psi_2 \vert $
associated with two states
$\vert \psi_1 \rangle$   and
$\vert \psi_2 \rangle$
 have a dual role as
observables -- actually, these observables are dichotomic propositions~\cite{birkhoff-36} --
``the quantized system is in the state $\vert \psi_1 \rangle$,''  and
``the quantized system is in the state $\vert \psi_2 \rangle$,'' respectively.


Instead of
what the  authors of Ref.~\cite{Pusey-arXiv:1111.3328} seem to have in mind
when they talk about compatibility of two states
$\vert \psi_1 \rangle$   and
$\vert \psi_2 \rangle$
is not comeasurability but
the {\em consistency} (i.e., free of contradictions)
of the assumption
that, after an observation,
the measured quantum might have been in any of the two states before that measurement.
Therefore, in what follows, the term  ``consistency'' is used for this property.

It is evident that if some agent emitting states, say Alice,
decides to deliver to some other agent, say Bob,
two incompatible quantum states,
Bob is not always in the position to identify uniquely which one of the states Alice has delivered.
More, specifically,
if Alice chooses only to deliver states in the nonorthogonal
basis
$\{
\vert 0 \rangle
,
\vert + \rangle
\}$
,
then Bob,
by measuring, say,
$0$,
cannot tell which one of the two states
$
\vert 0 \rangle
$
or
$
\frac{1}{\sqrt{2} }\left( \vert 0 \rangle + \vert 1 \rangle \right)
$
 he actually received.
For Bob,
the assumption that either one of these states was given to him is consistent
with his observations.
Note also
that, if Bob would measure $1$, he can unanimously and uniquely identify the state $\vert + \rangle  $
delivered by Alice.)

The probability $q$ for this kind of occurrence of
events which can consistently interpreted ambivalently
does, in two dimensions~\cite{WooFie,durt},
not exceed $\frac{1}{{2}}$.
Another option for Alice would be to send out the state
\begin{equation}
\vert \psi \rangle =\frac{1}{\sqrt{2+\sqrt{2}} }
\left( \vert 0 \rangle + \vert + \rangle \right) = \frac{1}{\sqrt{2+\sqrt{2}}  }
\left(1+\frac{1}{\sqrt{2} } ,1\right)
\end{equation}
whose vector is located ``inbetween''
$\vert 0 \rangle$ and
$\vert + \rangle$.
In this case, Bob cannot exclude anything from any observation he could make.

Let us now finally consider
the case when Bob performs  two-partite measurements of the
observables associated with the following states:
\begin{equation}
\begin{array}{l}
\vert \xi_1 \rangle =\frac{1}{\sqrt{2} }
\left( \vert 0 \rangle \vert 1 \rangle + \vert 1 \rangle \vert 0 \rangle \right)
= \frac{1}{\sqrt{2}} (0,1,1,0)
, \\
\vert \xi_2 \rangle =\frac{1}{\sqrt{2} }
\left( \vert 0 \rangle \vert - \rangle + \vert 1 \rangle \vert + \rangle \right)
= \frac{1}{{2}} (1,-1,1,1)
, \\
\vert \xi_3 \rangle =\frac{1}{\sqrt{2} }
\left( \vert + \rangle \vert 1 \rangle + \vert - \rangle \vert 0 \rangle \right)
= \frac{1}{{2}} (1,1,-1,1)
, \\
\vert \xi_4 \rangle =\frac{1}{\sqrt{2} }
\left( \vert + \rangle \vert - \rangle + \vert - \rangle \vert + \rangle \right)
= \frac{1}{\sqrt{2}} (1,0,0,-1)
.
\end{array}
\end{equation}
These observables
$\textsf{\textbf{E}}_i = \vert \xi_i \rangle  \langle \xi_i   \vert $,
$i=1,\ldots, 4$
 are mutually orthogonal and thus compatible.

As input states Alice choses to send the following five nonorthogonal two-partite states
\begin{equation}
\begin{array}{l}
\vert \psi_1 \rangle =
\vert 0 \rangle \vert 0  \rangle
= (1,0,0,0)
, \\
\vert \psi_2 \rangle =
\vert 0 \rangle \vert +   \rangle
=\frac{1}{\sqrt{2} } (1,1,0,0)
, \\
\vert \psi_3 \rangle =
\vert + \rangle \vert 0   \rangle
=\frac{1}{\sqrt{2} } (1,0,1,0)
, \\
\vert \psi_4 \rangle =
\vert + \rangle \vert +   \rangle
=\frac{1}{ {2} } (1,1,1,1)
, \\
\vert \psi_5 \rangle =
\vert \psi \rangle \vert \psi   \rangle
=\left(\frac{1}{4}\left(2+\sqrt{2}\right),
\frac{1}{2\sqrt{2}},\frac{1}{2\sqrt{2}},\frac{1}{4+2\sqrt{2}}\right)
.
\end{array}
\end{equation}


In order to be able to compute the quantum predictions
according to the Born rule for pure states
we have to take the (absolute) square of the scalar products of all $\xi_i$s
and
$\psi_i$s, as enumerated in Table~\ref{2011-comment-rudolph-t1}.
\begin{table}
\begin{tabular}{lcccc}
\hline\hline
$P_{\xi_i}(\psi_j)$&$\xi_1$ &$\xi_2$&$\xi_3$&$\xi_4$\\
\hline
$\psi_1$&$0 $&$ \frac{1}{4} $&$ \frac{1}{4} $&$ \frac{1}{2} $\\
$\psi_2$&$\frac{1}{4} $&$ 0 $&$ \frac{1}{2} $&$ \frac{1}{4} $\\
$\psi_3$&$\frac{1}{4} $&$ \frac{1}{2} $&$ 0 $&$ \frac{1}{4} $\\
$\psi_4$&$\frac{1}{2} $&$ \frac{1}{4} $&$ \frac{1}{4} $&$ 0 $\\
\hline
$\psi_5$&$\frac{1}{4} $&$ \frac{1}{4} $&$ \frac{1}{4} $&$ \frac{1}{4}$  \\
\hline\hline
\end{tabular}
\caption{Probabilities of the occurrence of the measurement of observable $\xi_i$, $i=1,\ldots 4$
on input of states $\psi_j$, $j=1,\ldots 5$.
Note that the row sums must be one.
Because of duality,
also the column sums of rows one to four must be one. }
\label{2011-comment-rudolph-t1}
\end{table}

Suppose for a moment that Alice does not send any $\psi_5$ states
and Bob knows about this.
Then, just as for the single quantum case,
if Bob, by measuring Alice's quanta in the $\xi$-base $\{\xi_1,\ldots ,\xi_4\}$,
obtains a $\xi_1$,
he can be sure that Alice has {\em not} send him a quantum prepared in
the $\psi_1$ state.
Of course, if Alice chooses to send also $\psi_5$ states
and Bob knows about this, then Bob would not be allowed to exclude anything.

In any case, beyond these conditions, I fail to see that Bob may be uncertain
about Alice's choices.
In particular, I fail to see any contradictions or inconsistency,
or the alleged fact that~\cite{Pusey-arXiv:1111.3328}
{\em ``no physical state [$\ldots$]
of the system can be compatible [I suppose this means ``consistent'']
with both the quantum states
$
\vert 0 \rangle
$
and
$
\vert + \rangle
$.''}
What about the quantum state $ \vert \psi \rangle =\frac{1}{\sqrt{2+\sqrt{2}} }
\left( \vert 0 \rangle + \vert + \rangle \right) $?


If it comes to state identification problems, it is interesting to
note some quasi-classical analogies~\cite{e-f-moore,wright,svozil-2008-ql}
of quantum situations,
but this would exceed the frame of this little comment.


\appendix

\section{Mathematica code of the computation}

{
\footnotesize
\begin{verbatim}
(*Definition of the Tensor Product*)
TensorProduct[a_, b_] :=
  Table[(*a,b are nxn and mxm-matrices*)
   a[[Ceiling[s/Length[b]], Ceiling[t/Length[b]]]]*
    b[[s - Floor[(s - 1)/Length[b]]*Length[b],
      t - Floor[(t - 1)/Length[b]]*Length[b]]], {s, 1,
    Length[a]*Length[b]}, {t, 1, Length[a]*Length[b]}];


(*Definition of the Tensor Product between two vectors*)

TensorProductVec[x_, y_] :=
  Flatten[Table[
    x[[i]] y[[j]], {i, 1, Length[x]}, {j, 1, Length[y]}]];


(*Definition of the Dyadic Product*)

DyadicProductVec[x_] :=
  Table[x[[i]] Conjugate[x[[j]]], {i, 1, Length[x]}, {j, 1,
    Length[x]}];

(* definition state vectors *)

s0= {1,0};
s1= {0,1};

splus = (1/Sqrt[2])(s0+s1)
sminus = (1/Sqrt[2])(s0-s1)

(* define vector \[Psi] which is maximally apart from both s0 and splus *)

\[Psi]  = Simplify[(1/Sqrt[2+Sqrt[2]]) (s0 + splus)]

(* check if the normalization of \[Psi] is correct *)

Simplify[ \[Psi].\[Psi] ]  == 1

(* Define some input states *)

\[Psi]1 = TensorProductVec[s0,s0]
\[Psi]2 = TensorProductVec[s0,splus]
\[Psi]3 = TensorProductVec[splus,s0]
\[Psi]4 = TensorProductVec[splus,splus]
\[Psi]5 = Simplify[TensorProductVec[\[Psi],\[Psi]]]

(* Now define two-partite base states *)

\[Xi]1 =  (1/Sqrt[2]) (TensorProductVec[s0,s1]+TensorProductVec[s1,s0])
\[Xi]2 =  (1/Sqrt[2]) (TensorProductVec[s0,sminus]+TensorProductVec[s1,splus])
\[Xi]3 =  (1/Sqrt[2]) (TensorProductVec[splus,s1]+TensorProductVec[sminus,s0])
\[Xi]4 =  (1/Sqrt[2]) (TensorProductVec[splus,sminus]+TensorProductVec[sminus,splus])

(*check if they are indeed orthogonal *)

Simplify[ \[Xi]1.\[Xi]2 ]  == Simplify[ \[Xi]1.\[Xi]3 ] == Simplify[ \[Xi]1.\[Xi]4 ]
== Simplify[ \[Xi]2.\[Xi]3 ] == Simplify[ \[Xi]2.\[Xi]4 ] == Simplify[ \[Xi]3.\[Xi]4 ] == 0

a[1,1] = TensorProductVec[s0,s0]. \[Xi]1;
a[1,2] = TensorProductVec[s0,s0]. \[Xi]2;
a[1,3] = TensorProductVec[s0,s0]. \[Xi]3;
a[1,4] = TensorProductVec[s0,s0]. \[Xi]4;

a[2,1] = TensorProductVec[s0,splus]. \[Xi]1;
a[2,2] = TensorProductVec[s0,splus]. \[Xi]2;
a[2,3] = TensorProductVec[s0,splus]. \[Xi]3;
a[2,4] = TensorProductVec[s0,splus]. \[Xi]4;

a[3,1] = TensorProductVec[splus,s0]. \[Xi]1;
a[3,2] = TensorProductVec[splus,s0]. \[Xi]2;
a[3,3] = TensorProductVec[splus,s0]. \[Xi]3;
a[3,4] = TensorProductVec[splus,s0]. \[Xi]4;

a[4,1] = TensorProductVec[splus,splus]. \[Xi]1;
a[4,2] = TensorProductVec[splus,splus]. \[Xi]2;
a[4,3] = TensorProductVec[splus,splus]. \[Xi]3;
a[4,4] = TensorProductVec[splus,splus]. \[Xi]4;

a[5,1] = TensorProductVec[\[Psi],\[Psi]]. \[Xi]1;
a[5,2] = TensorProductVec[\[Psi],\[Psi]]. \[Xi]2;
a[5,3] = TensorProductVec[\[Psi],\[Psi]]. \[Xi]3;
a[5,4] = TensorProductVec[\[Psi],\[Psi]]. \[Xi]4;

MatrixForm[Simplify[Table[a[i, j]^2, {i, 1, 5}, {j, 1, 4}]]]
TeXForm[Simplify[Table[a[i, j]^2, {i, 1, 5}, {j, 1, 4}]]]
\end{verbatim}

}

\bibliography{svozil}
\end{document}

