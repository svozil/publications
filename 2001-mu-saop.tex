\documentclass[a4paper,12pt]{article}

\usepackage{ngerman}

\textwidth16.5cm \oddsidemargin0cm \evensidemargin0cm
\topmargin0cm \textheight24cm

\begin{document}
\begin{center}
Handout zu Methoden der theoretischen Physik\\
Libisch Florian, November 2001
\end{center}
\section*{Selbstadjungierte Differentialoperatoren}
In der linearen Algebra ist die zu einer linearen Abbildung adjungierte Abbildung \"uber das Skalarprodukt definiert [1]: Sei V ein euklidischer Vektorraum mit Skalarprodukt $<.,.>$. Dann gilt :  Zu jeder linearen Abbildung $\phi : V \rightarrow V$ existiert eine eindeutig bestimmte lineare Abbildung $\phi^\dag : V \rightarrow V$ mit
\begin{equation}
\left<\phi(\vec{x}),\vec{y}\right> = \left<\vec{x},\phi^\dag(\vec{y})\right>
\end{equation}
$\phi^\dag$ hei"st zu $\phi$ adjungierte Abbildung.
Dies legt eine analoge Definition f\"ur Differentialoperatoren nahe. Sei L ein allgemeiner Differentialoperator zweiter Ordnung
\begin{equation}
L=a(x) \frac{d^2}{dx^2}+b(x)\frac{d}{dx}+c(x)
\end{equation}
 auf dem Intervall $[A,B]$ mit stetigen, reellwertigen Funktionen $a(x)$,$b(x)$,$c(x)$ und $a(x) \neq 0$ in $]A,B[$. Als Skalarprodukt betrachtet man
\begin{equation}
\left<f(x),g(x)\right> = \int_A^B dx \ f(x)g(x)
\end{equation}
Verwendet man nun in $\left<Lf(x),g(x)\right>$ partielle Integration, so erh\"alt man, wobei Ableitungen nach x durch ' abgek\"urzt werden:
\begin{eqnarray}
\left<Lf(x),g(x)\right> &=& \int_A^B dx\  L(f) * g\nonumber\\
&=&\int_A^B dx\  \left(a f''+b f'+c f\right) * g\nonumber\\
&=&a f' g\bigg|_A^B + b f g\bigg|_A^B  - \int_A^B dx\ f'\left(a g\right)' + f \left(b g\right)' - c f g\nonumber\\
&=&\underbrace{(f' (a g) - f (a g)' +  b f g)\bigg|_A^B}_{\textrm{Oberfl\"achenterm}}+\int_A^B dx f (a g)''  - f (b g)'  +  f c g\nonumber\\
&=&\overbrace{Q(f,g;B)-Q(f,g;A)}+\left<f(x),L^\dag g(x)\right>
\end{eqnarray}
Der Oberfl"achenterm ist oft durch geeignete Randbedingungen zum Verschwinden zu bringen. F"ur $L^\dag$ ergibt sich durch Vergleich :
\begin{eqnarray}
  L^\dag(g) &=& \frac{d^2}{dx^2}\left(a(x)g(x)\right)-\frac{d}{dx}\left(b(x)g(x)\right)+c(x)g(x)\nonumber\\&=&a'' g + 2 a' g' + a g'' - b' g + b g' -2 b g'+ c g\nonumber \\
&=&L(g)+2(a'-b)g'+(a'-b)'g \label{LdagBedingung}
\end{eqnarray}
Ist $L = L^\dag$, also $a' = b$, dann hei"st der Operator selbstadjungiert. Selbstadjungierte Differentialoperatoren spielen in der Quantenmechanik eine wichtige Rolle, weil sie reelle Eigenwerte besitzen, die physikalischen Observablen entsprechen.

\section*{Die Selbstadjungierte Form einer Differentialgleichung}
Ist L selbstadjungiert, so kann man L laut (\ref{LdagBedingung}) schreiben als
\begin{eqnarray}
L(g(x))=L^\dag(g(x))&=&a(x) \frac{d^2}{dx^2} g(x) + \left(\frac{d}{dx}a(x)\right)\left(\frac{d}{dx}g(x)\right)+c(x)g(x)\nonumber\\&=&
\frac{d}{dx}\left(a(x)\frac{d}{dx}f(x)\right) + c(x)f(x)\label{selbstadj}
\end{eqnarray}
Gegeben sei nun eine Differentialgleichung zweiter Ordnung
\begin{equation}
  f''(x)+P(x)f'(x)+Q(x)f(x) = 0 \label{Aufgabe}
\end{equation}
die auf selbstadjungierte Form gebracht werden soll. Hierf\"ur formen wir (\ref{selbstadj}) um
\begin{eqnarray}
  (a f')' +cf &=& 0\nonumber\\
  a \ f'' + a' \ f'+c f &=&0\nonumber\\
  f'' + \left(\frac{a'}{a}\right)f'+\frac{c}{a}f = 0\label{Umformung}
\end{eqnarray}
wobei im letzten Schritt durch a(x) dividiert wurde. Vergleich mit (\ref{Aufgabe}) lehrt (C ist Integrationskonstante):
\begin{eqnarray*}
  \frac{a'(x)}{a(x)}&=&P(x)\\
  \ln(a(x)) -\ln(C)&=&\int_{x_0}^x dx\ P(x)\\
  a(x) &=& C e^{\int dx \ P(x)}
\end{eqnarray*}
F\"ur c(x) ergibt sich durch Vergleich von (\ref{Aufgabe}) mit (\ref{Umformung}):
\begin{equation}
c(x) = a(x)   Q(x) =  C e^{\int dx \ P(x)} Q(x)
\end {equation}
Damit ist aber auch c proportional zu C, die Konstante kann somit herausgek\"urzt werden und man erh\"alt schlussendlich:
\begin{equation}
  \frac{d}{dx}\left(e^{\int dx \ P(x)}\frac{df(x)}{dx}\right)+e^{\int dx \ P(x)} Q(x) f(x)=0
\end{equation}
\newline
Beispiele : Siehe \"Ubungsskriptum Bsp 1.4.3.
\newline
Quellen : \newline
[1] G. Schranz-Kirlinger und P. Szmolyan, Vorlesungsskriptum Algebra f"ur technische Physiker \newline
[2] M. Schweda, Vorlesungsskriptum Methoden der theoretischen Physik
\end{document}







