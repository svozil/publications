
%%tth:\begin{html}<LINK REL=STYLESHEET HREF="/~svozil/ssh.css">\end{html}
\documentclass[12pt,a4paper]{article}
%\documentstyle[amsfonts]{article}
%\RequirePackage{times}
%\RequirePackage{courier}
%\RequirePackage{mathptm}
%\renewcommand{\baselinestretch}{1.3}
\usepackage{amsmath}
\usepackage{amssymb}
\begin{document}

%\def\frak{\cal }
%\def\Bbb{\bf }
\sloppy



\title{A Generalization to Zeilinger's "it from bit"-concept}
\author{Nikolaus Donath and Karl Svozil\\
 {\small Institut f\"ur Theoretische Physik, University of Technology Vienna }     \\
  {\small Wiedner Hauptstra\ss e 8-10/136,}
  {\small A-1040 Vienna, Austria   }            \\
  {\small e-mail: svozil@tuwien.ac.at}}
\date{ }
\maketitle

%\begin{flushright}
%{\scriptsize http://tph.tuwien.ac.at/$\widetilde{\;\;}\,$svozil/publ/2000-cesena.$\{$htm,ps,tex$\}$}
%\end{flushright}

Given a euklidean vector space in $N=2^n$ dimensions, we pose the following question:
Give all propostions to uniquely distinguish then $N$ vectors that form the basis. How many different sets of propsitions can be defined? What are the propostions for an arbitrary basis?

An algorithm for defining $N$ propostions to decompose the $N$-dimesnional euklidean vectors space can simply be given as follows:
separate the first ${N \over 2}$ vectors from the second ${N \over 2}$
within every block separate the first ${N \over 4}$ vectors from the second ${N \over 4}$
etc. which corresponds to the following operators for an $N=3$ (3-particle case)
\begin{equation}
O_1=\textrm{diag}\Big(1,1,1,1,0,0,0,0\Big)
\end{equation}
\begin{equation}
O_2=\textrm{diag}\Big(1,1,0,0,1,1,0,0\Big)
\end{equation}
\begin{equation}
O_3=\textrm{diag}\Big(1,0,1,0,1,0,1,0\Big)
\end{equation}
It is easy to verify that these operators are projection operators and commute mutually. Are there other possible sets of propositions? No! Indeed, it is possible show that only diagonal matrices containing just $1$s and $0$s in the principal diagonal can be eigenmatrices to the euklidean basis. All we could do is change the order of the vectors, thus we have $8!=40320$ equivalent propositional systems (3 propositions each).
\newline
\newline
Each basis of an $N$-dimensional vector space can be defined in terms of the euklidean basis rotated by a unitary matrix $U: (U^+=U^{-1})$
\begin{equation}
|b_i\rangle=U |e_i\rangle
\end{equation}
Thus the problem of finding the $N$ propostions for the basis $|b_i\rangle$ can simply be solved by transforming the propostions for the euklidean space.
\begin{equation}
O^b_i=U O^e_i U^{-1}
\end{equation}
These propostions have the same eigenvalues as the euklidean propostions
\begin{equation}
O^e |e_i\rangle = \lambda |e_i \rangle
\end{equation}
\begin{equation}
\to U O^e U^{-1} U |e_i \rangle = \lambda U |e_i \rangle
\end{equation}
\begin{equation}
\to O^b |b_i \rangle = \lambda |b_i \rangle
\end{equation}

The transformation matrix for the GHZ-case takes the form
\begin{equation}
U^{GHZ}=\left(\begin{array}{cccccccc}
{1 \over \sqrt{2}} & {1 \over \sqrt{2}} & 0 & 0 & 0 & 0 & 0 & 0 \\
0 & 0 & {1 \over \sqrt{2}} & {1 \over \sqrt{2}} & 0 & 0 & 0 & 0 \\
0 & 0 & 0 & 0 & {1 \over \sqrt{2}} & {1 \over \sqrt{2}} & 0 & 0 \\
0 & 0 & 0 & 0 & 0 & 0 & {1 \over \sqrt{2}} & {1 \over \sqrt{2}} \\
0 & 0 & 0 & 0 & 0 & 0 & {1 \over \sqrt{2}} & -{1 \over \sqrt{2}} \\
0 & 0 & 0 & 0 & {1 \over \sqrt{2}} & -{1 \over \sqrt{2}} & 0 & 0 \\
0 & 0 & {1 \over \sqrt{2}} & -{1 \over \sqrt{2}} & 0 & 0 & 1 & 0 \\
{1 \over \sqrt{2}} & -{1 \over \sqrt{2}} & 0 & 0 & 0 & 0 & 0 & 1
\end{array}\right)
\end{equation}
Explicit calculation shows that the matrices $O^{GHZ}_i=U O^e_i U^{-1}$ are exactly the same as those proposed in [unser paper].
A more physical way to find the propostions was proposed in [Cereceda,Mermin]:
\begin{eqnarray}
T_1&=\sigma_{1x}\sigma_{2y}\sigma_{3y} \\
T_2&=\sigma_{1y}\sigma_{2x}\sigma_{3y} \\
T_3&=\sigma_{1y}\sigma_{2y}\sigma_{3x}
\end{eqnarray}
which, of course, turn out to be a permutation of those given in [unser paper].



\end{document}
