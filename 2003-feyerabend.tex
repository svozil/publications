%%tth:\begin{html}<LINK REL=STYLESHEET HREF="/~svozil/ssh.css">\end{html}
\documentclass[pra,showpacs,showkeys,amsfonts,12pt]{revtex4}
\usepackage{graphicx}
\RequirePackage{times}
\RequirePackage{mathptm}
\RequirePackage{textcomp}
%\renewcommand{\baselinestretch}{1.3}
\begin{document}
\title{Feyerabend and physics}
\author{Karl Svozil}
\email{svozil@tuwien.ac.at}
\homepage{http://tph.tuwien.ac.at/~svozil}
\affiliation{Institut f\"ur Theoretische Physik, University of Technology Vienna,
Wiedner Hauptstra\ss e 8-10/136, A-1040 Vienna, Austria}


\begin{abstract}
Feyerabend frequently discussed physics.
He also referred to the history of the subject when motivating his philosophy of science.
Alas, as some examples show, his understanding of physics remained superficial.
In this respect, Feyerabend is like Popper;
the difference being his self-criticism later on,
and the much more tolerant attitude toward the allowance of methods.
Quite generally, partly due to the complexity of the formalism and the new challenges of their findings,
which left philosophy proper at a loss,
physicists have attempted to developed their own
meaning of their subject.
For instance, in recent years, the interpretation of quantum mechanics
has stimulated a new type of experimental philosophy,
which seeks to operationalize emerging philosophical issues;
issues which are incomprehensible for most philosophers.
In this respect, physics often appears to be a continuation of philosophy by other means.
Yet, Feyerabend has also expressed profound insights into the possibilities for the progress of physics,
a legacy which remains to be implemented in the times to come:
the conquest of abundance, the richness of reality,
the many worlds which still await discovery,
and the vast openness of the physical universe.
\end{abstract}

\pacs{01.75.+m,01.78.+p,01.65.+g,01.70.+w,03.65.Ta}
\keywords{Philosophy of science, science and society, science and government}

\maketitle

\newpage

% Fischer Oct. 2003 How did Feyerabend's anarchcal ideas develop; was he always an anarchist?
% Fischer Oct. Development of his ideas from analytic philosophy to ``anything goes''.

%\tableofcontents
\end{document}

We ponder upon a peculiar discrepancy:
Paul Feyerabend has become almost a shooting star of philosophy of science,
an icon of freedom and heresy
to a generation coming of age in the late period of the twentieth century.
Yet he never quite managed to obtain influence and convince scientific peers, governments and electorates to implement
his recommendations regarding the selection of science funding and the implementation
of science in general.
In addition to the invitation to allow also less stringent and less traditional criteria,
Feyerabend had a system of lay assessors (lay judges)
for the evaluation of scientific research proposals in mind,
which came close to procedures well known from the courts of lay assessors.
These kind of initiatives have been implemented nowhere.
As publicity and marketing did not seem to be the problems in Feyerabend's case,
there might be other reasons which prevented even tiny steps towards these
goals to unfold.
Indeed, the behemoth created by the Sixth Framework Programme (FP6)
and the establishment of the European Research Area (ERA),
appears to be totally opposed to any such strategies.
One reason for this might be the inadequacy and falsity of the
original suggestions; yet another
reason might be conflicts of interests within well established research communities.
We shall present Feyerabend's ideas in a temporal context,
and discuss possible implementation strategies and the difficulties they may meet.
