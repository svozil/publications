%%tth:\begin{html}<LINK REL=STYLESHEET HREF="/~svozil/ssh.css">\end{html}
\documentstyle[12pt]{article}
%\documentstyle[amsfonts]{article}
%\RequirePackage{times}
%\RequirePackage{courier}
%\RequirePackage{mathptm}
%\renewcommand{\baselinestretch}{1.3}
\begin{document}

%\def\frak{\cal }
\def\Bbb{\bf }
\sloppy



\title{Boole-Bell type inequalities for the Greenberger-Horne-Zeilinger and 3-3 cases}
\author{Itamar Pitowsky \\
 {\small Department of Philosophy, The Hebrew University,}
  {\small Mount Scopus, Jerusalem 91905, Israel }     \\
  {\small e-mail: itamarp@vms.huji.ac.il}   \\
{\small and}\\
Karl Svozil\\
 {\small Institut f\"ur Theoretische Physik, University of Technology Vienna }     \\
  {\small Wiedner Hauptstra\ss e 8-10/136,}
  {\small A-1040 Vienna, Austria   }            \\
  {\small e-mail: svozil@tuwien.ac.at}}
\date{ }
\maketitle

\begin{flushright}
{\scriptsize http://tph.tuwien.ac.at/$\widetilde{\;\;}\,$svozil/publ/2000-poly.$\{$htm,ps,tex$\}$}
\end{flushright}

\begin{abstract}
We derive all inequalities in the Greenberger-Horne-Zeilinger (GHZ) case
as well as for two particles with
 spin state measurements in three directions.
\end{abstract}


\newpage
\section{Methods}

Consider two agents called Alice and Bob.
Alice performs some experiments
and observes the relative frequencies of some elementary events,
as well as the relative frequencies of joint occurrences of these events.
For example, Alice may observe
the following propositions: $A=$ {\em  ``snow is falling in Vienna,''} %I hate snow!!!!!!!!!!!!!!!
and $B=$ {\em  ``the sun shines in Jerusalem,''} and the joint proposition
$AB=$ {\em  ``snow is falling in Vienna and the sun shines in Jerusalem.''}
In a more physical context,
she may observe the relative frequency to find an electron in a spin up state
in one, two or more particular directions.

Suppose Alice communicates the observed relative frequencies to Bob and asks him
whether the observed data are reasonable and consistent; i.e.,
whether this could be a possible classical experience.
(In what follows, a classical experience is one based on a classical Boolean propositional
structure, in particular one in which the distributive law is a tautology.)

Let $p(A)$ denote the relative frequency or probability estimate of proposition $A$.
If, for example, Alice states that her observation is
$p(A)=p(B)=0.99$ yet $p(AB)=0.01$, the classical consistency of her data would most likely be
rejected by Bob, at least intuitively: after all,
it cannot happen that snow is falling  in Vienna most of the time, and the sun shines in Jerusalem
most of the time; yet both events almost never happen at the same time.

Already 150 years ago, George Boole has considered these kind of questions \cite{Boole,Boole-62,Hailperin,pitowsky,Pit-94}.
He formulated ``conditions of possible experience'' which must be fulfilled by the relative
frequencies of logically connected events.
For the simple case mentioned above, the conditions are
\begin{eqnarray}
&&0\le p(AB)\le p(A),p(B),\nonumber\\
&&p(A)+ p(A)-p(AB)\le 1,\nonumber
\end{eqnarray}
the latter one being not satisfied by Alice's ``unreasonable'' data given above.
It should be stressed that any such system of inequalities
represents necessary and sufficient criteria for consistency {\em within a given single sample},
classical, quantum or otherwise \cite[Chapter10]{svozil-ql}.
They are the {\em best possible upper bound} for the conceivable probabilities \cite{Hailperin,pitowsky}.
Any violation of Boole's inequalities would result
in an outright contradiction and inconsistency and thus in nonsense.
Therefore, if only data within a single sample are considered,
not even quantum events violate the conditions of possible experience.
Yet, if counterfactuality is involved and  {\em  data from different samples are mixed},
under certain circumstances,
no immediate phenomenological inconsistency may result if the inequalities are violated.
This latter  case occurs in quantum mechanics \cite{Pit-94},
where different samples may correspond to
complementary observables.
In the foundational context, the
equations and inequalities Boole derived as criteria for
``conditions of possible experience''
can be identified with Bell-type inequalities.

Boole's (in)equalities correspond to the faces of convex correlation polytopes.
The vertices (extremal points) of the polytope are vectors  which correspond to the rows
of the associated  truth tables,
whereby the logical truth values "{\em true}'' and ``{\em false}''
are identified with the digits $0$ and $1$, respectively.
In our simple example, the truth values of $A\wedge B$ (``$\wedge$'' stands for the logical {\em and} operator),
\begin{center}
\begin{tabular}{ccc}
 \hline\hline
$A$&$B$&$A\wedge B$\\
 \hline
$0$&$0$&0\\
$0$&$1$&0\\
$1$&$0$&0\\
$1$&$1$&1\\
 \hline\hline
\end{tabular}
\end{center}
correspond to the vertices
$(0,0,0)$,
$(0,1,0)$,
$(1,0,0)$ and
$(1,1,1)$.
Intuitively speaking, every row can be associated with one scenario.
Let the scenario associated with the $n$'th row have probability $P_i$.
For instance, $P_1$ stands for the the scenario {\em ``it rains in Vienna and the sun does not shine in Jerusalem.''}
Since all rows exhaust all possible scenarios which are mutually incompatible,
the sum over these probabilities must be one; i.e.,
$P_1+P_2+P_3+P_4=1$.
This, as well as positivity $0\le P_i\le 1$, are  conditions imposed on the $P_i$'s.
Additional conditions are imposed on the probabilities from
\begin{eqnarray}
&&(p(A),p(B),p(AB))\nonumber\\
&&=P_1(0,0,0)+P_2(0,1,0)+P_3(1,0,0)+P_4(1,1,1)=\nonumber\\
&&=(1-P_1+P_2,1-P_1+P_3,1-P_1-P_2-P_3)\nonumber\\
&&\ge (0,0,0) ,\nonumber
\end{eqnarray}
which is a weighted average over the vertices ($P_4$ has been eliminated).
Thus the associated polytope is convex; and its faces are the extremal conditions
which have to be satisfied by the probabilities.
These faces are represented by the inequalities associated
with the Boole-Bell type conditions of possible experience.
For a general and  rigorous derivation, see Pitowsky \cite{pitowsky}.

Every convex polyhedron has two representations:
one as the intersection of finite halfspaces and the other as Minkowski
sum of the convex hull of finite points and the nonnegative hull of finite directions.
These are called H-representation and V-representation, respectively.
The problem of finding the faces of a complex polyhedron from its vertices is known as the {\em hull problem.}
It can be effectively solved by one of the available algorithms.
We have chosen the {\tt cdd} package \cite{cdd-pck} which is an efficient
implementation of the double description method  \cite{MRTT53}
by Komei Fukuda \cite{FP96,fukuda-94,fukuda-pr}.

Boole-Bell type inequalities represent a very general and powerful criterium for classicality
which can be applied to a variety of physical events in a straightforward manner.
In what follows we shall investigate the general cases of three particles with two propositions each (GHZ-case),
as well as of two particles with three propositions each.


\section{GHZ case: three particles at two angles}



In the Mermin version \cite{mermin,mermin-93} of the GHZ case \cite{ghz,ghsz},
the relevant propositions involve three particles,
denoted by $A,B,C$,
and two properties, denoted by $1,2$, respectively.
The set of 26 propositions involve all three-particle events and is given by
$\{
 A_1        ,$ $
 A_2        ,$ $
 B_1        ,$ $
 B_2        ,$ $
 C_1        ,$ $
 C_2        ,$ $
 A_1B_1     ,$ $
 A_1C_1     ,$ $
 A_1B_2     ,$ $
 A_1C_2     ,$ $
 A_2B_1     ,$ $
 A_2C_1     ,$ $
 A_2B_2     ,$ $
 A_2C_2     ,$ $
 B_1C_1     ,$ $
 B_1C_2     ,$ $
 B_2C_1     ,$ $
 B_2C_2     ,$ $
 A_1B_1C_1  ,$ $
 A_1B_1C_2  ,$ $
 A_1B_2C_1  ,$ $
 A_1B_2C_2  ,$ $
 A_2B_1C_1  ,$ $
 A_2B_1C_2  ,$ $
 A_2B_2C_1  ,$ $
 A_2B_2C_2
\}$.


The resulting correlation polytope is 26-dimensional and has 64 vertices and 53856 faces
corresponding to an equal amount of Boole-Bell type inequalities.
For a complete listing of all Boole-Bell type inequalities, see Ref. \cite{pit-svo-list1}.
Many of these inequalities are trivial; e.g.,   $p(A_1B_1) \ge  p(A_1B_1C_1)  \ge  0$ or
$ p(A_1) + p(A_1B_1C_1)\ge   p(A_1B_1) + p(A_1C_1) $.
Many equations can be reduced to others due to symmetries;
e.g., by a permutation of particles.
Inequalities which have been discussed in this context
by Larsson and Semitecolos \cite{lars-semi} and by
de Barros and Suppes \cite{deBarros-Suppes} have similar counterparts in the enumeration.
We stress here that our method produces {\em optimal} Boole-Bell inequalities in the sense that they
represent
the {\em best possible upper bounds} for the conceivable classical probabilities.
In what follows we shall enumerate some new Boole-Bell inequalities not published so far.
\begin{eqnarray}
0&\ge& -p(A_2)+p(A_2B_1)+p(A_2C_1)-p(B_2C_2)-p(A_2B_1C_1)-p(A_1B_2C_1)
\nonumber\\&&\quad
-p(A_1B_1C_2)+p(A_1B_2C_2)+p(A_2B_2C_2)
\label{eghz-0}
,\\
0&\ge& -p(A_1)-p(B_1)-p(C_2)+p(A_1B_1)+p(A_1B_2)+2p(A_1C_2)+p(A_2B_1)
\nonumber\\&&\quad
-p(A_2B_2)+2p(B_1C_2)+p(A_1B_1C_1)-p(A_2B_1C_1)-p(A_1B_2C_1)
\nonumber\\&&\quad
-3p(A_1B_1C_2)+p(A_2B_2C_1)-p(A_2B_1C_2)-p(A_1B_2C_2)+p(A_2B_2C_2)
\label{eghz-1}
,\\
0&\ge& -2p(A_1)-p(A_2)-p(B_1)-p(C_1)+p(A_1C_1)+2p(A_1B_2)+2p(A_1C_2)
\nonumber\\&&\quad
+2p(A_2B_1)+2p(A_2C_1)-p(A_2C_2)+p(B_1C_1)+p(B_2C_1)+p(B_1C_2)
\nonumber\\&&\quad
-p(B_2C_2)   +p(A_1B_1C_1)-2p(A_2B_1C_1)-2p(A_1B_2C_1)-2p(A_1B_1C_2)
\nonumber\\&&\quad
-p(A_2B_2C_1)   -p(A_2B_1C_2)-p(A_1B_2C_2)+2p(A_2B_2C_2)
\label{eghz-2}
,\\
0&\ge& -p(A_1)-p(B_1)-p(C_1)-p(C_2)+p(A_1B_1)+p(A_1C_1)+p(A_1B_2)
\nonumber\\&&\quad
+2p(A_1C_2)+p(A_2B_1)-p(A_2B_2)+p(B_1C_1)+2p(B_1C_2)-2p(A_2B_1C_1)
\nonumber\\&&\quad
-p(A_1B_2C_1)-3p(A_1B_1C_2)+2p(A_2B_2C_1)-p(A_2B_1C_2)
\nonumber\\&&\quad
-2p(A_1B_2C_2)+p(A_2B_2C_2)
\label{eghz-3}
,\\
0&\ge& -3p(A_1)-3p(B_1)-p(C_1)+3p(A_1B_1)+p(A_1C_1)+3p(A_1B_2)
\nonumber\\&&\quad
+3p(A_1C_2)+3p(A_2B_1)+p(A_2C_1)-3p(A_2B_2)-p(A_2C_2)+3p(B_1C_1)
\nonumber\\&&\quad
+p(B_2C_1)+p(B_1C_2)-p(B_2C_2)-2p(A_1B_1C_1)-3p(A_2B_1C_1)
\nonumber\\&&\quad
-2p(A_1B_2C_1) -2p(A_1B_1C_2)+2p(A_2B_2C_1)-p(A_2B_1C_2)
\nonumber\\&&\quad
-2p(A_1B_2C_2)+2p(A_2B_2C_2)
\label{eghz-4}
,\\
2&\ge& -p(A_1)+2p(A_2)+p(B_1)+p(B_2)-p(C_1)+2p(C_2)-p(A_1B_1)
\nonumber\\&&\quad
+p(A_1C_1)+2p(A_1B_2)+p(A_1C_2)-p(A_2B_1)+p(A_2C_1)-2p(A_2B_2)
\nonumber\\&&\quad
-3p(A_2C_2)+p(B_1C_1)- p(B_2C_1)- p(B_1C_2)-2B_2C_2)+2p(A_1B_1C_1)
\nonumber\\&&\quad
-2p(A_2B_1C_1)-2A_1B_2C_1)-2p(A_1B_1C_2)+2p(A_2B_2C_1)+2p(A_2B_1C_2)
\nonumber\\&&\quad
-p(A_1B_2C_2)+3p(A_2B_2C_2)
\label{eghz-5}
,\\
3&\ge& +2p(A_2)+3p(B_2)+2p(C_2)+2p(A_1C_1)-p(A_1C_2)+p(A_2B_1)
\nonumber\\&&\quad
-p(A_2C_1)-3p(A_2B_2)-p(A_2C_2)+p(B_1C_2)-3p(B_2C_2)+p(A_1B_1C_1)
\nonumber\\&&\quad
-2A_2B_1C_1)-3p(A_1B_2C_1)-2p(A_1B_1C_2)+2p(A_2B_2C_1)-2p(A_2B_1C_2)
\nonumber\\&&\quad
+2p(A_1B_2C_2)+2p(A_2B_2C_2)
\label{eghz-6}
,\\
 0 &\ge&  -3p(A_1) -2p(B_1) - p(C_1) +2p(A_1B_1) + p(A_1C_1)
\nonumber\\&&\quad
 +3p(A_1B_2) +3p(A_1C_2) +2p(A_2B_1) + p(A_2C_1) -2p(A_2B_2)
\nonumber\\&&\quad
- p(A_2C_2) + p(B_1C_1) + p(B_2C_1) +2p(B_1C_2) -2p(B_2C_2)
\nonumber\\&&\quad
+ p(A_1B_1C_1) -2p(A_2B_1C_1) -3p(A_1B_2C_1) -4p(A_1B_1C_2)
\nonumber\\&&\quad
+ p(A_2B_2C_1) - p(A_2B_1C_2) - p(A_1B_2C_2) +3p(A_2B_2C_2)
\label{eghz-7}
\end{eqnarray}

Suppose the elementary experiences or propositions are
clicks in a counter of a three particle interferometer as  discussed by
Greenberger, Horne, Shimony and Zeilinger \cite{ghsz}.
In the interferometric case \cite{ghsz},
$p(A_i)=p(B_i)=p(C_i)=1/2$ and
$p(A_iB_j)=p(A_iC_j)=p(B_iC_j)=1/4$, where $i,j=1,2$.
The joint quantum probabilities of events depend on three angles
$\phi_1,\phi_2,\phi_3$ in each one of the detector groups $A,B,C$, respectively.
They are given by
$
p(A_iB_jC_k)=
(1/8)[1+\sin (\phi_i+\phi_j+\phi_k)]
$, where again $i,j,k =1,2$.
For example, $C_2$ corresponds to the  proposition,
{\em ``the first detector of the detector group $C$ at angle $\phi_2$ clicks''}
(we only consider clicks in the first one of the two detectors here).
Yet it should be stressed that the derived inequalities are in no way dependent
on this particular interpretation. Any other, in particular one evolving spin state measurements,
would do just as well.
Let us specify the angles at $\phi_1=0$ and $\phi_3=\pi /2$.
Then,
(\ref{eghz-0})--(\ref{eghz-7})
are among the  1329 equalities (out of 53856) which
violate  Boole's condition of possible experience.
The corresponding
factors are
$ 0 :  1/8$,
$ 0 :  1/4$,
$ 0 :  1/4$,
$ 0 :  1/8$,
$ 0 :  1/8$,
 $2 :  9/8$,
 $3 :  25/8$,
 $0 :  1/2$,
respectively.
Notice that the inequalities can also be written in a form containing only
coincidence probabilities of three events.
For instance, (\ref{eghz-7}) yields
\begin{eqnarray}
{1\over 4} &\ge&
 p(A_1B_1C_1) -2p(A_2B_1C_1) -3p(A_1B_2C_1) -4p(A_1B_1C_2)
\nonumber\\&&\quad
+ p(A_2B_2C_1) - p(A_2B_1C_2) - p(A_1B_2C_2) +3p(A_2B_2C_2).
\label{eghz-7a}
\end{eqnarray}
%-3/2 -2/2 - 1/2 +2/4 + 1/4 +3/4 +3/4 +2/4 + 1/4 -2/4 - 1/4 + 1/4 + 1/4 +2/4 -2/4 + a_1b_1c_1 -2a_2b_1c_1 -3a_1b_2c_1 -4a_1b_1c_2 + a_2b_2c_1 - a_2b_1c_2 - a_1b_2c_2 +3a_2b_2c_2$ \\
%-12/4    +11/4   + a_1b_1c_1 -2a_2b_1c_1 -3a_1b_2c_1 -4a_1b_1c_2 + a_2b_2c_1 - a_2b_1c_2 - a_1b_2c_2 +3a_2b_2c_2$ \\

We find that it is not possible to obtain  a violation of Boole-Bell type inequalities
if only  single-particle and three-particle coincidences are taken into account.
This occurs only if also the two-particle coincidences are added.


\section{Two particles at three angles}

We shall next consider the case of two particles, labeled by
$A,B$,
and three properties per particle, denoted by $1,2,3$, respectively.
The set of 15 propositions involve all three-particle events and is given by
$\{
 A_1        ,$ $
 A_2        ,$ $
 A_3        ,$ $
 B_1        ,$ $
 B_2        ,$ $
 B_3        ,$ $
 A_1B_1     ,$ $
 A_1B_2     ,$ $
 A_1B_3     ,$ $
 A_2B_1     ,$ $
 A_2B_2     ,$ $
 A_2B_3     ,$ $
 A_3B_1     ,$ $
 A_3B_2     ,$ $
 A_3B_3
\}$.

The resulting correlation polytope is 15-dimensional and has 684 faces,
corresponding to 684 Boole-Bell type inequalities.
For a complete listing of all Boole-Bell type inequalities, see Ref. \cite{pit-svo-list2}.
Again, many of these inequalities are trivial; e.g.,   $p(A_2) \ge p(A_1B_3)  \ge  0$.
Many inequalities are   familiar ones,
such as the inequalities associated with the Bell-Wigner polytope
($\{A_1,A_2,A_3,A_1A_2,A_1A_3,A_2A_3\}$); i.e.,
\begin{eqnarray}
1 &\ge&  + p(A_2) + p(B_3) + p(A_1B_1) - p(A_1B_3) - p(A_2B_1) - p(A_2B_3)\label{e3-3bw}\\
  &\ge&  + p(A_1) + p(A_2) + p(A_3) - p(A_1A_2) - p(A_1A_3)  - p(A_2A_3),\nonumber
\end{eqnarray}
if one identifies $A_i\equiv B_i$, $i=1,2,3$
[recall that $p(A_1A_1)=p(A_1)$].
The following Boole-Bell inequalities are less known.
\begin{eqnarray}
3 & \ge & 2p(A_1) + p(A_2) +p(B_2) +2p(B_3) - p(A_1B_1) - p(A_1B_2) - p(A_1B_3)
\nonumber\\ &&\quad
 + p(A_2B_1) - p(A_2B_2) - p(A_2B_3) + p(A_3B_2) - p(A_3B_3)
,\\
1 & \ge & - p(A_1) + p(A_2) -p(B_2) +p(B_3) + p(A_1B_1) + p(A_1B_2) - p(A_2B_1)
\nonumber\\ &&\quad
 + p(A_2B_2) - p(A_2B_3) + p(A_3B_1) - p(A_3B_2) - p(A_3B_3)
,\\
1 & \ge &  p(A_2) - p(A_3) -2p(B_1) +p(B_3) + p(A_1B_1) + p(A_1B_2) - p(A_1B_3)
\nonumber\\ &&\quad
+ p(A_2B_1) - p(A_2B_2) - p(A_2B_3) + p(A_3B_1) + p(A_3B_3)
,\\
2 & \ge &  p(A_2) + p(A_3) + p(B_1) + p(B_3) + p(A_1B_1) - p(A_1B_2) - p(A_1B_3)
\nonumber\\ &&\quad
- p(A_2B_1) + p(A_2B_2) - p(A_2B_3) - p(A_3B_1) - p(A_3B_2)
,       \label{e3-3c}\\
0 & \ge & - p(A_1) - p(A_2) -p(B_1) -p(B_2) - p(A_1B_1) + p(A_1B_2) + p(A_1B_3)
\nonumber\\ &&\quad
 + p(A_2B_1) + p(A_2B_3) + p(A_3B_1) + p(A_3B_2) - p(A_3B_3)
,\label{e3-3a}\\
0 & \ge & - p(A_1) -p(B_3) + p(A_1B_2) + p(A_1B_3) - p(A_2B_2) + p(A_2B_3)
.       \label{e3-3b}
\end{eqnarray}


Let us specify our experiment now by choosing
the common spin state measurements of two spin $1/2$-particles prepared in a singlet state.
Thereby, every elementary proposition $A_{\bf x}$ can be stated as,
``the spin of  particle $A$ in the direction $\bf x$ is {\tt up}.''
It is well known that, for the singlet state of spin $1/2$-particles, the probability
to find the particles  both either in spin ``up'' or both in spin ``down'' states
is given by
$
p^{\uparrow \uparrow} (\theta)=
p^{\downarrow \downarrow} (\theta)=
(1/2)\sin^2[(\theta /2)]
$, where $\theta$ is the angle between the measurement directions.
Likewise, the probabilities for different spin states is given by
$
p^{\uparrow \downarrow} (\theta)=
p^{\downarrow \uparrow} (\theta)=
(1/2)\cos^2[(\theta /2)]
$.
In searching for possible violations of the inequalities,
one may choose a symmetric configuration such as
$\theta(A_1 = B_1) = 0$,
$\theta(A_2 = B_2) = 2\pi / 3 $,
$\theta(A_3 = B_3) = 4\pi / 3 $,
in which case one obtains for the parallel case ($\uparrow \uparrow$ or $\downarrow \downarrow$) a violation
of
$0:1/4$ for (\ref{e3-3a}) and of
$0:1/8$ for (\ref{e3-3b}).
For the opposite case ($\uparrow \downarrow$ or $\downarrow \uparrow$), the violation of
(\ref{e3-3bw}) is $1:9/8$ and of
(\ref{e3-3c}) is $2:5/4$.
In the less symmetric configuration
$\theta(B_1) = -\pi /4$,
$\theta(A_1) = 0$,
$\theta(B_2) = \pi / 4$,
$\theta(B_3) = \pi / 3 $,
$\theta(A_2) = \pi / 2$,
$\theta(A_3) = 2\pi /3$,
more inequalities violate the Bell inequalities, although to a lesser degree.

\section{Summary and outlook}
Correlation polytopes are are royal road to the understanding and generation of Boole-Bell type inequalities,
These inequalities can be interpreted as maximal bounds for consistent  ``conditions of possible experience,''
classical, quantum or otherwise, within a single sample.
In the complementary,  nondistributive, case, data from different samples
may consistently violate the Boole-Bell inequalities.

We have presented powerful numerical methods to evaluate the faces of correlation polytopes and thus Boole-Bell type
inequalities.
They were applied to a calculation of all
Boole-Bell type inequalities for the Greenberger-Horne-Zeilinger and 3-3 cases.


\bibliography{svozil}
\bibliographystyle{unsrt}
%\bibliographystyle{plain}


\end{document}
