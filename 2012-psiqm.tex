\documentclass[%
 %reprint,
 %superscriptaddress,
 %groupedaddress,
 %unsortedaddress,
 %runinaddress,
 %frontmatterverbose,
 preprint,
 showpacs,
 showkeys,
 preprintnumbers,
 %nofootinbib,
 %nobibnotes,
 %bibnotes,
 amsmath,amssymb,
 aps,
 %prl,
  pra,
 % prb,
 % rmp,
 %prstab,
 %prstper,
  longbibliography,
 %floatfix,
 %lengthcheck,%
 ]{revtex4-1}

%\usepackage{cdmtcs-pdf}

\usepackage[breaklinks=true,colorlinks=true,anchorcolor=blue,citecolor=blue,filecolor=blue,menucolor=blue,pagecolor=blue,urlcolor=blue,linkcolor=blue]{hyperref}
\usepackage{graphicx}% Include figure files
\usepackage{xcolor}
\usepackage[left]{eurosym}
\usepackage{url}


\begin{document}


\title{The present situation in quantum mechanics and the ontological single pure state conjecture}

%\cdmtcsauthor{Karl Svozil}
%\cdmtcsaffiliation{Vienna University of Technology}
%\cdmtcstrnumber{407}
%\cdmtcsdate{September 2011}
%\coverpage

\author{Karl Svozil}
\affiliation{Institute of Theoretical Physics, Vienna
    University of Technology, Wiedner Hauptstra\ss e 8-10/136, A-1040
    Vienna, Austria}
\affiliation{Dipartimento di Scienze Pedagogiche e Filosofiche, Universit\'a  di Cagliari,\\
 Via Is Mirrionis, 1, I-09123, Cagliari, Sardinia, Italy}
\email{svozil@tuwien.ac.at} \homepage[]{http://tph.tuwien.ac.at/~svozil}


\pacs{03.65.Ta, 03.65.Ud}
\keywords{quantum  measurement theory}
%\preprint{CDMTCS preprint nr. 407/2011}

\begin{abstract}
Despite its excessive success in predicting experimental frequencies and certain single outcomes,
the ``new quantum mechanics'' is haunted by several conceptual and technical issues; among them
(i) the (non-)existence of measurement and the cut between observer and object in an environment globally covered by a unitary (i.e. one-to-one Laplacian deterministic) evolution;
related to the question of how many-to-one mappings could possibly ``emerge'' from one-to-one functions; and also where exactly ``randomness resides;''
(ii) what constitutes a pure quantum state;
(iii) the epistemic or ontic (non-)existence of mixed states; related to the question of how non-pure states can be ``produced'' from pure ones; as well as
(iv) the epistemic or ontic existence of pure but entangled and/or coherent states containing classically mutually exclusive states;
an issue the late Schr\"odinger has called ``quantum quagmire'' or ``jellification;''
(v) the (non-)existence of quantum value indefiniteness and its purported ``resolution'' by quantum contextuality; and finally
(vi) the claim that the best interpretation of the quantum formalism is its non-interpretation.
All of these can be overcome by assuming that, at any given time, only a single pure state exists;
and that the quantum evolution ``permutes'' this state in its Hilbert space.
\end{abstract}

\maketitle


\section{``Dogmatism as doctrine of mental reservation''}

Quite rightly quantum mechanics has been lauded as one of the most successful physical theories ever developed by human thought.
Yet at the same time as it presents itself as a progressive research program
\cite{lakatosch} through corroborated predictions of an ever increasing phenomenology,
non-negligible areas remain conceptually fuzzy and unclear.
In this respect, the situation is not too different today than it has been almost a century ago,
when Schr\"odinger published his famous series of articles
\cite{schrodinger}
on the conceptual difficulties troubling a formal framework which he had helped to create.

One of the most disturbing aspects of the reception of quantum mechanics by the physics community appears to be
its willingness to accept incomplete knowledge {\em not as a challenge but as an inpenetrable, irreducible dogma.}
Because, while quantum mechanics has profited from the thrust and momentum of a strong rationalistic paradigm
that governed the mechanistic and industrial periods originating from the Renaissance and Enlightenment,
it has also fostered an antirational undertow, a sentiment towards
irrationality that seems to grow stronger by the day.

Presently we have a situation in which it is proclaimed in one of the most venerable scientific journals
by one of the most venerable physicists of our time that \cite{zeil-05_nature_ofQuantum}
``the discovery that individual events are
irreducibly random is probably one of the
most significant findings of the twentieth
century. $\ldots$~For the individual event in quantum physics,
not only do we not know the cause, there is no cause.''
Nowhere it is mentioned that, unlike logical, metamathematical and algorithmic undecidability \cite{davis,smullyan-92},
any such claim is a {\em belief} remaining {conjectural};
and that it is {\em provable unprovable} \cite{svozil-07-physical_unknowables} for the quite obvious reason that
any formal proof of even lawlessness (let alone ``randomness by compressed lawlessness'' \cite{calude:02})
would require transfinite capacities.


Actually, at my matured life as a physicist, I feel myself confronted with a situation
which the late Ernst Specker \cite{specker-ges,specker-60} in a personal conversation once
termed a ``Jesuit lie''\footnote{This is often related to the {\em doctrine of mental reservation.} The Jesuits, who I value highly, are very kindly asked to please forgive me for using the name of their order in this context; I myself being catholic.}:
very often, some facts are revealed that are favourable to some party's opinion,
while other facts which are not so favourable are not mentioned at all, or remain  ``conveniently unmentioned.''
As a consequence, misinterpretations if not errorneous perceptions abound.

The situation has reached a point where I find it unbearable not to stand up and mention the issues about ``our new quantum mechanics'' \cite[p.~866]{born-26-1}.
I am well aware that what I am going to say will not please everybody; indeed some colleagues, even those whom I owe much and whose scientific talents I value so highly,
may be upset about me and my direct and critical tone in this communication.
Yet, as pathetic as this may sound,  I am convinced that I cannot maintain my personal scientific integrity,
as well as the demand for truth that I myself, my peers, as well as the taxpayers in two countries expect,
by maintaining silent.
Alas I am afraid, here I have to stand up and state pointedly what others do not dare to say; as they are
either feeling either too comfortable with, or too intimidated by, or too endangered through,
the academic situation they find themselves in.
Thereby I confess outrightly that I cannot claim to be correct on all, or even many, or even any, of the following topics.



Indeed, entire related fields of quantum mechanics appear to be awash with claims which obfuscate or even block new vistas of thought.
Take, for instance, {\em quantum cryptography.}
It is often motivated by the possibility to compromise classical channels by cryptanalytic attacks.
Yet, major protocols in quantum cryptography,
such as the original protocol by Bennett and  Brassard introduced in 1984 (BB84) \cite{benn-84,benn-92}, in order to prevent
middleman attacks, have to assume uncompromised classical channels.
That is, they require the very similar confidence as those (classical) methods
against which they are presenting themselves superior to.
And although middleman attacks were already mentioned in the original BB84 publication,
there exist numerous papers and reviews \cite{gisin-qc-rmp}
discussing in great length the ``unconditional security of quantum cryptography'' without ever mentioning
the underlying rather unrealistic assumptions about the superficially decreased capacity of conceivable cryptanalytic attacks.
In more realistic scenarios \cite{ell-co-05} quantum cryptography makes necessary classical authentification
and thus is at most good for  ``key growing.''
In such ways, both the general public as well as potential customers remain in ignorance about the potential security threats to
certain quantum cryptographic protocols.


Another issue appears to be a rather relaxed notion of what constitutes a {\em proof}.
This is related to the fact that in many cases, the phenomena are reconstructed from detector clicks;
and it is not always clear what kind of confidence about various physical properties is associated with certain such clicks.
For the sake of an example, take the discussion \cite{Kimble-aposterioriQT,Bouwm-aposterioriQTReply} about an alleged {\it a posteriori} realization of
quantum teleportation \cite{BBCJPW}.


Still another group of examples are the alleged ``empirical'' proofs of the Kochen-Specker theorem \cite{kochen1}, which can be formally derived by a finite, constructive
proof by contradiction (or {\it reductio ad absurdum}).
Already
1994 at a {\it International Quantum Structures Association} meeting in Prague,
Rob Cliften privately mused about conceivable ``experimental proofs'' by jokingly asking, ``how can you measure a contradiction?''
Yet, as time passed by, some papers have been published which allegedly seem to be doing just that -- (proposing to) experimentally
``prove the Kochen-Specker theorem''
\cite{cabello-98,huang-2003}.

Moreover, by seemingly taking for granted the wrong assumption that the Kochen-Specker theorem somehow ``implies,''
or is a necessary and sufficient criterion for, {\em quantum contextuality}, ``experimental proofs of quantum contextuality''
(measuring some events or frequencies associated with quantum correlations which contradict classical predictions)
abound
\cite{hasegawa:230401,cabello:210401,Bartosik-09,kirch-09,PhysRevLett.103.160405,Lapkiewicz-11}.
Make no mistake -- all of these experiments are beautifully and professionally performed and deserve publication.
Alas unfortunately they do not convincingly prove what they are purporting to demonstrate.
They fall short of being able to convincingly argue the connections between the observed detector clicks on the one hand,
and the alleged physical property on the other hand.

Still another issue is related to the use of beam splitters for the generation of quantum random number sequences
\cite{svozil-qct,rarity-94,zeilinger:qct,stefanov-2000}.
As these beam splitters are essentially reversible gates
-- this can be demonstrated by serially combining two of them to a Mach-Zehnder device --
described by unitary transformations \cite{green-horn-zei},
it remains unclear exactly how and where the randomness generated ``comes about.''
These issues will be discussed in more detail later.
For the moment it suffices to say that it is not at all clear what kind of quantum certification comes from such devices \cite{PhysRevA.82.022102},
although they are hailed as representing absolute, irreducible randomness \cite{zeil-05_nature_ofQuantum}.
Again, this is related to the issue of (im)proper interpretation of detector clicks.

\section{Do measurements exist?}

Without measurement, the late Schr\"odinger polemically noted \cite{schroedinger-interpretation}, quantum theorists would
be continuously troubled that their
``surroundings rapidly turning into a quagmire, a sort of a featureless jelly or plasma,
all contours becoming blurred, [they themselves] probably becoming jelly fish.''
Time and again, the measurement problem has been
presented \cite{wheeler-Zurek:83} and proclaimed solved in various ways (e.g. \cite{RevModPhys.75.715}),
yet it pops up every now and then as remaining an open question.
The late Bell's sarcastic observation  \cite{bell-a} that most quantum physicists appear to be
``why bother?'ers'' \cite{dirac-noworries} who might just as well neglect the
measurement problem by considering it solved ``for all practical purposes'' (FAPP)
resonates well with Feynman's advise
to (young, as he seemed to have assumed that the older ones are sufficiently brainwashed anyway)
physicists \cite[p.~129]{feynman-law} that, while ``$\ldots$ nobody
understands quantum mechanics,'' to stop thinking about these issues ``$\ldots$ if  you  can possibly avoid it.
because you will get
`down the drain', into a blind alley from which nobody has
yet escaped. Nobody knows how it can be like that.''  --
Even if one grants Feynman some rhetoric benefits,
the appeal to ``stop thinking''
is a truly remarkable advice from one of the most popular scientists of his time!



If one insists in an exact treatment of these issues, one is soon meddling with the following problem,
which has already disturbed Schr\"odinger \cite{schrodinger,schroedinger-interpretation}, Everett \cite[p.~454]{everett}, and Wigner \cite[p.~173]{wigner:mb}:
{\em if quantum mechanics is universally valid, and if it is governed by unitary, reversible, one-to-one evolution,
how does irreversibility arise from reversibility?}


That problem is notorious also for classical statistical mechanics, but in quantum mechanics it seems
to appear even more pressing.
It has two aspects; one formal and the other empirical.

\subsection{`Emergence' of many-to-one from one-to-one functions?}

The formal aspect is related to question of whether it is possible to obtain an irreversible many-to-one function
from reversible one-to-one (injective) functions.
Pointedly stated: ``how can many-to-one-ness possibly `emerge' from one-to-one-ness?''

More specifically, as unitary (quantum) transformations (in-between measurements) are bijections,
the question is if any many-to-one function (modelling the ``state reduction''
or the ``wavefunction collapse'')  can be constructed from bijections.

I believe that any exact solution of the quantum measurement problem would
have to come up with some kind of scenario addressing these formal questions.
I also believe that, under very mild side assumptions,
(possibly excluding infinite limits),
no formal `emergence' of many-to-one-ness  from one-to-one-ness can exist.

Actually, consider the following very elementary proof by contradiction.
Suppose (wrongly) a hypothetical many-to-one function $h(x)=h(y)$ for $x\neq y$ exists which would somehow
`emerge' from injective functions.
Any such function would have to originate from the domain of one-to-one functions such that,
for all functions $f$ of this class,  $x\neq y$ implies  $f(x)\neq f(y)$
-- or, equivalently, the contrapositive statement (provable by comparison of truth tables \cite[chapter~3]{Daepp})
$f(x) = f(y)$ implies $x = y$,  a clear contradiction with the assumption.

No (finite) functional concatenation or injective (distinctness preserving) operation will be able to change that.
Indeed, any bijection from the set of reals (or complex numbers) to the set of real (or complex) numbers
is called a {\em permutation.}
The permutation group is a group whose elements are the bijections from a given set to itself,
and whose group operation is the composition of bijections.
Injections form semigroups because their inverse need not exist.


Hence, I believe, that the hypothesis or believe that
irreversible measurements can be reconciled with the assumption that the
(unitary) quantum evolution is universally valid will eventually
be identified as being what it is -- an illusory {\em red herring.}
The time when this will be acknowledged by the physics community is determined
by the subjective willingness of physicists to acknowledge
and not neglect an otherwise rather trivial issue.


\subsection{`Undoing' measurements}

Another, more empirical, question is:
``is there a principle (and not only practical or technological FAPP) limit to `undo' measurements?''

I believe that there is none,
as various quantum erasure experiments seem to indicate
\citep{PhysRevD.22.879,PhysRevA.25.2208,greenberger2,Nature351,Zajonc-91,PhysRevA.45.7729,PhysRevLett.73.1223,PhysRevLett.75.3783,hkwz}.
And thus,
what one calls ``measurement,''
as well as the cut between observer and object, is purely conventional \cite{svozil-2001-convention}.

Because, as has been already argued by  Everett \cite[p.~454]{everett}, and Wigner \cite[p.~173]{wigner:mb} and alluded to earlier,
even if one has located such a {\em Cartesian cut}, and drew the line between object and measurement apparatus,
this divide is whisked away into thin air by merely considering a larger, quantized system containing both the
aforementioned  object  and measurement apparatus.


\section{What constitutes a pure quantum state?}

Arguably the most fundamental property of a Hilbert space is its dimension.
For quantized systems, the (minimal) dimension required is the number of possible, mutually exclusive outcomes.
That is, in a generalized beam splitter scenario \cite{rzbb} which serves as a robust analogue of any quantized system,
whenever the number of (input and output) ports is $d$, so is the dimension of the Hilbert space modelling that beam splitter.
Operationally, in an ideal setup (no losses in the beam paths {\it et cetera}) one (and only one) detector, out of an array of $d$ detectors
(located after the output ports) clicks.

Any such system represents a maximal knowledge about a quantized system, which (ideally) is certain;
as well as a complete control of the preparation.
Here the terms ``maximal'' and ``complete'' refer to the fact that there is no operational procedure which could improve either the magnitude of definite knowledge,
or the precision of the preparation.

At the same time, any such an array of $d$ detectors can be represented by some single yet arbitrarily oriented orthonormal basis
containing $d$ orthonormal vectors in $d$-dimensional Hilbert space  \cite{Schwinger.60}.
It is therefore suggested that {\em a pure state is characterized by  an orthonormal basis} \cite{svozil-2002-statepart-prl}.
Synonymously one could also define a pure state as
(i) a {\em maximal operator}  from which all commuting operators can be functionally derived
\cite[sect.~84]{halmos-vs}, or
(ii) as a {\em context, subalgebra} or {\em block} \cite{svozil-2006-omni,svozil-2008-ql}, or
(iii) as a unitary transform associated with that orthonormal basis.

That maximal operators represent orthormal bases can be seen by constructing some non-degenerate spectral form
of some maximal operator
$\textsf{\textbf{M}} =\sum_{i=1}^d \lambda_i\textsf{\textbf{E}}_{i}$
by decomposing it into the projectors
$\textsf{\textbf{E}}_{i}= \vert {\bf e}_i \rangle \langle {\bf e}_i \vert$
corresponding to the orthonormal basis
${\mathfrak B} =\{ \vert {\bf e}_1 \rangle, \vert {\bf e}_2 \rangle, \ldots , \vert {\bf e}_d \rangle \}$
mentioned earlier; with mutually distinct eigenvalues $\lambda_i$.


Identifying a {\em context, subalgebra} or {\em block} with an orthonormal coordinate system is just another, algebraic, way of representing that basis.

Finally, the unitary operator
$\textsf{\textbf{U}}_{{\mathfrak B}}=   \sum_{i=1}^d  \vert {\bf e}_i\rangle \langle {\bf e}_i \vert $
(or, alternatively,   $\textsf{\textbf{U}} = e^{i \textsf{\textbf{M}}}$)
is still another representation of the orthonormal basis ${\mathfrak B}$.
Note that, in this picture, the composition property
for unitary operators (meaning that the concatenation of unitary operators yields a unitary operator)
assures the isometric quantum evolution  in a formally homogenuous way by considering isometries throughout the entire calculation (including quantum states).
Indeed, in such a framework there is no difference between a state (preparation), and its evolution and ``measurement'' (represented by another
unitary matrix).
In that sense, a (pure) state, its evolution and ``measurement'' are all but different ways of perceiving that state from different epistemic
viewpoints \cite{DallaChiara-epistemic}.

Note that projectors have a {\it dichotomic} meaning: they stand for an incomplete (in the sense that they are just part of
some context or state) preparation; and at the same time they serve as formal representations of (incomplete) observables.

This definition of a pure state in terms of a single orthonormal basis/maximal operator/context
is different from the usual definition in terms of a (unit) vector $\vert {\bf x}\rangle$
spanning a one-dimensional subspace ${\mathfrak M}_{\bf x}= \{ \vert {\bf y}\rangle \mid  \vert {\bf y}\rangle= c  \vert {\bf x}\rangle; \; c\in {\mathbb C} \}$
of the Hilbert space ${\mathfrak H}$; or the projector $\textsf{\textbf{E}}_{\bf x}= \vert {\bf x} \rangle \langle {\bf x}\vert$ associated with that
one-dimensional subspace by $\textsf{\textbf{E}}_{\bf x}{\mathfrak H}={\mathfrak M}_{\bf x}$:
Whereas pure states are usually just defined as {\em single elements} of a basis,
we propose to define it {\em via} the {\em entire basis.}
This includes not only the basis vector associated with the detector which clicks,
but all other (mutually exclusive observable) detectors which {\em do not click};
so in this sense it includes the full specification of all positive and null results recordable simultaneously.

Besides encoding the maximal knowledge, this definition has a variety of other advantages:
for dimensions higher than two it removes the arbitrariness of the choice of the other basis vectors, as compared to the
case in which merely a single such vector is defined to be the representation of the pure state;
and it already includes the notion of context, thereby representing it as a single, unified observable
corresponding to the associated maximal operator.

\section{The epistemic or ontic (non-)existence of mixed states}

It is rather evident that FAPP all physical state preparations and measurements are incomplete and thus give rise to
{\em epistemic uncertainties} about a prepared and measured (pure) state.
An entirely different issue is whether ``mixed states can be ontic;'' that is,
whether it is possible to ``produce truly'' mixed states which
are not merely disguised pure states.

Again, from a purely formal point of view,
it is impossible to obtain a mixed state from a pure one.
Because again, any unitary operation amounts to a mere basis transformation \cite{Schwinger.60},
and this cannot give rise to any increase in stochasticity or ``ignorance.''
Since the generation of ``ontologically mixed states'' from pure ones would require a many-to-one functional mapping,
we conclude that, just as irreversible measurements, genuine ``ontological mixed states'' originating from pure states cannot exist.
Therefore, any ontological mixed state has to be either carried through from previously existing mixed states; if they exist.

I would like to challenge anyone not convinced by the formal argument involving bijections and the permutation group to come
up with a concrete experiment that would ``produce'' a mixed state from a pure one.





\section{Epistemic or ontic existence of pure but entangled and/or coherent states}

It might not be too unreasonable to state that Schr\"odinger's position articles on the ``new quantum mechanics''
carrying the same name as this presentation
\cite{schrodinger}
raised two main concerns about
(i) the {\em coherence} of classically distinct and mutually exclusive states, such as  $\vert 0 \rangle + \vert 1 \rangle$;
such a superposition was called ``quantum quagmire'' or ``quantum jelly'' by the late Schr\"odinger
and
(ii) the entangled nature of certain states \cite{schroedinger-interpretation}; i.e., the possibility
to encode classical information ``across'' different quanta \cite{zeil-99}.
More formally, for two-quantum states with two states per quantum, by comparing the coefficients of
the {\em product state}
$
%\begin{equation}
\left(
a_0 \vert 0 \rangle + a_1 \vert 1 \rangle
\right)
\otimes
\left(
b_0 \vert 0 \rangle + b_1 \vert 1 \rangle
\right)
=
a_0b_0 \vert 00 \rangle + a_0b_1 \vert 01 \rangle
+
a_1b_0 \vert 10 \rangle + a_1b_1 \vert 11 \rangle
%\end{equation}
$
with the most general
two-quantum state
\begin{equation}
\vert \Psi\rangle =
\alpha_{00} \vert 00 \rangle + \alpha_{01}  \vert 01 \rangle
+
\alpha_{10}  \vert 10 \rangle + \alpha_{11}  \vert 11 \rangle
,
\label{2012-psqm-e2}
\end{equation}
in order to represent  a non-entangled (product) state,
those coefficients must satisfy
$
%\begin{equation}
\alpha_{00}\alpha_{11} = \alpha_{01} \alpha_{10}
%\end{equation}
$,
otherwise the state  cannot be multiplicatively ``decomposed;''
and thus the ``individual quanta'' must not be considered individually
\cite[pp.~17-18]{mermin-07}.

Note that any particular two-quantum wave function $\vert \Psi\rangle$ defined in Eq.~\ref{2012-psqm-e2}
can also be considered as a coherent state with respect to the basis
${\mathfrak B}=\{
\vert 00 \rangle, \vert 01 \rangle , \vert 10 \rangle ,  \vert 11 \rangle
\}$,
and hence can be ``rotated'' into a new basis  ${\mathfrak B}'=\{\vert \Psi\rangle ,\ldots
\}$  by a unitary transformation \cite{Schwinger.60}
$
\textsf{\textbf{U}}_{{\mathfrak B}' \rightarrow {\mathfrak B}}
=
\sum_{i=1}^4
\vert {\bf e}_i\rangle
\langle {\bf f}_i \vert
$,
where
$\vert {\bf e}_i\rangle \in {\mathfrak B}$ and
$\vert {\bf f}_i\rangle \in {\mathfrak B}'$.
In this new basis ${\mathfrak B}'$, $\vert \Psi '\rangle = \textsf{\textbf{U}}\vert \Psi\rangle = \vert 00 \rangle$ is no longer ``entangled,''
as the cofficients in that basis are $\alpha_{00}'=1$, and $\alpha_{11}' = \alpha_{01}' = \alpha_{10}' = 0$.
Of course, this new ``epistemic viewpoint'' corresponding to  ${\mathfrak B}'$ \cite{DallaChiara-epistemic} would not be localized any longer.


\section{The (non-)existence of quantum value indefiniteness and its purported ``resolution'' by quantum contextuality}

The Kochen-Specker theorem (and related constructions involving non-separable sets of two-valued states \cite[$\Gamma_3$]{kochen1}),
as well as other arguments (e.g. Bell- and Greenberger-Horne-Zeilinger type constructions \cite{mermin-93})
reveal that it is impossible for certain even finite observables to simultaneously co-exist -- in such a way
that
(i) different observables (propositions)  in the same block (state) behave classical; and
(ii) everywhere in this configuration of observables (propositions), an observable occurring in some particular but arbitrary context (block, basis, subalgebra)
must have precisely the same (truth) values
as that same observable in different contexts \cite{svozil-2008-ql}.
Alternatively one could say that a global consistent truth table (or value assignment)
cannot exist for certain collections of observables \cite{peres222,svozil_2010-pc09,svozil-2011-enough}.

Rather than assuming the most obvious conjecture that certain observables cannot simultaneously co-exist,
in the literature \cite{bohr-1949,bell-66,hey-red,redhead}
this is mostly interpreted as indication for {\em contextuality};
that is, as somehow ``implying'' that a certain observable may yield different outcomes,
depending on what other observables are measured alongside of it.
Contextuality is not present on a statistical level -- a quantum operator (projector) is context independent.
Rather, contextuality only ``reveales'' itself for individual events \cite{svozil:040102,svozil_2010-pc09,svozil-2011-enough}.
Maybe this ``contextual resolution'' of the issues raised by Kochen and Specker and others
has been the preferred hypothesis because it
(i) allows one to maintain a revised realism by believing in the ``existence'' of observables;
regardless of preparation and measurement  \cite{stace1};
(ii) conforms with the experience that,  any particular one of the observables occurring in the
Kochen-Specker proof {\em can be actually ``measured;'' regardless of the state prepared}.
That is,
any such type of ``measurement'' results in an ``answer'' or ``outcome.''


\section{The ontological single pure state conjecture}

It needs, however, not be taken for granted that the outcome of a measurement
reflects, or is a sole function of, the state prepared.
Rather, it appears to be much more natural to assume that only in the case of a
perfect match between preparation and measurement (context), the results are identical.
In all other cases, the measurement apparatus, through its many degrees of freedom,
might introduce a kind of ``stochasticity''
\cite{svozil-2003-garda,svozil-2011-jesuitlies}.
But this kind of FAPP ``stochasticity'' is epistemic and not ontic;
although it is often believed to originate from
essentially nothing \cite{zeil-99,zeil-05_nature_ofQuantum}, in which case it must be irreducible.


I propose here what can be called the {\em ontological single pure state conjecture:}
at any given time the system is in a definite pure state.

The potential infinity of (counterfactual) measurement outcomes
are (i) either precisely determined in the case when there is a perfect {\em match} between the context or state prepared on the one hand,
and the context or propositions measured on the other hand;
(ii) or, in the case of a context mismatch, epistemic stochasticity enters through the environment (measuring apparatus)
translating \cite{svozil-2003-garda} between the information prepared and the question asked.
This could, at least in principle, be tested by changing the capability of the environment to translate contexts.

While it may be impossible to consider the entire state of the Universe,
one could still maintain that it makes sense to consider certain quasi-isolated parts of it.
And while strictly (microcanonically) isolated quantized systems which would, for instance,
justify a finite dimensionality of the Hilbert space,
do not exist, some systems can be considered isolated enough FAPP; although one has to keep in mind the bigger picture, and thus
at least conceptually include the environment of such a FAPP-isolation.
This would, in a certain sense, justify measurements.
But one should always keep in mind that there cannot be any measurement in total isolation, as
for totally isolated systems there is no information/energy/quantum flow in and out of their boundaries.
And once there is an exchange through the cut or boundary,
there is no sharp distinction between observer and object
\cite{svozil-2001-convention}.



\section{Is the best interpretation of the quantum formalism its non-interpretation?}

Time and again, the quantum theorist is reminded by prestigious peers that the best
interpretation of quantum mechanics appears to be its non-interpretation \cite{Fuchs-Peres}.
Indeed, as has been pointed out earlier, it has become almost fashionable to discredit interpretation
and causality in the quantum domain.
Already Sommerfeld warned his students not to get
into these issues,
and not long ago scientists working in that field
have had a hard time not to appear as ``quacks'' \cite{clauser-talkvie}.

In favour of non-interpretation one could maintain that
interpretation is to the formalism what a
scaffolding in architecture and
building construction is to the completed building.
Very often the scaffolding has to be erected because
it is an indispensable part of the building process.
Once the completed building is in place, the scaffolding is torn down and
the {\em opus} stands in its own full glory.
No need for auxiliary scaffold any longer.

However, it is my conviction
that nobody in quantum mechanics gets along without any interpretation
and intuition.
Indeed, a {\it Haltung} towards non-interpretation of the formalism would inevitably cripple further physical progress.
Because by historic analogy we can suspect that, as time goes by, so will be our physical formalisms \cite{lakatosch},
and that, as has been expressed by Greenberger \cite{greenberger-talk-99}, four hundred years from now
all our present physical theories will appear ``laughable.''

Yet we are confronted with a situation in which an orthodoxy tries to supress and avoid thinking about the ``how'' \cite[p.~129]{feynman-law},
or at least advises ``not worry too much''  \cite{dirac-noworries},
while at the same time expressing the opinion that certain events occur {\it ex nihilo} (out of nothing), fundamentally
inexpicably, and irreducibly random \cite{zeil-05_nature_ofQuantum}.
This, I believe, is tantamount to dogmatism,
and contradictory to all rationalistic principles on which scientific progress thrives.

It is my conviction that
without interpretations, the application of the formalism would be restricted to
automatic proofing, to an ``a thousand monkeys typing-away scenario.''
Quantum mechanics would become another application of (rather elementary) linear algebra.
Hence   there will be no significant scientific progress in this area without
attempts to give meaning to what is formalized.
Therefore, interpretation should not be discredited outrightly,
but considered wisely with {\em evenly-suspended attention}
\cite{Freud-1912}.

\section{Conclusion}

Let me finally discuss the situation with regard to certain practical implementations of quantum random number generators.
Although, as has been mentioned earlier,
quantum randomness, or at least Turing incomputability from quantum coin tosses involving beam splitters,
is often judged to be the ``ultimate'' randomness,
and even ``the (nihilistic) message of the quantum,''
some questions with regard to the conceptual foundations arise.
For instance, an ideal beam splitter can be modelled by a unitary transformation,
which is a (Laplacian type causal) one-to-one isometric transformation in Hilbert space.

No information is lost in such devices, which are (at least ideally) incapable of irreversibility.
Operationally this can be demonstrated by the serial composition of two beam splitters,
which effectively renders a Mach-Zehnder interferometer yielding the original quantum state in its output ports.

One may speculate that the ``randomness resides'' in the (classical) detection of,
say, the photons, after the half-silvered mirror.
But, as has for instance been pointed out by Everett, this is (self-)delusional,
as quantum mechanics,
and in particular the unitary quantum evolution of all components involved in the detection,
at least in principle must hold uniformly.

The  situation discussed is also related to the issue of  ``quantum jellification''
posed by the late Schr\"odinger with regards to the coherent superposition
and co-existence of classically contradictory physical states.
But how can irreversibility possibly ``emerge'' from reversibility?
Besides "for-all-practical-purposes" being effectively irreversible,
in principle there does not seem to exist any conceivable unitary route to irreversibility;
at least if quantum theory is universally valid.

We have therefore posed two simple questions; one formal and one empirical:
the formal one is about the (im)possible ``emergence'' of noninvertible mappings from invertible ones;
the empirical question is about the (non-)existence of principal bounds on reconstructing a state after ``measurement.''

Thus, in view of a causal, bijective Laplacian type quantum evolution, what guarantees a quantum coin toss, say,
on a half-silvered mirror, to perform irreducibly random,
and where exactly does this randomness originate?
The issues appear as marred today as in Wigner's times \cite{wigner:mb},
but they are much more pressing now, as the associated technologies \cite{zeilinger:qct,stefanov-2000}
have been deployed
for experiments \cite{wjswz-98}
as well as cryptanalysis and industry by various spin-offs.

One possibility to circumvent this conundrum is by postulating
(i) that at every instant, only a single state (or context) exists;
and
(ii) through {\em context translation,}
in which a mismatch between the preparation and the measurement results in the ``translation''
of the original information encoded by a quantum system into the answer requested,
noise is introduced by the many degrees of freedom of a suitable ``quasi-classical'' measurement apparatus.
But this would be an altogether different source of randomness than irreducibly creation {\it ex nihilo} (``out of nothing'')
that is favoured by the present quantum orthodoxy.

\begin{acknowledgements}
 This research has been partly supported by FP7-PEOPLE-2010-IRSES-269151-RANPHYS.
\end{acknowledgements}

\bibliography{svozil}

\end{document}
