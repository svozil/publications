\documentclass[prl,preprint,amsfonts,showpacs,showkeys]{revtex4}
%\documentclass[pra,showpacs,showkeys,amsfonts]{revtex4}
\usepackage{graphicx}
%\documentstyle[amsfonts]{article}
\RequirePackage{times}
%\RequirePackage{courier}
\RequirePackage{mathptm}
%\renewcommand{\baselinestretch}{1.3}

\begin{document}
%\sloppy



\title{Communication cost of breaking the Bell barrier}

\author{Karl Svozil}
\email{svozil@tuwien.ac.at}
\homepage{http://tph.tuwien.ac.at/~svozil}
\affiliation{Institut f\"ur Theoretische Physik, University of Technology Vienna,
Wiedner Hauptstra\ss e 8-10/136, A-1040 Vienna, Austria}


\begin{abstract}
Correlations in an Einstein-Podolsky-Rosen-Bohm experiment can be made stronger than quantum correlations by allowing a single bit of classical communication between the two sides of the experiment.
\end{abstract}

\pacs{03.67.Hk,03.65.Ud,03.65.Ta,03.67.Mn}
\keywords{Quantum information, quantum communication, quantum teleportation}





\maketitle


From an operational point of view,
the nonlocal quantum correlations giving rise to
violations of Bell-type inequalities amount to the fact that
certain joint events at spacelike separated locations
occur with greater or smaller frequencies than can possibly be
expected from classical, local realistic models.
Two detectors at different locations register pairs
of particles or particle properties
more frequently or infrequently as can be explained by
the usual classical assumptions such as value definiteness.

With the rise of quantum algorithms and quantum information theory
\cite{nielsen-book},
the emphasis shifted to the communication cost and to the quantum communication complexity
related to those quantum correlations.
The question of the expense of obtaining quantum-type correlations
from classical systems was stimulated by quantum
\cite{BBCJPW}
and classical \cite{Brassard-Cleve-Tapp,cerf-gisin-massar-00,cerf-gisin-massar-pop-04,bru-gis-sca-04}
teleportation.
In a recent Letter~\cite{toner-bacon-03}, Toner and Bacon,
based on Refs. \cite{schatten-93,cerf-gisin-massar-00}, argue
that classical systems could mimic quantum systems by reproducing the
cosine law for correlation functions with the exchange of just one bit of classical information.

The formal coincidence of the quantum correlation function with classical
correlations augmented with the exchange of a single classical bit might indicate a deep structure
in quantum correlations.
One could, for instance, speculate that two-partite quantum systems may be capable of
conferring a single bit, a property which is reflected by the cosine form of the expectation function.
In what follows it will be argued that, while this may still be the case for the Toner-Bacon
protocol~\cite{toner-bacon-03}, in general the exchange of a single classical bit can give rise to
stronger than quantum correlations.



Since the systems discussed are entirely planar, whenever possible,
polar angles are used to represent the associated unit vectors.
The same symbols denote polar angles (without hat)
and the associated vectors (with hat).
Consider two correlated and spatially separated classical subsystems
sharing common
directions $ { \lambda }_i$, $i=1,\ldots$
which are chosen independently of each other and
 are distributed uniformly.
All parameters $ {\lambda }_i$ are assumed to be identical on each one of the two subsystems.
There are two measurement directions ${ a}$ and ${ b}$
of two
dichotomic observables with values ``-1'' and ``1''
at two spatially separated locations.
The measurement direction ${a}$ at ``Alice's location''
is unknown to an observer ``Bob'' measuring ${ b}$ and {\it vice versa}.
A two-particle correlation function $E(\theta )$
with $\theta =\vert {a} - { b}\vert $
is defined by averaging the product of the outcomes $O({ a})_i, O({ b} )_i\in {-1,1}$
in the $i$th experiment; i.e.,  $E(\theta )=(1/N)\sum_{i=1}^N O({ a})_i O({ b})_i$.


%%%%%%%%%%%%%%%%%%%%%%%%%%%%%%%%%%%%%%%%%%%%%%%%%%%%%%%%%%

The following nonadaptive, memoryless protocols
could give rise to stronger-than-quantum correlations
by allowing the exchange of a single bit per experiment.
%Thereby, planar configurations will be considered.
The protocols are similar to the one discussed by
Toner and Bacon~\cite{toner-bacon-03}, but require only a single share ${ \lambda }$,
and an additional direction ${ \Delta} (\delta )$, which is obtained by rotating
${\hat \lambda }$ clockwise around the origin by an angle $\delta$ which is a constant
shift for all experiments.
That is, ${\Delta} (\delta )=\lambda +\delta$.
Alice's observable is given by
$\alpha  = {\rm sgn}({\hat a} \cdot {\hat \lambda } )
={\rm sgn}\left[\cos ({a} - { \lambda } )\right]$.
The bit communicated by Alice is given by
$
c(\delta) =
{\rm sgn}({\hat a} \cdot {\hat \lambda } )
{\rm sgn}\left[{\hat a} \cdot {\hat \Delta} (\delta)\right]=
{\rm sgn}\left[\cos ({ a} - { \lambda } )\right]
{\rm sgn}\cos \left[{ a} - { \Delta} (\delta)\right]
$.
Bob's observable is  defined by
$\beta (\delta )=  {\rm sgn}[{\hat b} \cdot ({\hat \lambda } +c(\delta){\hat \Delta} (\delta))]$.
This protocol becomes Toner and Bacon's if $\delta$
is allowed to vary randomly, with uniform distribution.

The strongest correlations are obtained for $\delta =\pi /2$; i.e., in the case
where the two directions associated with ${ \lambda }$ and ${ \Delta} ={ \lambda }+ \pi /2$
are orthogonal
and the information obtained by $c (\pi /2)$ is about the location of ${ a}$
within two opposite quadrants.
Let $H(x)$ stand for the Heaviside step function of $x$.
The effective shift in the parameter direction
${\hat \lambda }\longrightarrow {\hat \lambda } \pm {\hat \lambda }^\perp$
yields a correlation function of the form
\begin{equation}
E\left(\theta ,\delta = {\pi \over 2}\right)
=
H\left(\theta - {3 \pi\over 4}\right)
 - H\left({\pi \over 4} - \theta \right)  -
  2\left(1 - {2\over \pi}\theta \right)
H\left(\theta - {\pi\over 4}\right) H\left({3 \pi\over 4} - \theta \right)
.
\label{e-2004-brainteaser-2a}
\end{equation}
To obtain a better understanding for the shift mechanism, in
Fig.~\ref{2004-brainteaser-f0} a configuration is drawn which, without the shift
${\hat \lambda } \longrightarrow {\hat \lambda }-{\hat \lambda }^\perp$,
${\rm sgn}({\hat b} \cdot {\hat \lambda })={\rm sgn} \cos({b} - { \lambda })$
would have contributed the factor $-1$.
The shift
results in
a positive contribution ${\rm sgn}[{\hat b} \cdot ({\hat \lambda }-{\hat \lambda }^\perp )]$
to the expectation value.
This shift mechanism always yields the strongest correlations $\pm 1$ as long as
the angle $\theta$ does not lie between $\pi /4$ and $3 \pi /4$.
\begin{figure}[htbp]
  \centering
%TexCad Options
%\grade{\off}
%\emlines{\off}
%\beziermacro{\off}
%\reduce{\on}
%\snapping{\off}
%\quality{6.00}
%\graddiff{0.01}
%\snapasp{1}
%\zoom{2.00}
\unitlength 1.00mm
\linethickness{0.5pt}
\begin{picture}(63.00,65.84)
%\circle(30.00,30.00){60.00}
\put(30.00,60.00){\line(1,0){1.61}}
\multiput(31.61,59.96)(0.80,-0.06){2}{\line(1,0){0.80}}
\multiput(33.22,59.83)(0.80,-0.11){2}{\line(1,0){0.80}}
\multiput(34.81,59.61)(0.53,-0.10){3}{\line(1,0){0.53}}
\multiput(36.39,59.31)(0.39,-0.10){4}{\line(1,0){0.39}}
\multiput(37.96,58.93)(0.39,-0.12){4}{\line(1,0){0.39}}
\multiput(39.50,58.46)(0.30,-0.11){5}{\line(1,0){0.30}}
\multiput(41.01,57.91)(0.25,-0.11){6}{\line(1,0){0.25}}
\multiput(42.50,57.27)(0.24,-0.12){6}{\line(1,0){0.24}}
\multiput(43.94,56.56)(0.20,-0.11){7}{\line(1,0){0.20}}
\multiput(45.35,55.78)(0.17,-0.11){8}{\line(1,0){0.17}}
\multiput(46.71,54.92)(0.16,-0.12){8}{\line(1,0){0.16}}
\multiput(48.02,53.98)(0.14,-0.11){9}{\line(1,0){0.14}}
\multiput(49.28,52.98)(0.13,-0.12){9}{\line(1,0){0.13}}
\multiput(50.49,51.91)(0.11,-0.11){10}{\line(1,0){0.11}}
\multiput(51.64,50.78)(0.11,-0.12){10}{\line(0,-1){0.12}}
\multiput(52.72,49.59)(0.11,-0.14){9}{\line(0,-1){0.14}}
\multiput(53.74,48.34)(0.12,-0.16){8}{\line(0,-1){0.16}}
\multiput(54.69,47.04)(0.11,-0.17){8}{\line(0,-1){0.17}}
\multiput(55.57,45.69)(0.12,-0.20){7}{\line(0,-1){0.20}}
\multiput(56.37,44.30)(0.10,-0.21){7}{\line(0,-1){0.21}}
\multiput(57.10,42.86)(0.11,-0.25){6}{\line(0,-1){0.25}}
\multiput(57.75,41.39)(0.11,-0.30){5}{\line(0,-1){0.30}}
\multiput(58.33,39.88)(0.10,-0.31){5}{\line(0,-1){0.31}}
\multiput(58.82,38.35)(0.10,-0.39){4}{\line(0,-1){0.39}}
\multiput(59.22,36.79)(0.11,-0.53){3}{\line(0,-1){0.53}}
\multiput(59.54,35.21)(0.12,-0.80){2}{\line(0,-1){0.80}}
\multiput(59.78,33.62)(0.08,-0.80){2}{\line(0,-1){0.80}}
\put(59.93,32.01){\line(0,-1){1.61}}
\put(60.00,30.40){\line(0,-1){1.61}}
\put(59.98,28.79){\line(0,-1){1.61}}
\multiput(59.87,27.18)(-0.10,-0.80){2}{\line(0,-1){0.80}}
\multiput(59.67,25.59)(-0.09,-0.53){3}{\line(0,-1){0.53}}
\multiput(59.39,24.00)(-0.09,-0.39){4}{\line(0,-1){0.39}}
\multiput(59.03,22.43)(-0.11,-0.39){4}{\line(0,-1){0.39}}
\multiput(58.58,20.88)(-0.11,-0.30){5}{\line(0,-1){0.30}}
\multiput(58.05,19.36)(-0.10,-0.25){6}{\line(0,-1){0.25}}
\multiput(57.44,17.87)(-0.12,-0.24){6}{\line(0,-1){0.24}}
\multiput(56.75,16.42)(-0.11,-0.20){7}{\line(0,-1){0.20}}
\multiput(55.98,15.00)(-0.11,-0.17){8}{\line(0,-1){0.17}}
\multiput(55.14,13.63)(-0.11,-0.17){8}{\line(0,-1){0.17}}
\multiput(54.22,12.30)(-0.11,-0.14){9}{\line(0,-1){0.14}}
\multiput(53.24,11.03)(-0.12,-0.14){9}{\line(0,-1){0.14}}
\multiput(52.19,9.81)(-0.11,-0.12){10}{\line(0,-1){0.12}}
\multiput(51.07,8.64)(-0.12,-0.11){10}{\line(-1,0){0.12}}
\multiput(49.89,7.54)(-0.14,-0.12){9}{\line(-1,0){0.14}}
\multiput(48.66,6.51)(-0.14,-0.11){9}{\line(-1,0){0.14}}
\multiput(47.37,5.54)(-0.17,-0.11){8}{\line(-1,0){0.17}}
\multiput(46.03,4.64)(-0.20,-0.12){7}{\line(-1,0){0.20}}
\multiput(44.65,3.82)(-0.20,-0.11){7}{\line(-1,0){0.20}}
\multiput(43.22,3.07)(-0.24,-0.11){6}{\line(-1,0){0.24}}
\multiput(41.76,2.40)(-0.30,-0.12){5}{\line(-1,0){0.30}}
\multiput(40.26,1.81)(-0.31,-0.10){5}{\line(-1,0){0.31}}
\multiput(38.73,1.30)(-0.39,-0.11){4}{\line(-1,0){0.39}}
\multiput(37.18,0.87)(-0.52,-0.11){3}{\line(-1,0){0.52}}
\multiput(35.61,0.53)(-0.53,-0.09){3}{\line(-1,0){0.53}}
\multiput(34.02,0.27)(-0.80,-0.09){2}{\line(-1,0){0.80}}
\put(32.41,0.10){\line(-1,0){1.61}}
\put(30.81,0.01){\line(-1,0){1.61}}
\put(29.19,0.01){\line(-1,0){1.61}}
\multiput(27.59,0.10)(-0.80,0.09){2}{\line(-1,0){0.80}}
\multiput(25.98,0.27)(-0.53,0.09){3}{\line(-1,0){0.53}}
\multiput(24.39,0.53)(-0.52,0.11){3}{\line(-1,0){0.52}}
\multiput(22.82,0.87)(-0.39,0.11){4}{\line(-1,0){0.39}}
\multiput(21.27,1.30)(-0.31,0.10){5}{\line(-1,0){0.31}}
\multiput(19.74,1.81)(-0.30,0.12){5}{\line(-1,0){0.30}}
\multiput(18.24,2.40)(-0.24,0.11){6}{\line(-1,0){0.24}}
\multiput(16.78,3.07)(-0.20,0.11){7}{\line(-1,0){0.20}}
\multiput(15.35,3.82)(-0.20,0.12){7}{\line(-1,0){0.20}}
\multiput(13.97,4.64)(-0.17,0.11){8}{\line(-1,0){0.17}}
\multiput(12.63,5.54)(-0.14,0.11){9}{\line(-1,0){0.14}}
\multiput(11.34,6.51)(-0.14,0.12){9}{\line(-1,0){0.14}}
\multiput(10.11,7.54)(-0.12,0.11){10}{\line(-1,0){0.12}}
\multiput(8.93,8.64)(-0.11,0.12){10}{\line(0,1){0.12}}
\multiput(7.81,9.81)(-0.12,0.14){9}{\line(0,1){0.14}}
\multiput(6.76,11.03)(-0.11,0.14){9}{\line(0,1){0.14}}
\multiput(5.78,12.30)(-0.11,0.17){8}{\line(0,1){0.17}}
\multiput(4.86,13.63)(-0.11,0.17){8}{\line(0,1){0.17}}
\multiput(4.02,15.00)(-0.11,0.20){7}{\line(0,1){0.20}}
\multiput(3.25,16.42)(-0.12,0.24){6}{\line(0,1){0.24}}
\multiput(2.56,17.87)(-0.10,0.25){6}{\line(0,1){0.25}}
\multiput(1.95,19.36)(-0.11,0.30){5}{\line(0,1){0.30}}
\multiput(1.42,20.88)(-0.11,0.39){4}{\line(0,1){0.39}}
\multiput(0.97,22.43)(-0.09,0.39){4}{\line(0,1){0.39}}
\multiput(0.61,24.00)(-0.09,0.53){3}{\line(0,1){0.53}}
\multiput(0.33,25.59)(-0.10,0.80){2}{\line(0,1){0.80}}
\put(0.13,27.18){\line(0,1){1.61}}
\put(0.02,28.79){\line(0,1){1.61}}
\put(0.00,30.40){\line(0,1){1.61}}
\multiput(0.07,32.01)(0.08,0.80){2}{\line(0,1){0.80}}
\multiput(0.22,33.62)(0.12,0.80){2}{\line(0,1){0.80}}
\multiput(0.46,35.21)(0.11,0.53){3}{\line(0,1){0.53}}
\multiput(0.78,36.79)(0.10,0.39){4}{\line(0,1){0.39}}
\multiput(1.18,38.35)(0.10,0.31){5}{\line(0,1){0.31}}
\multiput(1.67,39.88)(0.11,0.30){5}{\line(0,1){0.30}}
\multiput(2.25,41.39)(0.11,0.25){6}{\line(0,1){0.25}}
\multiput(2.90,42.86)(0.10,0.21){7}{\line(0,1){0.21}}
\multiput(3.63,44.30)(0.12,0.20){7}{\line(0,1){0.20}}
\multiput(4.43,45.69)(0.11,0.17){8}{\line(0,1){0.17}}
\multiput(5.31,47.04)(0.12,0.16){8}{\line(0,1){0.16}}
\multiput(6.26,48.34)(0.11,0.14){9}{\line(0,1){0.14}}
\multiput(7.28,49.59)(0.11,0.12){10}{\line(0,1){0.12}}
\multiput(8.36,50.78)(0.11,0.11){10}{\line(1,0){0.11}}
\multiput(9.51,51.91)(0.13,0.12){9}{\line(1,0){0.13}}
\multiput(10.72,52.98)(0.14,0.11){9}{\line(1,0){0.14}}
\multiput(11.98,53.98)(0.16,0.12){8}{\line(1,0){0.16}}
\multiput(13.29,54.92)(0.17,0.11){8}{\line(1,0){0.17}}
\multiput(14.65,55.78)(0.20,0.11){7}{\line(1,0){0.20}}
\multiput(16.06,56.56)(0.24,0.12){6}{\line(1,0){0.24}}
\multiput(17.50,57.27)(0.25,0.11){6}{\line(1,0){0.25}}
\multiput(18.99,57.91)(0.30,0.11){5}{\line(1,0){0.30}}
\multiput(20.50,58.46)(0.39,0.12){4}{\line(1,0){0.39}}
\multiput(22.04,58.93)(0.39,0.10){4}{\line(1,0){0.39}}
\multiput(23.61,59.31)(0.53,0.10){3}{\line(1,0){0.53}}
\multiput(25.19,59.61)(0.80,0.11){2}{\line(1,0){0.80}}
\multiput(26.78,59.83)(1.61,0.09){2}{\line(1,0){1.61}}
%\end
\put(30.00,30.00){\vector(2,1){26.33}}
\put(16.50,57.00){\line(1,-2){26.83}}
\put(30.00,65.84){\makebox(0,0)[cc]{${\hat b}$}}
\put(63.00,46.00){\makebox(0,0)[cc]{${\hat a}$}}
\put(46.00,51.83){\makebox(0,0)[cc]{$+1$}}
\put(7.67,43.01){\makebox(0,0)[cc]{$-1$}}
\put(52.83,17.17){\makebox(0,0)[cc]{$-1$}}
\put(11.83,10.67){\makebox(0,0)[cc]{$+1$}}
%\circle(30.00,30.00){30.00}
\put(30.00,45.00){\line(1,0){1.00}}
\put(31.00,44.97){\line(1,0){0.99}}
\multiput(31.99,44.87)(0.49,-0.08){2}{\line(1,0){0.49}}
\multiput(32.97,44.70)(0.49,-0.11){2}{\line(1,0){0.49}}
\multiput(33.94,44.47)(0.32,-0.10){3}{\line(1,0){0.32}}
\multiput(34.90,44.18)(0.31,-0.12){3}{\line(1,0){0.31}}
\multiput(35.83,43.82)(0.23,-0.10){4}{\line(1,0){0.23}}
\multiput(36.73,43.40)(0.22,-0.12){4}{\line(1,0){0.22}}
\multiput(37.61,42.93)(0.17,-0.11){5}{\line(1,0){0.17}}
\multiput(38.45,42.39)(0.16,-0.12){5}{\line(1,0){0.16}}
\multiput(39.25,41.80)(0.13,-0.11){6}{\line(1,0){0.13}}
\multiput(40.02,41.16)(0.12,-0.12){6}{\line(1,0){0.12}}
\multiput(40.74,40.47)(0.11,-0.12){6}{\line(0,-1){0.12}}
\multiput(41.41,39.74)(0.10,-0.13){6}{\line(0,-1){0.13}}
\multiput(42.03,38.96)(0.11,-0.16){5}{\line(0,-1){0.16}}
\multiput(42.60,38.14)(0.10,-0.17){5}{\line(0,-1){0.17}}
\multiput(43.11,37.28)(0.11,-0.22){4}{\line(0,-1){0.22}}
\multiput(43.57,36.40)(0.10,-0.23){4}{\line(0,-1){0.23}}
\multiput(43.96,35.48)(0.11,-0.31){3}{\line(0,-1){0.31}}
\multiput(44.30,34.54)(0.09,-0.32){3}{\line(0,-1){0.32}}
\multiput(44.57,33.58)(0.10,-0.49){2}{\line(0,-1){0.49}}
\multiput(44.77,32.60)(0.07,-0.49){2}{\line(0,-1){0.49}}
\put(44.91,31.62){\line(0,-1){0.99}}
\put(44.99,30.62){\line(0,-1){1.00}}
\put(45.00,29.63){\line(0,-1){1.00}}
\multiput(44.94,28.63)(-0.06,-0.49){2}{\line(0,-1){0.49}}
\multiput(44.81,27.64)(-0.09,-0.49){2}{\line(0,-1){0.49}}
\multiput(44.62,26.66)(-0.08,-0.32){3}{\line(0,-1){0.32}}
\multiput(44.37,25.70)(-0.11,-0.32){3}{\line(0,-1){0.32}}
\multiput(44.05,24.75)(-0.09,-0.23){4}{\line(0,-1){0.23}}
\multiput(43.67,23.83)(-0.11,-0.22){4}{\line(0,-1){0.22}}
\multiput(43.23,22.94)(-0.10,-0.17){5}{\line(0,-1){0.17}}
\multiput(42.73,22.07)(-0.11,-0.17){5}{\line(0,-1){0.17}}
\multiput(42.18,21.24)(-0.10,-0.13){6}{\line(0,-1){0.13}}
\multiput(41.57,20.45)(-0.11,-0.12){6}{\line(0,-1){0.12}}
\multiput(40.91,19.71)(-0.12,-0.12){6}{\line(-1,0){0.12}}
\multiput(40.20,19.00)(-0.13,-0.11){6}{\line(-1,0){0.13}}
\multiput(39.45,18.35)(-0.13,-0.10){6}{\line(-1,0){0.13}}
\multiput(38.65,17.75)(-0.17,-0.11){5}{\line(-1,0){0.17}}
\multiput(37.82,17.20)(-0.17,-0.10){5}{\line(-1,0){0.17}}
\multiput(36.95,16.71)(-0.22,-0.11){4}{\line(-1,0){0.22}}
\multiput(36.06,16.28)(-0.23,-0.09){4}{\line(-1,0){0.23}}
\multiput(35.13,15.90)(-0.32,-0.10){3}{\line(-1,0){0.32}}
\multiput(34.18,15.59)(-0.32,-0.08){3}{\line(-1,0){0.32}}
\multiput(33.22,15.35)(-0.49,-0.09){2}{\line(-1,0){0.49}}
\put(32.24,15.17){\line(-1,0){0.99}}
\put(31.25,15.05){\line(-1,0){1.00}}
\put(30.25,15.00){\line(-1,0){1.00}}
\put(29.25,15.02){\line(-1,0){0.99}}
\multiput(28.26,15.10)(-0.49,0.07){2}{\line(-1,0){0.49}}
\multiput(27.27,15.25)(-0.49,0.11){2}{\line(-1,0){0.49}}
\multiput(26.30,15.46)(-0.32,0.09){3}{\line(-1,0){0.32}}
\multiput(25.34,15.74)(-0.31,0.11){3}{\line(-1,0){0.31}}
\multiput(24.40,16.08)(-0.23,0.10){4}{\line(-1,0){0.23}}
\multiput(23.49,16.49)(-0.22,0.12){4}{\line(-1,0){0.22}}
\multiput(22.61,16.95)(-0.17,0.10){5}{\line(-1,0){0.17}}
\multiput(21.76,17.47)(-0.16,0.12){5}{\line(-1,0){0.16}}
\multiput(20.94,18.04)(-0.13,0.10){6}{\line(-1,0){0.13}}
\multiput(20.17,18.67)(-0.12,0.11){6}{\line(-1,0){0.12}}
\multiput(19.44,19.35)(-0.11,0.12){6}{\line(0,1){0.12}}
\multiput(18.75,20.07)(-0.11,0.13){6}{\line(0,1){0.13}}
\multiput(18.12,20.84)(-0.12,0.16){5}{\line(0,1){0.16}}
\multiput(17.54,21.65)(-0.11,0.17){5}{\line(0,1){0.17}}
\multiput(17.01,22.50)(-0.12,0.22){4}{\line(0,1){0.22}}
\multiput(16.54,23.38)(-0.10,0.23){4}{\line(0,1){0.23}}
\multiput(16.13,24.29)(-0.12,0.31){3}{\line(0,1){0.31}}
\multiput(15.78,25.22)(-0.10,0.32){3}{\line(0,1){0.32}}
\multiput(15.50,26.18)(-0.11,0.49){2}{\line(0,1){0.49}}
\multiput(15.27,27.15)(-0.08,0.49){2}{\line(0,1){0.49}}
\put(15.12,28.13){\line(0,1){0.99}}
\put(15.03,29.13){\line(0,1){1.00}}
\put(15.00,30.12){\line(0,1){1.00}}
\put(15.04,31.12){\line(0,1){0.99}}
\multiput(15.15,32.11)(0.09,0.49){2}{\line(0,1){0.49}}
\multiput(15.32,33.09)(0.12,0.48){2}{\line(0,1){0.48}}
\multiput(15.56,34.06)(0.10,0.32){3}{\line(0,1){0.32}}
\multiput(15.86,35.01)(0.09,0.23){4}{\line(0,1){0.23}}
\multiput(16.23,35.94)(0.11,0.23){4}{\line(0,1){0.23}}
\multiput(16.65,36.84)(0.10,0.17){5}{\line(0,1){0.17}}
\multiput(17.14,37.71)(0.11,0.17){5}{\line(0,1){0.17}}
\multiput(17.68,38.55)(0.12,0.16){5}{\line(0,1){0.16}}
\multiput(18.27,39.35)(0.11,0.13){6}{\line(0,1){0.13}}
\multiput(18.92,40.11)(0.12,0.12){6}{\line(0,1){0.12}}
\multiput(19.62,40.82)(0.12,0.11){6}{\line(1,0){0.12}}
\multiput(20.36,41.49)(0.13,0.10){6}{\line(1,0){0.13}}
\multiput(21.14,42.11)(0.16,0.11){5}{\line(1,0){0.16}}
\multiput(21.97,42.67)(0.17,0.10){5}{\line(1,0){0.17}}
\multiput(22.83,43.17)(0.22,0.11){4}{\line(1,0){0.22}}
\multiput(23.72,43.62)(0.23,0.10){4}{\line(1,0){0.23}}
\multiput(24.64,44.01)(0.31,0.11){3}{\line(1,0){0.31}}
\multiput(25.58,44.33)(0.32,0.09){3}{\line(1,0){0.32}}
\multiput(26.54,44.60)(0.49,0.10){2}{\line(1,0){0.49}}
\multiput(27.52,44.79)(0.49,0.07){2}{\line(1,0){0.49}}
\put(28.51,44.93){\line(1,0){1.49}}
%\end
\put(30.00,30.00){\vector(4,-1){14.33}}
\put(30.00,30.00){\vector(-1,-4){3.58}}
\put(30.00,30.00){\line(1,4){3.63}}
\put(30.00,30.00){\line(-4,1){14.50}}
\put(42.33,33.67){\makebox(0,0)[cc]{$-$}}
\put(37.33,20.50){\makebox(0,0)[cc]{$+$}}
\put(19.17,24.17){\makebox(0,0)[cc]{$-$}}
\put(22.67,40.17){\makebox(0,0)[cc]{$+$}}
\put(47.00,25.67){\makebox(0,0)[cc]{${\hat \lambda}$}}
\put(25.67,12.00){\makebox(0,0)[cc]{${\hat \lambda}^\perp$}}
\put(48.00,44.50){\makebox(0,0)[cc]{${\hat \lambda}-{\hat \lambda}^\perp$}}
\put(30.00,30.02){\vector(3,2){18.60}}
\put(44.73,26.36){\vector(1,4){3.95}}
\put(30.00,30.00){\vector(0,1){30.00}}
\put(0.00,30.00){\line(1,0){60.00}}
\end{picture}
  \caption{Demonstration of the shift mechanism.
Concentric circles represent the measurement directions
${\hat a}$ and ${\hat  b}$ (outer circle), as well as parameter directions
${\hat  \lambda }$ and ${\hat  \lambda }^\perp$ (inner circle)
and their associated projective sign regions.
The four measurement regions spanned by
${\hat a}$ and ${\hat b}$ are indicated by ``$\pm 1$,'' respectively.
Positive and negative octants spanned by
${\hat  \lambda }$ and ${\hat  \lambda }^\perp$ are indiated in the inner circle by ``$\pm$,'' respectively.
In this configuration, the shift ${\hat \lambda } \longrightarrow  {\hat \lambda }-{\hat \lambda }^\perp$
pushes ${\hat \lambda }$ into a positive region.}
\label{2004-brainteaser-f0}
\end{figure}

\begin{figure}[htbp]
  \centering
  \includegraphics[width=80mm]{2004-brainteaser-f2a}
  \caption{Classical and stronger-than-quantum correlation functions
obtained through the exchange of a single
bit in the memoryless regime for values of $\delta \in \{0,
{\pi \over 10},
{\pi \over 5},
{3\pi \over 10},
{2\pi \over 5},
{\pi \over 2}\}$ between $0$ (straigt line) and $\pi /2$ [cf. Eq~(\ref{e-2004-brainteaser-2a})].}
  \label{2004-brainteaser-f2}
\end{figure}
For general $0\le \delta \le \pi /2$, Fig.~\ref{2004-brainteaser-f2} depicts numerical evaluations
which  fit the correlation function
\begin{equation}
E(\theta ,\delta )= \left\{
\begin{array}{ll}
-1 & \text{ for } \;\; 0\le \theta \le {\delta \over 2} ,  \\
-1 +{2\over \pi}(\theta -{\delta \over 2} )&\text{ for } \;\; {\delta \over 2} < \theta \le {1 \over 2}(\pi - \delta) ,   \\
-2(1-{2 \over \pi } \theta ) &\text{ for } \; \; {1 \over 2}(\pi - \delta) < \theta \le {1 \over 2}(\pi   + \delta ) , \\
1+ {2\over \pi }(\theta-\pi +{\delta \over 2} ) &\text{ for } \;\; {1 \over 2}(\pi   + \delta )  < \theta \le \pi - {\delta \over 2} , \\
1 & \text{ for } \;\; \pi - {\delta \over 2} < \theta \le \pi .
\end{array}        \right.
\label{e-2004-brainteaser-2}
\end{equation}
Its domains are depicted in Fig.~\ref{2004-brainteaser-f2b}.
\begin{figure}[htbp]
  \centering
  \includegraphics[width=90mm]{2004-brainteaser-f2b}
  \caption{Domains of the correlation function $E(\theta ,\delta )$.
Consecutive sections from left to right cover the domains
           of Eq.~(\ref{e-2004-brainteaser-2}) from bottom to top:
(1) $0\le \theta \le {\delta \over 2}$,
(2) ${\delta \over 2} < \theta \le {1 \over 2}(\pi - \delta)$,
(3) ${1 \over 2}(\pi - \delta) < \theta \le {1 \over 2}(\pi   + \delta )$,
(4) ${1 \over 2}(\pi   + \delta )  < \theta \le \pi - {\delta \over 2}$,
(5) $\pi - {\delta \over 2} < \theta \le \pi$.
}
  \label{2004-brainteaser-f2b}
\end{figure}
For all nonzero $\delta$, $E(\theta ,\delta )$ correlates stronger than quantized systems
for some values of $\theta$.
With $\delta =\pi /2$, the Clauser-Horne-Shimony-Holt (CHSH) inequality
$
\vert
E({ a} ,{ b} )+
E({ a} ,{ b} ' )+
E({ a}' ,{ b} )-
E({a} ',{ b} ')
\vert
\le 2
$
for ${a} =\pi /2 $, ${a}'=0$, ${b} =\pi /4$, ${b}'= 3\pi /4$
is violated by the maximal algebraic value of 4 \cite{popescu-97b}, a larger value
than the Tsirelson bound for quantum violations $2\sqrt{2}$.
This is due to the fact that, phenomenologically, the strategy allows for
certain joint events to occur with greater or smaller frequencies as can be
expected from quantum entangled state measurements.
For $\delta =0$, the classical linear correlation function
$E(\theta )=  2\theta /\pi -1$ is recovered, as can be expected.

The average over all $0\le \delta \le \pi /2$
yields a very similar, but not identical behaviour
as the   quantum cosine correlation function.
More precisely,
assume two independent,
uniformly distributed shares ${\lambda}_1$ and ${\lambda}_2$, and
a single communicated bit $H\left[c({\hat  \lambda}_1,{\hat  \lambda}_2)\right]$ with
$c({\hat  \lambda}_1,{\hat  \lambda}_2)=
{\rm sgn}  ({\hat a} \cdot {\hat \lambda}_1){\rm sgn} ({\hat a} \cdot {\hat \lambda}_2)
= {\rm sgn} \cos ({ a} - { \lambda }_1 ){\rm sgn} \cos ({ a} - { \lambda }_2 )$
per measurement of
$
{\rm sgn} ({\hat a}\cdot {\hat \lambda}_1)
{\rm sgn}
[{\hat b}\cdot ({\hat \lambda}_2-c{\hat \lambda}_1)]
$.
The associated correlation function is
\begin{equation}
\begin{array}{cll}
&&E(\theta ) =
{1\over (2 \pi)^2}
\int d {\hat\lambda_1}  d {\hat\lambda_2}  \;
{\rm sgn}
({\hat a}\cdot {\hat \lambda}_1)
{\rm sgn}
[{\hat b}\cdot ({\hat \lambda}_2-c{\hat \lambda}_1)] \\
&&\qquad \qquad \qquad \qquad  =
{1\over 2 \pi^2}
\int d {\hat\lambda_1}  \;
{\rm sgn}
({\hat a}\cdot {\hat \lambda}_1)
\int d {\hat\lambda_2}    \;
{\rm sgn}
[{\hat b}\cdot ({\hat \lambda}_2-{\hat \lambda}_1)].
\end{array}
\label{e-2004-brainteaser-3}
\end{equation}

The elemination of $c$ at the cost of the prefactor $2$
on the right hand side of Eq.~(\ref{e-2004-brainteaser-3}) is achieved by
using the symmetries of the outcomes under the exchange
${\hat \lambda }_1 \longleftrightarrow -{\hat  \lambda }_1$
and
${\hat  \lambda }_2 \longleftrightarrow -{\hat  \lambda }_2$
as outlined in Ref.~\cite{toner-bacon-03}.
Note that, although the correlation function $E(\theta )$ is
nonlocal
(Bob's output depends on $c$, which contains ${ a}$),
after some recasting, despite the prefactor of 2 which accounts for nonlocality,
it appears to be perfectly local,
since it is the product of two sign functions
containing merely ${ a}$ and  (separately) ${b}$, respectively.
(The two parameters $ {\lambda }_1, {\lambda }_2$ are common shares.)

The ${\lambda}_2$ integration can be performed by
arranging
${\hat b}$ along the positive $y$-axis $(0,1)$ and by the parameterization
${\hat \lambda}_1=(\sin t ,\cos t )$
and
${\hat \lambda}_2=(\sin \tau ,\cos \tau )$.
The positive contributions amount to $A_+=2\int_0^t d\tau =2t$;
thus the negative contributions are $A_-= 2\pi -A_+$, and the entire integral
is $(A_+-A_-)/(2\pi )=2t/\pi -1=2\cos^{-1}({\hat b}\cdot {\hat \lambda}_1)/\pi -1$.

For the ${\lambda}_1$ integration,
${\hat a}$ is arranged along the positive $y$-axis $(0,1)$,
${\hat b}$ along $(\sin r,\cos r)$, and
${\hat \lambda}_1$ along $(\sin \tau ,\cos \tau )$.
Since $\cos^{-1} (\cos (x))=\vert x\vert$ for $-1\le x\le 1$,
one obtains for the correlation value
$(4/\pi^2)\int_0^\pi d \tau \;{\rm sgn}(\cos \tau ) \vert \tau -r\vert$.
After evaluating all cases, the ${\hat \lambda}_1$ integration,
for $0\le \theta =\vert{ a}- { b}\vert \le \pi $, yields
\begin{equation}
E(\theta ) =
{4\over \pi^2}
\left[
\left(\theta ^2 -{\pi^2 \over 4}\right)
-2
H\left(\theta -{\pi \over 2}\right)
\left(\theta -{\pi \over 2} \right)^2
\right],
\label{e-2004-brainteaser-4}
\end{equation}
which is plotted in Fig.~\ref{2004-brainteaser-f3}.
\begin{figure}[htbp]
  \centering
 \includegraphics[width=80mm]{2004-brainteaser-f3}
  \caption{The correlation $E(\theta )$ of Eq.~(\ref{e-2004-brainteaser-4}) as a function of
$\theta =\vert{ a}- { b}\vert$ for the memoryless exchange of a single
bit per experiment in the planar configuration.
Note that, although the shape resembles the quantum cosine law,
the function is piecewise quadratic.}
  \label{2004-brainteaser-f3}
\end{figure}
An alternative derivation of Eq.~(\ref{e-2004-brainteaser-4}) is via
$(2/\pi )\int_0^{\pi /2} d\delta \; E(\theta,\delta )$ with $E(\theta,\delta )$
from Eq.~(\ref{e-2004-brainteaser-2}).
The difference to the cosine law obtained by Toner and Bacon~\cite{toner-bacon-03}
is due to the reduced dimensionality of the problem.

%%%%%%%%%%%%%%%%%%%%%%%%%%%%%%%%%%%%%%%%%%%%%%%%%

Let us briefly sketch a protocol requiring memory which,
by the exchange of more than one bit,
could give rise to maximal violations \cite{svozil-krenn}
of the CHSH
inequality.
The protocol  is based on locating and communicating information about
Alice's measurement direction
to Bob, who then rotates his subsystem
(or alternatively his measurement direction)
so as to obtain the desired correlation function.

When compared to the protocol discussed by Toner and Bacon \cite{toner-bacon-03}
or to the memoryless protocoll introduced above,
the adaptive protocols share some similarities.
Both exchange very similar information,
as can for instance be seen by a comparison between $c_2$ above
and the bit
$c= {\rm sgn} \cos ({ a} - { \lambda }_1 ){\rm sgn} \cos ({ a} - { \lambda }_2 )$
exchanged in the  Toner and Bacon protocol.
After the exchange of just a few bits, the adaptive protocols
appear to be more efficient
from the communication complexity point of view.
However, these strategies  require memory.
Operationally, this presents no problem for Alice and Bob,
but if one insist on nonadaptive, single particle strategies,
these protocols must be excluded.


In summary we have found that, as long as adaptive protocols
requiring memory at the receiver side are allowed,
the CHSH inequality can be violated maximally.
Furthermore, we have presented a type of memoryless, nonadaptive protocol
giving rise to stronger-than-quantum correlations which
yields the maximal algebraic violation $4$ of the CHSH inequality,
as compared to the quantum Tsirelson bound $2\sqrt{2}$.
The question  about the effect of an information exchange between entangled quantum subsystems still remains open.





The author acknowledges Akbar Fahmi for pointing out Ref.~\cite{popescu-97b}.


%\bibliography{svozil}
%\bibliographystyle{osa}
%\bibliographystyle{apsrev}
%\bibliographystyle{unsrt}
%\bibliographystyle{plain}


\begin{thebibliography}{10}
\newcommand{\enquote}[1]{``#1''}
\expandafter\ifx\csname url\endcsname\relax
  \def\url#1{\texttt{#1}}\fi
\expandafter\ifx\csname urlprefix\endcsname\relax\def\urlprefix{URL }\fi
\providecommand{\eprint}[2][]{\url{#2}}

\bibitem{nielsen-book}
M.~A. Nielsen and I.~L. Chuang, \emph{Quantum Computation and Quantum
  Information} (Cambridge University Press, Cambridge, 2000).

\bibitem{BBCJPW}
C.~H. Bennett, G.~Brassard, C.~Cr{\'{e}}peau, R.~Jozsa, A.~Peres, and W.~K.
  Wootters, \enquote{Teleporting an unknown quantum state via dual classical
  and {E}instein-{P}odolsky-{R}osen channels,} Physical Review Letters
  \textbf{70}, 1895--1899 (1993).
  \urlprefix\url{http://dx.doi.org/10.1103/PhysRevLett.70.1895}.

\bibitem{Brassard-Cleve-Tapp}
G.~Brassard, R.~Cleve, and A.~Tapp, \enquote{Cost of Exactly Simulating Quantum
  Entanglement with Classical Communication,} Physical Review Letters
  \textbf{83}, 1874--1877 (1999).
  \urlprefix\url{http://dx.doi.org/10.1103/PhysRevLett.83.1874}.

\bibitem{cerf-gisin-massar-00}
N.~J. Cerf, N.~Gisin, and S.~Massar, \enquote{Classical Teleportation of a
  Quantum Bit,} Physical Review Letters \textbf{84}, 2521--2524 (2000).
  \urlprefix\url{http://dx.doi.org/10.1103/PhysRevLett.84.2521}.

\bibitem{cerf-gisin-massar-pop-04}
N.~J. Cerf, N.~Gisin, S.~Massar, and S.~Popescu, \enquote{Quantum entanglement
  can be simulated without communication,} Physical Review Letters \textbf{84},
  2220,403 (2005). \eprint{quant-ph/0410027},
  \urlprefix\url{http://dx.doi.org/10.1103/PhysRevLett.94.220403}.

\bibitem{bru-gis-sca-04}
N.~Brunner, N.~Gisin, and V.~Scarani, \enquote{Quantum non-locality and
  entanglement are different resources,} New J. Phys. \textbf{7}, 88 (2005).
  \eprint{quant-ph/0412109},
  \urlprefix\url{http://dx.doi.org/10.1088/1367-2630/7/1/088}.

\bibitem{toner-bacon-03}
B.~F. Toner and D.~Bacon, \enquote{Communication Cost of Simulating Bell
  Correlations,} Physical Review Letters \textbf{91}, 187,904 (2003).
  \urlprefix\url{http://dx.doi.org/10.1103/PhysRevLett.91.187904}.

\bibitem{schatten-93}
K.~H. Schatten, \enquote{Communication Cost of Simulating Bell Correlations,}
  Physical Review A \textbf{48}, 103--104 (1993).
  \urlprefix\url{http://dx.doi.org/10.1103/PhysRevA.48.103}.

\bibitem{popescu-97b}
S.~Popescu and D.~Rohrlich, \enquote{Action and passion at a distance: an essay
  in honor of professor {A}bner {S}himony,} Boston Studies in the Philosophy of
  Science \textbf{194}, 197--206 (1997). \eprint{quant-ph/9605004},
  \urlprefix\url{http://arxiv.org/abs/quant-ph/9605004}.

\bibitem{svozil-krenn}
G.~Krenn and K.~Svozil, \enquote{Stronger-than-quantum correlations,}
  Foundations of Physics \textbf{28}(6), 971--984 (1998).
  \urlprefix\url{http://dx.doi.org/10.1023/A:1018821314465}.

\end{thebibliography}


\end{document}

 The first bit
$c_1= H [{\rm sgn}({\hat a} \cdot {\hat \lambda } )]$
characterizes the location of ${\hat a}$ within the halfspaces
defined by the subspace orthogonal to $ {\hat \lambda }$; e.g., whether
${\rm sgn}({\hat a} \cdot {\hat \lambda } )$ is positive or negative.
($H$ stands for the Heaviside unit step function.)
The second bit
$c_2= H \left({\rm sgn}({\hat a} \cdot {\hat \lambda } )
{\rm sgn}[{\hat a} \cdot {\hat \Delta} (\pi /2) ]\right)$
characterizes the location of ${\hat a}$
within two opposite
quadrants spanned by $ {\hat \lambda }$ and the subspace orthogonal to $ {\hat \lambda }$.

The third bit of information is about the location of ${\hat a}$ with respect to
the octants formed by dividing the quadrants used to determine the second bit,
and so on.
In general, the $n$th bit
\begin{equation}
c_n= H \left(
\prod_{i=0}^{n-1}{\rm sgn}[{\hat a} \cdot {\hat \Delta} ((\pi /2)^n) ]\right)
\end{equation}
stands for an equipartition of the sphere into
$2^n$ similar slices, which are associated alternatively with the two bit values.
If the protocol stops after communicating the $n$th bit,
then by convention ${\hat a}$ is identified with the direction which is in the middle
of the section sorted out by the previous segments.
%Fig.~\ref{2004-brainteaser-f1} depicts this construction for the planar  configuration considered.
%%%%%%%%%%%%%%%%%%%%%%%%%%%%%%%%%%%%%%%%%%Fig. 1


The communicated information determines Alice's measurement direction ${\hat a}$ for Bob
with arbitrary precision.
In order to obtain a stronger-than-classical correlation,
Bob can use the information about the location ${\hat a}$ by rotating
his subsystem to compensates for the difference
between his measurement direction ${\hat b}$ and ${\hat a}$.
In that way, either perfect correlation or perfect anticorrelation
can be obtained, as Bob rotates his subsystem by either $\delta =\alpha -\beta$
or  $\delta + \pi $, respectively.
This yields an effective shift of ${\hat \lambda }$ by the opposite amount.

The number of required initial experiments increases proportional
to any specified finite precision of the correlation function.
Yet, any finite sequence of bits contributes little to the entire correlation function
$E$ if it is part of large sequence of $M+N$ outcomes in an experiment to determine
$E$: The relative cost $M/(M+N)$ decreases as $N$ increases.

%%%%%%%%%%%%%%%%%%%%%%%%%%

Another, rather obvious,  adaptive strategy  requiring  memory
is to communicate Alice's outcome to Bob.
The individual outcome is just an indication of
whether Alice's measurement direction is located ``above'' or below the
orthogonal subspace of the direction $ {\hat \lambda }$,
which is shared both by Alice and Bob.
As more bits from different ${\hat \lambda }$ directions are communicated,
the direction of ${\hat a}$ can be inferred  with increasing precision.

\section{Rules}

\begin{enumerate}

\item
Existence of parameter directions $ {\hat \lambda }_i$, $i=1,\ldots$;

\item
The directions $ {\hat \lambda }_i$ are chosen independently and are distributed uniformly over the unit sphere (circle);


\item
Two measurement directions ${\hat a}$ and ${\hat b}$

\item
dichotomic observables ``-1'' and ``+1'' at two spatially separated locations

\item
The direction ${\hat a}$ is unknown to an observer measuring ${\hat b}$ and {\it vice versa}.

\end{enumerate}

\section{Classical correlations without communication}

\begin{enumerate}

\item
Planar problem

\item
One direction $ {\hat \lambda }=\cdot {\hat \lambda }_1$

\item
$o_1 = {\rm sgn} ({\hat a} \cdot \cdot {\hat \lambda } )$
and
$o_2 = {\rm sgn} ({\hat b} \cdot \cdot {\hat \lambda } )$

\item
$E({\hat a} ,{\hat b} ) =E(\vert {\hat a} -{\hat b} \vert )=1-{2\vert {\hat a} -{\hat b} \vert \over \pi}$

\end{enumerate}


\begin{figure}
\begin{tabular}{ccc}
%TexCad Options
%\grade{\off}
%\emlines{\off}
%\beziermacro{\off}
%\reduce{\on}
%\snapping{\off}
%\quality{6.00}
%\graddiff{0.01}
%\snapasp{1}
%\zoom{0.80}
\unitlength 0.7mm
\linethickness{0.4pt}
\begin{picture}(60.00,65.67)
%\circle(30.00,30.00){60.00}
\put(30.00,60.00){\line(1,0){1.61}}
\multiput(31.61,59.96)(0.80,-0.06){2}{\line(1,0){0.80}}
\multiput(33.22,59.83)(0.80,-0.11){2}{\line(1,0){0.80}}
\multiput(34.81,59.61)(0.53,-0.10){3}{\line(1,0){0.53}}
\multiput(36.39,59.31)(0.39,-0.10){4}{\line(1,0){0.39}}
\multiput(37.96,58.93)(0.39,-0.12){4}{\line(1,0){0.39}}
\multiput(39.50,58.46)(0.30,-0.11){5}{\line(1,0){0.30}}
\multiput(41.01,57.91)(0.25,-0.11){6}{\line(1,0){0.25}}
\multiput(42.50,57.27)(0.24,-0.12){6}{\line(1,0){0.24}}
\multiput(43.94,56.56)(0.20,-0.11){7}{\line(1,0){0.20}}
\multiput(45.35,55.78)(0.17,-0.11){8}{\line(1,0){0.17}}
\multiput(46.71,54.92)(0.16,-0.12){8}{\line(1,0){0.16}}
\multiput(48.02,53.98)(0.14,-0.11){9}{\line(1,0){0.14}}
\multiput(49.28,52.98)(0.13,-0.12){9}{\line(1,0){0.13}}
\multiput(50.49,51.91)(0.11,-0.11){10}{\line(1,0){0.11}}
\multiput(51.64,50.78)(0.11,-0.12){10}{\line(0,-1){0.12}}
\multiput(52.72,49.59)(0.11,-0.14){9}{\line(0,-1){0.14}}
\multiput(53.74,48.34)(0.12,-0.16){8}{\line(0,-1){0.16}}
\multiput(54.69,47.04)(0.11,-0.17){8}{\line(0,-1){0.17}}
\multiput(55.57,45.69)(0.12,-0.20){7}{\line(0,-1){0.20}}
\multiput(56.37,44.30)(0.10,-0.21){7}{\line(0,-1){0.21}}
\multiput(57.10,42.86)(0.11,-0.25){6}{\line(0,-1){0.25}}
\multiput(57.75,41.39)(0.11,-0.30){5}{\line(0,-1){0.30}}
\multiput(58.33,39.88)(0.10,-0.31){5}{\line(0,-1){0.31}}
\multiput(58.82,38.35)(0.10,-0.39){4}{\line(0,-1){0.39}}
\multiput(59.22,36.79)(0.11,-0.53){3}{\line(0,-1){0.53}}
\multiput(59.54,35.21)(0.12,-0.80){2}{\line(0,-1){0.80}}
\multiput(59.78,33.62)(0.08,-0.80){2}{\line(0,-1){0.80}}
\put(59.93,32.01){\line(0,-1){1.61}}
\put(60.00,30.40){\line(0,-1){1.61}}
\put(59.98,28.79){\line(0,-1){1.61}}
\multiput(59.87,27.18)(-0.10,-0.80){2}{\line(0,-1){0.80}}
\multiput(59.67,25.59)(-0.09,-0.53){3}{\line(0,-1){0.53}}
\multiput(59.39,24.00)(-0.09,-0.39){4}{\line(0,-1){0.39}}
\multiput(59.03,22.43)(-0.11,-0.39){4}{\line(0,-1){0.39}}
\multiput(58.58,20.88)(-0.11,-0.30){5}{\line(0,-1){0.30}}
\multiput(58.05,19.36)(-0.10,-0.25){6}{\line(0,-1){0.25}}
\multiput(57.44,17.87)(-0.12,-0.24){6}{\line(0,-1){0.24}}
\multiput(56.75,16.42)(-0.11,-0.20){7}{\line(0,-1){0.20}}
\multiput(55.98,15.00)(-0.11,-0.17){8}{\line(0,-1){0.17}}
\multiput(55.14,13.63)(-0.11,-0.17){8}{\line(0,-1){0.17}}
\multiput(54.22,12.30)(-0.11,-0.14){9}{\line(0,-1){0.14}}
\multiput(53.24,11.03)(-0.12,-0.14){9}{\line(0,-1){0.14}}
\multiput(52.19,9.81)(-0.11,-0.12){10}{\line(0,-1){0.12}}
\multiput(51.07,8.64)(-0.12,-0.11){10}{\line(-1,0){0.12}}
\multiput(49.89,7.54)(-0.14,-0.12){9}{\line(-1,0){0.14}}
\multiput(48.66,6.51)(-0.14,-0.11){9}{\line(-1,0){0.14}}
\multiput(47.37,5.54)(-0.17,-0.11){8}{\line(-1,0){0.17}}
\multiput(46.03,4.64)(-0.20,-0.12){7}{\line(-1,0){0.20}}
\multiput(44.65,3.82)(-0.20,-0.11){7}{\line(-1,0){0.20}}
\multiput(43.22,3.07)(-0.24,-0.11){6}{\line(-1,0){0.24}}
\multiput(41.76,2.40)(-0.30,-0.12){5}{\line(-1,0){0.30}}
\multiput(40.26,1.81)(-0.31,-0.10){5}{\line(-1,0){0.31}}
\multiput(38.73,1.30)(-0.39,-0.11){4}{\line(-1,0){0.39}}
\multiput(37.18,0.87)(-0.52,-0.11){3}{\line(-1,0){0.52}}
\multiput(35.61,0.53)(-0.53,-0.09){3}{\line(-1,0){0.53}}
\multiput(34.02,0.27)(-0.80,-0.09){2}{\line(-1,0){0.80}}
\put(32.41,0.10){\line(-1,0){1.61}}
\put(30.81,0.01){\line(-1,0){1.61}}
\put(29.19,0.01){\line(-1,0){1.61}}
\multiput(27.59,0.10)(-0.80,0.09){2}{\line(-1,0){0.80}}
\multiput(25.98,0.27)(-0.53,0.09){3}{\line(-1,0){0.53}}
\multiput(24.39,0.53)(-0.52,0.11){3}{\line(-1,0){0.52}}
\multiput(22.82,0.87)(-0.39,0.11){4}{\line(-1,0){0.39}}
\multiput(21.27,1.30)(-0.31,0.10){5}{\line(-1,0){0.31}}
\multiput(19.74,1.81)(-0.30,0.12){5}{\line(-1,0){0.30}}
\multiput(18.24,2.40)(-0.24,0.11){6}{\line(-1,0){0.24}}
\multiput(16.78,3.07)(-0.20,0.11){7}{\line(-1,0){0.20}}
\multiput(15.35,3.82)(-0.20,0.12){7}{\line(-1,0){0.20}}
\multiput(13.97,4.64)(-0.17,0.11){8}{\line(-1,0){0.17}}
\multiput(12.63,5.54)(-0.14,0.11){9}{\line(-1,0){0.14}}
\multiput(11.34,6.51)(-0.14,0.12){9}{\line(-1,0){0.14}}
\multiput(10.11,7.54)(-0.12,0.11){10}{\line(-1,0){0.12}}
\multiput(8.93,8.64)(-0.11,0.12){10}{\line(0,1){0.12}}
\multiput(7.81,9.81)(-0.12,0.14){9}{\line(0,1){0.14}}
\multiput(6.76,11.03)(-0.11,0.14){9}{\line(0,1){0.14}}
\multiput(5.78,12.30)(-0.11,0.17){8}{\line(0,1){0.17}}
\multiput(4.86,13.63)(-0.11,0.17){8}{\line(0,1){0.17}}
\multiput(4.02,15.00)(-0.11,0.20){7}{\line(0,1){0.20}}
\multiput(3.25,16.42)(-0.12,0.24){6}{\line(0,1){0.24}}
\multiput(2.56,17.87)(-0.10,0.25){6}{\line(0,1){0.25}}
\multiput(1.95,19.36)(-0.11,0.30){5}{\line(0,1){0.30}}
\multiput(1.42,20.88)(-0.11,0.39){4}{\line(0,1){0.39}}
\multiput(0.97,22.43)(-0.09,0.39){4}{\line(0,1){0.39}}
\multiput(0.61,24.00)(-0.09,0.53){3}{\line(0,1){0.53}}
\multiput(0.33,25.59)(-0.10,0.80){2}{\line(0,1){0.80}}
\put(0.13,27.18){\line(0,1){1.61}}
\put(0.02,28.79){\line(0,1){1.61}}
\put(0.00,30.40){\line(0,1){1.61}}
\multiput(0.07,32.01)(0.08,0.80){2}{\line(0,1){0.80}}
\multiput(0.22,33.62)(0.12,0.80){2}{\line(0,1){0.80}}
\multiput(0.46,35.21)(0.11,0.53){3}{\line(0,1){0.53}}
\multiput(0.78,36.79)(0.10,0.39){4}{\line(0,1){0.39}}
\multiput(1.18,38.35)(0.10,0.31){5}{\line(0,1){0.31}}
\multiput(1.67,39.88)(0.11,0.30){5}{\line(0,1){0.30}}
\multiput(2.25,41.39)(0.11,0.25){6}{\line(0,1){0.25}}
\multiput(2.90,42.86)(0.10,0.21){7}{\line(0,1){0.21}}
\multiput(3.63,44.30)(0.12,0.20){7}{\line(0,1){0.20}}
\multiput(4.43,45.69)(0.11,0.17){8}{\line(0,1){0.17}}
\multiput(5.31,47.04)(0.12,0.16){8}{\line(0,1){0.16}}
\multiput(6.26,48.34)(0.11,0.14){9}{\line(0,1){0.14}}
\multiput(7.28,49.59)(0.11,0.12){10}{\line(0,1){0.12}}
\multiput(8.36,50.78)(0.11,0.11){10}{\line(1,0){0.11}}
\multiput(9.51,51.91)(0.13,0.12){9}{\line(1,0){0.13}}
\multiput(10.72,52.98)(0.14,0.11){9}{\line(1,0){0.14}}
\multiput(11.98,53.98)(0.16,0.12){8}{\line(1,0){0.16}}
\multiput(13.29,54.92)(0.17,0.11){8}{\line(1,0){0.17}}
\multiput(14.65,55.78)(0.20,0.11){7}{\line(1,0){0.20}}
\multiput(16.06,56.56)(0.24,0.12){6}{\line(1,0){0.24}}
\multiput(17.50,57.27)(0.25,0.11){6}{\line(1,0){0.25}}
\multiput(18.99,57.91)(0.30,0.11){5}{\line(1,0){0.30}}
\multiput(20.50,58.46)(0.39,0.12){4}{\line(1,0){0.39}}
\multiput(22.04,58.93)(0.39,0.10){4}{\line(1,0){0.39}}
\multiput(23.61,59.31)(0.53,0.10){3}{\line(1,0){0.53}}
\multiput(25.19,59.61)(0.80,0.11){2}{\line(1,0){0.80}}
\multiput(26.78,59.83)(1.61,0.09){2}{\line(1,0){1.61}}
%\end
\put(30.00,30.00){\vector(1,2){13.17}}
\put(57.00,16.50){\line(-2,1){54.00}}
\put(46.67,65.67){\makebox(0,0)[cc]{${\hat a}$}}
\put(30.33,48.00){\makebox(0,0)[cc]{$+1$}}
\put(46.33,14.67){\makebox(0,0)[cc]{$-1$}}
\end{picture}
$\quad$
&
$\quad$
%TexCad Options
%\grade{\off}
%\emlines{\off}
%\beziermacro{\off}
%\reduce{\on}
%\snapping{\off}
%\quality{6.00}
%\graddiff{0.01}
%\snapasp{1}
%\zoom{1.00}
\unitlength 0.70mm
\linethickness{0.4pt}
\begin{picture}(64.00,60.00)
%\circle(30.00,30.00){60.00}
\put(30.00,60.00){\line(1,0){1.61}}
\multiput(31.61,59.96)(0.80,-0.06){2}{\line(1,0){0.80}}
\multiput(33.22,59.83)(0.80,-0.11){2}{\line(1,0){0.80}}
\multiput(34.81,59.61)(0.53,-0.10){3}{\line(1,0){0.53}}
\multiput(36.39,59.31)(0.39,-0.10){4}{\line(1,0){0.39}}
\multiput(37.96,58.93)(0.39,-0.12){4}{\line(1,0){0.39}}
\multiput(39.50,58.46)(0.30,-0.11){5}{\line(1,0){0.30}}
\multiput(41.01,57.91)(0.25,-0.11){6}{\line(1,0){0.25}}
\multiput(42.50,57.27)(0.24,-0.12){6}{\line(1,0){0.24}}
\multiput(43.94,56.56)(0.20,-0.11){7}{\line(1,0){0.20}}
\multiput(45.35,55.78)(0.17,-0.11){8}{\line(1,0){0.17}}
\multiput(46.71,54.92)(0.16,-0.12){8}{\line(1,0){0.16}}
\multiput(48.02,53.98)(0.14,-0.11){9}{\line(1,0){0.14}}
\multiput(49.28,52.98)(0.13,-0.12){9}{\line(1,0){0.13}}
\multiput(50.49,51.91)(0.11,-0.11){10}{\line(1,0){0.11}}
\multiput(51.64,50.78)(0.11,-0.12){10}{\line(0,-1){0.12}}
\multiput(52.72,49.59)(0.11,-0.14){9}{\line(0,-1){0.14}}
\multiput(53.74,48.34)(0.12,-0.16){8}{\line(0,-1){0.16}}
\multiput(54.69,47.04)(0.11,-0.17){8}{\line(0,-1){0.17}}
\multiput(55.57,45.69)(0.12,-0.20){7}{\line(0,-1){0.20}}
\multiput(56.37,44.30)(0.10,-0.21){7}{\line(0,-1){0.21}}
\multiput(57.10,42.86)(0.11,-0.25){6}{\line(0,-1){0.25}}
\multiput(57.75,41.39)(0.11,-0.30){5}{\line(0,-1){0.30}}
\multiput(58.33,39.88)(0.10,-0.31){5}{\line(0,-1){0.31}}
\multiput(58.82,38.35)(0.10,-0.39){4}{\line(0,-1){0.39}}
\multiput(59.22,36.79)(0.11,-0.53){3}{\line(0,-1){0.53}}
\multiput(59.54,35.21)(0.12,-0.80){2}{\line(0,-1){0.80}}
\multiput(59.78,33.62)(0.08,-0.80){2}{\line(0,-1){0.80}}
\put(59.93,32.01){\line(0,-1){1.61}}
\put(60.00,30.40){\line(0,-1){1.61}}
\put(59.98,28.79){\line(0,-1){1.61}}
\multiput(59.87,27.18)(-0.10,-0.80){2}{\line(0,-1){0.80}}
\multiput(59.67,25.59)(-0.09,-0.53){3}{\line(0,-1){0.53}}
\multiput(59.39,24.00)(-0.09,-0.39){4}{\line(0,-1){0.39}}
\multiput(59.03,22.43)(-0.11,-0.39){4}{\line(0,-1){0.39}}
\multiput(58.58,20.88)(-0.11,-0.30){5}{\line(0,-1){0.30}}
\multiput(58.05,19.36)(-0.10,-0.25){6}{\line(0,-1){0.25}}
\multiput(57.44,17.87)(-0.12,-0.24){6}{\line(0,-1){0.24}}
\multiput(56.75,16.42)(-0.11,-0.20){7}{\line(0,-1){0.20}}
\multiput(55.98,15.00)(-0.11,-0.17){8}{\line(0,-1){0.17}}
\multiput(55.14,13.63)(-0.11,-0.17){8}{\line(0,-1){0.17}}
\multiput(54.22,12.30)(-0.11,-0.14){9}{\line(0,-1){0.14}}
\multiput(53.24,11.03)(-0.12,-0.14){9}{\line(0,-1){0.14}}
\multiput(52.19,9.81)(-0.11,-0.12){10}{\line(0,-1){0.12}}
\multiput(51.07,8.64)(-0.12,-0.11){10}{\line(-1,0){0.12}}
\multiput(49.89,7.54)(-0.14,-0.12){9}{\line(-1,0){0.14}}
\multiput(48.66,6.51)(-0.14,-0.11){9}{\line(-1,0){0.14}}
\multiput(47.37,5.54)(-0.17,-0.11){8}{\line(-1,0){0.17}}
\multiput(46.03,4.64)(-0.20,-0.12){7}{\line(-1,0){0.20}}
\multiput(44.65,3.82)(-0.20,-0.11){7}{\line(-1,0){0.20}}
\multiput(43.22,3.07)(-0.24,-0.11){6}{\line(-1,0){0.24}}
\multiput(41.76,2.40)(-0.30,-0.12){5}{\line(-1,0){0.30}}
\multiput(40.26,1.81)(-0.31,-0.10){5}{\line(-1,0){0.31}}
\multiput(38.73,1.30)(-0.39,-0.11){4}{\line(-1,0){0.39}}
\multiput(37.18,0.87)(-0.52,-0.11){3}{\line(-1,0){0.52}}
\multiput(35.61,0.53)(-0.53,-0.09){3}{\line(-1,0){0.53}}
\multiput(34.02,0.27)(-0.80,-0.09){2}{\line(-1,0){0.80}}
\put(32.41,0.10){\line(-1,0){1.61}}
\put(30.81,0.01){\line(-1,0){1.61}}
\put(29.19,0.01){\line(-1,0){1.61}}
\multiput(27.59,0.10)(-0.80,0.09){2}{\line(-1,0){0.80}}
\multiput(25.98,0.27)(-0.53,0.09){3}{\line(-1,0){0.53}}
\multiput(24.39,0.53)(-0.52,0.11){3}{\line(-1,0){0.52}}
\multiput(22.82,0.87)(-0.39,0.11){4}{\line(-1,0){0.39}}
\multiput(21.27,1.30)(-0.31,0.10){5}{\line(-1,0){0.31}}
\multiput(19.74,1.81)(-0.30,0.12){5}{\line(-1,0){0.30}}
\multiput(18.24,2.40)(-0.24,0.11){6}{\line(-1,0){0.24}}
\multiput(16.78,3.07)(-0.20,0.11){7}{\line(-1,0){0.20}}
\multiput(15.35,3.82)(-0.20,0.12){7}{\line(-1,0){0.20}}
\multiput(13.97,4.64)(-0.17,0.11){8}{\line(-1,0){0.17}}
\multiput(12.63,5.54)(-0.14,0.11){9}{\line(-1,0){0.14}}
\multiput(11.34,6.51)(-0.14,0.12){9}{\line(-1,0){0.14}}
\multiput(10.11,7.54)(-0.12,0.11){10}{\line(-1,0){0.12}}
\multiput(8.93,8.64)(-0.11,0.12){10}{\line(0,1){0.12}}
\multiput(7.81,9.81)(-0.12,0.14){9}{\line(0,1){0.14}}
\multiput(6.76,11.03)(-0.11,0.14){9}{\line(0,1){0.14}}
\multiput(5.78,12.30)(-0.11,0.17){8}{\line(0,1){0.17}}
\multiput(4.86,13.63)(-0.11,0.17){8}{\line(0,1){0.17}}
\multiput(4.02,15.00)(-0.11,0.20){7}{\line(0,1){0.20}}
\multiput(3.25,16.42)(-0.12,0.24){6}{\line(0,1){0.24}}
\multiput(2.56,17.87)(-0.10,0.25){6}{\line(0,1){0.25}}
\multiput(1.95,19.36)(-0.11,0.30){5}{\line(0,1){0.30}}
\multiput(1.42,20.88)(-0.11,0.39){4}{\line(0,1){0.39}}
\multiput(0.97,22.43)(-0.09,0.39){4}{\line(0,1){0.39}}
\multiput(0.61,24.00)(-0.09,0.53){3}{\line(0,1){0.53}}
\multiput(0.33,25.59)(-0.10,0.80){2}{\line(0,1){0.80}}
\put(0.13,27.18){\line(0,1){1.61}}
\put(0.02,28.79){\line(0,1){1.61}}
\put(0.00,30.40){\line(0,1){1.61}}
\multiput(0.07,32.01)(0.08,0.80){2}{\line(0,1){0.80}}
\multiput(0.22,33.62)(0.12,0.80){2}{\line(0,1){0.80}}
\multiput(0.46,35.21)(0.11,0.53){3}{\line(0,1){0.53}}
\multiput(0.78,36.79)(0.10,0.39){4}{\line(0,1){0.39}}
\multiput(1.18,38.35)(0.10,0.31){5}{\line(0,1){0.31}}
\multiput(1.67,39.88)(0.11,0.30){5}{\line(0,1){0.30}}
\multiput(2.25,41.39)(0.11,0.25){6}{\line(0,1){0.25}}
\multiput(2.90,42.86)(0.10,0.21){7}{\line(0,1){0.21}}
\multiput(3.63,44.30)(0.12,0.20){7}{\line(0,1){0.20}}
\multiput(4.43,45.69)(0.11,0.17){8}{\line(0,1){0.17}}
\multiput(5.31,47.04)(0.12,0.16){8}{\line(0,1){0.16}}
\multiput(6.26,48.34)(0.11,0.14){9}{\line(0,1){0.14}}
\multiput(7.28,49.59)(0.11,0.12){10}{\line(0,1){0.12}}
\multiput(8.36,50.78)(0.11,0.11){10}{\line(1,0){0.11}}
\multiput(9.51,51.91)(0.13,0.12){9}{\line(1,0){0.13}}
\multiput(10.72,52.98)(0.14,0.11){9}{\line(1,0){0.14}}
\multiput(11.98,53.98)(0.16,0.12){8}{\line(1,0){0.16}}
\multiput(13.29,54.92)(0.17,0.11){8}{\line(1,0){0.17}}
\multiput(14.65,55.78)(0.20,0.11){7}{\line(1,0){0.20}}
\multiput(16.06,56.56)(0.24,0.12){6}{\line(1,0){0.24}}
\multiput(17.50,57.27)(0.25,0.11){6}{\line(1,0){0.25}}
\multiput(18.99,57.91)(0.30,0.11){5}{\line(1,0){0.30}}
\multiput(20.50,58.46)(0.39,0.12){4}{\line(1,0){0.39}}
\multiput(22.04,58.93)(0.39,0.10){4}{\line(1,0){0.39}}
\multiput(23.61,59.31)(0.53,0.10){3}{\line(1,0){0.53}}
\multiput(25.19,59.61)(0.80,0.11){2}{\line(1,0){0.80}}
\multiput(26.78,59.83)(1.61,0.09){2}{\line(1,0){1.61}}
%\end
\put(30.00,30.00){\vector(2,1){26.33}}
\put(16.50,57.00){\line(1,-2){26.83}}
\put(64.00,48.67){\makebox(0,0)[cc]{${\hat b}$}}
\put(30.33,48.00){\makebox(0,0)[cc]{$+1$}}
\put(12.67,45.67){\makebox(0,0)[cc]{$-1$}}
\end{picture}
$\quad$
&
$\quad$
%TexCad Options
%\grade{\off}
%\emlines{\off}
%\beziermacro{\off}
%\reduce{\on}
%\snapping{\off}
%\quality{6.00}
%\graddiff{0.01}
%\snapasp{1}
%\zoom{1.00}
\unitlength 0.7mm
\linethickness{0.4pt}
\begin{picture}(64.00,65.67)
%\circle(30.00,30.00){60.00}
\put(30.00,60.00){\line(1,0){1.61}}
\multiput(31.61,59.96)(0.80,-0.06){2}{\line(1,0){0.80}}
\multiput(33.22,59.83)(0.80,-0.11){2}{\line(1,0){0.80}}
\multiput(34.81,59.61)(0.53,-0.10){3}{\line(1,0){0.53}}
\multiput(36.39,59.31)(0.39,-0.10){4}{\line(1,0){0.39}}
\multiput(37.96,58.93)(0.39,-0.12){4}{\line(1,0){0.39}}
\multiput(39.50,58.46)(0.30,-0.11){5}{\line(1,0){0.30}}
\multiput(41.01,57.91)(0.25,-0.11){6}{\line(1,0){0.25}}
\multiput(42.50,57.27)(0.24,-0.12){6}{\line(1,0){0.24}}
\multiput(43.94,56.56)(0.20,-0.11){7}{\line(1,0){0.20}}
\multiput(45.35,55.78)(0.17,-0.11){8}{\line(1,0){0.17}}
\multiput(46.71,54.92)(0.16,-0.12){8}{\line(1,0){0.16}}
\multiput(48.02,53.98)(0.14,-0.11){9}{\line(1,0){0.14}}
\multiput(49.28,52.98)(0.13,-0.12){9}{\line(1,0){0.13}}
\multiput(50.49,51.91)(0.11,-0.11){10}{\line(1,0){0.11}}
\multiput(51.64,50.78)(0.11,-0.12){10}{\line(0,-1){0.12}}
\multiput(52.72,49.59)(0.11,-0.14){9}{\line(0,-1){0.14}}
\multiput(53.74,48.34)(0.12,-0.16){8}{\line(0,-1){0.16}}
\multiput(54.69,47.04)(0.11,-0.17){8}{\line(0,-1){0.17}}
\multiput(55.57,45.69)(0.12,-0.20){7}{\line(0,-1){0.20}}
\multiput(56.37,44.30)(0.10,-0.21){7}{\line(0,-1){0.21}}
\multiput(57.10,42.86)(0.11,-0.25){6}{\line(0,-1){0.25}}
\multiput(57.75,41.39)(0.11,-0.30){5}{\line(0,-1){0.30}}
\multiput(58.33,39.88)(0.10,-0.31){5}{\line(0,-1){0.31}}
\multiput(58.82,38.35)(0.10,-0.39){4}{\line(0,-1){0.39}}
\multiput(59.22,36.79)(0.11,-0.53){3}{\line(0,-1){0.53}}
\multiput(59.54,35.21)(0.12,-0.80){2}{\line(0,-1){0.80}}
\multiput(59.78,33.62)(0.08,-0.80){2}{\line(0,-1){0.80}}
\put(59.93,32.01){\line(0,-1){1.61}}
\put(60.00,30.40){\line(0,-1){1.61}}
\put(59.98,28.79){\line(0,-1){1.61}}
\multiput(59.87,27.18)(-0.10,-0.80){2}{\line(0,-1){0.80}}
\multiput(59.67,25.59)(-0.09,-0.53){3}{\line(0,-1){0.53}}
\multiput(59.39,24.00)(-0.09,-0.39){4}{\line(0,-1){0.39}}
\multiput(59.03,22.43)(-0.11,-0.39){4}{\line(0,-1){0.39}}
\multiput(58.58,20.88)(-0.11,-0.30){5}{\line(0,-1){0.30}}
\multiput(58.05,19.36)(-0.10,-0.25){6}{\line(0,-1){0.25}}
\multiput(57.44,17.87)(-0.12,-0.24){6}{\line(0,-1){0.24}}
\multiput(56.75,16.42)(-0.11,-0.20){7}{\line(0,-1){0.20}}
\multiput(55.98,15.00)(-0.11,-0.17){8}{\line(0,-1){0.17}}
\multiput(55.14,13.63)(-0.11,-0.17){8}{\line(0,-1){0.17}}
\multiput(54.22,12.30)(-0.11,-0.14){9}{\line(0,-1){0.14}}
\multiput(53.24,11.03)(-0.12,-0.14){9}{\line(0,-1){0.14}}
\multiput(52.19,9.81)(-0.11,-0.12){10}{\line(0,-1){0.12}}
\multiput(51.07,8.64)(-0.12,-0.11){10}{\line(-1,0){0.12}}
\multiput(49.89,7.54)(-0.14,-0.12){9}{\line(-1,0){0.14}}
\multiput(48.66,6.51)(-0.14,-0.11){9}{\line(-1,0){0.14}}
\multiput(47.37,5.54)(-0.17,-0.11){8}{\line(-1,0){0.17}}
\multiput(46.03,4.64)(-0.20,-0.12){7}{\line(-1,0){0.20}}
\multiput(44.65,3.82)(-0.20,-0.11){7}{\line(-1,0){0.20}}
\multiput(43.22,3.07)(-0.24,-0.11){6}{\line(-1,0){0.24}}
\multiput(41.76,2.40)(-0.30,-0.12){5}{\line(-1,0){0.30}}
\multiput(40.26,1.81)(-0.31,-0.10){5}{\line(-1,0){0.31}}
\multiput(38.73,1.30)(-0.39,-0.11){4}{\line(-1,0){0.39}}
\multiput(37.18,0.87)(-0.52,-0.11){3}{\line(-1,0){0.52}}
\multiput(35.61,0.53)(-0.53,-0.09){3}{\line(-1,0){0.53}}
\multiput(34.02,0.27)(-0.80,-0.09){2}{\line(-1,0){0.80}}
\put(32.41,0.10){\line(-1,0){1.61}}
\put(30.81,0.01){\line(-1,0){1.61}}
\put(29.19,0.01){\line(-1,0){1.61}}
\multiput(27.59,0.10)(-0.80,0.09){2}{\line(-1,0){0.80}}
\multiput(25.98,0.27)(-0.53,0.09){3}{\line(-1,0){0.53}}
\multiput(24.39,0.53)(-0.52,0.11){3}{\line(-1,0){0.52}}
\multiput(22.82,0.87)(-0.39,0.11){4}{\line(-1,0){0.39}}
\multiput(21.27,1.30)(-0.31,0.10){5}{\line(-1,0){0.31}}
\multiput(19.74,1.81)(-0.30,0.12){5}{\line(-1,0){0.30}}
\multiput(18.24,2.40)(-0.24,0.11){6}{\line(-1,0){0.24}}
\multiput(16.78,3.07)(-0.20,0.11){7}{\line(-1,0){0.20}}
\multiput(15.35,3.82)(-0.20,0.12){7}{\line(-1,0){0.20}}
\multiput(13.97,4.64)(-0.17,0.11){8}{\line(-1,0){0.17}}
\multiput(12.63,5.54)(-0.14,0.11){9}{\line(-1,0){0.14}}
\multiput(11.34,6.51)(-0.14,0.12){9}{\line(-1,0){0.14}}
\multiput(10.11,7.54)(-0.12,0.11){10}{\line(-1,0){0.12}}
\multiput(8.93,8.64)(-0.11,0.12){10}{\line(0,1){0.12}}
\multiput(7.81,9.81)(-0.12,0.14){9}{\line(0,1){0.14}}
\multiput(6.76,11.03)(-0.11,0.14){9}{\line(0,1){0.14}}
\multiput(5.78,12.30)(-0.11,0.17){8}{\line(0,1){0.17}}
\multiput(4.86,13.63)(-0.11,0.17){8}{\line(0,1){0.17}}
\multiput(4.02,15.00)(-0.11,0.20){7}{\line(0,1){0.20}}
\multiput(3.25,16.42)(-0.12,0.24){6}{\line(0,1){0.24}}
\multiput(2.56,17.87)(-0.10,0.25){6}{\line(0,1){0.25}}
\multiput(1.95,19.36)(-0.11,0.30){5}{\line(0,1){0.30}}
\multiput(1.42,20.88)(-0.11,0.39){4}{\line(0,1){0.39}}
\multiput(0.97,22.43)(-0.09,0.39){4}{\line(0,1){0.39}}
\multiput(0.61,24.00)(-0.09,0.53){3}{\line(0,1){0.53}}
\multiput(0.33,25.59)(-0.10,0.80){2}{\line(0,1){0.80}}
\put(0.13,27.18){\line(0,1){1.61}}
\put(0.02,28.79){\line(0,1){1.61}}
\put(0.00,30.40){\line(0,1){1.61}}
\multiput(0.07,32.01)(0.08,0.80){2}{\line(0,1){0.80}}
\multiput(0.22,33.62)(0.12,0.80){2}{\line(0,1){0.80}}
\multiput(0.46,35.21)(0.11,0.53){3}{\line(0,1){0.53}}
\multiput(0.78,36.79)(0.10,0.39){4}{\line(0,1){0.39}}
\multiput(1.18,38.35)(0.10,0.31){5}{\line(0,1){0.31}}
\multiput(1.67,39.88)(0.11,0.30){5}{\line(0,1){0.30}}
\multiput(2.25,41.39)(0.11,0.25){6}{\line(0,1){0.25}}
\multiput(2.90,42.86)(0.10,0.21){7}{\line(0,1){0.21}}
\multiput(3.63,44.30)(0.12,0.20){7}{\line(0,1){0.20}}
\multiput(4.43,45.69)(0.11,0.17){8}{\line(0,1){0.17}}
\multiput(5.31,47.04)(0.12,0.16){8}{\line(0,1){0.16}}
\multiput(6.26,48.34)(0.11,0.14){9}{\line(0,1){0.14}}
\multiput(7.28,49.59)(0.11,0.12){10}{\line(0,1){0.12}}
\multiput(8.36,50.78)(0.11,0.11){10}{\line(1,0){0.11}}
\multiput(9.51,51.91)(0.13,0.12){9}{\line(1,0){0.13}}
\multiput(10.72,52.98)(0.14,0.11){9}{\line(1,0){0.14}}
\multiput(11.98,53.98)(0.16,0.12){8}{\line(1,0){0.16}}
\multiput(13.29,54.92)(0.17,0.11){8}{\line(1,0){0.17}}
\multiput(14.65,55.78)(0.20,0.11){7}{\line(1,0){0.20}}
\multiput(16.06,56.56)(0.24,0.12){6}{\line(1,0){0.24}}
\multiput(17.50,57.27)(0.25,0.11){6}{\line(1,0){0.25}}
\multiput(18.99,57.91)(0.30,0.11){5}{\line(1,0){0.30}}
\multiput(20.50,58.46)(0.39,0.12){4}{\line(1,0){0.39}}
\multiput(22.04,58.93)(0.39,0.10){4}{\line(1,0){0.39}}
\multiput(23.61,59.31)(0.53,0.10){3}{\line(1,0){0.53}}
\multiput(25.19,59.61)(0.80,0.11){2}{\line(1,0){0.80}}
\multiput(26.78,59.83)(1.61,0.09){2}{\line(1,0){1.61}}
%\end
\put(30.00,30.00){\vector(1,2){13.17}}
\put(30.00,30.00){\vector(2,1){26.33}}
\put(16.50,57.00){\line(1,-2){26.83}}
\put(57.00,16.50){\line(-2,1){54.00}}
\put(46.67,65.67){\makebox(0,0)[cc]{${\hat a}$}}
\put(64.00,48.67){\makebox(0,0)[cc]{${\hat b}$}}
\put(2.33,56.00){\makebox(0,0)[cc]{$\vert {\hat a} -{\hat b} \vert$}}
\put(30.33,48.00){\makebox(0,0)[cc]{$+1$}}
\put(12.67,45.67){\makebox(0,0)[cc]{$-1$}}
\put(46.33,14.67){\makebox(0,0)[cc]{$-1$}}
\put(17.00,17.00){\makebox(0,0)[cc]{$+1$}}
\put(59.00,5.67){\makebox(0,0)[cc]{$\vert {\hat a} -{\hat b} \vert $}}
\end{picture}
\\
a)
&
b)
&
c)
\end{tabular}
\end{figure}

\section{Strategies towards breaking the Bell barrier}

Locate ${\hat a}$, then rotate system such that measurement of ${\hat b}$ yields the desired correlations.


Successive partitions of the unit circle by introducing more directions $ {\hat \lambda }_i$.

\section{Ultimate goal: stronger than quantum correlations without information transfer}
brainteaser


(*********** linear ***********)

nn = 300;(*precision*)
Plot[(*theta = 2.0;*)
  vecbob = {Cos[theta + 10], Sin[theta + 10]};
  vecalice = {Cos[10], Sin[10]};
  iaux = 0;
  Do[{tlambda = Random[Real, {0, 2Pi}];
      tvec = {Cos[tlambda], Sin[tlambda]};
      iaux += N[Sign[vecbob.tvec]]*N[Sign[vecalice.tvec]]}, {nn}];
  iaux/nn, {theta, 0, Pi}]


(*********** 1/2 Heaviside ***********)
nn = 20;(*precision*)
vecalice = {1, 0};
Plot[(*theta = 2.0;*)
  vecbob = {Cos[theta], Sin[theta]};
  iaux = 0;
  Do[{tlambda = Random[Real, {0, 2Pi}];
      tvec = {Cos[tlambda], Sin[tlambda]};
      signaux = Sign[vecalice.tvec];
      tvecast = {Cos[tlambda - signaux*(tlambda)],
          Sin[tlambda - signaux*(tlambda)]};
      iiaux = iaux + N[Sign[vecbob.tvecast]]*N[Sign[vecalice.tvec]];
      iaux = iiaux}, {nn}];
  iaux/nn, {theta, 0, Pi}]


(***********  Heaviside ***********)
nn = 20;(*precision*)
Plot[(*theta = 2.0;*)
  vecbob = {Cos[theta], Sin[theta]};
  vecalice = {Cos[0], Sin[0]};
  iaux = 0;
  Do[{tlambda = Random[Real, {0, 2Pi}];
      tvec = {Cos[tlambda], Sin[tlambda]};
      If[Sign[vecbob.tvec] >= 0,
        If[Sign[vecalice.tvec] >= 0, shift = theta - tlambda,
          shift = theta - tlambda + Pi],
        If[Sign[vecalice.tvec] >= 0, shift = theta - tlambda + Pi,
          shift = theta - tlambda]];
      tvecast = {Cos[tlambda + shift], Sin[tlambda + shift]};
      iaux += N[Sign[vecbob.tvecast]]*N[Sign[vecalice.tvec]]}, {nn}];
  iaux/nn, {theta, 0, Pi}]


(**************************************)
nn = 40;(*precision*)
Plot[{
    vecbob = {Cos[theta], Sin[theta]};
    alice = 0;
    vecalice = {Cos[alice], Sin[alice]};
    iaux = 0;
    Do[{tlambda = Random[Real, {0, 2*Pi}];
        tvec = {Cos[tlambda], Sin[tlambda]};
        (* If[Sign[vecbob.vecalice] >= 0, shift = alice,
              shift = alice + Pi]; *)
        If[Sign[vecalice.tvec] >= 0,
          If[Sign[tvec.vecbob] >= 0, shift = tlambda, shift = tlambda + Pi],
          If[Sign[tvec.vecbob] < 0, shift = tlambda, shift = tlambda + Pi]
          ];
        iaux +=
          N[Sign[tvec.{Cos[shift], Sin[shift]}]]*
            N[Sign[vecalice.tvec]]}, {nn}];
    iaux/nn}, {theta, 0, Pi}]



(********* Toner & Bacon original 2 dim ****************************)

nn = 100;(*precision*)
Plot[
  vecbob = {Cos[theta], Sin[theta]};
  vecalice = {Cos[0], Sin[0]};
  iaux = 0;
  Do[{tlambda1 = Random[Real, {0, 2Pi}];
        tlambda2 = Random[Real, {0, 2Pi}];
      tvec1 = {Cos[tlambda1], Sin[tlambda1]};
      tvec2 = {Cos[tlambda2], Sin[tlambda2]};
      cc = Sign[vecalice.tvec1]*Sign[vecalice.tvec2];
      iaux += -N[Sign[vecbob.(tvec1 + cc*tvec2)]]*
          N[Sign[vecalice.tvec1]]}, {nn}];
  iaux/nn, {theta, 0, Pi}]

(********* Toner & Bacon original 3 dim ****************************)
nn = 10000;(*precision*)Plot[{-Cos[theta],
    vecbob = {Cos[theta], Sin[theta], 0};
    vecalice = {Cos[0], Sin[0], 0};
    iaux = 0;
    Do[{tlambda1 = Random[Real, {0, 2Pi}];
        tlambda2 = Random[Real, {0, 2Pi}];
        tphi1 = Random[Real, {0, Pi}];
        tphi2 = Random[Real, {0, Pi}];
        tvec1 = {Cos[tlambda1] Sin[tphi1], Sin[tlambda1] Sin[tphi1],
            Cos[tphi1]};
        tvec2 = {Cos[tlambda2] Sin[tphi2], Sin[tlambda2]  Sin[tphi2],
            Cos[tphi2]};
        cc = Sign[vecalice.tvec1]*Sign[vecalice.tvec2];
        iaux += -N[Sign[vecbob.(tvec1 + cc*tvec2)]]*
            N[Sign[vecalice.tvec1]]}, {nn}];
    iaux/nn}, {theta, 0, Pi}]


(********* Toner & Bacon Eq.(2) ****************************)


nn = 400;(*precision*)
Plot[
  vecbob = {Cos[theta], Sin[theta]};
  vecalice = {Cos[0], Sin[0]};
  iaux = 0;
  Do[{tlambda1 = Random[Real, {0, 2Pi}];
        tlambda2 = Random[Real, {0, 2Pi}];
      tvec1 = {Cos[tlambda1], Sin[tlambda1]};
      tvec2 = {Cos[tlambda2], Sin[tlambda2]};
      iaux +=
        2*N[Sign[vecbob.(tvec2 - tvec1)]]*N[Sign[vecalice.tvec1]]}, {nn}];
  iaux/nn, {theta, 0, Pi}]



(********* two orthogonal directions ****************************)
nn = 200;(*precision*)Plot[vecbob = {Cos[theta + 10], Sin[theta + 10]};
  vecalice = {Cos[0 + 10], Sin[0 + 10]};
  iaux = 0;
  Do[{tlambda1 = Random[Real, {0, 2Pi}];
      tlambda2 = tlambda1 + Pi/2 ;
      tvec1 = {Cos[tlambda1], Sin[tlambda1]};
      tvec2 = {Cos[tlambda2], Sin[tlambda2]};
      cc = Sign[vecalice.tvec1]*Sign[vecalice.tvec2];
      iaux += -N[Sign[vecbob.(tvec1 + cc*tvec2)]]*
          N[Sign[vecalice.tvec1]]}, {nn}];
  iaux/nn, {theta, 0, Pi}]


(********* two arbitrary directions ****************************)
nn = 2000;(*precision*)
Plot3D[vecbob = {Cos[theta + 10], Sin[theta + 10]};
  vecalice = {Cos[0 + 10], Sin[0 + 10]};
  iaux = 0;
  Do[{tlambda1 = Random[Real, {0, 2Pi}];
      tlambda2 = tlambda1 + delta;
      tvec1 = {Cos[tlambda1], Sin[tlambda1]};
      tvec2 = {Cos[tlambda2], Sin[tlambda2]};
      cc = Sign[vecalice.tvec1]*Sign[vecalice.tvec2];
      iaux += -N[Sign[vecbob.(tvec1 + cc*tvec2)]]*
          N[Sign[vecalice.tvec1]]}, {nn}];
  iaux/nn, {theta, 0, Pi}, {delta, 0, Pi},
  AxesLabel -> {\[Theta], \[Delta], E} ]

(********* two  directions ****************************)
nn = 10000;(*precision*)
mm = 5;
Show[Table[ Plot[ delta = N[Pi * idelta  / (2*mm)];
      vecbob = {Cos[theta + 10], Sin[theta + 10]};
      vecalice = {Cos[0 + 10], Sin[0 + 10]};
      iaux = 0;
      Do[{tlambda1 = Random[Real, {0, 2Pi}];
          tlambda2 = tlambda1 + delta;
          tvec1 = {Cos[tlambda1], Sin[tlambda1]};
          tvec2 = {Cos[tlambda2], Sin[tlambda2]};
          cc = Sign[vecalice.tvec1]*Sign[vecalice.tvec2];

          iaux += -N[Sign[vecbob.(tvec1 + cc*tvec2)]]*
              N[Sign[vecalice.tvec1]]}, {nn}];
      iaux/nn, {theta, 0, Pi}], {idelta, 0, mm}]]

or

\!\(nn = 1000000;  (*precision*) mm = 5;\[IndentingNewLine]
  Show[Table[
      Plot[delta =
          N[Pi*idelta/\((2*mm)\)]; \[IndentingNewLine]vecbob = {Cos[
              theta + 10],
            Sin[theta + 10]}; \[IndentingNewLine]vecalice = {Cos[0 + 10],
            Sin[0 + 10]}; \[IndentingNewLine]iaux =
          0; \[IndentingNewLine]Do[{tlambda1 =
              Random[Real, {0, 2  Pi}]; \[IndentingNewLine]tlambda2 =
              tlambda1 + delta; \[IndentingNewLine]tvec1 = {Cos[tlambda1],
                Sin[tlambda1]}; \[IndentingNewLine]tvec2 = {Cos[tlambda2],
                Sin[tlambda2]}; \[IndentingNewLine]cc =
              Sign[vecalice . tvec1]*
                Sign[vecalice . tvec2]; \[IndentingNewLine]iaux += \(-N[
                    Sign[vecbob . \((tvec1 + cc*tvec2)\)]]\)*
                N[Sign[vecalice . tvec1]]}, {nn}]; \[IndentingNewLine]iaux/
          nn, {theta, 0,
          Pi}, \[IndentingNewLine]FrameTicks -> {{{0.0001, "\<0\>"}, \
\[Pi]\/4, \[Pi]\/2, \(3  \[Pi]\)\/4, \[Pi]}, {\(-1\), {0.0001, "\<0\>"}, 1},
            None, None}, Frame -> True, DefaultFont -> {"\<Times\>", 24},
        FrameLabel -> {"\<\[Theta] [rad]\>", "\<E(\[Theta],\[Delta])\>"},
        AspectRatio -> 1,
        PlotStyle -> {{Hue[\((mm - idelta)\)/\((mm + 3)\), 1, 1],
              Thickness[0.004]}}], {idelta, 0, mm}]]\)

Export["c:/mytex/2004-brainteaser-f2a.eps", Out[39], "EPS", ImageSize -> 500]


(********* more heaviside, but shift ****************************)
nn = 20;(*precision*)Plot[vecbob = {Cos[theta + 10], Sin[theta + 10]};
  vecalice = {Cos[10], Sin[10]};
  iaux = 0;
  Do[{tlambda1 = Random[Real, {0, Pi/16}];
      tlambda2 = Random[Real, {0, Pi/16}];
      tvec1 = {Cos[tlambda1], Sin[tlambda1]};
      tvec2 = {Cos[tlambda2], Sin[tlambda2]};
      cc = Sign[vecalice.tvec1]*Sign[vecalice.tvec2];
      iaux += -N[Sign[vecbob.(tvec2 + cc*tvec1)]]*
          N[Sign[vecalice.tvec1]]}, {nn}];
  iaux/nn, {theta, 0, Pi}, PlotRange -> {{0, Pi}, {-1, 1}}]

(********* FIRST INTEGRAL OVER l2 ****************************)

nn = 100;(*precision*)Plot[
  vecbob = {0, 1};
  iaux = 0;
  Do[tlambda2 = Random[Real, {0, 2Pi}];
    tvec2 = {Sin[tlambda2], Cos[tlambda2]};
    tvec1 = {Sin[theta], Cos[theta]};
    iaux += N[Sign[vecbob.(tvec2 - tvec1)]], {nn}];
  iaux/nn, {theta, 0, Pi}]

(********* Second INTEGRAL OVER l1 ****************************)

nn = 100;(*precision*)Plot[{2*(theta)/Pi - 1,
    vecalice = {0, 1};
    vecbob = {Sin[theta], Cos[theta]};
    iaux = 0;
    Do[tlambda1 = Random[Real, {0, 2Pi}];
      tvec1 = {Sin[tlambda1], Cos[tlambda1]};
      iaux +=
        2*N[Sign[vecalice.tvec1]]*(2*(ArcCos[vecbob.tvec1])/Pi - 1), {nn}];
    iaux/nn}, {theta, 0, Pi}]

or

nn = 100;(*precision*)Plot[{2*(theta)/Pi - 1,
    vecalice = {0, 1};
    vecbob = {Sin[theta], Cos[theta]};
    iaux = 0;
    Do[tlambda1 = Random[Real, {0, 2Pi}];
      tvec1 = {Sin[tlambda1], Cos[tlambda1]};
      iaux +=
        2*N[Sign[Cos[tlambda1]]]*(2*(ArcCos[Cos[(theta - tlambda1)]])/Pi -
              1), {nn}];
    iaux/nn}, {theta, 0, Pi}]

or

nn = 100;(*precision*)Plot[{2*(theta)/Pi - 1,
    vecalice = {0, 1};
    vecbob = {Sin[theta], Cos[theta]};
    iaux = 0;
    Do[tlambda1 = Random[Real, {0, 2Pi}];
      tvec1 = {Sin[tlambda1], Cos[tlambda1]};
      iaux +=
        2*N[Sign[Cos[tlambda1]]]*(2*(ArcCos[Cos[(theta - tlambda1)]])/Pi), {nn}];
    iaux/nn}, {theta, 0, Pi}]

or

nn = 1000;(*precision*)Plot[{2*(theta)/Pi - 1, vecalice = {0, 1};
    vecbob = {Sin[theta], Cos[theta]};
    iaux = 0;
    Do[tlambda1 = Random[Real, {0, Pi}];
      tvec1 = {Sin[tlambda1], Cos[tlambda1]};
      iaux +=
        2*N[Sign[Cos[tlambda1]]]*(2*(Abs[(theta - tlambda1)])/Pi), {nn}];
    iaux/nn}, {theta, 0, Pi}]

or

Plot[(4/Pi^2) *
      UnitStep[Pi/2 - x]*  (x*x - x*Pi/2 +
          Pi^2/8 - (3 Pi^2/8 - (Pi/2)x)) + (4/Pi^2) *
      UnitStep[x - Pi/2]*(Pi/2  x -
          Pi^2/8 - (x^2 - (3*Pi/2)*x + 5Pi^2/8)), {x, 0, Pi}]

or

\!\(Plot[\((4/Pi^2)\)*
        UnitStep[Pi/2 - x]*\((x*x - Pi^2/4)\) - \((4/Pi^2)\)\ \((1/4)\)*
        UnitStep[x - Pi/2]*\((\((Pi - 2*x)\)*\((3*Pi - 2*x)\))\), {x,
      0, \ \[Pi]}, \[IndentingNewLine]PlotStyle -> {{Thickness[0.009]}, {Hue[
            0.2, 1, 0.5], Thickness[0.015]}, {Hue[0.65, 1, 0.7],
          Thickness[0.006]}, {Hue[0.85, 1, 0.8], Thickness[0.015]}},
    FrameTicks -> {{{0.0001, "\<0\>"}, \[Pi]\/4, \[Pi]\/2, \(3  \[Pi]\)\/4, \
\[Pi]}, {\(-1\), {0.0001, "\<0\>"}, 1}, None, None}, Frame -> True,
    DefaultFont -> {"\<Times\>", 24},
    FrameLabel -> {"\<\[Theta] [rad]\>", "\<E(\[Theta])\>"},
    AspectRatio -> 1]\)

Export["c:/mytex/2004-brainteaser-f3.eps", Out[58], "EPS", ImageSize -> 500]

(****************** alternative Route *******************)

ff[t_] = Integrate[-1*UnitStep[(d/2) - t] + (-1 + 2/Pi(t - (d/2)))*
          UnitStep[t - (d/2)]*UnitStep[Pi/2 - (d/2) - t] + (-2 + 4/Pi t)*
          UnitStep[t - (Pi/2 - (d/2))]*
          UnitStep[Pi/2 + (d/2) - t] + (1 + 2/Pi (t - Pi + (d/2)))UnitStep[
            t - (Pi/2 + (d/2))]*UnitStep[Pi - (d/2) - t] +
        UnitStep[t - (Pi - (d/2))], {d, 0, Pi/2}] // FullSimplify


Plot[(4/Pi)*ff[t], {t, 0, Pi}]

(**************************)
Plot[UnitStep[t - 3 Pi/4] - UnitStep[Pi/4 - t] -
    2(1 - 2 t/Pi)UnitStep[t - Pi/4]UnitStep[3 Pi/4 - t], {t, 0, Pi}]


(**************************)

Plot[4/Pi^2((t^2 - Pi^2/4) - 2UnitStep[t - Pi/2](t - Pi/2)^2),
  {t, 0, Pi}]

(*********** theta-delta plane ***************)

Plot[{0, d/2, Pi/2 - d/2, Pi/2 + d/2, Pi - d/2, Pi}, {d, 0, Pi/2},
  AspectRatio -> 1, Frame -> True]




Show[Plot[{t*2, Pi - t*2, -Pi + 2 t, 2 Pi - t*2}, {t, 0, \[Pi]},
    FrameTicks -> {{{0.0001, "0"}, \[Pi]/4, \[Pi]/2,
          3 \[Pi]/4, \[Pi]}, {{0.0001, "0"}, \[Pi]/4, \[Pi]/2}, None, None},
    PlotRange -> {{0, Pi}, {0, Pi/2}}, Frame -> True,
    DefaultFont -> {"Times", 24},
    PlotStyle -> {{Hue[0, 1, 0.9], Thickness[0.015]}, {Hue[0.2, 1, 0.5],
          Thickness[0.015]}, {Hue[0.65, 1, 0.7],
          Thickness[0.015]}, {Hue[0.85, 1, 0.8], Thickness[0.015]}},
    FrameLabel -> {"\[Theta] [rad]", "\[Delta] [rad]"}, AspectRatio -> 0.5],
  Graphics[{Text[
        StyleForm["1", FontSize -> 34, FontWeight -> "Bold"], {Pi/12,
          3 Pi/8}]}, AspectRatio -> 1],
  Graphics[{Thickness[0.01], Circle[{Pi/12, 3 Pi/8 + 0.025}, 0.2]},
    AspectRatio -> 1],
  Graphics[{Text[
        StyleForm["2", FontSize -> 34, FontWeight -> "Bold"], {Pi/4, Pi/8}]},
    AspectRatio -> 1],
  Graphics[{Thickness[0.01], Circle[{Pi/4, Pi/8 + 0.025}, 0.2]},
    AspectRatio -> 1],
  Graphics[{Text[
        StyleForm["3", FontSize -> 34, FontWeight -> "Bold"], {Pi/2,
          3 Pi/8}]}, AspectRatio -> 1],
  Graphics[{Thickness[0.01], Circle[{Pi/2, 3 Pi/8 + 0.025}, 0.2]},
    AspectRatio -> 1],
  Graphics[{Text[
        StyleForm["4", FontSize -> 34, FontWeight -> "Bold"], {3Pi/4, Pi/8}]},
     AspectRatio -> 1],
  Graphics[{Thickness[0.01], Circle[{3Pi/4, Pi/8 + 0.025}, 0.2]},
    AspectRatio -> 1],
  Graphics[{Text[
        StyleForm["5", FontSize -> 34, FontWeight -> "Bold"], {Pi - Pi/12,
          3 Pi/8}]}, AspectRatio -> 1],
  Graphics[{Thickness[0.01], Circle[{Pi - Pi/12, 3 Pi/8 + 0.025}, 0.2]},
    AspectRatio -> 1]]
]


Export["c:/mytex/2004-brainteaser-f2b.eps",Out[111],"EPS",ImageSize\[Rule]500]
