\documentclass{article}
\begin{document}


\section*{Preface}


The topic of this special issue on {\em Physics and Computation}
is the presentation of some aspects of the varied and extensive
-- and to some researchers even mind boggling --
connections between the disciplines of physics and computer science.
Whereas the former  one is rooted in empirical evidence, the latter subject
depends on some suitable physical substrate but transcends any
concrete realization
by reducing and ``purifying'' the act of computation to its formal aspects.


In comparing the concepts of both domains and interpreting any
physical performance as computation,
and any computation as a physical process, new aspects, insights and
stimuli can be obtained for both fields.



{\em Physics and Computation 2010}, which has been a continuation of
previous conferences,
took place on the River Nile,
on a boat that brought its participants from ancient Thebes (today's Luxor) to
Aswan and allowed them to fly on even further to Abu Simbel.
Thus the event took place in three worlds:
under the aegis of Ramses II,
in overpopulated contemporary Egypt on the edge of social turmoil that
lead to a period of political instability,
and in the scientific realm; all in parallel.



It is not easy to render the resulting amalgam of experiences enjoyed
by the participants.
The lectures were staged in the central saloon of the cruise boat,
which was almost entirely darkened and cooled down to freezing frenzy.
If one stepped outside of this lecture theatre one could see through
bullet eye windows the legs of idle participants as they were swimming
or rather stirring in the
tiny greenish pool that was hanging from the top ceiling and filling
the upper part of the space formed by the central staircase.
The pool served also as a great communication mediator.
I still remembers a vivid discussion we had on physical
quantum random
number generators
while we were slowly paddling along the pool's circumference.


Anecdotes aside, as the boat silently tucked away there were interesting ideas floating over the river Nile,
some of which are communicated in the following articles.


Proving that a dynamical system is chaotic is a central problem in
chaos theory.
In {\em Fermat's last theorem and chaoticity} by Elena Calude
the author applies a specific computational method  to
evaluate the algorithmic complexity of the Fermat's last theorem and
proves that the theorem is in the lowest complexity class associated
with that measure.
Using this result a two dimensional Hamiltonian system   is conceived
for which the proof that the system has a Smale horseshoe has a very
low complexity.


In {\em the physical Church thesis as an explanation of the Galileo
thesis} Gilles Dowek
discussed Galileio's assumption that the (book of the) Universe ``is
written in the language of mathematics''
and its specification in terms of a physical Church thesis very
pointedly stating that the Universe computes.


{\em Membrane system models
for super-Turing paradigms}  by Marian  Gheorghe and Mike Stannet
continues 2004 paper of Cristian Calude and Gheorghe Paun on using
biologically inspired models of computation
for building accelerated membrane systems able to simulate
super-Turing machines.
In this paper multiple accelerated membrane systems are built in order
to solve multiple inputs in parallel and in finite time.
The resulting systems have hyperarithmetical computational power.


The paper {\em how much contextuality} discusses certain quantitative
bounds on the (non)classical behavior of quantized systems.
Stated differently, it attempts to answer the question: ``how many
violations (if any) of the classical logical operations
and truth assignments are necessary in order to obtain certain
expectations, or frequencies, or
(in a certain counterfactual, nonsimultaneous sense) operational
expressions occurring in quantum theory?''



The contributors gratefully acknowledge support and contributions of
Alexandria University in Egypt,
of the
Centre for Discrete Mathematics and Theoretical Computer Science of the
University of Auckland in New Zealand,
of the
Centre for Applied Mathematics and Information Technologies of the
University of Azores in Portugal,
as well as of the
Centro de Matem\`atica e Aplica\c{c}\~{o}es Fundamentais of the
University of Lisbon in Portugal.
\\
\\
\\
Karl Svozil, svozil@tuwien.ac.at\\
Guest Editor

\end{document}
