%%%%%%%%%%%%%%%%%%%%% chapter.tex %%%%%%%%%%%%%%%%%%%%%%%%%%%%%%%%%
%
% sample chapter
%
% Use this file as a template for your own input.
%
%%%%%%%%%%%%%%%%%%%%%%%% Springer-Verlag %%%%%%%%%%%%%%%%%%%%%%%%%%
%\motto{Use the template \emph{chapter.tex} to style the various elements of your chapter content.}
\chapter{Some visual photos of objects that have not (yet?) been identified}
%\chapter{UFO legends after the Robertson Panel (January 1953) and before the USS Nimitz incident}
\label{2023-UFO-part-History-photos} % Always give a unique label
% use \chaptermark{}
% to alter or adjust the chapter heading in the running head


% https://www.afnwc.af.mil/About-Us/History/Trinity-Nuclear-Test/

% https://www.atomicarchive.com/media/photographs/trinity/index.html

\abstract*{Here is a selection of visual photographs depicting alleged UFO sightings, all of which are currently consistent with the hypothesis that they depict a craft not associated with any known human origin.}


\abstract{Here is a selection of visual photographs depicting alleged UFO sightings, all of which are currently consistent with the hypothesis that they depict a craft not associated with any known human origin.}

\newpage


\section{Trent phtotos near near McMinnville, Oregon on May 11, 1950}
\label{2023-UFO-part-History-photos-1950-tr}

On May 11, 1950, at approximately 7:30 pm, while returning to her farmhouse after feeding her caged rabbits,
Evelyn Trent spotted a slow-moving, metallic disk-shaped object approaching her from the northeast.
She alerted her husband, Paul, who was inside the house, and he too observed the object.
After watching it for a brief period, he returned to their home to retrieve a camera and managed
to capture two photographs of the object before it rapidly departed to the west.
Paul Trent's father also reported briefly glimpsing the object before it flew away~\cite[Case~46]{Condon-report-Bantam}.


% For figures use
%
\begin{figure}[b]
\sidecaption
% Use the relevant command for your figure-insertion program
% to insert the figure file.
% For example, with the option graphics use
%\includegraphics{2023-UFO-part-History-photos-1971-cr-c}
\resizebox{1\textwidth}{!}{ \includegraphics{2023-UFO-part-History-photos-1950-tr1-c}  }
%
% If not, use
%\picplace{5cm}{2cm} % Give the correct figure height and width in cm
%
\caption{First Trent photo of a UFO hovering  near McMinnville~\cite{TrentArchivScans,Sheaffer2015Aug}.}

\label{2023-UFO-part-History-photos-1950-tr1-c}       % Give a unique label
\end{figure}

% For figures use
%
\begin{figure}[b]
\sidecaption
% Use the relevant command for your figure-insertion program
% to insert the figure file.
% For example, with the option graphics use
\resizebox{.6\textwidth}{!}{ \includegraphics{2023-UFO-part-History-photos-1950-tr1-cu} }
%
% If not, use
%\picplace{5cm}{2cm} % Give the correct figure height and width in cm
%
\caption{Closeup of UFO depicted in~\ref{2023-UFO-part-History-photos-1950-tr1-c}.}
\label{2023-UFO-part-History-photos-1950-tr1-cu}       % Give a unique label
\end{figure}



% For figures use
%
\begin{figure}[b]
\sidecaption
% Use the relevant command for your figure-insertion program
% to insert the figure file.
% For example, with the option graphics use
%\includegraphics{2023-UFO-part-History-photos-1971-cr-c}
\resizebox{1\textwidth}{!}{ \includegraphics{2023-UFO-part-History-photos-1950-tr2-c} }
%
% If not, use
%\picplace{5cm}{2cm} % Give the correct figure height and width in cm
%
\caption{Second Trent photo of a UFO hovering  near McMinnville~\cite{TrentArchivScans,Sheaffer2015Aug}.}

\label{2023-UFO-part-History-photos-1950-tr2-c}       % Give a unique label
\end{figure}

% For figures use
%
\begin{figure}[b]
\sidecaption
% Use the relevant command for your figure-insertion program
% to insert the figure file.
% For example, with the option graphics use
\resizebox{.6\textwidth}{!}{ \includegraphics{2023-UFO-part-History-photos-1950-tr2-cu} }
%
% If not, use
%\picplace{5cm}{2cm} % Give the correct figure height and width in cm
%
\caption{Closeup of UFO depicted in~\ref{2023-UFO-part-History-photos-1950-tr2-c}.}
\label{2023-UFO-part-History-photos-1950-tr2-cu}       % Give a unique label
\end{figure}



\clearpage
%%%%%%%%%%%%%%%%%%%%%%%%%%%%%%%%%%%%%%%%%%%%%%%%%%%%%%%%%%%%%%%%%%%%%%%%%%%%%%%%%%%%%%%%%%%%%%%%%%%%%%%%%%%%%%%%



\section{Saucer-shaped object moving above the sea towards Trindade Island on January 16, 1958}
\label{2023-UFO-part-History-photos-1958-tr}

On January 16, 1958, at approximately  11:00 am,  the crew of the Brazilian Navy training ship ``Almirante Saldanha'' witnessed a strange, silver,
saucer-shaped object moving above the sea towards Trindade Island.
The object didn't make any sound, was shiny and sometimes it moved quickly,
then slowly, up and slightly down.
When it accelerated, it would leave a white phosphorescent trail that would disappear shortly.
The crew, including retired Forca A\'erea Brasileira captain-aviator Jos\'e Viegas and photographer Almiro Bara\'una, an expert in trick photography,
saw the object and took pictures.
The electronic equipment of the ship stopped working during the sighting,
and after the ship left the island, the equipment malfunctioned three more times.
The Navy wanted to keep the pictures secret, but a reporter took copies and showed them to then-president Juscelino Kubitscheck
who released them to the public~\cite{Trinitate1958Jan}.

However, In 2009, two Brazilian skeptics interviewed a crew member of the Almirante Saldanha,
who said that only 15 people saw the UFO and not the reported 48. The skeptics suggest the sighting may have been a case of mass hysteria.
In 2010, a relative of photographer Almiro Bara\'una revealed that he had confessed to faking the UFO photos using two plates stacked on
top of each other and photographed in front of a refrigerator door with the right lighting.

\newpage

% For figures use
%
\begin{figure}[b]
\sidecaption
% Use the relevant command for your figure-insertion program
% to insert the figure file.
% For example, with the option graphics use
%\includegraphics{2023-UFO-part-History-photos-1971-cr-c}
\resizebox{1\textwidth}{!}{ \includegraphics{2023-UFO-part-History-photos-1958-tr-c}  }
%
% If not, use
%\picplace{5cm}{2cm} % Give the correct figure height and width in cm
%
\caption{Photo of a UFO hovering above the sea towards Trindade Island, Brazil~\cite{Trinitate1958Jan}.}

\label{2023-UFO-part-History-photos-1958-tr-c}       % Give a unique label
\end{figure}

% For figures use
%
\begin{figure}[b]
\sidecaption
% Use the relevant command for your figure-insertion program
% to insert the figure file.
% For example, with the option graphics use
\resizebox{.6\textwidth}{!}{ \includegraphics{2023-UFO-part-History-photos-1958-tr-cu} }
%
% If not, use
%\picplace{5cm}{2cm} % Give the correct figure height and width in cm
%
\caption{Closeup of UFO depicted in~\ref{2023-UFO-part-History-photos-1958-tr-c}.}
\label{2023-UFO-part-History-photos-1958-tr-cu}       % Give a unique label
\end{figure}



\clearpage
%%%%%%%%%%%%%%%%%%%%%%%%%%%%%%%%%%%%%%%%%%%%%%%%%%%%%%%%%%%%%%%%%%%%%%%%%%%%%%%%%%%%%%%%%%%%%%%%%%%%%%%%%%%%%%%%



\section{Instituto Geogr\'afico Nacional de Costa Rica survey photo on September 4, 1971}
\label{2023-UFO-part-History-photos-1971-cr}

On September 4, 1971, an automated camera aboard a plane captured an image of a flying saucer near the Arenal Volcano in Costa Rica.
The camera was being used by the Instituto Geogr\'afico Nacional de Costa Rica to study a hydroelectric project in the area.
The occupants of the plane were not aware of the image at the time, and it wasn't until later when they were
examining the negatives that they noticed the object hovering over  Lago Cote.
The object has never been debunked and is estimated to be between 120-220 feet in diameter~\cite{Adams2022May}.

% For figures use
%
\begin{figure}[b]
\sidecaption
% Use the relevant command for your figure-insertion program
% to insert the figure file.
% For example, with the option graphics use
%\includegraphics{2023-UFO-part-History-photos-1971-cr-c}
\resizebox{.6\textwidth}{!}{ \includegraphics{2023-UFO-part-History-photos-1971-cr-c}  }
%
% If not, use
%\picplace{5cm}{2cm} % Give the correct figure height and width in cm
%
\caption{Cropped photo with the UFO inside, hovering over  Lago Cote~\cite{Adams2022May}.}

\label{2023-UFO-part-History-photos-1971-cr-cr-c}       % Give a unique label
\end{figure}

% For figures use
%
\begin{figure}[b]
\sidecaption
% Use the relevant command for your figure-insertion program
% to insert the figure file.
% For example, with the option graphics use
\resizebox{.6\textwidth}{!}{ \includegraphics{2023-UFO-part-History-photos-1971-cr-cu} }
%
% If not, use
%\picplace{5cm}{2cm} % Give the correct figure height and width in cm
%
\caption{Closeup of UFO depicted in~\ref{2023-UFO-part-History-photos-1971-cr-cr-c}.}
\label{2023-UFO-part-History-photos-1971-cr-cr-cu}       % Give a unique label
\end{figure}



\clearpage
%%%%%%%%%%%%%%%%%%%%%%%%%%%%%%%%%%%%%%%%%%%%%%%%%%%%%%%%%%%%%%%%%%%%%%%%%%%%%%%%%%%%%%%%%%%%%%%%%%%%%%%%%%%%%%%%




\section{Calvine  UFO photos  on August 4,  1990}
\label{2023-UFO-part-History-photos-1990-ca}

On the evening of Saturday, August 4,  1990, at approximately 9 pm,
two unknown men, one allegedly called Kevin Russell~\cite{YTDTMarch23}, were  located on a hillside near Calvine, a small hamlet situated approximately 35 miles north-west of Perth in Scotland,
captured six color photographs of a sizable diamond-shaped craft during the diminishing summer daylight.
The object remained stationary and silent near their location for around
ten minutes before suddenly shooting off vertically at a relatively high speed~\cite{AdamsCalvin20222Aug,ClarkeCalvin22,YTDTMarch23}.

% For figures use
%
\begin{figure}[b]
\sidecaption
% Use the relevant command for your figure-insertion program
% to insert the figure file.
% For example, with the option graphics use
%\includegraphics{2023-UFO-part-History-photos-1971-cr-c}
\resizebox{0.6\textwidth}{!}{ \includegraphics{2023-UFO-part-History-photos-1990-calvin-c}  }
%
% If not, use
%\picplace{5cm}{2cm} % Give the correct figure height and width in cm
%
\caption{Cropped photo of a diamond-shaped UFO hovering over the Scottish Highlands near Calvin~\cite{AdamsCalvin20222Aug,ClarkeCalvin22}
To this date, the author of the photo, an alleged Kevin Russell~\cite{YTDTMarch23}, has not been identified.}

\label{2023-UFO-part-History-photos-1990-calvin-c}       % Give a unique label
\end{figure}

% For figures use
%
\begin{figure}[b]
\sidecaption
% Use the relevant command for your figure-insertion program
% to insert the figure file.
% For example, with the option graphics use
\resizebox{.6\textwidth}{!}{ \includegraphics{2023-UFO-part-History-photos-1990-calvin-cu} }
%
% If not, use
%\picplace{5cm}{2cm} % Give the correct figure height and width in cm
%
\caption{Closeup of UFO depicted in~\ref{2023-UFO-part-History-photos-1990-calvin-c}.}
\label{2023-UFO-part-History-photos-1990-calvin-cu}       % Give a unique label
\end{figure}


\clearpage
%%%%%%%%%%%%%%%%%%%%%%%%%%%%%%%%%%%%%%%%%%%%%%%%%%%%%%%%%%%%%%%%%%%%%%%%%%%%%%%%%%%%%%%%%%%%%%%%%%%%%%%%%%%%%%%%




\section{Cigar shaped object south of Spain on January 23,  2010}
\label{2023-UFO-part-History-photos-2010-csssp}

On the late afternoon of January 23, 2010, during a flight from Amsterdam to Malaga on a Boing 737, an object caught
the attention of the pilot, who asked the then copilot Christiaan van Heijst~\cite{vanHeijst2010Jan,Heijst2023Feb}\index{van Heijst, Christiaan}
if he could see the object as well.
It appeared as a rounded rectangular or cigar shape, similar to the rear side of a contrail, far ahead of them.
However, they were flying rather high for a commercial aircraft at 41,000 ft,
and there was minimal to no traffic around.
They were on a direct route from Pamplona (PPN) in the Madrid Flight Information Region (LECM) to Malaga (AGP),
and they observed the object without changing direction.
Estimating its distance was difficult, but they first guessed it could be 100 (nautical) miles ahead of them,
or possibly much further.
Considering the distance, the object was much higher than any commercial traffic, arousing our curiosity.

While flying over the north of Spain, they inquired with Madrid Air Traffic Control
if there was any known traffic ahead and above of their aircraft, but they reported no traffic and were puzzled by this inquiry.
They informed Madrid Air Traffic Control of the object, and were directed to contact military air traffic control asap.
The military was prompt, interested, and took serious note of what Christiaan van Heijst observed.
They confirmed no other traffic in the area, including military activity, commercial traffic, or weather balloons.
Van Heijst and the pilot observed the stationary object for at least an hour until they descended into the clouds over Malaga,
specifically at position CRISA, far to the south of the Iberian Peninsula.
Van Heijst mentioned that the cigar shape of the object was more distinguishable than in the photo~\cite{vanHeijst2010Jan},
and that it remained motionless above the horizon.


\newpage
% For figures use
%
\begin{figure}[b]
\sidecaption
% Use the relevant command for your figure-insertion program
% to insert the figure file.
% For example, with the option graphics use
%\includegraphics{2023-UFO-part-History-photos-1971-cr-c}
\resizebox{1\textwidth}{!}{ \includegraphics{2023-UFO-part-History-photos-2010-csssp-c}  }
%
% If not, use
%\picplace{5cm}{2cm} % Give the correct figure height and width in cm
%
\caption{Cigar shaped object to the south of Spanish air space, recorded from a Boing 737 on January 23, 2010.
Courtesy of Christiaan van Heijst~\cite{vanHeijst2010Jan}.}

\label{2023-UFO-part-History-photos-2010-csssp-c}       % Give a unique label
\end{figure}

\vskip 1 cm

% For figures use
%
\begin{figure}[b]
\sidecaption
% Use the relevant command for your figure-insertion program
% to insert the figure file.
% For example, with the option graphics use
\resizebox{.6\textwidth}{!}{ \includegraphics{2023-UFO-part-History-photos-2010-csssp-cu} }
%
% If not, use
%\picplace{5cm}{2cm} % Give the correct figure height and width in cm
%
\caption{Closeup of object depicted in~\ref{2023-UFO-part-History-photos-2010-csssp-c}.}
\label{2023-UFO-part-History-photos-2010-csssp-cu}       % Give a unique label
\end{figure}
