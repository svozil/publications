\documentclass[%
 reprint,
 %twocolumn,
 %superscriptaddress,
 %groupedaddress,
 %unsortedaddress,
 %runinaddress,
 %frontmatterverbose,
  %preprint,
 showpacs,
 showkeys,
 preprintnumbers,
 %nofootinbib,
 %nobibnotes,
 %bibnotes,
 amsmath,amssymb,
 aps,
 prl,
 %  pra,
 % prb,
 % rmp,
 %prstab,
 %prstper,
  longbibliography,
 %floatfix,
 %lengthcheck,%
 ]{revtex4-1}

%\usepackage{cdmtcs-pdf}

\usepackage{amssymb,amsthm,amsmath}


\theoremstyle{definition}
\newtheorem{definition}{Definition}
\newtheorem{theorem}{Theorem}
\newtheorem{conjecture}{Conjecture}

\theoremstyle{remark}
\newtheorem*{motivation}{Motivation}
\newtheorem*{note}{Note}

\usepackage{tikz}
\usepackage[breaklinks=true,colorlinks=true,anchorcolor=blue,citecolor=blue,filecolor=blue,menucolor=blue,pagecolor=blue,urlcolor=blue,linkcolor=blue]{hyperref}
\usepackage{graphicx}% Include figure files
\usepackage{url}

\usepackage{xcolor}

\begin{document}


\title{The plasticity of length, and the velocity of light}

%\cdmtcsauthor{Karl Svozil}
%\cdmtcsaffiliation{Vienna University of Technology}
%\cdmtcstrnumber{407}
%\cdmtcsdate{September 2011}
%\coverpage

\author{Karl Svozil}
\affiliation{Institute for Theoretical Physics, Vienna
    University of Technology, Wiedner Hauptstra\ss e 8-10/136, A-1040
    Vienna, Austria}

\email{svozil@tuwien.ac.at} \homepage[]{http://tph.tuwien.ac.at/~svozil}


\pacs{41.20.Jb, 42.25.Bs,11.10.Gh}
\keywords{velocity of light, electromagnetic wave propagation, vacuum polarization, renormalization}
%\preprint{CDMTCS preprint nr. 407/2011}

\begin{abstract}
We discuss the plasticity of the velocity of light in vacuum.
\end{abstract}

\maketitle

The postulate of the constancy of the velocity of light in vacuum is one of the pillars of contemporary physics.
Since 1983 it has been {\em conventionalized}
by the {\em International System of Units}, which defines the meter, the base unit of length,
through {\em fixing} the velocity of light at exactly 299.792.458 meters per second
\cite{pet-83,0026-1394-19-4-004}.
% http://www.bipm.org/utils/common/pdf/si_brochure_8_en.pdf
This choice has two immediate consequences:
(i)
the relativistic Lorentz transformations
can be derived
by Alexandrov's theorem of incidence geometry \cite{lester},
and
(ii)
the meter as the base unit of length has become an operational quantity
\cite{peres-84,pet-row--84}.
Its invariance is supported by the physical principles of relativistic kinematics;
in particular, by the form invariance of the equations of motions, such as
Maxwell's equations.

Alas in what follows we shall regress to the (with some translation equivalent)
position expressed in Einstein's centennial paper \cite{ein-05},
in which the constant velocity of light is taken to be empiricial rather then conventional.
Thereby, we shall be less concerned with changes of inertial frames,
but rather investigate the physical process causing
or contributing to light propagation in vacuum.

In (naive) operational terms, the velocity of light
can be defined by the quotient of the ``distance travelled''
(measured by the a factor with respect to some fixed unit of length)
and the ``time taken''
(measured by a factor with respect to some time cycles)
by light in vacuum.
As it turns out it is very difficult to substanciate this definition
in situations where, for instance,
wave-particle duality is critical \cite{PhysRevA.48.R34,Chiao:02},
and various concepts of velocity become almost arbitrarily ``faster''
or ``slower'' than what is usually taken to be ``the velocity of light.''

Also, in field theory,
there is no comprehensive dynamic theory of the velocity of light.
Often, the (phase) velocity of light is taken to be inversely proportional
to the index of refraction, which in turn
is the related to the square of the product of the vacuum permittivity and the permeability.
The motivation for this identification is the occurrence of a
``velocity parameter characteristic of the isotropic linear propagation medium''
in the Maxwell equations [e.g., Eqs. (7.1)--(7.5) in Ref. \cite{jackson}] of classical electrodynamics,
yielding the Helmholtz wave equation.

Quantum field theory, in particular also perturbation theory,
``(re)constructs'' physical entities (such as mass or charge) by renormalizing
the parameters of the ``free theory'' by also taking into account the interactions
of all involved fields. Thereby, rather subtle corrections modify the free parameter,
and make it dependent on the environment
(e.g., the electron mass decreases in the presence of a conducting plate
\cite{PhysRevD.34.1429}).

Indeed, before considering the velocity of light one needs to clarify what ``light actually is.''
There appear to exist at least two alternatives.
(i) High energy physics defines the photon, the quantum of electromagnetic radiation,
as {\it ``atomos,''} an indivisible, irreducible particle
\cite[p. 27]{PhysRevD.86.010001} coupling to charges.
(ii) Photons may be considered to be ``excitations'' of the Dirac sea of electrons;
just as phonons might be considered to be represent ``collective excitations'' in some
elastic arrangement of particles.

Let us consider the orthodox, quantum field theoretical, case first.
In quantum electrodynamics, the radiative corrections,  in particular vacuum polarization ${\bf \Pi}$
-- the ``decay'' and ``recombination'' of a virtual (i.e., off-shell) electron-positron pair --
induces a kind of ``effective photon mass'' \cite{PhysRev.82.664,PhysRevD.10.492,PhysRevD.12.1132}
$
M(k)=\epsilon^\mu {\bf \Pi}_{\mu \nu}(k)\epsilon^\nu
$
with the effective dispersion relation
$
{\bf k}^2+ M(k)=(k^0)^2
$,
so that $k^\mu=({\bf k},k^0=\omega)$;
which can be approximated by the linear expansion
\begin{equation}
\vert  {\bf k} \vert \approx \omega - \frac{1}{2 \omega} M(k).
\end{equation}
If the refractive index is
$n(\omega )= \vert {\bf k} \vert / \omega  $,
then
\begin{equation}
n(\omega )\approx 1 - \frac{1}{2 \omega^2}\epsilon^\mu {\bf \Pi}_{\mu \nu}(k)\epsilon^\nu .
\end{equation}
Hence the effective  velocity of light in terms of the change of the vacuum polarization is
\begin{equation}
c^\ast
=
c\left[1+  \frac{1}{2 \omega^2}\epsilon^\mu {\rm \Delta }{\bf \Pi}_{\mu \nu}(k)\epsilon^\nu \right]
.
\label{2013-c}
\end{equation}

In this framework, the effective velocity of light $c^\ast$ might exceed the vacuum value $c$ if,
for whatever reason,
the vacuum polarization ${\bf \Pi}$ increases.
So far, two such physical scenarios have been proposed:
(i) in the proximity of parallel conductors \citep{scharnhorst,milonni,Scharnhorst-1998,barton,0305-4470-26-8-024},
and
(ii) in a regien occupied by fermions
\cite{svozil-putz-sol}.
Both involve situations in which the polarizability of the vacuum is {\em decreased}
--
either through less phonon modes in the Casimir vacuum inbetween conductors,
or through the Pauli exclusion principle in the presence of real (on-shell) fermions.

This suggest that {\em photons are slowed down by vacuum polarization}.
One could extend this observation by speculating that an attenuated photon has infinite speed,
thereby connecting space-time points {\em instantaneously}
(with respect to inertial frames generated by light propagating in normal vacuum).

An alternative model of the photon would be motivated by analogy to phonons.
Thereby, the photon is perceived as a collective excitation of the Dirac sea,
which can be approximated by a linear chain \cite{Henley-Thirring-EQFT}.


It is often argued that effective velocities of light exceeding the vacuum velocity
allow time paradoxa \cite{recami:87}.
However, time paradoxa are based on very specific assumptions:
on the one hand, they suppose the existence of a ``dominating'' interaction spreading with velocities faster-than-light,
and
on the other hand they hold space-time coordinates conventionalized by the assumption of the
constancy of the velocity light.
In no way they allow the construction of space-time scales by the greater velocities, which would,
with the same conventions and by Alexandrov's theorem,
result in an analogue theory of relativity \cite{svozil-relrel}.
It is as if one would claim that time paradoxes occur due to the construction of space-time scales
with sound instead of light.
Or, suppose gravity waves (if they exist) travel faster-than-light.
That would rather not give aise to time paradoxes,
but to a further conventionalization of our view on space-time scales \cite{svozil-2001-convention}.

One may also suppose that, due to extensional reasons such as quantum complementarity \cite{milonni},
for all practical purposes (FAPP \cite{bell-a}) time paradoxes do not occur.








\begin{acknowledgments}
 This research has been partly supported by FP7-PEOPLE-2010-IRSES-269151-RANPHYS.
\end{acknowledgments}

 \bibliography{svozil}


\end{document}
