\documentclass[pra,amssymb,twocolumn]{revtex4}
%\documentclass[pra,showpacs,showkeys,amsfonts]{revtex4}
\usepackage[T1]{fontenc}
\usepackage{textcomp}
\usepackage{graphicx}
%\usepackage{longtable}
%\documentstyle[amsfonts]{article}
%\RequirePackage{times}
%\RequirePackage{courier}
%\RequirePackage{mathptm}


\begin{document}

\title{Dimensional reduction via dimensional shadowing}

\author{Karl Svozil}
\email{svozil@tuwien.ac.at}
\homepage{http://tph.tuwien.ac.at/~svozil}
\affiliation{Institut f\"ur Theoretische Physik, University of Technology Vienna,
Wiedner Hauptstra\ss e 8-10/136, A-1040 Vienna, Austria}

\begin{abstract}
Dimensional shadowing is introduced as a formal method
to reduce extra dimensions in configuration space
by considering a fractal subset.
The Hausdorff dimension of the fractal is then perceived
as the physical dimension of configuration space.
\end{abstract}


\maketitle


1.
Unified field theories suggest a $N$--dimensional configuration
space with $\bar N=N-4>0$ extra dimensions not perceived in
nature.
The common heuristic reasons for this proposition are:
(i) the ``volume'' [Hausdorff measure] does not scale like
$\mu_H(\delta ,{\Bbb R}^N)=\delta^N\, \mu_H(1,{\Bbb R}^N)$,
where $\delta $ is some length scale and 1 stands for its unity.
Instead, experience tells us that increasing (or decreasing)
the spacial size of an object by $\delta $,
changes its volume by approximately $\delta^3$, corresponding
to a spacial [Hausdorff, if not denoted otherwise]
dimension $D_s=3$;
(ii) the number of spacial
degrees of freedom $D^L$ is not $N$ but three,
corresponding to a threedimensional vector space;
(iii) longrange static potentials around a [conserved] point
charge, behaving as $r^{2-D^P}$ for $D^P>2$, when $r$ is the
distance from the charge, suggest a
dimensional value $D^P$ of approximately three.
There is good evidence, that all these parameters
coincide and $D_s\approx D^L\approx D^P\approx 3$.


According to these observations, physical configuration space
is modelled as a product space ${\Bbb R}^4={\Bbb R}^3_s\times {\Bbb R}_t$,
where ${\Bbb R}_t$ stands for the time ``continuum''.
The dimension $D$  of its Cartesian product is \cite{falconer1}
$D\ge D_s+D_t\approx 4$.
Hence,
some kind of ``dimensional reduction'' has to effectively decrease the
number of operational attainable dimensions.
These may be defined via the Hausdorff dimension, or the
maximal number of linear independent vectors of a vector space
[this assumes the existence of a vector space], via the distance dependence of potentials, or otherwise.
However, it is in no way trivial, that all these definitions coincide.
The common notion of dimensional reduction in the Kaluza-Klein approach assumes
{\it compactification}:
Configuration space is assumed as ${\Bbb R}^4\times S^{\bar N}$,
where $S^{\bar N}$ is a compact $\bar N$--dimensional
manifold.
These
extra dimensions are assumed to be ``{\it curled up}''
to very small sizes, such that these additional degrees
of freedom could be observed only in the high energy regime.


2.
In this brief communication a very different approach to
dimensional reduction is pursued:
configuration space $X$ is assumed to be a fractal
{\it embedded} in a higherdimensional space ${\Bbb R}^N$
with arbitrary integer
dimension $D({\Bbb R}^N)=N\ge 4$.
It is then assumed, that due to some [yet unknown]
mechanism, the dimension of the configuration space $X$ is approximately
equal to four $D(X)\approx 4$.


Assumed is a parametrization of ${\Bbb R}^N$ as usual;
i.e. points are written in $N$--component vector notation
$\vec a=(a_1,\ldots ,a_N)$.
The standard Euclidean metric $d^N(x,y)=[\sum_{i=1}^N(y_i-x_i)^2]^{1/2}$
can be applied.
When the components of $N$ orthogonal basis vectors $\vec e^{\; (i)},
\; i=1,\ldots ,N$ are given by $e^{(i)}_j=\delta_{ij}$,
any vector may be written as $\vec a=\sum_{i=1}^Na^{(i)}\vec e^{\; (i)}$
with $a^{(i)}=a_i$.
In a vector space [which is closed under addition of arbitrary vectors
and multiplication of scalars], a dimension $D^L$ can be defined as the
maximal number of linear independent vectors $\vec b^{(i)}$,
for which $\sum_{i=1}^{D^L}\alpha^{(i)}\vec b^{(i)}=\vec 0$ if and only
if all scalars $\alpha^{(i)}=0$. Note however, that $D^L$ and $D$
need not coincide, as can be inferred from
rational scalars, where $D^L=N$, but $D=0$
[since ${\Bbb Q}^N$ is a countable point set, $D({\Bbb Q}^N)=0$].


$X$ has been modelled to reproduce the observed scaling
property of the volume $\mu_H(\delta , X)\approx \delta^4\mu_H
(1,X)$.
Concepts of linear independent vectors cannot be directly applied,
since $X$ is no vector space [with trivial exceptions such as $X={\Bbb R}^4$].
However, it may be conjectured, that the restrictions on $X$
reduce the maximal number of linear independent vectors from
$N$ to $n<N$, presumably four.
It has indeed be shown$^{1,3}$, that associated to every integerdimensional
{\it regular}
[{\it rectifiable}] $n$--dimensional fractal
embedded in ${\Bbb R}^N$, is a locally defined tangential $n$--dimensional
vectorsubspace of ${\Bbb R}^N$.


3.
When $D(X)=n$ is an integer, it can be shown \cite{WarrenSiegel}
that the standard calculus, such as integration and Fourier
analysis on $n$--dimensional manifolds, can be applied to $X$.
This holds true even for generalizations to noninteger
dimensions.
Quantum mechanical matrix elements would be identical
to standard calculations in ${\Bbb R}^4$ Minkowski space--time.


4.
The question is, do [{\it Lipschitz}] maps exist which project
$X$ onto a lowerdimensional manifold,
thereby preserving its measure theoretic {\it and} its
topologic structure [is $X$ rectifiable] ?
It can be shown$^1$,
that an orthogonal projection $\pi (X)$ onto ${\Bbb R}^n$,
yields for a very general class of fractals [Souslin sets],
$D(\pi (X))=\min (D(X),n)$.
However, orthogonal projections [such as
$\pi ((a_1,\ldots ,a_4,a_5,\ldots ,a_N))=(a_1,\ldots ,a_4)$]
are not preserving the topological structure of $X$.
In the low energy regime, orthogonal projection is
equivalent to standard compactification,
where effectively ${\Bbb R}^4\times S^{\bar N}\rightarrow {\Bbb R}^4$ is assumed.



The following general result has been stated quite recently$^1$,
although specific lowdimensional examples [$N=2, \; n=1$ etc.]
were proven much earlier \cite[see in particular 3. 2. 19 and 3. 3. 22]{federer1}:
Let $X$ be a $n$--dimensional subset of ${\Bbb R}^N$, where $n$ is an
integer.
The following statements are equivalent:
\begin{itemize}
\item[(i)]$X$ is {\it regular}, that is its density
$\lim_{r\rightarrow 0}r^{-n}\mu_H(B_r^N(x))$ exists
almost everywhere. $B_r^N(x)=\{y:\, y\in X,\, d^N(x,y)\le r\} $
is a ball in $X$ with radius $r$ and center $x$;
\item[(ii)]$X$ is {\it countable n rectifiable},
i.e. it can be decomposed into
$X=\bigcup_{i=1}^\infty \varphi_i(Y_i)\cup G$,
where Lipschitz functions $\varphi_i$
map bounded subsets $Y_i$ of ${\Bbb R}^n$ onto $X$ and $\mu_H(G)=0$
[i.e. this decomposition into Lipschitz functions holds in
almost all of $X$].
A Lipschitz function $\varphi $ requires
$d^N(\varphi (a),\varphi (b))\le {\rm Lip}(\varphi )
d^n(a,b)$. Here, ${\rm Lip}(\varphi )$ is some Lipschitz constant,
$a,b\in {\Bbb R}^n$, and $\varphi (a),\varphi (b)\in X$.
Hence, when $Y_i=B_\epsilon ^n(a)$ is a neighborhood of $a\in Y_i$,
$\varphi_i(Y_i)=B_\delta ^N(\varphi_i(a))\subset X$
is a neighborhood of $\varphi_i(a)$ in $X$ with
$\delta \le \epsilon {\rm Lip}(\varphi_i )$ [see Fig. 1].
More generally, when $\{ Y_i\} $ is a filter in ${\Bbb R}^n$,
$\{ \varphi_i(Y_i)\} $ is a filter in $X$;
\item[(iii)]$X$ has a $n$--dimensional tangent vectorsubspace
of ${\Bbb R}^N$ almost everywhere$^3$.
\end{itemize}


Hence, (i)--(iii) suggest, that every regular $n$--integerdimensional
fractal subset of ${\Bbb R}^N$
is {\it locally} perceived as a $n$--dimensional
vectorspace ${\Bbb R}^n$.


5.
By increasing the dimension of $X$ [heuristically speaking,
by ``filling up more and more'' of ${\Bbb R}^N$],
the $\bar N=N-4$ dimensions of the theory open up.
They correspond to additional degrees of freedom in configuration space.
Dimensional {\it saturation} occurs at $D(X)=N$.
A similar argument holds true for decreasing elements of $X$.
In particular, when $X$ becomes countable [it still could
be dense], $D(X)=0$.



6.
A lowerdimensional configuration space has been
``emulated'' by a fractal subset of a higherdimensional
manifold, yielding a sort of ``{\it shadowing}''
of ${\Bbb R}^N$ onto a smallerdimensional set
which is locally perceived as ${\Bbb R}^n\subset {\Bbb R}^N$.
Dimensional shadowing may present an alternative way of
dimensional reduction.
Like reduction by ``curling up'' extra ``compactified'' dimensions,
it is a formal procedure so far, which would have to be motivated
by physical reasoning in order to
transcend its purely technical virtue.


The author acknowledges discussions with Anton Zeilinger.
This work was supported by BMWF, project number 19.153/3-26/85.


\begin{figure}
\begin{center}
\end{center}
\caption{A Lipschitz function $\varphi $ maps the
1--dimensional ball $B^1_\epsilon $ onto
$B^2_\delta \subset X$ with diameter $\delta \le
\epsilon {\rm Lip}(\varphi )$.
\label{1996-shatex-f1}}
\end{figure}

\bibliography{svozil}
\bibliographystyle{osajnl}




\end{document}
