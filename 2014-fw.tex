\documentclass[%
 %reprint,
  twocolumn,
 %superscriptaddress,
 %groupedaddress,
 %unsortedaddress,
 %runinaddress,
 %frontmatterverbose,
 % preprint,
 showpacs,
 showkeys,
 preprintnumbers,
 %nofootinbib,
 %nobibnotes,
 %bibnotes,
 amsmath,amssymb,
 aps,
 % prl,
  pra,
 % prb,
 % rmp,
 %prstab,
 %prstper,
  longbibliography,
 %floatfix,
 %lengthcheck,%
 ]{revtex4-1}

%\usepackage{cdmtcs-pdf}

\usepackage{mathptmx}% http://ctan.org/pkg/mathptmx

\usepackage{amssymb,amsthm,amsmath}

\usepackage{tikz}
\usepackage[breaklinks=true,colorlinks=true,anchorcolor=blue,citecolor=blue,filecolor=blue,menucolor=blue,pagecolor=blue,urlcolor=blue,linkcolor=blue]{hyperref}
\usepackage{graphicx}% Include figure files
\usepackage{url}

\usepackage{xcolor}

\begin{document}


\title{Dualistic free will: necessary physical and computational conditions for miracles and oracles}

%\cdmtcsauthor{Karl Svozil}
%\cdmtcsaffiliation{Vienna University of Technology}
%\cdmtcstrnumber{407}
%\cdmtcsdate{September 2011}
%\coverpage

\author{Karl Svozil}
\affiliation{Institute for Theoretical Physics, Vienna
    University of Technology, Wiedner Hauptstra\ss e 8-10/136, A-1040
    Vienna, Austria}
\email{svozil@tuwien.ac.at} \homepage[]{http://tph.tuwien.ac.at/~svozil}


\pacs{01.70.+w}
\keywords{free will, determinism, indeterminism, choice, agent, oracle, miracle}
%\preprint{CDMTCS preprint nr. 407/2011}

\begin{abstract}
Free will based on a dualistic scenario avoids the problems encountered in totally (in)deterministic universes. It requires a fine-tuned mixture of determinism and indeterminism allowing gaps in the natural laws. These gaps facilitate amd permit the insertion and communication of intentions and choices via interfaces serving as Cartesian cuts. Algorithmic models are computer games, taking signals as input from some agent acting as an oracle.
\end{abstract}

\maketitle


\section{Dualistic scenario}

Like {\it Odysseus} trapped between {\it Scylla} and {\it Charybdis},
free will is severely restricted by, if not inconsistent with,
physical determinism as well as complete indeterminism.
Ontologically a clockwork universe, as well as one pushing uncontrollable chance,
leaves no room for willable alternatives.

Because on the one hand determinism blocks free will by the principle of sufficient reason.
In its most extreme form expressed, for instance, by the unitary quantum mechanical state evolution
(amounting to mere orthonormal base changes~\cite{Schwinger.60})
a universe evolving through permutations,
that is, one-to-one transitions between physical states, might be beautiful or
``rich'' in the sense of allowing ornamentation,
but it lacks any freedom of choice.
On the other hand, indeterminism,
at least in the form of the {\it creatio ex nihilo} of events without any cause,
leaves no room for choice either: because if events emerge {\it ex nihilo} and uncontrollably,
there is no freedom of choice between alternatives.


For both scenarios,
the only consistent way of maintaining free will appears to be epistemic,
idealistic~\cite{stace1}, and maybe existentialistic: consciousness
free will always (or at least sometimes~\cite{libet}) is a subjective illusion~\cite{Wegner-illusion,baer-et-al-free-will}.
Alas, to quote Goethe, {\it ``none are more hopelessly enslaved than those who falsely believe they are free.''}
%Elective Affinities, Die Wahlverwandtschaften, Hamburger Ausgabe, Bd 6 (Romane und Novellen I), dtv Verlag, M�nchen, 1982, S. 397 (II,5)
For Camus~\cite{camus-mos}, recognition of and revolt against any locked {\it conditio humana} (beyond suicide)
suggests a kind of dignifying absurd freedom.
Thereby the agent consciously realizes
that all hope to escape this {\em Cartesian prison}~\cite[Meditation~1.12]{descartes-meditation}
is doomed, yet resents it.

Despite this gloomy perspective, there is a third alternative which we shall discuss here:
the possibility that {\em transcendent agents} interact with a(n) (in)deterministic universe via suitable {\em interfaces.}
Immanence refers to all operational, intrinsic means available to embedded observers~\cite{toffoli:79,svozil-94} from within some universe;
whereas transcendence goes beyond these means.
In what follows we shall refer to the transcendental universe as the beyond.
We shall also adopt the terminology of Calude and Poznanovi\'{c}~\cite{CaludePoznanovic}
which identifies four components for a formal discussion of free will:
agents, objects, contexts and choices.
Informally speaking, an agent chooses an object in some context or environment.


For the sake of  metaphorical models,
take Eccles' mind-brain model~\cite{eccles:papal},
or consider a virtual reality, more particular, a computer game with various human players represented
by avatars.
There, the universe is identified with the game world created by an algorithm (essentially, some computer program),
and the transcendental agent is identified with the human gamer.
The interface essentially consists of any kind of device and method connecting the human body with the avatar.
It possesses two faces or handles: one into the universe, and a second one into the beyond.

Human players constantly input or inject choices through the interface, and {\it vice versa.}
In this {\em hierarchical, dualistic} scenario, such choices need not solely (or even entirely) be determined
by any conditions of the game world:
human players are transcendental with respect to the context of the game world,
and are subject to their own universe they live in (including the interface).
Alas the game world itself is totally deterministic in a very specific way:
it allows the player's input from beyond; but other than that it is created by a computation.
One may think of a player as a specific sort of indeterministic (with respect to intrinsic means)
{\em oracle}, or subprogram, or functional library.

Another algorithmic metaphor is an {\em operating system},
or a {\em real-time computer system}, serving as context.
(This is different from a classical Turing machine, whose emphasis is not on interaction with some user-agent.
The user is identified with the agent.
Any user not embedded within the context is thus transcendent with respect to this computation context.
In all these cases the  real-time computer system acts deterministically on any input received from the agent.
It observes and obeys commands of the agent handed over to it {\em via} some interface.
An interface could be anything allowing communication between the real-time computer system and the (human) agent;
say a touch screen, a typewriter(/display), or any brain-computer interface.
One may also say that without any such intervention the operating system remains dormant or idle.
The ``meaning'' of the real-time computer system is the interaction with, and response to, the agent.
The agent here has the function of an oracle which is constantly monitored.


If we translate these algorithmic metaphors into the context of our own universe,
we have to observe whether all the respective components are physically feasible.
In particular, we need to ask the following questions:
(i) What might serve as a context; that is, do there exist natural laws which could be identified with the game universe/operating/real-time computer system?
(ii) Do there exist potential interfaces in our universe allowing communication with some (supposedly transcendental) agent?
(iii) Are there constraints on such interventions~\cite{maryland,greenberger-svozil,svozil-07-physical_unknowables}?
We may also speculate about the transcendental nature of any agent communicating with our universe {\em via} such interface.
Whereas (i) essentially is the subject of natural science, issues regarding (ii) are seldomly mentioned explicitly.
Therefore, in what follows emphasis will be on this latter subject.

A necessary (but not sufficient) condition for the possibility of this scenario, and, in particular, for the existence of suitable interfaces, is the
existence of {\em gaps in the natural laws} and the causal fabric of the universe, as exposed by Frank~\cite[Chapter~{III}, Sec.~12]{frank,franke}.
Because if there were no possibilities to inject information or other matter or content
into the universe from beyond, there would be no possibility to manipulate the universe,
and therefore no substantial choice.

Gaps in the causal fabric are no sufficient condition for the existence of free will,
because these gaps may be completely filled up with {\em creatio ex nihilo:}
in that way, an otherwise deterministic universe is not steered by an agent,
but erratically pushed around by chance.
Indeed, this latter scenario seems to be the foundational,
metaphysical basis of the orthodox concepts
underlying deterministic chaos, spontaneous symmetry breaking, and instabilities due to discontinuities~\cite[p.~211-212]{Campbell-1882}
as well as of the irreducible indeterminism characterizing certain individual quantum outcomes.


\section{Individual, singular events}

How can one intrinsically decide between chance and an agent executing free will?
The interface must in both cases employ gaps in the intrinsic laws of the universe,
thereby allowing steering and communicating with it in a feasible, consistent manner.
That excludes any kind of absolute predictability of the signals emanating from it.
(Otherwise, the behaviour across the interface would be predictable and deterministic.)
Hence, for an embedded observer~\cite{toffoli:79} employing intrinsic  means
which are operationally available in his universe~\cite{toffoli:79},
no definite criterion can exist to either prove or falsify claims regarding mere
chance (by {\it creatio ex nihilo}) {\it versus} the free choice of an agent.
Both cases
--
free will of some agent as well as complete chance
--
express themselves by irreducible intrinsic indeterminism.

Nevertheless,
whether any such metaphysical claim is strictly non-operational and theological in the (game) universe,
as expressed by Frank, remains conjectural. In particular, suppose dualistic transgressions are feasible.
Thereby, the context associated with intrinsic means
may be extended beyond the interface to include (parts of) the beyond.
One such scenario is the flattening of dualism discussed below.

As has already been observed by Frank~\cite[Kapitel~{III}, Secs.~14,~15]{frank},
in order for any {\em miracle} or free will to manifest itself
through any such gap in the natural laws, it needs to be {\em systematic,}
{\em according to a plan}
and
{\em intentional} (German {\it planm\"a\ss ig}).
In its purest form, any dualistic choice manifests itself in a single bit.
Such a dichotomic signal may be communicated through a noisy channel requiring more than one bit,
or directly by the communication of a classical bit.
Again, it is important to stress that the occurrence of a single bit (or any finite concatenation thereof)
cannot differentiate between chance or choice.
(Due to Ramsey theory, absolute randomness is vacuous even for infinite sequences~\cite{2014-nobit,CaludePoznanovic}.)


\section{Physical interfaces}

We are now turning our attention to concrete physical
interfaces which are not forbidden within a lawful universe.
All such possibilities utilize gaps in the natural laws.





\subsection{Spontaneous symmetry breaking}

Already in 1873, Maxwell identified a certain kind of {\em instability} at {\em singular points}
as rendering a gap in the natural laws \cite[211-212]{Campbell-1882}:
{\em ``$\ldots$~when an infinitely small variation in the present state may bring about a finite difference in the state of the
system in a finite time, the condition of the system is said to be unstable.
It is manifest that the existence of unstable conditions renders impossible the prediction of future events, if our
knowledge of the present state is only approximate, and not accurate.
$\ldots$~the system has a quantity of potential energy, which is
capable of being transformed into motion, but which cannot begin to be so transformed till the system has reached
a certain configuration, to attain which requires an expenditure of work, which in certain cases may be
infinitesimally small, and in general bears no definite proportion to the energy developed in consequence thereof.''}

Fig.~\ref{fig:2014-fw-instability} depicts a one dimensional gap configuration envisioned by Maxwell: a
{\em ``rock loosed by frost and balanced on a singular point of the mountain-side, the little spark which
kindles the great forest,~$\ldots$''}
On top, the rock is in perfect balanced symmetry.
A small perturbation or fluctuation causes this symmetry to be broken,
thereby pushing the rock either to the left or to the right hand side of the potential divide.
This dichotomic alternative can be coded by $0$ and by $1$, respectively.
	\begin{figure}
		\begin{centering}
\unitlength 3mm % = 2.845pt
\linethickness{0.4pt}
\ifx\plotpoint\undefined\newsavebox{\plotpoint}\fi % GNUPLOT compatibility
\begin{picture}(9,7)(0,0)
\thicklines
\put(0,0){\color{blue}\line(1,0){3.0}}
\put(9,0){\color{blue}\line(1,0){3.0}}
\put(3,0){\color{orange}\line(1,2){3.0}}
\put(9,0){\color{orange}\line(-1,2){3.0}}
\put(6,6.9){\color{black}\circle*{2}}
\put(1.5,1){\color{gray}\circle*{2}}
\put(10.5,1){\color{gray}\circle*{2}}
\put(1.5,1){\color{white}\makebox(0,0)[cc]{$0$}}
\put(10.5,1){\color{white}\makebox(0,0)[cc]{$1$}}
\end{picture}
		\end{centering}
		\caption{(Color online) A gap created by a black particle sitting on top of a potential well.
The two final states are indicated by grey circles. Their positions can be coded by $0$ and $1$, respectively.}
		\label{fig:2014-fw-instability}
	\end{figure}

One may object to this scenario of {\em spontaneous symmetry breaking}
by maintaining that, if indeed the symmetry is perfect, there is no movement,
and the particle or rock stays on top of the tip (potential).
Any slightest movement might either result from a microscopic asymmetry of the initial state of the particle,
or from fluctuations of any form, either in the particle's position,
or by the surrounding environment of the particle.
For instance, any collision of gas molecules with the rock may push the latter over the edge
by thermal fluctuations.



\subsection{Quantum oracles}

A quantum mechanical gap can be realized by a {\em half-silvered mirror}~\cite{svozil-qct,stefanov-2000,zeilinger:qct},
with a 50:50 chance of transmission and reflection,
as depicted in Fig.~\ref{fig:2014-fw-qcointoss}.
A gap certified by quantum value indefiniteness necessarily has to operate with more than two exclusive outcomes~\cite{PhysRevA.89.032109}.
Ref.~\cite{2012-incomput-proofsCJ} presents such a qtrit configuration.
	\begin{figure}
		\begin{centering}
%\grade{\on}
%\emlines{\off}
%\epic{\off}
%\beziermacro{\on}
%\reduce{\on}
%\snapping{\off}
%\pvinsert{% Your \input, \def, etc. here}
%\quality{8.000}
%\graddiff{0.005}
%\snapasp{1}
%\zoom{8.0000}
\unitlength 0.7mm % = 2.845pt
\linethickness{0.4pt}
\ifx\plotpoint\undefined\newsavebox{\plotpoint}\fi % GNUPLOT compatibility
\begin{picture}(94.75,44.875)(0,0)

\put(10,10){\circle{10}}
\thinlines
%\emline(6.452,6.452)(13.568,13.523)
\multiput(6.452,6.452)(.033885714,.033671429){210}{\line(1,0){.033885714}}
%\end
%\emline(13.523,6.452)(6.408,13.523)
\multiput(13.523,6.452)(-.033880952,.033671429){210}{\line(-1,0){.033880952}}
%\end

\thicklines
{\color{blue}
\put(15,10){\line(1,0){44.5}}
%\dottedline(59.75,10)(94.75,10)
\multiput(59.68,9.93)(.972222,0){37}{{\rule{.8pt}{.8pt}}}
%\end
%\dottedline(59.875,9.875)(59.875,44.875)
\multiput(59.805,9.805)(0,.972222){37}{{\rule{.8pt}{.8pt}}}
%\end
}
\put(30,10){\color{black}\circle*{8}}
\put(59.68,39.93){\color{gray}\circle*{8}}
\put(89.68,9.93){\color{gray}\circle*{8}}
\put(59.68,39.93){\color{white}\makebox(0,0)[cc]{$0$}}
\put(89.68,9.93){\color{white}\makebox(0,0)[cc]{$1$}}

\thicklines
{\color{orange}
%\dashline{1}(50,0)(70,20)
\multiput(49.93,-.07)(.0322581,.0322581){20}{\line(1,0){.0322581}}
\multiput(51.22,1.22)(.0322581,.0322581){20}{\line(1,0){.0322581}}
\multiput(52.51,2.51)(.0322581,.0322581){20}{\line(1,0){.0322581}}
\multiput(53.801,3.801)(.0322581,.0322581){20}{\line(0,1){.0322581}}
\multiput(55.091,5.091)(.0322581,.0322581){20}{\line(0,1){.0322581}}
\multiput(56.381,6.381)(.0322581,.0322581){20}{\line(1,0){.0322581}}
\multiput(57.672,7.672)(.0322581,.0322581){20}{\line(1,0){.0322581}}
\multiput(58.962,8.962)(.0322581,.0322581){20}{\line(0,1){.0322581}}
\multiput(60.252,10.252)(.0322581,.0322581){20}{\line(0,1){.0322581}}
\multiput(61.543,11.543)(.0322581,.0322581){20}{\line(0,1){.0322581}}
\multiput(62.833,12.833)(.0322581,.0322581){20}{\line(1,0){.0322581}}
\multiput(64.123,14.123)(.0322581,.0322581){20}{\line(1,0){.0322581}}
\multiput(65.414,15.414)(.0322581,.0322581){20}{\line(0,1){.0322581}}
\multiput(66.704,16.704)(.0322581,.0322581){20}{\line(0,1){.0322581}}
\multiput(67.994,17.994)(.0322581,.0322581){20}{\line(0,1){.0322581}}
\multiput(69.285,19.285)(.0322581,.0322581){20}{\line(0,1){.0322581}}
%\end
}
\end{picture}
		\end{centering}
		\caption{(Color online) A gap created by a quantum coin toss. A single quantum (symbolized by a black circle
from a source (left crossed circle)
impinges on a semi-transparent mirror (dashed line), where it is reflected and transmitted with a 50:50 chance.
The two final states are indicated by grey circles. The exit ports of the mirror can be coded by $0$ and $1$, respectively.}
		\label{fig:2014-fw-qcointoss}
	\end{figure}

One may object to this scenario of {\em quantum indeterminism} by pointing out
that it is merely based on a believe
--
actually, Born's {\em inclinations
``to give up determinism in the world of atoms''}~\cite[p.~866]{born-26-1}
(English translation in \cite[p.~54]{wheeler-Zurek:83})
--
with provable formal improvability~\cite{svozil-2013-omelette}.
We shall come back to related issues later.

One may also object that the 50:50 mirror has a quantum mechanical representation as a unitary Hadamard transformation;
that is, it is totally deterministic.
This can be demonstrated by serially concatenating two such 50:50 mirrors so that the output ports of the first mirror
are the input ports of the second mirror. The result is a Mach-Zehnder interferometer reconstructing the original
quantum state of the particle.
In this line of thought, the choice of the ports by the photon decays into thin air,
and one wonders where exaclty the actual measurement takes place.
Indeed, Wigner~\cite{wigner:mb} argued that the measurement eventually must
be located and resides in some complex consciousness.


\subsection{Vacuum fluctuations}

As stated by Milonni~\cite[p.~xiii]{milonni-book} and others~\cite{einstein-aether,dirac-aether}, {\em ``$\ldots$~there is no vacuum in the ordinary sense of
tranquil nothingness. There is instead a fluctuating quantum vacuum.''}
One of the observable vacuum effects is the {\em spontaneous emission of radiation}~\cite{Weinberg-search}:
{\em ``$\ldots$~the process of spontaneous emission of radiation is one in which ``particles'' are actually created.
Before the event, it consists of an excited atom, whereas after the event, it consists of an atom in a state of lower energy, plus a photon.''}
Recent experiment achieve single photon production by spontaneous emission~\cite{PhysRevLett.39.691,PhysRevLett.85.290,Buckley-12,Stevenson-spontemi,Sanguinetti},
for instance by electroluminiscence.
Indeed, most of the visible light emitted by the sun or other sources of blackbody radiation, including incandescent bulbs,
is due to spontaneous emission~\cite[p.~78]{milonni-book} and thus subject to {\em creatio ex nihilo}.

It might not be too unreasonable to speculate that all gap scenarios, including spontaneous symmetry breaking and quantum oracles, are ultimately based on vacuum fluctuations.
We also mention without discussion that Jack Sarfatti has built what he called an  {\em  Eccles telegraph} by connecting a
radioactive source to a typewriter.

A gap based on vacuum fluctuations is schematically depicted in Fig.~\ref{fig:2014-fw-vacuumfluctuation}.
It consists of an atom in an excited state, which transits into a state of lower energy, thereby producing a photon.
The photon (non-)creation can be coded by the symbols $0$ and $1$, respectively.
	\begin{figure}
		\begin{centering}
% This is a LaTeX picture output by TeXCAD.
% File name: [1.pic].
% Version of TeXCAD: 4.3
% Reference / build: 30-Jun-2012 (rev. 105)
% For new versions, check: http://texcad.sf.net/
% Options on the following lines.
%\grade{\on}
%\emlines{\off}
%\epic{\off}
%\beziermacro{\on}
%\reduce{\on}
%\snapping{\off}
%\pvinsert{% Your \input, \def, etc. here}
%\quality{8.000}
%\graddiff{0.005}
%\snapasp{1}
%\zoom{4.0000}
\unitlength 0.6mm % = 2.845pt
\linethickness{0.4pt}
\ifx\plotpoint\undefined\newsavebox{\plotpoint}\fi % GNUPLOT compatibility
\begin{picture}(79.526,40)(0,0)
\thicklines
\put(0,0){\color{blue}\line(1,0){50}}
\put(0,40){\color{orange}\line(1,0){50}}
\put(25,40){\color{gray}\vector(0,-1){39}}
\thinlines
{\color{gray}
\qbezier(30,20)(35,13)(40,20)
\qbezier(70,20)(65,27)(60,20)
\qbezier(50,20)(45,27)(40,20)
\qbezier(50,20)(55,13)(60,20)
\put(70,20){\vector(1,-1){2}}
}
\put(77,20){\color{gray}\circle*{8}}
\put(77,20){\color{white}\makebox(0,0)[cc]{$1$}}
\end{picture}
		\end{centering}
		\caption{(Color online) A gap created by the spontaneous creation of a photon.}
		\label{fig:2014-fw-vacuumfluctuation}
	\end{figure}


%\subsection{Miscellaneous proposals}




\section{Analogies in physics}

In the following we shall look at two seemingly unrelated physical issues
--
the purported ``irreversibility''
of quantum measurements~\cite{PhysRevD.22.879,PhysRevA.25.2208,greenberger2,Nature351,Zajonc-91,PhysRevA.45.7729,PhysRevLett.73.1223,PhysRevLett.75.3783,hkwz},
as well the character of the second law of thermodynamics~\cite{Myrvold2011237}.

For the sake of demonstration, we again consider the case of a game universe introduced earlier.
Suppose we flatten out the configuration in such a way that everything is included in a hyper-universe.
This hyper-universe comprises
(i)
the algorithmic deterministic game universe,
(ii) the player and the beyond;
as well as (iii)  the interface between the player and the universe.
Thereby the Cartesian cut is dissolved.
Transcendence, in particular,
the hierarchical dualism sketched previously, is lost; it either vanishes with a uniform causality
governing both the universe and the beyond, or relegates the discussion to the nature of the player.

If we are willing to take this step, then we are back to the two-horn situation~\cite[p.~14]{russell-freedom}
of having to deal with either determinism or total indeterminism.
Note that, in the terminology of Calude and Poznanovi\'{c}~\cite{CaludePoznanovic}
the transition from dualism to uniformity the {\em context} has changed:
whereas for the dualistic model the context has been the game universe,
in the uniform, flattened hyper-universe the context includes
the original universe, the interface, and the beyond.

\subsection{Wigner's and Everett's arguments with regard to quantum measurement}

The extension of the observation context is not dissimilar to what Wigner~\cite{wigner:mb}
and, in particular, Everett~\cite{everett,everett-collw} had in mind when they argued against (irreversible
and, in principal, for reversible) measurement.
In analogy to the dualism allowing free will through interaction with some interface,
quantum mechanics, at least as it is usually presented, allows for two types of evolution:
the first type comprises irreversible measurements,
whereas the second mode is characterized by the unitary, that is, reversible permutation, of quantum states
inbetween aforementioned measurements.

Alas, although constantly repeated by the quantum orthodoxy, this is true only {for all practical purposes}~\cite{bell-a},
that is, {relative to the physical means}~\cite{Myrvold2011237} available to resolve the huge number of degrees of freedom involving a
macroscopic measurement apparatus.
And yet, at least in principle, if the unitary quantum evolution is taken to be universally valid,
then any distinction or cut between the observer and the measurement apparatus on the one side,
and the quantized object on the other side, is not absolute or ontic,
but epistemic, means-relative, subjective and conventional~\cite{svozil-2013-omelette}.


\subsection{Analogies to the second law of thermodynamics}

There are good reasons to believe that also irreversibility in statistical physics
is means relative~\cite{Myrvold2011237} and thus epistemic: if we cannot resolve individual constituents of a group, and their  degrees of freedom,
then irreversibility is the epistemic expression of our incapacity to do so.
However, if we take the molecules individually, the second law might decay into thin air.
In Maxwell's own words~\cite[Document~15, p.~422]{garber}
{\em ``I carefully abstain from asking the molecules which enter where they last started
from. I only count them and register their mean velocities, avoiding all personal
enquiries which would only get me into trouble.''}

The analogy to dualistic free will is exposed by identifying irreversibility with free will.
The context varies with the precision of description -- from probabilistic groups to individual molecules.

\section{Afterthoughts}

We have proposed a possibility for free will based on dualism.
This scenario avoids the problems encountered in totally (in)deterministic universes
by a fine-tuned mixture of determinism and indeterminism, thereby
allowing gaps in the natural laws
in which intentions and choices can be communicated via interfaces serving as Cartesian cuts.
By intrinsic means alone, any such signals cannot be differentiated from irreducible chance.


These considerations are related to other, logic constraints on free will and omniscience
discussed elsewhere~\cite{maryland,greenberger-svozil,svozil-07-physical_unknowables}
which can be derived by diagonalization and reduction to the halting problem.
Because the simultaneous enactment of omniscience, omnipotence on the one hand,
as well as free will on the other hand, are inconsistent:
any agent commanding the omniscience and omnipotence may freely choose to counteract its own predictions.
Thus consistency demands the absence of at least one of these features.

We do not suggest that the existence of the aforementioned gaps are either necessary or sufficient for the occurrence
of free will, miracles, or oracles. At the moment there does not seem to exist any consolidated gap mechanism for consciousness
or free will. Even its location is unclear, although most authors seem to agree that it must be the human brain.
To mention an anecdote: connecting a telegraph to a radioactive source has turned out to be not very helpful.

One related issue is a theologic one: if God created a clockwork universe, or one by universal chance,
the sole remaining option would be to ``throw the universe and watch it evolve;'' very similar with throwing a dice.
Again, a dualistic free will scenario presents ways of interference with the universe beyond non-intervention in a harmonized, consistent manner.


\begin{acknowledgments}
This research has been partly supported by FP7-PEOPLE-2010-IRSES-269151-RANPHYS.
The author is grateful to conversations with  Cristian S. Calude and Nemanja Poznanovi\'{c}, and for making their paper available prior to publication.
\end{acknowledgments}

\bibliography{svozil}

\end{document}
