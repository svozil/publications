%\documentclass [11pt]{article}
\documentclass [11pt]{llncs}

%\usepackage{utopia}
\sloppy
\raggedbottom


\pagestyle{empty}
\begin{document}

\title{Quantum information: the new frontier}
\author{K. Svozil\\
 {\small Institut f\"ur Theoretische Physik}  \\
  {\small University of Technology Vienna }     \\
  {\small Wiedner Hauptstra\ss e 8-10/136}    \\
  {\small A-1040 Vienna, Austria   }            \\
  {\small e-mail: svozil@tph.tuwien.ac.at}\\
  {\small www: http://tph.tuwien.ac.at/$\widetilde{\;}$svozil}}
\date{ }
\maketitle

\begin{flushright}
{\scriptsize
http://tph.tuwien.ac.at/$\widetilde{\;}$svozil/publ/2000-qitnf.tex
.tex}
\end{flushright}

\begin{abstract}
{\em
Quantum information and computation is the new hype in physics.
It is promising, mindboggling and even already applicable in cryptography,
with good prospects ahead.
A brief, rather subjective outline is presented.
}
\end{abstract}



\section{Is Nature telling us something?}

Friends in experimental physics tell me that the essence
of their observations are clicks in some
counter. There is a click or there is none.
This is experimental physics in a nutshell.
There may be some magic in properly designing experiments
and some magic in interpreting those clicks,
but that is all there is.

A single click represents some elementary physical proposition.
It is also an answer to a question which might not even have been posed consciously.
It is tempting to state that ``Nature wants to tell us something'' with these clicks about
some formal structures, symmetries, music or numbers beyond the phenomena.
Maybe that is the case, and most physicists tend to believe so;
but maybe we are just observing crap,
erratic emanations devoid of any meaning \cite{calude-meyerstein}.

Anyway, we have to deal with those clicks,
and one way to deal with them is to interpret them as information.
For example, in  an experimental  input-output scheme, information is received, transformed and communicated by the system.
One might think of a physical system as a black box with an input and an
output interface \cite{svozil-2000interface}.
The experimenter inputs some information and the black box responds with some information as output.

If we are dealing with mechanical systems, all the conceivable  gadgets inside the black box
can be isomorphically translated into a sheet or a tape of
paper on which finite computations are performed and vice versa.
This was  Turing's insight.

But if the black box is essentially quantum driven, then the paper metaphor becomes questionable.
The quantum is illusive and highly nonintuitive.
In the words of John Archibald Wheeler,
one is capturing a ``smoky [[quantum]] dragon'' \cite{wheeler} inside the black box.
Or, in Danny Greenberger's dictum, ``quantum mechanics is magic'' \cite{greenberger:pr2}.
In addition, quantized systems such as the quantized electromagnetic
field have ``more'' degrees of freedom
as compared to their classical correspondents.
Therefore, any isomorphic translation into
classical mechanistic devices remains very expensive in terms of paper consumption, at best.
To make things worse, under certain reasonable side assumption, it can be proven that a complete
``mechanical'' paper set of all quantum answers is inconsistent.

Because of these novel non-classical features
it is so exiting to pursue the quantum information concept.
But even if we look aside and do not want to be bothered with the quantum,
the quantum catches up on us: due to the progressing miniaturization of
circuits forming logical gates, we shall soon be confronted with
quantum phenomena there.
In the following, some of the recent developments are reviewed below;
and some speculations and prospects are mentioned.


\section{Formalization of quantum information}

In order to be applicable, any formalization of information has to be based on its proper realization
in physical terms; i.e., as states of a physical system.
In this view, information theory is part of physics; or conversely, physics is part of information theory.
And just as the classical bit represents the distinction between two classical physical states,
the quantum bit, henceforth often
abbreviated by the term {\em`qubit,'} represents the conceivable states of the most
elementary quantized system.
As we shall see, qubits feature quantum
mechanics `in a nutshell.'
Quantum bits are more general structures than classical bits.
That is,  classical bits can be represented as the limit of qubits, but not vice versa.


Classical information theory is based on the
classical bit as
fundamental atom. This classical bit, henceforth called
{\em `cbit,'} is in one of two
classical states $t$ (often interpreted as ``true'') and $f$ (often
interpreted as ``false'').
It is customary to code the classical logical states by
$\# ( t) =1$ and $\# ( f) =0$ ($\# (s)$ stands for the code of $s$).
The states can, for instance, be realized by some  condenser
which is discharged ($\equiv$ cbit state $0$) or charged ($\equiv$ cbit state $1$).

In quantum information theory
(see Appendix A for a brief outline of quantum mechanics)
qubits can be physically represented by a {\em `coherent superposition'}
of the two orthonormal\footnote{
$(t,t)=(f,f)=1$ and $(t,f)=0$.}
 states $t$ and $f$.
The qubit states
\begin{equation}
x_{\alpha}  =\alpha t+\beta f
\end{equation}
form a continuum, with
$ \vert \alpha \vert^2+\vert \beta \vert^2=1$, $\alpha ,\beta \in {\bf C}$.


What is a coherent superposition?
Formally it is just a sum of two elements (representing quantum states) in Hilbert space,
which results in an element (a quantum state) again per definition.
So, formally we are on the safe side.
Informally speaking, a coherent superposition of two different
and classically distinct states contains them both.
Now classically this sounds like outright nonsense!
A classical bit cannot be true and false at the same time.
This would be inconsistent, and inconsistencies in physics sound as
absurd as in mathematics \cite{hilbert-26}.

Yet, quantum mechanics (so far consistently) achieves the implementation of
classically inconsistent information into a single quantum bit.
Why is that possible?
Maybe we get a better feeling for this when we take up Erwin Schr\"odinger's
interpretation of the quantum wave function (in our terms: of the qubit states)
as a sort of ``catalogue of expectation values'' \cite{schrodinger}.
That is, the qubit appears to be a {\em representation of the state of our knowledge} about a physical system
rather than what may be called ``its true state.''
(Indeed we have to be extremely careful here with what we say.
The straightforward classical pretension that quantum systems must have ``a true state'',
albeit hidden to us, yields to outright contradictions!)

Why have I mentioned quantum superpositions here?
Because they lie at the heart of quantum parallelism.
And quantum parallelism lies at the heart of the quantum speedups
which caused so much hype recently.


The coding of qubits is discussed in Appendix C.


The classical and the quantum mechanical
concept of information differ from each other in several aspects.
Intuitively and classically, a unit of information is context-free.
That is, it is independent of what other information is or
might be present. A classical bit remains unchanged, no matter by what
methods it is inferred.
It obeys classical logic.
It can be copied. No doubts can be left.

By contrast, to mention just a few nonclassical properties of qubits:
\begin{itemize}
\item
Qubits are contextual \cite{kochen1}.
A quantum bit may appear different,
depending on the method by which it is inferred.
\item
Qubits cannot be copied or ``cloned''
\cite{wo-zu,dieks,mandel:83,mil-hard,glauber,caves}.
This due to the fact that the quantum evolution is reversible, i.e., one-to-one.
\item
Qubits do not necessarily satisfy
classical tautologies such as the distributive law \cite{kochen2,kochen3}.
\item
Qubits obey quantum logic \cite{birkhoff-36} which is different from classical logic.
\item
Qubits are coherent superpositions of
classically distinct, contradicting information.
\item
Qubits are subject to complementarity.
\end{itemize}


\section{Complementarity and quantum cryptography}

Before we proceed to quantum computing,
which makes heavy use of the possibility to superpose classically distinct information,
we shall mention an area of quantum information theory which
has already matured to the point where the applications have
almost become commercially available:
quantum cryptography.
At the moment, this might be seen as
{\em the} ``killer app'' of quantum information theory.

Quantum cryptography
(for a detailed review see \cite{benn-92})
is based on the quantum mechanical feature of
{\em complementarity.}
A formalization of quantum complementarity has been attempted by Edward Moore
\cite{e-f-moore} who started finite automata theory with this.
(Recent results are contained in Ref. \cite{cal-sv-yu} and \cite[chapter 10]{svozil-ql}; see also Appendix B.)


Informally speaking, quantum complementarity
stands for the principal impossibility to measure two observables
at the same time with arbitrary position.
If you decide to precisely measure the first observable, you ``loose control''
over the second one and vice versa.
By measuring one observable, the state of the system undergoes a ``state reduction''
or, expressed differently,  ``the wave function collapses''
and becomes  different from the original one.
This randomizes a  subsequent measurement of the second, complementary
observable: in performing the subsequent measurement,
one obtains some measurement results (i.e., clicks,
you remember?), but they dont tell us much about the original qubit, they are unusable crap.
There is no other way of recoving the original state than by completely
``undoing'' the first measurement in such a way that no trace is left of the previous
measurement result; not even a copy of the ``classical measurement''\footnote{I put
a quote here because if one is able to ``undo a measurement'', then
this process cannot be classical: per definition,
classicality means irreversibility, many-to-oneness.}
 result!

So how can this quantum property of complentarity can be put to use in cryptography?
The answer is straightforward (if one knows it already):
By taking advantage of complementarity,
the sender ``Alice'' of a secret and the receiver ``Bob''
are able to monitor the secure quantum communication channel and to know when an eavesdropper is present.

This can be done as follows.
Assume that Alice sends Bob a qubit
and an eavesdropper is present.
This eavesdropper is in an inescapable dilemma:
neither can the qubit be copied, nor can it be measured.
The former case is forbidden in quantum information theory
and the letter case would result in a state reduction which modifies Alice's qubit
to the point where it is nonsense for Bob.
Bob and Alice can realize this by comparing some of their results  over a classical
(insecure) channel.\footnote{Actually, if the eavesdropper has total control over the classical channel,
this might be used for a reasonable attack strategy.}
The exact protocol can for instance be found in \cite{benn-92}.
Another scheme \cite{ekert91} operates with entangled pairs of
qubits. Here
entanglement  means that whatever measurement of a particular type is performed on one qubit,
if you perform the same measurement on the other qubit of the pair, the result is the same.


Actually, in the real world, the communication over
the insecure classical channel has to go back and forth,
and they have to constantly compare a certain amount of their measured qubits in order
to be able to assure a guaranteed amount of certainty that no eavesdropper is present.
That is by no means trivial \cite{gilbert-hamrick-200009}.
But besides this necessary overhead, the quantum channel can be certified to be secure,
at least up to some desired amount of certainty and
up to the point where someone comes up with a theory which is ``better than quantum mechanics''
and which circumvents complementarity somehow.
Of course, the contemporaries always believe and assure the authorities that there will never be such a theory!


Quantum cryptographic schemes of the above type have already been demonstrated
to work for distances of 1000m (and longer) and net key sizes (after error correction) of 59000 Bits
at sustained (105 s) production rates of 850 Bits/s \cite{zeilinger-qc}.
Yet there is  no commercially available solution so far.


\section{Quantum computing}

Quantum computers operate with qubits.
We have dealt with  qubits already.
Now what about the operation of quantum computers on qubits?
We have to find something similar than Turing's ``paper-and-pencil-operations''
on paper or tape.
The most natural candidate for a formalization is the unitary time evolution of the quantum states.
This is all there is (maybe besides measurement
\cite{svozil-2000int}), because there is nothing beyond the unitary time evolution.
Unitary operators stand for generalized rotations in complex Hilbert spaces.
Therefore, a universal quantum computer can just be represented by the most general unitary
operator!

That is a straightforward concept: given a finite dimensional Hilbert space of, say, dimension $n$,
then the most general unitary operator $U(n)$ can for instance be parameterized by composition of unitary operations
in two (sub)dimensions $U(2)$ \cite{murnaghan}. Now we all know how $U(2)$ looks like (cf. Appendix D), so we know
how $U(n)$ looks like. Hence we all know how to properly formalize a universal quantum computer!


This looks simple enough, but where is the  advantage?
Of course one immediate answer is that it is perfectly all right to simulate
a quantized system with a quantum computer --- we all know that every system is a perfect
copy of itself!

But that is not the whole story.
What is really challenging here is that we may be able to use quantum parallelism for
speedups.
And, as mentioned already,
at the heart of quantum parallelism is the superposition principle
and quantum entanglement.
Superposition enables the quantum programmer to ``squeeze''
$2^N$ classical bits into $N$ qubits.
In processing $1$ qubit state $\alpha t+\beta f$, the computer processes $2$ classical bit states $t$ and $f$ at once.
In processing $N$ qubit states, the computer may be able to processes $2^N$ classical bit states at once.
Many researchers in quantum computing interpret this (in the so-called ``Everett interpretation of quantum mechanics'')
as an indication that $2^N$ seperate computer run in $2^N$ seperate worlds (one computer in each world);
thereby running through each one of the computational passes in parallel.
That might certainly be a big advantage as compared to a classical routine which might only be able to
process the cases consecutively, one after the other.

There are indeed indications that speedups are possible. The most prominent examples
are Shor's quantum algorithm for prime factoring
\cite{shor:94,ekerj96} and  Grover's  search algorithm \cite{grover}
for a single item satisfying a given condition in an unsorted database.
A detailed review of the suggested quantum algorithms exceeds the scope of this brief discussion
and can for instance be found in Gruska's book \cite{Gruska}.

One fundamental feature of the unitary evolution is its bijectivity, its one-to-oneness.
This is the reason why copying is not allowed, but this is also the reason why there is
no big waste basked where information vanishis into oblivion or nirvana forever.
In a quantum computer, one and the same ``message'' is constantly permutated.
It always remains the same but expresses itself through different forms.
Information is neither created nor discarded but remains constant at all times.\footnote{This
implicit time symmetry spoils the very notion of
``progress'' or ``achievement,''  since what
is a valuable output is purely determined by the
subjective meaning the observer associates with it and is devoid of any
syntactic relevance.}

Is there a price to be paid for parallelism?
Let me just mention one important problem here: the problem of the  readout of the result.
This is no issue in classical computation.
But in quantum computation, to use the Everett metaphor, it is by no means trivial how the
many parallel entangled universes communicate with each other in such a way that the classical
result can be properly communicated.
In many cases one has to make sure that, through positive interference, the proper probability amplitudes
indicating this result build up.
One may even speculate that there is no sufficient buildup of the states if the problem
allows for many nonunique solutions \cite{gottlob,cal-mike-svo}.

Another problem is the physical overhead which has to be invested in order for the system to remain "quantum"
and not turn classical \cite{mahler-priv}.
One may speculate that  the necessary equipment to process more qubits
grows exponentially with the number of qubits.
If that would be the case, then the advantages of quantum parallelism would be essentially nullified.

\section{Summary and outlook}

Let me close with a few observations.
So far, quantum information theory has applied the quantum features of
complementarity, entanglement and quantum parallelism to more or less real-world applications.
Certain other quantum features such as contextuality have not been put to use so far.

There are good prospects for quantum computing; if not for other reasons but because our computer parts will
finally reach the quantum domain.
We may be just at the very beginning, having conceived the quantum analogies of classical
tubes (e.g., quantum optical devices).
Maybe in the near future someone comes up with a revolutionary design such as a ``quantum transistor''
which will radically change the technology of information processing.

This is a very exciting and challenging new field of physics and computer sciences.



\section*{Appendix A: All (and probably more that) you ever wanted to know about quantum mechanics}

 \label{appendix-a}
``Quantization'' has been introduced by Max Planck around 1900 \cite{planck:1901,planck:1901b,planck:1901a}.
In a courageous, bold step Planck assumed
a {\em discretization} of the total energy
$U_N$ of
$N$
linear oscillators (``Resonatoren''),
$$
U_N= P\epsilon \in \{ 0,\epsilon ,2\epsilon ,3\epsilon ,4\epsilon
,\ldots
\}
,$$
where $P\in {\bf N}_0$ is zero or a positive
integer and $\epsilon$ stands for the {\em smallest quantum of energy}.
$\epsilon$ is a linear function of frequency $\omega$ and proportional
to Planck's fundamental constant
$\hbar \approx 10^{-34}$~Js; i.e.,
$$
\epsilon = \hbar \omega
.$$
That was a bold step in a time of the predominant continuum models of
classical mechanics.


In extension of Planck's discretized resonator energy model,
Einstein \cite{ein-5} proposed a quantization of the
electromagnetic field.
According to the light quantum hypothesis,
energy  in an electric field mode
characterized by the frequency $\omega$ can be produced, absorbed and
exchanged only in a discrete number $n$ of ``lumps'' or ``quanta'' or
``photons''
$$E_n=n\hbar \omega \, ,\; n=0,1,2,3,\ldots \; .$$





The following is a very brief introduction to the principles of quantum
mechanics
for logicians and computer scientists, as well as a reminder for
physicists.\footnote{ Introductions to quantum mechanics can be found in
Feynman, Leighton \& M. Sands \cite{feynman-III},
Harris \cite{har},
Lipkin \cite{lipkin},
Ballentine  \cite{ba-89},
Messiah  \cite{messiah-61},
Davydov \cite{davydov},
Dirac \cite{dirac},
Peres \cite{peres},
Mackey \cite{mackey:63},
von Neumann \cite{v-neumann-49}, and
Bell \cite{bell-87}, among many other expositions.
The history of quantum mechanics is reviewed by
Jammer \cite{jammer1}.
Wheeler \& Zurek \cite{wheeler-Zurek:83} published a helpful resource
book.}
To avoid a shock from a too early exposure  to ``exotic''
nomenclature prevalent in physics --- the Dirac bra-ket notation --- the
notation of Dunford-Schwartz
\cite{dunford-schwartz} is adopted.\footnote{
The bra-ket
notation introduced by Dirac is widely used in physics. To
translate expressions into the bra-ket notation, the following
identifications work for most practical purposes: for the scalar
product,
``$\langle  \equiv \;($'',
``$\rangle  \equiv \; )$'',
``$, \equiv \; \mid $''.
States are written as
$\mid  \psi \rangle  \equiv \psi$, operators as
$\langle  i\mid  A\mid  j \rangle  \equiv A_{ij}$.}

Quantum mechanics, just as classical mechanics, can be formalized in
terms of a linear space structure, in particular by Hilbert spaces
\cite{v-neumann-49}.
That is, all objects of quantum physics, in
particular the ones used by quantum logic, ought to be expressed
in terms of objects based on concepts of Hilbert space theory---scalar
products, linear summations, subspaces, operators, measures and so on.

Unless stated differently, only
finite-dimensional Hilbert spaces are considered.\footnote{
Infinite dimensional cases and continuous spectra are nontrivial
extensions of the finite
dimensional Hilbert space treatment. As a heuristic rule, which is not
always correct, it might be
stated that the sums become integrals, and the Kronecker delta function
$\delta_{ij}$
becomes the Dirac delta function $\delta (i-j)$, which is a
generalized function in the continuous variables $i,j$.
In the Dirac bra-ket notation, unity is given by
${\bf 1}=\int_{-\infty}^{+\infty} \vert i)( i\vert \, di$.
For a careful treatment, see, for instance,
the books by
Reed and Simon \cite{reed-sim1,reed-sim2}.}

A quantum mechanical {\em Hilbert space} is a linear
\index{Hilbert space}
\index{{{\cal H}}}
vector space ${\cal H}$ over the field ${\bf C}$ of complex numbers
(with vector addition
and scalar multiplication), together  with a complex function
$(\cdot ,\cdot
)$, the {\em scalar} or {\em inner product}, defined on ${\cal
H}\times{\cal H}$ such that
(i)
$(x,x)=0$ if and only if $x=0$;
(ii)
$(x,x)\ge 0$ for all $x \in{\cal H}$;
(iii)
$(x+y,z)=(x,z)+(y,z)$ for all $x,y,z \in {\cal H}$;
(iv)
$(\alpha x,y)=\alpha (x,y)$ for all $x,y \in {\cal H}, \alpha \in {\bf C}$;
(v)
$(x,y)={(y,x)}^\ast $ for all $x,y \in {\cal H}$
(${\alpha }^\ast $ stands for the complex conjugate of $\alpha$);
(vi)
If $x_n\in {\cal H}$, $n=1,2,\ldots$, and if $\lim_{n,m\rightarrow
\infty} (x_n-x_m,x_n-x_m)=0$, then there exists an $x\in {\cal H}$ with
$\lim_{n\rightarrow \infty} (x_n-x,x_n-x)=0$.



We shall make the following identifications between physical and
theoretical
objects (a {\it caveat:} this is an incomplete list).

\begin{description}
\item[(0)]
The dimension of the Hilbert space
corresponds to
the number of degrees of freedom.
\item[(I)]
A {\em pure physical state} $x$ is represented
either by the
one-dimensional linear subspace (closed linear manifold)
$\sp (x)=\{y\mid y=\alpha x,\; \alpha \in {\bf C},\; x\in {\cal H}\}$
spanned by a
(normalized)
vector $x$   of  the Hilbert space ${\cal H} $ or by the orthogonal
projection operator $E_x$ onto $\sp (x)$.
Thus, a vector $x\in {\cal H}$ represents a pure physical state.

Every one-dimensional projection $E_x$ onto a one-dimensional
linear subspace
 $\sp (x)$ spanned by $x\in {\cal H}$ can be
represented by the dyadic product
$E_x = \vert x)(x\vert$.

If two nonparallel vectors $x,y\in {\cal H}$ represent
pure physical states, their vector sum
$z=x+y\in{\cal H}$ is again a vector representing a pure physical
state.
This state $z$ is called the {\em superposition} of state $x$
and $y$.\footnote{
$x+y$ is sometimes referred to as ``coherent''
superposition to indicate the difference to ``incoherent'' mixtures of
state vectors, in which
the absolute squares $\vert x\vert^2 +\vert y\vert^2$
are
summed up.}


Elements $b_i,b_j\in {\cal H}$ of the set of orthonormal base vectors
satisfy
$(b_i, b_j) =\delta_{ij}$,
where $\delta_{ij}=
\left\{
\begin{array}{cc}
1&\textrm{if }i=j\\
0&\textrm{if }i\neq j
\end{array}
\right.$
is the Kronecker delta function.
Any pure state $x$ can be written as a linear
combination of
the set of orthonormal base vectors $\{b_1,b_2,\cdots \}$,
i.e.,
$x =\sum_{i=1}^n   \beta_i b_i$, where $n$ is the dimension of ${\cal
H}$ and
$\beta_i=(b_i,x) \in {\bf C}$.
In the Dirac bra-ket notation, unity is given by
${\bf 1}=\sum_{i=1}^n \vert b_i )( b_i\vert $.


In the nonpure state case, the system is characterized by the density
operator $\rho$, which is nonnegative and of trace class.\footnote{
Nonnegativity means $(\rho x,x)=(x,\rho x)\ge 0$ for all $x\in {\cal
H}$, and trace class means
$\textrm{trace}(\rho )=1$.}
If the system is in a nonpure state, then the preparation procedure
does not specify
the decomposition into projection operators (depending on the choice of basis)
precisely.
$\rho$ can be brought into its spectral form
$\rho =\sum_{i=1}^n P_i E_i$, where $E_i$ are projection operators and
the
$P_i$'s are the associated probabilities
 (nondegenerate
case\footnote{If the same eigenvalue of an operator
occurs more than once, it is called {\em degenerate.}}).

\item[(II)]
{\em Observables} $A$ are represented by hermitian
operators $A$
on the Hilbert space ${\cal H}$ such that $(Ax,y)=(x,Ay)$ for all
$x,y\in {\cal H}$. (Observables and their corresponding operators are
identified.) In matrix notation, the adjoint matrix $A^\dagger $ is the
complex conjugate of the transposed matrix of $A$; i.e., $(A^\dagger
)_{ij}=(A^\ast )_{ji}$.
Hermiticity means that $(A^\dagger)_{ij} =A_{ij}$.


Any hermitian operator has a spectral representation
$A=\sum_{i=1}^n \alpha_i E_i$,
where the $E_i$'s  are orthogonal projection operators onto the
orthonormal eigenvectors $a_i$ of $A$
(nondegenerate
case).

Note that the projection operators, as well as their corresponding
vectors and subspaces, have a double r\^{o}le as pure state and
elementary proposition (that the system is in that pure state).


\index{comeasurability}
\index{compatibility}
Observables are said to be {\em compatible} or {\em comeasurable} if
they can be defined simultaneously with arbitrary accuracy.
Compatible observables are polynomials (Borel measurable functions in
the infinite dimensional case) of a single ``Ur''-observable.

A criterion for compatibility is the {\em commutator.}
Two observables ${A},{B}$ are compatible if their {\em
commutator} vanishes; i.e.,
if $\left[
{A},
{B}
\right] =
{A}
{B}  -
{B}
{A}   =0$.
In this case, the hermitian matrices  $A$ and $B$ can be
simultaneously diagonalized, symbolizing that the observables
corresponding to $A$ and
$B$ are
simultaneously measurable.\footnote{ Let us first diagonalize $A$; i.e., $A_{ij}=\textrm{diag
}(A_{11},A_{22},\ldots ,A_{nn})_{ij}=
\left\{
\begin{array}{cc}
A_{ii}&\textrm{if }i=j\\
0&\textrm{if }i\neq j
\end{array}
\right.$.
Then, if $A$ commutes with $B$, the commutator
$[A,B]_{ij}= (AB-BA)_{ij}=$ $A_{ik}B_{ki}-B_{ik}A_{kj}=$
$(A_{ii}-A_{jj})B_{ij}=0$ vanishes. If $A$ is nondegenerate, then
$A_{ii}\neq A_{jj}$ and thus $B_{ij}=0$ for $i\neq j$.
In the degenerate case, $B$ can only be block diagonal. That is, each
one of
the blocks of $B$ corresponds to a set of equal eigenvalues of $A$ such
that the corresponding subblockmatrix of $A$ is proportional to the unit
matrix. Thus, each block of $B$ can be diagonalized separately without
affecting $A$ \cite[p. 71]{peres}.\label{fnl1}}



It has recently been demonstrated that
(by an analog embodiment using
particle beams) every hermitian operator in a finite dimensional Hilbert
space can be experimentally realized \cite{rzbb}.

Actually, one can also measure normal operators $N$ which can be decomposed
into the sum of two commuting operators $A,B$ according to $N=A+iB$, with $[A,B]=0$.

\item[(III)]
The result of any single measurement of the observable $A$
on an arbitrary state $x\in {\cal H}$
can only be one of the real eigenvalues of the corresponding
hermitian operator $A$.
(Actually, one can also measure normal operators which can be decomposed
into the sum of two commuting
If $x=\beta_1a_1+\cdots +\beta_ia_i+\cdots  +\beta_na_n$ is in a
superposition of eigenstates $\{ a_1,\ldots ,a_n\}$ of
$A$, the
particular outcome of any such single measurement is indeterministic;
i.e.,
it cannot be predicted with certainty. As a
result of the measurement,
the system is in the state $a_i$ which corresponds to
the associated real-valued eigenvalue
$\alpha_i$ which is the measurement outcome; i.e.,
$$x\rightarrow a_i .$$

The arrow symbol ``$\rightarrow$'' denotes an irreversible measurement;
usually interpreted as a ``transition'' or ``reduction'' of the state
due to an irreversible interaction of the microphysical quantum
system with a classical, macroscopic  measurement apparatus.
This ``reduction''  has given rise to speculations
concerning the
``collapse
of the wave function (state).''

As has been argued recently
(e.g., by
Greenberger and YaSin
\cite{greenberger2}, and by
Herzog, Kwiat, Weinfurter and Zeilinger
\cite{hkwz}),
 it is
possible to reconstruct the state of the physical system before the
measurement; i.e., to
``reverse the
collapse of the wave function,'' if the process of measurement is
reversible. After this reconstruction, no information about the
measurement is left, not even in principle.

\label{l-schroe}
How did Schr\"odinger, the creator of wave mechanics, perceive
the quantum physical state, or, more specifically,  the
$\psi$-function? In his
1935 paper
``Die gegenw\"artige
Situation in der Quantenmechanik'' (``The present situation in quantum
mechanics''
\cite[p. 823]{schrodinger}), Schr\"odinger states,\footnote{
{\em Die $\psi$-Funktion als Katalog der Erwartung:}
$\ldots$
Sie [[die $\psi$-Funktion]] ist jetzt das Instrument zur Voraussage der
Wahrscheinlichkeit von Ma\ss zahlen. In ihr ist die jeweils erreichte
Summe theoretisch begr\"undeter Zukunftserwartung verk\"orpert,
gleichsam wie in einem {\em Katalog} niedergelegt.
$\ldots$
Bei jeder Messung ist man gen\"otigt, der $\psi$-Funktion ($=$dem
Voraussagenkatalog) eine eigenartige, etwas pl\"otzliche Ver\"anderung
zuzuschreiben, die von der {\em gefundenen Ma\ss zahl} abh\"angt und
sich {\em nicht vorhersehen l\"a\ss t;} woraus allein schon deutlich
ist, da\ss~ diese zweite Art von Ver\"anderung der $\psi$-Funktion mit
ihrem regelm\"a\ss igen Abrollen {\em zwischen} zwei Messungen nicht das
mindeste zu tun hat. Die abrupte Ver\"anderung durch die Messung
$\ldots$ ist der interessanteste Punkt der ganzen Theorie. Es ist genau
{\em der} Punkt, der den Bruch mit dem naiven Realismus verlangt. Aus
{\em diesem} Grund kann man die $\psi$-Funktion {\em nicht} direkt an
die Stelle des Modells oder des Realdings setzen. Und zwar nicht etwa
weil man einem Realding oder einem Modell nicht abrupte unvorhergesehene
\"Anderungen zumuten d\"urfte, sondern weil vom realistischen Standpunkt
die Beobachtung ein Naturvorgang ist wie jeder andere und nicht per se
eine Unterbrechung des regelm\"a\ss igen Naturlaufs hervorrufen darf.
}
\begin{quote}
{\em The $\psi$-function as expectation-catalog:}
\index{catalogue of expectation values}
$\ldots$
In it [[the $\psi$-function]] is embodied the momentarily-attained sum
of theoretically based future expectation, somewhat as laid down in a
{\em catalog.}
$\ldots$
For each measurement one is required to ascribe to the $\psi$-function
($=$the prediction catalog) a characteristic, quite sudden change,
which {\em depends on the measurement result obtained,} and so {\em
cannot be foreseen;} from which alone it is already quite clear
that this second kind of change of the $\psi$-function has nothing
whatever in common with its orderly development {\em between} two
measurements. The abrupt change [[of the $\psi$-function ($=$the
prediction catalog)]] by measurement $\ldots$ is the most interesting
point of the entire theory. It is precisely {\em the} point that demands
the break with naive realism. For {\em this} reason one cannot put the
$\psi$-function directly in place of the model or of the physical thing.
And indeed not because one might never dare impute abrupt unforeseen
changes to a physical thing or to a model, but because in the realism
point of view observation is a natural process like any other and cannot
{\em per se} bring about an interruption of the orderly flow of natural
events.
\end{quote}
It therefore seems not unreasonable to state that, epistemologically,
quantum mechanics appears more as a theory of knowledge of an
(intrinsic) observer rather than the Platonic physics ``God knows.''
The  wave function, i.e., the state of the physical system in a
particular
representation (base), is a representation of the observer's knowledge;
it is a representation or name or code or index of
the information or knowledge the observer has access to.


\item[(IV)]
The probability $P_x(y)$ to find a system represented by a  normalized
pure state
$x$ in some normalized pure state $y$ is
given by
$$P_x(y)=\vert (x,y) \vert^2 , \quad \vert x\vert^2=\vert y\vert^2=1.$$
In the nonpure state case, The probability $P(y)$ to find a system
characterized by $\rho$
in a pure state associated with a projection operator
$E_y$ is
$$P_\rho (y)=\textrm{trace}(\rho E_y).$$


\item[(V)]
The {\em average value} or {\em expectation value} of an observable
${A}$
represented by a hermitian operator
$A$ in the normalized pure state
$ x$
is given by
$$\langle A\rangle_ x =
\sum_{i=1}^n \alpha_i
\vert (x,a_i) \vert^2, \quad \vert x\vert^2=\vert a_i\vert^2=1.$$

The {\em average value} or {\em expectation value} of an observable
${A}$
represented by a hermitian operator
$A$ in the nonpure state
$\rho$
is given by
$$\langle A\rangle =\textrm{trace}(\rho A)
=\sum_{i=1}^n \alpha_i\textrm{trace}(\rho E_i)
.$$

\item[(VI)]
The dynamical law or equation of motion between subsequent, irreversible,
measurements can be written in the form
$x (t) =Ux (t_0) $,
where $U^\dagger =U^{-1}$ (``$\dagger $ stands for transposition and
complex conjugation) is a
linear {\em unitary evolution operator}.\footnote{
Any unitary operator $U(n)$ in finite-dimensional Hilbert space can be
represented by the product --- the serial composition --- of unitary
operators  $U(2)$ acting in twodimensional subspaces
\cite{murnaghan,rzbb}.}
Per definition, this evolution is reversible; i.e., bijective,
one-to-one.
So, in quantum mechanics we have to distinguish between unitary,
reversible evolution of the system inbetween measurements, and the
``collapse of the wave function'' at an irreversible measurement.


The {\em Schr\"odinger equation}
$
i\hbar {\partial \over \partial t}  \psi (t)    =
H \psi (t) $
for some state $\psi$  is obtained by identifying $U$ with
$U=e^{-iHt/\hbar }$,
where $H$ is a hermitian  Hamiltonian (``energy'') operator,
by partially differentiating the equation of motion
with respect to the time variable $t$;
i.e.,
$
 {\partial \over \partial t} \psi (t) =-\,{iH\over
\hbar
}e^{-i{H}t/\hbar}
\psi (t_0 ) = -\,{i{H}\over \hbar } \psi (t)
$.
In terms of the
set of orthonormal base vectors $\{ b_1, b_2, \ldots
\}$, the Schr\"odinger equation can be written as
$i\hbar {\partial \over \partial t} ( b_i , \psi (t) )   =
\sum_{j}
H_{ij}( b_j, \psi (t) )$.

For stationary states $ \psi_n
(t)=
e^{-(i/\hbar )E_nt}  \psi_n $, the Schr\"odinger equation
can be brought into its time-independent form
$H\, \psi_n
=
E_m\, \psi_m $ (nondegenerate case).
Here,
$i\hbar {\partial \over \partial t} \psi_m (t)
=
E_m \, \psi_m (t) $  has been used;
$E_m$
and $\psi_m $
stand for the $m$'th eigenvalue and eigenstate of
$H$, respectively.

\end{description}

Usually, a physical problem is defined by the Hamiltonian ${H}$ and the
Hilbert space in question.
The problem of finding the physically relevant states reduces to finding
a complete set of eigenvalues and eigenstates of ${H}$.

\section*{Appendix B: Complementarity and automaton logic}

 \label{appendix-b}
\label{d:parlog}

A systematic, formal investigation of the black box system
or any finite input/output system can be given by finite automata.
Indeed, the study of finite automata was motivated from the very
beginning
by their analogy to quantum systems \cite{e-f-moore}. Finite automata
are
universal with respect to the class of computable functions. That is,
universal
networks of automata can compute any effectively (Turing-) computable
function. Conversely, any feature emerging from finite automata is
reflected by any other universal computational device.
In this sense, they are ``robust''.
All rationally conceivable finite games can be modeled by finite
automata.


{\em Computational complementarity,}
\index{computational complementarity} as it is sometimes called
\cite{finkelstein-83}, can be introduced as a game between Alice and
Bob. The rules of the game are as follows.
Before the actual game, Alice gives Bob all he needs to know about
the intrinsic workings of the
automaton. For example, Alice tells Bob, {\it ``if the automaton is in
state
1 and
you input the symbol 2, then the automaton will make a transition into
state 2 and output the symbol 0,''} and so on.
Then Alice presents Bob a black box which contains a realization of
the automaton. Attached to the black box are two interfaces:  a
keyboard for the
input of  symbols, and an output display, on which
the output symbols appear. Again, no other interfaces are allowed.
In particular, Bob is not allowed to ``screw the box open.''


Suppose now that Alice chooses some initial state of the
automaton.
She may either throw a dice, or she may make clever choices using some
formalized system.
In any case, Alice does not tell Bob about her choice. All Bob
has at his disposal are the input-output interfaces.

Bob's goal is to
find out which state Alice has chosen.
Alice's goal is to fool Bob.

Bob may
simply guess
or rely on his luck by throwing a dice. But Bob can also perform
clever input-output experiments and analyze
his data in order to find out. Bob wins if he gives the correct answer.
Alice wins if Bob's guess is incorrect. (So, Alice has to be really mean
and select worst-case scenarios).

Suppose that Bob tries very hard. Is cleverness sufficient?
Will Bob always be able to uniquely determine the initial automaton
state?

The answer to that question is ``no.'' The reason
is that there may be situations when Bob's input causes an irreversible
transition into a black box state which does not allow any
further queries about the initial state.

What has been introduced here as a game between Alice and Bob is
what the mathematicians have called the
{\em state identification problem}
\index{state identification problem}
\cite{e-f-moore,chaitin-65,conway,brauer-84}:
given a finite deterministic
automaton, the task is  to locate
an unknown initial state.  Thereby it is assumed that
only {\em a single} automaton copy is available for inspection.  That
is, no second, identical, example of the automaton can be used for
further examination.  Alternatively, one may think of it as choosing at
random a single automaton from a collection of automata in an ensemble
differing only by
their initial state.  The task then is to find out which was the initial
state of the chosen automaton.

The logico-algebraic structure of the state identification problem
has been introduced in
\cite{svozil-93}, and subsequently studied in
\cite{svozil-93,schaller-92,schaller-95,schaller-96,dvur-pul-svo,svo:za,svozil-tkadlec,cal-sv-yu}.
We shall deal with it next.



\subsubsection*{Step 1:
Computation of the experimental equivalence classes.}
In the propositional structure of sequential machines, state partitions
play an important r\^{o}le. Indeed, the set of states is partitioned
into equivalence classes with respect to a particular input-output
experiment.

Suppose again that the only unknown feature of an automaton is its
initial
state; all else is known. The automaton is presented in a black box,
with input and output interfaces. The task in this {\em complementary
game} is to find (partial) information about the initial state of the
automaton \cite{e-f-moore}.

To illustrate this, consider the Mealy automaton $M_s$ discussed above.
Input/output experiments can be performed by the input of just one
symbol $i$
(in this example, more inputs yield no finer partitions).
Suppose again that
Bob does not know the automaton's initial state. So, Bob has to
choose between the input of symbols
1,2, or 3.
If Bob inputs, say, symbol 1,
then he obtains a definite answer whether the automaton was
in state~1 --- corresponding to
output 1; or whether the automaton was not in state~1 --- corresponding
to
output 0. The latter proposition ``not~1'' can be identified with the
proposition that the automaton was either in state~2 or in state~3.

Likewise, if Bob inputs symbol 2,
he obtains a definite answer whether the automaton was
in state~2 --- corresponding to
output 1; or whether the automaton was not in state~2 --- corresponding
to
output 0. The latter proposition ``not~2'' can be identified with the
proposition that the automaton was either in state~1 or in state~3.
Finally, if Bob inputs symbol 3,
he obtains a definite answer whether the automaton was
in state~3 --- corresponding to
output 1; or whether the automaton was not in state~3 --- corresponding
to
output 0. The latter proposition ``not~3'' can be identified with the
proposition that the automaton was either in state~1 or in state~2.

Recall that Bob can actually perform only one of these input-output
experiments. This experiment will irreversibly destroy the initial
automaton state (with the exception of a ``hit''; i.e., of output 1).
Let us thus describe the three possible types of experiment as follows.
\begin{description}
\item[$\bullet$] Bob inputs the symbol 1.
\item[$\bullet$]  Bob inputs the symbol 2.
\item[$\bullet$]  Bob inputs the symbol 3.
\end{description}
The corresponding observable propositions are:
\begin{description}
\item{$p_{\{1\}}\equiv \{1\}$}: On input 1, Bob receives  the output
symbol 1.
\item{$p_{\{2,3\}}\equiv \{2,3\}$}: On input 1, Bob receives
the output symbol 0.
\item{$p_{\{2\}}\equiv \{2\}$}: On input 2, Bob receives  the output
symbol 1.
\item{$p_{\{1,3\}}\equiv \{1,3\}$}: On input 2, Bob receives  the output
symbol 0.
\item{$p_{\{3\}}\equiv \{3\}$}: On input 3, Bob receives  the output
symbol 1.
\item{$p_{\{1,2\}}\equiv \{1,2\}$}: On input 3, Bob receives  the output
symbol 0.
\end{description}
Note that, in particular,
$p_{\{1\}},p_{\{2\}},p_{\{3\}}$ are not comeasurable.
Note also that,
for $\epsilon_{ijk}\neq 0$,
$p_{\{i\}}'=p_{\{j,k\}}$ and
$p_{\{j,k\}}=p_{\{i\}}'$;
or equivalently ${\{i\}}'={\{j,k\}}$ and ${\{j,k\}}={\{i\}}'$.

%\begin{eqnarray*}
%&&v(1)=\{ \{1\},\{2,3\}\},\\
%&&v(2)=\{ \{2\},\{1,3\}\},\\
%&&v(3)=\{ \{3\},\{1,2\}\}.
%\end{eqnarray*}
In that way, we naturally arrive at the notion of a {\em partitioning}
of automaton states according to the information obtained from
input/output
experiments. Every element of the partition stands for the proposition
that the automaton is in (one of) the state(s) contained in that
partition. Every partition corresponds to a quasi-classical
Boolean block. Let us denote by $v(x)$ the block corresponding to input
(sequence) $x$. Then we obtain
\begin{description}
\item
no input:
$$v(\emptyset )=\{\{1,2,3\}\},$$
\item
one input symbol:
\begin{center}
\begin{tabular}{ccccc}
\hline\hline
input &&output&& output \\
 &&{$1$}&&   {$0$}\\
\hline
{$v(1)$}&${=}$&{$\{\{1\}$}&{,}&${\{2,3\}\}}$\\
{$v(2)$}&${=}$&{$\{\{2\}$}&{,}&${\{1,3\}\}}$\\
{$v(3)$}&${=}$&{$\{\{3\}$}&,&${\{1,2\}\}}$.\\
\hline\hline
\end{tabular}
\end{center}
\end{description}
Conventionally, only the finest partitions are included into
the set of state partitions.


\subsubsection*{Step 2: Pasting of the partitions.}

Just as in quantum logic, the {\em automaton propositional calculus} and
\index{automaton propositional calculus}
the associated
{\em partition logic} is the {\em pasting} of all the blocks of
\index{partition logic}
partitions
$v(i)$ on the atomic level. That is, elements of two blocks are
identified if and only if the corresponding atoms are
identical.

The automaton partition logic based on {\em atomic} pastings  differs
\index{atomic pasting}
from previous approaches
\cite{svozil-93,schaller-92,schaller-95,schaller-96,dvur-pul-svo,svo:za,svozil-tkadlec,cal-sv-yu}.
Atomic pasting guarantees that there is no mixing of elements
belonging to two different order levels.
Such confusions can give rise
to the nontransitivity of the order relation \cite{svozil-93} in cases
where both $p\rightarrow q$ and $q\rightarrow r$ are operational
but incompatible, i.e., complementary, and hence $p\rightarrow r$ is not
operational.
%%%%%%%%%%%%%%%%%%%

For the
Mealy automaton $M_s$ discussed above,
the pasting renders just the
horizontal sum --- only the least and greatest elements $0,1$
of each $2^2$ is identified---and
one obtains a ``Chinese lantern'' lattice $MO_3$.
The Hasse
diagram of the propositional calculus
is drawn in Figure \ref{f-mds}.
\begin{figure}[h]
\begin{center}
%TexCad Options
%\grade{\off}
%\emlines{\off}
%\beziermacro{\off}
%\reduce{\on}
%\snapping{\off}
%\quality{2.00}
%\graddiff{0.01}
%\snapasp{1}
%\zoom{1.00}
\unitlength 0.90mm
\linethickness{0.4pt}
\begin{picture}(111.00,91.00)
%------------------------------------------------- RED
\put(0.00,75.00){{\circle*{2.00}}                                  }
\put(15.00,60.00){{\circle*{2.00}}                                 }
\put(15.00,90.00){{\circle*{2.00}}                                 }
\put(30.00,75.00){{\circle*{2.00}}                                 }
\put(0.00,75.00){{\line(1,1){15.00}}                               }
\put(15.00,90.00){{\line(1,-1){15.00}}                             }
\put(30.00,75.00){{\line(-1,-1){15.00}}                            }
\put(15.00,60.00){{\line(-1,1){15.00}}                             }
\put(35.00,75.00){\makebox(0,0)[cc]{$\oplus$}}
\put(20.00,90.00){{\makebox(0,0)[lc]{$\{1,2,3\}$}}                 }
\put(20.00,60.00){{\makebox(0,0)[cc]{$0$}}                 }
\put(30.00,70.00){{\makebox(0,0)[cc]{$\{2,3\}$}}                   }
\put(0.00,70.00){{\makebox(0,0)[cc]{$\{1\}$}}                      }
%------------------------------------------------- GREEN, BLUE
\put(40.00,75.00){{\circle*{2.00}}                               }
\put(80.00,75.00){{\circle*{2.00}}                        }
\put(55.00,60.00){{\circle*{2.00}}                               }
\put(95.00,60.00){{\circle*{2.00}}                        }
\put(55.00,90.00){{\circle*{2.00}}                               }
\put(95.00,90.00){{\circle*{2.00}}                        }
\put(70.00,75.00){{\circle*{2.00}}                               }
\put(110.00,75.00){{\circle*{2.00}}                       }
\put(40.00,75.00){{\line(1,1){15.00}}                            }
\put(80.00,75.00){{\line(1,1){15.00}}                     }
\put(55.00,90.00){{\line(1,-1){15.00}}                           }
\put(95.00,90.00){{\line(1,-1){15.00}}                    }
\put(70.00,75.00){{\line(-1,-1){15.00}}                          }
\put(110.00,75.00){{\line(-1,-1){15.00}}                  }
\put(55.00,60.00){{\line(-1,1){15.00}}                           }
\put(95.00,60.00){{\line(-1,1){15.00}}                    }
\put(75.00,75.00){\makebox(0,0)[cc]{$\oplus$}}
\put(60.00,90.00){{\makebox(0,0)[lc]{$\{1,2,3\}$}}               }
\put(100.00,90.00){{\makebox(0,0)[lc]{$\{1,2,3\}$}}       }
\put(60.00,60.00){{\makebox(0,0)[cc]{$0$}}               }
\put(100.00,60.00){{\makebox(0,0)[cc]{$0$}}       }
\put(70.00,70.00){{\makebox(0,0)[cc]{$\{1,3\}$}}                 }
\put(110.00,70.00){{\makebox(0,0)[cc]{$\{1,2\}$}}         }
\put(40.00,70.00){{\makebox(0,0)[cc]{$\{2\}$}}                   }
\put(80.00,70.00){{\makebox(0,0)[cc]{$\{3\}$}}            }
%----------------------------------------------------
\put(55.00,50.00){\circle*{2.00}}
\put(55.00,10.00){\circle*{2.00}}
\put(55.00,50.00){{\line(-2,-1){40.00}}                            }
\put(15.00,30.00){{\line(2,-1){40.00}}                             }
\put(55.00,10.00){{\line(2,1){40.00}}                     }
\put(95.00,30.00){{\line(-2,1){40.00}}                    }
\put(55.00,50.00){{\line(-5,-4){25.00}}                            }
\put(30.00,30.00){{\line(5,-4){25.00}}                             }
\put(55.00,10.00){{\line(5,4){25.00}}                     }
\put(80.00,30.00){{\line(-5,4){25.00}}                    }
\put(55.00,50.00){{\line(-1,-2){10.00}}                          }
\put(45.00,30.00){{\line(1,-2){10.00}}                           }
\put(55.00,50.00){{\line(1,-2){10.00}}                           }
\put(65.00,30.00){{\line(-1,-2){10.00}}                          }
\put(15.00,30.00){{\circle*{2.00}}}
\put(30.00,30.00){{\circle*{2.00}}}
\put(45.00,30.00){{\circle*{2.00}}                               }
\put(65.00,30.00){{\circle*{2.00}}                               }
\put(80.00,30.00){{\circle*{2.00}}                        }
\put(95.00,30.00){{\circle*{2.00}}                        }
\put(30.00,25.00){\makebox(0,0)[cc]{{$\{2,3\}$}}}
\put(45.00,25.00){\makebox(0,0)[cc]{{$\{2\}$}}                   }
\put(65.00,25.00){\makebox(0,0)[cc]{{$\{1,3\}$}}                 }
\put(80.00,25.00){\makebox(0,0)[cc]{{$\{3\}$}}            }
\put(95.00,25.00){\makebox(0,0)[cc]{{$\{1,2\}$}}          }
\put(15.00,25.00){\makebox(0,0)[cc]{{$\{1\}$}}   }
\put(60.00,10.00){\makebox(0,0)[cc]{$0$}}
\put(60.00,50.00){\makebox(0,0)[lc]{$\{1,2,3\}$}}
\put(5.00,30.00){\makebox(0,0)[cc]{$=$}}
\end{picture}
\end{center}
\caption{Hasse
diagram of the propositional calculus of the Mealy automaton.\label{f-mds}}
\end{figure}


Let us give a formal definition for the procedures sketched so far.
Assume a set $S$ and a family of partitions ${\cal B}$ of $S$.
Every partition $E\in {\cal B}$ can be identified with a Boolean
algebra $B_E$ in a natural way by identifying the elements of the
partition with the atoms of the Boolean algebra.
The pasting of the Boolean algebras $B_E,E \in {\cal B}$ on the atomic
level is called a partition logic, denoted by $(S,{\cal B})$.
\index{partition logic}


The logical structure of the complementarity game (initial-state
identification problem) can
be defined as follows. Let us call a proposition concerning the initial
state of the machine
{\em experimentally decidable} if there is an experiment $E$ which
determines the truth value of that proposition.
This can be done by performing $E$, i.e., by the input of a sequence of
input symbols $i_1,i_2,i_3,\ldots ,i_n$ associated with $E$, and by
observing the output sequence
$$\lambda_E(s)=\lambda(s,i_1),\lambda(\delta (s,i_1),i_2), \ldots
,\lambda(\underbrace{\delta
(\cdots \delta (s,i_1)\cdots ,i_{n-1})}_{n-1 \mbox{ times}},i_n).$$
The most general form of a prediction concerning the
initial state $s$
of the machine is that the initial state $s$ is contained in a subset
$P$ of the state set $S$.
Therefore, we may identify propositions concerning the initial state
with subsets of $S$.
A subset $P$ of $S$ is then  identified with the proposition that the
initial state is contained in $P$.

Let $E$ be an experiment (a preset or adaptive one), and let
$\lambda_E(s)$
denote the obtained output of an initial
state $s$.
$\lambda_E$ defines a mapping of $S$ to the set of output sequences
$O^*$. We define an equivalence relation on the state set $S$ by
$$s \stackrel{E}{\equiv} t \mbox{ if and only if }\lambda_E(s) =
\lambda_E(t)$$ for any $s,t \in S$.
We denote the partition of $S$ corresponding to $\stackrel{E}{\equiv}$
by $S/\stackrel{E}{\equiv}$.
Obviously, the propositions decidable by the experiment $E$ are
the elements of the Boolean algebra generated by
$S/\stackrel{E}{\equiv}$, denoted by $B_E$.

There is also another way to construct the experimentally decidable
propositions of an experiment $E$.
Let $\lambda_E(P)  = \bigcup\limits_{s \in P}\lambda_E(s)$ be the direct
image of $P$ under $\lambda_E$ for any $P \subseteq S$.
We denote the direct image of $S$ by $O_E$; i.e.,  $O_E = \lambda_E(S)$.

It follows that the most general form of a prediction concerning
the outcome $W$ of the experiment $E$ is that $W$ lies in a subset of
$O_E$.
Therefore, the experimentally decidable propositions consist of all
inverse images $\lambda_E^{-1}(Q)$ of subsets $Q$ of $O_E$,
a procedure which can be constructively formulated (e.g., as an
effectively computable algorithm), and which also
leads to the Boolean algebra $B_E$.

Let ${\cal B}$ be the set of all Boolean algebras $B_E$.
We call the partition logic $R= (S,{\cal B})$ an {\em automaton
propositional calculus.}




\section*{Appendix C: Quantum coding}
 \label{appendix-C}
In the usual Hilbert space formulization,
qubits can then be written as
\begin{equation}
\# (x_{\alpha })  = e^{i\varphi } (\sin \omega  ,e^{i\delta } \cos \omega )\in {\bf C}^2  ,
\end{equation}
with
$\alpha=\alpha (\omega ,\varphi ,\delta)$, $\omega ,\varphi ,\delta \in {\bf R}$
Qubits can be identified with cbits as follows
\begin{equation}
\# (x_{\alpha(\pi/2, \varphi,\delta)})=(a,0)\equiv 1
\mbox{ and }
\# (x_{\alpha(0, \varphi,\delta)})=(0,b)\equiv 0
\quad , \qquad
\vert a\vert,
\vert b\vert =1\quad ,
\end{equation}
where the complex numbers $a$ and $b$ are of modulus one.
The quantum mechanical  states associated with the classical states $0$
and $1$ are mutually orthogonal.


Notice that, provided that $\alpha,\beta \neq 0$, a
qubit is not in a pure classical state. Therefore,
any practical determination of the qubit $x_{\alpha  }$
amounts to a measurement of the state amplitude of $t$ or $f$.
According to the quantum postulates,
any such {\em single} measurement will be
indeterministic (provided again that $\alpha,\beta \neq 0$). That is,
the outcome of a single measurement occurs unpredictably.
The probabilities
that the qubit $x_{\alpha  }$ is measured in states $t$
and
$f$ are
$P_t(x_{\alpha })=
\vert (x_{\alpha  },t)\vert^2
$ and
$P_f(x_{\alpha  })=
\vert (x_{\alpha  },f)\vert^2
=1-P_t(_{\alpha ,\beta })
$, respectively.




\section*{Appendix D: Universal manipulation of a single qubit: the $U(2)$-gate}

 \label{appendix-d}

It is well known that
any $n$-dimensional unitary matrix $U$ can be composed from elementary
unitary transformations in two-dimensional subspaces of ${\bf C}^n$.
This is usually shown in the context of parameterization of the
$n$-dimensional unitary groups
(cf. \cite[chapter 2]{murnaghan} and
\cite{rzbb,reck-94}).
Thereby, a transformation in $n$-dimensional spaces is decomposed into
transformations in $2$-dimensional subspaces.
This amounts to a
successive array of $U(2)$ elements, which in their entirety forms an
arbitrary time evolution
$U(n)$ in n-dimensional Hilbert space.

Hence, all quantum processes and computation tasks which can possibly be
executed must be representable by unitary transformations. Indeed,
unitary
transformations of qubits are a necessary and sufficient condition for
quantum computing. {\em The group of unitary transformations in
arbitrary- but finite-dimensional Hilbert space is a model of
universal quantum computer}.



It remains to be shown that the universal $U(2)$-gate is physically
operationalizable. This can be done
in the framework of Mach-Zehnder interferometry.
Note that the number of elementary $U(2)$-transformations is
polynomially bounded and does not exceed
$\left(
\begin{array}{c}
n\\2
\end{array}
\right) ={n\,(n-1)/ 2} =O(n^2)
$.



In what follows, a lossless {\em Mach-Zehnder} interferometer drawn in
Fig.
\ref{f:m-z} is discussed.
\begin{figure}
\begin{center}
%TexCad Options
%\grade{\off}
%\emlines{\off}
%\beziermacro{\off}
%\reduce{\on}
%\snapping{\off}
%\quality{0.20}
%\graddiff{0.01}
%\snapasp{1}
%\zoom{1.00}
\unitlength 0.70mm
\linethickness{0.4pt}
\begin{picture}(83.67,61.00)
%\emline(62.67,40.00)(62.67,15.00)
\put(62.67,40.00){\line(0,-1){25.00}}
%\end
\put(10.00,55.00){\makebox(0,0)[cc]{$L$}}
\put(10.00,55.00){\circle{10.00}}
%\emline(15.00,55.00)(55.00,55.00)
\put(15.00,55.00){\line(1,0){40.00}}
%\end
%\emline(44.67,55.00)(62.67,55.00)
\put(44.67,55.00){\line(1,0){18.00}}
%\end
%\emline(25.00,55.00)(25.00,30.00)
\put(25.00,55.00){\line(0,-1){25.00}}
%\end
%\emline(25.00,30.00)(38.00,30.00)
\put(25.00,30.00){\line(1,0){13.00}}
%\end
%\emline(62.67,55.00)(62.67,30.00)
\put(62.67,55.00){\line(0,-1){25.00}}
%\end
%\emline(62.67,30.00)(27.67,30.00)
\put(62.67,30.00){\line(-1,0){35.00}}
%\end
%\emline(62.67,30.00)(75.67,30.00)
\put(62.67,30.00){\line(1,0){13.00}}
%\end
%\emline(62.67,30.00)(62.67,17.00)
\put(62.67,30.00){\line(0,-1){13.00}}
%\end
%\emline(58.67,59.00)(66.67,51.00)
\multiput(58.67,59.00)(0.12,-0.12){67}{\line(0,-1){0.12}}
%\end
%\emline(21.00,35.00)(29.00,25.00)
\multiput(21.00,35.00)(0.12,-0.15){67}{\line(0,-1){0.15}}
%\end
\put(80.17,30.00){\oval(7.00,8.00)[r]}
\put(83.67,36.00){\makebox(0,0)[cc]{$D_1$}}
\put(28.00,42.00){\makebox(0,0)[cc]{$c$}}
\put(25.00,61.00){\makebox(0,0)[cc]{$S_1$}}
\put(56.67,38.00){\makebox(0,0)[cc]{$S_2$}}
%\emline(24.00,56.00)(26.00,54.00)
\multiput(24.00,56.00)(0.12,-0.12){17}{\line(0,-1){0.12}}
%\end
%\emline(23.00,57.00)(21.00,59.00)
\multiput(23.00,57.00)(-0.12,0.12){17}{\line(0,1){0.12}}
%\end
%\emline(27.00,53.00)(29.00,51.00)
\multiput(27.00,53.00)(0.12,-0.12){17}{\line(0,-1){0.12}}
%\end
%\emline(61.67,31.00)(63.67,29.00)
\multiput(61.67,31.00)(0.12,-0.12){17}{\line(0,-1){0.12}}
%\end
%\emline(60.67,32.00)(58.67,34.00)
\multiput(60.67,32.00)(-0.12,0.12){17}{\line(0,1){0.12}}
%\end
%\emline(64.67,28.00)(66.67,26.00)
\multiput(64.67,28.00)(0.12,-0.12){17}{\line(0,-1){0.12}}
%\end
\put(18.00,51.00){\makebox(0,0)[cc]{$a$}}
\put(60.67,41.00){\framebox(4.00,5.00)[cc]{}}
\put(68.67,43.00){\makebox(0,0)[cc]{$\varphi$}}
\put(58.67,43.00){\makebox(0,0)[cc]{$P$}}
\put(70.67,33.00){\makebox(0,0)[cc]{$d$}}
\put(62.67,13.33){\oval(8.67,8.00)[b]}
\put(70.00,9.00){\makebox(0,0)[cc]{$D_2$}}
\put(65.33,20.33){\makebox(0,0)[cc]{$e$}}
\put(43.00,61.00){\makebox(0,0)[cc]{$b$}}
\put(62.00,61.00){\makebox(0,0)[cc]{$M$}}
\put(27.33,21.00){\makebox(0,0)[cc]{$M$}}
\end{picture}
\end{center}
\caption{Mach-Zehnder interferometer.
A single quantum (photon, neutron, electron {\it etc}) is emitted in $L$
and meets a lossless beam splitter (half-silvered mirror) $S_1$, after
which its wave function
is in a coherent superposition of $  b $ and $  c $. In beam
path $b$ a phase shifter shifts the phase of state $  b $ by
$\varphi$. The two beams are then recombined at a second lossless
beam splitter (half-silvered
mirror) $S_2$. The quant is detected at either $D_1$ or $D_2$,
corresponding to the states $d $ and $ e $, respectively.
 \label{f:m-z}}
\end{figure}
The computation proceeds by successive substitution (transition) of
states; i.e.,
\begin{eqnarray}
S_1:\; a  &\rightarrow& ( b  +i c
)/\sqrt{2}\quad , \\
P:\; b  &\rightarrow&  b e^{i \varphi
}\quad ,\\
S_2:\; b  &\rightarrow& ( e  + i
d )/\sqrt{2}\quad ,\\
S_2:\; c  &\rightarrow& ( d  + i
e )/\sqrt{2}\quad .
\end{eqnarray}
The resulting transition is
\begin{equation}
  a  \rightarrow \psi =i\left( {e^{i\varphi} +1\over
2}\right)
d  +
\left( {e^{i\varphi} -1\over 2}\right)
e  \quad .
\label{e:mz}
\end{equation}
Assume that $\varphi =0$, i.e., there is no phase shift at all.
Then, equation (\ref{e:mz}) reduces to
$ a  \rightarrow i d $, and the emitted quant is detected
only by $D_1$.
Assume that $\varphi =\pi $.
Then, equation (\ref{e:mz}) reduces to
$ a  \rightarrow -  e  $, and the emitted quant is detected
only by $D_2$.
If one varies the phase shift $\varphi$, one obtains the following
detection probabilities:
\begin{equation}
P_{D_1}(\varphi )=\vert ( d, \psi ) \vert^2=\cos^2({\varphi
\over 2})
\quad ,
\quad
P_{D_2}(\varphi )=\vert ( e, \psi ) \vert^2=\sin^2({\varphi
\over 2})
\quad .
\end{equation}

For some ``mindboggling'' features of Mach-Zehnder interferometry,
see \cite{benn:94}.

\label{a:u(2)}

\begin{figure}
\begin{center}
\unitlength=0.80mm
\special{em:linewidth 0.4pt}
\linethickness{0.4pt}
\begin{picture}(120.00,200.00)
\put(20.00,120.00){\framebox(80.00,80.00)[cc]{}}
\put(57.67,160.00){\line(1,0){5.00}}
\put(64.33,160.00){\line(1,0){5.00}}
\put(50.67,160.00){\line(1,0){5.00}}
\put(78.67,170.00){\framebox(8.00,4.33)[cc]{}}
\put(82.67,178.00){\makebox(0,0)[cc]{$P_3,\varphi$}}
\put(73.33,160.00){\makebox(0,0)[lc]{$S(\omega )$}}
\put(8.33,183.67){\makebox(0,0)[cc]{${\bf 0}$}}
\put(110.67,183.67){\makebox(0,0)[cc]{${\bf 0}'$}}
\put(110.67,143.67){\makebox(0,0)[cc]{${\bf 1}'$}}
\put(8.00,143.67){\makebox(0,0)[cc]{${\bf 1}$}}
\put(24.33,195.67){\makebox(0,0)[lc]{${\bf T}_{21}^{bs}(\omega ,\alpha ,\beta ,\varphi )$}}
\put(0.00,179.67){\vector(1,0){20.00}}
\put(0.00,140.00){\vector(1,0){20.00}}
\put(100.00,180.00){\vector(1,0){20.00}}
\put(100.00,140.00){\vector(1,0){20.00}}
\put(20.00,14.67){\framebox(80.00,80.00)[cc]{}}
\put(20.00,34.67){\line(1,1){40.00}}
\put(60.00,74.67){\line(1,-1){40.00}}
\put(20.00,74.67){\line(1,-1){40.00}}
\put(60.00,34.67){\line(1,1){40.00}}
\put(55.00,74.67){\line(1,0){10.00}}
\put(55.00,34.67){\line(1,0){10.00}}
\put(37.67,54.67){\line(1,0){5.00}}
\put(44.33,54.67){\line(1,0){5.00}}
\put(30.67,54.67){\line(1,0){5.00}}
\put(77.67,54.67){\line(1,0){5.00}}
\put(84.33,54.67){\line(1,0){5.00}}
\put(70.67,54.67){\line(1,0){5.00}}
\put(88.67,64.67){\framebox(8.00,4.33)[cc]{}}
\put(85.67,66.67){\makebox(0,0)[rc]{$P_4,\varphi$}}
\put(60.00,80.67){\makebox(0,0)[cc]{$M$}}
\put(59.67,29.67){\makebox(0,0)[cc]{$M$}}
\put(31.67,57.67){\makebox(0,0)[rc]{$S_1({1\over 2})$}}
\put(85.33,57.67){\makebox(0,0)[lc]{$S_2({1\over 2})$}}
\put(8.33,78.34){\makebox(0,0)[cc]{${\bf 0}$}}
\put(110.67,78.34){\makebox(0,0)[cc]{${\bf 0}'$}}
\put(110.67,38.34){\makebox(0,0)[cc]{${\bf 1}'$}}
\put(8.00,38.34){\makebox(0,0)[cc]{${\bf 1}$}}
\put(49.00,39.67){\makebox(0,0)[cc]{$c$}}
\put(71.33,68.67){\makebox(0,0)[cc]{$b$}}
\put(24.33,90.34){\makebox(0,0)[lc]{${\bf T}_{21}^{MZ}(\alpha ,\beta ,\omega,\varphi )$}}
\put(0.00,74.34){\vector(1,0){20.00}}
\put(0.00,34.67){\vector(1,0){20.00}}
\put(100.00,74.67){\vector(1,0){20.00}}
\put(100.00,34.67){\vector(1,0){20.00}}
\put(48.67,64.67){\framebox(8.00,4.33)[cc]{}}
\put(56.67,60.67){\makebox(0,0)[lc]{$P_3,\omega$}}
\put(10.00,110.00){\makebox(0,0)[cc]{a)}}
\put(10.00,4.67){\makebox(0,0)[cc]{b)}}
\put(20.00,140.00){\line(2,1){80.00}}
\put(20.00,180.00){\line(2,-1){80.00}}
\put(32.67,170.00){\framebox(8.00,4.33)[cc]{}}
\put(36.67,178.00){\makebox(0,0)[cc]{$P_1,\alpha +\beta $}}
\put(24.67,64.67){\framebox(8.00,4.33)[cc]{}}
\put(29.67,72.67){\makebox(0,0)[lc]{$P_1,\alpha +\beta$}}
\put(24.67,41.67){\framebox(8.00,4.33)[cc]{}}
\put(29.34,37.67){\makebox(0,0)[lc]{$P_2,\beta$}}
\put(32.67,147.00){\framebox(8.00,4.33)[cc]{}}
\put(36.67,155.00){\makebox(0,0)[cc]{$P_2,\beta$}}
\end{picture}
\end{center}
\caption{Elementary quantum interference device.
An elementary quantum interference device can be realized by a
4-port interferometer with two input ports ${\bf 0} ,{\bf 1} $
and two
output ports
${\bf 0} ',{\bf 1} '$.
Any two-dimensional unitary transformation can be realized by the
devices.
a) shows a realization
by a single beam
splitter $S(T)$
with variable transmission $t$
and three phase shifters $P_1,P_2,P_3$;
b) shows a realization with 50:50 beam
splitters $S_1({1\over 2}) $ and $S_2 ({1\over 2})$ and four phase
shifters
$P_1,P_2,P_3,P_4$.
 \label{f:qid}}
\end{figure}
The
elementary quantum interference device ${\bf T}_{21}^{bs}$ depicted in
Fig.
(\ref{f:qid}.a)
is just a beam splitter followed by a phase shifter in one of the output
ports.

Alternatively, the action of a lossless beam splitter may be
described by the matrix $
\left(
\begin{array}{cc}
T(\omega ) &i \, R( \omega )
\\
i\, R(\omega ) &T( \omega )
 \end{array}
\right)
=
\left(
\begin{array}{cc}
\cos \omega  &i \, \sin \omega
\\
i\, \sin \omega  &\cos \omega
 \end{array}
\right)
$.
A phase shifter in a two-dimensional Hilbert space is represented by
either
$
\left(
\begin{array}{cc}
e^{i\varphi }&0
\\
0&1
 \end{array}
\right)
$
or
$
\left(
\begin{array}{cc}
1&0
\\
0&e^{i\varphi }
 \end{array}
\right)
$.
 The action of the entire device consisting of such elements is
calculated by multiplying the matrices in reverse order in which the
quanta pass these elements \cite{yurke-86,teich:90}.
%%%%%%%%%%%%%%%%%%%%%}
\begin{eqnarray}
P_1:\; {\bf 0}  &\rightarrow&  {\bf 0} e^{i
\alpha +\beta}
\quad , \\
P_2:\; {\bf 1}  &\rightarrow&  {\bf 1}
e^{i \beta}
\quad , \\
S:\; {\bf 0}
&\rightarrow& T\, {\bf 1}'  +iR\, {\bf 0}'
\quad , \\
S:\; {\bf 1}  &\rightarrow& T\, {\bf 0}'  +iR\,
{\bf 1}'
\quad , \\
P_3:\; {\bf 0}'  &\rightarrow&  {\bf 0}' e^{i
\varphi
}\quad .
\end{eqnarray}
If
$ {\bf 0}  \equiv    {\bf 0}' \equiv
\left(
\begin{array}{c}
1 \\
0
 \end{array}
\right)
$
and
$ {\bf 1} \equiv    {\bf 1}' \equiv
\left(
\begin{array}{c}
0 \\
1
 \end{array}
\right)
$ and $R(\omega )=\sin \omega $, $T(\omega )=\cos \omega$, then
the corresponding unitary evolution matrix
which transforms any coherent superposition of $ {\bf 0} $
and $ {\bf 1} $
into a superposition of
$ {\bf 0}' $
and
$ {\bf 1}' $
 is given by
\begin{eqnarray}
{\bf T}_{21}^{bs} (\omega ,\alpha ,\beta ,\varphi )&=&
\left[
e^{i\, \beta }\,
\left(
\begin{array}{cc}
i \, e^{i(\alpha +\varphi)} \,  \sin \omega  &e^{i\alpha }\,
  \cos \omega
\\ e^{i\varphi }\, \cos \omega  & i\,\sin \omega
 \end{array}
\right)
\right]^{-1}
\nonumber \\
&=&
e^{-i\, \beta }\,
\left(
\begin{array}{cc}
-i \, e^{-i(\alpha +\varphi)} \,  \sin \omega  &e^{-i\varphi }\,
  \cos \omega
\\ e^{-i\alpha }\, \cos \omega  & -i\,\sin \omega
 \end{array}
\right)
 \quad .
\label{e:quid1}
\end{eqnarray}


The
elementary quantum interference device ${\bf T}_{21}^{MZ}$ depicted in
Fig.
(\ref{f:qid}.b)
is a (rotated) Mach-Zehnder interferometer with {\em two}
input and output ports and three phase shifters.
According to the ``toolbox'' rules, the process can
be quantum mechanically described by
\begin{eqnarray}
P_1:\; {\bf 0}  &\rightarrow&  {\bf 0} e^{i
\alpha +\beta}
\quad , \\
P_2:\; {\bf 1}  &\rightarrow&  {\bf 1} e^{i
\beta}
\quad , \\
S_1:\; {\bf 1}  &\rightarrow& ( b  +i\,
c )/\sqrt{2}
\quad , \\
S_1:\; {\bf 0}  &\rightarrow& ( c  +i\,
b )/\sqrt{2}
\quad , \\
P_3:\; c  &\rightarrow&  c e^{i \omega
}\quad ,\\
S_2:\; b  &\rightarrow& ( {\bf 1}'  + i\,
{\bf 0}' )/\sqrt{2}\quad ,\\
S_2:\; c  &\rightarrow& ( {\bf 0}'  + i\,
{\bf 1}' )/\sqrt{2}\quad ,\\
P_4:\; {\bf 0}'  &\rightarrow&  {\bf 0}' e^{i
\varphi
}\quad .
\end{eqnarray}
When again
$ {\bf 0} \equiv   {\bf 0}' \equiv
\left(
\begin{array}{c}
1 \\
0
 \end{array}
\right)
$
and
$ {\bf 1} \equiv  {\bf 1}' \equiv
\left(
\begin{array}{c}
0 \\
1
 \end{array}
\right)
$, then
the corresponding unitary evolution matrix
which transforms any coherent superposition of $ {\bf 0} $
and $ {\bf 1} $
into a superposition of
$ {\bf 0}' $
and
$ {\bf 1}' $
 is given by
\begin{equation}
{\bf T}_{21}^{MZ} (\alpha ,\beta ,\omega ,\varphi )=
-i\,e^{-i(\beta +{\omega \over 2})}\;\left(
\begin{array}{cc}
-{e^{-i\,({\alpha +\varphi })}}\,\sin {{\omega }\over 2}
&
   {e^{-i\,{\varphi}}}\,\cos {{\omega }\over 2} \\
  e^{-i\,{\alpha }}\,\cos {{\omega }\over 2}&\sin {{\omega }\over
2}
 \end{array}
\right)
 \quad .
\label{e:quid2}
\end{equation}


The correspondence between
${\bf T}_{21}^{bs} (T(\omega ),\alpha ,\beta ,\varphi )$ with
${\bf T}_{21}^{MZ} (\alpha ',\beta ',\omega ',\varphi ')$ in equations
(\ref{e:quid1})
(\ref{e:quid2}) can be verified by comparing the elements of these
matrices.
The resulting four equations can be used to eliminate the four unknown
parameters
$\omega '=2\omega $,
$\beta '=\beta -\omega$,
$\alpha '=\alpha -\pi /2$,
$\beta '=\beta -\, \omega$ and
$\varphi '=\varphi -\pi /2$; i.e.,
\begin{equation}
{\bf T}_{21}^{bs} (\omega ,\alpha ,\beta ,\varphi ) =
{\bf T}_{21}^{MZ} (\alpha -{\pi \over 2}, \beta
-\omega ,2\omega
,\varphi
-{\pi \over 2})
\quad .
\end{equation}


Both elementary quantum interference devices are {\em universal} in the
sense that
{\em every} unitary quantum
evolution operator in two-dimensional Hilbert space can be brought into
a one-to-one correspondence to  ${\bf T}^{bs}_{21}$ and
${\bf T}^{MZ}_{21}$; with corresponding values of
$T,\alpha ,\beta ,\varphi$ or
$\alpha ,\omega ,\beta ,\varphi $.
This can be easily seen by a similar calculation as before; i.e., by
comparing
equations
(\ref{e:quid1})
(\ref{e:quid2}) with the ``canonical''  form of a unitary matrix, which
is the product of a $U(1)=e^{-i\,\beta}$ and
 of the unimodular unitary
matrix $SU(2)$ \cite{murnaghan}
\begin{equation}
{\bf T} (\omega ,\alpha ,\varphi )=
\left(
\begin{array}{cc}
{e^{i\,\alpha }}\,\cos \omega
&
{-e^{-i\,\varphi }}\,\sin \omega
\\
{e^{i\,\varphi }}\,\sin \omega
&
{e^{-i\,\alpha }}\,\cos \omega
 \end{array}
\right)
 \quad ,
\end{equation}
where $-\pi \le \beta ,\omega \le \pi$,
$-\, {\pi \over 2} \le  \alpha ,\varphi \le {\pi \over 2}$.
Let
\begin{equation}
{\bf T} (\omega ,\alpha ,\beta ,\varphi )=
e^{-i\,\beta}{\bf T} (\omega ,\alpha ,\varphi )
\quad .
\end{equation}
A proper identification of the parameters
$\alpha ,\beta ,\omega ,\varphi $ yields
\begin{equation}
{\bf T} (\omega ,\alpha ,\beta ,\varphi )=
{\bf T}_{21}^{bs} (\omega -{\pi \over 2} ,-\alpha -\varphi -{\pi \over
2},
\beta + \alpha  +{\pi  \over 2} ,\varphi -\alpha +{\pi \over 2}
)
\quad .
\end{equation}


Let us examine the realization of a few primitive logical ``gates''
corresponding to (unitary) unary operations on qubits.
The ``identity'' element ${\bf I}$ is defined by
$ {\bf 0}  \rightarrow   {\bf 0} $,
$ {\bf 1}  \rightarrow   {\bf 1} $ and can be realized by
\begin{equation}
{\bf I} =
T^{bs}_{21}(-{\pi \over 2},-{\pi \over 2},{\pi \over 2},{\pi \over 2})=
T^{MZ}_{21}(-\pi ,\pi ,-\pi ,0)
=
\left(
\begin{array}{cc}
1&0
\\
0&1
 \end{array}
\right)
\quad .
\end{equation}

The ``${\tt not}$'' element is defined by
$ {\bf 0}  \rightarrow   {\bf 1} $,
$ {\bf 1}  \rightarrow   {\bf 0} $ and can be realized by
\begin{equation}
{\tt not} =
T^{bs}_{21}(0,0,0,0)=
T^{MZ}_{21}(-\,{\pi \over 2},0,0,-\,{\pi \over 2})
=
\left(
\begin{array}{cc}
0&1
\\
1&0
 \end{array}
\right)
\quad .
\end{equation}


The next element, ``$\sqrt{{\tt not}}$'' is a truly quantum
mechanical; i.e., nonclassical, one, since it converts a classical bit
into
a coherent superposition of $ {\bf 0} $ and $ {\bf 1} $.
$\sqrt{{\tt not}}$ is defined by
$ {\bf 0}  \rightarrow   {\bf 0}  +  {\bf 1} $,
$ {\bf 1}  \rightarrow  - {\bf 0}  +  {\bf 1} $ and can
be realized by
\begin{equation}
\sqrt{{\tt not}} =
T^{bs}_{21}(-{\pi \over 4},-{\pi \over 2},
{\pi \over 2},
{\pi \over 2})=
%%%%{-1+i\over \sqrt{2}}\,
T^{MZ}_{21}(-\pi , {3\pi \over 4} ,-{\pi \over 2},0 )=
{1 \over \sqrt{2}}
\left(
\begin{array}{cc}
1&-1
\\
1&1
 \end{array}
\right)
\quad .
\end{equation}
Note that $\sqrt{{\tt not}}\cdot \sqrt{{\tt not}} = {\tt not}\cdot
{\rm diag}(1,-1)={\tt not}\, ({\rm mod }\, 1)$.
The relative phases in the output ports showing up in ${\rm
diag}(1,-1)$ can be avoided by defining
\begin{equation}
\sqrt{{\tt not}}' =
T^{bs}_{21}(-\,{\pi \over 4},0,
{\pi \over 4},
0)=
T^{MZ}_{21}(-\,{\pi \over 2} , {\pi \over 2} ,-\,{\pi \over 2},
-\,{\pi \over 2}
 )=
{1 \over 2}
\left(
\begin{array}{cc}
1+i&1-i
\\
1-i&1+i
 \end{array}
\right)
\quad .
\end{equation}
With this definition,
$
\sqrt{{\tt not}}'
\sqrt{{\tt not}}' = {\tt not}$.




It is very important that the elementary quantum
interference device realizes an arbitrary
quantum time evolution  of a two-dimensional system.
The  performance of the quantum interference device is determined by
four parameters, corresponding to the phases
$\alpha ,\beta ,\varphi, \omega$.


%\bibliography{svozil}
%\bibliographystyle{unsrt}
%\bibliographystyle{plain}

\begin{thebibliography}{10}

\bibitem{calude-meyerstein}
Cristian Calude and F.~Walter Meyerstein.
\newblock Is the universe lawful?
\newblock {\em Chaos, Solitons \& Fractals}, 10(6):1075--1084, 1999.

\bibitem{svozil-2000interface}
Karl Svozil.
\newblock Quantum interfaces.
\newblock e-print {\tt arXiv:quant-ph/0001064} available
  {http://arxiv.org/abs/quant-ph/0001064}, 2000.

\bibitem{wheeler}
John~A. Wheeler.
\newblock Law without law.
\newblock In John~A. Wheeler and W.~H. Zurek, editors, {\em Quantum Theory and
  Measurement}, pages 182--213. Princeton University Press, Princeton, 1983.
\newblock \cite{wheeler-Zurek:83}.

\bibitem{greenberger:pr2}
Daniel~M. Greenberger.
\newblock Private communication.

\bibitem{hilbert-26}
David Hilbert.
\newblock {\"{U}}ber das {U}nendliche.
\newblock {\em Mathematische Annalen}, 95:161--190, 1926.

\bibitem{schrodinger}
Erwin Schr{\"{o}}dinger.
\newblock Die gegenw{\"{a}}rtige {S}ituation in der {Q}uantenmechanik.
\newblock {\em Naturwissenschaften}, 23:807--812, 823--828, 844--849, 1935.
\newblock English translation in \cite{trimmer} and \cite[pp.
  152-167]{wheeler-Zurek:83}.

\bibitem{kochen1}
Simon Kochen and Ernst~P. Specker.
\newblock The problem of hidden variables in quantum mechanics.
\newblock {\em Journal of Mathematics and Mechanics}, 17(1):59--87, 1967.
\newblock Reprinted in \cite[pp. 235--263]{specker-ges}.

\bibitem{wo-zu}
W.~K. Wooters and W.~H. Zurek.
\newblock A single quantum cannot be cloned.
\newblock {\em Nature}, 299:802--803, 1982.

\bibitem{dieks}
D.~Dieks.
\newblock Communication by {EPR} devices.
\newblock {\em Physics Letters}, 92A(6):271--272, 1982.

\bibitem{mandel:83}
L.~Mandel.
\newblock Is a photon amplifier always polarization dependent?
\newblock {\em Nature}, 304:188, 1983.

\bibitem{mil-hard}
Peter~W. Milonni and M.~L. Hardies.
\newblock Photons cannot always be replicated.
\newblock {\em Physics Letters}, 92A(7):321--322, 1982.

\bibitem{glauber}
R.~J. Glauber.
\newblock Amplifiers, attenuators and the quantum theory of measurement.
\newblock In E.~R. Pikes and S.~Sarkar, editors, {\em Frontiers in Quantum
  Optics}. Adam Hilger, Bristol, 1986.

\bibitem{caves}
C.~M. Caves.
\newblock Quantum limits on noise in linear amplifiers.
\newblock {\em Physical Review}, D26:1817--1839, 1982.

\bibitem{kochen2}
Simon Kochen and Ernst~P. Specker.
\newblock Logical structures arising in quantum theory.
\newblock In {\em Symposium on the Theory of Models, Proceedings of the 1963
  International Symposium at Berkeley}, pages 177--189, Amsterdam, 1965. North
  Holland.
\newblock Reprinted in \cite[pp. 209--221]{specker-ges}.

\bibitem{kochen3}
Simon Kochen and Ernst~P. Specker.
\newblock The calculus of partial propositional functions.
\newblock In {\em Proceedings of the 1964 International Congress for Logic,
  Methodology and Philosophy of Science, Jerusalem}, pages 45--57, Amsterdam,
  1965. North Holland.
\newblock Reprinted in \cite[pp. 222--234]{specker-ges}.

\bibitem{birkhoff-36}
Garrett Birkhoff and John von Neumann.
\newblock The logic of quantum mechanics.
\newblock {\em Annals of Mathematics}, 37(4):823--843, 1936.

\bibitem{benn-92}
Charles~H. Bennett, F.~Bessette, G.~Brassard, L.~Salvail, and J.~Smolin.
\newblock Experimental quantum cryptography.
\newblock {\em Journal of Cryptology}, 5:3--28, 1992.

\bibitem{e-f-moore}
Edward~F. Moore.
\newblock Gedanken-experiments on sequential machines.
\newblock In C.~E. Shannon and J.~McCarthy, editors, {\em Automata Studies}.
  Princeton University Press, Princeton, 1956.

\bibitem{cal-sv-yu}
Cristian Calude, Elena Calude, Karl Svozil, and Sheng Yu.
\newblock Physical versus computational complementarity {I}.
\newblock {\em International Journal of Theoretical Physics}, 36(7):1495--1523,
  1997.

\bibitem{svozil-ql}
K.~Svozil.
\newblock {\em Quantum Logic}.
\newblock Springer, Singapore, 1998.

\bibitem{ekert91}
Artur Ekert.
\newblock Quantum cryptography based on {B}ell's theorem.
\newblock {\em Physical Review Letters}, 67:661--663, 1991.

\bibitem{gilbert-hamrick-200009}
M.~Hamrick G.~Gilbert.
\newblock Practical quantum cryptography: A comprehensive analysis (part one).
\newblock MITRE report MTR 00W0000052 and e-print {\tt arXiv:quant-ph/0009027}
  available at {\tt http://arxiv.org/abs/quant-ph/0009027}, 2000.

\bibitem{zeilinger-qc}
T.~Jenewein, G.~Weihs, C.~Simon, H.~Weinfurter, and A.~Zeilinger.
\newblock Poster, 1998.

\bibitem{svozil-2000int}
Karl Svozil.
\newblock The information interpretation of quantum mechanics.
\newblock e-print {\tt arXiv:quant-ph/0006033} available
  {http://arxiv.org/abs/quant-ph/0006033}, 2000.

\bibitem{murnaghan}
F.~D. Murnaghan.
\newblock {\em The Unitary and Rotation Groups}.
\newblock Spartan Books, Washington, 1962.

\bibitem{shor:94}
Peter~W. Shor.
\newblock Algorithms for quantum computation: discrete logarithms and
  factoring.
\newblock In {\em Proceedings of the 35th Annual Symposium of on Foundations of
  Computer Science, Santa Fe, NM, Nov. 20-22, 1994}. IEEE Computer Society
  Press, November 1994.
\newblock {\tt arXiv:quant-ph/9508027}.

\bibitem{ekerj96}
Artur Ekert and Richard Jozsa.
\newblock Quantum computation and {S}hor's factoring algorithm.
\newblock {\em Reviews of Modern Physics}, 68(3):733--753, 1996.

\bibitem{grover}
L.~K. Grover.
\newblock A fast quantum mechanical algorithm for database search.
\newblock In {\em Proceedings of the Twenty-Eighth Annual ACM Symposium on the
  Theory of Computing}, pages 212--219. 1996.

\bibitem{Gruska}
Josef Gruska.
\newblock {\em uantum Computing}.
\newblock McGraw-Hill, London, 1999.

\bibitem{gottlob}
Georg Gottlob.
\newblock Private communication.

\bibitem{cal-mike-svo}
Cristian~S. Calude, Michael~J. Dinneen, and Karl Svozil.
\newblock Reflections on quantum computing.
\newblock {\em Complexity}, 2000.
\newblock in print;\\e-print {\tt
  http://www.cs.auckland.ac.nz/CDMTCS//researchreports/130cris.pdf}.

\bibitem{mahler-priv}
G{\"{u}}nther Mahler.
\newblock private communication.

\bibitem{planck:1901}
Max Planck.
\newblock Ueber eine {V}erbesserung der {W}ien'schen {S}pectralgleichung.
\newblock {\em Verhandlungen der deutschen physikalischen Gesellschaft}, 2:202,
  1900.
\newblock See also \cite{planck:1901b}.

\bibitem{planck:1901b}
Max Planck.
\newblock Ueber das {G}esetz der {E}nergieverteilung im {N}ormalspectrum.
\newblock {\em Annalen der Physik}, 4:553--566, 1901.

\bibitem{planck:1901a}
Max Planck.
\newblock Zur {T}heorie des {G}esetzes der {E}nergieverteilung im
  {N}ormalspectrum.
\newblock {\em Verhandlungen der deutschen physikalischen Gesellschaft}, 2:237,
  1900.
\newblock See also \cite{planck:1901b}.

\bibitem{ein-5}
Albert Einstein.
\newblock {\"{U}}ber einen die {E}rzeugung und {V}erwandlung des {L}ichtes
  betreffenden heuristischen {G}esichtspunkt.
\newblock {\em Annalen der Physik}, 17:132--148, 1905.

\bibitem{feynman-III}
Richard~P. Feynman, Robert~B. Leighton, and Matthew Sands.
\newblock {\em The Feynman Lectures on Physics. Quantum Mechanics}, volume III.
\newblock Addison-Wesley, Reading, MA, 1965.

\bibitem{har}
E.~G. Harris.
\newblock {\em A Pedestrian Approach to Quantum Field Theory}.
\newblock Wiley-Interscience, New York, 1971.

\bibitem{lipkin}
H.~J. Lipkin.
\newblock {\em Quantum Mechanics, New Approaches to Selected Topics}.
\newblock North-Holland, Amsterdam, 1973.

\bibitem{ba-89}
L.~E. Ballentine.
\newblock {\em Quantum Mechanics}.
\newblock Prentice Hall, Englewood Cliffs, NJ, 1989.

\bibitem{messiah-61}
A.~Messiah.
\newblock {\em Quantum Mechanics}, volume~I.
\newblock North-Holland, Amsterdam, 1961.

\bibitem{davydov}
A.~S. Davydov.
\newblock {\em Quantum Mechanics}.
\newblock Addison-Wesley, Reading, MA, 1965.

\bibitem{dirac}
P.~A.~M. Dirac.
\newblock {\em The Principles of Quantum Mechanics}.
\newblock Oxford University Press, Oxford, 1947.

\bibitem{peres}
Asher Peres.
\newblock {\em Quantum Theory: Concepts and Methods}.
\newblock Kluwer Academic Publishers, Dordrecht, 1993.

\bibitem{mackey:63}
George~W. Mackey.
\newblock {\em The Mathematical Foundations of Quantum Mechanics}.
\newblock W. A. Benjamin, Reading, MA, 1963.

\bibitem{v-neumann-49}
John von Neumann.
\newblock {\em Mathematische Grundlagen der Quantenmechanik}.
\newblock Springer, Berlin, 1932.
\newblock English translation: {\sl Mathematical Foundations of Quantum
  Mechanics}, Princeton University Press, Princeton, 1955.

\bibitem{bell-87}
John~S. Bell.
\newblock {\em Speakable and Unspeakable in Quantum Mechanics}.
\newblock Cambridge University Press, Cambridge, 1987.

\bibitem{jammer1}
Max Jammer.
\newblock {\em The Philosophy of Quantum Mechanics}.
\newblock John Wiley \& Sons, New York, 1974.

\bibitem{wheeler-Zurek:83}
John~Archibald Wheeler and Wojciech~Hubert Zurek.
\newblock {\em Quantum Theory and Measurement}.
\newblock Princeton University Press, Princeton, 1983.

\bibitem{dunford-schwartz}
N.~Dunford and J.~T. Schwartz.
\newblock {\em Linear Operators I}.
\newblock Interscience Publishers, New York, 1958.

\bibitem{reed-sim1}
Michael Reed and Barry Simon.
\newblock {\em Methods of Mathematical Physics I: Functional Analysis}.
\newblock Academic Press, New York, 1972.

\bibitem{reed-sim2}
Michael Reed and Barry Simon.
\newblock {\em Methods of Mathematical Physics II: Fourier Analysis,
  Self-Adjointness}.
\newblock Academic Press, New York, 1975.

\bibitem{rzbb}
M.~Reck, Anton Zeilinger, H.~J. Bernstein, and P.~Bertani.
\newblock Experimental realization of any discrete unitary operator.
\newblock {\em Physical Review Letters}, 73:58--61, 1994.
\newblock See also \cite{murnaghan}.

\bibitem{greenberger2}
Daniel~B. Greenberger and A.~YaSin.
\newblock ``{H}aunted'' measurements in quantum theory.
\newblock {\em Foundation of Physics}, 19(6):679--704, 1989.

\bibitem{hkwz}
Thomas~J. Herzog, Paul~G. Kwiat, Harald Weinfurter, and Anton Zeilinger.
\newblock Complementarity and the quantum eraser.
\newblock {\em Physical Review Letters}, 75(17):3034--3037, 1995.

\bibitem{finkelstein-83}
David Finkelstein and Shlomit~R. Finkelstein.
\newblock Computational complementarity.
\newblock {\em International Journal of Theoretical Physics}, 22(8):753--779,
  1983.

\bibitem{chaitin-65}
Gregory~J. Chaitin.
\newblock An improvement on a theorem by {E}. {F}. {M}oore.
\newblock {\em IEEE Transactions on Electronic Computers}, EC-14:466--467,
  1965.

\bibitem{conway}
J.~H. Conway.
\newblock {\em Regular Algebra and Finite Machines}.
\newblock Chapman and Hall Ltd., London, 1971.

\bibitem{brauer-84}
W.~Brauer.
\newblock {\em Automatentheorie}.
\newblock Teubner, Stuttgart, 1984.

\bibitem{svozil-93}
Karl Svozil.
\newblock {\em Randomness \& Undecidability in Physics}.
\newblock World Scientific, Singapore, 1993.

\bibitem{schaller-92}
Martin Schaller and Karl Svozil.
\newblock Partition logics of automata.
\newblock {\em Il Nuovo Cimento}, 109B:167--176, 1994.

\bibitem{schaller-95}
Martin Schaller and Karl Svozil.
\newblock Automaton partition logic versus quantum logic.
\newblock {\em International Journal of Theoretical Physics}, 34(8):1741--1750,
  August 1995.

\bibitem{schaller-96}
Martin Schaller and Karl Svozil.
\newblock Automaton logic.
\newblock {\em International Journal of Theoretical Physics}, 35(5):911--940,
  May 1996.

\bibitem{dvur-pul-svo}
Anatolij Dvure{\v{c}}enskij, Sylvia Pulmannov{\'{a}}, and Karl Svozil.
\newblock Partition logics, orthoalgebras and automata.
\newblock {\em Helvetica Physica Acta}, 68:407--428, 1995.

\bibitem{svo:za}
Karl Svozil and Roman~R. Zapatrin.
\newblock Empirical logic of finite automata: microstatements versus
  macrostatements.
\newblock {\em International Journal of Theoretical Physics}, 35(7):1541--1548,
  1996.

\bibitem{svozil-tkadlec}
Karl Svozil and Josef Tkadlec.
\newblock Greechie diagrams, nonexistence of measures in quantum logics and
  {K}ochen--{S}pecker type constructions.
\newblock {\em Journal of Mathematical Physics}, 37(11):5380--5401, November
  1996.

\bibitem{reck-94}
M.~Reck and Anton Zeilinger.
\newblock Quantum phase tracing of correlated photons in optical multiports.
\newblock In F.~De Martini, G.~Denardo, and Anton Zeilinger, editors, {\em
  Quantum Interferometry}, Singapore, 1994. World Scientific.

\bibitem{benn:94}
Charles~H. Bennett.
\newblock Night thoughts, dark sight.
\newblock {\em Nature}, 371:479--480, 1994.

\bibitem{yurke-86}
B.~Yurke, S.~L. McCall, and J.~R. Klauder.
\newblock {SU(2)} and {SU(1,1)} interferometers.
\newblock {\em Physical Review}, A33:4033--4054, 1986.

\bibitem{teich:90}
R.~A. Campos, B.~E.~A. Saleh, and M.~C. Teich.
\newblock Fourth-order interference of joint single-photon wave packets in
  lossless optical systems.
\newblock {\em Physical Review}, A42:4127, 1990.

\bibitem{trimmer}
J.~D. Trimmer.
\newblock The present situation in quantum mechanics: a translation of
  {S}chr{\"{o}}dinger's ``cat paradox''.
\newblock {\em Proc. Am. Phil. Soc.}, 124:323--338, 1980.
\newblock Reprinted in \cite[pp. 152-167]{wheeler-Zurek:83}.

\bibitem{specker-ges}
Ernst Specker.
\newblock {\em Selecta}.
\newblock Birkh{\"{a}}user Verlag, Basel, 1990.

\end{thebibliography}

\end{document}
