\documentclass[%
 %reprint,
 %twocolumn,
 %superscriptaddress,
 %groupedaddress,
 %unsortedaddress,
 %runinaddress,
 %frontmatterverbose,
  preprint,
 showpacs,
 showkeys,
 preprintnumbers,
 %nofootinbib,
 %nobibnotes,
 %bibnotes,
 amsmath,amssymb,
 aps,
 %prl,
   pra,
 % prb,
 % rmp,
 %prstab,
 %prstper,
  longbibliography,
 %floatfix,
 %lengthcheck,%
 ]{revtex4-1}

%\usepackage{cdmtcs-pdf}

\usepackage[breaklinks=true,colorlinks=true,anchorcolor=blue,citecolor=blue,filecolor=blue,menucolor=blue,pagecolor=blue,urlcolor=blue,linkcolor=blue]{hyperref}
\usepackage{graphicx}% Include figure files
\usepackage{url}

\usepackage[dvipsnames]{xcolor}
\usepackage{tikz}

\newtheorem{question}{Question}
\newtheorem{conjecture}[question]{Principle}
\newtheorem{challenge}[question]{Challenge}

\begin{document}


\title{Relazione finale sulle attivit\`{a} del / Final report of \\Visiting Professor Karl Svozil}

%\cdmtcsauthor{Karl Svozil}
%\cdmtcsaffiliation{Vienna University of Technology}
%\cdmtcstrnumber{407}
%\cdmtcsdate{September 2011}
%\coverpage

\author{Karl Svozil}
\affiliation{Institute for Theoretical Physics, Vienna
    University of Technology, Wiedner Hauptstra\ss e 8-10/136, A-1040
    Vienna, Austria}
\affiliation{Dipartimento di Scienze Pedagogiche e Filosofiche, Universit\'a  di Cagliari,\\
 Via Is Mirrionis, 1, I-09123, Cagliari, Sardinia, Italy}
\email{svozil@tuwien.ac.at} \homepage[]{http://tph.tuwien.ac.at/~svozil}



\begin{abstract}
In what follows a final report on the teaching and research activities of Professor Karl Svozil
during his three months stay from May-July 2012 will be given.
\end{abstract}


\maketitle
\tableofcontents

%newpage

\section{Overview}

Visiting Professor Karl Svozil (K.S.) has visited the Dipartimento di Scienze Pedagogiche e Filosofiche, Universit\'a  di Cagliari (UNICA)
from May to July 2012,and has conducted scientific research as well as teaching.
One of the main scientific topics discussed and pursued  was the theory of quantum information and computation,
as well as conceptional issues related to physical value (in-)definiteness, as it concerns concrete physical quantum random number generators.

K.S.'s main cooperation partners have been members of the group of Professor Roberto Giuntini.

K.S. has offered a course and seminars for UNICA students and post-docs; and contributed talks two conferences in Cagliari.


\section{Teaching and lecturing}

As announced by Professor Marco Giunti, K.S. has offered a course on
{\em ``Physical aspects of classical and quantum information and computation''}
with the following syllabus

\begin{flushleft}
I. Foundations of Computer Science                                                \\
A. What is an algorithm?                                                          \\
B. Uncomputability                                                                \\
C. Karp-Cook Thesis                                                               \\
II. Physical foundations of computation                                           \\
A. Reversible computation and statistical physics                                 \\
B. Quantum Mechanics                                                              \\
C. Various theorems relating to hidden parameter models                           \\
III. Fundamental Properties of Cbits and Qbits                                    \\
A. Cbits and their state                                                          \\
B. Reversible Operations on Cbit                                                  \\
C. Qbits and their state                                                          \\
D. Reversible Operations on Qbit                                                  \\
E. The measurement of Qbit                                                        \\
IV. Quantum Computation: General features and some simple examples                \\
A. General computational process                                                  \\
B. No-cloning theorem                                                             \\
C. Deutsch's Problem  and Parity                                                  \\
D. Bernstein-Vazirani problem                                                     \\
E. Factoring                                                                      \\
V. Quantum recursion theory                                                       \\
VI. Counterfactual quantum  computation                                           \\
VII. Quantum cryptography                                                         \\
A.  Wiesner's conjugate coding                                                    \\
B. BB84 Protocol                                                                  \\
C. Wright's generalized urn model and chocolate ball cryptography
\end{flushleft}

Some of these topics were also covered by K.S.'s review presentations on related scientific articles at the
weekly reading seminars of the group.



\section{Conference participation}

During his stay at UNICA, K.S. contributed to two conferences and gave  talks there.


\subsection{First Joint Cagliari-Olmouc Workshop on Algebraic Logic}

\begin{flushleft}
UNICA Cagliari, 14-16 maggio 2012,        \\
URL {\tt http://www.unica.it/pub/7/show.jsp?id=18496}\\
Marted� 15 maggio 2012, aula 6                                      \\
10.30--11.30 Karl Svozil (TU Wien e Universit� di Cagliari),          \\
{\em ``Quantification of contextuality''}
\end{flushleft}

\subsection{11th Biennial IQSA Meeting Quantum Structures Cagliari 2012}

\begin{flushleft}
UNICA Cagliari 23 - 27 July\\
URL {\tt http://www.iqsa2012.org/}\\
July 23rd, Monday, aula 11          \\
16:40--17:15 Karl Svozil (TU Wien e Universit� di Cagliari),\\
{\em ``The Present Situation in
Quantum Mechanics and the
Ontological Single Pure State
Conjecture''}
\end{flushleft}


\section{Scientific research}
During his time at UNICA, K.S. conducted research in quantum logic and quantum computation, as well as
in the foundations of quantum mechanics; both alone and with members of the group of Professor Giuntini.

It was also very interesting for K.S.
to communicate with Professor Michele Camerota about his article (together with Mario Helbing from the ETH Z\"urich)
entitled
{\em ``{G}alileo and {P}isan {A}ristotelianism: {G}alileo's
De Motu Antiquiora and the Quaestiones De Motu Elementorum of the {P}isan Professors,''}
which, in some for K.S. unexpected ways, characterizes a situation not dissimilar from today's scientific disputes
about some unresolved aspects of quantum mechanics.


As a result, the following two papers appeared as preprints.

\subsection{arXiv:1206.6024}

Quantum Physics:
The present situation in quantum mechanics and the ontological single pure state conjecture

Submitted on 26 June 2012

Abstract:
Despite its excessive success in predicting experimental frequencies and certain single outcomes, the "new quantum mechanics" is haunted by several conceptual and technical issues; among them (i) the (non-)existence of measurement and the cut between observer and object in an environment globally covered by a unitary (i.e. one-to-one Laplacian deterministic) evolution; related to the question of how many-to-one mappings could possibly "emerge" from one-to-one functions; and also where exactly "randomness resides;" (ii) what constitutes a pure quantum state; (iii) the epistemic or ontic (non-)existence of mixed states; related to the question of how non-pure states can be "produced" from pure ones; as well as (iv) the epistemic or ontic existence of pure but entangled and/or coherent states containing classically mutually exclusive states; an issue the late Schr\"odinger has called "quantum quagmire" or "jellification;" (v) the (non-)existence of quantum value indefiniteness and its purported "resolution" by
quantum contextuality; and finally (vi) the claim that the best interpretation of the quantum formalism is its non-interpretation. All of these can be overcome by assuming that, at any given time, only a single pure state exists; and that the quantum evolution "permutes" this state in its Hilbert space.

URL: {\tt http://arxiv.org/abs/1206.6024}

\subsection{arXiv:1207.2029}

Quantum Physics:
Kochen-Specker Theorem Revisited and Strong Incomputability of Quantum Randomness

together with Alastair A. Abbott, Cristian S. Calude, Jonathan Conder

Submitted on 9 July 2012

Abstract:
We present a stronger variant of the Kochen-Specker theorem in which some quantum observables are identified to be provably value indefinite. This result is utilised for the construction and certification of a dichotomic quantum random number generator operating in a three-dimensional Hilbert space.

URL: {\tt http://arxiv.org/abs/1207.2029}


Further work on epistemic perceptions of quantum states will be jointly (by K.S. and the UNICA researchers) conducted in the future.

\section{General remarks and observations}

The time at UNICA has been very productive and insightful; K.S. would like to express his admiration for the achievements
of the research group hosting him.

K.S. is deeply thankful for the opportinity this Visiting Professorship has offered
to get into contact with students and fellow researchers an UNICA.
Further academic joint activities are planned and will follow.

\bibliography{svozil}

\end{document}
