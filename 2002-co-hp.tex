\documentclass{epl}

\title{Comment on ``{E}xclusion of time in the theorem of {B}ell'' by {K}.
  {H}ess and {W}. {P}hilipp}
\author{K. Svozil\inst{1}}
\institute{
  \inst{1} Institut f\"ur Theoretische Physik, University of Technology Vienna,
Wiedner Hauptstra\ss e 8-10/136, A-1040 Vienna, Austria
}
\pacs{03.65.Ud}{Entanglement and quantum nonlocality}

\begin{document}

\maketitle

\begin{abstract}
Bell's inequalities are a variant of Boole's legendary
consistency ``conditions of possible experience.''
Although they do not specifically refer to spatially
separated subsystems, they apply to spatially separated particles as well.
Such an interpretation appears to be immune to arguments involving time dependencies
put forward recently by Hess and Philipp \cite{Hess&Philipp2002}.
\end{abstract}


In the middle of the 19th century the English mathematician George Boole
formulated a theory of ``conditions of possible experience'' (COPE)
\cite{Boole,Boole-62,Pit-94}.
These conditions subsume the consistency requirements
satisfied by relative frequencies or probabilities of
classical events.
They are expressed by certain equations or inequalities.

More recently, COPE for a particular setup relevant in the
quantum mechanical context have been studied by Bell and others.
To name a specific example, the Clauser, Horne, Shimony, and Holt (CHSH) inequalities are just COPE for a particular
physical setup studied by CHSH.
Pitowsky, by pointing out this fact,  has given a geometrical interpretation
of COPE in terms of correlation polytopes
\cite{pitowsky}.
Thereby, the rows of the truth tables of events and their joints are interpreted
as vectors of Euclidean space.
A correlation polytope is defined by taking all such vectors and interpreting them as
the extreme points of the polytope.
%The Minkowski-Weyl representation theorem
%states that  every convex polytope has a dual description:
%either as the convex hull of its vertices,
%or as the intersection of a finite number of half-spaces,
%each one given by a linear inequality.
%The problem to obtain all inequalities from the vertices of a convex
%polytope is known as the  hull problem.


The physical interpretation of the inequalities representing the boundaries of the
Pitowsky correlation polytope is  this:
Any face of the polytope has an ``inside'' and an ``outside,''
and corresponds to a Boole-Bell type inequality.
It can be viewed as a sort of demarcation line,
a maximal border, between the classically allowed probabilities
and the ones (outside of the polytope) which are inconsistent
with a classical description
of observables as a Boolean algebra.
Quantum probabilities exceed the borders of the classical correlation
polytopes: they "lie outside."
But also quantum probabilities are subject to certain constraints
%\cite{cirelson,pit:range-2001}
and do not violate
the inequalities maximally.


The elementary algebraic derivation of
Boole's COPE and hence also of Bell-type inequalities do not refer to any spatial or temporal dependence whatsoever.
They just state consistency requirements for classical probabilities
and correlations of a multiparticle system.
Thereby, the particles could, but need not be spatially separated.
However, note that these arguments apply also to systems which are spatially separated and thus also
to the Einstein-Podolsky-Rosen (EPR) setup.

But as there is no spatial or temporal dependence involved in the argument, attempts to reproduce
quantum-like behaviours by considering certain time dependencies and correlations
originating from purely classical local operations are in vain.
It should again be stressed that the argument involving COPE does not make any use of locality whatsoever,
neither does it make use of any nonlocality: it is completely independent of such possible features.
It enumerates the purely classical constraints imposed by consistency requirements,
which are derived by elementary and purely algebraic means.
Whether or not the single particles forming the whole system are spatially or temporally separated
does not matter at all.
As long as the particles are classical, they have to obey COPE.
They are necessary constraints on classical physical systems.
(This, however, does not exclude more exotic possibilities such as
the nonconstructive probability measures introduced by Pitowsky \cite{pitowsky-82},
which could give rise to paradoxical decompositions.
%But in these cases one allows the use of highly nonoperational means, for which Boole's assumptions do not apply.
Nothing of that type is mentioned by  Hess and Philipp.)

Pointedly stated: in pretending to be able to put forward an argument
invalidating COPE, more specifically the CHSH or Bell's inequalities (in a spatially separated setup) by
purely classical means, amounts to prentending to be able to prove that ``$5+3\neq 3+5$ by
starting out with a system of axioms for arithmetic which includes the commutative law of addition.
Consistency requires that no specific example can contradict
a proposition which has been proven in full generality which also applies to this specific case.



Admittedly, independence of the Bell inequalities from spatial separateness is not
the most evident feature, in particular not in the spirit of EPR.
Spatial separateness has not been considered a flaw
of the setup but rather an essential property to assure statistical independence,
although  very early arrangements
had problems with intrinsic time correlations \cite{zeilinger-86}.
What may remain from the arguments of
Hess and Philipp
(see, for example, \cite{Gill-Weihs-Z-Z,Mermin-2002}
%\cite{Suarez-2002,Khrennikov-2002,Myrvold-2002,Gill-Weihs-Z-Z,Mermin-2002}
for other criticism, and  \cite{Hess&Philipp2002a} for a response), although these authors did not intend this,
is the suggestion to look at nonclassical, nonlocal quantum entanglement
between particles  with varying distances.



%\bibliography{svozil}
%%\bibliographystyle{apsrev}
%\bibliographystyle{unsrt}
%%\bibliographystyle{plain}

\begin{thebibliography}{10}

\bibitem{Hess&Philipp2002}
K.~Hess and W.~Philipp.
\newblock Exclusion of time in the theorem of {B}ell.
\newblock {\em Europhysics Letters}, 57(6):775--781, 2002.

\bibitem{Boole}
George Boole.
\newblock {\em An investigation of the laws of thought}.
\newblock Dover edition, New York, 1958.

\bibitem{Boole-62}
George Boole.
\newblock On the theory of probabilities.
\newblock {\em Philosophical Transactions of the Royal Society of London},
  152:225--252, 1862.

\bibitem{Pit-94}
Itamar Pitowsky.
\newblock {G}eorge {B}oole's `conditions od possible experience' and the
  quantum puzzle.
\newblock {\em Brit. J. Phil. Sci.}, 45:95--125, 1994.

\bibitem{pitowsky}
Itamar Pitowsky.
\newblock {\em Quantum Probability---Quantum Logic}.
\newblock Springer, Berlin, 1989.

\bibitem{pitowsky-82}
Itamar Pitowsky.
\newblock Resolution of the {E}instein-{P}odolsky-{R}osen and {B}ell paradoxes.
\newblock {\em Physical Review Letters}, 48:1299--1302, 1982.
\newblock Cf. N. D. Mermin, {\sl Physical Review Letters} {\bf 49}, 1214
  (1982); A. L. Macdonald, {\sl Physical Review Letters} {\bf 49}, 1215 (1982);
  Itamar Pitowsky, {\sl Physical Review Letters} {\bf 49}, 1216 (1982).

\bibitem{zeilinger-86}
Anton Zeilinger.
\newblock Testing {B}ell's inequalities with periodic switching.
\newblock {\em Phys.Lett. A}, 118:1--2, 1986.

\bibitem{Gill-Weihs-Z-Z}
R.D. Gill, G.~Weihs, A.~Zeilinger, and M.~Zukowski.
\newblock Comment on ``{E}xclusion of time in the theorem of {B}ell'' by {K}.
  {H}ess and {W}. {P}hilipp.
\newblock eprint quant-ph/0204169, 2002.

\bibitem{Mermin-2002}
N.~David Mermin.
\newblock Shedding (red and green) light on time related hidden parameters.
\newblock {\tt arXiv:quant-ph/0206118}, 2002.

\bibitem{Hess&Philipp2002a}
K.~Hess and W.~Philipp.
\newblock Logical inconsistencies in proofs of the theorem of {B}ell, 2002.
\newblock eprint quant-ph/0206046.

\end{thebibliography}


\end{document}

%See fig.~\ref{f.1}, table~\ref{t.1} and eq.~(\ref{e.1}).
%See also~\cite{b.a,b.b}.
%\begin{equation}
%\label{e.1}
%0\neq1
%\end{equation}
%
%\begin{figure}
%\caption{Figure caption.}
%\label{f.1}
%\end{figure}
%
%\begin{table}
%\caption{Table caption.}
%\label{t.1}
%\begin{center}
%\begin{tabular}{lcr}
%first  & table & row\\
%second & table & row
%\end{tabular}
%\end{center}
%\end{table}
%
%%\acknowledgments
%%Paper text.
%%
%%\begin{thebibliography}{0}
%
%\bibitem{b.a}
% \Name{Author F., Author S. \and Author T.}
% \REVIEW{Some Rev. A}{69}{1969}{9691}.
%
%\bibitem{b.b}
% \Name{Author F. \and Author S.}
% \Book{Some Book of Interest}
% \Editor{A. Editor}
% \Vol{9}
% \Publ{Publishing house, City}
% \Year{1939}
% \Page{666}.
%
%\bibitem{b.c}
% \Editor{Editor A.}
% \Book{Some Book of Interest}
% \Vol{9}
% \Publ{Publishing house, City}
% \Year{1939}
% \Page{666}.
%
%\end{thebibliography}
