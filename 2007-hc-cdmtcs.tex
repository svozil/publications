\documentclass[12pt]{article}

\usepackage{cdmtcs-pdf}

\usepackage{amsfonts}

\RequirePackage{times}
%\RequirePackage{courier}
\RequirePackage{mathptm}

\newtheorem{definition}{{\bf Definition} }



\begin{document}

\title{How to acknowledge hypercomputation?}
\author{Alexander Leitsch\footnote{leitsch@logic.at}, G{\"{u}}nter Schachner \\
 {\small Institut f{\"{u}}r Computersprachen, Vienna University of Technology,}\\ {\small   Favoritenstr.9/185, 1040 Vienna, Austria}\\
{and}\\
Karl Svozil\footnote{svozil@tuwien.ac.at} \\
 {\small Institute for Theoretical Physics, Vienna University of Technology,}\\ {\small   Wiedner Hauptstra\ss e 8-10/136, 1040 Vienna, Austria}
      }

\cdmtcsauthor{Alexander Leitsch, Guenter Schachner and Karl Svozil}
\cdmtcsaffiliation{University of Technology, Vienna }
\cdmtcstrnumber{315}
\cdmtcsdate{December 2007}
\coverpage

\date{ }
\maketitle

%\begin{flushright}
%{\scriptsize http://tph.tuwien.ac.at/$\widetilde{\;\;}\,$svozil/publ/2000-cesena.$\{$htm,ps,tex$\}$}
%\end{flushright}

\begin{abstract}
We discuss the question of how to operationally validate whether or not a ``hypercomputer'' performs better than the known discrete computational models.
\end{abstract}

\section{Introduction}

It is widely acknowledged \cite{wolfram-2002,svozil-2005-cu} that every physical system corresponds to a computational process,
and that every computational process, if applicable, has to be physically and operationally feasible in some concrete realization.
In this sense, the physical and computational capacities should match;
because if one is lagging behind the other,
there is a lack in the formalism and its potential scientific and ultimatively technological applicability.
Therefore, the exact correspondence of the mathematical formalism on the one hand, and
the particular physical system which is represented by that formalism  on the other hand, demands careful attention.



If one insists on operationalizability
\cite{bridgman},
one needs not go very far in the history of mathematics to encounter
suspicious mathematical objects.
Surely enough, the number $\pi$ can be defined and effectively computed as the ratio of the
circumference to the diameter of a ``perfect (platonic)'' circle.
Likewise, the numbers $\sqrt{2}$ and $\sqrt{3}$ can be interpreted as the ratio
between the length of the diagonal to
the side length of any square and cube.
But it is not totally unjustified to ask whether or not these numbers have
any operational meaning
in a strict physical sense; i.e., whether such numbers could, at least in
principle, be constructed and measured with arbitrary or even with absolute
precision.

At the heart of most of the problems seems to lie the ancient issue of the ``very large/small'' or even potential
infinite versus the actual infinite.
Whereas the mathematical formalism postulates the existence of  actual
infinite constructions and methods,
such as the summation of a (convergent) geometric series, or diagonalization,
the physical processes, methods and techniques are never infinite.

Suppose, as an example, one would attempt the operationalization of $\pi$.
Any construction of a ``real'' circle and one of its diameters, and a
subsequent measurement thereof,
would find its natural scale bound from below by the atomistic structure of
matter upon which any such  circle
is based.
Long before those molecular or atomic scales,
the physical geometry might turn out to be not as straightforward as it
appears from
larger scales; e.g., the object might turn out to be a fractal.

Chaitin's omega number \cite{chaitin3}, which is
interpretable as the halting probability of a universal computer,
can be ``computed in the limit'' (without any computable radius of convergence)
by a finite-size program in infinite time and with infinite space.
Just as for pi --- the difference being the absence of any computable radius of convergence --- the first digits of omega are well known
\cite{calude-dinneen06}, yet
omega is provable algorithmically incompressible and thus random.
Nevertheless, presently, for all practical purposes, the statement that
``the $10^{10^{10^{10}}}$th digit in a decimal expansion of pi is $5$''
is as unverifiable as a similar statement for omega.
Omega encodes all decision problems which can be algorithmically interpreted.
For instance, for a particular universal computer,
Goldbach's conjecture and Riemann's hypothesis could be decided with programs of size 3484 and 7780 bits,
respectively \cite{calude-elena-dinneen06}.
Yet, omega  appears to have two features which are normally considered contradictory:
it is one of the most informative mathematical numbers imaginable.
Yet at the same time this information is so compressed that it cannot be deciphered;
thus omega appears to be totally structureless and random.
In this sense, for omega,  total information and total randomness seem to be ``two sides of the same coin.''
On a more pragmatic level, it seems impossible here to differentiate between order and chaos,
between knowledge and chance.
This gives a taste of what can be expected from any ``hypercomputation''
beyond universal computability as defined by Turing.


It should be always kept in mind that all our sense perceptions are derived from elementary discrete events,
such as clicks in photon or particle detectors, even if they appear to be
analog:
the apparently smooth behavior has a discrete  fine structure.
Among other issues, such as finiteness of system resources,
this discreteness seems to prohibit the ``physical realization'' of any actual infinities.

What is the physical meaning of infinite  concepts,
such as space-time singularities, point particles, or infinite precision?
For instance, are infinity machines with geometrically squeezed time cycles,
such as the ones envisioned by Weyl \cite{weyl:49}
and others \cite{gruenbaum:74,thom:54,benna:62,rucker,pit:90,ear-nor:93,hogarth1,hogarth2,beth-59,le-91,sv-aut-rev}
physically feasible?
Motivated by recent proposals to utilize quantum computation for trespassing the Turing barrier
\cite{2002-cal-pav,ad-ca-pa,kieu-02,kieu-02a},
these accelerating Turing machines have been intensively discussed \cite{ord-2006}
among other forms of hypercomputation \cite{Davis-2004,Doria-2006,Davis-2006}.


Certainly, the almost ruthless and consequential application of seemingly
mind-boggling theories
such as quantum mechanics, as  far as finitistic methods are concerned, has
yielded one victory after another.
But it should be kept in mind that the use of actual
transfinite concepts and methods remains highly conjectural.


A priori, while it may appear rash to exclude the transfinite in general, and transfinite set theory
in particular from physics proper,
one should be aware of its counterintuitive consequences, such as for instance
the Banach-Tarski paradox, and be careful in claiming its physical
applicability.
Recall the old phrase attributed to Einstein and Infeld
(Ref.~\cite{ein-in}, p.31),
{\em ``Physical concepts are free creations of the human
   mind, and are not, however it may seem,
   uniquely determined by the external world.''}

To this point, we are not aware of any test, let alone any application, of the actual
transfinite in Nature.
While general contemplations about hypercomputations and the applicability of transfinite concepts
for physics may appear philosophically interesting,
our main concern will be operational {\em testability}:
if presented with claims that hypercomputers exist, how could we possibly falsify, or even verify and test such propositions
\cite{Chow-2004}?

In what follows, hypercomputation will be conceptualized in terms of a black box with its input/output behavior.
Several tests and their rather limited scope will be evaluated.
Already in 1958, Davis \cite[p. 11]{davis-58}
sets the stage of the following discussion by pointing out
 {\em `` $\ldots$ how can we ever exclude the possibility of our being presented,
 some day (perhaps by some extraterrestrial visitors), with a (perhaps
 extremely complex) device or ``oracle'' that ``computes'' a
 non-computable function?''}
While this may have been a remote, amusing issue in the days written,
the advancement of physical theory in the past decades
has made necessary
a careful evaluation of the possibilities and options for
verification and falsification of certain claims that a concrete physical system
``computes''   a  non-computable function.



\section{On black boxes which are hypercomputers}

In what follows we shall consider a
 device, an agent, or an oracle which one knows nothing about, has no
rational understanding (in the traditional algorithmic sense) of
the intrinsic working.
This device may, for the sake of the further discussion, be presented to us as an alleged  ``hypercomputer.''

The following  notation is introduced. Let $B$ be a subset of $X\colon\ X_1 \times \ldots
\times X_m$. The $i$-th projection of $B$ (for $i=1,\ldots , m$),
written as $B_i$, is defined by:
%
\[
\begin{array}{lcl}
B_i &=& \{x \mid x \in X_i, \ (\exists y \in X_1 \times \ldots
\times X_{i-1})(\exists z \in X_{i+1} \times \cdots \times X_m)\\
& & (y,x,z) \in X\}.
\end{array}
\]
For any $x \in N^m$ we define
$$|x| = \max\{x_i \mid i \in \{1,\ldots,m\}\}.$$

Then, a hypercomputer can be defined via its input/output behavior of black boxes as follows.

\begin{definition}[black box]\label{def.blackb}
Let $X,Y$ be sets and ${\Bbb N}$ be the set of natural numbers. A subset
$B$ of $X \times Y \times {\Bbb N}$ is called a {\em black box} if $B_1
= X$. $X$ is called the {\em input set} and $Y$ the {\em output
set.}
\end{definition}
Note that the condition $B_1 = X$ models a computing device
which is total, i.e. there exists always an output.
%
\begin{definition}
Let $B$ be a black box. We define
%
\begin{eqnarray*}
f_B &=& \{(x,y) \mid (\exists z)(x,y,z) \in B\},\\[1ex]
%
t_B &=& \{(x,z) \mid (\exists y)(x,y,z) \in B\}.
\end{eqnarray*}
%
$f_B$ is called the {\em input-output relation} of $B$ and $t_B$
the {\em computing time} of $B$. If $f_B$ and $t_B$ are functions
then $B$ is called deterministic.
\end{definition}
Every halting deterministic Turing machine defines a
black box. Indeed, let $M$ be a Turing machine (computing a total
function), $f_M\colon X \to Y$ be the function computed by $M$ and
$t_M$ be the computing time of $M$. Then
$$\{(x,f_M(x),t_M(x)) \mid x \in X\}$$
is a (deterministic) black box. Similarly all halting
non-deterministic Turing machines define black boxes.


\begin{definition}[hypercomputer]\label{def.hyperc}
A {\em strong hypercomputer} is a black box $B$ where $f_B$ is not
Turing-computable.
\end{definition}

\begin{definition}
Let ${\cal C}$ be a class of computable monotone functions ${\Bbb N} \to
{\Bbb N}$ containing the polynomials (over ${\Bbb N}$ with non-negative
coefficients). Then ${\cal C}$ is called a {\em bound class.}
\end{definition}



\begin{definition}
A {\em weak hypercomputer} is a black box $B$ with the
following property:
%
There exists a bound class ${\cal C}$ s.t.
\begin{itemize}
\item $t_M(x) > g(|x|)$ a.e. for all $g \in {\cal C}$ and for
    all Turing machines $M$ with $f_M =f_B$.
\item There exists an $h \in {\cal C}$ s.t. $t_B(x) \leq h(|x|)$ for
        all $x \in B_1$.
\end{itemize}
\end{definition}

A strong hypercomputer computes either a
non-computable function or decides an undecidable problem. A weak
hypercomputation outperforms all Turing machines. A possible
scenario for a weak hypercomputer $B$ is the following:
%
$f_B$ is an ${\bf EXPTIME}$-complete problem, therefore there exists no
polynomial $p$ and no Turing machine $M$ computing $f_B$ with
$t_M(x) \leq p(|x|)$ for all $x \in X$ , but $t_B(x) \leq p(|x|)$
for all $x \in X$ and for a polynomial $p$.

For non-deterministic hypercomputers we may distinguish between the following cases:
\begin{itemize}
\item $f_B$ is not a function,
\item $f_B$ is a function, but $t_B$ is not.
\end{itemize}

For stochastic hypercomputers, either $t_B$ or both $f_B$ and
$t_B$ are random variables, and the requirements on the computation have to be
specified.




\section{Tests}

Having set the stage for a general investigation into hypercomputers which are presented to us as
black boxes, we shall consider a few cases and tests.
These test methods will be essentially heuristic and present no way of systematically addressing
the issue of falsifying or even verifying hypercomputation.



\subsection{NP-complete cases}

It may be conjectured that, by operational means,  it is not possible to go beyond
tests of hyper-NP-completeness.
Even for an NP-complete problem as for instance SAT, it is not trivial to
verify that a hypercomputer solves the problem in polynomial time.
Without insight into the internal structure of the hypercomputer
we cannot obtain a proof of polynomial time computation,
which is an asymptotic result. Even here we rely on
experiments to test a ``large'' number of
problems. A central problem consists in the right selection of
problem sequences. If the selection is based on random generators
we merely obtain results on average complexity, which would not be
significant.

Furthermore, we need at least some information about the polynomial
in question (e.g., its maximum degree). Otherwise it remains
impossible to decide by finite means whether some behavior
is polynomially or not.


\subsection{Harder cases with tractable verification}

Do there exist (decision) problems which are harder
than the known NP-complete cases,
possibly having no recursively enumerable solution and proof methods,
whose results nevertheless are tractable verifiable?
For example, the problem of \emph{graph non-isomorphism} (GNI) is one that is not
known to be in NP, not even in NP $\cup$ BPP. Nevertheless, it is
possible to ``efficiently verify'' whether a ``prover''
solves this problem correctly.

If the prover claims that two graphs $G_1$ and $G_2$ are isomorphic,
he can convince us by providing a graph isomorphism. That can be checked
in polynomial time, which also means that GNI $\in$ coNP.
If, on the other hand, the prover claims that $G_1$ and $G_2$ are
non-isomorphic, we can verify this by the following \emph{interactive proof}:

\begin{enumerate}
\item Choose one of the graphs $G_1$ and $G_2$ with equal probability. \label{I:1}
\item Apply an arbitrary permutation to its vertices; this yields graph $H$.
\item The prover must decide whether $H$ is equivalent to $G_1$ or $G_2$. \label{I:3}
\item Repeat for $N$ rounds.
\end{enumerate}

If the initial answer was wrong and the graphs $G_1$ and $G_2$ are actually isomorphic,
the prover can in step \ref{I:3} only \emph{guess} which graph
was chosen in step \ref{I:1} (since now $H$ could have been derived from
\emph{either}). Hence, after $N$ rounds we can be sure with probability
$1-2^{-N}$ that the graphs $G_1$ and $G_2$ are non-isomorphic.

By denoting the class of interactive proofs by IP, we have shown that
GNI $\in$ IP. Interactive proofs further exist for \emph{every} language in
PSPACE (which is assumed to be \emph{much} larger than NP). In fact, it can be shown
\cite{sha:92} that IP equals PSPACE. This means, in particular, that
IP is closed under complement.

The protocol in the example above has the property that in each round a constant number
of messages is sent. In a generic interactive proof system for
PSPACE this is \emph{not} necessarily true; but at any instance the number of messages depends
polynomially on the input length.

In the literature, specific classes of interactive proof systems are investigated
as well, e.g.\ the \emph{Arthur-Merlin class} \cite{bab:85} and
the \emph{GMR class} \cite{GMR:85}. The former uses public coin tosses, with the
intention to accomodate certain languages in as low complexity classes as possible.
The latter uses private coin tosses, with the intention to cover the widest
possible class of efficiently verifiable languages;
additionally, it has the feature of providing \emph{zero-knowledge proofs,}
which is of great significance in cryptography. (The protocol presented above does
not have the zero-knowledge property -- unless GNI $\in$ BPP --, but can be modified
to have.) For further information on interactive proof systems see \cite{BM:88,gold:01}.

\subsection{Interference of problems}

One may confront the hypercomputer with the problem
of comparing the solutions of multiple tasks.
Such a comparison needs not necessarily involve the separate computation of the solutions of these multiple tasks.

As an analogy, consider Deutsch's problem as one of the first problems which
quantum computers could solve effectively.
Consider a function that takes a single (classical) bit into a single (classical) bit.
There are four such functions $f_1,\ldots ,f_4$, corresponding to all variations.
One can specify or ``prepare'' a function bitwise, or alternatively,
one may specify it by requiring that, for instance, such a function
acquires different values on different inputs, such as $f(0)\neq f(1)$.
Thereby, we may, even in principle, learn nothing about the individual functional values alone.
% cf http://people.ccmr.cornell.edu/~mermin/qcomp/chap2.pdf



\subsection{Generation of random sequences}
By implementation of Chaitin's ``algorithm'' to compute
Chaitin's $\Omega$  \cite{chaitin:01}
or variants thereof \cite{calude:94},
it would in principle be possible to ``compute'' the first bits of random sequences.
Such random sequences could in principle be subject to the usual
tests of stochasticity \cite{svozil-qct,calude-dinneen05}.

Note that in quantum mechanics, the randomness of certain microphysical events,
such as the spontaneous decay of excited quantum states
\cite{erber-95,berkeland:052103},
or the quantum coin toss experiments in complete context mismatches
\cite{svozil-qct} is postulated as an axiom.
This postulate is then used as the basis for the production of quantum randomness oracles
such as the commercially available {\it Quantis}\textsuperscript{\texttrademark} interface \cite{Quantis}.


\section{Impossibility of unsolvable problems whose ``solution'' is polynomially verifiable}
%
Let $\Sigma_0$ be a finite (non-empty) alphabet and $X \subset \Sigma_0^*$
be a semi-decidable, but not decidable set. That means there exists a
Turing machine which accepts
the language $X$, but does not terminate on all $x \in \Sigma_0^*$.
The concept
of {\em acceptable by Turing machines} is equivalent to {\em derivable by
inference systems} or {\em producible by grammars}. We choose the approach of
a {\em universal proof system}, i.e. of a system which simulates every
Turing machine.

Let $P$ be such a proof system. Let $V$ be an infinite set of variables
(over strings in $\Sigma^*$). A {\em meta-string} is an object
$x_1 \ldots x_n$ where
$x_i \in \Sigma$ or $x_i \in V$. If $X$ is a meta-string and $\theta$ is a
substitution (i.e. a mapping $V \to (V \cup \Sigma)^*$)
then $X\theta$ is called an
instance of $X$. If $X\theta \in \Sigma^*$ we call $X\theta$ a {\em ground
instance} of $X$.\\[1ex]
%
We may define $P = ({\cal Y},{\cal X},\Sigma,\Sigma_0)$ where
${\cal Y}$ is a finite sets of meta-strings (the axioms) and
${\cal X}$ is a finite set of
{\em rules}, i.e. expressions of the form:
\[
{X_1 \ldots X_n\over X}
\]
%
where $X_1,\ldots,X_n,X$ are meta-strings s.t. the set of variables in $X$ is
contained in the set of variables in $X_1,\ldots,X_n$.

$\Sigma_0$ is a (non-empty) subset of $\Sigma$ (defining the strings of the
theory to be generated).
\\[1ex]
%
A {\em derivation $\varphi$ in $P$} is a tree s.t. all nodes are labelled
by strings in $\Sigma^*$. In particular:
%
\begin{itemize}
\item the leaves of $\varphi$ are labelled by ground instances of axioms.
\item Let $N$ be a node in $\varphi$ which is not a leaf and
    $(N,N_1),\ldots,(N,N_k)$ be the nodes from $N$ then
\[
{N_1 \ldots N_k\over N}
\]
%
is a ground instance of a rule in ${\cal X}$.
\end{itemize}

A proof of an $x$ in $\Sigma_0^*$ in $P$ is a derivation in $P$
with the root node labelled
by $x$. $x$ is called {\em provable} in $P$ if there exists a proof of
$x$ in $P$.\\[1ex]
%
{\bf fact:} as $\Sigma$ is finite there are only finitely many derivations
of length $\leq k$ for any natural number $k$, where {\em length} is the
number of symbol occurrences. Let $P[k]$ be the set of all derivations of
length $\leq k$.\\[1ex]
%
We prove now:\\
There is no recursive function $g$ s.t. for all $x \in X$:
\begin{itemize}
\item[$(*)$] $x$ is provable in $P$ iff there exists a proof $\varphi$ of $x$
    with $|\varphi| \leq g(|x|)$.
\end{itemize}
%
{\bf proof:}\\
assume that there exists a recursive $g$ s.t. $(*)$ holds. We construct a
decision procedure of $X$:
%
\[
\begin{array}{l}
\mbox{input:}\ x \in X.\\
\bullet\ \mbox{compute}\ g(|x|).\\
\bullet\ \mbox{construct}\ P[g(|x|)].\\
\bullet\ \mbox{if } P[g(|x|)] \mbox{ contains a proof of }x \mbox{ then }
    x \in X\\
\hspace*{1cm} \mbox{ else } x \not \in X.
\end{array}
\]
%
But we assumed $X$ to be undecidable, thus we arrive at a complete contradiction. {Q.E.D.}

It follows as a corollary that there exists no proof system which generates an undecidable
problem $X$ and $X$ is polynomially verifiable.

The result above illustrates one of the problems in acknowledging hypercomputation. Even if we have a strong hypercomputer solving, let us say, the halting problem, the verification of its correctness is ultimately unfeasible. Due to the absence of recursive bounds we cannot expect to obtain a full proof of the corresponding property (halting or non-halting) from the hypercomputer itself.

When we consider the halting problem and the property of non-halting, this can only be verified by a proof (and not by simulating a Turing machine). By the undecidability of the problem there is no complete (recursive) proof system doing the job. So when we obtain a verification from the hypercomputer concerning non-halting, the form of this verification lies outside computational proof systems.

However we might think about the follwing test procedure for hypercomputers: humans create a test set of problems for an undecidable problem $X$, i.e. a finite set $Y$ with $Y \cap X \neq \emptyset$ and $Y \cap X^c \neq \emptyset$. The humans are in possession of the solutions, preferably of proofs $\varphi_y$ of $y \in X$ or of $y \not \in X$ for any $y \in Y$. This finite set may at least serve the purpose of {\em falsifying} hypercomputation (provided the hypercomputer is not stochastic and wrong answers are admitted). Beyond the possibility of falsification we might consider the follwoing scenario: the hypercomputer answers all questions concerning the test set $Y$ correctly, and its computing time is independent of the complexity of the proofs $\varphi_y$. Such a phenomenon-would, of course, not yield a verification of the hypercomputer but at least indicate a behavior structurally differing from computable proof systems.

But the ultimate barrier of verifying a hypercomputer is that of verifying a black box, characterized by the attempt to induce a property of infinitely many input-output pairs by a finite test set.

\section{Discussion and summary}

The considerations presented here may be viewed as special cases of a very general black box identification problem:
is it possible to deduce certain features of a black box, without screwing the box open and without knowing the intrinsic
working of the black box, from its input/output behavior alone?
Several issues of this general problem have already been discussed.
For instance, in an effort to formalize the uncertainty principle,
Moore \cite{e-f-moore} considered initial state identification problems of (given) deterministic finite automata.
Gold  \cite{go-67,blum75blum,angluin:83,ad-91,li:92} considered a question related to induction:
if one restricts black boxes to computable functions, then the rule inference problem,
i.e., the problem to find out which function
is implemented by the black box, is in general unsolvable.
The halting problem \cite{turing-36,rogers1,odi:89}
can be translated into a black box problem: given a black box
with known partial recursive  function,
then its future behavior is generally unpredictable.
Even the problem to determine whether or not a black box system is polynomial in computation space and
time appears to be far from being trivial.

So, if presented with a hypercomputer or oracle, we could only assert heuristic information, nothing more.
We have to accept the fact that more general assertions, or even proofs for computational capacities
beyond very limited finite computational capacities remain impossible, and will possibly remain so forever.

The situation is not dissimilar from claims of absolute indeterminism and randomness on a microphysical scale
\cite{svozil-qct}, where a few, albeit subtle tests of time series \cite{calude-dinneen05} generated by
quantum randomness oracles such as {\it Quantis}\textsuperscript{\texttrademark}  \cite{Quantis} can be compared against
advanced algorithmic random number generators
such as the  Rule30CA Wolfram rule 30 generator  implemented by {\it Mathematica}\textsuperscript{\textregistered}.
Beyond heuristic testing, any general statement about quantum randomness remains conjectural.



\section{Acknowledgments}
This manuscript grew out of discussions between computer scientists and physicists at the Vienna University of Technology,
including, among others, Erman Acar, Bernhard Gramlich, Markus Moschner, and Gernot Salzer.

%\bibliography{svozil}
%\bibliographystyle{osa}



\begin{thebibliography}{10}
\newcommand{\enquote}[1]{``#1''}
\expandafter\ifx\csname url\endcsname\relax
  \def\url#1{{#1}}\fi
\expandafter\ifx\csname urlprefix\endcsname\relax\def\urlprefix{}\fi

\bibitem{wolfram-2002}
S.~Wolfram, {\em A New Kind of Science\/} (Wolfram Media, Inc., Champaign, IL,
  2002).

\bibitem{svozil-2005-cu}
K.~Svozil, \enquote{Computational universes,} Chaos, Solitons \& Fractals {\bf
  25}, 845--859 (2006).
\newline http://dx.doi.org/10.1016/j.chaos.2004.11.055

\bibitem{bridgman}
P.~W. Bridgman, \enquote{A Physicist's Second Reaction to {M}engenlehre,}
  Scripta Mathematica {\bf 2}, 101--117, 224--234 (1934), cf. R. Landauer
  \cite{landauer-95}.

\bibitem{chaitin3}
G.~J. Chaitin, {\em Algorithmic Information Theory\/} (Cambridge University
  Press, Cambridge, 1987).

\bibitem{calude-dinneen06}
C.~S. Calude and M.~J. Dinneen, \enquote{Exact Approximations of Omega
  Numbers,} International Journal of Bifurcation and Chaos {\bf 17}, 1937--1954
  (2007), {CDMTCS} report series 293.
\newline http://dx.doi.org/10.1142/S0218127407018130

\bibitem{calude-elena-dinneen06}
C.~S. Calude and E.~C. M.~J. Dinneen, \enquote{A new measure of the difficulty
  of problems,} Journal for Multiple-Valued Logic and Soft Computing {\bf 12},
  285--307 (2006), {CDMTCS} report series 277.
\newline http://www.cs.auckland.ac.nz/CDMTCS//researchreports/277cris.pdf

\bibitem{weyl:49}
H.~Weyl, {\em Philosophy of Mathematics and Natural Science\/} (Princeton
  University Press, Princeton, 1949).

\bibitem{gruenbaum:74}
A.~Gr{\"{u}}nbaum, {\em Philosophical problems of space and time (Boston
  Studies in the Philosophy of Science, vol. 12)\/} (D. Reidel,
  Dordrecht/Boston, 1974), second, enlarged edition edn.

\bibitem{thom:54}
J.~F. Thomson, \enquote{Tasks and supertasks,} Analysis {\bf 15}, 1--13 (1954).

\bibitem{benna:62}
P.~Benacerraf, \enquote{Tasks and supertasks, and the modern {E}leatics,}
  Journal of Philosophy {\bf LIX}, 765--784 (1962).

\bibitem{rucker}
R.~Rucker, {\em Infinity and the Mind\/} (Birkh{\"{a}}user, Boston, 1982),
  reprinted by Bantam Books, 1986.

\bibitem{pit:90}
I.~Pitowsky, \enquote{The physical {C}hurch-{T}uring thesis and physical
  computational complexity,} Iyyun {\bf 39}, 81--99 (1990).

\bibitem{ear-nor:93}
J.~Earman and J.~D. Norton, \enquote{Forever is a day: supertasks in {P}itowsky
  and {M}alament-{H}ogart spacetimes,} Philosophy of Science {\bf 60}, 22--42
  (1993).

\bibitem{hogarth1}
M.~Hogarth, \enquote{Predicting the future in relativistic spacetimes,} Studies
  in History and Philosophy of Science. Studies in History and Philosophy of
  Modern Physics {\bf 24}, 721--739 (1993).

\bibitem{hogarth2}
M.~Hogarth, \enquote{Non-{T}uring computers and non-{T}uring computability,}
  PSA {\bf 1}, 126--138 (1994).

\bibitem{beth-59}
E.~W. Beth, {\em The Foundations of Metamathematics\/} (North-Holland,
  Amsterdam, 1959).

\bibitem{le-91}
E.~G.~K. L{\'{o}}pez-Escobar, \enquote{{Z}eno's Paradoxes: Pre {G}{\"{o}}delian
  Incompleteness,} Yearbook 1991 of the Kurt-G{\"{o}}del-Society {\bf 4},
  49--63 (1991).

\bibitem{sv-aut-rev}
K.~Svozil, \enquote{The {C}hurch-{T}uring Thesis as a Guiding Principle for
  Physics,} in {\em Unconventional Models of Computation\/}, C.~S. Calude,
  J.~Casti, and M.~J. Dinneen, eds.,  pp. 371--385 (1998).

\bibitem{2002-cal-pav}
C.~S. Calude and B.~Pavlov, \enquote{Coins, Quantum Measurements, and
  {T}uring's Barrier,} Quantum Information Processing {\bf 1}, 107--127 (2002).
\newline http://arxiv.org/abs/quant-ph/0112087

\bibitem{ad-ca-pa}
V.~A. Adamyan and B.~S.~P. Cristian S.~Calude, \enquote{Transcending the Limits
  of {T}uring Computability,}  (1998).
\newline http://arxiv.org/abs/quant-ph/0304128

\bibitem{kieu-02}
T.~D. Kieu, \enquote{Quantum Algorithm for {H}ilbert's Tenth Problem,}
  International Journal of Theoretical Physics {\bf 42}, 1461--1478 (2003).
\newline http://arxiv.org/abs/quant-ph/0110136

\bibitem{kieu-02a}
T.~D. Kieu, \enquote{Computing the Noncomputable,} Contemporary Physics {\bf
  44}, 51--71 (2003).
\newline http://arxiv.org/abs/quant-ph/0203034

\bibitem{ord-2006}
T.~Ord, \enquote{The many forms of hypercomputation,} Applied Mathematics and
  Computation {\bf 178}, 143--153 (2006).
\newline http://dx.doi.org/10.1016/j.amc.2005.09.076

\bibitem{Davis-2004}
M.~Davis, \enquote{The myth of hypercomputation,} in {\em Alan Turing: Life and
  Legacy of a Great Thinker\/}, C.~Teuscher, ed.  (Springer, Berlin, 2004), pp.
  195--212.

\bibitem{Doria-2006}
F.~A. Doria and J.~F. Costa, \enquote{Introduction to the special issue on
  hypercomputation,} Applied Mathematics and Computation {\bf 178}, 1--3
  (2006).
\newline http://dx.doi.org/10.1016/j.amc.2005.09.065

\bibitem{Davis-2006}
M.~Davis, \enquote{Why there is no such discipline as hypercomputation,}
  Applied Mathematics and Computation {\bf 178}, 4--7 (2006).
\newline http://dx.doi.org/10.1016/j.amc.2005.09.066

\bibitem{ein-in}
A.~Einstein and L.~Infeld, {\em The evolution of physics\/} (Cambridge
  University Press, Cambridge, 1938).

\bibitem{Chow-2004}
T.~Y. Chow, \enquote{The Myth of Hypercomputation,}  (2004), contribution to a
  discussion group on hypercomputation.
\newline http://cs.nyu.edu/pipermail/fom/2004-February/007883.html

\bibitem{davis-58}
M.~Davis, {\em Computability and Unsolvability\/} (McGraw-Hill, New York,
  1958).

\bibitem{sha:92}
A.~Shamir, \enquote{IP = PSPACE,} J. ACM {\bf 39}, 869--877 (1992).
\newline http://dx.doi.org/10.1145/146585.146609

\bibitem{bab:85}
L.~Babai, \enquote{Trading group theory for randomness,} in {\em STOC '85:
  Proceedings of the seventeenth annual ACM symposium on theory of computing\/}
   pp. 421--429 (1985).
\newline http://dx.doi.org/10.1145/22145.22192

\bibitem{GMR:85}
S.~Goldwasser, S.~Micali, and C.~Rackoff, \enquote{The knowledge complexity of
  interactive proof systems,} SIAM J. Comput. {\bf 18}, 186--208 (1989).
\newline http://dx.doi.org/10.1137/0218012

\bibitem{BM:88}
L.~Babai and S.~Moran, \enquote{Arthur--Merlin games: A randomized proof
  system, and a hierarchy of complexity classes,} Journal of Comp. and Syst.
  Sci. {\bf 36}, 254--276 (1988).

\bibitem{gold:01}
O.~Goldreich, {\em Foundations of Cryptography: Basic Tools\/} (Cambridge
  University Press, Cambridge, 2001).

\bibitem{chaitin:01}
G.~J. Chaitin, {\em Exploring Randomness\/} (Springer Verlag, London, 2001).

\bibitem{calude:94}
C.~Calude, {\em Information and Randomness---An Algorithmic Perspective\/}
  (Springer, Berlin, 1994).

\bibitem{svozil-qct}
K.~Svozil, \enquote{The quantum coin toss---Testing microphysical
  undecidability,} Physics Letters A {\bf 143}, 433--437 (1990).
\newline http://dx.doi.org/10.1016/0375-9601(90)90408-G

\bibitem{calude-dinneen05}
C.~S. Calude and M.~J. Dinneen, \enquote{Is quantum randomness algorithmic
  random? A preliminary attack,} in {\em Proceedings 1st International
  Conference on Algebraic Informatics\/}, S.~Bozapalidis, A.~Kalampakas, and
  G.~Rahonis, eds.,  pp. 195--196 (2005).

\bibitem{erber-95}
T.~Erber, \enquote{Testing the Randomness of Quantum Mechanics: Nature's
  Ultimate Cryptogram?} in {\em Annals of the New York Academy of Sciences.
  {V}olume 755 Fundamental Problems in Quantum Theory\/}, D.~M. Greenberger and
  A.~Zeilinger, eds.  (Springer, Berlin, Heidelberg, New York, 1995), Vol. 755,
  pp. 748--756.
\newline http://dx.doi.org/10.1111/j.1749-6632.1995.tb39016.x

\bibitem{berkeland:052103}
D.~J. Berkeland, D.~A. Raymondson, and V.~M. Tassin, \enquote{Tests for
  nonrandomness in quantum jumps,} Physical Review A (Atomic, Molecular, and
  Optical Physics) {\bf 69}, 052\,103 (2004).
\newline http://dx.doi.org/10.1103/PhysRevA.69.052103

\bibitem{Quantis}
id~Quantique, \enquote{Quantis - Quantum Random Number Generators,}  (2004).
\newline http://www.idquantique.com

\bibitem{e-f-moore}
E.~F. Moore, \enquote{Gedanken-Experiments on Sequential Machines,} in {\em
  Automata Studies\/}, C.~E. Shannon and J.~McCarthy, eds.  (Princeton
  University Press, Princeton, 1956).

\bibitem{go-67}
M.~E. Gold, \enquote{Language identification in the limit,} Information and
  Control {\bf 10}, 447--474 (1967).
\newline http://dx.doi.org/10.1016/S0019-9958(67)91165-5

\bibitem{blum75blum}
L.~Blum and M.~Blum, \enquote{Toward a mathematical theory of inductive
  inference,} Information and Control {\bf 28}, 125--155 (1975).

\bibitem{angluin:83}
D.~Angluin and C.~H. Smith, \enquote{A Survey of Inductive Inference: Theory
  and Methods,} Computing Surveys {\bf 15}, 237--269 (1983).

\bibitem{ad-91}
L.~M. Adleman and M.~Blum, \enquote{Inductive Inference and Unsolvability,} The
  Journal of Symbolic Logic {\bf 56}, 891--900 (1991).
\newline http://dx.doi.org/10.2307/2275058

\bibitem{li:92}
M.~Li and P.~M.~B. Vit{\'{a}}nyi, \enquote{Inductive reasoning and {K}olmogorov
  complexity,} Journal of Computer and System Science {\bf 44}, 343--384
  (1992).
\newline http://dx.doi.org/10.1016/0022-0000(92)90026-F

\bibitem{turing-36}
A.~M. Turing, \enquote{On computable numbers, with an application to the
  {E}ntscheidungsproblem,} Proceedings of the London Mathematical Society,
  Series 2 {\bf 42 and 43}, 230--265 and 544--546 (1936-7 and 1937), reprinted
  in \cite{davis}.

\bibitem{rogers1}
H.~{Rogers, Jr.}, {\em Theory of Recursive Functions and Effective
  Computability\/} (MacGraw-Hill, New York, 1967).

\bibitem{odi:89}
P.~Odifreddi, {\em Classical Recursion Theory, Vol. 1\/} (North-Holland,
  Amsterdam, 1989).

\bibitem{landauer-95}
R.~Landauer, \enquote{Advertisement For a Paper {I} Like,} in {\em On
  Limits\/}, J.~L. Casti and J.~F. Traub, eds.  (Santa Fe Institute Report
  94-10-056, Santa Fe, NM, 1994), p.~39.
\newline
  http://www.santafe.edu/research/publications/workingpapers/94-10-056.pdf

\bibitem{davis}
M.~Davis, {\em The Undecidable\/} (Raven Press, New York, 1965).

\end{thebibliography}
\end{document}
