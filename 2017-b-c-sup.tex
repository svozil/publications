\newif\ifsup
%\suptrue
\ifsup

\documentclass[%
 %reprint,
  twocolumn,
 %superscriptaddress,
 %groupedaddress,
 %unsortedaddress,
 %runinaddress,
 %frontmatterverbose,
 % preprint,
 showpacs,
 showkeys,
 preprintnumbers,
 %nofootinbib,
 %nobibnotes,
 %bibnotes,
 amsmath,amssymb,
 aps,
 % prl,
  pra,
 % prb,
 % rmp,
 %prstab,
 %prstper,
  longbibliography,
 floatfix,
 %lengthcheck,%
 ]{revtex4-1}

%\usepackage{cdmtcs-pdf}

\usepackage{mathptmx}% http://ctan.org/pkg/mathptmx

\usepackage{amssymb,amsthm,amsmath}


\usepackage{listings}
%\lstset{numbers=left, numberstyle=\tiny, stepnumber=1, numbersep=5pt}
%\lstset{emph={V-representation,begin,end}, emphstyle=\color{red}, emph={[2]root,base},emphstyle={[2]\color{blue}}}
\lstset{keywords={V,H,\*,representation,begin,end}}
%\lstset{emph={V-representation,begin,end}, emphstyle=\color{red}}
\lstset{% general command to set parameter(s)
basicstyle=\small, % print whole listing small
keywordstyle=\color{blue}\bfseries,   %\underbar
% underlined bold black keywords
identifierstyle=, % nothing happens
commentstyle=\color{white}, % white comments
stringstyle=\ttfamily, % typewriter type for strings
showstringspaces=false} % no special string space


%\usepackage{tikz}
\usepackage{graphicx}% Include figure files

%\usepackage{url}

\usepackage{xcolor}
\usepackage{eepic}

\usepackage{hyperref}
\hypersetup{
    colorlinks,
    linkcolor={blue!80!black},
    citecolor={red!75!black},
    urlcolor={blue!80!black}
}


\begin{document}


\title{Supplemental Material: Classical versus quantum probabilities \& correlations}


\author{Karl Svozil}
\email{svozil@tuwien.ac.at}
\homepage{http://tph.tuwien.ac.at/~svozil}

\affiliation{Institute for Theoretical Physics,
Vienna  University of Technology,
Wiedner Hauptstrasse 8-10/136,
1040 Vienna,  Austria}

\affiliation{Department of Computer Science, University of Auckland,
Private Bag 92019, Auckland, New Zealand}

\date{\today}


\maketitle

%\widetext
\subsection{The cddlib package}
\fi

% 2017-b-c-sup.tex

%\section{Supplemental Material: cddlib codes of examples}
%\label{2017-b-ccoe}


Fukuda's {\em cddlib package cddlib-094h} can be obtained from the package homepage~\cite{cdd-pck}. Installation on Unix-type operating systems is with {\em gcc};
the free library for arbitrary precision arithmetic {\em GMP} (currently 6.1.2)~\cite{gmplib}, must be installed first.

In its elementary form of the  {\em V-representation},  {\em cddlib}
takes in the $k$ vertices $\vert {\bf v}_1 \rangle , \ldots , \vert {\bf v}_k \rangle$ of a convex polytope in an $m$-dimensional
vector space as follows (note that all rows of vector components start with ``$1$''):

%\begin{verbatim}
%\begin{lstlisting}[caption={Vertex representation of cddlib~\cite{cdd-pck}.},label=2017-b-c-supvcddlib]
\begin{lstlisting}[backgroundcolor=\color{yellow!10},framerule=0pt,breaklines=true, frame=tb]
V-representation
begin
k  m+1  numbertype
1 v_11 ... v_1m
...............
1 v_k1 ... v_km
end
\end{lstlisting}

%\end{verbatim}

{\em cddlib} responds with the faces (boundaries of halfspaces), as encoded by  $n$  inequalities $ \textsf{\textbf{A}} \vert {\bf x} \rangle   \le \vert {\bf b} \rangle $
in the  {\em H-representation} as follows:

{\begin{lstlisting}[backgroundcolor=\color{yellow!10},framerule=0pt,breaklines=true, frame=tb]
H-representation
begin
n     m+1  numbertype
b    -A
end
\end{lstlisting}  }

Comments appear after an asterisk.


\subsection{Trivial examples}
\label{2017-b-teap}


\subsubsection{One observable}
\label{2017-b-ooa}


The case of a single variable has two extreme cases: false$\equiv 0$ and true$\equiv 1$,
resulting in $0\le p_1 \le 1$:

{ \begin{lstlisting}[backgroundcolor=\color{yellow!10},framerule=0pt,breaklines=true, frame=tb]

* one variable
*
V-representation
begin
2   2   integer
1   0
1   1
end

~~~~~~ cddlib response

H-representation
begin
 2 2 real
  1 -1
  0  1
end

\end{lstlisting}  }

\subsubsection{Two observables}
\label{2017-b-toa}

The case of two variables $p_1$ and $p_2$, and a joint variable $p_{12}$,
result in
% -A <= b
\begin{eqnarray}
%\begin{aligned}
p_1 + p_2 - p_{12} &\le& 1,
\\
- p_1 + p_{12} &\le& 0,
\\
- p_2 + p_{12} &\le& 0,
\\
- p_{12} &\le& 0,
\label{2017-b-1-2-p-ca}
%\end{aligned}
\end{eqnarray}
and thus $0  \le  p_{12}  \le  p_1 , p_2$.


{ \begin{lstlisting}[backgroundcolor=\color{yellow!10},framerule=0pt,breaklines=true, frame=tb]

* two variables: p1, p2, p12=p1*p2
*
V-representation
begin
4   4   integer
1   0   0   0
1   0   1   0
1   1   0   0
1   1   1   1
end

~~~~~~ cddlib response

H-representation
begin
 4 4 real
  1 -1 -1  1
  0  1  0 -1
  0  0  1 -1
  0  0  0  1
end

\end{lstlisting}  }

For dichotomic expectation values $\pm 1$,
{ \begin{lstlisting}[backgroundcolor=\color{yellow!10},framerule=0pt,breaklines=true, frame=tb]

* two expectation values: E1, E2, E12=E1*E2
*
V-representation
begin
4   4   integer
1   -1   -1   1
1   -1   1   -1
1   1   -1   -1
1   1   1   1
end

~~~~~~ cddlib response

H-representation
begin
 4 4 real
  1 -1 -1  1
  1  1 -1 -1
  1 -1  1 -1
  1  1  1  1
end

\end{lstlisting}  }

\subsubsection{Bounds on the (joint) probabilities and expectations of three observables}
\label{2017-b-tevoa}


{ \begin{lstlisting}[backgroundcolor=\color{yellow!10},framerule=0pt,breaklines=true, frame=tb]

* four joint expectations:
* p1, p2, p3,
* p12=p1*p2, p13=p1*p3, p23=p2*p3,
* p123=p1*p2*p3
V-representation
begin
8   8    integer
1       0    0    0    0    0    0    0
1       0    0    1    0    0    0    0
1       0    1    0    0    0    0    0
1       0    1    1    0    0    1    0
1       1    0    0    0    0    0    0
1       1    0    1    0    1    0    0
1       1    1    0    1    0    0    0
1       1    1    1    1    1    1    1
end

~~~~~~ cddlib response

H-representation
begin
 8 8 real
  1 -1 -1 -1  1  1  1 -1
  0  1  0  0 -1 -1  0  1
  0  0  1  0 -1  0 -1  1
  0  0  0  1  0 -1 -1  1
  0  0  0  0  1  0  0 -1
  0  0  0  0  0  1  0 -1
  0  0  0  0  0  0  1 -1
  0  0  0  0  0  0  0  1
end

\end{lstlisting}  }

If single observable  expectations  are set to zero by assumption (axiom) and are not-enumerated,
the table of expectation values may be redundand.

The case of three expectation value observables
$E_1$, $E_2$  and $E_3$ (which are not explicitly enumerated),
as well as all joint expectations $E_{12}$, $E_{13}$, $E_{23}$, and $E_{123}$,
result in
% -A <= b
\begin{eqnarray}
%\begin{aligned}
- E_{12}- E_{13}- E_{23} &\le& 1
\\
- E_{123} &\le& 1,
\\
E_{123} &\le& 1,
\\
- E_{12}+ E_{13}+ E_{23} &\le& 1,
\\
 E_{12}- E_{13}+ E_{23} &\le& 1,
\\
 E_{12}+ E_{13}- E_{23} &\le& 1
.
\label{2017-b-1-3-e-ia}
%\end{aligned}
\end{eqnarray}

{ \begin{lstlisting}[backgroundcolor=\color{yellow!10},framerule=0pt,breaklines=true, frame=tb]

* four joint expectations:
* [E1, E2, E3, not explicitly enumerated]
* E12=E1*E2, E13=E1*E3, E23=E2*E3,
* E123=E1*E2*E3
V-representation
begin
8   5    integer
1    1    1    1    1
1    1   -1   -1   -1
1   -1    1   -1   -1
1   -1   -1    1    1
1   -1   -1    1   -1
1   -1    1   -1    1
1    1   -1   -1    1
1    1    1    1   -1
end

~~~~~~ cddlib response

H-representation
begin
 6 5 real
  1  1  1  1  0
  1  0  0  0  1
  1  0  0  0 -1
  1  1 -1 -1  0
  1 -1  1 -1  0
  1 -1 -1  1  0
end

\end{lstlisting}  }


\subsection{2 observers, 2 measurement configurations per observer}
\label{2017-b-totmcpoa}
From a quantum physical standpoint the first relevant case is that of 2 observers and 2 measurement configurations per observer.

\subsubsection{Bell-Wigner-Fine case: probabilities for 2 observers, 2 measurement configurations per observer}
\label{2017-b-bwfa}

The case of four probabilities
$p_1$, $p_2$, $p_3$  and $p_4$,
as well as four joint probabilities $p_{13}$, $p_{14}$, $p_{23}$, and $p_{24}$
result in
% -A <= b
\begin{eqnarray}
%\begin{aligned}
                                          -p_{14}                        &\le&  0\\
                                                            -p_{24}      &\le&  0\\
 +p_{1}                  +p_{4}  -p_{13}  -p_{14}  +p_{23}  -p_{24}      &\le&  1\\
         +p_{2}          +p_{4}  +p_{13}  -p_{14}  -p_{23}  -p_{24}      &\le&  1\\
         +p_{2}  +p_{3}          -p_{13}  +p_{14}  -p_{23}  -p_{24}      &\le&  1\\
 +p_{1}          +p_{3}          -p_{13}  -p_{14}  -p_{23}  +p_{24}      &\le&  1\\
                                 -p_{13}                                 &\le&  0\\
                                                   -p_{23}               &\le&  0\\
 -p_{1}                  -p_{4}  +p_{13}  +p_{14}  -p_{23}  +p_{24}      &\le&  0\\
         -p_{2}          -p_{4}  -p_{13}  +p_{14}  +p_{23}  +p_{24}      &\le&  0\\
         -p_{2}  -p_{3}          +p_{13}  -p_{14}  +p_{23}  +p_{24}      &\le&  0\\
 -p_{1}          -p_{3}          +p_{13}  +p_{14}  +p_{23}  -p_{24}      &\le&  0\\
 -p_{1}                                   +p_{14}                        &\le&  0\\
         -p_{2}                                             +p_{24}      &\le&  0\\
                 -p_{3}                            +p_{23}               &\le&  0\\
                 -p_{3}          +p_{13}                                 &\le&  0\\
 -p_{1}                          +p_{13}                                 &\le&  0\\
         -p_{2}                                    +p_{23}               &\le&  0\\
                         -p_{4}                             +p_{24}      &\le&  0\\
                         -p_{4}           +p_{14}                        &\le&  0\\
         +p_{2}          +p_{4}                             -p_{24}      &\le&  1\\
 +p_{1}                  +p_{4}           -p_{14}                        &\le&  1\\
         +p_{2}  +p_{3}                            -p_{23}               &\le&  1\\
 +p_{1}          +p_{3}          -p_{13}                                 &\le&  1
.
\label{2017-b-2-2-p-c}
%\end{aligned}
\end{eqnarray}

{ \begin{lstlisting}[backgroundcolor=\color{yellow!10},framerule=0pt,breaklines=true, frame=tb]

* eight variables: p1, p2, p3, p4,
* p13, p14, p23, p24
*
V-representation
begin
16   9   integer
1      0    0    0    0    0    0    0    0
1      0    0    0    1    0    0    0    0
1      0    0    1    0    0    0    0    0
1      0    0    1    1    0    0    0    0
1      0    1    0    0    0    0    0    0
1      0    1    0    1    0    0    0    1
1      0    1    1    0    0    0    1    0
1      0    1    1    1    0    0    1    1
1      1    0    0    0    0    0    0    0
1      1    0    0    1    0    1    0    0
1      1    0    1    0    1    0    0    0
1      1    0    1    1    1    1    0    0
1      1    1    0    0    0    0    0    0
1      1    1    0    1    0    1    0    1
1      1    1    1    0    1    0    1    0
1      1    1    1    1    1    1    1    1
end

~~~~~~ cddlib response

H-representation
begin
 24 9 real
  0  0  0  0  0  0  1  0  0
  0  0  0  0  0  0  0  0  1
  1 -1  0  0 -1  1  1 -1  1
  1  0 -1  0 -1 -1  1  1  1
  1  0 -1 -1  0  1 -1  1  1
  1 -1  0 -1  0  1  1  1 -1
  0  0  0  0  0  1  0  0  0
  0  0  0  0  0  0  0  1  0
  0  1  0  0  1 -1 -1  1 -1
  0  0  1  0  1  1 -1 -1 -1
  0  0  1  1  0 -1  1 -1 -1
  0  1  0  1  0 -1 -1 -1  1
  0  1  0  0  0  0 -1  0  0
  0  0  1  0  0  0  0  0 -1
  0  0  0  1  0  0  0 -1  0
  0  0  0  1  0 -1  0  0  0
  0  1  0  0  0 -1  0  0  0
  0  0  1  0  0  0  0 -1  0
  0  0  0  0  1  0  0  0 -1
  0  0  0  0  1  0 -1  0  0
  1  0 -1  0 -1  0  0  0  1
  1 -1  0  0 -1  0  1  0  0
  1  0 -1 -1  0  0  0  1  0
  1 -1  0 -1  0  1  0  0  0
end

\end{lstlisting}  }


\subsubsection{Clauser-Horne-Shimony-Holt case: expectation values for 2 observers, 2 measurement configurations per observer}
\label{2017-b-chshcevta}


The case of four expectation values
$E_1$, $E_2$, $E_3$  and $E_4$ (which are not explicitly enumerated),
as well as all joint expectations $E_{13}$, $E_{14}$, $E_{23}$, and $E_{24}$
result in
% -A <= b
\begin{eqnarray}
%\begin{aligned}
   + E_{13} - E_{14} - E_{23} - E_{24}       &\le&      2    \\
                              - E_{24}       &\le&      1    \\
                     - E_{23}                &\le&      1    \\
   - E_{13} + E_{14} - E_{23} - E_{24}       &\le&      2    \\
            - E_{14}                         &\le&      1    \\
   - E_{13} - E_{14} + E_{23} - E_{24}       &\le&      2    \\
   - E_{13} - E_{14} - E_{23} + E_{24}       &\le&      2    \\
   - E_{13}                                  &\le&      1    \\
   - E_{13} + E_{14} + E_{23} + E_{24}       &\le&      2    \\
                              + E_{24}       &\le&      1    \\
                     + E_{23}                &\le&      1    \\
   + E_{13} - E_{14} + E_{23} + E_{24}       &\le&      2    \\
            + E_{14}                         &\le&      1    \\
   + E_{13} + E_{14} - E_{23} + E_{24}       &\le&      2    \\
   + E_{13} + E_{14} + E_{23} - E_{24}       &\le&      2    \\
   + E_{13}                                  &\le&      1
.
\label{2017-b-2-2-e-i}
%\end{aligned}
\end{eqnarray}

{ \begin{lstlisting}[backgroundcolor=\color{yellow!10},framerule=0pt,breaklines=true, frame=tb]

* four joint expectations:
* E13, E14, E23, E24
*
V-representation
begin
16   5   integer
1   1    1    1    1
1   1   -1    1   -1
1  -1    1   -1    1
1  -1   -1   -1   -1
1   1    1   -1   -1
1   1   -1   -1    1
1  -1    1    1   -1
1  -1   -1    1    1
1  -1   -1    1    1
1  -1    1    1   -1
1   1   -1   -1    1
1   1    1   -1   -1
1  -1   -1   -1   -1
1  -1    1   -1    1
1   1   -1    1   -1
1   1    1    1    1
end

~~~~~~ cddlib response

H-representation
begin
 16 5 real
  2 -1  1  1  1
  1  0  0  0  1
  1  0  0  1  0
  2  1 -1  1  1
  1  0  1  0  0
  2  1  1 -1  1
  2  1  1  1 -1
  1  1  0  0  0
  2  1 -1 -1 -1
  1  0  0  0 -1
  1  0  0 -1  0
  2 -1  1 -1 -1
  1  0 -1  0  0
  2 -1 -1  1 -1
  2 -1 -1 -1  1
  1 -1  0  0  0
end

\end{lstlisting}  }


\subsubsection{Beyond the Clauser-Horne-Shimony-Holt case: 2 observers, 3 measurement configurations per observer}
\label{2017-b-chshc1ba}

{ \begin{lstlisting}[backgroundcolor=\color{yellow!10},framerule=0pt,breaklines=true, frame=tb]

* 6  expectations:
* E1, ... , E6
* 9 joint expectations:
* E14, E15, E16, E24, E25, E26, E34, E35, E36
* 1,2,3 on one side
* 4,5,6 on other side
*
V-representation
begin
64   16    integer
1       1    1    1    1    1    1    1    1    1    1    1    1    1    1    1
1       1    1    1    1    1   -1    1    1   -1    1    1   -1    1    1   -1
1       1    1    1    1   -1    1    1   -1    1    1   -1    1    1   -1    1
1       1    1    1    1   -1   -1    1   -1   -1    1   -1   -1    1   -1   -1
1       1    1    1   -1    1    1   -1    1    1   -1    1    1   -1    1    1
1       1    1    1   -1    1   -1   -1    1   -1   -1    1   -1   -1    1   -1
1       1    1    1   -1   -1    1   -1   -1    1   -1   -1    1   -1   -1    1
1       1    1    1   -1   -1   -1   -1   -1   -1   -1   -1   -1   -1   -1   -1
1       1    1   -1    1    1    1    1    1    1    1    1    1   -1   -1   -1
1       1    1   -1    1    1   -1    1    1   -1    1    1   -1   -1   -1    1
1       1    1   -1    1   -1    1    1   -1    1    1   -1    1   -1    1   -1
1       1    1   -1    1   -1   -1    1   -1   -1    1   -1   -1   -1    1    1
1       1    1   -1   -1    1    1   -1    1    1   -1    1    1    1   -1   -1
1       1    1   -1   -1    1   -1   -1    1   -1   -1    1   -1    1   -1    1
1       1    1   -1   -1   -1    1   -1   -1    1   -1   -1    1    1    1   -1
1       1    1   -1   -1   -1   -1   -1   -1   -1   -1   -1   -1    1    1    1
1       1   -1    1    1    1    1    1    1    1   -1   -1   -1    1    1    1
1       1   -1    1    1    1   -1    1    1   -1   -1   -1    1    1    1   -1
1       1   -1    1    1   -1    1    1   -1    1   -1    1   -1    1   -1    1
1       1   -1    1    1   -1   -1    1   -1   -1   -1    1    1    1   -1   -1
1       1   -1    1   -1    1    1   -1    1    1    1   -1   -1   -1    1    1
1       1   -1    1   -1    1   -1   -1    1   -1    1   -1    1   -1    1   -1
1       1   -1    1   -1   -1    1   -1   -1    1    1    1   -1   -1   -1    1
1       1   -1    1   -1   -1   -1   -1   -1   -1    1    1    1   -1   -1   -1
1       1   -1   -1    1    1    1    1    1    1   -1   -1   -1   -1   -1   -1
1       1   -1   -1    1    1   -1    1    1   -1   -1   -1    1   -1   -1    1
1       1   -1   -1    1   -1    1    1   -1    1   -1    1   -1   -1    1   -1
1       1   -1   -1    1   -1   -1    1   -1   -1   -1    1    1   -1    1    1
1       1   -1   -1   -1    1    1   -1    1    1    1   -1   -1    1   -1   -1
1       1   -1   -1   -1    1   -1   -1    1   -1    1   -1    1    1   -1    1
1       1   -1   -1   -1   -1    1   -1   -1    1    1    1   -1    1    1   -1
1       1   -1   -1   -1   -1   -1   -1   -1   -1    1    1    1    1    1    1
1      -1    1    1    1    1    1   -1   -1   -1    1    1    1    1    1    1
1      -1    1    1    1    1   -1   -1   -1    1    1    1   -1    1    1   -1
1      -1    1    1    1   -1    1   -1    1   -1    1   -1    1    1   -1    1
1      -1    1    1    1   -1   -1   -1    1    1    1   -1   -1    1   -1   -1
1      -1    1    1   -1    1    1    1   -1   -1   -1    1    1   -1    1    1
1      -1    1    1   -1    1   -1    1   -1    1   -1    1   -1   -1    1   -1
1      -1    1    1   -1   -1    1    1    1   -1   -1   -1    1   -1   -1    1
1      -1    1    1   -1   -1   -1    1    1    1   -1   -1   -1   -1   -1   -1
1      -1    1   -1    1    1    1   -1   -1   -1    1    1    1   -1   -1   -1
1      -1    1   -1    1    1   -1   -1   -1    1    1    1   -1   -1   -1    1
1      -1    1   -1    1   -1    1   -1    1   -1    1   -1    1   -1    1   -1
1      -1    1   -1    1   -1   -1   -1    1    1    1   -1   -1   -1    1    1
1      -1    1   -1   -1    1    1    1   -1   -1   -1    1    1    1   -1   -1
1      -1    1   -1   -1    1   -1    1   -1    1   -1    1   -1    1   -1    1
1      -1    1   -1   -1   -1    1    1    1   -1   -1   -1    1    1    1   -1
1      -1    1   -1   -1   -1   -1    1    1    1   -1   -1   -1    1    1    1
1      -1   -1    1    1    1    1   -1   -1   -1   -1   -1   -1    1    1    1
1      -1   -1    1    1    1   -1   -1   -1    1   -1   -1    1    1    1   -1
1      -1   -1    1    1   -1    1   -1    1   -1   -1    1   -1    1   -1    1
1      -1   -1    1    1   -1   -1   -1    1    1   -1    1    1    1   -1   -1
1      -1   -1    1   -1    1    1    1   -1   -1    1   -1   -1   -1    1    1
1      -1   -1    1   -1    1   -1    1   -1    1    1   -1    1   -1    1   -1
1      -1   -1    1   -1   -1    1    1    1   -1    1    1   -1   -1   -1    1
1      -1   -1    1   -1   -1   -1    1    1    1    1    1    1   -1   -1   -1
1      -1   -1   -1    1    1    1   -1   -1   -1   -1   -1   -1   -1   -1   -1
1      -1   -1   -1    1    1   -1   -1   -1    1   -1   -1    1   -1   -1    1
1      -1   -1   -1    1   -1    1   -1    1   -1   -1    1   -1   -1    1   -1
1      -1   -1   -1    1   -1   -1   -1    1    1   -1    1    1   -1    1    1
1      -1   -1   -1   -1    1    1    1   -1   -1    1   -1   -1    1   -1   -1
1      -1   -1   -1   -1    1   -1    1   -1    1    1   -1    1    1   -1    1
1      -1   -1   -1   -1   -1    1    1    1   -1    1    1   -1    1    1   -1
1      -1   -1   -1   -1   -1   -1    1    1    1    1    1    1    1    1    1
end

~~~~~~ cddlib response

H-representation
begin
 684 16 real
  4  0 -1  1 -1 -1  0  1 -1  0  1  1  1 -1 -1  1
  [...]
  4  1  1  0  1  1  0  1  1  1  1  1 -1  1 -1  0
  [...]
end

\end{lstlisting}  }


\subsection{Pentagon logic}

\subsection{Probabilities but no joint probabilities}

Here is a computation which includes all probabilities but no joint probabilities:



{ \begin{lstlisting}[backgroundcolor=\color{yellow!10},framerule=0pt,breaklines=true, frame=tb]

* ten probabilities:
* p1 ... p10
*
begin
11  11  integer
1       1    0    0    1    0    1    0    1    0    0
1       1    0    0    0    1    0    0    1    0    0
1       1    0    0    1    0    0    1    0    0    0
1       0    0    1    0    0    1    0    1    0    1
1       0    0    1    0    0    0    1    0    0    1
1       0    0    1    0    0    1    0    0    1    0
1       0    1    0    0    1    0    0    1    0    1
1       0    1    0    0    1    0    0    0    1    0
1       0    1    0    1    0    0    1    0    0    1
1       0    1    0    1    0    1    0    0    1    0
1       0    1    0    1    0    1    0    1    0    1
end

~~~~~~ cddlib response

H-representation
linearity 5  12 13 14 15 16
begin
 16 11 real
  0  0  0  0  0  0  1  0  0  0  0
  0  0  0  0  0  0  0  0  1  0  0
  0 -1  0  0  1  0  0  0  1  0  0
  0  0  0  0  1  0  0  0  0  0  0
  0  1  0  0  0  0  0  0  0  0  0
  1 -1 -1  0  1  0 -1  0  0  0  0
  0  0  1  0  0  0  0  0  0  0  0
  1 -2 -1  0  1  0 -1  0  1  0  0
  0  1  1  0 -1  0  0  0  0  0  0
  0  1  1  0 -1  0  1  0 -1  0  0
  1 -1 -1  0  0  0  0  0  0  0  0
 -1  1  1  1  0  0  0  0  0  0  0
  0 -1 -1  0  1  1  0  0  0  0  0
 -1  1  1  0 -1  0  1  1  0  0  0
  0 -1 -1  0  1  0 -1  0  1  1  0
 -1  2  1  0 -1  0  1  0 -1  0  1
end

\end{lstlisting}  }


\begin{eqnarray}
%\begin{aligned}
                                    + p_6                                    \ge           0       \\
                                                  + p_8                      \ge           0       \\
  - p_1               + p_4                       + p_8                      \ge           0       \\
                      + p_4                                                  \ge           0       \\
  + p_1                                                                      \ge           0       \\
  - p_1 - p_2         + p_4         - p_6                                    \ge           -1      \\
        + p_2                                                                \ge           0       \\
  -2p_1 - p_2         + p_4         - p_6         + p_8                      \ge           -1      \\
  + p_1 + p_2         - p_4                                                  \ge           0       \\
  + p_1 + p_2         - p_4         + p_6         - p_8                      \ge           0       \\
  - p_1 - p_2                                                                \ge           -1      \\
  + p_1 + p_2  + p_3                                                         \ge           1       \\
  - p_1 - p_2         + p_4  + p_5                                           \ge           0       \\
  + p_1 + p_2         - p_4         + p_6  + p_7                             \ge           1       \\
  - p_1 - p_2         + p_4         - p_6         + p_8  + p_9               \ge           0       \\
   2p_1 + p_2         - p_4         + p_6         - p_8         + p_{10}     \ge           1
.
\label{2017-b-kl-p-c}
%\end{aligned}
\end{eqnarray}


\subsection{Joint Expectations on all atoms}

This is a full hull computation taking all joint expectations into account:


{\footnotesize \begin{lstlisting}[backgroundcolor=\color{yellow!10},framerule=0pt,breaklines=true, frame=tb]

* 45 pair expectations:
* E12 ... E910
*
V-representation
begin
11  46  real
1 -1 -1 1 -1 1 -1 1 -1 -1 1 -1 1 -1 1 -1 1 1 -1 1 -1 1 -1 1 1 -1 1 -1 1 -1 -1 -1 1 -1 1 1 -1 1 -1 -1 -1 1 1 -1 -1 1
1 -1 -1 -1 1 -1 -1 1 -1 -1 1 1 -1 1 1 -1 1 1 1 -1 1 1 -1 1 1 -1 1 1 -1 1 1 -1 -1 1 -1 -1 1 -1 1 1 -1 1 1 -1 -1 1
1 -1 -1 1 -1 -1 1 -1 -1 -1 1 -1 1 1 -1 1 1 1 -1 1 1 -1 1 1 1 -1 -1 1 -1 -1 -1 1 -1 1 1 1 -1 1 1 1 -1 -1 -1 1 1 1
1 1 -1 1 1 -1 1 -1 1 -1 -1 1 1 -1 1 -1 1 -1 -1 -1 1 -1 1 -1 1 1 -1 1 -1 1 -1 -1 1 -1 1 -1 -1 1 -1 1 -1 1 -1 -1 1 -1
1 1 -1 1 1 1 -1 1 1 -1 -1 1 1 1 -1 1 1 -1 -1 -1 -1 1 -1 -1 1 1 1 -1 1 1 -1 1 -1 1 1 -1 -1 1 1 -1 -1 -1 1 1 -1 -1
1 1 -1 1 1 -1 1 1 -1 1 -1 1 1 -1 1 1 -1 1 -1 -1 1 -1 -1 1 -1 1 -1 1 1 -1 1 -1 1 1 -1 1 -1 -1 1 -1 1 -1 1 -1 1 -1
1 -1 1 1 -1 1 1 -1 1 -1 -1 -1 1 -1 -1 1 -1 1 1 -1 1 1 -1 1 -1 -1 1 1 -1 1 -1 -1 -1 1 -1 1 1 -1 1 -1 -1 1 -1 -1 1 -1
1 -1 1 1 -1 1 1 1 -1 1 -1 -1 1 -1 -1 -1 1 -1 1 -1 1 1 1 -1 1 -1 1 1 1 -1 1 -1 -1 -1 1 -1 1 1 -1 1 1 -1 1 -1 1 -1
1 -1 1 -1 1 1 -1 1 1 -1 -1 1 -1 -1 1 -1 -1 1 -1 1 1 -1 1 1 -1 -1 -1 1 -1 -1 1 1 -1 1 1 -1 -1 1 1 -1 -1 -1 1 1 -1 -1
1 -1 1 -1 1 -1 1 1 -1 1 -1 1 -1 1 -1 -1 1 -1 -1 1 -1 1 1 -1 1 -1 1 -1 -1 1 -1 -1 1 1 -1 1 -1 -1 1 -1 1 -1 1 -1 1 -1
1 -1 1 -1 1 -1 1 -1 1 -1 -1 1 -1 1 -1 1 -1 1 -1 1 -1 1 -1 1 -1 -1 1 -1 1 -1 1 -1 1 -1 1 -1 -1 1 -1 1 -1 1 -1 -1 1 -1
end

~~~~~~ cddlib response

H-representation
linearity 35  12 13 14 15 16 17 18 19 20 21 22 23 24 25 26 27 28 29 30 31 32 33 34 35 36 37 38 39 40 41 42 43 44 45 46
begin
 46 46 real
  1 0 -1 -1 0 0 0 0 0 0 0 0 0 0 0 0 0 0 1 0 0 0 0 0 0 0 0 0 0 0 0 0 0 0 0 0 0 0 0 0 0 0 0 0 0 0
  1 1 0 0 0 0 0 -1 0 0 0 0 0 0 0 -1 0 0 0 0 0 0 0 0 0 0 0 0 0 0 0 0 0 0 0 0 0 0 0 0 0 0 0 0 0 0
  1 0 0 -1 0 0 0 -1 0 0 0 0 0 0 0 0 0 0 0 0 0 0 0 0 0 0 0 0 1 0 0 0 0 0 0 0 0 0 0 0 0 0 0 0 0 0
 -1 -1 0 1 0 0 0 0 0 0 0 0 0 1 0 0 0 0 -1 0 1 0 0 0 0 0 0 0 0 0 0 0 0 0 0 0 0 0 0 0 0 0 0 0 0 0
  0 -1 -1 0 0 0 0 0 0 0 0 0 0 -1 0 0 0 0 0 0 -1 0 0 0 0 0 0 0 0 0 0 0 0 0 0 0 0 0 0 0 0 0 0 0 0 0
  0 0 1 1 0 -1 0 1 0 0 0 0 0 0 0 0 0 0 0 0 -1 0 0 0 0 0 0 0 -1 0 0 0 0 0 0 0 0 0 0 0 0 0 0 0 0 0
  1 1 0 0 0 1 0 0 0 0 0 0 0 1 0 0 0 0 0 0 0 0 0 0 0 0 0 0 0 0 0 0 0 0 0 0 0 0 0 0 0 0 0 0 0 0
  0 0 0 0 0 -1 0 1 0 0 0 0 0 -1 0 1 0 0 0 0 0 0 0 0 0 0 0 0 0 0 0 0 0 0 0 0 0 0 0 0 0 0 0 0 0 0
  1 0 0 0 0 0 0 0 0 0 0 0 0 -1 0 1 0 0 1 0 -1 0 0 0 0 0 0 0 -1 0 0 0 0 0 0 0 0 0 0 0 0 0 0 0 0 0
  0 0 0 -1 0 1 0 0 0 0 0 0 0 0 0 0 0 0 -1 0 1 0 0 0 0 0 0 0 0 0 0 0 0 0 0 0 0 0 0 0 0 0 0 0 0 0
  0 0 1 1 0 0 0 0 0 0 0 0 0 1 0 -1 0 0 0 0 1 0 0 0 0 0 0 0 1 0 0 0 0 0 0 0 0 0 0 0 0 0 0 0 0 0
 -1 -1 0 1 1 0 0 0 0 0 0 0 0 0 0 0 0 0 0 0 0 0 0 0 0 0 0 0 0 0 0 0 0 0 0 0 0 0 0 0 0 0 0 0 0 0
  0 0 -1 -1 0 1 1 0 0 0 0 0 0 0 0 0 0 0 0 0 0 0 0 0 0 0 0 0 0 0 0 0 0 0 0 0 0 0 0 0 0 0 0 0 0 0
 -1 -1 0 1 0 -1 0 1 1 0 0 0 0 0 0 0 0 0 0 0 0 0 0 0 0 0 0 0 0 0 0 0 0 0 0 0 0 0 0 0 0 0 0 0 0 0
  1 0 -1 -1 0 1 0 -1 0 1 0 0 0 0 0 0 0 0 0 0 0 0 0 0 0 0 0 0 0 0 0 0 0 0 0 0 0 0 0 0 0 0 0 0 0 0
  1 1 1 0 0 0 0 0 0 0 1 0 0 0 0 0 0 0 0 0 0 0 0 0 0 0 0 0 0 0 0 0 0 0 0 0 0 0 0 0 0 0 0 0 0 0
 -1 -1 0 1 0 0 0 0 0 0 0 1 0 0 0 0 0 0 0 0 0 0 0 0 0 0 0 0 0 0 0 0 0 0 0 0 0 0 0 0 0 0 0 0 0 0
  0 0 0 -1 0 0 0 0 0 0 0 0 1 0 0 0 0 0 0 0 0 0 0 0 0 0 0 0 0 0 0 0 0 0 0 0 0 0 0 0 0 0 0 0 0 0
  0 0 1 1 0 0 0 0 0 0 0 0 0 1 1 0 0 0 0 0 0 0 0 0 0 0 0 0 0 0 0 0 0 0 0 0 0 0 0 0 0 0 0 0 0 0
  0 0 0 -1 0 0 0 0 0 0 0 0 0 -1 0 1 1 0 0 0 0 0 0 0 0 0 0 0 0 0 0 0 0 0 0 0 0 0 0 0 0 0 0 0 0 0
  0 1 1 1 0 0 0 0 0 0 0 0 0 1 0 -1 0 1 0 0 0 0 0 0 0 0 0 0 0 0 0 0 0 0 0 0 0 0 0 0 0 0 0 0 0 0
  1 1 0 0 0 0 0 0 0 0 0 0 0 0 0 0 0 0 1 1 0 0 0 0 0 0 0 0 0 0 0 0 0 0 0 0 0 0 0 0 0 0 0 0 0 0
 -1 0 0 0 0 0 0 0 0 0 0 0 0 0 0 0 0 0 -1 0 1 1 0 0 0 0 0 0 0 0 0 0 0 0 0 0 0 0 0 0 0 0 0 0 0 0
  1 0 -1 -1 0 0 0 0 0 0 0 0 0 -1 0 1 0 0 1 0 -1 0 1 0 0 0 0 0 0 0 0 0 0 0 0 0 0 0 0 0 0 0 0 0 0 0
  0 1 1 1 0 0 0 0 0 0 0 0 0 1 0 -1 0 0 0 0 0 0 0 1 0 0 0 0 0 0 0 0 0 0 0 0 0 0 0 0 0 0 0 0 0 0
  0 0 0 -1 0 0 0 0 0 0 0 0 0 -1 0 1 0 0 0 0 0 0 0 0 1 0 0 0 0 0 0 0 0 0 0 0 0 0 0 0 0 0 0 0 0 0
  0 -1 0 0 0 0 0 0 0 0 0 0 0 0 0 0 0 0 0 0 0 0 0 0 0 1 0 0 0 0 0 0 0 0 0 0 0 0 0 0 0 0 0 0 0 0
 -1 0 0 0 0 0 0 0 0 0 0 0 0 0 0 0 0 0 -1 0 1 0 0 0 0 0 1 0 0 0 0 0 0 0 0 0 0 0 0 0 0 0 0 0 0 0
  0 0 0 0 0 0 0 0 0 0 0 0 0 0 0 0 0 0 0 0 -1 0 0 0 0 0 0 1 0 0 0 0 0 0 0 0 0 0 0 0 0 0 0 0 0 0
 -1 -1 0 0 0 0 0 0 0 0 0 0 0 0 0 0 0 0 -1 0 1 0 0 0 0 0 0 0 1 1 0 0 0 0 0 0 0 0 0 0 0 0 0 0 0 0
  0 0 0 1 0 0 0 0 0 0 0 0 0 0 0 0 0 0 0 0 -1 0 0 0 0 0 0 0 -1 0 1 0 0 0 0 0 0 0 0 0 0 0 0 0 0 0
  1 0 0 0 0 -1 0 0 0 0 0 0 0 -1 0 0 0 0 1 0 -1 0 0 0 0 0 0 0 0 0 0 1 0 0 0 0 0 0 0 0 0 0 0 0 0 0
  0 0 0 0 0 1 0 0 0 0 0 0 0 1 0 0 0 0 0 0 1 0 0 0 0 0 0 0 0 0 0 0 1 0 0 0 0 0 0 0 0 0 0 0 0 0
  0 0 0 0 0 0 0 -1 0 0 0 0 0 0 0 -1 0 0 0 0 0 0 0 0 0 0 0 0 1 0 0 0 0 1 0 0 0 0 0 0 0 0 0 0 0 0
  0 0 0 0 0 -1 0 1 0 0 0 0 0 -1 0 1 0 0 1 0 -1 0 0 0 0 0 0 0 -1 0 0 0 0 0 1 0 0 0 0 0 0 0 0 0 0 0
  1 1 0 -1 0 1 0 -1 0 0 0 0 0 1 0 -1 0 0 0 0 1 0 0 0 0 0 0 0 1 0 0 0 0 0 0 1 0 0 0 0 0 0 0 0 0 0
  0 0 0 0 0 0 0 0 0 0 0 0 0 0 0 0 0 0 -1 0 0 0 0 0 0 0 0 0 0 0 0 0 0 0 0 0 1 0 0 0 0 0 0 0 0 0
  0 0 0 0 0 -1 0 1 0 0 0 0 0 -1 0 1 0 0 1 0 -1 0 0 0 0 0 0 0 -1 0 0 0 0 0 0 0 0 1 0 0 0 0 0 0 0 0
  0 0 0 0 0 0 0 -1 0 0 0 0 0 0 0 -1 0 0 0 0 0 0 0 0 0 0 0 0 1 0 0 0 0 0 0 0 0 0 1 0 0 0 0 0 0 0
  0 0 0 0 0 0 0 1 0 0 0 0 0 -1 0 1 0 0 0 0 -1 0 0 0 0 0 0 0 -1 0 0 0 0 0 0 0 0 0 0 1 0 0 0 0 0 0
  1 0 -1 -1 0 1 0 -1 0 0 0 0 0 0 0 0 0 0 0 0 0 0 0 0 0 0 0 0 0 0 0 0 0 0 0 0 0 0 0 0 1 0 0 0 0 0
 -1 0 1 1 0 0 0 1 0 0 0 0 0 1 0 0 0 0 -1 0 1 0 0 0 0 0 0 0 0 0 0 0 0 0 0 0 0 0 0 0 0 1 0 0 0 0
  0 0 0 0 0 0 0 -1 0 0 0 0 0 0 0 0 0 0 0 0 0 0 0 0 0 0 0 0 0 0 0 0 0 0 0 0 0 0 0 0 0 0 1 0 0 0
  1 0 0 0 0 -1 0 0 0 0 0 0 0 -1 0 0 0 0 1 0 -1 0 0 0 0 0 0 0 0 0 0 0 0 0 0 0 0 0 0 0 0 0 0 1 0 0
  0 0 -1 -1 0 1 0 0 0 0 0 0 0 0 0 0 0 0 0 0 0 0 0 0 0 0 0 0 0 0 0 0 0 0 0 0 0 0 0 0 0 0 0 0 1 0
  1 1 1 0 0 0 0 0 0 0 0 0 0 0 0 0 0 0 0 0 0 0 0 0 0 0 0 0 0 0 0 0 0 0 0 0 0 0 0 0 0 0 0 0 0 1
end

\end{lstlisting}  }


\begin{eqnarray}
%\begin{aligned}
 E_{13} + E_{14} - E_{34} &\le&  1,                                                                          \\
-E_{12} + E_{18} + E_{28} &\le&  1,                                                                          \\
 E_{14} + E_{18} - E_{48} &\le&  1,                                                                          \\
 E_{12} - E_{14} - E_{26} + E_{34} - E_{36} &\le&  -1,                                                       \\
 E_{12} + E_{13} + E_{26} + E_{36} &\le&  0,                                                                 \\
-E_{13} - E_{14} + E_{16} - E_{18} + E_{36} + E_{48} &\le&    0,                                             \\
-E_{12} - E_{16} - E_{26} &\le&  1,                                                                          \\
E_{16} - E_{18} + E_{26} - E_{28} &\le&  0,                                                                  \\
 E_{26} - E_{28} - E_{34} + E_{36} + E_{48} &\le&  1,                                                        \\
 E_{14} - E_{16} + E_{34} - E_{36} &\le&  0,                                                                 \\
-E_{13} - E_{14} - E_{26} + E_{28} - E_{36} - E_{48} &\le&  0,                                               \\
 E_{12} - E_{14} - E_{15} &\le&  -1,                                                                         \\
 E_{13} + E_{14} - E_{16} - E_{17} &\le&  0,                                                                 \\
 E_{12} - E_{14} + E_{16} - E_{18} - E_{19} &\le&  -1,                                                       \\
-E_{1,10} + E_{13} + E_{14} - E_{16} + E_{18} &\le&    1,                                                    \\
-E_{12} - E_{13} - E_{23} &\le&  1,                                                                          \\
 E_{12} - E_{14} - E_{24} &\le&  -1,                                                                         \\
 E_{14} - E_{25} &\le&  0,                                                                                   \\
-E_{13} - E_{14} - E_{26} - E_{27} &\le&  0,                                                                 \\
 E_{14} + E_{26} - E_{28} - E_{29} &\le&  0,                                                                 \\
-E_{12} - E_{13} - E_{14} - E_{2,10} - E_{26} + E_{28} &\le&    0,                                           \\
-E_{12} - E_{34} - E_{35} &\le&  1,                                                                          \\
 E_{34} - E_{36} - E_{37} &\le&  -1,                                                                         \\
 E_{13} + E_{14} + E_{26} - E_{28} - E_{34} + E_{36} - E_{38} &\le&   1,                                     \\
-E_{12} - E_{13} - E_{14} - E_{26} + E_{28} - E_{39} &\le&  0,                                               \\
 E_{14} + E_{26} - E_{28} - E_{3,10} &\le&  0,                                                               \\
 E_{12} - E_{45} &\le&  0,                                                                                   \\
 E_{34} - E_{36} - E_{46} &\le&  -1,                                                                         \\
 E_{36} - E_{47} &\le&  0,                                                                                   \\
 E_{12} + E_{34} - E_{36} - E_{48} - E_{49} &\le&  -1,                                                       \\
-E_{14} + E_{36} - E_{4,10} + E_{48} &\le&  0,                                                               \\
 E_{16} + E_{26} - E_{34} + E_{36} - E_{56} &\le&  1,                                                        \\
-E_{16} - E_{26} - E_{36} - E_{57} &\le&  0,                                                                 \\
 E_{18} + E_{28} - E_{48} - E_{58} &\le&  0,                                                                 \\
 E_{16} - E_{18} + E_{26} - E_{28} - E_{34} + E_{36} + E_{48} - E_{59} &\le&   0,                            \\
-E_{12} + E_{14} - E_{16} + E_{18} - E_{26} + E_{28} - E_{36} - E_{48} - E_{5,10} &\le&  1,                  \\
 E_{34} - E_{67} &\le&  0,                                                                                   \\
 E_{16} - E_{18} + E_{26} - E_{28} - E_{34} + E_{36} + E_{48} - E_{68} &\le&  0,                             \\
 E_{18} + E_{28} - E_{48} - E_{69} &\le&  0,                                                                 \\
-E_{18} + E_{26} - E_{28} + E_{36} + E_{48} - E_{6,10} &\le&  0,                                             \\
 E_{13} + E_{14} - E_{16} + E_{18} - E_{78} &\le&  1,                                                        \\
-E_{13} - E_{14} - E_{18} - E_{26} + E_{34} - E_{36} - E_{79} &\le&  -1,                                     \\
 E_{18} - E_{7,10} &\le&  0,                                                                                 \\
 E_{16} + E_{26} - E_{34} + E_{36} - E_{89} &\le&  1,                                                        \\
 E_{13} + E_{14} - E_{16} - E_{8,10} &\le&  0,                                                               \\
-E_{12} - E_{13} - E_{9,10} &\le&  1
.
\label{2017-b-kl-e-c}
%\end{aligned}
\end{eqnarray}


\subsection{Bub-Stairs inequality}

If one considers only the five probabilities on the intertwining atoms,
then the following Bub-Stairs inequality  $p_1+p_3+p_5+p_7+p_9  \le  2$, among others,
results:

{ \begin{lstlisting}[backgroundcolor=\color{yellow!10},framerule=0pt,breaklines=true, frame=tb]

* five probabilities on intertwining contexts
* p1, p3, p5, p7, p9
*
V-representation
begin
11  6  integer
1       1    0    0    0    0
1       1    0    1    0    0
1       1    0    0    1    0
1       0    1    0    0    0
1       0    1    0    1    0
1       0    1    0    0    1
1       0    0    1    0    0
1       0    0    1    0    1
1       0    0    0    1    0
1       0    0    0    0    1
1       0    0    0    0    0
end

~~~~~~ cddlib response

H-representation
begin
 11 6 real
  0  0  0  1  0  0
  1  0  0  0 -1 -1
  0  1  0  0  0  0
  1  0 -1 -1  0  0
  2 -1 -1 -1 -1 -1
  1 -1 -1  0  0  0
  0  0  0  0  1  0
  1 -1  0  0  0 -1
  1  0  0 -1 -1  0
  0  0  1  0  0  0
  0  0  0  0  0  1
end

\end{lstlisting}  }

One could also consider probabilities on the non-intertwining atoms yielding; in particular,
$p_2+p_4+p_6+p_8+p_{10} \ge 1$.

{ \begin{lstlisting}[backgroundcolor=\color{yellow!10},framerule=0pt,breaklines=true, frame=tb]

* five probabilities
* on non-intertwining atoms
* p2, p4, p6, p8, p10
*
V-representation
begin
11  6  integer
1       0    1    1    1    0
1       0    0    0    1    0
1       0    1    0    0    0
1       0    0    1    1    1
1       0    0    0    0    1
1       0    0    1    0    0
1       1    0    0    1    1
1       1    0    0    0    0
1       1    1    0    0    1
1       1    1    1    0    0
1       1    1    1    1    1
end

~~~~~~ cddlib response

H-representation
begin
 11 6 real
  0  0  0  0  1  0
  0  0  0  0  0  1
  0  0  1  0  0  0
 -1  1  1  1  1  1
  0  1  0  0  0  0
  0  0  0  1  0  0
  1  1 -1  1 -1 -1
  1 -1  1 -1 -1  1
  1  1 -1 -1  1 -1
  1 -1  1 -1  1 -1
  1 -1 -1  1 -1  1
end

\end{lstlisting}  }




\subsubsection{Klyachko-Can-Biniciogolu-Shumovsky inequalities}
\label{2017-b-kcbsia}


The following hull computation is limited to adjacent pair expectations;
it yields the Klyachko-Can-Biniciogolu-Shumovsky inequality
$  E_{13}  + E_{35}  + E_{57}  + E_{79}  + E_{91}   \ge 3   $:

{ \begin{lstlisting}[backgroundcolor=\color{yellow!10},framerule=0pt,breaklines=true, frame=tb]

* five joint Expectations:
* E13 E35 E57 E79 E91
*
V-representation
begin
11  6  real
1      -1    1    1    1   -1
1      -1   -1   -1    1   -1
1      -1    1   -1   -1   -1
1      -1   -1    1    1    1
1      -1   -1   -1   -1    1
1      -1   -1    1   -1   -1
1       1   -1   -1    1    1
1       1   -1   -1   -1   -1
1       1    1   -1   -1    1
1       1    1    1   -1   -1
1       1    1    1    1    1
end

~~~~~~ cddlib response

H-representation
begin
 11 6 real
  1  0  0  0  1  0
  1  0  0  0  0  1
  1  0  1  0  0  0
  3  1  1  1  1  1
  1  1  0  0  0  0
  1  0  0  1  0  0
  1  1 -1  1 -1 -1
  1 -1  1 -1 -1  1
  1  1 -1 -1  1 -1
  1 -1  1 -1  1 -1
  1 -1 -1  1 -1  1
end

\end{lstlisting}  }

\begin{eqnarray}
%\begin{aligned}
                                 - E_{79}             &\le& 1       \\
                                           - E_{91}   &\le& 1       \\
             - E_{35}                                 &\le& 1       \\
   - E_{13}  - E_{35}  - E_{57}  - E_{79}  - E_{91}   &\le& 3       \\
   - E_{13}                                           &\le& 1       \\
                       - E_{57}                       &\le& 1       \\
   - E_{13}  + E_{35}  - E_{57}  + E_{79}  + E_{91}   &\le& 1       \\
   + E_{13}  - E_{35}  + E_{57}  + E_{79}  - E_{91}   &\le& 1       \\
   - E_{13}  + E_{35}  + E_{57}  - E_{79}  + E_{91}   &\le& 1       \\
   + E_{13}  - E_{35}  + E_{57}  - E_{79}  + E_{91}   &\le& 1       \\
   + E_{13}  + E_{35}  - E_{57}  + E_{79}  - E_{91}   &\le& 1
.
\label{2017-b-kl-e-i}
%\end{aligned}
\end{eqnarray}

\subsection{Two intertwined pentagon logics forming a Specker K\"afer (bug) or cat's cradle logic}

\subsubsection{Probabilities on the Specker bug logic}

A {\em Mathematica}~\cite{Mathematica11.1} code to reduce probabilities on the Specker bug logic:


{ \begin{lstlisting}[backgroundcolor=\color{yellow!10},framerule=0pt,breaklines=true, frame=tb]

Reduce[
p1 + p2 + p3 == 1
&& p3 + p4 + p5 == 1
&& p5 + p6 + p7 == 1
&&   p7 + p8 + p9 == 1
&& p9 + p10 + p11 == 1
&& p11 + p12 + p1 == 1
&&  p4 + p10 + p13 == 1,
{p3, p11, p5, p9, p4, p10}, Reals]

~~~~~~ Mathematica response

p1 == 3/2 - p12/2 - p13/2 - p2/2 - p6/2 - p7 - p8/2 &&
 p3 == -(1/2) + p12/2 + p13/2 - p2/2 + p6/2 + p7 + p8/2 &&
 p11 == -(1/2) - p12/2 + p13/2 + p2/2 + p6/2 + p7 + p8/2 &&
 p5 == 1 - p6 - p7 && p9 == 1 - p7 - p8 &&
 p4 == 1/2 - p12/2 - p13/2 + p2/2 + p6/2 - p8/2 &&
 p10 == 1/2 + p12/2 - p13/2 - p2/2 - p6/2 + p8/2

\end{lstlisting}  }

Computation of all the two-valued states thereon:


{ \begin{lstlisting}[backgroundcolor=\color{yellow!10},framerule=0pt,breaklines=true, frame=tb]

Reduce[p1 + p2 + p3 == 1 && p3 + p4 + p5 == 1 && p5 + p6 + p7 == 1 &&
  p7 + p8 + p9 == 1 && p9 + p10 + p11 == 1 && p11 + p12 + p1 == 1 &&
  p4 + p10 + p13 == 1 && p1^2 == p1 && p2^2 == p2 && p3^2 == p3 &&
  p4^2 == p4 && p5^2 == p5 && p6^2 == p6 && p7^2 == p7 && p8^2 == p8 &&
   p9^2 == p9 && p10^2 == p10 && p11^2 == p11 && p12^2 == p12 &&
  p13^2 == p13]

~~~~~~ Mathematica response

(p9 == 0 && p8 == 0 && p7 == 1 && p6 == 0 && p5 == 0 && p4 == 0 &&
   p3 == 1 && p2 == 0 && p13 == 0 && p12 == 1 && p11 == 0 &&
   p10 == 1 && p1 == 0) || (p9 == 0 && p8 == 0 && p7 == 1 && p6 == 0 &&
    p5 == 0 && p4 == 0 && p3 == 1 && p2 == 0 && p13 == 1 && p12 == 0 &&
    p11 == 1 && p10 == 0 && p1 == 0) || (p9 == 0 && p8 == 0 &&
   p7 == 1 && p6 == 0 && p5 == 0 && p4 == 1 && p3 == 0 && p2 == 1 &&
   p13 == 0 && p12 == 0 && p11 == 1 && p10 == 0 &&
   p1 == 0) || (p9 == 0 && p8 == 1 && p7 == 0 && p6 == 0 && p5 == 1 &&
    p4 == 0 && p3 == 0 && p2 == 0 && p13 == 0 && p12 == 0 &&
   p11 == 0 && p10 == 1 && p1 == 1) || (p9 == 0 && p8 == 1 &&
   p7 == 0 && p6 == 0 && p5 == 1 && p4 == 0 && p3 == 0 && p2 == 1 &&
   p13 == 0 && p12 == 1 && p11 == 0 && p10 == 1 &&
   p1 == 0) || (p9 == 0 && p8 == 1 && p7 == 0 && p6 == 0 && p5 == 1 &&
    p4 == 0 && p3 == 0 && p2 == 1 && p13 == 1 && p12 == 0 &&
   p11 == 1 && p10 == 0 && p1 == 0) || (p9 == 0 && p8 == 1 &&
   p7 == 0 && p6 == 1 && p5 == 0 && p4 == 0 && p3 == 1 && p2 == 0 &&
   p13 == 0 && p12 == 1 && p11 == 0 && p10 == 1 &&
   p1 == 0) || (p9 == 0 && p8 == 1 && p7 == 0 && p6 == 1 && p5 == 0 &&
    p4 == 0 && p3 == 1 && p2 == 0 && p13 == 1 && p12 == 0 &&
   p11 == 1 && p10 == 0 && p1 == 0) || (p9 == 0 && p8 == 1 &&
   p7 == 0 && p6 == 1 && p5 == 0 && p4 == 1 && p3 == 0 && p2 == 1 &&
   p13 == 0 && p12 == 0 && p11 == 1 && p10 == 0 &&
   p1 == 0) || (p9 == 1 && p8 == 0 && p7 == 0 && p6 == 0 && p5 == 1 &&
    p4 == 0 && p3 == 0 && p2 == 0 && p13 == 1 && p12 == 0 &&
   p11 == 0 && p10 == 0 && p1 == 1) || (p9 == 1 && p8 == 0 &&
   p7 == 0 && p6 == 0 && p5 == 1 && p4 == 0 && p3 == 0 && p2 == 1 &&
   p13 == 1 && p12 == 1 && p11 == 0 && p10 == 0 &&
   p1 == 0) || (p9 == 1 && p8 == 0 && p7 == 0 && p6 == 1 && p5 == 0 &&
    p4 == 0 && p3 == 1 && p2 == 0 && p13 == 1 && p12 == 1 &&
   p11 == 0 && p10 == 0 && p1 == 0) || (p9 == 1 && p8 == 0 &&
   p7 == 0 && p6 == 1 && p5 == 0 && p4 == 1 && p3 == 0 && p2 == 0 &&
   p13 == 0 && p12 == 0 && p11 == 0 && p10 == 0 &&
   p1 == 1) || (p9 == 1 && p8 == 0 && p7 == 0 && p6 == 1 && p5 == 0 &&
    p4 == 1 && p3 == 0 && p2 == 1 && p13 == 0 && p12 == 1 &&
   p11 == 0 && p10 == 0 && p1 == 0)

\end{lstlisting}  }


\subsubsection{Hull calculation for the probabilities on the Specker bug logic}

{ \begin{lstlisting}[backgroundcolor=\color{yellow!10},framerule=0pt,breaklines=true, frame=tb]

* 13 probabilities on atoms a1...a13:
* p1 ... p13
*
V-representation
begin
14  14  real
1 1 0 0 0 1 0 0 0 1 0 0 0 1
1 1 0 0 1 0 1 0 0 1 0 0 0 0
1 1 0 0 0 1 0 0 1 0 1 0 0 0
1 0 1 0 0 1 0 0 0 1 0 0 1 1
1 0 1 0 0 1 0 0 1 0 0 1 0 1
1 0 1 0 1 0 1 0 0 1 0 0 1 0
1 0 1 0 1 0 0 1 0 0 0 1 0 0
1 0 1 0 1 0 1 0 1 0 0 1 0 0
1 0 1 0 0 1 0 0 1 0 1 0 1 0
1 0 0 1 0 0 0 1 0 0 0 1 0 1
1 0 0 1 0 0 1 0 1 0 0 1 0 1
1 0 0 1 0 0 1 0 0 1 0 0 1 1
1 0 0 1 0 0 0 1 0 0 1 0 1 0
1 0 0 1 0 0 1 0 1 0 1 0 1 0
end

~~~~~~ cddlib response


H-representation
linearity 7  17 18 19 20 21 22 23
begin
 23 14 real
  0  0  0  0  1  0  0  0  0  0  0  0  0  0
  0  0  0  0  0  0  1  0  0  0  0  0  0  0
  0  1  1  0 -1  0  1  0 -1  0  0  0  0  0
  0  1  0  0  0  0  0  0  0  0  0  0  0  0
  0  1  1  0 -1  0  0  0  0  0  0  0  0  0
  0  1  2  0 -2  0  1  0 -1  0  0  0  0  0
  0  0  1  0 -1  0  1  0  0  0  0  0  0  0
  0  0  1  0  0  0  0  0  0  0  0  0  0  0
  0  0  0  0  0  0  0  0  0  0  1  0  0  0
  0  0  0  0  0  0  0  0  1  0  0  0  0  0
  0  0  1  0 -1  0  1  0 -1  0  1  0  0  0
  1  0  0  0 -1  0  0  0  0  0 -1  0  0  0
  1 -1 -1  0  1  0 -1  0  1  0 -1  0  0  0
  1 -1 -1  0  0  0  0  0  1  0 -1  0  0  0
  1 -1 -1  0  0  0  0  0  0  0  0  0  0  0
  1 -1 -1  0  1  0 -1  0  0  0  0  0  0  0
 -1  1  1  1  0  0  0  0  0  0  0  0  0  0
  0 -1 -1  0  1  1  0  0  0  0  0  0  0  0
 -1  1  1  0 -1  0  1  1  0  0  0  0  0  0
  0 -1 -1  0  1  0 -1  0  1  1  0  0  0  0
 -1  1  1  0 -1  0  1  0 -1  0  1  1  0  0
  0  0 -1  0  1  0 -1  0  1  0 -1  0  1  0
 -1  0  0  0  1  0  0  0  0  0  1  0  0  1
end

\end{lstlisting}  }

The resulting face inequalities are
\begin{eqnarray}
%\begin{aligned}
                     - p_4                                                                                 \le     0 , \\
                                   - p_6                                                                   \le     0 , \\
- p_1  - p_2         + p_4         - p_6         + p_8                                                     \le     0 , \\
- p_1                                                                                                      \le     0 , \\
- p_1  - p_2         + p_4                                                                                 \le     0 , \\
- p_1  -2p_2         +2p_4         - p_6         + p_8                                                     \le     0 , \\
       - p_2         + p_4         - p_6                                                                   \le     0 , \\
       - p_2                                                                                               \le     0 , \\
                                                               - p_{10}                                    \le     0 , \\
                                                 - p_8                                                     \le     0 , \\
       - p_2         + p_4         - p_6         + p_8         - p_{10}                                    \le     0 , \\
                     + p_4                                     + p_{10}                                    \le    +1 , \\
+ p_1  + p_2         - p_4         + p_6         - p_8         + p_{10}                                    \le    +1 , \\
+ p_1  + p_2                                     - p_8         + p_{10}                                    \le    +1 , \\
+ p_1  + p_2                                                                                               \le    +1 , \\
+ p_1  + p_2         - p_4         + p_6                                                                   \le    +1 , \\
- p_1  - p_2  - p_3                                                                                        \le    -1 , \\
+ p_1  + p_2         - p_4  - p_5                                                                          \le     0 , \\
- p_1  - p_2         + p_4         - p_6  - p_7                                                            \le    -1 ,  \\
+ p_1  + p_2         - p_4         + p_6         - p_8  - p_9                                              \le     0 , \\
- p_1  - p_2         + p_4         - p_6         + p_8         - p_{10}  - p_{11}                          \le    -1 ,  \\
       + p_2         - p_4         + p_6         - p_8         + p_{10}            - p_{12}                \le     0 , \\
                     - p_4                                     - p_{10}                      - p_{13}      \le    -1
.
\label{2017-b-sb-p-c}
%\end{aligned}
\end{eqnarray}

\subsubsection{Hull calculation for the expectations on the Specker bug logic}

{ \begin{lstlisting}[backgroundcolor=\color{yellow!10},framerule=0pt,breaklines=true, frame=tb]

* (13 expectations on atoms a1...a13:
* E1 ... E13 not enumerated)
* 6 joint expectations  E1*E3, E3*E5, ..., E11*E1
*
V-representation
begin
14  7  integer
1      -1   -1   -1   -1   -1   -1
1      -1    1    1   -1   -1   -1
1      -1   -1   -1    1    1   -1
1       1   -1   -1   -1   -1    1
1       1   -1   -1    1   -1   -1
1       1    1    1   -1   -1    1
1       1    1   -1   -1   -1   -1
1       1    1    1    1   -1   -1
1       1   -1   -1    1    1    1
1      -1   -1   -1   -1   -1   -1
1      -1   -1    1    1   -1   -1
1      -1   -1    1   -1   -1    1
1      -1   -1   -1   -1    1    1
1      -1   -1    1    1    1    1
end

~~~~~~ cddlib response


H-representation
linearity 1  18
begin
 18 7 real
  1  0  0  0  1  0  0
  1 -1  0  0  1 -1  0
  1 -1  1 -1  1 -1  0
  1  0  0 -1  1 -1  0
  1  0  1  0  0  0  0
  1  1  0  0  0  0  0
  1  1 -1  1  0  0  0
  1  0  0  1  0  0  0
  1  1 -1  0 -1  0  0
  1  0  0  0 -1  0  0
  1  0 -1  1 -1  0  0
  1  1 -1  1 -1  1  0
  1  0  0 -1  0  0  0
  1 -1  1 -1  0  0  0
  1 -1  0  0  0  0  0
  1  0  0  0  0  1  0
  0  0 -1  0  0 -1  0
  0 -1  1 -1  1 -1  1
end

\end{lstlisting}  }

\subsubsection{Extended Specker bug logic}

Here is the {\em Mathematica}~\cite{Mathematica11.1} code to reduce probabilities on the extended (by two contexts) Specker bug logics:

{ \begin{lstlisting}[backgroundcolor=\color{yellow!10},framerule=0pt,breaklines=true, frame=tb]

Reduce[
p1 + p2 + p3 == 1
&& p3 + p4 + p5 == 1
&& p5 + p6 + p7 == 1
&&   p7 + p8 + p9 == 1
&& p9 + p10 + p11 == 1
&& p11 + p12 + p1 == 1
&&  p4 + p10 + p13 == 1
&& p1 + pc + q7 ==1
&& p7 + pc + q1 ==1,
{p3, p11, p5, p9, p4, p10, q3, q11, q5, q9, q4, q10, p13, q13, pc}]

~~~~~~ Mathematica response

p1 == p7 + q1 - q7 && p3 == 1 - p2 - p7 - q1 + q7 &&
 p11 == 1 - p12 - p7 - q1 + q7 && p5 == 1 - p6 - p7 &&
 p9 == 1 - p7 - p8 && p4 == -1 + p2 + p6 + 2 p7 + q1 - q7 &&
 p10 == -1 + p12 + 2 p7 + p8 + q1 - q7 &&
 p13 == 3 - p12 - p2 - p6 - 4 p7 - p8 - 2 q1 + 2 q7 &&
 pc == 1 - p7 - q1

\end{lstlisting}  }

Computation of all the 112 two-valued states thereon:


{ \begin{lstlisting}[backgroundcolor=\color{yellow!10},framerule=0pt,breaklines=true, frame=tb]

Reduce[p1 + p2 + p3 == 1 && p3 + p4 + p5 == 1 && p5 + p6 + p7 == 1 &&
  p7 + p8 + p9 == 1 && p9 + p10 + p11 == 1 && p11 + p12 + p1 == 1 &&
  p4 + p10 + p13 == 1 && p1^2 == p1 && p2^2 == p2 && p3^2 == p3 &&
  p4^2 == p4 && p5^2 == p5 && p6^2 == p6 && p7^2 == p7 && p8^2 == p8 &&
   p9^2 == p9 && p10^2 == p10 && p11^2 == p11 && p12^2 == p12 &&
  p13^2 == p13 &&  q1^2 == q1 &&  q7^2 == q7 &&  pc^2 == pc]

~~~~~~ Mathematica response

q7 == 0 && q1 == 0 && pc == 0 && p9 == 0 && p8 == 0 && p7 == 1 &&
   p6 == 0 && p5 == 0 && p4 == 0 && p3 == 1 && p2 == 0 && p13 == 0 &&
   p12 == 1 && p11 == 0 && p10 == 1 && p1 == 0) || (q7 == 0 &&
   q1 == 0 && pc == 0 && p9 == 0 && p8 == 0 && p7 == 1 && p6 == 0 &&
   p5 == 0 && p4 == 0 && p3 == 1 && p2 == 0 && p13 == 1 && p12 == 0 &&
    p11 == 1 && p10 == 0 && p1 == 0) ||
    [...]
  || (q7 == 1 && q1 == 1 && pc == 1 && p9 == 1 && p8 == 0 &&
    p7 == 0 && p6 == 1 && p5 == 0 && p4 == 1 && p3 == 0 && p2 == 1 &&
   p13 == 0 && p12 == 1 && p11 == 0 && p10 == 0 && p1 == 0)

\end{lstlisting}  }



\subsection{Two intertwined Specker bug logics}
\label{2017-b-bugscombinoa}

Here is the {\em Mathematica}~\cite{Mathematica11.1} code to reduce probabilities on two intertwined Specker bug logics:

{ \begin{lstlisting}[backgroundcolor=\color{yellow!10},framerule=0pt,breaklines=true, frame=tb]

Reduce[
p1 + p2 + p3 == 1
&& p3 + p4 + p5 == 1
&& p5 + p6 + p7 == 1
&&   p7 + p8 + p9 == 1
&& p9 + p10 + p11 == 1
&& p11 + p12 + p1 == 1
&&  p4 + p10 + p13 == 1
&& q1 + q2 + q3 == 1
&& q3 + q4 + q5 == 1
&& q5 + q6 + q7 == 1
&&   q7 + q8 + q9 == 1
&& q9 + q10 + q11 == 1
&& q11 + q12 + q1 == 1
&&  q4 + q10 + q13 == 1
&& p1 + pc + q7 ==1
&& p7 + pc + q1 ==1,
{p3, p11, p5, p9, p4, p10, q3, q11, q5, q9, q4, q10, p13, q13, pc}]

~~~~~~ Mathematica response

p1 == p7 + q1 - q7 && p3 == 1 - p2 - p7 - q1 + q7 &&
 p11 == 1 - p12 - p7 - q1 + q7 && p5 == 1 - p6 - p7 &&
 p9 == 1 - p7 - p8 && p4 == -1 + p2 + p6 + 2 p7 + q1 - q7 &&
 p10 == -1 + p12 + 2 p7 + p8 + q1 - q7 && q3 == 1 - q1 - q2 &&
 q11 == 1 - q1 - q12 && q5 == 1 - q6 - q7 && q9 == 1 - q7 - q8 &&
 q4 == -1 + q1 + q2 + q6 + q7 && q10 == -1 + q1 + q12 + q7 + q8 &&
 p13 == 3 - p12 - p2 - p6 - 4 p7 - p8 - 2 q1 + 2 q7 &&
 q13 == 3 - 2 q1 - q12 - q2 - q6 - 2 q7 - q8 && pc == 1 - p7 - q1

\end{lstlisting}  }

%All two-valued states thereon:
%
%{ \begin{lstlisting}[backgroundcolor=\color{yellow!10},framerule=0pt,breaklines=true, frame=tb]
%
%Reduce[
%p1 + p2 + p3 == 1
%&& p3 + p4 + p5 == 1
%&& p5 + p6 + p7 == 1
%&&   p7 + p8 + p9 == 1
%&& p9 + p10 + p11 == 1
%&& p11 + p12 + p1 == 1
%&&  p4 + p10 + p13 == 1
%&& q1 + q2 + q3 == 1
%&& q3 + q4 + q5 == 1
%&& q5 + q6 + q7 == 1
%&&   q7 + q8 + q9 == 1
%&& q9 + q10 + q11 == 1
%&& q11 + q12 + q1 == 1
%&&  q4 + q10 + q13 == 1
%&& p1 + pc + q7 ==1
%&& p7 + pc + q1 ==1
%&& (p1  == 0 ||  p1   == 1     )
%&& (p2  == 0 ||  p2   == 1     )
%&& (p3  == 0 ||  p3   == 1     )
%&& (p4  == 0 ||  p4   == 1     )
%&& (p5  == 0 ||  p5   == 1     )
%&& (p6  == 0 ||  p6   == 1     )
%&& (p7  == 0 ||  p7   == 1     )
%&& (p8  == 0 ||  p8   == 1     )
%&& (p9  == 0 ||  p9   == 1     )
%&& (p10 == 0 ||  p10  == 1     )
%&& (p11 == 0 ||  p11  == 1     )
%&& (p12 == 0 ||  p12  == 1     )
%&& (p13 == 0 ||  p13  == 1     )
%&& (q1  == 0 ||  q1   == 1     )
%&& (q2  == 0 ||  q2   == 1     )
%&& (q3  == 0 ||  q3   == 1     )
%&& (q4  == 0 ||  q4   == 1     )
%&& (q5  == 0 ||  q5   == 1     )
%&& (q6  == 0 ||  q6   == 1     )
%&& (q7  == 0 ||  q7   == 1     )
%&& (q8  == 0 ||  q8   == 1     )
%&& (q9  == 0 ||  q9   == 1     )
%&& (q10 == 0 ||  q10  == 1     )
%&& (q11 == 0 ||  q11  == 1     )
%&& (q12 == 0 ||  q12  == 1     )
%&& (q13 == 0 ||  q13  == 1     )
%&& (pc  == 0 ||  pc   == 1     )
%]
%
%~~~~~~ Mathematica response
%
%nil ;-(
%
%\end{lstlisting}  }



\subsubsection{Hull calculation for the contexual inequalities corresponding to the Cabello, Estebaranz and Garc{\'{i}}a-Alcaine logic}

{ \begin{lstlisting}[backgroundcolor=\color{yellow!10},framerule=0pt,breaklines=true, frame=tb]

* (13 expectations on atoms A1...A18:
*  not enumerated)
*  9 4th order expectations  A1A2A3A4 A4A5A6A7 ... A2A9A11A18
*
V-representation
begin
262144  10   real
1   1    1    1    1    1    1    1    1    1
1   1    1    1    1    1   -1   -1    1    1
1   1    1    1    1    1   -1    1    1   -1
[[...]]
1   1    1    1    1    1   -1    1    1   -1
1   1    1    1    1    1   -1   -1    1    1
1   1    1    1    1    1    1    1    1    1
end

~~~~~~ cddlib response

H-representation
begin
 274 10 real
  1  0  0  0  0  0  0  0  0  1
  1  0  0  0  0  0  0  0  1  0
  7 -1 -1 -1 -1 -1 -1  1  1  1
  7 -1 -1 -1 -1 -1  1 -1  1  1
  7 -1 -1 -1 -1  1 -1 -1  1  1
  7 -1 -1 -1  1 -1 -1 -1  1  1
  7 -1 -1  1 -1 -1 -1 -1  1  1
  7 -1  1 -1 -1 -1 -1 -1  1  1
  7  1 -1 -1 -1 -1 -1 -1  1  1
  1  0  0  0  0  0  0  1  0  0
  7 -1 -1 -1 -1 -1  1  1 -1  1
  7 -1 -1 -1 -1  1 -1  1 -1  1
  7 -1 -1 -1  1 -1 -1  1 -1  1
  7 -1 -1  1 -1 -1 -1  1 -1  1
  7 -1  1 -1 -1 -1 -1  1 -1  1
  7  1 -1 -1 -1 -1 -1  1 -1  1
  7 -1 -1 -1 -1 -1  1  1  1 -1
  7 -1 -1 -1 -1  1 -1  1  1 -1
  7 -1 -1 -1  1 -1 -1  1  1 -1
  7 -1 -1  1 -1 -1 -1  1  1 -1
  7 -1  1 -1 -1 -1 -1  1  1 -1
  7  1 -1 -1 -1 -1 -1  1  1 -1
  1  0  0  0  0  0  1  0  0  0
  7 -1 -1 -1 -1  1  1 -1 -1  1
  7 -1 -1 -1  1 -1  1 -1 -1  1
  7 -1 -1  1 -1 -1  1 -1 -1  1
  7 -1  1 -1 -1 -1  1 -1 -1  1
  7  1 -1 -1 -1 -1  1 -1 -1  1
  7 -1 -1 -1 -1  1  1 -1  1 -1
  7 -1 -1 -1  1 -1  1 -1  1 -1
  7 -1 -1  1 -1 -1  1 -1  1 -1
  7 -1  1 -1 -1 -1  1 -1  1 -1
  7  1 -1 -1 -1 -1  1 -1  1 -1
  7 -1 -1 -1 -1  1  1  1 -1 -1
  7 -1 -1 -1  1 -1  1  1 -1 -1
  7 -1 -1  1 -1 -1  1  1 -1 -1
  7 -1  1 -1 -1 -1  1  1 -1 -1
  7  1 -1 -1 -1 -1  1  1 -1 -1
  7 -1 -1 -1 -1  1  1  1  1  1
  7 -1 -1 -1  1 -1  1  1  1  1
  7 -1 -1  1 -1 -1  1  1  1  1
  7 -1  1 -1 -1 -1  1  1  1  1
  7  1 -1 -1 -1 -1  1  1  1  1
  1  0  0  0  0  1  0  0  0  0
  7 -1 -1 -1  1  1 -1 -1 -1  1
  7 -1 -1  1 -1  1 -1 -1 -1  1
  7 -1  1 -1 -1  1 -1 -1 -1  1
  7  1 -1 -1 -1  1 -1 -1 -1  1
  7 -1 -1 -1  1  1 -1 -1  1 -1
  7 -1 -1  1 -1  1 -1 -1  1 -1
  7 -1  1 -1 -1  1 -1 -1  1 -1
  7  1 -1 -1 -1  1 -1 -1  1 -1
  7 -1 -1 -1  1  1 -1  1 -1 -1
  7 -1 -1  1 -1  1 -1  1 -1 -1
  7 -1  1 -1 -1  1 -1  1 -1 -1
  7  1 -1 -1 -1  1 -1  1 -1 -1
  7 -1 -1 -1  1  1 -1  1  1  1
  7 -1 -1  1 -1  1 -1  1  1  1
  7 -1  1 -1 -1  1 -1  1  1  1
  7  1 -1 -1 -1  1 -1  1  1  1
  7 -1 -1 -1  1  1  1 -1 -1 -1
  7 -1 -1  1 -1  1  1 -1 -1 -1
  7 -1  1 -1 -1  1  1 -1 -1 -1
  7  1 -1 -1 -1  1  1 -1 -1 -1
  7 -1 -1 -1  1  1  1 -1  1  1
  7 -1 -1  1 -1  1  1 -1  1  1
  7 -1  1 -1 -1  1  1 -1  1  1
  7  1 -1 -1 -1  1  1 -1  1  1
  7 -1 -1 -1  1  1  1  1 -1  1
  7 -1 -1  1 -1  1  1  1 -1  1
  7 -1  1 -1 -1  1  1  1 -1  1
  7  1 -1 -1 -1  1  1  1 -1  1
  7 -1 -1 -1  1  1  1  1  1 -1
  7 -1 -1  1 -1  1  1  1  1 -1
  7 -1  1 -1 -1  1  1  1  1 -1
  7  1 -1 -1 -1  1  1  1  1 -1
  1  0  0  0  1  0  0  0  0  0
  7 -1 -1  1  1 -1 -1 -1 -1  1
  7 -1  1 -1  1 -1 -1 -1 -1  1
  7  1 -1 -1  1 -1 -1 -1 -1  1
  7 -1 -1  1  1 -1 -1 -1  1 -1
  7 -1  1 -1  1 -1 -1 -1  1 -1
  7  1 -1 -1  1 -1 -1 -1  1 -1
  7 -1 -1  1  1 -1 -1  1 -1 -1
  7 -1  1 -1  1 -1 -1  1 -1 -1
  7  1 -1 -1  1 -1 -1  1 -1 -1
  7 -1 -1  1  1 -1 -1  1  1  1
  7 -1  1 -1  1 -1 -1  1  1  1
  7  1 -1 -1  1 -1 -1  1  1  1
  7 -1 -1  1  1 -1  1 -1 -1 -1
  7 -1  1 -1  1 -1  1 -1 -1 -1
  7  1 -1 -1  1 -1  1 -1 -1 -1
  7 -1 -1  1  1 -1  1 -1  1  1
  7 -1  1 -1  1 -1  1 -1  1  1
  7  1 -1 -1  1 -1  1 -1  1  1
  7 -1 -1  1  1 -1  1  1 -1  1
  7 -1  1 -1  1 -1  1  1 -1  1
  7  1 -1 -1  1 -1  1  1 -1  1
  7 -1 -1  1  1 -1  1  1  1 -1
  7 -1  1 -1  1 -1  1  1  1 -1
  7  1 -1 -1  1 -1  1  1  1 -1
  7 -1 -1  1  1  1 -1 -1 -1 -1
  7 -1  1 -1  1  1 -1 -1 -1 -1
  7  1 -1 -1  1  1 -1 -1 -1 -1
  7 -1 -1  1  1  1 -1 -1  1  1
  7 -1  1 -1  1  1 -1 -1  1  1
  7  1 -1 -1  1  1 -1 -1  1  1
  7 -1 -1  1  1  1 -1  1 -1  1
  7 -1  1 -1  1  1 -1  1 -1  1
  7  1 -1 -1  1  1 -1  1 -1  1
  7 -1 -1  1  1  1 -1  1  1 -1
  7 -1  1 -1  1  1 -1  1  1 -1
  7  1 -1 -1  1  1 -1  1  1 -1
  7 -1 -1  1  1  1  1 -1 -1  1
  7 -1  1 -1  1  1  1 -1 -1  1
  7  1 -1 -1  1  1  1 -1 -1  1
  7 -1 -1  1  1  1  1 -1  1 -1
  7 -1  1 -1  1  1  1 -1  1 -1
  7  1 -1 -1  1  1  1 -1  1 -1
  7 -1 -1  1  1  1  1  1 -1 -1
  7 -1  1 -1  1  1  1  1 -1 -1
  7  1 -1 -1  1  1  1  1 -1 -1
  7 -1 -1  1  1  1  1  1  1  1
  7 -1  1 -1  1  1  1  1  1  1
  7  1 -1 -1  1  1  1  1  1  1
  1  0  0  1  0  0  0  0  0  0
  7 -1  1  1 -1 -1 -1 -1 -1  1
  7  1 -1  1 -1 -1 -1 -1 -1  1
  7 -1  1  1 -1 -1 -1 -1  1 -1
  7  1 -1  1 -1 -1 -1 -1  1 -1
  7 -1  1  1 -1 -1 -1  1 -1 -1
  7  1 -1  1 -1 -1 -1  1 -1 -1
  7 -1  1  1 -1 -1 -1  1  1  1
  7  1 -1  1 -1 -1 -1  1  1  1
  7 -1  1  1 -1 -1  1 -1 -1 -1
  7  1 -1  1 -1 -1  1 -1 -1 -1
  7 -1  1  1 -1 -1  1 -1  1  1
  7  1 -1  1 -1 -1  1 -1  1  1
  7 -1  1  1 -1 -1  1  1 -1  1
  7  1 -1  1 -1 -1  1  1 -1  1
  7 -1  1  1 -1 -1  1  1  1 -1
  7  1 -1  1 -1 -1  1  1  1 -1
  7 -1  1  1 -1  1 -1 -1 -1 -1
  7  1 -1  1 -1  1 -1 -1 -1 -1
  7 -1  1  1 -1  1 -1 -1  1  1
  7  1 -1  1 -1  1 -1 -1  1  1
  7 -1  1  1 -1  1 -1  1 -1  1
  7  1 -1  1 -1  1 -1  1 -1  1
  7 -1  1  1 -1  1 -1  1  1 -1
  7  1 -1  1 -1  1 -1  1  1 -1
  7 -1  1  1 -1  1  1 -1 -1  1
  7  1 -1  1 -1  1  1 -1 -1  1
  7 -1  1  1 -1  1  1 -1  1 -1
  7  1 -1  1 -1  1  1 -1  1 -1
  7 -1  1  1 -1  1  1  1 -1 -1
  7  1 -1  1 -1  1  1  1 -1 -1
  7 -1  1  1 -1  1  1  1  1  1
  7  1 -1  1 -1  1  1  1  1  1
  7 -1  1  1  1 -1 -1 -1 -1 -1
  7  1 -1  1  1 -1 -1 -1 -1 -1
  7 -1  1  1  1 -1 -1 -1  1  1
  7  1 -1  1  1 -1 -1 -1  1  1
  7 -1  1  1  1 -1 -1  1 -1  1
  7  1 -1  1  1 -1 -1  1 -1  1
  7 -1  1  1  1 -1 -1  1  1 -1
  7  1 -1  1  1 -1 -1  1  1 -1
  7 -1  1  1  1 -1  1 -1 -1  1
  7  1 -1  1  1 -1  1 -1 -1  1
  7 -1  1  1  1 -1  1 -1  1 -1
  7  1 -1  1  1 -1  1 -1  1 -1
  7 -1  1  1  1 -1  1  1 -1 -1
  7  1 -1  1  1 -1  1  1 -1 -1
  7 -1  1  1  1 -1  1  1  1  1
  7  1 -1  1  1 -1  1  1  1  1
  7 -1  1  1  1  1 -1 -1 -1  1
  7  1 -1  1  1  1 -1 -1 -1  1
  7 -1  1  1  1  1 -1 -1  1 -1
  7  1 -1  1  1  1 -1 -1  1 -1
  7 -1  1  1  1  1 -1  1 -1 -1
  7  1 -1  1  1  1 -1  1 -1 -1
  7 -1  1  1  1  1 -1  1  1  1
  7  1 -1  1  1  1 -1  1  1  1
  7 -1  1  1  1  1  1 -1 -1 -1
  7  1 -1  1  1  1  1 -1 -1 -1
  7 -1  1  1  1  1  1 -1  1  1
  7  1 -1  1  1  1  1 -1  1  1
  7 -1  1  1  1  1  1  1 -1  1
  7  1 -1  1  1  1  1  1 -1  1
  7 -1  1  1  1  1  1  1  1 -1
  7  1 -1  1  1  1  1  1  1 -1
  1  0  1  0  0  0  0  0  0  0
  7  1  1 -1 -1 -1 -1 -1 -1  1
  7  1  1 -1 -1 -1 -1 -1  1 -1
  7  1  1 -1 -1 -1 -1  1 -1 -1
  7  1  1 -1 -1 -1 -1  1  1  1
  7  1  1 -1 -1 -1  1 -1 -1 -1
  7  1  1 -1 -1 -1  1 -1  1  1
  7  1  1 -1 -1 -1  1  1 -1  1
  7  1  1 -1 -1 -1  1  1  1 -1
  7  1  1 -1 -1  1 -1 -1 -1 -1
  7  1  1 -1 -1  1 -1 -1  1  1
  7  1  1 -1 -1  1 -1  1 -1  1
  7  1  1 -1 -1  1 -1  1  1 -1
  7  1  1 -1 -1  1  1 -1 -1  1
  7  1  1 -1 -1  1  1 -1  1 -1
  7  1  1 -1 -1  1  1  1 -1 -1
  7  1  1 -1 -1  1  1  1  1  1
  7  1  1 -1  1 -1 -1 -1 -1 -1
  7  1  1 -1  1 -1 -1 -1  1  1
  7  1  1 -1  1 -1 -1  1 -1  1
  7  1  1 -1  1 -1 -1  1  1 -1
  7  1  1 -1  1 -1  1 -1 -1  1
  7  1  1 -1  1 -1  1 -1  1 -1
  7  1  1 -1  1 -1  1  1 -1 -1
  7  1  1 -1  1 -1  1  1  1  1
  7  1  1 -1  1  1 -1 -1 -1  1
  7  1  1 -1  1  1 -1 -1  1 -1
  7  1  1 -1  1  1 -1  1 -1 -1
  7  1  1 -1  1  1 -1  1  1  1
  7  1  1 -1  1  1  1 -1 -1 -1
  7  1  1 -1  1  1  1 -1  1  1
  7  1  1 -1  1  1  1  1 -1  1
  7  1  1 -1  1  1  1  1  1 -1
  7  1  1  1 -1 -1 -1 -1 -1 -1
  7  1  1  1 -1 -1 -1 -1  1  1
  7  1  1  1 -1 -1 -1  1 -1  1
  7  1  1  1 -1 -1 -1  1  1 -1
  7  1  1  1 -1 -1  1 -1 -1  1
  7  1  1  1 -1 -1  1 -1  1 -1
  7  1  1  1 -1 -1  1  1 -1 -1
  7  1  1  1 -1 -1  1  1  1  1
  7  1  1  1 -1  1 -1 -1 -1  1
  7  1  1  1 -1  1 -1 -1  1 -1
  7  1  1  1 -1  1 -1  1 -1 -1
  7  1  1  1 -1  1 -1  1  1  1
  7  1  1  1 -1  1  1 -1 -1 -1
  7  1  1  1 -1  1  1 -1  1  1
  7  1  1  1 -1  1  1  1 -1  1
  7  1  1  1 -1  1  1  1  1 -1
  7  1  1  1  1 -1 -1 -1 -1  1
  7  1  1  1  1 -1 -1 -1  1 -1
  7  1  1  1  1 -1 -1  1 -1 -1
  7  1  1  1  1 -1 -1  1  1  1
  7  1  1  1  1 -1  1 -1 -1 -1
  7  1  1  1  1 -1  1 -1  1  1
  7  1  1  1  1 -1  1  1 -1  1
  7  1  1  1  1 -1  1  1  1 -1
  7  1  1  1  1  1 -1 -1 -1 -1
  7  1  1  1  1  1 -1 -1  1  1
  7  1  1  1  1  1 -1  1 -1  1
  7  1  1  1  1  1 -1  1  1 -1
  7  1  1  1  1  1  1 -1 -1  1
  7  1  1  1  1  1  1 -1  1 -1
  7  1  1  1  1  1  1  1 -1 -1
  7  1  1  1  1  1  1  1  1  1
  1  1  0  0  0  0  0  0  0  0
  7  1 -1 -1 -1 -1 -1 -1 -1 -1
  7 -1  1 -1 -1 -1 -1 -1 -1 -1
  7 -1 -1  1 -1 -1 -1 -1 -1 -1
  7 -1 -1 -1  1 -1 -1 -1 -1 -1
  7 -1 -1 -1 -1  1 -1 -1 -1 -1
  7 -1 -1 -1 -1 -1  1 -1 -1 -1
  7 -1 -1 -1 -1 -1 -1  1 -1 -1
  7 -1 -1 -1 -1 -1 -1 -1  1 -1
  7 -1 -1 -1 -1 -1 -1 -1 -1  1
  1  0  0  0  0  0  0  0  0 -1
  1  0  0  0  0  0  0  0 -1  0
  1  0  0  0  0  0  0 -1  0  0
  1  0  0  0  0  0 -1  0  0  0
  1  0  0  0  0 -1  0  0  0  0
  1  0  0  0 -1  0  0  0  0  0
  1  0  0 -1  0  0  0  0  0  0
  1  0 -1  0  0  0  0  0  0  0
  1 -1  0  0  0  0  0  0  0  0
end

~~~~~~ cddlib reverse vertex computation


V-representation
begin
 256 10 real
  1 -1 -1 -1 -1 -1 -1  1  1  1
  1 -1 -1 -1 -1 -1  1 -1  1  1
  1 -1 -1 -1 -1  1 -1 -1  1  1
  1 -1 -1 -1  1 -1 -1 -1  1  1
  1 -1 -1  1 -1 -1 -1 -1  1  1
  1 -1  1 -1 -1 -1 -1 -1  1  1
  1  1 -1 -1 -1 -1 -1 -1  1  1
  1  1 -1 -1 -1 -1 -1  1  1 -1
  1 -1  1 -1 -1 -1 -1  1  1 -1
  1 -1 -1  1 -1 -1 -1  1  1 -1
  1 -1 -1 -1  1 -1 -1  1  1 -1
  1 -1 -1 -1 -1  1 -1  1  1 -1
  1 -1 -1 -1 -1 -1  1  1  1 -1
  1  1 -1 -1 -1 -1  1 -1  1 -1
  1 -1  1 -1 -1 -1  1 -1  1 -1
  1 -1 -1  1 -1 -1  1 -1  1 -1
  1 -1 -1 -1  1 -1  1 -1  1 -1
  1 -1 -1 -1 -1  1  1 -1  1 -1
  1  1 -1 -1 -1 -1  1  1  1  1
  1 -1  1 -1 -1 -1  1  1  1  1
  1 -1 -1  1 -1 -1  1  1  1  1
  1 -1 -1 -1  1 -1  1  1  1  1
  1 -1 -1 -1 -1  1  1  1  1  1
  1  1 -1 -1 -1  1 -1 -1  1 -1
  1 -1  1 -1 -1  1 -1 -1  1 -1
  1 -1 -1  1 -1  1 -1 -1  1 -1
  1 -1 -1 -1  1  1 -1 -1  1 -1
  1  1 -1 -1 -1  1 -1  1  1  1
  1 -1  1 -1 -1  1 -1  1  1  1
  1 -1 -1  1 -1  1 -1  1  1  1
  1 -1 -1 -1  1  1 -1  1  1  1
  1  1 -1 -1 -1  1  1 -1  1  1
  1 -1  1 -1 -1  1  1 -1  1  1
  1 -1 -1  1 -1  1  1 -1  1  1
  1 -1 -1 -1  1  1  1 -1  1  1
  1  1 -1 -1 -1  1  1  1  1 -1
  1 -1  1 -1 -1  1  1  1  1 -1
  1 -1 -1  1 -1  1  1  1  1 -1
  1 -1 -1 -1  1  1  1  1  1 -1
  1  1 -1 -1  1 -1 -1 -1  1 -1
  1 -1  1 -1  1 -1 -1 -1  1 -1
  1 -1 -1  1  1 -1 -1 -1  1 -1
  1  1 -1 -1  1 -1 -1  1  1  1
  1 -1  1 -1  1 -1 -1  1  1  1
  1 -1 -1  1  1 -1 -1  1  1  1
  1  1 -1 -1  1 -1  1 -1  1  1
  1 -1  1 -1  1 -1  1 -1  1  1
  1 -1 -1  1  1 -1  1 -1  1  1
  1  1 -1 -1  1 -1  1  1  1 -1
  1 -1  1 -1  1 -1  1  1  1 -1
  1 -1 -1  1  1 -1  1  1  1 -1
  1  1 -1 -1  1  1 -1 -1  1  1
  1 -1  1 -1  1  1 -1 -1  1  1
  1 -1 -1  1  1  1 -1 -1  1  1
  1  1 -1 -1  1  1 -1  1  1 -1
  1 -1  1 -1  1  1 -1  1  1 -1
  1 -1 -1  1  1  1 -1  1  1 -1
  1  1 -1 -1  1  1  1 -1  1 -1
  1 -1  1 -1  1  1  1 -1  1 -1
  1 -1 -1  1  1  1  1 -1  1 -1
  1  1 -1 -1  1  1  1  1  1  1
  1 -1  1 -1  1  1  1  1  1  1
  1 -1 -1  1  1  1  1  1  1  1
  1  1 -1  1 -1 -1 -1 -1  1 -1
  1 -1  1  1 -1 -1 -1 -1  1 -1
  1  1 -1  1 -1 -1 -1  1  1  1
  1 -1  1  1 -1 -1 -1  1  1  1
  1  1 -1  1 -1 -1  1 -1  1  1
  1 -1  1  1 -1 -1  1 -1  1  1
  1  1 -1  1 -1 -1  1  1  1 -1
  1 -1  1  1 -1 -1  1  1  1 -1
  1  1 -1  1 -1  1 -1 -1  1  1
  1 -1  1  1 -1  1 -1 -1  1  1
  1  1 -1  1 -1  1 -1  1  1 -1
  1 -1  1  1 -1  1 -1  1  1 -1
  1  1 -1  1 -1  1  1 -1  1 -1
  1 -1  1  1 -1  1  1 -1  1 -1
  1  1 -1  1 -1  1  1  1  1  1
  1 -1  1  1 -1  1  1  1  1  1
  1  1 -1  1  1 -1 -1 -1  1  1
  1 -1  1  1  1 -1 -1 -1  1  1
  1  1 -1  1  1 -1 -1  1  1 -1
  1 -1  1  1  1 -1 -1  1  1 -1
  1  1 -1  1  1 -1  1 -1  1 -1
  1 -1  1  1  1 -1  1 -1  1 -1
  1  1 -1  1  1 -1  1  1  1  1
  1 -1  1  1  1 -1  1  1  1  1
  1  1 -1  1  1  1 -1 -1  1 -1
  1 -1  1  1  1  1 -1 -1  1 -1
  1  1 -1  1  1  1 -1  1  1  1
  1 -1  1  1  1  1 -1  1  1  1
  1  1 -1  1  1  1  1 -1  1  1
  1 -1  1  1  1  1  1 -1  1  1
  1  1 -1  1  1  1  1  1  1 -1
  1 -1  1  1  1  1  1  1  1 -1
  1  1  1 -1 -1 -1 -1 -1  1 -1
  1  1  1 -1 -1 -1 -1  1  1  1
  1  1  1 -1 -1 -1  1 -1  1  1
  1  1  1 -1 -1 -1  1  1  1 -1
  1  1  1 -1 -1  1 -1 -1  1  1
  1  1  1 -1 -1  1 -1  1  1 -1
  1  1  1 -1 -1  1  1 -1  1 -1
  1  1  1 -1 -1  1  1  1  1  1
  1  1  1 -1  1 -1 -1 -1  1  1
  1  1  1 -1  1 -1 -1  1  1 -1
  1  1  1 -1  1 -1  1 -1  1 -1
  1  1  1 -1  1 -1  1  1  1  1
  1  1  1 -1  1  1 -1 -1  1 -1
  1  1  1 -1  1  1 -1  1  1  1
  1  1  1 -1  1  1  1 -1  1  1
  1  1  1 -1  1  1  1  1  1 -1
  1  1  1  1 -1 -1 -1 -1  1  1
  1  1  1  1 -1 -1 -1  1  1 -1
  1  1  1  1 -1 -1  1 -1  1 -1
  1  1  1  1 -1 -1  1  1  1  1
  1  1  1  1 -1  1 -1 -1  1 -1
  1  1  1  1 -1  1 -1  1  1  1
  1  1  1  1 -1  1  1 -1  1  1
  1  1  1  1 -1  1  1  1  1 -1
  1  1  1  1  1 -1 -1 -1  1 -1
  1  1  1  1  1 -1 -1  1  1  1
  1  1  1  1  1 -1  1 -1  1  1
  1  1  1  1  1 -1  1  1  1 -1
  1  1  1  1  1  1 -1 -1  1  1
  1  1  1  1  1  1 -1  1  1 -1
  1  1  1  1  1  1  1 -1  1 -1
  1  1  1  1  1  1  1  1  1  1
  1  1  1  1  1  1  1  1 -1 -1
  1  1  1  1  1  1  1 -1 -1  1
  1  1  1  1  1  1 -1  1 -1  1
  1  1  1  1  1  1 -1 -1 -1 -1
  1  1  1  1  1 -1  1  1 -1  1
  1  1  1  1  1 -1  1 -1 -1 -1
  1  1  1  1  1 -1 -1  1 -1 -1
  1  1  1  1  1 -1 -1 -1 -1  1
  1  1  1  1 -1  1  1  1 -1  1
  1  1  1  1 -1  1  1 -1 -1 -1
  1  1  1  1 -1  1 -1  1 -1 -1
  1  1  1  1 -1  1 -1 -1 -1  1
  1  1  1  1 -1 -1  1  1 -1 -1
  1  1  1  1 -1 -1  1 -1 -1  1
  1  1  1  1 -1 -1 -1  1 -1  1
  1  1  1  1 -1 -1 -1 -1 -1 -1
  1  1  1 -1  1  1  1  1 -1  1
  1  1  1 -1  1  1  1 -1 -1 -1
  1  1  1 -1  1  1 -1  1 -1 -1
  1  1  1 -1  1  1 -1 -1 -1  1
  1  1  1 -1  1 -1  1  1 -1 -1
  1  1  1 -1  1 -1  1 -1 -1  1
  1  1  1 -1  1 -1 -1  1 -1  1
  1  1  1 -1  1 -1 -1 -1 -1 -1
  1  1  1 -1 -1  1  1  1 -1 -1
  1  1  1 -1 -1  1  1 -1 -1  1
  1  1  1 -1 -1  1 -1  1 -1  1
  1  1  1 -1 -1  1 -1 -1 -1 -1
  1  1  1 -1 -1 -1  1  1 -1  1
  1  1  1 -1 -1 -1  1 -1 -1 -1
  1  1  1 -1 -1 -1 -1  1 -1 -1
  1  1  1 -1 -1 -1 -1 -1 -1  1
  1 -1  1  1  1  1  1  1 -1  1
  1  1 -1  1  1  1  1  1 -1  1
  1 -1  1  1  1  1  1 -1 -1 -1
  1  1 -1  1  1  1  1 -1 -1 -1
  1 -1  1  1  1  1 -1  1 -1 -1
  1  1 -1  1  1  1 -1  1 -1 -1
  1 -1  1  1  1  1 -1 -1 -1  1
  1  1 -1  1  1  1 -1 -1 -1  1
  1 -1  1  1  1 -1  1  1 -1 -1
  1  1 -1  1  1 -1  1  1 -1 -1
  1 -1  1  1  1 -1  1 -1 -1  1
  1  1 -1  1  1 -1  1 -1 -1  1
  1 -1  1  1  1 -1 -1  1 -1  1
  1  1 -1  1  1 -1 -1  1 -1  1
  1 -1  1  1  1 -1 -1 -1 -1 -1
  1  1 -1  1  1 -1 -1 -1 -1 -1
  1 -1  1  1 -1  1  1  1 -1 -1
  1  1 -1  1 -1  1  1  1 -1 -1
  1 -1  1  1 -1  1  1 -1 -1  1
  1  1 -1  1 -1  1  1 -1 -1  1
  1 -1  1  1 -1  1 -1  1 -1  1
  1  1 -1  1 -1  1 -1  1 -1  1
  1 -1  1  1 -1  1 -1 -1 -1 -1
  1  1 -1  1 -1  1 -1 -1 -1 -1
  1 -1  1  1 -1 -1  1  1 -1  1
  1  1 -1  1 -1 -1  1  1 -1  1
  1 -1  1  1 -1 -1  1 -1 -1 -1
  1  1 -1  1 -1 -1  1 -1 -1 -1
  1 -1  1  1 -1 -1 -1  1 -1 -1
  1  1 -1  1 -1 -1 -1  1 -1 -1
  1 -1  1  1 -1 -1 -1 -1 -1  1
  1  1 -1  1 -1 -1 -1 -1 -1  1
  1 -1 -1  1  1  1  1  1 -1 -1
  1 -1  1 -1  1  1  1  1 -1 -1
  1  1 -1 -1  1  1  1  1 -1 -1
  1 -1 -1  1  1  1  1 -1 -1  1
  1 -1  1 -1  1  1  1 -1 -1  1
  1  1 -1 -1  1  1  1 -1 -1  1
  1 -1 -1  1  1  1 -1  1 -1  1
  1 -1  1 -1  1  1 -1  1 -1  1
  1  1 -1 -1  1  1 -1  1 -1  1
  1 -1 -1  1  1  1 -1 -1 -1 -1
  1 -1  1 -1  1  1 -1 -1 -1 -1
  1  1 -1 -1  1  1 -1 -1 -1 -1
  1 -1 -1  1  1 -1  1  1 -1  1
  1 -1  1 -1  1 -1  1  1 -1  1
  1  1 -1 -1  1 -1  1  1 -1  1
  1 -1 -1  1  1 -1  1 -1 -1 -1
  1 -1  1 -1  1 -1  1 -1 -1 -1
  1  1 -1 -1  1 -1  1 -1 -1 -1
  1 -1 -1  1  1 -1 -1  1 -1 -1
  1 -1  1 -1  1 -1 -1  1 -1 -1
  1  1 -1 -1  1 -1 -1  1 -1 -1
  1 -1 -1  1  1 -1 -1 -1 -1  1
  1 -1  1 -1  1 -1 -1 -1 -1  1
  1  1 -1 -1  1 -1 -1 -1 -1  1
  1 -1 -1 -1  1  1  1  1 -1  1
  1 -1 -1  1 -1  1  1  1 -1  1
  1 -1  1 -1 -1  1  1  1 -1  1
  1  1 -1 -1 -1  1  1  1 -1  1
  1 -1 -1 -1  1  1  1 -1 -1 -1
  1 -1 -1  1 -1  1  1 -1 -1 -1
  1 -1  1 -1 -1  1  1 -1 -1 -1
  1  1 -1 -1 -1  1  1 -1 -1 -1
  1 -1 -1 -1  1  1 -1  1 -1 -1
  1 -1 -1  1 -1  1 -1  1 -1 -1
  1 -1  1 -1 -1  1 -1  1 -1 -1
  1  1 -1 -1 -1  1 -1  1 -1 -1
  1 -1 -1 -1  1  1 -1 -1 -1  1
  1 -1 -1  1 -1  1 -1 -1 -1  1
  1 -1  1 -1 -1  1 -1 -1 -1  1
  1  1 -1 -1 -1  1 -1 -1 -1  1
  1 -1 -1 -1 -1  1  1  1 -1 -1
  1 -1 -1 -1  1 -1  1  1 -1 -1
  1 -1 -1  1 -1 -1  1  1 -1 -1
  1 -1  1 -1 -1 -1  1  1 -1 -1
  1  1 -1 -1 -1 -1  1  1 -1 -1
  1 -1 -1 -1 -1  1  1 -1 -1  1
  1 -1 -1 -1  1 -1  1 -1 -1  1
  1 -1 -1  1 -1 -1  1 -1 -1  1
  1 -1  1 -1 -1 -1  1 -1 -1  1
  1  1 -1 -1 -1 -1  1 -1 -1  1
  1 -1 -1 -1 -1 -1  1  1 -1  1
  1 -1 -1 -1 -1  1 -1  1 -1  1
  1 -1 -1 -1  1 -1 -1  1 -1  1
  1 -1 -1  1 -1 -1 -1  1 -1  1
  1 -1  1 -1 -1 -1 -1  1 -1  1
  1  1 -1 -1 -1 -1 -1  1 -1  1
  1 -1 -1 -1 -1 -1 -1  1 -1 -1
  1 -1 -1 -1 -1 -1  1 -1 -1 -1
  1 -1 -1 -1 -1  1 -1 -1 -1 -1
  1 -1 -1 -1  1 -1 -1 -1 -1 -1
  1 -1 -1  1 -1 -1 -1 -1 -1 -1
  1 -1  1 -1 -1 -1 -1 -1 -1 -1
  1  1 -1 -1 -1 -1 -1 -1 -1 -1
  1 -1 -1 -1 -1 -1 -1 -1  1 -1
  1 -1 -1 -1 -1 -1 -1 -1 -1  1
end

\end{lstlisting}  }



\subsubsection{Hull calculation for the contexual inequalities corresponding to the pentagon logic}

{ \begin{lstlisting}[backgroundcolor=\color{yellow!10},framerule=0pt,breaklines=true, frame=tb]

* (10 expectations on atoms A1...A10:
*  not enumerated)
*  5 3th order expectations  A1A2A3 A3A4A5 ... A9A10A1
*  obtained through reverse Hull computation
V-representation
begin
 32 6 real
  1  1 -1 -1 -1 -1
  1  1 -1 -1 -1  1
  1  1 -1 -1  1 -1
  1  1 -1 -1  1  1
  1  1 -1  1 -1 -1
  1  1 -1  1 -1  1
  1  1 -1  1  1 -1
  1  1 -1  1  1  1
  1  1  1  1 -1 -1
  1  1  1  1 -1  1
  1  1  1  1  1  1
  1  1  1  1  1 -1
  1  1  1 -1  1  1
  1  1  1 -1  1 -1
  1  1  1 -1 -1  1
  1  1  1 -1 -1 -1
  1 -1  1  1  1  1
  1 -1  1  1  1 -1
  1 -1  1  1 -1  1
  1 -1  1  1 -1 -1
  1 -1  1 -1  1  1
  1 -1  1 -1  1 -1
  1 -1  1 -1 -1  1
  1 -1  1 -1 -1 -1
  1 -1 -1  1  1  1
  1 -1 -1  1  1 -1
  1 -1 -1  1 -1  1
  1 -1 -1  1 -1 -1
  1 -1 -1 -1  1  1
  1 -1 -1 -1  1 -1
  1 -1 -1 -1 -1  1
  1 -1 -1 -1 -1 -1
end

~~~~~~ cddlib response

H-representation
begin
 10 6 real
  1  0  0  0  0  1
  1  0  0  0  1  0
  1  0  0  1  0  0
  1  0  1  0  0  0
  1  1  0  0  0  0
  1  0  0  0  0 -1
  1  0  0  0 -1  0
  1  0  0 -1  0  0
  1  0 -1  0  0  0
  1 -1  0  0  0  0
end

\end{lstlisting}  }

\subsubsection{Hull calculation for the contexual inequalities corresponding to Specker bug logics}

{ \begin{lstlisting}[backgroundcolor=\color{yellow!10},framerule=0pt,breaklines=true, frame=tb]

* (13 expectations on atoms A1...A13:
*  not enumerated)
*  7 3th order expectations  A1A2A3 A3A4A5 ... A11A12A1 A4A13A10
*  obtained through reverse Hull computation
V-representation
begin
 128 8 real
  1  1 -1 -1 -1 -1 -1 -1
  1  1 -1 -1 -1 -1 -1  1
  1  1 -1 -1 -1 -1  1 -1
  1  1 -1 -1 -1 -1  1  1
  1  1 -1 -1 -1  1 -1 -1
  1  1 -1 -1 -1  1 -1  1
  1  1 -1 -1 -1  1  1 -1
  1  1 -1 -1 -1  1  1  1
  1  1 -1 -1  1 -1 -1 -1
  1  1 -1 -1  1 -1 -1  1
  1  1 -1 -1  1 -1  1 -1
  1  1 -1 -1  1 -1  1  1
  1  1 -1 -1  1  1 -1 -1
  1  1 -1 -1  1  1 -1  1
  1  1 -1 -1  1  1  1 -1
  1  1 -1 -1  1  1  1  1
  1  1 -1  1 -1 -1 -1 -1
  1  1 -1  1 -1 -1 -1  1
  1  1 -1  1 -1 -1  1 -1
  1  1 -1  1 -1 -1  1  1
  1  1 -1  1 -1  1 -1 -1
  1  1 -1  1 -1  1 -1  1
  1  1 -1  1 -1  1  1 -1
  1  1 -1  1 -1  1  1  1
  1  1 -1  1  1 -1 -1 -1
  1  1 -1  1  1 -1 -1  1
  1  1 -1  1  1 -1  1 -1
  1  1 -1  1  1 -1  1  1
  1  1 -1  1  1  1 -1 -1
  1  1 -1  1  1  1 -1  1
  1  1 -1  1  1  1  1 -1
  1  1 -1  1  1  1  1  1
  1  1  1  1 -1 -1 -1 -1
  1  1  1  1 -1 -1 -1  1
  1  1  1  1 -1 -1  1 -1
  1  1  1  1 -1 -1  1  1
  1  1  1  1 -1  1 -1 -1
  1  1  1  1 -1  1 -1  1
  1  1  1  1 -1  1  1 -1
  1  1  1  1 -1  1  1  1
  1  1  1  1  1  1 -1 -1
  1  1  1  1  1  1 -1  1
  1  1  1  1  1  1  1  1
  1  1  1  1  1  1  1 -1
  1  1  1  1  1 -1  1  1
  1  1  1  1  1 -1  1 -1
  1  1  1  1  1 -1 -1  1
  1  1  1  1  1 -1 -1 -1
  1  1  1 -1  1  1  1  1
  1  1  1 -1  1  1  1 -1
  1  1  1 -1  1  1 -1  1
  1  1  1 -1  1  1 -1 -1
  1  1  1 -1  1 -1  1  1
  1  1  1 -1  1 -1  1 -1
  1  1  1 -1  1 -1 -1  1
  1  1  1 -1  1 -1 -1 -1
  1  1  1 -1 -1  1  1  1
  1  1  1 -1 -1  1  1 -1
  1  1  1 -1 -1  1 -1  1
  1  1  1 -1 -1  1 -1 -1
  1  1  1 -1 -1 -1  1  1
  1  1  1 -1 -1 -1  1 -1
  1  1  1 -1 -1 -1 -1  1
  1  1  1 -1 -1 -1 -1 -1
  1 -1  1  1  1  1  1  1
  1 -1  1  1  1  1  1 -1
  1 -1  1  1  1  1 -1  1
  1 -1  1  1  1  1 -1 -1
  1 -1  1  1  1 -1  1  1
  1 -1  1  1  1 -1  1 -1
  1 -1  1  1  1 -1 -1  1
  1 -1  1  1  1 -1 -1 -1
  1 -1  1  1 -1  1  1  1
  1 -1  1  1 -1  1  1 -1
  1 -1  1  1 -1  1 -1  1
  1 -1  1  1 -1  1 -1 -1
  1 -1  1  1 -1 -1  1  1
  1 -1  1  1 -1 -1  1 -1
  1 -1  1  1 -1 -1 -1  1
  1 -1  1  1 -1 -1 -1 -1
  1 -1  1 -1  1  1  1  1
  1 -1  1 -1  1  1  1 -1
  1 -1  1 -1  1  1 -1  1
  1 -1  1 -1  1  1 -1 -1
  1 -1  1 -1  1 -1  1  1
  1 -1  1 -1  1 -1  1 -1
  1 -1  1 -1  1 -1 -1  1
  1 -1  1 -1  1 -1 -1 -1
  1 -1  1 -1 -1  1  1  1
  1 -1  1 -1 -1  1  1 -1
  1 -1  1 -1 -1  1 -1  1
  1 -1  1 -1 -1  1 -1 -1
  1 -1  1 -1 -1 -1  1  1
  1 -1  1 -1 -1 -1  1 -1
  1 -1  1 -1 -1 -1 -1  1
  1 -1  1 -1 -1 -1 -1 -1
  1 -1 -1  1  1  1  1  1
  1 -1 -1  1  1  1  1 -1
  1 -1 -1  1  1  1 -1  1
  1 -1 -1  1  1  1 -1 -1
  1 -1 -1  1  1 -1  1  1
  1 -1 -1  1  1 -1  1 -1
  1 -1 -1  1  1 -1 -1  1
  1 -1 -1  1  1 -1 -1 -1
  1 -1 -1  1 -1  1  1  1
  1 -1 -1  1 -1  1  1 -1
  1 -1 -1  1 -1  1 -1  1
  1 -1 -1  1 -1  1 -1 -1
  1 -1 -1  1 -1 -1  1  1
  1 -1 -1  1 -1 -1  1 -1
  1 -1 -1  1 -1 -1 -1  1
  1 -1 -1  1 -1 -1 -1 -1
  1 -1 -1 -1  1  1  1  1
  1 -1 -1 -1  1  1  1 -1
  1 -1 -1 -1  1  1 -1  1
  1 -1 -1 -1  1  1 -1 -1
  1 -1 -1 -1  1 -1  1  1
  1 -1 -1 -1  1 -1  1 -1
  1 -1 -1 -1  1 -1 -1  1
  1 -1 -1 -1  1 -1 -1 -1
  1 -1 -1 -1 -1  1  1  1
  1 -1 -1 -1 -1  1  1 -1
  1 -1 -1 -1 -1  1 -1  1
  1 -1 -1 -1 -1  1 -1 -1
  1 -1 -1 -1 -1 -1  1  1
  1 -1 -1 -1 -1 -1  1 -1
  1 -1 -1 -1 -1 -1 -1  1
  1 -1 -1 -1 -1 -1 -1 -1
end

~~~~~~ cddlib response

H-representation
begin
 14 8 real
  1  0  0  0  0  0  0  1
  1  0  0  0  0  0  1  0
  1  0  0  0  0  1  0  0
  1  0  0  0  1  0  0  0
  1  0  0  1  0  0  0  0
  1  0  1  0  0  0  0  0
  1  1  0  0  0  0  0  0
  1  0  0  0  0  0  0 -1
  1  0  0  0  0  0 -1  0
  1  0  0  0  0 -1  0  0
  1  0  0  0 -1  0  0  0
  1  0  0 -1  0  0  0  0
  1  0 -1  0  0  0  0  0
  1 -1  0  0  0  0  0  0
end

\end{lstlisting}  }



\subsubsection{Min-max calculation for the quantum bounds of two-two-state particles}


{ \begin{lstlisting}[backgroundcolor=\color{yellow!10},framerule=0pt,breaklines=true, frame=tb]


(* ~~~~~~~~~~~~~~~~~~~~~~~~~~~~~~~~~~~~~~~~~~~~~~~~~~~~~~~~~~~~~~~~~~~~~~~ *)
(* ~~~~~~~~~~~~~~~~~~Start Mathematica Code~~~~~~~~~~~~~~~~~~~~~~~~~~~~~~~ *)
(* ~~~~~~~~~~~~~~~~~~~~~~~~~~~~~~~~~~~~~~~~~~~~~~~~~~~~~~~~~~~~~~~~~~~~~~~ *)

(* old stuff

<<Algebra`ReIm`

Normalize[z_]:= z/Sqrt[z.Conjugate[z]];    *)

(*Definition of "my" Tensor Product*)
(*a,b are nxn and mxm-matrices*)

MyTensorProduct[a_, b_] :=
  Table[
   a[[Ceiling[s/Length[b]], Ceiling[t/Length[b]]]]*
    b[[s - Floor[(s - 1)/Length[b]]*Length[b],
      t - Floor[(t - 1)/Length[b]]*Length[b]]], {s, 1,
    Length[a]*Length[b]}, {t, 1, Length[a]*Length[b]}];


(*Definition of the Tensor Product between two vectors*)

TensorProductVec[x_, y_] :=
  Flatten[Table[
    x[[i]] y[[j]], {i, 1, Length[x]}, {j, 1, Length[y]}]];


(*Definition of the Dyadic Product*)

DyadicProductVec[x_] :=
  Table[x[[i]] Conjugate[x[[j]]], {i, 1, Length[x]}, {j, 1,
    Length[x]}];

(*Definition of the sigma matrices*)


vecsig[r_, tt_, p_] :=
 r*{{Cos[tt], Sin[tt] Exp[-I p]}, {Sin[tt] Exp[I p], -Cos[tt]}}

(*Definition of some vectors*)

BellBasis = (1/Sqrt[2]) {{1, 0, 0, 1}, {0, 1, 1, 0}, {0, 1, -1,
     0}, {1, 0, 0, -1}};

Basis = {{1, 0, 0, 0}, {0, 1, 0, 0}, {0, 0, 1, 0}, {0, 0, 0, 1}};



(*~~~~~~~~~~~~~~~~~~~~~~~~~  2  PARTICLES ~~~~~~~~~~~~~~~~~~~~~~~~~~~~~~~~~~~~~~~*)

(*~~~~~~~~~~~~~~~~~~~~~~~~~  2  State System ~~~~~~~~~~~~~~~~~~~~~~~~~~~~~~~~~~~~~~~

% ~~~~~~~~~~~~~~~   2 x 2
% ~~~~~~~~~~~~~~~   2 x 2
% ~~~~~~~~~~~~~~~   2 x 2
% ~~~~~~~~~~~~~~~   2 x 2
% ~~~~~~~~~~~~~~~   2 x 2
% ~~~~~~~~~~~~~~~   2 x 2

*)


(*Definition of singlet state*)
vp = {1,0};
vm = {0,1};
psi2s = (1/Sqrt[2])*(TensorProductVec[vp, vm] -
    TensorProductVec[vm, vp])

DyadicProductVec[psi2s]

(*Definition of operators*)

(* Definition of one-particle operator *)

M2X = (1/2) {{0, 1}, {1, 0}};
M2Y = (1/2) {{0, -I}, {I, 0}};
M2Z = (1/2) {{1, 0}, {0, -1}};


Eigenvectors[M2X]
Eigenvectors[M2Y]
Eigenvectors[M2Z]

S2[t_, p_] := FullSimplify[M2X *Sin[t] Cos[p] + M2Y *Sin[t] Sin[p] + M2Z *Cos[t]]

FullSimplify[S2[\[Theta], \[Phi]]] // MatrixForm

FullSimplify[ComplexExpand[S2[Pi/2, 0]]] // MatrixForm
FullSimplify[ComplexExpand[S2[Pi/2, Pi/2]]] // MatrixForm
FullSimplify[ComplexExpand[S2[0, 0]]] // MatrixForm

Assuming[{0 <= \[Theta] <= Pi, 0 <= \[Phi] <= 2 Pi}, FullSimplify[Eigensystem[S2[\[Theta], \[Phi]]], {Element[\[Theta], Reals],
  Element[\[Phi], Reals]}]]



FullSimplify[
 Normalize[
  Eigenvectors[S2[\[Theta], \[Phi]]][[1]]], {Element[\[Theta], Reals],
   Element[\[Phi], Reals]}]

ES2M[\[Theta]_,\[Phi]_] := {-E^(-I \[Phi]) Tan[\[Theta]/2], 1}*Cos[\[Theta]/2]*E^(I \[Phi]/2)
ES2P[\[Theta]_,\[Phi]_] := {E^(-I \[Phi]) Cot[\[Theta]/2], 1}*Sin[\[Theta]/2]*E^(I \[Phi]/2)

FullSimplify[ES2M[\[Theta],\[Phi]] .Conjugate[ES2M [\[Theta],\[Phi]]], {Element[\[Theta], Reals],
  Element[\[Phi], Reals]}]
FullSimplify[ES2P[\[Theta],\[Phi]] .Conjugate[ES2P [\[Theta],\[Phi]]], {Element[\[Theta], Reals],
  Element[\[Phi], Reals]}]
FullSimplify[ES2P[\[Theta],\[Phi]] .Conjugate[ES2M[\[Theta],\[Phi]]], {Element[\[Theta], Reals],
  Element[\[Phi], Reals]}]


ProjectorES2M[\[Theta]_,\[Phi]_] := FullSimplify[DyadicProductVec[ES2M[\[Theta],\[Phi]]], {Element[\[Theta], Reals],
  Element[\[Phi], Reals]}]
ProjectorES2P[\[Theta]_,\[Phi]_] := FullSimplify[DyadicProductVec[ES2P[\[Theta],\[Phi]]], {Element[\[Theta], Reals],
  Element[\[Phi], Reals]}]

 ProjectorES2M[\[Theta],\[Phi]] //MatrixForm
 ProjectorES2P[\[Theta],\[Phi]] //MatrixForm


(* verification of spectral form *)

FullSimplify[(-1/2)ProjectorES2M[\[Theta],\[Phi]] + (1/2)ProjectorES2P[\[Theta],\[Phi]], {Element[\[Theta], Reals],
  Element[\[Phi], Reals]}]


SingleParticleSpinOneHalfeObservable[x_, p_] :=   FullSimplify[(1/2) (IdentityMatrix[2] + vecsig[1, x, p])] ;

SingleParticleSpinOneHalfeObservable[\[Theta], \[Phi]] // MatrixForm

Eigensystem[FullSimplify[SingleParticleSpinOneHalfeObservable[x, p]]]


(*Definition of single operators for occurrence of spin up*)

SingleParticleProjector2first[x_, p_, pm_] :=   MyTensorProduct[1/2 (IdentityMatrix[2] + pm*vecsig[1, x, p]),  IdentityMatrix[2]]

SingleParticleProjector2second[x_, p_, pm_] :=  MyTensorProduct[IdentityMatrix[2], 1/2 (IdentityMatrix[2] + pm*vecsig[1, x, p])]



(*Definition of two-particle joint operator for occurrence of spin up \
and down*)

JointProjector2[x1_, x2_, p1_, p2_, pm1_, pm2_] :=  MyTensorProduct[1/2 (IdentityMatrix[2] + pm1*vecsig[1, x1, p1]),  1/2 (IdentityMatrix[2] + pm2*vecsig[1, x2, p2])]


(*Definition of probabilities*)


(*Probability of concurrence of two equal events for two-particle \
probability in singlet Bell state for occurrence of spin up*)

JointProb2s[x1_, x2_, p1_, p2_, pm1_, pm2_] :=
 FullSimplify[
  Tr[DyadicProductVec[psi2s].JointProjector2[x1, x2, p1, p2, pm1,
     pm2]]]

JointProb2s[x1, x2, p1, p2, pm1, pm2]

JointProb2s[x1, x2, p1, p2, pm1, pm2] // TeXForm

(*sum of joint probabilities add up to one*)

FullSimplify[
 Sum[JointProb2s[x1, x2, p1, p2, pm1, pm2], {pm1, -1, 1, 2}, {pm2, -1,
    1, 2}]]

(*Probability of concurrence of two equal events*)

P2Es[x1_, x2_, p1_, p2_] =
  FullSimplify[
   Sum[UnitStep[pm1*pm2]*
     JointProb2s[x1, x2, p1, p2, pm1, pm2], {pm1, -1, 1, 2}, {pm2, -1,
      1, 2}]];

P2Es[x1, x2, p1, p2]

(*Probability of concurrence of two non-equal events*)

P2NEs[x1_, x2_, p1_, p2_] =
  FullSimplify[
   Sum[UnitStep[-pm1*pm2]*
     JointProb2s[x1, x2, p1, p2, pm1, pm2], {pm1, -1, 1, 2}, {pm2, -1,
      1, 2}]];

P2NEs[x1, x2, p1, p2]

(*Expectation function*)

Expectation2s[x1_, x2_, p1_, p2_] =
 FullSimplify[P2Es[x1, x2, p1, p2] - P2NEs[x1, x2, p1, p2]]


(* ~~~~~~~~~~~~~~~~~~~~~~~ Min-Max calculation of the quantum correlation function ~~~~~~~~~~~~~~~~~~~~~~~ *)

JointExpectation2[t1_, t2_, p1_, p2_] :=  MyTensorProduct[ 2 * S2[t1, p1] , 2 * S2[t2, p2] ]


FullSimplify[
 Eigensystem[
 JointExpectation2[t1 , t2 , p1 , p2 ]  ]]    // MatrixForm

FullSimplify[
 Eigensystem[
 DyadicProductVec[psi2s]. JointExpectation2[t1, t2, p1 , p2 ] . DyadicProductVec[psi2s]  ]]   // MatrixForm

FullSimplify[
 Eigensystem[
 JointExpectation2[Pi/2 , Pi/2 , p1 , p2 ]  ]]        // MatrixForm

FullSimplify[
  Eigensystem[
   DyadicProductVec[psi2s].JointExpectation2[Pi/2, Pi/2, p1,  p2 ].DyadicProductVec[psi2s]]] // MatrixForm


psi2mp = (1/Sqrt[2])*(TensorProductVec[vp, vm] +
    TensorProductVec[vm, vp])

psi2mm = (1/Sqrt[2])*(TensorProductVec[vp, vp] -
    TensorProductVec[vm, vm])

psi2pp = (1/Sqrt[2])*(TensorProductVec[vp, vp] +
    TensorProductVec[vm, vm])


FullSimplify[
  Eigensystem[
   DyadicProductVec[psi2mp].JointExpectation2[Pi/2, Pi/2, p1,
     p2].DyadicProductVec[psi2mp]]] // MatrixForm

FullSimplify[
  Eigensystem[
   DyadicProductVec[psi2mm].JointExpectation2[Pi/2, Pi/2, p1,
     p2].DyadicProductVec[psi2mm]]] // MatrixForm

FullSimplify[
  Eigensystem[
   DyadicProductVec[psi2pp].JointExpectation2[Pi/2, Pi/2, p1,
     p2].DyadicProductVec[psi2pp]]] // MatrixForm

(* ~~~~~~~~~~~~~~~~~~~~~~~ Min-Max calculation of the Tsirelson bound ~~~~~~~~~~~~~~~~~~~~~~~ *)

JointProjector2Red[ p1_, p2_, pm1_, pm2_] :=  JointProjector2[ Pi/2 , Pi/2 , p1, p2, pm1, pm2]

FullSimplify[ JointProjector2Red[ p1 , p2 , pm1 , pm2 ]]

(* ~~~~~~~~~~~~~~~~~~~~~~~~~ plausibility check *)

JointProb2sRed[p1_, p2_, pm1_, pm2_] :=
 FullSimplify[
  Tr[DyadicProductVec[psi2s].JointProjector2Red[p1, p2, pm1, pm2]]]

JointProb2sRed[p1, p2, pm1, pm2]

FullSimplify[
 JointProb2sRed[p1, p2, 1, 1] + JointProb2sRed[p1, p2, -1, -1] -
  JointProb2sRed[p1, p2, -1, 1] - JointProb2sRed[p1, p2, 1, -1]]

(* ~~~~~~~~~~~~~~~~~~~~~~~~~ end plausibility check *)


TwoParticleExpectationsRed[ p1_, p2_] := JointProjector2Red[ p1, p2, 1, 1]  + JointProjector2Red[ p1, p2, -1, -1] -
                                         JointProjector2Red[ p1, p2, -1, 1] - JointProjector2Red[ p1, p2, 1, -1]


(* ~~~~~~~~~~~~~~~~~~~~~~~~~ plausibility check *)

FullSimplify[ Tr[DyadicProductVec[psi2s].TwoParticleExpectationsRed[A1, B1]] ]

(* ~~~~~~~~~~~~~~~~~~~~~~~~~ end plausibility check *)

TwoParticleExpectationsRed[A1, B1] // MatrixForm
TwoParticleExpectationsRed[A1, B1] // TeXForm

Eigenvalues[
 ComplexExpand[
  TwoParticleExpectationsRed[A1, B1] +
   TwoParticleExpectationsRed[A2, B1] +
   TwoParticleExpectationsRed[A1, B2] -
   TwoParticleExpectationsRed[A2, B2] ]]

FullSimplify[
 Eigenvalues[
  ComplexExpand[
   TwoParticleExpectationsRed[A1, B1] +
    TwoParticleExpectationsRed[A2, B1] +
    TwoParticleExpectationsRed[A1, B2] -
    TwoParticleExpectationsRed[A2, B2] ]]]


FullSimplify[
   TwoParticleExpectationsRed[A1, B1] +
    TwoParticleExpectationsRed[A2, B1] +
    TwoParticleExpectationsRed[A1, B2] -
    TwoParticleExpectationsRed[A2, B2] ]

(*  observables along psi_singlet *)


Eigenvalues[
 ComplexExpand[
  DyadicProductVec[
    psi2s].(TwoParticleExpectationsRed[A1, B1] +
     TwoParticleExpectationsRed[A2, B1] +
     TwoParticleExpectationsRed[A1, B2] -
     TwoParticleExpectationsRed[A2, B2]).DyadicProductVec[psi2s]]]

FullSimplify[
 TrigExpand[
  Eigenvalues[
   ComplexExpand[
    DyadicProductVec[
      psi2s].(TwoParticleExpectationsRed[0, Pi/4] +
       TwoParticleExpectationsRed[Pi/2, Pi/4] +
       TwoParticleExpectationsRed[0, -Pi/4] -
       TwoParticleExpectationsRed[Pi/2, -Pi/4]).DyadicProductVec[
      psi2s]]]]]

(*  observables along psi_+ *)


Eigenvalues[
 ComplexExpand[
  DyadicProductVec[
    psi2mp].(TwoParticleExpectationsRed[A1, B1] +
     TwoParticleExpectationsRed[A2, B1] +
     TwoParticleExpectationsRed[A1, B2] -
     TwoParticleExpectationsRed[A2, B2]).DyadicProductVec[psi2mp]]]

FullSimplify[
 TrigExpand[
  Eigenvalues[
   ComplexExpand[
    DyadicProductVec[
      psi2mp].(TwoParticleExpectationsRed[0, Pi/4] +
       TwoParticleExpectationsRed[Pi/2, Pi/4] +
       TwoParticleExpectationsRed[0, -Pi/4] -
       TwoParticleExpectationsRed[Pi/2, -Pi/4]).DyadicProductVec[
      psi2mp]]]]]

(*** observables along phi_+ ***)


Eigenvalues[
 ComplexExpand[
  DyadicProductVec[
    psi2mm].(TwoParticleExpectationsRed[A1, B1] +
     TwoParticleExpectationsRed[A2, B1] +
     TwoParticleExpectationsRed[A1, B2] -
     TwoParticleExpectationsRed[A2, B2]).DyadicProductVec[psi2mm]]]

FullSimplify[
 TrigExpand[
  Eigenvalues[
   ComplexExpand[
    DyadicProductVec[
      psi2mm].(TwoParticleExpectationsRed[0, -Pi/4] +
       TwoParticleExpectationsRed[Pi/2, -Pi/4] +
       TwoParticleExpectationsRed[0, Pi/4] -
       TwoParticleExpectationsRed[Pi/2, Pi/4]).DyadicProductVec[
      psi2mm]]]]]


(*** observables along phi_+ ***)


Eigenvalues[
 ComplexExpand[
  DyadicProductVec[
    psi2pp].(TwoParticleExpectationsRed[A1, B1] +
     TwoParticleExpectationsRed[A2, B1] +
     TwoParticleExpectationsRed[A1, B2] -
     TwoParticleExpectationsRed[A2, B2]).DyadicProductVec[psi2pp]]]

FullSimplify[
 TrigExpand[
  Eigenvalues[
   ComplexExpand[
    DyadicProductVec[
      psi2pp].(TwoParticleExpectationsRed[0, -Pi/4] +
       TwoParticleExpectationsRed[Pi/2, -Pi/4] +
       TwoParticleExpectationsRed[0, Pi/4] -
       TwoParticleExpectationsRed[Pi/2, Pi/4]).DyadicProductVec[
      psi2pp]]]]]

\end{lstlisting}  }


\subsubsection{Min-max calculation for the quantum bounds of two three-state particles}

{ \begin{lstlisting}[backgroundcolor=\color{yellow!10},framerule=0pt,breaklines=true, frame=tb]

(* ~~~~~~~~~~~~~~~~~~~~~~~~~~~~~~~~~~~~~~~~~~~~~~~~~~~~~~~~~~~~~~~~~~~~~~~ *)
(* ~~~~~~~~~~~~~~~~~~Start Mathematica Code~~~~~~~~~~~~~~~~~~~~~~~~~~~~~~~ *)
(* ~~~~~~~~~~~~~~~~~~~~~~~~~~~~~~~~~~~~~~~~~~~~~~~~~~~~~~~~~~~~~~~~~~~~~~~ *)

(* old stuff

<<Algebra`ReIm`

Normalize[z_]:= z/Sqrt[z.Conjugate[z]];    *)

(*Definition of "my" Tensor Product*)
(*a,b are nxn and mxm-matrices*)

MyTensorProduct[a_, b_] :=
  Table[
   a[[Ceiling[s/Length[b]], Ceiling[t/Length[b]]]]*
    b[[s - Floor[(s - 1)/Length[b]]*Length[b],
      t - Floor[(t - 1)/Length[b]]*Length[b]]], {s, 1,
    Length[a]*Length[b]}, {t, 1, Length[a]*Length[b]}];


(*Definition of the Tensor Product between two vectors*)

TensorProductVec[x_, y_] :=
  Flatten[Table[
    x[[i]] y[[j]], {i, 1, Length[x]}, {j, 1, Length[y]}]];


(*Definition of the Dyadic Product*)

DyadicProductVec[x_] :=
  Table[x[[i]] Conjugate[x[[j]]], {i, 1, Length[x]}, {j, 1,
    Length[x]}];

(*Definition of the sigma matrices*)


vecsig[r_, tt_, p_] :=
 r*{{Cos[tt], Sin[tt] Exp[-I p]}, {Sin[tt] Exp[I p], -Cos[tt]}}

(*Definition of some vectors*)

BellBasis = (1/Sqrt[2]) {{1, 0, 0, 1}, {0, 1, 1, 0}, {0, 1, -1,
     0}, {1, 0, 0, -1}};

Basis = {{1, 0, 0, 0}, {0, 1, 0, 0}, {0, 0, 1, 0}, {0, 0, 0, 1}};


(*~~~~~~~~~~~~~~~~~~~~~~~~~  3  State System ~~~~~~~~~~~~~~~~~~~~~~~~~~~~~~~~~~~~~~~

% ~~~~~~~~~~~~~~~   2 x 3
% ~~~~~~~~~~~~~~~   2 x 3
% ~~~~~~~~~~~~~~~   2 x 3
% ~~~~~~~~~~~~~~~   2 x 3
% ~~~~~~~~~~~~~~~   2 x 3
% ~~~~~~~~~~~~~~~   2 x 3
% ~~~~~~~~~~~~~~~   2 x 3
% ~~~~~~~~~~~~~~~   2 x 3

*)



(*Definition of operators*)

(* Definition of one-particle operator *)

M3X = (1/Sqrt[2]) {{0, 1, 0}, {1, 0, 1},{0, 1, 0}};
M3Y = (1/Sqrt[2]) {{0, -I, 0}, {I, 0, -I}, {0, I, 0}};
M3Z =  {{1, 0, 0}, {0, 0, 0},{0, 0, -1}};


Eigenvectors[M3X]
Eigenvectors[M3Y]
Eigenvectors[M3Z]

S3[t_, p_] := M3X *Sin[t] Cos[p] + M3Y *Sin[t] Sin[p] + M3Z *Cos[t]

FullSimplify[S3[\[Theta], \[Phi]]] // MatrixForm

FullSimplify[ComplexExpand[S3[Pi/2, 0]]] // MatrixForm
FullSimplify[ComplexExpand[S3[Pi/2, Pi/2]]] // MatrixForm
FullSimplify[ComplexExpand[S3[0, 0]]] // MatrixForm

Assuming[{0 <= \[Theta] <= Pi, 0 <= \[Phi] <= 2 Pi}, FullSimplify[Eigensystem[S3[\[Theta], \[Phi]]], {Element[\[Theta], Reals],
  Element[\[Phi], Reals]}]]



FullSimplify[ComplexExpand[
 Normalize[
  Eigenvectors[S3[\[Theta], \[Phi]]][[1]]], {Element[\[Theta], Reals],
   Element[\[Phi], Reals]}]]

ES3M[\[Theta]_,\[Phi]_] := FullSimplify[ ComplexExpand[
 Normalize[
  Eigenvectors[S3[\[Theta], \[Phi]]][[1]]]*E^(I \[Phi])  , {Element[\[Theta], Reals], Element[\[Phi], Reals]}]]

ES3M[\[Theta],\[Phi]]


ES3P[\[Theta]_,\[Phi]_] := FullSimplify[ComplexExpand[
 Normalize[
  Eigenvectors[S3[\[Theta], \[Phi]]][[2]]]*E^(I \[Phi])  , {Element[\[Theta], Reals], Element[\[Phi], Reals]}]]

ES3P[\[Theta],\[Phi]]

ES30[\[Theta]_,\[Phi]_] := FullSimplify[ComplexExpand[
 Normalize[
  Eigenvectors[S3[\[Theta], \[Phi]]][[3]]]*E^(I \[Phi])  , {Element[\[Theta], Reals], Element[\[Phi], Reals]}]]

ES30[\[Theta],\[Phi]]

FullSimplify[ES3M[\[Theta],\[Phi]] .Conjugate[ES3M [\[Theta],\[Phi]]], {Element[\[Theta], Reals],
  Element[\[Phi], Reals]}]
FullSimplify[ES3P[\[Theta],\[Phi]] .Conjugate[ES3P [\[Theta],\[Phi]]], {Element[\[Theta], Reals],
  Element[\[Phi], Reals]}]
FullSimplify[ES30[\[Theta],\[Phi]] .Conjugate[ES30 [\[Theta],\[Phi]]], {Element[\[Theta], Reals],
  Element[\[Phi], Reals]}]
FullSimplify[ES3P[\[Theta],\[Phi]] .Conjugate[ES3M[\[Theta],\[Phi]]], {Element[\[Theta], Reals],
  Element[\[Phi], Reals]}]
FullSimplify[ES3P[\[Theta],\[Phi]] .Conjugate[ES30[\[Theta],\[Phi]]], {Element[\[Theta], Reals],
  Element[\[Phi], Reals]}]
FullSimplify[ES30[\[Theta],\[Phi]] .Conjugate[ES3M[\[Theta],\[Phi]]], {Element[\[Theta], Reals],
  Element[\[Phi], Reals]}]


ProjectorES30[\[Theta]_,\[Phi]_] := FullSimplify[ComplexExpand[DyadicProductVec[ES30[\[Theta],\[Phi]]], {Element[\[Theta], Reals],
  Element[\[Phi], Reals]}]]
ProjectorES3M[\[Theta]_,\[Phi]_] := FullSimplify[ComplexExpand[DyadicProductVec[ES3M[\[Theta],\[Phi]]], {Element[\[Theta], Reals],
  Element[\[Phi], Reals]}]]
ProjectorES3P[\[Theta]_,\[Phi]_] := FullSimplify[ComplexExpand[DyadicProductVec[ES3P[\[Theta],\[Phi]]], {Element[\[Theta], Reals],
  Element[\[Phi], Reals]}]]

 ProjectorES30[\[Theta],\[Phi]] //MatrixForm
 ProjectorES3M[\[Theta],\[Phi]] //MatrixForm
 ProjectorES3P[\[Theta],\[Phi]] //MatrixForm

ProjectorES30[\[Theta], \[Phi]] // MatrixForm // TeXForm
ProjectorES3M[\[Theta], \[Phi]] // MatrixForm // TeXForm
ProjectorES3P[\[Theta], \[Phi]] // MatrixForm // TeXForm

(* verification of spectral form *)

FullSimplify[0 * ProjectorES30[\[Theta],\[Phi]] +  (-1) * ProjectorES3M[\[Theta],\[Phi]] + (+1) * ProjectorES3P[\[Theta],\[Phi]], {Element[\[Theta], Reals],
  Element[\[Phi], Reals]}] //MatrixForm


(*  ~~~~~~~~~~~~~~~~~~~ general operator ~~~~~~~~~~~~~~~~~~~~~~~  *)

Operator3GEN[\[Theta]_,\[Phi]_] := FullSimplify[LM * ProjectorES3M[\[Theta],\[Phi]] + L0 * ProjectorES30[\[Theta],\[Phi]] + LP * ProjectorES3P[\[Theta],\[Phi]], {Element[\[Theta], Reals], Element[\[Phi], Reals]}];

Operator3GEN[\[Theta],\[Phi]]

JointProjector3GEN[x1_, x2_, p1_, p2_] :=  MyTensorProduct[Operator3GEN[x1,p1],Operator3GEN[x2,p2]];

v3p = {1,0,0};
v30 = {0,1,0};
v3m = {0,0,1};

psi3s = (1/Sqrt[3])*(-TensorProductVec[v30, v30] + TensorProductVec[v3m, v3p] + TensorProductVec[v3p, v3m])


Expectation3sGEN[x1_, x2_, p1_, p2_] := FullSimplify[ Tr[DyadicProductVec[psi3s].JointProjector3GEN[x1, x2, p1, p2]]];

Expectation3sGEN[x1, x2, p1, p2]


Ex3[LM_,L0_,LP_,x1_,x2_,p1_,p2_]:=FullSimplify[1/192 (24 L0^2 + 40 L0 (LM + LP) + 22 (LM + LP)^2 -
   32 (LM - LP)^2 Cos[x1] Cos[x2] +
   2 (-2 L0 + LM + LP)^2 Cos[
     2 x2] ((3 + Cos[2 (p1 - p2)]) Cos[2 x1] + 2 Sin[p1 - p2]^2) +
   2 (-2 L0 + LM + LP)^2 (Cos[2 (p1 - p2)] +
      2 Cos[2 x1] Sin[p1 - p2]^2) -
   32 (LM - LP)^2 Cos[p1 - p2] Sin[x1] Sin[x2] +
   8 (-2 L0 + LM + LP)^2 Cos[p1 - p2] Sin[2 x1] Sin[2 x2])];

Ex3[-1,0,1,x1,x2,p1,p2]



(* ~~~~~~~~~~~ natural spin observables ~~~~~~~~~~~~~~~~~~~~~ *)



JointProjector3NAT[x1_, x2_, p1_, p2_] :=  MyTensorProduct[S3[x1,p1],S3[x2,p2]];

Expectation3sNAT[x1_, x2_, p1_, p2_] := FullSimplify[ Tr[DyadicProductVec[psi3s].JointProjector3NAT[x1, x2, p1, p2]]];

Expectation3sNAT[x1, x2, p1, p2]



(* ~~~~~~~~~~~ Kochen-Specker observables ~~~~~~~~~~~~~~~~~~~~~ *)


(*
S3[t_, p_] := M3X *Sin[t] Cos[p] + M3Y *Sin[t] Sin[p] + M3Z *Cos[t]

MM3X[ \[Alpha]_ ] := FullSimplify[S3[Pi/2, \[Alpha]]];
MM3Y[ \[Alpha]_ ] := FullSimplify[S3[Pi/2, \[Alpha]+Pi/2]];
MM3Z[ \[Alpha]_ ] := FullSimplify[S3[0, 0]];

SKS[ \[Alpha]_ ] := FullSimplify[ MM3X[\[Alpha]].MM3X[\[Alpha]] + MM3Y[\[Alpha]].MM3Y[\[Alpha]] + MM3Z[\[Alpha]].MM3Z[\[Alpha]] ];

FullSimplify[SKS[ \[Alpha] ]] // MatrixForm

FullSimplify[ComplexExpand[SKS[ 0]]] // MatrixForm
FullSimplify[ComplexExpand[SKS[ Pi/2]]] // MatrixForm

Assuming[{0 <= \[Theta] <= Pi, 0 <= \[Phi] <= 2 Pi}, FullSimplify[Eigensystem[SKS[ \[Alpha] ]], {Element[\[Alpha], Reals]}]]

*)

Ex3[1, 0, 1, \[Theta]1, \[Theta]2, \[CurlyPhi]1, \[CurlyPhi]2]

Ex3[0, 1, 0, \[Theta]1, \[Theta]2, \[CurlyPhi]1, \[CurlyPhi]2]

Ex3[1, 0, 1, Pi/2, Pi/2, \[CurlyPhi]1, \[CurlyPhi]2]

Ex3[0, 1, 0, Pi/2, Pi/2, \[CurlyPhi]1, \[CurlyPhi]2]

Ex3[1, 0, 1, \[Theta]1, \[Theta]2, 0, 0]

Ex3[0, 1, 0, \[Theta]1, \[Theta]2, 0, 0]




(* min-max computation *)

(* define dichotomic operator based on spin-1  expectation value , take \[Phi] = Pi/2 *)

(* old, invalid parameterization
A[ \[Theta]1_ , \[Theta]2_  ] :=   MyTensorProduct[ S3[\[Theta]1, Pi/2] ,  S3[\[Theta]2, Pi/2] ]

(* Form the Klyachko-Can-Biniciogolu-Shumovsky operator  *)

T[\[Theta]1_, \[Theta]3_, \[Theta]5_, \[Theta]7_, \[Theta]9_] :=
 A[\[Theta]1,\[Theta]3] + A[\[Theta]3,\[Theta]5] +
  A[\[Theta]5,\[Theta]7] + A[\[Theta]7,\[Theta]9] +
  A[\[Theta]9,\[Theta]1]


FullSimplify[
 Eigenvalues[
  FullSimplify[
  T[\[Theta]1, \[Theta]3, \[Theta]5, \[Theta]7, \[Theta]9]]]]

FullSimplify[
 Eigenvalues[
  T[2 Pi/5 , 4 Pi/5, 6 Pi/5, 8 Pi/5, 2 Pi]]]

 *)

A[ \[Theta]1_ , \[Theta]2_ ,\[CurlyPhi]1_, \[CurlyPhi]2_  ] :=   MyTensorProduct[ S3[\[Theta]1, \[CurlyPhi]1] ,  S3[\[Theta]2, \[CurlyPhi]2] ]

(* Form the Klyachko-Can-Biniciogolu-Shumovsky operator  *)

T[\[Theta]1_, \[Theta]3_, \[Theta]5_, \[Theta]7_, \[Theta]9_,\[CurlyPhi]1_, \[CurlyPhi]3_,\[CurlyPhi]5_, \[CurlyPhi]7_,\[CurlyPhi]9_] :=
 A[\[Theta]1,\[Theta]3, \[CurlyPhi]1 ,\[CurlyPhi]3] + A[\[Theta]3,\[Theta]5,\[CurlyPhi]3,\[CurlyPhi]5] +
  A[\[Theta]5,\[Theta]7,\[CurlyPhi]5,\[CurlyPhi]7] + A[\[Theta]7,\[Theta]9,\[CurlyPhi]7,\[CurlyPhi]9] +
  A[\[Theta]9,\[Theta]1,\[CurlyPhi]9,\[CurlyPhi]1]


A1   =  CoordinateTransformData[ "Cartesian" -> "Spherical", "Mapping", {1,0,0  }] ;
A2   =  CoordinateTransformData[ "Cartesian" -> "Spherical", "Mapping", {0,1,0  }] ;
A3   =  (* CoordinateTransformData[ "Cartesian" -> "Spherical", "Mapping", {0,0,1  }] *)  {1,0,Pi/2} ;
A4   =  CoordinateTransformData[ "Cartesian" -> "Spherical", "Mapping", {1,-1,0 }] ;
A5   =  CoordinateTransformData[ "Cartesian" -> "Spherical", "Mapping", {1,1,0  }] ;
A6   =  CoordinateTransformData[ "Cartesian" -> "Spherical", "Mapping", {1,-1,2 }] ;
A7   =  CoordinateTransformData[ "Cartesian" -> "Spherical", "Mapping", {-1,1,1 }] ;
A8   =  CoordinateTransformData[ "Cartesian" -> "Spherical", "Mapping", {2,1,1  }] ;
A9   =  CoordinateTransformData[ "Cartesian" -> "Spherical", "Mapping", {0,1,-1 }] ;
A10  =  CoordinateTransformData[ "Cartesian" -> "Spherical", "Mapping", {0,1,1  }] ;



FullSimplify[
 Eigenvalues[
  FullSimplify[
  T[ A1[[2]],  A3[[2]],  A5[[2]],  A7[[2]],  A9[[2]] , A1[[3]],  A3[[3]],  A5[[3]],  A7[[3]],  A9[[3]]]]]]



{A1,
A2 ,
A3 ,
A4 ,
A5 ,
A6 ,
A7 ,
A8 ,
A9 ,
A10} //TexForm

\end{lstlisting}  }




\subsubsection{Min-max calculation for two four-state particles}

{ \begin{lstlisting}[backgroundcolor=\color{yellow!10},framerule=0pt,breaklines=true, frame=tb]

(* ~~~~~~~~~~~~~~~~~~~~~~~~~~~~~~~~~~~~~~~~~~~~~~~~~~~~~~~~~~~~~~~~~~~~~~~ *)
(* ~~~~~~~~~~~~~~~~~~Start Mathematica Code~~~~~~~~~~~~~~~~~~~~~~~~~~~~~~~ *)
(* ~~~~~~~~~~~~~~~~~~~~~~~~~~~~~~~~~~~~~~~~~~~~~~~~~~~~~~~~~~~~~~~~~~~~~~~ *)

(* old stuff

<<Algebra`ReIm`

Normalize[z_]:= z/Sqrt[z.Conjugate[z]];    *)

(*Definition of "my" Tensor Product*)
(*a,b are nxn and mxm-matrices*)

MyTensorProduct[a_, b_] :=
  Table[
   a[[Ceiling[s/Length[b]], Ceiling[t/Length[b]]]]*
    b[[s - Floor[(s - 1)/Length[b]]*Length[b],
      t - Floor[(t - 1)/Length[b]]*Length[b]]], {s, 1,
    Length[a]*Length[b]}, {t, 1, Length[a]*Length[b]}];


(*Definition of the Tensor Product between two vectors*)

TensorProductVec[x_, y_] :=
  Flatten[Table[
    x[[i]] y[[j]], {i, 1, Length[x]}, {j, 1, Length[y]}]];


(*Definition of the Dyadic Product*)

DyadicProductVec[x_] :=
  Table[x[[i]] Conjugate[x[[j]]], {i, 1, Length[x]}, {j, 1,
    Length[x]}];

(*Definition of the sigma matrices*)


vecsig[r_, tt_, p_] :=
 r*{{Cos[tt], Sin[tt] Exp[-I p]}, {Sin[tt] Exp[I p], -Cos[tt]}}

(*Definition of some vectors*)

BellBasis = (1/Sqrt[2]) {{1, 0, 0, 1}, {0, 1, 1, 0}, {0, 1, -1,
     0}, {1, 0, 0, -1}};

Basis = {{1, 0, 0, 0}, {0, 1, 0, 0}, {0, 0, 1, 0}, {0, 0, 0, 1}};


(*~~~~~~~~~~~~~~~~~~~~~~~~~  4  State System ~~~~~~~~~~~~~~~~~~~~~~~~~~~~~~~~~~~~~~~

% ~~~~~~~~~~~~~~~   2 x 4
% ~~~~~~~~~~~~~~~   2 x 4
% ~~~~~~~~~~~~~~~   2 x 4
% ~~~~~~~~~~~~~~~   2 x 4
% ~~~~~~~~~~~~~~~   2 x 4
% ~~~~~~~~~~~~~~~   2 x 4
% ~~~~~~~~~~~~~~~   2 x 4
% ~~~~~~~~~~~~~~~   2 x 4

*)





(*Definition of operators*)

(* Definition of one-particle operator *)

M4X = (1/2) {{0,Sqrt[3],0,0 },{Sqrt[3],0,2,0 },{0,2,0,Sqrt[3] },{0,0,Sqrt[3],0 }};
M4Y = (1/2) {{0,-Sqrt[3]I,0,0 },{Sqrt[3]I,0,-2I,0},{0,2I,0,-Sqrt[3]I},{0,0,Sqrt[3]I,0  } };
M4Z = (1/2) {{3,0,0,0 },{0,1,0,0 },{0,0,-1,0},{0,0,0,-3}};

Eigenvectors[M4X]
Eigenvectors[M4Y]
Eigenvectors[M4Z]

S4[t_, p_] :=  FullSimplify[M4X *Sin[t] Cos[p] + M4Y *Sin[t] Sin[p] + M4Z *Cos[t]];



(*  ~~~~~~~~~~~~~~~~~~~ general operator ~~~~~~~~~~~~~~~~~~~~~~~  *)

LM32 =-3/2;
LM12 =-1/2;
LP32 =3/2;
LP12 =1/2;

ES4M32[\[Theta]_, \[Phi]_] :=   FullSimplify[   Assuming[{0 < \[Theta] < Pi, 0 <= \[Phi] <= 2 Pi},    Normalize[     Eigenvectors[S4[\[Theta], \[Phi]]][[1]]]], {Element[\[Theta],     Reals], Element[\[Phi], Reals]}];
ES4P32[\[Theta]_, \[Phi]_] :=  FullSimplify[   Assuming[{0 < \[Theta] < Pi, 0 <= \[Phi] <= 2 Pi},    Normalize[     Eigenvectors[S4[\[Theta], \[Phi]]][[2]]]], {Element[\[Theta],     Reals], Element[\[Phi], Reals]}];
ES4M12[\[Theta]_, \[Phi]_] :=  FullSimplify[   Assuming[{0 < \[Theta] < Pi, 0 <= \[Phi] <= 2 Pi},    Normalize[     Eigenvectors[S4[\[Theta], \[Phi]]][[3]]]], {Element[\[Theta],     Reals], Element[\[Phi], Reals]}];
ES4P12[\[Theta]_, \[Phi]_] :=  FullSimplify[   Assuming[{0 < \[Theta] < Pi, 0 <= \[Phi] <= 2 Pi},    Normalize[     Eigenvectors[S4[\[Theta], \[Phi]]][[4]]]], {Element[\[Theta],     Reals], Element[\[Phi], Reals]}];



JointProjector4GEN[x1_, x2_, p1_, p2_] :=  TensorProduct[S4[x1,p1],S4[x2,p2]];

v4P32 = ES4P32[0,0]
v4P12 = ES4P12[0,0]
v4M12 = ES4M12[0,0]
v4M32 = ES4M32[0,0]



psi4s = (1/2)*(TensorProductVec[v4P32, v4M32]-TensorProductVec[v4M32, v4P32] - TensorProductVec[v4P12 , v4M12 ] + TensorProductVec[v4M12 , v4P12 ])


Expectation4sGEN[x1_, x2_, p1_, p2_] := Tr[DyadicProductVec[psi4s].JointProjector4GEN[x1, x2, p1, p2]];

FullSimplify[Expectation4sGEN[x1, x2, p1, p2]]


(* ~~~~~~~~ general case ~~~~~~~~~ *)

EPPMM1[L4M32_ , L4M12_ , L4P12_ , L4P32_ ,  \[Theta]_, \[Phi]_] :=   Assuming[{0 < \[Theta] < Pi, 0 <= \[Phi] <= 2 Pi}, FullSimplify[
L4M32 * Assuming[{0 < \[Theta] < Pi, 0 <= \[Phi] <= 2 Pi},
 FullSimplify[
  DyadicProductVec[
   ES4M32[\[Theta], \[Phi]]], {Element[\[Theta], Reals],
   Element[\[Phi], Reals]}] ]   + L4M12 *    Assuming[{0 < \[Theta] < Pi, 0 <= \[Phi] <= 2 Pi},
 FullSimplify[
  DyadicProductVec[
   ES4M12[\[Theta], \[Phi]]], {Element[\[Theta], Reals],
   Element[\[Phi], Reals]}] ]+
L4P32 * Assuming[{0 < \[Theta] < Pi, 0 <= \[Phi] <= 2 Pi},
 FullSimplify[
  DyadicProductVec[
   ES4P32[\[Theta], \[Phi]]], {Element[\[Theta], Reals],
   Element[\[Phi], Reals]}] ]+
L4P12 * Assuming[{0 < \[Theta] < Pi, 0 <= \[Phi] <= 2 Pi},
 FullSimplify[
  DyadicProductVec[
   ES4P12[\[Theta], \[Phi]]], {Element[\[Theta], Reals],
   Element[\[Phi], Reals]}] ]
]]


EPPMM1[-1,-1,1,1,\[Theta], \[Phi]] //MatrixForm

JointProjector4PPMM1[L4M32_ , L4M12_ , L4P12_ , L4P32_ , x1_, x2_, p1_, p2_] :=  Assuming[{0 < \[Theta] < Pi, 0 <= \[Phi] <= 2 Pi},
 FullSimplify[TensorProduct[EPPMM1[L4M32 , L4M12 , L4P12 , L4P32 , x1,p1],EPPMM1[L4M32 , L4M12 , L4P12 , L4P32 ,x2,p2]], {Element[\[Theta], Reals],
   Element[\[Phi], Reals]}] ];

Expectation4PPMM1[L4M32_ , L4M12_ , L4P12_ , L4P32_ , x1_, x2_, p1_, p2_] := Tr[DyadicProductVec[psi4s].JointProjector4PPMM1[L4M32 , L4M12 , L4P12 , L4P32 ,x1, x2, p1, p2]];

FullSimplify[Expectation4PPMM1[-1,-1,1,1,x1, x2, p1, p2]]

Emmpp[x1_ ]= FullSimplify[Expectation4PPMM1[-1, -1, 1, 1, x1, 0, 0, 0]];
Emppm[x1_ ]= FullSimplify[Expectation4PPMM1[-1, 1, 1, -1, x1, 0, 0, 0]];
Empmp[x1_ ]= FullSimplify[Expectation4PPMM1[-1, 1, -1, 1, x1, 0, 0, 0]];


(*********** minmax calculation  *************)

v12   = Normalize [ { 1,0,0,0     } ]  ;
v18   = Normalize [ { 0,1,0,0     } ]  ;
v17   = Normalize [ { 0,0,1,1     } ] ;
v16   = Normalize [ { 0,0,1,-1   } ]  ;
v67   = Normalize [ { 1,-1,0,0   } ]  ;
v69   = Normalize [ { 1,1,-1,-1  } ]  ;
v56   = Normalize [ { 1,1,1,1    } ]  ;
v59   = Normalize [ { 1,-1,1,-1  } ]  ;
v58   = Normalize [ { 1,0,-1,0   } ]  ;
v45   = Normalize [ { 0,1,0,-1   } ]  ;
v48   = Normalize [ { 1,0,1,0    } ]  ;
v47   = Normalize [ { 1,1,-1,1   } ]  ;
v34   = Normalize [ { -1,1,1,1   } ]  ;
v37   = Normalize [ { 1,1,1,-1   } ]  ;
v39   = Normalize [ { 1,0,0,1    } ]  ;
v23   = Normalize [ { 0,1,-1,0   } ]  ;
v29   = Normalize [ { 0,1,1,0    } ]  ;
v28   = Normalize [ { 0,0,0,1     } ]  ;

A12   = 2 * DyadicProductVec[ v12 ] -  IdentityMatrix[4];
A18   = 2 * DyadicProductVec[ v18 ] -  IdentityMatrix[4];
A17   = 2 * DyadicProductVec[ v17 ] -  IdentityMatrix[4];
A16   = 2 * DyadicProductVec[ v16 ] -  IdentityMatrix[4];
A67   = 2 * DyadicProductVec[ v67 ] -  IdentityMatrix[4];
A69   = 2 * DyadicProductVec[ v69 ] -  IdentityMatrix[4];
A56   = 2 * DyadicProductVec[ v56 ] -  IdentityMatrix[4];
A59   = 2 * DyadicProductVec[ v59 ] -  IdentityMatrix[4];
A58   = 2 * DyadicProductVec[ v58 ] -  IdentityMatrix[4];
A45   = 2 * DyadicProductVec[ v45 ] -  IdentityMatrix[4];
A48   = 2 * DyadicProductVec[ v48 ] -  IdentityMatrix[4];
A47   = 2 * DyadicProductVec[ v47 ] -  IdentityMatrix[4];
A34   = 2 * DyadicProductVec[ v34 ] -  IdentityMatrix[4];
A37   = 2 * DyadicProductVec[ v37 ] -  IdentityMatrix[4];
A39   = 2 * DyadicProductVec[ v39 ] -  IdentityMatrix[4];
A23   = 2 * DyadicProductVec[ v23 ] -  IdentityMatrix[4];
A29   = 2 * DyadicProductVec[ v29 ] -  IdentityMatrix[4];
A28   = 2 * DyadicProductVec[ v28 ] -  IdentityMatrix[4];


T=- MyTensorProduct[ A12, MyTensorProduct[ A16, MyTensorProduct[ A17,  A18]]]-
    MyTensorProduct[ A34, MyTensorProduct[ A45, MyTensorProduct[ A47,  A48]]]-
    MyTensorProduct[ A17, MyTensorProduct[ A37, MyTensorProduct[ A47,  A67]]]-
    MyTensorProduct[ A12, MyTensorProduct[ A23, MyTensorProduct[ A28,  A29]]]-
    MyTensorProduct[ A45, MyTensorProduct[ A56, MyTensorProduct[ A58,  A59]]]-
    MyTensorProduct[ A18, MyTensorProduct[ A28, MyTensorProduct[ A48,  A58]]]-
    MyTensorProduct[ A23, MyTensorProduct[ A34, MyTensorProduct[ A37,  A39]]]-
    MyTensorProduct[ A16, MyTensorProduct[ A56, MyTensorProduct[ A67,  A69]]]-
    MyTensorProduct[ A29, MyTensorProduct[ A39, MyTensorProduct[ A59,  A69]]];


Sort[N[ Eigenvalues[FullSimplify[T]] ]]

~~~~~~~~~~~~~~~ Mathematica responds with

-6.94177, -6.67604, -6.33701, -6.28615, -6.23127, -6.16054, -6.03163, \
-5.96035, -5.93383, -5.84682, -5.73132, -5.69364, -5.56816, -5.51187, \
-5.41033, -5.37887, -5.30655, -5.19379, -5.16625, -5.14571, -5.10303, \
-5.05058, -4.94995, -4.88683, -4.81198, -4.76875, -4.64477, -4.59783, \
-4.51564, -4.46342, -4.44793, -4.36655, -4.33535, -4.26487, -4.24242, \
-4.18346, -4.11958, -4.05858, -4.00766, -3.94818, -3.91915, -3.86835, \
-3.83409, -3.77134, -3.7264, -3.68635, -3.63589, -3.59371, -3.54261, \
-3.48718, -3.47436, -3.4259, -3.35916, -3.35162, -3.29849, -3.24756, \
-3.23809, -3.18265, -3.14344, -3.09402, -3.07889, -3.03559, -3.02288, \
-2.98647, -2.88163, -2.84532, -2.80141, -2.76377, -2.72709, -2.67779, \
-2.65641, -2.64092, -2.5736, -2.53695, -2.48594, -2.46943, -2.42826, \
-2.40909, -2.3199, -2.27146, -2.26781, -2.23017, -2.19853, -2.14537, \
-2.1276, -2.1156, -2.08393, -2.02886, -2.01068, -1.95272, -1.90585, \
-1.8751, -1.81924, -1.80788, -1.77317, -1.71073, -1.67061, -1.61881, \
-1.58689, -1.56025, -1.52167, -1.47029, -1.43804, -1.41839, -1.39628, \
-1.33188, -1.2978, -1.26275, -1.24332, -1.17988, -1.16121, -1.12508, \
-1.06344, -1.04392, -0.981618, -0.9452, -0.93099, -0.902773, \
-0.866424, -0.847618, -0.797269, -0.749678, -0.718776, -0.667079, \
-0.655403, -0.621519, -0.563475, -0.535886, -0.505914, -0.488961, \
-0.477695, -0.438752, -0.413149, -0.385094, -0.329761, -0.313382, \
-0.267465, -0.251247, -0.186771, -0.162663, -0.135313, -0.115949, \
-0.0388241, -0.0285473, 0.0336107, 0.0472502, 0.0664514, 0.0818923, \
0.137393, 0.170784, 0.18296, 0.254586, 0.311604, 0.337846, 0.347853, \
0.351775, 0.395505, 0.422414, 0.481815, 0.515078, 0.57488, 0.600515, \
0.655748, 0.703362, 0.727865, 0.763394, 0.782482, 0.81889, 0.844406, \
0.888659, 0.920904, 1.00356, 1.02312, 1.03976, 1.08469, 1.1021, \
1.11609, 1.14654, 1.20192, 1.22992, 1.28624, 1.29287, 1.32196, \
1.36147, 1.43187, 1.52158, 1.5859, 1.61094, 1.62377, 1.66645, \
1.68222, 1.77266, 1.8082, 1.86793, 1.92219, 1.94603, 1.98741, \
2.04197, 2.06058, 2.12728, 2.16917, 2.20299, 2.20934, 2.2568, \
2.34362, 2.38008, 2.38999, 2.44382, 2.47456, 2.49679, 2.57822, \
2.62572, 2.63375, 2.67809, 2.73929, 2.81403, 2.82569, 2.87209, \
2.94084, 2.94773, 2.99356, 3.03768, 3.0484, 3.09975, 3.2194, 3.26743, \
3.2782, 3.30107, 3.41633, 3.43565, 3.49832, 3.62058, 3.6639, 3.7087, \
3.78394, 3.83644, 3.94999, 3.98744, 4.01948, 4.12536, 4.33452, \
4.37928, 4.42565, 4.47313, 4.53695, 4.71925, 4.84841, 4.90328, \
4.95742, 5.0169, 5.17123, 5.28471, 5.39555, 5.68376, 5.78503, 6.023}

\end{lstlisting}  }

\ifsup

 \bibliography{svozil}
 \bibliographystyle{apsrev}

\end{document}

\fi
