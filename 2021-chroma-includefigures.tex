\documentclass[%
12pt,
prereprint,
%twocolumn,
%superscriptaddress,
%groupedaddress,
%unsortedaddress,
%runinaddress,
%frontmatterverbose,
% preprint,
showpacs,
showkeys,
preprintnumbers,
%nofootinbib,
%nobibnotes,
%bibnotes,
amsmath,amssymb,
aps,
% prl,
pra,
% prb,
% rmp,
%prstab,
%prstper,
longbibliography,
%floatfix,
%lengthcheck,%
notitlepage
]{revtex4-1}

%\usepackage{cdmtcs-pdf}

\usepackage[dvipsnames]{xcolor}

\usepackage{mathptmx}% http://ctan.org/pkg/mathptmx

\usepackage{amssymb,amsthm,amsmath,bm}


\usepackage{float}
\makeatletter
\let\newfloat\newfloat@ltx
\makeatother


\usepackage[english]{babel}
\usepackage[utf8]{inputenc}
%\usepackage[boxed]{algorithm}
%\usepackage{arevmath}     % For math symbols
%\usepackage[noend]{algpseudocode}
%\usepackage{algpseudocode}

\usepackage{algorithm}
\usepackage{algpseudocode}

\usepackage{etoolbox}



\makeatletter
% start with some helper code
% This is the vertical rule that is inserted
\newcommand*{\algrule}[1][\algorithmicindent]{%
  \makebox[#1][l]{%
    \hspace*{.2em}% <------------- This is where the rule starts from
    \vrule height .75\baselineskip depth .25\baselineskip
  }
}

\newcount\ALG@printindent@tempcnta
\def\ALG@printindent{%
    \ifnum \theALG@nested>0% is there anything to print
    \ifx\ALG@text\ALG@x@notext% is this an end group without any text?
    % do nothing
    \else
    \unskip
    % draw a rule for each indent level
    \ALG@printindent@tempcnta=1
    \loop
    \algrule[\csname ALG@ind@\the\ALG@printindent@tempcnta\endcsname]%
    \advance \ALG@printindent@tempcnta 1
    \ifnum \ALG@printindent@tempcnta<\numexpr\theALG@nested+1\relax
    \repeat
    \fi
    \fi
}
% the following line injects our new indent handling code in place of the default spacing
\patchcmd{\ALG@doentity}{\noindent\hskip\ALG@tlm}{\ALG@printindent}{}{\errmessage{failed to patch}}
\patchcmd{\ALG@doentity}{\item[]\nointerlineskip}{}{}{} % no spurious vertical space
% end vertical rule patch for algorithmicx
\makeatother



\usepackage{tikz}
\usetikzlibrary{calc,decorations.pathreplacing,decorations.markings,positioning,shapes,snakes,external}
\tikzexternalize

\usepackage[breaklinks=true,colorlinks=true,anchorcolor=blue,citecolor=blue,filecolor=blue,menucolor=blue,urlcolor=blue,linkcolor=blue]{hyperref}
\usepackage{graphicx}% Include figure files
\usepackage{url}

\usepackage{soul}
\newcommand{\karl}[1]{{\color{blue}#1}}

\definecolor{TITLE}{rgb}{0,0,0}
\definecolor{midblue}{rgb}{0.00,0.0,0.80}
\definecolor{darkblue}{rgb}{0.00,0.00,0.45}
\definecolor{SECTION}{rgb}{0.50,0.00,1.00}
\definecolor{THM}{rgb}{0.8,0,0.1}
\definecolor{SEC}{rgb}{0,0,1}

\newtheorem{theorem}{{\color{THM} Theorem}}%[section]
\newtheorem{procedure}{{\color{THM} Procedure}}%[section]

\newtheorem{lemma}[theorem]{{\color{THM}Lemma}}
\newtheorem{proposition}[theorem]{{\color{THM}Proposition}}
\newtheorem{corollary}[theorem]{{\color{THM}Corollary}}
\newtheorem{conjecture}[theorem]{{\color{THM}Conjecture}}
\theoremstyle{definition}
\newtheorem{definition}[theorem]{{\color{THM}Definition\ }}
\newtheorem{criterion}[theorem]{{\color{THM}Criterion\ }}
\newtheorem{example}[theorem]{{\color{THM}Example}}
\newtheorem{xca}[theorem]{{\color{THM}Exercise}}
\newtheorem{problem}[theorem]{{\color{THM}Problem}}
\newtheorem{remark}[theorem]{{\color{THM}Remark}}

\setcounter{MaxMatrixCols}{20}




\begin{document}

        \title{Noncontextual coloring of orthogonality hypergraphs}



        \author{Mohammad H. Shekarriz}
        \email{mhshekarriz@yazd.ac.ir}
        \affiliation{Department of Mathematics, Yazd University, 89195-741, Yazd, Iran}

        \author{Karl Svozil}
        \email{svozil@tuwien.ac.at}
        \homepage{http://tph.tuwien.ac.at/~svozil}

        \affiliation{Institute for Theoretical Physics,
                Vienna  University of Technology,
                Wiedner Hauptstrasse 8-10/136,
                1040 Vienna,  Austria}


        \date{\today}

        \begin{abstract}
                We discuss chromatic constructions on orthogonality hypergraphs which are classical set representable or have a faithful orthogonal representation. The latter ones have a quantum mechanical realization in terms of intertwined contexts or maximal observables. Structure reconstruction of these hypergraphs from their table of two-valued states is possible for a class of hypergraphs, namely perfectly separable hypergraphs. Some examples from exempt categories that either cannot be reconstructed by two-valued states or whose set of two-valued states does not yield a coloring are presented.
        \end{abstract}

        \keywords{%Quantum mechanics, Gleason theorem, Born rule,
                Kochen-Specker theorem, gadget hypergraphs, hypergraph reconstruction, (perfectly) separable hypergraphs}
        \pacs{03.65.Ca, 02.50.-r, 02.10.-v, 03.65.Aa, 03.67.Ac, 03.65.Ud}

        \maketitle


        \section{Colorings of quantum contexts}

        Graph and hypergraph representations of (intertwining) configurations of physical observables facilitate a reliable,
        compact identification and recognition of important properties of these collections.
        For instance, proofs of (extended) Kochen-Specker theorem, in its various forms~\cite{kochen1,cabello-96,hru-pit-2003,2015-AnalyticKS},
        can be structurally represented compactly by (hyper)graphs.

        The chromatic properties of such (hyper)graphs are directly related to their (non)classical aspects.
        For instance, in quantum logic~\cite{birkhoff-36}
        certain colorings---with the number of colors equal to the clique number that can be identified with the Hilbert space dimension---can be ``reduced'' to or ``collapsed'' into two-valued dispersionless states, which in turn are interpretable as noncontextual classical truth assignments of the elementary propositions represented by vertices of such graphs. If the chromatic number exceeds  the Hilbert space dimension then no uniform, global two-valued state and also no corresponding uniform, global classical truth value assignment exists.
        Thereby, the chromatic number signifies important properties and (non)existence of classical interpretations of the aforementioned
        configurations of observables; in particular, arrangements of observables realizable by quantum means.

        One example of the usefulness of these reduction techniques is the construction of a dense yet discontinuous coloring
        of the rational sphere~\cite{meyer:99}---with the resulting states and propositions
        represented by unit vectors with rational coordinates---based on Pythagorean triples~\cite{havlicek-2000}.
        Thereby, a two-valued dispersionless state can be straightforwardly obtained by identifying all but one colors
        of colorings of the sphere~\cite{godsil-zaks}.

        The usefulness of colorings is not restricted to reductions of the colors but can be extended to certain properties of operators.
        The chromaticity of observables within a given context---which in quantum mechanics can be essentially identified with an orthonormal basis
        of the associated finite-dimensional Hilbert space---can be associated
        with particular types of spectral forms of maximal operators~\cite[\S~84]{halmos-vs} formed by the
        elements of the contexts.
        Thereby, unit vectors are interpreted as the orthogonal projection operators formed by the respective dyadic products.
        Those orthogonal projection operators within any given context are mutually orthogonal and can be
        inserted into the spectral sum of a (normal, or, more specifically) self-adjoint operator.
        The respective (real) eigenvalues can be identified with some real-valued encoding of the color or chromaticity of the element of the context.
        Thereby a uniform way of defining (intertwining) context identifiable with maximal quantum operators representing quantum observables is obtained.

        If the chromatic number equals the dimension of the Hilbert
        space---which is, at the same time, identical to the clique number of the graph---this yields
        a uniform way of defining (maximal) observables even among complementary observables and physical properties
        (associated with nonidentical contexts).
        However, if the chromatic number exceeds the dimension of Hilbert space,
        the construction yields entirely new potential features of observables:
        if one insists in the uniform global simultaneous (yet counterfactual~\cite{specker-60}) existence of observables
        within such structures, then consistency dictates the abandonment of uniform eigenvalues associated with observables within such contexts.
        This holds even though the chromatic number of a (hyper)graph and its respective encoding by real values for the spectral sum within a single such context
        cannot exceed the dimension of the pertinent Hilbert space.
        As of today, neither such ``omni-realistic'' escape or generalization of the Kochen-Specker theorem has been discussed nor observed;
        so we might assume that it is an inapplicable option. And yet, it is feasible in principle.

        In what follows we shall first develop the necessary nomenclature and then present some criteria and results with regards to the reconstruction of (hypergraphs) representing
        logics---aka, collections of (intertwined) contexts containing a uniform number of observables.
        We shall also find criteria for the algorithmic (constructive) generation of colorings
        by the set of two-valued states on such (hyper)graphs or logics.



        \section{Nomenclature}

        The following terms are used by authors synonymuously:
        context, block, (maximal Boolean) subalgebra, (maximal) clique, complete subgraph and \emph{hyperedge}. The same is applied to atom, element and \emph{vertex}.

        Greechie has suggested~\cite{greechie:71} to (amendments are indicated by square brackets ``[$\ldots$]'')
        \begin{quote}
                [$\ldots$]
                present [$\ldots$] lattices as unions of [contexts]
                intertwined or pasted together in some fashion
                [$\ldots$]
                by replacing, for example, the $2^n$ elements in the Hasse diagram of the power set
                of an $n$-element set with the [context aka] complete [sub]graph [$K_n$] on $n$ elements.
                The reduction in numbers of elements is considerable but the number of remaining ``links''
                or ``lines'' is still too cumbersome for our purposes.
                We replace the [context aka] complete [sub]graph on $n$ elements by a single smooth curve (usually a straight line)
                containing $n$ distinguished points. Thus we replace $n(n + 1)/2$ ``links'' with a single smooth curve.
                This representation is propitious and uncomplicated provided that
                the intersection of any pair of blocks contains at most one atom.
        \end{quote}

        In what follows, we shall refer to such a structure as {Greechie diagram}~\cite{kalmbach-83} (References~~\cite{Greechie1968,svozil-tkadlec,Mckay2000,Pavicic-2005,Bretto-MR3077516,2018-minimalYIYS}
        contain variants thereof). The Greechie diagrammatical representation of such, possibly intertwined, collection of blocks, is a {\em hypergraph}, which is a well-known structure in discrete mathematics. We shall briefly introduce the required terms here, but an interested reader might take a look at
7
 \cite{Bretto-MR3077516} for further theory of hypergraphs.

        A hypergraph  $H$  is an ordered pair $H=(V ;E)$ where $V=V(H)$ is the set of vertices and $E=E(H)$ is a family of subsets of $V$ called the hyperedges.
        It depicts a collection of quantum contexts faithfully represented in an $n$-dimensional Hilbert space, whereby the vertices are identified with elementary quantum propositions whose label assignments are in terms of vectors (or with their respective one-dimensional orthogonal projection operators), and the hyperedges are identified with quantum contexts. An $n$-subset of atoms forms a hyperedge if its elements are mutually orthogonal.

        Let $H = (V;E)$ be a hypergraph. The \emph{2-section}  of $H$ is a graph, denoted by $[H]_2$, whose vertices are the same as $V(H)$, and two distinct vertices form an edge if and only if they are on the same hyperedge of $H$. One may think of the 2-section of a hypergraph as the graph associated with the hypergraph. A hypergraph $H$ is called \emph{conformal} if any maximal clique (with respect to the inclusion) of its 2-section $[H]_2$ is on a hyperedge of $H$. If all the hyperedges of a hypergraph $H$ consist of exactly $n$ elements, then $H$ is called \emph{$n$-uniform} \cite{Bretto-MR3077516}.

        Quantum orthogonality hypergraphs are conformal $n$-uniform. Some authors referred to them by their associated graphs, i.~e., their 2-sections. Quantum contexts are represented by hyperedges of these quantum hypergraphs, that is, by the maximal cliques of their 2-sections. We also reserve the letter $n$ for the clique number of those 2-sections, so we always mean $n=\omega ([H]_2 )$, which is a constant integer. Now, to formally state a constraint that has to be met by an orthogonality hypergraph $H$ on $k$ vertices, we remind the reader that a \emph{vertex labeling} of a (hyper)graph $H$ is a function $f:V(H)\longrightarrow L$ that assigns labels from a set $L$ to vertices of $H$.

        \begin{criterion}\label{orthogonality}
                There is a vertex labeling $\bm{x}:V(H)\longrightarrow L$ that assigns a set of $k$ mutually non-colinear vectors in an $n$-dimensional Hilbert space $L$ to vertices of $H$ such that any pair of vertices $a$ and $b$ are adjacent if and only if $\bm{x}(a)$ is orthogonal to $\bm{x}(b)$.
        \end{criterion}

        Such a vertex labeling is called an \emph{$n$-dimensional faithful orthogonal representation}~\cite{lovasz-79,lovasz-89,Portillo-2015} (FOR) for~$H$.

        To represent an orthogonality hypergraphs, we shall concentrate on Greechie diagrams which are pasting~\cite{Greechie1968} constructions~\cite[Chapter~2]{greechie-66-PhD}
        of a homogeneous  single type
        of contexts $K_n$
        where the clique number $n$ is fixed. This means that every hyperedge is shown by a straight line segment or, more general, by a smooth curve which has exactly $n$ elements as nodes on it.

        A hypergraph coloring of $H$ is a \emph{proper vertex coloring} which associates colors to vertices of $H$ so that every two vertices lying on a hyperedge receive different colors. That is, the $n$ distinguished points of any single smooth curve in the hypergraph have $n$ different colors.
        The coloring is noncontextual; that is, the coloring of atomic elements common to two or more contexts (intertwining there)
        is independent of the context.
        %In requiring uniformity we shall implicitly also exclude partial colorings~\cite{Abbott:2010uq,PhysRevA.89.032109,2015-AnalyticKS} where partiality is understood as allowing for undefinedness in the sense of Kleene~\cite{Kleene1936}. %%%%%% Iman: I didn't find uniformity in the coloring!!!

        For a hypergraph $H$, the \emph{chromatic number} $m=\chi(H)$ is the minimum number of colors required for a proper coloring of vertices of $H$. Obviously the clique number $n=\omega([H]_2 )$ is a lower~bound. If these numbers are the same, that is, if $m=\chi(H)=\omega([H]_2 )=n$, then one could obtain two-valued measures from colorings by ``projecting'' one of the colors into the value 1, and all the other $n-1$ colors into the value 0~\cite{godsil-zaks,meyer:99,havlicek-2000}. A hypergraph $H$, whose chromatic and clique numbers are equal, is called here \emph{semi-perfect}.


        Finite examples for which the chromatic number exceeds the clique number, that is, $m>n$,
        are the logical structures involved in proofs of the Kochen-Specker theorem.
        Explicit constructions are, for instance,
        $\Gamma_2$ of Ref.~\cite{kochen1},
        as well as the configurations enumerated in
        Figure~9 of~\cite{svozil-tkadlec},
        Figure~1--3 of~\cite{tkadlec-00},
        Ref.~\cite{cabello-96},
        as well as Table~I, Figure~2 of Ref.~\cite{2015-AnalyticKS},
        among numerous others
        which have a faithful orthogonal representation~\cite{lovasz-79,lovasz-89,Portillo-2015}
        in ``small dimensions'' greater than two.

        A logic has a \emph{separable set of two-valued states} if for any pair $a_i$ and $a_j$, $i\neq j$, of different propositions, there is at least one two-valued state, say $t$, such that $t( a_i ) \neq t(a_j )$ \cite{svozil-tkadlec}. Such a logic is called \emph{unitary} if for each proposition $a$ there is a state $t$ for which we have $t(a)=1$. A hypergraph $H$ is said to be \emph{separable} (respectively unitary) if there is a logic with a separable (resp. unitary) set of two-valued states on its vertices. %We use these notions in Section \ref{construction-conj}.

        A ``true implies false set'' $(a,b)$-TIFS (gadget~\cite{tutte_1954,SZABO2009436,Ramanathan-18}) is a conformal $n$-uniform orthogonality hypergraph $H$ containing two vertices $a$ and $b$ such that for all two-valued states of $H$, we have $b$ is true only if $a$ is false, and conversely, $a$ is true only if $b$ is false. That is, the relation is symmetrical: $a$ exclusive or  $b$ true implies $b$ exclusive or $a$ false, respectively.

        Similarly, a ``true implies true set'' $(c,d)$-TITS (gadget) is a conformal $n$-uniform orthogonality hypergraph $H'$ containing two vertices $c$ and $d$ such that for all two-valued states of $H'$, we have $d$ is true whenever $c$ is true~\cite{2018-minimalYIYS}
        (in this case the converse need not be true as both $c$ and $d$ could be false, or $d$ could be true and $c$ false). That is, $c$ true implies $d$ true.


        The hypergraphs considered here are assumed to be on $k$ vertices, and all their hyperedges (contexts) uniformly contain exactly $n$ vertices. Equivalently, we can assume that our objects are connected graphs like $G$ on $k$ vertices with clique number $\omega (G)=n$, with the assumption that every vertex $v\in V(G)$ lies on at least one maximal clique (context) of size $n$. This assumption is not always necessary, but it is not harmful to our argument, as we can always add vertices to those contexts that have less number of vertices. Therefore, we can state \emph{the completion criterion} as follows:

        \begin{criterion}\label{completion}
                If $a$ and $b$ are adjacent in an orthogonality hypergraph $H$, then there is a hyperedge that contains $a$ and $b$ along with $n-2$ other vertices.
        \end{criterion}

        In other words, the completion criterion says that there must be no hyperedge with less than $n$ vertices. The importance of this simple criterion becomes clear in Section \ref{Rec-PSH}.

      \section{Hypergraphs of low dimensions}

        Some assertions that are true in larger dimensions might be wrong when dimensions are as low as 1 or 2. The only connected 1-uniform (hyper)graph is the trivial singleton whose two-valued states consists of only one state that assigns true to the only vertex. Obviously, a 2-uniform hypergraph is actually a graph (that is its 2-section) whose clique number is 2, so we can call it ``a graph'' instead of ``a hypergraph''. Clearly, a 2-dimensional orthogonality graph cannot have an induced triangle.

        For an obvious reason, we only consider connected structures. Consequently, we only have one FOR graph on 2 vertices ($P_2$), no FOR graph on 3 vertices and two FOR graphs on 4 vertices ($P_4$ and $C_4$).

        Triangle-free graphs can have chromatic numbers as large as is desired. For example for any $r\geq 2$, using the Mycielski's method~\cite{Mycielski1955}, we can construct a triangle-free graph whose chromatic number is $r$. The smallest triangle-free graph whose chromatic number is bigger than 2 is $C_5$, the cycle on 5 vertices. This graph has no admissible two-valued state, so is the smallest KS-set if we allow our dimension be 2. Furthermore, since any graph with an induced odd cycle have chromatic number bigger than 2, such graphs also do not have any admissible two-valued state.

        We know that a graph is 2-colorable if and only if it is bipartite. If it is also connected, the bi-partition can be determined uniquely --- in other word every connected graph whose chromatic number is 2 is also uniquely colorable graph~\cite{HARARY1969264}. This means that in any two-valued state of a bipartite graph, vertices of a bi-partition set receive the same value. Therefore, when a bi-partition sets of a bipartite graph $G$ consist of more than one vertex, we are sure that $G$ is not separable and adjacencies among vertices cannot be certainly determined from admissible two-valued states. For example consider graphs of Figure~\ref{fig:hexagon}. Although these four graphs are not mutually isomorphic, their Travis matrix $T_{ij}$ of Formula~\ref{trav:hexagon}.

        \begin{equation}\label{trav:hexagon}
                T_{ij}=\begin{pmatrix}
                        1 &  0 & 1 & 0 & 1 & 0  \\
                        0 &  1 & 0 & 1 & 0 & 1  \\
                \end{pmatrix}
        \end{equation}

                \begin{figure}
        \begin{center}
                \begin{tabular}{ c c c c c c c }

                        \begin{tikzpicture} [scale=1]

                                \tikzstyle{every path}=[line width=1pt]

                                \newdimen\ms
                                \ms=0.15cm
                                \tikzstyle{s1}=[color=black,fill,circle,inner sep=1]
                                \tikzstyle{c1}=[draw=black,fill=white,circle,inner sep={1}]

                                % Define positions of all observables


                                \coordinate (a1) at (1,0);
                                \coordinate (a2) at (0.5,0.866025);
                                \coordinate (a3) at (-0.5,0.866025);
                                \coordinate (a4) at (-1,0);
                                \coordinate (a5) at (-0.5,-0.866025);
                                \coordinate (a6) at (0.5,-0.866025);
                                % draw contexts


                                \draw (a1) -- (a2);
                                \draw (a3) -- (a2);
                                \draw (a3) -- (a4);
                                \draw (a4) -- (a5);
                                \draw (a5) -- (a6);
                                \draw (a1) -- (a6);


                                % draw atoms

                                \draw (a1) coordinate[s1];
                                \draw (a2) coordinate[c1];
                                \draw (a3) coordinate[s1];
                                \draw (a4) coordinate[c1];
                                \draw (a5) coordinate[s1];
                                \draw (a6) coordinate[c1];
                        \end{tikzpicture}

        & \hspace{0.6 cm} &

                        \begin{tikzpicture} [scale=1]

                                \tikzstyle{every path}=[line width=1pt]

                                \newdimen\ms
                                \ms=0.15cm
                                \tikzstyle{s1}=[color=black,fill,circle,inner sep=1]
                                \tikzstyle{c1}=[draw=black,fill=white,circle,inner sep={1}]

                                % Define positions of all observables


                                \coordinate (a1) at (1,0);
                                \coordinate (a2) at (0.5,0.866025);
                                \coordinate (a3) at (-0.5,0.866025);
                                \coordinate (a4) at (-1,0);
                                \coordinate (a5) at (-0.5,-0.866025);
                                \coordinate (a6) at (0.5,-0.866025);
                                % draw contexts


                                \draw (a1) -- (a2);
                                \draw (a3) -- (a2);
                                \draw (a3) -- (a4);
                                \draw (a4) -- (a5);
                                \draw (a5) -- (a6);
                                \draw (a1) -- (a6);
                                \draw (a2) -- (a5);



                                % draw atoms

                                \draw (a1) coordinate[s1];
                                \draw (a2) coordinate[c1];
                                \draw (a3) coordinate[s1];
                                \draw (a4) coordinate[c1];
                                \draw (a5) coordinate[s1];
                                \draw (a6) coordinate[c1];
                        \end{tikzpicture}

                & \hspace{0.6 cm} &

                        \begin{tikzpicture} [scale=1]

                                \tikzstyle{every path}=[line width=1pt]

                                \newdimen\ms
                                \ms=0.15cm
                                \tikzstyle{s1}=[color=black,fill,circle,inner sep=1]
                                \tikzstyle{c1}=[draw=black,fill=white,circle,inner sep={1}]

                                % Define positions of all observables


                                \coordinate (a1) at (1,0);
                                \coordinate (a2) at (0.5,0.866025);
                                \coordinate (a3) at (-0.5,0.866025);
                                \coordinate (a4) at (-1,0);
                                \coordinate (a5) at (-0.5,-0.866025);
                                \coordinate (a6) at (0.5,-0.866025);
                                % draw contexts


                                \draw (a1) -- (a2);
                                \draw (a3) -- (a2);
                                \draw (a3) -- (a4);
                                \draw (a4) -- (a5);
                                \draw (a5) -- (a6);
                                \draw (a1) -- (a6);
                \draw (a2) -- (a5);
                                \draw (a3) -- (a6);

                                % draw atoms

                                \draw (a1) coordinate[s1];
                                \draw (a2) coordinate[c1];
                                \draw (a3) coordinate[s1];
                                \draw (a4) coordinate[c1];
                                \draw (a5) coordinate[s1];
                                \draw (a6) coordinate[c1];
                        \end{tikzpicture}


                        & \hspace{0.6 cm} &

                        \begin{tikzpicture} [scale=1]

                                \tikzstyle{every path}=[line width=1pt]

                                \newdimen\ms
                                \ms=0.15cm
                                \tikzstyle{s1}=[color=black,fill,circle,inner sep=1]
                                \tikzstyle{c1}=[draw=black,fill=white,circle,inner sep={1}]

                                % Define positions of all observables


                                \coordinate (a1) at (1,0);
                                \coordinate (a2) at (0.5,0.866025);
                                \coordinate (a3) at (-0.5,0.866025);
                                \coordinate (a4) at (-1,0);
                                \coordinate (a5) at (-0.5,-0.866025);
                                \coordinate (a6) at (0.5,-0.866025);
                                % draw contexts


                                \draw (a1) -- (a2);
                                \draw (a3) -- (a2);
                                \draw (a3) -- (a4);
                                \draw (a4) -- (a5);
                                \draw (a5) -- (a6);
                                \draw (a1) -- (a6);
                \draw (a2) -- (a5);
                                \draw (a3) -- (a6);
                                \draw (a1) -- (a4);

                                % draw atoms

                                \draw (a1) coordinate[s1];
                                \draw (a2) coordinate[c1];
                                \draw (a3) coordinate[s1];
                                \draw (a4) coordinate[c1];
                                \draw (a5) coordinate[s1];
                                \draw (a6) coordinate[c1];
                        \end{tikzpicture}

        \\

                (a)&&(b)&&(c)&&(d)
                \end{tabular}
        \end{center}
        \caption{\label{fig:hexagon}
                Four bipartite graph on 6 vertices whose two-valued states are the same. Therefore, they are not reconstructable from their Travis matrix.}
        \end{figure}


        Here in this paper, we are interested in reconstructing hypergraphs from their table of two-valued states, so it must be noted that apart from $P_2=K_{1,1}$, all other connected 2-colorable graphs are nonseparable and therefore cannot be reconstructed from their two-valued states unless $G$ is the star graph $K_{1,r}$ for $r>1$ and we are sure that the dimension is 2, which is the only exception that is nonseparable but reconstructable.

        So, hereinafter, we suppose that the dimension of our space $n$, which is the uniform number of vertices on hyperedges, is bigger than 2.



        \section{Preliminary conjectures regarding the chromatic construction from two-valued states}

        In this section we consider two statements. The first one is about the relation between a hypergraph's semi-perfectness and existence of a certain number of two-valued states that induce a partition logic on its vertex set.


        \begin{conjecture}\label{c1} The following statements are equivalent:

                \begin{itemize}
                        \item[(i)]
                        The hypergraph $H$ is semi-perfect, i.e., its chromatic number and 2-section clique number are equal.%; that is, the associated 2-section is colorable by $n$ distinct colors.
                        \item[(ii)]
                        The set of two-valued states contains $n$ members which correspond
                        to, or induce, a partitioning of all elements of the partition logic;
                        the equivalence relation defined by each one of these $n$ states evaluating to $1$ on some element of every context.
                        That is, those $n$ states are $1$ on different atoms of every context.
                \end{itemize}
        \end{conjecture}

        This does not exclude the existence of partition logics which are not semi-perfect.
        Indeed, in general, their chromatic number can exceed their 2-section's clique number.
        A concrete example is
        Greechie's $G_{32}$~\cite[Figure~6, p.~121]{greechie:71} mentioned in the Appendix, Section~\ref{2021-chroma-G32}, and depicted in
        Figure~\ref{2020-f-GreechieG32}.

        Another statement is about the possibility of reconstructing the hypergraph from its table of two-valued states---its Travis~\cite{travis-mt-62} matrix~\cite{greechie-66-PhD}---if it is separable. For two-valued states, whenever a vertex is assigned true, all other vertices adjacent to it have to be assigned false. This, which is called later the adjacency criterion, is the most evident property that every pair of adjacent vertices has. For a nonseparable hypergraph, however, this does not let us to reconstruct the hypergraph because for two distinct vertices $a$ and $b$ with $t(a)=t(b)$ for every two-valued state $t$, it cannot be understood whether or not $a$ is adjacent to all the neighbors of $b$.

        \begin{conjecture} \label{c2} A hypergraph $H$ is reconstructable from the table of its two-valued states if and only if $H$ is separable.
        \end{conjecture}

        Knowing that an orthogonality hypergraph is reconstructable from its logical states gives us the opportunity to find perpendicular pairs of atoms without actually measuring angles between them.

        We examine these two conjectures in sections \ref{construction-conj} and \ref{color-conj}.







        \section{Structure reconstruction from two-valued states}\label{construction-conj}

        Suppose we have a table of all two-valued states of a hypergraph $H$. Under what condition(s) can we reconstruct the graph theoretical structure of $H$? We saw that Conjecture \ref{c2} claims that whenever $H$ is separable, we may be able to reconstruct its structure from the set of its two-valued states. This is what we are going to examine here.

        \subsection{Perfectly separable hypergraphs}




        In this section, we introduce a type of hypergraphs for which Conjecture \ref{c2} holds.

        Let $H$ be a hypergraph with $V(H)=\{a_1, a_2,\ldots, a_k\}$, and its set of two-valued states contains $s$ elements (which can be shown by saying that $nTS(H)=s$). Its Travis matrix $T(H) =[t_{ij}]_{s\times k}$ enumerates all two-valued states which are represented by row vectors on the vertices of $H$, arranged in the columns, such that each vertex corresponds to one column. Then, \emph{separability} of $H$ can be extended to the following statement:

        \begin{definition}\label{separability}
                The hypergraph $H$ is \emph{perfectly separable} if and only if for all pairs $a_i$ and $a_j$ of vertices of $H$ we get:
                \begin{itemize}
                        \item[1.] There is an $r_1 \in \{1,\ldots,s\}$ such that $t_{r_1 i}=0$ and $t_{r_1 j}=1$.
                        \item[2.] There is an $r_2 \in \{1,\ldots,s\}$  such that $t_{r_2 i}=1$ and $t_{r_2 j}=0$.
                        \item[3.] If $a_i$ and $a_j$ are not adjacent in $H$, then there is an $r_3 \in \{1,\ldots,s\}$  such that $t_{r_3 i}=1$ and $t_{r_3 j}=1$.
                        \item[4.] If $a_i$ and $a_j$ are not adjacent in $H$, then there is an $r_4 \in \{1,\ldots,s\}$ such that $t_{r_4 i}=0$ and $t_{r_4 j}=0$.
                \end{itemize}

                Note that a hypergraph is separable if item 1 or 2 of Definition \ref{separability} holds. Therefore, every perfectly separable orthogonality hypergraph is also separable, but the converse is not always true.
                Elementary counterexamples are true-implies-false gadgets such as the Specker bug depicted in Figure~\ref{2020-f-SpeckerBug}.

        \end{definition}

        \subsection{Reconstruction of perfectly separable hypergraphs}\label{Rec-PSH}

        We know that whenever $a_i$ and $a_j$ are adjacent in $H$ (or, equivalently, they are on the same context), then $t_{ri}=1$  implies $t_{rj}=0$ for $r=1,\ldots,s$. We want to know under what conditions the converse is true as well, i.e., the following \emph{adjacency criterion} holds:
        \begin{criterion}
                \label{AdCr} For all $r=1,\ldots,s$, if $t_{ri}=1$ implies $t_{rj}=0$ (which means its contraposition, $t_{rj}=1$ implies $t_{ri}=0$, is also true) then $a_i$ and $a_j$ are adjacent.
        \end{criterion}

        The next theorem asserts that when $H$ is a perfectly separable hypergraph, we can always reconstruct $H$ from the information presented in its Travis matrix $T(H)$ and the adjacency criterion. And conversely, a reconstructable hypergraph using Criterion \ref{AdCr} has to be a perfectly separable hypergraph. In other words, Conjecture \ref{c2} is true for perfectly separable hypergraphs.

        \begin{theorem}\label{reconstruct}
                Let $H$ be a hypergraph on $\{a_1, a_2,\ldots, a_k\}$ whose Travis matrix $T(H)$ is available and $\omega([H]_2 )=n\geq 3$. Moreover, suppose that $H'$ is the hypergraph on $\{a_1, a_2,\ldots, a_k\}$ whose adjacency is defined by Criterion \ref{AdCr}. Then $H=H'$ if and only if $H$ is perfectly separable.
        \end{theorem}
        \begin{proof}
                First suppose that $H$ is perfectly separable. Then, because of item 3 of Definition \ref{separability}, for all $a_i$ and $a_j$ that are not adjacent in $H$ there is a row $r$ in $T(H)$ such that $t_{r i}=1$ and $t_{r j}=1$. Therefore, Criterion \ref{AdCr} in $H'$ can only be satisfied for those vertices that are already adjacent in $H$. Consequently, $H=H'$.

                Conversely, suppose that $H=H'$ and $a_i$ and $a_j$ are two distinct non-adjacent vertices. Then the adjacency criterion does not meet for $a_i$ and $a_j$. As a result, there is an $r$, $1\leq r\leq s$ such that $t_{ri}=1$ and $t_{rj}=1$, i.e., item 3 of Definition \ref{separability} already holds for $a_i$ and $a_j$. Moreover, we have the following two cases:

                \begin{itemize}
                        \item[Case 1.] \emph{Item 1 (or 2) of Definition \ref{separability} does not hold for $i$ and $j$}. Therefore, the adjacency criterion makes $a_i$ adjacent to all vertices already adjacent to $a_j$, which cannot happen unless $a_i = a_j$, or $a_i$ is colinear with $a_j$, which is a contradiction.
                        \item[Case 2.] \emph{Item 4 of Definition \ref{separability} does not hold of $i$ and $j$}. Therefore, whenever $t_{ri}=0$ we have $t_{rj}=1$ and when $t_{rj}=0$ we have $t_{ri}=1$. We distinguish the following two cases:
                        \begin{itemize}
                                \item[Case 2.1.] \emph{Both $t_{ri}$ and $t_{rj}$ are $1$ for all $r=1,\ldots,t$.} Therefore, $a_i$ is adjacent to all $n-1 \geq 2$ vertices of a context $a_j$ lies on, which in its turn means that $a_i$ and $a_j$ are colinear, a contradiction.

                                \item[Case 2.2.] \emph{At least one of $t_{ri}$ or $t_{rj}$ is $0$ for some $r=1,\ldots,t$.} Without loss of generality, let it be $t_{ri}$ that is sometimes $0$. Therefore, there is an $a_l$ on a context containing $a_i$ that is assigned $1$ in a two-valued state $t_r$ while it is not on any contexts containing $a_j$ which is also assigned $1$ by $t_r$. As a result, all other vertices on a context containing $a_j$ must be adjacent to $a_l$ and consequently, $a_l$ and $a_j$ must be colinear, a contradiction.
                        \end{itemize}
                \end{itemize}
                As we see, both these cases lead to complete  contradictions. Therefore, for every pair of non-adjacent vertices of $H$, items 1 to 4 of Definition \ref{separability} hold. If for any pair of adjacent vertices in $H$, namely $a_i$ and $a_j$, also  both items~1 and~2 of Definition \ref{separability} are true, then it can be inferred that $H$ must be perfectly separable.

                To see this, first notice that if one of items 1 or 2 does not hold for adjacent vertices $a_i$ and $a_j$, then one them, say $a_j$, is always assigned $0$. Therefore, because $H=H'$, and $H'$ is constructed via Criterion \ref{AdCr}, $a_j$ must be adjacent to any vertex that is assigned $1$ at least once. Hence, we can distinguish the following two cases:

                \begin{itemize}
                        \item[Case 1.] \emph{The vertex $a_j$ belongs to all contexts of $H$}. Then there is a two-valued state that assigns $1$ to $a_j$ and $0$ to all other vertices, a contradiction to our assumption that $a_j$ is always assigned $0$.

                        \item[Case 2.] \emph{There is at least one context, $\mathcal{A}$, which does not contain $a_j$}. Since $H=H'$, only one vertex of $\mathcal{A}$, say $a_p$, can be assigned always $0$. Then $a_j$ is adjacent to at least $n-1$ vertices of $\mathcal{A}$ other than $a_p$, and consequently, $a_j$ and $a_p$ are colinear, a contradiction.
                \end{itemize}
                Therefore, items 1 and 2 of Definition \ref{separability} also hold for every pair of adjacent vertices of $H$. Now the conclusion is evident.
        \end{proof}

        From the proof of Theorem \ref{reconstruct} we see that, for a reconstructable hypergraph $H$, the most important factor is that at least item 1 or 2 is true for every pair of non-adjacent vertices. It should be noted that item 2 is actually the contraposition for item 1. Yet conversely, if  we only want to imply Criterion \ref{AdCr}, we see that for separable hypergraphs which are not perfectly separable, the reconstruction produces some extra hyperedges for $H'$ and makes it different from $H$. For example, when we try to reconstruct the Specker bug from its two-valued states using the adjacency criterion, we produce an extra hyperedge on two vertices, as depicted in Figure~\ref{Spec-Rec}.

        \begin{figure}
                \begin{center}
                        \begin{tikzpicture}  [scale=0.6]

                                \newdimen\ms
                                \ms=0.07cm

                                \tikzstyle{every path}=[line width=1pt]

                                \tikzstyle{c3}=[circle,inner sep={\ms/8},minimum size=6*\ms]
                                \tikzstyle{c2}=[circle,inner sep={\ms/8},minimum size=4*\ms]
                                \tikzstyle{c1}=[circle,inner sep={\ms/8},minimum size=0.8*\ms]

                                % Radius of regular polygons
                                \newdimen\R
                                \R=30mm     % outer circle

                                %\r= { \R * sqrt(3) }     % inner circle
                                %\newdimen\r
                                %\r=    {\R * sqrt(3)/2}       % inner circle

                                %\newdimen\K
                                %\K=3cm

                                % Define positions of all observables
                                \path
                                ({ 180 - 0 * 360 /6}:\R      ) coordinate(1)
                                ({ 180 - 30 - 0 * 360 /6}:{\R * sqrt(3)/2}      ) coordinate(2)
                                ({ 180 - 1 * 360 /6}:\R   ) coordinate(3)
                                ({ 180 - 30 - 1 * 360 /6}:{\R * sqrt(3)/2}   ) coordinate(4)
                                ({ 180 - 2 * 360 /6}:\R  ) coordinate(5)
                                ({ 180 - 30 - 2 * 360 /6}:{\R * sqrt(3)/2}  ) coordinate(6)
                                ({ 180 - 3 * 360 /6}:\R  ) coordinate(7)
                                ({ 180 - 30 - 3 * 360 /6}:{\R * sqrt(3)/2}  ) coordinate(8)
                                ({ 180 - 4 * 360 /6}:\R     ) coordinate(9)
                                ({ 180 - 30 - 4 * 360 /6}:{\R * sqrt(3)/2}     ) coordinate(10)
                                ({ 180 - 5 * 360 /6}:\R     ) coordinate(11)
                                ({ 180 - 30 - 5 * 360 /6}:{\R * sqrt(3)/2}     ) coordinate(12)
                                ;

                                % draw contexts

                                \draw [color=cyan] (1) -- (2) -- (3);
                                \draw [color=red] (3) -- (4) -- (5);
                                \draw [color=green] (5) -- (6) -- (7);
                                \draw [color=blue] (7) -- (8) -- (9);
                                \draw [color=magenta] (9) -- (10) -- (11);    %
                                \draw [color=olive] (11) -- (12) -- (1);    %
                                \draw [color=teal] (4) -- (10)  coordinate[pos=0.3]  (13);
                                \draw [color=gray, snake=coil,segment aspect=0] (1) -- (7);

                                %
                                %%
                                %% draw atoms
                                %%
                                %
                                \draw (1) coordinate[c2,draw=red,fill=red];   %
                                %
                                \draw (2) coordinate[c2,draw=gray,fill=white];    %
                                %
                                \draw (3) coordinate[c2,draw=gray,fill=white]; %
                                %
                                \draw (4) coordinate[c2,draw=gray,fill=white];  %
                                %
                                \draw (5) coordinate[c2,draw=gray,fill=white];  %
                                %
                                \draw (6) coordinate[c2,draw=gray,fill=white];
                                %
                                \draw (7) coordinate[c2,draw=green,fill=green];  %
                                %
                                \draw (8) coordinate[c2,draw=gray,fill=white];  %
                                %
                                \draw (9) coordinate[c2,draw=gray,fill=white];
                                %
                                \draw (10) coordinate[c2,draw=gray,fill=white];  %
                                %
                                \draw (11) coordinate[c2,draw=gray,fill=white];  %
                                %
                                \draw (12) coordinate[c2,draw=gray,fill=white];
                                %
                                %
                                \draw (13) coordinate[c2,draw=gray,fill=white];  %
                                %                       \draw (13) coordinate[c3,fill=red,label={below right: $11,12,{\color{red}14}\}$}];  %
                                %
                        \end{tikzpicture}
                \end{center}
                \caption{\label{Spec-Rec}
                        Reconstruction of the Specker bug using Criterion \ref{AdCr} which gives the gray snake-shaped extra hyperedge that contains only two (the red and green) vertices. This extra hyperedge can be easily eliminated if we use  Criterion \ref{completion}.
                }
        \end{figure}


        However, one might think that we may be able to reconstruct $H$ from $H'$ by finding the extra hyperedge(s) that must be eliminated. To do this, suppose that $\omega([H]_2 )=n\geq 3$ and every vertex is on a hyperedge of size $n$. Then the reconstructed $H'$ must have the same property, i.e., we must have Criterion \ref{completion} for all hyperedges of $H'$, and if a hyperedge does not meet this criterion, then it has to be eliminated. These ``unwanted'' adjacency between $a_i$ and $a_j$ in $H'$ appear only when these two are not on the same context but whenever one is assigned $1$ the other has to be $0$ and vice versa---a typical situation in a TIFS gadget---like what is happening when we try to reconstruct the Specker bug from its two-valued states. However, we show in the next section that this method cannot always guarantee that we can identify these unwanted hyperedges from the table of two-valued states.


        \subsection{Examples of non-reconstructable hypergraphs using Criteria \ref{completion} and  \ref{AdCr}}\label{Rec-B(H)}


        In this section we want to give a counterexample to Conjecture \ref{c2}: a hypergraph that is separable but not perfectly separable. We construct the hypergraph $B(H)$ of the dimension $n=3$, but for higher dimensions the construction should be analogous.

        A counterexample to Conjecture \ref{c2}, say $H$, is an orthogonality hypergraph that has some non-adjacent vertices but Criteria \ref{completion} and \ref{AdCr} detect them as they are on a hyperedge in the reconstructed hypergraph $H'$; and consequently these two hypergraphs are not the same. Note that, up to permutations of rows and columns, their Travis matrices are the same. Since our dimension is 3, we must have at least 3 vertices $a$, $b$ and $c$ which are not adjacent in $H$, but whenever one is assigned true by a two-valued state, the other two  have to be assigned false. Moreover, there must not be a two-valued state that simultaneously assigns $a$, $b$ and $c$ false.

        Suppose $G$ is an $(a,b)$-TIFS gadget such that whenever $a$ is assigned true by a two-valued state, $b$ has to be false. As mentioned earlier, the reverse is also true; that is, whenever $b$ is assigned true by a two-valued state, $a$ has to be false. This means that $a$ and $b$ cannot be assigned true at the same time (but they can both be false). Therefore, if we make a triangle, using a true implies false set (TIFS) as its edges, then we have the desired property that whenever one end is true, the other two are false. However, this hypergraph might have several two-valued states that assign false to all its three ends $a$, $b$ and $c$.  Figure~\ref{layer-graph} depicted this layer hypergraph, with $a=a_1 b_3$, $b=a_2 b_1$ and $c=a_3 b_2$.

        \begin{figure}
                \begin{center}
                        \begin{tikzpicture}  [scale=1]

                                \tikzstyle{every path}=[line width=1pt]

                                \newdimen\ms
                                \ms=0.1cm
                                \tikzstyle{s1}=[color=black,fill,rectangle,inner sep=3]
                                \tikzstyle{c1}=[draw=gray,fill=white,circle,inner sep={\ms/1}]
                                %\tikzstyle{c1}=[color=black,fill,circle,inner sep={\ms/8},minimum size=2*\ms]

                                % Define positions of all observables


                                \coordinate (a1) at (0,0);
                                \coordinate (a2) at (3,0);
                                \coordinate (a3) at (1.5,2.5981);


                                % draw contexts

                                %\draw [color=olive, ->,snake=zigzag,segment amplitude=5pt,thick] (a1) to node[below] {$G_1$} (a2);
                                %\draw [color=cyan, ->,snake=zigzag,segment amplitude=5pt,thick] (a2) to node[right] {$ \textnormal{       } G_2$} (a3);
                                %\draw [color=orange, ->,snake=zigzag,segment amplitude=4pt,thick] (a3) to node[left] {$G_3 \textnormal{       }  $  } (a1);

                                \draw [color=olive, ->, snake=coil,segment aspect=0,thick] (a1) to node[below] {$G_1$} (a2);
                                \draw [color=cyan, ->, snake=coil,segment aspect=0,thick] (a2) to node[right] {$ \; \;  G_2$} (a3);
                                \draw [color=orange, ->, snake=coil,segment aspect=0,thick] (a3) to node[left] {$G_3 \; \; $  } (a1);

                                % draw atoms

                                \draw (a1) coordinate[c1,label=below left:$a_{1} b_{3} $];
                                \draw (a2) coordinate[c1,label=below right:$a_2 b_1$];
                                \draw (a3) coordinate[c1,label=above:$a_3 b_2$];
                        \end{tikzpicture}
                \end{center}
                \caption{\label{layer-graph}
                        A layer hypergraph serving as a quasi-block for the construction of a counterexample to Conjecture \ref{c2}. For $i=1,2,3$, the snake-shaped edges are distinct copies of an arbitrary separable $(a_i , b_i )$-TIFS gadget $G$ such that whenever $a_i$ is assigned true by a two-valued state, $b_i$ has to be false: their respective ends have been ``cyclically folded'' on each other, eg., $a_1$ from $G_1$ is identified with $b_3$ from $G_3$. Note that vertices $a_1 b_3$, $a_2 b_1$ and $a_3 b_2$ are not adjacent.}
        \end{figure}

        For an $(a_i ,b_i)$-TIFS gadget $G$, Criteria \ref{completion} and \ref{AdCr} do not detect a hyperedge containing $a_1 b_3$, $a_2 b_1$ and $a_3 b_2$ from the two-valued states of a hypergraph like Figure~\ref{layer-graph}. This is because usually there are some two-valued states that assign false to all of these three vertices. Therefore, the layer hypergraph of Figure~\ref{layer-graph} must become a part of a larger hypergraph so that in every two-valued state, exactly one of the vertices $a$, $b$ or $c$ be assigned true.

        To do so, one possibility is to use three copies of the layer hypergraph of Figure~\ref{layer-graph} and use three extra contexts to bind them together; a configuration drawn in Figure~\ref{TIFS-non-Rec}. When we use $G$ as an $(a_i , b_i)$-TIFS gadget the resulting hypergraph of this construction is denoted by $B(G)$.

        \begin{figure}
                \begin{center}
                        \resizebox{0.5\textwidth}{!}{%
                                \begin{tikzpicture} [scale=1]

                                        \tikzstyle{every path}=[line width=1pt]

                                        \newdimen\ms
                                        \ms=0.15cm
                                        \tikzstyle{s1}=[color=black,fill,rectangle,inner sep=3]
                                        \tikzstyle{c1}=[draw=gray,fill=white,circle,inner sep={\ms/1}]

                                        % Define positions of all observables


                                        \coordinate (a1) at (0,0);
                                        \coordinate (a2) at (2,0);
                                        \coordinate (a3) at (1,1.732);
                                        \coordinate (a4) at (-2,-1.5);
                                        \coordinate (a5) at (4,-1.5);
                                        \coordinate (a6) at (1,3.6961);
                                        \coordinate (a7) at (-4,-3);
                                        \coordinate (a8) at (6,-3);
                                        \coordinate (a9) at (1,5.6602);
                                        % draw contexts
                                        %\draw [color=magenta, ->,snake=zigzag,segment amplitude=5pt,thick] (a1) to node[below] {$G_1$} (a2);
                                        %\draw [color=cyan, ->,snake=zigzag,segment amplitude=5pt,thick] (a2) to node[right] {$ \textnormal{       } G_2$} (a3);
                                        %\draw [color=black, ->,snake=zigzag,segment amplitude=4pt,thick] (a3) to node[left] {$G_3 \textnormal{       }  $  } (a1);
                                        %\draw [color=purple, ->,snake=zigzag,segment amplitude=5pt,thick] (a4) to node[below] {$G_4$} (a5);
                                        %\draw [color=teal, ->,snake=zigzag,segment amplitude=5pt,thick] (a5) to node[right] {$ \textnormal{       } G_5$} (a6);
                                        %\draw [color=red, ->,snake=zigzag,segment amplitude=4pt,thick] (a6) to node[left] {$G_6 \textnormal{       }  $  } (a4);
                                        %\draw [color=violet, ->,snake=zigzag,segment amplitude=5pt,thick] (a7) to node[below] {$G_7$} (a8);
                                        %\draw [color=midblue, ->,snake=zigzag,segment amplitude=5pt,thick] (a8) to node[right] {$ \textnormal{       } G_8$} (a9);
                                        %\draw [color=orange, ->,snake=zigzag,segment amplitude=4pt,thick] (a9) to node[left] {$G_9 \textnormal{       }  $  } (a7);

                                        \draw [color=magenta, ->,snake=coil,segment aspect=0,thick] (a1) to node[below] {$G_1$} (a2);
                                        \draw [color=cyan, ->,snake=coil,segment aspect=0,thick] (a2) to node[right] {$ \textnormal{       } G_2$} (a3);
                                        \draw [color=black, ->,snake=coil,segment aspect=0,thick] (a3) to node[left] {$G_3 \textnormal{       }  $  } (a1);
                                        \draw [color=purple, ->,snake=coil,segment aspect=0,thick] (a4) to node[below] {$G_4$} (a5);
                                        \draw [color=teal, ->,snake=coil,segment aspect=0,thick] (a5) to node[right] {$ \textnormal{       } G_5$} (a6);
                                        \draw [color=red, ->,snake=coil,segment aspect=0,thick] (a6) to node[left] {$G_6 \textnormal{       }  $  } (a4);
                                        \draw [color=violet, ->,snake=coil,segment aspect=0,thick] (a7) to node[below] {$G_7$} (a8);
                                        \draw [color=midblue, ->,snake=coil,segment aspect=0,thick] (a8) to node[right] {$ \textnormal{       } G_8$} (a9);
                                        \draw [color=orange, ->,snake=coil,segment aspect=0,thick] (a9) to node[left] {$G_9 \textnormal{       }  $  } (a7);


                                        \draw [color=brown] (a1) -- (a7);
                                        \draw [color=olive] (a2) -- (a8);
                                        \draw [color=lightgray] (a3) -- (a9);

                                        % draw atoms

                                        \draw (a1) coordinate[c1,label=left:$a$];
                                        \draw (a2) coordinate[c1,label=right:$b$];
                                        \draw (a3) coordinate[c1,label=left:$c$];
                                        \draw (a4) coordinate[c1,label=left:$a'$];
                                        \draw (a5) coordinate[c1,label=right:$b'$];
                                        \draw (a6) coordinate[c1,label=left:$c'$];
                                        \draw (a7) coordinate[c1,label=left:$a''$];
                                        \draw (a8) coordinate[c1,label=right:$b''$];
                                        \draw (a9) coordinate[c1,label=above:$c''$];
                                \end{tikzpicture}
                        }
                \end{center}
                \caption{\label{TIFS-non-Rec}
                        A hypergraph $B(G)$ depicting a counterexample to Conjecture \ref{c2}. For $i=1,\ldots ,9$, the snake-shaped curves indicate different copies of an $(a_i , b_i )$-TIFS gadget $G$ with their terminals suitably identified, that is, $a_1$ from $G_1$ is identified with $b_3$ from $G_3$ as the vertex $a$. Straight lines are ordinary hyperedges, i.~e., vertices $a$, $a'$ and $a''$ are on a context, drawn in brown.}
        \end{figure}

        Since $G$ is a TIFS gadget, in every two-valued state of $B(G)$, exactly one of $a$, $b$ or $c$ has to be assigned true while the other two have to be false. This is because $\{a,a',a''\}$, $\{b,b',b''\}$ and $\{c,c',c''\}$ are contexts, and has to have exactly one true value in every two-valued states. Consequently, if for example $b$ and $c$ are assigned false by a two-valued state $t_r$, then one of $b'$ and $b''$, and one of $c'$ and $c''$ has to be assigned true by $t_r$. Without loss of generality, let $b'$ be the true one and $b''$ be false. then since $G_4$ and $G_5$ are TIFS gadgets, it can be inferred that $a'$ and $c'$ have to be assigned false by $t_r$. Therefore, $c''$ has also to be assigned true because $c$ and $c'$ are false. Again, since $G_9$ is a TIFS gadget, $a''$ has to be false. Now, $a$ has to be assigned true by $t_r$ because $a'$ and $a''$ are both false.

        Therefore, for nonadjacent vertices $a$, $b$ and $c$ we have exactly one of them has to be assigned true, and the other two are false. This is just like if these three vertices are on the same context. It means that if one tries to reconstruct $B(G)$ from its table of two-valued states using criteria \ref{completion} and \ref{AdCr}, he finds extra hyperedges of $\{a,b,c\}$, $\{a',b',c'\}$ and $\{a'',b'',c''\}$. This implies that $B(G)$ is not reconstructable using these criteria.

        One question that arises in this regard is the following: suppose we know that $G$ is separable, then does this imply that $B(G)$ is also separable? This is not too hard to answer, and it is always ``yes'' if it does not contain a true implies true set (TITS) gadget.

        To proceed, we need one more thing. A function $f:S\longrightarrow T$ defined on a subset $S\subsetneq X$ is said to be \emph{lifted} to $\tilde{f}:X\longrightarrow T$ if $\tilde{f}(a)=f(a)$ for each $a\in S$. It must be mentioned that when $f$ possesses a property, like if $f$ is a proper coloring~\cite{ALBERTSON1998189} or a two-valued state, there might not always be a lift with the same property.

        \begin{lemma}\label{Sep-lem}
                Let $G$ be a separable unitary TIFS gadget which does not contain a TITS gadget and $\omega ([G]_2)=3$. Then $B(G)$ is also separable.
        \end{lemma}

        \begin{proof}
                Since $G$ is unitary, there is no vertex that has to be assigned 0 by all the two-valued states of $G$. On the other hand, there is no vertex that is given 1 by all the two-valued states of $G$ because else it must lie on at least one hyperedge with two other vertices, those that have to be always assigned 0, a contradiction to separability (and being unitary) of $G$. Thus, it can be inferred that for a vertex $u$ of $G$, there are states $\varphi$ and $\psi$ such that $\varphi (u)=0$ and $\psi (u)=1$.

                For an $(a ,b )$-TIFS gadget $G$, the set of two-valued states are nonempty. Let $s_{1}$ be the set of those states in which $a$ is true, $s_{2}$ be the set of those states that $b$ is true and $s_{3}$ be the set of those states in which both $a$ and $b$ are false. Then these sets $s_{1}$ , $s_{2}$ and $s_{3}$ partition the set of two-valued states of $G$. For an $(a ,b )$-TIFS gadget copy $G_i$, these sets are shown here by $s_{i1}$, $s_{i2}$ and $s_{i3}$. Therefore, if for example a two-valued state is in $s_{1}$, it is a two-valued state of $G$ so that its head, i.~e. $a$, is assigned 1.

                To show that $B(G)$ is separable, we show that for any pair of distinct vertices $x$ and $y$, there is a two-valued state of it that gives them different values. There are the following cases:

                \begin{itemize}
                        \item[Case 1.] \emph{$x$ and $y$ belong to the same copy of $G$, say $G_i$.} \ Because of the symmetry, we can assume without loss of generality that $i=1$.  Since $G$ is separable, there must be a two-valued state, say $t$ on $G$, such that $t(x)\neq t(y)$. We know that $t\in s_{1}\cup s_{2}\cup s_{3}$. If $t\in s_l$ for $l=1,2,3$, then there is a two-valued state for the underlying hypergraph of $B(G)$ in Section~\ref{sec:underlying} such that it agrees with the values of $t(a)$ and $t(b)$. Now, using this two-valued state we define $\Tilde{t}$ for the end vertices $a$, $b$, $c$, $a'$, $b'$ ,$c'$, $a''$, $b''$ and $c''$. Then using appropriate two-valued states of $G$, we can find a suitable two-valued state for internal vertices of $G_2,\ldots,G_9$. Therefore, there are two-valued states such as $\Tilde{t}$ of $B(G)$ which is a lifting for $t$ and  $\Tilde{t}(x) \neq \Tilde{t}(y)$.


                        \item[Case 2.] \emph{$x$ and $y$ lie on different copies of $G$, say $G_i$ and $G_j$ respectively.} Then there are another three cases:
                        \begin{itemize}
                                \item[Case 2.1.] \emph{$G_i$ and $G_j$ lie on the same layer of $B(G)$.} Without loss of generality, suppose that it is the layer consisting of the vertices $a$, $b$ and $c$. Then $G_i$ and $G_j$ has a common vertex that again without loss of generality, we can assume that it is $a$. Let $G_i$'s head and tail be $a$ and $b$, and $G_j$'s head and tail be $c$ and $a$, respectively (see Figure \ref{fig-proof-sep-lem} (a)). Therefore, every two-valued state of the induced subhypergraph $G_i \cup G_j$ in $B(G)$ is a member of $s_{i1} \cup s_{j2}$, $s_{i2} \cup s_{j3}$ or $s_{i3}\cup s_{j1}$. Suppose on contrary that $x$ and $y$ receive the same value by all two-valued states of $B(G)$. Then, since there is a two-valued state of $B(G)$ that assigns 1 to $x$, at least one of the following statements holds:

                                \begin{itemize}
                                        \item[1.] If $x$ is assigned 1 by a two-valued state of $s_{i1}$, then $y$ has to be assigned 1 by all two-valued states of $s_{j2}$. This means that $G_j$ (and therefore $G$) is a $(a,y)$-TITS gadget, a contradiction to our assumption.
                                        \item[2.] If $x$ is assigned 1 by a two-valued state of $s_{i2}$, then $y$ has to be assigned 1 by all two-valued states of $s_{j3}$. Consequently, $x$ cannot be assigned 0 by a $s_{i2}$, because else, it can be lifted to the required separation of $x$ and $y$. Hence, $G_i$ (and therefore $G$) is a $(b,x)$-TITS gadget, a contradiction to our assumption.
                                        \item[3.] If $x$ is assigned 1 by a two-valued state of $s_{i3}$, then $y$ has to be assigned 1 by all two-valued states of $s_{j1}$. This means that $G_j$ (and therefore $G$) is a $(c,y)$-TITS gadget, again a contradiction to our assumption.
                                \end{itemize}


                                \item[Case 2.2] \emph{$G_i$ and $G_j$ lie on different layers of $B(G)$, and both ends of $G_i$ and $G_j$ lie on the same contexts of $B(G)$.} Without loss of generality suppose that $G_i$ has $a$ and $b$ and $G_j$ has $a'$ and $b'$ as their heads and tails, respectively (see Figure \ref{fig-proof-sep-lem} (b)). Therefore, every two-valued state of the induced subhypergraph $G_i \cup G_j$ in $B(G)$ is a member of $s_{i1} \cup s_{j2}\cup s_{j3}$, $s_{i2} \cup s_{j1}\cup s_{j3}$ or $s_{i3}\cup s_{j1}\cup s_{j2}$ (or $s_{j1} \cup s_{i2}\cup s_{i3}$, $s_{j2} \cup s_{i1}\cup s_{i3}$ or $s_{j3}\cup s_{i1}\cup s_{i2}$ which are completely similar). Suppose on contrary that $x$ and $y$ receive the same value by all two-valued states of $B(G)$. Then, since there is a two-valued state of $B(G)$ that assigns 1 to $x$, at least one of the following statements holds:

                                        \begin{itemize}
                                                \item[1.] If $x$ is assigned 1 by a two-valued state of $s_{i1}$, then $y$ has to be assigned 1 by all two-valued states of $s_{j2}\cup s_{j3}$. This means that $G_j$ (and therefore $G$) is a $(b',y)$-TITS gadget, a contradiction to our assumption.
                                                \item[2.] If $x$ is assigned 1 by a two-valued state of $s_{i2}$, then $y$ has to be assigned 1 by all two-valued states of $s_{j1}\cup s_{j3}$. Consequently, $x$ cannot be assigned 0 by a $s_{i2}$, because else, it can be lifted to the required separation of $x$ and $y$. Hence, $G_i$ (and therefore $G$) is a $(b,x)$-TITS gadget, a contradiction to our assumption.
                                                \item[3.] If $x$ is assigned 1 by a two-valued state of $s_{i3}$, then $y$ has to be assigned 1 by all two-valued states of $s_{j1}\cup s_{j2}$. This means that $G_j$ (and therefore $G$) is a $(b',y)$-TITS gadget (and also a $(c',y)$-TITS gadget), again a contradiction to our assumption.
                                        \end{itemize}
                                \item[Case 2.3] \emph{$G_i$ and $G_j$ lie on different layers of $B(G)$, but only one end from $G_i$ and one end from $G_j$ lie on the same context.} Again without loss of generality, suppose that $G_i$ has $a$ and $b$ and $G_j$ has $b'$ and $c'$ as their heads and tails, respectively (see Figure \ref{fig-proof-sep-lem} (c)). Therefore, every two-valued state of the induced subhypergraph $G_i \cup G_j$ in $B(G)$ are members of $s_{i1} \cup s_{j1}\cup s_{j2}\cup s_{j3}$, $s_{i2} \cup s_{j2}\cup s_{j3}$, or $s_{i3}\cup s_{j1}\cup s_{j2} \cup s_{j3}$ (or $s_{j1} \cup s_{i1}\cup s_{i2}\cup s_{i3}$, $s_{j2} \cup s_{i2}\cup s_{i3}$, or $s_{j3}\cup s_{i1}\cup s_{i2} \cup s_{i3}$ which can be treated similarly). Suppose on contrary that $x$ and $y$ receive the same value by all two-valued states of $B(G)$. Then, since there is a two-valued state of $B(G)$ that assigns 1 to $x$, at least one of the following statements holds:

                                        \begin{itemize}
                                                \item[1.] If $x$ is assigned 1 by a two-valued state of $s_{i1}$, then $y$ has to be assigned 1 by all two-valued states of $s_{j1}\cup s_{j2}\cup s_{j3}$, a contradiction to the fact that no vertex of $G$ can be assigned 1 by all the two-valued states.
                                                \item[2.] If $x$ is assigned 1 by a two-valued state of $s_{i2}$, then $y$ has to be assigned 1 by all two-valued states of $s_{j2}\cup s_{j3}$. Consequently, $G_j$ (and therefore $G$) is a $(c',y)$-TITS gadget, a contradiction to our assumption.
                                                \item[3.] If $x$ is assigned 1 by a two-valued state of $s_{i3}$, then $y$ has to be assigned 1 by all two-valued states of $s_{j1}\cup s_{j2}\cup s_{j3}$, a contradiction to the fact that no vertex of $G$ can be assigned 1 by all the two-valued states.
                                        \end{itemize}

                        \end{itemize}
                \end{itemize}


                We showed that in any case, $x$ and $y$ can be separated by a two-valued state of $B(G)$ (or else there is a contradiction) which concludes the proof.
        \end{proof}


        \begin{figure}
        \begin{center}
                \begin{tabular}{ c c c c c }

                        \begin{tikzpicture} [scale=0.44]

                                \tikzstyle{every path}=[line width=1pt]

                                \newdimen\ms
                                \ms=0.15cm
                                \tikzstyle{s1}=[color=black,fill,rectangle,inner sep=3]
                                \tikzstyle{c1}=[draw=gray,fill=white,circle,inner sep={1}]

                                % Define positions of all observables


                                \coordinate (a1) at (0,0);
                                \coordinate (a2) at (2,0);
                                \coordinate (a3) at (1,1.732);
                                \coordinate (a4) at (-2,-1.5);
                                \coordinate (a5) at (4,-1.5);
                                \coordinate (a6) at (1,3.6961);
                                \coordinate (a7) at (-4,-3);
                                \coordinate (a8) at (6,-3);
                                \coordinate (a9) at (1,5.6602);
                                % draw contexts


                                \draw [color=lightgray, ->,snake=coil,segment aspect=0,thick] (a1) to node[below] { } (a2);
                                \draw [color=lightgray, ->,snake=coil,segment aspect=0,thick] (a2) to node[right] { } (a3);
                                \draw [color=lightgray, ->,snake=coil,segment aspect=0,thick] (a3) to node[left] {  } (a1);
                                \draw [color=lightgray, ->,snake=coil,segment aspect=0,thick] (a4) to node[below] { } (a5);
                                \draw [color=lightgray, ->,snake=coil,segment aspect=0,thick] (a5) to node[right] { } (a6);
                                \draw [color=lightgray, ->,snake=coil,segment aspect=0,thick] (a6) to node[left] {  } (a4);
                                \draw [color=violet, ->,snake=coil,segment aspect=0,thick] (a7) to node[below] {$G_i$} (a8);
                                \draw [color=lightgray, ->,snake=coil,segment aspect=0,thick] (a8) to node[right] { } (a9);
                                \draw [color=orange, ->,snake=coil,segment aspect=0,thick] (a9) to node[left] {$G_j \textnormal{       }  $  } (a7);


                                \draw [color=lightgray] (a1) -- (a7);
                                \draw [color=lightgray] (a2) -- (a8);
                                \draw [color=lightgray] (a3) -- (a9);

                                % draw atoms

                                \draw (a1) coordinate[c1];
                                \draw (a2) coordinate[c1];
                                \draw (a3) coordinate[c1];
                                \draw (a4) coordinate[c1];
                                \draw (a5) coordinate[c1];
                                \draw (a6) coordinate[c1];
                                \draw (a7) coordinate[c1,label=below:$a$];
                                \draw (a8) coordinate[c1,label=below:$b$];
                                \draw (a9) coordinate[c1,label=above:$c$];
                        \end{tikzpicture}

        & \hspace{0.6 cm} &

                \begin{tikzpicture} [scale=0.44]

                        \tikzstyle{every path}=[line width=1pt]

                        \newdimen\ms
                        \ms=0.15cm
                        \tikzstyle{s1}=[color=black,fill,rectangle,inner sep=3]
                        \tikzstyle{c1}=[draw=gray,fill=white,circle,inner sep={1}]

                        % Define positions of all observables


                        \coordinate (a1) at (0,0);
                        \coordinate (a2) at (2,0);
                        \coordinate (a3) at (1,1.732);
                        \coordinate (a4) at (-2,-1.5);
                        \coordinate (a5) at (4,-1.5);
                        \coordinate (a6) at (1,3.6961);
                        \coordinate (a7) at (-4,-3);
                        \coordinate (a8) at (6,-3);
                        \coordinate (a9) at (1,5.6602);
                        % draw contexts


                        \draw [color=lightgray, ->,snake=coil,segment aspect=0,thick] (a1) to node[below] { } (a2);
                        \draw [color=lightgray, ->,snake=coil,segment aspect=0,thick] (a2) to node[right] { } (a3);
                        \draw [color=lightgray, ->,snake=coil,segment aspect=0,thick] (a3) to node[left] {  } (a1);
                        \draw [color=orange, ->,snake=coil,segment aspect=0,thick] (a4) to node[above] {$G_j$ } (a5);
                        \draw [color=lightgray, ->,snake=coil,segment aspect=0,thick] (a5) to node[right] { } (a6);
                        \draw [color=lightgray, ->,snake=coil,segment aspect=0,thick] (a6) to node[left] {  } (a4);
                        \draw [color=violet, ->,snake=coil,segment aspect=0,thick] (a7) to node[below] {$G_i$} (a8);
                        \draw [color=lightgray, ->,snake=coil,segment aspect=0,thick] (a8) to node[right] { } (a9);
                        \draw [color=lightgray, ->,snake=coil,segment aspect=0,thick] (a9) to node[left] {  } (a7);


                        \draw [color=lightgray] (a1) -- (a7);
                        \draw [color=lightgray] (a2) -- (a8);
                        \draw [color=lightgray] (a3) -- (a9);

                        % draw atoms

                        \draw (a1) coordinate[c1];
                        \draw (a2) coordinate[c1];
                        \draw (a3) coordinate[c1];
                        \draw (a4) coordinate[c1,label=below:$a'$];
                        \draw (a5) coordinate[c1,label=below:$b'$];
                        \draw (a6) coordinate[c1];
                        \draw (a7) coordinate[c1,label=below:$a$];
                        \draw (a8) coordinate[c1,label=below:$b$];
                        \draw (a9) coordinate[c1,label=above:$c$];
                \end{tikzpicture}

                & \hspace{0.6 cm} &

                \begin{tikzpicture} [scale=0.44]

                        \tikzstyle{every path}=[line width=1pt]

                        \newdimen\ms
                        \ms=0.15cm
                        \tikzstyle{s1}=[color=black,fill,rectangle,inner sep=3]
                        \tikzstyle{c1}=[draw=gray,fill=white,circle,inner sep={1}]

                        % Define positions of all observables


                        \coordinate (a1) at (0,0);
                        \coordinate (a2) at (2,0);
                        \coordinate (a3) at (1,1.732);
                        \coordinate (a4) at (-2,-1.5);
                        \coordinate (a5) at (4,-1.5);
                        \coordinate (a6) at (1,3.6961);
                        \coordinate (a7) at (-4,-3);
                        \coordinate (a8) at (6,-3);
                        \coordinate (a9) at (1,5.6602);
                        % draw contexts


                        \draw [color=lightgray, ->,snake=coil,segment aspect=0,thick] (a1) to node[below] { } (a2);
                        \draw [color=lightgray, ->,snake=coil,segment aspect=0,thick] (a2) to node[right] { } (a3);
                        \draw [color=lightgray, ->,snake=coil,segment aspect=0,thick] (a3) to node[left] {  } (a1);
                        \draw [color=orange, ->,snake=coil,segment aspect=0,thick] (a4) to node[below] { $G_j$ } (a5);
                        \draw [color=lightgray, ->,snake=coil,segment aspect=0,thick] (a5) to node[right] { } (a6);
                        \draw [color=lightgray, ->,snake=coil,segment aspect=0,thick] (a6) to node[left] {  } (a4);
                        \draw [color=lightgray, ->,snake=coil,segment aspect=0,thick] (a7) to node[below] { } (a8);
                        \draw [color=lightgray, ->,snake=coil,segment aspect=0,thick] (a8) to node[right] { } (a9);
                        \draw [color=violet, ->,snake=coil,segment aspect=0,thick] (a9) to node[left] {$G_i \textnormal{       }  $  } (a7);


                        \draw [color=lightgray] (a1) -- (a7);
                        \draw [color=lightgray] (a2) -- (a8);
                        \draw [color=lightgray] (a3) -- (a9);

                        % draw atoms

                        \draw (a1) coordinate[c1];
                        \draw (a2) coordinate[c1];
                        \draw (a3) coordinate[c1];
                        \draw (a4) coordinate[c1,label=below:$b'$];
                        \draw (a5) coordinate[c1,label=below:$c'$];
                        \draw (a6) coordinate[c1];
                        \draw (a7) coordinate[c1,label=below:$b$];
                        \draw (a8) coordinate[c1,label=below:$c$];
                        \draw (a9) coordinate[c1,label=above:$a$];
                \end{tikzpicture}

        \\

                (a)&&(b)&&(c)
                \end{tabular}
        \end{center}
        \caption{\label{fig-proof-sep-lem}
                Illustrations of cases 2.1 (a), 2.2 (b) and 2.3 (c) in the proof of Lemma \ref{Sep-lem}.}
        \end{figure}

        We can go further by calculating the number of two-valued states of $B(G)$ based on $G$'s. Suppose that $G$ is an $(a,b)$-TIFS and has respectively $n_a$, $n_b$ and $n_n$ two-valued states that give $a$ true, $b$ true and none of $a$ and $b$ true. In other words, $n_a =\vert s_1 \vert$, $n_b = \vert s_2 \vert$ and $n_n = \vert s_3 \vert$. Then, by the elementary methods of counting, the number of two-valued states of $B(G)$ is
        \begin{equation}\label{nTS}
        nTS(B(H))= (3+2+1)\cdot n_a^3 \cdot n_b^3 \cdot n_n^3 .
        \end{equation}
    For the first layer we have three choices for an end vertex to be true, for the second layer we only have two such choices and for the third layer only one such a choice we have got. That is why we have 3+2+1=6 choices.

        For the sake of an example take the Specker bug $G$  discussed in the Appendix Section~\ref{Specker-Bug}. From its Travis matrix \ref{Specker-Bug-Travice} we know that $n_a = 3$, $n_b =3$ and $n_n=8$. Therefore, Formula \ref{nTS} implies that $B(G)$, which is a separable hypergraph on 108 vertices and 66 contexts, has $$6\cdot 3^3 \cdot 3^3\cdot 8^3=2,239,488$$ two-valued states, a number that can easily be checked via an ordinary computer.

        However, it is not difficult to show that this hypergraph $B(G)$, when $G$ is the Specker bug, does not meet our requirements. This is because the two ends of the Specker bug cannot be orthogonal in the 3-space \cite{Cabello-1996-diss}. It can also be discussed using graph theoretical terminology; one orthogonality hypergraph cannot have a cycle of length 4 because else any pair of antipodal vertices of the cycle of length 4 have to be colinear. Therefore, even if $B(G)$ is an orthogonality hypergraph, the reconstructed hypergraph with an extra context of $\{a,b,c\}$ is certainly not.

        We have to find a TIFS gadget $H$, other than the Specker bug, in which the distance between its two ends are not 3 (so the two ends can be orthogonal in the reconstructed hypergraph). A candidate for such a hypergraph is shown in Figure~\ref{Baba-Taher}, which is a TIFS on 43 vertices whose end-points (say $a$ and $b$) are far enough, so that it is not only separable, but also probably a FOR. This hypergraph has 2589 two-valued states, 45 of which assign $a$ true, 504 give $b$ true and 2040 give both $a$ and $b$ false. In other words, for the hypergraph $H$ we have $n_a=45$, $n_b=504$ and $n_n=2048$.

        If we construct $B(H)$, then Lemma \ref{Sep-lem} implies that the resulting hypergraph on 378 vertices is separable and there would be enough space for the three end vertices to be perpendicular.

        \begin{figure}
                \begin{center}
                        \begin{tikzpicture}  [scale=0.8]

                                \newdimen\ms
                                \ms=0.1cm

                                \tikzstyle{every path}=[line width=1pt]

                                %\tikzstyle{c1}=[color=black,fill,circle,inner sep={\ms/8},minimum size=2*\ms]
                                \tikzstyle{c1}=[draw=gray,fill=white,circle,inner sep={\ms/1}]
                                \tikzstyle{c2}=[color=blue,fill,circle,inner sep={\ms/8},minimum size=2*\ms]
                                \tikzstyle{c3}=[color=red,fill,circle,inner sep={\ms/8},minimum size=2*\ms]

                                % Radius of regular polygons
                                \newdimen\R
                                \R=30mm     % outer circle

                                %\r= { \R * sqrt(3) }     % inner circle
                                %\newdimen\r
                                %\r=    {\R * sqrt(3)/2}       % inner circle

                                %\newdimen\K
                                %\K=3cm

                                % Define positions of all observables
                                %


                                \coordinate (a1) at (0,0);
                                \coordinate (a2) at (1,1);
                                \coordinate (a3) at (2,2);
                                \coordinate (a4) at (3,2);
                                \coordinate (a5) at (4,2);
                                \coordinate (a6) at (5,1.5);
                                \coordinate (a7) at (6,1);
                                \coordinate (a8) at (7,1.5);
                                \coordinate (a9) at (8,2);
                                \coordinate (a10) at (9,1);
                                \coordinate (a11) at (10,0);
                                \coordinate (a12) at (9,-1);
                                \coordinate (a13) at (8,-2);
                                \coordinate (a14) at (7,-1.5);
                                \coordinate (a15) at (6,-1);
                                \coordinate (a16) at (5,-1.5);
                                \coordinate (a17) at (4,-2);
                                \coordinate (a18) at (3,-2);
                                \coordinate (a19) at (2,-2);
                                \coordinate (a20) at (1,-1);
                                \coordinate (a21) at (3,0);
                                % draw contexts

                                %                       \draw[line width=5pt] (0,0) ..  (1,2) .. (2,3)  ..  (3,3) .. (5,2.5)  ..  (6,2) .. (7,1.5);
                                %\draw [color=magenta, dashed,snake=coil,segment amplitude=5pt,thick] (a1) to [bend left=70] (a8);
                                %\draw [color=cyan, dashed,snake=zigzag,segment amplitude=5pt,thick] (a1) to [bend right=70] (a14);
                                \draw [color=magenta, decoration = snake,decorate] (a1) to [bend left=90]  (a8);
                                \draw [color=cyan, decoration = snake,decorate] (a1) to [bend right=90]  (a14);

                                \draw [color=orange] (a1) -- (a3);
                                \draw [color=green] (a3) -- (a5);
                                \draw [color=brown] (a5) -- (a7);
                                \draw [color=olive] (a7) -- (a9);
                                \draw [color=darkblue] (a9) -- (a11);
                                \draw [color=teal] (a11) -- (a13);
                                \draw [color=red] (a13) -- (a15);
                                \draw [color=purple] (a15) -- (a17);
                                \draw [color=blue] (a17) -- (a19);
                                \draw [color=violet] (a19) -- (a1);
                                \draw [color=lightgray] (a4) -- (a18);
                                % draw atoms

                                \draw (a1) coordinate[c1,draw=red,fill=red,label=left:$a$];
                                \draw (a2) coordinate[c1];
                                \draw (a3) coordinate[c1];
                                \draw (a4) coordinate[c1];
                                \draw (a5) coordinate[c1];
                                \draw (a6) coordinate[c1];
                                \draw (a7) coordinate[c1];
                                \draw (a8) coordinate[c1,draw=green,fill=green,label=below:$c$];
                                \draw (a9) coordinate[c1];
                                \draw (a10) coordinate[c1];
                                \draw (a11) coordinate[c1,draw=green,fill=green,label=right:$b$];
                                \draw (a12) coordinate[c1];
                                \draw (a13) coordinate[c1];
                                \draw (a14) coordinate[c1,draw=green,fill=green,label=above:$d$];
                                \draw (a15) coordinate[c1];
                                \draw (a16) coordinate[c1];
                                \draw (a17) coordinate[c1];
                                \draw (a18) coordinate[c1];
                                \draw (a19) coordinate[c1];
                                \draw (a20) coordinate[c1];
                                \draw (a21) coordinate[c1];
                        \end{tikzpicture}
                \end{center}
                \caption{\label{Baba-Taher}
                        An $(a,b)$-TIFS gadget whose distance between its two terminal points $a$ and $b$ is at least five contexts. The snake-like decorated curves indicate Specker bugs, so this hypergraph has 43 vertices.
                }
        \end{figure}

        The hypergraph $B(H)$ is on 378 vertices and 228 contexts, and by Formula \ref{nTS} it has $$6\cdot 45^3\cdot 504^3\cdot 2040 ^3=594,252,343,817,330,688,000,000$$ two-valued states. This huge number makes it hard to quickly check by an ordinary computer.

        Another pertinent problem is to show that these hypergraphs---namely $B(H)$ and also its counterpart, the reconstructed hypergraph $B(H)'$ from $B(H)$'s table of two-valued states---have a faithful orthogonal representation; and to enumerate an explicit example of such a representation.

        However, it seems that Criterion \ref{orthogonality} is independent from Criteria \ref{completion} and \ref{AdCr}, having an example such as $B(H)$ on 378 vertices raises the possibility that Conjecture \ref{c2} is false for separable hypergraphs that are not perfectly-separable.

        \section{Two-valued states vs coloring}\label{color-conj}

        \subsection{Coloring vs independent partitions}
        A hypergraph (whose vertices all lie on at least one hyperedge of size $\omega([H]_2 )=n$) is called to have ``an $n$-partition system'' if there is a vertex partition $\mathcal{S}=\{S_1 , \ldots , S_n \}$ of $H$ (into exactly $n$ cells) with the following properties:
        \begin{itemize}
                \item[1.] whenever $v,w \in S_i$ for an $i=1,\ldots , n$, we have $v$ and $w$ are not adjacent, and
                \item[2.] for each $i=1,\ldots , n$ and every $v \in V(G)$, there is a vertex $w \in S_i$ such that either $v =w$ or $v$ and $w$ are adjacent.
        \end{itemize}

        When a hypergraph $H$ has an $n$-partition system, we might simply say that $H$ is $n$-partitionable. While $\mathcal{S}$ is a partition, every $S_i$ is non-empty and $\bigcup_{i=1}^{n} S_i = V(G)$. With Property 1, we can be sure that if we assign the true value to all the vertices in $S_i$ and the false value to the rest of vertices, then no two true vertices are adjacent, and consequently, every context has at most one true valued vertex. Moreover, Property 2 assures us that there is no context without a true-valued vertex. (Using graph theoretical terminology, these two properties mean that every $S_i$ is an \emph{independent dominating set} for $H$.)  Therefore, an $n$-partition system actually induces $n$ two-valued states on $H$.% and separates its vertices.

        We have the following theorem.

        \begin{theorem}\label{separable}
                A hypergraph $H$ is $n$-colorable if and only if it is $n$-partitionable.
        \end{theorem}
        \begin{proof}
                It is evident that if $\sigma$ is a proper vertex coloring of $G$ with $\{1,\ldots , n\}$, we can easily find $\mathcal{S}=\{S_1 , \ldots , S_k \}$ by putting $$S_i =\{v\in V(G)\; : \; \sigma(v)=i \}.$$ It is also clear that property 1 holds because $\sigma$ is a proper coloring. Moreover, we also have property 2 because of the assumption we made that, in $[H]_2$, every vertex is on a maximal clique of size $n$.

                To prove the converse, suppose that we have a partition $\mathcal{S}=\{S_1 , \ldots , S_n \}$ of vertices of $H$ which satisfies properties 1 and 2 above. Define, $\sigma: V(G) \longrightarrow \{1,\ldots , n\}$ such that

                \begin{center}
                        $\sigma(v)=i$ if $v\in S_i.$
                \end{center}

                While property 2 implies that every $S_i$ is non-empty, for $i=1\ldots , n$, it shows also that every hyperedge contains a vertex $v$ such that $\sigma (v)=i$. In other words, every color $i=1,\ldots,k$ is used in each hyperedge. Furthermore, property 1 implies that every $S_i$ is an independent set. Hence $\sigma$ is a proper coloring of $H$ and consequently, $H$ is $n$-colorable.
        \end{proof}

        As a result, we can say that for each proper $n$-coloring of $H$ we have $n$ different two-valued states on vertices of $H$. Conversely we can construct exactly $n!$ proper $n$-colorings for $H$ from an available $n$-partition system on vertices of $H$. %The factorial shows up here because every permutation of colors produces a new coloring, and there are exactly $n!$ permutations for $n$ colors.
        Therefore, Conjecture \ref{c1} is a corollary of Theorem~\ref{separable}.


        \subsection{Reconstructing coloring from logical assignments}

        From Theorem~\ref{separable} we know that when there is an $n$-coloring for an orthogonality hypergraph $H$, there are $n$ two-valued states corresponding to it so that they induce a partition logic. In other words, there are $n$ rows in the Travis matrix of $H$ such that when one of them assigns 1 to a vertex $u$, the rest of them assign 0 to $u$. Consequently, if $t_{s_1},\ldots,t_{s_n}$ are the rows of the Travis matrix $T(H)$ corresponding to an $n$-coloring and $t_{s_i} + t_{s_j}$ be the element-wise sum of $t_{s_i}$ and $t_{s_j}$ in $\mathbb{Z}_2$, then this yields the $\vert V(H)\vert$--tuple whose entries are one:
                \begin{equation}\label{color-all-two-states}
                \sum_{i=1}^{n} t_{s_i} = \Big(\underbrace{\raisebox{-0pt}{1,1,\ldots,1}}_{\vert V(H)\vert \text{ times}}\Big)=\begin{pmatrix}\mathbf{1}_{1\times \vert V(H)\vert}\end{pmatrix}.
        \end{equation}
        Therefore, when $H$ is $n$-colorable, there is at least one set of $n$ two-valued states that induce a partition logic on $H$.

        Algorithm \ref{algorithm1} searches for such states when $T(H)$ is available. It
        \begin{enumerate}
            \item[(1)] takes the Travis matrix $T(H)$ and the clique number $n$, and
            \item[(2)] gives a list of rows, $A$, from which we can retrieve an $n$ coloring for $H$.
        \end{enumerate}

        Variables of this algorithm are
        \begin{enumerate}
            \item[(1)] $AvailableRows$ which is a list of active rows in $T(H)$, with each such row representing a two-valued state of $H$,
            \item[(2)] $i$ which runs from $1$ to the clique number $n$,
            \item[(3)] $j$ which runs from $1$ to the number of two-valued states $nTS(H)$ which is the number of rows of the Travis matrix, and
            \item[(4)] $RemovedRows$ which is a list of lists, whose $i$th element is the rows of $T(H)$ that become inactive at the $i$th step of filling $A$.
        \end{enumerate}

        \def\NoNumber#1{{\def\alglinenumber##1{}\State #1}\addtocounter{ALG@line}{-1}}

        \begin{algorithm}
                \caption{Finding an $n$-coloring for $H$  from its set of two-valued states encoded by the Travis matrix}\label{algorithm1}
                \begin{flushleft} $\;$\\ \hspace*{\algorithmicindent}
                    \textbf{Input:} $T(H)$, $n$ \Comment{Travis matrix, clique number}\\
                        \hspace*{\algorithmicindent} \textbf{Output:} $A$ \Comment{a list of $n$ rows of $T(H)$}
                \end{flushleft}
                \begin{algorithmic}[1]

                        \State $i\gets 1$ \Comment{start of variable initialization}
                        \State $AvailableRows\gets (1,\ldots,nTS(H))$
                        \State $A\gets (\;)$
                        \State $RemovedRows\gets (\;)$ \Comment{end of variable initialization}
                        \NoNumber{ }
                        \While{$i\leq n$ and ($i\neq 1$ or $AvailableRows\neq \emptyset$)} \Comment{try all colors}

                        \If{$AvailableRows=\emptyset$} \Comment{start over again if all two-valued states are exhausted}

                        \State {\it Append} $RemovedRows[i]$ to $AvailableRows$
                        \State {\it Remove} $RemovedRows[i]$ from $RemovedRows$
                        \State {\it Remove} $A[i]$ from $A$
                        \State $i\gets i-1$

                        \Else \Comment{try to identify a new color assignment by the next available two-valued state}

                        \State $j\gets$ first available cell in $AvailableRows$
                        \State $A[i] \gets AvailableRows[j]$
                        \State {\it Append} $AvailableRows[j]$ to $RemovedRows[i]$
                        \State {\it Remove} $AvailableRows[j]$ from $AvailableRows$
                        \State $i\gets i+1$
                        \State {\it Append} to $RemovedRows[i]$ all $AvailableRows[s]$  for which there is a vertex $u$
                        \NoNumber{\hspace{1.2 cm}such that the state of rows}
                        $AvailableRows[s]$ and $AvailableRows[j]$ both assign 1 to $u$
                        \State {\it Remove} all elements of $RemovedRows[i]$ from $AvailableRows$
                        \EndIf

                        \EndWhile
                \end{algorithmic}
        \end{algorithm}

        If the output of Algorithm \ref{algorithm1} has less than $n$ elements, then $H$ has no admissible $n$-coloring---because else Theorem \ref{separable} guarantees that there are $n$ two-valued states partitioning the logic, in which case Algorithm \ref{algorithm1} would have given $\vert A \vert = n$. If $\vert A \vert =n$, then $A$ is a list of $n$ rows in $T(H)$, each of which corresponds to a color class of an $n$-coloring of $H$. In other words, when $s\in A$, the two-valued state $t_s$ presents the color class consisting all the vertices it assigns, or maps to, 1. It is evident that the resulting color classes are independent sets while Formula \ref{color-all-two-states} implies that they cover all the vertices. Consequently, $A$ induces a proper $n$-coloring on vertices of $H$.

        Algorithm \ref{algorithm1} is not highly efficient in finding an $n$-coloring for $H$. The main reason is that in the worst case study it has to check all the two-valued states of $H$ whose number, i.e. $nTS(H)$, can grow exponentially in terms of the clique number, number of vertices and hyperedges.

        Moreover, one could conjecture that Algorithm~\ref{algorithm1} could be modified to render a coloring even if the (hyper)graph is not $n$-partitionable, in which case Theorem~\ref{separable} does not apply.
    Because even if one has exhausted all combinations of two-valued states one could still attempt to ``complete'' the coloring by identifying the missing colors with
    ``suitable segments'' of the remaining two-valued states (if there are any leftovers). Of course in this way the column sums of all the respective two valued states cannot be 1; and hence formula~(\ref{color-all-two-states}) is no longer valid.
    In any case, Brooks' theorem~\cite{Brooks1941,Lovasz1975}---stating that for any connected undirected graph $G$ the chromatic number of $G$ is at most its maximum degree (the maximal number of edges that are incident to some vertex) $\Delta$  unless $G$ is a complete graph or an odd cycle, in which case the chromatic number is $\Delta + 1$---and its generalization to hypergraphs~\cite[page 45, Theorem 3.2]{Bretto-MR3077516} yield an upper bound for the chromatic number of such (hyper)graphs.

        %\clearpage

        \section{Summary and concluding remarks}


        We have presented a constructive, algorithmic way to generate a coloring of a (hyper)graph from its set of two-valued states. By this, we have been able to find a ``compact'' partition logic within the logical states of the hypergraph. The only criterion for a success of this approach is by ascertaining that the respective hypergraph is semi-perfect, that is, its chromatic number equals to the clique number of its 2-section.

        With regard to representing and reconstructing (hyper)graphs or logics in terms of their two-valued states, in particular, regarding separability of vertices or elementary propositions, we conjecture that there exist quantum logics with a separable set of two-valued states that cannot be reconstructed from these states. We have presented a hypergraph, namely  $B(G)$ depicted in Figure~\ref{TIFS-non-Rec} in Section~\ref{Rec-B(H)} with a TIFS gadget such as the one depicted in Figure~\ref{Baba-Taher}, that has this characteristic but we could not find a faithful orthogonal representation in a Hilbert space.

        Yet, stronger forms of separability, in particular, perfect separability, can be identified that allow (hyper)graphs or logics to be represented (by their Travis matrices) and reconstructed in terms of their two-valued states. And while the conditions on perfectly separable (hyper)graphs are rather strong, one can be certain that such a reconstruction exists.

        Indeed, such a reconstruction helps to directly identify mutually perpendicular elementary propositions, and thus the contexts corresponding to the maximal operators they form: if an orthogonality (hyper)graph is reconstructable from its set of two-valued states we can deduce the mutual orthogonality of the elementary quantum propositions by just looking at these two-valued states. This facilitates the construction of the (mutually perpendicular) orthogonal operators in the spectral sums associated with the contexts, and thus supports finding a global faithful orthogonal representation, i.~e., the assignment of vectors to vertices, of (hyper)graphs.

        So while for perfectly separable (hyper)graphs we can be certain that they can be reconstructed; and for Kochen-Specker type (hyper)graphs that they cannot be reconstructed, because there is no two-valued state associated with any classical value assignment, for the remaining (hyper)graphs reconstructability remains an open question.




        \appendix


        \section{Examples}
        \subsection{Triangle logic}

        The coloring procedure of the triangle hypergraph is depicted in Figure~\ref{2020-f-chroma-triangle3}.
        Consider the set of all four two-valued states on the six atoms which can be tabulated by a
        (compactified) Travis matrix $T_{ij}$
        whose rows indicate the
        $i$th state $s_i$ and whose columns
        indicate the atoms $a_j$, respectively; that is, $T_{ij}=s_i(a_j)$:
        \begin{equation}
                T_{ij}=\begin{pmatrix}
                        {\color{red}1}&{\color{red}0}&{\color{red}0}&{\color{red}1}&{\color{red}0}&{\color{red}0}\\
                        {\color{blue}0}&{\color{blue}0}&{\color{blue}1}&{\color{blue}0}&{\color{blue}0}&{\color{blue}1}\\
                        {\color{green}0}&{\color{green}1}&{\color{green}0}&{\color{green}0}&{\color{green}1}&{\color{green}0}\\
                        0&1&0&1&0&1
                \end{pmatrix}
                .
        \end{equation}
        It is not too difficult to see that the first three measures, represented by the first three row vectors of the
        Travis matrix, add up to
        $
        \begin{pmatrix}
                1,1,1,1,1,1
        \end{pmatrix}
        $. They can thus be taken as the basis of a coloring.

        \begin{figure}
                \begin{center}
                        \begin{tabular}{ c c c c }
                                \begin{tikzpicture}  [scale=0.8]

                                        \tikzstyle{every path}=[line width=1pt]

                                        \newdimen\ms
                                        \ms=0.1cm
                                        \tikzstyle{s1}=[fill=red,draw=red,circle,inner sep=2]
                                        \tikzstyle{c1}=[fill=white,draw=green,circle,inner sep={\ms/8},minimum size=2*\ms]

                                        % Define positions of all observables


                                        \coordinate (a1) at  (1,2);
                                        \coordinate (a2) at (1.5,1);
                                        \coordinate (a3) at (2,0);
                                        \coordinate (a4) at (1,0);
                                        \coordinate (a5) at (0,0);
                                        \coordinate (a6) at (0.5,1);
                                        \coordinate (c) at (1,0.6);

                                        % draw contexts

                                        \draw [color=olive] (a1) -- (a3);
                                        \draw [color=cyan] (a3) -- (a5);
                                        \draw [color=orange] (a5) -- (a1);


                                        % draw atoms

                                        \draw (a1) coordinate[s1,label=above:$a_1$];
                                        \draw (a2) coordinate[c1,label=right:$a_2$];
                                        \draw (a3) coordinate[c1,label=below:$a_3$];
                                        \draw (a4) coordinate[s1,label=below:$a_4$];
                                        \draw (a5) coordinate[c1,label=below:$a_5$];
                                        \draw (a6) coordinate[c1,label=left:$a_6$];
                                        \node at (c) {\large $s_1$};

                                \end{tikzpicture}
                                &
                                \begin{tikzpicture}  [scale=0.8]

                                        \tikzstyle{every path}=[line width=1pt]

                                        \newdimen\ms
                                        \ms=0.1cm
                                        \tikzstyle{s1}=[fill=red,draw=red,circle,inner sep=2]
                                        \tikzstyle{c1}=[fill=white,draw=green,circle,inner sep={\ms/8},minimum size=2*\ms]

                                        % Define positions of all observables


                                        \coordinate (a1) at  (1,2);
                                        \coordinate (a2) at (1.5,1);
                                        \coordinate (a3) at (2,0);
                                        \coordinate (a4) at (1,0);
                                        \coordinate (a5) at (0,0);
                                        \coordinate (a6) at (0.5,1);
                                        \coordinate (c) at (1,0.6);

                                        % draw contexts

                                        \draw [color=olive] (a1) -- (a3);
                                        \draw [color=cyan] (a3) -- (a5);
                                        \draw [color=orange] (a5) -- (a1);


                                        % draw atoms

                                        \draw (a1) coordinate[c1,label=above:$a_1$];
                                        \draw (a2) coordinate[c1,label=right:$a_2$];
                                        \draw (a3) coordinate[s1,label=below:$a_3$];
                                        \draw (a4) coordinate[c1,label=below:$a_4$];
                                        \draw (a5) coordinate[c1,label=below:$a_5$];
                                        \draw (a6) coordinate[s1,label=left:$a_6$];
                                        \coordinate (c) at (1,0.6);
                                        \node at (c) {\large $s_2$};

                                \end{tikzpicture}
                                &
                                \begin{tikzpicture}  [scale=0.8]

                                        \tikzstyle{every path}=[line width=1pt]

                                        \newdimen\ms
                                        \ms=0.1cm
                                        \tikzstyle{s1}=[fill=red,draw=red,circle,inner sep=2]
                                        \tikzstyle{c1}=[fill=white,draw=green,circle,inner sep={\ms/8},minimum size=2*\ms]

                                        % Define positions of all observables


                                        \coordinate (a1) at  (1,2);
                                        \coordinate (a2) at (1.5,1);
                                        \coordinate (a3) at (2,0);
                                        \coordinate (a4) at (1,0);
                                        \coordinate (a5) at (0,0);
                                        \coordinate (a6) at (0.5,1);
                                        \coordinate (c) at (1,0.6);

                                        % draw contexts

                                        \draw [color=olive] (a1) -- (a3);
                                        \draw [color=cyan] (a3) -- (a5);
                                        \draw [color=orange] (a5) -- (a1);


                                        % draw atoms

                                        \draw (a1) coordinate[c1,label=above:$a_1$];
                                        \draw (a2) coordinate[s1,label=right:$a_2$];
                                        \draw (a3) coordinate[c1,label=below:$a_3$];
                                        \draw (a4) coordinate[c1,label=below:$a_4$];
                                        \draw (a5) coordinate[s1,label=below:$a_5$];
                                        \draw (a6) coordinate[c1,label=left:$a_6$];
                                        \node at (c) {\large $s_3$};

                                \end{tikzpicture}
                                &
                                \begin{tikzpicture}  [scale=0.8]

                                        \tikzstyle{every path}=[line width=1pt]

                                        \newdimen\ms
                                        \ms=0.1cm
                                        \tikzstyle{s1}=[fill=red,draw=red,circle,inner sep=2]
                                        \tikzstyle{c1}=[fill=white,draw=green,circle,inner sep={\ms/8},minimum size=2*\ms]

                                        % Define positions of all observables


                                        \coordinate (a1) at  (1,2);
                                        \coordinate (a2) at (1.5,1);
                                        \coordinate (a3) at (2,0);
                                        \coordinate (a4) at (1,0);
                                        \coordinate (a5) at (0,0);
                                        \coordinate (a6) at (0.5,1);
                                        \coordinate (c) at (1,0.6);

                                        % draw contexts

                                        \draw [color=olive] (a1) -- (a3);
                                        \draw [color=cyan] (a3) -- (a5);
                                        \draw [color=orange] (a5) -- (a1);


                                        % draw atoms

                                        \draw (a1) coordinate[c1,label=above:$a_1$];
                                        \draw (a2) coordinate[s1,label=right:$a_2$];
                                        \draw (a3) coordinate[c1,label=below:$a_3$];
                                        \draw (a4) coordinate[s1,label=below:$a_4$];
                                        \draw (a5) coordinate[c1,label=below:$a_5$];
                                        \draw (a6) coordinate[s1,label=left:$a_6$];
                                        \node at (c) {\large $s_4$};

                                \end{tikzpicture}
                                %
                                \\
                                %
                                (a)&(b)&(c)&(d)\\
                                %
                        \end{tabular}
                        %
                        \\
                        %
                        \begin{tabular}{ c c c}
                                %
                                \begin{tikzpicture}  [scale=0.8]

                                        \tikzstyle{every path}=[line width=1pt]

                                        \newdimen\ms
                                        \ms=0.1cm
                                        \tikzstyle{c2}=[circle,inner sep={\ms/8},minimum size=3*\ms]
                                        \tikzstyle{c1}=[circle,inner sep={\ms/8},minimum size=2*\ms]

                                        % Define positions of all observables


                                        \coordinate (a1) at  (1,2);
                                        \coordinate (a2) at (1.5,1);
                                        \coordinate (a3) at (2,0);
                                        \coordinate (a4) at (1,0);
                                        \coordinate (a5) at (0,0);
                                        \coordinate (a6) at (0.5,1);

                                        % draw contexts

                                        \draw [color=olive] (a1) -- (a3);
                                        \draw [color=cyan] (a3) -- (a5);
                                        \draw [color=orange] (a5) -- (a1);


                                        % draw atoms

                                        \draw (a1) coordinate[c2,fill=orange,label=above:${a_1=\{1\}}$];
                                        \draw (a1) coordinate[c1,fill=olive];

                                        \draw (a2) coordinate[c1,fill=olive,label=right:${a_2=\{3,4\}}$];

                                        \draw (a3) coordinate[c2,fill=olive,label=below right:${a_3=\{2\}}$];
                                        \draw (a3) coordinate[c1,fill=cyan];

                                        \draw (a4) coordinate[c1,fill=cyan,label=below:${a_4=\{1,4\}}$];

                                        \draw (a5) coordinate[c2,fill=cyan,label=below left:${a_5=\{3\}}$];
                                        \draw (a5) coordinate[c1,fill=orange];

                                        \draw (a6) coordinate[c1,fill=orange,label=left:${a_6=\{2,4\}}$];

                                \end{tikzpicture}
                                &
                                $\qquad$
                                &
                                \begin{tikzpicture}  [scale=0.8]

                                        \tikzstyle{every path}=[line width=1pt]

                                        \newdimen\ms
                                        \ms=0.1cm
                                        \tikzstyle{c2}=[circle,inner sep={\ms/8},minimum size=3*\ms]
                                        \tikzstyle{c1}=[circle,inner sep={\ms/8},minimum size=2*\ms]

                                        % Define positions of all observables


                                        \coordinate (a1) at  (1,2);
                                        \coordinate (a2) at (1.5,1);
                                        \coordinate (a3) at (2,0);
                                        \coordinate (a4) at (1,0);
                                        \coordinate (a5) at (0,0);
                                        \coordinate (a6) at (0.5,1);

                                        % draw contexts

                                        \draw [color=olive] (a1) -- (a3);
                                        \draw [color=cyan] (a3) -- (a5);
                                        \draw [color=orange] (a5) -- (a1);


                                        % draw atoms

                                        \draw (a1) coordinate[c2,fill=red,label=above:${\{{\color{red} \bf 1}\}}$];

                                        \draw (a2) coordinate[c2,fill=green,label=right:${\{{\color{green} \bf 3},{\color{red!20!white} 4}\}}$];

                                        \draw (a3) coordinate[c2,fill=blue,label=below right:${\{{\color{blue} \bf 2}\}}$];


                                        \draw (a4) coordinate[c2,fill=red,label=below:${\{{\color{red} \bf 1},{\color{red!20!white} 4}\}}$];

                                        \draw (a5) coordinate[c2,fill=green,label=below left:${\{{\color{green} \bf 3}\}}$];


                                        \draw (a6) coordinate[c2,fill=blue,label=left:${\{{\color{blue} \bf 2},{\color{red!20!white} 4}\}}$];

                                \end{tikzpicture}
                                %
                                \\
                                %
                                (e)&&(f)
                        \end{tabular}
                \end{center}
                \caption{\label{2020-f-chroma-triangle3}
                        One (nonunique) coloring~(f) construction of
                        the triangle hypergraph of the logic: first compose a (nonunique)
                        canonical partition logic~(e) from enumerating the set of all 4 two-valued states depicted in (a)--(d).
                        Then choose the context $\{a_1,a_2,a_3\}$, and from this context choose the atom $a_1=\{1\}$.
                        Now identify the first color (red) with the index 1, thereby identifying $a_1=\{1\}$ as well as $a_4=\{1,4\}$ with red.
                        Then delete the index number $4$ from every atom; that is, $a_2=\{3,4\}\rightarrow \{3\}$ and $a_6=\{2,4\}\rightarrow \{2\}$.
                        Finally identify 3 with the second color (green) and 2 with the third color (blue),
                        thereby identifying $a_2$ and $a_5$ with green, and $a_3$ and $a_6$ with blue, respectively.
                        Note that $s_1$, $s_2$, and $s_3$ ``generate'' a 3-partitioning of the set of atoms $\{a_1,\ldots ,a_6\}$ of this logic.
                }
        \end{figure}

        \subsection{House or pentagon or pentagram logic}

        The Travis matrix of the house or pentagon or pentagram logic
        is a matrix representation of its 11 dispersion free states~\cite{wright:pent}
        \begin{equation}
                T_{ij}=\begin{pmatrix}
                        {\color{blue}1}& {\color{blue}0}& {\color{blue}0}& {\color{blue}1}& {\color{blue}0}& {\color{blue}1}& {\color{blue}0}& {\color{blue}1}& {\color{blue}0}& {\color{blue}0}  \\
                        1& 0& 0& 1& 0& 0& 1& 0& 0& 0  \\
                        1& 0& 0& 0& 1& 0& 0& 1& 0& 0  \\
                        0& 1& 0& 1& 0& 1& 0& 1& 0& 1  \\
                        0& 1& 0& 1& 0& 1& 0& 0& 1& 0  \\
                        0& 1& 0& 1& 0& 0& 1& 0& 0& 1  \\
                        0& 1& 0& 0& 1& 0& 0& 1& 0& 1  \\
                        {\color{green}0}& {\color{green}1}& {\color{green}0}& {\color{green}0}& {\color{green}1}& {\color{green}0}& {\color{green}0}& {\color{green}0}& {\color{green}1}& {\color{green}0}  \\
                        0& 0& 1& 0& 0& 1& 0& 1& 0& 1  \\
                        0& 0& 1& 0& 0& 1& 0& 0& 1& 0  \\
                        {\color{red}0}& {\color{red}0}& {\color{red}1}& {\color{red}0}& {\color{red}0}& {\color{red}0}& {\color{red}1}& {\color{red}0}& {\color{red}0}& {\color{red}1}
                \end{pmatrix}
                .
        \end{equation}
        A coloring can be constructed with the earlier mentioned construction
        which results in three states partitioning all 10 atoms.
        The associated 1st, the 8th and the 11th row vectors
        of $T_{ij}$  are partitioning the 10 atoms.


        \begin{figure}
                \begin{center}
                        \begin{tikzpicture}  [scale=1]

                                \tikzstyle{every path}=[line width=1pt]

                                \newdimen\ms
                                \ms=0.1cm
                                \tikzstyle{s1}=[color=red,rectangle,inner sep=3.5]
                                \tikzstyle{c3}=[circle,inner sep={\ms/8},minimum size=5*\ms]
                                \tikzstyle{c2}=[circle,inner sep={\ms/8},minimum size=3*\ms]
                                \tikzstyle{c1}=[circle,inner sep={\ms/8},minimum size=2*\ms]

                                % Define positions of all observables

                                \coordinate (a1) at (0,2);
                                \coordinate (a2) at (0,1);
                                \coordinate (a3) at (0,0);
                                \coordinate (a4) at (1,0);
                                \coordinate (a5) at (2,0);
                                \coordinate (a6) at (2,1);
                                \coordinate (a7) at (2,2);
                                \coordinate (a8) at (1.5,{2+(3.5-2)/2});
                                \coordinate (a9) at (1,3.5);
                                \coordinate (a10) at (0.5,{2+(3.5-2)/2});

                                % draw contexts

                                \draw [color=orange] (a1) -- (a3);
                                \draw [color=blue] (a3) -- (a5);
                                \draw [color=red] (a5) -- (a7);
                                \draw [color=green] (a7) -- (a9);
                                \draw [color=gray] (a9) -- (a1);

                                % draw atoms

                                \draw (a1) coordinate[c2,fill=blue,label=left:{$\{{\color{blue}1},2,3\}$}];

                                \draw (a2) coordinate[c2,fill=green,label=left:{$\{ 4,5,6,7,{\color{green}8} \}$}];

                                \draw (a3) coordinate[c2,fill=red,label=below left:{$\{  9,10,{\color{red}11}\}$}];

                                \draw (a4) coordinate[c2,fill=blue,label=below:{$\{ {\color{blue}1},2,4,5,6 \}$}];

                                \draw (a5) coordinate[c2,fill=green,label=below right:{$\{ 3,7,{\color{green}8} \}$}];

                                \draw (a6) coordinate[c2,fill=blue,label=right:{$\{ {\color{blue}1},4,5,9,10 \}$}];

                                \draw (a7) coordinate[c2,fill=red,label=right:{$\{ 2,6,{\color{red}11} \}$}];

                                \draw (a8) coordinate[c2,fill=blue,label=above right:{$\{ {\color{blue}1},3,4,7,9 \}$}];

                                \draw (a9) coordinate[c2,fill=green,label=above:{$\{ 5,{\color{green}8},10 \}$}];

                                \draw (a10) coordinate[c2,fill=red,label=above left:{$\{ 4,6,7,9,{\color{red}11} \}$}];

                        \end{tikzpicture}
                \end{center}
                \caption{\label{2020-f-chroma-pentagon3}
                        Coloring scheme of the house or pentagon or pentagram logic from the set of two-valued states.
                }
        \end{figure}


        \subsection{Specker bug gadget}
        \label{Specker-Bug}

        The hypergraph depicted in Figure~\ref{2020-f-SpeckerBug}
        is a minimal~\cite{2018-minimalYIYS} true-implies false gadget introduced by
        Kochen and Specker~\cite[Fig.~1, p.~182]{kochen2} (see also ~\cite[Figure~1, p.~123]{Greechie1974}, among others).
        It is a subgraph of $G_{32}$ introduced later in Figure~\ref{2020-f-GreechieG32}.
        Its Travis matrix is
        \begin{equation}\label{Specker-Bug-Travice}
                T_{ij}=\begin{pmatrix}
                        {\color{blue}1}& {\color{blue}0}& {\color{blue}0}& {\color{blue}1}& {\color{blue}0} &{\color{blue}1}& {\color{blue}0}& {\color{blue}0}& {\color{blue}1}& {\color{blue}0}& {\color{blue}0}& {\color{blue}0}& {\color{blue}0} \\
                        1& 0& 0& 0& 1 &0& 0& 1& 0& 1& 0& 0& 0 \\
                        1& 0& 0& 0& 1 &0& 0& 0& 1& 0& 0& 0& 1 \\
                        0& 1& 0& 1& 0 &1& 0& 1& 0& 0& 1& 0& 0 \\
                        0& 1& 0& 1& 0 &1& 0& 0& 1& 0& 0& 1& 0 \\
                        0& 1& 0& 1& 0 &0& 1& 0& 0& 0& 1& 0& 0 \\
                        {\color{green}0}& {\color{green}1}& {\color{green}0}& {\color{green}0}& {\color{green}1} &{\color{green}0}& {\color{green}0}& {\color{green}1}& {\color{green}0}& {\color{green}1}& {\color{green}0}& {\color{green}1}& {\color{green}0} \\
                        0& 1& 0& 0& 1 &0& 0& 1& 0& 0& 1& 0& 1 \\
                        0& 1& 0& 0& 1 &0& 0& 0& 1& 0& 0& 1& 1 \\
                        0& 0& 1& 0& 0 &1& 0& 1& 0& 1& 0& 1& 0 \\
                        0& 0& 1& 0& 0 &1& 0& 1& 0& 0& 1& 0& 1 \\
                        0& 0& 1& 0& 0 &1& 0& 0& 1& 0& 0& 1& 1 \\
                        0& 0& 1& 0& 0 &0& 1& 0& 0& 1& 0& 1& 0 \\
                        {\color{red}0}& {\color{red}0}& {\color{red}1}& {\color{red}0}& {\color{red}0} &{\color{red}0}& {\color{red}1}& {\color{red}0}& {\color{red}0}& {\color{red}0}& {\color{red}1}& {\color{red}0}& {\color{red}1}
                \end{pmatrix}
                .
        \end{equation}

        \begin{figure}
                \begin{center}
                        \begin{tikzpicture}  [scale=0.8]

                                \newdimen\ms
                                \ms=0.05cm

                                \tikzstyle{every path}=[line width=1pt]

                                \tikzstyle{c3}=[circle,inner sep={\ms/8},minimum size=6*\ms]
                                \tikzstyle{c2}=[circle,inner sep={\ms/8},minimum size=4*\ms]
                                \tikzstyle{c1}=[circle,inner sep={\ms/8},minimum size=0.8*\ms]

                                % Radius of regular polygons
                                \newdimen\R
                                \R=30mm     % outer circle

                                %\r= { \R * sqrt(3) }     % inner circle
                                %\newdimen\r
                                %\r=    {\R * sqrt(3)/2}       % inner circle

                                %\newdimen\K
                                %\K=3cm

                                % Define positions of all observables
                                \path
                                ({ 180 - 0 * 360 /6}:\R      ) coordinate(1)
                                ({ 180 - 30 - 0 * 360 /6}:{\R * sqrt(3)/2}      ) coordinate(2)
                                ({ 180 - 1 * 360 /6}:\R   ) coordinate(3)
                                ({ 180 - 30 - 1 * 360 /6}:{\R * sqrt(3)/2}   ) coordinate(4)
                                ({ 180 - 2 * 360 /6}:\R  ) coordinate(5)
                                ({ 180 - 30 - 2 * 360 /6}:{\R * sqrt(3)/2}  ) coordinate(6)
                                ({ 180 - 3 * 360 /6}:\R  ) coordinate(7)
                                ({ 180 - 30 - 3 * 360 /6}:{\R * sqrt(3)/2}  ) coordinate(8)
                                ({ 180 - 4 * 360 /6}:\R     ) coordinate(9)
                                ({ 180 - 30 - 4 * 360 /6}:{\R * sqrt(3)/2}     ) coordinate(10)
                                ({ 180 - 5 * 360 /6}:\R     ) coordinate(11)
                                ({ 180 - 30 - 5 * 360 /6}:{\R * sqrt(3)/2}     ) coordinate(12)
                                ;

                                % draw contexts

                                \draw [color=cyan] (1) -- (2) -- (3);
                                \draw [color=red] (3) -- (4) -- (5);
                                \draw [color=green] (5) -- (6) -- (7);
                                \draw [color=blue] (7) -- (8) -- (9);
                                \draw [color=magenta] (9) -- (10) -- (11);    %
                                \draw [color=olive] (11) -- (12) -- (1);    %
                                \draw [color=teal] (4) -- (10)  coordinate[pos=0.5]  (13);

                                %
                                %%
                                %% draw atoms
                                %%
                                %
                                \draw (1) coordinate[c3,fill=blue,label={left: $\{ {\color{blue}1},2,3\} $}];   %
                                %
                                \draw (2) coordinate[c3,fill=green,label={above left: $\{ 4,5,6,{\color{green}7},8,9 \}$}];    %
                                %
                                \draw (3) coordinate[c3,fill=red,label={above left: $\{10,11,12,13,{\color{red}14} \} $}]; %
                                %
                                \draw (4) coordinate[c3,fill=blue,label={above: $\{ {\color{blue}1},4,5,6\}$}];  %
                                %
                                \draw (5) coordinate[c3,fill=green,label={above right: $\{ 2,3,{\color{green}7},8,9\} $}];  %
                                %
                                \draw (6) coordinate[c3,fill=blue,label={above right: $\{ {\color{blue}1},4,5,10,11,12\} $}];
                                %
                                \draw (7) coordinate[c3,fill=red,label={right: $\{ 6,13,{\color{red}14}\}$}];  %
                                %
                                \draw (8) coordinate[c3,fill=green,label={below right: $\{ 2,4,{\color{green}7},8,10,11\}$}];  %
                                %
                                \draw (9) coordinate[c3,fill=blue,label={below right: $\{ {\color{blue}1},3,5,9,12\}$}];
                                %
                                \draw (10) coordinate[c3,fill=green,label={below: $\{ 2,{\color{green}7},10,13\}$}];  %
                                %
                                \draw (11) coordinate[c3,fill=red,label={below left: $\{ 4,6,8,11,{\color{red}14}\}$}];  %
                                %
                                \draw (12) coordinate[c3,fill=green,label={below left: $\{ 5,{\color{green}7},9,10,12,13\}$}];
                                %
                                %
                                \draw (13) coordinate[c3,fill=red,label={right: $\{3,8,9,$}];  %
                                \draw (13) coordinate[c3,fill=red,label={below right: $11,12,{\color{red}14}\}$}];  %
                                %
                        \end{tikzpicture}
                \end{center}
                \caption{\label{2020-f-SpeckerBug}
                        Coloring scheme of the ``Specker bug'' gadget~\cite{kochen2,Greechie1974} from two-valued states.
                        The set-theoretic representation is in terms of
                        the canonical partition logic as an equipartitioning of the set $\{1,2,\ldots,14\}$
                        obtained from all 14 two-valued states on this gadget.
                }
        \end{figure}

        \subsection{The underlying hypergraph of $B(G)$}\label{sec:underlying}

        All the two-valued states of the structure introduced in Figure~\ref{TIFS-non-Rec} have to assign the vertices $a$, $b$, $c$, $a'$, $b'$, $c'$ and $a''$, $b''$, $c''$ the same combination of two-valued states as for the hypergraph of Figure~\ref{Fig:underlying}.

        \begin{figure}
                \begin{center}
                        \begin{tikzpicture}  [scale=0.9]

                                \tikzstyle{every path}=[line width=1pt]

                                \newdimen\ms
                                \ms=0.1cm
                                \tikzstyle{s1}=[color=red,rectangle,inner sep=3.5]
                                \tikzstyle{c3}=[circle,inner sep={\ms/8},minimum size=4*\ms]
                                \tikzstyle{c2}=[circle,inner sep={\ms/8},minimum size=3*\ms]
                                %\tikzstyle{c1}=[circle,inner sep={\ms/8},minimum size=2*\ms]
                                \tikzstyle{c1}=[draw=gray,fill=white,circle,inner sep={\ms/1}]
                                \tikzstyle{cs1}=[circle,inner sep={\ms/8},minimum size=1*\ms]


                                % Define positions of all observables

                                \coordinate (a) at ( 0,4);
                                \coordinate (b) at ( 2,4);
                                \coordinate (c) at ( 4,4);


                                \coordinate (a1) at ( 0,2);
                                \coordinate (b1) at ( 2,2);
                                \coordinate (c1) at ( 4,2);


                                \coordinate (a2) at ( 0,0);
                                \coordinate (b2) at ( 2,0);
                                \coordinate (c2) at ( 4,0);





                                % draw contexts

                                \draw [color=green] (a) -- (c);
                                \draw [color=blue] (a1) -- (c1);
                                \draw [color=red] (a2) -- (c2);


                                \draw [color=brown] (a) -- (a2);
                                \draw [color=olive] (b) -- (b2);
                                \draw [color=lightgray] (c) -- (c2);



                                %draw atoms




                                 % \draw (a)  coordinate[c2,fill=blue,draw=blue,label=above:{$a$}];
                              % \draw (b)  coordinate[c2,fill=red,draw=red,label=above:{$b$}];
                              % \draw (c)  coordinate[c2,fill=green,draw=green,label=above:{$c$}];
                              % \draw (a1) coordinate[c2,fill=green,draw=green,label=left:{$a'$}];
                              % \draw (b1) coordinate[c2,fill=blue,draw=blue,label=above right:{$b'$}];
                              % \draw (c1) coordinate[c2,fill=red,draw=red,label=right:{$c'$}];
                              % \draw (a2) coordinate[c2,fill=red,draw=red,label=below:{$a''$}];
                              % \draw (b2) coordinate[c2,fill=green,draw=green,label=below:{$b''$}];
                              % \draw (c2) coordinate[c2,fill=blue,draw=blue,label=below:{$c''$}];

                                \draw (a)  coordinate[c1,label=above:{$a$}];
                                \draw (b)  coordinate[c1,label=above:{$b$}];
                                \draw (c)  coordinate[c1,label=above:{$c$}];
                                \draw (a1) coordinate[c1,label=left:{$a'$}];
                                \draw (b1) coordinate[c1,label=above right:{$b'$}];
                                \draw (c1) coordinate[c1,label=right:{$c'$}];
                                \draw (a2) coordinate[c1,label=below:{$a''$}];
                                \draw (b2) coordinate[c1,label=below:{$b''$}];
                                \draw (c2) coordinate[c1,label=below:{$c''$}];



                        \end{tikzpicture}
                \end{center}
                \caption{\label{Fig:underlying}
                        From the point of view of two-valued states, for which all the nine $G_i$'s in Figure~\ref{TIFS-non-Rec}, $1\le i\le 9$  are TIFS, the three ``new'' contexts
                        $\{a, b, c\}$, $\{a', b', c'\}$ and $\{a'', b'',c''\}$ are formed
                        through three triples of TIFS
                        $\{G_1,G_2,G_3\}$, $\{G_4,G_5,G_6\}$, and $\{G_7,G_8,G_9\}$, respectively; thereby rendering a tightly bi-connected hypergraph underlying the one depicted in Figure~\ref{TIFS-non-Rec}. Note that the vertices of each row in the original graph of $B(G)$ do not lie on a context, but here in the underlying hypergraph they are.
                }
        \end{figure}

        The Travis matrix of this tightly bi-connected hypergraph is as follows (columns from left to right correspond to $a$, $b$, $c$, $a'$, $b'$, $c'$, $a''$, $b''$ and $c''$):
        \begin{equation}\label{underlying-Travice}
                T_{ij}=\begin{pmatrix}
                        1 &  0 & 0 & 0 & 1 & 0 & 0 & 0 & 1    \\
                        1 &  0 & 0 & 0 & 0 & 1 & 0 & 1 & 0    \\
                        0 &  1 & 0 & 1 & 0 & 0 & 0 & 0 & 1    \\
                        0 &  1 & 0 & 0 & 0 & 1 & 1 & 0 & 0    \\
                        0 &  0 & 1 & 1 & 0 & 0 & 0 & 1 & 0    \\
                        0 &  0 & 1 & 0 & 1 & 0 & 1 & 0 & 0    \\
                \end{pmatrix}
                .
        \end{equation}

        \subsection{``Tight GHZ'' logic}

        The hypergraph depicted in Figure~\ref{2020-f-GHZ}
        is a sublogic  of the observables in the Greenberger-Horn-Zeilinger setup~\cite{svozil-2020-ghz}.
        Its Travis matrix is
        \begin{equation}\label{GHZ-tight-Travice}
                T_{ij}=\begin{pmatrix}
                        {\color{blue}1 } & {\color{blue}  0 } & {\color{blue} 0 } & {\color{blue} 0 } & {\color{blue} 0 } & {\color{blue} 0 } & {\color{blue} 1 } & {\color{blue} 0 } & {\color{blue} 0 } & {\color{blue} 0 } & {\color{blue} 0 } & {\color{blue} 1 } & {\color{blue} 0 } & {\color{blue} 1 } & {\color{blue} 0 } & {\color{blue} 0 }   \\
                        1 &  0 & 0 & 0 & 0 & 0 & 0 & 1 & 0 & 1 & 0 & 0 & 0 & 0 & 1 & 0    \\
                        0 &  1 & 0 & 0 & 0 & 0 & 1 & 0 & 1 & 0 & 0 & 0 & 0 & 0 & 0 & 1    \\
                        {\color{green}0 } & {\color{green}  1 } & {\color{green} 0 } & {\color{green} 0 } & {\color{green} 0 } & {\color{green} 0 } & {\color{green} 0 } & {\color{green} 1 } & {\color{green} 0 } & {\color{green} 0 } & {\color{green} 1 } & {\color{green} 0 } & {\color{green} 1 } & {\color{green} 0 } & {\color{green} 0 } & {\color{green} 0 }   \\
                        {\color{red}0 } & {\color{red}  0 } & {\color{red} 1 } & {\color{red} 0 } & {\color{red} 1 } & {\color{red} 0 } & {\color{red} 0 } & {\color{red} 0 } & {\color{red} 0 } & {\color{red} 1 } & {\color{red} 0 } & {\color{red} 0 } & {\color{red} 0 } & {\color{red} 0 } & {\color{red} 0 } & {\color{red} 1   } \\
                        0 &  0 & 1 & 0 & 0 & 1 & 0 & 0 & 0 & 0 & 0 & 1 & 1 & 0 & 0 & 0    \\
                        0 &  0 & 0 & 1 & 1 & 0 & 0 & 0 & 0 & 0 & 1 & 0 & 0 & 1 & 0 & 0    \\
                        {\color{cyan}0 } & {\color{cyan}  0 } & {\color{cyan} 0 } & {\color{cyan} 1 } & {\color{cyan} 0 } & {\color{cyan} 1 } & {\color{cyan} 0 } & {\color{cyan} 0 } & {\color{cyan} 1 } & {\color{cyan} 0 } & {\color{cyan} 0 } & {\color{cyan} 0 } & {\color{cyan} 0 } & {\color{cyan} 0 } & {\color{cyan} 1 } & {\color{cyan} 0 }
                \end{pmatrix}
                .
        \end{equation}

        The coloring is depicted in Figure~\ref{2020-f-GHZ}.
        \begin{figure}
                \begin{center}
                        \begin{tikzpicture}  [scale=0.7]

                                \tikzstyle{every path}=[line width=1pt]

                                \newdimen\ms
                                \ms=0.1cm
                                \tikzstyle{s1}=[color=red,rectangle,inner sep=3.5]
                                \tikzstyle{c3}=[circle,inner sep={\ms/8},minimum size=4*\ms]
                                \tikzstyle{c2}=[circle,inner sep={\ms/8},minimum size=3*\ms]
                                \tikzstyle{c1}=[circle,inner sep={\ms/8},minimum size=2*\ms]
                                \tikzstyle{cs1}=[circle,inner sep={\ms/8},minimum size=1*\ms]


                                % Define positions of all observables

                                \coordinate (uuu) at ( 0,6);
                                \coordinate (uuv) at ( 2,6);
                                \coordinate (uvu) at ( 4,6);
                                \coordinate (uvv) at ( 6,6);
                                \coordinate (vuu) at ( 8,6);
                                \coordinate (vuv) at (10,6);
                                \coordinate (vvu) at (12,6);
                                \coordinate (vvv) at (14,6);


                                \coordinate (ucc) at ( 0,4);
                                \coordinate (vcc) at ( 2,4);
                                \coordinate (ucd) at ( 4,4);
                                \coordinate (vcd) at ( 6,4);
                                \coordinate (udc) at ( 8,4);
                                \coordinate (vdc) at (10,4);
                                \coordinate (udd) at (12,4);
                                \coordinate (vdd) at (14,4);

                                \coordinate (cuc) at ( 0,2);
                                \coordinate (cvc) at ( 2,2);
                                \coordinate (cud) at ( 4,2);
                                \coordinate (cvd) at ( 6,2);
                                \coordinate (duc) at ( 8,2);
                                \coordinate (dvc) at (10,2);
                                \coordinate (dud) at (12,2);
                                \coordinate (dvd) at (14,2);

                                \coordinate (ccu) at ( 0,0);
                                \coordinate (ccv) at ( 2,0);
                                \coordinate (cdu) at ( 4,0);
                                \coordinate (cdv) at ( 6,0);
                                \coordinate (dcu) at ( 8,0);
                                \coordinate (dcv) at (10,0);
                                \coordinate (ddu) at (12,0);
                                \coordinate (ddv) at (14,0);




                                % draw contexts

                                \draw [color=cyan] (ccu) -- (cdv);
                                \draw [color=blue] (cuc) -- (cvd);
                                \draw [color=red] (ucc) -- (vcd);
                                \draw [color=green] (uuu) -- (uvv);

                                \draw [color=Gray] (uuu) -- (ccu);
                                \draw [color=Plum] (uuv) -- (ccv);
                                \draw [color=CornflowerBlue] (uvu) -- (cdu);
                                \draw [color=YellowGreen] (uvv) -- (cdv);


                                \draw [color=Tan] (uuu) -- (cdv);
                                \draw [color=Brown] (uvv) -- (ccu);


                                \draw [color=ForestGreen] (uuv) -- (ucc);
                                \draw [color=ForestGreen](cdu) -- (cvd);
                                \draw [rotate=225,color=ForestGreen] (cvd) arc (90:270:4.5 and 2.82);
                                \draw[rotate=45,color=ForestGreen] (ucc) arc (90:270:4.5 and 2.82);

                                \draw [color=Magenta] (cuc) -- (ccv);
                                \draw [color=Magenta] (uvu) -- (vcd);
                                \draw[rotate=315,color=Magenta] (uvu) arc (90:270:4.5 and 2.82);
                                \draw[rotate=135,color=Magenta] (ccv) arc (90:270:4.5 and 2.82);


                                %draw atoms




                                \draw (uuu) coordinate[c2,fill=blue,draw=blue,label=above:{$\{{\color{blue}1 },2\}$}];
                                %\draw (uuu) coordinate[c1,fill=Gray,draw=green];
                                \draw (uuv) coordinate[c2,fill=green,draw=green,label={[xshift=-3.7mm]90:$\{3,{\color{green}4}\}$}];
                                %\draw (uuv) coordinate[c1,fill=green,draw=Plum];
                                \draw (uvu) coordinate[c2,fill=red,draw=red,label={[xshift=+4mm]90:$\{{\color{red}5},6\}$}];
                                %\draw (uvu) coordinate[c1,fill=CornflowerBlue,draw=green];
                                \draw (uvv) coordinate[c2,fill=cyan,draw=cyan,label=above:{$\{7,{\color{cyan}8}\}$}];
                                %\draw (uvv) coordinate[c1,fill=YellowGreen,draw=green];


                                % \draw (ucc) coordinate[c2,fill=ForestGreen,draw=ForestGreen,label={[xshift=-7.5mm,distance=0mm]90:$\{x_-  y_-  y_-\}$}];
                                \draw (ucc) coordinate[c2,fill=red,draw=red,label=below right:{$\{{\color{red}5},7\}$}];
                                % \draw (ucc) coordinate[c1,fill=Gray,draw=red];
                                \draw (vcc) coordinate[c2,fill=cyan,draw=cyan,label=above right:{$\{6,{\color{cyan}8}\}$}];
                                % \draw (vcc) coordinate[c1,fill=Plum,draw=red];
                                \draw (ucd) coordinate[c2,fill=blue,draw=blue,label=below right:{$\{{\color{blue}1 },3\}$}];
                                % \draw (ucd) coordinate[c1,fill=CornflowerBlue,draw=red];
                                % \draw (vcd) coordinate[c2,fill=Magenta,draw=Magenta,label=above right:{$\{x_+  y_+  y_-\}$}];
                                \draw (vcd) coordinate[c2,fill=green,draw=green,label={[xshift=+5mm,distance=10mm]90:$\{2,{\color{green}4} \}$}];
                                %\draw (vcd) coordinate[c1,fill=red,draw=YellowGreen];

                                \draw (cuc) coordinate[c2,fill=cyan,draw=cyan,label=above right:{$\{3,{\color{cyan}8}\}$}];
                                % \draw (cuc) coordinate[c1,fill=Gray,draw=blue];
                                \draw (cvc) coordinate[c2,fill=red,draw=red,label=below right:{$\{2,{\color{red}5}\}$}];
                                % \draw (cvc) coordinate[c1,fill=Plum,draw=blue];
                                \draw (cud) coordinate[c2,fill=green,draw=green,label=above right:{$\{{\color{green}4},7\}$}];
                                % \draw (cud) coordinate[c1,fill=CornflowerBlue,draw=blue];
                                \draw (cvd) coordinate[c2,fill=blue,draw=blue,label={[xshift=+5mm,distance=10mm]270:$\{{\color{blue}1 },6\}$}];
                                %\draw (cvd) coordinate[c1,fill=YellowGreen,draw=blue];

                                \draw (ccu) coordinate[c2,fill=green,draw=green,label=below:{$\{{\color{green}4},6\}$}];
                                %\draw (ccu) coordinate[c1,fill=Gray,draw=cyan];
                                \draw (ccv) coordinate[c2,fill=blue,draw=blue,label={[xshift=-3.5mm]270:$\{{\color{blue}1 },7\}$}];
                                %\draw (ccv) coordinate[c1,fill=Plum,draw=cyan];
                                \draw (cdu) coordinate[c2,fill=cyan,draw=cyan,label={[xshift=+3.5mm]270:$\{2,{\color{cyan}8}\}$}];
                                %\draw (cdu) coordinate[c1,fill=CornflowerBlue,draw=cyan];
                                \draw (cdv) coordinate[c2,fill=red,draw=red,label=below:{$\{3,{\color{red}5}\}$}];
                                %\draw (cdv) coordinate[c1,fill=YellowGreen,draw=cyan];



                        \end{tikzpicture}
                \end{center}
                \caption{\label{2020-f-GHZ}
                        Coloring scheme of the ``tight GHZ'' logic~\cite{svozil-2020-ghz} from two-valued states.
                        The set-theoretic representation is in terms of
                        the canonical partition logic as an equipartitioning of the set $\{1,2,\ldots,8\}$
                        obtained from all eight two-valued states on this gadget.
                }
        \end{figure}

        \section{A counterexample: Greechie's $G_{32}$}\label{2021-chroma-G32}

        It is quite straightforward to demonstrate that the logic $G_{32}$ introduced by Greechie~\cite[Figure~6, p.~121]{greechie:71}
        (see also Refs.~\cite{Holland1975,Bennett-MC-1970,Greechie1974,Greechie-Suppes1976})
        whose hypergraph is depicted in Figure~\ref{2020-f-GreechieG32}(a) has a chromatic number larger than three;
        and, in particular, while having a separating and a unital set of two-valued states, cannot be colored by two-valued states in the algorithmic way proposed earlier.
        Consider the set of all six two-valued states which can be tabulated by the Travis matrix
        \begin{equation}
                T_{ij}=\begin{pmatrix}
                        1&0&0&1&0&1&0&0&1&0&0&0&0&0&1\\
                        1&0&0&0&1&0&0&1&0&1&0&0&0&1&0\\
                        0&1&0&1&0&0&1&0&0&0&1&0&0&1&0\\
                        0&1&0&0&1&0&0&0&1&0&0&1&1&0&0\\
                        0&0&1&0&0&1&0&1&0&0&1&0&1&0&0\\
                        0&0&1&0&0&0&1&0&0&1&0&1&0&0&1
                \end{pmatrix}
                .
        \end{equation}
        There is no way how three of these six row vectors add up to
        a vector whose components are all one; that is,
        $
        \begin{pmatrix}
                1,1,1,1,1,1,1,1,1,1,1,1,1,1,1
        \end{pmatrix}
        $.
        ``Completing'' the partition logic and ``extending''
        $G_{32}$ by adding five more contexts
        $\{\{1, 2\}, \{3, 6\}, \{4, 5\}\}$,
        $\{\{1, 4\}, \{2, 3\}, \{5, 6\}\}$,
        $\{\{1, 3\}, \{2, 5\}, \{4, 6\}\}$,
        $\{\{1, 5\}, \{2, 6\}, \{3, 4\}\}$, and
        $\{\{1, 6\}, \{2, 4\}, \{3, 5\}\}$
        does not change the set of two-valued states and thus the Travis matrix.


        Another way of seeing this is to associate a color to, say, the first state.
        As a consequence, all other states, namely states number
        $2$,
        $3$,
        $4$,
        $5$, and
        $6$, need to be eliminated,
        leaving no state which can be associated with
        another color.

        One possibility for finding a proper coloring is to drop ``exclusivity'', or rather, the unique association of two-valued states with colors; but not entirely. This can be achieved by not eliminating two-valued states if they appear in association with previous colors. A construction identifying state numbers 1 with red, 3 with green, 5 with blue, and then 2 or four with cyan is depicted in Figure~\ref{2020-f-GreechieG32}(b).

        \begin{figure}
                \begin{center}
                        \begin{tabular}{ c c c }
                                \begin{tikzpicture}  [scale=0.8]

                                        \newdimen\ms
                                        \ms=0.05cm

                                        \tikzstyle{every path}=[line width=1pt]

                                        \tikzstyle{c3}=[circle,inner sep={\ms/8},minimum size=6*\ms]
                                        \tikzstyle{c2}=[circle,inner sep={\ms/8},minimum size=4*\ms]
                                        \tikzstyle{c1}=[circle,inner sep={\ms/8},minimum size=0.8*\ms]

                                        % Radius of regular polygons
                                        \newdimen\R
                                        \R=30mm     % outer circle

                                        %\r= { \R * sqrt(3) }     % inner circle
                                        %\newdimen\r
                                        %\r=    {\R * sqrt(3)/2}       % inner circle

                                        %\newdimen\K
                                        %\K=3cm

                                        % Define positions of all observables
                                        \path
                                        ({ 180 - 0 * 360 /6}:\R      ) coordinate(1)
                                        ({ 180 - 30 - 0 * 360 /6}:{\R * sqrt(3)/2}      ) coordinate(2)
                                        ({ 180 - 1 * 360 /6}:\R   ) coordinate(3)
                                        ({ 180 - 30 - 1 * 360 /6}:{\R * sqrt(3)/2}   ) coordinate(4)
                                        ({ 180 - 2 * 360 /6}:\R  ) coordinate(5)
                                        ({ 180 - 30 - 2 * 360 /6}:{\R * sqrt(3)/2}  ) coordinate(6)
                                        ({ 180 - 3 * 360 /6}:\R  ) coordinate(7)
                                        ({ 180 - 30 - 3 * 360 /6}:{\R * sqrt(3)/2}  ) coordinate(8)
                                        ({ 180 - 4 * 360 /6}:\R     ) coordinate(9)
                                        ({ 180 - 30 - 4 * 360 /6}:{\R * sqrt(3)/2}     ) coordinate(10)
                                        ({ 180 - 5 * 360 /6}:\R     ) coordinate(11)
                                        ({ 180 - 30 - 5 * 360 /6}:{\R * sqrt(3)/2}     ) coordinate(12)
                                        ;

                                        % draw contexts

                                        \draw [color=cyan] (1) -- (2) -- (3);
                                        \draw [color=red] (3) -- (4) -- (5);
                                        \draw [color=green] (5) -- (6) -- (7);
                                        \draw [color=blue] (7) -- (8) -- (9);
                                        \draw [color=magenta] (9) -- (10) -- (11);    %
                                        \draw [color=olive] (11) -- (12) -- (1);    %
                                        \draw [color=orange] (2) -- (8)  coordinate[pos=0.85]  (15);
                                        \draw [color=teal] (4) -- (10)  coordinate[pos=0.15]  (13);
                                        \draw [color=MidnightBlue] (6) -- (12)  coordinate[pos=0.325]  (14);
                                        \draw [color=gray] (13) --(15);

                                        %
                                        %%
                                        %% draw atoms
                                        %%
                                        %
                                        \draw (1) coordinate[c3,fill=cyan,label={left: $\{ 1,2\} $}];   %
                                        \draw (1) coordinate[c2,fill=olive];  %
                                        %
                                        \draw (2) coordinate[c3,fill=cyan,label={above left: $\{ 3,4\}$}];    %
                                        \draw (2) coordinate[c2,fill=orange];    %
                                        %
                                        \draw (3) coordinate[c3,fill=red,label={above: $\{ 5,6\} $}]; %
                                        \draw (3) coordinate[c2,fill=cyan];  %
                                        %
                                        \draw (4) coordinate[c3,fill=red,label={above: $\{ 1,3\}$}];  %
                                        \draw (4) coordinate[c2,fill=teal];  %
                                        %
                                        \draw (5) coordinate[c3,fill=green,label={above: $\{ 2,4\} $}];  %
                                        \draw (5) coordinate[c2,fill=red];  %
                                        %
                                        \draw (6) coordinate[c3,fill=green,label={above right: $\{ 1,5\} $}];
                                        \draw (6) coordinate[c2,fill=MidnightBlue];
                                        %
                                        \draw (7) coordinate[c3,fill=blue,label={right: $\{ 3,6\}$}];  %
                                        \draw (7) coordinate[c2,fill=green];  %
                                        %
                                        \draw (8) coordinate[c3,fill=blue,label={below right: $\{ 2,5\}$}];  %
                                        \draw (8) coordinate[c2,fill=orange];  %
                                        %
                                        \draw (9) coordinate[c3,fill=magenta,label={below: $\{ 1,4\}$}];
                                        \draw (9) coordinate[c2,fill=blue];  %
                                        %
                                        \draw (10) coordinate[c3,fill=magenta,label={below: $\{ 2,6\}$}];  %
                                        \draw (10) coordinate[c2,fill=teal];  %
                                        %
                                        \draw (11) coordinate[c3,fill=olive,label={below: $\{ 3,5\}$}];  %
                                        \draw (11) coordinate[c2,fill=magenta];  %
                                        %
                                        \draw (12) coordinate[c3,fill=olive,label={below left: $\{ 4,6\}$}];
                                        \draw (12) coordinate[c2,fill=MidnightBlue];
                                        %
                                        \draw (13) coordinate[c3,fill=MidnightBlue,label={right: $\{ 4,5\}$}];  %
                                        \draw (13) coordinate[c2,fill=gray];  %
                                        %
                                        \draw (14) coordinate[c3,fill=teal,label=0:{$\{ 2,3\}$}];  %
                                        \draw (14) coordinate[c2,fill=gray];  %
                                        %
                                        \draw (15) coordinate[c3,fill=orange,label={below left: $\{1,6\}$}];  %
                                        \draw (15) coordinate[c2,fill=gray];  %
                                        %
                                \end{tikzpicture}
                                &$\qquad$&
                                \begin{tikzpicture}  [scale=0.8]

                                        \newdimen\ms
                                        \ms=0.05cm

                                        \tikzstyle{every path}=[line width=1pt]

                                        \tikzstyle{c3}=[circle,inner sep={\ms/8},minimum size=6*\ms]
                                        \tikzstyle{c2}=[circle,inner sep={\ms/8},minimum size=4*\ms]
                                        \tikzstyle{c1}=[circle,inner sep={\ms/8},minimum size=0.8*\ms]

                                        % Radius of regular polygons
                                        \newdimen\R
                                        \R=30mm     % outer circle

                                        %\r= { \R * sqrt(3) }     % inner circle
                                        %\newdimen\r
                                        %\r=    {\R * sqrt(3)/2}       % inner circle

                                        %\newdimen\K
                                        %\K=3cm

                                        % Define positions of all observables
                                        \path
                                        ({ 180 - 0 * 360 /6}:\R      ) coordinate(1)
                                        ({ 180 - 30 - 0 * 360 /6}:{\R * sqrt(3)/2}      ) coordinate(2)
                                        ({ 180 - 1 * 360 /6}:\R   ) coordinate(3)
                                        ({ 180 - 30 - 1 * 360 /6}:{\R * sqrt(3)/2}   ) coordinate(4)
                                        ({ 180 - 2 * 360 /6}:\R  ) coordinate(5)
                                        ({ 180 - 30 - 2 * 360 /6}:{\R * sqrt(3)/2}  ) coordinate(6)
                                        ({ 180 - 3 * 360 /6}:\R  ) coordinate(7)
                                        ({ 180 - 30 - 3 * 360 /6}:{\R * sqrt(3)/2}  ) coordinate(8)
                                        ({ 180 - 4 * 360 /6}:\R     ) coordinate(9)
                                        ({ 180 - 30 - 4 * 360 /6}:{\R * sqrt(3)/2}     ) coordinate(10)
                                        ({ 180 - 5 * 360 /6}:\R     ) coordinate(11)
                                        ({ 180 - 30 - 5 * 360 /6}:{\R * sqrt(3)/2}     ) coordinate(12)
                                        ;

                                        % draw contexts

                                        \draw [color=cyan] (1) -- (2) -- (3);
                                        \draw [color=red] (3) -- (4) -- (5);
                                        \draw [color=green] (5) -- (6) -- (7);
                                        \draw [color=blue] (7) -- (8) -- (9);
                                        \draw [color=magenta] (9) -- (10) -- (11);    %
                                        \draw [color=olive] (11) -- (12) -- (1);    %
                                        \draw [color=orange] (2) -- (8)  coordinate[pos=0.85]  (15);
                                        \draw [color=teal] (4) -- (10)  coordinate[pos=0.15]  (13);
                                        \draw [color=MidnightBlue] (6) -- (12)  coordinate[pos=0.325]  (14);
                                        \draw [color=gray] (13) --(15);

                                        %
                                        %%
                                        %% draw atoms
                                        %%
                                        %
                                        \draw (1) coordinate[c3,fill=red,label={left: $\{ 1,2\} $}];   %
                                        %
                                        \draw (2) coordinate[c3,fill=green,label={above left: $\{ 3,4\}$}];    %
                                        %
                                        \draw (3) coordinate[c3,fill=blue,label={above: $\{ 5,6\} $}]; %
                                        %
                                        \draw (4) coordinate[c3,fill=red,label={above: $\{ 1,3\}$}];  %
                                        %
                                        \draw (5) coordinate[c3,fill=cyan,label={above: $\{ 2,4\} $}];  %
                                        %
                                        \draw (6) coordinate[c3,fill=red,label={above right: $\{ 1,5\} $}];
                                        %
                                        \draw (7) coordinate[c3,fill=green,label={right: $\{ 3,6\}$}];  %
                                        %
                                        \draw (8) coordinate[c3,fill=blue,label={below right: $\{ 2,5\}$}];  %
                                        %
                                        \draw (9) coordinate[c3,fill=red,label={below: $\{ 1,4\}$}];
                                        %
                                        \draw (10) coordinate[c3,fill=cyan,label={below: $\{ 2,6\}$}];  %
                                        %
                                        \draw (11) coordinate[c3,fill=green,label={below: $\{ 3,5\}$}];  %
                                        %
                                        \draw (12) coordinate[c3,fill=cyan,label={below left: $\{ 4,6\}$}];
                                        %
                                        \draw (13) coordinate[c3,fill=blue,label={right: $\{ 4,5\}$}];  %
                                        %
                                        \draw (14) coordinate[c3,fill=green,label=0:{$\{ 2,3\}$}];  %
                                        %
                                        \draw (15) coordinate[c3,fill=red,label={below left: $\{1,6\}$}];  %
                                        %
                                \end{tikzpicture}\\
                                (a)&&(b)
                        \end{tabular}
                \end{center}
                \caption{\label{2020-f-GreechieG32}
                        (a) Greechie diagram of $G_{32}$ introduced by Greechie~\cite[Figure~6, p.~121]{greechie:71}.
                        The overlaid set theoretic representation is in terms of
                        the canonical partition logic as an equipartitioning of the set $\{1,2,3,4,5,6\}$
                        obtained from all 6 two-valued states on $G_{32}$;
                        (b) a nonunique coloring by four colors.
                }
        \end{figure}

        $G_{32}$  cannot have a faithful orthogonal representation because of the following proof by contradiction:
        Suppose $G_{32}$  has a faithful orthogonal representation.
        Then each one of the nine biconnected contexts of $G_{32}$ can be  uniformly represented by a
        maximal operator~\cite[\S~84, p.~171,172]{halmos-vs}. In order for a faithful orthogonal
        representation to exist the spectral decomposition of two ``intertwining'' maximal
        operators must have (i) (at least) one common projector (ii) with identical eigenvalues which can be identified with identical colors.
        If the logic can be consistently ``covered'' or colored by three colors then the eigenvalues associated with the maximal operators can be the same--that is, these three values (or colors) would occur uniformly in all nine contexts.
        But this is not the case for $G_{32}$. Therefore, a uniform representation cannot be be given in terms of nine maximal operators with just three eigenvalues per context (that is, maximal operator).
        This is a form of nonclassicality based on a chromatic number exceeding the 2-section's clique number.
        In this case Brooks' theorem~\cite{Brooks1941,Lovasz1975} yields an upper bound of 4 for the chromatic number of $G_{32}$.




        \begin{acknowledgments}
            The authors would like to thank the anonymous referees for their  invaluable comments on the earlier version of this paper.

            The first author thanks the Institute for Theoretical Physics, Vienna University of Technology for their hospitality during his visit in September 2019, without which writing this paper would not have been possible.

                The second author is supported in whole, or in part, by the Austrian Science Fund (FWF), Project No. I 4579-N. For the purpose of open access, the author has applied a CC BY public copyright licence to any Author Accepted Manuscript version arising from this submission.

                The authors declare no conflict of interest.

                We kindly acknowledge enlightening discussions with Adan Cabello Jos\'{e}, R. Portillo,
                Alexander~Svozil, Josef Tkadlec and Sebastian Matkovich.
                We are grateful to Josef Tkadlec for providing a {\em Pascal} program
                which computes and analyses the set of two-valued states of collections of contexts.
                All misconceptions and errors are ours.
        \end{acknowledgments}



        % MDPI \funding{The author acknowledges the support by the Austrian Science Fund (FWF): project I 4579-N and the Czech Science Foundation (GA\v CR): project 20-09869L.}

        %%%%%%%%%%%%%%%%%%%%%%%%%%%%%%%%%%%%%%%
        % MDPI \acknowledgments{I kindly acknowledge enlightening discussions with Adan Cabello, Jos\'{e} R. Portillo, and Mohammad Hadi Shekarriz.
        % MDPI I am grateful to Josef Tkadlec for providing a {\em Pascal} program
        % MDPI which computes and analyses the set of two-valued states of collections of contexts.
        % MDPI All misconceptions and errors are mine.
        % MDPI The author declares no conflict of interest.
        % MDPI }

        %%%%%%%%%%%%%%%%%%%%%%%%%%%%%%%%%%%%%%%
        % MDPI \conflictsofinterest{The author declares no conflict of interest.}


        %%%%%%%%%%%%%%%%%%%%%%%%%%%%%%%%%%%%%%%
        % MDPI \reftitle{References}

        % Please provide either the correct journal abbreviation (e.g. according to the “List of Title Word Abbreviations” http://www.issn.org/services/online-services/access-to-the-ltwa/) or the full name of the journal.
        % Citations and References in Supplementary files are permitted provided that they also appear in the reference list here.

        %=====================================
        % References, variant A: external bibliography
        %=====================================
        %\externalbibliography{yes}

        \bibliography{svozil}




        %%%%%%%%%%%%%%%%%%%%%%%%%%%%%%%%%%%%%%%

\end{document}
