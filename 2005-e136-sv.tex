\documentclass[pra,amsfonts,showpacs,preprint,showkeys]{revtex4}
%\documentclass[pra,showpacs,showkeys,amsfonts]{revtex4}
\usepackage[T1]{fontenc}
\usepackage{textcomp}
\usepackage{graphicx}
%\documentstyle[amsfonts]{article}
\RequirePackage{times}
%\RequirePackage{courier}
\RequirePackage{mathptm}
%\renewcommand{\baselinestretch}{1.3}

\begin{document}

{\tt
http://tph.tuwien.ac.at/~svozil/publ/2005-e136-sv.tex
}

\section{Quantum information}

Suppose we would be able to unleash the power of the quantum world in ways which would have been unthinkable
only a few years ago.
For instance, we could use quantum superposition, the possibility for a quantum bit to contain
all conceivable and mutually excluding classical states in itself.
Then, in a single computational step, we could realize the parallel processing of all these classical states,
whose number grows exponentially with the number of classical bits involved,
through the quantum state evolution of this single state.
That is the vision of quantum parallelism,
which is one of the driving forces of quantum computing, and at the same time
one of the fastest growing  areas of research in the last decade or so.

These strategies have all been made possible and operationalizable with new techniques
capable to produce, manipulate, and detect single quanta, such as photons, neutrons and electrons.

There are other prospects as well.
Quantum processes and in particular the quantum state evolution in-between irreversible measurements
are one-to-one, i.e., reversible.
The ``message'' encoded into a quantum state merely gets permuted and transformed such that nothing gets lost.
Thus, processes such as state copy or state deletion, which appear so familiar from classical computing,
are not allowed in quantum information theory.
Copy, for instance, is one-to-two, or one-to-many.
Deletion is many-to-one.
As a consequence, information transmission has to rely on processes which are strictly one-to-one.
This elementary, innocently looking fact of quantum state evolution, can be put to practical use
in areas such as cryptography, where it is tantamount to keep a secret secret; i.e., by not allowing
potential eavesdroppers to divert, copy, and resubmit messages.
Actually, quantum cryptography uses another mind-boggling feature of the quantum: complementarity;
the impossibility to measure all classical observables of a state at once with arbitrary accuracy.
So it is the scarcity of the quantum processes which could be harvested for new technologies.
Even potential cryptanalytic techniques---such as man-in-the-middle attacks
on quantum cryptography---could be perceived
as a challenge to cope with the structure of the quantum world in detail.

The basis of these potential exciting new technologies is the quantum world and its relation
to the performance of classical systems.
Already George Boole, one hundred and fifty years ago, mused over issues which became most important today.
He figured out that there are some constraints on the joint frequency of classical events
which come from the requirement of consistency.

Suppose someone claims that the chances of rain in Vienna
and Budapest are 0.1 in each one of the cities alone, and
the joint probability of rainfall in both cities is 0.99. Would
such a proposition appear reasonable? Certainly not, for even
intuitively it does not make much sense to claim that it rains
almost never in one of the cities, yet almost always in both of
them. The worrying question remains: which numbers could
be considered reasonable and consistent? Surely, the joint
probability should not exceed any single probability. This
certainly appears to be a necessary condition, but is it a sufficient
one?
Boole, and much  later Bell---already in the quantum mechanical context
and with a specific class of experiment in mind---derived constraints on the classical probabilities from the formalization of such considerations.
In a way, these bounds originate from the conception that all classical probability distributions
are just convex sums of extreme ones, which can be characterized by two-valued measures
interpretable as classical truth values.
They form a convex polytope bounded by Boole-Bell-type inequalities.

Remarkable, quantum probability theory is entirely different from classical probability theory,
as it allows a statistics of the joint occurrence of events which
extends and violates Boole's and Bell's classical constraints.
Alas,  quantum mechanics does not violate the constraints maximally,
quantum bounds fall just ``in-between'' the classical and maximal bounds.

The question is: how much exactly and quantitatively does quantum mechanics violate these bounds?
We have derived numerical as well as analytical bounds on the norm of quantum operators
associated with classical Bell-type inequalities
can be derived from their maximal eigenvalues.
This quantitative method enables detailed
predictions of the maximal violations of Bell-type inequalities,
and generalizes Tsirelson's result $2\sqrt{2}$
for the maximal violation of the Clauser-Horn-Shimony-Holt inequality.

We have also developed new protocols for quantum cryptography using interferometers.
Thereby, we have considered sets of quantum observables corresponding to {\em eutactic stars}.
Eutactic stars are systems of vectors which are the lower-dimensional ``shadow'' image,
the orthogonal view, of higher-dimensional orthonormal bases.
Although these vector systems are not comeasurable,
they represent redundant coordinate bases with remarkable properties.
One application is quantum secret sharing.
Fig. \ref{2003-eu-f1} depicts a typical configuration (boxes standing for a 50:50 mixing).
\begin{figure}
\begin{center}
\begin{tabular}{c}
%TexCad Options
%\grade{\off}
%\emlines{\off}
%\beziermacro{\on}
%\reduce{\on}
%\snapping{\off}
%\quality{2.00}
%\graddiff{0.01}
%\snapasp{1}
%\zoom{1.00}
\unitlength 0.75mm
\linethickness{0.8pt}
\begin{picture}(210.00,50.01)
\put(15.00,15.00){\framebox(20.00,30.00)[cc]{$R_{13}({\pi \over 4})$}}
\put(55.00,5.00){\framebox(20.00,40.00)[cc]{$R_{14}({\pi \over 4})$}}
\put(95.00,25.00){\framebox(20.00,20.00)[cc]{$R_{12}({\pi \over 4})$}}
\put(135.00,15.00){\framebox(20.00,30.00)[cc]{$R_{13}({\pi \over 4})$}}
\put(2.00,45.00){\makebox(0,0)[cc]{1}}
\put(2.00,35.00){\makebox(0,0)[cc]{2}}
\put(2.00,25.00){\makebox(0,0)[cc]{3}}
\put(2.00,15.00){\makebox(0,0)[cc]{4}}
\put(0.00,40.00){\vector(1,0){15.00}}
\put(0.00,20.00){\vector(1,0){15.00}}
\put(0.00,10.00){\vector(1,0){55.00}}
\put(0.00,30.00){\vector(1,0){5.00}}
\put(5.00,30.00){\line(2,5){8.00}}
\put(13.00,50.00){\line(1,0){67.00}}
\put(80.00,50.00){\line(1,-4){5.00}}
\put(85.00,30.00){\line(1,0){10.00}}
\put(115.00,30.00){\line(1,0){10.00}}
\put(125.00,30.00){\line(1,3){6.67}}
\put(131.67,50.00){\line(1,0){27.67}}
\put(35.00,40.00){\line(1,0){20.00}}
\put(75.00,40.00){\line(1,0){20.00}}
\put(115.00,40.00){\line(1,0){20.00}}
\put(35.00,20.00){\line(1,0){10.00}}
\put(45.00,20.00){\line(2,-5){8.00}}
\put(53.00,0.00){\line(1,0){26.00}}
\put(79.00,0.00){\line(1,3){6.67}}
\put(85.67,20.00){\line(1,0){49.33}}
\put(159.67,50.00){\line(1,-4){5.00}}
\put(164.67,30.00){\line(1,0){5.33}}
\put(155.00,40.00){\line(1,0){15.00}}
\put(155.00,20.00){\line(1,0){14.67}}
\put(75.00,10.00){\line(1,0){95.00}}
\put(165.00,40.00){\vector(1,0){5.00}}
\put(165.00,30.00){\vector(1,0){5.00}}
\put(165.00,20.00){\vector(1,0){5.00}}
\put(165.00,10.00){\vector(1,0){5.00}}
\bezier{28}(172.33,27.00)(174.67,27.67)(174.67,32.00)
\bezier{20}(174.67,32.00)(175.00,35.00)(177.00,35.00)
\bezier{28}(172.33,43.00)(174.67,42.33)(174.67,38.00)
\bezier{20}(174.67,38.00)(175.00,35.00)(177.00,35.00)
\bezier{28}(172.33,23.00)(174.67,22.33)(174.67,18.00)
\bezier{20}(174.67,18.00)(175.00,15.00)(177.00,15.00)
\bezier{28}(172.33,7.00)(174.67,7.67)(174.67,12.00)
\bezier{20}(174.67,12.00)(175.00,15.00)(177.00,15.00)
\put(183.00,35.00){\makebox(0,0)[lc]{share \# 2}}
\put(183.00,15.00){\makebox(0,0)[lc]{share \# 1}}
\end{picture}
\\
(a)
\\
\\
%TexCad Options
%\grade{\off}
%\emlines{\off}
%\beziermacro{\on}
%\reduce{\on}
%\snapping{\off}
%\quality{2.00}
%\graddiff{0.01}
%\snapasp{1}
%\zoom{1.00}
\unitlength 0.75mm
\linethickness{0.8pt}
\begin{picture}(185.00,50.01)
\put(150.00,15.00){\framebox(20.00,30.00)[cc]{$R_{13}( -{\pi \over 4})$}}
\put(110.00,5.00){\framebox(20.00,40.00)[cc]{$R_{14}(-{\pi \over 4})$}}
\put(70.00,25.00){\framebox(20.00,20.00)[cc]{$R_{12}( -{\pi\over 4})$}}
\put(30.00,15.00){\framebox(20.00,30.00)[cc]{$R_{13}(-{\pi \over 4})$}}
\put(183.00,45.00){\makebox(0,0)[cc]{1}}
\put(183.00,35.00){\makebox(0,0)[cc]{2}}
\put(183.00,25.00){\makebox(0,0)[cc]{3}}
\put(183.00,15.00){\makebox(0,0)[cc]{4}}
\put(170.00,40.00){\vector(1,0){15.00}}
\put(170.00,20.00){\vector(1,0){15.00}}
\put(130.00,10.00){\vector(1,0){55.00}}
\put(180.00,30.00){\vector(1,0){5.00}}
\put(180.00,30.00){\line(-2,5){8.00}}
\put(172.00,50.00){\line(-1,0){67.00}}
\put(105.00,50.00){\line(-1,-4){5.00}}
\put(100.00,30.00){\line(-1,0){10.00}}
\put(70.00,30.00){\line(-1,0){10.00}}
\put(60.00,30.00){\line(-1,3){6.67}}
\put(53.33,50.00){\line(-1,0){27.67}}
\put(150.00,40.00){\line(-1,0){20.00}}
\put(110.00,40.00){\line(-1,0){20.00}}
\put(70.00,40.00){\line(-1,0){20.00}}
\put(150.00,20.00){\line(-1,0){10.00}}
\put(140.00,20.00){\line(-2,-5){8.00}}
\put(132.00,0.00){\line(-1,0){26.00}}
\put(106.00,0.00){\line(-1,3){6.67}}
\put(99.33,20.00){\line(-1,0){49.33}}
\put(25.33,50.00){\line(-1,-4){5.00}}
\put(20.33,30.00){\line(-1,0){5.33}}
\put(30.00,40.00){\line(-1,0){15.00}}
\put(30.00,20.00){\line(-1,0){14.67}}
\put(110.00,10.00){\line(-1,0){95.00}}
\put(20.00,40.00){\vector(1,0){5.00}}
\put(20.00,30.00){\vector(1,0){0.50}}
\put(20.00,20.00){\vector(1,0){5.00}}
\put(20.00,10.00){\vector(1,0){5.00}}
\bezier{28}(12.67,27.00)(10.33,27.67)(10.33,32.00)
\bezier{20}(10.33,32.00)(10.00,35.00)(8.00,35.00)
\bezier{28}(12.67,43.00)(10.33,42.33)(10.33,38.00)
\bezier{20}(10.33,38.00)(10.00,35.00)(8.00,35.00)
\bezier{28}(12.67,23.00)(10.33,22.33)(10.33,18.00)
\bezier{20}(10.33,18.00)(10.00,15.00)(8.00,15.00)
\bezier{28}(12.67,7.00)(10.33,7.67)(10.33,12.00)
\bezier{20}(10.33,12.00)(10.00,15.00)(8.00,15.00)
\put(2.00,35.00){\makebox(0,0)[rc]{share \# 2}}
\put(2.00,15.00){\makebox(0,0)[rc]{share \# 1}}
\end{picture}
\\
(b)
\end{tabular}
\end{center}
\caption{Experimental realization of
(a) the encoding stage of a two-component two-share configuration
by an array of effectively two-dimensional beam splitters depicted as boxes.
The decoding stage (b) is just the encoding stage (a) in reverse order,
with inverse beam splitters. \label{2003-eu-f1}}
\end{figure}

------------------------------

Stefan Filipp and Karl Svozil, ``Generalizing Tsirelson's Bound on Bell Inequalities Using a Min-Max Principle'', Physical Review Letters 93, 130407 (2004). [CrossRef DOI:10.1103/PhysRevLett.93.130407], [arXiv], [pdf], [pdf], [tex], [ps]. This article has been selected by the APS for the Virtual Journal of Quantum Information 4(10) (October 2004) [VJQI issue].

Karl Svozil, ``Eutactic quantum codes'', Physical Review A 69, 034303 (2004). [CrossRef DOI:10.1103/PhysRevA.69.034303], [arXiv], [html], [pdf], [pdf], [tex], [ps]. This article has been selected by the APS for the Virtual Journal of Quantum Information 4(3) (March 2004) [VJQI issue].

Karl Svozil, ``Quantum information via state partitions and the context translation principle'', Journal of Modern Optics 51, 811-819 (2004). [CrossRef DOI:10.1080/09500340410001664179], [arXiv], [html], [pdf], [pdf], [tex], [ps].

Stefan Filipp and Karl Svozil, ``Testing the bounds on quantum probabilities'', Physical Review A 69, 032101 (2004) [CrossRef DOI:10.1103/PhysRevA.69.032101], [arXiv], [html], [pdf], [pdf], [tex], [ps]. This article has been selected by the APS for the Virtual Journal of Quantum Information 4(3) (March 2004) [VJQI issue].

------------------------------


\section{Coding strategies for Cochlear Implants}

This area of research is motivated by one of the first direct artificial interfaces to the
human brain, cochlear implants.
It deals with certain coding strategies which might be adequate for a signal transduction to
those structures in the cortex which interprets them.

One particular area of research is stochastic interference,
a phenomenon by which certain geometric properties of the
signal, such as self-similar ``fractal'' stochastic patterns and in particular
their characteristic dimension gets very sensitive to a change of the input signal
in a multi-channel configuration.
Such processes are immune to white noise and may quite well reflect the
successive convergence and divergence of neurons.

------------------------------

Geneva-talk

\end{document}
