%\documentclass[pra,showpacs,showkeys,amsfonts,amsmath,twocolumn]{revtex4}
\documentclass[amsmath,blue,table,sans]{beamer}
%\documentclass[pra,showpacs,showkeys,amsfonts]{revtex4}
\usepackage[T1]{fontenc}
%%\usepackage{beamerthemeshadow}
\usepackage[headheight=1pt,footheight=10pt]{beamerthemeboxes}
\addfootboxtemplate{\color{structure!80}}{\color{white}\tiny \hfill Karl Svozil (TU Vienna)\hfill}
\addfootboxtemplate{\color{structure!65}}{\color{white}\tiny \hfill Sensory Constraints on Aesthetics $\ldots$\hfill}
\addfootboxtemplate{\color{structure!50}}{\color{white}\tiny \hfill Alpbach Technology Forum, 2007-08-24\hfill}
%\usepackage[dark]{beamerthemesidebar}
%\usepackage[headheight=24pt,footheight=12pt]{beamerthemesplit}
%\usepackage{beamerthemesplit}
%\usepackage[bar]{beamerthemetree}
\usepackage{graphicx}
\usepackage{pgf}
%\usepackage[usenames]{color}
%\newcommand{\Red}{\color{Red}}  %(VERY-Approx.PANTONE-RED)
%\newcommand{\Green}{\color{Green}}  %(VERY-Approx.PANTONE-GREEN)

%\RequirePackage[german]{babel}
%\selectlanguage{german}
%\RequirePackage[isolatin]{inputenc}

\pgfdeclareimage[height=0.5cm]{logo}{tu-logo}
\logo{\pgfuseimage{logo}}
\beamertemplatetriangleitem
%\beamertemplateballitem

\beamerboxesdeclarecolorscheme{alert}{red}{red!15!averagebackgroundcolor}
%\begin{beamerboxesrounded}[scheme=alert,shadow=true]{}
%\end{beamerboxesrounded}

%\beamersetaveragebackground{yellow!10}

%\beamertemplatecircleminiframe

\begin{document}

\title{\bf \textcolor{blue}{Some Sensory Constraints on the Industrial Production of Aesthetics}}
%\subtitle{Naturwissenschaftlich-Humanisticher Tag am BG 19\\Weltbild und Wissenschaft\\http://tph.tuwien.ac.at/\~{}svozil/publ/2005-BG18-pres.pdf}
\subtitle{\textcolor{orange!60}{\small http://tph.tuwien.ac.at/$\sim$svozil/publ/2007-alpbach-pres.pdf\\
http://arxiv.org/abs/physics/0505088}}
\author{Karl Svozil}
\institute{Institut f\"ur Theoretische Physik, University of Technology Vienna, \\
Wiedner Hauptstra\ss e 8-10/136, A-1040 Vienna, Austria\\
svozil@tuwien.ac.at
%{\tiny Disclaimer: Die hier vertretenen Meinungen des Autors verstehen sich als Diskussionsbeitr�ge und decken sich nicht notwendigerweise mit den Positionen der Technischen Universit�t Wien oder deren Vertreter.}
}
\date{Alpbach Technology Forum, August 24, 2007}
\maketitle



\frame{
\frametitle{Contents}
\tableofcontents
}


\frame{
\frametitle{Main results}

\begin{itemize}
\item<1->
Every work of art can be seen as a structure created
by the artist and needing decryption and understanding from the audience.



\item<1->
Thus human aesthetics can be developed as a function of decryption.

\item<1->
Decryption is analyzed in terms of the required computational resources.



\item<1->
Due to human predisposition, aesthetics is inevitably based on natural forms,
thus providing some guidelines for and constraints on acceptable degrees of
complexity as interpreted by algorithmic information density.

\item<1->
While too condensed coding makes a decryption of a work of art impossible
and is perceived as random and chaotic by the untrained mind,
too regular structures are perceived as monotonous, too orderly and not very stimulating.


\end{itemize}

}

\frame{
\frametitle{Main results cntd.}

\begin{itemize}

\item<1->
Different arts have developed differently:


\begin{itemize}
\item<1->
wherever the costs of complexity are relatively low, such as in music or painting, the complexity increased, resulting in random, incomprehensible creations whose consumption
requires repetition and effort.

\item<1->
Whenever the expenses are high, such as in architecture and in ``contemporary state--of--the--art'' virtual space, the complexity has decreased or remains low, mostly connected to the pressure of cost and the scarcity of resources.


\end{itemize}
\end{itemize}

}


%%%%%%%%%%%%%%%%%%%%%%%%%%%%%%%%%%%%%%%%%%%%%%%%%%%%%%%%%%%%%%%%%%%%%%%%%%%%%%%%%%%%%%%%%%%%%%%%%%%%%%%%%%%%%%%%%%%%%



\section{Design acceptance depends on decryptability}

\frame{
\begin{center}
{\Huge I. Design acceptance depends on decryptability}
\end{center}
}


\subsection{Scheme}
\frame{
\frametitle{Scheme}

\begin{center}
\begin{tabular}{ccccc}
\hline\hline
idea                &
$\Rightarrow$    &
encryption            &
$\Rightarrow$      &
decryption
\\
\hline
artist  &
         &
artist    &
&
recipient, audience
\\
\hline
demand&&product design&&consumption
\\
\hline
no idea&&random patterns&&infinite effort\\
no effort&&no or infinite  effort\\
\hline
satisfaction&&easy or &&frustration\\
&&impossible task      \\
\hline\hline
\end{tabular}
\end{center}



}

\subsection{Complexity measures}
\frame{
\frametitle{Two types of complexity measures}

\begin{itemize}

\item<1->
Algorithmic Information: length of description in terms of a computer program;
``algorithmic incompressibility'' of design


\item<1->
Computational complexity: duration of (en- \&) decoding, time consumption thereof

\item<1->
In what follows, ``complexity'' is interpreted as high in algorithmic information, or high in computational complexity,
or high in both.

\item<1->
In terms of computation, encoding is expensive; so is decoding.


\end{itemize}
}


\subsection{Law of aesthetic complexity}

\frame{
\frametitle{Law of aesthetic complexity}
\begin{center}
\includegraphics<1>[height=6cm]{2007-alpbach-patt.jpg}\\
{\bf Too low-complex patterns appear monotonous and dull; too high-complex patterns appear irritating and chaotic.}
\end{center}
}





%%%%%%%%%%%%%%%%%%%%%%%%%%%%%%%%%%%%%%%%%%%%%%%%%%%%%%%%%%%%%%%%%%%%%%%%%%%%%%%%%%%%%%%%%%%%%%%%%%%%%%%%%%%%%%%%%%%%%

\section{``Nature Beauty'' versus ``Art Beauty'' (das Natur-Sch�ne versus das Kunst-Sch�ne according to Hegel)}

\frame{
\begin{center}
{\Huge II. ``Nature Beauty'' versus ``Art Beauty'' (das Natur-Sch�ne versus das Kunst-Sch�ne according to Hegel)}
\end{center}
}

\subsection{``Nature Beauty'' (das Natur-Sch�ne according to Hegel)}

\frame{
\begin{center}
\includegraphics<1>[height=8cm]{2005-ae-foliage.jpg}\\
{\tiny ``Nature Beauty:'' Autumn foliage near Baden, Lower Austria, Oct. 15, 2000}
\end{center}
}

\frame{
\begin{center}
\includegraphics<1>[height=8cm]{2005-ae-China_MountEverest_MER_FR_Orbit09148_20031130_hires_s.jpg}\\
{\tiny ``Nature Beauty:'' Mount Everest as seen by MERIS at orbit \# 09148 on Nov. 30th, 2003
(\copyright ESA/MERIS)}
\end{center}
}


\subsection{``Art Beauty'' (das Kunst-Sch�ne according to Hegel)}

\frame{
\begin{center}
\includegraphics<1>[height=8cm]{2005-ae-parkett-l.jpg}\\
{\tiny ``Art Beauty:'' Parquet flooring in the gallery rooms of the Garden Palais
 Liechtenstein, late 18th century, Vienna, Austria}
\end{center}
}


\frame{
\begin{center}
\includegraphics<1>[height=8cm]{2005-ae-bospiral.jpg}\\
{\tiny ``Art Beauty:'' Santino Bussi (1664-1736) Stucco detail in the Sala Terrena of the Garden Palais
 Liechtenstein, after 1700, Vienna, Austria}
\end{center}
}


\frame{
\begin{center}
\includegraphics<1>[height=5cm]{2005-ae-greekornament2.jpg}\\
{\tiny ``Art Beauty:'' Greek ornament from left to right: upper part of a stele,
termination of the marble tiles of the Pantheon;
the upper part of a stele;
by Lewis Vulliamy and
reprinted by Owen Jones, Grammar of Ornament}
\end{center}
}

\frame{
\begin{center}
\includegraphics<1>[height=8cm]{2005-ae-JanVanHuysum_Blumenstrauss.jpg}\\
{\tiny Jan Van Huysum, Flowers}
\end{center}
}




\subsection{Two examples to formalize \& utilize Nature Beauty}
\frame{
\frametitle{Two examples to formalize \& utilize Nature Beauty: Divine Proportion / Golden ratio (Pythagoras, Euclid, da Vinci, D�rer, $\ldots$)}

\begin{center}
\includegraphics<1>[height=3cm]{2007-alpbach-gr.jpg} \qquad
\includegraphics<1>[height=3cm]{2007-alpbach-dv.jpg}\\
{\tiny }
\end{center}

$$
\begin{array}{l}
{a+b\over a}={a\over b}=\varphi \Longrightarrow \varphi^2-\varphi -1=0 \\
\qquad
\qquad
\qquad
\qquad
\Longrightarrow \varphi ={1+\sqrt{5}\over 2}
\approx 1.618033989 \ldots
\end{array}
$$
}

\frame{
\frametitle{Examples to formalize \& utilize Natur Beauty: $1/f$ ``fractal'' noise (Voss, Clark, $\ldots$)}

\begin{center}
\includegraphics<1>[height=6cm]{2007-alpbach-voss1.jpg} \qquad
\includegraphics<1>[height=6cm]{2007-alpbach-voss2.jpg}\\
\end{center}



}
%%%%%%%%%%%%%%%%%%%%%%%%%%%%%%%%%%%%%%%%%%%%%%%%%%%%%%%%%%%%%%%%%%%%%%%%%%%%%%%%%%%%%%%%%%%%%%%%%%%%%%%%%%%%%%%%%%%%%

\section{Industrial design strategies and the scarcity of resources}


\frame{
\begin{center}
{\Huge III. Industrial design strategies and the scarcity of resources}
\end{center}
}

\subsection{Loos' principle}
\frame{
\frametitle{Loos' principle}

\begin{itemize}
\item<1->
Adolf Loos, in ``Ornament and Crime (1908)'' (``Ornament und Verbrechen'')
essentially asked the following question:

{\it ... why build one pretty house (with ornaments) if you can have
two ugly ones for the same price?}

\item<1->
Loos strongly endorsed monotony and cost effectiveness over beauty; to the point that he pretended
that ornamentation is so expensive it cripples the workforce and therefore has to be considered ugly.
So, he identified beauty with cheapness.

\item<1->
Even if one disagrees, cost is an important, if not sometimes dominant criterion and has to be taken seriously.

\end{itemize}

}


\subsection{Marx's principle}
\frame{
\frametitle{Joseph Marx: ''Weltsprache Musik'' (1964)}

\begin{itemize}
\item<1->
Marx calls himself a ``romantic realist''

\item<1->
Every work of art, if it claims to be aesthetically appealing, has to cope with the physiological apparatus of the recipient.

\item<1->
In the case of humans, this apparatus is limited, not unlimited, and it is bound to certain ``natural''
forms

\item<1->
This essentially supports tonal music.


\end{itemize}

}


\subsection{Second law of aesthetic complexity}
\frame{
\frametitle{Second law of aesthetic complexity}



\begin{center}
{\bf Aesthetic complexity trends are dominated by cost \& scarcity}
\end{center}
\begin{center}
\begin{tabular}{ccccc}
\hline\hline
art type                &
towards higher or    &expense
\\
&lower complexity\\
&(encode/decodeability) \\
\hline
painting & $\uparrow$&low\\
&&as practiced ;-)\\
music & $\uparrow$&low\\
architecture \& building & $\downarrow$&high\\
virtual space & ``ridiculously low''&extremely high \\
&but $\uparrow$\\
\hline\hline
\end{tabular}
\end{center}


}

\frame{
\frametitle{Some strategies for automated pattern formation and automated ornamentation}

\begin{center}
\includegraphics<1>[height=3cm]{2005-ae-greekornament1.jpg} \qquad
\end{center}
\begin{itemize}
\item<1->{Randomness and mutation}
\item<1->{Morphing and crossing of existing (natural) forms}
\item<1->{Permutation}
\item<1->{Self-similarity}
\item<1->{Repetition}
\item<1->{Symmetry}
\end{itemize}

}


\frame{
\frametitle{Summary of main results}

\begin{center}
\includegraphics[height=4cm]{2007-alpbach-thonet.jpg}
\end{center}
\begin{itemize}
\item<1->

A too complex work of art is impossible to decode
and is perceived as random and chaotic.
Too regular structures are perceived as monotonous, too orderly and not very stimulating.




\item<1->
The trend of complexity is dominated by cost.

\end{itemize}

}


%%%%%%%%%%%%%%%%%%%%%%%%%%

\frame{
\begin{center}
\includegraphics<1>[height=7cm]{2007-alpbach-panorama.jpg}
\end{center}
\centerline{\Large Thank you for your attention!}
 }


\end{document}
