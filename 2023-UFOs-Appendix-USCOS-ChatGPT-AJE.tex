%%%%%%%%%%%%%%%%%%%%% appendix.tex %%%%%%%%%%%%%%%%%%%%%%%%%%%%%%%%%
%
% sample appendix
%
% Use this file as a template for your own input.
%
%%%%%%%%%%%%%%%%%%%%%%%% Springer-Verlag %%%%%%%%%%%%%%%%%%%%%%%%%%


\chapter{US categories of secrecy}
\label{2023-UFO-Appendix-USCOS} % Always give a unique label
% use \chaptermark{}
% to alter or adjust the chapter heading in the running head

%Use the template \emph{appendix.tex} together with the Springer document class SVMono (monograph-type books) or SVMult (edited books) to style appendix of your book.




\section{Congressional power and oversight}
\label{2023-UFO-part-Perception-crash-retreivals-cpao}



Constitutional rights and executive secrecy and necessities may appear to be at odds. For instance, the Constitution of the United States delineates war powers between Congress and the President. Congress is authorized to declare war and allocate funds for the military, while the President serves as the commander in chief of the armed forces. This separation of powers reflects the framers' aim to balance the influence of the executive and legislative branches and to guarantee that decisions concerning war and peace are made with the advice and consent of both government branches.
In particular:
\begin{enumerate}
\item
Article I, Section 8 of the Constitution gives Congress the sole power ``to declare war, grant letters of marque and reprisal, and make rules concerning captures on land and water.'' This means that Congress has the authority to decide whether the United States should go to war and to authorize the use of military force.
\item
Article II, Section 2 of the Constitution, on the other hand, grants the president the power to be the commander in chief of the armed forces. This means that the president has the authority to direct the military and to make decisions about the use of military force, subject to the limitations and constraints imposed by Congress and the Constitution.
\end{enumerate}
Therefore, the US Constitution attempts to establish a system of checks and balances between Congress and the President and between the legislative and executive branches of government, with each branch having its own powers and responsibilities in matters related to war and national defense.

However, the division of power between Congress and the Presidency,
as well as executive branches such as the military and private entities, has fluid boundaries.
For example, what happens if Congress is not willing to declare war, but the President and associated agencies are?
The latter may have many pathways to get what they want.
One such pathway would be to turn the declaration of war ``upside down'' by stimulating or almost forcing an opponent
to declare war themselves.
These methods could involve sanctions and other aggressive measures that are
not classified as ``war'' and therefore do not need to be approved by Congress.
A controversial example of this is the Hull note, which provoked the Empire of Japan to attack,
ultimately leading to the US's entry into WWII.


Another pathway
is to distribute (mis)information, as has happened with the alleged
``throwing the babies out of the incubator testimony''~\cite{guyjohn592010Jun}
of the
daughter of the Kuwaiti ambassador to the US (at the time undisclosed and disguised as ``Nurse Nayirah''),
allegedly prepped by the PR agency Hill \& Knowlton before Congress that contributed
to launching the First Iraq War~\cite{krazyhandz6662011Sep},
or claims of the existence of ``Saddam's weapons of mass destruction,''
presented before the UN General Assembly that contributed to launching the Second Iraq War,
or
the Gulf of Tonkin incident that led to the US engaging more directly in the Vietnam War.
Once direct confrontations have commenced, in the fog of war~\cite{Morris-fog-of-war},
anybody doubting its beginning is called out as a traitor.


An anecdote suggests G\"odel's reservations regarding the lack of consistency of the US Constitution and its democratic robustness~\cite{Morgenstern1971Jan}; perhaps, in particular, his thoughts on the effects of executive orders. (However, this is just speculation.)

These historic events call into question the democratic oversight Congress is supposed to yield. This is particularly pressing in regard to highly sensitive information and undertakings of the executive branch of government. Therefore, it is of the utmost importance to understand the structure, categories, and hierarchy of compartmented secrecy within the three branches of the US executive: the Department of Energy, the Department of Defense, and, last but not least, the US intelligence community.

\section{Mandatory lies---from need-to-know to need-to-lie}
\label{2023-UFO-part-Perception-crash-retreivals-fntktntl}

As will be detailed later, US regulations mandate that individuals with knowledge of extremely confidential topics must deceive
those who are not authorized to access such information~\cite{vanderReijden2005,Sweetman2000,Dolan-MrX-Disclosure2020Jul}.
This rule applies regardless of rank, meaning that even lower-ranking personnel must deceive those of higher rank if they lack the necessary clearance.

In certain cases, it is also necessary to deceive the public, members of Congress (with some exceptions), and the press.
The president is exempt from this rule; however, if confidential information is disclosed, they must face the consequences.

This requirement to deceive is often justified as a means of preserving plausible deniability.
By deceiving those who are not authorized to access certain programs or events, individuals can deny knowledge of their existence if necessary.

Only a select group of individuals have knowledge of how many Special Access Programs (SAPs) are unacknowledged or waived.
This group includes the ``Gang of Eight'' of Congress---the leaders of each of the two parties from both the Senate and House of Representatives,
and the chairs and ranking minority members of
both the Senate Committee and House Committee for intelligence as set forth by 50 U.S.C. \S 3093(c)(2)---as well as the members of the Special Access Program Oversight Committee (SAPOC),\index{SAPOC} and the Secretary of Defense.

\section{US Department of Defense categories of secrecy}\label{2023-UFO-part-Perception-crash-retreivals-USDOFCOS}

Security clearances are like a Faustian trade with the devil (even if it is ``your kind of devil''): you gain access to information and capabilities, but you are bound by the confidentiality agreements you had to sign.

Ask yourself: how would you react if you were offered access to a ``sports model'' extraterrestrial craft
in exchange for keeping your lips sealed? If I were under pressure,
I might choose to accept the offer, which would mean compromising my freedom of expression.
(Perhaps I should clarify that nobody has approached me with any such offer, and it is possible that I might be excluded from it simply because of my Austrian citizenship---or even more simply, because no "sports model" extraterrestrial craft actually exists.)

\subsection{Ordinary secrecy classifications: ``Confidential,'' ``Secret,'' and ``Top Secret''}

Ordinary secrecy classifications are used for the protection of information that is sensitive or classified. There are three main levels of classification: ``Confidential,'' ``Secret,'' and ``Top Secret.''

\begin{enumerate}
\item
``Confidential''\index{confidential} is the lowest level of classification, and it is used for information that could potentially damage national security or cause other harm if it were to be released to the public. This could include information about military operations, diplomatic negotiations, or scientific research.

\item
``Secret''\index{secret} is a higher level of classification, and it is used for information that could cause serious damage to national security or cause other harm if it were to be released. This could include information about military plans, intelligence operations, or diplomatic relations.

\item
``Top Secret''\index{top secret} is the highest level of classification, and it is used for information that could cause extremely grave damage to national security or cause other harm if it were to be released. This could include information about military capabilities, intelligence sources, or diplomatic negotiations.
\end{enumerate}
In general, the higher the level of classification, the more sensitive the information is considered to be, and the more measures are taken to protect it.

\subsection{Beyond ordinary secrecy classifications: special access programs}

When normal protective measures, including those used for top secret programs, are not considered sufficient~\cite{vanderReijden2005},
a Special Access Program (SAP)~\cite{DCSA-SAPguide,DODDirective5205.07} can be created that has even stricter (and more expensive) rules, and uses more stringent safeguards for program security.

Many private contractors, such as Lockheed Martin (Corporation, LMT), The Boeing Company (BA), Raytheon Company (RTN), United Technologies Corporation (UTX), and Battelle (Memorial Institute), execute several of these programs. The level of oversight for SAPs varies. ``Blackest of black'' projects with the highest secrecy, called carve-out USAPs, have very little or no effective oversight by anyone, including the US Department of Defense.

In the past, there have been complaints about excessively strict restrictions. Is it possible that SAPs cause deployment delays? The Joint Security Commission, which was convened by then-deputy Secretary of Defense Bill Perry in 1993 and reported in 1994, stated that~\cite[p.~18-19]{RedefiningSecurity}:
\begin{svgraybox}
``even when military elements are briefed, they are put under
such tight constraints that they are unable to use the compartmented access
information in any practical way. This prohibits field elements from being
able to incorporate these capabilities into war planning and other crisis activities.
$\ldots$~A senior military officer on the Joint Staff expressed concern that current
classification and security procedures constrict the flow of operational
information to the warfighter at the tactical level. He felt that we still treat
certain capabilities as pearls too precious to wear---we acknowledge their
value, but because of their value, we lock them up and don't use them for
fear of losing them.''
\end{svgraybox}

\subsubsection{Acknowledged special access program}


An acknowledged Special Access Program (SAP) is one that has been officially recognized and acknowledged by the government. This means that the existence of the program has been officially acknowledged and is known to a limited number of individuals who have a need to know. However, the details of an acknowledged SAP, such as its subject matter and nature, may still be highly classified and not disclosed to the general public.


The legal basis for SAPs is derived from the Department of Defense Directive 5205.07~\cite{DODDirective5205.07}, defining
an ``acknowledged SAP [[as a]] SAP whose existence is acknowledged but its specific details
(technologies, materials, techniques, etc.) are classified as specified in the applicable security
classification guide.''

For example, if an acknowledged SAP has a code word, such as ``Blue Tomato'', it can be acknowledged that a program with this code word exists, but the details of what the program involves cannot be disclosed~\cite{Dolan-MrX-Disclosure2020Jul}.

To further protect against the disclosure of information, the Department of Defense uses a multilevel nomenclature system for its SAPs. Each SAP has an unclassified nickname, which is composed of two unclassified words, such as ``Have Blue'' or ``Rivet Joint''. These nicknames may be used on badges and in secure rooms to control access to information and physical facilities, even in programs that have a standard designation.

The nickname may consist of words such as ``Have'', ``Senior'', and ``Constant'', which are often used as the first word in Air Force programs,
``Tractor'' in Army programs, and ``Chalk'' in Navy programs~\cite{Sweetman2000}.

\subsubsection{Unacknowledged special access program}

An unacknowledged special access program (USAP)\index{USAP} is one that has not been officially recognized or acknowledged by the government. This means that the existence of the program is not officially known or publicly disclosed. An unacknowledged SAP may also be referred to as a ``black'' program, as it is not formally acknowledged or accounted for in the government's budget.

The legal basis for USAPs is derived from the Department of Defense Directive 5205.07~\cite{DODDirective5205.07}, defining
an ``unacknowledged SAP [[as a]]  SAP having protective controls ensuring the existence of the program
is not acknowledged, affirmed, or made known to any person not authorized for such  information.''
Directive 5205.07 establishes the policy and procedures for the governance, management, and oversight of DoD SAPs.
The directive also establishes the Special Access Program Oversight Committee (SAPOC),
which advises and assists the Secretary of Defense and the Deputy Secretary of Defense with USAPs.

According to Directive 5205.07, USAPs must be authorized by one of the following officials:
the Secretary of Defense, the Deputy Secretary of Defense, the Secretaries of the Military Departments, the Chairman of the Joint Chiefs of Staff, or the Combatant Commanders.


The level of secrecy and security surrounding an unacknowledged SAP is likely to be even higher than that of an acknowledged SAP. Access to an unacknowledged SAP may be strictly limited to a small number of individuals with a need to know and may require additional security clearances and background checks. As a consequence, for unacknowledged SAPs, both the nature of the program (as in acknowledged SAPs) and its various system components, such as its code name, must not be disclosed. In particular, the phrase ``no comment'' or similar is not sufficient. Persons in the know have to make up a total lie about an unacknowledged SAP, denying its existence, and they must not just say nothing---such as a ``no comment'' comment---or merely remain silent~\cite{Dolan-MrX-Disclosure2020Jul}.

This is mandatory even if asked by a supervisor who is not on the bigot list\index{bigot list}---a list of individuals who have been granted the necessary security clearance and are authorized to be informed about specific operations or sensitive information. The intentional dissemination of false information, or disinformation, is supported by two mechanisms within unacknowledged SAPs. If someone in the know is questioned about an unauthorized use of a program name or specific question of a USAP, they must deny any knowledge of a USAP. The person questioning them may not be aware of this requirement to deny knowledge and may therefore believe the denial and spread it further~\cite{Sweetman2000}.

Therefore, individuals may genuinely believe that there are no secret programs in their area of responsibility due to a lack of access to the information. As a result, when faced with a lack of confirmation and no way to distinguish between intentional and unintentional disinformation, many media outlets have stopped attempting to investigate classified programs~\cite{Sweetman2000}.


\subsubsection{Title 10 versus Title 50 of the United States Code}
\label{2023-UFO-part-Perception-crash-retreivals--t1v50}

Title 50 is the section of the United States Code that covers war and national defense,
including intelligence activities and covert action.
According to some sources, most USAPs are conducted under Title 50 authority,
as they involve sensitive intelligence operations that require a high degree of secrecy and plausible deniability.

However, some USAPs may also involve military operations or support under Title 10,
which is the section of the United States Code that covers armed forces.
Thus  informally,  Title 10  pertains to the Department of Defense  and military operations, whereas  Title 50  pertains
to intelligence agencies like the CIA, intelligence activities, and covert action~\cite{wall2011demystifying}.

It is plausible to suppose that the majority of UFO-related USAPs,
if indeed they exist, would fall under the Title 50 category. As highlighted by
\index{AARO}AARO's Sean Kirkpatrick~\cite[time = 2410~s]{Kirkpatrick2023Apr}, the current operations of this agency are conducted under Title 10.
However, incorporating Title 50 authority could significantly expand its scope of legitimate inquiry.



\subsubsection{Waived (unacknowledged) SAP}

Escalating secrecy levels further, waived SAPs are granted a special status that exempts them from most reporting requirements, in accordance with statutory authority granted by the Secretary of Defense~\cite{Dolan-MrX-Disclosure2020Jul}. As a result, waived SAPs are not disclosed to the general public or most members of Congress. The only individuals who are required to be informed of waived SAPs are the chairpersons and ranking committee members of the Senate Appropriations Committee, Senate Armed Services Committee, House Appropriations Committee, and House Armed Services Committee. These notifications are typically made orally rather than in writing.


The legal basis for waived SAPs is derived from the Department of Defense Directive 5205.07~\cite{DODDirective5205.07}, defining
a ``waived SAP [[as an unacknowledged]] SAP for which the Secretary of Defense has waived applicable reporting in
accordance with Reference (c) [[Section 119 of title 10, United States Code~\cite{HouseofRepresentatives2021Dec,CornellLaw-10/119-2023Jan}]]
following a determination of adverse effect to national security.
An unacknowledged SAP that has more restrictive reporting and access controls.''

It is important to note that waived SAPs are still subject to oversight by Congress. The chairpersons and ranking committee members of the relevant committees have the authority to request information about these programs as needed~\cite{Dolan-MrX-Disclosure2020Jul}. However, it has been suggested that many of these individuals do not request to be briefed on these programs and may not even be aware of their existence. This is often because they have numerous other responsibilities and may not see the relevance of being informed about these programs.

Additionally, it is possible for these programs to evade congressional oversight by providing irrelevant, distracting, or incomplete information and entries into the list of SAPs maintained by the Under Secretary of Defense for Acquisition and Sustainment. Despite this, waived SAPs are still subjected to inspections by the Department of Defense on an annual basis~\cite{Dolan-MrX-Disclosure2020Jul}.

%The contractors and participants carrying out such programs may attempt to evade congressional oversight by providing irrelevant, distracting, or incomplete information and entries into the list of SAPs maintained by the Under Secretary of Defense for Acquisition and Sustainment.

%However, even waived SAPs appear to be subject to (annual) inspections by the Department of Defense (DOD).

\subsubsection{Carve-out (unacknowledged) SAP}
\label{2023-UFO-part-Perception-crash-retreivals--cousap}

SAPs that have been carved out of Defense Security Service (DSS)\index{Defense Security Service} cognizance~\cite{saps2020} are called carve-out SAPs\index{carve-out SAPs}.
As a result, the responsibility for inspection is shifted from the DOD to the contractor, thereby transferring the duty of inspection and compliance to the mostly private contractor.

The legal basis for carve-out SAPs is derived from the Department of Defense Directive 5205.07~\cite{DODDirective5205.07}, defining
a ``carve-out [[as a]]  provision approved by the Secretary or Deputy Secretary of Defense that relieves
DSS of its National Industrial Security Program obligation to perform industrial security
oversight functions for a DoD SAP.''



\section{US intelligence categories of secrecy}
\label{2023-UFO-part-Perception-crash-retreivals--USICOS}


The intelligence community operates under the direction of the Director of National Intelligence (DNI),
who oversees and coordinates the activities of 17 US intelligence agencies,
including the CIA, NSA, and DIA. Much of the information produced by the intelligence community
is classified as Sensitive Compartmented Information (SCI).
SCI is used to protect highly sensitive intelligence information related to sources, methods, and activities.
Access to SCI is strictly limited to individuals who have undergone special security clearance procedures
and have a specific need to know. This type of information is marked with a special caveat, such as NOFORN,
indicating its high level of sensitivity.

SCI is handled within a strict security system called a Sensitive Compartmented Information Facility (SCIF), which is a specially designed and constructed secure area equipped with specialized security systems. The information contained within these facilities is marked with specific caveats to indicate its level of sensitivity. There are various types of SCI, each designed to protect specific types of sensitive intelligence information:
\begin{enumerate}
\item
Code Word SCI: This type of SCI is used to protect information that is considered to be among the most sensitive within the intelligence community. It is typically marked with a code word, such as GAMMA, to indicate the high level of sensitivity and security surrounding the information.

\item
Special Intelligence (SI) SCI: This type of SCI is used to protect information that relates to sensitive intelligence sources, methods, and activities. It is typically marked with the caveat SI (for Special Intelligence).

\item
TALENT KEYHOLE (TK) SCI: This type of SCI is used to protect information that relates to satellite intelligence. It is typically marked with the caveat TK (for Talent Keyhole).

\item
HUMINT Control System (HCS) SCI: This type of SCI is used to protect information that relates to human intelligence sources and methods. It is typically marked with the caveat HCS (for Human Intelligence Collection Systems).

\item
FGI SCI: This type of SCI is used to protect information that has been provided to the United States by a foreign government or international organization and is considered to be particularly sensitive. It is typically marked with the caveat FGI (for Foreign Government Information).
\end{enumerate}

Just as the Department of Defense (DOD) has its Special Access Program (SAP),
the United States intelligence community has a corresponding system known as Controlled Access Program (CAP)~\cite{ICD906,CRSCAP22}. CAPs operate as compartmentalized control systems within specific levels of classification, which involve compartments and sub-compartments.

These levels of classification include TOP SECRET, SECRET, and CONFIDENTIAL, signifying the relative sensitivity of intelligence activities, sources, and methods described within a document, as well as the potential damage to United States national security that could result from unauthorized disclosure.

Within the intelligence community (IC), CAPs are most commonly found as compartments or sub-compartments within the TOP SECRET level of classified intelligence.

Similarly to DOD's USAPs, as outlined in the Intelligence Community Directive 906~\cite{ICD906},
Unacknowledged Controlled Access Programs (UCAPs) are those CAPs
with protective controls that ensure the existence of the program is not acknowledged,
affirmed, or made known to any person not authorized for such information.
They are different from acknowledged CAPs,
which are CAPs whose existence is known or can be made known to individuals who
do not have a need to know.


\section{Department of Energy categories of secrecy}
\label{2023-UFO-part-Perception-crash-retreivals-DOECOS}


The Department of Energy (DOE) employs a classification and declassification system to safeguard sensitive information concerning
its missions and operations. The DOE classification system consists of three categories:
\begin{enumerate}
\item
Confidential: This is the lowest level of classification and is used to protect information that, if disclosed, could cause damage to the national security of the United States. This category is used for information that requires protection for a period of at least ten years.

\item
Secret: This is the intermediate level of classification and is used to protect information that, if disclosed, could cause serious damage to the national security of the United States. This category is used for information that requires protection for a period of at least twenty years.

\item
Top Secret: This is the highest level of classification and is used to protect information that, if disclosed, could cause exceptionally grave damage to the national security of the United States. This category is used for information that requires protection for a period of at least thirty years.
\end{enumerate}
In addition to these three categories, the DOE classification system also includes several special designations, such as ``Formerly Restricted Data'' and ``Transclassified Foreign Nuclear Information,'' that serve to safeguard information related to nuclear weapons. These designations play a critical role in preserving national security and preventing the spread of nuclear weapons:
\begin{enumerate}
\item
Restricted Data (RD).

\item
Formerly Restricted Data.

\item
Critical Nuclear Weapons Design Information (CNWDI).
\end{enumerate}
Nonetheless, there have been concerns raised about the potential misuse of these designations, which could lead to the withholding of information that should be accessible to the public.

According to the Department of Energy (DOE) website~\cite{DOEHFMSP22}, the DOE Special Access Program (SAP)
is a program established for a specific class of classified information that imposes
safeguarding and access requirements exceeding those normally required for information at the same classification level.
The DOE SAP is administered by the Executive Secretary of the Special Access Program Oversight Committee (SAPOC),
which conducts its activities in accordance with the requirements of DOE Order 471.5.
This directive is classified as ``Official Use Only (OUO)'' and can
only be accessed by people who have a legitimate need to know.
The DOE SAP is established under Executive Order 13526 or the Atomic Energy Act of 1954, as amended.



\section{Miscellaneous nomenclature}
\label{2023-UFO-part-Perception-crash-retreivals-MN}

\subsection{Core secret}

A core secret\index{core secret} is defined as any item, progress, strategy, or piece of information, the compromise of which would result in unrecoverable failure~\cite{Sweetman2000}.
If someone is asked about an unacknowledged SAP and they are not supposed to know about it, they are required to deny any knowledge of it. Such information is considered a core secret.
Note that if the person being questioned responds with ``no comment,''
such a statement could be taken as confirmation that the program or information does exist.
Therefore, a ``no comment'' response is not good enough.
The correct answer is falsity: ``No, it does not exist.''
If the questioner is unaware that the person being questioned is not supposed to have knowledge of the program or information, he or she may believe the denial and spread disinformation that the program or information does not exist, effectively allowing plausible deniability or an excuse to legally lie.

\subsection{Limited hangout}

A limited hangout\index{limited hangout} occurs when an organization or individual releases a limited amount of information to deflect attention from more sensitive or essential information that is being withheld. The term is often used in the context of large institutions or governments releasing partial or selective information in response to criticism or public pressure.

Speculation~\cite{Dolan-MrX-Disclosure2020Jul} suggests that a high-ranking military official may have been subjected to a limited hangout in his unconfirmed attempt to learn about crash retrievals by a private group. It is possible that he was provided with partial or selective information to distract from more sensitive or crucial information that was being withheld. The military official may have been informed that the group had little or no success, had been working for an extended period with minimal progress, faced difficulty obtaining assistance from outside experts, and had to overcome a challenging work environment. The aim may have been to implant this information in his mind and convince him that the group had never succeeded, possibly to reverse-engineer the technology. If the military official had been informed that the group had indeed developed the technology successfully, it would have been a game-changer.


%Admiral Wilson may have acted differently and demanded full access to all information for the Department of Defense.

Somewhat related to limited hangouts is lying by omission\index{lying by omission}. It is a type of deception in which someone intentionally leaves out important information or fails to disclose it, with the intent to mislead or deceive others. This can be done through words, actions, or both. It is often considered a form of lying because the person withholding the information is not being truthful about the complete picture and is therefore being dishonest.


\subsection{Bigot list}

A ``BIGOT'' list, also known as a bigot list, is a term utilized by the military and intelligence community to refer to a roster of personnel with clearance to access sensitive information or operations. The term ``BIGOT'' is an acronym used to identify individuals who have the required security clearance and need-to-know. This is to ensure they are informed about a particular operation or piece of sensitive information. The commanding officer or security officer responsible for the operation or information usually maintains the bigot list. It is used to keep track of who has been granted access to the sensitive material. The contents of a bigot list are generally classified, and access to the list is strictly controlled.


\subsection{Gatekeepers: program manager, security director, corporate lawyer}

In the day-to-day operations of Special Access Programs (SAPs), there are several key roles that are responsible for various aspects of the program. These include the Government Program Manager (GPM), who is responsible for all aspects of the SAP; the Contractor Program Manager (CPM), who is responsible for managing the program within the contractor facility; the Program Security Officer (PSO), who is responsible for all aspects of security; the Government SAP Security Officer (GSSO), who is responsible for security management and training for SAP-accessed individuals at government program facilities; and the Contractor Program Security Officer (CPSO), who is responsible for security management and training for SAP-accessed individuals at contractor facilities. All of these roles have specific duties and responsibilities related to maintaining the security and integrity of the SAP.

\subsection{Stovepiping}

Stovepiping is a term that refers to the way information is shared or disseminated within an organization. It often refers to a system in which information is passed vertically through a series of hierarchical levels rather than being shared horizontally among different units or departments. This can lead to a lack of communication and coordination within the organization, as well as a lack of integration and synthesis of information. Stovepiping can also lead to the siloing of information, where different units or departments have their own separate sources of information and do not share them with each other. This can result in a lack of situational awareness and a lack of understanding of the bigger picture.



\subsection{Compartmentalization}

Compartmentalization refers to the separation of different parts or functions within an organization or system. It can involve the separation of responsibilities, information, or resources to prevent the spread of problems or errors or to protect sensitive information or resources.

Compartmentalization is often used in the context of cybersecurity, where it is used to protect against the spread of malware or other malicious software. It can also be used in the management of organizations, where it can be used to clearly define roles and responsibilities or to separate different business functions to improve efficiency and control.

Compartmentalization can be an effective way to manage complex systems or organizations, but it can also have downsides, such as making it more difficult to share information or resources across different compartments. It is important to carefully consider the benefits and trade-offs of compartmentalization when designing or implementing systems or organizations.


\subsection{Sensitive compartmented information facility (SCIF)}

A Sensitive Compartmented Information Facility\index{Sensitive Compartmented Information Facility} (SCIF) is a secure area within a building that is used to store, process, and discuss sensitive compartmented information (SCI). SCI is a type of classified information that is protected by special security measures and controls and can only be accessed by individuals who have been specifically authorized to do so. SCIFs are designed to prevent unauthorized access to SCI and to protect it from espionage, sabotage, and other threats. They typically have physical security measures such as locked doors, security cameras, and access controls, as well as technical measures such as secure communications systems and secure computer networks. SCIFs may be used by government agencies, military units, or contractors who handle SCI on a regular basis.

White noise is a type of random noise that is often used in SCIFs to mask the sounds of conversations and other activities taking place within the SCIF. White noise is a continuous, broadband noise signal that contains a wide range of frequencies. It is designed to mask sounds by overwhelming them with constant, unobtrusive background noise. The use of white noise in SCIFs is intended to prevent people outside the SCIF from overhearing conversations or other sensitive information that may be discussed inside the facility. It is also used to prevent electronic eavesdropping, as white noise can interfere with the operation of listening devices. White noise is typically generated by a noise generator or by playing a recording of white noise through speakers in the SCIF.
