\PassOptionsToPackage{dvipsnames}{xcolor}
%\documentclass[amsmath,table,sans,amsfonts, handout]{beamer}
\documentclass[amsmath,table,sans,amsfonts]{beamer}
\usepackage[T1]{fontenc}
%%\usepackage{beamerthemeshadow}
%%\usepackage[headheight=1pt,footheight=10pt]{beamerthemeboxes}
%%\addfootboxtemplate{\color{structure!80}}{\color{white}\tiny \hfill Karl Svozil (TU Vienna)\hfill}
%%\addfootboxtemplate{\color{structure!65}}{\color{white}\tiny \hfill mur.sat \hfill}
%%\addfootboxtemplate{\color{structure!50}}{\color{white}\tiny \hfill Graz, 2010-12-11\hfill}
%\usepackage[dark]{beamerthemesidebar}
%\usepackage[headheight=24pt,footheight=12pt]{beamerthemesplit}
%\usepackage{beamerthemesplit}
%\usepackage[bar]{beamerthemetree}
\usepackage{graphicx}

\usepackage{wrapfig,lipsum,booktabs}

%Global Background must be put in preamble
{%
%\usebackgroundtemplate%      \includegraphics[width=\paperwidth,height=\paperheight]{HaK-Urkaos-da}%
}
 \setbeamercolor{background canvas}{bg=black}

%\usepackage{eepic}
%\usepackage[usenames]{color}
%\newcommand{\Red}{\color{Red}}  %(VERY-Approx.PANTONE-RED)
%\newcommand{\Green}{\color{Green}}  %(VERY-Approx.PANTONE-GREEN)


%\RequirePackage[german]{babel}
%\selectlanguage{german}
%\RequirePackage[isolatin]{inputenc}

%\pgfdeclareimage[height=0.5cm]{logo}{tu-logo}
%\logo{\pgfuseimage{logo}}
\beamertemplatetriangleitem
%\beamertemplateballitem

\beamerboxesdeclarecolorscheme{alert}{red}{red!15!averagebackgroundcolor}
%\begin{beamerboxesrounded}[scheme=alert,shadow=true]{}
%\end{beamerboxesrounded}

%\beamersetaveragebackground{yellow!10}

%\beamertemplatecircleminiframe

\newtheorem{question}{Question}
\newtheorem{conjecture}[question]{Principle}
\newtheorem{challenge}[question]{Challenge}

\usepackage{tikz}
\usetikzlibrary{calc,decorations.pathreplacing,decorations.markings,positioning,shapes}




\definecolor{orange(webcolor)}{rgb}{1.0, 0.65, 0.0}
\setbeamercolor{normal text}{fg=white}
\setbeamercolor{structure}{fg=yellow}

\setbeamertemplate{footline}
{
  \hbox{\begin{beamercolorbox}[wd=1\paperwidth,ht=2.25ex,dp=1ex,right]{framenumber}%
      \usebeamerfont{framenumber}\insertframenumber{} / \inserttotalframenumber\hspace*{2ex}
    \end{beamercolorbox}}%
  \vskip0pt%
}
\beamertemplatenavigationsymbolsempty

\setbeamersize{text margin left=0.5cm}

\begin{document}




\title{\bf \color{white} {Physical aspects of Kelly's bowling {\it vs.} curling metaphor \\-- creatio ex nihilo {\it vs.} continua}   }
\subtitle{ \color{white}
\footnotesize http://tph.tuwien.ac.at/\textasciitilde{}svozil/publ/2019-Svozil-Casablanca-pres.pdf
 \\
\footnotesize based on https://doi.org/10.1007/978-3-319-70815-7\_22 (OA)
}
\author{{Karl Svozil}}
\institute{{ITP/Vienna University of Technology, Austria\\
svozil@tuwien.ac.at
}
%{\tiny Disclaimer: Die hier vertretenen Meinungen des Autors verstehen sich als Diskussionsbeitr�ge und decken sich nicht notwendigerweise mit den Positionen der Technischen Universit�t Wien oder deren Vertreter.}
}
\date{\textcolor{white!100}{Casablanca, Morocco, June 18th, 2019}}

\maketitle

% \frame{
% \frametitle{}
%
% }

\begin{frame}
\frametitle{Some {\it caveats}}
    \begin{itemize}
\item
my {\em \color{red}ignorance} both as a person and as a contemporary:
{\it Ei mihi, qui nescio saltem quid nesciam!}
{ (Alas for me, that I do not at least know the extent of my own ignorance!)}
{  --~Aurelius Augustinus, 354--430, ``Confessiones'' (Book XI, chapter 25)}
\\
Besides {\em \color{red}unknown unknows} there are even {\em \color{red}unknown knowns};
that is, things we believe we know but actually don't know
(cf. Donald H. Rumsfeld, February 12, 2002; documentary 2013)


\item
what constitutes a {\em  \color{red}message} -- that is, (non)randomness?  --- {\em \color{red}means relativity} of (non)randomness

\item
there appears to be an obvious continuum between a {\em  \color{red}``chaotic''} universe (Exner, 1909) on the one hand,
and a {\em  \color{red}clockwork} universe; with {\em  \color{red}miracles} and
some sort of {\em  \color{red}Ark of the Covenant} in-between.
{\em \color{red}On what point in this bracket are we?}

    \end{itemize}

 \end{frame}

\begin{frame}
\frametitle{Mathematics of indeterminism/randomness: non-operational \& non-constructive \& blocked by provable non-provability}
    \begin{itemize}
\item
there is no mathematical definition for a {\em finite} sequence of events

\item
{\em (a)causal (in)dependence} of two or more events is subject to (spurious) correlations
https://doi.org/10.3390/philosophies4020017

\item
transfinite definition of (in)determinism {\it via} infinite sequences and theory of (in)computability

\item
transfinite definition of randomness {\it via} infinite sequences and algorithmic incompressibility
https://doi.org/10.1016/0030-4018(87)90271-9

\item
formal proofs of (in)determinism are in general blocked by G\"odel/Tarski/Turing-type incomputability

\item
formal arguments {\em depend on the assumptions} (axioms, rules of derivation) made -- aka ``garbage-in-garbage-out''
--
there is no {\em ``archimedian ontological anchor''} on which to base whatever

    \end{itemize}

 \end{frame}


 \begin{frame}
 \frametitle{Bowling {\it vs.} curling / gap scenario in classical physics: uniqueness of solution of ordinary differential equation}

According to the Picard-Lindel\"of theorem %~\cite[Theorem~1.6.2]{nagy-ODE}
\index{Picard-Lindel\"of theorem}
an {\em initial value problem}
\index{initial value problem}
\index{Cauchy problem}
defined by a first orderordinary differential equation of the form $y'(t)=f(t,y(t))$
and the initial value $y(t_0)=y_0$
has a {\em unique} solution if $f$  satisfies the
Lipschitz condition and is continuous as a function of $t$.
A mapping $f$ satisfies (global/local) {\em Lipschitz continuity} (or, used synonymously,   {\em Lipschitz condition})
\index{Lipschitz continuity}
\index{Lipschitz condition}
with finite positive constant $0<k<\infty$ if
it increases the distance between any two points $y_1$ and $y_2$ (of its entire domain/some neighbourhood)
by a factor at most $k$:
$$
\vert f(t,y_2)-f(t,y_1) \vert \le k \vert y_2 - y_1 \vert
.
$$
That is,
$f$ may be nonlinear as long as it does not separate different points $y_1$ and $y_2$ ``too much.''

\begin{center}\color{orange}
$\widetilde{\qquad \qquad }$
$\widetilde{\qquad \qquad}$
$\widetilde{\qquad \qquad }$
\end{center}

{\color{red}Recent example for non-uniqueness: {\em Norton dome} https://www.pitt.edu/\textasciitilde{}jdnorton/Goodies/Dome/
}
 \end{frame}

 \begin{frame}
 \frametitle{Bowling  scenario in classical physics II: deterministic chaos}

\begin{itemize}
\item
assume classical continuum; select (as per the axiom of choice) one element thereof as a ``seed'' or initial value
\item
deterministically ``reveal'' the information of the seed such that initially ``close'' states become ``hugely separated''

\end{itemize}

 \end{frame}

 \begin{frame}
 \frametitle{Curling / gap scenarios in quantum mechanics -- how to market weakness as strength}


\begin{itemize}
\item
quantum complementarity: in certain situations characterized by finite physical means
you can't have your cake and eat it too
\item
quantum value indefiniteness (aka contextuality): attempts to interpret certain finite configurations of
quantum observables as classical value definite properties fail miserably
https://doi.org/10.1063/1.4931658


\item
(radioactive) decay and spontaneous as well as stimulated emissions: no causes found so far

\end{itemize}

 \end{frame}


 \begin{frame}
 \frametitle{Remarks regarding quantum (in)determinism}
\begin{itemize}
\item
``quantum mechanics only'' is inconsistent (permutative state evolution vs. measurement; nesting)
https://doi.org/10.1103/RevModPhys.29.454

\item
quantum physics is ``vector world'' -- different from classical logic based on power sets


\item
many ``evangelical theoreticians'' preach various quantum gospels; no consolidated ``interpretation'' (aka semantics)
https://doi.org/10.1007/978-3-662-05032-3\_6

\item
general (deterministic) extensions of quantum mechanics exist and cannot be excluded; specific ones can

\end{itemize}

 \end{frame}


 \begin{frame}
 \frametitle{Executive summary in one short phrase}

 \begin{center}
{\color{red} \Huge (almost) ``anything goes''}\\
(ask Paul Feyerabend \& Cole Porter, in that order)
$\;$\\
$\;$\\
{\color{yellow} \Huge Of the many scientific narratives conceived so far,
not much is of any relevance for theology; but what they tell has a great utility (technology-wise).
}
 \end{center}

 \end{frame}





\frame{

\centerline{\huge {\color{yellow} Thank you for your attention!}}

\begin{center}\color{orange}
$\widetilde{\qquad \qquad }$
$\widetilde{\qquad \qquad}$
$\widetilde{\qquad \qquad }$
\end{center}

 }
 \end{document}








 \frame{
 \frametitle{}

 }



