\PassOptionsToPackage{usenames,dvipsnames}{xcolor}
%\documentclass[amsmath,table,sans,amsfonts, handout]{beamer}
\documentclass[amsmath,table,sans,amsfonts,hyperref={colorlinks,citecolor=blue,linkcolor=blue,urlcolor=purple}]{beamer}
\usepackage[T1]{fontenc}
%%\usepackage{beamerthemeshadow}
%%\usepackage[headheight=1pt,footheight=10pt]{beamerthemeboxes}
%%\addfootboxtemplate{\color{structure!80}}{\color{white}\tiny \hfill Karl Svozil (TU Vienna)\hfill}
%%\addfootboxtemplate{\color{structure!65}}{\color{white}\tiny \hfill mur.sat \hfill}
%%\addfootboxtemplate{\color{structure!50}}{\color{white}\tiny \hfill Graz, 2010-12-11\hfill}
%\usepackage[dark]{beamerthemesidebar}
%\usepackage[headheight=24pt,footheight=12pt]{beamerthemesplit}
%\usepackage{beamerthemesplit}
%\usepackage[bar]{beamerthemetree}
\usepackage{graphicx}
\usepackage{pgf}
%\usepackage{eepic}
%\newcommand{\Red}{\color{Red}}  %(VERY-Approx.PANTONE-RED)
%\newcommand{\Green}{\color{Green}}  %(VERY-Approx.PANTONE-GREEN)

\definecolor{applegreen}{rgb}{0.55, 0.71, 0.0}

\usepackage{fourier-orns}  %fancy symbols https://mirror.easyname.at/ctan/fonts/fourier-GUT/doc/fourier-orns-doc.pdf

%\usepackage{musixtex}

\newcommand{\Abschnitt}[1]{{\section #1}}

%%%%%%%%%%%%%%%%%%%%%%%%%%%%%
\usepackage{iftex}
\ifxetex
\usepackage{fontspec}% Schriftumschaltung mit den nativen XeTeX-Anweisungen
                     % vornehmen. Voreinstellung: Latin Modern
%\usepackage[ngerman]{babel}% Sprachumschaltung: Deutsch nach neuer Rechtschreibung



%
% XeLaTeX
%
\XeTeXinputencoding cp1252
\usepackage{fontspec}
%%\setmainfont{Times New Roman}
\setmainfont{Garamond}
\setsansfont{Garamond}
%\setmainfont{EB Garamond}
%\setsansfont{EB Garamond}
%
\else
\usepackage[latin1]{inputenc}
\usepackage[T1]{fontenc}
\fi
%%%%%%%%%%%%%%%%%%%%%%%%%%%%%

%\RequirePackage[german]{babel}
%\selectlanguage{german}
%\RequirePackage[isolatin]{inputenc}

%\pgfdeclareimage[height=0.5cm]{logo}{tu-logo}
%\logo{\pgfuseimage{logo}}
\beamertemplatetriangleitem
%\beamertemplateballitem

\beamerboxesdeclarecolorscheme{alert}{red}{red!15!averagebackgroundcolor}
%\begin{beamerboxesrounded}[scheme=alert,shadow=true]{}
%\end{beamerboxesrounded}

%\beamersetaveragebackground{yellow!10}

%\beamertemplatecircleminiframe

\newtheorem{question}{Question}
\newtheorem{conjecture}[question]{Principle}
\newtheorem{challenge}[question]{Challenge}
\usepackage{tikz}
\newcommand{\bra}[1]{\left< #1 \right|}
\newcommand{\ket}[1]{\left| #1 \right>}

\newcommand{\iprod}[2]{\langle #1 | #2 \rangle}
\newcommand{\mprod}[3]{\langle #1 | #2 | #3 \rangle}
\newcommand{\oprod}[2]{| #1 \rangle\langle #2 |}

\newcommand{\proj}[3]{\begin{smallmatrix} #1 & #2 & #3 \end{smallmatrix}}
\newcommand{\projbf}[3]{\begin{smallmatrix} \mathbf{#1} & \mathbf{#2} & \mathbf{#3} \end{smallmatrix}}

\sloppy
\parskip .7em %vskip between paragraphs

\newcommand{\seq}[1]{\mathbf{#1}}
\newcommand{\floor}[1]{\left\lfloor #1 \right\rfloor}
\newcommand{\ceil}[1]{\left\lceil #1 \right\rceil}
\newcommand{\m}[1]{\widetilde{#1}}
%\newcommand{\p}[1]{\scriptsize\textcolor{black}{$[#1]$}}

\usepackage[most]{tcolorbox}
\begin{document}

\title{\textcolor{black}{\bf Utilizing Quantum Distinctness to Create Space and Time Frames}}
\subtitle{\footnotesize \url{http://tph.tuwien.ac.at/~svozil/publ/2023-QIP23-pres1.pdf}
%%%\\
%%%\footnotesize based on \href{https://arxiv.org/abs/1903.10424}{arXiv:1903.10424}
}
\author{\textcolor{black}{Karl Svozil}}
\institute{\normalsize \textcolor{black}{Institute for Theoretical Physics, TU Wien}\\
\textcolor{black}{svozil@tuwien.ac.at}
%{\tiny Disclaimer: Die hier vertretenen Meinungen des Autors verstehen sich als Diskussionsbeitr�ge und decken sich nicht notwendigerweise mit den Positionen der Technischen Universit�t Wien oder deren Vertreter.}
}
\date{{\color{purple}Tuesday, June 13, 2023,
Quantum Information and Probability: from Foundations to Engineering (QIP23), V\"axj\"o, Sweden}}
\maketitle


% \frame{
% \frametitle{Contents}
% \tableofcontents
% }

\section{Category formation -- the ``bigger'' picture}

\begin{frame}[shrink=8]
 \frametitle{Category formation -- the ``bigger'' picture}

 {\color{black}
Evolutionary psychology ``pushed'' or ``stipulated'' a particular ``worldview''
or formation of category of perceptions in terms of what we might call ``objects''
or ``images'' of our thinking.

Cf Hertz: {\it ``We form for ourselves images or symbols of external objects;
and the form which we give them is such that the necessary
consequents of the images in thought are always the images of
the necessary consequents in nature of the things pictured. In
order that this requirement may be satisfied, there must be a
certain conformity between nature and our thought.''}

Those objects have no stringent ontological status, although we might believe so.
In particular, space-time frames are such images.
We need to be careful in conceptualizing them.

One possible ``reasonable'' guidance principle: operationalism -- cf. Einstein synchronicity, Bridgman, Zeilinger.

Therefore, it is not totally unreasonable to attempt to construct space-time frames in terms of quantum mechanics.
}

\end{frame}

\section{65+ years of contextuality craze}

\frame{
 \frametitle{65+ years of contextuality craze---the informal part}

 {\color{black}
Since Gleason's 1957 theorem (and before) physicists and mathematicians like
Specker, Bell, Greenberger, Horne {\&} Zeilinger exploited the consequences of the following fact:

That we are living in a vector world, that all entities like states need to be expressed in terms of (normalized) vectors.

This is in contrast to (quasi)classical underpinnings, dominated by scalars and power sets, partition logics, and set theoretic operations.

This difference is called {\color{magenta}``contextuality''}.
}

}

\begin{frame}[shrink=8]
 \frametitle{65+ years of contextuality craze---the formal part: ``localized'' contextuality}

 {\color{black}
``Localized'' contextuality by
Specker,  Zierler and Schlessinger, Kochen {\&} Specker, Hardy etc:
algebra of quantum logic---with propositions as (normalized) vectors, and logical operations identified as orthogonal subspace of a Hilbert space ($\equiv$not),
intersection ($\equiv$and), and linear span ($\equiv$or) cannot be homomorphically (faithfully) embedded into Boolean algebras.

Kochen {\&} Specker's demarcation criterium ``Theorem 0'': classical two-valued states on a quantum logic are {\color{magenta} non/separating} the atoms (propositions)---if they are, the embeddibility is guaranteed, otherwise ``contextual''.

Contextuality goes beyond complementarity because there are complementary sets of propositions (logics) that are {\color{magenta} not embeddable},
but all complementary sets of propositions are complementary.

%Specker and Sch\"utte were interested also in classical tautologies which are no quantum tautologies.

Weaker forms of contextuality involve only probabilistic discrepancies.
}


\end{frame}

\frame{
 \frametitle{65+ years of contextuality craze---the formal part: ``non-localized'' contextuality}

 {\color{black}
``Non-localized'' contextuality by
Einstein, Podolsky {\&} Rosen, Bell, Greenberger, Horne {\&} Zeilinger
considered {\color{magenta} multi-partite} configurations---explosion views of ``localized'' contextuality of sorts.

Formally, multi-partite states are quantum mechanically in the tensor product space of the single particle spaces.
However, not all such vectors in the tensor product space can be represented as Cartesian product of single particle states;
but rather, as a coherent superposition aka sum thereof.
This was found and pointed out by Schr\"odinger and termed {\color{magenta}``entanglement''}.

In particular, these non-localized configurations of operator valued observables yield
``counterfactually testable''---one complementary term after another at different time---(total) contradictions between
classical and quantum predictions.

}

}

\section{Challenges to space-time formation for ``non-localized'' contextuality}
\begin{frame}[shrink=10]
 \frametitle{Challenges to space-time formation for ``non-localized'' contextuality}

 {\color{black}
``Non-localized'' contextuality exploiting entanglement---ie non-factorizable quantum states---present a potential problem to classical space-time constructions,
eg via Einstein synchronization in conjunction with randomness of the outcomes:

\begin{itemize}
\item[$\bullet$]
The (pure) quantum state of the entangled constituents essentially
(due to non-local unitary transformation of localized product states) re-encodes or scrambles those state to be {\color{magenta} relational}
and (unlike classical states) {\color{magenta} devoid of indidual definite ``localized'' properties or local shares}---only
the relational properties among its constituents are value definite.

\item[$\bullet$]
If any single random outcome, under strict Einstein locality conditions,
is somehow only subjected to or influenced by the ``local'' environment of the particle constituent's
environment, and at the same time (see earlier) the entire state lacks any individual local property or local share of the constituent particles whatsoever---then~$\ldots$

\item[$\bullet$]
{\color{magenta} How come relationality is maintained and strictly observed?}

\end{itemize}
}
\end{frame}


\frame{
 \frametitle{Challenges to space-time formation for ``non-localized'' contextuality cntd.}

{\color{black}

Imo this kind of  ``transfer of relationality without individual definiteness'' by a ``transfer of entanglement'' between
spacially separated under strict Einstein locality conditions,
is the gist of the conundrum.

Recall that classical states, as for instance pointed out by Peres in ``unperformed experiments have no result'',
perform relationally by {\color{magenta} possessing definite local shares},
which are revealed by measurements, and, therefore, result in (perfect) relational correllations.

(Even cosine-type correlations are insufficient; but communication of contexts are.)
}
}



\begin{frame}%[shrink=8]
 \frametitle{Challenges to space-time formation for ``non-localized'' contextuality cntd.}

 {\color{black}
In my opinion, as speculated earlier, in order to reasonably categorize space-time (frames) we need to take
quantum mechanics as primary, and develope and operationalize space-time frames entirely by quantum means.

This is different from Kantian conceptualizations of space-time as ``intuited a priori''.

As a consequence we need to observe what, in quantum terms, may be considered {\color{magenta} separate}, and what not.

I postulate that constituents in entangled states {\color{magenta} as well as measurement outcomes on such states cannot be considered separate}.

}

\end{frame}

\begin{frame}%[shrink=8]
 \frametitle{Challenges to space-time formation for ``non-localized'' contextuality cntd.}

 {\color{black}

Therefore, at least in the entangled observables, the constituents are not spatially separated at all: in other words,
for such ``affected'' observables, spatial distances shrink to zero.

This does not necessarily mean that with respect to other observables of these constituents, the separation is zero.

In this view {\color{magenta} spatial separation is means relative}, and thus {\color{magenta}  space-time frames are means relative} with respect to
the quantum shares involved.

}

\end{frame}


\section{Suggestions for a new protocol of clock synchronization}
\frame{
 \frametitle{Suggestions for a new protocol of clock synchronization}

 {\color{black}
For entangled shares I therefore suggest to abandon Einstein clock synchronization by exchanges of (light) signals.

I suggest to employ a Bennett-Brassard-Eckert-type protocol utilizing random outcomes of entangled multi-partite states
as a time standard.
Thereby, local entangled time is successively made precise and generated by the correlated outcomes of entangled states.

As a result, relativity theory is ``relativized'' further by into a multitude of means relative ``patches'' of space-time (frames).

}

}

\frame{

\centerline{\Large {\color{magenta} Thank you for your attention!}}

\begin{center}\color{orange}
$\widetilde{\qquad \qquad }$
$\widetilde{\qquad \qquad}$
$\widetilde{\qquad \qquad }$
\end{center}
 }
 \end{document}


















\section{ }

\frame{
 \frametitle{ }

\begin{itemize}
\item[$\bullet$] {
%\color{purple}
}
\pause
\item[$\bullet$] {
%\color{purple}
}
\end{itemize}
}

\section{ }

\frame{
 \frametitle{ }

\begin{itemize}
\item[$\bullet$] {
%\color{purple}
}
\pause
\item[$\bullet$] {
%\color{purple}
}
\end{itemize}
}

\section{ }

\frame{
 \frametitle{ }

\begin{itemize}
\item[$\bullet$] {
%\color{purple}
}
\pause
\item[$\bullet$] {
%\color{purple}
}
\end{itemize}
}

\section{ }

\frame{
 \frametitle{ }

\begin{itemize}
\item[$\bullet$] {
%\color{purple}
}
\pause
\item[$\bullet$] {
%\color{purple}
}
\end{itemize}
}

