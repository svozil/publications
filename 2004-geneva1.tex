%%tth:\begin{html}<LINK REL=STYLESHEET HREF="/~svozil/ssh.css">\end{html}
\documentstyle[a4]{article}
\RequirePackage{graphicx}
%\documentstyle[amsfonts]{article}
%\RequirePackage{times}
%\RequirePackage{courier}
%\RequirePackage{mathptm}
%\renewcommand{\baselinestretch}{1.3}
\begin{document}

%\def\frak{\cal }
%\def\Bbb{\bf }
\sloppy

Stochastic interference and auditory perception
(ESPCI 2004 // Nr. 161)

Klaus Ehrenberger
Universitaetsklinik fuer Hals-, Nasen- und Ohrenkrankheiten
Medizinische Universitaet Wien
1090 Vienna, Waehringer G�rtel 18-20
email: klaus.ehrenberger@meduniwien.ac.at

and

Karl Svozil
Institut fuer Theoretische Physik
Technische Universit�t Wien
1040 Vienna, Wiedner Hauptstrasse 8-10/136
email: svozil@tuwien.ac.at


------------------------

- Why fractal noise in CI?

- What is Stochastic Interference?

- How can it be implemented?

------------------------
   The Why

Due to physiologic memory effects, the formation of spike patterns of nerve activities can be characterized by fractal geometry.

Acoustic stimuli such as music obey fractal geometry.

The converging and diverging neuronal pathways, in which such fractal geometries are processed present additional challenge to an understanding of the processes contributing to auditory perception; in particular to speech and to music.

Any method which attempts a faithful representation and reconstruction of the electric activity pattern has to implement fractal geometric stimuli of auditory nerves, and also has to cope with the converging and diverging processing of those stimuli.

Ref. (among others):
R. F. Voss, J. Clarke, 1/f noise in music and speech, Nature 258 (1975) 317-318.
R. F. Voss, J. Clarke, 1/f noise in music: music from 1/f noise, Journal of Acoustical Society of America 63 (1978) 258-263.
K. Ehrenberger, D. Felix and K. Svozil, ``Origin of Auditory Fractal Random Signals in Guinea Pigs'', NeuroReport 6, 2117-2120 (1995).
Wolfgang Gstoettner, Wolf Baumgartner, Jafar Hamzavi, Dominik Felix, Karl Svozil, Reiner Meyer und Klaus Ehrenberger, ``Auditory fractal random signals: Experimental data and clinical application", Acta Oto-Laryngologica 116, 222-223 (1996).

------------------------
   The What

Utilization of fractal noise

Take N channels of stochastic fractal noise $C_k = c_{1k}c_{2k}c_{3k}\ldots $, $1\le k\le N$.

Code these signals by $c_{ik} =$ "0" and "1," characterizing activity, respectively.

At each time, generate a combined signal $S= s_{1}s_{2}s_{3}\ldots $ by taking the logical AND operation
represented by a multiplication $s_i = c_{i1}c_{i2}\cdots c_{iN}$.

Small changes in the fractal geometry of input signals yield a potentially large change in the fractal geometry of the output signal.

$\Delta D_S = N \Delta D$, where $D$ and $D_S$ are fractal dimensions of the original and the combined signals (obtained e.g. by box counting techniques).

Variation of the fractal dimension of the secondary spike pattern is proportional to the number of converging input signals and to the variation of their fractal dimension.

Tradeoff: signal gets weaker.

The effect remains unchanged if white noise is added.

K. Svozil, D. Felix, K. Ehrenberger, Amplification by stochastic interference, Journal of Physics A 29 (1996) L351-L354.
------------------------
   The How

Multi-electrode stimulation of nerve cells.

Additional strategy for fine resolution of sound by the input of more than one information channels.

Thank you for your attention.
------------------------




\end{document}

