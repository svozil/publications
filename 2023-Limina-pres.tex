\PassOptionsToPackage{usenames,dvipsnames}{xcolor}
%\documentclass[amsmath,table,sans,amsfonts, handout]{beamer}
\documentclass[amsmath,table,sans,amsfonts,hyperref={colorlinks,citecolor=blue,linkcolor=blue,urlcolor=purple}]{beamer}
\usepackage[T1]{fontenc}
%%\usepackage{beamerthemeshadow}
%%\usepackage[headheight=1pt,footheight=10pt]{beamerthemeboxes}
%%\addfootboxtemplate{\color{structure!80}}{\color{white}\tiny \hfill Karl Svozil (TU Vienna)\hfill}
%%\addfootboxtemplate{\color{structure!65}}{\color{white}\tiny \hfill mur.sat \hfill}
%%\addfootboxtemplate{\color{structure!50}}{\color{white}\tiny \hfill Graz, 2010-12-11\hfill}
%\usepackage[dark]{beamerthemesidebar}
%\usepackage[headheight=24pt,footheight=12pt]{beamerthemesplit}
%\usepackage{beamerthemesplit}
%\usepackage[bar]{beamerthemetree}
\usepackage{graphicx}
\usepackage{pgf}
%\usepackage{eepic}
%\newcommand{\Red}{\color{Red}}  %(VERY-Approx.PANTONE-RED)
%\newcommand{\Green}{\color{Green}}  %(VERY-Approx.PANTONE-GREEN)

\definecolor{applegreen}{rgb}{0.55, 0.71, 0.0}

\usepackage{fourier-orns}  %fancy symbols https://mirror.easyname.at/ctan/fonts/fourier-GUT/doc/fourier-orns-doc.pdf

%\usepackage{musixtex}

\newcommand{\Abschnitt}[1]{{\section #1}}

%%%%%%%%%%%%%%%%%%%%%%%%%%%%%
\usepackage{iftex}
\ifxetex
\usepackage{fontspec}% Schriftumschaltung mit den nativen XeTeX-Anweisungen
                     % vornehmen. Voreinstellung: Latin Modern
%\usepackage[ngerman]{babel}% Sprachumschaltung: Deutsch nach neuer Rechtschreibung



%
% XeLaTeX
%
\XeTeXinputencoding cp1252
\usepackage{fontspec}
%%\setmainfont{Times New Roman}
\setmainfont{Garamond}
\setsansfont{Garamond}
%\setmainfont{EB Garamond}
%\setsansfont{EB Garamond}
%
\else
\usepackage[latin1]{inputenc}
\usepackage[T1]{fontenc}
\fi
%%%%%%%%%%%%%%%%%%%%%%%%%%%%%

%\RequirePackage[german]{babel}
%\selectlanguage{german}
%\RequirePackage[isolatin]{inputenc}

%\pgfdeclareimage[height=0.5cm]{logo}{tu-logo}
%\logo{\pgfuseimage{logo}}
\beamertemplatetriangleitem
%\beamertemplateballitem

\beamerboxesdeclarecolorscheme{alert}{red}{red!15!averagebackgroundcolor}
%\begin{beamerboxesrounded}[scheme=alert,shadow=true]{}
%\end{beamerboxesrounded}

%\beamersetaveragebackground{yellow!10}

%\beamertemplatecircleminiframe

\newtheorem{question}{Question}
\newtheorem{conjecture}[question]{Principle}
\newtheorem{challenge}[question]{Challenge}
\usepackage{tikz}
\newcommand{\bra}[1]{\left< #1 \right|}
\newcommand{\ket}[1]{\left| #1 \right>}

\newcommand{\iprod}[2]{\langle #1 | #2 \rangle}
\newcommand{\mprod}[3]{\langle #1 | #2 | #3 \rangle}
\newcommand{\oprod}[2]{| #1 \rangle\langle #2 |}

\newcommand{\proj}[3]{\begin{smallmatrix} #1 & #2 & #3 \end{smallmatrix}}
\newcommand{\projbf}[3]{\begin{smallmatrix} \mathbf{#1} & \mathbf{#2} & \mathbf{#3} \end{smallmatrix}}

\sloppy
\parskip .7em %vskip between paragraphs

\newcommand{\seq}[1]{\mathbf{#1}}
\newcommand{\floor}[1]{\left\lfloor #1 \right\rfloor}
\newcommand{\ceil}[1]{\left\lceil #1 \right\rceil}
\newcommand{\m}[1]{\widetilde{#1}}
%\newcommand{\p}[1]{\scriptsize\textcolor{black}{$[#1]$}}

\usepackage[most]{tcolorbox}
\begin{document}

\title{\textcolor{blue}{\bf Is Revising Inertia The Key To Zigzag Motion, And What About ``Anti-Gravity''?}}
\subtitle{\footnotesize \url{http://tph.tuwien.ac.at/~svozil/publ/2023-Limina-pres.pdf}
\\
\footnotesize based on some chapters of my forthcoming book compressed with techniques from Generative Pre-trained Transformer~3
}
\author{\textcolor{blue}{Karl Svozil}}
\institute{\normalsize \textcolor{blue}{Institute for Theoretical Physics, TU Wien}\\
\textcolor{blue}{svozil@tuwien.ac.at}
%{\tiny Disclaimer: Die hier vertretenen Meinungen des Autors verstehen sich als Diskussionsbeitr�ge und decken sich nicht notwendigerweise mit den Positionen der Technischen Universit�t Wien oder deren Vertreter.}
}
\date{{\color{purple}Sunday, February 5th, 2022, Limina's Inaugural Symposium 2023, February 3rd, 4th, and 5th 2023, Virtual--Only
}}
\maketitle


% \frame{
% \frametitle{Contents}
% \tableofcontents
% }


\section{About myself}

\frame{
 \frametitle{About myself}


\begin{itemize}

\item[$\bullet$]
I am a theoretical physicist interested in ``fast'' (if possible superluminal) space travel; most of my
works available through my \href{http://orcid.org/0000-0001-6554-2802}{Orcid ID: 0000-0001-6554-2802}

\item[$\bullet$]
$\ldots $~some background also in Freudian-type psychoanalysis (pa 1977-82);

\item[$\bullet$]
$\ldots $~no experiencer/abductee: I newer saw a ``life UFO'' (as far as I can remember ;-)

\end{itemize}

}

\frame{
 \frametitle{Some of my unconventional papers --- Zeilinger's ``weirdest'' paper: Measuring the Dimension of Space-Time (with Anton Zeilinger/NP 2022) \href{http://doi.org/10.1103/PhysRevLett.54.2553}{DOI: 10.1103/PhysRevLett.54.2553}}

\begin{center}
\includegraphics[width=0.8\textwidth]{2023-Limina-pres-mtdost}
\end{center}
}




\frame{
 \frametitle{Dimensional Shadowing \href{http://doi.org/10.1088/0305-4470/19/18/002}{DOI: 10.1088/0305-4470/19/18/002}}

\begin{center}
\includegraphics[width=0.8\textwidth]{2023-Limina-pres-ds}
\end{center}
}

\frame{
 \frametitle{Fractal Gravity \href{https://doi.org/10.1007/s10699-019-09609-4}{DOI: 10.1007/s10699-019-09609-4}}

\begin{center}
\includegraphics[width=0.8\textwidth]{2023-Limina-pres-ds}
\end{center}
}

\frame{
 \frametitle{(Casimir-ZPF-induces) Change of Mass (aka Inertia) of Electrons In-between Conducting Plates
\href{http://doi.org/10.1103/PhysRevLett.54.742}{DOI: 10.1103/PhysRevLett.54.742}, see also with Max Kreuzer
\href{http://doi.org/10.1103/PhysRevD.34.1429}{DOI: 10.1103/PhysRevD.34.1429}}

\begin{center}
\includegraphics[width=0.8\textwidth]{2023-Limina-pres-comicp}
\end{center}
}

\frame{
 \frametitle{Interdimensionality
\href{https://doi.org/10.3390/axioms10040300}{DOI: 10.3390/axioms10040300}}

\begin{center}
\includegraphics[width=0.8\textwidth]{2023-Limina-pres-interdim}
\end{center}
}

\frame{
 \frametitle{Relativizing Relativity
\href{http://doi.org/10.1023/A:1003600519752}{DOI: 10.1023/A:1003600519752}}

\begin{center}
\includegraphics[width=0.8\textwidth]{2023-Limina-pres-relrel}
\end{center}
}

\frame{
 \frametitle{Supercavitation in the quantum {\ae}ther
\href{http://doi.org/10.48550/arXiv.physics/0210091}{DOI: 10.48550/arXiv.physics/0210091}}

\begin{center}
\includegraphics[width=0.8\textwidth]{2023-Limina-pres-supcav}
\end{center}
}

\frame{
 \frametitle{Are we living in a virtual reality?
\href{http://doi.org/10.1007/978-94-017-3327-4\_6}{DOI: 10.1007/978-94-017-3327-4\_6}}

\begin{center}
\includegraphics[width=0.5\textwidth]{2023-Limina-pres-maze}
\end{center}
}

\section{Caveat}

\frame{
 \frametitle{A Caveat}

{\huge
The following speculations are based on/relative to the assumption of alien visits with (space)craft.
}

}

\frame{
 \frametitle{China Teapot allegory, by Bertrand Russell (1958) }

{\color{blue}
``$\ldots$~nobody can prove that there is not between the Earth and Mars a china teapot revolving in an elliptical orbit,
but nobody thinks this sufficiently likely to be taken into account in practice.''}

Therefore, if we claim extraordinary things we need to come forward with extraordinary strong corroborations.
Otherwise common folks would not believe such claims.

Furthermore, the burden of proof for exceptional claims falls on the person making the claim, not the skeptic.
}

\section{Three Flying Saucer/UFO/UAP--related traps}

\frame{
 \frametitle{Three Flying Saucer/UFO/UAP--related challenges}

{\Large

\begin{itemize}

\item[$\bullet$]
The Explanation Trap

\item[$\;$]

\item[$\bullet$]
The (data) Volume Trap

\item[$\;$]

\item[$\bullet$]
The Experiencer/Abduction issue---a tale of two worlds
\item[$\;$]

\item[$\bullet$]
Nonfraternization

\end{itemize}
}
}

\section{The Explanation Trap}

\frame{
 \frametitle{The Explanation Trap: Kuhn versus Lakatos}

Kuhn believed that science goes through phases of "normal" and "revolutionary" science, with normal science being focused on solving puzzles using familiar methods and problems, and revolutionary science involving a revision of current beliefs and practices. On the other hand, Lakatos modified Kuhn's views by proposing that a research program, rather than isolated theories, should be the unit of appraisal. In his view, a program is progressive if the new theory resolves empirical anomalies, is independently testable, and expands the empirical realm. He emphasized the importance of logico-methodological terms and the existence of competing programs or paradigms.

}

\frame{
 \frametitle{The Explanation Trap: Temporal succession of research programs with no semantic convergence}

Both Kuhn and Lakatos might agree that
\begin{enumerate}
\item
for prolonged periods there exist beliefs, hard-core assumptions, and practices that constitute a dominant scientific program;
\item
any such dominant scientific program
\begin{enumerate}
\item
consists of core semantical concepts which translate into theoretical, syntactic formalizations;
\item
eventually will be overturned by another scientific program;
\item
the semantical concepts of competing or successive scientific programs are un(cor)related (whereas their theoretical, syntactic formalizations might, in some approximations, coincide).
\end{enumerate}
\end{enumerate}


Therefore, effectively, we are dealing with a temporal succession of scientific research programs
whose concepts are entirely distinct and inconsistent.
No ``semantic convergence'' can be recognized.
}

\frame{
 \frametitle{The Explanation Trap: Conceptual and theoretical overreach}

{\color{magenta}
Individuals or groups pursuing a specific research program will not be able to understand or reconstruct phenomena
and technology associated with a program more than one step ahead.
Therefore, trying to understand advanced technology from a civilization with more than one scientific revolution ahead will likely fail.}

As a corollary,
{\color{blue}consulting contemporary ``experts'', such as theoretical physicists, astronomers or rocket scientists}, on ``crashed alien craft''
{\color{magenta} may even negatively affect one's understanding of a phenomenon} due to bias and ego investment in their field of expertise.
This can result in inappropriate
and distractive thinking and a waste of opportunity costs.

}

\frame{
 \frametitle{The Explanation Trap: Two fictional examples}

Example~I: For an analogy imagine asking a shaman medicine man of Borneo to explain a World War II airplane flying over his head.

Example~II: Consider the USA after Project SIGN (and before GRUDGE and BLUE BOOK), in a  1971 recollection Oliver Harry Turner
mentions a rather ``astounding''
US investments into anti-gravity research:

{\color{blue}
``14.  A more astounding decision on the part of the U.S. Government
was to allocate considerable funds to investigate gravity and a
means of controlling gravity. Despite the fact that science had
not attained a level of competence to deal with either gravity or
anti-gravity problems and the only theory that might be applicable
was Einstein's Unified Field Theory which was still incomplete at the
time of his death, the U.S. chose to support six universities and
government agencies in an all- out drive to conquer the problem.~$\ldots$
}
}


\frame{
 \frametitle{The Explanation Trap: Turner's testimony (to the Australian government) cntd.}

{\color{blue}
``17. Such an intensive onslaught on the gravity enigma was
entirely irrational from the standpoint of conventional science,
and can only be rationalized within the context of a firm belief
that UFOs were real and that the intelligences behind them knew
how to control gravity. The drive to harness this power before
the USSR could do so would be a strong incentive for the U.S.
Government to fully support an anti-gravity program. By 1966,
46 separate projects of this nature were being financially supported,
33 of which were under the supervision of the U.S. Air Force.
Although details of most of these projects have been kept classified
it would appear that generally they have not been successful.~$\ldots$''
}
}



\begin{frame}[shrink=15]{My wild speculation: Did the USA ``process'' alien artefacts in the same manner as NAZI technology?}


\begin{enumerate}
\item
The USA  attempted to ``process'' Alien artifacts similarly to NAZI and other Earth-bound (eg Soviet) technology.

\item
In doing this ``Paperclip''-style was inapplicable because of alien non-fraternization: nobody explained it to them
like the former NAZI party member  SS Sturmbannf\"uhrer von Braun did.

\item
Eisenhower's military-industrial (MI) complex handled most of these efforts---coordinated by what was then the Air Materiel Command at Wright-Patterson Field, Ohio---at various DOE/DOD/industry (Batelle/Lockheed?) ratios.


\item
The MI complex succeeded with some ``proximity technology'' that may have been ``on the verge of discovery'' in any case.

\item
The MI complex failed with some more advanced technology---such as propulsion---which was based on concepts two or more scientific revolutions ahead of contemporary Earth-limited resources.

\item
It does so until today. Therefore, the USA might possess some craft (this is pure speculation!) but cannot mass-reproduce them.

\item
Maybe the propulsion units of some crafts were salvaged to build an ``own'' craft around them?

\end{enumerate}


\end{frame}

\frame{
 \frametitle{The Explanation Trap: Resolution and one question}

(Re)Solution: Follow Freud's advice and
adopt an analytical approach called {\color{magenta}``evenly-suspended attention.''}
This approach involves observing the phenomenon with as few as possible projections of the mind---cf.
Edwin Thompson Jaynes' {\color{blue}``Mind Projection Fallacy''}---and
without preconceived ideas or biases, and allowing the observations to settle in,
even without immediate explanations or category formations.

}


\section{Inset I: Why gain/manipulation of inertia rather than anti-gravity?}

\frame{
 \frametitle{Inset I: Why gain/manipulation of inertia rather than anti-gravity?}

Anti-gravity may not deliver the sudden changes in direction reported in UFO sightings.
Anti-gravity could counteract the pull towards the center of the Earth,
allowing for hovering or slow cruising, but not sudden changes in direction.

According to Newton's second law of motion, any propulsion, including anti-gravity,
must obey the law stating that acceleration is proportional to the force applied and inversely proportional to mass.

This means that sudden changes in motion would require either applying a large
amount of (whatever) force or reducing the craft's mass in the desired direction.
(This is true relative to the validity of Newton's second law of motion which needs not be universally valid.)

}



\frame{
 \frametitle{Is gain/manipulation of inertia feasible?}

I, therefore, suggest examining the possibilities of changing inertia or mass.

The problem: our present concept of inertia or mass is based on electroweak and strong field energy (as the mass of the ``virtual clouds'' surrounding
respective charges), or by couplings to the Higgs field.

How to modulate mass or inertia under such circumstances? I believe that quantum field theory is not good enough for that.

We are still striving for unification, regardless of what contemporary theoretical physicists might tell you.

In any case: the show goes on! Maybe somebody sometime will break their silence.

}



\section{Inset II: A fresh look at space-time operationalization}

\frame{
 \frametitle{Inset II: A fresh look at space-time operationalization: time synchronization by entanglement}

The following proposal for time synchronization
 is independent of/contradicts Poincare-Einstein Synchronization (1905, DOI 10.1002/andp.19053221004)
by, say, exchange of light rays (patented long time ago).

It involves a series of entangled particles, like, say, in the singlet Bell state
$\vert \Psi_- \rangle = (1/\sqrt{2})(\vert 01 \rangle - \vert 10 \rangle )$.
The constituent particles of respective pairs are made to spatially drift apart in an Einstein-Podolsky-Rosen (EPR)
\href{https://doi.org/10.1103/PhysRev.47.777}{DOI: 10.1103/PhysRev.47.777}
``explosion'' type setup.
They are measured at two locations by ``Alice'' and ``Bob.''
Alice and Bob set their  clocks (of identical type) according to the measurement outcomes.
This kind of time synchronization requires $n$ bits for a time synchronization precision of $n$ bits.

}

\frame{
 \frametitle{Time synchronization by entanglement -- inspirations}


1935 (shortly after publication of EPR), in a  letter to Schr\"odinger, Einstein  pointed out
\href{https://doi.org/10.1007/978-3-642-04335-2}{(Old Lyme, dated 19.6.35, Einstein Archives 22-047)};
cf also Howard \href{https://doi.org/10.1016/0039-3681(85)90001-9}{DOI: 10.1016/0039-3681(85)90001-9})
that in his view, the EPR paper {\color{blue}``was written by Podolsky after many discussions. But
still it has not come out as well as I really wanted; on the contrary, the main point
was, so to speak, buried by the erudition [[: $\ldots$]] The real state
of $B$ [[Bob's particle]] thus cannot depend upon the kind of measurement I carry out on $A$ [[Alice's particle]].''}

I suggest that Einstein was mistaken because he emphasized/prioritized relativity theory over quantum mechanics.
But, imo, it should just be the other way round: quantum mechanics might be essential to ``(re)construct'' (an emergent) space-time
by defining temporal ``proximity'' via entanglement.

}


\frame{
 \frametitle{Time synchronization by entanglement -- inspirations cntd.}

As quantum entaglement is about relational proximity, we need a new concept of what is ``near'' or ``far'',
both in space and time.

Please have a look at: (i) Wheeler's ``delayed choice'' measurement when applied to the construction of
non/entanglement, as envisioned by Peres' \href{http://dx.doi.org/10.1080/09500340008244032}
{Delayed choice for entanglement swapping DOI: 10.1080/09500340008244032}
(ii) Zeilinger's recent Nobel Prize lecture 2022.

The challenge: perfect correlations, including relational encoding, can be simulated classically---cf eg, Peres'
\href{https://doi.org/10.1119/1.11393}{``Unperformed Experiments have no results'' DOI: 10.1119/1.11393}
--
BUT: maximally entangled particles carry only RELATIONAL info about joint properties;
they have NO INDIVIDUAL (!) properties (unlike classical examples).


}


\frame{
 \frametitle{Time synchronization by entanglement -- inspirations cntd.}

[[The situation with nonentangled states versus entangled ones is a bit like the age-old tradeoff between individuality versus the collective; or the existence of poor states with rich citizens versus
rich states with poor citizens.]]

This indicates is a zero-sum game-- in quantum mechanics due to unitarity of the state evolution which is esentially one-to-one.


So, if Alice \& Bob observe individual RANDOM outcomes at both spatially (say, under strict Einstein locality contitions) separated ends of
the EPR configuration,  which, nevertheless, at the same time
obey  (even perfect if desired) RELATIONAL  correlations of pairs of such outcomes -- how does this relationality come about?

My hypothesis: relationality at both ends of the EPR ports comes about because those ends, for the encoded particle pair, are
NOT SEPARATED AT ALL!
That indicates imo
the need to turn fixed Kantian space-time concepts ``upside down'' in favour of patchworks of entanglement-supported/mediated space-time frames.
}

\begin{frame}[shrink=1.05]{The role of conventionality in the construction of space-time frames}

The International System of Units (SI) nowadays DEFINES (rather than measures) the velocity of light $c$ to be constant in all inertial frames;
cf \href{https://doi.org/10.1038/303373a0}{DOI: 10.1038/303373a0}.
This is because SI needed a better length standard:
the ur-meter in Paris tended to change its size (shrink?) over time.

So constancy of light in vacuum $c$ in all inertial frames is now true per definition/convention, according to SI.
This stirred a small exchange in Nature -- cf. \href{https://doi.org/10.1038/312010b0}{DOI: 10.1038/312010b0}
versus  \href{https://doi.org/10.1038/312490a0}{DOI: 10.1038/312490a0} ;-)

Alexandrov's theorem of incidence geometry: constancy of light in vacuum $c$ in all inertial frames (and bijectivity of space-time transformations)
essentially renders/implies Lorentz transformations.

So, the kinematic part of relativity is purely conventional nowadays. Cf also Bell,
\href{https://doi.org/10.1017/CBO9780511815676.011}{``How to teach relativity'' DOI: 10.1017/CBO9780511815676.011}.


\end{frame}

\section{The (data) Volume Trap}

\frame{
 \frametitle{The (data) Volume Trap}

I agree with the evaluation of the Condon report by the
Technical Committee on Atmospheric Environment and the Technical Committee on Space and Atmospheric Physics
of the American Institute of Aeronautics and Astronautics (AIAA) [[my punctuation]]:

{\color{blue}
``Taking all evidence which has come to the Subcommittee's attention into account,
we find it difficult to ignore the small residue of
well-documented but unexplainable cases which form the hard core of the UFO controverse.''

$\ldots$

``In fact, the Subcommittee finds that the opposite conclusion could have been drawn from its content, namely,
that a phenomenon with such a high ratio of unexplained cases (about 30
{\%}
[[in the Condon Report]])
should arouse sufficient scientific curiosity to continue its study.''
}
}



\frame{
 \frametitle{The (data) Volume Trap --- AIAA on the Condon Report}

``{\color{blue} The issue seems to boil down to the question:    }
\begin{itemize}
\item[$\bullet$]
{\color{magenta} Are we justified to extrapolate from 0.99 to 1.00, implying that if
99{\%}
of all observations can be explained, the remaining
1{\%}
could also be explained}; or
\item[$\bullet$]
{\color{magenta} do we face a severe problem of signal-to-noise ratio
(order of magnitude 0.01)?}''
\end{itemize}

}


\frame{
 \frametitle{The (data) Volume Trap: drowning in data with 1/100 signal-to-noise ratio}

\begin{itemize}
\item[$\bullet$]
Why do we need a database? We need single ``core cases'', not Big Data---quality over quantity!
But, for maybe good reasons, those data may be stowed away in carve-out  USAPs.
\item[$\bullet$]
Monetizing data is bad scientific practice. Data need to be Open Access!
\item[$\bullet$]
Databases very likely will be filled with a lot of ``very bad''---ie prosaic, mundane--sightings;
\item[$\bullet$]
if the signal-to-noise ratio of ``core cases'' versus prosaic, mundane cases is under 1/100:
will even AI be able to sort out the wheat from the chaff?
I think not.
\item[$\bullet$]
Efforts drowning in data may very well be a good strategy for ``kicking the can down the road once more.''
\end{itemize}



}


%%%%%%%%%%%%%%%%%%%%%%%%%%%%%%%%%%%%%%%%%%%%%%%%%%%%%%%%%%%%%%%%%%%%%%%%%%%%%%%%%%%%%%%%%%%

\section{The Experiencer/Abduction issue---a tale of two worlds}

\frame{
 \frametitle{The Experiencer/Abduction issue---a tale of two worlds}

If you had an ``experience'' or have been ``abducted'' there is no doubt about the existence of the ``phenomenon'' for
you---but what about the
other common folks?



Many experiencers ---those on the inside---don't understand that persons
who  have not shared similar experiences---those on the outside---are skeptical.


}

\frame{
 \frametitle{The Experiencer/Abduction issue---Similarity to Freud's seduction hypothesis}

In 1896, Sigmund Freud presented his theory that acts of sexual abuse and violence on children caused adult mental illness,
known as the ``seduction theory,'' in his article ``The Aetiology of Hysteria''.
This was not well received, as it implicated clients and members of the Viennese society,
leading Freud to later reject the theory and blame patients for deceptive memories.

In his controversial  book ``The Assault on Truth'', Jeffrey Masson criticizes Freud for rejecting the theory and
argues that it was due to pressure from colleagues and unconscious conflicts.

The similarity with supposed alien abductions is that both raise questions
of what occurred, whether abductees experienced them or had false memories and delusions.

Maybe hypnosis is not a proper method to cope with abduction cases, in the same way as Freud abandoned hypnosis in favor of ``free associations'' and other techniques.




}

\section{Nonfraternization issue---what if the Others (if they exist) don't want to communicate? }

\begin{frame}[shrink=1.05]{Nonfraternization issue---what if the Others (if they exist) don't want to communicate? }


\begin{itemize}

\item[$\bullet$]
Suppose there are ``alien visitations' by Others. Even if some of us in power knew---they
might not have the autonomy to publicly acknowledge the existence of Others.

\item[$\bullet$]
For Others to pursue their goal it might not be favorable to fraternize.
Why should they?
Their goals or motives might not align with our interests.

\item[$\bullet$]
There might be consequences for disclosure projects.
Suppose the Others had telepathic abilities---maybe they could be ``in our head'' (cf. Terry Lovelace)
or could steer us?


\item[$\bullet$]
Maybe our governments try to cope with ``them'' while pretending ``normalcy'' for the rest of us. These are hypothetical,
highly speculative, paranoic scenarios. But what if~$\ldots$?

\end{itemize}

\end{frame}



\frame{
 \frametitle{A Caveat}

{\huge
All speculations made here were based on/relative to the assumption of alien visits with (space)craft.
}

}




\frame{

\centerline{\Large {\color{magenta} Thank you for your attention!}}

\begin{center}\color{orange}
$\widetilde{\qquad \qquad }$
$\widetilde{\qquad \qquad}$
$\widetilde{\qquad \qquad }$
\end{center}
 }
 \end{document}


















\section{ }



\frame{
 \frametitle{ }

\begin{itemize}
\item[$\bullet$] {
%\color{purple}
}
\pause
\item[$\bullet$] {
%\color{purple}
}
\end{itemize}
}

\section{ }

\frame{
 \frametitle{ }

\begin{itemize}
\item[$\bullet$] {
%\color{purple}
}
\pause
\item[$\bullet$] {
%\color{purple}
}
\end{itemize}
}

\section{ }

\frame{
 \frametitle{ }

\begin{itemize}
\item[$\bullet$] {
%\color{purple}
}
\pause
\item[$\bullet$] {
%\color{purple}
}
\end{itemize}
}


