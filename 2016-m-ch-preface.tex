\chapter*{Introduction}
\addcontentsline{toc}{chapter}{Unreasonable effectiveness of mathematics in the natural sciences}
\markboth{Introduction}{Unreasonable effectiveness of mathematics in the natural sciences}






\newthought{This is an ongoing attempt}
\marginnote{{\em ``It is not enough to have no concept,
one must also be capable of expressing it.''}
From the German original in {\em Karl Kraus, {\em Die Fackel} {\bf 697}, 60 (1925)}:
{\em ``Es gen\"ugt nicht, keinen Gedanken zu haben: man muss ihn auch ausdr\"ucken k\"onnen.''
}}
to provide some written material of a course in mathemathical methods of theoretical physics.
I have  presented this course to an undergraduate audience at the Vienna University of Technology.
Only God knows (see Ref. \cite{Aquinas} part one, question 14, article 13; and  also Ref. \cite{specker-60}, p. 243)
if I have succeeded to teach them the subject!
I kindly ask the perplexed to please be patient, do not panic under any circumstances,
and do not allow themselves to be too  upset with mistakes, omissions \& other problems of this text.
At the end of the day, everything will be fine, and in the long run we will  be dead anyway.

\newthought{The problem}
with all such presentations is to learn the formalism in sufficient depth, yet not to get buried by the formalism.
As every individual has his or her own mode of comprehension there is no canonical answer to this challenge.

\newthought{I am releasing this} text to the public domain because it is my conviction and experience that content can no longer be held back,
 and access to it be restricted, as its creators see fit.
On the contrary, in the {\em attention economy} -- subject to the scarcity as well as the compound accumulation of attention --
we experience a push toward so much content that we can hardly bear this information flood, so we have to be selective
and restrictive rather than aquisitive.
I hope that there are some readers out there who actually enjoy and profit from the text, in whatever form and way they find appropriate.

\newthought{Such university texts
as this one} -- and even recorded video transcripts of lectures -- present a transitory,
almost outdated form of teaching.
Future generations of students will most likely enjoy
{\em massive open online courses} (MOOCs) that might integrate interactive elements
and will allow a more individualized -- and at the same time automated -- form of learning.
What is most important from the viewpoint of university administrations
is that (i) MOOCs are cost-effective (that is, cheaper than standard tuition)
and  (ii) the know-how of university teachers and researchers gets transferred to the
university administration and management; thereby the dependency of the university management
on teaching staff is considerably alleviated.
In the latter way, MOOCs are the implementation of
assembly line methods (first introduced by Henry Ford for the production of affordable cars)
in the university setting.
Together with ``scientometric'' methods which have their origin in both Bolshevism as well as in Taylorism \cite{taylor-1911},
automated teaching
is transforming schools and universites, and in particular, the old Central European universities,
as much as the {\em Ford Motor Company} (NYSE:F)  has transformed
the car industry, and the Soviets have transformed Czarist Russia.
To this end, for better or worse, university teachers become accountants \cite{svozil-2011-sklaverei},
and {\it ``science becomes bureaucratized; indeed, a higher police function.
The retrieval is taught to the professors. \cite{juenger-Heliopolis}''}
\marginnote{German original
{\it ``die Wissenschaft
wird b�rokratisiert, ja Funktion der h�heren Polizei. Den Professoren
wird das Apportieren beigebracht.''}
}


\newthought{To newcomers} in the area of theoretical physics (and beyond)
I strongly recommend to consider and acquire two related proficiencies:
\marginnote{If you excuse a maybe utterly displaced comparison, this might  be tantamount only to
studying the Austrian family code (``Ehegesetz'')
from \S 49
onward, available through {\tt http://www.ris.bka.gv.at/Bundesrecht/}
before getting married.}
\begin{itemize}
\item
to learn to speak and publish in \LaTeX\ and BibTeX;
in particular, in the implementation of TeX Live.
%\marginnote{https://www.tug.org/texlive/}
\index{LaTeX}
\index{BibTeX}
\LaTeX's various dialects and formats,
such as {REVTeX}, provide a kind of template for structured scientific texts,
thereby assisting you writing and publishing consistently and with methodologic rigour;
\item
to subsribe to and browse through preprints published at the website {\tt arXiv.org},
which provides open access to more than three quarters of a million scientific texts;
most of them written in and compiled by \LaTeX.
Over time, this database has emerged as a {\it de facto} standard
from the initiative of an individual researcher working at the
{\em Los Alamos National Laboratory}
(the site at which also the first nuclear bomb has been developed and assembled).
Presently it happens to be administered by {\em Cornell University.}
I suspect (this is a personal subjective opinion)
that (the successors of) {\tt arXiv.org}
will eventually bypass if not supersede most scientific journals of today.
\end{itemize}
So this very text is written in \LaTeX\
and accessible freely {\it via} {\tt arXiv.org} under eprint number {\em arXiv:1203.4558}.\marginnote{\url{http://arxiv.org/abs/1203.4558}}

\newthought{My own encounter} with many researchers of different fields and different degrees of formalization
has convinced me that there is no single, unique ``optimal'' way of formally comprehending a subject \cite{anderson:73}.
With regards to formal rigour, there appears to be a rather questionable chain of contempt --
all too often
theoretical physicists look upon the experimentalists suspiciously,
mathematical physicists look upon the theoreticians skeptically,
and
mathematicians look upon the mathematical physicists dubiously.
I have even experienced the distrust formal logicians expressed about their collegues in mathematics!
For an anectodal evidence, take the claim of a prominent member of the mathematical physics community,
who once dryly remarked in front of a fully packed audience,
``what other people call `proof' I call `conjecture'!''

\newthought{So please be aware} that not all I present here will be acceptable to everybody; for various reasons.
Some people will claim that I am too confusing and utterly formalistic, others will claim my arguments are in desparate need of rigour.
Many formally fascinated readers will demand to go deeper into the meaning of the subjects;
others may want some easy-to-identify pragmatic, syntactic rules of deriving results.
I apologise to both groups from the outset.
This is the best I can do; from certain different perspectives, others, maybe even some tutors or students, might perform much better.


\newthought{I am calling} for more tolerance and a greater unity in physics;
as well as for a greater esteem on ``both sides of the same effort;''
I am also opting for more pragmatism;
one that acknowledges the mutual benefits and oneness of
theoretical and empirical physical world perceptions.
Schr�dinger \cite{schroed:natgr}
cites  Democritus with arguing against a too great separation of the  intellect ($\delta \iota {\alpha}\nu o \iota \alpha$, dianoia) and the senses
($\alpha \iota \sigma \theta {\eta} \sigma \epsilon \iota \varsigma$, aitheseis).
In fragment D 125 from Galen \cite{Diels-fdv}, p. 408, footnote 125 , the intellect claims
``ostensibly there is color, ostensibly sweetness, ostensibly bitterness, actually only atoms and the void;''
to which the senses retort:
``Poor intellect, do you hope to defeat us while from us you borrow your evidence? Your victory is your defeat.''
\marginnote{German: Nachdem D. [[Demokritos]] sein Mi\ss trauen gegen die Sinneswahrnehmungen in
dem Satze ausgesprochen: `Scheinbar (d. i. konventionell) ist Farbe,
scheinbar S\"u\ss igkeit, scheinbar Bitterkeit: wirklich nur Atome und
Leeres'' l\"a\ss t er die Sinne gegen den Verstand reden: `Du armer Verstand, von uns nimmst du deine Beweisst\"ucke und willst uns damit
besiegen? Dein Sieg ist dein Fall!'}

In his 1987 {\it Abschiedsvorlesung} professor Ernst Specker
at the {\it Eidgen\"ossische Hochschule Z\"urich}
remarked that
the many books authored by David Hilbert carry his name first,
and the name(s) of his co-author(s) second,
although the subsequent author(s) had actually written these books;
the only exception of this rule being Courant and Hilbert's 1924 book
{\em Methoden der mathematischen Physik},
comprising around 1000 densly packed pages,
which allegedly none of these authors had actually written.
It appears to be some sort of collective effort of scholars from the University of G\"ottingen.

So, in sharp distinction from these activities,
I most humbly present my own version of what is important for standard courses of contemporary physics.
Thereby, I am quite aware that, not dissimilar with some attempts of that sort undertaken so far, I might fail miserably.
Because even if I manage to induce some interest, affaction, passion and understanding in the audience -- as Danny Greenberger put it,
inevitably
four hundred years from now, all our present physical theories of today will appear transient \cite{lakatosch}, if not laughable.
And thus in the long run, my efforts will be forgotten (although, I do hope, not totally futile); and some other brave, courageous guy
will continue attempting to (re)present the most important mathematical methods in theoretical physics.



\newthought{All things considered,} it is mind-boggling why formalized thinking and numbers utilize our comprehension
of nature.
Even today eminent researchers muse about the {\em
``unreasonable effectiveness of mathematics in the natural sciences''} \cite{wigner}.



Zeno of Elea and Parmenides, for instance, wondered how there can be motion if
our universe is either infinitely divisible or discrete.
Because, in the dense case (between any two points there is another point),
the slightest finite move would require an infinity of actions.
Likewise in the discrete case,
how can there be motion if everything is not moving at all times \cite{zeno,benna:62,gruenbaum:68,Sainsbury}?


The Pythagoreans are often cited to have believed that the universe is natural numbers or simple fractions thereof, and thus physics is just a part of mathematics; or that
there is no difference between these realms.
They took their conception of numbers and world-as-numbers so seriously that the existence of irrational numbers which cannot
be written as some ratio of integers shocked them; so much so that they allegedly drowned the poor guy who had discovered this fact.
That appears to be a saddening case of a state of mind in which a subjective metaphysical belief in and
wishful thinking about one's own constructions of the world overwhelms critical thinking;
and what should be wisely taken as an epistemic finding is taken to be ontologic truth.


The relationship between physics and formalism has been debated by
Bridgman \cite{bridgman},
Feynman \cite{feynman-computation},
and  Landauer \cite{landauer},
among many others.
It has many twists, anecdotes and opinions.
Take, for instance, Heaviside's not uncontroversial stance \cite{heaviside-EMT} on it:
\begin{quote}
{\em
I suppose all workers
in mathematical physics have noticed how the mathematics
seems made for the physics, the latter suggesting the former, and
that practical ways of working arise naturally. $\ldots$ But then the
rigorous logic of the matter is not plain! Well, what of that?
Shall I refuse my dinner because I do not fully understand the
process of digestion? No, not if I am satisfied with the result.
Now a physicist may in like manner employ unrigorous processes with satisfaction and usefulness if he, by the application
of tests, satisfies himself of the accuracy of his results. At
the same time he may be fully aware of his want of infallibility,
and that his investigations are largely of an experimental character, and may be repellent to unsympathetically
constituted mathematicians accustomed to a different kind
of work.  [\S 225]
}
\label{2013-m-ch-intro-cooking}
\end{quote}


Dietrich K\"uchemann,
the ingenious German-British aerodynamicist and
one of the main contributors to the wing design of the {\em Concord} supersonic civil aercraft, tells us
\cite{Kuchemann}
\begin{quote}
{\em
[Again,] the most drastic simplifying assumptions must be made before we can even think about
the flow of gases and arrive at equations which are amenable to treatment. Our whole
science lives on highly-idealised concepts and ingenious abstractions and approximations.
We should remember this in all modesty at all times, especially when somebody claims to
have obtained ``the right answer'' or ``the exact solution''.
At the same time, we must acknowledge and admire the intuitive art of those scientists
to whom we owe the many useful concepts and approximations with which we work [page 23].
}
\end{quote}



The question, for instance, is imminent whether we should take the formalism very seriously and literally,
using it as a guide to new territories, which might even appear absurd, inconsistent and mind-boggling;
just like Carroll's {\em Alice's Adventures in Wonderland.}
Should we expect that all the wild things formally imaginable have a physical realization?


It might be prudent to
adopt a contemplative strategy of {\em evenly-suspended attention}
outlined by  Freud \cite{Freud-1912}, who admonishes analysts to be aware of the dangers
caused by {\em ``temptations to project,
what  [the analyst]  in dull self-perception recognizes as the peculiarities of his own personality,
as generally valid theory into science.''}
Nature is thereby treated as a  client-patient,  and whatever findings come up are accepted  as is  without any
immediate emphasis or judgment.
This also alleviates the dangers of becoming embittered with the reactions of ``the peers,''
a problem sometimes encountered when ``surfing on the edge'' of contemporary knowledge; such as, for
example, Everett's case \cite{everett-collw}.

Jaynes has warned of the {\em ``Mind Projection Fallacy''} \cite{jaynes-89,jaynes-90}, pointing out that
{\em ``we are all under an ego-driven temptation to project our private
thoughts out onto the real world, by supposing that the creations of one's own imagination are real
properties of Nature, or that one's own ignorance signifies some kind of indecision on the part of
Nature.''}

And yet, despite all aforementioned {\it provisos}, science finally succeeded to do what the alchemists sought for so long:
we are capable of producing gold from mercury \cite{PhysRev.60.473}.

\begin{center}
{\color{lightgray}   \Huge
\aldine
 %\decofourright \decofourleft
%\aldine X \decoone c \floweroneright
% \aldineleft ] \decosix g \leafleft
% \aldineright Y \decothreeleft f \leafNE
% \aldinesmall Z \decothreeright h \leafright
% \decofourleft a \decotwo d \starredbullet
% \decofourright b \floweroneleft
}
\end{center}
