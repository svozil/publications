\documentclass[rmp,amsfonts,showpacs,showkeys,preprint]{revtex4}
\usepackage{cdmtcs-pdf}
%\documentclass[pra,showpacs,showkeys,amsfonts]{revtex4}
\usepackage{graphicx}% Include figure files
\usepackage{dcolumn}% Align table columns on decimal point
\usepackage{bm}% bold math

\RequirePackage{times}
%\RequirePackage{courier}
\RequirePackage{mathptm}


\usepackage[usenames]{color}
\newcommand{\Red}{\color{Red}}  %(VERY-Approx.PANTONE-RED)
\newcommand{\Green}{\color{Green}}  %(VERY-Approx.PANTONE-GREEN)

\begin{document}
%\sloppy

\title{Physical unknowables\footnote{Contribution to
the international symposium ``Horizons of Truth''
celebrating the 100th birthday of Kurt G\"odel
at  the University of Vienna from April 27th-29th, 2006}}

\cdmtcsauthor{Karl Svozil}
\cdmtcsaffiliation{University of Technology, Vienna }
\cdmtcstrnumber{299}
\cdmtcsdate{January 2007}
\coverpage

\author{Karl Svozil}
\email{svozil@tuwien.ac.at}
\homepage{http://tph.tuwien.ac.at/~svozil}
\affiliation{Institut f\"ur Theoretische Physik, University of Technology Vienna,
Wiedner Hauptstra\ss e 8-10/136, A-1040 Vienna, Austria}


\begin{abstract}
Different types of physical unknowables are discussed.
Provable unknowables are derived from reduction to problems which are known to be recursively unsolvable.
Recent series solutions to the $n$-body problem and related to it, chaotic systems, may have no computable radius of convergence.
Quantum unknowables include the random occurrence of single events, complementarity and value indefiniteness.
\end{abstract}


\pacs{01.70.+w,01.65.+g,02.30.Lt,03.65.Ta}
\keywords{Unknowables, Church-Turing thesis, induction and forecast, n-body problem, quantum indeterminism}

\maketitle

\tableofcontents

\begin{quote}
\begin{flushright}
{\footnotesize
$\ldots$ as we know, there are known knowns; \\
there are things we know we know. \\
We also know there are known unknowns; \\
that is to say we know there are some things we do not know. \\
But there are also unknown unknowns --\\
the ones we don't know we don't know.  \\
{\em United States Secretary of Defense Donald H. Rumsfeld \\
at a Department of Defense news briefing on February 12, 2002}
% http://www.defenselink.mil/Transcripts/Transcript.aspx?TranscriptID=2636
 }
\end{flushright}
\end{quote}

\section{Islands of preliminary insights in an ocean of ignorance}

Throughout history, the demand to form the physical world according to people's wishes
has been counterbalanced with the inability to predict and manipulate
large portions of the human habitat.
As time passed, humankind was able to figure out ways to tune
ever increasing fragments of the world according to its needs.
From a purely behavioral perspective, this is brought about in the way of
opportunistic quasi-causal rules of the following kind,
``if we do this, we obtain that.''
A typical example of such a rule is, ``if we move a particular kind of  electric on/off switch,
the lights go on, and the room turns from dark to bright.''


How do we arrive at those kinds of rules?
Guided by our suspicions, thoughts, formalisms and by pure chance,
we ``fiddle'' and ``roam around,'' inspecting portions of our world
and examining their behavior.
We observe repeating patterns of behavior and pin them down by reproducing them.
A physical behavior is anything that can be observed and thus operationally obtained and measured;
e.g.,
the rise and fall of the sun, the ignition of fire, the formation and the melting of ice.
Note that, due to the finiteness of the resolution, all kinds of physical behaviors,
even the ones that appear continuous, can be discretized.
Ultimately, all physical experiences can be broken down into yes-no propositions
representable by zeroes and ones, by sequences of single clicks in detectors.


As we observe physical behaviors,
we attempt to ``understand'' them by trying to figure out
the ``cause'' \cite{frank} or ``reason'' for the behavioral  patterns.
That is, we invent virtual parallel worlds of thoughts
and intellectual concepts such as ``electric field'' or ``mechanical force''
to ``explain'' the behavioral patterns.
We call these creations of our minds ``physical theories.''
Contemporary physical theories are heavily formalized,
utilizing almost every branch of mathematics and formal logic
which could have been imagined so far.
A ``good theory'' provides us with the feeling of a key properly fitting into the lock of a treasure chest,
a key unlocking new ways of world comprehension and manipulation.

The methods we employ are pretty reliable.
Reliability yields a feeling of security and consolation.
It strengthens the belief in the applicability and the overall validity of the method.

Ideally, an ``explanation'' should be as compact as possible
and should apply to as many behavioral patterns as possible.
We also have the feeling that, as we are able to manipulate more and more fragments
of our habitat, we are converging to some final truth.
Ultimately, we seek theories of everything \cite{barrow-TOE}
predicting and manipulating the phenomena at large.

In the extreme form, we envision ourselves as becoming empowered with omniscience and omni-influence:
we presume that our ability to manipulate and tune the world is limited by our own phantasies alone,
and any constraints whatsoever can be bypassed or overcome one way or another.
Indeed, some of what in the past has been called magic, mystery and the beyond has been realized in everyday life.
Many wonders of witchcraft have been transferred into the realm of the physical sciences.
Take, for example, our abilities to fly,
the capability to transmute lead into gold,
to listen and speak to far away friends,
or to cure bacterial diseases with a few pills of antibiotics.

Thereby, we not only trust the rules merely syntactically in the operational sense,
but we take for granted the semantic significance of the physical theories that ``let us understand'' the
behavioral patterns and even lead us to novel predictions of behaviors.
Pointedly stated, we not only accept physical theories as pure abstractions and constructions of our own mind,
but we associate meaning and truth to them.
We grant absolute status to our own constructions of mind,
purporting that they somehow are metaphysically real and eternal;
so much so that only very reluctantly we admit their preliminary character.

Alas, the possibility to formulate theories per se;
and in particular the applicability of formal, mathematical models, comes as a surprise.
There appears to be an unreasonable effectiveness of mathematics in the natural sciences \cite{wigner}
which seems difficult to explain within science proper.
It is not too unreasonable to speculate that any such reasoning might be metaphysical.

Sometimes we have the strength to face suspicions that,
to put it in analogy to Shakespeare's poetry,
our own constructions and the baseless fabric of our vision,
just like the great globe itself, shall dissolve
and leave not a rack behind.
Our physical theories are such stuff as dreams are made on, and our little islands of transient insights
are rounded with an ocean of ignorance.


\section{Provable physical unknowables}

In the past century, unknowability has been formally derived along the notion of unprovability,
accompanied by a precise meaning of provability \cite{rogers1,odi:89,odi:99}.
In formal logic \cite{godel1} and the foundations of mathematics \cite{tarski:32,tarski:56}
as well as theoretical computer sciences \cite{turing-36,chaitin3},
unprovability has been established as a concept proper.
Those theoretical frameworks proved strong enough to derive some of their own limitations;
among them their incompleteness and overall consistency.

This is a remarkable departure from informal suspicions and observations regarding the limitations
of our worldview.
No longer is one reduced to informal, heuristic contemplations and comparisons about what one knows or can do versus
one's unpredictability and incapability.
Formal unknowability is about formal proofs of unpredictability and impossibility.

Almost since its discovery, attempts \cite{popper-50i,popper-50ii} have been made to translate
formal incompleteness into the physical science,
mostly by reduction to the halting problem \cite{moore,casti:94-onlimits_book,casti:96-onlimits}.
Here reduction means that physical undecidability is linked or reduced to logical undecidability.
A typical example for such a reduction is the embedding of a Turing machine or any type of computer capable of
universal computation into a physical system.
As a consequence, the physical system inherits
any type of unsolvability derivable for universal computers, such as the
unsolvability of the halting problem:
since the computer is part of the physical system, so are its behavioral patterns.


A clear distinction should be made from the onset regarding two different types of unknowables in
the natural sciences: unknowables about physical systems and their phenomena and behaviors on one hand,
and unknowables of the formal theoretical descriptions and models on the other hand.
This section will mostly concern the first type of physical unknowability;
the one which is associated with deterministic physical systems.





\subsection{Intrinsic self-referential observers}



Every physical observation is essentially (i) discrete, (ii) finite and (iii) self-referential.
Whereas finiteness and discreteness has been briefly mentioned earlier,
self-referentiality is a seldom recognized system science aspect of physical world perception.

Let us start with the assumption that there exist observers measuring objects, and that
observer and object are distinct from one another.
That is, there exists a ``cut'' between observer and object.
Through that cut, information is exchanged.

If we insist on idealized one-way observation,
information is transferred from the object to the observer via the cut.
%, as sketched in Fig.~\ref{2007-miracles-fc}a).
In this scenario, the object is a transmitter,
and the observer is the receiver.

Symbolically, we may regard the object as an agent contained in a ``black box,''
whose only relevant emanations are representable by finite strings of zeroes and ones
appearing on the cut, which can be modeled by any kind of screen or display.
In this purely syntactic point of view,
a physical theory should be able to render identical symbols like the ones appearing through the cut.
That is, a physical theory should be able to mimic or emulate the black box it purports to apply to.
This view is often adapted in quantum mechanics,
where it is difficult to find any meaning \cite{feynman-law} for the theory.


A sharp distinction between a physical object and an extrinsic,
outside observer is a rarely affordable abstraction.
Examples are astronomy, blackbody radiation and classical physical configurations
allowing an infinitely small (relative to the entire system) subsystem to convey the information transfer.

We are mostly interested in another scenario, in which the observer is part of the system to be observed.
There,
%as drawn in Fig.~\ref{2007-miracles-fc}b),
the measurement process is modeled symmetrically,
and information is exchanged between observer and object bidirectionally.

The symmetrical configuration makes a distinction between observer and object purely conventional.
The cut is constituted by the information exchanged.
We tend to associate with the ``measurement apparatus''
one of the two subsystems which in comparison is ``larger'' and ``more classical''
and up-linked with some conscious observer.
The rest of the system we call the ``measured object.''






Intrinsic observers face all kinds of self-referential situations.
Among the most interesting are paradoxical self-referential statements.
These have been known
both informally as puzzling amusement and artistic perplexion,
as well as a formalized scientifically valuable resource.
There is an English phrase stating that one should not bite the hand that feeds oneself.
In German, the saying amounts to the advice not to cut the very tree branch one is sitting on.
The liar paradox is already mentioned in the Bible's Epistle to Titus, 1:12 stating that,
``one of Crete's own prophets has said it: `Cretans are always liars, evil brutes, lazy gluttons.'
He has surely told the truth.''

In what follows, paradoxical self-referentiality will be used to argue
against the solvability of the general induction problem,
as well as for a pandemonium of undecidabilities related to physical systems
and their behaviors. All of them are based on intrinsic observers embedded
in the system they observe.

It is not totally unreasonable to speculate that the
limits of ``intrinsic self-expression'' seems to be
what G\"odel himself
considered the gist of his incompleteness theorems.
In a reply to a letter by Burks
(reprinted in Ref.~\cite[p. 55]{v-neumann-66}; see also Ref.~\cite[p.554]{fef-84}),
G\"odel states,
 \begin{quote}
 {\em
 ``$\ldots$ that a complete epistemological description
 of a language $A$ cannot be given in the same language $A$, because
 the concept of truth of sentences of $A$ cannot be defined in $A$. It
 is this theorem which is the true reason for the existence of
 undecidable propositions in the formal systems containing arithmetic.''}
 \end{quote}







\subsection{What is an acceptable form of proof?}

There exist ancient and informal notions of proof.
An example \cite{baats1} is the Babylonian notion to ``prove'' arithmetical statements
by considering ``large number'' cases
of algebraic formulae such as \cite[Chapter V]{neugeb},
$\sum_i^n i^2 = (1/3)(1+2n)\sum_i^n i$.
Another example  is knowledge acquired by revelation or by authority.
Oracles occur in modern computer science,
but only as idealized concepts whose physical realization is highly
questionable if not forbidden.

The contemporary notion of proof is formalized and algorithmic.
Around 1930 mathematicians could still hope for a
``mathematical theory of everything''
which consists of a finite number of axioms and algorithmic derivation rules
by which all true mathematical statements could formally be derived.
In particular, as expressed in Hilbert's 2nd problem,
it should be possible to prove the consistency of the axioms of arithmetic.

Shortly afterwards, G\"odel \cite{godel1}, Tarski \cite{tarski:32}, and Turing
\cite{turing-36} put an end to this formalist program.
They first formalized the concepts of proof and computation in general,
equating them with algorithmic content.
Then, they translated self-referential statements of the
kind mentioned above into the formalism.


From a purely syntactic point of view,
every formal system of mathematics  can be identified with a computation
and {\it vice versa}.
Indeed, as stated by K. G\"odel in a {\sl Postscript,} dated from  June 3rd, 1964 \cite[pp. 369-370]{godel-ges1},
 \begin{quote}
 {\em
 $\ldots $ due to A. M. Turing's work,
 a precise and unquestionably
 adequate definition of the general concept of formal system can now be
 given, the existence of undecidable arithmetical propositions and the
 non-demonstrability of the consistency of a system in the same system
 can now be proved rigorously for {\em every} consistent formal system
 containing a certain amount of finitary number theory.

 Turing's work gives an analysis of the
 concept of ``mechanical
 procedure'' (alias ``algorithm'' or ``computation procedure'' or
 ``finite combinatorial procedure''). This concept is shown to be
 equivalent with that of a ``Turing machine.'' A formal system can
 simply be defined to be any mechanical procedure for producing
 formulas, called provable formulas.}
 \end{quote}

What is an algorithm? In Turing's own words \cite{turing-36},
 \begin{quote}
{\em
``whatever  can (in principle) be calculated on a
 sheet of paper by  the usual rules is  computable.''}
\end{quote}

These concretions limit the expressiveness of any formalism,
for either it is too restricted to allow the representation of rich patterns of behavior,
or it is bounded by self-referentiality.
They, however, do not exclude revelations
and knowledge of truth transcending the algorithmic formalism.



\subsection{Undecidability of the general forecasting problem}


Logical and undecidabilities are based on intrinsic paradoxical self-reference.
Can we make use of paradoxical self-reference in physics?
Is it possible to find physical expressions corresponding to,
for instance, the liar paradox?
Can we apply the ``G\"odelian program'' to physics?

Indeed, we can argue that for any deterministic system strong enough to support
universal computation,  the general forecast or prediction
problem is provable unsolvable.
This will be shown by reduction to the halting problem.

G\"odel had doubts about the relevance of formal incompleteness to physics,
in particular to quantum mechanics.
The author was told by professor Wheeler that this resentment
(also mentioned in Ref.~\cite[pp. 140-141]{bernstein})
may have been due to Einstein's negative opinion of quantum theory;
to the extend that Einstein may have ``brainwashed'' G\"odel
into believing that all efforts in this direction were in vain.

One of the first researchers getting interested in the application
of paradoxical self-reference to physics
was the philosopher Popper,
who published two almost forgotten papers
\cite{popper-50i,popper-50ii}
discussing, among other issues, Russell's Paradox of
Tristram Shandy \cite{sterne}:
In Volume 1, Chapter XIV, Shandy finds that he could publish
two volumes of his life every year,
covering a time span far smaller than the time it took him to write
these volumes. This de-synchronization, Shandy concedes,
will rather increase than diminish as he advances; and one may thus have serious doubts
whether he will ever complete his autobiography.

More recently, there have been attempts to bring together researchers
interested in the relevance of G\"odelian incompleteness in physics.
One of those meetings took place in Santa Fe
\cite{casti:94-onlimits_book}, another one in Abisko
\cite{casti:96-onlimits}.

A straightforward embedding of a universal computer
into a physical system results in the fact that,
due to the reduction to the recursive undecidability of the halting problem,
certain future events cannot be forecasted
and are thus provable indeterministic.
Here reduction again means that physical undecidability is linked or reduced
to logical undecidability.

For the sake of getting an (algorithmic) taste
of what paradoxical self-reference is like,
we present the sketch of an algorithmic proof (by contradiction)
of the unsolvability of the halting problem.
Consider a universal computer $U$ and  an arbitrary algorithm
$B(X)$ whose input is a string of symbols $X$.  Assume that there exists
a ``halting algorithm'' ${\tt HALT}$ which is able to decide whether $B$
terminates on $X$ or not.
The domain of ${\tt HALT}$  is the set of legal programs.
The range of ${\tt HALT}$ are classical bits.

Using ${\tt HALT}(B(X))$ we shall construct another deterministic
computing agent $A$, which has as input any effective program $B$ and
which proceeds as follows:  Upon reading the program $B$ as input, $A$
makes a copy of it.  This can be readily achieved, since the program $B$
is presented to $A$ in some encoded form
$\ulcorner B\urcorner $,
i.e., as a string of
symbols.  In the next step, the agent uses the code
$\ulcorner B\urcorner $
 as input
string for $B$ itself; i.e., $A$ forms  $B(\ulcorner B\urcorner )$,
henceforth denoted by
$B(B)$.  The agent now hands $B(B)$ over to its subroutine ${\tt HALT}$.
Then, $A$ proceeds as follows:  if ${\tt HALT}(B(B))$ decides that
$B(B)$ halts, then the agent $A$ does not halt; this can for instance be
realized by an infinite {\tt DO}-loop; if ${\tt HALT}(B(B))$ decides
that $B(B)$ does {\em not} halt, then $A$ halts.

The agent $A$ will now be confronted with the following paradoxical
task:  take the own code as input and proceed to determine whether or not it halts.
Then, whenever $A(A)$
halts, ${\tt HALT}(A(A))$, by the definition of $A$, would force $A(A)$ not to halt.
Conversely,
whenever $A(A)$ does not halt, then ${\tt HALT}(A(A))$ would steer
$A(A)$ into the halting mode.  In both cases one arrives at a complete
contradiction.  Classically, this contradiction can only be consistently
avoided by assuming the nonexistence of $A$ and, since the only
nontrivial feature of $A$ is the use of the peculiar halting algorithm
${\tt HALT}$, the impossibility of any such halting algorithm.


A universal computer can in principle be embedded into or realized by
physical systems \cite{moore}.
An example for such a physical system is the computer
on which I am currently typing this manuscript.
It follows by reduction that there exist physical observables,
in particular forecasts about whether or not such computer will ever
halt in the sense sketched above,
which are provable undecidable.




\subsection{The busy beaver function as the maximal recurrence time}

The busy beaver function \cite{rado,chaitin-ACM,dewdney,brady}
addresses the following
question: given a finite system;
i.e., a system whose algorithmic description is of finite length.
What is the biggest number producible by such a system before halting?

Let $\Sigma (n)$ denote the busy beaver function of $n$.
 Originally, T. Rado \cite{rado}
 asked how
 many $1$'s a Turing machine with $n$ possible states and an empty
 input tape
 could print on that tape before halting.
 The first values of the Turing busy beaver function $\Sigma _T(x)$
 are finite and are known \cite{dewdney,brady}:
  $\Sigma _T(1)=1$,
 $\Sigma _T(2)= 4$,
  $\Sigma _T(3)=6$,
 $\Sigma _T(4)= 13$,
 $\Sigma _T(5) \ge 1915$,
 $\Sigma_T(7)\ge 22961$,
 $\Sigma_T(8)\ge 3\cdot (7\cdot 3^{92}-1)/2$.

Consider a related question: what is the upper bound of running time --- or,
alternatively, recurrence time --- of a program of length $n$ bits before
terminating?
An answer to that question confers a feeling of how long we have to
wait for the most time-consuming program of length $n$ bits to
hold. That, of course, is a worst-case scenario. Many programs of
length $n$ bits will have halted long  before the maximal halting time.

We mention without proof \cite{chaitin-ACM,chaitin-bb}  that
this bound can be represented by the busy beaver function:
${\tt TMAX}(n)=\Sigma (n+O(1))$ is the minimum time at which all
programs of size smaller than or equal to $n$ bits which halt have done so.

Knowledge of ${\tt TMAX}$ would ``solve'' the halting
problem quantitatively.
Because if the maximal halting time would be known
and bounded by any computable function of the program size of $n$ bits,
one would have to wait
just a little bit longer than ${\tt TMAX}(n)$ to make sure
that every program of length $n$ --- also this particular program ---
would have terminated.
Otherwise, the program would run forever.
In this sense, knowledge of ${\tt TMAX}$ is equivalent to  a
perfect predictor. Since the latter one does not exist,
we may expect that ${\tt TMAX}$ cannot be a computable function.
Indeed, for large values of $n$, $\Sigma (n)$
grows faster than any computable function  of $n$.


By reduction we obtain upper bounds for the recurrence of any kind of physical behavior:
for deterministic systems representable by $n$ bits,
the recurrence time grows faster than any computable number
of $n$.
This bound from below for possible behaviors may be interpreted as a qualitative measure
of the impossibility to predict and forecast such behaviors by algorithmic means.


\subsection{Undecidability of the induction problem}

Induction in physics is the inference of general rules
dominating and generating physical behaviors from these behaviors.
For any deterministic system strong enough to support
universal computation, the general induction problem
is provable unsolvable.
Induction is thereby reduced to the unsolvability of
the rule inference problem \cite{go-67,blum75blum,angluin:83,ad-91,li:92},

Informally, the algorithmic idea of the proof is to take any sufficiently powerful
rule or method of induction and, in using it, define some
functional behavior which is not identified by it.
This amounts
to constructing an algorithm which
(passively!)
 ``fakes'' the ``guesser'' by simulating some particular function $\varphi $
until the guesser
pretends to guess this function correctly.
In a second,  diagonalization step, the ``faking'' algorithm then switches to a different
 function $\varphi^\ast  \neq \varphi $, such that the guesser's guesses become incorrect.


More formally, assume two (universal) computers $U$ and $U'$.
Suppose that the second computer $U'$ executes an arbitrary
algorithm $p$ unknown to computer $U$, the ``guesser.''
 The task of $U$,
 which is called the rule inference problem,
 is to conjecture the ``law'' or algorithm $p$ by analysing the
 behavior of $U'(p)$.
 The recursive unsolvability of the rule inference problem \cite{go-67} states that this task cannot be
 performed by any effective computation.

For the sake of contradiction, assume \cite{li:92}
that there exists a ``perfect guesser'' $U$ which can identify
all total recursive functions (wrong).
Then it is possible to construct a function $\varphi^\ast:{\Bbb N} \rightarrow
\{0,1\}$, such that the guesses of $U$ are wrong infinitely often,
thereby contradicting the above assumption.

Define $\varphi^\ast (0)=0$.
One may construct $\varphi^\ast $ by simulating $U$. Suppose the values
$\varphi^\ast (0)$, $\varphi^\ast (1)$, $\varphi^\ast (2)$, $\cdots$,
$\varphi^\ast (n-1)$ have already been constructed. Then, on input $n$,
simulate $U$, based on the previous series
$
\{0, \varphi^\ast (0)\},
\{1, \varphi^\ast (1)\},
\{2, \varphi^\ast (2)\},
\cdots ,
\{n-1, \varphi^\ast (n-1)\}$,
 and define
$\varphi^\ast (n)$ equal to 1 plus the guess of $U$ of
$\varphi^\ast (n)$ mod 2. In this way, $U$ can never guess
$\varphi^\ast $ correctly; thereby making an infinite number of mistakes.

One can also interpret this result in terms of the recursive
unsolvability of the halting problem, which in turn is related to the busy beaver function:
there is no recursive bound on the
time the guesser has to wait in order to make sure that his guess is
correct.


\subsection{Results in classical recursion theory with implications for theoretical physics}


The following theorems of recursive (i.e., computable) analysis have some implications to theoretical physics \cite{kreisel}.
(i)
There exist recursive monotone bounded sequences of rational numbers
whose limit is no computable number
\cite{Specker49}.
A concrete example of such a number is Chaitin's Omega number \cite{chaitin3,calude:94},
the ``halting probability'' for a computer (using prefix-free code),
which can be defined by a sequence of rational numbers
with no computable radius of convergence.

(ii)
There exist a recursive real function which has its maximum in the unit interval
at no recursive real number \cite{Specker57}.
This has implication for the principle of least action.

(iii)
There exists a real number $r$ such that $G(r) = 0$ is recursively undecidable for $G(x)$
in a class of functions which involves polynomials and the sine function
\cite{wang}.
This again has some bearing on  the principle of least action.

(iv)
There exist uncomputable solutions of the wave equations for computable initial values
\cite{pr1,bridges1}.

%\end{description}



\section{Behavior of three or more classical bodies}

An extreme deterministic position
was formulated by Laplace,  stating that \cite[Chapter II]{laplace-prob}
\begin{quote}
{\em
Present events are connected with preceding ones
by a tie based upon the evident principle that a thing
cannot occur without a cause which produces it. This
axiom, known by the name of the principle of sufficient
reason, extends even to actions which are considered
indifferent $\ldots$


We ought then to regard the present state of the
universe as the effect of its anterior state and as the
cause of the one which is to follow. Given for one
instant an intelligence which could comprehend all the
forces by which nature is animated and the respective
situation of the beings who compose it an intelligence
sufficiently vast to submit these data to analysis it
would embrace in the same formula the movements of
the greatest bodies of the universe and those of the
lightest atom; for it, nothing would be uncertain and
the future, as the past, would be present to its eyes.
}
\end{quote}

In the late 18th hundred,
the issue seemed worthy and pressing enough to establish a prize by
King Oscar II of Sweden, advised by Martin Leffler, who published the following
question formulated by Weierstrass:
\begin{quote}
{\em
Given a system of arbitrarily many mass points that attract each
according to Newton's law, under the assumption that no two points ever collide,
try to find a representation of the coordinates of each point
as a series in a variable that is some known function of time and for
all of whose values the series converges uniformly.
}
\end{quote}
Poincar{\'e}'s original prize--winning contribution contained errors.
The necessary corrections led the author to the conclusion that sometimes small
variations in the initial values could lead to huge variations in the
evolution of a physical system in later times.
In Poincar{\'e}'s own words  \cite[Chapter 4, Sect. II, pp.56-57]{poincare14}\footnote{
W\"urden wir die Gesetze der Natur und den Zustand des Universums
f\"ur einen gewissen Zeitpunkt genau kennen, so
k\"onnten wir den Zustand dieses Universums f\"ur
irgendeinen sp\"ateren Zeitpunkt genau voraussagen.
Aber
[[~$\ldots$~]]
 es kann der Fall eintreten,
da\ss $\;$ kleine Unterschiede in den Anfangsbedingungen
gro\ss e Unterschiede in den sp\"ateren Erscheinungen bedingen;
ein kleiner Irrtum in den ersteren kann einen au\ss erordentlich gro\ss en
Irrtum f\"ur den letzteren nach sich ziehen.
Die Vorhersage wird unm\"oglich und wir haben eine
``zuf\"allige Erscheinung''.}:
\begin{quote}
{\em
If we would know the laws of Nature and the state of the Universe precisely
for a certain time,
we would be able to predict with certainty
the state of the Universe for any later time.
But
[[~$\ldots$~]]
it can be the case that small differences in the initial values
produce great differences in the later phenomena;
a small error in the former may result in a large error in the latter.
The prediction becomes impossible and we have a ``random phenomenon.''}
\end{quote}

\subsection{Deterministic chaos}

Poincar{\'e}'s recognition of possible instabilities
in $n$-body problems was the first indication of what today is called ``deterministic chaos.''
In chaotic systems it is practically impossible to specify
the initial value precise enough to allow long-term predictions.

A stronger assumption supposes that the initial values are elements of
a continuum and thus are not representable by any algorithmically compressible number;
in short, that they are random \cite{calude:02}.
Classical, deterministic chaos results from ``unfolding'' such a random initial value
drawn from the ``continuum urn'' by a recursive, deterministic function.

A weaker form of deterministic chaos just expresses the fact
that linear deviations of initial values which lie within the measurement precision
result in exponential divergences in the future evolution of the system.
For further discussions, the interested reader is referred to
Refs.~\cite{shaw,crutchFaPaShaw,schuster1,bricmont,Crutchfield90}


\subsection{Convergence of the general solution}

More than one hundred years after its formulation as quoted above,
the $n$--body problem has been solved by Wang \cite{Wang91,Diacu96,Wang01}.
The $3$--body problem was already solved in 1912 \cite{Sundman12}.
The solutions are given in terms of power series.

Yet, in order to be practically applicable,
the radius of convergence of the series must be known.
We might already expect from deterministic chaos
that these series solution have a ``very slow'' convergence.
Even the prediction of behaviors in insignificantly short times
may require the summation of a huge number of terms,
making these series unusable for any practical work
\cite{Diacu96}.

Alas, the complications regarding convergence of the series solutions are far more serious.
Suppose we are able to construct a universal computer based on the $n$--body problem.
This can, for instance, be achieved by ballistic computation, such as the
``Billiard Ball'' model of computation
\cite{fred-tof-82,margolus-02}
which effectively ``embeds'' a universal computer into a system of $n$--bodies.
Then, by reduction, it follows that certain predictions are impossible.

What are the consequences of this reduction for the convergence of the series solutions?
It can be expected that not only do the series converge ``very slowly,''
like in deterministic chaotic systems,
but that in general there does not exist any computable radius of convergence
for the series solutions.
This is very similar to Chaitin's Omega number \cite{chaitin3,calude:94}
representing the halting probability of a universal computer, or the busy beaver function.
The Omega number can be ``enumerated''
by series solutions from ``pseudo-algorithms''
computing its very first digits.
Yet, due to the uncomputable growth of the time
required to determine whether or not terms possibly contribute,
the series lack any computable radius of convergence.



\section{Quantum unknowables}


A third group of physical unknowables arise in the quantum context.
Throughout its development, although a highly successful theory,
quantum mechanics, in particular its interpretation and meaning,
has been received controversially within the community.
Some of its founding fathers, such as Schr\"odinger,
De Broglie and Einstein had a very critical view on its
validity and considered quantum mechanics a preliminary theory which should give
way to a more complete one.
Others, among them Bohr and Heisenberg,
claimed that quantum unknowables will stay with us forever.
Over the years, the latter view seems to have prevailed
\cite{fuchs-peres}, although it was not totally unchallenged
\cite{jammer:66,jammer1,jammer-92}.
Already Sommerfeld warned his students not to get
into the ``meaning behind'' quantum mechanics,
and, as mentioned by Clauser \cite{clauser-talkvie},
not long ago scientists working in that field
had to be very careful not to become discredited as ``quacks.''
Richard Feynman \cite[p. 129]{feynman-law}
once mentioned the
\begin{quote}
{\em ``$\ldots$ perpetual torment that results
from [[the question]], `But how can it be like that?' which
is a reflection of uncontrolled but utterly vain desire to see
[[quantum mechanics]] in terms of an analogy with something familiar
$\ldots$
Do not keep saying to yourself, if you can possibly avoid it,
`But how can it be like that?'
because you will get `down the drain', into a blind alley from which nobody has yet
escaped.''}
\end{quote}

In what follows, we shall discuss three main quantum unknowables:
(i) randomness of single events,
(ii) complementarity, and
(iii) value indefiniteness.

\subsection{Random events}

The quantum formalism does not predict the outcome of single events
when there is a mismatch between the context in which a state was prepared,
and the context in which it is measured.
Here, context means maximal observable, or more technically,
the maximal operator from which all commuting operators can be functionally derived
\cite[\S 84]{halmos-vs}.

In the absence of other explanations, one is thus lead to the conclusion
that these single events occur without any causation
and thus at random.
Such random ``quantum coin toss'' \cite{svozil-qct}
have been used for various purposes, among them delayed choice experiments
\cite{wjswz-98,zeilinger:qct}.
Commercial interface cards \cite{Quantis} perform at a rate of 4 to 16 Mbit/s.

Note that randomness of this type \cite{Cris04,calude-dinneen05}
is postulated rather than proven.
This is necessarily so, for any claim of randomness can only be corroborated
with respect to a more or less large class of laws or behaviors;
it is impossible to inspect the hypothesis against an infinity of conceivable laws.
How can we ever exclude the possibility of our
presented, some day (perhaps by some extraterrestrial visitors), with a (perhaps
extremely complex) device  that ``computes'' and ``predicts''
a certain type of hitherto ``random'' physical behavior?


\subsection{Complementarity}

Another quantum indeterminism is complementarity.
Complementarity is the principal impossibility to measure
two or more complementary observables
with arbitrary precision simultaneously.

Complementarity was first encountered in quantum mechanics,
but it is a phenomenon also observable in the classical world.
To get a better feeling for complementarity, we shall consider generalized urn models
\cite{wright,wright:pent} or, equivalently \cite{svozil-2001-eua},
finite Moore and Mealy automata \cite{e-f-moore,schaller-96,dvur-pul-svo,cal-sv-yu}.
Both quasi-classic examples mimic complementarity and even quasi-quantum cryptography
\cite{svozil-2005-ln1e}.


A generalized urn model is
characterized by an ensemble of balls with black background color.
Printed on these balls are some color symbols from a symbolic alphabet.
The colors are elements of a set of colors.
A particular ball type is associated with a unique combination of mono-spectrally
(no mixture of wavelength) colored symbols
printed on the black ball background.
Every ball contains just one single symbol per color.

Assume further some mono-spectral filters or eyeglasses which are
``perfect'' by totally absorbing light of all other colors
but a particular single one.
In that way, every color can be associated with a particular eyeglass and vice versa.
%This, of course, is a system science trick related to intrinsic color perception.

When a spectator looks at a particular ball through such an eyeglass,
the only operationally recognizable symbol will be the one in the particular
color which is transmitted through the eyeglass.
All other colors are absorbed, and the symbols printed in them will appear black
and therefore cannot be differentiated from the black background.
Hence the ball appears to carry a different ``message'' or symbol,
depending on the color at which it is viewed.
An explicit example is
enumerated in Table~\ref{2005-nl1-t1}.
\begin{table}
\begin{tabular}{ccc}
\hline\hline
 \hspace{0.5 true cm} ball type \hspace{0.5 true cm} &
 \hspace{0.5 true cm} {\Red red} \hspace{0.5 true cm} &
\hspace{0.5 true cm} {\Green green} \hspace{0.5 true cm} \\
\hline
1&{\Red 0}&{\Green 0}\\
2&{\Red 0}&{\Green 1}\\
3&{\Red 1}&{\Green 0}\\
4&{\Red 1}&{\Green 1}\\
\hline\hline
\end{tabular}
\caption{Schema of imprinting of four ball types filling a generalized urn.
Whenever the spectator looks through the red eyeglass,
the red symbols printed on the balls appear, whereas the green symbols
merge in their black background.
Conversely,
the spectator may choose to look at the green symbols through
the green eyeglass. In the latter case, the red symbols become unrecognizable.
\label{2005-nl1-t1}}
\end{table}


The difference between the balls and the quanta is the possibility
to view all the different symbols on the balls
in all different colors by taking off the eyeglasses.
Quantum mechanics does not provide us with such a possibility.
On the contrary, there are strong formal arguments suggesting
that the assumption of a simultaneous
physical existence of such complementary observables yields a complete contradiction.
These issues will be discussed next.

\subsection{Value indefiniteness versus omniscience}

Still another quantum unknowable results from the fact that no ``global'' classical truth
assignment exists which is consistent with even a finite number of ``local'' ones.
That is, no consistent classical truth table can be given by pasting together commeasurable observables.
This phenomenon is also known as value indefiniteness or contextuality.

Already scholastic philosophy \cite{specker-60},
for instance Thomas Aquinas  Ref.~\cite{Aquinas},
considered questions such as whether God has knowledge of non-existing things
(Part 1, Question 14, Article 9) or things
that are not yet (Part 1, Question 14, Article 13).
Classical omniscience, at least its naive expression that
``if a proposition is true, then an omniscient agent (such as God) knows that it is true''
is plagued by paradoxical self-referential.

The empirical sciences implement classical omniscience by assuming that
in principle, all observables of classical physics are (co-)measurable without any restrictions,
and regardless of whether they are actually measured or not.
No distinction is made between an observable obtained by an ``actual'' and a ``potential'' measurement.
(In contrast compare Schr\"odinger's own interpretation of the wave function \cite[\S  7]{schrodinger} as a
``catalogue of expectations.'')
Precision and (co-)measurability are limited only by the technical capacities of the experimenter.
The principle of empirical classical omniscience has given rise to the realistic believe that
all observables ``exist'' regardless of their observation; i.e., regardless and independent of
any particular measurement.
Physical (co-)existence is thereby related to the realistic assumption \cite{stace}
(sometimes referred to as the ``ontic'' \cite{atman:05} viewpoint) that such physical entities exist
even without being experienced by any finite mind.

The formal expression of classical omniscience is the Boolean algebra of observable propositions
\cite{Boole}, the distributive law satisfied by the classical logical operations,
and in particular the ``abundance'' of two-valued states interpretable
as omniscience about the system.
Thereby, any such ``dispersionless'' two-valued state --- associated with a ``truth table''
--- can be defined on all observables,
regardless of whether they have been actually observed or not.

Historically, the discovery of the uncertainty principle and complementarity
seem to have been first indications of the decline of classical omniscience.
A formal expression of complementarity is the nondistributive algebra of quantum observables.
Alas,
nondistributivity of the empirical logical structure is no sufficient
condition for the impossibility of omniscience.
The generalized urn
as well as equivalent finite automaton models discussed above
possess two-valued states
interpretable as omniscience.

A further blow to quantum omniscience came from Boole's
``conditions of possible experience'' \cite{Boole-62,Pit-94} for quantum probabilities and
expectation functions.
In particular, Bell was the first to point to experiments which,
based on counterfactually inferred elements of physical reality
discussed by Einstein, Podolsky and Rosen \cite{epr},
seemed to indicate the impossibility
to faithfully embed quantum observables into classical Boolean algebras.
To state the issues pointedly, under some (presumably mild) side assumptions,
``unperformed experiments have no results''
\cite{peres222} --- there cannot exist a table enumerating all
actual and hypothetical experimental
outcomes consistent with the observed quantum frequencies \cite{zeilinger-epr-98}.
Any such table could be interpreted as omniscience with respect to
the observables in the Bell-type experiments.
The impossibility to construct such tables appears to be a very serious indication against
quantum omniscience.

The reason for the impossibility to describe all quantum observables
simultaneously by classical tables of experimental outcomes
can be understood in terms of a ``stronger'' result stating that,
for quantum systems whose Hilbert space is of dimension greater than two,
there does not exist any dispersionless, two-valued state
interpretable as truth table.
This result, which is known as the Kochen-Specker theorem
\cite{specker-60,kochen1,ZirlSchl-65,Alda,Alda2,kamber64,kamber65,cabello-96},
has a finitistic proof by contradiction.
To get a flavor of the argument,
a short version of the proof is depicted in Fig.~\ref{2007-miracles-ksc}.
It is a brain teaser
to argue that no coloring of the points in this diagram exists which would include only one red point
per smooth, unbroken curve; the other three points all remaining green.
\begin{figure*}
\begin{center}
%TeXCAD Picture [4.pic]. Options:
%\grade{\on}
%\emlines{\off}
%\epic{\off}
%\beziermacro{\on}
%\reduce{\on}
%\snapping{\off}
%\quality{8.00}
%\graddiff{0.01}
%\snapasp{1}
%\zoom{5.6569}
\unitlength 1mm % = 2.85pt
\linethickness{0.8pt}
\ifx\plotpoint\undefined\newsavebox{\plotpoint}\fi % GNUPLOT compatibility
\begin{picture}(134.09,125.99)(0,0)
%\emline(86.39,101.96)(111.39,58.46)
\multiput(86.39,101.96)(.067385445,-.117250674){371}{\line(0,-1){.117250674}}
%\end
%\emline(86.39,14.96)(111.39,58.46)
\multiput(86.39,14.96)(.067385445,.117250674){371}{\line(0,1){.117250674}}
%\end
%\emline(36.47,101.96)(11.47,58.46)
\multiput(36.47,101.96)(-.067385445,-.117250674){371}{\line(0,-1){.117250674}}
%\end
%\emline(36.47,14.96)(11.47,58.46)
\multiput(36.47,14.96)(-.067385445,.117250674){371}{\line(0,1){.117250674}}
%\end
\put(86.39,101.71){\line(-1,0){50}}
\put(86.39,15.21){\line(-1,0){50}}
\put(86.28,101.76){\circle{2.97}}
\put(86.28,15.16){\circle{2.97}}
\put(93.53,89.21){\circle{2.97}}
\put(93.53,27.71){\circle{2.97}}
\put(29.24,89.21){\circle{2.97}}
\put(29.24,27.71){\circle{2.97}}
\put(102.37,73.47){\circle{2.97}}
\put(102.37,43.44){\circle{2.97}}
\put(20.4,73.47){\circle{2.97}}
\put(20.4,43.44){\circle{2.97}}
\put(111.21,58.45){\circle{2.97}}
\put(11.56,58.45){\circle{2.97}}
\put(36.34,101.76){\circle{2.97}}
\put(36.34,15.16){\circle{2.97}}
\put(52.99,101.76){\circle{2.97}}
\put(52.99,15.16){\circle{2.97}}
\put(69.68,101.76){\circle{2.97}}
\put(69.68,15.16){\circle{2.97}}
\qbezier(29.2,27.73)(23.55,-5.86)(52.99,15.24)
\qbezier(93.57,27.73)(99.22,-5.86)(69.78,15.24)
\qbezier(29.2,27.88)(36.93,75)(69.63,101.91)
\qbezier(93.57,27.88)(85.84,75)(53.13,101.91)
\qbezier(52.69,15.24)(87.47,40.96)(93.72,89.27)
\qbezier(70.08,15.24)(35.3,40.96)(29.05,89.27)
\qbezier(93.72,89.27)(98.4,125.99)(69.49,102.06)
\qbezier(29.05,89.27)(24.37,125.99)(53.28,102.06)
\qbezier(20.15,73.72)(-11.67,58.52)(20.15,43.31)
\qbezier(20.33,73.72)(61.34,93.16)(102.36,73.72)
\qbezier(102.36,73.72)(134.09,58.52)(102.53,43.31)
\qbezier(102.53,43.31)(60.99,23.43)(20.15,43.49)
\put(30.41,114.02){\makebox(0,0)[cc]{$(0,1,-1,0)$}}
\put(30.41,2.65){\makebox(0,0)[cc]{$(0,0,1,-1)$}}
\put(52.68,114.38){\makebox(0,0)[cc]{$(1,0,0,1)$}}
\put(52.68,2.3){\makebox(0,0)[cc]{$(1,-1,0,0)$}}
\put(91.93,114.2){\makebox(0,0)[cc]{$(-1,1,1,1)$}}
\put(91.93,2.48){\makebox(0,0)[cc]{$(1,1,1,1)$}}
\put(69.65,114.38){\makebox(0,0)[cc]{$(1,1,1,-1)$}}
\put(73.65,2.3){\makebox(0,0)[cc]{$(1,1,-1,-1)$}}
\put(103.24,94.22){\makebox(0,0)[cc]{$(1,1,-1,1)$}}
\put(19.45,94.22){\makebox(0,0)[cc]{$(0,1,1,0)$}}
\put(106.24,22.45){\makebox(0,0)[cc]{$(1,-1,1,-1)$}}
\put(19.45,22.45){\makebox(0,0)[cc]{$(0,0,1,1)$}}
\put(110.13,77.96){\makebox(0,0)[cc]{$(1,0,1,0)$}}
\put(12.55,77.96){\makebox(0,0)[cc]{$(0,0,0,1)$}}
\put(110.13,38.72){\makebox(0,0)[cc]{$(1,0,-1,0)$}}
\put(12.55,38.72){\makebox(0,0)[cc]{$(0,1,0,0)$}}
\put(120.92,57.98){\makebox(0,0)[l]{$(1,1,0,-1)$}}
\put(1.77,57.98){\makebox(0,0)[rc]{$(1,0,0,0)$}}
\end{picture}
\end{center}
\caption{Proof of the Kochen-Specker theorem \cite{cabello-96,cabello-99} in four-dimensional real vector space.
Nine interconnected contexts (or four-pods) are represented by smooth, unbroken curves.
The graph contains possible quantum observables represented by 18 points, which are explicitly enumerated.
It cannot be colored by the two colors red (associated with truth)
and green (associated with falsity) such that every context contains exactly one red and three green points.
For, by construction, on the one hand, every red point occurs in exactly two contexts (four-pods), and hence
there is an even number of red points in a table containing the points of the contexts as columns
and the enumeration of contexts as rows.
On the other hand, there are nine contexts involved; thus by the rules it follows that there
is an odd number of red points in this table (exactly one per context).
Thus, our assumption about the colorability and therefore about
possible consistent truth assignments for this finite set of quantum observables
leads to a complete contradiction.
\label{2007-miracles-ksc} }
\end{figure*}



The violations of conditions of possible classical experience or
the Kochen-Specker theorem do not exclude realism restricted to a single context,
but realistic omniscience beyond it.
It might be a classical anachronism to assume that outside of a single context
in which the particle was prepared, all observables are (pre-)defined.


\section{Miracles due to gaps in causal description}

A different issue, discussed by Philipp Frank,
is the possible occurrence of miracles in the presence of gaps of physical determinism.
One might perceive singular events occurring
within the bounds of classical and quantum physics without any apparent cause as miracles.
For, if there is no cause to an event,
why should such an event occur altogether rather than not occur?

Although such thoughts remain highly speculative, miracles,
if they exist,
could be the basis for a directed evolution in otherwise deterministic physical systems.
Similar models have also been applied to dualistic models of the mind \cite{eccles:papal,popper-eccles}.

There exist bounds on miracles and on behavioral patterns in general due to the self-referential
perception of intrinsic observers endowed with free will:
if such an observer is omniscient and has absolute predictive power,
then free will could counteract omniscience, and in particular the own predictions.
The only consistent alternative seems either to abandon free will,
stating that it is an idealistic illusion,
or to accept that omniscience and absolute predictive power is bound by paradoxical self-reference.

\section{Summary}


Hilbert's 6th problem is about the axiomatization of all of physics.
We still do not know whether or not this goal is achievable.
All we know is that even if it could be achieved, omniscience cannot be gained
via the formalized, syntactic route to infer and predict physical behaviors.
It will remain blocked forever by paradoxical self-reference
which intrinsic observers and operational methods are bound to,
It remains to be seen whether or not these G\"odelian-type physical unknowables
are relevant for the practical development of physics proper.

%\bibliography{svozil}
%\bibliographystyle{osa}


\begin{thebibliography}{100}
\newcommand{\enquote}[1]{``#1''}
\expandafter\ifx\csname url\endcsname\relax
  \def\url#1{\texttt{#1}}\fi
\expandafter\ifx\csname urlprefix\endcsname\relax\def\urlprefix{URL }\fi
\providecommand{\eprint}[2][]{\url{#2}}

\bibitem{frank}
P.~Frank, \emph{Das Kausalgesetz und seine Grenzen} (Springer, Vienna, 1932).
  English translation in Ref.~\cite{franke}.

\bibitem{barrow-TOE}
J.~D. Barrow, \emph{Theories of Everything} (Oxford University Press, Oxford,
  1991).

\bibitem{wigner}
E.~P. Wigner, \enquote{The unreasonable effectiveness of mathematics in the
  natural sciences. {R}ichard {C}ourant {L}ecture delivered at {N}ew {Y}ork
  {U}niversity, {M}ay 11, 1959,} Communications on Pure and Applied Mathematics
  \textbf{13}, 1 (1960).

\bibitem{rogers1}
H.~{Rogers, Jr.}, \emph{Theory of Recursive Functions and Effective
  Computability} (MacGraw-Hill, New York, 1967).

\bibitem{odi:89}
P.~Odifreddi, \emph{Classical Recursion Theory, Vol. 1} (North-Holland,
  Amsterdam, 1989).

\bibitem{odi:99}
P.~Odifreddi, \emph{Classical Recursion Theory, Vol. 2} (North-Holland,
  Amsterdam, 1999).

\bibitem{godel1}
K.~G{\"{o}}del, \enquote{{\"{U}}ber formal unentscheidbare {S\"{a}}tze der
  {P}rincipia {M}athematica und verwandter {S}ysteme,} Monatshefte f{\"{u}}r
  Mathematik und Physik \textbf{38}, 173--198 (1931). {E}nglish translation in
  \cite{godel-ges1}, and in \cite{davis}.

\bibitem{tarski:32}
A.~Tarski, \enquote{Der Wahrheitsbegriff in den Sprachen der deduktiven
  Disziplinen,} Akademie der Wissenschaften in Wien.
  Mathematisch-naturwissenschaftliche Klasse, Anzeiger \textbf{69}, 24 (1932).

\bibitem{tarski:56}
A.~Tarski, \emph{Logic, Semantics and Metamathematics} (Oxford University
  Press, Oxford, 1956).

\bibitem{turing-36}
A.~M. Turing, \enquote{On computable numbers, with an application to the
  {E}ntscheidungsproblem,} Proceedings of the London Mathematical Society,
  Series 2 \textbf{42 and 43}, 230--265 and 544--546 (1936-7 and 1937).
  Reprinted in \cite{davis}.

\bibitem{chaitin3}
G.~J. Chaitin, \emph{Algorithmic Information Theory} (Cambridge University
  Press, Cambridge, 1987).

\bibitem{popper-50i}
K.~R. Popper, \enquote{Indeterminism in Quantum Physics and in Classical
  Physics I,} The British Journal for the Philosophy of Science \textbf{1},
  117--133 (1950).

\bibitem{popper-50ii}
K.~R. Popper, \enquote{Indeterminism in Quantum Physics and in Classical
  Physics II,} The British Journal for the Philosophy of Science \textbf{1},
  173--195 (1950).

\bibitem{moore}
C.~D. Moore, \enquote{Unpredictability and undecidability in dynamical
  systems,} Physical Review Letters \textbf{64}, 2354--2357 (1990). Cf. Ch.
  Bennett, {\sl Nature}, {\bf 346}, 606 (1990),
  \urlprefix\url{http://link.aps.org/abstract/PRL/v64/p2354}.

\bibitem{casti:94-onlimits_book}
J.~L. Casti and J.~F. Traub, \emph{On Limits} (Santa Fe Institute, Santa Fe,
  NM, 1994). Report 94-10-056,
  \urlprefix\url{http://www.santafe.edu/research/publications/workingpapers/94%
-10-056.pdf}.

\bibitem{casti:96-onlimits}
J.~L. Casti and A.~Karlquist, \emph{Boundaries and Barriers. On the Limits to
  Scientific Knowledge} (Addison-Wesley, Reading, MA, 1996).

\bibitem{feynman-law}
R.~P. Feynman, \emph{The Character of Physical Law} (MIT Press, Cambridge, MA,
  1965).

\bibitem{v-neumann-66}
J.~von Neumann, \emph{Theory of Self-Reproducing Automata} (University of
  Illinois Press, Urbana, 1966). A. W. Burks, editor.

\bibitem{fef-84}
S.~Feferman, \enquote{{K}urt {G}\"odel: conviction and caution,} Philosophia
  Naturalis \textbf{21}, 546--562 (1984).

\bibitem{baats1}
M.~Baaz, \enquote{{\"U}ber den allgemeinen Gehalt von Beweisen,} in
  \emph{Contributions to General Algebra}, vol.~6
  (H{\"{o}}lder-Pichler-Tempsky, Vienna, 1988).

\bibitem{neugeb}
O.~Neugebauer, \emph{Vorlesungen {\"{u}}ber die Geschichte der antiken
  mathematischen Wissenschaften. 1. Band: Vorgriechische Mathematik} (Springer,
  Berlin, 1934). Page 172.

\bibitem{godel-ges1}
K.~G{\"{o}}del, in \emph{Collected Works. Publications 1929-1936. Volume {I}},
  S.~Feferman, J.~W. Dawson, S.~C. Kleene, G.~H. Moore, R.~M. Solovay, and
  J.~van Heijenoort, eds. (Oxford University Press, Oxford, 1986).

\bibitem{bernstein}
J.~Bernstein, \emph{Quantum Profiles} (Princeton University Press, Princeton,
  1991).

\bibitem{sterne}
L.~Sterne, \emph{The Life and Opinions of Tristram Shandy, Gentleman} (London,
  1760-1767). \urlprefix\url{http://www.gutenberg.org/etext/1079}.

\bibitem{rado}
T.~Rado, \enquote{On Non-Computable Functions,} The Bell System Technical
  Journal \textbf{XLI(41)}(3), 877--884 (1962).

\bibitem{chaitin-ACM}
G.~J. Chaitin, \enquote{Information-theoretic limitations of formal systems,}
  Journal of the Association of Computing Machinery \textbf{21}, 403--424
  (1974). Reprinted in \cite{chaitin2},
  \urlprefix\url{http://www.cs.auckland.ac.nz/CDMTCS/chaitin/acm74.pdf}.

\bibitem{dewdney}
A.~K. Dewdney, \enquote{Computer Recreations: A Computer Trap for the Busy
  Beaver, the Hardest-Working {T}uring Machine,} Scientific American
  \textbf{251}(2), 19--23 (1984).

\bibitem{brady}
A.~H. Brady, \enquote{The Busy Beaver Game and the Meaning of Life,} in
  \emph{The Universal Turing Machine. A Half-Century Survey}, R.~Herken, ed.,
  p. 259 (Kammerer und Unverzagt, Hamburg, 1988).

\bibitem{chaitin-bb}
G.~J. Chaitin, \enquote{Computing the Busy Beaver Function,} in \emph{Open
  Problems in Communication and Computation}, T.~M. Cover and B.~Gopinath,
  eds., p. 108 (Springer, New York, 1987). Reprinted in \cite{chaitin2}.

\bibitem{go-67}
E.~M. Gold, \enquote{Language identification in the limit,} Information and
  Control \textbf{10}, 447--474 (1967).
  \urlprefix\url{http://dx.doi.org/10.1016/S0019-9958(67)91165-5}.

\bibitem{blum75blum}
L.~Blum and M.~Blum, \enquote{Toward a mathematical theory of inductive
  inference,} Information and Control \textbf{28}(2), 125--155 (1975).

\bibitem{angluin:83}
D.~Angluin and C.~H. Smith, \enquote{A Survey of Inductive Inference: Theory
  and Methods,} Computing Surveys \textbf{15}, 237--269 (1983).

\bibitem{ad-91}
L.~M. Adleman and M.~Blum, \enquote{Inductive Inference and Unsolvability,} The
  Journal of Symbolic Logic \textbf{56}, 891--900 (1991).
  \urlprefix\url{http://dx.doi.org/10.2307/2275058}.

\bibitem{li:92}
M.~Li and P.~M.~B. Vit{\'{a}}nyi, \enquote{Inductive reasoning and {K}olmogorov
  complexity,} Journal of Computer and System Science \textbf{44}, 343--384
  (1992). \urlprefix\url{http://dx.doi.org/10.1016/0022-0000(92)90026-F}.

\bibitem{kreisel}
G.~Kreisel, \enquote{A notion of mechanistic theory,} Synthese \textbf{29},
  11--26 (1974). \urlprefix\url{http://dx.doi.org/10.1007/BF00484949}.

\bibitem{Specker49}
E.~Specker, \enquote{Nicht konstruktiv beweisbare {S}\"atze der {A}nalysis,}
  The Journal of Smbolic Logic \textbf{14}, 145--158 (1949). Reprinted in
  \cite[pp. 35--48]{specker-ges}; {E}nglish translation: {\it Theorems of
  Analysis which cannot be proven constructively}.

\bibitem{calude:94}
C.~Calude, \emph{Information and Randomness---An Algorithmic Perspective}
  (Springer, Berlin, 1994).

\bibitem{Specker57}
E.~Specker, \enquote{Der {S}atz vom {M}aximum in der rekursiven {A}nalysis,} in
  \emph{Constructivity in mathematics : proceedings of the colloquium held at
  Amsterdam, 1957}, A.~Heyting, ed., pp. 254--265 (North-Holland Publishing
  Company, Amsterdam, 1959). Reprinted in \cite[pp. 148--159]{specker-ges};
  {E}nglish translation: {\it Theorems of Analysis which cannot be proven
  constructively}.

\bibitem{wang}
P.~S. Wang, \enquote{The undecidability of the existence of zeros of real
  elementary functions,} Journal of the ACM (JACM) \textbf{21}, 586--589
  (1974). \urlprefix\url{http://dx.doi.org/10.1145/321850.321856}.

\bibitem{pr1}
M.~B. Pour-El and J.~I. Richards, \emph{Computability in Analysis and Physics}
  (Springer, Berlin, 1989).

\bibitem{bridges1}
D.~S. Bridges, \enquote{Constructive mathematics and unbounded operators---a
  reply to Hellman,} Journal of Philosophical Logic \textbf{28}(5) (1999).
  \urlprefix\url{http://dx.doi.org/10.1023/A:1004420413391}.

\bibitem{laplace-prob}
P.-S. Laplace, \emph{Philosophical Essay on Probabilities. {T}ranslated from
  the fifth {F}rench edition of 1825} (Springer, Berlin, New York, 1995,1998).
  \urlprefix\url{http://www.archive.org/details/philosophicaless00lapliala}.

\bibitem{poincare14}
H.~Poincar{\'{e}}, \emph{Wissenschaft und Hypothese} (Teubner, Leipzig, 1914).

\bibitem{calude:02}
C.~Calude, \emph{Information and Randomness---An Algorithmic Perspective}, 2nd
  ed. (Springer, Berlin, 2002).

\bibitem{shaw}
R.~S. Shaw, \enquote{Strange Attractors, Chaotic Behavior, and Information
  Flow,} Zeitschrift f{\"{u}}r Naturforschung \textbf{36a}, 80--110 (1981).

\bibitem{crutchFaPaShaw}
J.~P. Crutchfield, J.~D. Farmer, N.~H. Packard, and R.~S. Shaw,
  \enquote{Chaos,} Scientific American \textbf{255}, 46--57 (1986).

\bibitem{schuster1}
H.~G. Schuster, \emph{Deterministic Chaos} (Physik Verlag, Weinheim, 1984).

\bibitem{bricmont}
J.~Bricmont, \enquote{Science of Chaos or Chaos in Science?} Annals of the New
  York Academy of Sciences \textbf{775}, 131--176 (1996). Also reprinted in
  \cite{bricmont2}, \eprint{chao-dyn/9603009},
  \urlprefix\url{http://arxiv.org/abs/chao-dyn/9603009}.

\bibitem{Crutchfield90}
J.~P. Crutchfield, \enquote{Chaos and Complexity,} in \emph{Handbook of
  Metaphysics and Ontology}, H.~Burkhardt and B.~Smith, eds., p. 549
  (Philosophia Verlag, M\"unchen, 1991).
  \urlprefix\url{http://cse.ucdavis.edu/~cmg/papers/COCC.pdf}.

\bibitem{Wang91}
Q.~D. Wang, \enquote{The global solution of the n-body problem,} Celestial
  Mechanics \textbf{50}, 73--88 (1991).

\bibitem{Diacu96}
F.~Diacu, \enquote{The Solution of the N-Body Problem,} The Mathematical
  Intelligencer \textbf{18}(3), 66--70 (1996).

\bibitem{Wang01}
Q.~D. Wang, \enquote{Power Series Solutions and Integral Manifold of the n-body
  Problem,} Regular \& Chaotic Dynamics \textbf{6}(4), 433--442 (2001).
  \urlprefix\url{http://dx.doi.org/10.1070/RD2001v006n04ABEH000187}.

\bibitem{Sundman12}
K.~E. Sundman, \enquote{Memoire sur le probl{\`{e}}me de trois corps,} Acta
  Mathematica \textbf{36}, 105--179 (1912).

\bibitem{fred-tof-82}
E.~Fredkin and T.~Toffoli, \enquote{Conservative Logic,} International Journal
  of Theoretical Physics \textbf{21}(3-4), 219--253 (1982). Reprinted in
  \cite[Part I, Chapter 3]{adama02}.,
  \urlprefix\url{http://dx.doi.org/10.1007/BF01857727}.

\bibitem{margolus-02}
N.~Margolus, \enquote{Universal cellular automata based on the collisions of
  soft spheres,} in \emph{Collision-based computing}, A.~Adamatzky, ed.
  (Springer, London, 2002).
  \urlprefix\url{http://people.csail.mit.edu/nhm/cca.pdf}.

\bibitem{fuchs-peres}
C.~A. Fuchs and A.~Peres, \enquote{Quantum theory needs no `Interpretation�,}
  Physics Today \textbf{53}(4), 70--71 (2000). Further discussions of and
  reactions to the article can be found in the September issue of Physics
  Today, {\it 53}, 11-14 (2000),
  \urlprefix\url{http://www.aip.org/web2/aiphome/pt/vol-53/iss-9/p11.html and
  http://www.aip.org/web2/aiphome/pt/vol-53/iss-9/p14.html}.

\bibitem{jammer:66}
M.~Jammer, \emph{The Conceptual Development of Quantum Mechanics} (McGraw-Hill
  Book Company, New York, 1966).

\bibitem{jammer1}
M.~Jammer, \emph{The Philosophy of Quantum Mechanics} (John Wiley \& Sons, New
  York, 1974).

\bibitem{jammer-92}
M.~Jammer, \enquote{John {S}teward {B}ell and the debate on the significance of
  his contributions to the foundations of quantum mechanics,} in \emph{Bell's
  Theorem and the Foundations of Modern Physics}, A.~van~der Merwe, F.~Selleri,
  and G.~Tarozzi, eds., pp. 1--23 (World Scientific, Singapore, 1992).

\bibitem{clauser-talkvie}
J.~Clauser, \enquote{Early History of {B}ell�s Theorem,} in \emph{Quantum
  (Un)speakables. {F}rom {B}ell to Quantum Information}, pp. 61--96 (Springer,
  Berlin, 2002).

\bibitem{halmos-vs}
P.~R. Halmos, \emph{Finite-dimensional vector spaces} (Springer, New York,
  Heidelberg, Berlin, 1974).

\bibitem{svozil-qct}
K.~Svozil, \enquote{The quantum coin toss---Testing microphysical
  undecidability,} Physics Letters A \textbf{143}, 433--437 (1990).
  \urlprefix\url{http://dx.doi.org/10.1016/0375-9601(90)90408-G}.

\bibitem{wjswz-98}
G.~Weihs, T.~Jennewein, C.~Simon, H.~Weinfurter, and A.~Zeilinger,
  \enquote{Violation of {B}ell's Inequality under Strict Einstein Locality
  Conditions,} Phys. Rev. Lett. \textbf{81}, 5039--5043 (1998).
  \urlprefix\url{http://dx.doi.org/10.1103/PhysRevLett.81.5039}.

\bibitem{zeilinger:qct}
T.~Jennewein, U.~Achleitner, G.~Weihs, H.~Weinfurter, and A.~Zeilinger,
  \enquote{A Fast and Compact Quantum Random Number Generator,} Review of
  Scientific Instruments \textbf{71}, 1675--1680 (2000).
  \eprint{quant-ph/9912118},
  \urlprefix\url{http://dx.doi.org/10.1063/1.1150518}.

\bibitem{Quantis}
id~Quantique, \enquote{Quantis - Quantum Random Number Generators,}  (2004).
  \urlprefix\url{http://www.idquantique.com}.

\bibitem{Cris04}
C.~S. Calude, \enquote{{A}lgorithmic randomness, quantum physics, and
  incompleteness,} in \emph{Proceedings of the Conference ``Machines,
  Computations and Universality'' (MCU'2004)}, M.~Margenstern, ed., pp. 1--17
  (Lectures Notes in Comput. Sci. 3354, Springer, Berlin, 2004).

\bibitem{calude-dinneen05}
C.~S. Calude and M.~J. Dinneen, \enquote{Is quantum randomness algorithmic
  random? A preliminary attack,} in \emph{Proceedings 1st International
  Conference on Algebraic Informatics}, S.~Bozapalidis, A.~Kalampakas, and
  G.~Rahonis, eds., pp. 195--196 (Aristotle University of Thessaloniki, 2005).

\bibitem{wright}
R.~Wright, \enquote{Generalized urn models,} Foundations of Physics
  \textbf{20}, 881--903 (1990).

\bibitem{wright:pent}
R.~Wright, \enquote{The state of the pentagon. {A} nonclassical example,} in
  \emph{Mathematical Foundations of Quantum Theory}, A.~R. Marlow, ed., pp.
  255--274 (Academic Press, New York, 1978).

\bibitem{svozil-2001-eua}
K.~Svozil, \enquote{Logical equivalence between generalized urn models and
  finite automata,} International Journal of Theoretical Physics \textbf{44},
  745--754 (2005). \eprint{quant-ph/0209136},
  \urlprefix\url{http://dx.doi.org/10.1007/s10773-005-7052-0}.

\bibitem{e-f-moore}
E.~F. Moore, \enquote{Gedanken-Experiments on Sequential Machines,} in
  \emph{Automata Studies}, C.~E. Shannon and J.~McCarthy, eds. (Princeton
  University Press, Princeton, 1956).

\bibitem{schaller-96}
M.~Schaller and K.~Svozil, \enquote{Automaton logic,} International Journal of
  Theoretical Physics \textbf{35}(5), 911--940 (1996).

\bibitem{dvur-pul-svo}
A.~Dvure{\v{c}}enskij, S.~Pulmannov{\'{a}}, and K.~Svozil, \enquote{Partition
  Logics, Orthoalgebras and Automata,} Helvetica Physica Acta \textbf{68},
  407--428 (1995).

\bibitem{cal-sv-yu}
C.~Calude, E.~Calude, K.~Svozil, and S.~Yu, \enquote{Physical versus
  Computational Complementarity {I},} International Journal of Theoretical
  Physics \textbf{36}(7), 1495--1523 (1997). \eprint{quant-ph/9412004}.

\bibitem{svozil-2005-ln1e}
K.~Svozil, \enquote{Staging quantum cryptography with chocolate balls,}
  American Journal of Physics Vol. , No. , September \textbf{74}(9), 800--803
  (2006). \eprint{physics/0510050},
  \urlprefix\url{http://dx.doi.org/10.1119/1.2205879}.

\bibitem{specker-60}
E.~Specker, \enquote{{D}ie {L}ogik nicht gleichzeitig entscheidbarer
  {A}ussagen,} Dialectica \textbf{14}, 175--182 (1960). Reprinted in \cite[pp.
  175--182]{specker-ges}; {E}nglish translation: {\it The logic of propositions
  which are not simultaneously decidable}, reprinted in \cite[pp.
  135-140]{hooker}.

\bibitem{Aquinas}
T.~Aquinas, \emph{Summa Theologica. {T}ranslated by {F}athers of the {E}nglish
  {D}ominican {P}rovince} (Christian Classics, USA, 1981).
  \urlprefix\url{http://www.ccel.org/ccel/aquinas/summa.html}.

\bibitem{schrodinger}
E.~Schr{\"{o}}dinger, \enquote{Die gegenw{\"{a}}rtige {S}ituation in der
  {Q}uantenmechanik,} Naturwissenschaften \textbf{23}, 807--812, 823--828,
  844--849 (1935). {E}nglish translation in \cite{trimmer} and \cite[pp.
  152-167]{wheeler-Zurek:83},
  \urlprefix\url{http://wwwthep.physik.uni-mainz.de/~matschul/rot/schroedinger%
.pdf}.

\bibitem{stace}
W.~T. Stace, \enquote{The Refutation of Realism,} in \emph{Readings in
  philosophical analysis}, H.~Feigl and W.~Sellars, eds.
  (Appleton--Century--Crofts, New York, 1949). Previously published in {\em
  Mind} {\bf 53}, 1934.

\bibitem{atman:05}
H.~Atmanspacher and H.~Primas, \enquote{Epistemic and Ontic Quantum Realities,}
  in \emph{Foundations of Probability and Physics -- 3, AIP Conference
  Proceedings Volume 750}, A.~Khrennikov, ed., pp. 49--62 (Springer, New York,
  Berlin, 2005). \urlprefix\url{http://dx.doi.org/10.1063/1.1874557}.

\bibitem{Boole}
G.~Boole, \emph{An investigation of the laws of thought} (Dover edition, New
  York, 1958).

\bibitem{Boole-62}
G.~Boole, \enquote{On the theory of probabilities,} Philosophical Transactions
  of the Royal Society of London \textbf{152}, 225--252 (1862).

\bibitem{Pit-94}
I.~Pitowsky, \enquote{{G}eorge {B}oole's `Conditions od Possible Experience'
  and the Quantum Puzzle,} The British Journal for the Philosophy of Science
  \textbf{45}, 95--125 (1994).
  \urlprefix\url{http://dx.doi.org/10.1093/bjps/45.1.95}.

\bibitem{epr}
A.~Einstein, B.~Podolsky, and N.~Rosen, \enquote{Can quantum-mechanical
  description of physical reality be considered complete?} Physical Review
  \textbf{47}, 777--780 (1935).
  \urlprefix\url{http://dx.doi.org/10.1103/PhysRev.47.777}.

\bibitem{peres222}
A.~Peres, \enquote{Unperformed experiments have no results,} American Journal
  of Physics \textbf{46}, 745--747 (1978).
  \urlprefix\url{http://dx.doi.org/10.1119/1.11393}.

\bibitem{zeilinger-epr-98}
G.~Weihs, T.~Jennewein, C.~Simon, H.~Weinfurter, and A.~Zeilinger,
  \enquote{Violation of {B}ell's Inequality under Strict {E}instein Locality
  Conditions,} Phys. Rev. Lett. \textbf{81}, 5039--5043 (1998).
  \urlprefix\url{http://dx.doi.org/10.1103/PhysRevLett.81.5039}.

\bibitem{kochen1}
S.~Kochen and E.~P. Specker, \enquote{The Problem of Hidden Variables in
  Quantum Mechanics,} Journal of Mathematics and Mechanics \textbf{17}(1),
  59--87 (1967). Reprinted in \cite[pp. 235--263]{specker-ges}.

\bibitem{ZirlSchl-65}
N.~Zierler and M.~Schlessinger, \enquote{Boolean embeddings of orthomodular
  sets and quantum logic,} Duke Mathematical Journal \textbf{32}, 251--262
  (1965).

\bibitem{Alda}
V.~Alda, \enquote{On\/ {\rm 0-1} measures for projectors I,} Aplik. mate.
  \textbf{25}, 373--374 (1980).

\bibitem{Alda2}
V.~Alda, \enquote{On\/ {\rm 0-1} measures for projectors II,} Aplik. mate.
  \textbf{26}, 57--58 (1981).

\bibitem{kamber64}
F.~Kamber, \enquote{Die {S}truktur des {A}ussagenkalk{\"{u}}ls in einer
  physikalischen {T}heorie,} Nachr. Akad. Wiss. G{\"{o}}ttingen \textbf{10},
  103--124 (1964).

\bibitem{kamber65}
F.~Kamber, \enquote{Zweiwertige {W}ahrscheinlichkeitsfunktionen auf
  orthokomplement{\"{a}}ren {V}erb{\"{a}}nden,} Mathematische Annalen
  \textbf{158}, 158--196 (1965).

\bibitem{cabello-96}
A.~Cabello, J.~M. Estebaranz, and G.~Garc{\'{i}}a-Alcaine,
  \enquote{{B}ell-{K}ochen-{S}pecker theorem: A proof with 18 vectors,} Physics
  Letters A \textbf{212}(4), 183--187 (1996).
  \urlprefix\url{http://dx.doi.org/10.1016/0375-9601(96)00134-X}.

\bibitem{cabello-99}
A.~Cabello, \enquote{{K}ochen-{S}pecker theorem and experimental test on hidden
  variables,} International Journal of Modern Physics \textbf{A 15}(18),
  2813--2820 (2000). \eprint{quant-ph/9911022},
  \urlprefix\url{http://dx.doi.org/10.1142/S0217751X00002020}.

\bibitem{eccles:papal}
J.~C. Eccles, \enquote{The Mind-Brain Problem Revisited: The Microsite
  Hypothesis,} in \emph{The Principles of Design and Operation of the Brain},
  J.~C. Eccles and O.~Creutzfeldt, eds., p. 549 (Springer, Berlin, 1990).

\bibitem{popper-eccles}
K.~R. Popper and J.~C. Eccles, \emph{The Self and Its Brain} (Springer, Berlin,
  Heidelberg, London, New York, 1977).

\bibitem{franke}
P.~Frank and R.~C. (Editor), \emph{The Law of Causality and its Limits (Vienna
  Circle Collection)} (Springer, Vienna, 1997).

\bibitem{davis}
M.~Davis, \emph{The Undecidable} (Raven Press, New York, 1965).

\bibitem{chaitin2}
G.~J. Chaitin, \emph{Information, Randomness and Incompleteness}, 2nd ed.
  (World Scientific, Singapore, 1990). This is a collection of G. Chaitin's
  early publications.

\bibitem{specker-ges}
E.~Specker, \emph{Selecta} (Birkh{\"{a}}user Verlag, Basel, 1990).

\bibitem{bricmont2}
J.~Bricmont, \enquote{Science of Chaos or Chaos in Science?} in \emph{Flight
  from Science and Reason}, P.~R. Gross, N.~Levitt, and M.~W. Lewis, eds. (John
  Hopkins University Press, 1997).

\bibitem{adama02}
A.~Adamatzky, \emph{Collision-based computing} (Springer, London, 2002).

\bibitem{hooker}
C.~A. Hooker, \emph{The Logico-Algebraic Approach to Quantum Mechanics.
  {V}olume {I}: Historical Evolution} (Reidel, Dordrecht, 1975).

\bibitem{trimmer}
J.~D. Trimmer, \enquote{The present situation in quantum mechanics: a
  translation of {S}chr{\"{o}}dinger's ``cat paradox'',} Proc. Am. Phil. Soc.
  \textbf{124}, 323--338 (1980). Reprinted in \cite[pp.
  152-167]{wheeler-Zurek:83}.

\bibitem{wheeler-Zurek:83}
J.~A. Wheeler and W.~H. Zurek, \emph{Quantum Theory and Measurement} (Princeton
  University Press, Princeton, 1983).

\end{thebibliography}

\end{document}


\begin{figure}
\begin{center}
\begin{tabular}{ccccccc}
%TeXCAD Picture [1.pic]. Options:
%\grade{\on}
%\emlines{\off}
%\epic{\off}
%\beziermacro{\on}
%\reduce{\on}
%\snapping{\off}
%\quality{8.00}
%\graddiff{0.01}
%\snapasp{1}
%\zoom{4.0000}
\unitlength .4mm % = 1.14pt
\linethickness{0.4pt}
\ifx\plotpoint\undefined\newsavebox{\plotpoint}\fi % GNUPLOT compatibility
\begin{picture}(113.25,136.75)(0,0)
\qbezier(67.5,136.75)(92.5,108.88)(67.5,75.5)
\qbezier(67.5,75.5)(47.88,49.38)(66.75,26.75)
\put(74.75,136){\makebox(0,0)[lc]{cut}}
\thicklines
%\vector{dash}{1}(86.25,70.5)(53.75,86)
\put(53.75,86){\vector(-2,1){.18}}\multiput(86.07,70.32)(-.17568,.08378){5}{\line(-1,0){.17568}}
\multiput(84.32,71.16)(-.17568,.08378){5}{\line(-1,0){.17568}}
\multiput(82.56,72)(-.17568,.08378){5}{\line(-1,0){.17568}}
\multiput(80.8,72.84)(-.17568,.08378){5}{\line(-1,0){.17568}}
\multiput(79.05,73.68)(-.17568,.08378){5}{\line(-1,0){.17568}}
\multiput(77.29,74.51)(-.17568,.08378){5}{\line(-1,0){.17568}}
\multiput(75.53,75.35)(-.17568,.08378){5}{\line(-1,0){.17568}}
\multiput(73.78,76.19)(-.17568,.08378){5}{\line(-1,0){.17568}}
\multiput(72.02,77.03)(-.17568,.08378){5}{\line(-1,0){.17568}}
\multiput(70.26,77.86)(-.17568,.08378){5}{\line(-1,0){.17568}}
\multiput(68.51,78.7)(-.17568,.08378){5}{\line(-1,0){.17568}}
\multiput(66.75,79.54)(-.17568,.08378){5}{\line(-1,0){.17568}}
\multiput(64.99,80.38)(-.17568,.08378){5}{\line(-1,0){.17568}}
\multiput(63.24,81.22)(-.17568,.08378){5}{\line(-1,0){.17568}}
\multiput(61.48,82.05)(-.17568,.08378){5}{\line(-1,0){.17568}}
\multiput(59.72,82.89)(-.17568,.08378){5}{\line(-1,0){.17568}}
\multiput(57.97,83.73)(-.17568,.08378){5}{\line(-1,0){.17568}}
\multiput(56.21,84.57)(-.17568,.08378){5}{\line(-1,0){.17568}}
\multiput(54.45,85.41)(-.17568,.08378){5}{\line(-1,0){.17568}}
%\end
\thinlines
%\dottedbox(91.5,58.5)(21.75,10)
\put(91.5,58.5){\makebox(21.75,10)[cc]{object}}
\multiput(91.32,58.32)(.9886,0){23}{{\rule{.4pt}{.4pt}}}
\multiput(91.32,68.32)(.9886,0){23}{{\rule{.4pt}{.4pt}}}
\multiput(91.32,68.32)(0,-.9091){12}{{\rule{.4pt}{.4pt}}}
\multiput(113.07,68.32)(0,-.9091){12}{{\rule{.4pt}{.4pt}}}
%\end
%\dottedbox(15.25,89.25)(31.25,11.25)
\put(15.25,89.25){\makebox(31.25,11.25)[cc]{observer}}
\multiput(15.07,89.07)(.97656,0){33}{{\rule{.4pt}{.4pt}}}
\multiput(15.07,100.32)(.97656,0){33}{{\rule{.4pt}{.4pt}}}
\multiput(15.07,100.32)(0,-.9375){13}{{\rule{.4pt}{.4pt}}}
\multiput(46.32,100.32)(0,-.9375){13}{{\rule{.4pt}{.4pt}}}
%\end
\end{picture}
&
$\qquad $
&
%TeXCAD Picture [2.pic]. Options:
%\grade{\on}
%\emlines{\off}
%\epic{\off}
%\beziermacro{\on}
%\reduce{\on}
%\snapping{\off}
%\quality{8.00}
%\graddiff{0.01}
%\snapasp{1}
%\zoom{4.0000}
\unitlength 0.4mm % = 1.14pt
\linethickness{0.4pt}
\ifx\plotpoint\undefined\newsavebox{\plotpoint}\fi % GNUPLOT compatibility
\begin{picture}(129.25,150.25)(0,0)
\qbezier(67.5,136.75)(92.5,108.88)(67.5,75.5)
\qbezier(67.5,75.5)(47.88,49.38)(66.75,26.75)
\put(27.75,89.75){\makebox(0,0)[cc]{observer}}
\put(116,65.5){\makebox(0,0)[cc]{object}}
\put(74.75,136){\makebox(0,0)[lc]{cut}}
\thicklines
%\vector{dash}{1}(86.25,70.5)(53.75,86)
\put(53.75,86){\vector(-2,1){.07}}\multiput(86.18,70.43)(-.067568,.032225){13}{\line(-1,0){.067568}}
\multiput(84.42,71.27)(-.067568,.032225){13}{\line(-1,0){.067568}}
\multiput(82.67,72.11)(-.067568,.032225){13}{\line(-1,0){.067568}}
\multiput(80.91,72.94)(-.067568,.032225){13}{\line(-1,0){.067568}}
\multiput(79.15,73.78)(-.067568,.032225){13}{\line(-1,0){.067568}}
\multiput(77.4,74.62)(-.067568,.032225){13}{\line(-1,0){.067568}}
\multiput(75.64,75.46)(-.067568,.032225){13}{\line(-1,0){.067568}}
\multiput(73.88,76.29)(-.067568,.032225){13}{\line(-1,0){.067568}}
\multiput(72.13,77.13)(-.067568,.032225){13}{\line(-1,0){.067568}}
\multiput(70.37,77.97)(-.067568,.032225){13}{\line(-1,0){.067568}}
\multiput(68.61,78.81)(-.067568,.032225){13}{\line(-1,0){.067568}}
\multiput(66.86,79.65)(-.067568,.032225){13}{\line(-1,0){.067568}}
\multiput(65.1,80.48)(-.067568,.032225){13}{\line(-1,0){.067568}}
\multiput(63.34,81.32)(-.067568,.032225){13}{\line(-1,0){.067568}}
\multiput(61.59,82.16)(-.067568,.032225){13}{\line(-1,0){.067568}}
\multiput(59.83,83)(-.067568,.032225){13}{\line(-1,0){.067568}}
\multiput(58.07,83.84)(-.067568,.032225){13}{\line(-1,0){.067568}}
\multiput(56.31,84.67)(-.067568,.032225){13}{\line(-1,0){.067568}}
\multiput(54.56,85.51)(-.067568,.032225){13}{\line(-1,0){.067568}}
%\end
%\vector{dash}{1}(57.5,93.5)(89.5,78)
\put(89.5,78){\vector(2,-1){.07}}\multiput(57.43,93.43)(.066528,-.032225){13}{\line(1,0){.066528}}
\multiput(59.16,92.59)(.066528,-.032225){13}{\line(1,0){.066528}}
\multiput(60.89,91.75)(.066528,-.032225){13}{\line(1,0){.066528}}
\multiput(62.62,90.92)(.066528,-.032225){13}{\line(1,0){.066528}}
\multiput(64.35,90.08)(.066528,-.032225){13}{\line(1,0){.066528}}
\multiput(66.08,89.24)(.066528,-.032225){13}{\line(1,0){.066528}}
\multiput(67.81,88.4)(.066528,-.032225){13}{\line(1,0){.066528}}
\multiput(69.54,87.56)(.066528,-.032225){13}{\line(1,0){.066528}}
\multiput(71.27,86.73)(.066528,-.032225){13}{\line(1,0){.066528}}
\multiput(73,85.89)(.066528,-.032225){13}{\line(1,0){.066528}}
\multiput(74.73,85.05)(.066528,-.032225){13}{\line(1,0){.066528}}
\multiput(76.46,84.21)(.066528,-.032225){13}{\line(1,0){.066528}}
\multiput(78.19,83.38)(.066528,-.032225){13}{\line(1,0){.066528}}
\multiput(79.92,82.54)(.066528,-.032225){13}{\line(1,0){.066528}}
\multiput(81.65,81.7)(.066528,-.032225){13}{\line(1,0){.066528}}
\multiput(83.38,80.86)(.066528,-.032225){13}{\line(1,0){.066528}}
\multiput(85.11,80.02)(.066528,-.032225){13}{\line(1,0){.066528}}
\multiput(86.84,79.19)(.066528,-.032225){13}{\line(1,0){.066528}}
\multiput(88.56,78.35)(.066528,-.032225){13}{\line(1,0){.066528}}
%\end
\thinlines
%\dottedbox(9,11)(120.25,139.25)
\put(9,11){\makebox(120.25,139.25)[cc]{}}
\multiput(8.93,10.93)(.9938,0){122}{{\rule{.4pt}{.4pt}}}
\multiput(8.93,150.18)(.9938,0){122}{{\rule{.4pt}{.4pt}}}
\multiput(8.93,150.18)(0,-.99464){141}{{\rule{.4pt}{.4pt}}}
\multiput(129.18,150.18)(0,-.99464){141}{{\rule{.4pt}{.4pt}}}
%\end
\end{picture}
&
$\qquad $
&
%TeXCAD Picture [3.pic]. Options:
%\grade{\on}
%\emlines{\off}
%\epic{\off}
%\beziermacro{\on}
%\reduce{\on}
%\snapping{\off}
%\quality{8.00}
%\graddiff{0.01}
%\snapasp{1}
%\zoom{4.0000}
\unitlength .4mm % = 1.14pt
\linethickness{0.4pt}
\ifx\plotpoint\undefined\newsavebox{\plotpoint}\fi % GNUPLOT compatibility
\begin{picture}(129.25,150.25)(0,0)
\qbezier(67.5,136.75)(92.5,108.88)(67.5,75.5)
\qbezier(67.5,75.5)(47.88,49.38)(66.75,26.75)
\put(27.75,89.75){\makebox(0,0)[cc]{observer}}
\put(116,65.5){\makebox(0,0)[cc]{object}}
\put(74.75,136){\makebox(0,0)[lc]{cut}}
\thicklines
%\vector{dash}{1}(86.25,70.5)(53.75,86)
\put(53.75,86){\vector(-2,1){.18}}\multiput(86.07,70.32)(-.17568,.08378){5}{\line(-1,0){.17568}}
\multiput(84.32,71.16)(-.17568,.08378){5}{\line(-1,0){.17568}}
\multiput(82.56,72)(-.17568,.08378){5}{\line(-1,0){.17568}}
\multiput(80.8,72.84)(-.17568,.08378){5}{\line(-1,0){.17568}}
\multiput(79.05,73.68)(-.17568,.08378){5}{\line(-1,0){.17568}}
\multiput(77.29,74.51)(-.17568,.08378){5}{\line(-1,0){.17568}}
\multiput(75.53,75.35)(-.17568,.08378){5}{\line(-1,0){.17568}}
\multiput(73.78,76.19)(-.17568,.08378){5}{\line(-1,0){.17568}}
\multiput(72.02,77.03)(-.17568,.08378){5}{\line(-1,0){.17568}}
\multiput(70.26,77.86)(-.17568,.08378){5}{\line(-1,0){.17568}}
\multiput(68.51,78.7)(-.17568,.08378){5}{\line(-1,0){.17568}}
\multiput(66.75,79.54)(-.17568,.08378){5}{\line(-1,0){.17568}}
\multiput(64.99,80.38)(-.17568,.08378){5}{\line(-1,0){.17568}}
\multiput(63.24,81.22)(-.17568,.08378){5}{\line(-1,0){.17568}}
\multiput(61.48,82.05)(-.17568,.08378){5}{\line(-1,0){.17568}}
\multiput(59.72,82.89)(-.17568,.08378){5}{\line(-1,0){.17568}}
\multiput(57.97,83.73)(-.17568,.08378){5}{\line(-1,0){.17568}}
\multiput(56.21,84.57)(-.17568,.08378){5}{\line(-1,0){.17568}}
\multiput(54.45,85.41)(-.17568,.08378){5}{\line(-1,0){.17568}}
%\end
%\vector{dash}{1}(57.5,93.5)(89.5,78)
\put(89.5,78){\vector(2,-1){.18}}\multiput(57.32,93.32)(.17297,-.08378){5}{\line(1,0){.17297}}
\multiput(59.05,92.49)(.17297,-.08378){5}{\line(1,0){.17297}}
\multiput(60.78,91.65)(.17297,-.08378){5}{\line(1,0){.17297}}
\multiput(62.51,90.81)(.17297,-.08378){5}{\line(1,0){.17297}}
\multiput(64.24,89.97)(.17297,-.08378){5}{\line(1,0){.17297}}
\multiput(65.97,89.14)(.17297,-.08378){5}{\line(1,0){.17297}}
\multiput(67.7,88.3)(.17297,-.08378){5}{\line(1,0){.17297}}
\multiput(69.43,87.46)(.17297,-.08378){5}{\line(1,0){.17297}}
\multiput(71.16,86.62)(.17297,-.08378){5}{\line(1,0){.17297}}
\multiput(72.89,85.78)(.17297,-.08378){5}{\line(1,0){.17297}}
\multiput(74.62,84.95)(.17297,-.08378){5}{\line(1,0){.17297}}
\multiput(76.35,84.11)(.17297,-.08378){5}{\line(1,0){.17297}}
\multiput(78.08,83.27)(.17297,-.08378){5}{\line(1,0){.17297}}
\multiput(79.81,82.43)(.17297,-.08378){5}{\line(1,0){.17297}}
\multiput(81.54,81.59)(.17297,-.08378){5}{\line(1,0){.17297}}
\multiput(83.27,80.76)(.17297,-.08378){5}{\line(1,0){.17297}}
\multiput(85,79.92)(.17297,-.08378){5}{\line(1,0){.17297}}
\multiput(86.73,79.08)(.17297,-.08378){5}{\line(1,0){.17297}}
\multiput(88.46,78.24)(.17297,-.08378){5}{\line(1,0){.17297}}
%\end
\thinlines
%\dottedbox(9,11)(120.25,139.25)
\put(9,11){\makebox(120.25,139.25)[cc]{}}
\multiput(8.82,10.82)(.9938,0){122}{{\rule{.4pt}{.4pt}}}
\multiput(8.82,150.07)(.9938,0){122}{{\rule{.4pt}{.4pt}}}
\multiput(8.82,150.07)(0,-.99464){141}{{\rule{.4pt}{.4pt}}}
\multiput(129.07,150.07)(0,-.99464){141}{{\rule{.4pt}{.4pt}}}
%\end
\put(16.25,141.5){\makebox(0,0)[lc]{cell 1}}
\put(119.75,140.75){\makebox(0,0)[rc]{cell 2}}
\put(14.5,18){\makebox(0,0)[lc]{cell 3}}
\put(120.5,18){\makebox(0,0)[rc]{cell 4}}
%\dottedline(69,11.25)(69,150)
\multiput(68.82,11.07)(0,.9982){140}{{\rule{.4pt}{.4pt}}}
%\end
%\dottedline(9.25,81)(129.25,81)
\multiput(9.07,80.82)(.99174,0){122}{{\rule{.4pt}{.4pt}}}
%\end
\end{picture}
\\
a)&&b)&&c)
\end{tabular}
\end{center}
\caption{Relation between observer and observed object:
a) clear-cut distinction between observer and object;
b) both observers are embedded in the same system; there is a bidirectional exchange between observer and object.
In the latter case, the cut is merely conventional;
c) tessellation of the system into four cells.
\label{2007-miracles-fc}}
\end{figure}

We note in by-passing that certain quantum mechanical measurement concept are based on
a classical measurement apparatus which has no detailed quantum mechanical description.
It serves as an irreversible instance; that is, a reconstruction \cite{greenberger2,hkwz} of the
original quantum mechanical state is impossible after any such measurement.
At all other times, the evolution of the quantum mechanical state is reversible.
Suppose that one cannot exclude the principal possibility
of an all-encompassing quantum mechanical
description.
Then it suffices to consider the
formerly ``classical'' measurement apparatus and the object to be part of
a single quantized system to render this entire system to behave reversibly
and thus in contradiction with the assumption of irreversible measurements \cite{everett}.
This has lead to speculations \cite{wigner:mb} that only transcendental consciousness
could limit quantum reversibility.

For intrinsic observers, reversibility puts constraints on the measurement process.
One of the most striking restrictions is due to the impossibility to copy the state of
the object subsystem without changing the state of the observer subsystem
such that for the reconstruction of the original state of the object subsystem
all information has to be re-processed,
leaving not a trace or impression of the copied state at the observer.
To put it differently, in order to reconstruct the original state of the object,
all measurements on the observer's side have to be used.

For the sake of formalization, consider a tessellation of
the entire system
into finite sized cells which are numbered as depicted in Fig.~\ref{2007-miracles-fc}c).
Suppose that the observer is located at cell number 1 and is in state number $s_1$ originally,
whereas the object is located in cell number 4 and is in state number $s_4$ originally.
We may write all states into a pseudo-vector form
$(s_1,s_2,s_3,s_4)^T$, where the superscript $T$ denotes transposition.
Now there are $4!$ permutation of four elements,
some of them are capable of exchanging state information
from the first to the fourth cell,
among them the permutation representable by the permutation matrix
$\left(
\begin{array}{cccc} 0&0&0&1\\ 0&1&0&0\\ 0&0&1&0\\ 1&0&0&0 \end{array}
\right)$
such that
$\left(
\begin{array}{cccc} 0&0&0&1\\ 0&1&0&0\\ 0&0&1&0\\ 1&0&0&0
\end{array}
\right)
\left(
\begin{array}{c}
s_1\\s_2\\ s_3\\ s_4\end{array}
\right)
=
\left(
\begin{array}{c}
s_4\\s_2\\ s_3 \\ s_1
\end{array}
\right)
$.
If only permutations are allowed, there is no possibility to copy the state $s_4$
into any other cell without dislocating it from cell 4; i.e.,
by state exchange.
Note that copying is not excluded in a wider framework of universal computation
\cite{maxwell-demon}
allowing not only exchanges of cell states but also
algebraic manipulations among cell values.
