%%tth:\begin{html}<LINK REL=STYLESHEET HREF="/~svozil/ssh.css">\end{html}
\documentstyle[12pt]{article}
%\documentstyle[amsfonts]{article}
%\RequirePackage{times}
%\RequirePackage{courier}
%\RequirePackage{mathptm}
%\renewcommand{\baselinestretch}{1.3}
\begin{document}

%\def\frak{\cal }
%\def\Bbb{\bf }
\sloppy



\title{Censorship and the peer review system}
\author{Karl Svozil\\
 {\small Institut f\"ur Theoretische Physik, University of Technology Vienna }     \\
  {\small Wiedner Hauptstra\ss e 8-10/136,}
  {\small A-1040 Vienna, Austria   }            \\
  {\small e-mail: svozil@tuwien.ac.at}}
\date{ }
\maketitle

%\begin{flushright}
%{\scriptsize http://tph.tuwien.ac.at/$\widetilde{\;\;}\,$svozil/publ/2000-cesena.$\{$htm,ps,tex$\}$}
%\end{flushright}

\begin{abstract}
This is a critical essay against the so-called ``peer review system'' in science.
It is argued that, despite all its obvious advantages for quality control,
the method is outdated and in some cases even counterproductive.
New, more interactive methods, are proposed for the communication of scientific research
which seem to be more appropriate for an effective dissemination  than the old paper journals.
Sooner or later, electronic interactive scientific media
will become a de facto standard in science proper.
It is further argued that the peer review system for distributing money to the scientific
community
is ineffective and wasteful and needs to be substantially overhowled
even against massive resentment by the ``peers.''
It is proposed to spend, at least for a limited time period,
additional money (e.g., ten percent of the ``peered'' money)
through two totally different channels:
(i) through the decision of lay assesors (a suggestion already put forward by Feyerabend)
and
(ii) by absolutely random choice. The scientific output of all three channels, peered, laymen, and random
should then be publicly compared by the media.
\end{abstract}


Journals are not willing to implement the very standard of scientific conduct they are requesting from researchrs; in particular double blind refereeing procedures.

\end{document}
