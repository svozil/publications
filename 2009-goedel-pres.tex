%\documentclass[pra,showpacs,showkeys,amsfonts,amsmath,twocolumn,handou]{revtex4}
\documentclass[slidestop,compress,mathserif,blackandwhite]{beamer}
%\documentclass[amsmath,red,table,sans,handout]{beamer}
%\documentclass[pra,showpacs,showkeys,amsfonts]{revtex4}
%\documentclass[pra,showpacs,showkeys,amsfonts]{revtex4}
%\usepackage[utf8]{inputenc}
\usepackage[T1]{fontenc}
\usepackage{fancybox}
%%\usepackage{beamerthemeshadow}
\usepackage[headheight=1pt,footheight=10pt]{beamerthemeboxes}
\addfootboxtemplate{\color{structure!80}}{\color{white}\tiny \hfill Karl Svozil (TU Vienna)\hfill}
\addfootboxtemplate{\color{structure!65}}{\color{white}\tiny \hfill Undecidability in Physics\hfill}
\addfootboxtemplate{\color{structure!50}}{\color{white}\tiny \hfill December 12, 2009, Vienna, Austria\hfill}
%\usepackage[dark]{beamerthemesidebar}
%\usepackage[headheight=24pt,footheight=12pt]{beamerthemesplit}
%\usepackage{beamerthemesplit}
%\usepackage[bar]{beamerthemetree}
\usepackage{graphicx}
\usepackage{pgf}
%\usepackage[usenames]{color}
%\newcommand{\Red}{\color{Red}}  %(VERY-Approx.PANTONE-RED)
%\newcommand{\Green}{\color{Green}}  %(VERY-Approx.PANTONE-GREEN)

%\RequirePackage[german]{babel}
%\selectlanguage{german}
%\RequirePackage[isolatin]{inputenc}

\pgfdeclareimage[height=0.5cm]{logo}{TU_Signet_SW_rgb}
\logo{\pgfuseimage{logo}}
\beamertemplatetriangleitem
%\beamertemplateballitem

\beamerboxesdeclarecolorscheme{alert}{red}{red!15!averagebackgroundcolor}
\beamerboxesdeclarecolorscheme{alert2}{purple}{purple!15!averagebackgroundcolor}
%\begin{beamerboxesrounded}[scheme=alert,shadow=true]{}
%\end{beamerboxesrounded}

%\beamersetaveragebackground{green!10}

%\beamertemplatecircleminiframe

\newcounter{nc}[part]
\setcounter{nc}{1}
\begin{document}

\title{\bf \textcolor{purple}{Undecidability in Physics}}
%\subtitle{Naturwissenschaftlich-Humanisticher Tag am BG 19\\Weltbild und Wissenschaft\\http://tph.tuwien.ac.at/\~{}svozil/publ/2005-BG18-pres.pdf}
\subtitle{\textcolor{structure!65}{\small http://tph.tuwien.ac.at/$\sim$svozil/publ/2009-goedel-pres.pdf}
\\ \textcolor{structure!50}{\footnotesize http://tph.tuwien.ac.at/~svozil/publ/2007-miracles.pdf}
}
\author{Karl Svozil}
\institute{Institut f\"ur Theoretische Physik, Vienna University of Technology, \\
Wiedner Hauptstra\ss e 8-10/136, A-1040 Vienna, Austria\\
svozil@tuwien.ac.at
%{\tiny Disclaimer: Die hier vertretenen Meinungen des Autors verstehen sich als Diskussionsbeitr�ge und decken sich nicht notwendigerweise mit den Positionen der Technischen Universit�t Wien oder deren Vertreter.}
}
\date{17. Symposion Kulturraum-Donau \\
Kurt G�del oder die Unvollst�ndigkeit der Wissenschaft?\\
am 12. und 13. Dezember 2009 \\
Wiener Urania, Dachsaal, Uraniastra\ss e 1, 1010 Wien
\\   $\,$  \\
\textcolor{structure!65}{\footnotesize In commemoration of Erich Wolfgang Korngold (1897---1957), \\ creator of "Das Wunder der Heliane" (Op.20, 1927)} }
\maketitle



\frame{
\frametitle{Contents}
\tableofcontents
}


\section{Provable physical unknowables}

\frame{
\begin{center}
$\;$\\
$\;$\\
$\;$\\
$\;$\\
{\color{purple}
\Huge Part \Roman{nc}: \\
Provable physical unknowables}
\end{center}
\addtocounter{nc}{1}
\begin{center}{\color{lime}
$\widetilde{\qquad \qquad }$
$\widetilde{\qquad \qquad}$
$\widetilde{\qquad \qquad }$ }
\end{center}
 }


\frame{
\frametitle{Provable physical unknowables ...}
\begin{itemize}

\item<1->
by ``reduction'' to the halting problem
\item<1->
{\it via} embedding a Turing machine into the physical universe
\item<1->
A straightforward embedding of a universal computer
into a physical system results in the fact that,
due to the reduction to the recursive undecidability of the halting problem,
certain future events or tasks cannot be forecasted or performed
and are thus provable indeterministic or impossible.
\item<1->
Here ``reduction'' means that physical undecidability is linked or reduced
to logical undecidability.
\end{itemize}
}

\subsection{Intrinsic self-referential observers}

\frame{
\frametitle{Paradoxical self-reference of intrinsic observers}
\begin{itemize}

\item<1->
Every physical observation is essentially discrete,  finite and  self-referential.
\item<1->
Intrinsic observers face all kinds of self-referential situations;
among the most interesting are paradoxical self-referential statements.
\item<1->
G\"odel states,
 \begin{quote}
 {\it
 ``$\ldots$ that a complete epistemological description
 of a language $A$ cannot be given in the same language $A$, because
 the concept of truth of sentences of $A$ cannot be defined in $A$. It
 is this theorem which is the true reason for the existence of
 undecidable propositions in the formal systems containing arithmetic.''}
 \end{quote}
\end{itemize}
}

\subsection{Undecidability of the general forecasting problem}

\frame[shrink=2]{
\frametitle{Undecidability of the general forecasting problem}

\textcolor{purple!60}{``Executive summary:'' Assuming total forecasting power \& free will (omnipotence), it would be possible to counter-act
the forecast, thereby invalidating it.}


\begin{block}<+->{{\footnotesize Algorithmic proof:}\\ \tiny
Consider a universal computer $U$ and an arbitrary algorithm
$B(X)$ whose input is a string of symbols $X$.  Assume that there exists
a ``halting algorithm'' ${\tt HALT}$ which is able to decide whether $B$
terminates on $X$ or not.
The domain of ${\tt HALT}$  is the set of legal programs.
The range of ${\tt HALT}$ are classical bits.
$\;$\\
$\;$\\
Using ${\tt HALT}(B(X))$ we shall construct another deterministic
computing agent $A$, which has as input any effective program $B$ and
which proceeds as follows:  Upon reading the program $B$ as input, $A$
makes a copy of it.  This can be readily achieved, since the program $B$
is presented to $A$ in some encoded form
$\ulcorner B\urcorner $,
i.e., as a string of
symbols.  In the next step, the agent uses the code
$\ulcorner B\urcorner $
 as input
string for $B$ itself; i.e., $A$ forms  $B(\ulcorner B\urcorner )$,
henceforth denoted by
$B(B)$.  The agent now hands $B(B)$ over to its subroutine ${\tt HALT}$.
Then, $A$ proceeds as follows:  if ${\tt HALT}(B(B))$ decides that
$B(B)$ halts, then the agent $A$ does not halt; this can for instance be
realized by an infinite {\tt DO}-loop; if ${\tt HALT}(B(B))$ decides
that $B(B)$ does {\em not} halt, then $A$ halts.
$\;$\\
$\;$\\
The agent $A$ will now be confronted with the following paradoxical
task:  take the own code as input and proceed to determine whether or not it halts.
Then, whenever $A(A)$
halts, ${\tt HALT}(A(A))$, by the definition of $A$, would force $A(A)$ not to halt.
Conversely,
whenever $A(A)$ does not halt, then ${\tt HALT}(A(A))$ would steer
$A(A)$ into the halting mode.  In both cases one arrives at a complete
contradiction.  Classically, this contradiction can only be consistently
avoided by assuming the nonexistence of $A$ and, since the only
nontrivial feature of $A$ is the use of the peculiar halting algorithm
${\tt HALT}$, the impossibility of any such halting algorithm.
}
\end{block}


}

\subsection{The busy beaver function as the maximal recurrence time}

\frame{
\frametitle{The busy beaver function as the maximal recurrence time}
\begin{itemize}

\item<1->
What is the biggest number $\Sigma (n)$ producible by  a program of length $n$ before halting?
$\Sigma$ is called the {\it busy beaver function}.
\item<1->
$\Sigma (n)$ grows faster than any computable function of $n$; its first values are known or estimated by:
  $\Sigma _T(1)=1$,
 $\Sigma _T(2)= 4$,
  $\Sigma _T(3)=6$,
 $\Sigma _T(4)= 13$,
 $\Sigma _T(5) \ge 1915$,
 $\Sigma_T(7)\ge 22961$,
 $\Sigma_T(8)\ge 3\cdot (7\cdot 3^{92}-1)/2$.
\item<1->
Consider a related question: what is the upper bound of running time --- or,
alternatively, recurrence time --- of a program of length $n$ bits before
terminating, or, alternatively, recurring?
\item<1->
An answer to that question confers a feeling of how long we have to
wait for the most time-consuming program of length $n$ bits to
hold. This bound can be represented by the busy beaver function:
${\tt TMAX}(n)=\Sigma (n+O(1))$.
\end{itemize}
}

\subsection{Undecidability of the induction problem}

\frame{
\frametitle{Undecidability of the induction problem}

\textcolor{purple!60}{``Executive summary:'' We just cannot be sure whether we have waited ``long enough''  that some law we have guessed is correct.}\\$\;$ \\


Induction in physics is the inference of general rules
dominating and generating physical behaviors from these behaviors.
$\;$\\
$\;$\\
For any deterministic system strong enough to support
universal computation, the general induction problem
is provable unsolvable.
$\;$\\
Induction is thereby reduced to the unsolvability of
the rule inference problem
of identifying a rule or law reproducing the behavior of a deterministic system
by observing its input/output performance by purely algorithmic means;
and ultimately to the halting problem.
}

\subsection{Results in classical recursion theory with implications for physics}

\frame[squeeze]{
\frametitle{Results in classical recursion theory with implications for physics. On what ``G�del believed''}
{\small
\begin{itemize}

\item<1->
There exist recursive monotone bounded sequences of rational numbers
whose limit is no computable number~(Specker  1957).
A concrete example of such a number is Chaitin's Omega number,
the ``halting probability'' for a computer (using prefix-free code),
which can be defined by a sequence of rational numbers
with no computable rate of convergence.

\item<1->
There exist a recursive real function which has its maximum in the unit interval
at no recursive real number~(Specker  1957, Kreisel 1974).
This has implication for the principle of least action.

\item<1->
There exists a real number $r$ such that $G(r) = 0$ is recursively undecidable for $G(x)$
in a class of functions which involves polynomials and the sine function
(Wang 1974).
This again has some bearing on  the principle of least action.

\item<1->
There exist incomputable solutions of the wave equations for computable initial values (Pour-El \& Richards 1980's)
\end{itemize}
}

}
\section{Behavior of three or more classical bodies}

\frame{
\begin{center}
$\;$\\
$\;$\\
$\;$\\
$\;$\\
{\color{purple}
\Huge Part \Roman{nc}: \\
Behavior of three or more classical bodies}
\end{center}
\addtocounter{nc}{1}
\begin{center}{\color{lime}
$\widetilde{\qquad \qquad }$
$\widetilde{\qquad \qquad}$
$\widetilde{\qquad \qquad }$ }
\end{center}
 }


\frame{
\frametitle{Behavior of three or more classical bodies}
{\tiny
\begin{itemize}
\item<1->
Laplace:
\begin{quote}
{\it
Present events are connected with preceding ones
by a tie based upon the evident principle that a thing
cannot occur without a cause which produces it. This
axiom, known by the name of the principle of sufficient
reason, extends even to actions which are considered
indifferent $\ldots$


We ought then to regard the present state of the
universe as the effect of its anterior state and as the
cause of the one which is to follow. Given for one
instant an intelligence which could comprehend all the
forces by which nature is animated and the respective
situation of the beings who compose it an intelligence
sufficiently vast to submit these data to analysis it
would embrace in the same formula the movements of
the greatest bodies of the universe and those of the
lightest atom; for it, nothing would be uncertain and
the future, as the past, would be present to its eyes.}
\end{quote}

\item<1->  Poincar{\'e}:
\begin{quote}
{\it
W\"urden wir die Gesetze der Natur und den Zustand des Universums
f\"ur einen gewissen Zeitpunkt genau kennen, so
k\"onnten wir den Zustand dieses Universums f\"ur
irgendeinen sp\"ateren Zeitpunkt genau voraussagen.
Aber
[[~$\ldots$~]]
 es kann der Fall eintreten,
da\ss $\;$ kleine Unterschiede in den Anfangsbedingungen
gro\ss e Unterschiede in den sp\"ateren Erscheinungen bedingen;
ein kleiner Irrtum in den ersteren kann einen au\ss erordentlich gro\ss en
Irrtum f\"ur den letzteren nach sich ziehen.
Die Vorhersage wird unm\"oglich und wir haben eine
``zuf\"allige Erscheinung''.\\
$\;$ \\
English translation:
If we would know the laws of Nature and the state of the Universe precisely
for a certain time,
we would be able to predict with certainty
the state of the Universe for any later time.
But
[[~$\ldots$~]]
it can be the case that small differences in the initial values
produce great differences in the later phenomena;
a small error in the former may result in a large error in the latter.
The prediction becomes impossible and we have a ``random phenomenon.''
}
\end{quote}
\end{itemize}
}
}


\subsection{Deterministic chaos}

\frame{
\frametitle{Deterministic chaos}
Poincar{\'e}'s recognition of possible instabilities
in $n$-body problems was the first indication of what today is called ``deterministic chaos.''
In chaotic systems it is practically impossible to specify
the initial value precise enough to allow long-term predictions.
$\;$ \\
$\;$ \\
A stronger assumption supposes that the initial values are elements of
a continuum, and thus ``almost all'' (of measure one) of them
are not representable by any algorithmically compressible number;
in short, that they are random.
$\;$ \\
$\;$ \\
Classical, deterministic chaos results from ``unfolding'' such a random initial value
drawn from the ``continuum urn'' by a recursive, deterministic function.
}

\subsection{Convergence of the general solution of the $N$-body problem}

\frame{
\frametitle{Convergence of the general solution of the $N$-body problem}
{\small
\begin{itemize}

\item<1->   Classical $N$-body problem:
\begin{quote}
{\it
Given a system of arbitrarily many mass points that attract each other
according to Newton's law, under the assumption that no two points ever collide,
try to find a representation of the coordinates of each point
as a series in a variable that is some known function of time and for
all of whose values the series converges uniformly.
}
\end{quote}
\item<1->
More than one hundred years after its formulation as quoted above,
the $n$--body problem has been solved by Wang in 1991.
The $3$--body problem was already solved in 1912 by Sundman.
The solutions are given in terms of power series.
\item<1->
What is the convergence of the series solutions?
Suppose we are able to construct a universal computer based on the $n$--body problem.
This can, for instance, be achieved by ballistic computation, such as the
``Billiard Ball'' model of computation
which effectively ``embeds'' a universal computer into a system of $n$--bodies.
It follows by reduction that certain predictions are impossible.
\end{itemize}
}
}

\section{Quantum unknowables}

\frame{
\begin{center}
$\;$\\
$\;$\\
$\;$\\
$\;$\\
{\color{purple}
\Huge Part \Roman{nc}: \\
Quantum unknowables }
\end{center}
\addtocounter{nc}{1}
\begin{center}{\color{lime}
$\widetilde{\qquad \qquad }$
$\widetilde{\qquad \qquad}$
$\widetilde{\qquad \qquad }$ }
\end{center}
 }


\frame{
\frametitle{``Brainwashed'' by Einstein against quantum theory, G\"odel threw Wheeler ``out of his office''}


\begin{itemize}

\item<1->
Several attempts to ``translate'' or ``reduce'' G\"odel-type incompleteness (in)to the quantum diomain; none successful;
maybe this is a ``red herring?''
\item<1->
Computational complementarity (Moore 1956) and generalized Urn Models (Wright  1978)

\begin{center}
\pgfdeclareimage[height=4cm]{2009-qchocolate-box}{2009-qchocolate-box}
\pgfuseimage{2009-qchocolate-box}
\end{center}
\end{itemize}
}

\subsection{Quantum indeterminacy I: Random individual quantum events}

\frame{
\frametitle{Quantum indeterminacy I: Random individual quantum events}
{\tiny
\begin{itemize}
\item<1->
Born 1926:
\begin{quote}
{\it
 ``Vom Standpunkt unserer Quantenmechanik gibt es keine Gr\"o\ss e, die im {\em Einzelfalle} den Effekts eines Sto\ss es
kausal festlegt; aber auch in der Erfahrung haben wir keinen Anhaltspunkt daf\"ur, da\ss~ es innere Eigenschaften
der Atome gibt, die einen bestimmten Sto\ss erfolg bedingen.
Sollen wir hoffen, sp\"ater solche Eigenschaften
[[$\ldots$]] zu entdecken und im Einzelfalle zu bestimmen?
Oder sollen wir glauben, dass die \"Ubereinstimmung von Theorie und Erfahrung
in der Unf\"ahigkeit, Bedingungen f\"ur den kausalen Ablauf anzugeben, eine pr\"astabilisierte Harmonie ist,
die auf der Nichtexistenz solcher Bedingungen beruht?
Ich selber neige dazu,die Determiniertheit in der atomaren Welt aufzugeben.''


 ``Die Bewegung der Partikel folgt Wahrscheinlichkeitsgesetzen,
die Wahrscheinlichkeit selbst aber breitet sich im Einklang mit dem Kausalgesetz  aus.
[Das hei\ss t, da\ss~ die Kenntnis des Zustandes in allen Punkten in einem Augenblick
die Verteilung des Zustandes zu allen sp{\"a}teren Zeiten festlegt.]''}
\end{quote}

\item<1-> Zeilinger 2005:
\begin{quote}
{\it
 ``the discovery that individual events are
irreducibly random is probably one of the
most significant findings of the twentieth
century. [[$\ldots$]]~for the individual event in quantum physics, not only do we not know the cause, there is no cause.''
}
\end{quote}
\item<1->
No type of randomness is specified; e.g., its relation to ``classical randomness.''
Due to recursion theoretic restrictions, it is impossible to ``formally prove'' this type of randomness by observing finite sequences alone.
\end{itemize}
}
}

\subsection{Quantum indeterminacy II: Complementarity}

\frame{
\frametitle{Quantum indeterminacy II: Complementarity}
{\tiny
\begin{itemize}
\item<1->
Pauli 1933:
\begin{quote}
{\it
 ``Bei der Unbestimmtheit einer Eigenschaft eines Systems bei einer bestimmten Anordnung
(bei einem bestimmten Zustand eines Systems) vernichtet jeder Versuch, die betreffende Eigenschaft zu messen,
(mindestens teilweise) den Einflu\ss~
der fr{\"u}heren Kenntnisse vom System auf die (eventuell statistischen) Aussagen
{\"u}ber sp{\"a}tere m{\"o}gliche Messungsergebnisse.
[[$\ldots$]]
So m{\"u}ssen, um den Ort eines Teilchens zu bestimmen und um seinen Impuls zu bestimmen,
{\em einander ausschlie\ss ende Versuchsanordnungen benutzt werden.}
[[$\ldots$]]
Die Beeinflussung des Systems durch den Messaparat f{\"u}r den Impuls (Ort)
ist eine solche, da\ss~ innerhalb der durch die Ungenauigkeitsrelationen gegebenen Grenzen
die Benutzbarkeit der fr{\"u}heren Orts- (Impuls-)
Kenntnis f{\"u}r die Voraussagbarkeit der Ergebnisse sp{\"a}terer Orts- (Impuls-) Messungen verlorengegangen ist.
Wenn aus diesem Grunde die Benutzbarkeit {\em eines} klassischen Begriffes in einem
ausschlie\ss enden Verh{\"a}ltnis zu einem {\em anderen} steht, nennen wir diese beiden Begriffe (z.B. Orts- und
Impulskoordinaten eines Teilchens) mit Bohr {\em komplement{\"a}r.}'' }
\end{quote}

\item<1->  Dirac 1927:
\begin{quote}
{\it
 ``$\ldots$ Causality applies only to a system which is
left undisturbed. If a system is small, we cannot observe it without
producing a serious disturbance and hence we cannot expect to find
any causal connexion between the results of our observations.
Causality will still be assumed to apply to undisturbed systems and
the equations which will be set up to describe an undisturbed system
will be differential equations expressing a causal connexion between
conditions at one time and conditions at a later time. $\ldots$ There is an unavoidable indeterminacy in the calculation
of observational results, the theory enabling us to calculate in
general only the probability of our obtaining a particular result when
we make an observation.''
}
\end{quote}
\end{itemize}
}
}

\subsection{Quantum indeterminacy III: Value indefiniteness versus omniscience}

\frame{
\frametitle{Quantum indeterminacy III: Value indefiniteness versus omniscience}


\begin{center}
%TeXCAD Picture [4.pic]. Options:
%\grade{\on}
%\emlines{\off}
%\epic{\off}
%\beziermacro{\on}
%\reduce{\on}
%\snapping{\off}
%\quality{8.00}
%\graddiff{0.01}
%\snapasp{1}
%\zoom{5.6569}
\unitlength 0.3mm % = 2.85pt
\linethickness{0.7pt}
\ifx\plotpoint\undefined\newsavebox{\plotpoint}\fi % GNUPLOT compatibility
\begin{picture}(134.09,125.99)(0,0)
%\emline(86.39,101.96)(111.39,58.46)
\multiput(86.39,101.96)(.119617225,-.208133971){209}{{\color{green}\line(0,-1){.208133971}}}
%\end
%\emline(86.39,14.96)(111.39,58.46)
\multiput(86.39,14.96)(.119617225,.208133971){209}{{\color{red}\line(0,1){.208133971}}}
%\end
%\emline(36.47,101.96)(11.47,58.46)
\multiput(36.47,101.96)(-.119617225,-.208133971){209}{{\color{brown}\line(0,-1){.208133971}}}
%\end
%\emline(36.47,14.96)(11.47,58.46)
\multiput(36.47,14.96)(-.119617225,.208133971){209}{{\color{pink}\line(0,1){.208133971}}}
%\end
\color{blue}\put(86.39,15.21){\color{blue}\line(-1,0){50}}
\put(86.39,101.71){\color{violet}\line(-1,0){50}}
%
\put(36.34,15.16){\color{pink}\circle{6}}
\put(36.34,15.16){\color{blue}\circle{4}}
\put(52.99,15.16){\color{blue}\circle{4}}
\put(52.99,15.16){\color{cyan}\circle{6}}
\put(69.68,15.16){\color{blue}\circle{4}}
\put(69.68,15.16){\color{orange}\circle{6}}
\put(86.28,15.16){\color{blue}\circle{4}}
\put(86.28,15.16){\color{red}\circle{6}}
%
\put(93.53,27.71){\color{red}\circle{4}}
\put(93.53,27.71){\color{orange}\circle{6}}
\put(102.37,43.44){\color{red}\circle{4}}
\put(102.37,43.44){\color{olive}\circle{6}}
\put(111.21,58.45){\color{red}\circle{4}}
\color{green}\put(111.21,58.45){\circle{6}}
%
\put(102.37,73.47){\color{green}\circle{4}}
\put(102.37,73.47){\color{olive}\circle{6}}
\put(93.53,89.21){\color{green}\circle{4}}
\put(93.53,89.21){\color{cyan}\circle{6}}
\put(86.28,101.76){\color{green}\circle{4}}
\put(86.28,101.76){\color{violet}\circle{6}}
%
\put(69.68,101.76){\color{violet}\circle{4}}
\put(69.68,101.76){\color{cyan}\circle{6}}
\put(52.99,101.76){\color{violet}\circle{4}}
\put(52.99,101.76){\color{orange}\circle{6}}
\put(36.34,101.76){\color{violet}\circle{4}}
\put(36.34,101.76){\color{brown}\circle{6}}
%
\put(29.24,89.21){\color{brown}\circle{4}}
\put(29.24,89.21){\color{orange}\circle{6}}
\put(20.4,73.47){\color{brown}\circle{4}}
\put(20.4,73.47){\color{olive}\circle{6}}
\put(11.56,58.45){\color{brown}\circle{4}}
\put(11.56,58.45){\color{pink}\circle{6}}

\put(20.4,43.44){\color{pink}\circle{4}}
\put(20.4,43.44){\color{olive}\circle{6}}
\put(29.24,27.71){\color{pink}\circle{4}}
\put(29.24,27.71){\color{cyan}\circle{6}}

\color{cyan}
\qbezier(29.2,27.73)(23.55,-5.86)(52.99,15.24)
\qbezier(29.2,27.88)(36.93,75)(69.63,101.91)
\qbezier(52.69,15.24)(87.47,40.96)(93.72,89.27)
\qbezier(93.72,89.27)(98.4,125.99)(69.49,102.06)
\color{orange}
\qbezier(93.57,27.73)(99.22,-5.86)(69.78,15.24)
\qbezier(93.57,27.88)(85.84,75)(53.13,101.91)
\qbezier(70.08,15.24)(35.3,40.96)(29.05,89.27)
\qbezier(29.05,89.27)(24.37,125.99)(53.28,102.06)
\color{olive}
\qbezier(20.15,73.72)(-11.67,58.52)(20.15,43.31)
\qbezier(20.33,73.72)(61.34,93.16)(102.36,73.72)
\qbezier(102.36,73.72)(134.09,58.52)(102.53,43.31)
\qbezier(102.53,43.31)(60.99,23.43)(20.15,43.49)
{\color{black} \tiny
\put(12.41,116.02){\makebox(0,0)[rc]{$(0,1,-1,0)$}}
\put(12.41,2.65){\makebox(0,0)[rc]{$(0,0,1,-1)$}}
\put(58.68,116.38){\makebox(0,0)[rc]{$(1,0,0,1)$}}
\put(58.68,2.3){\makebox(0,0)[rc]{$(1,-1,0,0)$}}
\put(115.93,116.2){\makebox(0,0)[lc]{$(-1,1,1,1)$}}
\put(115.93,2.48){\makebox(0,0)[lc]{$(1,1,1,1)$}}
\put(65.65,116.38){\makebox(0,0)[lc]{$(1,1,1,-1)$}}
\put(65.65,2.3){\makebox(0,0)[lc]{$(1,1,-1,-1)$}}
\put(108.24,94.22){\makebox(0,0)[lc]{$(1,1,-1,1)$}}
\put(17.45,94.22){\makebox(0,0)[rc]{$(0,1,1,0)$}}
\put(108.24,22.45){\makebox(0,0)[lc]{$(1,-1,1,-1)$}}
\put(16.45,22.45){\makebox(0,0)[rc]{$(0,0,1,1)$}}
\put(114.13,77.96){\makebox(0,0)[lc]{$(1,0,1,0)$}}
\put(8.55,77.96){\makebox(0,0)[rc]{$(0,0,0,1)$}}
\put(114.13,38.72){\makebox(0,0)[lc]{$(1,0,-1,0)$}}
\put(8.55,38.72){\makebox(0,0)[rc]{$(0,1,0,0)$}}
\put(120.92,57.98){\makebox(0,0)[lc]{$(0,1,0,-1)$}}
\put(1.77,57.98){\makebox(0,0)[rc]{$(1,0,0,0)$}}
}
\end{picture}
\end{center}

{ \small Orthogonality diagram of a ``short'' proof of the Kochen-Specker theorem by {\it Cabello et al.} (PRL 101 (2008) 210401;
arXiv:0808.2456) in four dimensions.
The graph cannot be colored by the two colors red (associated with truth)
and green (associated with falsity) such that every context contains exactly one red and three green points.}
}

\section{Miracles due to gaps in causal description}

\frame{
\frametitle{Miracles due to gaps in causal description}
{\footnotesize
An alltogether different issue, discussed by Philipp Frank,
is the possible occurrence of miracles in the presence of gaps of physical determinism.
One might perceive singular events occurring
within the bounds of classical and quantum physics without any apparent cause as miracles.
For, if there is no cause to an event,
why should such an event occur altogether rather than not occur?

Although such thoughts remain highly speculative, miracles,
if they exist,
could be the basis for an ``operator-directed'' evolution in otherwise deterministic physical systems.
Since in this scenario
the actions of the operator as well as the operating agent itself are ``located outside'' of the physical domain,
this presents a viable option for a transcendental influence in an otherwise deterministic universe.
Similar models have also been applied to dualistic models of the mind.

There exist bounds on miracles and on behavioral patterns in general due to the self-referential
perception of intrinsic observers endowed with free will.
}
}









\frame{

$\;$\\
$\;$\\
$\;$\\
$\;$\\
$\;$\\

\centerline{\Large Thank you for your attention!}

\begin{center}
$\widetilde{\qquad \qquad }$
$\widetilde{\qquad \qquad}$
$\widetilde{\qquad \qquad }$
\end{center}
 }

\end{document}
