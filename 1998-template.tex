\documentstyle[amsfonts]{elsart}
\sloppy
\renewcommand{\floatpagefraction}{0.91}
\renewcommand{\baselinestretch}{2}



\begin{document}
\begin{frontmatter}


\title{Analogues of Quantum Complementarity in the Theory of Automata}
\author{K. Svozil}
 \address{Institut f\"ur Theoretische Physik,
  University of Technology Vienna,
  Wiedner Hauptstra\ss e 8-10/136,
  A-1040 Vienna, Austria,
  e-mail: svozil@tph.tuwien.ac.at}


\begin{abstract}
{\em
Complementarity is not only a feature of quantum mechanical
systems but occurs also in the context of finite automata.}

\begin{flushright}
{\scriptsize
http://tph.tuwien.ac.at/$\widetilde{\;\;}\,$svozil/publ/template.tex}
\end{flushright}
\end{abstract}
\end{frontmatter}

\newpage

\section{Motivation}


The aim of this paper is to present to philosophers of physics some
results in the theory of automata, especially the theory concerned with
determining the initial state of the automaton: results which are
analogons to the phenomena of ``complementarity'' or ``non-Booleanness''
which occur in quantum mechanics.


It has long been known that any finite input/output system can be
modelled by finite automata \cite{paz}.
The study of finite automata was motivated from the very beginning
by their analogy to quantum systems
\cite{e-f-moore,Foulis-Randall,ran-foul-73}. Finite automata
are
universal with respect to the class of computable functions in the
(usual) sense that universal
networks of automata can compute any effectively (Turing-) computable
function. Conversely, any feature emerging from finite automata is
reflected by any other universal computational device.
Their non-Boolean intrinsic propositional calculus closely resembles
finite quantum mechanical systems
\cite{svozil-93,schaller-92,schaller-95,schaller-96,dvur-pul-svo}.

The considerations to follow in this article are not technically
complicated.
 Nevertheless,
the corresponding ideas turn out to be highly nontrivial
and nonclassical, sometimes mindboggling \cite{green-horn-zei}.


\section{Construction of automaton logics}
In this Section, I will first summarize some elements of the theory of
finite automata; then discuss the so-called state-identification
problem, and how it gives rise to to non-Boolean lattices, analogons to
those occurring in quantum theory.
Then I explicitly consider quantum logic in general and
give some examples.


\subsection{Machines}

\subsubsection{Moore and Mealy automata, state machines and combinatorial
circuits}

A {\em finite deterministic sequential machine} or {\em automaton}
\cite{e-f-moore,hopcroft,hartmanis}
is a device
with
a finite set of inputs which can be applied in a sequence, with
a finite set of internal configurations or states, and with
a finite set of outputs.
Furthermore
the present internal configuration and input uniquely determine the next
internal configuration and the output.

A {\em Mealy automaton} is a quintuple $M=(S,I,O,\delta
,\lambda
)$, where
\begin{enumerate}
\item[(i)]
$S$ is a finite (nonempty) set of states;
\item[(ii)]
$I$ is a finite (nonempty) set of inputs;
\item[(iii)]
$O$ is a finite (nonempty) set of outputs;
\item[(iv)]
$\delta : S\times I\longrightarrow S$ is a computable transition
function;
\item[(v)]
$\lambda : S\times I\longrightarrow O$ is a computable output
function.
\end{enumerate}

A {\em state machine} is a triplet of  $M=(S,I,\delta )$ with no outputs
and no output function.

A {\em combinatorial circuit} or {\em gate} is a triplet
of $M=(I,O,\lambda
)$, which
maps inputs into outputs, regardless of past history. It can also be
modelled as a one state Mealy automaton.


In what follows and if not mentioned otherwise, $s,i,o$ stand for
a particular internal state, input and output, respectively.


Mealy machines are represented by flow tables and state graphs.
To illustrate this, consider a
Mealy machine $M_e=(S,I,O,\delta ,\lambda )$ which has $n$ states, $n$
inputs and
2 outputs. That is
       \begin{eqnarray*}
S&=&\{ 1,2, ..., n \},\\
I&=&\{ 1,2, ..., n \},   \\
O&=&\{ 0,1 \}.
       \end{eqnarray*}
Its transition and output functions are ($ \delta_{s,x}$ stands for the
Kronecker delta function)
       \begin{eqnarray*}
\delta (s,i)&=&i,\\
\lambda (s,i) &=& \delta_{s,i} = \left\{
\begin{array}{cc}
1&\mbox{if }s=i\\
0&\mbox{if }s\neq i
\end{array}
\right.
       \end{eqnarray*}
The
flow table and state
graph of this
Mealy automaton is given in Fig. \ref{xx1}, where
%TexCad Options
%\grade{\off}
%\emlines{\off}
%\beziermacro{\off}
%\reduce{\on}
%\snapping{\off}
%\quality{2.00}
%\graddiff{0.01}
%\snapasp{1}
%\zoom{1.00}
\unitlength 0.7mm
\linethickness{0.4pt}
\begin{picture}(46.00,10.00)
\put(5.00,5.00){\circle{10.00}}
\put(5.00,5.00){\makebox(0,0)[cc]{$m$}}
\put(11.67,5.00){\vector(1,0){21.00}}
\put(20.00,9.67){\makebox(0,0)[cc]{$\alpha$/$\beta$}}
\put(41.00,5.00){\circle{10.00}}
\put(41.00,5.00){\makebox(0,0)[cc]{$n$}}
\end{picture}
indicates that when in state $m$, you receive input $\alpha$ you enter
state $n$ and produce output $\beta$.

\begin{figure}
\begin{center}
$M_e=$
\begin{tabular}{|c|cccc||cccc|}
 \hline
s$\backslash$c  &$\delta$ &&& & $\lambda$&&&\\
&1&2&$\cdots$&n & 1&2&$\cdots$&n\\
 \hline
1&1&2&$\cdots$&n & 1&0&$\cdots$&0\\
2&1&2&$\cdots$&n & 0&1&$\cdots$&0\\
$\vdots$&1&2&$\cdots$&n & 0&0&$\cdots$&0\\
n&1&2&$\cdots$&n & 0&0&$\cdots$&1\\
 \hline
\end{tabular}


%TexCad Options
%\grade{\off}
%\emlines{\off}
%\beziermacro{\off}
%\reduce{\on}
%\snapping{\on}
%\quality{6.00}
%\graddiff{0.01}
%\snapasp{1}
%\zoom{1.00}
\unitlength 1.00mm
\linethickness{0.4pt}
\begin{picture}(63.67,73.33)
%\circle(10.00,10.00){14.14}
\put(10.00,17.07){\line(1,0){0.54}}
\put(10.54,17.05){\line(1,0){0.53}}
\put(11.07,16.99){\line(1,0){0.53}}
\multiput(11.60,16.89)(0.26,-0.07){2}{\line(1,0){0.26}}
\multiput(12.12,16.75)(0.25,-0.09){2}{\line(1,0){0.25}}
\multiput(12.62,16.56)(0.25,-0.11){2}{\line(1,0){0.25}}
\multiput(13.12,16.35)(0.16,-0.09){3}{\line(1,0){0.16}}
\multiput(13.59,16.09)(0.15,-0.10){3}{\line(1,0){0.15}}
\multiput(14.04,15.80)(0.14,-0.11){3}{\line(1,0){0.14}}
\multiput(14.47,15.48)(0.13,-0.12){3}{\line(1,0){0.13}}
\multiput(14.87,15.12)(0.09,-0.10){4}{\line(0,-1){0.10}}
\multiput(15.25,14.74)(0.11,-0.14){3}{\line(0,-1){0.14}}
\multiput(15.59,14.32)(0.10,-0.15){3}{\line(0,-1){0.15}}
\multiput(15.91,13.89)(0.09,-0.15){3}{\line(0,-1){0.15}}
\multiput(16.18,13.43)(0.08,-0.16){3}{\line(0,-1){0.16}}
\multiput(16.43,12.95)(0.10,-0.25){2}{\line(0,-1){0.25}}
\multiput(16.63,12.45)(0.08,-0.26){2}{\line(0,-1){0.26}}
\multiput(16.80,11.94)(0.06,-0.26){2}{\line(0,-1){0.26}}
\put(16.93,11.42){\line(0,-1){0.53}}
\put(17.01,10.89){\line(0,-1){0.54}}
\put(17.06,10.35){\line(0,-1){0.54}}
\put(17.07,9.81){\line(0,-1){0.54}}
\put(17.03,9.28){\line(0,-1){0.53}}
\put(16.96,8.74){\line(0,-1){0.53}}
\multiput(16.84,8.22)(-0.08,-0.26){2}{\line(0,-1){0.26}}
\multiput(16.69,7.70)(-0.10,-0.25){2}{\line(0,-1){0.25}}
\multiput(16.49,7.20)(-0.12,-0.24){2}{\line(0,-1){0.24}}
\multiput(16.26,6.72)(-0.09,-0.16){3}{\line(0,-1){0.16}}
\multiput(15.99,6.25)(-0.10,-0.15){3}{\line(0,-1){0.15}}
\multiput(15.69,5.81)(-0.11,-0.14){3}{\line(0,-1){0.14}}
\multiput(15.36,5.39)(-0.09,-0.10){4}{\line(0,-1){0.10}}
\multiput(14.99,4.99)(-0.10,-0.09){4}{\line(-1,0){0.10}}
\multiput(14.60,4.63)(-0.14,-0.11){3}{\line(-1,0){0.14}}
\multiput(14.17,4.29)(-0.15,-0.10){3}{\line(-1,0){0.15}}
\multiput(13.73,3.99)(-0.16,-0.09){3}{\line(-1,0){0.16}}
\multiput(13.26,3.73)(-0.24,-0.11){2}{\line(-1,0){0.24}}
\multiput(12.77,3.50)(-0.25,-0.10){2}{\line(-1,0){0.25}}
\multiput(12.27,3.31)(-0.26,-0.08){2}{\line(-1,0){0.26}}
\put(11.76,3.15){\line(-1,0){0.53}}
\put(11.23,3.04){\line(-1,0){0.53}}
\put(10.70,2.96){\line(-1,0){0.54}}
\put(10.16,2.93){\line(-1,0){0.54}}
\put(9.63,2.94){\line(-1,0){0.54}}
\put(9.09,2.99){\line(-1,0){0.53}}
\multiput(8.56,3.08)(-0.26,0.06){2}{\line(-1,0){0.26}}
\multiput(8.04,3.21)(-0.26,0.08){2}{\line(-1,0){0.26}}
\multiput(7.53,3.38)(-0.25,0.10){2}{\line(-1,0){0.25}}
\multiput(7.03,3.58)(-0.16,0.08){3}{\line(-1,0){0.16}}
\multiput(6.55,3.83)(-0.15,0.09){3}{\line(-1,0){0.15}}
\multiput(6.09,4.11)(-0.15,0.10){3}{\line(-1,0){0.15}}
\multiput(5.66,4.42)(-0.14,0.12){3}{\line(-1,0){0.14}}
\multiput(5.25,4.77)(-0.10,0.09){4}{\line(-1,0){0.10}}
\multiput(4.86,5.14)(-0.12,0.13){3}{\line(0,1){0.13}}
\multiput(4.51,5.55)(-0.11,0.14){3}{\line(0,1){0.14}}
\multiput(4.18,5.98)(-0.10,0.15){3}{\line(0,1){0.15}}
\multiput(3.90,6.43)(-0.08,0.16){3}{\line(0,1){0.16}}
\multiput(3.64,6.91)(-0.11,0.25){2}{\line(0,1){0.25}}
\multiput(3.43,7.40)(-0.09,0.25){2}{\line(0,1){0.25}}
\multiput(3.25,7.91)(-0.07,0.26){2}{\line(0,1){0.26}}
\put(3.11,8.42){\line(0,1){0.53}}
\put(3.01,8.95){\line(0,1){0.53}}
\put(2.95,9.49){\line(0,1){1.07}}
\put(2.95,10.56){\line(0,1){0.53}}
\put(3.02,11.10){\line(0,1){0.53}}
\multiput(3.12,11.62)(0.07,0.26){2}{\line(0,1){0.26}}
\multiput(3.26,12.14)(0.09,0.25){2}{\line(0,1){0.25}}
\multiput(3.44,12.65)(0.11,0.25){2}{\line(0,1){0.25}}
\multiput(3.66,13.14)(0.09,0.16){3}{\line(0,1){0.16}}
\multiput(3.92,13.61)(0.10,0.15){3}{\line(0,1){0.15}}
\multiput(4.21,14.06)(0.11,0.14){3}{\line(0,1){0.14}}
\multiput(4.54,14.49)(0.12,0.13){3}{\line(0,1){0.13}}
\multiput(4.90,14.89)(0.10,0.09){4}{\line(1,0){0.10}}
\multiput(5.28,15.27)(0.14,0.11){3}{\line(1,0){0.14}}
\multiput(5.70,15.61)(0.15,0.10){3}{\line(1,0){0.15}}
\multiput(6.13,15.92)(0.15,0.09){3}{\line(1,0){0.15}}
\multiput(6.60,16.20)(0.16,0.08){3}{\line(1,0){0.16}}
\multiput(7.08,16.44)(0.25,0.10){2}{\line(1,0){0.25}}
\multiput(7.57,16.64)(0.26,0.08){2}{\line(1,0){0.26}}
\multiput(8.09,16.81)(0.26,0.06){2}{\line(1,0){0.26}}
\put(8.61,16.93){\line(1,0){0.53}}
\put(9.14,17.02){\line(1,0){0.86}}
%\end
\put(10.00,10.00){\makebox(0,0)[cc]{$1$}}
%\circle(10.00,10.00){14.14}
\put(10.00,17.07){\line(1,0){0.54}}
\put(10.54,17.05){\line(1,0){0.53}}
\put(11.07,16.99){\line(1,0){0.53}}
\multiput(11.60,16.89)(0.26,-0.07){2}{\line(1,0){0.26}}
\multiput(12.12,16.75)(0.25,-0.09){2}{\line(1,0){0.25}}
\multiput(12.62,16.56)(0.25,-0.11){2}{\line(1,0){0.25}}
\multiput(13.12,16.35)(0.16,-0.09){3}{\line(1,0){0.16}}
\multiput(13.59,16.09)(0.15,-0.10){3}{\line(1,0){0.15}}
\multiput(14.04,15.80)(0.14,-0.11){3}{\line(1,0){0.14}}
\multiput(14.47,15.48)(0.13,-0.12){3}{\line(1,0){0.13}}
\multiput(14.87,15.12)(0.09,-0.10){4}{\line(0,-1){0.10}}
\multiput(15.25,14.74)(0.11,-0.14){3}{\line(0,-1){0.14}}
\multiput(15.59,14.32)(0.10,-0.15){3}{\line(0,-1){0.15}}
\multiput(15.91,13.89)(0.09,-0.15){3}{\line(0,-1){0.15}}
\multiput(16.18,13.43)(0.08,-0.16){3}{\line(0,-1){0.16}}
\multiput(16.43,12.95)(0.10,-0.25){2}{\line(0,-1){0.25}}
\multiput(16.63,12.45)(0.08,-0.26){2}{\line(0,-1){0.26}}
\multiput(16.80,11.94)(0.06,-0.26){2}{\line(0,-1){0.26}}
\put(16.93,11.42){\line(0,-1){0.53}}
\put(17.01,10.89){\line(0,-1){0.54}}
\put(17.06,10.35){\line(0,-1){0.54}}
\put(17.07,9.81){\line(0,-1){0.54}}
\put(17.03,9.28){\line(0,-1){0.53}}
\put(16.96,8.74){\line(0,-1){0.53}}
\multiput(16.84,8.22)(-0.08,-0.26){2}{\line(0,-1){0.26}}
\multiput(16.69,7.70)(-0.10,-0.25){2}{\line(0,-1){0.25}}
\multiput(16.49,7.20)(-0.12,-0.24){2}{\line(0,-1){0.24}}
\multiput(16.26,6.72)(-0.09,-0.16){3}{\line(0,-1){0.16}}
\multiput(15.99,6.25)(-0.10,-0.15){3}{\line(0,-1){0.15}}
\multiput(15.69,5.81)(-0.11,-0.14){3}{\line(0,-1){0.14}}
\multiput(15.36,5.39)(-0.09,-0.10){4}{\line(0,-1){0.10}}
\multiput(14.99,4.99)(-0.10,-0.09){4}{\line(-1,0){0.10}}
\multiput(14.60,4.63)(-0.14,-0.11){3}{\line(-1,0){0.14}}
\multiput(14.17,4.29)(-0.15,-0.10){3}{\line(-1,0){0.15}}
\multiput(13.73,3.99)(-0.16,-0.09){3}{\line(-1,0){0.16}}
\multiput(13.26,3.73)(-0.24,-0.11){2}{\line(-1,0){0.24}}
\multiput(12.77,3.50)(-0.25,-0.10){2}{\line(-1,0){0.25}}
\multiput(12.27,3.31)(-0.26,-0.08){2}{\line(-1,0){0.26}}
\put(11.76,3.15){\line(-1,0){0.53}}
\put(11.23,3.04){\line(-1,0){0.53}}
\put(10.70,2.96){\line(-1,0){0.54}}
\put(10.16,2.93){\line(-1,0){0.54}}
\put(9.63,2.94){\line(-1,0){0.54}}
\put(9.09,2.99){\line(-1,0){0.53}}
\multiput(8.56,3.08)(-0.26,0.06){2}{\line(-1,0){0.26}}
\multiput(8.04,3.21)(-0.26,0.08){2}{\line(-1,0){0.26}}
\multiput(7.53,3.38)(-0.25,0.10){2}{\line(-1,0){0.25}}
\multiput(7.03,3.58)(-0.16,0.08){3}{\line(-1,0){0.16}}
\multiput(6.55,3.83)(-0.15,0.09){3}{\line(-1,0){0.15}}
\multiput(6.09,4.11)(-0.15,0.10){3}{\line(-1,0){0.15}}
\multiput(5.66,4.42)(-0.14,0.12){3}{\line(-1,0){0.14}}
\multiput(5.25,4.77)(-0.10,0.09){4}{\line(-1,0){0.10}}
\multiput(4.86,5.14)(-0.12,0.13){3}{\line(0,1){0.13}}
\multiput(4.51,5.55)(-0.11,0.14){3}{\line(0,1){0.14}}
\multiput(4.18,5.98)(-0.10,0.15){3}{\line(0,1){0.15}}
\multiput(3.90,6.43)(-0.08,0.16){3}{\line(0,1){0.16}}
\multiput(3.64,6.91)(-0.11,0.25){2}{\line(0,1){0.25}}
\multiput(3.43,7.40)(-0.09,0.25){2}{\line(0,1){0.25}}
\multiput(3.25,7.91)(-0.07,0.26){2}{\line(0,1){0.26}}
\put(3.11,8.42){\line(0,1){0.53}}
\put(3.01,8.95){\line(0,1){0.53}}
\put(2.95,9.49){\line(0,1){1.07}}
\put(2.95,10.56){\line(0,1){0.53}}
\put(3.02,11.10){\line(0,1){0.53}}
\multiput(3.12,11.62)(0.07,0.26){2}{\line(0,1){0.26}}
\multiput(3.26,12.14)(0.09,0.25){2}{\line(0,1){0.25}}
\multiput(3.44,12.65)(0.11,0.25){2}{\line(0,1){0.25}}
\multiput(3.66,13.14)(0.09,0.16){3}{\line(0,1){0.16}}
\multiput(3.92,13.61)(0.10,0.15){3}{\line(0,1){0.15}}
\multiput(4.21,14.06)(0.11,0.14){3}{\line(0,1){0.14}}
\multiput(4.54,14.49)(0.12,0.13){3}{\line(0,1){0.13}}
\multiput(4.90,14.89)(0.10,0.09){4}{\line(1,0){0.10}}
\multiput(5.28,15.27)(0.14,0.11){3}{\line(1,0){0.14}}
\multiput(5.70,15.61)(0.15,0.10){3}{\line(1,0){0.15}}
\multiput(6.13,15.92)(0.15,0.09){3}{\line(1,0){0.15}}
\multiput(6.60,16.20)(0.16,0.08){3}{\line(1,0){0.16}}
\multiput(7.08,16.44)(0.25,0.10){2}{\line(1,0){0.25}}
\multiput(7.57,16.64)(0.26,0.08){2}{\line(1,0){0.26}}
\multiput(8.09,16.81)(0.26,0.06){2}{\line(1,0){0.26}}
\put(8.61,16.93){\line(1,0){0.53}}
\put(9.14,17.02){\line(1,0){0.86}}
%\end
\put(10.00,10.00){\makebox(0,0)[cc]{$1$}}
%\circle(45.00,10.00){14.14}
\put(45.00,17.07){\line(1,0){0.54}}
\put(45.54,17.05){\line(1,0){0.53}}
\put(46.07,16.99){\line(1,0){0.53}}
\multiput(46.60,16.89)(0.26,-0.07){2}{\line(1,0){0.26}}
\multiput(47.12,16.75)(0.25,-0.09){2}{\line(1,0){0.25}}
\multiput(47.62,16.56)(0.25,-0.11){2}{\line(1,0){0.25}}
\multiput(48.12,16.35)(0.16,-0.09){3}{\line(1,0){0.16}}
\multiput(48.59,16.09)(0.15,-0.10){3}{\line(1,0){0.15}}
\multiput(49.04,15.80)(0.14,-0.11){3}{\line(1,0){0.14}}
\multiput(49.47,15.48)(0.13,-0.12){3}{\line(1,0){0.13}}
\multiput(49.87,15.12)(0.09,-0.10){4}{\line(0,-1){0.10}}
\multiput(50.25,14.74)(0.11,-0.14){3}{\line(0,-1){0.14}}
\multiput(50.59,14.32)(0.10,-0.15){3}{\line(0,-1){0.15}}
\multiput(50.91,13.89)(0.09,-0.15){3}{\line(0,-1){0.15}}
\multiput(51.18,13.43)(0.08,-0.16){3}{\line(0,-1){0.16}}
\multiput(51.43,12.95)(0.10,-0.25){2}{\line(0,-1){0.25}}
\multiput(51.63,12.45)(0.08,-0.26){2}{\line(0,-1){0.26}}
\multiput(51.80,11.94)(0.06,-0.26){2}{\line(0,-1){0.26}}
\put(51.93,11.42){\line(0,-1){0.53}}
\put(52.01,10.89){\line(0,-1){0.54}}
\put(52.06,10.35){\line(0,-1){0.54}}
\put(52.07,9.81){\line(0,-1){0.54}}
\put(52.03,9.28){\line(0,-1){0.53}}
\put(51.96,8.74){\line(0,-1){0.53}}
\multiput(51.84,8.22)(-0.08,-0.26){2}{\line(0,-1){0.26}}
\multiput(51.69,7.70)(-0.10,-0.25){2}{\line(0,-1){0.25}}
\multiput(51.49,7.20)(-0.12,-0.24){2}{\line(0,-1){0.24}}
\multiput(51.26,6.72)(-0.09,-0.16){3}{\line(0,-1){0.16}}
\multiput(50.99,6.25)(-0.10,-0.15){3}{\line(0,-1){0.15}}
\multiput(50.69,5.81)(-0.11,-0.14){3}{\line(0,-1){0.14}}
\multiput(50.36,5.39)(-0.09,-0.10){4}{\line(0,-1){0.10}}
\multiput(49.99,4.99)(-0.10,-0.09){4}{\line(-1,0){0.10}}
\multiput(49.60,4.63)(-0.14,-0.11){3}{\line(-1,0){0.14}}
\multiput(49.17,4.29)(-0.15,-0.10){3}{\line(-1,0){0.15}}
\multiput(48.73,3.99)(-0.16,-0.09){3}{\line(-1,0){0.16}}
\multiput(48.26,3.73)(-0.24,-0.11){2}{\line(-1,0){0.24}}
\multiput(47.77,3.50)(-0.25,-0.10){2}{\line(-1,0){0.25}}
\multiput(47.27,3.31)(-0.26,-0.08){2}{\line(-1,0){0.26}}
\put(46.76,3.15){\line(-1,0){0.53}}
\put(46.23,3.04){\line(-1,0){0.53}}
\put(45.70,2.96){\line(-1,0){0.54}}
\put(45.16,2.93){\line(-1,0){0.54}}
\put(44.63,2.94){\line(-1,0){0.54}}
\put(44.09,2.99){\line(-1,0){0.53}}
\multiput(43.56,3.08)(-0.26,0.06){2}{\line(-1,0){0.26}}
\multiput(43.04,3.21)(-0.26,0.08){2}{\line(-1,0){0.26}}
\multiput(42.53,3.38)(-0.25,0.10){2}{\line(-1,0){0.25}}
\multiput(42.03,3.58)(-0.16,0.08){3}{\line(-1,0){0.16}}
\multiput(41.55,3.83)(-0.15,0.09){3}{\line(-1,0){0.15}}
\multiput(41.09,4.11)(-0.15,0.10){3}{\line(-1,0){0.15}}
\multiput(40.66,4.42)(-0.14,0.12){3}{\line(-1,0){0.14}}
\multiput(40.25,4.77)(-0.10,0.09){4}{\line(-1,0){0.10}}
\multiput(39.86,5.14)(-0.12,0.13){3}{\line(0,1){0.13}}
\multiput(39.51,5.55)(-0.11,0.14){3}{\line(0,1){0.14}}
\multiput(39.18,5.98)(-0.10,0.15){3}{\line(0,1){0.15}}
\multiput(38.90,6.43)(-0.08,0.16){3}{\line(0,1){0.16}}
\multiput(38.64,6.91)(-0.11,0.25){2}{\line(0,1){0.25}}
\multiput(38.43,7.40)(-0.09,0.25){2}{\line(0,1){0.25}}
\multiput(38.25,7.91)(-0.07,0.26){2}{\line(0,1){0.26}}
\put(38.11,8.42){\line(0,1){0.53}}
\put(38.01,8.95){\line(0,1){0.53}}
\put(37.95,9.49){\line(0,1){1.07}}
\put(37.95,10.56){\line(0,1){0.53}}
\put(38.02,11.10){\line(0,1){0.53}}
\multiput(38.12,11.62)(0.07,0.26){2}{\line(0,1){0.26}}
\multiput(38.26,12.14)(0.09,0.25){2}{\line(0,1){0.25}}
\multiput(38.44,12.65)(0.11,0.25){2}{\line(0,1){0.25}}
\multiput(38.66,13.14)(0.09,0.16){3}{\line(0,1){0.16}}
\multiput(38.92,13.61)(0.10,0.15){3}{\line(0,1){0.15}}
\multiput(39.21,14.06)(0.11,0.14){3}{\line(0,1){0.14}}
\multiput(39.54,14.49)(0.12,0.13){3}{\line(0,1){0.13}}
\multiput(39.90,14.89)(0.10,0.09){4}{\line(1,0){0.10}}
\multiput(40.28,15.27)(0.14,0.11){3}{\line(1,0){0.14}}
\multiput(40.70,15.61)(0.15,0.10){3}{\line(1,0){0.15}}
\multiput(41.13,15.92)(0.15,0.09){3}{\line(1,0){0.15}}
\multiput(41.60,16.20)(0.16,0.08){3}{\line(1,0){0.16}}
\multiput(42.08,16.44)(0.25,0.10){2}{\line(1,0){0.25}}
\multiput(42.57,16.64)(0.26,0.08){2}{\line(1,0){0.26}}
\multiput(43.09,16.81)(0.26,0.06){2}{\line(1,0){0.26}}
\put(43.61,16.93){\line(1,0){0.53}}
\put(44.14,17.02){\line(1,0){0.86}}
%\end
%\circle(45.00,10.00){14.14}
\put(45.00,17.07){\line(1,0){0.54}}
\put(45.54,17.05){\line(1,0){0.53}}
\put(46.07,16.99){\line(1,0){0.53}}
\multiput(46.60,16.89)(0.26,-0.07){2}{\line(1,0){0.26}}
\multiput(47.12,16.75)(0.25,-0.09){2}{\line(1,0){0.25}}
\multiput(47.62,16.56)(0.25,-0.11){2}{\line(1,0){0.25}}
\multiput(48.12,16.35)(0.16,-0.09){3}{\line(1,0){0.16}}
\multiput(48.59,16.09)(0.15,-0.10){3}{\line(1,0){0.15}}
\multiput(49.04,15.80)(0.14,-0.11){3}{\line(1,0){0.14}}
\multiput(49.47,15.48)(0.13,-0.12){3}{\line(1,0){0.13}}
\multiput(49.87,15.12)(0.09,-0.10){4}{\line(0,-1){0.10}}
\multiput(50.25,14.74)(0.11,-0.14){3}{\line(0,-1){0.14}}
\multiput(50.59,14.32)(0.10,-0.15){3}{\line(0,-1){0.15}}
\multiput(50.91,13.89)(0.09,-0.15){3}{\line(0,-1){0.15}}
\multiput(51.18,13.43)(0.08,-0.16){3}{\line(0,-1){0.16}}
\multiput(51.43,12.95)(0.10,-0.25){2}{\line(0,-1){0.25}}
\multiput(51.63,12.45)(0.08,-0.26){2}{\line(0,-1){0.26}}
\multiput(51.80,11.94)(0.06,-0.26){2}{\line(0,-1){0.26}}
\put(51.93,11.42){\line(0,-1){0.53}}
\put(52.01,10.89){\line(0,-1){0.54}}
\put(52.06,10.35){\line(0,-1){0.54}}
\put(52.07,9.81){\line(0,-1){0.54}}
\put(52.03,9.28){\line(0,-1){0.53}}
\put(51.96,8.74){\line(0,-1){0.53}}
\multiput(51.84,8.22)(-0.08,-0.26){2}{\line(0,-1){0.26}}
\multiput(51.69,7.70)(-0.10,-0.25){2}{\line(0,-1){0.25}}
\multiput(51.49,7.20)(-0.12,-0.24){2}{\line(0,-1){0.24}}
\multiput(51.26,6.72)(-0.09,-0.16){3}{\line(0,-1){0.16}}
\multiput(50.99,6.25)(-0.10,-0.15){3}{\line(0,-1){0.15}}
\multiput(50.69,5.81)(-0.11,-0.14){3}{\line(0,-1){0.14}}
\multiput(50.36,5.39)(-0.09,-0.10){4}{\line(0,-1){0.10}}
\multiput(49.99,4.99)(-0.10,-0.09){4}{\line(-1,0){0.10}}
\multiput(49.60,4.63)(-0.14,-0.11){3}{\line(-1,0){0.14}}
\multiput(49.17,4.29)(-0.15,-0.10){3}{\line(-1,0){0.15}}
\multiput(48.73,3.99)(-0.16,-0.09){3}{\line(-1,0){0.16}}
\multiput(48.26,3.73)(-0.24,-0.11){2}{\line(-1,0){0.24}}
\multiput(47.77,3.50)(-0.25,-0.10){2}{\line(-1,0){0.25}}
\multiput(47.27,3.31)(-0.26,-0.08){2}{\line(-1,0){0.26}}
\put(46.76,3.15){\line(-1,0){0.53}}
\put(46.23,3.04){\line(-1,0){0.53}}
\put(45.70,2.96){\line(-1,0){0.54}}
\put(45.16,2.93){\line(-1,0){0.54}}
\put(44.63,2.94){\line(-1,0){0.54}}
\put(44.09,2.99){\line(-1,0){0.53}}
\multiput(43.56,3.08)(-0.26,0.06){2}{\line(-1,0){0.26}}
\multiput(43.04,3.21)(-0.26,0.08){2}{\line(-1,0){0.26}}
\multiput(42.53,3.38)(-0.25,0.10){2}{\line(-1,0){0.25}}
\multiput(42.03,3.58)(-0.16,0.08){3}{\line(-1,0){0.16}}
\multiput(41.55,3.83)(-0.15,0.09){3}{\line(-1,0){0.15}}
\multiput(41.09,4.11)(-0.15,0.10){3}{\line(-1,0){0.15}}
\multiput(40.66,4.42)(-0.14,0.12){3}{\line(-1,0){0.14}}
\multiput(40.25,4.77)(-0.10,0.09){4}{\line(-1,0){0.10}}
\multiput(39.86,5.14)(-0.12,0.13){3}{\line(0,1){0.13}}
\multiput(39.51,5.55)(-0.11,0.14){3}{\line(0,1){0.14}}
\multiput(39.18,5.98)(-0.10,0.15){3}{\line(0,1){0.15}}
\multiput(38.90,6.43)(-0.08,0.16){3}{\line(0,1){0.16}}
\multiput(38.64,6.91)(-0.11,0.25){2}{\line(0,1){0.25}}
\multiput(38.43,7.40)(-0.09,0.25){2}{\line(0,1){0.25}}
\multiput(38.25,7.91)(-0.07,0.26){2}{\line(0,1){0.26}}
\put(38.11,8.42){\line(0,1){0.53}}
\put(38.01,8.95){\line(0,1){0.53}}
\put(37.95,9.49){\line(0,1){1.07}}
\put(37.95,10.56){\line(0,1){0.53}}
\put(38.02,11.10){\line(0,1){0.53}}
\multiput(38.12,11.62)(0.07,0.26){2}{\line(0,1){0.26}}
\multiput(38.26,12.14)(0.09,0.25){2}{\line(0,1){0.25}}
\multiput(38.44,12.65)(0.11,0.25){2}{\line(0,1){0.25}}
\multiput(38.66,13.14)(0.09,0.16){3}{\line(0,1){0.16}}
\multiput(38.92,13.61)(0.10,0.15){3}{\line(0,1){0.15}}
\multiput(39.21,14.06)(0.11,0.14){3}{\line(0,1){0.14}}
\multiput(39.54,14.49)(0.12,0.13){3}{\line(0,1){0.13}}
\multiput(39.90,14.89)(0.10,0.09){4}{\line(1,0){0.10}}
\multiput(40.28,15.27)(0.14,0.11){3}{\line(1,0){0.14}}
\multiput(40.70,15.61)(0.15,0.10){3}{\line(1,0){0.15}}
\multiput(41.13,15.92)(0.15,0.09){3}{\line(1,0){0.15}}
\multiput(41.60,16.20)(0.16,0.08){3}{\line(1,0){0.16}}
\multiput(42.08,16.44)(0.25,0.10){2}{\line(1,0){0.25}}
\multiput(42.57,16.64)(0.26,0.08){2}{\line(1,0){0.26}}
\multiput(43.09,16.81)(0.26,0.06){2}{\line(1,0){0.26}}
\put(43.61,16.93){\line(1,0){0.53}}
\put(44.14,17.02){\line(1,0){0.86}}
%\end
%\circle(10.00,45.00){14.14}
\put(10.00,52.07){\line(1,0){0.54}}
\put(10.54,52.05){\line(1,0){0.53}}
\put(11.07,51.99){\line(1,0){0.53}}
\multiput(11.60,51.89)(0.26,-0.07){2}{\line(1,0){0.26}}
\multiput(12.12,51.75)(0.25,-0.09){2}{\line(1,0){0.25}}
\multiput(12.62,51.56)(0.25,-0.11){2}{\line(1,0){0.25}}
\multiput(13.12,51.35)(0.16,-0.09){3}{\line(1,0){0.16}}
\multiput(13.59,51.09)(0.15,-0.10){3}{\line(1,0){0.15}}
\multiput(14.04,50.80)(0.14,-0.11){3}{\line(1,0){0.14}}
\multiput(14.47,50.48)(0.13,-0.12){3}{\line(1,0){0.13}}
\multiput(14.87,50.12)(0.09,-0.10){4}{\line(0,-1){0.10}}
\multiput(15.25,49.74)(0.11,-0.14){3}{\line(0,-1){0.14}}
\multiput(15.59,49.32)(0.10,-0.15){3}{\line(0,-1){0.15}}
\multiput(15.91,48.89)(0.09,-0.15){3}{\line(0,-1){0.15}}
\multiput(16.18,48.43)(0.08,-0.16){3}{\line(0,-1){0.16}}
\multiput(16.43,47.95)(0.10,-0.25){2}{\line(0,-1){0.25}}
\multiput(16.63,47.45)(0.08,-0.26){2}{\line(0,-1){0.26}}
\multiput(16.80,46.94)(0.06,-0.26){2}{\line(0,-1){0.26}}
\put(16.93,46.42){\line(0,-1){0.53}}
\put(17.01,45.89){\line(0,-1){0.54}}
\put(17.06,45.35){\line(0,-1){0.54}}
\put(17.07,44.81){\line(0,-1){0.54}}
\put(17.03,44.28){\line(0,-1){0.53}}
\put(16.96,43.74){\line(0,-1){0.53}}
\multiput(16.84,43.22)(-0.08,-0.26){2}{\line(0,-1){0.26}}
\multiput(16.69,42.70)(-0.10,-0.25){2}{\line(0,-1){0.25}}
\multiput(16.49,42.20)(-0.12,-0.24){2}{\line(0,-1){0.24}}
\multiput(16.26,41.72)(-0.09,-0.16){3}{\line(0,-1){0.16}}
\multiput(15.99,41.25)(-0.10,-0.15){3}{\line(0,-1){0.15}}
\multiput(15.69,40.81)(-0.11,-0.14){3}{\line(0,-1){0.14}}
\multiput(15.36,40.39)(-0.09,-0.10){4}{\line(0,-1){0.10}}
\multiput(14.99,39.99)(-0.10,-0.09){4}{\line(-1,0){0.10}}
\multiput(14.60,39.63)(-0.14,-0.11){3}{\line(-1,0){0.14}}
\multiput(14.17,39.29)(-0.15,-0.10){3}{\line(-1,0){0.15}}
\multiput(13.73,38.99)(-0.16,-0.09){3}{\line(-1,0){0.16}}
\multiput(13.26,38.73)(-0.24,-0.11){2}{\line(-1,0){0.24}}
\multiput(12.77,38.50)(-0.25,-0.10){2}{\line(-1,0){0.25}}
\multiput(12.27,38.31)(-0.26,-0.08){2}{\line(-1,0){0.26}}
\put(11.76,38.15){\line(-1,0){0.53}}
\put(11.23,38.04){\line(-1,0){0.53}}
\put(10.70,37.96){\line(-1,0){0.54}}
\put(10.16,37.93){\line(-1,0){0.54}}
\put(9.63,37.94){\line(-1,0){0.54}}
\put(9.09,37.99){\line(-1,0){0.53}}
\multiput(8.56,38.08)(-0.26,0.06){2}{\line(-1,0){0.26}}
\multiput(8.04,38.21)(-0.26,0.08){2}{\line(-1,0){0.26}}
\multiput(7.53,38.38)(-0.25,0.10){2}{\line(-1,0){0.25}}
\multiput(7.03,38.58)(-0.16,0.08){3}{\line(-1,0){0.16}}
\multiput(6.55,38.83)(-0.15,0.09){3}{\line(-1,0){0.15}}
\multiput(6.09,39.11)(-0.15,0.10){3}{\line(-1,0){0.15}}
\multiput(5.66,39.42)(-0.14,0.12){3}{\line(-1,0){0.14}}
\multiput(5.25,39.77)(-0.10,0.09){4}{\line(-1,0){0.10}}
\multiput(4.86,40.14)(-0.12,0.13){3}{\line(0,1){0.13}}
\multiput(4.51,40.55)(-0.11,0.14){3}{\line(0,1){0.14}}
\multiput(4.18,40.98)(-0.10,0.15){3}{\line(0,1){0.15}}
\multiput(3.90,41.43)(-0.08,0.16){3}{\line(0,1){0.16}}
\multiput(3.64,41.91)(-0.11,0.25){2}{\line(0,1){0.25}}
\multiput(3.43,42.40)(-0.09,0.25){2}{\line(0,1){0.25}}
\multiput(3.25,42.91)(-0.07,0.26){2}{\line(0,1){0.26}}
\put(3.11,43.42){\line(0,1){0.53}}
\put(3.01,43.95){\line(0,1){0.53}}
\put(2.95,44.49){\line(0,1){1.07}}
\put(2.95,45.56){\line(0,1){0.53}}
\put(3.02,46.10){\line(0,1){0.53}}
\multiput(3.12,46.62)(0.07,0.26){2}{\line(0,1){0.26}}
\multiput(3.26,47.14)(0.09,0.25){2}{\line(0,1){0.25}}
\multiput(3.44,47.65)(0.11,0.25){2}{\line(0,1){0.25}}
\multiput(3.66,48.14)(0.09,0.16){3}{\line(0,1){0.16}}
\multiput(3.92,48.61)(0.10,0.15){3}{\line(0,1){0.15}}
\multiput(4.21,49.06)(0.11,0.14){3}{\line(0,1){0.14}}
\multiput(4.54,49.49)(0.12,0.13){3}{\line(0,1){0.13}}
\multiput(4.90,49.89)(0.10,0.09){4}{\line(1,0){0.10}}
\multiput(5.28,50.27)(0.14,0.11){3}{\line(1,0){0.14}}
\multiput(5.70,50.61)(0.15,0.10){3}{\line(1,0){0.15}}
\multiput(6.13,50.92)(0.15,0.09){3}{\line(1,0){0.15}}
\multiput(6.60,51.20)(0.16,0.08){3}{\line(1,0){0.16}}
\multiput(7.08,51.44)(0.25,0.10){2}{\line(1,0){0.25}}
\multiput(7.57,51.64)(0.26,0.08){2}{\line(1,0){0.26}}
\multiput(8.09,51.81)(0.26,0.06){2}{\line(1,0){0.26}}
\put(8.61,51.93){\line(1,0){0.53}}
\put(9.14,52.02){\line(1,0){0.86}}
%\end
%\circle(10.00,45.00){14.14}
\put(10.00,52.07){\line(1,0){0.54}}
\put(10.54,52.05){\line(1,0){0.53}}
\put(11.07,51.99){\line(1,0){0.53}}
\multiput(11.60,51.89)(0.26,-0.07){2}{\line(1,0){0.26}}
\multiput(12.12,51.75)(0.25,-0.09){2}{\line(1,0){0.25}}
\multiput(12.62,51.56)(0.25,-0.11){2}{\line(1,0){0.25}}
\multiput(13.12,51.35)(0.16,-0.09){3}{\line(1,0){0.16}}
\multiput(13.59,51.09)(0.15,-0.10){3}{\line(1,0){0.15}}
\multiput(14.04,50.80)(0.14,-0.11){3}{\line(1,0){0.14}}
\multiput(14.47,50.48)(0.13,-0.12){3}{\line(1,0){0.13}}
\multiput(14.87,50.12)(0.09,-0.10){4}{\line(0,-1){0.10}}
\multiput(15.25,49.74)(0.11,-0.14){3}{\line(0,-1){0.14}}
\multiput(15.59,49.32)(0.10,-0.15){3}{\line(0,-1){0.15}}
\multiput(15.91,48.89)(0.09,-0.15){3}{\line(0,-1){0.15}}
\multiput(16.18,48.43)(0.08,-0.16){3}{\line(0,-1){0.16}}
\multiput(16.43,47.95)(0.10,-0.25){2}{\line(0,-1){0.25}}
\multiput(16.63,47.45)(0.08,-0.26){2}{\line(0,-1){0.26}}
\multiput(16.80,46.94)(0.06,-0.26){2}{\line(0,-1){0.26}}
\put(16.93,46.42){\line(0,-1){0.53}}
\put(17.01,45.89){\line(0,-1){0.54}}
\put(17.06,45.35){\line(0,-1){0.54}}
\put(17.07,44.81){\line(0,-1){0.54}}
\put(17.03,44.28){\line(0,-1){0.53}}
\put(16.96,43.74){\line(0,-1){0.53}}
\multiput(16.84,43.22)(-0.08,-0.26){2}{\line(0,-1){0.26}}
\multiput(16.69,42.70)(-0.10,-0.25){2}{\line(0,-1){0.25}}
\multiput(16.49,42.20)(-0.12,-0.24){2}{\line(0,-1){0.24}}
\multiput(16.26,41.72)(-0.09,-0.16){3}{\line(0,-1){0.16}}
\multiput(15.99,41.25)(-0.10,-0.15){3}{\line(0,-1){0.15}}
\multiput(15.69,40.81)(-0.11,-0.14){3}{\line(0,-1){0.14}}
\multiput(15.36,40.39)(-0.09,-0.10){4}{\line(0,-1){0.10}}
\multiput(14.99,39.99)(-0.10,-0.09){4}{\line(-1,0){0.10}}
\multiput(14.60,39.63)(-0.14,-0.11){3}{\line(-1,0){0.14}}
\multiput(14.17,39.29)(-0.15,-0.10){3}{\line(-1,0){0.15}}
\multiput(13.73,38.99)(-0.16,-0.09){3}{\line(-1,0){0.16}}
\multiput(13.26,38.73)(-0.24,-0.11){2}{\line(-1,0){0.24}}
\multiput(12.77,38.50)(-0.25,-0.10){2}{\line(-1,0){0.25}}
\multiput(12.27,38.31)(-0.26,-0.08){2}{\line(-1,0){0.26}}
\put(11.76,38.15){\line(-1,0){0.53}}
\put(11.23,38.04){\line(-1,0){0.53}}
\put(10.70,37.96){\line(-1,0){0.54}}
\put(10.16,37.93){\line(-1,0){0.54}}
\put(9.63,37.94){\line(-1,0){0.54}}
\put(9.09,37.99){\line(-1,0){0.53}}
\multiput(8.56,38.08)(-0.26,0.06){2}{\line(-1,0){0.26}}
\multiput(8.04,38.21)(-0.26,0.08){2}{\line(-1,0){0.26}}
\multiput(7.53,38.38)(-0.25,0.10){2}{\line(-1,0){0.25}}
\multiput(7.03,38.58)(-0.16,0.08){3}{\line(-1,0){0.16}}
\multiput(6.55,38.83)(-0.15,0.09){3}{\line(-1,0){0.15}}
\multiput(6.09,39.11)(-0.15,0.10){3}{\line(-1,0){0.15}}
\multiput(5.66,39.42)(-0.14,0.12){3}{\line(-1,0){0.14}}
\multiput(5.25,39.77)(-0.10,0.09){4}{\line(-1,0){0.10}}
\multiput(4.86,40.14)(-0.12,0.13){3}{\line(0,1){0.13}}
\multiput(4.51,40.55)(-0.11,0.14){3}{\line(0,1){0.14}}
\multiput(4.18,40.98)(-0.10,0.15){3}{\line(0,1){0.15}}
\multiput(3.90,41.43)(-0.08,0.16){3}{\line(0,1){0.16}}
\multiput(3.64,41.91)(-0.11,0.25){2}{\line(0,1){0.25}}
\multiput(3.43,42.40)(-0.09,0.25){2}{\line(0,1){0.25}}
\multiput(3.25,42.91)(-0.07,0.26){2}{\line(0,1){0.26}}
\put(3.11,43.42){\line(0,1){0.53}}
\put(3.01,43.95){\line(0,1){0.53}}
\put(2.95,44.49){\line(0,1){1.07}}
\put(2.95,45.56){\line(0,1){0.53}}
\put(3.02,46.10){\line(0,1){0.53}}
\multiput(3.12,46.62)(0.07,0.26){2}{\line(0,1){0.26}}
\multiput(3.26,47.14)(0.09,0.25){2}{\line(0,1){0.25}}
\multiput(3.44,47.65)(0.11,0.25){2}{\line(0,1){0.25}}
\multiput(3.66,48.14)(0.09,0.16){3}{\line(0,1){0.16}}
\multiput(3.92,48.61)(0.10,0.15){3}{\line(0,1){0.15}}
\multiput(4.21,49.06)(0.11,0.14){3}{\line(0,1){0.14}}
\multiput(4.54,49.49)(0.12,0.13){3}{\line(0,1){0.13}}
\multiput(4.90,49.89)(0.10,0.09){4}{\line(1,0){0.10}}
\multiput(5.28,50.27)(0.14,0.11){3}{\line(1,0){0.14}}
\multiput(5.70,50.61)(0.15,0.10){3}{\line(1,0){0.15}}
\multiput(6.13,50.92)(0.15,0.09){3}{\line(1,0){0.15}}
\multiput(6.60,51.20)(0.16,0.08){3}{\line(1,0){0.16}}
\multiput(7.08,51.44)(0.25,0.10){2}{\line(1,0){0.25}}
\multiput(7.57,51.64)(0.26,0.08){2}{\line(1,0){0.26}}
\multiput(8.09,51.81)(0.26,0.06){2}{\line(1,0){0.26}}
\put(8.61,51.93){\line(1,0){0.53}}
\put(9.14,52.02){\line(1,0){0.86}}
%\end
%\bezvec{104}(16.00,6.00)(26.67,0.00)(39.00,6.00)
\put(39.00,6.00){\vector(2,1){0.2}}
\multiput(16.00,6.00)(0.23,-0.12){9}{\line(1,0){0.23}}
\multiput(18.07,4.96)(0.30,-0.12){7}{\line(1,0){0.30}}
\multiput(20.17,4.14)(0.43,-0.12){5}{\line(1,0){0.43}}
\multiput(22.29,3.54)(0.54,-0.09){4}{\line(1,0){0.54}}
\multiput(24.45,3.16)(1.10,-0.08){2}{\line(1,0){1.10}}
\put(26.64,3.00){\line(1,0){2.22}}
\multiput(28.86,3.07)(0.75,0.10){3}{\line(1,0){0.75}}
\multiput(31.12,3.36)(0.46,0.10){5}{\line(1,0){0.46}}
\multiput(33.40,3.87)(0.33,0.10){7}{\line(1,0){0.33}}
\multiput(35.71,4.60)(0.27,0.12){12}{\line(1,0){0.27}}
%\end
%\bezvec{96}(38.00,12.00)(28.67,18.00)(17.00,12.00)
\put(17.00,12.00){\vector(-2,-1){0.2}}
\multiput(38.00,12.00)(-0.20,0.11){10}{\line(-1,0){0.20}}
\multiput(36.03,13.12)(-0.25,0.11){8}{\line(-1,0){0.25}}
\multiput(34.01,13.98)(-0.41,0.12){5}{\line(-1,0){0.41}}
\multiput(31.94,14.58)(-0.71,0.11){3}{\line(-1,0){0.71}}
\put(29.82,14.92){\line(-1,0){2.17}}
\multiput(27.65,14.99)(-1.11,-0.09){2}{\line(-1,0){1.11}}
\multiput(25.42,14.81)(-0.57,-0.11){4}{\line(-1,0){0.57}}
\multiput(23.15,14.37)(-0.39,-0.12){6}{\line(-1,0){0.39}}
\multiput(20.83,13.67)(-0.27,-0.12){14}{\line(-1,0){0.27}}
%\end
%\bezvec{156}(39.00,14.00)(19.67,21.33)(14.00,39.00)
\put(14.00,39.00){\vector(-1,3){0.2}}
\multiput(39.00,14.00)(-0.27,0.11){9}{\line(-1,0){0.27}}
\multiput(36.58,14.98)(-0.26,0.12){9}{\line(-1,0){0.26}}
\multiput(34.27,16.05)(-0.22,0.12){10}{\line(-1,0){0.22}}
\multiput(32.07,17.20)(-0.19,0.11){11}{\line(-1,0){0.19}}
\multiput(29.99,18.44)(-0.16,0.11){12}{\line(-1,0){0.16}}
\multiput(28.01,19.76)(-0.16,0.12){12}{\line(-1,0){0.16}}
\multiput(26.15,21.17)(-0.13,0.11){13}{\line(-1,0){0.13}}
\multiput(24.40,22.66)(-0.12,0.11){14}{\line(-1,0){0.12}}
\multiput(22.77,24.24)(-0.12,0.13){13}{\line(0,1){0.13}}
\multiput(21.24,25.90)(-0.12,0.15){12}{\line(0,1){0.15}}
\multiput(19.83,27.65)(-0.12,0.17){11}{\line(0,1){0.17}}
\multiput(18.53,29.48)(-0.12,0.19){10}{\line(0,1){0.19}}
\multiput(17.34,31.40)(-0.12,0.22){9}{\line(0,1){0.22}}
\multiput(16.27,33.40)(-0.11,0.23){9}{\line(0,1){0.23}}
\multiput(15.31,35.48)(-0.12,0.32){11}{\line(0,1){0.32}}
%\end
%\bezvec{164}(16.00,42.00)(37.67,33.00)(41.00,16.00)
\put(41.00,16.00){\vector(1,-4){0.2}}
\multiput(16.00,42.00)(0.26,-0.11){10}{\line(1,0){0.26}}
\multiput(18.57,40.87)(0.24,-0.12){10}{\line(1,0){0.24}}
\multiput(21.01,39.69)(0.21,-0.11){11}{\line(1,0){0.21}}
\multiput(23.31,38.44)(0.20,-0.12){11}{\line(1,0){0.20}}
\multiput(25.48,37.13)(0.17,-0.11){12}{\line(1,0){0.17}}
\multiput(27.51,35.77)(0.16,-0.12){12}{\line(1,0){0.16}}
\multiput(29.40,34.34)(0.14,-0.11){13}{\line(1,0){0.14}}
\multiput(31.16,32.86)(0.12,-0.12){13}{\line(1,0){0.12}}
\multiput(32.78,31.32)(0.11,-0.12){13}{\line(0,-1){0.12}}
\multiput(34.26,29.71)(0.11,-0.14){12}{\line(0,-1){0.14}}
\multiput(35.61,28.05)(0.11,-0.16){11}{\line(0,-1){0.16}}
\multiput(36.82,26.33)(0.12,-0.20){9}{\line(0,-1){0.20}}
\multiput(37.89,24.55)(0.12,-0.23){8}{\line(0,-1){0.23}}
\multiput(38.83,22.70)(0.11,-0.27){7}{\line(0,-1){0.27}}
\multiput(39.63,20.80)(0.11,-0.33){6}{\line(0,-1){0.33}}
\multiput(40.30,18.84)(0.12,-0.47){6}{\line(0,-1){0.47}}
%\end
%\bezvec{104}(6.00,16.00)(0.00,28.33)(6.00,39.00)
\put(6.00,39.00){\vector(1,2){0.2}}
\multiput(6.00,16.00)(-0.12,0.26){9}{\line(0,1){0.26}}
\multiput(4.96,18.36)(-0.12,0.33){7}{\line(0,1){0.33}}
\multiput(4.14,20.68)(-0.12,0.46){5}{\line(0,1){0.46}}
\multiput(3.54,22.98)(-0.09,0.57){4}{\line(0,1){0.57}}
\multiput(3.16,25.24)(-0.08,1.12){2}{\line(0,1){1.12}}
\put(3.00,27.47){\line(0,1){2.20}}
\multiput(3.07,29.67)(0.10,0.72){3}{\line(0,1){0.72}}
\multiput(3.36,31.85)(0.10,0.43){5}{\line(0,1){0.43}}
\multiput(3.87,33.99)(0.10,0.30){7}{\line(0,1){0.30}}
\multiput(4.60,36.10)(0.12,0.24){12}{\line(0,1){0.24}}
%\end
%\bezvec{96}(12.00,38.00)(17.67,25.00)(13.00,16.00)
\put(13.00,16.00){\vector(-1,-2){0.2}}
\multiput(12.00,38.00)(0.12,-0.30){9}{\line(0,-1){0.30}}
\multiput(13.07,35.34)(0.11,-0.32){8}{\line(0,-1){0.32}}
\multiput(13.91,32.76)(0.10,-0.42){6}{\line(0,-1){0.42}}
\multiput(14.53,30.27)(0.10,-0.60){4}{\line(0,-1){0.60}}
\multiput(14.93,27.86)(0.09,-1.16){2}{\line(0,-1){1.16}}
\put(15.10,25.54){\line(0,-1){2.23}}
\multiput(15.05,23.31)(-0.09,-0.71){3}{\line(0,-1){0.71}}
\multiput(14.77,21.17)(-0.10,-0.41){5}{\line(0,-1){0.41}}
\multiput(14.27,19.11)(-0.12,-0.28){11}{\line(0,-1){0.28}}
%\end
\put(5.00,33.00){\makebox(0,0)[lc]{$n/0$}}
\put(13.00,20.00){\makebox(0,0)[rc]{$1/0$}}
\put(23.00,26.00){\makebox(0,0)[lc]{$n/0$}}
\put(35.00,31.00){\makebox(0,0)[lc]{$2/0$}}
\put(27.00,5.00){\makebox(0,0)[cc]{$2/0$}}
\put(27.00,12.00){\makebox(0,0)[cc]{$1/0$}}
%\bezvec{124}(49.00,4.00)(63.67,-0.67)(52.00,9.00)
\put(52.00,9.00){\vector(-4,3){0.2}}
\multiput(49.00,4.00)(0.37,-0.11){6}{\line(1,0){0.37}}
\multiput(51.19,3.34)(0.46,-0.12){4}{\line(1,0){0.46}}
\multiput(53.05,2.87)(0.50,-0.10){3}{\line(1,0){0.50}}
\put(54.56,2.58){\line(1,0){1.17}}
\put(55.72,2.48){\line(1,0){0.82}}
\multiput(56.55,2.57)(0.16,0.09){3}{\line(1,0){0.16}}
\multiput(57.03,2.84)(0.07,0.23){2}{\line(0,1){0.23}}
\multiput(57.17,3.30)(-0.10,0.32){2}{\line(0,1){0.32}}
\multiput(56.97,3.94)(-0.11,0.17){5}{\line(0,1){0.17}}
\multiput(56.42,4.78)(-0.11,0.13){8}{\line(0,1){0.13}}
\multiput(55.53,5.79)(-0.11,0.11){11}{\line(-1,0){0.11}}
\multiput(54.30,7.00)(-0.14,0.12){17}{\line(-1,0){0.14}}
%\end
%\bezvec{188}(14.00,51.00)(10.67,73.33)(5.00,50.00)
\put(5.00,50.00){\vector(-1,-4){0.2}}
\multiput(14.00,51.00)(-0.09,0.56){4}{\line(0,1){0.56}}
\multiput(13.64,53.25)(-0.09,0.50){4}{\line(0,1){0.50}}
\multiput(13.27,55.23)(-0.10,0.43){4}{\line(0,1){0.43}}
\multiput(12.88,56.96)(-0.10,0.37){4}{\line(0,1){0.37}}
\multiput(12.48,58.44)(-0.10,0.30){4}{\line(0,1){0.30}}
\multiput(12.06,59.65)(-0.11,0.24){4}{\line(0,1){0.24}}
\multiput(11.64,60.60)(-0.11,0.17){4}{\line(0,1){0.17}}
\multiput(11.20,61.30)(-0.11,0.11){4}{\line(-1,0){0.11}}
\multiput(10.74,61.74)(-0.23,0.09){2}{\line(-1,0){0.23}}
\put(10.28,61.92){\line(-1,0){0.48}}
\multiput(9.80,61.84)(-0.16,-0.11){3}{\line(-1,0){0.16}}
\multiput(9.30,61.50)(-0.10,-0.12){5}{\line(0,-1){0.12}}
\multiput(8.80,60.90)(-0.10,-0.17){5}{\line(0,-1){0.17}}
\multiput(8.28,60.05)(-0.11,-0.22){5}{\line(0,-1){0.22}}
\multiput(7.74,58.94)(-0.11,-0.27){5}{\line(0,-1){0.27}}
\multiput(7.20,57.57)(-0.11,-0.33){5}{\line(0,-1){0.33}}
\multiput(6.64,55.94)(-0.11,-0.38){5}{\line(0,-1){0.38}}
\multiput(6.06,54.05)(-0.12,-0.45){9}{\line(0,-1){0.45}}
%\end
%\bezvec{128}(3.00,8.00)(-5.33,-5.67)(8.00,3.00)
\put(8.00,3.00){\vector(2,1){0.2}}
\multiput(3.00,8.00)(-0.12,-0.20){10}{\line(0,-1){0.20}}
\multiput(1.83,6.00)(-0.11,-0.22){8}{\line(0,-1){0.22}}
\multiput(0.93,4.27)(-0.11,-0.24){6}{\line(0,-1){0.24}}
\multiput(0.29,2.82)(-0.09,-0.30){4}{\line(0,-1){0.30}}
\put(-0.09,1.64){\line(0,-1){0.91}}
\multiput(-0.20,0.73)(0.08,-0.32){2}{\line(0,-1){0.32}}
\multiput(-0.05,0.09)(0.10,-0.09){4}{\line(1,0){0.10}}
\put(0.37,-0.27){\line(1,0){0.68}}
\multiput(1.05,-0.36)(0.47,0.09){2}{\line(1,0){0.47}}
\multiput(1.99,-0.18)(0.30,0.11){4}{\line(1,0){0.30}}
\multiput(3.20,0.28)(0.21,0.10){7}{\line(1,0){0.21}}
\multiput(4.68,1.00)(0.20,0.12){17}{\line(1,0){0.20}}
%\end
\put(10.00,65.00){\makebox(0,0)[cc]{$n/1$}}
\put(61.00,2.00){\makebox(0,0)[cc]{$2/1$}}
\put(10.00,0.00){\makebox(0,0)[cc]{$1/1$}}
\put(10.00,45.00){\makebox(0,0)[cc]{$n$}}
\put(45.00,10.00){\makebox(0,0)[cc]{$2$}}
\put(45.00,45.00){\makebox(0,0)[cc]{$\cdots$}}
\end{picture}
\end{center}
\caption{ \label{xx1}Automaton of the Mealy type.}\end{figure}


\subsubsection{Machine isomorphism, serial and parallel decompositions,
networks and universality}

Two automata
$M_1=(S_1,I_1,O_1,\delta_1,\lambda_1)$
and
$M_2=(S_2,I_2,O_2,\delta_2,\lambda_2)$
of the same type are {\em isomorphic}
if and only if there exist three one-to-one mappings
$f:S_1\longrightarrow S_2$,
$g:I_1\longrightarrow I_2$,
$h:O_1\longrightarrow O_2$ such that
$f[\delta_1(s_1,i_1)]=\delta_2 [f(s_1),g(i_1)]$ and
$h[\lambda_1(s_1,i_1)]=\lambda_2 [f(s_1),g(i_1)]$, where $s_j\in S_j$
and $i_j\in I_j$, $j\in \{1,2\}$.
The triple $(f,g,h)$ is an {\em isomorphism} between $M_1$ and $M_2$.
An isomorphism just renames the states, the inputs and the outputs.
From a purely input/output point of view, $g$ as well as $h$ (or
$h^{-1}$) are combinatory circuits and $M_1$ performs similarly to
the serial connection (see below) $h^{-1}M_2 g$ of the machines $g$,
$M_2$ and
$h^{-1}$.


The {\em serial connection} of the two machines
$M_1=(S_1,I_1,O_1,\delta_1,\lambda_1)$
and
$M_2=(S_2,I_2,O_2,\delta_2,\lambda_2)$
for which $O_1=I_2$
is the machine \cite[p. 42]{hartmanis}
$$M=M_1 \rightarrow M_2 = (S_1\times S_2, I_1, O_2,\delta ,\lambda )$$
where
$\delta [(s_1,s_2),i]=\left( \delta_1(s_1,i),\delta_2[s_2,\lambda
(s_1,i)]\right)$ and
$\lambda [(s_1,s_2),i]=\lambda_2[s_2,\lambda_1(s_1,i)]$.

The {\em parallel
 connection} of the two machines
$M_1=(S_1,I_1,O_1,\delta_1,\lambda_1)$
and
$M_2=(S_2,I_2,O_2,\delta_2,\lambda_2)$
is the machine \cite[p. 48]{hartmanis}
$$M=M_1 \| M_2 = (S_1\times S_2, I_1\times I_2,O_1\times O_2,\delta
,\lambda )$$
where
$\delta [(s_1,s_2),(i_1,i_2)]=(\delta_1(s_1,i_1),\delta_2(s_2,i_2))$
and
$\lambda [(s_1,s_2),(i_1,i_2)]=(\lambda_1(s_1,i_1),\lambda_2(s_2,i_2))$.



By suitable serial and parallel connections it is  possible to
construct networks of automata or
combinatorial circuits (gates) which are universal relative to the class
of Turing-computable algorithms.
That is, all algorithms computable on a Turing machine are computable by
serial and parallel connections of finite automata and {\it vice versa}.



\subsection{Construction of automaton partition logics}



\subsubsection*{Introduction by example}

Suppose that the only unknown feature of an automaton is its initial
state; all else is known. The automaton is presented in a black box,
with input and output interfaces. The task in this {\em complementarity
game} is to find (partial) information about the initial state of the
automaton \cite{e-f-moore}.
This is sometimes referred to as the state identification problem
\cite{conway,brauer-84}.

To illustrate this, consider the Mealy automaton $M_e$ discussed above.
Input/output experiments can be performed by inputting of one symbol $i$
(in this example, more inputs
yield no finer partitions).  Let us assume that one inputs $i = 5$. This
experiment is able to distinguish between state $s = 5$ and all the
other states; hence it induces a partition (suppose $n > 5$)
$$v(5) = \{ \{ 5 \} , \{1,2,3,4,6, ..., n\}\}.$$
After this experiment, information about the initial state is lost
(so that the model is ``irreversible'' in some sense).  Now
consider the partitions
$v(i)$ of all possible
experiments with one input x (all of them non-co-measurable).  Every one
of them generates a Boolean algebra of events with two atoms; e.g.,
$v(5)$ generates a four-element Boolean algebra $2^2$ whose Hasse
diagram is drawn in Fig. \ref{xx0}.
\begin{figure}
\begin{center}
%TexCad Options
%\grade{\off}
%\emlines{\off}
%\beziermacro{\off}
%\reduce{\on}
%\snapping{\on}
%\quality{6.00}
%\graddiff{0.01}
%\snapasp{1}
%\zoom{1.00}
\unitlength 1.00mm
\linethickness{0.4pt}
\begin{picture}(52.00,41.00)
\put(27.00,0.00){\circle*{2.00}}
\put(47.00,20.00){\circle*{2.00}}
\put(27.00,40.00){\circle*{2.00}}
\put(7.00,20.00){\circle*{2.00}}
\put(27.00,0.00){\line(1,1){20.00}}
\put(47.00,20.00){\line(-1,1){20.00}}
\put(27.00,40.00){\line(-1,-1){20.00}}
\put(7.00,20.00){\line(1,-1){20.00}}
\put(32.00,0.00){\makebox(0,0)[lc]{$0$}}
\put(32.00,40.00){\makebox(0,0)[lc]{$1=\{1,2,\ldots ,n\}$}}
\put(2.00,20.00){\makebox(0,0)[cc]{$\{5\}$}}
\put(52.00,20.00){\makebox(0,0)[lc]{$\{1,2,3,4,6,\ldots ,n\}$}}
\end{picture}
\end{center}
\caption{ \label{xx0} Boolean algebra $2^2$}
\end{figure}

The {\em automaton propositional calculus} and the associated
{\em partition logic} is the set of all partitions
$$P=\{ v(i) \mid i\in I\}.$$
Lattice theoretically, this amounts to a
{\em pasting} \cite{nav:91}
of all the $v(i)$'s.
In the specific example, the pasting is just the
 horizontal sum---only the least and greatest elements $0$ and $1$
of each $2^2$ are identified with each other---and
one obtains a Chinese lantern lattice $MO_n$.


\subsubsection*{Formal definition}
The logical structure of the complementarity game (initial-state
identification problem) can
be defined as follows. Let us call a proposition concerning the initial
state of the machine
{\em experimentally decidable} if there is an experiment $E$ which
determines the truth value of that proposition.
This can be done by performing $E$, i.e., by the input of a sequence of
input symbols $i_1,i_2,i_3,\ldots ,i_n$ associated with $E$, and by
observing the output sequence
$$\lambda_E(s)=\lambda(s,i_1),\lambda(\delta (s,i_1),i_2), \ldots
,\lambda(\underbrace{\delta
(\cdots
\delta
(s,i_1)\cdots ,i_{n-1})}_{n-1 \mbox{ times}},i_n).$$
The most general form of a prediction concerning the
initial state $s$
of the machine is that the initial state $s$ is contained in a subset
$P$ of the state set $S$.
Therefore, we may identify propositions concerning the initial state
with subsets of $S$.
A subset $P$ of $S$ is then  identified with the proposition that the
initial state is contained in $P$.

Let $E$ be an experiment (a preset or adaptive one), and let
$\lambda_E(s)$
denote the output obtained when one performs $E$ on an initial
state $s$.
$\lambda_E$ defines a mapping of $S$ to the set of output sequences
$O^*$. We define an equivalence relation on the state set $S$ by
$$s \stackrel{E}{\equiv} t \mbox{ iff }\lambda_E(s) = \lambda_E(t)$$
for any $s,t \in S$.
We denote the partition of $S$ corresponding to $\stackrel{E}{\equiv}$
by $S/\stackrel{E}{\equiv}$.
Obviously, the propositions decidable by the experiment $E$ are
the elements of the Boolean algebra generated by
$S/\stackrel{E}{\equiv}$, denoted by $B_E$.

There is also another way to construct the experimentally decidable
propositions of an experiment $E$.
Let $\lambda_E(P)  = \bigcup\limits_{s \in P}\lambda_E(s)$ be the direct
image of $P$ under $\lambda_E$ for any $P \subseteq S$.
We denote the direct image of $S$ by $O_E$; i.e.,  $O_E = \lambda_E(S)$.

It follows that the most general form of a prediction concerning
the outcome $W$ of the experiment $E$ is that $W$ lies in a subset of
$O_E$.
Therefore, the experimentally decidable propositions consist of all
inverse images $\lambda_E^{-1}(Q)$ of subsets $Q$ of $O_E$,
a procedure which can be constructively formulated (e.g., as an
effectively computable algorithm), and which also
leads to the Boolean algebra $B_E$.

Let ${\frak B}$ be the set of all Boolean algebras $B_E$.
We call the partition logic $R= (S,{\frak B})$ an {\em automaton
propositional calculus.} That is, we paste all Boolean subalgebras
together. For instance, in the particular example discussed above, the
Boolean subalgebras are $v(1),v(2),\ldots ,v(n)$.


If one does not know the automaton's initial state, one has to
choose which experiment to perform.
Computational complementarity manifests itself in the
following way.
Let us assume that no experiment gives a definite answer to the
initial-state identification problem.
(The classical ``initial value problem'' has a very different meaning in
physics.)
Suppose further that the actual performance of any one
experiment makes impossible all the other experimental
measurements---this can, for instance be achieved by irreversible
transition and output functions ($\delta$ and/or $\lambda$ are
many--to--one). Then the first (and only) experiment decides which one
of the possible observables is actually being measured.
``Observable'' here means a statement such as {\em ``the automaton is
in state $m$ or in state $n$.''}
 After
this measurement, the other remaining observables cannot be measured any
more.
We shall refer to such a class of observables as {\em complementary}
ones.

\subsection{Construction of quantum logics}

Quantum logic, as pioneered by Birkhoff and von Neumann
\cite{birkhoff-36}, is usually derived from Hilbert space. There, the
logical primitives, such as propositions and the logical operators ``and'',
``or'' and ``not'' are defined by Hilbert space entities. For instance,
consider the threedimensional, real Hilbert space ${\Bbb R}^3$ with the
usual
scalar product $(v,w):=\sum_{i=1}^3v_iw_i$, $v,w\in {\Bbb R}^3$.
Any
proposition is identified with a closed linear subspace of ${\Bbb R}^3$.
For instance, the zero
vector corresponds to a false statement. Any line spanned by a nonzero
vector corresponds to the statement that the physical system has an
observable property
associated with the projection operator corresponding to the
ondimensional subspace spanned by the vector. Any plane formed by
linear combinations
of two (non-collinear) vectors $v,w$ corresponds to the statement that
the
physical system has either the property corresponding to $v$ {\em or\/}
the property corresponding to
$w$. The whole Hilbert space ${\Bbb R}^3$ corresponds to the tautology
(true
propositions). The logical ``and''-operation is identified with the set
theoretical intersection of two propositions; e.g., with the intersection of
two planes. The logical ``not''-operation, or the ``complement'', is
identified with taking the orthogonal subspace; e.g., the complement of a
line is the planes orthogonal to that line.

In this top-down approach, one arrives at a propositional calculus which
resembles the classical one, but differs from it in several important
aspects. They are non-Boolean, i.e.,  non-distributive, algebraic
structures.

Furthermore, as was first pointed out by Kochen and Specker in the
context of partial algebras \cite{kochen1,ZirlSchl-65,redhead,mermin-93},
there exist certain
{\em finite\/} sets of lines, such that the
partial Boolean algebra generated by this set does not admit of any
monomorphism into the two--element Boolean algebra.

It has been demonstrated recently \cite{svozil-tkadlec} that no
Kochen-Specker-type constructions are possible in automaton partition
logic. This can be understood intuitively as arising from the
definiteness and
context-independence of any proposition regarding an automaton state:
automaton partition logic is nonclassical (e.g., nondistributive) but
context-independent. The context-dependence associated with the
Kochen-Specker construction is deeply rooted in the infinite
propositional structure of quantum logic derived from Hilbert space.
Although the explicit construction operates with a finite number of rays
(corresponding to elementary true-false propositions), it generates an
infinite number of such propositions \cite{havlicek}.

\subsection{Algebraic structure of logics}

Let $({\frak L}, \vee  ,\wedge  , ' ,0,1)$ be an algebraic structure.
Thereby, ${\frak L}$ is a non-empty set of elements to be interpreted
as propositions which are, at
least in principle, operational. $\vee  ,\wedge  $ are binary operations
interpretable as ``or'' and ``and,''  respectively.
$'$ is a unary operation interpretable as ``not.''
$0,1$ are elements of  ${\frak L}$ interpretes as the proposition
which is always false and always true (tautology), respectively.

 A  partially ordered set (poset)
 is a system ${\frak L}$ in which a binary order relation
 $\le$ (inverse $\ge$)
 is defined, which satisfies
(i)
$a\le a$, for all $a\in {\frak L}$ (reflexivity);
(ii)
$a\le  b$ and $ b\le c\Longrightarrow a\le c$ (transitivity);
(iii)
$a\le b$ and $b\le a\Longrightarrow a=b$ (antisymmetry).

A  partially ordered system ${\frak L}$ with order relation
 $\le $ (inverse $\ge$)
 is a  lattice if and only if any pair $a,b$  of its elements
 has
 (i)
 a greatest lower bound  $a\wedge  b$
 and
(ii)
a least upper bound
 $a\vee b$.

$a'$ is called the orthocomplement (orthogonal complement) of $a$, if
$a\vee
a'=1$, $a\wedge  a' =0$, $(a')'=a$ and if $a\le b\Rightarrow a' \ge b'$.
The structure $({\frak L}, \vee  ,\wedge  , ' ,0,1)$ is called the  {\em
ortholattice}.


A Boolean algebra is an ortholattice which satisfies the
distributive laws
$a \vee  (b\wedge  c)=(a\vee  b)\wedge  (a\vee  c)$ and
$a \wedge  (b\vee  c)=(a\wedge  b)\vee  (a\wedge  c)$.
A Boolean algebra with
$n$ atoms is denoted by
$2^n$. An {\em atom} $a$ of a lattice ${\frak L}$
covers the
least element
${
0}$; i.e.,
 $0\le a$ and that $0\le x\le a$ implies $x=a$.

The structure is modular if the modular law
$(a\vee b)\wedge  c= a\vee     (b \wedge     c)$
is satisfied for all $a\le c$.
The structure is orthomodular if the orthomodular law
$a \le b \Rightarrow b= a\vee  (b\wedge  a')$
is satisfied.


\subsection{Construction by examples}

Besides automaton logics, there are other ``quasi-classical'' examples
of non-Boolean algebras, such as
Wright's
generalized urn models \cite{wright,wright:pent}
and Aerts' models \cite{aerts}.
Another interesting example is
Cohen's ``firefly in a box'' scenario \cite{cohen} with a three-chamber
box
\cite{dvur-pul-svo}
as depicted in Fig. \ref{fig-2}.
\begin{figure}
\begin{center}
%TexCad Options
%\grade{\off}
%\emlines{\off}
%\beziermacro{\off}
%\reduce{\on}
%\snapping{\off}
%\quality{0.20}
%\graddiff{0.01}
%\snapasp{1}
%\zoom{3.00}
\unitlength 1.00mm
\linethickness{0.4pt}
\begin{picture}(41.17,45.00)
%\emline(0.17,10.00)(40.17,10.00)
\put(0.17,10.00){\line(1,0){40.00}}
%\end
%\emline(20.17,45.00)(20.17,45.00)
\put(20.17,45.00){\line(0,1){0.00}}
%\end
%\emline(20.17,45.00)(20.17,45.00)
\put(20.17,45.00){\line(0,1){0.00}}
%\end
%\emline(20.17,45.00)(40.17,10.00)
\multiput(20.17,45.00)(0.12,-0.21){167}{\line(0,-1){0.21}}
%\end
%\emline(40.17,10.00)(40.17,10.00)
\put(40.17,10.00){\line(0,1){0.00}}
%\end
%\emline(20.17,45.00)(0.17,10.00)
\multiput(20.17,45.00)(-0.12,-0.21){167}{\line(0,-1){0.21}}
%\end
%\emline(0.17,10.00)(0.17,10.00)
\put(0.17,10.00){\line(0,1){0.00}}
%\end
\put(5.17,5.00){\makebox(0,0)[cc]{$l_A$}}
\put(34.17,5.00){\makebox(0,0)[cc]{$r_A$}}
\put(41.17,16.00){\makebox(0,0)[cc]{$l_C$}}
\put(27.17,40.00){\makebox(0,0)[cc]{$r_C$}}
\put(12.17,40.00){\makebox(0,0)[cc]{$l_B$}}
\put(-0.83,16.00){\makebox(0,0)[cc]{$r_B$}}
\put(20.00,10.00){\line(0,1){5.00}}
\put(20.00,22.50){\line(0,-1){5.00}}
\put(26.00,25.50){\line(2,1){4.00}}
%\put(13.83,25.67){\line(-2,1){3.67}}
\put(13.83,25.50){\line(-2,1){3.67}}
\put(20.00,22.56){\line(-2,1){3.67}}
\put(20.00,22.56){\line(2,1){4.00}}
\put(20.00,0.00){\makebox(0,0)[cc]{window $A$}}
\put(37.67,30.00){\makebox(0,0)[lc]{window $C$}}
\put(3.67,30.00){\makebox(0,0)[rc]{window $B$}}
\end{picture}
\end{center}
\caption{Firefly in a three-chamber box. \label{fig-2}}
\end{figure}

The firefly flies around the three chambers.
Furthermore, it is free to light up or not to light up. The sides of the
box are windows with vertical
lines down their centers. Consider the three  experiments,
corresponding to the three windows $A$, $B$ and $C$.
For each experiment $E$, one records $l_E,\, r_E, \, n_E$ if one sees
a light to the left ($l_E$) or to the right ($r_E$) of the center line
or if one sees no light at all ($n_e$).
One can identify $r_A =l_C =: e$, $r_C = l_B =: c$,
$r_B = l_A =: a$ (but one should not identify $f:= n_A,\, b := n_B,\,
d:= n_C$).
The propositional logic of this model is represented by the
 Hasse diagram drawn in Fig. \ref{fig-3}.
\begin{figure}
\begin{center}
%TexCad Options
%\grade{\off}
%\emlines{\off}
%\beziermacro{\off}
%\reduce{\on}
%\snapping{\off}
%\quality{0.20}
%\graddiff{0.01}
%\snapasp{1}
%\zoom{2.00}
\unitlength 1.00mm
\linethickness{0.4pt}
\begin{picture}(103.00,84.00)
\put(8.00,38.00){\circle*{2.00}}
\put(23.00,38.00){\circle*{2.00}}
\put(38.00,38.00){\circle*{2.00}}
\put(53.00,38.00){\circle*{2.00}}
\put(68.00,38.00){\circle*{2.00}}
\put(83.00,38.00){\circle*{2.00}}
\put(98.00,38.00){\circle*{2.00}}
\put(98.00,58.00){\circle*{0.00}}
\put(98.00,58.00){\circle*{2.00}}
\put(83.00,58.00){\circle*{2.00}}
\put(68.00,58.00){\circle*{2.00}}
\put(53.00,58.00){\circle*{2.00}}
\put(38.00,58.00){\circle*{2.00}}
\put(23.00,58.00){\circle*{2.00}}
\put(8.00,58.00){\circle*{2.00}}
\put(53.00,78.00){\circle*{2.00}}
\put(53.00,18.00){\circle*{2.00}}
%\emline(53.00,18.00)(53.00,38.00)
\put(53.00,18.00){\line(0,1){20.00}}
%\end
%\emline(53.00,38.00)(38.00,58.00)
\multiput(53.00,38.00)(-0.12,0.16){126}{\line(0,1){0.16}}
%\end
%\emline(38.00,58.00)(23.00,38.00)
\multiput(38.00,58.00)(-0.12,-0.16){126}{\line(0,-1){0.16}}
%\end
%\emline(23.00,38.00)(8.00,58.00)
\multiput(23.00,38.00)(-0.12,0.16){126}{\line(0,1){0.16}}
%\end
%\emline(8.00,58.00)(38.00,38.00)
\multiput(8.00,58.00)(0.18,-0.12){167}{\line(1,0){0.18}}
%\end
%\emline(38.00,38.00)(23.00,58.00)
\multiput(38.00,38.00)(-0.12,0.16){126}{\line(0,1){0.16}}
%\end
%\emline(23.00,58.00)(8.00,38.00)
\multiput(23.00,58.00)(-0.12,-0.16){126}{\line(0,-1){0.16}}
%\end
%\emline(8.00,38.00)(38.00,58.00)
\multiput(8.00,38.00)(0.18,0.12){167}{\line(1,0){0.18}}
%\end
%\emline(38.00,58.00)(53.00,78.00)
\multiput(38.00,58.00)(0.12,0.16){126}{\line(0,1){0.16}}
%\end
%\emline(53.00,78.00)(53.00,58.00)
\put(53.00,78.00){\line(0,-1){20.00}}
%\end
%\emline(53.00,58.00)(38.00,38.00)
\multiput(53.00,58.00)(-0.12,-0.16){126}{\line(0,-1){0.16}}
%\end
%\emline(53.00,38.00)(53.00,38.00)
\put(53.00,38.00){\line(0,1){0.00}}
%\end
%\emline(38.00,58.00)(68.00,38.00)
\multiput(38.00,58.00)(0.18,-0.12){167}{\line(1,0){0.18}}
%\end
%\emline(68.00,38.00)(98.00,58.00)
\multiput(68.00,38.00)(0.18,0.12){167}{\line(1,0){0.18}}
%\end
%\emline(98.00,58.00)(83.00,38.00)
\multiput(98.00,58.00)(-0.12,-0.16){126}{\line(0,-1){0.16}}
%\end
%\emline(83.00,38.00)(68.00,58.00)
\multiput(83.00,38.00)(-0.12,0.16){126}{\line(0,1){0.16}}
%\end
%\emline(68.00,58.00)(98.00,38.00)
\multiput(68.00,58.00)(0.18,-0.12){167}{\line(1,0){0.18}}
%\end
%\emline(98.00,38.00)(83.00,58.00)
\multiput(98.00,38.00)(-0.12,0.16){126}{\line(0,1){0.16}}
%\end
%\emline(83.00,58.00)(68.00,38.00)
\multiput(83.00,58.00)(-0.12,-0.16){126}{\line(0,-1){0.16}}
%\end
%\emline(68.00,38.00)(53.00,18.00)
\multiput(68.00,38.00)(-0.12,-0.16){126}{\line(0,-1){0.16}}
%\end
%\emline(53.00,18.00)(38.00,38.00)
\multiput(53.00,18.00)(-0.12,0.16){126}{\line(0,1){0.16}}
%\end
\put(23.00,38.00){\vector(3,-2){30.00}}
\put(53.00,18.00){\vector(3,2){30.00}}
%\emline(98.00,38.00)(53.00,18.00)
\multiput(98.00,38.00)(-0.27,-0.12){167}{\line(-1,0){0.27}}
%\end
%\emline(53.00,18.00)(8.00,38.00)
\multiput(53.00,18.00)(-0.27,0.12){167}{\line(-1,0){0.27}}
%\end
%\emline(8.00,58.00)(53.00,78.00)
\multiput(8.00,58.00)(0.27,0.12){167}{\line(1,0){0.27}}
%\end
%\emline(53.00,78.00)(23.00,58.00)
\multiput(53.00,78.00)(-0.18,-0.12){167}{\line(-1,0){0.18}}
%\end
%\emline(68.00,58.00)(53.00,78.00)
\multiput(68.00,58.00)(-0.12,0.16){126}{\line(0,1){0.16}}
%\end
%\emline(53.00,78.00)(53.00,78.00)
\put(53.00,78.00){\line(0,1){0.00}}
%\end
%\emline(53.00,78.00)(83.00,58.00)
\multiput(53.00,78.00)(0.18,-0.12){167}{\line(1,0){0.18}}
%\end
%\emline(83.00,58.00)(83.00,58.00)
\put(83.00,58.00){\line(0,1){0.00}}
%\end
%\emline(98.00,58.00)(53.00,78.00)
\multiput(98.00,58.00)(-0.27,0.12){167}{\line(-1,0){0.27}}
%\end
%\emline(53.00,78.00)(53.00,78.00)
\put(53.00,78.00){\line(0,1){0.00}}
%\end
%\emline(53.00,58.00)(68.00,38.00)
\multiput(53.00,58.00)(0.12,-0.16){126}{\line(0,-1){0.16}}
%\end
\put(3.00,34.00){\makebox(0,0)[cc]{$a$}}
\put(19.00,34.00){\makebox(0,0)[cc]{$b$}}
\put(34.00,34.00){\makebox(0,0)[cc]{$c$}}
\put(50.00,34.00){\makebox(0,0)[cc]{$d$}}
\put(62.00,34.00){\makebox(0,0)[cc]{$e$}}
\put(101.00,34.00){\makebox(0,0)[cc]{$a$}}
\put(53.00,12.00){\makebox(0,0)[cc]{0}}
\put(53.00,84.00){\makebox(0,0)[cc]{1}}
\put(2.00,54.00){\makebox(0,0)[cc]{$a'$}}
\put(18.00,55.00){\makebox(0,0)[cc]{$b'$}}
\put(32.00,56.00){\makebox(0,0)[cc]{$c'$}}
\put(48.00,56.00){\makebox(0,0)[cc]{$d'$}}
\put(63.00,56.00){\makebox(0,0)[cc]{$e'$}}
\put(103.00,55.00){\makebox(0,0)[cc]{$a'$}}
\put(87.00,56.00){\makebox(0,0)[cc]{$f'$}}
\put(87.00,35.00){\makebox(0,0)[cc]{$f$}}
\put(38.00,38.00){\line(3,2){30.00}}
\put(53.00,38.00){\line(3,4){15.00}}
\end{picture}
\end{center}
\caption{Hasse diagram of the scenario for a  firefly in
a three-chamber box.
\label{fig-3}}
\end{figure}




\section{Miniatlas of low-complex Hasse diagrams}

The following miniatlas contains a sample collection of Hasse diagrams. It is
by no means intended as a complete collection of Hasse diagram features.

One difference between automaton logic and quantum logic should be kept
in mind. The Hasse diagrams originating from finite automata are finite
almost by definition. The Hasse diagrams originating from Hilbert-space
quantum mechanics \cite{birkhoff-36} are continuously
($\aleph_1$) infinite.
Furthermore, any finite quantum propositional structure which does not
allow a two-valued measure (classically interpretable as
the logical values ``true'' and ``false'') and therefore
implements a Kochen-Specker type contradiction is embedded
into an countably infinite ($\aleph_0$) propositional structure
\cite{svozil-tkadlec,havlicek}.
Therefore,  it will never be possible to
completely reduce quantum logic to automaton logic.

Nevertheless, finite structures are worth studying. They can serve
as models for complementarity. They show non-classical features not
observed
in quantum physics. For instance, the propositional structure needs not
be a partially ordered set  (cf. section \ref{n-tr}). It could be
transitive
and Boolean, but in a peculiar way feature complementary  (cf. section
\ref{B-c}).



It can be shown by a straightforward construction
\cite[pp. 154--155]{svozil-93} that every partition
logic corresponds to an automaton logic.


\subsection{Pastings}
\subsubsection{$\oplus_n 2^2$}


\subsubsection*{Hasse diagram}
$\;$\\
%\begin{figure}
\begin{center}
%TexCad Options
%\grade{\off}
%\emlines{\off}
%\beziermacro{\off}
%\reduce{\on}
%\snapping{\off}
%\quality{0.20}
%\graddiff{0.01}
%\snapasp{1}
%\zoom{0.90}
\unitlength 0.80mm
\linethickness{0.4pt}
\begin{picture}(171.67,126.67)
\put(85.00,0.00){\line(-3,2){60.00}}
\put(25.00,40.00){\line(3,2){60.00}}
\put(85.00,80.00){\line(-1,-1){40.00}}
\put(45.00,40.00){\line(1,-1){40.00}}
\put(85.00,0.00){\line(-1,2){20.00}}
\put(65.00,40.00){\line(1,2){20.00}}
\put(85.00,80.00){\line(-1,-5){8.00}}
\put(77.00,40.00){\line(1,-5){8.00}}
\put(85.00,0.00){\line(1,4){10.00}}
\put(95.00,40.00){\line(-1,4){9.92}}
\put(85.08,79.67){\line(3,-4){29.75}}
\put(114.83,40.00){\line(-3,-4){30.00}}
\put(84.83,0.00){\line(5,4){50.17}}
\put(135.00,40.13){\line(-5,4){50.00}}
\put(85.00,0.00){\line(5,3){67.00}}
\put(152.00,40.20){\line(-5,3){67.00}}
\put(85.00,80.00){\circle*{3.33}}
\put(25.00,40.00){\circle*{3.33}}
\put(45.00,40.00){\circle*{3.33}}
\put(65.00,40.00){\circle*{3.33}}
\put(77.00,40.00){\circle*{3.33}}
\put(95.00,40.00){\circle*{3.33}}
\put(115.00,40.00){\circle*{3.33}}
\put(135.00,40.00){\circle*{3.33}}
\put(152.33,40.00){\circle*{3.33}}
\put(85.00,-0.33){\circle*{3.33}}
\put(93.33,80.00){\makebox(0,0)[lc]{$1=\cup_{i=1}^n 1_i$}}
\put(93.33,0.00){\makebox(0,0)[lc]{$0 =\cup_{i=1}^n
0_i$}}
\put(25.00,35.00){\makebox(0,0)[cc]{$a_1$}}
\put(45.00,35.00){\makebox(0,0)[cc]{$b_1$}}
\put(65.00,35.00){\makebox(0,0)[cc]{$a_2$}}
\put(75.00,35.00){\makebox(0,0)[cc]{$b_2$}}
\put(97.00,35.00){\makebox(0,0)[cc]{$a_3$}}
\put(115.67,35.00){\makebox(0,0)[cc]{$b_3$}}
\put(135.00,35.00){\makebox(0,0)[cc]{$a_n$}}
\put(152.67,35.00){\makebox(0,0)[cc]{$b_n$}}
\put(14.67,95.00){\circle*{3.33}}
\put(-0.33,110.00){\circle*{3.33}}
\put(29.67,110.00){\circle*{3.33}}
\put(14.67,125.00){\circle*{3.33}}
\put(-0.33,110.00){\line(1,1){15.00}}
\put(14.67,125.00){\line(1,-1){15.00}}
\put(29.67,110.00){\line(-1,-1){15.00}}
\put(14.67,95.00){\line(-1,1){15.00}}
\put(21.00,95.00){\makebox(0,0)[cc]{$0_1 $}}
\put(21.00,125.00){\makebox(0,0)[cc]{$1_1$}}
\put(-0.33,104.67){\makebox(0,0)[cc]{$a_1$}}
\put(29.67,105.00){\makebox(0,0)[cc]{$b_1$}}
\put(35.50,110.00){\makebox(0,0)[cc]{$\oplus$}}
\put(55.92,95.00){\circle*{3.33}}
\put(98.83,95.00){\circle*{3.33}}
\put(155.00,95.00){\circle*{3.33}}
\put(40.92,110.00){\circle*{3.33}}
\put(83.83,110.00){\circle*{3.33}}
\put(140.00,110.00){\circle*{3.33}}
\put(70.92,110.00){\circle*{3.33}}
\put(113.83,110.00){\circle*{3.33}}
\put(170.00,110.00){\circle*{3.33}}
\put(55.92,125.00){\circle*{3.33}}
\put(98.83,125.00){\circle*{3.33}}
\put(155.00,125.00){\circle*{3.33}}
\put(40.92,110.00){\line(1,1){15.00}}
\put(83.83,110.00){\line(1,1){15.00}}
\put(140.00,110.00){\line(1,1){15.00}}
\put(55.92,125.00){\line(1,-1){15.00}}
\put(98.83,125.00){\line(1,-1){15.00}}
\put(155.00,125.00){\line(1,-1){15.00}}
\put(70.92,110.00){\line(-1,-1){15.00}}
\put(113.83,110.00){\line(-1,-1){15.00}}
\put(170.00,110.00){\line(-1,-1){15.00}}
\put(55.92,95.00){\line(-1,1){15.00}}
\put(98.83,95.00){\line(-1,1){15.00}}
\put(155.00,95.00){\line(-1,1){15.00}}
\put(62.25,95.00){\makebox(0,0)[cc]{$0_2 $}}
\put(105.16,95.00){\makebox(0,0)[cc]{$0 _3$}}
\put(161.33,95.00){\makebox(0,0)[cc]{$0 _n$}}
\put(62.25,125.00){\makebox(0,0)[cc]{$1_2$}}
\put(105.16,125.00){\makebox(0,0)[cc]{$1_3$}}
\put(161.33,125.00){\makebox(0,0)[cc]{$1_n$}}
\put(40.92,104.67){\makebox(0,0)[cc]{$a_2$}}
\put(83.83,104.67){\makebox(0,0)[cc]{$a_3$}}
\put(140.00,104.67){\makebox(0,0)[cc]{$a_n$}}
\put(70.92,105.00){\makebox(0,0)[cc]{$b_2$}}
\put(113.83,105.00){\makebox(0,0)[cc]{$b_3$}}
\put(170.00,105.00){\makebox(0,0)[cc]{$b_n$}}
\put(77.58,110.00){\makebox(0,0)[cc]{$\oplus$}}
\put(119.67,110.00){\makebox(0,0)[cc]{$\oplus$}}
\put(134.67,110.00){\makebox(0,0)[cc]{$\oplus$}}
\put(121.33,40.00){\circle*{0.67}}
\put(123.00,40.00){\circle*{0.67}}
\put(124.67,40.00){\circle*{0.67}}
\put(125.50,110.00){\circle*{0.67}}
\put(127.17,110.00){\circle*{0.67}}
\put(128.84,110.00){\circle*{0.67}}
\put(11.67,40.00){\makebox(0,0)[cc]{$=$}}
\end{picture}
\end{center}
%\caption{ \label{xx3}}\end{figure}


\subsubsection*{Realization}

\subsubsection*{(i) Quantum mechanics}
The quantum mechanics of spin-$1/2$ particles in $n$ different
directions. $\{ MO_n \mid MO_n= \oplus (2^2)^n , n\in {\Bbb N}\}$,
together with the trivial lattice $2^1$
form all {\em finite} sublattices of
twodimensional Hilbert space ${\Bbb R}^2$. [The complete sublattice
structure of ${\Bbb R}^2$ contains a continuum of (undenumerable many)
$2^2$;
$n\in
{\Bbb R}$ becomes a continuous variable.]

\subsubsection*{(ii) Partition (automaton) logic}
We return to the example at the start of Section
2.3: that is, to the partition$P$ on the states of $M_e$
\begin{eqnarray*}
P&=&\{
\{\{1\},\{2,3,\ldots, n\}\},       \\
&&\{\{2\},\{1,3,\ldots ,n\}\},       \\
&&\{\{3\},\{1,2, \ldots,n\}\},       \\
&&\vdots \\
&&\{\{n\},\{1,2,3,\ldots , n-1\}\}
\}.
\end{eqnarray*}


This lattice $MO_n$ occurs in quantum mechanics (logic) if one considers
the
measurement of the spin component of an electron in n directions.  So,
in a finitistic sense,
the ``Mealy electron'' $M_e$ defined in Fig. \ref{xx1} faithfully
represents the spin observables of an electron.
But  quantum mechanics supposes that the spin component of an electron
can be measured along an  arbitrary, continuous direction.  In this
sense, already two-dimensional Hilbert space implies that a complete
representation
of a quantum object such as spin cannot be given by finitistic entities.





\clearpage
\subsubsection{Horizontal sum $\oplus_n 2^3$}
Cf. below with $m=0$.
\subsubsection{$(\oplus_{i=1}^n 2^{3_i})\oplus (\oplus_{j=1}^m 2^{2_j})$}
\subsubsection*{Hasse diagram}
%TexCad Options
%\grade{\off}
%\emlines{\off}
%\beziermacro{\off}
%\reduce{\on}
%\snapping{\off}
%\quality{0.20}
%\graddiff{0.01}
%\snapasp{1}
%\zoom{1.00}
\unitlength 0.800mm
\linethickness{0.4pt}
\begin{picture}(151.06,121.06)
\put(0.00,100.00){\circle*{2.11}}
\put(40.00,100.00){\circle*{2.11}}
\put(10.00,100.00){\circle*{2.11}}
\put(50.00,100.00){\circle*{2.11}}
\put(20.00,100.00){\circle*{2.11}}
\put(60.00,100.00){\circle*{2.11}}
\put(0.00,110.00){\circle*{2.11}}
\put(40.00,110.00){\circle*{2.11}}
\put(10.00,110.00){\circle*{2.11}}
\put(50.00,110.00){\circle*{2.11}}
\put(20.00,110.00){\circle*{2.11}}
\put(60.00,110.00){\circle*{2.11}}
\put(10.00,119.67){\circle*{2.11}}
\put(50.00,119.67){\circle*{2.11}}
\put(10.00,90.00){\circle*{2.11}}
\put(50.00,90.00){\circle*{2.11}}
\put(10.00,90.00){\line(-1,1){10.00}}
\put(50.00,90.00){\line(-1,1){10.00}}
\put(0.00,100.00){\line(1,1){10.00}}
\put(40.00,100.00){\line(1,1){10.00}}
\put(10.00,110.00){\line(0,1){10.00}}
\put(50.00,110.00){\line(0,1){10.00}}
\put(10.00,120.00){\line(1,-1){10.00}}
\put(50.00,120.00){\line(1,-1){10.00}}
\put(20.00,110.00){\line(-2,-1){20.00}}
\put(60.00,110.00){\line(-2,-1){20.00}}
\put(10.00,90.00){\line(0,1){10.00}}
\put(50.00,90.00){\line(0,1){10.00}}
\put(10.00,100.00){\line(-1,1){10.00}}
\put(50.00,100.00){\line(-1,1){10.00}}
\put(0.00,110.00){\line(1,1){10.00}}
\put(40.00,110.00){\line(1,1){10.00}}
\put(10.00,100.00){\line(1,1){10.33}}
\put(50.00,100.00){\line(1,1){10.33}}
\put(10.00,110.00){\line(1,-1){10.00}}
\put(50.00,110.00){\line(1,-1){10.00}}
\put(20.00,100.00){\line(-2,1){20.00}}
\put(60.00,100.00){\line(-2,1){20.00}}
\put(20.00,100.00){\line(-1,-1){10.00}}
\put(60.00,100.00){\line(-1,-1){10.00}}
\put(0.00,95.00){\makebox(0,0)[cc]{$a_1$}}
\put(40.00,95.00){\makebox(0,0)[cc]{$a_n$}}
\put(13.00,97.33){\makebox(0,0)[cc]{$b_1$}}
\put(53.00,97.33){\makebox(0,0)[cc]{$b_n$}}
\put(20.00,95.00){\makebox(0,0)[cc]{$c_1$}}
\put(60.00,95.00){\makebox(0,0)[cc]{$c_n$}}
\put(15.00,90.00){\makebox(0,0)[cc]{$0_1$}}
\put(55.00,90.00){\makebox(0,0)[cc]{$0_n$}}
\put(0.00,115.00){\makebox(0,0)[cc]{$a_1'$}}
\put(40.00,115.00){\makebox(0,0)[cc]{$a_n'$}}
\put(13.00,112.33){\makebox(0,0)[cc]{$b_1'$}}
\put(53.00,112.33){\makebox(0,0)[cc]{$b_n'$}}
\put(20.00,115.00){\makebox(0,0)[cc]{$c_1'$}}
\put(60.00,115.00){\makebox(0,0)[cc]{$c_n'$}}
\put(15.33,120.00){\makebox(0,0)[cc]{$1_1$}}
\put(55.33,120.00){\makebox(0,0)[cc]{$1_n$}}
\put(25.00,105.00){\makebox(0,0)[cc]{$\oplus$}}
\put(35.00,105.00){\makebox(0,0)[cc]{$\oplus$}}
\put(20.00,40.00){\circle*{2.11}}
\put(30.00,40.00){\circle*{2.11}}
\put(40.00,40.00){\circle*{2.11}}
\put(20.00,50.00){\circle*{2.11}}
\put(30.00,50.00){\circle*{2.11}}
\put(40.00,50.00){\circle*{2.11}}
\put(20.00,40.00){\line(1,1){10.00}}
\put(40.00,50.00){\line(-2,-1){20.00}}
\put(30.00,40.00){\line(-1,1){10.00}}
\put(30.00,40.00){\line(1,1){10.33}}
\put(30.00,50.00){\line(1,-1){10.00}}
\put(40.00,40.00){\line(-2,1){20.00}}
\put(20.00,35.00){\makebox(0,0)[cc]{$a_1$}}
\put(33.00,37.33){\makebox(0,0)[cc]{$b_1$}}
%\put(40.00,35.00){\makebox(0,0)[cc]{$c_1$}}
\put(20.00,55.00){\makebox(0,0)[cc]{$a_1'$}}
\put(33.00,52.33){\makebox(0,0)[cc]{$b_1'$}}
%\put(40.00,55.00){\makebox(0,0)[cc]{$c_1'$}}
\put(40.00,40.00){\circle*{2.11}}
\put(60.00,40.00){\circle*{2.11}}
\put(80.00,40.00){\circle*{2.11}}
\put(70.00,40.00){\circle*{2.11}}
\put(60.00,40.00){\circle*{2.11}}
\put(80.00,40.00){\circle*{2.11}}
\put(40.00,50.00){\circle*{2.11}}
\put(60.00,50.00){\circle*{2.11}}
\put(80.00,50.00){\circle*{2.11}}
\put(70.00,50.00){\circle*{2.11}}
\put(60.00,50.00){\circle*{2.11}}
\put(80.00,50.00){\circle*{2.11}}
\put(60.00,40.00){\line(1,1){10.00}}
\put(80.00,50.00){\line(-2,-1){20.00}}
\put(70.00,40.00){\line(-1,1){10.00}}
\put(70.00,40.00){\line(1,1){10.33}}
\put(70.00,50.00){\line(1,-1){10.00}}
\put(80.00,40.00){\line(-2,1){20.00}}
\put(40.00,35.00){\makebox(0,0)[cc]{$c_1$}}
\put(60.00,35.00){\makebox(0,0)[cc]{$a_n$}}
\put(80.00,35.00){\makebox(0,0)[cc]{$c_n$}}
\put(73.00,37.33){\makebox(0,0)[cc]{$b_n$}}
\put(40.00,55.00){\makebox(0,0)[cc]{$c_1'$}}
\put(60.00,55.00){\makebox(0,0)[cc]{$a_n'$}}
\put(80.00,55.00){\makebox(0,0)[cc]{$c_n'$}}
\put(73.00,52.33){\makebox(0,0)[cc]{$b_n'$}}
\put(110.33,45.00){\circle*{0.67}}
\put(112.67,45.00){\circle*{0.67}}
\put(115.00,45.00){\circle*{0.67}}
\put(27.33,105.00){\circle*{0.67}}
\put(29.67,105.00){\circle*{0.67}}
\put(32.00,105.00){\circle*{0.67}}
\put(80.00,80.00){\circle*{2.11}}
\put(80.00,10.00){\circle*{2.11}}
\put(20.00,50.00){\line(2,1){60.00}}
\put(80.00,80.00){\line(-5,-3){50.00}}
\put(40.00,50.00){\line(4,3){40.00}}
\put(60.00,50.00){\line(2,3){20.00}}
\put(80.00,80.00){\line(-1,-3){10.00}}
\put(80.00,50.00){\line(0,1){30.00}}
\put(20.00,40.00){\line(2,-1){60.00}}
\put(80.00,10.00){\line(-5,3){50.00}}
\put(40.00,40.00){\line(4,-3){40.00}}
\put(60.00,40.00){\line(2,-3){20.00}}
\put(80.00,10.00){\line(-1,3){10.00}}
\put(80.00,40.00){\line(0,-1){30.00}}
\put(90.00,80.00){\makebox(0,0)[lc]{$1=\cup_{i=1}^{n+m} 1_i$}}
\put(90.00,10.00){\makebox(0,0)[lc]{$0=\cup_{i=1}^{n+m}0_i$}}
\put(10.00,45.00){\makebox(0,0)[cc]{$=$}}
%\put(20.00,10.00){\makebox(0,0)[lc]{Baroque diagram}}
\put(65.00,105.00){\makebox(0,0)[cc]{$\oplus$}}
\put(85.00,120.00){\circle*{2.11}}
\put(135.00,120.00){\circle*{2.11}}
\put(85.00,90.00){\circle*{2.11}}
\put(135.00,90.00){\circle*{2.11}}
\put(70.00,105.00){\circle*{2.11}}
\put(100.00,105.00){\circle*{2.11}}
\put(120.00,105.00){\circle*{2.11}}
\put(150.00,105.00){\circle*{2.11}}
\put(105.33,105.00){\makebox(0,0)[cc]{$\oplus$}}
\put(115.33,105.00){\makebox(0,0)[cc]{$\oplus$}}
\put(107.66,105.00){\circle*{0.67}}
\put(110.00,105.00){\circle*{0.67}}
\put(112.33,105.00){\circle*{0.67}}
\put(85.00,90.00){\line(-1,1){15.00}}
\put(70.00,105.00){\line(1,1){15.00}}
\put(85.00,120.00){\line(1,-1){15.00}}
\put(100.00,105.00){\line(-1,-1){15.00}}
\put(135.00,90.00){\line(-1,1){15.00}}
\put(120.00,105.00){\line(1,1){15.00}}
\put(135.00,120.00){\line(1,-1){15.00}}
\put(150.00,105.00){\line(-1,-1){15.00}}
\put(90.00,90.00){\makebox(0,0)[cc]{$0_{n+1}$}}
\put(140.00,90.00){\makebox(0,0)[cc]{$0_{n+m}$}}
\put(90.00,120.00){\makebox(0,0)[cc]{$1_{n+1}$}}
\put(140.00,120.00){\makebox(0,0)[cc]{$1_{n+m}$}}
\put(70.00,110.00){\makebox(0,0)[cc]{$d_1$}}
\put(100.00,110.00){\makebox(0,0)[cc]{$d_1'$}}
\put(120.00,110.00){\makebox(0,0)[cc]{$d_m$}}
\put(150.00,110.00){\makebox(0,0)[cc]{$d_m'$}}
\put(47.33,45.00){\circle*{0.67}}
\put(49.67,45.00){\circle*{0.67}}
\put(52.00,45.00){\circle*{0.67}}
\put(80.00,80.00){\line(1,-3){11.67}}
\put(91.67,45.00){\line(-1,-3){11.67}}
\put(80.00,80.00){\line(2,-3){23.33}}
\put(103.33,45.00){\line(-2,-3){23.33}}
\put(80.00,80.00){\line(4,-3){46.67}}
\put(126.67,45.00){\line(-4,-3){46.67}}
\put(80.00,10.00){\line(2,1){70.00}}
\put(150.00,45.00){\line(-2,1){70.00}}
\put(91.67,45.00){\circle*{2.11}}
\put(103.33,45.00){\circle*{2.11}}
\put(126.33,45.00){\circle*{2.11}}
\put(150.00,45.00){\circle*{2.11}}
\put(91.67,50.00){\makebox(0,0)[cc]{$d_1$}}
\put(103.33,50.00){\makebox(0,0)[cc]{$d_1'$}}
\put(126.33,50.00){\makebox(0,0)[cc]{$d_m$}}
\put(150.00,50.00){\makebox(0,0)[cc]{$d_m'$}}
\end{picture}
%\end{figure}

\subsubsection*{Realization}

\subsubsection*{(i) Quantum mechanics}
The lattices are not modular but orthomodular and have finite length.

\subsubsection*{(ii) Partition (automaton) logic}
Exercise.




\subsubsection{${\frak L}_{1n}=\oplus_{i=1}^n (2^{3_i})$}
\subsubsection*{Baroque Hasse diagram}
%\begin{figure}
\begin{center}
%TexCad Options
%\grade{\off}
%\emlines{\off}
%\beziermacro{\off}
%\reduce{\on}
%\snapping{\off}
%\quality{0.20}
%\graddiff{0.01}
%\snapasp{1}
%\zoom{1.00}
\unitlength 0.80mm
\linethickness{0.4pt}
\begin{picture}(151.06,120.73)
\put(0.00,100.00){\circle*{2.11}}
\put(10.00,100.00){\circle*{2.11}}
\put(20.00,100.00){\circle*{2.11}}
\put(0.00,110.00){\circle*{2.11}}
\put(10.00,110.00){\circle*{2.11}}
\put(20.00,110.00){\circle*{2.11}}
\put(10.00,119.67){\circle*{2.11}}
\put(10.00,90.00){\circle*{2.11}}
\put(10.00,90.00){\line(-1,1){10.00}}
\put(0.00,100.00){\line(1,1){10.00}}
\put(10.00,110.00){\line(0,1){10.00}}
\put(10.00,120.00){\line(1,-1){10.00}}
\put(20.00,110.00){\line(-2,-1){20.00}}
\put(10.00,90.00){\line(0,1){10.00}}
\put(10.00,100.00){\line(-1,1){10.00}}
\put(0.00,110.00){\line(1,1){10.00}}
\put(10.00,100.00){\line(1,1){10.33}}
\put(10.00,110.00){\line(1,-1){10.00}}
\put(20.00,100.00){\line(-2,1){20.00}}
\put(20.00,100.00){\line(-1,-1){10.00}}
\put(0.00,95.00){\makebox(0,0)[cc]{$a_1$}}
\put(13.00,97.33){\makebox(0,0)[cc]{$b_1$}}
\put(20.00,95.00){\makebox(0,0)[cc]{$a_2$}}
\put(15.00,90.00){\makebox(0,0)[cc]{$0_1$}}
\put(0.00,115.00){\makebox(0,0)[cc]{$a_1'$}}
\put(13.00,112.33){\makebox(0,0)[cc]{$b_1'$}}
\put(20.00,115.00){\makebox(0,0)[cc]{$a_2'$}}
\put(15.33,120.00){\makebox(0,0)[cc]{$1_1$}}
\put(30.00,100.00){\circle*{2.11}}
\put(60.00,100.00){\circle*{2.11}}
\put(90.00,100.00){\circle*{2.11}}
\put(130.00,100.00){\circle*{2.11}}
\put(40.00,100.00){\circle*{2.11}}
\put(70.00,100.00){\circle*{2.11}}
\put(100.00,100.00){\circle*{2.11}}
\put(140.00,100.00){\circle*{2.11}}
\put(50.00,100.00){\circle*{2.11}}
\put(80.00,100.00){\circle*{2.11}}
\put(110.00,100.00){\circle*{2.11}}
\put(150.00,100.00){\circle*{2.11}}
\put(30.00,110.00){\circle*{2.11}}
\put(60.00,110.00){\circle*{2.11}}
\put(90.00,110.00){\circle*{2.11}}
\put(130.00,110.00){\circle*{2.11}}
\put(40.00,110.00){\circle*{2.11}}
\put(70.00,110.00){\circle*{2.11}}
\put(100.00,110.00){\circle*{2.11}}
\put(140.00,110.00){\circle*{2.11}}
\put(50.00,110.00){\circle*{2.11}}
\put(80.00,110.00){\circle*{2.11}}
\put(110.00,110.00){\circle*{2.11}}
\put(150.00,110.00){\circle*{2.11}}
\put(40.00,119.67){\circle*{2.11}}
\put(70.00,119.67){\circle*{2.11}}
\put(100.00,119.67){\circle*{2.11}}
\put(140.00,119.67){\circle*{2.11}}
\put(40.00,90.00){\circle*{2.11}}
\put(70.00,90.00){\circle*{2.11}}
\put(100.00,90.00){\circle*{2.11}}
\put(140.00,90.00){\circle*{2.11}}
\put(40.00,90.00){\line(-1,1){10.00}}
\put(70.00,90.00){\line(-1,1){10.00}}
\put(100.00,90.00){\line(-1,1){10.00}}
\put(140.00,90.00){\line(-1,1){10.00}}
\put(30.00,100.00){\line(1,1){10.00}}
\put(60.00,100.00){\line(1,1){10.00}}
\put(90.00,100.00){\line(1,1){10.00}}
\put(130.00,100.00){\line(1,1){10.00}}
\put(40.00,110.00){\line(0,1){10.00}}
\put(70.00,110.00){\line(0,1){10.00}}
\put(100.00,110.00){\line(0,1){10.00}}
\put(140.00,110.00){\line(0,1){10.00}}
\put(40.00,120.00){\line(1,-1){10.00}}
\put(70.00,120.00){\line(1,-1){10.00}}
\put(100.00,120.00){\line(1,-1){10.00}}
\put(140.00,120.00){\line(1,-1){10.00}}
\put(50.00,110.00){\line(-2,-1){20.00}}
\put(80.00,110.00){\line(-2,-1){20.00}}
\put(110.00,110.00){\line(-2,-1){20.00}}
\put(150.00,110.00){\line(-2,-1){20.00}}
\put(40.00,90.00){\line(0,1){10.00}}
\put(70.00,90.00){\line(0,1){10.00}}
\put(100.00,90.00){\line(0,1){10.00}}
\put(140.00,90.00){\line(0,1){10.00}}
\put(40.00,100.00){\line(-1,1){10.00}}
\put(70.00,100.00){\line(-1,1){10.00}}
\put(100.00,100.00){\line(-1,1){10.00}}
\put(140.00,100.00){\line(-1,1){10.00}}
\put(30.00,110.00){\line(1,1){10.00}}
\put(60.00,110.00){\line(1,1){10.00}}
\put(90.00,110.00){\line(1,1){10.00}}
\put(130.00,110.00){\line(1,1){10.00}}
\put(40.00,100.00){\line(1,1){10.33}}
\put(70.00,100.00){\line(1,1){10.33}}
\put(100.00,100.00){\line(1,1){10.33}}
\put(140.00,100.00){\line(1,1){10.33}}
\put(40.00,110.00){\line(1,-1){10.00}}
\put(70.00,110.00){\line(1,-1){10.00}}
\put(100.00,110.00){\line(1,-1){10.00}}
\put(140.00,110.00){\line(1,-1){10.00}}
\put(50.00,100.00){\line(-2,1){20.00}}
\put(80.00,100.00){\line(-2,1){20.00}}
\put(110.00,100.00){\line(-2,1){20.00}}
\put(150.00,100.00){\line(-2,1){20.00}}
\put(50.00,100.00){\line(-1,-1){10.00}}
\put(80.00,100.00){\line(-1,-1){10.00}}
\put(110.00,100.00){\line(-1,-1){10.00}}
\put(150.00,100.00){\line(-1,-1){10.00}}
\put(30.00,95.00){\makebox(0,0)[cc]{$a_2$}}
\put(60.00,95.00){\makebox(0,0)[cc]{$a_3$}}
\put(90.00,95.00){\makebox(0,0)[cc]{$a_4$}}
\put(130.00,95.00){\makebox(0,0)[cc]{$a_n$}}
\put(43.00,97.33){\makebox(0,0)[cc]{$b_2$}}
\put(73.00,97.33){\makebox(0,0)[cc]{$b_3$}}
\put(103.00,97.33){\makebox(0,0)[cc]{$b_4$}}
\put(143.00,97.33){\makebox(0,0)[cc]{$b_n$}}
\put(50.00,95.00){\makebox(0,0)[cc]{$a_3$}}
\put(80.00,95.00){\makebox(0,0)[cc]{$a_4$}}
\put(110.00,95.00){\makebox(0,0)[cc]{$a_5$}}
\put(150.00,95.00){\makebox(0,0)[cc]{$c_n$}}
\put(45.00,90.00){\makebox(0,0)[cc]{$0_2$}}
\put(75.00,90.00){\makebox(0,0)[cc]{$0_3$}}
\put(105.00,90.00){\makebox(0,0)[cc]{$0_4$}}
\put(145.00,90.00){\makebox(0,0)[cc]{$0_n$}}
\put(30.00,115.00){\makebox(0,0)[cc]{$a_2'$}}
\put(60.00,115.00){\makebox(0,0)[cc]{$a_3'$}}
\put(90.00,115.00){\makebox(0,0)[cc]{$a_4'$}}
\put(130.00,115.00){\makebox(0,0)[cc]{$a_n'$}}
\put(43.00,112.33){\makebox(0,0)[cc]{$b_2'$}}
\put(73.00,112.33){\makebox(0,0)[cc]{$b_3'$}}
\put(103.00,112.33){\makebox(0,0)[cc]{$b_4'$}}
\put(143.00,112.33){\makebox(0,0)[cc]{$b_n'$}}
\put(50.00,115.00){\makebox(0,0)[cc]{$a_3'$}}
\put(80.00,115.00){\makebox(0,0)[cc]{$a_4'$}}
\put(110.00,115.00){\makebox(0,0)[cc]{$a_5'$}}
\put(150.00,115.00){\makebox(0,0)[cc]{$c_n'$}}
\put(45.33,120.00){\makebox(0,0)[cc]{$1_2$}}
\put(75.33,120.00){\makebox(0,0)[cc]{$1_3$}}
\put(105.33,120.00){\makebox(0,0)[cc]{$1_4$}}
\put(145.33,120.00){\makebox(0,0)[cc]{$1_n$}}
\put(25.00,105.00){\makebox(0,0)[cc]{$\oplus$}}
\put(55.00,105.00){\makebox(0,0)[cc]{$\oplus$}}
\put(85.00,105.00){\makebox(0,0)[cc]{$\oplus$}}
\put(115.00,105.00){\makebox(0,0)[cc]{$\oplus$}}
\put(125.00,105.00){\makebox(0,0)[cc]{$\oplus$}}
\put(20.00,40.00){\circle*{2.11}}
\put(30.00,40.00){\circle*{2.11}}
\put(40.00,40.00){\circle*{2.11}}
\put(20.00,50.00){\circle*{2.11}}
\put(30.00,50.00){\circle*{2.11}}
\put(40.00,50.00){\circle*{2.11}}
\put(20.00,40.00){\line(1,1){10.00}}
\put(40.00,50.00){\line(-2,-1){20.00}}
\put(30.00,40.00){\line(-1,1){10.00}}
\put(30.00,40.00){\line(1,1){10.33}}
\put(30.00,50.00){\line(1,-1){10.00}}
\put(40.00,40.00){\line(-2,1){20.00}}
\put(20.00,35.00){\makebox(0,0)[cc]{$a_1$}}
\put(33.00,37.33){\makebox(0,0)[cc]{$b_1$}}
%\put(40.00,35.00){\makebox(0,0)[cc]{$c_1$}}
\put(20.00,55.00){\makebox(0,0)[cc]{$a_1'$}}
\put(33.00,52.33){\makebox(0,0)[cc]{$b_1'$}}
%\put(40.00,55.00){\makebox(0,0)[cc]{$c_1'$}}
\put(40.00,40.00){\circle*{2.11}}
\put(60.00,40.00){\circle*{2.11}}
\put(80.00,40.00){\circle*{2.11}}
\put(50.00,40.00){\circle*{2.11}}
\put(70.00,40.00){\circle*{2.11}}
\put(90.00,40.00){\circle*{2.11}}
\put(60.00,40.00){\circle*{2.11}}
\put(80.00,40.00){\circle*{2.11}}
\put(100.00,40.00){\circle*{2.11}}
\put(40.00,50.00){\circle*{2.11}}
\put(60.00,50.00){\circle*{2.11}}
\put(80.00,50.00){\circle*{2.11}}
\put(50.00,50.00){\circle*{2.11}}
\put(70.00,50.00){\circle*{2.11}}
\put(90.00,50.00){\circle*{2.11}}
\put(60.00,50.00){\circle*{2.11}}
\put(80.00,50.00){\circle*{2.11}}
\put(100.00,50.00){\circle*{2.11}}
\put(40.00,40.00){\line(1,1){10.00}}
\put(60.00,40.00){\line(1,1){10.00}}
\put(80.00,40.00){\line(1,1){10.00}}
\put(60.00,50.00){\line(-2,-1){20.00}}
\put(80.00,50.00){\line(-2,-1){20.00}}
\put(100.00,50.00){\line(-2,-1){20.00}}
\put(50.00,40.00){\line(-1,1){10.00}}
\put(70.00,40.00){\line(-1,1){10.00}}
\put(90.00,40.00){\line(-1,1){10.00}}
\put(50.00,40.00){\line(1,1){10.33}}
\put(70.00,40.00){\line(1,1){10.33}}
\put(90.00,40.00){\line(1,1){10.33}}
\put(50.00,50.00){\line(1,-1){10.00}}
\put(70.00,50.00){\line(1,-1){10.00}}
\put(90.00,50.00){\line(1,-1){10.00}}
\put(60.00,40.00){\line(-2,1){20.00}}
\put(80.00,40.00){\line(-2,1){20.00}}
\put(100.00,40.00){\line(-2,1){20.00}}
\put(40.00,35.00){\makebox(0,0)[cc]{$a_2$}}
\put(60.00,35.00){\makebox(0,0)[cc]{$a_3$}}
\put(80.00,35.00){\makebox(0,0)[cc]{$a_4$}}
\put(53.00,37.33){\makebox(0,0)[cc]{$b_2$}}
\put(73.00,37.33){\makebox(0,0)[cc]{$b_3$}}
\put(93.00,37.33){\makebox(0,0)[cc]{$b_4$}}
\put(100.00,35.00){\makebox(0,0)[cc]{$a_5$}}
\put(40.00,55.00){\makebox(0,0)[cc]{$a_2'$}}
\put(60.00,55.00){\makebox(0,0)[cc]{$a_3'$}}
\put(80.00,55.00){\makebox(0,0)[cc]{$a_4'$}}
\put(53.00,52.33){\makebox(0,0)[cc]{$b_2'$}}
\put(73.00,52.33){\makebox(0,0)[cc]{$b_3'$}}
\put(93.00,52.33){\makebox(0,0)[cc]{$b_4'$}}
\put(100.00,55.00){\makebox(0,0)[cc]{$a_5'$}}
\put(120.00,40.00){\circle*{2.11}}
\put(130.00,40.00){\circle*{2.11}}
\put(140.00,40.00){\circle*{2.11}}
\put(120.00,50.00){\circle*{2.11}}
\put(130.00,50.00){\circle*{2.11}}
\put(140.00,50.00){\circle*{2.11}}
\put(120.00,40.00){\line(1,1){10.00}}
\put(140.00,50.00){\line(-2,-1){20.00}}
\put(130.00,40.00){\line(-1,1){10.00}}
\put(130.00,40.00){\line(1,1){10.33}}
\put(130.00,50.00){\line(1,-1){10.00}}
\put(140.00,40.00){\line(-2,1){20.00}}
\put(120.00,35.00){\makebox(0,0)[cc]{$a_n$}}
\put(133.00,37.33){\makebox(0,0)[cc]{$b_n$}}
\put(140.00,35.00){\makebox(0,0)[cc]{$c_n$}}
\put(120.00,55.00){\makebox(0,0)[cc]{$a_n'$}}
\put(133.00,52.33){\makebox(0,0)[cc]{$b_n'$}}
\put(140.00,55.00){\makebox(0,0)[cc]{$c_n'$}}
\put(107.33,45.00){\circle*{0.67}}
\put(109.67,45.00){\circle*{0.67}}
\put(112.00,45.00){\circle*{0.67}}
\put(117.33,105.00){\circle*{0.67}}
\put(119.67,105.00){\circle*{0.67}}
\put(122.00,105.00){\circle*{0.67}}
\put(80.00,80.00){\circle*{2.11}}
\put(80.00,10.00){\circle*{2.11}}
\put(20.00,50.00){\line(2,1){60.00}}
\put(80.00,80.00){\line(-5,-3){50.00}}
\put(40.00,50.00){\line(4,3){40.00}}
\put(80.00,80.00){\line(-1,-1){30.00}}
\put(60.00,50.00){\line(2,3){20.00}}
\put(80.00,80.00){\line(-1,-3){10.00}}
\put(80.00,50.00){\line(0,1){30.00}}
\put(80.00,80.00){\line(1,-3){10.00}}
\put(100.00,50.00){\line(-2,3){20.00}}
\put(80.00,80.00){\line(4,-3){40.00}}
\put(130.00,50.00){\line(-5,3){50.00}}
\put(80.00,80.00){\line(2,-1){60.00}}
\put(20.00,40.00){\line(2,-1){60.00}}
\put(80.00,10.00){\line(-5,3){50.00}}
\put(40.00,40.00){\line(4,-3){40.00}}
\put(80.00,10.00){\line(-1,1){30.00}}
\put(60.00,40.00){\line(2,-3){20.00}}
\put(80.00,10.00){\line(-1,3){10.00}}
\put(80.00,40.00){\line(0,-1){30.00}}
\put(80.00,10.00){\line(1,3){10.00}}
\put(100.00,40.00){\line(-2,-3){20.00}}
\put(80.00,10.00){\line(4,3){40.00}}
\put(130.00,40.00){\line(-5,-3){50.00}}
\put(80.00,10.00){\line(2,1){60.00}}
\put(90.00,80.00){\makebox(0,0)[lc]{$1=\cup_{i=1}^n 1_i$}}
\put(90.00,10.00){\makebox(0,0)[lc]{$0 =\cup_{i=1}^n0_i$}}
\put(10.00,45.00){\makebox(0,0)[cc]{$=$}}
%\put(20.00,10.00){\makebox(0,0)[lc]{Baroque diagram}}
\end{picture}
\end{center}
%\caption{ \label{xx4}}\end{figure}


\subsubsection*{Realization}

\subsubsection*{(i) Quantum mechanics}
${\frak L}_{12}$ is a
subortholattice of threedimensional Hilbert space ${\frak H}_3$.
It is therefore
embeddible into the quantum logic of
threedimensional
Hilbert space.
A quantum mechanical realization has been given
by Foulis
and Randall \cite[example III]{Foulis-Randall}. Consider a device which, from time
to time, emits a particle and projects it along a linear scale.
Suppose  two types of experiments are performed.
Experiment A measures whether or not  there is a particle present.
If there is no particle present, one records the
outcome of A as the symbol $a_2$. If there is, one measures its
position coordinate $x$. If $x\ge 1$, we record the outcome of $A$ as
the symbol $a_1$, otherwise one records the symbol $b_1$. Similarly
for experiment B: If no particle is present, one records the outcome
of B as the symbol $a_2$ (same as for no particle in A). If a particle
is detected, then one measures the
$x$--component
$p_x$ of its momentum. If $p_x \ge 1,$ one records $b_2$, otherwise
one records $a_3$. The resulting propositional logic is
${\frak L}_{12}$.
For a further physical realization, see
\cite[p. 159-162]{giuntini-91}.

${\frak L}_{1n>2}$ is not a
subortholattice of threedimensional Hilbert space ${\frak H}_3$
\cite{svozil-tkadlec}.
It is a nontrivial pasting. It is not  a horizontal sum as the logics
before.

\subsubsection*{(ii) Partition (automaton) logic}
We again mention that every partition logic corresponds to an automaton
logic.
In the next particular example, let
\begin{eqnarray*}
P&=&\{
\{\{1\},\{ 2\} ,\{ 3,\ldots, n\}\},       \\
&&\{\{ 2 \},\{ 3\},\{1,4,\ldots ,n\}\},       \\
&&\{\{ 3 \},\{ 4\},\{1,2,5,\ldots ,n\}\},       \\
&&\vdots \\
&&\{\{n-1\},\{ n\},\{ 1,2,3,\ldots , n-2\}\}
\}.
\end{eqnarray*}

One (but not the only one) particular way to construct a corresponding
automaton
logic would be to define a Mealy automaton with as many input symbols as
there are elements of $P$. The number of output symbols should be
three. The transition function could be trivial; i.e., $\delta (s)=1$
for all $s_in S$. The output function should reflect the partitions;
e.g.,
$\lambda (1,1)=1$,
$\lambda (2,1)=2$,
$\lambda (2,1)=
\cdots =
\lambda (2,1)=3$.

\clearpage

\subsubsection{$\oplus_{i=1}^2 2^{3_i}$}
\label{n-tr}


\subsubsection*{Hasse diagram}
%\begin{figure}
\begin{center}
%TexCad Options
%\grade{\off}
%\emlines{\off}
%\beziermacro{\off}
%\reduce{\on}
%\snapping{\on}
%\quality{6.00}
%\graddiff{0.01}
%\snapasp{1}
%\zoom{1.00}
\unitlength 1.00mm
\linethickness{0.4pt}
\begin{picture}(105.00,141.00)
\put(17.00,20.00){\circle*{2.00}}
\put(37.00,20.00){\circle*{2.00}}
\put(57.00,20.00){\circle*{2.00}}
\put(77.00,20.00){\circle*{2.00}}
\put(17.00,60.00){\circle*{2.00}}
\put(37.00,60.00){\circle*{2.00}}
\put(57.00,60.00){\circle*{2.00}}
\put(77.00,60.00){\circle*{2.00}}
\put(77.00,40.00){\circle*{2.00}}
\put(17.00,40.00){\circle*{2.00}}
\put(47.00,80.00){\circle*{2.00}}
\put(47.00,0.00){\circle*{2.00}}
\put(47.00,0.00){\line(-3,2){30.00}}
\put(17.00,20.00){\line(0,1){40.00}}
\put(17.00,60.00){\line(3,2){30.00}}
\put(47.00,80.00){\line(3,-2){30.00}}
\put(77.00,60.00){\line(0,-1){40.00}}
\put(77.00,20.00){\line(-3,-2){30.00}}
\put(47.00,0.00){\line(-1,2){10.00}}
\put(37.00,20.00){\line(1,1){40.00}}
\put(47.00,0.00){\line(1,2){10.00}}
\put(57.00,20.00){\line(-1,1){40.00}}
\put(47.00,80.00){\line(-1,-2){10.00}}
\put(37.00,60.00){\line(-1,-1){20.00}}
\put(17.00,40.00){\line(1,-1){20.00}}
\put(57.00,20.00){\line(1,1){20.00}}
\put(77.00,40.00){\line(-1,1){20.00}}
\put(57.00,60.00){\line(-1,2){10.00}}
\put(47.00,80.00){\line(0,0){0.00}}
\put(37.00,60.00){\line(1,-1){40.00}}
\put(57.00,60.00){\line(-1,-1){40.00}}
\put(0.00,110.00){\circle*{2.00}}
\put(20.00,110.00){\circle*{2.00}}
\put(20.00,95.00){\circle*{2.00}}
\put(20.00,140.00){\circle*{2.00}}
\put(25.00,95.00){\makebox(0,0)[lc]{$0$}}
\put(25.00,140.00){\makebox(0,0)[cc]{$1$}}
\put(40.00,110.00){\circle*{2.00}}
\put(0.00,125.00){\circle*{2.00}}
\put(20.00,125.00){\circle*{2.00}}
\put(40.00,125.00){\circle*{2.00}}
\put(20.00,95.00){\line(-4,3){20.00}}
\put(0.00,110.00){\line(4,3){20.00}}
\put(20.00,125.00){\line(4,-3){20.00}}
\put(40.00,110.00){\line(-4,-3){20.00}}
\put(20.00,95.00){\line(0,1){15.00}}
\put(20.00,110.00){\line(4,3){20.00}}
\put(40.00,125.00){\line(-4,3){20.00}}
\put(20.00,140.00){\line(-4,-3){20.00}}
\put(0.00,125.00){\line(4,-3){20.00}}
\put(40.00,110.00){\line(-5,2){40.00}}
\put(0.00,110.00){\line(5,2){40.00}}
\put(20.00,125.00){\line(0,1){15.00}}
\put(0.00,105.00){\makebox(0,0)[cc]{$a$}}
\put(23.00,105.00){\makebox(0,0)[lc]{$b$}}
\put(40.00,105.00){\makebox(0,0)[cc]{$\{c,d\}$}}
\put(40.00,130.00){\makebox(0,0)[cc]{$\{c,d\} '$}}
\put(23.00,130.00){\makebox(0,0)[lc]{$b'$}}
\put(0.00,130.00){\makebox(0,0)[cc]{$a'$}}
\put(55.00,110.00){\circle*{2.00}}
\put(75.00,110.00){\circle*{2.00}}
\put(75.00,95.00){\circle*{2.00}}
\put(75.00,140.00){\circle*{2.00}}
\put(80.00,95.00){\makebox(0,0)[lc]{$0$}}
\put(80.00,140.00){\makebox(0,0)[cc]{$1$}}
\put(95.00,110.00){\circle*{2.00}}
\put(55.00,125.00){\circle*{2.00}}
\put(75.00,125.00){\circle*{2.00}}
\put(95.00,125.00){\circle*{2.00}}
\put(75.00,95.00){\line(-4,3){20.00}}
\put(55.00,110.00){\line(4,3){20.00}}
\put(75.00,125.00){\line(4,-3){20.00}}
\put(95.00,110.00){\line(-4,-3){20.00}}
\put(75.00,95.00){\line(0,1){15.00}}
\put(75.00,110.00){\line(4,3){20.00}}
\put(95.00,125.00){\line(-4,3){20.00}}
\put(75.00,140.00){\line(-4,-3){20.00}}
\put(55.00,125.00){\line(4,-3){20.00}}
\put(95.00,110.00){\line(-5,2){40.00}}
\put(55.00,110.00){\line(5,2){40.00}}
\put(75.00,125.00){\line(0,1){15.00}}
\put(55.00,105.00){\makebox(0,0)[cc]{$\{a,b\}$}}
\put(78.00,105.00){\makebox(0,0)[lc]{$c$}}
\put(95.00,105.00){\makebox(0,0)[cc]{$d$}}
\put(95.00,130.00){\makebox(0,0)[cc]{$d'$}}
\put(78.00,130.00){\makebox(0,0)[lc]{$c'$}}
\put(55.00,130.00){\makebox(0,0)[cc]{$\{a,b\} '$}}
\put(47.00,118.00){\makebox(0,0)[cc]{$\oplus$}}
\put(105.00,120.00){\makebox(0,0)[cc]{$=$}}
\put(52.00,80.00){\makebox(0,0)[lc]{$1=1_1\cup 1_2$}}
\put(52.00,0.00){\makebox(0,0)[lc]{$0=0_1\cup 0_2$}}
\put(17.00,15.00){\makebox(0,0)[cc]{$a$}}
\put(37.00,15.00){\makebox(0,0)[cc]{$b$}}
\put(57.00,15.00){\makebox(0,0)[cc]{$c$}}
\put(77.00,15.00){\makebox(0,0)[cc]{$d$}}
\put(82.00,40.00){\makebox(0,0)[lc]{$\{c,d\}$}}
\put(12.00,40.00){\makebox(0,0)[rc]{$\{a,b\}$}}
\put(17.00,65.00){\makebox(0,0)[rc]{$d'=\{a,b,c\}$}}
\put(34.00,65.00){\makebox(0,0)[cc]{$c'=\{a,b,d\}$}}
\put(60.00,65.00){\makebox(0,0)[cc]{$b'=\{a,c,d\}$}}
\put(77.00,65.00){\makebox(0,0)[lc]{$a'=\{b,c,d\}$}}
\end{picture}
\end{center}
%\end{figure}


\subsubsection*{Realization}
\subsubsection*{Partition (automaton) logic}
\begin{eqnarray*}
P&=&\{
\{\{1\},\{2\},\{3,4\}\},       \\
&&\{\{1,2\},\{3\},\{4\}\}
\}
\end{eqnarray*}
The resulting propositional structure is not transitive, since there is
an experiment deciding the ``implication'' $1$ ``$\rightarrow$'' $
(1\vee
2)$ and another
one deciding the ``implication'' $(1\vee 2) $ ``$\rightarrow$'' $(1\vee 2\vee
3)$, but none deciding the ``implication''
$1 $ ``$\rightarrow$'' $(1\vee 2\vee 3)$.
The reason for this is that
the last ``relation'' is not experimentally testable.


\subsubsection{$\oplus_{i=1}^6 2^{3_i} =2^4$---A
classical Boolean system featuring complementarity}
\label{B-c}
\subsubsection*{Hasse diagram}
%\begin{figure}
\begin{center}
%TexCad Options
%\grade{\off}
%\emlines{\off}
%\beziermacro{\off}
%\reduce{\on}
%\snapping{\off}
%\quality{0.20}
%\graddiff{0.01}
%\snapasp{1}
%\zoom{0.80}
\unitlength 0.85mm
\linethickness{0.4pt}
\begin{picture}(171.06,125.73)
\put(0.00,105.00){\circle*{2.11}}
\put(10.00,105.00){\circle*{2.11}}
\put(20.00,105.00){\circle*{2.11}}
\put(0.00,115.00){\circle*{2.11}}
\put(10.00,115.00){\circle*{2.11}}
\put(20.00,115.00){\circle*{2.11}}
\put(10.00,124.67){\circle*{2.11}}
\put(10.00,95.00){\circle*{2.11}}
\put(10.00,95.00){\line(-1,1){10.00}}
\put(0.00,105.00){\line(1,1){10.00}}
\put(10.00,115.00){\line(0,1){10.00}}
\put(10.00,125.00){\line(1,-1){10.00}}
\put(20.00,115.00){\line(-2,-1){20.00}}
\put(10.00,95.00){\line(0,1){10.00}}
\put(10.00,105.00){\line(-1,1){10.00}}
\put(0.00,115.00){\line(1,1){10.00}}
\put(10.00,105.00){\line(1,1){10.33}}
\put(10.00,115.00){\line(1,-1){10.00}}
\put(20.00,105.00){\line(-2,1){20.00}}
\put(20.00,105.00){\line(-1,-1){10.00}}
\put(0.00,100.00){\makebox(0,0)[cc]{$a$}}
\put(13.00,102.33){\makebox(0,0)[cc]{$b$}}
\put(20.00,100.00){\makebox(0,0)[cc]{$\{c,d\}$}}
\put(15.00,95.00){\makebox(0,0)[cc]{$0$}}
\put(0.00,120.00){\makebox(0,0)[cc]{$a'$}}
\put(13.00,117.33){\makebox(0,0)[cc]{$b'$}}
\put(20.00,120.00){\makebox(0,0)[cc]{$\{c,d\} '$}}
\put(15.33,125.00){\makebox(0,0)[cc]{$1$}}
\put(30.00,105.00){\circle*{2.11}}
\put(60.00,105.00){\circle*{2.11}}
\put(90.00,105.00){\circle*{2.11}}
\put(40.00,105.00){\circle*{2.11}}
\put(70.00,105.00){\circle*{2.11}}
\put(100.00,105.00){\circle*{2.11}}
\put(50.00,105.00){\circle*{2.11}}
\put(80.00,105.00){\circle*{2.11}}
\put(110.00,105.00){\circle*{2.11}}
\put(30.00,115.00){\circle*{2.11}}
\put(60.00,115.00){\circle*{2.11}}
\put(90.00,115.00){\circle*{2.11}}
\put(40.00,115.00){\circle*{2.11}}
\put(70.00,115.00){\circle*{2.11}}
\put(100.00,115.00){\circle*{2.11}}
\put(50.00,115.00){\circle*{2.11}}
\put(80.00,115.00){\circle*{2.11}}
\put(110.00,115.00){\circle*{2.11}}
\put(40.00,124.67){\circle*{2.11}}
\put(70.00,124.67){\circle*{2.11}}
\put(100.00,124.67){\circle*{2.11}}
\put(40.00,95.00){\circle*{2.11}}
\put(70.00,95.00){\circle*{2.11}}
\put(100.00,95.00){\circle*{2.11}}
\put(40.00,95.00){\line(-1,1){10.00}}
\put(70.00,95.00){\line(-1,1){10.00}}
\put(100.00,95.00){\line(-1,1){10.00}}
\put(30.00,105.00){\line(1,1){10.00}}
\put(60.00,105.00){\line(1,1){10.00}}
\put(90.00,105.00){\line(1,1){10.00}}
\put(40.00,115.00){\line(0,1){10.00}}
\put(70.00,115.00){\line(0,1){10.00}}
\put(100.00,115.00){\line(0,1){10.00}}
\put(40.00,125.00){\line(1,-1){10.00}}
\put(70.00,125.00){\line(1,-1){10.00}}
\put(100.00,125.00){\line(1,-1){10.00}}
\put(50.00,115.00){\line(-2,-1){20.00}}
\put(80.00,115.00){\line(-2,-1){20.00}}
\put(110.00,115.00){\line(-2,-1){20.00}}
\put(40.00,95.00){\line(0,1){10.00}}
\put(70.00,95.00){\line(0,1){10.00}}
\put(100.00,95.00){\line(0,1){10.00}}
\put(40.00,105.00){\line(-1,1){10.00}}
\put(70.00,105.00){\line(-1,1){10.00}}
\put(100.00,105.00){\line(-1,1){10.00}}
\put(30.00,115.00){\line(1,1){10.00}}
\put(60.00,115.00){\line(1,1){10.00}}
\put(90.00,115.00){\line(1,1){10.00}}
\put(40.00,105.00){\line(1,1){10.33}}
\put(70.00,105.00){\line(1,1){10.33}}
\put(100.00,105.00){\line(1,1){10.33}}
\put(40.00,115.00){\line(1,-1){10.00}}
\put(70.00,115.00){\line(1,-1){10.00}}
\put(100.00,115.00){\line(1,-1){10.00}}
\put(50.00,105.00){\line(-2,1){20.00}}
\put(80.00,105.00){\line(-2,1){20.00}}
\put(110.00,105.00){\line(-2,1){20.00}}
\put(50.00,105.00){\line(-1,-1){10.00}}
\put(80.00,105.00){\line(-1,-1){10.00}}
\put(110.00,105.00){\line(-1,-1){10.00}}
\put(30.00,100.00){\makebox(0,0)[cc]{$a$}}
\put(60.00,100.00){\makebox(0,0)[cc]{$a$}}
\put(90.00,100.00){\makebox(0,0)[cc]{$b$}}
\put(43.00,102.33){\makebox(0,0)[cc]{$c$}}
\put(73.00,102.33){\makebox(0,0)[cc]{$d$}}
\put(103.00,102.33){\makebox(0,0)[cc]{$c$}}
\put(50.00,100.00){\makebox(0,0)[cc]{$\{b,d\}$}}
\put(80.00,100.00){\makebox(0,0)[cc]{$\{b,c\}$}}
\put(110.00,100.00){\makebox(0,0)[cc]{$\{a,d\}$}}
\put(45.00,95.00){\makebox(0,0)[cc]{$0$}}
\put(75.00,95.00){\makebox(0,0)[cc]{$0$}}
\put(105.00,95.00){\makebox(0,0)[cc]{$0$}}
\put(30.00,120.00){\makebox(0,0)[cc]{$a'$}}
\put(60.00,120.00){\makebox(0,0)[cc]{$a'$}}
\put(90.00,120.00){\makebox(0,0)[cc]{$b'$}}
\put(43.00,117.33){\makebox(0,0)[cc]{$c'$}}
\put(73.00,117.33){\makebox(0,0)[cc]{$d'$}}
\put(103.00,117.33){\makebox(0,0)[cc]{$c'$}}
\put(50.00,120.00){\makebox(0,0)[cc]{$\{b,d\} '$}}
\put(80.00,120.00){\makebox(0,0)[cc]{$\{b,c\} '$}}
\put(110.00,120.00){\makebox(0,0)[cc]{$\{a,d\} '$}}
\put(45.33,125.00){\makebox(0,0)[cc]{$1$}}
\put(75.33,125.00){\makebox(0,0)[cc]{$1$}}
\put(105.33,125.00){\makebox(0,0)[cc]{$1$}}
\put(25.00,110.00){\makebox(0,0)[cc]{$\oplus$}}
\put(55.00,110.00){\makebox(0,0)[cc]{$\oplus$}}
\put(85.00,110.00){\makebox(0,0)[cc]{$\oplus$}}
\put(115.00,110.00){\makebox(0,0)[cc]{$\oplus$}}
\put(15.00,45.00){\makebox(0,0)[cc]{$=$}}
\put(120.00,105.00){\circle*{2.11}}
\put(150.00,105.00){\circle*{2.11}}
\put(130.00,105.00){\circle*{2.11}}
\put(160.00,105.00){\circle*{2.11}}
\put(140.00,105.00){\circle*{2.11}}
\put(170.00,105.00){\circle*{2.11}}
\put(120.00,115.00){\circle*{2.11}}
\put(150.00,115.00){\circle*{2.11}}
\put(130.00,115.00){\circle*{2.11}}
\put(160.00,115.00){\circle*{2.11}}
\put(140.00,115.00){\circle*{2.11}}
\put(170.00,115.00){\circle*{2.11}}
\put(130.00,124.67){\circle*{2.11}}
\put(160.00,124.67){\circle*{2.11}}
\put(130.00,95.00){\circle*{2.11}}
\put(160.00,95.00){\circle*{2.11}}
\put(130.00,95.00){\line(-1,1){10.00}}
\put(160.00,95.00){\line(-1,1){10.00}}
\put(120.00,105.00){\line(1,1){10.00}}
\put(150.00,105.00){\line(1,1){10.00}}
\put(130.00,115.00){\line(0,1){10.00}}
\put(160.00,115.00){\line(0,1){10.00}}
\put(130.00,125.00){\line(1,-1){10.00}}
\put(160.00,125.00){\line(1,-1){10.00}}
\put(140.00,115.00){\line(-2,-1){20.00}}
\put(170.00,115.00){\line(-2,-1){20.00}}
\put(130.00,95.00){\line(0,1){10.00}}
\put(160.00,95.00){\line(0,1){10.00}}
\put(130.00,105.00){\line(-1,1){10.00}}
\put(160.00,105.00){\line(-1,1){10.00}}
\put(120.00,115.00){\line(1,1){10.00}}
\put(150.00,115.00){\line(1,1){10.00}}
\put(130.00,105.00){\line(1,1){10.33}}
\put(160.00,105.00){\line(1,1){10.33}}
\put(130.00,115.00){\line(1,-1){10.00}}
\put(160.00,115.00){\line(1,-1){10.00}}
\put(140.00,105.00){\line(-2,1){20.00}}
\put(170.00,105.00){\line(-2,1){20.00}}
\put(140.00,105.00){\line(-1,-1){10.00}}
\put(170.00,105.00){\line(-1,-1){10.00}}
\put(120.00,100.00){\makebox(0,0)[cc]{$b$}}
\put(150.00,100.00){\makebox(0,0)[cc]{$c$}}
\put(133.00,102.33){\makebox(0,0)[cc]{$d$}}
\put(163.00,102.33){\makebox(0,0)[cc]{$d$}}
\put(140.00,100.00){\makebox(0,0)[cc]{$\{a,c\}$}}
\put(170.00,100.00){\makebox(0,0)[cc]{$\{a,b\}$}}
\put(135.00,95.00){\makebox(0,0)[cc]{$0$}}
\put(165.00,95.00){\makebox(0,0)[cc]{$0$}}
\put(120.00,120.00){\makebox(0,0)[cc]{$b'$}}
\put(150.00,120.00){\makebox(0,0)[cc]{$c'$}}
\put(133.00,117.33){\makebox(0,0)[cc]{$d'$}}
\put(163.00,117.33){\makebox(0,0)[cc]{$d'$}}
\put(140.00,120.00){\makebox(0,0)[cc]{$\{a,c\} '$}}
\put(170.00,120.00){\makebox(0,0)[cc]{$\{a,b\} '$}}
\put(135.33,125.00){\makebox(0,0)[cc]{$1$}}
\put(165.33,125.00){\makebox(0,0)[cc]{$1$}}
\put(145.00,110.00){\makebox(0,0)[cc]{$\oplus$}}
\put(85.00,85.00){\circle*{2.11}}
\put(55.00,65.00){\circle*{2.11}}
\put(75.00,65.00){\circle*{2.11}}
\put(95.00,65.00){\circle*{2.11}}
\put(115.00,65.00){\circle*{2.11}}
\put(55.00,45.00){\circle*{2.11}}
\put(55.00,25.00){\circle*{2.11}}
\put(75.00,45.00){\circle*{2.11}}
\put(75.00,25.00){\circle*{2.11}}
\put(95.00,45.00){\circle*{2.11}}
\put(95.00,25.00){\circle*{2.11}}
\put(115.00,45.00){\circle*{2.11}}
\put(115.00,25.00){\circle*{2.11}}
\put(85.00,5.00){\circle*{2.11}}
\put(35.00,45.00){\circle*{2.11}}
\put(135.00,45.00){\circle*{2.11}}
\put(85.00,5.00){\line(-3,2){30.00}}
\put(55.00,25.00){\line(-1,1){20.00}}
\put(35.00,45.00){\line(1,1){20.00}}
\put(55.00,65.00){\line(3,2){30.00}}
\put(85.00,85.00){\line(3,-2){30.00}}
\put(115.00,65.00){\line(1,-1){20.00}}
\put(135.00,45.00){\line(-1,-1){20.00}}
\put(115.00,25.00){\line(-3,-2){30.00}}
\put(85.00,5.00){\line(-1,2){10.00}}
\put(75.00,25.00){\line(-2,1){40.00}}
\put(35.00,45.00){\line(2,1){40.00}}
\put(75.00,65.00){\line(2,-1){40.00}}
\put(115.00,45.00){\line(-2,-1){40.00}}
\put(75.00,25.00){\line(1,1){40.00}}
\put(115.00,65.00){\line(0,-1){40.00}}
\put(115.00,25.00){\line(-2,1){40.00}}
\put(75.00,45.00){\line(0,1){20.00}}
\put(75.00,65.00){\line(1,2){10.00}}
\put(85.00,85.00){\line(1,-2){10.00}}
\put(95.00,65.00){\line(-1,-1){40.00}}
\put(55.00,25.00){\line(0,1){40.00}}
\put(55.00,65.00){\line(2,-1){40.00}}
\put(85.00,5.00){\line(1,2){10.00}}
\put(95.00,25.00){\line(-2,1){40.00}}
\put(55.00,45.00){\line(2,1){40.00}}
\put(95.00,65.00){\line(2,-1){40.00}}
\put(135.00,45.00){\line(-2,-1){40.00}}
\put(95.00,25.00){\line(0,1){20.00}}
\put(95.00,5.00){\makebox(0,0)[cc]{$0$}}
\put(95.00,85.00){\makebox(0,0)[lc]{$1=\{a,b,c,d\}$}}
\put(55.00,20.00){\makebox(0,0)[cc]{$a$}}
\put(75.00,20.00){\makebox(0,0)[cc]{$b$}}
\put(95.00,20.00){\makebox(0,0)[cc]{$c$}}
\put(115.00,20.00){\makebox(0,0)[cc]{$d$}}
\put(35.00,40.00){\makebox(0,0)[cc]{$\{a,b\}$}}
\put(56.33,40.00){\makebox(0,0)[lc]{$\{a,c\}$}}
\put(75.00,40.00){\makebox(0,0)[cc]{$\{a,d\}$}}
\put(96.00,40.00){\makebox(0,0)[lc]{$\{b,c\}$}}
\put(116.00,40.00){\makebox(0,0)[lc]{$\{b,d\}$}}
\put(135.00,40.00){\makebox(0,0)[cc]{$\{c,d\}$}}
\put(115.00,70.00){\makebox(0,0)[lc]{$\{b,c,d\}=a'$}}
\put(97.00,70.00){\makebox(0,0)[cc]{$\{a,c,d\}=b'$}}
\put(73.00,70.00){\makebox(0,0)[cc]{$\{a,b,d\}=c'$}}
\put(55.00,70.00){\makebox(0,0)[rc]{$\{a,b,c\}=d'$}}
\end{picture}
\end{center}
%\caption{ \label{xx5}}\end{figure}

\subsubsection*{Realization}
\subsubsection*{Partition (automaton) logic}
\begin{eqnarray*}
P&=&\{
\{\{1\},\{2\},\{3,4\}\},       \\
&&\{\{1\},\{3\},\{2,4\}\},       \\
&&\{\{1\},\{4\},\{2,3\}\},       \\
&&\{\{2\},\{3\},\{1,4\}\},       \\
&&\{\{2\},\{4\},\{1,3\}\},       \\
&&\{\{3\},\{4\},\{1,2\}\}
\}
\end{eqnarray*}

The resulting propositional calculus is Boolean, but has
a non-classical feature of complementarity insofar as there exists no
experiment deciding between any one of the different initial states.
As in the last example, the reason for this feature is that
certain ``relations'' are not experimentally testable. That is, there is
simply no experiment which could be made to verify, for instance,
1 ``$\rightarrow$'' $\{1,2,3\}$,
although the statements
1 ``$\rightarrow$'' $\{1,3\}$ and
$\{1,3\}$ ``$\rightarrow$'' $\{1,2,3\}$ are testable singularly.

\subsection{Products}

\clearpage
\subsubsection{$2^1\otimes x=x$}
\subsubsection*{Hasse diagram}
%\begin{figure}
\begin{center}
%TexCad Options
%\grade{\off}
%\emlines{\off}
%\beziermacro{\off}
%\reduce{\on}
%\snapping{\on}
%\quality{0.20}
%\graddiff{0.01}
%\snapasp{1}
%\zoom{1.00}
\unitlength 1.00mm
\linethickness{0.4pt}
\begin{picture}(61.00,121.00)
\put(0.00,100.00){\circle*{2.00}}
\put(0.00,120.00){\circle*{2.00}}
\put(0.00,100.00){\line(0,1){20.00}}
\put(10.00,100.00){\circle*{2.00}}
\put(20.00,100.00){\circle*{2.00}}
\put(10.00,120.00){\circle*{2.00}}
\put(20.00,120.00){\circle*{2.00}}
\put(10.00,100.00){\line(0,1){20.00}}
\put(20.00,100.00){\line(0,1){20.00}}
\put(5.00,110.00){\makebox(0,0)[cc]{$\otimes$}}
\put(15.00,110.00){\makebox(0,0)[cc]{$=$}}
\put(0.00,70.00){\circle*{2.00}}
\put(0.00,90.00){\circle*{2.00}}
\put(0.00,70.00){\line(0,1){20.00}}
\put(5.00,80.00){\makebox(0,0)[cc]{$\otimes$}}
\put(20.00,70.00){\circle*{2.00}}
\put(20.00,90.00){\circle*{2.00}}
\put(10.00,80.00){\circle*{2.00}}
\put(30.00,80.00){\circle*{2.00}}
\put(20.00,70.00){\line(1,1){10.00}}
\put(30.00,80.00){\line(-1,1){10.00}}
\put(20.00,90.00){\line(-1,-1){10.00}}
\put(10.00,80.00){\line(1,-1){10.00}}
\put(3.00,120.00){\makebox(0,0)[cc]{$1$}}
\put(3.00,100.00){\makebox(0,0)[cc]{$0$}}
\put(13.00,120.00){\makebox(0,0)[cc]{$1$}}
\put(23.00,120.00){\makebox(0,0)[cc]{$1$}}
\put(13.00,100.00){\makebox(0,0)[cc]{$0$}}
\put(23.00,100.00){\makebox(0,0)[cc]{$0$}}
\put(3.00,90.00){\makebox(0,0)[cc]{$1$}}
\put(3.00,70.00){\makebox(0,0)[cc]{$0$}}
\put(23.00,90.00){\makebox(0,0)[cc]{$1$}}
\put(23.00,70.00){\makebox(0,0)[cc]{$0$}}
\put(10.00,75.00){\makebox(0,0)[cc]{$a$}}
\put(30.00,75.00){\makebox(0,0)[cc]{$b$}}
\put(35.00,80.00){\makebox(0,0)[cc]{$=$}}
\put(50.00,70.00){\circle*{2.00}}
\put(50.00,90.00){\circle*{2.00}}
\put(40.00,80.00){\circle*{2.00}}
\put(60.00,80.00){\circle*{2.00}}
\put(50.00,70.00){\line(1,1){10.00}}
\put(60.00,80.00){\line(-1,1){10.00}}
\put(50.00,90.00){\line(-1,-1){10.00}}
\put(40.00,80.00){\line(1,-1){10.00}}
\put(53.00,90.00){\makebox(0,0)[cc]{$1$}}
\put(53.00,70.00){\makebox(0,0)[cc]{$0$}}
\put(40.00,75.00){\makebox(0,0)[cc]{$a$}}
\put(60.00,75.00){\makebox(0,0)[cc]{$b$}}
\put(0.00,60.00){\makebox(0,0)[cc]{$\vdots $}}
\put(0.00,0.00){\circle*{2.00}}
\put(0.00,20.00){\circle*{2.00}}
\put(0.00,0.00){\line(0,1){20.00}}
\put(5.00,10.00){\makebox(0,0)[cc]{$\otimes$}}
\put(15.00,10.00){\makebox(0,0)[cc]{$=$}}
\put(3.00,20.00){\makebox(0,0)[cc]{$1$}}
\put(3.00,0.00){\makebox(0,0)[cc]{$0$}}
\put(10.00,10.00){\makebox(0,0)[cc]{$x$}}
\put(20.00,10.00){\makebox(0,0)[cc]{$x$}}
\put(0.00,30.00){\circle*{2.00}}
\put(0.00,50.00){\circle*{2.00}}
\put(0.00,30.00){\line(0,1){20.00}}
\put(5.00,40.00){\makebox(0,0)[cc]{$\otimes$}}
\put(15.00,40.00){\makebox(0,0)[cc]{$=$}}
\put(3.00,50.00){\makebox(0,0)[cc]{$1$}}
\put(3.00,30.00){\makebox(0,0)[cc]{$0$}}
\put(10.00,40.00){\makebox(0,0)[cc]{$2^n$}}
\put(20.00,40.00){\makebox(0,0)[cc]{$2^n$}}
\end{picture}
\end{center}
%\caption{ \label{xx6}}\end{figure}

\clearpage
\subsubsection{$2^2\otimes 2^2$}
\subsubsection*{Hasse diagram}
%\begin{figure}
\begin{center}
%TexCad Options
%\grade{\off}
%\emlines{\off}
%\beziermacro{\off}
%\reduce{\on}
%\snapping{\off}
%\quality{0.20}
%\graddiff{0.01}
%\snapasp{1}
%\zoom{1.00}
\unitlength 0.80mm
\linethickness{0.4pt}
\begin{picture}(145.00,201.42)
\put(25.00,30.00){\circle*{2.83}}
\put(55.00,30.00){\circle*{2.83}}
\put(85.00,30.00){\circle*{2.83}}
\put(115.00,30.00){\circle*{2.83}}
\put(25.00,60.00){\circle*{2.83}}
\put(25.00,90.00){\circle*{2.83}}
\put(55.00,60.00){\circle*{2.83}}
\put(55.00,90.00){\circle*{2.83}}
\put(85.00,60.00){\circle*{2.83}}
\put(85.00,90.00){\circle*{2.83}}
\put(115.00,60.00){\circle*{2.83}}
\put(115.00,90.00){\circle*{2.83}}
\put(25.00,30.00){\line(1,1){60.00}}
\put(85.00,90.00){\line(1,-1){30.00}}
\put(115.00,60.00){\line(-1,-1){30.00}}
\put(85.00,30.00){\line(-1,1){60.00}}
\put(25.00,90.00){\line(0,-1){60.00}}
\put(25.00,30.00){\line(3,-2){45.00}}
\put(70.00,0.00){\line(-1,2){15.00}}
\put(55.00,30.00){\line(-1,1){30.00}}
\put(25.00,60.00){\line(1,1){30.00}}
\put(55.00,90.00){\line(1,-1){60.00}}
\put(115.00,30.00){\line(0,1){60.00}}
\put(115.00,90.00){\line(-1,-1){60.00}}
\put(85.00,30.00){\line(-1,-2){15.00}}
\put(70.00,0.00){\line(3,2){45.00}}
\put(25.00,90.00){\line(3,2){45.00}}
\put(70.00,120.00){\line(-1,-2){15.00}}
\put(70.00,120.00){\line(1,-2){15.00}}
\put(70.00,120.00){\line(3,-2){45.00}}
\put(70.00,120.00){\circle*{2.83}}
\put(70.00,0.00){\circle*{2.83}}
\put(30.00,30.00){\makebox(0,0)[lc]{$a_1$}}
\put(60.00,30.00){\makebox(0,0)[lc]{$a_2$}}
\put(90.00,30.00){\makebox(0,0)[lc]{$b_2$}}
\put(120.00,30.00){\makebox(0,0)[lc]{$b_1$}}
\put(30.00,60.00){\makebox(0,0)[lc]{$\{ a_1,a_2\}$}}
\put(60.00,60.00){\makebox(0,0)[lc]{$\{ a_1,b_2\}$}}
\put(90.00,60.00){\makebox(0,0)[lc]{$\{ b_1,a_2\}$}}
\put(120.00,60.00){\makebox(0,0)[lc]{$\{ b_1,b_2\}$}}
\put(30.00,90.00){\makebox(0,0)[lc]{$\{ a_1,1_2\}$}}
\put(60.00,90.00){\makebox(0,0)[lc]{$\{ 1_1,a_2\}$}}
\put(90.00,90.00){\makebox(0,0)[lc]{$\{ 1_1,b_2\}$}}
\put(120.00,90.00){\makebox(0,0)[lc]{$\{ b_1,1_2\}$}}
\put(80.00,120.00){\makebox(0,0)[lc]{$1=\{ 1_1,1_2\}$}}
\put(80.00,0.00){\makebox(0,0)[lc]{$0 =\{ 0_1,
0_2\}$}}
\put(30.00,140.00){\circle*{2.83}}
\put(0.00,170.00){\circle*{2.83}}
\put(30.00,200.00){\circle*{2.83}}
\put(60.00,170.00){\circle*{2.83}}
\put(30.00,140.00){\line(-1,1){30.00}}
\put(0.00,170.00){\line(1,1){30.00}}
\put(35.00,140.00){\makebox(0,0)[lc]{$0_1$}}
\put(5.00,170.00){\makebox(0,0)[lc]{$a_1$}}
\put(65.00,170.00){\makebox(0,0)[lc]{$b_1$}}
\put(35.00,200.00){\makebox(0,0)[lc]{$1_1$}}
\put(30.00,140.00){\line(1,1){30.00}}
\put(60.00,170.00){\line(-1,1){30.00}}
\put(110.00,140.00){\circle*{2.83}}
\put(80.00,170.00){\circle*{2.83}}
\put(110.00,200.00){\circle*{2.83}}
\put(140.00,170.00){\circle*{2.83}}
\put(110.00,140.00){\line(-1,1){30.00}}
\put(80.00,170.00){\line(1,1){30.00}}
\put(115.00,140.00){\makebox(0,0)[lc]{$0_2$}}
\put(85.00,170.00){\makebox(0,0)[lc]{$a_2$}}
\put(145.00,170.00){\makebox(0,0)[lc]{$b_2$}}
\put(115.00,200.00){\makebox(0,0)[lc]{$1_2$}}
\put(110.00,140.00){\line(1,1){30.00}}
\put(140.00,170.00){\line(-1,1){30.00}}
\put(73.00,170.00){\makebox(0,0)[cc]{$\bigotimes$}}
\put(10.00,60.00){\makebox(0,0)[cc]{$=$}}
\end{picture}
\end{center}
%\caption{ \label{xx7}}\end{figure}

$\{1_1\}=\{a_1,b_1\}$ and
$\{1_2\}=\{a_2,b_2\}$
do not belong to the diagram, because---as they do not include
propositions about the second or first automaton factor---they cannot
be realized in any experiment.


\subsubsection*{Realization}
\subsubsection*{Partition (automaton) logic}

\begin{eqnarray*}
P&=&\{
\{\{1\},\{2\},\{3,4\}\},\\
&&\{\{1\},\{3\},\{2,4\}\},\\
&&\{\{2\},\{4\},\{1,3\}\},  \\
&&\{\{3\},\{4\},\{1,2\}\}
\}
\end{eqnarray*}

%\begin{figure}
%\begin{center}
%TexCad Options
%\grade{\off}
%\emlines{\off}
%\beziermacro{\off}
%\reduce{\on}
%\snapping{\off}
%\quality{0.20}
%\graddiff{0.01}
%\snapasp{1}
%\zoom{1.00}
%\unitlength 0.80mm
%\linethickness{0.4pt}
%\begin{picture}(145.00,201.42)
%\put(25.00,30.00){\circle*{2.83}}
%\put(55.00,30.00){\circle*{2.83}}
%\put(85.00,30.00){\circle*{2.83}}
%\put(115.00,30.00){\circle*{2.83}}
%\put(25.00,60.00){\circle*{2.83}}
%\put(25.00,90.00){\circle*{2.83}}
%\put(55.00,60.00){\circle*{2.83}}
%\put(55.00,90.00){\circle*{2.83}}
%\put(85.00,60.00){\circle*{2.83}}
%\put(85.00,90.00){\circle*{2.83}}
%\put(115.00,60.00){\circle*{2.83}}
%\put(115.00,90.00){\circle*{2.83}}
%\put(25.00,30.00){\line(1,1){60.00}}
%\put(85.00,90.00){\line(1,-1){30.00}}
%\put(115.00,60.00){\line(-1,-1){30.00}}
%\put(85.00,30.00){\line(-1,1){60.00}}
%\put(25.00,90.00){\line(0,-1){60.00}}
%\put(25.00,30.00){\line(3,-2){45.00}}
%\put(70.00,0.00){\line(-1,2){15.00}}
%\put(55.00,30.00){\line(-1,1){30.00}}
%\put(25.00,60.00){\line(1,1){30.00}}
%\put(55.00,90.00){\line(1,-1){60.00}}
%\put(115.00,30.00){\line(0,1){60.00}}
%\put(115.00,90.00){\line(-1,-1){60.00}}
%\put(85.00,30.00){\line(-1,-2){15.00}}
%\put(70.00,0.00){\line(3,2){45.00}}
%\put(25.00,90.00){\line(3,2){45.00}}
%\put(70.00,120.00){\line(-1,-2){15.00}}
%\put(70.00,120.00){\line(1,-2){15.00}}
%\put(70.00,120.00){\line(3,-2){45.00}}
%\put(70.00,120.00){\circle*{2.83}}
%\put(70.00,0.00){\circle*{2.83}}
%\put(30.00,30.00){\makebox(0,0)[lc]{$\{1\}$}}
%\put(60.00,30.00){\makebox(0,0)[lc]{$\{2\}$}}
%\put(90.00,30.00){\makebox(0,0)[lc]{$\{3\}$}}
%\put(120.00,30.00){\makebox(0,0)[lc]{$\{4\}$}}
%\put(30.00,60.00){\makebox(0,0)[lc]{$\{1,2\}$}}
%\put(60.00,60.00){\makebox(0,0)[lc]{$\{1,3\}$}}
%\put(90.00,60.00){\makebox(0,0)[lc]{$\{2,4\}$}}
%\put(120.00,60.00){\makebox(0,0)[lc]{$\{3,4\}$}}
%\put(30.00,90.00){\makebox(0,0)[lc]{$\{1,2,3\}$}}
%\put(60.00,90.00){\makebox(0,0)[lc]{$\{1,2,4\}$}}
%\put(90.00,90.00){\makebox(0,0)[lc]{$\{1,3,4\}$}}
%\put(120.00,90.00){\makebox(0,0)[lc]{$\{2,3,4\}$}}
%\put(80.00,120.00){\makebox(0,0)[lc]{$\{1,2,3,4\}$}}
%\put(80.00,0.00){\makebox(0,0)[lc]{$0$}}
%\put(30.00,140.00){\circle*{2.83}}
%\put(0.00,170.00){\circle*{2.83}}
%\put(30.00,200.00){\circle*{2.83}}
%\put(60.00,170.00){\circle*{2.83}}
%\put(30.00,140.00){\line(-1,1){30.00}}
%\put(0.00,170.00){\line(1,1){30.00}}
%\put(35.00,140.00){\makebox(0,0)[lc]{$0$}}
%\put(5.00,170.00){\makebox(0,0)[lc]{$\{1\}$}}
%\put(65.00,170.00){\makebox(0,0)[cc]{$\{4\}$}}
%\put(35.00,200.00){\makebox(0,0)[lc]{$\{1,4\}$}}
%\put(30.00,140.00){\line(1,1){30.00}}
%\put(60.00,170.00){\line(-1,1){30.00}}
%\put(110.00,140.00){\circle*{2.83}}
%\put(80.00,170.00){\circle*{2.83}}
%\put(110.00,200.00){\circle*{2.83}}
%\put(140.00,170.00){\circle*{2.83}}
%\put(110.00,140.00){\line(-1,1){30.00}}
%\put(80.00,170.00){\line(1,1){30.00}}
%\put(115.00,140.00){\makebox(0,0)[lc]{$0$}}
%\put(85.00,170.00){\makebox(0,0)[lc]{$\{2\}$}}
%\put(145.00,170.00){\makebox(0,0)[lc]{$\{3\}$}}
%\put(115.00,200.00){\makebox(0,0)[lc]{$\{2,3\}$}}
%\put(110.00,140.00){\line(1,1){30.00}}
%\put(140.00,170.00){\line(-1,1){30.00}}
%\put(75.00,170.00){\makebox(0,0)[cc]{{\large $\otimes$}}}
%\put(10.00,60.00){\makebox(0,0)[cc]{$=$}}
%\end{picture}
%\end{center}
%%\caption{ \label{xx8}}\end{figure}


 One automaton realization is the Mealy automaton $M$, which can be
 parallel decomposed into two Mealy automata $M_1, M_2$ such that
 $M=M_1 \| M_2$ according to
 %\begin{figure}
\begin{center}
$M=$
\begin{tabular}{|c|cccc||cccc|}
 \hline
s/i &1&2&3&4 & 1&2&3&4\\
 \hline
1&1&2&3&4 & 1&1&1&1\\
2&1&2&3&4 & 2&2&2&1\\
3&1&2&3&4 & 3&3&1&2\\
4&1&2&3&4 & 3&2&3&3\\
 \hline
\end{tabular}
,\\
$\,$\\
$M_1=$
\begin{tabular}{|c|cc||cc|}
 \hline
s/i &a&b & a&b\\
 \hline
$A$&$A$&$B$ & 0&0\\
$B$&$A$&$B$ & 1&1\\
 \hline
\end{tabular}
,  $\quad$
$M_2=$
\begin{tabular}{|c|cc||cc|}
 \hline
s/i &i&ii& i&ii\\
 \hline
$I$&$I$&$II$ & 0&0\\
$II$&$I$&$II$ & 1&1\\
 \hline
\end{tabular}
.
\end{center}
%\caption{ \label{xx9}}\end{figure}
The proper identifications relating the states of $M,M_1,M_2$ are
$A\equiv \{1,2\}$,
$B\equiv \{3,4\}$,
$I\equiv \{1,3\}$,
$II\equiv \{2,4\}$ and
$1\equiv A\cdot I$,
$2\equiv A\cdot II$,
$3\equiv B\cdot I$,
$4\equiv B\cdot II$.
Here, the ``$\cdot$''-product of two sets of states is their set
theoretic intersection
\cite[p. 4,5]{hartmanis}.
The proper identifications relating the input symbols of $M_1,M_2,M$ are
$ai\equiv 1$,
$aii\equiv 2$,
$bi\equiv 3$,
$bii\equiv 4$.

Note that the output table of $M$ reproduces the partition logic $P$.
The $i$'th input generates the $i$'th partition by associating the
output symbol $j$ to the $j$'th element of the $i$'th partition.


\clearpage
\subsubsection{$2^2\otimes 2^2\otimes 2^2$}
\subsubsection*{Hasse diagram}
%\begin{figure}
\begin{center}
%TexCad Options
%\grade{\off}
%\emlines{\off}
%\beziermacro{\off}
%\reduce{\on}
%\snapping{\on}
%\quality{0.20}
%\graddiff{0.01}
%\snapasp{1}
%\zoom{1.00}
\unitlength 1.00mm
\linethickness{0.4pt}
\begin{picture}(140.67,170.67)
\put(45.00,20.00){\circle*{1.33}}
\put(55.00,20.00){\circle*{1.33}}
\put(65.00,20.00){\circle*{1.33}}
\put(75.00,20.00){\circle*{1.33}}
\put(85.00,20.00){\circle*{1.33}}
\put(95.00,20.00){\circle*{1.33}}
\put(15.00,40.00){\circle*{1.33}}
\put(25.00,40.00){\circle*{1.33}}
\put(35.00,40.00){\circle*{1.33}}
\put(45.00,40.00){\circle*{1.33}}
\put(55.00,40.00){\circle*{1.33}}
\put(65.00,40.00){\circle*{1.33}}
\put(75.00,40.00){\circle*{1.33}}
\put(85.00,40.00){\circle*{1.33}}
\put(95.00,40.00){\circle*{1.33}}
\put(105.00,40.00){\circle*{1.33}}
\put(115.00,40.00){\circle*{1.33}}
\put(125.00,40.00){\circle*{1.33}}
\put(35.00,60.00){\circle*{1.33}}
\put(45.00,60.00){\circle*{1.33}}
\put(55.00,60.00){\circle*{1.33}}
\put(65.00,60.00){\circle*{1.33}}
\put(75.00,60.00){\circle*{1.33}}
\put(85.00,60.00){\circle*{1.33}}
\put(95.00,60.00){\circle*{1.33}}
\put(105.00,60.00){\circle*{1.33}}
\put(15.00,80.00){\circle*{1.33}}
\put(25.00,80.00){\circle*{1.33}}
\put(35.00,80.00){\circle*{1.33}}
\put(45.00,80.00){\circle*{1.33}}
\put(55.00,80.00){\circle*{1.33}}
\put(65.00,80.00){\circle*{1.33}}
\put(75.00,80.00){\circle*{1.33}}
\put(85.00,80.00){\circle*{1.33}}
\put(95.00,80.00){\circle*{1.33}}
\put(105.00,80.00){\circle*{1.33}}
\put(115.00,80.00){\circle*{1.33}}
\put(125.00,80.00){\circle*{1.33}}
\put(45.00,100.00){\circle*{1.33}}
\put(55.00,100.00){\circle*{1.33}}
\put(65.00,100.00){\circle*{1.33}}
\put(75.00,100.00){\circle*{1.33}}
\put(85.00,100.00){\circle*{1.33}}
\put(95.00,100.00){\circle*{1.33}}
\put(70.00,120.00){\circle*{1.33}}
\put(70.00,0.00){\circle*{1.33}}
\put(70.00,0.00){\line(-5,4){25.00}}
\put(45.00,20.00){\line(-3,2){30.00}}
\put(15.00,40.00){\line(1,1){20.00}}
\put(35.00,60.00){\line(-1,1){20.00}}
\put(15.00,80.00){\line(3,2){30.00}}
\put(45.00,100.00){\line(5,4){25.00}}
\put(70.00,120.00){\line(5,-4){25.00}}
\put(95.00,100.00){\line(3,-2){30.00}}
\put(125.00,80.00){\line(-1,-1){20.00}}
\put(105.00,60.00){\line(1,-1){20.00}}
\put(125.00,40.00){\line(-3,-2){30.00}}
\put(95.00,20.00){\line(-5,-4){25.00}}
\put(70.00,0.00){\line(-3,4){15.00}}
\put(25.00,40.00){\line(1,2){10.00}}
\put(35.00,60.00){\line(-1,2){10.00}}
\put(25.00,80.00){\line(1,1){20.00}}
\put(55.00,20.00){\line(-2,1){40.00}}
\put(20.00,130.00){\circle*{1.33}}
\put(0.00,150.00){\circle*{1.33}}
\put(40.00,150.00){\circle*{1.33}}
\put(20.00,170.00){\circle*{1.33}}
\put(20.00,130.00){\line(-1,1){20.00}}
\put(0.00,150.00){\line(1,1){20.00}}
\put(20.00,170.00){\line(1,-1){20.00}}
\put(40.00,150.00){\line(-1,-1){20.00}}
\put(45.00,150.00){\makebox(0,0)[cc]{$\otimes$}}
\put(25.00,130.00){\makebox(0,0)[cc]{$0_1$}}
\put(25.00,170.00){\makebox(0,0)[cc]{$1_1$}}
\put(40.00,145.00){\makebox(0,0)[rc]{$b_1=a_1 '$}}
\put(0.00,145.00){\makebox(0,0)[cc]{$a_1$}}
\put(70.00,130.00){\circle*{1.33}}
\put(120.00,130.00){\circle*{1.33}}
\put(50.00,150.00){\circle*{1.33}}
\put(100.00,150.00){\circle*{1.33}}
\put(90.00,150.00){\circle*{1.33}}
\put(140.00,150.00){\circle*{1.33}}
\put(70.00,170.00){\circle*{1.33}}
\put(120.00,170.00){\circle*{1.33}}
\put(70.00,130.00){\line(-1,1){20.00}}
\put(120.00,130.00){\line(-1,1){20.00}}
\put(50.00,150.00){\line(1,1){20.00}}
\put(100.00,150.00){\line(1,1){20.00}}
\put(70.00,170.00){\line(1,-1){20.00}}
\put(120.00,170.00){\line(1,-1){20.00}}
\put(90.00,150.00){\line(-1,-1){20.00}}
\put(140.00,150.00){\line(-1,-1){20.00}}
\put(95.00,150.00){\makebox(0,0)[cc]{$\otimes$}}
\put(75.00,130.00){\makebox(0,0)[cc]{$0_2$}}
\put(125.00,130.00){\makebox(0,0)[cc]{$0_3$}}
\put(75.00,170.00){\makebox(0,0)[cc]{$1_2$}}
\put(125.00,170.00){\makebox(0,0)[cc]{$1_3$}}
\put(90.00,145.00){\makebox(0,0)[rc]{$b_2=a_2 '$}}
\put(140.00,145.00){\makebox(0,0)[rc]{$b_3=a_3 '$}}
\put(50.00,145.00){\makebox(0,0)[cc]{$a_2$}}
\put(100.00,145.00){\makebox(0,0)[cc]{$a_3$}}
\put(5.00,60.00){\makebox(0,0)[cc]{$=$}}
\put(75.00,120.00){\makebox(0,0)[lc]{$1$}}
\put(75.00,0.00){\makebox(0,0)[lc]{$0$}}
\put(45.00,20.00){\line(-1,1){20.00}}
\put(45.00,20.00){\line(-1,2){10.00}}
\put(45.00,20.00){\line(0,1){60.00}}
\put(55.00,20.00){\line(0,1){20.00}}
\put(55.00,40.00){\line(-1,1){20.00}}
\put(35.00,60.00){\line(0,1){20.00}}
\put(35.00,80.00){\line(1,1){20.00}}
\put(55.00,100.00){\line(3,4){15.00}}
\put(55.00,20.00){\line(1,2){10.00}}
\put(65.00,40.00){\line(1,2){10.00}}
\put(75.00,60.00){\line(1,1){20.00}}
\put(95.00,80.00){\line(-5,2){50.00}}
\put(95.00,100.00){\line(0,-1){60.00}}
\put(95.00,40.00){\line(-1,-2){10.00}}
\put(85.00,20.00){\line(-3,-4){15.00}}
\put(55.00,20.00){\line(1,1){20.00}}
\put(75.00,40.00){\line(1,2){10.00}}
\put(85.00,60.00){\line(3,2){30.00}}
\put(115.00,80.00){\line(-1,1){20.00}}
\put(70.00,0.00){\line(-1,4){5.00}}
\put(65.00,20.00){\line(-1,2){10.00}}
\put(55.00,40.00){\line(1,1){20.00}}
\put(75.00,60.00){\line(-3,1){60.00}}
\put(65.00,20.00){\line(-2,1){40.00}}
\put(25.00,40.00){\line(3,2){30.00}}
\put(55.00,60.00){\line(-3,2){30.00}}
\put(25.00,80.00){\line(2,1){40.00}}
\put(65.00,100.00){\line(1,4){5.00}}
\put(65.00,20.00){\line(1,1){20.00}}
\put(85.00,40.00){\line(-1,2){10.00}}
\put(75.00,60.00){\line(3,2){30.00}}
\put(105.00,80.00){\line(-1,2){10.00}}
\put(65.00,20.00){\line(3,2){30.00}}
\put(95.00,40.00){\line(-2,1){40.00}}
\put(55.00,60.00){\line(1,1){20.00}}
\put(75.00,80.00){\line(-1,2){10.00}}
\put(70.00,0.00){\line(1,4){5.00}}
\put(75.00,20.00){\line(-1,2){10.00}}
\put(65.00,40.00){\line(1,1){20.00}}
\put(85.00,60.00){\line(1,1){20.00}}
\put(105.00,80.00){\line(-5,2){50.00}}
\put(75.00,20.00){\line(1,2){10.00}}
\put(85.00,40.00){\line(1,2){10.00}}
\put(95.00,60.00){\line(3,2){30.00}}
\put(75.00,20.00){\line(3,2){30.00}}
\put(105.00,40.00){\line(-1,2){10.00}}
\put(95.00,60.00){\line(-3,2){30.00}}
\put(65.00,80.00){\line(-1,1){20.00}}
\put(75.00,20.00){\line(2,1){40.00}}
\put(115.00,40.00){\line(-1,2){10.00}}
\put(105.00,60.00){\line(-1,1){20.00}}
\put(85.00,80.00){\line(-1,2){10.00}}
\put(75.00,100.00){\line(-1,4){5.00}}
\put(85.00,20.00){\line(-5,2){50.00}}
\put(35.00,40.00){\line(1,1){20.00}}
\put(55.00,60.00){\line(1,2){10.00}}
\put(65.00,80.00){\line(1,1){20.00}}
\put(85.00,100.00){\line(-3,4){15.00}}
\put(85.00,20.00){\line(1,1){20.00}}
\put(105.00,40.00){\line(0,1){20.00}}
\put(105.00,60.00){\line(1,2){10.00}}
\put(115.00,80.00){\line(-2,1){40.00}}
\put(75.00,100.00){\line(-3,-2){30.00}}
\put(45.00,80.00){\line(2,-1){40.00}}
\put(85.00,60.00){\line(3,-2){30.00}}
\put(115.00,40.00){\line(-1,-1){20.00}}
\put(95.00,20.00){\line(-1,1){20.00}}
\put(75.00,40.00){\line(-3,2){30.00}}
\put(45.00,60.00){\line(-1,2){10.00}}
\put(35.00,80.00){\line(3,2){30.00}}
\put(65.00,100.00){\line(-1,-2){10.00}}
\put(55.00,80.00){\line(1,-2){10.00}}
\put(65.00,60.00){\line(-1,-1){20.00}}
\put(45.00,40.00){\line(5,-2){50.00}}
\put(85.00,20.00){\line(2,1){40.00}}
\put(125.00,40.00){\line(-3,1){60.00}}
\put(65.00,60.00){\line(1,2){10.00}}
\put(75.00,80.00){\line(1,2){10.00}}
\put(85.00,100.00){\line(0,-1){20.00}}
\put(85.00,80.00){\line(-1,-1){20.00}}
\put(15.00,40.00){\line(3,2){30.00}}
\put(45.00,60.00){\line(1,2){10.00}}
\put(55.00,80.00){\line(1,1){20.00}}
\put(35.00,40.00){\line(3,2){30.00}}
\put(15.00,80.00){\line(2,1){40.00}}
\put(45.00,80.00){\line(1,2){10.00}}
\put(125.00,80.00){\line(-2,1){40.00}}
\put(45.00,15.00){\makebox(0,0)[cc]{$a_1$}}
\put(55.00,15.00){\makebox(0,0)[cc]{$a_2$}}
\put(65.00,15.00){\makebox(0,0)[cc]{$a_3$}}
\put(75.00,15.00){\makebox(0,0)[cc]{$b_1$}}
\put(85.00,15.00){\makebox(0,0)[cc]{$b_2$}}
\put(95.00,15.00){\makebox(0,0)[cc]{$b_3$}}
\put(45.00,105.00){\makebox(0,0)[cc]{$b_3 '$}}
\put(55.00,105.00){\makebox(0,0)[cc]{$b_2 '$}}
\put(65.00,105.00){\makebox(0,0)[cc]{$b_1 '$}}
\put(75.00,105.00){\makebox(0,0)[cc]{$a_3 '$}}
\put(85.00,105.00){\makebox(0,0)[cc]{$a_2 '$}}
\put(95.00,105.00){\makebox(0,0)[cc]{$a_1 '$}}
\put(15.00,35.00){\makebox(0,0)[cc]{$\{a_1,a_2\}$}}
\put(125.00,35.00){\makebox(0,0)[cc]{$\{b_2,b_3\}$}}
\put(15.00,85.00){\makebox(0,0)[cc]{$\{b_2,b_3\} '$}}
\put(125.00,85.00){\makebox(0,0)[cc]{$\{a_1,a_2\} '$}}
\put(110.00,60.00){\makebox(0,0)[lc]{$\{b_1,b_2,b_3\}$}}
\put(30.00,60.00){\makebox(0,0)[rc]{$\{a_1,a_2,a_3\}$}}
\end{picture}
\end{center}
%\caption{ \label{x10}}\end{figure}




\subsubsection*{Realization}
\subsubsection*{Partition (automaton) logic}

Let
$a_1\equiv 1$,
$a_2\equiv 2$,
$a_3\equiv 3$,
$b_1\equiv 4$,
$b_2\equiv 5$,
$b_3\equiv 6$.
A partition logic isomorphic
$2^2\otimes 2^2\otimes 2^2$ is
 \begin{eqnarray*}
P&=&\{\{\{1\},\{2\},\{3\},\{4,5,6\}\},   \\
&&\{\{1\},\{5\},\{3\},\{2,4,6\}\},      \\
&&\{\{1\},\{2\},\{6\},\{3,4,5\}\},      \\
&&\{\{1\},\{5\},\{6\},\{2,3,4\}\},      \\
&&\{\{4\},\{2\},\{3\},\{1,5,6\}\},      \\
&&\{\{4\},\{5\},\{3\},\{1,2,6\}\},      \\
&&\{\{4\},\{2\},\{6\},\{1,3,5\}\},      \\
&&\{\{4\},\{5\},\{6\},\{1,2,3\}\}\}.
 \end{eqnarray*}

One automaton realization is the Mealy automaton $M$, which can be
parallel decomposed into two Mealy automata $M_1, M_2$ such that
$M=M_1 \| M_2 \| M_3$ according to
%\begin{figure}
\begin{center}
$M=$
\begin{tabular}{|c|cccccccc||cccccccc|}
 \hline
s/i &1&2&3&4&5&6&7&8 & 1&2&3&4&5&6&7&8 \\
 \hline
1&1&2&3&4&5&6&1&2 & 1&1&1&1&1&1&1&1\\
2&1&2&3&4&5&6&1&2 & 2&4&2&4&2&1&2&1\\
3&1&2&3&4&5&6&1&2 & 3&3&4&4&3&2&1&1\\
4&1&2&3&4&5&6&1&2 & 4&4&4&4&4&3&3&2\\
5&1&2&3&4&5&6&1&2 & 4&2&4&2&1&4&1&3\\
6&1&2&3&4&5&6&7&8 & 4&4&3&3&1&1&4&4\\
 \hline
\end{tabular}
,\\
$\,$\\
$M_1=$
\begin{tabular}{|c|cc||cc|}
 \hline
s/i &a&b & a&b\\
 \hline
$A$&$A$&$B$ & 0&0\\
$B$&$A$&$B$ & 1&1\\
 \hline
\end{tabular}
,
$M_2=$
\begin{tabular}{|c|cc||cc|}
 \hline
s/i &i&ii& i&ii\\
 \hline
$I$&$I$&$II$ & 0&0\\
$II$&$I$&$II$ & 1&1\\
 \hline
\end{tabular}
,
$M_3=$
\begin{tabular}{|c|cc||cc|}
 \hline
s/i &$\gamma$&$\delta$&$\gamma$&$\delta$ \\
 \hline
$\Gamma$&$\Gamma$&$\Delta $ & 0&0\\
$\Delta$&$\Gamma $&$\Delta $ & 1&1\\
 \hline
\end{tabular}
.
\end{center}
%\caption{ \label{x11}}\end{figure}
The proper identifications relating the states of $M,M_1,M_2$ are
$A\equiv \{1,2,3\}$,
$B\equiv \{4,5,6\}$,
$I\equiv \{1,5,6\}$,
$II\equiv \{2,3,4\}$,
$\Gamma \equiv \{1,3,5\}$,
$\Delta \equiv \{2,4,6\}$,
 and
$1\equiv A\cdot I\cdot \Gamma $,
$2\equiv A\cdot II\cdot \Delta $,
$3\equiv A\cdot II\cdot \Gamma $,
$4\equiv B\cdot II\cdot  \Delta $,
$5\equiv B\cdot I\cdot  \Gamma $,
$6\equiv B\cdot I\cdot  \Delta $.
Note again that the output table of $M$ reproduces the partition logic
$P$.

\section{Conclusion}

I have attempted
to enumerate some rationally
conceivable forms of complementarity, or, more specifically,
of the logico--algebraic structure of propositions about observable
phenomena. This is in the spirit of Foulis and Randall
\cite{Foulis-Randall,ran-foul-73}, but with a definite algorithmic
flavour.    Thereby, structures in algorithmics have been related to and
compared with logical and  physical forms.
A small collection of  low-complexity structures has been discussed.
These examples mainly originate from quantum systems and automata
theory, including the serial and parallel composition of deterministic
Moore and Mealy automata.



It should be emphasized that
complementarity  is not directly
related to diagonalization \cite{godel1,turing-36,rogers1,odi:89}; it
is, rather,  a
second, independent source of undecidability. It is already
realizable at an elementary `pre-diagonalization' level,
 i.e., without
the requirement of computational universality or its arithmetic
equivalent.
The corresponding machine model is the class of finite automata.

Since any finite state
automaton can be simulated by a universal computer, complementarity is a
feature of sufficiently complex deterministic universes as well.
 To put it pointedly: if
the physical
universe is conceived as the product of a universal computation,
then complementarity is an inevitable feature of the
perception of
observers.



%\bibliography{svozil}
 %\bibliographystyle{plain}
%\bibliographystyle{natbib}


\begin{thebibliography}{}

\bibitem[Aerts(1995)]{aerts}
Aerts, D. (1995).
\newblock Quantum structures: an attempt to explain the origin of their
  appearence in nature.
\newblock {\em International Journal of Theoretical Physics}, {\bf 34},
  1165--1186.

\bibitem[Birkhoff and von Neumann(1936)]{birkhoff-36}
Birkhoff, G. and von Neumann, J. (1936).
\newblock The logic of quantum mechanics.
\newblock {\em Annals of Mathematics}, {\bf 37}(4), 823--843.

\bibitem[Brauer(1984)]{brauer-84}
Brauer, W. (1984).
\newblock {\em Automatentheorie}.
\newblock Teubner, Stuttgart.

\bibitem[Cohen(1989)]{cohen}
Cohen, D.~W. (1989).
\newblock {\em An Introduction to Hilbert Space and Quantum Logic}.
\newblock Springer, New York.

\bibitem[Conway(1971)]{conway}
Conway, J.~H. (1971).
\newblock {\em Regular Algebra and Finite Machines}.
\newblock Chapman and Hall Ltd., London.

\bibitem[Davis(1965)]{davis}
Davis, M. (1965).
\newblock {\em The Undecidable}.
\newblock Raven Press, New York.

\bibitem[Dvure{\v{c}}enskij {\em et~al.}(1995)]{dvur-pul-svo}
Dvure{\v{c}}enskij, A., Pulmannov{\'{a}}, S., and Svozil, K. (1995).
\newblock Partition logics, orthoalgebras and automata.
\newblock {\em Helvetica Physica Acta}, {\bf 68}, 407--428.

\bibitem[Foulis and Randall(1972)]{Foulis-Randall}
Foulis, D.~J. and Randall, C. (1972).
\newblock Operational statistics. {I}. basic concepts.
\newblock {\em Journal of Mathematical Physics}, {\bf 13}, 1667--1675.

\bibitem[Giuntini(1991)]{giuntini-91}
Giuntini, R. (1991).
\newblock {\em Quantum Logic and Hidden Variables}.
\newblock BI Wissenschaftsverlag, Mannheim.

\bibitem[G{\"{o}}del(1931)]{godel1}
G{\"{o}}del, K. (1931).
\newblock {\"{U}}ber formal unentscheidbare {S\"{a}}tze der {P}rincipia
  {M}athematica und verwandter {S}ysteme.
\newblock {\em Monatshefte f{\"{u}}r Mathematik und Physik}, {\bf 38},
  173--198.
\newblock English translation in \cite{godel-ges1}, and in \cite{davis}.

\bibitem[G{\"{o}}del(1986)]{godel-ges1}
G{\"{o}}del, K. (1986).
\newblock In S.~Feferman, J.~W. Dawson, S.~C. Kleene, G.~H. Moore, R.~M.
  Solovay, and J.~van Heijenoort, editors, {\em Collected Works. Publications
  1929-1936. Volume {I}}. Oxford University Press, Oxford.

\bibitem[Greenberger {\em et~al.}(1993)]{green-horn-zei}
Greenberger, D.~B., Horne, M., and Zeilinger, A. (1993).
\newblock Multiparticle interferometry and the superposition principle.
\newblock {\em Physics Today}, {\bf 46}, 22--29.

\bibitem[Hartmanis and Stearns(1966)]{hartmanis}
Hartmanis, J. and Stearns, R.~E. (1966).
\newblock {\em Algebraic Structure Theory of Sequential Machines}.
\newblock Prentice Hall, Englewood Cliffs, NJ.

\bibitem[Havlicek and Svozil(1996)]{havlicek}
Havlicek, H. and Svozil, K. (1996).
\newblock Density conditions for quantum propositions.
\newblock {\em Journal of Mathematical Physics}, {\bf 37}(11), 5337--5341.

\bibitem[Hopcroft and Ullman(1979)]{hopcroft}
Hopcroft, J.~E. and Ullman, J.~D. (1979).
\newblock {\em Introduction to Automata Theory, Languages, and Computation}.
\newblock Addison-Wesley, Reading, MA.

\bibitem[Kochen and Specker(1967)]{kochen1}
Kochen, S. and Specker, E.~P. (1967).
\newblock The problem of hidden variables in quantum mechanics.
\newblock {\em Journal of Mathematics and Mechanics}, {\bf 17}(1), 59--87.
\newblock Reprinted in \cite[pp. 235--263]{specker-ges}.

\bibitem[Mermin(1993)]{mermin-93}
Mermin, N.~D. (1993).
\newblock Hidden variables and the two theorems of {J}ohn {B}ell.
\newblock {\em Reviews of Modern Physics}, {\bf 65}, 803--815.

\bibitem[Moore(1956)]{e-f-moore}
Moore, E.~F. (1956).
\newblock Gedanken-experiments on sequential machines.
\newblock In C.~E. Shannon and J.~McCarthy, editors, {\em Automata Studies}.
  Princeton University Press, Princeton.

\bibitem[Navara and Rogalewicz(1991)]{nav:91}
Navara, M. and Rogalewicz, V. (1991).
\newblock The pasting constructions for orthomodular posets.
\newblock {\em Mathematische Nachrichten}, {\bf 154}, 157--168.

\bibitem[Odifreddi(1989)]{odi:89}
Odifreddi, P. (1989).
\newblock {\em Classical Recursion Theory}.
\newblock North-Holland, Amsterdam.

\bibitem[Paz(1971)]{paz}
Paz, A. (1971).
\newblock {\em Introduction to Probabilistic Automata}.
\newblock Academic Press, New York.

\bibitem[Randall and Foulis(1973)]{ran-foul-73}
Randall, C.~H. and Foulis, D.~J. (1973).
\newblock Operational statistics. {II}. {M}anual of operations and their
  logics.
\newblock {\em Journal of Mathematical Physics}, {\bf 14}, 1472--1480.

\bibitem[Redhead(1990)]{redhead}
Redhead, M. (1990).
\newblock {\em Incompleteness, Nonlocality, and Realism: A Prolegomenon to the
  Philosophy of Quantum Mechanics}.
\newblock Clarendon Press, Oxford.

\bibitem[{Rogers, Jr.}(1967)]{rogers1}
{Rogers, Jr.}, H. (1967).
\newblock {\em Theory of Recursive Functions and Effective Computability}.
\newblock MacGraw-Hill, New York.

\bibitem[Schaller and Svozil(1994)]{schaller-92}
Schaller, M. and Svozil, K. (1994).
\newblock Partition logics of automata.
\newblock {\em Il Nuovo Cimento}, {\bf 109B}, 167--176.

\bibitem[Schaller and Svozil(1995)]{schaller-95}
Schaller, M. and Svozil, K. (1995).
\newblock Automaton partition logic versus quantum logic.
\newblock {\em International Journal of Theoretical Physics}, {\bf 34}(8),
  1741--1750.

\bibitem[Schaller and Svozil(1996)]{schaller-96}
Schaller, M. and Svozil, K. (1996).
\newblock Automaton logic.
\newblock {\em International Journal of Theoretical Physics}, {\bf 35}(5),
  911--940.

\bibitem[Specker(1990)]{specker-ges}
Specker, E. (1990).
\newblock {\em Selecta}.
\newblock Birkh{\"{a}}user Verlag, Basel.

\bibitem[Svozil(1993)]{svozil-93}
Svozil, K. (1993).
\newblock {\em Randomness \& Undecidability in Physics}.
\newblock World Scientific, Singapore.

\bibitem[Svozil and Tkadlec(1996)]{svozil-tkadlec}
Svozil, K. and Tkadlec, J. (1996).
\newblock Greechie diagrams, nonexistence of measures in quantum logics and
  {K}ochen--{S}pecker type constructions.
\newblock {\em Journal of Mathematical Physics}, {\bf 37}(11), 5380--5401.

\bibitem[Turing(1937)]{turing-36}
Turing, A.~M. (1936-7 and 1937).
\newblock On computable numbers, with an application to the
  {E}ntscheidungsproblem.
\newblock {\em Proceedings of the London Mathematical Society, Series 2}, {\bf
  42 and 43}, 230--265 and 544--546.
\newblock reprinted in \cite{davis}.

\bibitem[Wright(1978)]{wright:pent}
Wright, R. (1978).
\newblock The state of the pentagon. {A} nonclassical example.
\newblock In A.~R. Marlow, editor, {\em Mathematical Foundations of Quantum
  Theory}, pages 255--274. Academic Press, New York.

\bibitem[Wright(1990)]{wright}
Wright, R. (1990).
\newblock Generalized urn models.
\newblock {\em Foundations of Physics}, {\bf 20}, 881--903.

\bibitem[Zierler and Schlessinger(1965)]{ZirlSchl-65}
Zierler, N. and Schlessinger, M. (1965).
\newblock Boolean embeddings of orthomodular sets and quantum logic.
\newblock {\em Duke Mathematical Journal}, {\bf 32}, 251--262.

\end{thebibliography}

\end{document}
