\chapter{Brief review of complex analysis}
\label{2011-m-ch-ca}

Is it not amazing that complex numbers \cite{Hlawka-zz} can be used for physics?
Robert Musil (a mathematician educated in Vienna), in  {\it ``Verwirrungen des Z\"ogling T\"orle\ss''}
\marginnote{German original (http://www.gutenberg.org/ebooks/34717):
{\it
``In solch einer Rechnung sind am Anfang ganz solide Zahlen,
die Meter oder Gewichte, oder irgend etwas anderes Greifbares darstellen
k\"onnen und wenigstens wirkliche Zahlen sind.
Am Ende der Rechnung stehen ebensolche.
Aber diese beiden h\"angen miteinander durch etwas zusammen, das es gar nicht gibt.
Ist das nicht wie eine Br\"ucke,
von der nur Anfangs- und Endpfeiler vorhanden sind
und die man dennoch so sicher \"uberschreitet,
als ob sie ganz dast�nde?
F�r mich hat so eine Rechnung etwas Schwindliges;
als ob es ein St�ck des Weges wei\ss{} Gott wohin ginge.
Das eigentlich Unheimliche ist mir aber die Kraft,
die in solch einer Rechnung steckt und einen so festhalt,
da\ss{} man doch wieder richtig landet.''}
},
expresses the amazement of a youngster confronted with the applicability of imaginaries,
states that, at the beginning of any computation involving imaginary numbers
are ``solid'' numbers which could represent something measurable, like lengths or weights,
or something else tangible; or are at least real numbers.
At the end of the computation there are also such ``solid'' numbers.
But the beginning and the end of the computation are connected by something
seemingly nonexisting.
Does this not appear, Musil's {\it  Z\"ogling T\"orle\ss} wonders,
like a bridge crossing an abyss
with only a bridge pier at the very beginning and one at the very end,
which could nevertheless be crossed with certainty and securely,
as if this bridge would exist entirely?


In what follows, a very brief review of
{\em complex analysis},
\index{complex analysis}
or, by another term,
{\em function theory},
\index{function theory}
will be presented.
For much more detailed introductions to complex analysis,
including proofs,
take, for instance, the ``classical'' books
\cite{freitag-busam,whittaker:1927:cma,Greene,Hille62,ahlfors:1966:ca},
among a zillion of other very good ones \cite{jaenich-ft,salamon-ft}.
We shall study complex analysis not only for its beauty, but also because it yields
very important analytical methods and tools;
for instance for the solution of (differential) equations and the
computation of definite integrals.
These methods will then be required for the computation of distributions and Green's functions,
as well for the solution of differential equations of mathematical physics -- such as the Schr\"odinger equation.

One motivation for introducing
{\em imaginary numbers}
\index{imaginary numbers}
is the (if you perceive it that way)
``malady''
that not every polynomial such as $P(x)=x^2+1$ has a root $x$
-- and thus not every (polynomial) equation $P(x)=x^2+1=0$ has a
solution $x$  --
which is a real number.
Indeed, you need the imaginary unit $i^2 =-1$ for a factorization $P(x)=(x+i)(x-i)$ yielding the two roots $\pm i$
to achieve this.
In that way, the introduction of imaginary numbers is a further step towards omni-solvability.
No wonder that the fundamental theorem of algebra, stating that every
non-constant polynomial with complex coefficients has at least one complex root
--
and thus total factorizability of polynomials into linear factors, follows!

If not mentioned otherwise, it is assumed that the
{\em Riemann surface,}
\index{Riemann surface}
representing a ``deformed version'' of the complex plane for functional purposes,
is simply connected.
Simple connectedness means that the Riemann surface
it is path-connected so that every path between two points can be continuously transformed, staying within the domain,
into any other path while preserving the two endpoints between the paths.
In particular, suppose that
there are no ``holes'' in the Riemann surface; it is not ``punctured.''

Furthermore, let $i$ be the
{\em imaginary unit}
\index{imaginary unit} with the property that
 $i^2=-1$ is the solution of the equation $x^2+1=0$.
With the introduction of imaginary numbers we can guarantee that all quadratic
equations have two roots (i.e., solutions).


By combining imaginary and real numbers,
any {\em complex number} can be defined to be some linear combination of the real unit number ``$1$''
\index{complex numbers}
with the imaginary unit number $i$
that is,
$
z= 1 \times (\Re z)  +  i \times (\Im z)
$,
with the real valued factors $(\Re z) $
and $(\Im z)$, respectively.
By this definition,
a complex number $z$ can be decomposed into real numbers $x$, $y$, $r$ and $\varphi$ such that
\begin{equation}
z \stackrel{{\tiny \textrm{ def }}}{=}  \Re z + i \Im  z  =x +iy  =r e^{i\varphi},
\end{equation}
with $x = r\cos \varphi$
and $y = r\sin \varphi$,
where {Euler's formula}
\index{Euler's formula}
\begin{equation}
e^{i\varphi} = \cos \varphi +i \sin \varphi
\label{2012-m-ch-ca-ef}
\end{equation}
has been used.
If  $z = \Re z $ we call $z$ a real number.
If  $z = i\Im z $ we call $z$ a purely imaginary number.

The {\em modulus} or {\em absolute value}
\index{modulus}
\index{absolute value}
of a complex number $z$ is defined by
\begin{equation}
| z |
\stackrel{{\tiny \textrm{ def }}}{=} +
\sqrt{ (\Re z)^2 + (\Im z)^2 }.
\label{2012-m-ch-ca-defmodulus}
\end{equation}

Many rules of classical arithmetic can be carried over to complex arithmetic
\cite{apostol,freitag-busam}.
Note, however,
that, for instance, $\sqrt{a}\sqrt{b}= \sqrt{ab}$
is only valid if at least one factor $a$ or $b$ is positive;
hence  $-1=i^2=\sqrt{i}\sqrt{i} =\sqrt{-1}\sqrt{-1} \neq \sqrt{(-1)^2}=1$.
More generally, for two arbitrary numbers, $u$ and $v$,
$\sqrt{u}\sqrt{v}$ is not always equal to $\sqrt{u v}$.
\marginnote{Nevertheless,
$\sqrt{|u|}\sqrt{|v|}=\sqrt{|u v|}$.
}

For many mathematicians
{\em Euler's identity}
\index{Euler identity}
\begin{equation}
e^{i\pi}=-1 \textrm{, or } e^{i\pi}+1=0,
\end{equation}
is the ``most beautiful'' theorem \cite{springerlink:10.1007/BF03023741}.

Euler's formula (\ref{2012-m-ch-ca-ef}) can be used to derive {\em de Moivre's formula}
\index{Moivre's formula} for integer $n$ (for non-integer $n$ the formula is multi-valued for different arguments $\varphi$):
\begin{equation}
e^{in\varphi} = (\cos \varphi +i \sin \varphi )^n      = \cos (n\varphi) +i \sin (n\varphi).
\end{equation}



It is quite suggestive to consider the complex numbers $z$, which are linear combinations
of the real and the imaginary unit,
in the {\em complex plane} ${\Bbb C} = {\Bbb R} \times {\Bbb R}$
\index{complex plane}
as a geometric representation of complex numbers.
Thereby,  the real and the imaginary unit are identified with the (orthonormal) basis vectors
of the
{\em standard (Cartesian) basis};
that is, with the tuples
\index{Cartesian basis}
\index{standard basis}
\begin{equation}
\begin{split}
1 \equiv (1,0),\\
i \equiv (0,1).\\
\end{split}
\end{equation}
The addition and multiplication of two complex numbers represented by $(x,y)$ and $(u,v)$ with $x,y,u,v \in {\Bbb R}$
are then defined by
\begin{equation}
\begin{split}
(x,y) + (u,v) = (x+u,y+v),\\
(x,y) \cdot (u,v) = (xu-yv,xv+yu),
\end{split}
\end{equation}
and the neutral elements for addition and multiplication are $(0,0)$ and $(1,0)$, respectively.


We shall also consider the {\em extended plane}  ${\overline{\Bbb C}} = {\Bbb C} \cup \{\infty \}$
\index{extended plane}
consisting of the entire complex plane ${\Bbb C}$ {\em together} with the point ``$\infty$''
representing infinity.
Thereby, $\infty$
is introduced as an ideal element,
completing the one-to-one (bijective) mapping $w=\frac{1}{z}$,
which otherwise would have no image at $z=0$, and no pre-image (argument)
at $w=0$.


\section{Differentiable, holomorphic (analytic) function}

Consider the function $f(z)$ on the domain $G\subset {\rm Domain}(f)$.

$f$
is called {\em differentiable} at the point $z_0$ if the  differential quotient
\begin{equation}
\left. {df\over dz}\right|_{z_0}=
\left. f'(z)\right|_{z_0} =
\left. {\partial f\over \partial  x}\right|_{z_0} =
\left. {1\over i}{\partial f\over \partial y}\right|_{z_0}
\end{equation}
 exists.
\index{differentiable}



If $f$ is differentiable in the  domain $G$ it is called {\em holomorphic,}
or, used synonymuously, {\em analytic} in the domain $G$.
\index{holomorphic function}
\index{analytic function}




 \section{Cauchy-Riemann equations}
\index{Cauchy-Riemann equations}
 The function $f(z)=u(z)+iv(z)$ (where $u$ and $v$ are real valued functions) is
 {analytic or holomorph} if and only if
 ($a_b=\partial a/\partial b$)
 \begin{equation}
u_x=v_y, \qquad u_y=-v_x\quad .
\end{equation}
{\color{OliveGreen}
\bproof
For a proof, differentiate along the real, and then along the complex axis,
taking
\begin{equation}
\begin{split}
f'(z) =\lim_{x\rightarrow 0}\frac{f(z+x)-f(z)}{x}=\frac{\partial f}{\partial x}=   \frac{\partial u}{\partial x}+i\frac{\partial v}{\partial x},\\
\textrm { and } f'(z) =\lim_{y\rightarrow 0}\frac{f(z+iy)-f(z)}{iy}=\frac{\partial f}{\partial iy}= -i\frac{\partial f}{\partial y}=   -i\frac{\partial u}{\partial y}+ \frac{\partial v}{\partial y}.
\end{split}
\end{equation}
For $f$ to be analytic, both partial derivatives have to be identical, and thus $\frac{\partial f}{\partial x}=\frac{\partial f}{\partial iy}$, or
\begin{equation}
\frac{\partial u}{\partial x}+i\frac{\partial v}{\partial x}=   -i\frac{\partial u}{\partial y}+ \frac{\partial v}{\partial y}.
\end{equation}
By comparing the real and imaginary parts of this equation, one obtains the two real Cauchy-Riemann equations
\begin{equation}
\begin{split}
\frac{\partial u}{\partial x}=   \frac{\partial v}{\partial y},\\
\frac{\partial v}{\partial x}=   -\frac{\partial u}{\partial y}.
\end{split}
\end{equation}
\eproof
}

  \section{Definition analytical function}
{\color{OliveGreen}
\bproof
If $f$ is analytic in $G$, all derivatives of $f$ exist, and all mixed derivatives are independent on the order of differentiations.
Then the  Cauchy-Riemann equations  imply that
\begin{equation}
\begin{split}
\frac{\partial }{\partial x}\left(\frac{\partial u}{\partial x}\right)=
\frac{\partial }{\partial x}\left(\frac{\partial v}{\partial y}\right)=
\frac{\partial }{\partial y}\left(\frac{\partial v}{\partial x}\right)=
 -\frac{\partial }{\partial y}\left(\frac{\partial u}{\partial y}\right),     \\
\textrm{ and }\frac{\partial }{\partial y}\left(\frac{\partial v}{\partial y}\right)=
\frac{\partial }{\partial y}\left(\frac{\partial u}{\partial x}\right)=
\frac{\partial }{\partial x}\left(\frac{\partial u}{\partial y}\right)=
-\frac{\partial }{\partial x}\left(\frac{\partial v}{\partial x}\right)
,
\end{split}
\end{equation}
and thus
\eproof
}
\begin{equation}
 \left({\partial^2\over \partial x^2}
 + {\partial^2\over \partial y^2}\right)u=0      \textrm{, and }
 \left({\partial^2\over \partial x^2}
 + {\partial^2\over \partial y^2}\right)v=0\quad .
 \end{equation}


 If $f=u+iv$ is analytic in $G$, then the lines of constant $u$ and $v$ are orthogonal.

 {\color{OliveGreen}
\bproof
 The tangential vectors of the lines of constant $u$ and $v$ in the two-dimensional complex plane are defined by the two-dimensional nabla operator
\index{nabla operator}
$\nabla u(x,y)$ and $\nabla v(x,y)$.
Since, by the  Cauchy-Riemann equations $u_x=v_y$ and $u_y=-v_x$
\begin{equation}
\nabla u(x,y)\cdot \nabla v(x,y)
=
\left(
\begin{array}{c}
u_x\\
u_y
\end{array}
\right)
\cdot
\left(
\begin{array}{c}
v_x\\
v_y
\end{array}
\right)
=  u_x  v_x + u_y v_y   =   u_x  v_x  + (-v_x) u_x =0
\end{equation}
these tangential vectors are normal.
\eproof
}


$f$
is
{\em angle (shape)  preserving}
\index{conformal map}
{\em conformal} if and only if it is holomorphic and its derivative is everywhere non-zero.

 {\color{OliveGreen}
\bproof

Consider an analytic function $f$ and an arbitrary path $C$ in the complex plane of the arguments parameterized
by $z(t)$, $t\in {\Bbb R}$.
The image of $C$ associated with $f$ is  $f(C) = C': f(z(t))$, $t\in {\Bbb R}$.

The tangent vector of $C'$ in $t=0$ and $z_0=z(0)$ is
\begin{equation}
\begin{split}
\left. \frac{d }{dt} f(z(t))\right|_{t=0}
=
\left. \frac{d }{dz} f(z)\right|_{z_0}
\left. \frac{d }{dt} z(t)\right|_{t=0}
\qquad =
\lambda_0
e^{i\varphi_0}
\left. \frac{d }{dt} z(t)\right|_{t=0} .
\end{split}
\end{equation}
Note that the first term $\left. \frac{d }{dz} f(z)\right|_{z_0}$
is independent of the curve $C$ and only depends on $z_0$.
Therefore, it can be written as a product of a  squeeze (stretch) $\lambda_0 $
and a rotation $e^{i\varphi_0}$.
This is independent of the curve; hence
two curves $C_1$ and $C_2$ passing through $z_0$ yield the same
transformation of the image $\lambda_0
e^{i\varphi_0}$.
\eproof
}






 \section{Cauchy's integral theorem}
\index{Cauchy's integral theorem}
 If $f$ is analytic on $G$ and on its borders $\partial G$, then any closed line integral of $f$ vanishes
 \begin{equation}
\oint_{\partial G}f(z)dz=0\quad .
\end{equation}

No proof is given here.



In particular,
 $\oint_{C\subset \partial G}f(z)dz
$ is independent of the particular curve, and only depends on the initial and the end points.

 {\color{OliveGreen}
\bproof
 For a proof, subtract two line integral which follow arbitrary paths  $C_1$ and $C_2$ to a common initial and end point,
and which have the same integral kernel.
Then reverse the integration direction of one of the line integrals.
According to Cauchy's integral theorem the resulting integral over the closed loop has to vanish.
\eproof
}

Often it is useful to parameterize a contour integral by some form of
 \begin{equation}
\int_{C}f(z)dz= \int_{a}^b f(z(t))\frac{dz(t)}{dt} dt.
\end{equation}


{
\color{blue}
\bexample
Let $f(z) = 1/z$ and $C: z(\varphi )=R e^{i\varphi}$, with $R>0$ and $-\pi < \varphi \le \pi$. Then
\begin{equation}
\begin{split}
\oint_{\vert z\vert =R}
f (z) dz
\qquad =
\int_{-\pi}^\pi
f (z(\varphi ))\frac{dz(\varphi )}{d\varphi } d\varphi   \\
\qquad =
\int_{-\pi}^\pi
\frac{1}{R e^{i\varphi}}R \, i\, e^{i\varphi} d\varphi   \\
\qquad =
\int_{-\pi}^\pi
i\varphi   \\
\qquad =    2\pi i
\end{split}
\end{equation}
is independent of $R$.
\eexample
}



 \section{Cauchy's integral formula}
\index{Cauchy's integral formula}

If $f$ is analytic on $G$ and on its borders $\partial G$, then
\begin{equation}
f(z_0)={1\over 2\pi i}\oint_{\partial G}{f(z)\over z-z_0}dz\quad
 .
\label{2012-m-ch-ca-cif}
\end{equation}

No proof is given here.

Note that because of Cauchy's integral formula, analytic
functions have an integral representation.
This might appear as not very exciting; alas it has far-reaching consequences,
because analytic functions have integral
representation, they have higher derivatives,
which also have integral representation.
% http://people.reed.edu/~jerry/311/basicideas.pdf
And, as a result,
if a function  has one complex derivative, then it has infnitely many complex derivatives.
This statement can be expressed formally precisely by
the generalized Cauchy's integral formula or, by another term,
 Cauchy's differentiation formula
 \index{Cauchy's differentiation formula}
 \index{generalized Cauchy integral formula}
states that if $f$ is analytic on $G$ and on its borders $\partial G$, then
\begin{equation}
f^{(n)}(z_0)={n!\over 2\pi i}\oint_{\partial G}{f(z)\over
 (z-z_0)^{n+1}}dz\quad
\label{2012-m-ch-cagcif}
 .\end{equation}

No proof is given here.

Cauchy's integral formula presents a powerful method to compute integrals.
Consider the following examples.

{
\color{blue}
\bexample

\renewcommand{\labelenumi}{(\roman{enumi})}
\begin{enumerate}

\item First,
let us calculate  $$\oint_{\vert z\vert =3} \frac{3z+2}{z(z+1)^3} dz.$$
The kernel has two poles at $z=0$ and $z=-1$ which are both inside the domain of the contour defined by $\vert z\vert =3$.
By using Cauchy's integral formula we obtain for ``small'' $\epsilon$
\begin{equation}
\begin{split}
\oint_{\vert z\vert =3} \frac{3z+2}{z(z+1)^3} dz  \\
\qquad =\oint_{\vert z\vert =\epsilon} \frac{3z+2}{z(z+1)^3} dz    + \oint_{\vert z+1\vert =\epsilon} \frac{3z+2}{z(z+1)^3} dz \\
\qquad =\oint_{\vert z\vert =\epsilon} \frac{3z+2}{(z+1)^3} \frac{1}{z} dz    + \oint_{\vert z+1\vert =\epsilon} \frac{3z+2}{z}\frac{1}{(z+1)^3} dz  \\
\qquad =
\left.
\frac{2\pi i}{0!}
[[\frac{d^0}{dz^0}]]
\frac{3z+2}{(z+1)^3}
\right|_{z=0}
+
\left.
\frac{2\pi i}{2!}
\frac{d^2}{dz^2}
\frac{3z+2}{z}
\right|_{z=-1} \\
\qquad =
\left.
\frac{2\pi i}{0!}
\frac{3z+2}{(z+1)^3}
\right|_{z=0}
+
\left.
\frac{2\pi i}{2!}
\frac{d^2}{dz^2}
\frac{3z+2}{z}
\right|_{z=-1} \\
\qquad = 4\pi i - 4 \pi i \\
\qquad =0.
\end{split}
\end{equation}

\item
Consider
\begin{equation}
\begin{split}
\oint_{\vert z\vert =3}
\frac{e^{2z}}{(z+1)^4 }dz\\
\qquad =
\frac{2\pi i}{3!}
\frac{3!}{2\pi i}
\oint_{\vert z\vert =3}
\frac{e^{2z}}{(z- (-1))^{3+1} }dz  \\
\qquad =
\frac{2\pi i}{3!}
\frac{d^3}{dz^3}
\left| e^{2z} \right|_{z=-1}  \\
\qquad =
\frac{2\pi i}{3!}
2^3  \left| e^{2z} \right|_{z=-1}    \\
\qquad =
\frac{8 \pi i e^{-2}}{3}.
\end{split}
\end{equation}

\end{enumerate}
\eexample
}


Suppose $g(z)$ is a function with a pole of order $n$ at the point
 $z_0$; that is
 \begin{equation}
g(z)= {f(z)\over (z-z_0)^n}
\end{equation}
 where $f(z)$ is an analytic function. Then,
 \begin{equation}
\oint_{\partial G}g(z)dz={2\pi i\over (n-1)!}f^{(n-1)}(z_0)\quad .
\end{equation}

 \section{Series representation of complex differentiable functions}

As a consequence of Cauchy's (generalized) integral formula,
analytic  functions have power series representations.

{\color{OliveGreen}
\bproof

For the sake of a proof,
we shall recast the denominator   $z-z_0$
in Cauchy's integral formula
(\ref{2012-m-ch-ca-cif})
as a geometric series as follows  (we shall assume that $|z_0 - a| < |z - a|$)
\begin{equation}
\begin{split}
\frac{1}{z - z_0}=
\frac{1}{(z - a) - (z_0 - a)}\\=
\frac{1}{ (z - a)} \left[\frac  {1}{1 - \frac{ z_0 - a }{ z - a }}\right]\\
=
\frac{1}{(z - a)} \left[ \sum_{n=0}^\infty \frac{ (z_0 - a)^n }{ (z - a)^n }\right]\\
=
 \sum_{n=0}^\infty \frac{ (z_0 - a)^n }{ (z - a)^{n+1} } .
\label{2012-m-ch-rgs}
\end{split}
\end{equation}
By substituting this in
Cauchy's integral formula
(\ref{2012-m-ch-ca-cif})
and using
Cauchy's generalized integral formula
(\ref{2012-m-ch-cagcif})
yields    an expansion of the analytical function $f$ around $z_0$ by a power series
\begin{equation}
\begin{split}
f(z_0)={1\over 2\pi i}\oint_{\partial G}{f(z)\over z-z_0}dz \\
={1\over 2\pi i}\oint_{\partial G} f(z) \sum_{n=0}^\infty \frac{ (z_0 - a)^n }{ (z - a)^{n+1}} dz \\
=  \sum_{n=0}^\infty (z_0 - a)^n  {1\over 2\pi i}\oint_{\partial G}  \frac{ f(z) }{ (z - a)^{n+1}} dz \\
=  \sum_{n=0}^\infty  \frac{ f^{n}(z_0) }{ n! }   (z_0 - a)^n  .
\end{split}
\label{2012-m-ch-ca-ssolut}
\end{equation}

\eproof
}


 \section{Laurent series}
 \index{Laurent series}

Every function $f$ which is analytic in a concentric region
$R_1< \vert z-z_0\vert <R_2$ can in this region be uniquely written as a {\em Laurent series}
 \begin{equation}
f(z)=\sum_{k=-\infty}^\infty (z-z_0)^k a_k
\label{011-m-ch-ca-else1}
\end{equation}
The coefficients $a_k$ are
 (the closed contour $C$ must be in the concentric region)
 \begin{equation}
a_k={1\over 2\pi i}\oint_C (\chi -z_0)^{-k-1}f(\chi ) d\chi \quad.
\end{equation}
The coefficient
\begin{equation}
\textrm{Res}(f(z_0)) = a_{-1}={1\over 2\pi i}\oint_C f(\chi )d\chi
\label{011-m-ch-ca-else2}
\end{equation}
is called the
{\em residue}, denoted by
\index{residue}
 ``$\textrm{Res}$.''

{\color{OliveGreen}
\bproof

For  a proof,   as in Eqs. (\ref{2012-m-ch-rgs})
we shall recast $(a-b)^{-1}$   for $|a|>|b|$
as a geometric series
\begin{equation}
\begin{split}
\frac{1}{a - b}=
\frac{1}{a} \left(\frac  {1}{1 - \frac{ b }{ a }}\right) =
=
\frac{1}{a} \left( \sum_{n=0}^\infty \frac{ b^n }{  a^n }\right)
=
\sum_{n=0}^\infty \frac{ b^n }{  a^{n+1} } \\
[\textrm{substitution }  n+1 \rightarrow -k, \, n \rightarrow -k-1\, k \rightarrow -n-1]
=
\sum_{k=-1}^{-\infty} \frac{ a^k }{  b^{k+1} },
\end{split}
\end{equation}
and, for   $|a|<|b|$,
\begin{equation}
\begin{split}
\frac{1}{a - b}= - \frac{1}{b - a}=
-
\sum_{n=0}^\infty \frac{ a^n }{  b^{n+1} } \\
[\textrm{substitution }  n+1 \rightarrow -k, \, n \rightarrow -k-1\, k \rightarrow -n-1]
=
-
\sum_{k=-1}^{-\infty} \frac{ b^k }{  a^{k+1} }.
\end{split}
\end{equation}
Furthermore since $a+b = a-(-b)$,
we obtain, for $|a|>|b|$,
\begin{equation}
\frac{1}{a + b}
=
\sum_{n=0}^\infty (-1)^n \frac{ b^n }{  a^{n+1} }
=
\sum_{k=-1}^{-\infty}(-1)^{-k-1} \frac{ a^k }{  b^{k+1} }
=
-
\sum_{k=-1}^{-\infty}(-1)^{ k } \frac{ a^k }{  b^{k+1} }
,
\end{equation}
and,  for $|a|<|b|$,
\begin{equation}
\begin{split}
\frac{1}{a + b}
=
-
\sum_{n=0}^\infty (-1)^{n+1} \frac{ a^n }{  b^{n+1} }
=
\sum_{n=0}^\infty (-1)^{n} \frac{ a^n }{  b^{n+1} }  \\
=
\sum_{k=-1}^{-\infty}(-1)^{-k-1} \frac{ b^k }{  a^{k+1} }
=
-
\sum_{k=-1}^{-\infty}(-1)^{ k } \frac{ b^k }{  a^{k+1} }
.
\end{split}
\end{equation}


Suppose that some function $f(z)$ is analytic in an annulus bounded by the radius $r_1$ and $r_2 > r_1$.
By substituting this in
Cauchy's integral formula
(\ref{2012-m-ch-ca-cif})
for an annulus bounded by the radius $r_1$ and $r_2 > r_1$
(note that the orientations of the boundaries with respect to the annulus are opposite,
rendering a relative factor ``$-1$'')
and using
Cauchy's generalized integral formula
(\ref{2012-m-ch-cagcif})
yields an expansion of the analytical function $f$ around $z_0$ by the Laurent series
for a point $a$ on the annulus; that is,
for a path  containing the point $z$ around a circle with radius
$r_1$, $|z - a| < |z_0 - a|$;
likewise,
for a path  containing the point $z$ around a circle with radius
$r_2 > a > r_1$, $|z - a| > |z_0 - a|$,
\begin{equation}
\begin{split}
f(z_0)=
{1\over 2\pi i}\oint_{r_1}{f(z)\over z-z_0}dz
-
{1\over 2\pi i}\oint_{r_2}{f(z)\over z-z_0}dz
\\
=
{1\over 2\pi i}\left[
\oint_{r_1} f(z) \sum_{n=0}^\infty \frac{ (z_0 - a)^n }{ (z - a)^{n+1}} dz
+
\oint_{r_2} f(z) \sum_{n=-1}^{-\infty} \frac{ (z_0 - a)^n }{ (z - a)^{n+1}} dz
\right]
\\
=
{1\over 2\pi i}\left[
\sum_{n=0}^\infty (z_0 - a)^n  \oint_{r_1}   \frac{  f(z) }{ (z - a)^{n+1}} dz
+
\sum_{n=-1}^{-\infty} (z_0 - a)^n \oint_{r_2} \frac{ f(z) }{ (z - a)^{n+1}} dz
\right]
\\
=
\sum_{-\infty}^\infty (z_0 - a)^n  \left[ {1\over 2\pi i} \oint_{r_1\le r \le r_2} \frac{ f(z) }{ (z - a)^{n+1}}  dz \right] .
\end{split}
\label{2012-m-ch-ca-ssolutlarent}
\end{equation}

\eproof
}

Suppose that  $g(z)$ is a function with a pole of order $n$ at the point
 $z_0$; that is
 $g(z)= {h(z)/ (z-z_0)^n}$ ,
where $h(z)$ is an analytic function. Then the terms  $k\le -(n+1)$
vanish in the Laurent series.
This follows from  Cauchy's integral formula
\begin{equation}
a_k ={1\over 2\pi i}\oint_c(\chi -z_0)^{-k-n-1}h(\chi )d\chi =0
\end{equation}
 for $-k-n-1\ge 0$.

Note that, if $f$ has a simple pole (pole of order 1) at $z_0$,
then it can be rewritten into $f(z)=g(z)/(z-z_0)$ for some analytic function $g(z)=(z-z_0) f(z)$
that remains after the singularity has been ``split'' from $f$.
Cauchy's integral formula (\ref{2012-m-ch-ca-cif}),
and the residue can be rewritten as
\begin{equation}
a_{-1}={1\over 2\pi i}
\oint_{\partial G}{g(z)\over z-z_0}dz =
g(z_0)
 .
\label{2012-m-ch-ca-cif1}
\end{equation}
For poles of higher order, the generalized Cauchy integral formula
(\ref{2012-m-ch-cagcif}) can be used.

 \section{Residue theorem}
 \index{Residue theorem}


Suppose $f$ is analytic on a  simply connected open subset $G$
with the exception of finitely many (or denumerably many) points  $z_i$.
Then,
\begin{equation}
\oint_{\partial G} f(z)dz=2\pi i \sum_{z_i} {\rm Res}f(z_i)\quad .
\end{equation}


 No proof is given here.

The residue theorem presents a powerful tool for calculating integrals, both real and complex.
Let us first mention a rather general case of a situation often used.
Suppose we are interested in the integral
  $$I=\int_{-\infty}^{\infty} R(x) dx$$
with rational kernel $R$; that is, $R(x)= P(x)/Q(x)$,
where $P(x)$ and $Q(x)$
are polynomials (or can at least be bounded by a polynomial) with no common root (and therefore factor).
Suppose further that the degrees of the polynomial is
$$
\textrm{deg} P(x) \le
\textrm{deg} Q(x) -2.
$$
This condition is needed to assure that the additional upper or lower path we want to add when completing the contour
does not contribute; that is, vanishes.

Now first let us analytically continue $R(x)$ to the complex plane $R(z)$; that is,
$$I=\int_{-\infty}^{\infty} R(x) dx =\int_{-\infty}^{\infty} R(z) dz.$$
Next let us close the contour by adding a (vanishing) path integral
$$\int_{\curvearrowleft} R(z) dz =
0$$
in the upper (lower) complex plane
$$I=\int_{-\infty}^{\infty} R(z) dz +\int_{\curvearrowleft} R(z) dz=\oint_{\rightarrow \& \curvearrowleft} R(z) dz.$$
The added integral vanishes because
it can be approximated by
$$\left| \int_{\curvearrowleft} R(z)  dz\right| \le \lim_{r\rightarrow \infty} \left(\frac{\textrm{const.}}{r^2} \pi r \right) =0.$$

With the contour closed the residue theorem can be applied  for an evaluation of $I$; that is,
$$I= 2\pi i \sum_{z_i} {\rm Res}R(z_i)$$
for all singularities $z_i$ in the region enclosed by ``$\rightarrow \& \curvearrowleft$. ''

Let us consider some examples.

{
\color{blue}
\bexample


\renewcommand{\labelenumi}{(\roman{enumi})}
\begin{enumerate}

\item  Consider   $$I=\int_{-\infty}^{\infty}\frac{dx}{x^2+1} .$$
The analytic continuation of the kernel and the addition with vanishing a semicircle ``far away'' closing the integration path
in the {\em upper} complex half-plane of $z$ yields
\begin{equation}
\begin{split}
I=\int_{-\infty}^{\infty}\frac{dx}{x^2+1} \\
\quad      =  \int_{-\infty}^{\infty}\frac{dz}{z^2+1}  \\
\quad      = \int_{-\infty}^{\infty}\frac{dz}{z^2+1} + \int_{\curvearrowleft} \frac{dz}{z^2+1} \\
\quad      = \int_{-\infty}^{\infty}\frac{dz}{(z+i)(z-i)} +  \int_{\curvearrowleft} \frac{dz}{(z+i)(z-i)} \\
\quad      = \oint\frac{1}{(z-i)} f(z) dz \textrm{ with } f(z)=\frac{1}{(z+i)} \\
\quad      = 2\pi i \textrm{Res}\left.\left(\frac{1}{(z+i)(z-i)} \right)\right|_{z=+i} \\
\quad      = 2\pi i f(+i)  \\
\quad      = 2\pi i \frac{1}{(2i)}  \\
\quad      = \pi.   \\
\end{split}
\end{equation}
Here, Eq. (\ref{2012-m-ch-ca-cif1}) has been used.
Closing the integration path
in the {\em lower} complex half-plane of $z$ yields (note that in this case the contour integral is negative because of the path orientation)
\begin{equation}
\begin{split}
I=\int_{-\infty}^{\infty}\frac{dx}{x^2+1} \\
\quad      =  \int_{-\infty}^{\infty}\frac{dz}{z^2+1}  \\
\quad      = \int_{-\infty}^{\infty}\frac{dz}{z^2+1}  \int_{\textrm{lower path}} \frac{dz}{z^2+1} \\
\quad      = \int_{-\infty}^{\infty}\frac{dz}{(z+i)(z-i)} + \int_{\textrm{lower path}} \frac{dz}{(z+i)(z-i)} \\
\quad      = \oint\frac{1}{(z+i)} f(z) dz \textrm{ with } f(z)=\frac{1}{(z-i)} \\
\quad      = 2\pi i \textrm{Res}\left.\left(\frac{1}{(z+i)(z-i)} \right)\right|_{z=-i} \\
\quad      = -2\pi i f(-i)  \\
\quad      = 2\pi i \frac{1}{(2i)}  \\
\quad      = \pi.   \\
\end{split}
\end{equation}


\item  Consider   $$F(p)=\int_{-\infty}^{\infty}\frac{ e^{ipx}}{x^2+a^2} dx$$ with $a\neq 0$.

The analytic continuation of the kernel yields
$$F(p)=\int_{-\infty}^{\infty}\frac{ e^{ipz}}{z^2+a^2} dz
=  \int_{-\infty}^{\infty}\frac{ e^{ipz}}{(z-ia)(z+ia)} dz
.$$

Suppose first that $p>0$. Then, if $z=x+iy$,  $e^{ipz}=e^{ipx}e^{-py}\rightarrow 0$
for $z \rightarrow \infty $
in the {\em upper} half plane.
Hence,  we can close the contour in the upper half plane and obtain  $F(p)$
with the help of the residue theorem.

If $a>0$ only the pole at $z=+ia$ is enclosed in the contour; thus we obtain
\begin{equation}
\begin{split}
F(p) =\left.  2\pi i  {\rm Res} \frac{ e^{ipz}}{(z+ia)} \right|_{z=+ia} \\
\quad      =  2\pi i   \frac{e^{i^2 pa}}{2ia} \\
\quad      =  \frac{\pi}{a}    e^{- pa}.
\end{split}
\end{equation}

If $a<0$ only the pole at $z=-ia$ is enclosed in the contour; thus we obtain
\begin{equation}
\begin{split}
F(p) =\left.  2\pi i  {\rm Res} \frac{ e^{ipz}}{(z-ia)} \right|_{z=-ia} \\
\quad      =  2\pi i   \frac{e^{-i^2 pa}}{-2ia} \\
\quad      =  \frac{\pi}{-a}    e^{-i^2 pa} \\
\quad      =  \frac{\pi}{-a}    e^{ pa}
.
\end{split}
\end{equation}
Hence, for $a\neq 0$,
\begin{equation}
F(p) =  \frac{\pi}{\vert a\vert }    e^{ -\vert p a\vert}.
\end{equation}

For $p<0$ a very similar consideration, taking the {\em lower} path for continuation --
and thus aquiring a minus sign because of the ``clockwork''
orientation of the path as compared to its interior --
yields
\begin{equation}
F(p) =  \frac{\pi}{\vert a\vert }    e^{  - \vert pa\vert }.
\end{equation}


\item  Not all singularities are ``nice'' poles.
Consider   $$\oint_{\vert z\vert =1} e^{1\over z} dz.$$

That is, let $f(z) = e^{1\over z}$ and $C: z(\varphi )=R e^{i\varphi}$, with $R=1$ and $-\pi < \varphi \le \pi$.
This function is singular only in the origin $z=0$,
but this is an {\em essential singularity} near which the function exhibits extreme behavior.
and can be expanded into a Laurent series
$$
f(z) = e^{1\over z} =\sum_{l=0}^\infty \frac{1}{l!}   \left(\frac{1}{z}\right)^l
$$
around this singularity.
In such a case the residue can be found only by using Laurent series of $f(z)$;
that is by {\em comparing} its coefficient  of the $1/z$ term.
Hence,  $\left.\textrm{ Res }\left( e^{1\over z} \right) \right|_{z=0}$
is the coefficient $1$ of the $1/z$ term.
The residue is {\em not},  with $z=e^{i\varphi}$,
\begin{equation}
\begin{split}
a_{-1}= \left.\textrm{ Res }\left( e^{1\over z} \right) \right|_{z=0}  \\
\qquad \neq {1\over 2\pi i}\oint_C e^{1\over z}dz \\
\qquad \qquad = {1\over 2\pi i}\int_{-\pi}^\pi  e^{1\over e^{i\varphi}}\frac{dz(\varphi )}{d\varphi } d\varphi  \\
\qquad \qquad = {1\over 2\pi i}\int_{-\pi}^\pi  e^{1\over e^{i\varphi}}\, i\, e^{i\varphi} d\varphi    \\
\qquad \qquad = {1\over 2\pi }\int_{-\pi}^\pi  e^{- e^{i\varphi}}\,  e^{i\varphi} d\varphi    \\
\qquad \qquad = {1\over 2\pi }\int_{-\pi}^\pi  e^{- e^{i\varphi}+i\varphi} d\varphi    \\
\qquad \qquad = {1\over 2\pi }\int_{-\pi}^\pi i\frac{d}{d\varphi}  e^{- e^{i\varphi}} d\varphi    \\
\qquad \qquad = \left.{i\over 2\pi } e^{- e^{i\varphi}}\right|_{-\pi}^\pi   \\
\qquad \qquad =  0.
\end{split}
\end{equation}


Thus, by the residue theorem,
\begin{equation}
\oint_{\vert z\vert =1}
e^{1\over z} dz
=
2\pi i
\textrm{ Res }\left. \left( e^{1\over z} \right)\right|_{z=0} = 2\pi i.
\end{equation}

For $f(z) = e^{-{1\over z}}$, the same argument yields  $\left.\textrm{ Res }\left( e^{-{1\over z}} \right) \right|_{z=0} = -1$
and thus  $
\oint_{\vert z\vert =1}
e^{-{1\over z}} dz =  -2\pi i$.



\end{enumerate}
\eexample
}



 \section{Multi-valued relationships, branch points and and branch cuts}

Suppose that the  Riemann surface of is {\em not} simply connected.

Suppose further that $f$ is a  multi-valued function (or multifunction).
\index{multifunction}
\index{multi-valued function}
An argument
$z$ of the function $f$ is called
{\em branch point}
\index{branch point}
if there is a closed curve $C_z$ around $z$ whose image $f(C_z)$ is an open curve.
That is, the multifunction $f$ is discontinuous in $z$.
Intuitively speaking, branch points are the points where the various sheets of a multifunction come together.

A {\em branch cut} is a curve (with ends possibly open, closed, or half-open)
in the complex plane across which an analytic multifunction is discontinuous.
Branch cuts are often taken as lines.

 \section{Riemann surface}
Suppose $f(z)$ is a multifunction.
Then the various $z$-surfaces on which $f(z)$ is uniquely defined,
together with their connections through branch points and branch cuts,
constitute  the Riemann surface of $f$.
The required leafs are called {\em Riemann sheet}.
\index{sheet}

A point $z$ of the function $f(z)$ is called a {\em branch point of order $n$} if through it and through the associated cut(s)
$n+1$ Riemann sheets are connected.


\section{Some special functional classes}

The requirement that a function is holomorphic (analytic, differentiable)
puts some stringent conditions on its type, form, and on its behaviour.
For instance,
let
$z_0\in G$ the limit of a sequence
$\{z_n\}\in G$, $z_n\ne z_0$.
Then it can be shown that,
if two analytic functions
$f$ und $g$
on the domain $G$  coincide in the points $z_n$,
then they coincide on the entire domain $G$.

\subsection{Entire function}

An function is said to be an {\em entire function}
if it is defined and differentiable (holomorphic, analytic)
in the entire {\em finite complex plane} ${\Bbb C}$.
\index{entire function}

An entire function may be
either a
{\em rational function}
\index{rational function}
$f(z)=P(z)/Q(z)$
which can be written as the ratio of two polynomial functions
$P(z)$ and $Q(z)$,
or it may be a
{\em transcendental function}
\index{transcendental function}
such as $e^z$ or $\sin z$.


The
{\em  Weierstrass factorization theorem}
\index{Weierstrass factorization theorem}
\index{power series}
states that an entire function can be represented by a
(possibly infinite \cite{Gamelin-ca})
product involving its zeroes [i.e., the points $z_k$
at which the function vanishes $f(z_k)=0$].
For example (for a proof, see Eq. (6.2) of
\cite{conway-focvI}),
\begin{equation}
\sin z = z \prod_{k=1}^\infty \left[ 1-  \left(\frac{z}{\pi k}\right)^2\right].
\end{equation}


\subsection{Liouville's theorem for bounded entire function}
\label{2012-m-ch-ca-lt}
{\em Liouville's theorem}
\index{Liouville theorem}
states that
a bounded (i.e., its absolute square is finite everywhere in ${\Bbb C}$)
entire function which is defined at infinity
is a constant.
Conversely, a nonconstant entire function cannot be bounded.

No proof is presented here.
\marginnote{It may (wrongly)
appear that $\sin z$ is nonconstant and bounded. However
it is only bounded on the real axis;
indeeed,  $\sin iy = (1/2)(e^y+e^{-y}) \rightarrow \infty$ for $y\rightarrow \infty$.}


 {\color{OliveGreen}
\bproof
For a proof, consider the integral representation of the derivative $f'(z)$
of some bounded entire function $f(z)<C$ (suppose the bound is $C$)
obtained through Cauchy's integral formula (\ref{2012-m-ch-cagcif}),
taken along a circular path with arbitrarily large radius $r$ of length $2\pi r$ in the limit of infinite radius;
that is,
 \begin{equation}
\begin{split}
\left| f'(z_0) \right| = \left|
{1\over 2\pi i}\oint_{\partial G}{f(z)\over
 (z-z_0)^{2}}dz
\right|   \\
<
{1\over 2\pi i}\oint_{\partial G}{ \left| f(z)\right|  \over
 (z-z_0)^{2}}dz
<  \frac{1}{2\pi i} 2\pi r  \frac{C}{r^2} =  \frac{C}{r} \stackrel{r\rightarrow \infty}{\longrightarrow} 0   .
\end{split}
 \end{equation}
As a result, $f(z_0)=0$ and thus $f = A \in {\Bbb C}$.
\eproof
}





\subsection{Picard's theorem}
\label{2012-m-ch-ca-pt}
{\em Picard's theorem}
\index{Picard theorem}
states that any entire function that misses two or more points
$f: {\Bbb C} \mapsto  {\Bbb C} - \{z_1,z_2,\ldots \}$
is constant.
% http://people.reed.edu/~jerry/311/mats.html
Conversely, any nonconstant entire function covers the entire
complex plane ${\Bbb C}$ except a single point.

An example for a nonconstant entire function is $e^z$ which never reaches the point $0$.

\subsection{Meromorphic function}

If $f$ has no singularities other than poles in the domain
$G$ it is called {\em meromorphic} in the domain $G$.
\index{meromorphic function}


We state without proof (e.g., Theorem 8.5.1 of \cite{Hille62})
that a function $f$ which is meromorphic in the extended plane
is a rational function $f(z)=P(z)/Q(z)$
which can be written as the ratio of two polynomial functions
$P(z)$ and $Q(z)$.



\section{Fundamental theorem of algebra}
\index{fundamental theorem of algebra}


The {\em factor theorem} states that a polynomial $f(z)$  in $z$  of degree $k$
\index{factor theorem}
has a factor $z-z_0$ if and only if $f(z_0)=0$, and can thus be written as $f(z)= (z-z_0)g(z)$,
where $g(z)$ is a polynomial  in $z$  of degree $k-1$.
Hence, by iteration,
\begin{equation}
f(z)= \alpha \prod_{i=1}^k \left(z-z_i\right),
\end{equation}
where $\alpha \in {\Bbb C}$.

No proof is presented here.

The {\em fundamental theorem of algebra} states that
every polynomial (with arbitrary complex coefficients) has a root [i.e. solution of $f(z)=0$] in the complex plane.
Therefore, by the factor theorem, the number of roots of a polynomial, up to multiplicity, equals its degree.

Again, no proof
\marginnote{{\tiny https://www.dpmms.cam.ac.uk/~wtg10/ftalg.html}}
is presented here.
