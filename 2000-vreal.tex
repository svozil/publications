\documentstyle[12pt]{article}
%\renewcommand{\baselinestretch}{2}
\begin{document}

\def\frak{\cal }
\def\Bbb{\bf }

\title{Science at the crossroad between randomness and determinism}
\author{Karl Svozil\\
 {\small Institut f\"ur Theoretische Physik,}
  {\small Technische Universit\"at Wien }     \\
  {\small Wiedner Hauptstra\ss e 8-10/136,}
  {\small A-1040 Vienna, Austria   }            \\
  {\small e-mail: svozil@tuwien.ac.at}}
\date{ }
\maketitle

\begin{flushright}
{\scriptsize http://tph.tuwien.ac.at/$\widetilde{\;\;}\,$svozil/publ/vreal.tex,dvi,htm,ps,tex$\}$}
\end{flushright}

\begin{abstract}
Time and again, man's understanding
of Nature is at the crossroad between total world-comprehension
and total randomness.
It is suggested that not only are the preferences influenced by
the theories and models of today, but also by the very personal subjective
inclinations of the people involved.
The second part deals with the principle of self-consistency and its consequences
for totally deterministic systems.
\end{abstract}



\section{Who is more afraid of what?}

Let me start with a question  to you, the reader of this article.
\begin{quote}
{\em ``What appears to be more frightening:
a clocklike universe which is totally governed by deterministic laws,
or a lawless universe which is totally unpredictable and random?''}
\end{quote}

\subsection{Clocklike universe}
In a totally deterministic ``clocklike'' universe, every single phenomenon
is predetermined by its previous state.
Once the initial stage is ``set up,''
its creator gets detached from it and watches---without in any way
influencing it---as time and events go by.

In particular, no room is left for free will at all.
To any kind of personality and conscious agent
imprisoned in such a universe, free will must be a subjective impression
which is an illusion---{\it Maya}.
If these agents could only look behind the scene, then they would know.
But as the
clocklike universe is hermetic, to them any such beyond
does not make any operational sense.


Clocklike universes are nowadays best described by the term ``algorithm''
\cite{kreisel,svozil-93}.
Via the Church-Thesis, they can be even formalized by recursive function theory
\cite{rogers1,odi:89}.
From this point of view,
the universe appears as a gigantic (from our perspective), presumably universal, computer.
Conscious agents are just temporary imprints or patterns
on whatever ``hyper-substance'' it may be made of.

If this indeed would be the case with the universe we are living in, then
what appears to be amazing is the mere possibility of our self to imagine these
scenarios; to phantasize about free will being an illusion and about a hierarchical
organization of reality; to express {\it Maya}.
This is not totally new:
already von Neumann considered the possibilities of implanting agents in a
universal cellular automata substratum capable of self-reproduction and introspection
\cite{v-neumann-66}.
Fredkin has developed  ``digital mechanics'' \cite{fredkin} and ``digital soal.''

Within totally deterministic systems, subjective indeterminism
may result from intrinsic undecidability.
There exist various forms of intrinsic indeterminism (see \cite{svozil-93} for a review);
among them undecidability
analogous to the recursive unsolvability of the halting problem, and
computational complementarity \cite{cal-sv-yu}.

Let me gear up this scenario by purporting that not only might the universe be clocklike,
but {\em reversibile.}
That means that every process therein, every single evolution step,
is one-to-one; in more formal terms, the evolution map between initial and final state
is bijective.

In such a reversible hermetic prison, the time
evolution is a constant permutation of one and the same ``message''
which always remains the same but expresses itself through different forms.
Information is neither created nor discarded but remains constant at all times.
The implicit time symmetry spoils the very notion of
``progress'' or ``achievement,''  since what
is a valuable output is purely determined by the
subjective meaning the observer associates with it and is devoid of any
syntactic relevance. In such a
scenario, any gain in knowledge remains a merely subjective
impression of ignorant observers.



Let us now turn to the other extreme.


\subsection{Lawless universe}


Both chaos theory and quantum mechanics assert that there is an irreducible randomness in nature.

One concrete example of this allegedly irreducible randomness is the ``quantum coin toss''
\cite{svozil-qct} realized recently be the group of Anton Zeilinger \cite{zeilinger:qct}.
It is a which-way detection of a single photon passing through a semitransparent mirror
or a calcit crystal.

A lawless universe is characterized by the---admittedly highly nonconstructive---property
that it is not governed by any law at all.
There could be no principle which could in any way ``explain'' or ``predict'' the
performance of such a universe.
More importantly: there could be no control over events.
Formally, a lawless universe can be represented by a Martin-L\"of/Solovay/Chaitin random
\cite{chaitin2,calude:94,chaitin-99}
bit string \cite{calude-meyerstein,calude:pr}.


This does not mean that on a {\em local} scale, say, for any finite number of
phenomenologic occurrences
or evolution steps, the lawless universe cannot appear to be governed by laws.
Indeed, some observers embedded in a totally lawless universe
\cite{bos,toffoli:79,svo5,svo-86,roessler-87,svozil-93,atman:93}
might figure out some local structure and believe that this could persist
for any finite time for any finite extension.
They, like us, might call this the cosmological principle.

Because of the lack of meaning, observers could experience total freedom.
This resembles the absurd freedom of existentialism.
Because if there is no law, there cannot be any convincing moral codex, at least globally.
Any kind of behaviour or decision  would at most make local sense,
but would be devoid of any deeper, permanent relevance.
From a global ethical point of view,
any decision would be reduced to the throwing of a fair coin.

It is not totally unreasonable to speculate that the
cosy little lawful local worlds some observers appear to be living in
could be a mere subjective fantasy, a subjective impression
which is an illusion---{\it Maya} again.
And physics and all natural sciences may  just amount
to pretentious talk about finite lawful
bubbles within an endless ocean of chaos.


This may be not the full story.
Consider a related question,  namely
\begin{quote}
{\em
``Can there be order out of chaos?''
}
\end{quote}

As of today, the answer to this question is unknown.
A quite straightforward positive answer can be given by applying the law of large numbers:
if, for instance, one is measuring the output of a random
source emitting the binary symbols ``0'' and ``1'', and if one
just waits long enough, then each one of these binary symbols occurs with
probability $1/2$.

Formally, Martin-L\"of/Solovay/Chaitin random
sequences are Borel normal; i.e., contain the code of any finite universe an infinite
number of times.
By the very way it was defined, any Martin-L\"of/Solovay/Chaitin random
sequence obeys all statistical laws associated with randomness.

If we are justified to derive more lawful structures out of such
random sources is debatable but challenging.
The most radical answer I can think of is that there is a {\em unique}
and {\em robust}
class of laws emerging, and
that these laws correspond to the physical universe we are living in.
Robust in this context means that the laws are not changed ``very much''
if we focus on different finite parts of the source code.


\subsection{Miracles}

Besides the clocklike and the lawless universe there appears to be
at least another variant:
A clocklike universe inspired by miracles.
In what follows, we shall denote by ``miracle'' all
{\it ad hoc} occurrances which can in no way be explained in an otherwise
clocklike universe.
Miracles have been studied by the Vienna Circle,
in particular by Philip Franck \cite{frank}.

Imagine the following example.
Suppose you are an actor in a virtual computer game (such as Quake)
in which a number of persons interact collectively.
Their virtual reality environment is totally lawful:
it is created by a single computer or a network of computers.
Yet, what is going on in this virtual environment is not totally determined by
the computer system alone, but decisively by the constant input of the players.
The players act and input via interfaces.
Since the interface is not total, ``part of'' the player will always be beyond
the scope of the game.
Thus many of the interventions of the players are beyond the scope of the
limited domain of the virtual reality interface through which they interact.

Let us consider a trivial example: one player feels hungry and decides to take
a break and order some Pizza in the ``real world.''
This act may come as a total surprise and cannot be precisely predicted or predetermined
within the ``virtual world'' of the game.

Almost needless to say, this picture is an old idea in a relatively new context---dualism.


\subsection{Personal preferences}

As the topic is far from being settled,
it is not unreasonable to assume that each individual researcher has
his or her personal preferences.
We take the position here that these preferences are mostly determined by the person's
fears and desires.

Clocklike universes may appear monotonic and dull, without any possibility to act freely.
Lawless universes may appear totally incomprehensive, arbitrary and weird.

On the other hand, at least to a certain extend, clocklike universes appear
(subjectively) controllable and predictable.
This possibility may bring about a certain kind of dignity felt by the Enlightenment:
man is not confronted with a totally random environment but can influence the world
according to his own desires.

Lawless universes seem to guarantee spontaneity and freedom.
They dont appear to be hermetic prisons and have an open
future which is constantly created.


\section{Limits to forcast and event control}

Are there limits to event forecast and event control
for observers embedded in totally deterministic systems?

Here we shall argue for  {\em complementarity} in such systems.
It is a robust notion insofar this feature does not depend on
the particular type of deterministic system.

Intuitively, complementarity states that it is
impossible to (irreversibly) observe certain observables simultaneously
with arbitrary accuracy. The more precisely one of these
observables is measured, the less precisely can be the measurement of
other---complementary---observables. Typical examples of complementary
observables are position/momentum (velocity), angular momentum in
the x/y/z direction, and particle number/phase
\cite{peres,wheeler-Zurek:83}.



Let us develop {\em computational complementarity,} as it is
often called \cite{e-f-moore,finkelstein-83}, as a game between you as the reader
and me as the author. The rules of the game are as follows.
I first give you all you need to know about the intrinsic workings of
the
automaton. For example, I tell you, ``if the automaton is in state 1 and
you input the symbol 2, then the automaton will make a transition into
state 2 and output the symbol 0;'' and so on.
Then I present you a black box which contains a realization of the
automaton. The black box has a keyboard, with
which you input the input symbols. It has an output display, on which
the output symbols appear. No other interfaces are allowed.
Suppose that I can choose in which initial state the automaton is at the
beginning of the game. I do not tell you this state. Your goal is to
find out by experiment which state I have chosen. You can simply guess
or relying on your luck by throwing a dice. But you can also perform
clever input-output experiments and analyze
your data in order to find out. You win if you give the correct answer.
I win if you guess incorrectly. (So, I have to be mean and select
worst-case examples).

Suppose that you try very hard. Is cleverness sufficient?
Will you always be able to uniquely determine the initial automaton
state?

The answer to that question is ``no.'' The reason for this
is that there may be situations when the input causes an irreversible
transition into a state which does not allow any
further queries about the initial state.
This is the meaning of the term
``self-interference'' mentioned above.
Any such irreversible loss of information about the initial value of the
automaton can be traced back
to many-to-one operations \cite{landauer}: different states
are mapped onto a single state with the same output. Many-to-one
operations such as ``deletion of information'' are the only
source of entropy increase
in mechanistic systems \cite{landauer,bennett-82}.


The reader is refered to much more detailed accounts in refs.
\cite{svozil-93,schaller-96,cal-sv-yu}.

\section{Principle of self-consistency}

Let us assume, for the rest of the article, that the universe is clocklike.

In this part we shall review consequences  of the basic and most evident consistency
requirement---that measured events cannot happen and not happen at the same
time.
As a consequence, particular, very general bounds on the forecast and
control of events within the known laws of physics are derived. These bounds are of a global, statistical
nature and need not affect singular events or groups of events.

An irreducible, atomic physical phenomenon
manifests itself as a click of some detector. There can either be a click or
there can be
no click. This yes-no scheme is experimental physics in-a-nutshell (at
least according to a theoretician). From
this type of elementary observation, all of our physical
evidence is accumulated.
Irreversibly observed events of physical reality (in the context in
which they can be defined
\cite{greenberger2,hkwz,sv-forthcoming})
are subject to the primary condition of {\em consistency} or
{\em self-consistency}.

\begin{quote}
{\em Any particular irreversibly observed event can either
happen or cannot happen, but it must not both happen and not happen.}
\end{quote}


Indeed, so trivial seems the requirement of
consistency for the set of physically recorded events that
David Hilbert polemicised against ``another
author'' with the following words \cite{hilbert-26}, ``...for me, the
opinion that the
[[physical]] facts and events
themselves can be contradictory is a good example of thoughtlessness.''


Just as in mathematics, inconsistency, i.e., the coexistence of  truth
and falseness of  propositions, is a fatal property of any
physical theory. Nevertheless, in a certain very precise sense, quantum
mechanics incorporates inconsistencies in a very subtle way  which
assures overall consistency. For instance, a particle
wave function or quantum state is said to ``pass'' a double slit through both
slits, which is classically impossible. (Such considerations may, however, be
considered as mere trickery quanum talk, devoid of any operational meaning.)
Yet, neither a particle wave
function nor quantum states are directly associable with any sort of
irreversible observed event of physical reality.


And just as in mathematics it can be argued
that too
strong capacities of event forecast and event control renders the
system overall inconsistent.


\subsection{Strong forecasting}
Let us consider forecasting the future first.
Even if physical phenomena occur
deterministically and can be accounted for ("computed") on a higher
level of abstraction, from within the system such a complete description
may not be of much practical, operational use.

Indeed, suppose there exists free will. Suppose further that an agent could
predict {\em all} future events, without exceptions. We shall call this
the
{\em strong form of forecasting.}
In this case, the agent could freely decide to
counteract in such a way as to invalidate that prediction.
Hence, in order to avoid inconsistencies and  paradoxes,  either  free
will
has to be abandoned or it has to be accepted that complete prediction is
impossible.

Another possibility would be to consider strong forms of forecasting
which are, however, not utilized to alter the system.
Effectively, this results in the abandonment of free will,
amounting to an extrinsic, detached viewpoint.
After all, what is knowledge and what is it good for if it cannot be
applied and made to use?

It should be mentioned that the above argument is of an
ancient type. It has been formalized recently in set theory, formal
logic and recursive function theory, where it is called
``diagonalization method.''

\subsection{Strong event control}
A very similar argument holds for event control and the production of
``miracles'' \cite{frank}.
Suppose there exists free will. Suppose further that an agent could
entirely control the future. We shall call this the {\em strong form of
event control.} Then
this observer could freely decide to invalidate the laws of physics.
In order to avoid a paradox,  either  free will or some
physical laws would have to
be abandoned,  or it has to be accepted that
complete event control is impossible.

\begin{quote}
{\em Stated differently, forecast and event control should be possible
only if this capacity cannot be associated with any paradox or contradiction.}
\end{quote}

Thus the requirement of consistency of the phenomena seems to impose
rather stringent conditions on  forecasting and
event control. Similar ideas have  already been discussed in the context
of time paradoxes in relativity theory (cf. \cite{friedetal} and
\cite[p. 272]{nahin}, {\em ``The only solutions to the laws of physics
that can occur locally $\ldots$ are those which are globally
self-consistent''}).



\subsection{Weak forcast and event control}
There is, however, a possibility that the forecast and control of future
events {\em is} conceivable for {\em singular} events within the
statistical bounds. Such occurrences
may be ``singular miracles'' which are well accountable within
classical physics. They will be called {\em weak forms of
forecasting and event control.}

It may be argued that, in order to obey overall
consistency, such a framework should not be extendable to any forms of strong forecast or
event control, because, as has been argued before, this could
either violate global consistency criteria
or would make necessary a revision of the known laws of physics.

It may be argued that weak forms of forecasting and event
control amount to nothing else than the impossibility of {\em any forms
of forecasting and event control} at all.

This, however, needs not to be the case. The laws of statistics impose
rather
lax constraints and do not exclude local, singular, improbable events.
For example, a binary sequence such as
$$11111111111111111111111111111111$$
is just as probable as the sequences
$$11100101110101000111000011010101$$
$$01010101010101010101010101010101$$
and its occurrence in a test is equally likely, although its statistical
property and the ``meaning'' an observer could ascribe to it is rather
outstanding.

Just as it is perfectly all right
to consider the statement ``This statement is true'' to be true, it may
thus be perfectly reasonable to speculate that certain events are
forecasted and controlled within the domain of statistical laws.
But in order to be  within the statistical laws, any such method {\em needs
not to be guaranteed} to work all the time.


To put it pointedly: it may be perfectly reasonable to become rich, say,
by
singular forecasts of the stock and future values or in horse races, but
such an ability
must necessarily be irreproducible and secretive. At least to such an
extend that no guarantee of an overall strategy can be derived from it.

The associated weak forms of forecasting and
event control are thus beyond any global statistical significance. Their
importance and meaning seem to lie mainly on a very subjective level of
singular events. This comes close to one aspect of what  Jung imagined
as the principle of ``Synchronicity'' \cite{jung1}.


\subsection{Against the odds}
This final paragraphs review a couple of experiments which suggest
themselves in the context of weak forecast and event control.
All are based on the observation that an agent forcasts or controls correctly
future events such as, say, the tossing of a fair coin.

In the first run of the experiment, no consequence is derived from the
agent's capacity despite the mere recording of the data.

The second run of the experiment is like the first run, but the {\em
meaning} of the forecasts or controlled events are different. They are
taken as outcomes of, say gambling, against other individuals (i) with
or (ii)
without similar capacities, or against (iii) an anonymous ``mechanic''
agent such as a casino or a stock exchange.

As a variant of this experiment, the partners or adversaries of the
agent are informed about the agent's intentions.

In the third run of experiments, the experimenter attempts to counteract
the agent's capacity. Let us assume the experimenter has total control
over the event. If the agent predicts or attempts to bring
about to happen a certain future event, the experimenter causes the
event not to happen and so on.

It might be interesting to record just how much the agent's capacity
is changed by the setup. Such a correlation might be defined from
a dichotomic observable
$$e(A,i) =\left\{
\begin{array}{cl}
+1&\qquad {\rm correct\; guess}\\
-1&\qquad {\rm incorrect\; guess}\\
\end{array}
\right.
$$
where $i$ stands for the $i$'th experiment and $A$ stands for the agent
$A$. A correlation function can then be defined as usual by the average
over $N$ experiments; i.e.,
$$C(A) = {1\over N}\sum_{i=1}^N e(A,i).$$

From the first to the second type of experiment it should become more
and
more unlikely that the agent operates correctly, since his performance
is leveled against other agents with more or less the same capacities.
The third type of experiment should produce a
total uncorrelation.


\section*{Postscript}
Instead of a summary, let me cite from a 1983 poem by
Erich Christian Schreibm�ller.

\begin{quote}
{\it Er nennt sich heimlich den ausgelassensten
Dentisten der Galaxie, doch wei� er
nat�rlich nichts von den wahren Verh�ltnissen.}
\end{quote}

\begin{quote}
English translation:
{\it Secretly he calls himself the most flamboyant
dentist of the galaxy, but of course he does not realize
the true circumstances.}
\end{quote}



%\bibliography{svozil}
%\bibliographystyle{unsrt}
%\bibliographystyle{plain}


\begin{thebibliography}{10}

\bibitem{kreisel}
G.~Kreisel.
\newblock A notion of mechanistic theory.
\newblock {\em Synthese}, 29:11--16, 1974.

\bibitem{svozil-93}
Karl Svozil.
\newblock {\em Randomness \& Undecidability in Physics}.
\newblock World Scientific, Singapore, 1993.

\bibitem{rogers1}
Hartley {Rogers, Jr.}
\newblock {\em Theory of Recursive Functions and Effective Computability}.
\newblock MacGraw-Hill, New York, 1967.

\bibitem{odi:89}
Piergiorgio Odifreddi.
\newblock {\em Classical Recursion Theory}.
\newblock North-Holland, Amsterdam, 1989.

\bibitem{v-neumann-66}
John von Neumann.
\newblock {\em Theory of Self-Reproducing Automata}.
\newblock University of Illinois Press, Urbana, 1966.
\newblock A. W. Burks, editor.

\bibitem{fredkin}
Edward Fredkin.
\newblock Digital information mechanics.
\newblock {\em Physica}, D45:254, 1990.
\newblock technical report, August 1989.

\bibitem{cal-sv-yu}
Cristian Calude, Elena Calude, Karl Svozil, and Sheng Yu.
\newblock Physical versus computational complementarity {I}.
\newblock {\em International Journal of Theoretical Physics}, 36(7):1495--1523,
  1997.

\bibitem{svozil-qct}
Karl Svozil.
\newblock The quantum coin toss-testing microphysical undecidability.
\newblock {\em Physics Letters}, A143:433--437, 1990.

\bibitem{zeilinger:qct}
Thomas Jennewein, Ulrich Achleitner, Gregor Weihs, Harald Weinfurter, and Anton
  Zeilinger.
\newblock A fast and compact quantum random number generator.
\newblock e-print http://xxx.lanl.gov/abs/quant-ph/9912118, 1999.

\bibitem{chaitin2}
Gregory~J. Chaitin.
\newblock {\em Information, Randomness and Incompleteness}.
\newblock World Scientific, Singapore, second edition, 1990.
\newblock This is a collection of G. Chaitin's publications.

\bibitem{calude:94}
Cristian Calude.
\newblock {\em Information and Randomness---An Algorithmic Perspective}.
\newblock Springer, Berlin, 1994.

\bibitem{chaitin-99}
Gregory~J. Chaitin.
\newblock {\em The Unknowable}.
\newblock Springer-Verlag, Singapore, 1999.

\bibitem{calude-meyerstein}
Cristian Calude and F.~Walter Meyerstein.
\newblock Is the universe lawful?
\newblock {\em Chaos, Solitons \& Fractals}, 10(6):1075--1084, 1999.

\bibitem{calude:pr}
Cristian Calude.
\newblock Private communication.

\bibitem{bos}
R.~J. Boskovich.
\newblock {\em De spacio et tempore, ut a nobis cognoscuntur}.
\newblock Vienna, 1755.
\newblock English translation in \cite{bos1}.

\bibitem{toffoli:79}
T.~Toffoli.
\newblock The role of the observer in uniform systems.
\newblock In G.~Klir, editor, {\em Applied General Systems Research}. Plenum
  Press, New York, London, 1978.

\bibitem{svo5}
Karl Svozil.
\newblock Connections between deviations from lorentz transformation and
  relativistic energy-momentum relation.
\newblock {\em Europhysics Letters}, 2:83--85, 1986.
\newblock excerpts from \cite{svo-83}.

\bibitem{svo-86}
Karl Svozil.
\newblock Operational perception of space-time coordinates in a quantum medium.
\newblock {\em Il Nuovo Cimento}, 96B:127--139, 1986.

\bibitem{roessler-87}
Otto~E. R{\"{o}}ssler.
\newblock Endophysics.
\newblock In John~L. Casti and A.~Karlquist, editors, {\em Real Brains,
  Artificial Minds}, page~25. North-Holland, New York, 1987.

\bibitem{atman:93}
Harald Atmanspacher and G.~Dalenoort, editors.
\newblock {\em Inside Versus Outside}, Berlin, 1994. Springer.

\bibitem{frank}
Philip Frank.
\newblock {\em Das Kausalgesetz und seine Grenzen}.
\newblock Springer, Vienna, 1932.

\bibitem{peres}
Asher Peres.
\newblock {\em Quantum Theory: Concepts and Methods}.
\newblock Kluwer Academic Publishers, Dordrecht, 1993.

\bibitem{wheeler-Zurek:83}
John~Archibald Wheeler and Wojciech~Hubert Zurek.
\newblock {\em Quantum Theory and Measurement}.
\newblock Princeton University Press, Princeton, 1983.

\bibitem{e-f-moore}
Edward~F. Moore.
\newblock Gedanken-experiments on sequential machines.
\newblock In C.~E. Shannon and J.~McCarthy, editors, {\em Automata Studies}.
  Princeton University Press, Princeton, 1956.

\bibitem{finkelstein-83}
David Finkelstein and Shlomit~R. Finkelstein.
\newblock Computational complementarity.
\newblock {\em International Journal of Theoretical Physics}, 22(8):753--779,
  1983.

\bibitem{landauer}
R.~Landauer.
\newblock Information is physical.
\newblock {\em Physics Today}, 44:23--29, May 1991.

\bibitem{bennett-82}
Charles~H. Bennett.
\newblock The thermodynamics of computation---a review.
\newblock In {\em International Journal of Theoretical Physics\/}
  \cite{maxwell-demon}, pages 905--940.
\newblock Reprinted in \cite[pp. 213-248]{maxwell-demon}.

\bibitem{schaller-96}
Martin Schaller and Karl Svozil.
\newblock Automaton logic.
\newblock {\em International Journal of Theoretical Physics}, 35(5):911--940,
  May 1996.

\bibitem{greenberger2}
Daniel~B. Greenberger and A.~YaSin.
\newblock ``{H}aunted'' measurements in quantum theory.
\newblock {\em Foundation of Physics}, 19(6):679--704, 1989.

\bibitem{hkwz}
Thomas~J. Herzog, Paul~G. Kwiat, Harald Weinfurter, and Anton Zeilinger.
\newblock Complementarity and the quantum eraser.
\newblock {\em Physical Review Letters}, 75(17):3034--3037, 1995.

\bibitem{sv-forthcoming}
Karl Svozil.
\newblock Quantum interfaces.
\newblock forthcoming, 2000.

\bibitem{hilbert-26}
David Hilbert.
\newblock {\"{U}}ber das {U}nendliche.
\newblock {\em Mathematische Annalen}, 95:161--190, 1926.

\bibitem{friedetal}
John Friedman, Michael~S. Morris, Igor~D. Novikov, Fernando Echeverria, Gunnar
  Klinkhammer, Kip~S. Thorne, and Ulvi Yurtsever.
\newblock Cauchy problem in spacetimes with closed timelike curves.
\newblock {\em Physical Review}, D42(6):1915--1930, 1990.

\bibitem{nahin}
Paul~J. Nahin.
\newblock {\em Time Travel (Second edition)}.
\newblock AIP Press and Springer, New York, 1998.

\bibitem{jung1}
Carl~Gustav Jung.
\newblock Synchronizit{\"{a}}t als ein {P}rinzip akausaler
  {Z}usammenh{\"{a}}nge.
\newblock In Carl~Gustav Jung and Wolfgang Pauli, editors, {\em
  Naturerkl{\"{a}}rung und Psyche}. Rascher, Z{\"{u}}rich, 1952.

\bibitem{bos1}
R.~J. Boskovich.
\newblock De spacio et tempore, ut a nobis cognoscuntur.
\newblock In J.~M. Child, editor, {\em A Theory of Natural Philosophy}, pages
  203--205. Open Court (1922) and MIT Press, Cambridge, MA, 1966.

\bibitem{svo-83}
Karl Svozil.
\newblock On the setting of scales for space and time in arbitrary quantized
  media.
\newblock {\em Lawrence Berkeley Laboratory preprint}, LBL-16097, May 1983.

\bibitem{maxwell-demon}
H.~S. Leff and A.~F. Rex.
\newblock {\em Maxwell's Demon}.
\newblock Princeton University Press, Princeton, 1990.

\end{thebibliography}
\end{document}

