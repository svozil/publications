\documentclass[pra,amsfonts,showpacs,showkeys,preprint]{revtex4}

\usepackage{xcolor}
\usepackage{eepic}
\usepackage{graphicx}% Include figure files

\usepackage{bm}% bold math
\RequirePackage{times}
\RequirePackage{mathptm}
\RequirePackage{courier}

\usepackage{longtable}
\usepackage{dcolumn}% Align table columns on decimal point

\renewcommand{\labelenumi}{(\roman{enumi})}
\bibpunct{[}{]}{,}{a}{}{;}




%\usepackage{cdmtcs}

\begin{document}


%\cdmtcsauthor{Karl Svozil}
%\cdmtcstitle{Some observations concerning the plasticity of nonlocal quantum correlations exceeding classical expectations}
%\cdmtcsaffiliation{Vienna University of Technology}
%\cdmtcstrnumber{346}
%\cdmtcsdate{February 2009}
%\colourcoverpage


\title{Plasticity of quantum correlations}

\author{Karl Svozil}
\email{svozil@tuwien.ac.at}
\homepage{http://tph.tuwien.ac.at/~svozil}
\affiliation{Institute for Theoretical Physics, Vienna University of Technology,  \\
Wiedner Hauptstra\ss e 8-10/136, A-1040 Vienna, Austria}


\begin{abstract}
A redefinition of the quantum correlations of four level systems present the possibility of stronger-than-classical expectations capable of violating Boole-Bell type inequalities even beyond Tsirelson's bound.
\end{abstract}

\pacs{03.67.Hk,03.65.Ud}
\keywords{Quantum information, quantum communication, singlet states, group theory, entanglement, quantum nonlocality}




\maketitle

%\tableofcontents
%\newpage

\section{Introduction}

The observation of stronger-than-classical correlations~\cite{clauser,aspect-82a,peres222} for nonlocal, i.e., spatially and even causally separated, quanta
in  ``delayed choice'' measurements has been experimentally verified~\cite{wjswz-98}.
A typical phenomenologic criterion of such correlations it the {\em increased} of {\em decreased} frequency of the occurrence of certain coincidences of outcomes,
such as the more- or less-often-than-classically expected recordings of joint spin up and down measurements labelled by ``$++$,'' ``$+-$,'' ``$-+$'' or ``$--$,'' respectively.

The physical meaning of statements referring to stronger-than-classical quantum correlations is at least threefold:
(i) First, as has been already mentioned, there is a direct, operational meaning:
certain joint outcomes of single particle measurements, when collected and compared to each other,
appear to occur more or less often than could be expected classically.
(ii) Second, the resulting frequencies, as well as the quantum theoretical probabilities and expectations derived from the Born rule or Gleason's theorem,
seem to contradict the {\em ``conditions of possible experience''} investigated by Boole~\cite{Boole,Boole-62} and in later times by
Bell and others~\cite{bell,Pit-94,2000-poly}.
(iii) Third, the quantum correlations indicate that quantum probabilities, unlike classical probabilities~\cite{pitowsky,svozil-2008-ql}
cannot be based upon the convex sum of classical two-valued measures,
because there are no two-valued measures (interpretable as classical global truth assignments) for quantized systems
with more than two mutually exclusive outcomes~\cite{specker-60,kochen1,ZirlSchl-65,Alda,Alda2,kamber64,kamber65,svozil-tkadlec,cabello-96}.


Stated pointedly, the ``magic'' behind the quantum correlations, as compared to classical correlations, resides in the fact that
for allmost all measurement directions (despite collinear or orthogonal ones), an observer ``Alice,''
when recording some outcome of a measurement, can be sure that her partner ``Bob,''
although spatially and causally disconnected from her, is either more or less likely to record a particular measurement outcome on his side.
However, because of the randomness~\cite{svozil-qct} and uncontrollability~\cite{svozil-slash} of the individual events, and
because of the no-cloning theorem(e.g., Ref.~\cite[pp.~39-40]{mermin-07}),
no classically useful information can be transferred from Alice to Bob, or {\it vice versa:}
The parameter independence~\cite{clauser,shimony3} and outcome dependence of otherwise random events ensures that
the nonlocal correlations among quanta cannot be directly used to communicate classical information~\cite{pop-rohr,grunhaus-96}.
The expectations of the joint outcomes on Alice's and Bob's sides can only be  verified by collecting all the different outcomes {\it ex post facto,}
recombining joint events one-by one~\cite{Gill-Larss-04}.
Nevertheless, there are hopes and visions to utilize nonlocal quantum correlations for a wide range of explanations and
applications; for instance in quantum information theory~\cite{bruk-06} and life sciences~\cite{sum-05}.


In what follows a few known and novel quantum correlations will be systematically enumerated.
We shall derive the correlations between two and four two-state particles in singlet states.
We also derive the correlations of two three-, four- and general $d$-state particles in a singlet state.
Singlets are states of two or more quantum particles whose total angular momentum is zero,
although the angular momenta of the constituents are not.
They have the advantage that
they are {\em form invariant} with respect to directional changes; i.e., they ``look the same,'' regardless of the measurement direction.
Singlet states of two particles have the additional advantage that they satisfy a {\em uniqueness property} \cite{svozil-2006-uniquenessprinciple}
in the sense that knowledge of an outcome of one particle observable entails the certainty that,
if this observable were measured on the other particle(s) as well,
the outcome of the measurement would be a unique function of the outcome of the measurement performed.
A {\em counterfactual argument}~\cite[p.~243]{specker-60} envisioned by
Einstein-Podolsky-Rosen (EPR)~\cite{epr}
claims to measure and infer with certainty two nonco-measurable, incompatible observables associated with noncommuting operators counterfactually.
One context is measured on one side of the EPR setup, the other context on the other side of it.
By the uniqueness property  of certain two-particle states,
knowledge of a property of one particle entails the certainty
that, if this property were measured on the other particle as well, the outcome of the measurement would be
a unique function of the outcome of the measurement performed.
This makes possible the measurement of one observable, {\em as well as} the {\em simultaneous counterfactual inference} of another incompatible observable.
Because, one could argue, that although one has actually measured on one side a different, incompatible observable compared to the observable measured on the other side,
{\em if} on both sides the same  observable {\em would be measured}, the outcomes on both sides {\em would be uniquely correlated}.
Hence measurement of one observable per side is sufficient, for the outcome could be counterfactually inferred from the measurements on the other side.







% ~~~~~~~~~~~~~~~   2 x 4
% ~~~~~~~~~~~~~~~   2 x 4
% ~~~~~~~~~~~~~~~   2 x 4
% ~~~~~~~~~~~~~~~   2 x 4
% ~~~~~~~~~~~~~~~   2 x 4
% ~~~~~~~~~~~~~~~   2 x 4
% ~~~~~~~~~~~~~~~   2 x 4
% ~~~~~~~~~~~~~~~   2 x 4


\section{Four level systems}

In what follows, a detailed analysis of the quantum expectations and the parametrization of the associated expectation functions
of two four level systems will be presented.

\subsection{Observables}
The spin three-half angular momentum observables in units of $\hbar$ are given by~\cite{schiff-55}
\begin{equation}
M_x=
\frac{1}{2}
\left(
\begin{array}{cccccccccc}
0&\sqrt{3}&0&0\\
\sqrt{3}&0&2&0\\
0&2&0&\sqrt{3}\\
0&0&\sqrt{3}&0
\end{array}
\right),
\;
M_y=
\frac{1}{2}
\left(
\begin{array}{ccccccccccr}
0&-\sqrt{3}i&0&0\\
\sqrt{3}i&0&-2i&0\\
0&2i&0&-\sqrt{3}i\\
0&0&\sqrt{3}i&0
\end{array}
\right),
\;
M_z=
\frac{1}{2}
\left(
\begin{array}{cccccccccc}
3&0&0&0\\
0&1&0&0\\
0&0&-1&0\\
0&0&0&-3
\end{array}
\right).
\end{equation}

Again, the angular momentum operator in arbitrary direction $\theta$, $\varphi$ can be written in its spectral form
\begin{equation}
\begin{array}{rcl}
S_\frac{3}{2} (\theta ,\varphi) &=&
xM_x
+
yM_y
+
zM_z
=
 M_x  \sin \theta \cos \varphi
+
M_y   \sin \theta \sin \varphi
+
M_z   \cos \theta
\\
&=&   \left(
\begin{array}{cccc}
 \frac{3 \cos \theta }{2} & \frac{\sqrt{3}}{2}  e^{-i \varphi } \sin \theta  & 0 & 0 \\
 \frac{\sqrt{3}}{2}  e^{i \varphi } \sin \theta  & \frac{\cos \theta }{2} & e^{-i \varphi } \sin \theta  & 0 \\
 0 & e^{i \varphi } \sin \theta  & -\frac{\cos \theta }{2} & \frac{\sqrt{3}}{2}  e^{-i \varphi } \sin \theta  \\
 0 & 0 & \frac{ \sqrt{3}}{2} e^{i \varphi } \sin \theta  & -\frac{3 \cos \theta }{2}
\end{array}
\right)  \\
&=& -\frac{3}{2}F_{-\frac{3}{2}}(\theta ,\varphi) - \frac{1}{2} F_{-\frac{1}{2}}(\theta ,\varphi) +
\frac{1}{2}F_{+\frac{1}{2}}(\theta ,\varphi)+ \frac{3}{2}F_{+\frac{3}{2}}(\theta ,\varphi).
\end{array}
\label{e-2009-gtq-s444}
\end{equation}


If one is only interested in spin state measurements with the associated outcomes of spin states in units of $\hbar$,
the associated two-particle operator is given by
\begin{equation}
 S_{1 1 } ({\hat \theta},{\hat \varphi} ) =
S_{1 }( \theta_1,\varphi_1 )
\otimes
S_{1 }( \theta_2,\varphi_2 ).
\label{2004-gtq-e3F2nat}
\end{equation}

More generally, one could define a two-particle operator by
\begin{equation}
F^2_{\lambda_{-\frac{3}{2}},\lambda_{-\frac{1}{2}}, \lambda_{+\frac{1}{2}}, \lambda_{+\frac{3}{2}} } ({\hat \theta},{\hat \varphi} ) =
F_{\lambda_{-\frac{3}{2}},\lambda_{-\frac{1}{2}}, \lambda_{+\frac{1}{2}}, \lambda_{+\frac{3}{2}} } ( \theta_1,\varphi_1)
\otimes
F_{\lambda_{-\frac{3}{2}},\lambda_{-\frac{1}{2}}, \lambda_{+\frac{1}{2}}, \lambda_{+\frac{3}{2}} } ( \theta_2,\varphi_2 ),
\label{2004-gtq-e3F2gen}
\end{equation}
where
\begin{equation}
F_{\lambda_{-\frac{3}{2}},\lambda_{-\frac{1}{2}}, \lambda_{+\frac{1}{2}}, \lambda_{+\frac{3}{2}} } ( \theta ,\varphi )
=
 \lambda_{-\frac{3}{2}}F_{-\frac{3}{2}}(\theta ,\varphi) +  \lambda_{-\frac{1}{2}} F_{-\frac{1}{2}}(\theta ,\varphi) +
 \lambda_{\frac{1}{2}}F_{+\frac{1}{2}}(\theta ,\varphi)+  \lambda_{\frac{3}{2}}F_{+\frac{3}{2}}(\theta ,\varphi)
.
\label{2004-gtq-e3F2gen2}
\end{equation}
For the sake of the physical interpretation of this operator~(\ref{2004-gtq-e3F2gen}), let us consider as a concrete example
a spin state measurement on two quanta as depicted in Fig.~\ref{2009-gtq-f5}:
$ F_{\lambda_{-\frac{3}{2}}}(\theta_1 ,\varphi_1)\otimes   F_{\lambda_{+\frac{3}{2}}}(\theta_2 ,\varphi_2 )$  stands for the proposition
\begin{quote}
{\em `The outcome of the first particle measured along $\theta_1,\varphi_1$ is ``$\lambda_{-\frac{3}{2}}$''
      and
      the outcome of the second particle measured along $\theta_2,\varphi_2$ is ``$\lambda_{+\frac{3}{2}}$''~.'
}
\end{quote}

\begin{figure}
\begin{center}
%TeXCAD Picture [1.pic]. Options:
%\grade{\off}
%\emlines{\off}
%\epic{\on}
%\beziermacro{\on}
%\reduce{\on}
%\snapping{\off}
%\quality{2.000}
%\graddiff{0.010}
%\snapasp{1}
%\zoom{9.5137}
\unitlength 1mm % = 2.845pt
\allinethickness{1pt} %\thicklines %\linethickness{0.4pt}
\ifx\plotpoint\undefined\newsavebox{\plotpoint}\fi % GNUPLOT compatibility
\begin{picture}(120.037,46.452)(0,0)
\put(56,20.333){\line(4,3){8}}
\put(64,20.333){\line(-4,3){8}}
\put(5,29.365){\oval(10,10)[l]}
\put(115.037,29.365){\oval(10,10)[r]}
\put(5,17.304){\oval(10,10)[l]}
\put(115.037,17.304){\oval(10,10)[r]}
\put(5,5.216){\oval(10,10)[l]}
\put(115.037,5.216){\oval(10,10)[r]}
\put(5,41.452){\oval(10,10)[l]}
\put(115.037,41.452){\oval(10,10)[r]}
\put(5,24.365){\line(0,1){10}}
\put(115.037,24.365){\line(0,1){10}}
\put(5,22.304){\line(0,-1){10}}
\put(115.037,22.304){\line(0,-1){10}}
\put(5,10.216){\line(0,-1){10}}
\put(115.037,10.216){\line(0,-1){10}}
\put(5,36.452){\line(0,1){10}}
\put(115.037,36.452){\line(0,1){10}}
\put(2.5,29.365){\makebox(0,0)[cc]{{\footnotesize $+\frac{1}{2}$}}}
\put(117.537,29.365){\makebox(0,0)[]{\footnotesize $+\frac{1}{2}$}}
\put(2.5,17.304){\makebox(0,0)[cc]{\footnotesize $-\frac{1}{2}$}}
\put(117.537,17.304){\makebox(0,0)[]{\footnotesize $-\frac{1}{2}$}}
\put(2.5,5.216){\makebox(0,0)[cc]{\footnotesize $-\frac{3}{2}$}}
\put(117.537,5.216){\makebox(0,0)[]{\footnotesize $-\frac{3}{2}$}}
\put(2.5,41.452){\makebox(0,0)[cc]{\footnotesize $+\frac{3}{2}$}}
\put(117.537,41.452){\makebox(0,0)[]{\footnotesize $+\frac{3}{2}$}}
\put(10,16.257){\framebox(10,15)[cc]{\footnotesize $\theta_1,\varphi_1$}}
\put(100,16.257){\framebox(10,15)[cc]{\footnotesize $\theta_2,\varphi_2$}}
\put(60.019,23.23){\circle{9.727}}
%\vector[middle]{\line}
\put(65.379,23.335){\line(1,0){33.846}}\put(82.302,23.335){\vector(1,0){.07}}
%\end
%\vector[middle]{\line}
\put(54.658,23.335){\line(-1,0){33.846}}\put(37.735,23.335){\vector(-1,0){.07}}
%\end
\end{picture}
\end{center}
\caption{Simultaneous measurement of
the two particles with four outcome per particle. Boxes indicate spin state analyzers such as Stern-Gerlach apparatus
oriented along the directions $\theta_1,\varphi_1 $ and
$\theta_2,\varphi_2 $;
their two output ports are occupied with detectors  associated
with the outcomes
``$\lambda_{+\frac{3}{2}}$,''
``$\lambda_{+\frac{1}{2}}$,''
``$\lambda_{-\frac{1}{2}}$'' and
``$\lambda_{-\frac{3}{2}}$,''
respectively.
\label{2009-gtq-f5}}
\end{figure}

\subsection{Singlet state}

The singlet state of two spin-$3/2$ observables
can be found by the general methods developed in Ref.~\cite{schimpf-svozil}.
In this case, this amounts to summing all possible two-partite states yielding zero angular momentum,
multiplied with the corresponding  Clebsch-Gordan coefficients
\begin{equation}
\langle j_1m_1j_2m_2\vert 00\rangle = \delta_{j_1,j_2}  \delta_{m_1,-m_2} \frac{(-1)^{j_1-m_1}}{\sqrt{2j_1+1}}
\label{2009-gtq-cgordon0}
\end{equation}
of mutually negative single particle states resulting in total angular momentum zero.
More explicitly,  for $j_1=j_2=\frac{3}{2}$,
\begin{equation}
\left|  \left. \psi_{4,2,1} \right\rangle  \right. =
\frac{1}{2} \left(
\left| \left. \frac{3}{2}, -\frac{3}{2}\right\rangle \right.
 - \left| \left.  -\frac{3}{2}, \frac{3}{2}\right\rangle    \right.
- \left| \left.  \frac{1}{2}, -\frac{1}{2}\right\rangle  \right.
+ \left| \left.  -\frac{1}{2}, \frac{1}{2}\right\rangle   \right.
\right).
\end{equation}
Again, this two-partite singlet state satisfies the uniqueness property.
The four different spin states can be identified with the cartesian basis of fourdimensional Hilbert space
$\left| \left. \frac{3}{2}\right\rangle \right. \equiv (1,0,0,0)$,
$\left| \left. \frac{1}{2}\right\rangle \right. \equiv (0,1,0,0)$,
$\left| \left. -\frac{1}{2}\right\rangle \right. \equiv (0,0,1,0)$,
and
$\left| \left. -\frac{3}{2}\right\rangle \right. \equiv (0,0,0,1)$,
respectively.

\subsection{Results}



For the sake of comparison, let us again specify the rather lengthy
expectation function in the case of general observables with arbitrary outcomes $\lambda_i$, $i=1,\ldots ,4$
to the standard  quantum mechanical expectations~(\ref{2009-gtq-sso2})
and~(\ref{2009-gtq-edosgc3})
by setting $\lambda_{+\frac{3}{2}} = +\frac{3}{2}$,
$\lambda_{+\frac{1}{2}}= +\frac{1}{2}$,
$\lambda_{-\frac{1}{2}}=-\frac{1}{2}$ and
$\lambda_{-\frac{3}{2}}=-\frac{3}{2}$; i.e., by substituting the general outcomes with spin state observables in units of $\hbar$.
With these identifications, the expectation functions can be directly calculated {\it via} $S_{\frac{3}{2}\frac{3}{2}}$; i.e.,
\begin{equation}
\begin{array}{rcl}
E_{{ \Psi_{4,2,1}}\,-\frac{3}{2},-\frac{1}{2}, +\frac{1}{2}, +\frac{3}{2} } ({\hat \theta},{\hat \varphi} )
&=&
{\rm Tr}\left\{ \rho_{ \Psi_{4,2,1}} \cdot \left[ S_{\frac{3}{2}}(\theta_1,\varphi_1) \otimes S_{\frac{3}{2}}(\theta_2,\varphi_2)\right]\right\} \\
&=& -\frac{5}{4} \left[\cos \theta_1 \cos \theta_2 + \cos (\varphi_1 - \varphi_2) \sin \theta_1 \sin \theta_2\right] \\
&=& \frac{8}{15}E_{{ \Psi_{2,3,1}}\,-1, +1 } ({\hat \theta},{\hat \varphi} ) \\
&=& 5 E_{{ \Psi_{2,2,1}}\,-\frac{1}{2}, +\frac{1}{2} } ({\hat \theta},{\hat \varphi} )
= \frac{5}{4}E_{{ \Psi_{2,2,1}}\,-1, +1 } ({\hat \theta},{\hat \varphi} )
\end{array}
\label{2009-gtq-edosgc4}
.
\end{equation}
This expectation function is again functionally identical with the spin one-half and spin one (two and three outcomes) expectation functions.

The plasticity of the general expectation function
\begin{equation}
E_{{ \Psi_{4,2,1}}\,\lambda_{-\frac{3}{2}},\lambda_{-\frac{1}{2}}, \lambda_{+\frac{1}{2}}, \lambda_{+\frac{3}{2}} } ({\hat \theta},{\hat \varphi} )
=
{\rm Tr}\left[ \rho_{ \Psi_{4,2,1}} \cdot  F^2_{\lambda_{-\frac{3}{2}},\lambda_{-\frac{1}{2}}, \lambda_{+\frac{1}{2}}, \lambda_{+\frac{3}{2}} } ({\hat \theta},{\hat \varphi})\right]
\label{2009-gtq-gpe4}
\end{equation}
can be demonstrated by enumerating special cases; e.g.,
\begin{equation}
\begin{array}{rcl}
E_{{ \Psi_{4,2,1}}\,-1,-1, +1, +1 } ( \theta ,0,0,0 )
&=& \frac{1}{8} \left[-7 \cos \theta -\cos (3 \theta )\right]
,
\\
E_{{ \Psi_{4,2,1}}\,-1,+1, +1, -1 } ( \theta ,0,0,0 )
&=& \frac{1}{4} \left[3 \cos (2 \theta )+1\right]
,
\\
E_{{ \Psi_{4,2,1}}\,+1,-1, +1, -1 } ( \theta ,0,0,0 )
&=& \frac{1}{2} \left[-\cos \theta -\cos (3 \theta )\right]
.
\\
\end{array}
\label{2009-gtq-e4-plast}
\end{equation}
These  expectation functions are drawn in Fig.~\ref{2009-gtq-gr4},
together with the spin state expectation function
$\frac{4}{5}E_{{ \Psi_{4,2,1}}\,-\frac{3}{2},-\frac{1}{2}, +\frac{1}{2}, +\frac{3}{2} }
( \theta ,0,0,0 ) = -\cos \theta $
and the classical linear expectation function
$E_{\text{cl},2,2}(\theta ) = {2 \theta / \pi} - 1$
in Eq.~(\ref{2009-gtq-eclass}).
\begin{figure}[htbp]
  \centering
\includegraphics[width=90mm]{2009-gtq-gr4}
\caption{Plasticity of the expectation function of two spin three-half quanta in a singlet state.
(a) $E_{{ \Psi_{4,2,1}}\,-1,-1, +1, +1 }$ is represented by the long-dashed blue curve,
(b) $E_{{ \Psi_{4,2,1}}\,-1,+1, +1, -1  }$ is represented by the dashed-dotted red curve,
(c) $E_{{ \Psi_{4,2,1}}\,+1,-1, +1, -1  }$ is represented by the short-dashed green curve,
(d) $\frac{4}{5}E_{{ \Psi_{4,2,1}}\,-\frac{3}{2},-\frac{1}{2}, +\frac{1}{2}, +\frac{3}{2} }$ is represented by the dotted orange curve,
and (e) $E_{\text{cl},2,2}(\theta )$ is represented by the classical linear  black line.
}
\label{2009-gtq-gr4}
\end{figure}

``raw''




\section{Summary}

Compared to the two-partite quantum correlations of two-state particles,
the plasticity of the quantum expectations of states of {\em more than  two particles }
originates in the dependency of the {\em multitude of angles} involved, as well as by the {\em multitude of singlet states} in this domain.
For states composed from particles of {\em more than two mutually exclusive outcomes,} the plasticity
is also increased by the {\em different values associated with the outcomes.}

We have explicitly derived the quantum correlation functions of two- and four-partite spin one-half, a well as two-partite systems of higher spin.
All quantum expectation functions of the two-partite spin observables have identical form, all being proportional to
$\cos \theta_1 \cos \theta_2 + \cos (\varphi_1 - \varphi_2) \sin \theta_1 \sin \theta_2$.
We have also argued that, by utilizing the plasticity of the quantum expectation functions for spins higher that one-half,
this well-known correlation function can be ``enhanced'' by defining sums of quantum expectation functions,
at least in some domains of the measurement angles.

It would be interesting to know whether this plasticity of the quantum expectations
$E_{{ \Psi_{l,2,1}}\,\lambda_{-l}, \ldots , \lambda_{+l} }$
for ``very high'' angular momentum $l$ observables
could be pushed to the point of maximal violation of
the Clauser-Horne-Shimony-Holt inequality {\em without} a  bit exchange
such as by using the
``building up '' of a step function from the individual expectation functions~\cite{svozil-krenn}; e.g., for  $0 \leq \theta \leq \pi $,
\begin{equation}
%\begin{array}{rcl}
{\rm sgn} (x)
=\left\{ \begin{array}{rl}
-1 \; &\text{  for } 0\le x<\frac{\pi}{2} \\
 0 \; &\text{  for }  x=\frac{\pi}{2}  \\
+1 \; &\text{  for }  \frac{\pi}{2} <  \theta \le \pi
   \end{array} \right.
={4\over \pi }\sum_{n=0}^\infty {(-1)^n\cos \left[(2n+1)\left( \theta +\frac{\pi}{2}\right)\right]\over
2n+1}\quad .
%\end{array}
\label{e:10}
\end{equation}
Any such violation of Boole-Bell type ``conditions of possible experience'' beyond the maximal quantum
violations, as for instance
derived by Tsirelson~\cite{cirelson} and generalized in Ref.~\cite{filipp-svo-04-qpoly-prl} not necessarily generalizes
to the multipartite, non dichotomic cases.
Note also that such a strong or even maximal violation of the Boole-Bell type ``conditions of possible experience'' beyond the maximal quantum
violations
needs
not necessarily violate relativistic causality~\cite{popescu-97,popescu-97b},
or be associated with a ``sharpening'' of the angular dependence of the joint occurrence of certain elementary dichotomic outcomes,
such as  ``$++$,'' ``$+-$,'' ``$-+$'' or ``$--$,'' respectively.

{\bf Acknowledgments:}
The author gratefully acknowledges support from the Centre for Discrete Mathematics and Theoretical Computer Science (CDMTCS) of
The University of Auckland, as well as discussions with Boris Kamenik, G\"unther Krenn and Johann Summhammer in Viennese coffee houses and elsewhere.
Many ideas emerged in those conversations and also during walks through the Vienna Woods and Auckland's Waitakere Ranges.





\appendix





\section{General definition of two particle correlations}

In what follows, spin state measurements along certain directions or angles in spherical coordinates will be considered.
Let us, for the sake of clarity, first specify and make precise what we mean by ``direction of measurement.''
Following, e.g., Ref.~\cite[p.~1, Fig.~1]{RAMACHANDRAN:61}, and Fig.~\ref{f-2009-gtq-f1}, when not specified otherwise,
we consider a particle travelling along the positive $z$-axis; i.e., along $0Z$, which is taken to be horizontal.
The $x$-axis   along $0X$ is also taken to be horizontal.
The remaining  $y$-axis is taken vertically along $0Y$.
The three axes together form a right-handed system of coordinates.
%
%
%
\begin{figure}
\begin{center}
%TeXCAD Picture [1.pic]. Options:
%\grade{\on}
%\emlines{\off}
%\epic{\off}
%\beziermacro{\on}
%\reduce{\on}
%\snapping{\on}
%\pvinsert{% Your \input, \def, etc. here}
%\quality{8.000}
%\graddiff{0.005}
%\snapasp{1}
%\zoom{6.7272}
\unitlength .6mm % = 1.707pt
\allinethickness{0.6pt} %\thicklines %\linethickness{0.4pt}
\ifx\plotpoint\undefined\newsavebox{\plotpoint}\fi % GNUPLOT compatibility
\begin{picture}(120,102)(0,0)
%\emline(0,8)(41,30)
\multiput(0,8)(.1045918367,.056122449){392}{\line(1,0){.1045918367}}
%\end
\put(41,29.5){\line(0,1){72.5}}
%\emline(41,102)(0,80)
\multiput(41,102)(-.1045918367,-.056122449){392}{\line(-1,0){.1045918367}}
%\end
\put(0,80.5){\line(0,-1){72.5}}
{\color{blue}
\put(20,51){\vector(1,0){100}}
\put(20,51){\vector(0,1){34}}
\put(20,51){\vector(3,2){20}}
}
{
%\bezvec{618}[middle](45,51)(50,62)(35,61)
\put(45,59){\color{red}\vector(-2,3){.117}}\color{red}\bezier{618}(45,51)(50,62)(35,61)
%\end
%\bezvec{487}[middle](35,61)(28.5,72)(20,71)
\put(28,69){\color{red}\vector(-4,3){.117}}\color{red}\bezier{487}(35,61)(28.5,72)(20,71)
%\end
}
{\color{blue}
\put(20,45){\makebox(0,0)[cc]{$0$}}
\put(106,43){\makebox(0,0)[cc]{$Z$}}
\put(23,84){\makebox(0,0)[cc]{$Y$}}
\put(37,68){\makebox(0,0)[cc]{$X$}}
}
{\color{red}
\put(49,63){\makebox(0,0)[cc]{$\theta$}}
\put(29,75){\makebox(0,0)[cc]{$\varphi$}}
}
\end{picture}
\end{center}
\caption{\label{f-2009-gtq-f1}Coordinate system for measurements of particles travelling along $0Z$}
\end{figure}

The Cartesian $(x  , y , z )$--coordinates can be translated into spherical coordinates
$(r, \theta ,\varphi )$ via
$x = r\sin \theta \cos \varphi$,
$y = r\sin \theta \sin \varphi$,
$z = r\cos \theta $,
whereby  $\theta$ is the polar angle in the $x$--$z$-plane measured
from the $z$-axis, with $0 \le \theta \le \pi$,
and $\varphi $ is  the azimuthal angle in the $x$--$y$-plane, measured
from the $x$-axis with $0 \le \varphi < 2 \pi$. We shall only consider directions taken from the origin $0$,
characterized by the angles
$\theta$ and $\varphi$, assuming a unit radius $r=1$.



Consider two particles or quanta. On each one of the two quanta, certain measurements
(such as the spin state or polarization) of
(dichotomic) observables
$O({ a})$ and
$O({ b})$
along the directions $a$ and $b$, respectively, are performed.
The individual outcomes are
encoded or labeled by the symbols ``$-$'' and  ``$+$,'' or values ``-1'' and ``+1'' are recorded along
the directions ${ a}$ for the first particle, and  ${ b}$ for the second particle, respectively.
(Suppose that the measurement direction ${a}$ at ``Alice's location''
is unknown to an observer ``Bob'' measuring ${ b}$ and {\it vice versa}.)
A two-particle correlation function $E(a,b )$
is defined by averaging over the product of the outcomes $O({ a})_i, O({ b} )_i\in \{-1,1\}$
in the $i$th experiment for a total of $N$ experiments; i.e.,
\begin{equation}
E(a,b )={1\over N}\sum_{i=1}^N O({ a})_i O({ b})_i.
\end{equation}


Quantum mechanically, we shall follow a standard procedure for obtaining the probabilities upon which the expectation functions are based.
We shall start from the angular momentum operators, as for instance defined in Schiff's {\em ``Quantum Mechanics''}~\cite[Chap.~VI, Sec.24]{schiff-55}
in arbitrary directions, given by the spherical angular momentum co-ordinates $\theta$ and $\varphi$, as defined above.
Then, the projection operators corresponding to the eigenstates associated with the different eigenvalues are derived
from the dyadic (tensor) product of the normalized eigenvectors.
In Hilbert space based~\cite{v-neumann-49} quantum logic~\cite{birkhoff-36}, every projector corresponds to
a proposition that the system is in a state corresponding to that observable.
The quantum probabilities associated with these eigenstates are derived from the Born rule, assuming singlet states for the physical reasons discussed above.
These probabilities contribute to the correlation and expectation functions.


% ~~~~~~~~~~~~~~~   2 x 2 classical
% ~~~~~~~~~~~~~~~   2 x 2 classical
% ~~~~~~~~~~~~~~~   2 x 2 classical
% ~~~~~~~~~~~~~~~   2 x 2 classical
% ~~~~~~~~~~~~~~~   2 x 2 classical
% ~~~~~~~~~~~~~~~   2 x 2 classical
% ~~~~~~~~~~~~~~~   2 x 2 classical
% ~~~~~~~~~~~~~~~   2 x 2 classical

\section{Classical two-particle correlations}


For the two-outcome (e.g., spin one-half case of photon polarization) case,
it is quite easy to demonstrate that the {\em classical} expectation function
in the plane perpendicular to the direction connecting the two particles is a {\em linear} function of the azimuthal measurement angle.
Assume uniform  distribution of (opposite but otherwise) identical ``angular momenta'' shared by the two particles and lying on the circumference
of the unit circle in the plane spanned by $0X$ and $0Y$, as depicted in Figs.~\ref{f-2009-gtq-f1} and~\ref{f-2009-gtq-f2}.
%
\begin{figure}
\begin{center}
%
%TeXCAD Picture [2.pic]. Options:
%\grade{\on}
%\emlines{\off}
%\epic{\off}
%\beziermacro{\on}
%\reduce{\on}
%\snapping{\off}
%\quality{8.000}
%\graddiff{0.010}
%\snapasp{1}
%\zoom{5.7082}
\unitlength .6mm % = 1.138pt
\allinethickness{1pt} %\thicklines %\linethickness{0.4pt}
\ifx\plotpoint\undefined\newsavebox{\plotpoint}\fi % GNUPLOT compatibility
%\begin{picture}(220.345,235.75)(0,0)
\begin{picture}(220.345,70)(0,0)
{\color{blue}
\put(30.25,29.75){\circle{61.53}}
%
\put(30.00,68.5){\makebox(0,0)[cc]{$a$}}
\put(30.25,30.25){\line(0,1){30.5}}
\put(-.091,29.825){\line(1,0){61}}
%\dottedline(1.75,235.75)(2,235.25)
%\multiput(1.574,235.574)(.125,-.25){3}{{\rule{.4pt}{.4pt}}}
%\end
\put(18.89,42.78){\makebox(0,0)[cc]{$+$}}
\put(29.44,12.22){\makebox(0,0)[cc]{$-$}}
}
{\color{red}
\put(109.92,29.75){\circle{61.53}}
%
%\emline(110,30)(128.33,54)
\multiput(110,30)(.084082569,.110091743){218}{\line(0,1){.110091743}}
%\end
%\emline(85.59,48.466)(134.056,11.196)
\multiput(85.59,48.466)(.1096526165,-.0843225288){442}{\line(1,0){.1096526165}}
%\end
\put(133.61,62.94){\makebox(0,0)[cc]{$b$}}
\put(110.56,46.67){\makebox(0,0)[cc]{$-$}}
\put(99.44,17.22){\makebox(0,0)[cc]{$+$}}
}
\put(189.58,29.75){\circle{61.53}}
%
\put(165.08,38){\makebox(0,0)[cc]{$+$}}
\put(182.83,13.5){\makebox(0,0)[cc]{{\color{blue}$-$}$\cdot${\color{red}$+$}$=-$}}
\put(207.5,35.5){\makebox(0,0)[cc]{{\color{blue}$+$}$\cdot${\color{red}$-$}$=-$}}
\put(211.58,21.25){\makebox(0,0)[cc]{$+$}}
{\color{blue}
\put(189.58,30.25){\line(0,1){30.5}}
\put(159.33,30){\line(1,0){61}}
\put(189.58,68.5){\makebox(0,0)[cc]{$a$}}
}

%\emline(189.44,30)(207.78,54)
\multiput(189.44,30)(.08412844,.110091743){218}{\color{red}\line(0,1){.110091743}}
%\end
%\emline(165.125,48.642)(213.591,11.371)
\multiput(165.125,48.642)(.1096526165,-.0843225288){442}{\color{red}\line(1,0){.1096526165}}
%\end

{\color{red}
\put(213.28,62.94){\makebox(0,0)[cc]{$b$}}
}
\put(193.00,40.00){\makebox(0,0)[cc]{$\theta$}}
\put(199.00,26.00){\makebox(0,0)[lc]{$\theta$}}
\put(178.00,34.00){\makebox(0,0)[rc]{$\theta$}}
\bezier{44}(189.44,45)(195,46.11)(198.89,42.22)
\bezier{106}(172.209,29.957)(171.946,35.651)(175.538,40.293)
\bezier{94}(204.794,29.782)(204.444,24.526)(201.29,20.672)
\end{picture}
\end{center}
\caption{Planar geometry demonstrating the classical two two-state particles correlation.}
\label{f-2009-gtq-f2}
\end{figure}

By considering the length  $A_+(a,b)$ and $A_-(a,b)$ of the positive and negative contributions to expectation function,
one obtains for
$0\le \theta=\vert a-b\vert \le \pi$,
\begin{equation}
\begin{array}{rcl}
E_{\text{cl},2,2}(\theta ) =E_{\text{cl},2,2}(a,b) &=& \frac{1}{2\pi} \left[A_+(a,b)-A_-(a,b)\right]\\
&& \quad =  \frac{1}{2\pi} \left[2A_+(a,b) -2\pi \right]=
{2\over \pi}\vert a-b\vert - 1 = {2\theta \over \pi} - 1,
\label{2009-gtq-eclass}
\end{array}
\end{equation}
where the subscripts stand for the number of mutually exclusive measurement outcomes per particle, and
for the number of particles, respectively.
Note that $A_+(a,b)+A_-(a,b)=2\pi$.

The exchange of a single bit between particles results in classical correlations of the form~\cite{svozil-2004-brainteaser}
\begin{equation}
E_{\text{cl, 1 bit exchange},2,2}\left(\theta \right)
=
H\left(\theta - {3 \pi\over 4}\right)
 - H\left({\pi \over 4} - \theta \right)  -
  2\left(1 - {2\over \pi}\theta \right)
H\left(\theta - {\pi\over 4}\right) H\left({3 \pi\over 4} - \theta \right)
,
\label{e-2004-brainteaser-2a}
\end{equation}
where $H$ stands for the Heaviside (unit) step function.
The bit exchange ``enhances'' the classical correlation $E_{\text{cl},2,2}(\theta )$ without a bit exchange to the extend
that they violate certain {\em ``conditions of possible experience;''} in particular
 the Clauser-Horne-Shimony-Halt inequalities, maximally~\cite{popescu-97b}.



% ~~~~~~~~~~~~~~~   2 x 2
% ~~~~~~~~~~~~~~~   2 x 2
% ~~~~~~~~~~~~~~~   2 x 2
% ~~~~~~~~~~~~~~~   2 x 2
% ~~~~~~~~~~~~~~~   2 x 2
% ~~~~~~~~~~~~~~~   2 x 2


\section{Quantum  two-particle correlations}

The two spin one-half particle case is one of the standard quantum mechanical exercises, although
it is seldomly computed explicitly.
For the sake of completeness and with the prospect to generalize the results to more particles of higher spin,
this case will be enumerated explicitly.
In what follows, we shall use the following notation:
Let
$
\vert +\rangle
$
denote the pure state corresponding to
$ {\hat {\bf e}}_1 =(0,1)
$,
and
$
\vert -\rangle $ denote the orthogonal pure state
corresponding to
${\hat {\bf e}}_2 =(1,0)
$.
The superscript
``$T$,''
``$\ast$'' and
``$\dagger$'' stand for transposition, complex and hermitian conjugation, respectively.

In finite-dimensional Hilbert space, the matrix representation of projectors $E_{\bf a}$
from normalized vectors ${\bf a}=(a_1,a_2,\ldots ,a_n)^T$ with respect to some basis of $n$-dimensional Hilbert space
is obtained by taking the dyadic product; i.e., by
\begin{equation}
E_{\bf a}= \left[{\bf a},{\bf a}^\dagger\right]=\left[{\bf a},({\bf a}^\ast)^T\right]=
{\bf a}\otimes {\bf a}^\dagger =
\left(
\begin{array}{cccccccccc}
a_1{\bf a}^\dagger \\
a_2{\bf a}^\dagger \\
\ldots  \\
a_n{\bf a}^\dagger
\end{array}
\right)
=
\left(
\begin{array}{cccccccccc}
a_1a_1^\ast & a_1a_2^\ast & \ldots & a_1a_n^\ast \\
a_2a_1^\ast & a_2a_2^\ast & \ldots & a_2a_n^\ast \\
\ldots & \ldots & \ldots & \ldots \\
a_na_1^\ast & a_na_2^\ast & \ldots & a_na_n^\ast
\end{array}
\right)
.
\end{equation}
The tensor or Kronecker product of two vectors ${\bf a}$ and ${\bf b} =(b_1,b_2,\ldots ,b_m)^T$ can be represented by
\begin{equation}
{\bf a} \otimes {\bf b} = (a_1{\bf b},a_2{\bf b},\ldots ,a_n{\bf b})^T = (a_1b_1,a_1b_2,\ldots ,a_nb_m)^T
\end{equation}
The tensor or Kronecker product of some operators
\begin{equation}
A=
\left(
\begin{array}{cccccccccc}
a_{11} & a_{12} & \ldots & a_{1n} \\
a_{21} & a_{22} & \ldots & a_{2n} \\
\ldots & \ldots & \ldots & \ldots \\
a_{n1} & a_{n2} & \ldots & a_{nn}
\end{array}
\right)
\text{ and  }B=
\left(
\begin{array}{cccccccccc}
b_{11} & b_{12} & \ldots & b_{1m} \\
b_{21} & b_{22} & \ldots & b_{2m} \\
\ldots & \ldots & \ldots & \ldots \\
b_{m1} & b_{m2} & \ldots & b_{mm}
\end{array}
\right)
\end{equation}
is represented by an $n\times n$-matrix
\begin{equation}
A\otimes B
=
\left(
\begin{array}{cccccccccc}
a_{11} B& a_{12} B& \ldots & a_{1n}B \\
a_{21} B& a_{22} B& \ldots & a_{2n}B \\
\ldots & \ldots & \ldots & \ldots \\
a_{n1} B& a_{n2} B& \ldots & a_{nn}B
\end{array}
\right)
=
\left(
\begin{array}{cccccccccc}
a_{11} b_{11}& a_{11} b_{12} & \ldots & a_{1n}b_{1m} \\
a_{11} b_{21}& a_{11} b_{22}& \ldots & a_{2n} b_{2m}\\
\ldots & \ldots & \ldots & \ldots \\
a_{nn} b_{m1}& a_{nn} b_{m2}& \ldots & a_{nn} b_{mm}
\end{array}
\right)
.
\end{equation}

\subsection{Observables}

Let us start with the spin one-half angular momentum observables of {\em a single} particle along an arbitrary direction
in spherical co-ordinates $\theta$ and $\varphi$
in units of $\hbar$~\cite{schiff-55}; i.e.,
\begin{equation}
M_x=
\frac{1}{2}
\left(
\begin{array}{cccccccccc}
0&1\\
1&0
\end{array}
\right),
\qquad
M_y=
\frac{1}{2}
\left(
\begin{array}{cccccccccc}
0&-i\\
i&0
\end{array}
\right),
\qquad
M_z=
\frac{1}{2}
\left(
\begin{array}{cccccccccc}
1&0\\
0&-1
\end{array}
\right).
\end{equation}
The angular momentum operator in arbitrary direction $\theta$, $\varphi$ is given by its spectral decomposition
\begin{equation}
\begin{array}{rcl}
S_\frac{1}{2} (\theta ,\varphi) &=&
xM_x
+
yM_y
+
zM_z
=
 M_x  \sin \theta \cos \varphi
+
M_y   \sin \theta \sin \varphi
+
M_z   \cos \theta
\\
&=&   \frac{1}{2}\sigma (\theta ,\varphi)=
{1\over 2}
\left(\begin{array}{rcl}
\cos \theta &  e^{-i \varphi }\sin \theta \\
e^{i \varphi }\sin \theta & - \cos \theta
\end{array}
\right)\\
&=&
-
\frac{1}{2}
\left(
\begin{array}{cc}
 \sin ^2 \frac{\theta }{2} & -\frac{1}{2} e^{-i \varphi } \sin \theta  \\
 -\frac{1}{2} e^{i \varphi } \sin \theta  & \cos ^2\frac{\theta  }{2}
\end{array}
\right)
+
\frac{1}{2}
 \left(
\begin{array}{cc}
 \cos ^2 \frac{\theta }{2} & \frac{1}{2} e^{-i \varphi } \sin \theta  \\
 \frac{1}{2} e^{i \varphi } \sin \theta  & \sin ^2 \frac{\theta }{2}
\end{array}
\right)\\
&=&
-
\frac{1}{2}
\left\{
\frac{1}{2}
\left[
{\Bbb I}_2 - \sigma (\theta ,\varphi)
\right]
\right\}
+
\frac{1}{2}
\left\{
\frac{1}{2}
\left[
{\Bbb I}_2 + \sigma (\theta ,\varphi)
\right]
\right\}
.
\end{array}
\label{e-2009-gtq-s2}
\end{equation}

The  orthonormal eigenstates (eigenvectors)  associated with the eigenvalues $-\frac{1}{2}$ and $+\frac{1}{2}$ of
$S_\frac{1}{2}(\theta , \varphi )$ in Eq.~(\ref{e-2009-gtq-s2})
are
\begin{equation}
\label{e-2009-gtq-s2ev}
\begin{array}{cccc}
\vert -\rangle_{\theta ,\varphi} \equiv {\bf x}_{-\frac{1}{2}}(\theta ,\varphi)&=e^{i\delta_{+}}& \left(-
e^{-\frac{i\varphi}{2}} \sin{\theta \over 2} ,e^{\frac{i\varphi}{2}}  \cos{\theta \over 2}
\right),\\
\vert +\rangle_{\theta ,\varphi} \equiv {\bf x}_{+\frac{1}{2}}(\theta ,\varphi)&=e^{i\delta_{-}}& \left(
e^{-\frac{i\varphi}{2}} \cos{\theta \over 2}, e^{\frac{i\varphi}{2}}\sin{\theta \over 2}
\right) ,
\end{array}
\end{equation}
respectively. $\delta_{+}$ and $\delta_{-}$ are arbitrary phases.
These orthogonal unit vectors correspond to the two orthogonal projectors
\begin{equation}
\label{e-2009-gtq-s2evproj}
F_\mp (\theta ,\varphi ) =
\frac{1}{2}
\left[
{\Bbb I}_2 \mp \sigma (\theta ,\varphi)
\right]
\end{equation}
for the spin down and up states along $\theta $ and $\varphi$, respectively.
By setting all the phases and angles to zero, one obtains the original
orthonormalized basis $\{\vert -\rangle,\vert +\rangle\}$.

In what follows, we shall consider two-partite correlation operators based on the spin observables discussed above.

\begin{enumerate}

\item{Two-partite angular momentum observable}

If we are only interested in spin state measurements with the associated outcomes of spin states in units of $\hbar$,
Eq.~(\ref{2004-gtq-e2F2}) can be rewritten to include all possible cases at once; i.e.,
\begin{equation}
 S_{\frac{1}{2} \frac{1}{2} } ({\hat \theta},{\hat \varphi} ) =
S_{\frac{1}{2} }( \theta_1,\varphi_1 )
\otimes
S_{\frac{1}{2} }( \theta_2,\varphi_2 ).
\label{2004-gtq-e2F2nat}
\end{equation}

\item{General two-partite observables}


The two-particle projectors
$F_{\pm \pm }$ or, by another notation, $F_{\pm_1 \pm_2 }$ to indicate the outcome on the first or the second particle,
corresponding to a two~spin-${1\over 2}$~particle joint measurement
aligned (``$+$'') or antialigned  (``$-$'') along arbitrary directions are
\begin{equation}
 F_{\pm_1 \pm_2 } ({\hat \theta},{\hat \varphi} ) =
{\frac{1}{2}}\left[{\mathbb I}_2 \pm_1 {\bf \sigma}( \theta_1,\varphi_1 )\right]
\otimes
{\frac{1}{2}}\left[{\mathbb I}_2 \pm_2 {\bf \sigma}( \theta_2,\varphi_2 )\right];
\label{2004-gtq-e2F2}
\end{equation}
where ``$\pm_i$,'' $i=1,2$ refers to the outcome on the $i$'th particle,
and the notation ${\hat \theta},{\hat \varphi}$ is used to indicate all angular parameters.

To demonstrate its physical interpretation, let us consider as a concrete example
a spin state measurement on two quanta as depicted in Fig.~\ref{2009-gtq-f3}:
$F_{- +  } ({\hat \theta},{\hat \varphi} )$ stands for the proposition
\begin{quote}
{\em `The spin state of the first particle measured along $\theta_1,\varphi_1$ is ``$-$''
      and
      the spin state of the second particle measured along $\theta_2,\varphi_2$ is ``$+$''~.'
}
\end{quote}

\begin{figure}
\begin{center}
%TeXCAD Picture [1.pic]. Options:
%\grade{\off}
%\emlines{\off}
%\epic{\on}
%\beziermacro{\on}
%\reduce{\on}
%\snapping{\off}
%\quality{2.000}
%\graddiff{0.010}
%\snapasp{1}
%\zoom{9.5137}
\unitlength 1mm % = 2.845pt
\allinethickness{1pt} %\thicklines %\linethickness{0.4pt}
\ifx\plotpoint\undefined\newsavebox{\plotpoint}\fi % GNUPLOT compatibility
\begin{picture}(120,25.01)(0,0)
\put(56,9.086){\line(4,3){8}}
\put(64,9.086){\line(-4,3){8}}
\put(5,5.01){\oval(10,10)[l]}
\put(5,.01){\line(0,1){10}}
\put(2.5,5.01){\makebox(0,0)[cc]{$-$}}
\put(5,20.01){\oval(10,10)[l]}
\put(5,15.01){\line(0,1){10}}
\put(2.5,20.01){\makebox(0,0)[cc]{$+$}}
\put(10,5.01){\framebox(10,15)[cc]{$\theta_1,\varphi_1$}}
\put(115,5.01){\oval(10,10)[r]}
\put(115,.01){\line(0,1){10}}
\put(117.5,5.01){\makebox(0,0)[cc]{$-$}}
\put(115,20.01){\oval(10,10)[r]}
\put(115,15.01){\line(0,1){10}}
\put(117.5,20.01){\makebox(0,0)[cc]{$+$}}
\put(100,5.01){\framebox(10,15)[cc]{$\theta_2,\varphi_2$}}
\put(60.019,11.983){\circle{9.727}}
%\vector[middle]{\line}
\put(65.379,12.088){\line(1,0){33.846}}\put(82.302,12.088){\vector(1,0){.07}}
%\end
%\vector[middle]{\line}
\put(54.658,12.088){\line(-1,0){33.846}}\put(37.735,12.088){\vector(-1,0){.07}}
%\end
\end{picture}
\end{center}
\caption{Simultaneous spin state measurement of
the two-partite state represented in Eq.~(\ref{2009-gtq-s1s21}).
Boxes indicate spin state analyzers such as Stern-Gerlach apparatus
oriented along the directions $\theta_1,\varphi_1 $ and
$\theta_2,\varphi_2 $;
their two output ports are occupied with detectors  associated
with the outcomes
``$+$''
and
``$-$'',
respectively.
\label{2009-gtq-f3}}
\end{figure}




More generally, we will consider two different numbers $\lambda_+$ and $\lambda_-$,
and the generalized single-particle operator
\begin{equation}
R_{\frac{1}{2}} (\theta ,\varphi) =
\lambda_-
\left\{
\frac{1}{2}
\left[
{\Bbb I}_2 - \sigma (\theta ,\varphi)
\right]
\right\}
+
\lambda_+
\left\{
\frac{1}{2}
\left[
{\Bbb I}_2 + \sigma (\theta ,\varphi)
\right]
\right\}
,
\label{e-2009-gtq-s2g}
\end{equation}
as well as the resulting two-particle operator
\begin{equation}
%\begin{array}{rcl}
R_{\frac{1}{2} \frac{1}{2}} ({\hat \theta},{\hat \varphi} ) =
R_{\frac{1}{2}}( \theta_1,\varphi_1 )
\otimes
R_{\frac{1}{2}} ( \theta_2,\varphi_2 )\\
=
\lambda_- \lambda_- F_{--} +
\lambda_- \lambda_+ F_{-+} +
\lambda_+ \lambda_- F_{+-} +
\lambda_+ \lambda_+ F_{++}
.
%\end{array}
\label{2004-gtq-e2F2g}
\end{equation}


\end{enumerate}

\subsection{Singlet state}




In what follows, singlet states $\vert \Psi_{d,n,i} \rangle$ will be labeled by three numbers $d$, $n$ and $i$,
denoting
the number $d$ of outcomes associated with the dimension of Hilbert space per particle,
the number $n$ of participating particles,
and the state count $i$ in an enumeration of all possible singlet states of $n$ particles of spin $j=(d-1)/2$, respectively.
For $n=2$, there is only one singlet state, and $i=1$ is always one.

Consider the {\em singlet} ``Bell'' state of two spin-${1\over 2}$
particles
\begin{equation}
\vert \Psi_{2,2,1} \rangle
=
 {1\over \sqrt{2}}
\bigl(
\vert +- \rangle -
\vert -+ \rangle
\bigr)
.
\label{2009-gtq-s1s21}
\end{equation}

With the identifications
$
\vert +\rangle
\equiv {\hat {\bf e}}_1 =(1,0)
$
and
$
\vert -\rangle \equiv {\hat {\bf e}}_2 =(0,1)
$ as before,
the Bell state has a vector representation as
\begin{equation}
\vert  \Psi_{2,2,1}\rangle
 \equiv
{1\over \sqrt{2}}\left({\hat {\bf e}}_1\otimes {\hat {\bf e}}_2-{\hat {\bf e}}_2\otimes {\hat {\bf e}}_1 \right)
= {1\over \sqrt{2}}\left[ (1,0)\otimes (0,1) - (0,1) \otimes (1,0)\right]
=\left( 0,\frac{1}{\sqrt{2}},- \frac{1}{\sqrt{2}} ,  0 \right).
\label{2005-hp-ep12s1v}
\end{equation}
The density operator $\rho_{\Psi_{2,2,1}}$
is just the projector of the dyadic product of this vector, corresponding to the one-dimensional
linear subspace spanned by  $\vert  \Psi_{2,2,1}\rangle $; i.e.,
\begin{equation}
%\begin{array}{lll}
\rho_{\Psi_{2,2,1}} = \vert  \Psi_{2,2,1}\rangle \langle  \Psi_{2,2,1} \vert
=
\left[ \vert  \Psi_{2,2,1}\rangle ,\vert  \Psi_{2,2,1}\rangle^\dagger \right]
=
\frac{1}{2}
 \left(
\begin{array}{rrrr}
0&0&0&0\\
0&1&-1&0\\
0&-1&1&0\\
0&0&0&0
\end{array}
\right)
.
%\end{array}
\end{equation}



Singlet states are form invariant with respect to arbitrary unitary
transformations in the single-particle Hilbert spaces and thus
also rotationally invariant in configuration space,
in particular under the rotations
$
\vert + \rangle =
e^{ i{\frac{\varphi}{2}} }
\left(
\cos \frac{\theta}{2} \vert +'  \rangle
-
\sin \frac{\theta}{2} \vert -'   \rangle
\right)
$
and
$
\vert - \rangle =
e^{ -i{\frac{\varphi}{2}} }
\left(
\sin \frac{\theta}{2} \vert +'   \rangle
+
\cos \frac{\theta}{2} \vert -'  \rangle
\right)
$
in the spherical coordinates $\theta , \varphi$ defined above
[e.\,g., Ref.~\cite{krenn1}, Eq.~(2), or Ref.~\cite{ba-89}, Eq.~(7--49)].

The Bell singlet state is unique in the sense that the outcome of a spin state measurement
along a particular direction on one particle ``fixes'' also the opposite outcome of a spin state measurement
along {\em the same} direction on its ``partner'' particle: (assuming lossless devices)
whenever a ``plus'' or a ``minus'' is recorded on one side,
a ``minus'' or a ``plus'' is recorded on the other side, and {\it vice versa.}




\subsection{Results}

We now turn to the calculation of quantum predictions.
The joint probability to register the spins of the two particles
in state $\rho_{\Psi_{2,2,1}}$
aligned or antialigned along the directions defined by
($\theta_1$, $\varphi_1 $) and
($\theta_2$, $\varphi_2 $)
can be evaluated by a straightforward calculation of
\begin{equation}
\begin{array}{rcl}
P_{{ \Psi_{2,2,1}}\,\pm_1 \pm_2 } ({\hat \theta},{\hat \varphi} )&=&
{\rm Tr}\left[\rho_{ \Psi_{2,2,1}} \cdot F_{\pm_1 \pm_2 } \left({\hat \theta},{\hat \varphi} \right)\right] \\
&&\qquad
=\frac{1}{4} \left\{ 1-(\pm_1 1)( \pm_2 1) \left[\cos \theta_1 \cos \theta_2 + \sin \theta_1 \sin \theta_2 \cos (\varphi_1-\varphi_2) \right]\right\}
.
\end{array}
\end{equation}
Again, ``$\pm_i$,'' $i=1,2$ refers to the outcome on the $i$'th particle.

Since $P_= + P_{\neq} = 1$ and $E= P_= - P_{\neq}$, the joint probabilities to find the two particles
in an even or in an odd number of
spin-``$-\frac{1}{2}$''-states when measured along
($\theta_1$, $\varphi_1 $) and
($\theta_2$, $\varphi_2 $)
are in terms of the expectation function given by
\begin{equation}
\begin{array}{rcl}
P_= &=& P_{++}+P_{--} =
{1\over2}\left(1 + E  \right)
=\frac{1}{2} \left\{ 1- \left[\cos \theta_1 \cos \theta_2 - \sin \theta_1 \sin \theta_2 \cos (\varphi_1-\varphi_2) \right]\right\}
,
\\
P_{\neq} &=& P_{+-}+P_{-+} =
{1\over2}\left(1 - E \right)
=\frac{1}{2} \left\{ 1+ \left[\cos \theta_1 \cos \theta_2 + \sin \theta_1 \sin \theta_2 \cos (\varphi_1-\varphi_2) \right]\right\}
.
\end{array}
\end{equation}
Finally, the quantum mechanical expectation function is obtained by  the difference $P_= -P_{\neq }$; i.e.,
\begin{equation}
E_{{ \Psi_{2,2,1}}\,-1,+1  }(\theta_1,\theta_2,\varphi_1 , \varphi_2)=
-\left[\cos \theta_1 \cos \theta_2 + \cos (\varphi_1 - \varphi_2) \sin \theta_1 \sin \theta_2\right]
.
\label{2009-gtq-gme22}
\end{equation}
By setting either the azimuthal angle differences equal to zero,
or by assuming measurements in the plane perpendicular to the direction of particle propagation,
i.e., with $\theta_1=\theta_2 =\frac{\pi}{2}$,
one obtains
\begin{equation}
\label{2009-gtq-edosgc}
\begin{array}{rcl}
E_{{ \Psi_{2,2,1}}\,-1,+1  }(\theta_1,\theta_2)&=& -\cos (\theta_1 - \theta_2),\\
E_{{ \Psi_{2,2,1}}\,-1,+1  }(\frac{\pi}{2},\frac{\pi}{2},\varphi_1 , \varphi_2) &=& - \cos (\varphi_1 - \varphi_2).
\end{array}
\end{equation}


The general computation of the quantum expectation function for operator~(\ref{2004-gtq-e2F2g})
yields
\begin{equation}
\begin{array}{rcl}
E_{{ \Psi_{2,2,1}}\,\lambda_1 \lambda_2 } ({\hat \theta},{\hat \varphi} )&=&
{\rm Tr}\left[\rho_{ \Psi_{2,2,1}} \cdot R_{\frac{1}{2}\frac{1}{2} } \left({\hat \theta},{\hat \varphi} \right)\right] =\\
&&\quad  =
\frac{1}{4} \left\{( \lambda_- + \lambda_+ )^2-( \lambda_- - \lambda_+ )^2 \left[\cos
    \theta_1  \cos  \theta_2 +\cos ( \varphi_1 - \varphi_2 ) \sin
    \theta_1  \sin  \theta_2 \right]\right\}.
\end{array}
\end{equation}
The standard two-particle quantum mechanical expectations~(\ref{2009-gtq-gme22}) based on the dichotomic outcomes
``$-1$''
and
``$+1$''
are obtained by setting
$  \lambda_+ = -  \lambda_- =1$.

A more ``natural'' choice of $\lambda_\pm$ would be in terms of the spin state observables~(\ref{2004-gtq-e2F2nat}) in units of $\hbar$;
i.e., $  \lambda_+ = -  \lambda_- =\frac{1}{2}$.
The expectation function of  these observables can be directly calculated {\it via} $S_{\frac{1}{2}}$; i.e.,
\begin{equation}
\begin{array}{rcl}
E_{{ \Psi_{2,2,1}}\,-\frac{1}{2},+ \frac{1}{2} } ({\hat \theta},{\hat \varphi} )&=&
{\rm Tr}\left\{ \rho_{ \Psi_{2,2,1}} \cdot \left[ S_{\frac{1}{2}}(\theta_1,\varphi_1) \otimes S_{\frac{1}{2}}(\theta_2,\varphi_2)\right]\right\} \\
&=&
\frac{1}{4} \left[\cos
    \theta_1  \cos  \theta_2 +\cos ( \varphi_1 - \varphi_2 ) \sin \theta_1  \sin  \theta_2 \right]
= \frac{1}{4}E_{{ \Psi_{2,2,1}}\,-1 ,+1 } ({\hat \theta},{\hat \varphi} )
.
\label{2009-gtq-sso2}
\end{array}
\end{equation}


% ~~~~~~~~~~~~~~~   2 x 3
% ~~~~~~~~~~~~~~~   2 x 3
% ~~~~~~~~~~~~~~~   2 x 3
% ~~~~~~~~~~~~~~~   2 x 3
% ~~~~~~~~~~~~~~~   2 x 3
% ~~~~~~~~~~~~~~~   2 x 3
% ~~~~~~~~~~~~~~~   2 x 3
% ~~~~~~~~~~~~~~~   2 x 3

\section{Three-state particle correlations}

\subsection{Observables}
The single particle  spin one angular momentum observables in units of $\hbar$ are given by~\cite{schiff-55}
\begin{equation}
M_x=
\frac{1}{\sqrt{2}}
\left(
\begin{array}{cccccccccc}
0&1&0\\
1&0&1\\
0&1&0
\end{array}
\right),
\;
M_y=
\frac{1}{\sqrt{2}}
\left(
\begin{array}{cccccccccc}
0&-i&0\\
i&0&-i\\
0&i&0
\end{array}
\right),
\;
M_z=
\left(
\begin{array}{cccccccccc}
1&0&0\\
0&0&0\\
0&0&-1
\end{array}
\right).
\end{equation}


Again, the angular momentum operator in arbitrary direction $\theta$, $\varphi$ is given by its spectral decomposition
\begin{equation}
\begin{array}{rcl}
S_1 (\theta ,\varphi) &=&
xM_x
+
yM_y
+
zM_z
=
 M_x  \sin \theta \cos \varphi
+
M_y   \sin \theta \sin \varphi
+
M_z   \cos \theta
\\
&=&   \left(
\begin{array}{cccc}
\cos \theta & {e^{-i\varphi}\sin \theta \over \sqrt{2}}& 0      \\
{e^{i\varphi}\sin \theta \over \sqrt{2}}& 0
& {e^{-i\varphi}\sin \theta \over \sqrt{2}}      \\
0& {e^{i\varphi}\sin \theta \over \sqrt{2}}& -\cos \theta
\end{array}\right)
= -F_{-}(\theta ,\varphi)+0\cdot F_0(\theta ,\varphi) +F_{+}(\theta ,\varphi),
\end{array}
\label{e-2009-gtq-s3}
\end{equation}
where the orthogonal projectors associated with the eigenstates of $S_1 (\theta ,\varphi)$ are
\begin{equation}
\begin{array}{rcl}
F_{-}(\theta ,\varphi) &=&
\left(
\begin{array}{ccc}
\sin ^4\frac{\theta }{2} & -\frac{e^{-i \varphi } \sin ^2\frac{\theta }{2} \sin \theta }{\sqrt{2}} & \frac{1}{4} e^{-2 i \varphi } \sin ^2\theta  \\
 -\frac{e^{i \varphi } \sin ^2\frac{\theta }{2} \sin \theta }{\sqrt{2}} & \frac{\sin ^2\theta }{2} & -\frac{e^{-i \varphi } \cos^2 \frac{\theta}{2} \sin \theta }{\sqrt{2}} \\
 \frac{1}{4} e^{2 i \varphi } \sin ^2\theta  & -\frac{e^{i \varphi } \cos^2 \frac{\theta}{2} \sin \theta }{ \sqrt{2}} & \cos ^4\frac{\theta }{2}
\end{array}
\right),
   \\
F_{0}(\theta ,\varphi) &=&
 \left(
\begin{array}{ccc}
\frac{\sin ^2\theta }{2} & -\frac{e^{-i \varphi } \cos \theta  \sin \theta }{\sqrt{2}} & -\frac{1}{2} e^{-2 i \varphi } \sin ^2\theta  \\
 -\frac{e^{i \varphi } \cos \theta  \sin \theta }{\sqrt{2}} & \cos ^2\theta  & \frac{e^{-i \varphi } \cos \theta  \sin \theta }{\sqrt{2}} \\
 -\frac{1}{2} e^{2 i \varphi } \sin ^2\theta  & \frac{e^{i \varphi } \cos \theta  \sin \theta }{\sqrt{2}} & \frac{\sin ^2\theta }{2}
\end{array}
\right),
\\
F_{+}(\theta ,\varphi) &=&
 \left(
\begin{array}{ccc}
\cos ^4\frac{\theta }{2} & \frac{e^{-i \varphi } \cos^2 \frac{\theta}{2} \sin \theta }{ \sqrt{2}} & \frac{1}{4} e^{-2 i \varphi } \sin ^2\theta  \\
 \frac{e^{i \varphi } \cos^2 \frac{\theta}{2} \sin \theta }{ \sqrt{2}} & \frac{\sin ^2\theta }{2} & \frac{e^{-i \varphi } \sin ^2\frac{\theta }{2} \sin \theta }{\sqrt{2}} \\
 \frac{1}{4} e^{2 i \varphi } \sin ^2\theta  & \frac{e^{i \varphi } \sin ^2\frac{\theta }{2} \sin \theta }{\sqrt{2}} & \sin ^4\frac{\theta }{2}
\end{array}
\right)
.
\end{array}
\label{e-2009-gtq-s2f}
\end{equation}

The orthonormal eigenstates associated with the eigenvalues $+1$, $0$, $-1$ of
$S_1(\theta , \varphi )$ in Eq.~(\ref{e-2009-gtq-s3})
are
\begin{equation}
\label{l-soksp-ev}
\begin{array}{cccccccc}
\vert +\rangle_{\theta ,\varphi} \equiv {\bf x}_{+1}&=&e^{i\delta_{+1}}& \left(
e^{-i\varphi} \cos^2{\theta \over 2}, {1\over \sqrt{2}}   \sin \theta ,e^{i\varphi}  \sin^2{\theta \over 2}
\right),\\
\vert 0\rangle_{\theta ,\varphi} \equiv {\bf x}_{0}&=&e^{i\delta_0}& \left(
-{1\over \sqrt{2}} e^{-i\varphi} \sin \theta , \cos \theta , {1\over \sqrt{2}} e^{i\varphi}\sin \theta
\right),\\
\vert -\rangle_{\theta ,\varphi} \equiv {\bf x}_{-1}&=&e^{i\delta_{-1}}& \left(
e^{-i\varphi} \sin^2{\theta \over 2}, - {1\over \sqrt{2}}     \sin \theta , e^{i\varphi}\cos^2{\theta \over 2}
\right) ,
\end{array}
\end{equation}
respectively.
For vanishing angles $\theta =\varphi =0$,
$\vert +\rangle = (1,0,0)$,
$\vert 0\rangle = (0,1,0)$, and
$\vert -\rangle = (0,0,1)$.


The generalized one-particle observable with the previous outcomes of spin state measurements ``coded''
into the map
\begin{equation}
\label{2009-gtq-c1}
-1 \mapsto  \lambda_{-} ,\qquad
0 \mapsto   \lambda_{0}   ,\qquad
+1 \mapsto   \lambda_{+}
\end{equation}
can be written as
\begin{equation}
\label{2009-gtq-sso1}
R_1(\theta ,\varphi) = \lambda_{-} F_{-}(\theta ,\varphi) + \lambda_{0} F_0(\theta ,\varphi) +  \lambda_{+} F_{+}(\theta ,\varphi)
.
\end{equation}

We now torn to the construction of two-partite operators.
\begin{enumerate}

\item{Two-partite angular momentum observable}

If one is only interested in spin state measurements with the associated outcomes of spin states in units of $\hbar$,
Eq.~(\ref{e-2009-gtq-s3}) can be used to build up the corresponding two-partite operators; i.e.,
\begin{equation}
 S_{1 1 } ({\hat \theta},{\hat \varphi} ) =
S_{1 }( \theta_1,\varphi_1 )
\otimes
S_{1 }( \theta_2,\varphi_2 ).
\label{2004-gtq-e3F3nat}
\end{equation}


\item{General two-partite observables}

The two-particle joint operator corresponding to $R_1(\theta ,\varphi$ is
\begin{equation}
\label{2009-gtq-tso1}
R_{11}(\hat \theta ,\hat \varphi) = R_1(\theta_1 ,\varphi_1)\otimes R_1(\theta_2 ,\varphi_2)
.
\end{equation}

For the sake of the physical interpretation of this generalized operator~(\ref{2009-gtq-tso1}), let us consider as a concrete example
a spin state measurement on two quanta as depicted in Fig.~\ref{2009-gtq-f4}:
$\lambda_{-} F_{-}(\theta_1 ,\varphi_1)\otimes  \lambda_{+} F_{+}(\theta_2 ,\varphi_2 )$  stands for the proposition
\begin{quote}
{\em `The outcome of the first particle measured along $\theta_1,\varphi_1$ is ``$\lambda_-$''
      and
      the outcome of the second particle measured along $\theta_2,\varphi_2$ is ``$\lambda_+$''~.'
}
\end{quote}

\begin{figure}
\begin{center}
%TeXCAD Picture [2.pic]. Options:
%\grade{\off}
%\emlines{\off}
%\epic{\on}
%\beziermacro{\on}
%\reduce{\on}
%\snapping{\off}
%\quality{2.000}
%\graddiff{0.010}
%\snapasp{1}
%\zoom{9.5137}
\unitlength 1mm % = 2.845pt
\allinethickness{1pt} %\thicklines %\linethickness{0.4pt}
\ifx\plotpoint\undefined\newsavebox{\plotpoint}\fi % GNUPLOT compatibility
\begin{picture}(120.037,34.785)(0,0)
\put(56,14.657){\line(4,3){8}}
\put(64,14.657){\line(-4,3){8}}
\put(5,17.698){\oval(10,10)[l]}
\put(115.037,17.698){\oval(10,10)[r]}
\put(5,5.637){\oval(10,10)[l]}
\put(115.037,5.637){\oval(10,10)[r]}
\put(5,29.785){\oval(10,10)[l]}
\put(115.037,29.785){\oval(10,10)[r]}
\put(5,12.698){\line(0,1){10}}
\put(115.037,12.698){\line(0,1){10}}
\put(5,10.637){\line(0,-1){10}}
\put(115.037,10.637){\line(0,-1){10}}
\put(5,24.785){\line(0,1){10}}
\put(115.037,24.785){\line(0,1){10}}
\put(2.5,17.698){\makebox(0,0)[cc]{{\footnotesize $0$}}}
\put(117.537,17.698){\makebox(0,0)[]{\footnotesize $0$}}
\put(2.5,5.637){\makebox(0,0)[cc]{\footnotesize $-$}}
\put(117.537,5.637){\makebox(0,0)[]{\footnotesize $-$}}
\put(2.5,29.785){\makebox(0,0)[cc]{\footnotesize $+$}}
\put(117.537,29.785){\makebox(0,0)[]{\footnotesize $+$}}
\put(10,10.581){\framebox(10,15)[cc]{\footnotesize $\theta_1,\varphi_1$}}
\put(100,10.581){\framebox(10,15)[cc]{\footnotesize $\theta_2,\varphi_2$}}
\put(60.019,17.554){\circle{9.727}}
%\vector[middle]{\line}
\put(65.379,17.659){\line(1,0){33.846}}\put(82.302,17.659){\vector(1,0){.07}}
%\end
%\vector[middle]{\line}
\put(54.658,17.659){\line(-1,0){33.846}}\put(37.735,17.659){\vector(-1,0){.07}}
%\end
\end{picture}
\end{center}
\caption{Simultaneous measurement of
the two particles with three outcome per particle. Boxes indicate spin state analyzers such as Stern-Gerlach apparatus
oriented along the directions $\theta_1,\varphi_1 $ and
$\theta_2,\varphi_2 $;
their two output ports are occupied with detectors  associated
with the outcomes
``$\lambda_+$,''
``$\lambda_0$''
and
``$\lambda_-$'',
respectively.
\label{2009-gtq-f4}}
\end{figure}

\item{Two-partite Kochen-Specker observables}

For the sake of an operationalization of the 117 contexts contained in their proof,
Kochen and Specker~\cite{kochen1} introduced an observable based on spin one
with degenerate eigenvalues corresponding to
$\lambda_+ = \lambda_- = 1$ and $\lambda_0 = 0$,
or its ``inverted'' form $\lambda_+ = \lambda_- = 0$ and $\lambda_0 = 1$.
The corresponding correlation functions will be discussed below.

\end{enumerate}

\subsection{Singlet state}

Consider the two spin-one particle singlet state
\begin{equation}
\label{2009-gtq-s1}
\vert \Psi_{3,2,1} \rangle  =  \frac{1}{\sqrt{3}}\left(-|00\rangle + |-+\rangle + |+-\rangle \right)
.
\end{equation}
Its vector space representation can be explicitly enumerated by taking the direction $\theta =\varphi =0$ and recalling that
$\vert +\rangle \equiv (1,0,0)$,
$\vert 0\rangle \equiv (0,1,0)$, and
$\vert -\rangle \equiv (0,0,1)$; i.e.,
\begin{equation}
\label{2009-gtq-s1ef}
\vert \Psi_{3,2,1} \rangle  \equiv  \frac{1}{\sqrt{3}}\left(0,0,1,0,-1,0,1,0,0 \right)
.
\end{equation}

\subsection{Results}


\begin{enumerate}

\item{Expectation of general two-partite observables}

The  general computation of the quantum expectation function for operator~(\ref{2009-gtq-tso1})
yields
\begin{equation}
\begin{array}{ll}
&E_{{ \Psi_{3,2,1}}\,\lambda_- \lambda_0 \lambda_+ } ({\hat \theta},{\hat \varphi} )= {\rm Tr}\left[\rho_{ \Psi_{3,2,1}} \cdot R_{1 1} \left({\hat \theta},{\hat \varphi} \right)\right] =\\
& \quad   =  \frac{1}{192} \left\{24 \lambda_0^2 + 40 \lambda_0 \left(\lambda_- + \lambda_+\right) + 22 \left(\lambda_- + \lambda_+\right)^2 -
   32 \left(\lambda_- - \lambda_+\right)^2 \cos \theta_1  \cos \theta_2  +   \right.                                                                                     \\
& \qquad \; + 2 \left(-2 \lambda_0 + \lambda_- + \lambda_+\right)^2 \cos\left(2  \theta_2 \right) \left[\left(3 + \cos\left(2 \left( \varphi_1  -
  \varphi_2 \right)\right)\right) \cos\left(2  \theta_1 \right) + 2 \sin\left( \varphi_1  -  \varphi_2 \right)^2\right] +                                          \\
& \qquad \; +  2 \left(-2 \lambda_0 + \lambda_- + \lambda_+\right)^2 \left[\cos\left(2 \left( \varphi_1  -  \varphi_2 \right)\right) +
     2 \cos\left(2  \theta_1 \right) \sin\left( \varphi_1  -  \varphi_2 \right)^2\right] -                                                                         \\
& \qquad \; -32 \left(\lambda_- - \lambda_+\right)^2 \cos\left( \varphi_1  -  \varphi_2 \right) \sin \theta_1  \sin \theta_2  +    \\
& \qquad \; \left.   + 8 \left(-2 \lambda_0 + \lambda_- + \lambda_+\right)^2 \cos\left( \varphi_1  -  \varphi_2 \right) \sin\left(2  \theta_1 \right) \sin\left(2  \theta_2 \right)\right\}
.
\end{array}
\label{e-2009-gtq-e3gen}
\end{equation}

\item{Expectation of two-partite angular momentum observable}

For the sake of comparison, let us relate the rather lengthy
expectation function in Eq.~(\ref{e-2009-gtq-e3gen})
to the standard  quantum mechanical expectations~(\ref{2009-gtq-gme22})
and~(\ref{2009-gtq-edosgc}) based on the dichotomic outcomes
by either using $ S_{1 1 }$ from Eq.~(\ref{2004-gtq-e3F3nat}),
or by setting $\lambda_0 = 0$, $  \lambda_+ = +1$ and  $\lambda_- =-1$.
With these identifications,
\begin{equation}
E_{{ \Psi_{3,2,1}}\,-1, 0, +1 } ({\hat \theta},{\hat \varphi} )= -\frac{2}{3}
\left[\cos \theta_1 \cos \theta_2 + \cos (\varphi_1 - \varphi_2) \sin \theta_1 \sin \theta_2\right]
= \frac{2}{3}E_{{ \Psi_{2,2,1}}\,-1, +1 } ({\hat \theta},{\hat \varphi} )
\label{2009-gtq-edosgc3}
.
\end{equation}
This expectation function is functionally identical with the spin one-half (two outcomes) expectation functions.

\item{Expectation of two-partite Kochen-Specker observables}

The expectation function resulting from the Kochen-Specker observable corresponding to
$\lambda_+ = \lambda_- = 1$ and $\lambda_0 = 0$ or its inverted form
$\lambda_+ = \lambda_- = 0$ and $\lambda_0 = 1$
is
\begin{equation}
\begin{array}{rcl}
E_{{ \Psi_{3,2,1}}\,+1, 0, +1 } ({\hat \theta},{\hat \varphi} )&=&
\frac{1}{24} \left\{
11
+\cos [2 (\varphi_1-\varphi_2)]
+4 \cos (\varphi_1-\varphi_2) \sin (2  \theta_1 ) \sin (2  \theta_2 ) +  \right.
\\
&&\;
+2 \left[\cos (2  \theta_1 )+\cos (2 \theta_2) \right] \sin ^2(\varphi_1-\varphi_2)+
\\
&&\; \left.
+\cos (2  \theta_1 )   \cos (2  \theta_2 ) \left[ \cos (2 (\varphi_1-\varphi_2))+3\right]\right\},
\\
E_{{ \Psi_{3,2,1}}\, 0, +1, 0 } ({\hat \theta},{\hat \varphi} )&=&
\frac{1}{3} \left[\cos \theta_1  \cos \theta_2)+\cos    (\varphi_1-\varphi_2) \sin  \theta_1  \sin  \theta_2 \right]^2 ,
\\
E_{{ \Psi_{3,2,1}}\,+1, 0, +1 } (\frac{\pi}{2},\frac{\pi}{2},{\hat \varphi} )&=&
\frac{1}{6} \left\{\cos \left[2 (\varphi_1-\varphi_2)\right]+3\right\},
\\
E_{{ \Psi_{3,2,1}}\, 0, +1, 0 } (\frac{\pi}{2},\frac{\pi}{2},{\hat \varphi} )&=&
\frac{1}{3} \cos ^2(\varphi_1-\varphi_2),
\\
E_{{ \Psi_{3,2,1}}\,+1, 0, +1 } ({\hat \theta},0,0 )&=&
\frac{1}{6} \left\{\cos \left[2 ( \theta_1 - \theta_2 )\right]+3\right\},
\\
E_{{ \Psi_{3,2,1}}\, 0, +1, 0 } ({\hat \theta},0,0 )&=&
\frac{1}{3} \cos ^2( \theta_1 - \theta_2 ).
\end{array}
\label{2009-gtq-edosgc3ks}
\end{equation}

\end{enumerate}


By comparing the quantum expectation function
$E_{{ \Psi_{3,2,1}}\,-1, 0, +1 } ({\hat \theta},0,0 )\propto - \cos (\theta_1 - \theta_2)$
of the spin operators in Eq.~(\ref{2009-gtq-edosgc3})
with the quantum expectation function  of the Kochen Specker operators
$E_{{ \Psi_{3,2,1}}\,+1, 0, +1 } ({\hat \theta},0,0 ) \propto
\cos \left[2 ( \theta_1 - \theta_2 )\right]$
of Eq.~(\ref{2009-gtq-edosgc3ks}),
one could, for higher-than one-half angular momentum observables, envision an ``enhancement'' of the quantum expectation function
by adding weighted expectation functions, generated from different labels $\lambda_i$.
Indeed, in the domain $\frac{\pi }{3}< | \theta_1 -\theta_2 | < \frac{\pi }{3}$,
the plasticity of
$E_{{ \Psi_{l,2,1}}\,\lambda_{-1},  \lambda_{0} , \lambda_{+1} }$
can be used to build up ``enhanced'' quantum correlations {\it via}
\begin{equation}
\begin{array}{rcl}
&&\frac{1}{2}\left\{
E_{{ \Psi_{3,2,1}}\,-1, 0, +1 } ({\hat \theta},0,0 )
+
3\left[2 E_{{ \Psi_{3,2,1}}\,+1, 0, +1 } ({\hat \theta},0,0 ) -1\right]\right\}
\\
&&\qquad =  \frac{1}{2}\left[-\cos (\theta_1 -\theta_2 ) + \cos 2 (\theta_1 -\theta_2 )
\right]
\\
&& \qquad \qquad
< -\cos (\theta_1 -\theta_2 ) =  E_{{ \Psi_{2,2,1}}\,-1, +1 } ({\hat \theta},0,0 )
\end{array}
\label{2009-gtq-eqcs3}
\end{equation}


% ~~~~~~~~~~~~~~~   2 x j
% ~~~~~~~~~~~~~~~   2 x j
% ~~~~~~~~~~~~~~~   2 x j
% ~~~~~~~~~~~~~~~   2 x j
% ~~~~~~~~~~~~~~~   2 x j
% ~~~~~~~~~~~~~~~   2 x j

\section{General case of two spin $j$ particles}

We shall next treat the general case of spin expectation values of two particles with arbitrary spin $j$ (see also Ref.~\cite{svozil-krenn}).

\subsection{Observables}

In full generality, the matrix representation of the spin $j$ angular momentum observables in units of $\hbar$ are given by
\begin{equation}
\begin{array}{rcl}
(M_x)_{m,n}&=&
{1\over 2}\sqrt{j(j+1)-m(m-1)}\delta_{m,n+1} + {1\over 2}\sqrt{j(j+1)-m(m+1)}\delta_{m,n-1},\\
(M_y)_{m,n}&=&
{i\over 2}\sqrt{j(j+1)-m(m-1)}\delta_{m,n+1} -{i\over 2}\sqrt{j(j+1)-m(m+1)}\delta_{m,n-1},\\
(M_z)_{m,n}&=&m\delta_{mn},
\end{array}
\end{equation}
where $m,n=-j,-j+1,\ldots ,j-1,j$.

Again, the angular momentum operator in arbitrary direction $\theta$, $\varphi$ in units of $\hbar$ can be written as
\begin{equation}
S_j (\theta ,\varphi) =
xM_x
+
yM_y
+
zM_z
=
 M_x  \sin \theta \cos \varphi
+
M_y   \sin \theta \sin \varphi
+
M_z   \cos \theta
.
\label{e-2009-gtq-sjj}
\end{equation}


If one is interested in spin state measurements with the associated outcomes of spin states in units of $\hbar$,
the associated two-particle operator is given by
\begin{equation}
 S_{j j } ({\hat \theta},{\hat \varphi} ) =
S_{j }( \theta_1,\varphi_1 )
\otimes
S_{j }( \theta_2,\varphi_2 ).
\label{2004-gtq-e3F2natjj}
\end{equation}
The physical interpretation of the operator~(\ref{2004-gtq-e3F2natjj}) is this:
\begin{quote}
{\em `The outcome of the first particle measured along $\theta_1,\varphi_1$ is some $\lambda_{m}$
      and
      the outcome of the second particle measured along $\theta_2,\varphi_2$ is some $\lambda_{m'}$,
      where  $\lambda_{m}, \lambda_{m'}\in \{-j,-j+1,\ldots ,j-1,j$ correspond to one of the $2j+1$ outcomes of a spin state measurement
      along the directions $\theta_1,\varphi_1$ and $\theta_2,\varphi_2$, respectively.'
}
\end{quote}


\subsection{Singlet state}

The singlet state of two spin-$j$ observables
can again be found by the general methods developed in Ref.~\cite{schimpf-svozil}.
A singlet state composed from just two particles can only be a ``zigzag'' state~\cite{schimpf-svozil}
of the form
\begin{equation}
\begin{array}{rcl}
\left|  \left. \Psi_{2,2j+1,1} \right\rangle  \right.
&=&
\sum_{m=-j}^{j}
\langle j+mj-m\vert 00\rangle
\;
\left| \left. +m ,-m \right\rangle \right. \\
&\equiv &
\sum_{m=1}^{2j+1}
\frac{(-1)^{1+j-m}}{\sqrt{2j+1}}
{\bf e}_m \otimes {\bf e}_{2(j+1)-m}
,
\end{array}
\label{e-2009-gtq-zz}
\end{equation}
where Eq.~(\ref{2009-gtq-cgordon0}) has been used,
and ${\bf e}_m$ is the $m$'th vector of the Cartesian basis in $2j+1$-dimensional
vector space, with $m$'th component $1$ and $0$ otherwise.



\subsection{Results}




With these identifications, the expectation functions can be directly calculated {\it via} $S_{jj}$ yielding
\begin{equation}
\begin{array}{rcl}
E_{{ \Psi_{2,2j+1,1}}\,-j,-j+1,\ldots,+j-1, +j } ({\hat \theta},{\hat \varphi} )
&=&
{\rm Tr}\left\{ \rho_{ \Psi_{2,2j+1,1}} \cdot \left[ S_{j}(\theta_1,\varphi_1) \otimes S_{j}(\theta_2,\varphi_2)\right]\right\} \\
&=& -\frac{j(1+j)}{3} \left[\cos \theta_1 \cos \theta_2 + \cos (\varphi_1 - \varphi_2) \sin \theta_1 \sin \theta_2\right]  .
\end{array}
\label{2009-gtq-edosgcjj}
\end{equation}
Thus, the functional form of the two-particle expectation functions based on spin state observables is {\em
independent} of the absolute spin value.




% ~~~~~~~~~~~~~~~   4 x 2
% ~~~~~~~~~~~~~~~   4 x 2
% ~~~~~~~~~~~~~~~   4 x 2
% ~~~~~~~~~~~~~~~   4 x 2
% ~~~~~~~~~~~~~~~   4 x 2
% ~~~~~~~~~~~~~~~   4 x 2


\section{Four spin one-half particle correlations}

\begin{figure}
\begin{center}
\begin{tabular}{ccc}
%TexCad Options
%\grade{\off}
%\emlines{\off}
%\beziermacro{\on}
%\reduce{\on}
%\snapping{\off}
%\quality{2.00}
%\graddiff{0.01}
%\snapasp{1}
%\zoom{1.00}
\unitlength 0.40mm
\allinethickness{1pt} %\linethickness{0.4pt}
\begin{picture}(150.00,150.00)
\put(15.00,2.00){\makebox(0,0)[cc]{$0$}}
\put(45.00,2.00){\makebox(0,0)[cc]{$1$}}
\put(75.00,2.00){\makebox(0,0)[cc]{$2$}}
\put(105.0,2.00){\makebox(0,0)[cc]{$3$}}
\put(135.00,2.00){\makebox(0,0)[cc]{$4$}}
\put(150.00,2.00){\makebox(0,0)[cc]{$N$}}
\put(2.00,45.00){\makebox(0,0)[cc]{${1\over 2}$}}
\put(2.00,75.00){\makebox(0,0)[cc]{$1$}}
\put(2.00,105.00){\makebox(0,0)[cc]{${3\over 2}$}}
\put(2.00,135.00){\makebox(0,0)[cc]{$2$}}
\put(2.00,150.00){\makebox(0,0)[cc]{$j$}}
\put(10.00,10.00){\line(0,1){130.00}}
\put(10.00,10.00){\line(1,0){130.00}}
%\put(15.00,15.00){\line(1,1){125.00}}
%\put(15.00,15.00){\color{blue} \circle*{4.00}}
\put(45.00,45.00){\color{blue} \circle*{4.00}} \put(75.00,15.00){\color{orange} \circle{4.00}}
\put(105.00,45.00){\color{blue} \circle*{4.00}}
\put(135.00,15.00){\color{blue} \circle*{4.00}}
%\put(15.00,15.00){\color{blue} \vector(1,1){28.00}}
\put(75.00,75.00){\color{blue} \vector(1,-1){28.00}}
\put(45.00,45.00){\color{blue} \vector(1,1){28.00}}
\put(105.00,45.00){\color{blue} \vector(1,-1){28.00}}
\put(75.00,75.00){\color{blue} \circle*{4.00}}
%\put(105.00,75.00){\color{orange} \circle{4.00}}
\put(135.00,75.00){\color{orange} \circle{4.00}}
\put(105.00,105.00){\color{orange} \circle{4.00}}
%\put(135.00,105.00){\color{orange} \circle{4.00}}
\put(135.00,135.00){\color{orange} \circle{4.00}}
%\put(75.00,45.00){\color{orange} \circle{4.00}}
%\put(135.00,45.00){\color{orange} \circle{4.00}}
\put(135.00,15.00){\color{red} \circle{8.00}}
\end{picture}
&
$\qquad$
&
%TexCad Options
%\grade{\off}
%\emlines{\off}
%\beziermacro{\on}
%\reduce{\on}
%\snapping{\off}
%\quality{2.00}
%\graddiff{0.01}
%\snapasp{1}
%\zoom{1.00}
\unitlength 0.40mm
\allinethickness{1pt} %\linethickness{0.4pt}
\begin{picture}(150.00,150.00)
\put(15.00,2.00){\makebox(0,0)[cc]{$0$}}
\put(45.00,2.00){\makebox(0,0)[cc]{$1$}}
\put(75.00,2.00){\makebox(0,0)[cc]{$2$}}
\put(105.0,2.00){\makebox(0,0)[cc]{$3$}}
\put(135.00,2.00){\makebox(0,0)[cc]{$4$}}
\put(150.00,2.00){\makebox(0,0)[cc]{$N$}}
\put(2.00,45.00){\makebox(0,0)[cc]{${1\over 2}$}}
\put(2.00,75.00){\makebox(0,0)[cc]{$1$}}
\put(2.00,105.00){\makebox(0,0)[cc]{${3\over 2}$}}
\put(2.00,135.00){\makebox(0,0)[cc]{$2$}}
\put(2.00,150.00){\makebox(0,0)[cc]{$j$}}
\put(10.00,10.00){\line(0,1){130.00}}
\put(10.00,10.00){\line(1,0){130.00}}
%\put(15.00,15.00){\line(1,1){125.00}}
\put(45.00,45.00){\color{blue} \circle*{4.00}}
\put(75.00,15.00){\color{blue} \circle*{4.00}}
\put(105.00,45.00){\color{blue} \circle*{4.00}}
\put(135.00,15.00){\color{blue} \circle*{4.00}}
\put(45.00,45.00){\color{blue} \vector(1,-1){28.00}}
\put(75.00,15.00){\color{blue} \vector(1,1){28.00}}
\put(105.00,45.00){\color{blue} \vector(1,-1){28.00}}
\put(75.00,75.00){\color{orange} \circle{4.00}}
%\put(105.00,75.00){\color{orange} \circle{4.00}}
\put(135.00,75.00){\color{orange} \circle{4.00}}
\put(105.00,105.00){\color{orange} \circle{4.00}}
%\put(135.00,105.00){\color{orange} \circle{4.00}}
\put(135.00,135.00){\color{orange} \circle{4.00}}
%\put(75.00,45.00){\color{orange} \circle{4.00}}
%\put(135.00,45.00){\color{orange} \circle{4.00}}
\put(135.00,15.00){\color{red} \circle{8.00}}
\end{picture}
\\
a)&&b)
\end{tabular}
\end{center}
\caption{Construction of both singlet states a) $\vert \Psi_{2,4,1} \rangle$ of Eq.~(\ref{2005-hp-ep24s2}) and
b)  $\vert \Psi_{2,4,2} \rangle$ of Eq.~(\ref{2004-gtq-s1})
of four
spin-${1\over 2}$ particles. Concentric circles indicate the
target states. The second state is a ``zigzag'' state composed by the product of two two-partite singlet states.
The two-dimensional diagram  represents the ``space'' or ``domain'' of all multi-partite states,
whereby the {\em number of particles} is represented by the abscissa (the $x$-coordinate) along the positive $x$-axis.
The ordinate (the $y$-coordinate) of the state is equal the total angular momentum of the state.
Note that a single point may represent many states; all corresponding to an equal number of particles,
and all having the same total angular momentum.
$N$-partite singlet states can be constructed by starting from the unique state of one particle,
then proceeding {\it via} all ``diagonal'' and, whenever possible for integer spins ,
also ``horizontal'' pathways  consisting of single substeps adding one particle after the other
--- either diagonally from the lower left to the upper right ``{\color{blue}$\nearrow$},''
or diagonally from the upper left to the lower right ``{\color{blue}$\searrow$},''
or, if possible, also horizontally from left to right ``{\color{blue}$\rightarrow$}'' ---
towards the zero momentum state of $N$ particles.
Every diagonal or horizontal substep corresponds to the addition of a single particle.
\label{2005-singlet-f12-e1}
}
\end{figure}

To begin with the analysis of four-partite correlations,  consider four spin-${1\over 2}$
particles in one of the two singlet states generated by the two ``paths''
in the multipartite state space depicted in Fig.~\ref{2005-singlet-f12-e1}
(See also Ref.~\cite{schimpf-svozil})
\begin{eqnarray}
\vert \Psi_{2,4,1} \rangle
&=&
{1\over \sqrt{3}}\Bigl[
\vert ++-- \rangle +
\vert --++ \rangle \nonumber \\
&&\qquad
\qquad
-  {1\over 2}
\bigl(
\vert +- \rangle +
\vert -+ \rangle
\bigr)
\bigl(
\vert +- \rangle +
\vert -+ \rangle
\bigr)
\Bigr],
\label{2005-hp-ep24s2}
\\
\vert \Psi_{2,4,2} \rangle
&=&
\left( \vert \Psi_{2,2,1} \rangle \right)^2
=
{1\over 2}
\bigl(
\vert +- \rangle -
\vert -+ \rangle
\bigr)
\bigl(
\vert +- \rangle -
\vert -+ \rangle
\bigr),
\label{2004-gtq-s1}
\end{eqnarray}
where
$\vert \Psi_{2,2,1} \rangle = \frac{1}{\sqrt{ 2}}
\bigl(
\vert +- \rangle -
\vert -+ \rangle
\bigr)
$
is the two particle singlet ``Bell'' state.
In what follows, we shall concentrate on the first state
$\vert \Psi_{2,4,1} \rangle$, since  $\vert \Psi_{2,4,2} \rangle$
is just the product of two two-partite singlet states,
thus presenting entanglement merely among two pairs of two quanta.

With the identification of
$
\vert +\rangle
\equiv {\hat {\bf e}}_1 =(1,0)
$
and
$
\vert -\rangle \equiv {\hat {\bf e}}_2 =(0,1)
$ as before,
the first singlet state has a vector representation
\begin{equation}
\begin{array}{rcl}
\hat  \Psi_{2,4,1}
&=&
{1\over \sqrt{3}}\Bigl[
{\hat {\bf e}}_1\otimes {\hat {\bf e}}_1\otimes {\hat {\bf e}}_2\otimes {\hat {\bf e}}_2+
{\hat {\bf e}}_2\otimes {\hat {\bf e}}_2\otimes {\hat {\bf e}}_1\otimes {\hat {\bf e}}_1
 \nonumber \\
&&\qquad
-  {1\over \sqrt{2}}\bigl({\hat {\bf e}}_1\otimes {\hat {\bf e}}_2+{\hat {\bf e}}_2\otimes {\hat {\bf e}}_1 \bigr)
\otimes
 {1\over \sqrt{2}}\bigl({\hat {\bf e}}_1\otimes {\hat {\bf e}}_2+{\hat {\bf e}}_2\otimes {\hat {\bf e}}_1 \bigr)
\Bigr]\nonumber \\
&=&\left( 0,0,0,\frac{1}{{\sqrt{3}}},0,
  -\frac{1}{2\,{\sqrt{3}}},-\frac{1}{2\,{\sqrt{3}}},0,
  0,-\frac{1}{2\,{\sqrt{3}}},-\frac{1}{2\,{\sqrt{3}}},
  0,\frac{1}{{\sqrt{3}}},0,0,0\right).
\label{2005-hp-ep24s2v}
\end{array}
\end{equation}
The density operators $\rho_{\Psi_{2,4,1}}$
is just the projector corresponding to the one-dimensional
linear subspaces spanned by
the vectors representing
$ \hat \Psi_{2,4,1}$
in Eq.~(\ref{2005-hp-ep24s2v}); i.e.,
it is the dyadic product
\begin{equation}
%\begin{array}{lll}
\rho_{\Psi_{2,4,1}} = \vert { \Psi}_{2,4,1}\rangle \langle { \Psi}_{2,4,1}\vert =
\left[\vert { \Psi}_{2,4,1}\rangle,\vert { \Psi}_{2,4,1}\rangle^\dagger \right].
%\end{array}
\end{equation}


\subsection{Observables}

In what follows, the operators corresponding to the spin state observables will be enumerated.



The projection operators $F$
corresponding to a four spin one-half particle joint measurement
aligned (``$+$'') or antialigned  (``$-$'') along those angles are
\begin{equation}
\begin{array}{lll}
 F_{\pm \pm \pm \pm} ({\hat \theta},{\hat \varphi} ) =
{\frac{1}{2}}\left[{\mathbb I}_2 \pm {\bf \sigma}( \theta_1,\varphi_1 )\right]
\otimes
{\frac{1}{2}}\left[{\mathbb I}_2 \pm {\bf \sigma}( \theta_2,\varphi_2 )\right] \otimes
\nonumber\\
\qquad\qquad\qquad\qquad\qquad
\otimes
{\frac{1}{2}}\left[{\mathbb I}_2 \pm {\bf \sigma}( \theta_3,\varphi_3 )\right]
\otimes
{\frac{1}{2}}\left[{\mathbb I}_2 \pm {\bf \sigma}( \theta_4,\varphi_4 )\right].
\end{array}
\label{2004-gtq-e2}
\end{equation}

To demonstrate its physical interpretation, let us consider a concrete example: $F_{- + - + } ({\hat \theta},{\hat \varphi} )$ stands for the proposition
\begin{quote}
{\em `The spin state of the first particle measured along $\theta_1,\varphi_1$ is ``$-$'',
      the spin state of the second particle measured along $\theta_2,\varphi_2$ is ``$+$'',
      the spin state of the third particle measured along $\theta_3,\varphi_3$ is ``$-$'',
      and the spin state of the fourth particle measured along $\theta_4,\varphi_4$ is ``$+$''~.'
}
\end{quote}
Fig.~\ref{2005-gtq-f1} depicts a measurement configuration
for a simultaneous measurement of spins along
$\theta_1,\varphi_1 $,
$\theta_2,\varphi_2 $,
$\theta_3,\varphi_3 $ and
$\theta_4,\varphi_4 $
of the state $\Psi_{2,4,1}$.
\begin{figure}[htbp]
\begin{center}

%TexCad Options
%\grade{\off}
%\emlines{\off}
%\beziermacro{\on}
%\reduce{\on}
%\snapping{\off}
%\quality{2.00}
%\graddiff{0.01}
%\snapasp{1}
%\zoom{1.00}
\unitlength 1.0mm
\allinethickness{1pt} %\linethickness{0.4pt}
\begin{picture}(120.00,68.00)
\put(60.00,35.00){\color{black} \circle{10.00}}
\put(56.00,32.00){\line(4,3){8.00}}
\put(64.00,32.00){\line(-4,3){8.00}}
\put(57.00,31.00){\line(-2,-1){32.00}}
\put(5.00,5.00){\oval(10.00,10.00)[l]}
\put(5.00,10.00){\line(0,-1){10.00}}
\put(2.50,5.00){\makebox(0,0)[cc]{$-$}}
\put(57.00,39.00){\line(-2,1){32.00}}
\put(5.00,48.00){\oval(10.00,10.00)[l]}
\put(5.00,43.00){\line(0,1){10.00}}
\put(2.50,48.00){\makebox(0,0)[cc]{$-$}}
\put(63.00,31.00){\line(2,-1){32.00}}
\put(63.00,39.00){\line(2,1){32.00}}
\put(5.00,20.00){\oval(10.00,10.00)[l]}
\put(5.00,25.00){\line(0,-1){10.00}}
\put(2.50,20.00){\makebox(0,0)[cc]{$+$}}
\put(5.00,63.00){\oval(10.00,10.00)[l]}
\put(5.00,58.00){\line(0,1){10.00}}
\put(2.50,63.00){\makebox(0,0)[cc]{$+$}}
\put(10.00,5.00){\framebox(10.00,15.00)[cc]{$\theta_2,\varphi_2$}}
\put(10.00,48.00){\framebox(10.00,15.00)[cc]{$\theta_1,\varphi_1$}}
\put(115.00,5.00){\oval(10.00,10.00)[r]}
\put(115.00,10.00){\line(0,-1){10.00}}
\put(117.50,5.00){\makebox(0,0)[cc]{$-$}}
\put(115.00,48.00){\oval(10.00,10.00)[r]}
\put(115.00,43.00){\line(0,1){10.00}}
\put(117.50,48.00){\makebox(0,0)[cc]{$-$}}
\put(115.00,20.00){\oval(10.00,10.00)[r]}
\put(115.00,25.00){\line(0,-1){10.00}}
\put(117.50,20.00){\makebox(0,0)[cc]{$+$}}
\put(115.00,63.00){\oval(10.00,10.00)[r]}
\put(115.00,58.00){\line(0,1){10.00}}
\put(117.50,63.00){\makebox(0,0)[cc]{$+$}}
\put(100.00,5.00){\framebox(10.00,15.00)[cc]{$\theta_4,\varphi_4$}}
\put(100.00,48.00){\framebox(10.00,15.00)[cc]{$\theta_3,\varphi_3$}}
\end{picture}
\end{center}
\caption{Simultaneous spin measurement of
the four-partite singlet state represented in Eq.~(\ref{2004-gtq-s1}).
Boxes indicate spin state analyzers such as Stern-Gerlach apparatus
oriented along the directions $\theta_1,\varphi_1 $,
$\theta_2,\varphi_2 $,
$\theta_3,\varphi_3 $ and
$\theta_4,\varphi_4 $;
their two output ports are occupied with detectors  associated
with the outcomes
``$+$''
and
``$-$'',
respectively.
\label{2005-gtq-f1}}
\end{figure}

\subsection{Probabilities and expectations}

The joint probability to register the spins of the four particles
in state $\Psi_{2,4,1}$
aligned or antialigned along the directions defined by
($\theta_1$, $\varphi_1 $),
($\theta_2$, $\varphi_2 $),
($\theta_3$, $\varphi_3 $),  and
($\theta_4$, $\varphi_4 $) can be evaluated by a straightforward calculation
of
\begin{equation}
P_{{\Psi_{2,4,1}} \pm 1 ,\pm 1,\pm 1\pm 1} ({\hat \theta},{\hat \varphi} )=
{\rm Tr}\left[\rho_{\Psi_{2,4,1}} \cdot F_{\pm \pm \pm \pm} \left({\hat \theta},{\hat \varphi} \right)\right].
\end{equation}

The expectation functions and joint probabilities to find the four particles
in an even or in an odd number of
spin-``$-$''-states when measured along
($\theta_1$, $\varphi_1 $),
($\theta_2$, $\varphi_2 $),
($\theta_3$, $\varphi_3 $),  and
($\theta_4$, $\varphi_4 $)
obey  $P_{ \rm even} + P_{ \rm odd}=1$,
as well as $E= P_{ \rm even} - P_{ \rm odd}$; hence
$
P_{ \rm even} =
{1\over2}\left[1 + E  \right]
$
and
$
P_{\rm odd} =
{1\over2}\left[1 - E  \right]
$.
Thus, the four particle quantum correlation is given by (cf. Table~\ref{2008-gtq-2part})
\begin{equation}
\begin{array}{rcl}
E_{ \Psi_{2,4,1}-1,+1}({\hat \theta} , {\hat \varphi } )  &=&
\frac{1}{3}
\left\{
\cos \theta_3 \sin \theta_1
\left[
-\cos \theta_4 \cos (\varphi_1 - \varphi_2) \sin \theta_2 +
          2 \cos \theta_2 \cos (\varphi_1 - \varphi_4) \sin \theta_4
\right] +
\right.
\\
&&\qquad
    \sin \theta_1 \sin \theta_3
\left[2 \cos \theta_2 \cos \theta_4 \cos (\varphi_1 - \varphi_3)  +
\right.
\\
&&\qquad
\qquad
\left.
\left(
2 \cos (\varphi_1 + \varphi_2 - \varphi_3 - \varphi_4) +
                \cos (\varphi_1 - \varphi_2)
                \cos (\varphi_3 - \varphi_4)
\right) \sin \theta_2 \sin \theta_4
\right]   +
\\
&&\qquad
    \cos \theta_1
\left[
2 \sin \theta_2
\left(
\cos \theta_4 \cos (\varphi_2 - \varphi_3) \sin \theta_3 +
                \cos \theta_3 \cos (\varphi_2 - \varphi_4) \sin \theta_4
\right) \right.
 +
\\
&&\qquad
\qquad
\left.
\left.
\cos \theta_2
\left(3 \cos \theta_3 \cos \theta_4 -
                \cos (\varphi_3 - \varphi_4) \sin \theta_3
\sin \theta_4
\right)
\right]
\right\} .
\end{array}
\label{2009-gtq-fpqcgen}
\end{equation}
If all the polar angles $\hat \theta$ are set to $\pi /2$,
then this correlation function yields
\begin{equation}
E_{{\Psi_{2,4,1}}-1,+1} ( \frac{\pi}{2},\frac{\pi}{2},\frac{\pi}{2},\frac{\pi}{2},\hat \varphi )=
\frac{1}{3} \left[2 \cos (\varphi_1+\varphi_2- \varphi_3 - \varphi_4)
+\cos (\varphi_1-\varphi_2) \cos (\varphi_3-\varphi_4)
\right]
.
\end{equation}
Likewise, if all the azimuthal angles $\hat \varphi$ are all set to zero, one obtains
\begin{equation}
E_{{\Psi_{2,4,1}}-1,+1} (\hat \theta )=
\frac{1}{3} \left[2 \cos (\theta_1+\theta_2- \theta_3 - \theta_4)
+\cos (\theta_1-\theta_2) \cos (\theta_3-\theta_4)
\right]
.
\label{2009-gtq-E241e}
\end{equation}
The plasticity of the expectation function
$E_{{\Psi_{2,4,1}}-1,+1} (\hat \theta )$ of Eq.~(\ref{2009-gtq-E241e})
for various parameter values $\theta$ is depicted in Fig.~\ref{2009-gtq-E241}.
\begin{figure}[htbp]
  \centering
\includegraphics[width=90mm]{2009-gtq-E241}
\caption{Plasticity of the expectation function of four spin one-half quanta in a singlet state.
(a) $E_{{\Psi_{2,4,1}}-1,+1}(\theta , \frac{\pi}{4}, -\theta , \theta )$ is represented by the long-dashed blue curve,
(b) $E_{{\Psi_{2,4,1}}-1,+1}(\theta , \theta , -\theta , \theta )$ is represented by the dashed-dotted red curve,
(c) $E_{{\Psi_{2,4,1}}-1,+1}(\theta , -\theta , -\theta , \theta )$ is represented by the short-dashed green curve,
(d) $E_{{\Psi_{2,4,1}}-1,+1}(\theta , -\theta , -\theta , 0)$ is represented by the dotted orange curve,
and
(e) $E_{{\Psi_{2,4,1}}-1,+1}(-\theta , -\theta , \frac{\pi}{4}, \theta )$ is represented by the solid magenta line.
}
\label{2009-gtq-E241}
\end{figure}


\begin{table}
\begin{tabular}{c}
\hline\hline
$
P_{ \rm even} =
{1\over2}\left[1 + E  \right]
\; ,\;
P_{\rm odd} =
{1\over2}\left[1 - E  \right]
\; ,\;
E=
P_{{\rm even}}
-
P_{{\rm odd}}
$
\\
\hline
$
\begin{array}{lll}
E_{{\Psi_{2,4,1}}-1,+1} ({\hat \theta} ,{\hat \varphi})  &=&
\frac{1}{3}
\left\{
\cos \theta_3 \sin \theta_1
\left[
-\cos \theta_4 \cos (\varphi_1 - \varphi_2) \sin \theta_2 +
          2 \cos \theta_2 \cos (\varphi_1 - \varphi_4) \sin \theta_4
\right] +
\right.
\\
&&\qquad
    \sin \theta_1 \sin \theta_3
\left[2 \cos \theta_2 \cos \theta_4 \cos (\varphi_1 - \varphi_3)  +
\right.
\\
&&\qquad
\qquad
\left.
\left(
2 \cos (\varphi_1 + \varphi_2 - \varphi_3 - \varphi_4) +
                \cos (\varphi_1 - \varphi_2)
                \cos (\varphi_3 - \varphi_4)
\right) \sin \theta_2 \sin \theta_4
\right]   +
\\
&&\qquad
    \cos \theta_1
\left[
2 \sin \theta_2
\left(
\cos \theta_4 \cos (\varphi_2 - \varphi_3) \sin \theta_3 +
                \cos \theta_3 \cos (\varphi_2 - \varphi_4) \sin \theta_4
\right) \right.
 +
\\
&&\qquad
\qquad
\left.
\left.
\cos \theta_2
\left(3 \cos \theta_3 \cos \theta_4 -
                \cos (\varphi_3 - \varphi_4) \sin \theta_3
\sin \theta_4
\right)
\right]
\right\}
\end{array}
$
\\
$E_{{\Psi_{2,4,1}}-1,+1} ({\hat \theta} )  =
\frac{1}{3} \left[2 \cos (\theta_1 +\theta_2 -\theta_3 -\theta_4 )+\cos
   (\theta_1 -\theta_2 ) \cos (\theta_3 -\theta_4 )\right].
$
\\
$
E_{{\Psi_{2,4,1}}-1,+1} ( \frac{\pi}{2},\frac{\pi}{2},\frac{\pi}{2},\frac{\pi}{2},\hat \varphi )=
\frac{1}{3} \left[2 \cos (\varphi_1+\varphi_2- \varphi_3 - \varphi_4)
+\cos (\varphi_1-\varphi_2) \cos (\varphi_3-\varphi_4)
\right]
$ \\
\hline
$
\begin{array}{lll}
E_{{\Psi_{2,4,2}}-1,+1}({\hat \theta} , {\hat \varphi } )  &=&
\left[\cos \theta_1 \cos \theta_2 +
          \cos ( \varphi_1 - \varphi_2) \sin \theta_1 \sin \theta_2\right]\cdot \\
&&\qquad  \qquad  \left[\cos \theta_3 \cos \theta_4 +
          \cos (\varphi_3 - \varphi_4) \sin \theta_3 \sin \theta_4
\right]
\end{array}
$
\\
$E_{{\Psi_{2,4,2}}-1,+1}({\hat \theta} )  =
\cos (\theta_1 -\theta_2 ) \cos (\theta_3 -\theta_4 ),
$
\\
$E_{{\Psi_{2,4,2}}-1,+1}( \frac{\pi}{2},\frac{\pi}{2},\frac{\pi}{2},\frac{\pi}{2},{\hat \varphi} )  =
\cos (\varphi_1 -\varphi_2 ) \cos (\varphi_3 -\varphi_4 ),
$
\\
\hline\hline
\end{tabular}
\caption{Probabilities and expectation functions
for finding an odd or even number of spin-``$-$''-states for both four-partite singlet states.
Omitted arguments are zero.
\label{2008-gtq-2part}
}
\end{table}


\bibliography{svozil}
%bibliographystyle{osa}
\bibliographystyle{revtex}



\end{document}




(* ~~~~~~~~~~~~~~~~~~~~~~~~~~~~~~~~~~~~~~~~~~~~~~~~~~~~~~~~~~~~~~~~~~~~~~~ *)
(* ~~~~~~~~~~~~~~~~~~Start Mathematica Code~~~~~~~~~~~~~~~~~~~~~~~~~~~~~~~ *)
(* ~~~~~~~~~~~~~~~~~~~~~~~~~~~~~~~~~~~~~~~~~~~~~~~~~~~~~~~~~~~~~~~~~~~~~~~ *)

(*<<Algebra`ReIm`*)

Normalize[z_]:= z/Sqrt[z.Conjugate[z]];

(*Definition of the Tensor Product*)
TensorProduct[a_, b_] :=
  Table[(*a,b are nxn and mxm-matrices*)
   a[[Ceiling[s/Length[b]], Ceiling[t/Length[b]]]]*
    b[[s - Floor[(s - 1)/Length[b]]*Length[b],
      t - Floor[(t - 1)/Length[b]]*Length[b]]], {s, 1,
    Length[a]*Length[b]}, {t, 1, Length[a]*Length[b]}];


(*Definition of the Tensor Product between two vectors*)

TensorProductVec[x_, y_] :=
  Flatten[Table[
    x[[i]] y[[j]], {i, 1, Length[x]}, {j, 1, Length[y]}]];


(*Definition of the Dyadic Product*)

DyadicProductVec[x_] :=
  Table[x[[i]] Conjugate[x[[j]]], {i, 1, Length[x]}, {j, 1,
    Length[x]}];

(*Definition of the sigma matrices*)


vecsig[r_, tt_, p_] :=
 r*{{Cos[tt], Sin[tt] Exp[-I p]}, {Sin[tt] Exp[I p], -Cos[tt]}}

(*Definition of some vectors*)

BellBasis = (1/Sqrt[2]) {{1, 0, 0, 1}, {0, 1, 1, 0}, {0, 1, -1,
     0}, {1, 0, 0, -1}};

Basis = {{1, 0, 0, 0}, {0, 1, 0, 0}, {0, 0, 1, 0}, {0, 0, 0, 1}};



(*~~~~~~~~~~~~~~~~~~~~~~~~~  2  PARTICLES ~~~~~~~~~~~~~~~~~~~~~~~~~~~~~~~~~~~~~~~*)
(*~~~~~~~~~~~~~~~~~~~~~~~~~  2  PARTICLES ~~~~~~~~~~~~~~~~~~~~~~~~~~~~~~~~~~~~~~~*)
(*~~~~~~~~~~~~~~~~~~~~~~~~~  2  PARTICLES ~~~~~~~~~~~~~~~~~~~~~~~~~~~~~~~~~~~~~~~*)



(*~~~~~~~~~~~~~~~~~~~~~~~~~  2  State System ~~~~~~~~~~~~~~~~~~~~~~~~~~~~~~~~~~~~~~~

% ~~~~~~~~~~~~~~~   2 x 2
% ~~~~~~~~~~~~~~~   2 x 2
% ~~~~~~~~~~~~~~~   2 x 2
% ~~~~~~~~~~~~~~~   2 x 2
% ~~~~~~~~~~~~~~~   2 x 2
% ~~~~~~~~~~~~~~~   2 x 2

*)


(*Definition of singlet state*)
vp = {1,0};
vm = {0,1};
psi2s = (1/Sqrt[2])*(TensorProductVec[vp, vm] -
    TensorProductVec[vm, vp])

DyadicProductVec[psi2s]

(*Definition of operators*)

(* Definition of one-particle operator *)

M2X = (1/2) {{0, 1}, {1, 0}};
M2Y = (1/2) {{0, -I}, {I, 0}};
M2Z = (1/2) {{1, 0}, {0, -1}};


Eigenvectors[M2X]
Eigenvectors[M2Y]
Eigenvectors[M2Z]

S2[t_, p_] := FullSimplify[M2X *Sin[t] Cos[p] + M2Y *Sin[t] Sin[p] + M2Z *Cos[t]]

FullSimplify[S2[\[Theta], \[Phi]]] // MatrixForm

FullSimplify[ComplexExpand[S2[Pi/2, 0]]] // MatrixForm
FullSimplify[ComplexExpand[S2[Pi/2, Pi/2]]] // MatrixForm
FullSimplify[ComplexExpand[S2[0, 0]]] // MatrixForm

Assuming[{0 <= \[Theta] <= Pi, 0 <= \[Phi] <= 2 Pi}, FullSimplify[Eigensystem[S2[\[Theta], \[Phi]]], {Element[\[Theta], Reals],
  Element[\[Phi], Reals]}]]



FullSimplify[
 Normalize[
  Eigenvectors[S2[\[Theta], \[Phi]]][[1]]], {Element[\[Theta], Reals],
   Element[\[Phi], Reals]}]

ES2M[\[Theta]_,\[Phi]_] := {-E^(-I \[Phi]) Tan[\[Theta]/2], 1}*Cos[\[Theta]/2]*E^(I \[Phi]/2)
ES2P[\[Theta]_,\[Phi]_] := {E^(-I \[Phi]) Cot[\[Theta]/2], 1}*Sin[\[Theta]/2]*E^(I \[Phi]/2)

FullSimplify[ES2M[\[Theta],\[Phi]] .Conjugate[ES2M [\[Theta],\[Phi]]], {Element[\[Theta], Reals],
  Element[\[Phi], Reals]}]
FullSimplify[ES2P[\[Theta],\[Phi]] .Conjugate[ES2P [\[Theta],\[Phi]]], {Element[\[Theta], Reals],
  Element[\[Phi], Reals]}]
FullSimplify[ES2P[\[Theta],\[Phi]] .Conjugate[ES2M[\[Theta],\[Phi]]], {Element[\[Theta], Reals],
  Element[\[Phi], Reals]}]


ProjectorES2M[\[Theta]_,\[Phi]_] := FullSimplify[DyadicProductVec[ES2M[\[Theta],\[Phi]]], {Element[\[Theta], Reals],
  Element[\[Phi], Reals]}]
ProjectorES2P[\[Theta]_,\[Phi]_] := FullSimplify[DyadicProductVec[ES2P[\[Theta],\[Phi]]], {Element[\[Theta], Reals],
  Element[\[Phi], Reals]}]

 ProjectorES2M[\[Theta],\[Phi]] //MatrixForm
 ProjectorES2P[\[Theta],\[Phi]] //MatrixForm


(* verification of spectral form *)

FullSimplify[(-1/2)ProjectorES2M[\[Theta],\[Phi]] + (1/2)ProjectorES2P[\[Theta],\[Phi]], {Element[\[Theta], Reals],
  Element[\[Phi], Reals]}]


SingleParticleSpinOneHalfeObservable[x_, p_] :=   FullSimplify[(1/2) (IdentityMatrix[2] + vecsig[1, x, p])] ;

SingleParticleSpinOneHalfeObservable[\[Theta], \[Phi]] // MatrixForm

Eigensystem[FullSimplify[SingleParticleSpinOneHalfeObservable[x, p]]]


(*Definition of single operators for occurrence of spin up*)

SingleParticleProjector2first[x_, p_, pm_] :=   TensorProduct[1/2 (IdentityMatrix[2] + pm*vecsig[1, x, p]),  IdentityMatrix[2]]

SingleParticleProjector2second[x_, p_, pm_] :=  TensorProduct[IdentityMatrix[2], 1/2 (IdentityMatrix[2] + pm*vecsig[1, x, p])]



(*Definition of two-particle joint operator for occurrence of spin up \
and down*)

JointProjector2[x1_, x2_, p1_, p2_, pm1_, pm2_] :=  TensorProduct[1/2 (IdentityMatrix[2] + pm1*vecsig[1, x1, p1]),  1/2 (IdentityMatrix[2] + pm2*vecsig[1, x2, p2])]


(*Definition of probabilities*)


(*Probability of concurrence of two equal events for two-particle \
probability in singlet Bell state for occurrence of spin up*)

JointProb2s[x1_, x2_, p1_, p2_, pm1_, pm2_] :=
 FullSimplify[
  Tr[DyadicProductVec[psi2s].JointProjector2[x1, x2, p1, p2, pm1,
     pm2]]]

JointProb2s[x1, x2, p1, p2, pm1, pm2]

JointProb2s[x1, x2, p1, p2, pm1, pm2] // TeXForm

(*sum of joint probabilities add up to one*)

FullSimplify[
 Sum[JointProb2s[x1, x2, p1, p2, pm1, pm2], {pm1, -1, 1, 2}, {pm2, -1,
    1, 2}]]

(*Probability of concurrence of two equal events*)

P2Es[x1_, x2_, p1_, p2_] =
  FullSimplify[
   Sum[UnitStep[pm1*pm2]*
     JointProb2s[x1, x2, p1, p2, pm1, pm2], {pm1, -1, 1, 2}, {pm2, -1,
      1, 2}]];

P2Es[x1, x2, p1, p2]

(*Probability of concurrence of two non-equal events*)

P2NEs[x1_, x2_, p1_, p2_] =
  FullSimplify[
   Sum[UnitStep[-pm1*pm2]*
     JointProb2s[x1, x2, p1, p2, pm1, pm2], {pm1, -1, 1, 2}, {pm2, -1,
      1, 2}]];

P2NEs[x1, x2, p1, p2]

(*Expectation function*)

Expectation2s[x1_, x2_, p1_, p2_] =
 FullSimplify[P2Es[x1, x2, p1, p2] - P2NEs[x1, x2, p1, p2]]


(* ~~~~~~~~~~~~~~~~~~~~~~~ generalization ~~~~~~~~~~~~~~~~~~~~~~~ *)

 OperatorGEN[\[Theta]_,\[Phi]_] = FullSimplify[LM * ProjectorES2M[\[Theta],\[Phi]] + LP * ProjectorES2P[\[Theta],\[Phi]], {Element[\[Theta], Reals],  Element[\[Phi], Reals]}];

 OperatorGEN[\[Theta],\[Phi]] //MatrixForm

 JointProjector2GEN[x1_, x2_, p1_, p2_] :=  TensorProduct[OperatorGEN[x1,p1],OperatorGEN[x2,p2]];

Expectation2sGEN[x1_, x2_, p1_, p2_] := FullSimplify[ Tr[DyadicProductVec[psi2s].JointProjector2GEN[x1, x2, p1, p2]]];

Expectation2sGEN[x1, x2, p1, p2]


(* ~~~~~~~~~~~ natural spin observables ~~~~~~~~~~~~~~~~~~~~~ *)



JointProjector2NAT[x1_, x2_, p1_, p2_] :=  TensorProduct[S2[x1,p1],S2[x2,p2]];

Expectation2sNAT[x1_, x2_, p1_, p2_] := FullSimplify[ Tr[DyadicProductVec[psi2s].JointProjector2NAT[x1, x2, p1, p2]]];

Expectation2sNAT[x1, x2, p1, p2]


(*~~~~~~~~~~~~~~~~~~~~~~~~~  3  State System ~~~~~~~~~~~~~~~~~~~~~~~~~~~~~~~~~~~~~~~

% ~~~~~~~~~~~~~~~   2 x 3
% ~~~~~~~~~~~~~~~   2 x 3
% ~~~~~~~~~~~~~~~   2 x 3
% ~~~~~~~~~~~~~~~   2 x 3
% ~~~~~~~~~~~~~~~   2 x 3
% ~~~~~~~~~~~~~~~   2 x 3
% ~~~~~~~~~~~~~~~   2 x 3
% ~~~~~~~~~~~~~~~   2 x 3

*)



(*Definition of operators*)

(* Definition of one-particle operator *)

M3X = (1/Sqrt[2]) {{0, 1, 0}, {1, 0, 1},{0, 1, 0}};
M3Y = (1/Sqrt[2]) {{0, -I, 0}, {I, 0, -I}, {0, I, 0}};
M3Z =  {{1, 0, 0}, {0, 0, 0},{0, 0, -1}};


Eigenvectors[M3X]
Eigenvectors[M3Y]
Eigenvectors[M3Z]

S3[t_, p_] := M3X *Sin[t] Cos[p] + M3Y *Sin[t] Sin[p] + M3Z *Cos[t]

FullSimplify[S3[\[Theta], \[Phi]]] // MatrixForm

FullSimplify[ComplexExpand[S3[Pi/2, 0]]] // MatrixForm
FullSimplify[ComplexExpand[S3[Pi/2, Pi/2]]] // MatrixForm
FullSimplify[ComplexExpand[S3[0, 0]]] // MatrixForm

Assuming[{0 <= \[Theta] <= Pi, 0 <= \[Phi] <= 2 Pi}, FullSimplify[Eigensystem[S3[\[Theta], \[Phi]]], {Element[\[Theta], Reals],
  Element[\[Phi], Reals]}]]



FullSimplify[
 Normalize[
  Eigenvectors[S3[\[Theta], \[Phi]]][[1]]], {Element[\[Theta], Reals],
   Element[\[Phi], Reals]}]

ES30[\[Theta]_,\[Phi]_] := FullSimplify[
 Normalize[
  Eigenvectors[S3[\[Theta], \[Phi]]][[1]]]*E^(I \[Phi])  , {Element[\[Theta], Reals], Element[\[Phi], Reals]}]

ES30[\[Theta],\[Phi]]


ES3M[\[Theta]_,\[Phi]_] := FullSimplify[
 Normalize[
  Eigenvectors[S3[\[Theta], \[Phi]]][[2]]]*E^(I \[Phi])  , {Element[\[Theta], Reals], Element[\[Phi], Reals]}]

ES3M[\[Theta],\[Phi]]

ES3P[\[Theta]_,\[Phi]_] := FullSimplify[
 Normalize[
  Eigenvectors[S3[\[Theta], \[Phi]]][[3]]]*E^(I \[Phi])  , {Element[\[Theta], Reals], Element[\[Phi], Reals]}]

ES3P[\[Theta],\[Phi]]

FullSimplify[ES3M[\[Theta],\[Phi]] .Conjugate[ES3M [\[Theta],\[Phi]]], {Element[\[Theta], Reals],
  Element[\[Phi], Reals]}]
FullSimplify[ES3P[\[Theta],\[Phi]] .Conjugate[ES3P [\[Theta],\[Phi]]], {Element[\[Theta], Reals],
  Element[\[Phi], Reals]}]
FullSimplify[ES30[\[Theta],\[Phi]] .Conjugate[ES30 [\[Theta],\[Phi]]], {Element[\[Theta], Reals],
  Element[\[Phi], Reals]}]
FullSimplify[ES3P[\[Theta],\[Phi]] .Conjugate[ES3M[\[Theta],\[Phi]]], {Element[\[Theta], Reals],
  Element[\[Phi], Reals]}]
FullSimplify[ES3P[\[Theta],\[Phi]] .Conjugate[ES30[\[Theta],\[Phi]]], {Element[\[Theta], Reals],
  Element[\[Phi], Reals]}]
FullSimplify[ES30[\[Theta],\[Phi]] .Conjugate[ES3M[\[Theta],\[Phi]]], {Element[\[Theta], Reals],
  Element[\[Phi], Reals]}]


ProjectorES30[\[Theta]_,\[Phi]_] := FullSimplify[DyadicProductVec[ES30[\[Theta],\[Phi]]], {Element[\[Theta], Reals],
  Element[\[Phi], Reals]}]
ProjectorES3M[\[Theta]_,\[Phi]_] := FullSimplify[DyadicProductVec[ES3M[\[Theta],\[Phi]]], {Element[\[Theta], Reals],
  Element[\[Phi], Reals]}]
ProjectorES3P[\[Theta]_,\[Phi]_] := FullSimplify[DyadicProductVec[ES3P[\[Theta],\[Phi]]], {Element[\[Theta], Reals],
  Element[\[Phi], Reals]}]

 ProjectorES30[\[Theta],\[Phi]] //MatrixForm
 ProjectorES3M[\[Theta],\[Phi]] //MatrixForm
 ProjectorES3P[\[Theta],\[Phi]] //MatrixForm

ProjectorES30[\[Theta], \[Phi]] // MatrixForm // TeXForm
ProjectorES3M[\[Theta], \[Phi]] // MatrixForm // TeXForm
ProjectorES3P[\[Theta], \[Phi]] // MatrixForm // TeXForm

(* verification of spectral form *)

FullSimplify[0 * ProjectorES30[\[Theta],\[Phi]] +  (-1) * ProjectorES3M[\[Theta],\[Phi]] + (+1) * ProjectorES3P[\[Theta],\[Phi]], {Element[\[Theta], Reals],
  Element[\[Phi], Reals]}] //MatrixForm


(*  ~~~~~~~~~~~~~~~~~~~ general operator ~~~~~~~~~~~~~~~~~~~~~~~  *)

Operator3GEN[\[Theta]_,\[Phi]_] := FullSimplify[LM * ProjectorES3M[\[Theta],\[Phi]] + L0 * ProjectorES30[\[Theta],\[Phi]] + LP * ProjectorES3P[\[Theta],\[Phi]], {Element[\[Theta], Reals], Element[\[Phi], Reals]}];

Operator3GEN[\[Theta],\[Phi]]

JointProjector3GEN[x1_, x2_, p1_, p2_] :=  TensorProduct[Operator3GEN[x1,p1],Operator3GEN[x2,p2]];

v3p = {1,0,0};
v30 = {0,1,0};
v3m = {0,0,1};

psi3s = (1/Sqrt[3])*(-TensorProductVec[v30, v30] + TensorProductVec[v3m, v3p] + TensorProductVec[v3p, v3m])


Expectation3sGEN[x1_, x2_, p1_, p2_] := FullSimplify[ Tr[DyadicProductVec[psi3s].JointProjector3GEN[x1, x2, p1, p2]]];

Expectation3sGEN[x1, x2, p1, p2]


Ex3[LM_,L0_,LP_,x1_,x2_,p1_,p2_]:=FullSimplify[1/192 (24 L0^2 + 40 L0 (LM + LP) + 22 (LM + LP)^2 -
   32 (LM - LP)^2 Cos[x1] Cos[x2] +
   2 (-2 L0 + LM + LP)^2 Cos[
     2 x2] ((3 + Cos[2 (p1 - p2)]) Cos[2 x1] + 2 Sin[p1 - p2]^2) +
   2 (-2 L0 + LM + LP)^2 (Cos[2 (p1 - p2)] +
      2 Cos[2 x1] Sin[p1 - p2]^2) -
   32 (LM - LP)^2 Cos[p1 - p2] Sin[x1] Sin[x2] +
   8 (-2 L0 + LM + LP)^2 Cos[p1 - p2] Sin[2 x1] Sin[2 x2])];

Ex3[-1,0,1,x1,x2,p1,p2]



(* ~~~~~~~~~~~ natural spin observables ~~~~~~~~~~~~~~~~~~~~~ *)



JointProjector3NAT[x1_, x2_, p1_, p2_] :=  TensorProduct[S3[x1,p1],S3[x2,p2]];

Expectation3sNAT[x1_, x2_, p1_, p2_] := FullSimplify[ Tr[DyadicProductVec[psi3s].JointProjector3NAT[x1, x2, p1, p2]]];

Expectation3sNAT[x1, x2, p1, p2]



(* ~~~~~~~~~~~ Kochen-Specker observables ~~~~~~~~~~~~~~~~~~~~~ *)


(*
S3[t_, p_] := M3X *Sin[t] Cos[p] + M3Y *Sin[t] Sin[p] + M3Z *Cos[t]

MM3X[ \[Alpha]_ ] := FullSimplify[S3[Pi/2, \[Alpha]]];
MM3Y[ \[Alpha]_ ] := FullSimplify[S3[Pi/2, \[Alpha]+Pi/2]];
MM3Z[ \[Alpha]_ ] := FullSimplify[S3[0, 0]];

SKS[ \[Alpha]_ ] := FullSimplify[ MM3X[\[Alpha]].MM3X[\[Alpha]] + MM3Y[\[Alpha]].MM3Y[\[Alpha]] + MM3Z[\[Alpha]].MM3Z[\[Alpha]] ];

FullSimplify[SKS[ \[Alpha] ]] // MatrixForm

FullSimplify[ComplexExpand[SKS[ 0]]] // MatrixForm
FullSimplify[ComplexExpand[SKS[ Pi/2]]] // MatrixForm

Assuming[{0 <= \[Theta] <= Pi, 0 <= \[Phi] <= 2 Pi}, FullSimplify[Eigensystem[SKS[ \[Alpha] ]], {Element[\[Alpha], Reals]}]]

*)

Ex3[1, 0, 1, \[Theta]1, \[Theta]2, \[CurlyPhi]1, \[CurlyPhi]2]

Ex3[0, 1, 0, \[Theta]1, \[Theta]2, \[CurlyPhi]1, \[CurlyPhi]2]

Ex3[1, 0, 1, Pi/2, Pi/2, \[CurlyPhi]1, \[CurlyPhi]2]

Ex3[0, 1, 0, Pi/2, Pi/2, \[CurlyPhi]1, \[CurlyPhi]2]

Ex3[1, 0, 1, \[Theta]1, \[Theta]2, 0, 0]

Ex3[0, 1, 0, \[Theta]1, \[Theta]2, 0, 0]








(*~~~~~~~~~~~~~~~~~~~~~~~~~  4  State System ~~~~~~~~~~~~~~~~~~~~~~~~~~~~~~~~~~~~~~~

% ~~~~~~~~~~~~~~~   2 x 4
% ~~~~~~~~~~~~~~~   2 x 4
% ~~~~~~~~~~~~~~~   2 x 4
% ~~~~~~~~~~~~~~~   2 x 4
% ~~~~~~~~~~~~~~~   2 x 4
% ~~~~~~~~~~~~~~~   2 x 4
% ~~~~~~~~~~~~~~~   2 x 4
% ~~~~~~~~~~~~~~~   2 x 4

*)





(*Definition of operators*)

(* Definition of one-particle operator *)

M4X = (1/2) {{0,Sqrt[3],0,0 },{Sqrt[3],0,2,0 },{0,2,0,Sqrt[3] },{0,0,Sqrt[3],0 }};
M4Y = (1/2) {{0,-Sqrt[3]I,0,0 },{Sqrt[3]I,0,-2I,0},{0,2I,0,-Sqrt[3]I},{0,0,Sqrt[3]I,0  } };
M4Z = (1/2) {{3,0,0,0 },{0,1,0,0 },{0,0,-1,0},{0,0,0,-3}};

Eigenvectors[M4X]
Eigenvectors[M4Y]
Eigenvectors[M4Z]

S4[t_, p_] :=  FullSimplify[M4X *Sin[t] Cos[p] + M4Y *Sin[t] Sin[p] + M4Z *Cos[t]];



(*  ~~~~~~~~~~~~~~~~~~~ general operator ~~~~~~~~~~~~~~~~~~~~~~~  *)

LM32 =-3/2;
LM12 =-1/2;
LP32 =3/2;
LP12 =1/2;

ES4M32[\[Theta]_, \[Phi]_] :=   FullSimplify[   Assuming[{0 <= \[Theta] <= Pi, 0 <= \[Phi] <= 2 Pi},    Normalize[     Eigenvectors[S4[\[Theta], \[Phi]]][[1]]]], {Element[\[Theta],     Reals], Element[\[Phi], Reals]}];
ES4P32[\[Theta]_, \[Phi]_] :=  FullSimplify[   Assuming[{0 <= \[Theta] <= Pi, 0 <= \[Phi] <= 2 Pi},    Normalize[     Eigenvectors[S4[\[Theta], \[Phi]]][[2]]]], {Element[\[Theta],     Reals], Element[\[Phi], Reals]}];
ES4M12[\[Theta]_, \[Phi]_] :=  FullSimplify[   Assuming[{0 <= \[Theta] <= Pi, 0 <= \[Phi] <= 2 Pi},    Normalize[     Eigenvectors[S4[\[Theta], \[Phi]]][[3]]]], {Element[\[Theta],     Reals], Element[\[Phi], Reals]}];
ES4P12[\[Theta]_, \[Phi]_] :=  FullSimplify[   Assuming[{0 <= \[Theta] <= Pi, 0 <= \[Phi] <= 2 Pi},    Normalize[     Eigenvectors[S4[\[Theta], \[Phi]]][[4]]]], {Element[\[Theta],     Reals], Element[\[Phi], Reals]}];



JointProjector4GEN[x1_, x2_, p1_, p2_] :=  TensorProduct[S4[x1,p1],S4[x2,p2]];

v4P32 = ES4P32[0,0]
v4P12 = ES4P12[0,0]
v4M12 = ES4M12[0,0]
v4M32 = ES4M32[0,0]



psi4s = (1/2)*(TensorProductVec[v4P32, v4M32]-TensorProductVec[v4M32, v4P32] - TensorProductVec[v4P12 , v4M12 ] + TensorProductVec[v4M12 , v4P12 ])


Expectation4sGEN[x1_, x2_, p1_, p2_] := Tr[DyadicProductVec[psi4s].JointProjector4GEN[x1, x2, p1, p2]];

FullSimplify[Expectation4sGEN[x1, x2, p1, p2]]


(* ~~~~~~~~ general case ~~~~~~~~~ *)

EPPMM1[L4M32_ , L4M12_ , L4P12_ , L4P32_ ,  \[Theta]_, \[Phi]_] :=   Assuming[{0 <= \[Theta] <= Pi, 0 <= \[Phi] <= 2 Pi}, FullSimplify[
L4M32 * Assuming[{0 <= \[Theta] <= Pi, 0 <= \[Phi] <= 2 Pi},
 FullSimplify[
  DyadicProductVec[
   ES4M32[\[Theta], \[Phi]]], {Element[\[Theta], Reals],
   Element[\[Phi], Reals]}] ]   + L4M12 *    Assuming[{0 <= \[Theta] <= Pi, 0 <= \[Phi] <= 2 Pi},
 FullSimplify[
  DyadicProductVec[
   ES4M12[\[Theta], \[Phi]]], {Element[\[Theta], Reals],
   Element[\[Phi], Reals]}] ]+
L4P32 * Assuming[{0 <= \[Theta] <= Pi, 0 <= \[Phi] <= 2 Pi},
 FullSimplify[
  DyadicProductVec[
   ES4P32[\[Theta], \[Phi]]], {Element[\[Theta], Reals],
   Element[\[Phi], Reals]}] ]+
L4P12 * Assuming[{0 <= \[Theta] <= Pi, 0 <= \[Phi] <= 2 Pi},
 FullSimplify[
  DyadicProductVec[
   ES4P12[\[Theta], \[Phi]]], {Element[\[Theta], Reals],
   Element[\[Phi], Reals]}] ]
]]


EPPMM1[-1,-1,1,1,\[Theta], \[Phi]] //MatrixForm

JointProjector4PPMM1[L4M32_ , L4M12_ , L4P12_ , L4P32_ , x1_, x2_, p1_, p2_] :=  Assuming[{0 <= \[Theta] <= Pi, 0 <= \[Phi] <= 2 Pi},
 FullSimplify[TensorProduct[EPPMM1[L4M32 , L4M12 , L4P12 , L4P32 , x1,p1],EPPMM1[L4M32 , L4M12 , L4P12 , L4P32 ,x2,p2]], {Element[\[Theta], Reals],
   Element[\[Phi], Reals]}] ];

Expectation4PPMM1[L4M32_ , L4M12_ , L4P12_ , L4P32_ , x1_, x2_, p1_, p2_] := Tr[DyadicProductVec[psi4s].JointProjector4PPMM1[L4M32 , L4M12 , L4P12 , L4P32 ,x1, x2, p1, p2]];

FullSimplify[Expectation4PPMM1[-1,-1,1,1,x1, x2, p1, p2]]

Emmpp[x1_ ]= FullSimplify[Expectation4PPMM1[-1, -1, 1, 1, x1, 0, 0, 0]];
Emppm[x1_ ]= FullSimplify[Expectation4PPMM1[-1, 1, 1, -1, x1, 0, 0, 0]];
Empmp[x1_ ]= FullSimplify[Expectation4PPMM1[-1, 1, -1, 1, x1, 0, 0, 0]];

Plot[{Emmpp[x1], Emppm[x1], Empmp[x1], -Cos[x1], 2*x1/Pi - 1}, {x1, 0,
   Pi}, PlotStyle -> {{Hue[0.65, 1, 1], Dashing[{0.04, 0.02}],
    Thickness[0.01]}, {Hue[0, 1, 1],
    Dashing[{0.04, 0.02, 0.005, 0.02}], Thickness[0.01]}, {Green,
    Thickness[0.01], Dashing[{0.02, 0.02}]}, {Orange, Thickness[0.01],
     Dashing[{0.01, 0.01}]}, {Black, Thickness[0.007]}},
 PlotRange -> {-1.03, 1.03}, PlotRange -> {-1.03, 1.03},
 Frame -> True,
 FrameLabel -> {Style["\[Theta] [rad]", FontSize -> 26],
   Style["E(\[Theta])", FontSize -> 26]}, AspectRatio -> 1,
 BaseStyle -> {FontFamily -> "Times", FontSize -> 24},
 FrameTicks -> {{{0.0001, "0"}, \[Pi]/4, \[Pi]/2,
    3 \[Pi]/4, \[Pi]}, {-1, -1/2, {0.0001, "0"}, 1/2, 1}, None, None}]

Export["c:/mytex/2009-gtq-gr4.ps", Out[166], "EPS", ImageSize -> 500]

Plot[{Emmpp[x1], Emppm[x1],
  UnitStep[Pi/2 - x1]*Min[-Cos[x1], Empmp[x1 - Pi/4]] +
   UnitStep[-Pi/2 + x1]*Max[-Cos[x1], Empmp[x1 + Pi/4]], -Cos[x1],
  2*x1/Pi - 1}, {x1, 0, Pi},
 PlotStyle -> {{Hue[0.65, 1, 1], Dashing[{0.04, 0.02}],
    Thickness[0.01]}, {Hue[0, 1, 1],
    Dashing[{0.04, 0.02, 0.005, 0.02}], Thickness[0.01]}, {Green,
    Thickness[0.01], Dashing[{0.02, 0.02}]}, {Orange, Thickness[0.01],
     Dashing[{0.01, 0.01}]}, {Black, Thickness[0.007]}},
 PlotRange -> {-1.03, 1.03}, PlotRange -> {-1.03, 1.03},
 Frame -> True,
 FrameLabel -> {Style["\[Theta] [rad]", FontSize -> 26],
   Style["E(\[Theta])", FontSize -> 26]}, AspectRatio -> 1,
 BaseStyle -> {FontFamily -> "Times", FontSize -> 24},
 FrameTicks -> {{{0.0001, "0"}, \[Pi]/4, \[Pi]/2,
    3 \[Pi]/4, \[Pi]}, {-1, -1/2, {0.0001, "0"}, 1/2, 1}, None, None}]

Plot[{UnitStep[Pi/2 - x1]*Min[-Cos[x1], Empmp[x1 - Pi/4]] +
   UnitStep[-Pi/2 + x1]*Max[-Cos[x1], Empmp[x1 + Pi/4]]}, {x1, 0, Pi},
  PlotStyle -> {{Hue[0.65, 1, 1], Dashing[{0.04, 0.02}],
    Thickness[0.01]}, {Hue[0, 1, 1],
    Dashing[{0.04, 0.02, 0.005, 0.02}], Thickness[0.01]}, {Green,
    Thickness[0.01], Dashing[{0.02, 0.02}]}, {Orange, Thickness[0.01],
     Dashing[{0.01, 0.01}]}, {Black, Thickness[0.007]}},
 PlotRange -> {-1.03, 1.03}, PlotRange -> {-1.03, 1.03},
 Frame -> True,
 FrameLabel -> {Style["\[Theta] [rad]", FontSize -> 26],
   Style["E(\[Theta])", FontSize -> 26]}, AspectRatio -> 1,
 BaseStyle -> {FontFamily -> "Times", FontSize -> 24},
 FrameTicks -> {{{0.0001, "0"}, \[Pi]/4, \[Pi]/2,
    3 \[Pi]/4, \[Pi]}, {-1, -1/2, {0.0001, "0"}, 1/2, 1}, None, None}]

(* ~~~~~~~~~~~~~~~~~~~~~~~~~~~~ GENERAL CASE j arbitrary ~~~~~~~~~~~~~~~~~~~~~~~~~~~~~~ *)
(* ~~~~~~~~~~~~~~~~~~~~~~~~~~~~ GENERAL CASE j arbitrary ~~~~~~~~~~~~~~~~~~~~~~~~~~~~~~ *)
(* ~~~~~~~~~~~~~~~~~~~~~~~~~~~~ GENERAL CASE j arbitrary ~~~~~~~~~~~~~~~~~~~~~~~~~~~~~~ *)
(* ~~~~~~~~~~~~~~~~~~~~~~~~~~~~ GENERAL CASE j arbitrary ~~~~~~~~~~~~~~~~~~~~~~~~~~~~~~ *)
(* ~~~~~~~~~~~~~~~~~~~~~~~~~~~~ GENERAL CASE j arbitrary ~~~~~~~~~~~~~~~~~~~~~~~~~~~~~~ *)
(* ~~~~~~~~~~~~~~~~~~~~~~~~~~~~ GENERAL CASE j arbitrary

% ~~~~~~~~~~~~~~~   2 x 2j+1
% ~~~~~~~~~~~~~~~   2 x 2j+1
% ~~~~~~~~~~~~~~~   2 x 2j+1
% ~~~~~~~~~~~~~~~   2 x 2j+1
% ~~~~~~~~~~~~~~~   2 x 2j+1
% ~~~~~~~~~~~~~~~   2 x 2j+1
% ~~~~~~~~~~~~~~~   2 x 2j+1


~~~~~~~~~~~~~~~~~~~~~~~~~~~~~~ *)



M1[j_] = Table[(1/2)(Sqrt[j(j+1)-m(m-1)] KroneckerDelta[m,n+1] + Sqrt[j(j+1)-m(m+1)] KroneckerDelta[m,n-1]), {m,-j,j,1}, {n,-j,j,1}];
M2[j_] = Table[(I/2)(Sqrt[j(j+1)-m(m-1)] KroneckerDelta[m,n+1] - Sqrt[j(j+1)-m(m+1)] KroneckerDelta[m,n-1]), {m,-j,j,1}, {n,-j,j,1}];
M3[j_] = Table[m KroneckerDelta[m,n], {m,-j,j,1}, {n,-j,j,1}];

S[j_,t_, p_] :=  FullSimplify[M1[j] *Sin[t] Cos[p] + M2[j] *Sin[t] Sin[p] + M3[j] *Cos[t]];

JointProjectorS[ j_ , x1_ , x2_ , p1_ , p2_ ] :=  TensorProduct[S[j,x1,p1],S[j,x2,p2]];
(*
M1[1/2] //MatrixForm
M2[1/2] //MatrixForm
M3[1/2] //MatrixForm
M1[1] //MatrixForm
M2[1] //MatrixForm
M3[1] //MatrixForm
M1[3/2] //MatrixForm
M2[3/2] //MatrixForm
M3[3/2] //MatrixForm
M1[2] //MatrixForm
M2[2] //MatrixForm
M3[2] //MatrixForm
*)

(* System of Eigenvectors of spin l particle *)

EigenVSystem[l_] := Table[Table[If[ i === ii ,1,0], {i, 1, l}], {ii, 1, l}];

ClebsG00[j_,m_] =(1/Sqrt[2*j+1])(-1)^(j-m);

(* two particle singlet zigzag state *)

TwoPartiteSingletState[l_] :=  Sum[TensorProductVec[ EigenVSystem[2 l + 1][[m]] ,    EigenVSystem[2 l + 1][[(2 l + 1) - (m - 1)]] ]*   ClebsG00[l, 1 + l - m], {m, 1, 2 l + 1}]


(* two-partite expectation function *)

ExpectationJsGEN[j_ ,x1_ , x2_ , p1_ , p2_ ] := FullSimplify[Tr[DyadicProductVec[TwoPartiteSingletState[j]].JointProjectorS[j, x1, x2, p1, p2]]];

Do[ Print[
  FullSimplify[
   ExpectationJsGEN[j, \[Theta]1, \[Theta]2, \[Phi]1, \[Phi]2]] ], {j,
   1/2, 4, 1}]

Do[ Print[
  FullSimplify[
   ExpectationJsGEN[j, \[Theta]1, \[Theta]2, \[Phi]1, \[Phi]2]] ], {j,
   1, 4, 1}]




(* ~~~~~~~~~~~~~~~~~~~~~~~~~~~~

% ~~~~~~~~~~~~~~~   4 x 2
% ~~~~~~~~~~~~~~~   4 x 2
% ~~~~~~~~~~~~~~~   4 x 2
% ~~~~~~~~~~~~~~~   4 x 2
% ~~~~~~~~~~~~~~~   4 x 2
% ~~~~~~~~~~~~~~~   4 x 2
% ~~~~~~~~~~~~~~~   4 x 2
% ~~~~~~~~~~~~~~~   4 x 2


~~~~~~~~~~~~~~~~~~~~~~~~~~~~~~ *)



(* 3 PARTICLES GHZM *)

(*Definition of three -
    particle joint operators for occurrence of spin up or down*)

JointProjector3[p1_, p2_, p3_, x1_, x2_, x3_, x1pm_, x2pm_, x3pm_] =
    TensorProduct[
      TensorProduct[1/2(IdentityMatrix[2] + x1pm*vecsig[1, x1, p1]),
        1/2(IdentityMatrix[2] + x2pm*vecsig[1, x2, p2])],
      1/2(IdentityMatrix[2] + x3pm*vecsig[1, x3, p3])];

(* GHZM-state
 *)

psiGHZM3 = (1/Sqrt[2]) (TensorProductVec[TensorProductVec[vm, vm], vm] +
        TensorProductVec[TensorProductVec[vp, vp], vp]);

(* probability *)

P4PMGHZM3[p1_, p2_, p3_, x1_, x2_, x3_, x1pm_, x2pm_, x3pm_] =
    Simplify[Tr[
        DyadicProductVec[psiGHZM3].JointProjector3[p1, p2, p3, x1, x2, x3, x1pm,
            x2pm, x3pm]]];

(* probabilities add up to one *)

Simplify[Sum[
    P4PMGHZM3[\[Phi]1, \[Phi]2, \[Phi]3, \[Alpha], \[Beta], \[Gamma], pm1,
      pm2, pm3], {pm1, -1, 1, 2}, {pm2, -1, 1, 2}, {pm3, -1, 1, 2}]]

(* Probability with no selection of third particle by sign *)

FullSimplify[
  Sum[UnitStep[pm1*pm2]*
      P4PMGHZM3[\[Phi]1, \[Phi]2, \[Phi]3, \[Alpha], \[Beta], \[Gamma], pm1,
        pm2, pm3], {pm1, -1, 1, 2}, {pm2, -1, 1, 2}, {pm3, -1, 1, 2}]]

(* Probability selection of third particle by sign *)

FullSimplify[
  Sum[UnitStep[pm1*pm2]*
      P4PMGHZM3[\[Phi]1, \[Phi]2, \[Phi]3, \[Alpha], \[Beta], \[Gamma], pm1,
        pm2, pm], {pm1, -1, 1, 2}, {pm2, -1, 1, 2}]]

(* Expectation with no selection of third particle by sign *)

FullSimplify[
  Sum[pm1*pm2*
      P4PMGHZM3[\[Phi]1, \[Phi]2, \[Phi]3, \[Alpha], \[Beta], \[Gamma], pm1,
        pm2, pm3], {pm1, -1, 1, 2}, {pm2, -1, 1, 2}, {pm3, -1, 1, 2}]]

(* Expectation selection of third particle by sign *)

FullSimplify[
  Sum[pm1*pm2*
      P4PMGHZM3[\[Phi]1, \[Phi]2, \[Phi]3, \[Alpha], \[Beta], \[Gamma], pm1,
        pm2, pm], {pm1, -1, 1, 2}, {pm2, -1, 1, 2}]]


(************************************************************************)

(* 4 PARTICLES *)


(* Definition of operators *)


(*Definition of four -
    particle joint operators for occurrence of spin up or down*)
  JointProjector4[x1_, x2_, x3_, x4_, p1_, p2_, p3_, p4_, x1pm_, x2pm_, x3pm_,
      x4pm_] :=
    TensorProduct[
      TensorProduct[
        TensorProduct[1/2(IdentityMatrix[2] + x1pm*vecsig[1, x1, p1]),
          1/2(IdentityMatrix[2] + x2pm*vecsig[1, x2, p2])],
        1/2(IdentityMatrix[2] + x3pm*vecsig[1, x3, p3])],
      1/2(IdentityMatrix[2] + x4pm*vecsig[1, x4, p4])];


(* Definition of the two four-particle singlet states *)

(* psi4s1 = DyadicProductVec[
          TensorProductVec[
           (1/Sqrt[2]) {0, 1, -1, 0}, (1/Sqrt[2]) {0, 1, -1,0}
                          ]
                ]; *)

psi4s1v = TensorProductVec[(1/Sqrt[2]) (TensorProductVec[vm, vp] -
          TensorProductVec[vp, vm]), (1/Sqrt[2]) (TensorProductVec[vm, vp] -
          TensorProductVec[vp, vm])]

psi4s1 = DyadicProductVec[psi4s1v];


psi4s2v = (1/Sqrt[6]) (
          Sqrt[2]*(
          TensorProductVec[  TensorProductVec[  TensorProductVec[vm,vm],vp],vp]+
          TensorProductVec[  TensorProductVec[  TensorProductVec[vp,vp],vm],vm]
                  )-
          (1/Sqrt[2])*(
          TensorProductVec[  TensorProductVec[  TensorProductVec[vm,vp],vm],vp]+
          TensorProductVec[  TensorProductVec[  TensorProductVec[vm,vp],vp],vm]+
          TensorProductVec[  TensorProductVec[  TensorProductVec[vp,vm],vm],vp]+
          TensorProductVec[  TensorProductVec[  TensorProductVec[vp,vm],vp],vm]
                      )               )

psi4s2 = DyadicProductVec[psi4s2v];


psi4sv[\[Tau]_] = Cos[\[Tau]]* psi4s1v + Sin[\[Tau]]*psi4s2v;

psi4s[\[Tau]_] = FullSimplify[DyadicProductVec[Cos[\[Tau]]*psi4s1v + Sin[\[Tau]]*psi4s2v]];

(* Probabilities of four-particle joint operators  for occurrence of spin up or down
   in one of the singlet states
 *)

P4PM[x1_, x2_, x3_, x4_, p1_, p2_, p3_, p4_, x1pm_, x2pm_, x3pm_, x4pm_] = Simplify[ComplexExpand[
Tr[psi4s1.JointProjector4[x1, x2, x3, x4, p1, p2, p3, p4, x1pm, x2pm, x3pm, x4pm]]]];

P4PMs2[x1_, x2_, x3_, x4_, p1_, p2_, p3_, p4_, x1pm_, x2pm_, x3pm_, x4pm_] = Simplify[ComplexExpand[
Tr[psi4s2.JointProjector4[x1, x2, x3, x4, p1, p2, p3, p4, x1pm, x2pm, x3pm, x4pm]]]];

P4PMs[\[Tau]_,x1_, x2_, x3_, x4_, p1_, p2_, p3_, p4_, x1pm_, x2pm_, x3pm_, x4pm_] = Simplify[ComplexExpand[
Tr[psi4s[\[Tau]].JointProjector4[x1, x2, x3, x4, p1, p2, p3, p4, x1pm, x2pm, x3pm, x4pm]]]];



(* All the probabilities of four-particle joint operators  for occurrence of spin up or down
   in one of the singlet states sum up to 1
 *)

FullSimplify[ComplexExpand[
    Sum[P4PM[\[Alpha], \[Beta], \[Gamma], \[Delta],
             \[CurlyPhi]1,\[CurlyPhi]2,\[CurlyPhi]3,\[CurlyPhi]4, pm1, pm2, pm3,
        pm4], {pm1, -1, 1, 2}, {pm2, -1, 1, 2}, {pm3, -1, 1, 2}, {pm4, -1, 1,2}
       ]
            ]            ]

FullSimplify[ComplexExpand[
  Sum[P4PMs2[\[Alpha], \[Beta], \[Gamma], \[Delta],
             \[CurlyPhi]1,\[CurlyPhi]2,\[CurlyPhi]3,\[CurlyPhi]4, pm1, pm2, pm3,
      pm4], {pm1, -1, 1, 2}, {pm2, -1, 1, 2}, {pm3, -1, 1, 2}, {pm4, -1, 1,2}
       ]
            ]            ]

Simplify[
  Sum[P4PMs[\[Tau],\[Alpha], \[Beta], \[Gamma], \[Delta],
             \[CurlyPhi]1,\[CurlyPhi]2,\[CurlyPhi]3,\[CurlyPhi]4, pm1, pm2, pm3,
      pm4], {pm1, -1, 1, 2}, {pm2, -1, 1, 2}, {pm3, -1, 1, 2}, {pm4, -1, 1,2}
       ]
            ]

(* Probabilities of odd or even number of negative outcomes of
   four-particle joint operators  for occurrence of spin up or down
   in one of the singlet on the left (particles 1&2) and right (particles 3&4) hand side
 *)

SumP4Plus[\[Alpha]_, \[Beta]_, \[Gamma]_, \[Delta]_,\[CurlyPhi]1_,\[CurlyPhi]2_,\[CurlyPhi]3_,\[CurlyPhi]4_] = FullSimplify[ComplexExpand[Sum[
    UnitStep[pm1*pm2*pm3*pm4]*
      P4PM[\[Alpha], \[Beta], \[Gamma], \[Delta],
             \[CurlyPhi]1,\[CurlyPhi]2,\[CurlyPhi]3,\[CurlyPhi]4, pm1, pm2, pm3,
        pm4], {pm1, -1, 1, 2}, {pm2, -1, 1, 2}, {pm3, -1, 1, 2}, {pm4, -1, 1,
      2}]]                ]

SumP4Minus[\[Alpha]_, \[Beta]_, \[Gamma]_, \[Delta]_,\[CurlyPhi]1_,\[CurlyPhi]2_,\[CurlyPhi]3_,\[CurlyPhi]4_] = FullSimplify[ComplexExpand[Sum[
    UnitStep[-pm1*pm2*pm3*pm4]*
      P4PM[\[Alpha], \[Beta], \[Gamma], \[Delta],
             \[CurlyPhi]1,\[CurlyPhi]2,\[CurlyPhi]3,\[CurlyPhi]4, pm1, pm2, pm3,
        pm4], {pm1, -1, 1, 2}, {pm2, -1, 1, 2}, {pm3, -1, 1, 2}, {pm4, -1, 1,
      2}]]              ]

SumP4s2Plus[\[Alpha]_, \[Beta]_, \[Gamma]_, \[Delta]_,\[CurlyPhi]1_,\[CurlyPhi]2_,\[CurlyPhi]3_,\[CurlyPhi]4_] =
  FullSimplify[ComplexExpand[
    Sum[UnitStep[pm1*pm2*pm3*pm4]*
        P4PMs2[\[Alpha], \[Beta], \[Gamma], \[Delta],
             \[CurlyPhi]1,\[CurlyPhi]2,\[CurlyPhi]3,\[CurlyPhi]4, pm1, pm2, pm3,
          pm4], {pm1, -1, 1, 2}, {pm2, -1, 1, 2}, {pm3, -1, 1, 2}, {pm4, -1,
        1, 2}]]          ]

SumP4s2Minus[\[Alpha]_, \[Beta]_, \[Gamma]_, \[Delta]_,\[CurlyPhi]1_,\[CurlyPhi]2_,\[CurlyPhi]3_,\[CurlyPhi]4_] =
  FullSimplify[ComplexExpand[
    Sum[UnitStep[-pm1*pm2*pm3*pm4]*
        P4PMs2[\[Alpha], \[Beta], \[Gamma], \[Delta],
             \[CurlyPhi]1,\[CurlyPhi]2,\[CurlyPhi]3,\[CurlyPhi]4, pm1, pm2, pm3,
          pm4], {pm1, -1, 1, 2}, {pm2, -1, 1, 2}, {pm3, -1, 1, 2}, {pm4, -1,
        1, 2}]]           ]



SumP4sPlus[\[Tau]_,\[Alpha]_, \[Beta]_, \[Gamma]_, \[Delta]_,\[CurlyPhi]1_,\[CurlyPhi]2_,\[CurlyPhi]3_,\[CurlyPhi]4_] =
  FullSimplify[ComplexExpand[
    Sum[UnitStep[pm1*pm2*pm3*pm4]*
        P4PMs[\[Tau] ,\[Alpha], \[Beta], \[Gamma], \[Delta],
             \[CurlyPhi]1,\[CurlyPhi]2,\[CurlyPhi]3,\[CurlyPhi]4, pm1, pm2, pm3,
          pm4], {pm1, -1, 1, 2}, {pm2, -1, 1, 2}, {pm3, -1, 1, 2}, {pm4, -1,
        1, 2}]]         ]

SumP4sMinus[\[Tau]_,\[Alpha]_, \[Beta]_, \[Gamma]_, \[Delta]_,\[CurlyPhi]1_,\[CurlyPhi]2_,\[CurlyPhi]3_,\[CurlyPhi]4_] =
  FullSimplify[ComplexExpand[
    Sum[UnitStep[-pm1*pm2*pm3*pm4]*
        P4PMs[\[Tau] ,\[Alpha], \[Beta], \[Gamma], \[Delta],
             \[CurlyPhi]1,\[CurlyPhi]2,\[CurlyPhi]3,\[CurlyPhi]4, pm1, pm2, pm3,
          pm4], {pm1, -1, 1, 2}, {pm2, -1, 1, 2}, {pm3, -1, 1, 2}, {pm4, -1,
        1, 2}]]          ]

(* The associated expectation values
 *)


P4PPMMGB[\[Alpha]_, \[Beta]_, \[Gamma]_, \[Delta]_,\[CurlyPhi]1_,\[CurlyPhi]2_,\[CurlyPhi]3_,\[CurlyPhi]4_] =
  FullSimplify[ComplexExpand[
    Sum[pm1*pm2*pm3*pm4*
        P4PM[\[Alpha], \[Beta], \[Gamma], \[Delta],
             \[CurlyPhi]1,\[CurlyPhi]2,\[CurlyPhi]3,\[CurlyPhi]4, pm1, pm2, pm3,
          pm4], {pm1, -1, 1, 2}, {pm2, -1, 1, 2}, {pm3, -1, 1, 2}, {pm4, -1,
        1, 2}]]           ]

(* Cos[\[Alpha] - \[Beta]] Cos[\[Gamma] - \[Delta]] *)


P4PPMMGBs2[\[Alpha]_, \[Beta]_, \[Gamma]_, \[Delta]_,\[CurlyPhi]1_,\[CurlyPhi]2_,\[CurlyPhi]3_,\[CurlyPhi]4_] =
  FullSimplify[ComplexExpand[
    Sum[pm1*pm2*pm3*pm4*
        P4PMs2[\[Alpha], \[Beta], \[Gamma], \[Delta],
             \[CurlyPhi]1,\[CurlyPhi]2,\[CurlyPhi]3,\[CurlyPhi]4, pm1, pm2, pm3,
          pm4], {pm1, -1, 1, 2}, {pm2, -1, 1, 2}, {pm3, -1, 1, 2}, {pm4, -1,
        1, 2}]]            ]

P4PPMMGBs[\[Tau]_,\[Alpha]_, \[Beta]_, \[Gamma]_, \[Delta]_,\[CurlyPhi]1_,\[CurlyPhi]2_,\[CurlyPhi]3_,\[CurlyPhi]4_] =
  FullSimplify[ComplexExpand[
    Sum[pm1*pm2*pm3*pm4*
        P4PMs[\[Tau] ,\[Alpha], \[Beta], \[Gamma], \[Delta],
             \[CurlyPhi]1,\[CurlyPhi]2,\[CurlyPhi]3,\[CurlyPhi]4, pm1, pm2, pm3,
          pm4], {pm1, -1, 1, 2}, {pm2, -1, 1, 2}, {pm3, -1, 1, 2}, {pm4, -1,
        1, 2}]]         ]


(********  expectation functions after selection  **********)

P4PPMMGBs2SELPM4[\[Alpha]_, \[Beta]_, \[Gamma]_, \[Delta]_, \[CurlyPhi]1_, \
\[CurlyPhi]2_, \[CurlyPhi]3_, \[CurlyPhi]4_] =
  Simplify[ComplexExpand[
      Sum[pm1*pm2*pm3*
          P4PMs2[\[Alpha], \[Beta], \[Gamma], \[Delta], \[CurlyPhi]1, \
\[CurlyPhi]2, \[CurlyPhi]3, \[CurlyPhi]4, pm1, pm2, pm3, pm4], {pm1, -1, 1,
          2}, {pm2, -1, 1, 2}, {pm3, -1, 1, 2}]]]


(* PLOTS *)

Plot[{SumP4Plus[\[Theta], 0, 0, 0], SumP4Minus[\[Theta], 0, 0, 0],
    SumP4Plus[\[Theta], 0, 0, 0] - SumP4Minus[\[Theta], 0, 0, 0]}, {\[Theta],
    0, Pi},
  PlotStyle -> {{Hue[0.65, 1, 1], Dashing[{0.04, 0.02}],
      Thickness[0.01]}, {Hue[0, 1, 1], Dashing[{0.04, 0.02, 0.005, 0.02}],
      Thickness[0.01]}, {Hue[0.5, 1, 0],
      Thickness[0.01]}, {Hue[0.2]}}, PlotRange -> {-1.03, 1.03},
  PlotRange -> {-1.03, 1.03}, Frame -> True, DefaultFont -> {"Times", 24},
  FrameLabel -> {"\[Theta] [rad]", "E(\[Theta])"}, AspectRatio -> 1,
  FrameTicks -> {{{0.0001, "0"}, \[Pi]/4, \[Pi]/2,
        3 \[Pi]/4, \[Pi]}, {-1, -1/2, {0.0001, "0"}, 1/2, 1}, None, None}]

Export["c:/mytex/2005-gtq-fr1-1.ps",Out[109],"EPS",ImageSize\[Rule]500]


Plot[{SumP4Plus[\[Theta],0,0,\[Pi]],SumP4Minus[\[Theta],0,0,\[Pi]],
    SumP4Plus[\[Theta],0,0,\[Pi]]-SumP4Minus[\[Theta],0,0,\[Pi]]},{\[Theta],0,Pi},
  PlotStyle -> {{Hue[0.65, 1, 1], Dashing[{0.04, 0.02}],
      Thickness[0.01]}, {Hue[0, 1, 1], Dashing[{0.04, 0.02, 0.005, 0.02}],
      Thickness[0.01]}, {Hue[0.5, 1, 0],
      Thickness[0.01]}, {Hue[0.2]}}, PlotRange -> {-1.03, 1.03},
  PlotRange\[Rule]{-1.03,1.03},Frame\[Rule]True,
  DefaultFont\[Rule]{"Times",24},
  FrameLabel\[Rule]{"\[Theta] [rad]","E(\[Theta])"},AspectRatio\[Rule]1,
  FrameTicks\[Rule]{{{0.0001,"0"},\[Pi]/4,\[Pi]/2,
        3 \[Pi]/4,\[Pi]},{-1,-1/2,{0.0001,"0"},1/2,1},None,None}]

Export["c:/mytex/2005-gtq-fr1-2.ps",Out[111],"EPS",ImageSize\[Rule]500]

Plot[{SumP4sPlus[\[Pi]/2, \[Theta], \[Theta], -\[Theta], \[Theta]],
    SumP4sMinus[\[Pi]/2, \[Theta], \[Theta], -\[Theta], \[Theta]],
    SumP4sPlus[\[Pi]/2, \[Theta], \[Theta], -\[Theta], \[Theta]] -
      SumP4sMinus[\[Pi]/
          2, \[Theta], \[Theta], -\[Theta], \[Theta]]}, {\[Theta], 0, Pi},
  PlotStyle -> {{Hue[0.65, 1, 1], Dashing[{0.04, 0.02}],
        Thickness[0.01]}, {Hue[0, 1, 1], Dashing[{0.04, 0.02, 0.005, 0.02}],
        Thickness[0.01]}, {Hue[0.5, 1, 0], Thickness[0.01]}, {Hue[0.2]}},
  PlotRange -> {-1.03, 1.03}, PlotRange -> {-1.03, 1.03}, Frame -> True,
  DefaultFont -> {"Times", 24},
  FrameLabel -> {"\[Theta] [rad]", "E(\[Theta])"}, AspectRatio -> 1,
  FrameTicks -> {{{0.0001, "0"}, \[Pi]/4, \[Pi]/2,
        3 \[Pi]/4, \[Pi]}, {-1, -1/2, {0.0001, "0"}, 1/2, 1}, None, None}]

Export["c:/mytex/2005-gtq-fr1-3.ps",Out[109],"EPS",ImageSize\[Rule]500]

Plot[{SumP4sPlus[\[Pi]/2, \[Theta], Pi/4, -\[Theta], \[Theta]],
    SumP4sMinus[\[Pi]/2, \[Theta], Pi/4, -\[Theta], \[Theta]],
    SumP4sPlus[\[Pi]/2, \[Theta], Pi/4, -\[Theta], \[Theta]] -
      SumP4sMinus[\[Pi]/2, \[Theta], Pi/4, -\[Theta], \[Theta]]}, {\[Theta],
    0, Pi}, PlotStyle -> {{Hue[0.65, 1, 1], Dashing[{0.04, 0.02}],
        Thickness[0.01]}, {Hue[0, 1, 1], Dashing[{0.04, 0.02, 0.005, 0.02}],
        Thickness[0.01]}, {Hue[0.5, 1, 0], Thickness[0.01]}, {Hue[0.2]}},
  PlotRange -> {-1.03, 1.03}, PlotRange -> {-1.03, 1.03}, Frame -> True,
  DefaultFont -> {"Times", 24},
  FrameLabel -> {"\[Theta] [rad]", "E(\[Theta])"}, AspectRatio -> 1,
  FrameTicks -> {{{0.0001, "0"}, \[Pi]/4, \[Pi]/2,
        3 \[Pi]/4, \[Pi]}, {-1, -1/2, {0.0001, "0"}, 1/2, 1}, None, None}]

Export["c:/mytex/2005-gtq-fr1-4.ps",Out[109],"EPS",ImageSize\[Rule]500]

Plot[{SumP4sPlus[\[Pi]/4, \[Theta], Pi/4, -\[Theta], \[Theta]],
    SumP4sMinus[\[Pi]/4, \[Theta], Pi/4, -\[Theta], \[Theta]],
    SumP4sPlus[\[Pi]/4, \[Theta], Pi/4, -\[Theta], \[Theta]] -
      SumP4sMinus[\[Pi]/4, \[Theta], Pi/4, -\[Theta], \[Theta]]}, {\[Theta],
    0, Pi}, PlotStyle -> {{Hue[0.65, 1, 1], Dashing[{0.04, 0.02}],
        Thickness[0.01]}, {Hue[0, 1, 1], Dashing[{0.04, 0.02, 0.005, 0.02}],
        Thickness[0.01]}, {Hue[0.5, 1, 0], Thickness[0.01]}, {Hue[0.2]}},
  PlotRange -> {-1.03, 1.03}, PlotRange -> {-1.03, 1.03}, Frame -> True,
  DefaultFont -> {"Times", 24},
  FrameLabel -> {"\[Theta] [rad]", "E(\[Theta])"}, AspectRatio -> 1,
  FrameTicks -> {{{0.0001, "0"}, \[Pi]/4, \[Pi]/2,
        3 \[Pi]/4, \[Pi]}, {-1, -1/2, {0.0001, "0"}, 1/2, 1}, None, None}]

Export["c:/mytex/2005-gtq-fr1-5.ps",Out[109],"EPS",ImageSize\[Rule]500]


Plot[{SumP4sPlus[\[Pi]/4, \[Theta], 0, -\[Theta], \[Theta]],
    SumP4sMinus[\[Pi]/4, \[Theta], 0, -\[Theta], \[Theta]],
    SumP4sPlus[\[Pi]/4, \[Theta], 0, -\[Theta], \[Theta]] -
      SumP4sMinus[\[Pi]/4, \[Theta], 0, -\[Theta], \[Theta]]}, {\[Theta], 0,
    Pi}, PlotStyle -> {{Hue[0.65, 1, 1], Dashing[{0.04, 0.02}],
        Thickness[0.01]}, {Hue[0, 1, 1], Dashing[{0.04, 0.02, 0.005, 0.02}],
        Thickness[0.01]}, {Hue[0.5, 1, 0], Thickness[0.01]}, {Hue[0.2]}},
  PlotRange -> {-1.03, 1.03}, PlotRange -> {-1.03, 1.03}, Frame -> True,
  DefaultFont -> {"Times", 24},
  FrameLabel -> {"\[Theta] [rad]", "E(\[Theta])"}, AspectRatio -> 1,
  FrameTicks -> {{{0.0001, "0"}, \[Pi]/4, \[Pi]/2,
        3 \[Pi]/4, \[Pi]}, {-1, -1/2, {0.0001, "0"}, 1/2, 1}, None, None}]

Export["c:/mytex/2005-gtq-fr1-6.ps",Out[109],"EPS",ImageSize\[Rule]500]

(*****************************)

(* GHZM states *)

psiGHZM = (1/ Sqrt[2]) (TensorProductVec[
          TensorProductVec[TensorProductVec[vm, vm], vm], vm] +
        TensorProductVec[TensorProductVec[TensorProductVec[vp, vp], vp], vp])

P4PMGHZM[x1_, x2_, x3_, x4_, x1pm_, x2pm_, x3pm_, x4pm_] =
    Simplify[Tr[
        DyadicProductVec[psiGHZM].JointProb4[x1, x2, x3, x4, x1pm, x2pm, x3pm,
             x4pm]]];

Simplify[Sum[
    P4PMGHZM[\[Alpha], \[Beta], \[Gamma], \[Delta], pm1, pm2, pm3,
      pm4], {pm1, -1, 1, 2}, {pm2, -1, 1, 2}, {pm3, -1, 1, 2}, {pm4, -1, 1,
      2}]]

(* Expectation *)
Simplify[Sum[
    pm1*pm2*pm3*pm4*
      P4PMGHZM[\[Alpha], \[Beta], \[Gamma], \[Delta], pm1, pm2, pm3,
        pm4], {pm1, -1, 1, 2}, {pm2, -1, 1, 2}, {pm3, -1, 1, 2}, {pm4, -1, 1,
      2}]]

(* Cos[\[Alpha]] Cos[\[Beta]] Cos[\[Gamma]] Cos[\[Delta]] +
  Sin[\[Alpha]] Sin[\[Beta]] Sin[\[Gamma]] Sin[\[Delta]] *)

(* Expectation for two particles (cf. Krenn & Zeilinger) *)

Simplify[Sum[
    pm1*pm2*P4PMGHZM[\[Alpha], \[Beta], \[Gamma], \[Delta], pm1, pm2, pm3,
        pm4], {pm1, -1, 1, 2}, {pm2, -1, 1, 2}, {pm3, -1, 1, 2}, {pm4, -1, 1,
      2}]]

(*
Export["c:/mytex/2005-gtq-fr1-1.ps", Out[28], "EPS", ImageSize -> 500]

http://mathworld.wolfram.com/HeavisideStepFunction.html
 *)

(* ~~~~~~~~~~~~~~~~~~~~~~~~~ plasticity ~~~~~~~~~~~~~~~~~~~~~~~ *)



sz[ t_ ] := (4/Pi )*
   Sum[ (-1)^n*Cos[(2 n + 1) (2*t + Pi/2)]/(2 n + 1), {n, 0, 18}];

Plot[sz[tt], {tt, 0, Pi}]

s2z[t_] := (1/2)*(-Cos[t] + Cos[2 t])

Plot[{-Cos[tt], s2z[tt]}, {tt, Pi/3, Pi/2}]



E241[t1_, t2_, t3_,  t4_] := (1/3) (2 Cos[t1 + t2 - t3 - t4] +       Cos[t1 - t2]*Cos[t3 - t4]);

Plot[{ E241[tt, Pi/4, -tt, tt], E241[tt, tt, -tt, tt],E241[tt, -tt, -tt, tt], E241[tt, -tt, -tt, 0], E241[-tt, -tt, Pi/4, tt]},
{tt, 0, Pi},
    PlotStyle -> {{Hue[0.65, 1, 1], Dashing[{0.04, 0.02}],
    Thickness[0.01]}, {Hue[0, 1, 1],
    Dashing[{0.04, 0.02, 0.005, 0.02}], Thickness[0.01]}, {Green,
    Thickness[0.01], Dashing[{0.02, 0.02}]}, {Orange, Thickness[0.01],
     Dashing[{0.01, 0.01}]}, {Magenta, Thickness[0.01]}},
 PlotRange -> {-1.03, 1.03}, PlotRange -> {-1.03, 1.03},
 Frame -> True,
 FrameLabel -> {Style["\[Theta] [rad]", FontSize -> 26],
   Style["E(\[Theta])", FontSize -> 26]}, AspectRatio -> 1,
 BaseStyle -> {FontFamily -> "Times", FontSize -> 24},
 FrameTicks -> {{{0.0001, "0"}, \[Pi]/4, \[Pi]/2,
    3 \[Pi]/4, \[Pi]}, {-1, -1/2, {0.0001, "0"}, 1/2, 1}, None, None}]

Export["c:/mytex/2009-gtq-E241.ps", %, "EPS", ImageSize -> 500]

















&&&&&&&&&&&&&&&&&&&&&&&&&&&&&&&&&&&&&&&&&&&&&&&&&&&&&&&&&&&&&&&&&&&&&&&&&&&&&&&&&&&&&&&&&&&&&&&&&&&&&&&&&&&&&&&&&&&&&&&&&&&&&&&&&&&&&&&&&&&&&&&&&&&&&&
&&&&&&&&&&&&&&&&&&&&&&&&&&&&&&&&&&&&&&&&&&&&&&&&&&&&&&&&&&&&&&&&&&&&&&&&&&&&&&&&&&&&&&&&&&&&&&&&&&&&&&&&&&&&&&&&&&&&&&&&&&&&&&&&&&&&&&&&&&&&&&&&&&&&&&
&&&&&&&&&&&&&&&&&&&&&&&&&&&&&&&&&&&&&&&&&&&&&&&&&&&&&&&&&&&&&&&&&&&&&&&&&&&&&&&&&&&&&&&&&&&&&&&&&&&&&&&&&&&&&&&&&&&&&&&&&&&&&&&&&&&&&&&&&&&&&&&&&&&&&&
&&&&&&&&&&&&&&&&&&&&&&&&&&&&&&&&&&&&&&&&&&&&&&&&&&&&&&&&&&&&&&&&&&&&&&&&&&&&&&&&&&&&&&&&&&&&&&&&&&&&&&&&&&&&&&&&&&&&&&&&&&&&&&&&&&&&&&&&&&&&&&&&&&&&&&
&&&&&&&&&&&&&&&&&&&&&&&&&&&&&&&&&&&&&&&&&&&&&&&&&&&&&&&&&&&&&&&&&&&&&&&&&&&&&&&&&&&&&&&&&&&&&&&&&&&&&&&&&&&&&&&&&&&&&&&&&&&&&&&&&&&&&&&&&&&&&&&&&&&&&&
&&&&&&&&&&&&&&&&&&&&&&&&&&&&&&&&&&&&&&&&&&&&&&&&&&&&&&&&&&&&&&&&&&&&&&&&&&&&&&&&&&&&&&&&&&&&&&&&&&&&&&&&&&&&&&&&&&&&&&&&&&&&&&&&&&&&&&&&&&&&&&&&&&&&&&
&&&&&&&&&&&&&&&&&&&&&&&&&&&&&&&&&&&&&&&&&&&&&&&&&&&&&&&&&&&&&&&&&&&&&&&&&&&&&&&&&&&&&&&&&&&&&&&&&&&&&&&&&&&&&&&&&&&&&&&&&&&&&&&&&&&&&&&&&&&&&&&&&&&&&&
(* ~~ ~~ ~~ ~~ ~~ ~~ ~~ ~~ ~~ ~~ ~~ ~~ ~~ ~~ ~~ ~~ ~~ ~~ ~~ ~~ ~~ ~~ ~~ \
~~ ~~ ~~ ~~ ~~ ~~ ~~ ~~ ~~ ~~ ~~ ~~ ~*)(* ~~~~~~~~~~~~~~~~~~Start \
Mathematica Code~~~~~~~~~~~~~~~~~~~~~~~~~~~~~~~*)(* \
~~~~~~~~~~~~~~~~~~~~~~~~~~~~~~~~~~~~~~~~~~~~~~~~~~~~~~~~~~~~~~~~~~~~~~~\
*)(*<<Algebra`ReIm`*)Normalize[z_] := z/Sqrt[z.Conjugate[z]];

(*Definition of the Tensor Product*)
TensorProduct[a_, b_] :=
  Table[(*a,b are nxn and mxm-matrices*)
   a[[Ceiling[s/Length[b]], Ceiling[t/Length[b]]]]*
    b[[s - Floor[(s - 1)/Length[b]]*Length[b],
      t - Floor[(t - 1)/Length[b]]*Length[b]]], {s, 1,
    Length[a]*Length[b]}, {t, 1, Length[a]*Length[b]}];


(*Definition of the Tensor Product between two vectors*)

TensorProductVec[x_, y_] :=
  Flatten[Table[x[[i]] y[[j]], {i, 1, Length[x]}, {j, 1, Length[y]}]];


(*Definition of the Dyadic Product*)

DyadicProductVec[x_] :=
  Table[x[[i]] Conjugate[x[[j]]], {i, 1, Length[x]}, {j, 1,
    Length[x]}];

(*Definition of the sigma matrices*)


vecsig[r_, tt_, p_] :=
 r*{{Cos[tt], Sin[tt] Exp[-I p]}, {Sin[tt] Exp[I p], -Cos[tt]}}

(*Definition of some vectors*)

BellBasis = (1/Sqrt[2]) {{1, 0, 0, 1}, {0, 1, 1, 0}, {0, 1, -1,
     0}, {1, 0, 0, -1}};

Basis = {{1, 0, 0, 0}, {0, 1, 0, 0}, {0, 0, 1, 0}, {0, 0, 0, 1}};


(*~~~~~~~~~~~~~~~~~~~~~~~~~4 State \
System~~~~~~~~~~~~~~~~~~~~~~~~~~~~~~~~~~~~~~~%~~~~~~~~~~~~~~~2 x 4 \
%~~~~~~~~~~~~~~~2 x 4 %~~~~~~~~~~~~~~~2 x 4 %~~~~~~~~~~~~~~~2 x 4 \
%~~~~~~~~~~~~~~~2 x 4 %~~~~~~~~~~~~~~~2 x 4 %~~~~~~~~~~~~~~~2 x 4 \
%~~~~~~~~~~~~~~~2 x 4*)(*Definition of operators*)(*Definition of \
one-particle operator*)
M4X = (1/2) {{0, Sqrt[3], 0, 0}, {Sqrt[3], 0, 2, 0}, {0, 2, 0,
     Sqrt[3]}, {0, 0, Sqrt[3], 0}};
M4Y = (1/2) {{0, -Sqrt[3] I, 0, 0}, {Sqrt[3] I, 0, -2 I, 0}, {0, 2 I,
     0, -Sqrt[3] I}, {0, 0, Sqrt[3] I, 0}};
M4Z = (1/2) {{3, 0, 0, 0}, {0, 1, 0, 0}, {0, 0, -1, 0}, {0, 0, 0, -3}};

Eigenvectors[M4X]
Eigenvectors[M4Y]
Eigenvectors[M4Z]

S4[t_, p_] :=
  FullSimplify[M4X*Sin[t] Cos[p] + M4Y*Sin[t] Sin[p] + M4Z*Cos[t]];



(*  ~~~~~~~~~~~~~~~~~~~general operator~~~~~~~~~~~~~~~~~~~~~~~*)

LM32 = -3/2;
LM12 = -1/2;
LP32 = 3/2;
LP12 = 1/2;

ES4M32[\[Theta]_, \[Phi]_] :=
  FullSimplify[
   Assuming[{0 < \[Theta] < Pi, 0 <= \[Phi] <= 2 Pi},
    Normalize[
     Eigenvectors[S4[\[Theta], \[Phi]]][[1]]]], {Element[\[Theta],
     Reals], Element[\[Phi], Reals]}];
ES4P32[\[Theta]_, \[Phi]_] :=
  FullSimplify[
   Assuming[{0 < \[Theta] < Pi, 0 <= \[Phi] <= 2 Pi},
    Normalize[
     Eigenvectors[S4[\[Theta], \[Phi]]][[2]]]], {Element[\[Theta],
     Reals], Element[\[Phi], Reals]}];
ES4M12[\[Theta]_, \[Phi]_] :=
  FullSimplify[
   Assuming[{0 < \[Theta] < Pi, 0 <= \[Phi] <= 2 Pi},
    Normalize[
     Eigenvectors[S4[\[Theta], \[Phi]]][[3]]]], {Element[\[Theta],
     Reals], Element[\[Phi], Reals]}];
ES4P12[\[Theta]_, \[Phi]_] :=
  FullSimplify[
   Assuming[{0 < \[Theta] < Pi, 0 <= \[Phi] <= 2 Pi},
    Normalize[
     Eigenvectors[S4[\[Theta], \[Phi]]][[4]]]], {Element[\[Theta],
     Reals], Element[\[Phi], Reals]}];



JointProjector4GEN[x1_, x2_, p1_, p2_] :=
  TensorProduct[S4[x1, p1], S4[x2, p2]];

v4P32 = ES4P32[0, 0]
v4P12 = ES4P12[0, 0]
v4M12 = ES4M12[0, 0]
v4M32 = ES4M32[0, 0]



psi4s = (1/2)*(TensorProductVec[v4P32, v4M32] -
    TensorProductVec[v4M32, v4P32] - TensorProductVec[v4P12, v4M12] +
    TensorProductVec[v4M12, v4P12])


Expectation4sGEN[x1_, x2_, p1_, p2_] :=
  Tr[DyadicProductVec[psi4s].JointProjector4GEN[x1, x2, p1, p2]];

FullSimplify[Expectation4sGEN[x1, x2, p1, p2]]


(* ~~~~~~~~general case~~~~~~~~~*)

EPPMM1[L4M32_, L4M12_, L4P12_, L4P32_, \[Theta]_, \[Phi]_] :=
 Assuming[{0 < \[Theta] < Pi, 0 <= \[Phi] <= 2 Pi},
  FullSimplify[
   L4M32*Assuming[{0 < \[Theta] < Pi, 0 <= \[Phi] <= 2 Pi},
      FullSimplify[
       DyadicProductVec[
        ES4M32[\[Theta], \[Phi]]], {Element[\[Theta], Reals],
        Element[\[Phi], Reals]}]] +
    L4M12*Assuming[{0 < \[Theta] < Pi, 0 <= \[Phi] <= 2 Pi},
      FullSimplify[
       DyadicProductVec[
        ES4M12[\[Theta], \[Phi]]], {Element[\[Theta], Reals],
        Element[\[Phi], Reals]}]] +
    L4P32*Assuming[{0 < \[Theta] < Pi, 0 <= \[Phi] <= 2 Pi},
      FullSimplify[
       DyadicProductVec[
        ES4P32[\[Theta], \[Phi]]], {Element[\[Theta], Reals],
        Element[\[Phi], Reals]}]] +
    L4P12*Assuming[{0 < \[Theta] < Pi, 0 <= \[Phi] <= 2 Pi},
      FullSimplify[
       DyadicProductVec[
        ES4P12[\[Theta], \[Phi]]], {Element[\[Theta], Reals],
        Element[\[Phi], Reals]}]]]]

JointProjector4PPMM1[L4M32_ , L4M12_ , L4P12_ , L4P32_ , x1_, x2_, p1_, p2_] :=  Assuming[{0 < \[Theta] < Pi, 0 <= \[Phi] <= 2 Pi},
 FullSimplify[TensorProduct[EPPMM1[L4M32 , L4M12 , L4P12 , L4P32 , x1,p1],EPPMM1[L4M32 , L4M12 , L4P12 , L4P32 ,x2,p2]], {Element[\[Theta], Reals],
   Element[\[Phi], Reals]}] ];

Expectation4PPMM1[L4M32_ , L4M12_ , L4P12_ , L4P32_ , x1_, x2_, p1_, p2_] := Tr[DyadicProductVec[psi4s].JointProjector4PPMM1[L4M32 , L4M12 , L4P12 , L4P32 ,x1, x2, p1, p2]];




pppp[x_] = FullSimplify[Expectation4PPMM1[+1,+1, +1, +1, x, 0, 0, 0]];
pppm[x_] = FullSimplify[Expectation4PPMM1[+1,+1, +1, -1, x, 0, 0, 0]];
ppmp[x_] = FullSimplify[Expectation4PPMM1[+1,+1, -1, +1, x, 0, 0, 0]];
ppmm[x_] = FullSimplify[Expectation4PPMM1[+1,+1, -1, -1, x, 0, 0, 0]];
pmpp[x_] = FullSimplify[Expectation4PPMM1[+1,-1, +1, +1, x, 0, 0, 0]];
pmpm[x_] = FullSimplify[Expectation4PPMM1[+1,-1, +1, -1, x, 0, 0, 0]];
pmmp[x_] = FullSimplify[Expectation4PPMM1[+1,-1, -1, +1, x, 0, 0, 0]];
pmmm[x_] = FullSimplify[Expectation4PPMM1[+1,-1, -1, -1, x, 0, 0, 0]];
mppp[x_] = FullSimplify[Expectation4PPMM1[-1,+1, +1, +1, x, 0, 0, 0]];
mppm[x_] = FullSimplify[Expectation4PPMM1[-1,+1, +1, -1, x, 0, 0, 0]];
mpmp[x_] = FullSimplify[Expectation4PPMM1[-1,+1, -1, +1, x, 0, 0, 0]];
mpmm[x_] = FullSimplify[Expectation4PPMM1[-1,+1, -1, -1, x, 0, 0, 0]];
mmpp[x_] = FullSimplify[Expectation4PPMM1[-1,-1, +1, +1, x, 0, 0, 0]];
mmpm[x_] = FullSimplify[Expectation4PPMM1[-1,-1, +1, -1, x, 0, 0, 0]];
mmmp[x_] = FullSimplify[Expectation4PPMM1[-1,-1, -1, +1, x, 0, 0, 0]];
mmmm[x_] = FullSimplify[Expectation4PPMM1[-1,-1, -1, -1, x, 0, 0, 0]];

all[x_]:= pppp[x]-
 pppm[x]
-ppmp[x]
+ppmm[x]
-pmpp[x]
+pmpm[x]
+pmmp[x]
-pmmm[x]
-mppp[x]
+mppm[x]
+mpmp[x]
-mpmm[x]
+mmpp[x]
-mmpm[x]
-mmmp[x]
+mmmm[x];


Plot[{all[x1],a[x1], -Cos[x1], 2*x1/Pi - 1}, {x1, 0, Pi},
 PlotStyle -> {{Hue[0.65, 1, 1], Dashing[{0.04, 0.02}],
    Thickness[0.01]}, {Hue[0, 1, 1],
    Dashing[{0.04, 0.02, 0.005, 0.02}], Thickness[0.01]}, {Green,
    Thickness[0.01], Dashing[{0.02, 0.02}]}, {Orange, Thickness[0.01],
     Dashing[{0.01, 0.01}]}, {Black, Thickness[0.007]}},
 PlotRange -> {-1.03, 1.03}, PlotRange -> {-1.03, 1.03},
 Frame -> True,
 FrameLabel -> {Style["\[Theta] [rad]", FontSize -> 26],
   Style["E(\[Theta])", FontSize -> 26]}, AspectRatio -> 1,
 BaseStyle -> {FontFamily -> "Times", FontSize -> 24},
 FrameTicks -> {{{0.0001, "0"}, \[Pi]/4, \[Pi]/2,
    3 \[Pi]/4, \[Pi]}, {-1, -1/2, {0.0001, "0"}, 1/2, 1}, None, None}]








mzpz[x_] = FullSimplify[Expectation4PPMM1[-1, 0, +1, 0, x, 0, 0, 0]];
mzzp[x_] = FullSimplify[Expectation4PPMM1[-1, 0, 0, +1, x, 0, 0, 0]];
zmpz[x_] = FullSimplify[Expectation4PPMM1[0, -1, +1, 0, x, 0, 0, 0]];
zmzp[x_] = FullSimplify[Expectation4PPMM1[0, -1, 0, +1, x, 0, 0, 0]];
mmzz[x_] = FullSimplify[Expectation4PPMM1[-1, -1, 0, 0, x, 0, 0, 0]];
zzpp[x_] = FullSimplify[Expectation4PPMM1[0, 0, +1, +1, x, 0, 0, 0]];


mzzz[x_] = FullSimplify[Expectation4PPMM1[-1, 0, 0, 0, x, 0, 0, 0]];
zmzz[x_] = FullSimplify[Expectation4PPMM1[0, -1, 0, 0, x, 0, 0, 0]];
zzpz[x_] = FullSimplify[Expectation4PPMM1[0, 0, +1, 0, x, 0, 0, 0]];
zzzp[x_] = FullSimplify[Expectation4PPMM1[0, 0, 0, +1, x, 0, 0, 0]];

mmpp[x_] = FullSimplify[Expectation4PPMM1[-1, -1, +1, +1, x, 0, 0, 0]];


all1[x_] :=
  mzpz[x] + mzzp[x] + zmpz[x] + zmzp[x] + mmzz[x] + zzpp[x] -
   2 mzzz[x] - 2 zmzz[x] - 2 zzpz[x] - 2 zzzp[x];
Plot[{FullSimplify[ all1[x1]], mmpp[x1], -Cos[x1], 2*x1/Pi - 1}, {x1,
  0, Pi}, PlotStyle -> {{Hue[0.65, 1, 1], Dashing[{0.04, 0.02}],
    Thickness[0.01]}, {Hue[0, 1, 1],
    Dashing[{0.04, 0.02, 0.005, 0.02}], Thickness[0.01]}, {Green,
    Thickness[0.01], Dashing[{0.02, 0.02}]}, {Orange, Thickness[0.01],
     Dashing[{0.01, 0.01}]}, {Black, Thickness[0.007]}},
 PlotRange -> {-1.03, 1.03}, PlotRange -> {-1.03, 1.03},
 Frame -> True,
 FrameLabel -> {Style["\[Theta] [rad]", FontSize -> 26],
   Style["E(\[Theta])", FontSize -> 26]}, AspectRatio -> 1,
 BaseStyle -> {FontFamily -> "Times", FontSize -> 24},
 FrameTicks -> {{{0.0001, "0"}, \[Pi]/4, \[Pi]/2,
    3 \[Pi]/4, \[Pi]}, {-1, -1/2, {0.0001, "0"}, 1/2, 1}, None, None}]
