\documentclass[pra,amsfonts,showpacs,preprint,showkeys]{revtex4}
%\documentclass[pra,showpacs,showkeys,amsfonts]{revtex4}
\usepackage[T1]{fontenc}
\usepackage{graphicx}

%\documentclass[12pt,a4paper]{article}
%\usepackage[english, USenglish]{babel}
\usepackage{amsmath}
%\linespread{1.3}
%\usepackage{amsfonts}
%\usepackage[dvips]{graphicx}
%\usepackage{hyperref}
%\usepackage{a4wide}
%\usepackage{rotating}
\usepackage{longtable}
%\usepackage{lscape}
\begin{document}
\title{All the singlet states}

\author{Peter Kasperkowitz}
\author{Maria Schimpf}
\author{Karl Svozil}
\affiliation{Institute of Theoretical Physics, Vienna
  University of Technology, Wiedner Hauptstra\ss e 8-10/136, A-1040
    Vienna, Austria}
\email{svozil@tuwien.ac.at}


\begin{abstract}
We present a group theoretic method to construct all
$N$-particle singlet states by recursion and iteration and we
derive the symmetries of $N$-particle singlet states.
\end{abstract}

\pacs{03.67.-a,02.20.-a,03.65.Ca }

\keywords{Quantum information, singlet states, group theory}

\maketitle


\section{Introduction}

Singlet states are among the most useful states in quantum
mechanics; yet their explicit structure---although well understood
in general terms in group theory---has up to now neither been
enumerated nor investigated beyond a few instances for
spin-${1\over 2}$ and spin-$1$ particles. Recent theoretical and
experimental studies in multi-particle production (e.g.,
Ref.~\cite{egbkzw}) elicit that a more systematic way to generate
the complete set of arbitrary $N$-particle singlet states is desirable.

In the present study we pursue an algorithmic generation strategy,
and tabulate some of the first singlet states. The recursive
method employed is based on triangle relations and
Clebsch-Gordan coefficients (e.g., Ch.~13, Sec.~27 of
Ref.~\cite{messiah-62}). With this approach, a complete table of
all angular momentum states is created. The singlet states stem
from the various pathways towards the $j= m=0$ states. The
procedure can best be illustrated in a triangular diagram where
the states in ascending order of angular momentum are drawn against the number
of particles. In such a diagram, the ``lowest'' states correspond
to singlets.




\subsection{Description of the algorithm for obtaining spin-${1\over 2}$ and spin-$1$ $N$-particle singlet states}

There always exist ``zigzag'' singlet states, which are
the product of $r$ two-particle singlet states stemming from the
rising and lowering of consecutive states. The situation is
depicted in Fig.~\ref{2005-singlet-f1-zigzag}. For $j=1$ and
$N=3r$ there exist ``zigzag'' singlet states, which are the
product of $r$ three-particle singlet states. For singlet states
with $N=2r+3t$ ($r,t$ integer) there exist singlet states being
the product of $r$
two-particle singlet states and $t$ three-particle singlet states.
\begin{figure}
\begin{center}
%TexCad Options
%\grade{\off}
%\emlines{\off}
%\beziermacro{\on}
%\reduce{\on}
%\snapping{\off}
%\quality{2.00}
%\graddiff{0.01}
%\snapasp{1}
%\zoom{0.80}
\unitlength 0.40mm \linethickness{0.4pt}
\begin{picture}(305.33,150.00)
\put(15.00,5.00){\makebox(0,0)[cc]{$0$}}
\put(45.00,5.00){\makebox(0,0)[cc]{$1$}}
\put(75.00,5.00){\makebox(0,0)[cc]{$2$}}
\put(105.00,5.00){\makebox(0,0)[cc]{$3$}}
\put(135.00,5.00){\makebox(0,0)[cc]{$4$}}
\put(5.00,15.00){\makebox(0,0)[cc]{${0}$}}
\put(5.00,45.00){\makebox(0,0)[cc]{${l}$}}
\put(5.00,60.00){\makebox(0,0)[cc]{$j$}}
\put(10.00,10.00){\line(0,1){40.00}}
\put(10.00,10.00){\line(1,0){130.00}}
%\put(15.00,15.00){\line(1,1){125.00}}
%\put(15.00,15.00){\circle*{4.00}}
\put(45.00,45.00){\circle*{4.00}}
\put(75.00,15.00){\circle*{4.00}}
\put(105.00,45.00){\circle*{4.00}}
\put(135.00,15.00){\circle*{4.00}}
%\put(15.00,15.00){\vector(1,1){28.00}}
\put(45.00,45.00){\vector(1,-1){28.00}}
\put(75.00,15.00){\vector(1,1){28.00}}
\put(105.00,45.00){\vector(1,-1){28.00}}
\put(170.33,5.00){\makebox(0,0)[cc]{$N-4$}}
\put(200.33,5.00){\makebox(0,0)[cc]{$N-3$}}
\put(230.33,5.00){\makebox(0,0)[cc]{$N-2$}}
\put(260.33,5.00){\makebox(0,0)[cc]{$N-1$}}
\put(290.33,5.00){\makebox(0,0)[cc]{$N$}}
\put(305.33,5.00){\makebox(0,0)[cc]{$N$}}
\put(165.33,10.00){\line(1,0){130.00}}
\put(170.33,15.00){\circle*{4.00}}
\put(200.33,45.00){\circle*{4.00}}
\put(230.33,15.00){\circle*{4.00}}
\put(260.33,45.00){\circle*{4.00}}
\put(290.33,15.00){\circle*{4.00}}
\put(170.33,15.00){\vector(1,1){28.00}}
\put(200.33,45.00){\vector(1,-1){28.00}}
\put(230.33,15.00){\vector(1,1){28.00}}
\put(260.33,45.00){\vector(1,-1){28.00}}
\put(290.33,15.00){\circle{8.00}}
\put(152.08,10.00){\makebox(0,0)[cc]{$\cdots$}}
\end{picture}
\end{center}
\caption{ Construction of the ``zigzag'' singlet state of $N$
particles which effectively is a product state of $\frac{N}{2}$
spin-$l$ particle states. \label{2005-singlet-f1-zigzag}}
\end{figure}


Here we present a method to construct all states for a given
number of particles. They are the basis to construct non-trivial,
e.g., non-zigzag  singlet states, which are not just
products of singlet states of a smaller number of particles.


We start by considering the spin state of a single spin-${1\over
2}$ particle. A second spin-${1\over 2}$ particle is added by
combining two angular momenta $\frac{1}{2}$ to all possible
angular momenta $j_{12}=0,1$. Next a third particle is introduced
by coupling a third angular momentum $\frac{1}{2}$ to all
previously derived states. Following the triangle equation, the
resulting $j$-values for each $j_{12}$ are: $|j_{12}-j_3|\leq
j\leq j_{12}+j_3$.

In order to obtain all $N$-particle singlet states, we
successively produce all states (not only singlets) of
$\frac{N}{2}$ particles. From this point on, only certain states
are necessary for the further procedure. For $\frac{N}{2}\leq
h\leq N$ particles we only need angular momentum states with
$0\leq j\leq\frac{N-h}{2}$.

Angular momentum states will be written as
$|h,j,m,i\rangle$, where $h$ denotes the particle number, $j$ the
angular momentum, $m$ the magnetic quantum number, $i$ the number
of state. The Clebsch-Gordan coefficient is denoted $\langle
j_1,j_2,m_1,m_2|j,m\rangle$. $f[j+1,h-1]$ denotes the number of
states at $h$ particles and angular momentum $\frac{j}{2}$.

For the sake of demonstration of the above method, we consider the
explicit procedure for obtaining the states $|h,j,m,i\rangle $. We
first consider the states of $h-1$ particles and angular momentum
$j+{1\over2}$. To produce the concrete state $|h,j,m,i\rangle$ we
multiply the Clebsch-Gordan coefficient $\langle
j+{1\over2},m-{1\over2},{1\over2},{1\over2}|j,m\rangle$ with the
product state
$|h-1,j+{1\over2},m-{1\over2},i\rangle\otimes|1,{1\over2},{1\over2},1\rangle$.
We take the state $|h-1,j+{1\over2},m+{1\over2},i\rangle$, build
the product state
$|h-1,j+{1\over2},m+{1\over2},i\rangle\otimes|1,{1\over2},-{1\over2},1\rangle$
and multiply it with the Clebsch-Gordan coefficient $\langle
j+{1\over2},m+{1\over2},{1\over2},-{1\over2}|j,m\rangle$. Adding
the two results we obtain the state $|h,j,m,i\rangle$. We do this
for $m= -j,j$ and $i= 1, f[(2j+1)+1,h-1]$. If $j$ is greater than
zero, we look at the states
$|h-1,j-{1\over2},m-{1\over2},i\rangle$ and
$|h-1,j-{1\over2},m+{1\over2},i\rangle$ and obtain the
$|h,j,m,i\rangle$ particle state as the sum of $\langle
j-{1\over2},m-{1\over2},{1\over2},{1\over2}|j,m\rangle|h-1,j-{1\over2},m-{1\over2},i\rangle\otimes|1,{1\over2},{1\over2},1\rangle$
and $\langle
j-{1\over2},m+{1\over2},{1\over2},-{1\over2}|j,m\rangle|h-1,j-{1\over2},m+{1\over2},i\rangle\otimes|1,{1\over2},-{1\over2},1\rangle$.
This procedure is carried out for $m= -j,j$ and $i=
f[(2j+1)+1,h-1]+1,f[(2j+1)+1,h-1]+f[(2j+1)-1,h-1]$.

%\begin{figure}
%\begin{align}
%&|h-1,j+{1\over2},m-{1\over2},i\rangle\rightarrow \langle
%j+{1\over2},m-{1\over2},{1\over2},{1\over2}|j,m\rangle|h-1,j+{1\over2},m-{1\over2}\rangle\otimes|{1\over2},{1\over2}\rangle\\
%&|h-1,j+{1\over2},m+{1\over2},i\rangle\rightarrow \langle
%j+{1\over2},m+{1\over2},{1\over2},-{1\over2}|j,m\rangle|h-1,j+{1\over2},m+{1\over2}\rangle\otimes|{1\over2},-{1\over2}\rangle
%\end{align}
%joining eq.1 and eq.2 $\rightarrow |h,j,m,i\rangle $ for $m= -j,j$
%and $i= 1, f[(2j+1)+1,h-1]$\\\\
%If $j>0$
%\begin{align}
%&|h-1,j-{1\over2},m-{1\over2},i\rangle\rightarrow \langle
%j-{1\over2},m-{1\over2},{1\over2},{1\over2}|j,m\rangle|h-1,j-{1\over2},m-{1\over2}\rangle\otimes|{1\over2},{1\over2}\rangle\\
%&|h-1,j-{1\over2},m+{1\over2},i\rangle\rightarrow \langle
%j-{1\over2},m+{1\over2},{1\over2},-{1\over2}|j,m\rangle|h-1,j-{1\over2},m+{1\over2}\rangle\otimes|{1\over2},-{1\over2}\rangle
%\end{align}
%joining eq.3 and eq.4 $\rightarrow |h,j,m,i\rangle $ for $m= -j,j$
%and $i= f[(2j+1)+1,h-1]+1,f[(2j+1)+1,h-1]+f[(2j+1)-1,h-1]$
%\caption{Sketch of the construction of the $|h,j,m\rangle$ product
%states for spin-${1\over 2}$ particles. $h$ denotes the particle
%number. f[j+1,h] denotes the number of states at $h$ particles and
%angular momentum $j$.}
%\end{figure}




A concrete example is drawn in Fig.~\ref{2005-singlet-f12-e1}.
\begin{figure}
\begin{center}
\begin{tabular}{ccc}
%TexCad Options
%\grade{\off}
%\emlines{\off}
%\beziermacro{\on}
%\reduce{\on}
%\snapping{\off}
%\quality{2.00}
%\graddiff{0.01}
%\snapasp{1}
%\zoom{1.00}
\unitlength 0.40mm
\linethickness{0.4pt}
\begin{picture}(150.00,150.00)
\put(15.00,5.00){\makebox(0,0)[cc]{$0$}}
\put(45.00,5.00){\makebox(0,0)[cc]{$1$}}
\put(75.00,5.00){\makebox(0,0)[cc]{$2$}}
\put(105.0,5.00){\makebox(0,0)[cc]{$3$}}
\put(135.00,5.00){\makebox(0,0)[cc]{$4$}}
\put(150.00,5.00){\makebox(0,0)[cc]{$N$}}
\put(5.00,45.00){\makebox(0,0)[cc]{${1\over 2}$}}
\put(5.00,75.00){\makebox(0,0)[cc]{$1$}}
\put(5.00,105.00){\makebox(0,0)[cc]{${3\over 2}$}}
\put(5.00,135.00){\makebox(0,0)[cc]{$2$}}
\put(5.00,150.00){\makebox(0,0)[cc]{$j$}}
\put(10.00,10.00){\line(0,1){130.00}}
\put(10.00,10.00){\line(1,0){130.00}}
%\put(15.00,15.00){\line(1,1){125.00}}
\put(45.00,45.00){\circle*{4.00}}
\put(75.00,15.00){\circle*{4.00}}
\put(105.00,45.00){\circle*{4.00}}
\put(135.00,15.00){\circle*{4.00}}
\put(45.00,45.00){\vector(1,-1){28.00}}
\put(75.00,15.00){\vector(1,1){28.00}}
\put(105.00,45.00){\vector(1,-1){28.00}}
\put(75.00,75.00){\circle{4.00}}
%\put(105.00,75.00){\circle{4.00}}
\put(135.00,75.00){\circle{4.00}}
\put(105.00,105.00){\circle{4.00}}
%\put(135.00,105.00){\circle{4.00}}
\put(135.00,135.00){\circle{4.00}}
%\put(75.00,45.00){\circle{4.00}}
%\put(135.00,45.00){\circle{4.00}}
\put(135.00,15.00){\circle{8.00}}
\end{picture}
&
$\qquad$
&
%TexCad Options
%\grade{\off}
%\emlines{\off}
%\beziermacro{\on}
%\reduce{\on}
%\snapping{\off}
%\quality{2.00}
%\graddiff{0.01}
%\snapasp{1}
%\zoom{1.00}
\unitlength 0.40mm
\linethickness{0.4pt}
\begin{picture}(150.00,150.00)
\put(15.00,5.00){\makebox(0,0)[cc]{$0$}}
\put(45.00,5.00){\makebox(0,0)[cc]{$1$}}
\put(75.00,5.00){\makebox(0,0)[cc]{$2$}}
\put(105.0,5.00){\makebox(0,0)[cc]{$3$}}
\put(135.00,5.00){\makebox(0,0)[cc]{$4$}}
\put(150.00,5.00){\makebox(0,0)[cc]{$N$}}
\put(5.00,45.00){\makebox(0,0)[cc]{${1\over 2}$}}
\put(5.00,75.00){\makebox(0,0)[cc]{$1$}}
\put(5.00,105.00){\makebox(0,0)[cc]{${3\over 2}$}}
\put(5.00,135.00){\makebox(0,0)[cc]{$2$}}
\put(5.00,150.00){\makebox(0,0)[cc]{$j$}}
\put(10.00,10.00){\line(0,1){130.00}}
\put(10.00,10.00){\line(1,0){130.00}}
%\put(15.00,15.00){\line(1,1){125.00}}
%\put(15.00,15.00){\circle*{4.00}}
\put(45.00,45.00){\circle*{4.00}} \put(75.00,15.00){\circle{4.00}}
\put(105.00,45.00){\circle*{4.00}}
\put(135.00,15.00){\circle*{4.00}}
%\put(15.00,15.00){\vector(1,1){28.00}}
\put(75.00,75.00){\vector(1,-1){28.00}}
\put(45.00,45.00){\vector(1,1){28.00}}
\put(105.00,45.00){\vector(1,-1){28.00}}
\put(75.00,75.00){\circle*{4.00}}
%\put(105.00,75.00){\circle{4.00}}
\put(135.00,75.00){\circle{4.00}}
\put(105.00,105.00){\circle{4.00}}
%\put(135.00,105.00){\circle{4.00}}
\put(135.00,135.00){\circle{4.00}}
%\put(75.00,45.00){\circle{4.00}}
%\put(135.00,45.00){\circle{4.00}}
\put(135.00,15.00){\circle{8.00}}
\end{picture}
\\
a)&&b)
\end{tabular}
\end{center}
\caption{ Construction of both singlet states of four
spin-${1\over 2}$ particles. Concentric circles indicate the
target states. \label{2005-singlet-f12-e1}}
\end{figure}
It contains the pathways leading to the construction of both
singlet states of four spin-${1\over 2}$ particles. Another
example is the construction of the  three spin-1 particle
singlet state drawn in
 Fig.~\ref{2005-singlet-f1-e2}.
\begin{figure}
\begin{center}
%TexCad Options
%\grade{\off}
%\emlines{\off}
%\beziermacro{\on}
%\reduce{\on}
%\snapping{\off}
%\quality{2.00}
%\graddiff{0.01}
%\snapasp{1}
%\zoom{1.00}
\unitlength 0.40mm
\linethickness{0.4pt}
\begin{picture}(120.00,120.00)
\put(15.00,5.00){\makebox(0,0)[cc]{$0$}}
\put(45.00,5.00){\makebox(0,0)[cc]{$1$}}
\put(75.00,5.00){\makebox(0,0)[cc]{$2$}}
\put(105.00,5.00){\makebox(0,0)[cc]{$3$}}
\put(120.00,5.00){\makebox(0,0)[cc]{$N$}}
\put(5.00,45.00){\makebox(0,0)[cc]{${1}$}}
\put(5.00,75.00){\makebox(0,0)[cc]{$2$}}
\put(5.00,105.00){\makebox(0,0)[cc]{${3}$}}
\put(5.00,120.00){\makebox(0,0)[cc]{$j$}}
\put(10.00,10.00){\line(0,1){100.00}}
\put(10.00,10.00){\line(1,0){100.00}}
%\put(15.00,15.00){\line(1,1){125.00}}
%\put(15.00,15.00){\circle*{4.00}}
%\put(45.00,15.00){\circle{4.00}}
\put(45.00,45.00){\circle*{4.00}}
\put(75.00,15.00){\circle{4.00}}
\put(105.00,15.00){\circle*{4.00}}
\put(105.00,15.00){\circle{8.00}}
%\put(15.00,15.00){\vector(1,1){28.00}}
\put(75.00,44.67){\vector(1,-1){28.00}}
\put(75.00,75.00){\circle{4.00}}
\put(105.00,75.00){\circle{4.00}}
\put(105.00,105.00){\circle{4.00}}
\put(75.00,45.00){\circle*{4.00}}
\put(105.00,45.00){\circle{4.00}}
%\put(45.00,15.00){\vector(1,1){28.00}}
\put(45.00,45.00){\vector(1,0){27.00}}
\end{picture}
\end{center}
\caption{ Construction of the singlet state of three spin-$1$
particles. \label{2005-singlet-f1-e2}}
\end{figure}
The singlet states of up to 6 spin-${1\over 2}$ and 4
spin-${1}$ particle are explicitly enumerated in
Tables \ref{2005-singlet-t12} and \ref{2005-singlet-t1}.

%\begin{table}
%\begin{tabular}{ccccc}
%\hline\hline
%k & \# & \\
%\hline\hline
%\hline\hline
%2& 1 &$-\frac{1}{2} \vert -+ \rangle +\frac{1}{2} \vert +- \rangle $\\
%\hline
%4& 1 &
%${\frac{1}{\sqrt{3}} \vert --++ \rangle }+{-\frac{1}{2\sqrt{3}} \vert -+-+ \rangle }
%+{-\frac{1}{2\sqrt{3}} \vert +--+ \rangle }$\\
%& 2 &
%$+{-\frac{1}{2\sqrt{3}} \vert -++- \rangle }
%+{-\frac{1}{2\sqrt{3}} \vert +-+- \rangle }+{\frac{1}{{\sqrt{3}}} \vert ++-- \rangle }$\\
%& 3 &
%${{\frac{1}{2} \vert -+-+ \rangle }+{-\frac{1}{2} \vert +--+ \rangle }
%+{-\frac{1}{2} \vert -++- \rangle }+{\frac{1}{2} \vert +-+- \rangle }}$

%\hline\hline
%\end{tabular}
%\caption{First singlet states of k spin-${1\over 2}$ quanta.
%\label{2005-singlet-t12}}
%\end{table}


%\begin{table}
%\begin{tabular}{ccccc}
%\hline\hline
%k & \# & \\


%\hline\hline
%\end{tabular}
%\caption{First singlet states of k  spin-${1}$ quanta.
%\label{2005-singlet-t1}}
%\end{table}
\clearpage
\begin{table}
\begin{tabular}{ccc}
\hline\hline
N & \# & \\
\hline\hline
2&1&$\frac{1}{{\sqrt{2}}}\big(|+,-\rangle-|-,+\rangle\big);$\\\hline
4&1&$-\frac{1}{2\sqrt{3}} \big(|-,+,-,+\rangle
+|-,+,+,-\rangle+|+,-,-,+\rangle +|+,-,+,-\rangle \big)+$\\&&$
+\frac{1}{\sqrt{3}}\big(|-,-,+,+\rangle +|+,+,-,-\rangle\big)
;$\\


4&2&$\big(-\frac{1}{{\sqrt{2}}}|-,+\rangle
+\frac{1}{{\sqrt{2}}}|+,-\rangle\big)^2
% -\frac{1}{2}\big(|-1,+,+,-1\rangle
%+|1,-1,-1,1\rangle \big) +\frac{1}{2}\big(|-1,1,-1,1\rangle+
%|1,-1,1,-1\rangle\big)
;$\\\hline


6&1& $-\frac{1}{2}|-,-,-,+,+,+\rangle +-\frac{1}{6}\big(
|-,+,+,-,-,+\rangle + |-,+,+,-,+,-\rangle+$\\&&$
+|-,+,+,+,-,-\rangle + |+,-,+,-,-,+\rangle +|+,-,+,-,+,-\rangle
+$\\&&$+ |+,-,+,+,-,-\rangle +|+,+,-,-,-,+\rangle +
|+,+,-,-,+,-\rangle +$\\&&$+|+,+,-,+,-,-\rangle\big) +
\frac{1}{6}\big(|-,-,+,-,+,+\rangle +|-,-,+,+,-,+\rangle +$\\&&$+
|-,-,+,+,+,-\rangle +|-,+,-,-,+,+\rangle + |-,+,-,+,-,+\rangle
+$\\&&$+|-,+,-,+,+,-\rangle + |+,-,-,-,+,+\rangle
+|+,-,-,+,-,+\rangle +$\\&&$+
|+,-,-,+,+,-\rangle\big) +\frac{1}{2}|+,+,+,-,-,-\rangle;$\\

6&2&$ -\frac{{\sqrt{2}}}{3}|-,-,+,-,+,+\rangle +-\frac{1}{3
{\sqrt{2}}}\big(|-,+,+,+,-,-\rangle +|+,-,+,+,-,-\rangle+$\\&&$
+|+,+,-,-,-,+\rangle+
 |+,+,-,-,+,-\rangle\big) +-\frac{1}{6 {\sqrt{2}}}\big(|-,+,-,+,-,+\rangle+$\\&&$ +|-,+,-,+,+,-\rangle
 +
 |+,-,-,+,-,+\rangle +|+,-,-,+,+,-\rangle\big)+$\\&&$ +\frac{1}{6 {\sqrt{2}}}\big(|-,+,+,-,-,+\rangle +
 |-,+,+,-,+,-\rangle +|+,-,+,-,-,+\rangle+$\\&&$ +|+,-,+,-,+,-\rangle\big)+
 \frac{1}{3 {\sqrt{2}}}\big(|-,-,+,+,-,+\rangle +|-,-,+,+,+,-\rangle+$\\&&$ +|-,+,-,-,+,+\rangle +
|+,-,-,-,+,+\rangle\big)+\frac{{\sqrt{2}}}{3}|+,+,-,+,-,-\rangle;$\\

6&3&$ -\frac{1}{{\sqrt{6}}}\big(|-,+,-,-,+,+\rangle
+|-,+,+,+,-,-\rangle\big)+-\frac{1}{2
{\sqrt{6}}}\big(|+,-,-,+,-,+\rangle +$\\&&$+ |+,-,-,+,+,-\rangle+
 |+,-,+,-,-,+\rangle +|+,-,+,-,+,-\rangle\big)+$\\&&$ +\frac{1}{2 {\sqrt{6}}}\big(|-,+,-,+,-,+\rangle +
 |-,+,-,+,+,-\rangle +|-,+,+,-,-,+\rangle
 +$\\&&$+|-,+,+,-,+,-\rangle\big)+
 \frac{1}{{\sqrt{6}}}\big(|+,-,-,-,+,+\rangle +|+,-,+,+,-,-\rangle\big)
 ;$\\

6&4&$ -\frac{1}{{\sqrt{6}}}\big(|-,-,+,+,-,+\rangle +
|+,+,-,-,-,+\rangle\big) +-\frac{1}{2
{\sqrt{6}}}\big(|-,+,-,+,+,-\rangle+$\\&&$ +|-,+,+,-,+,-\rangle +
|+,-,-,+,+,-\rangle +|+,-,+,-,+,-\rangle\big)+$\\&&$ +\frac{1}{2
{\sqrt{6}}}\big(|-,+,-,+,-,+\rangle +
 |-,+,+,-,-,+\rangle +|+,-,-,+,-,+\rangle+$\\&&$ +|+,-,+,-,-,+\rangle\big) +
 \frac{1}{{\sqrt{6}}}\big(|-,-,+,+,+,-\rangle
 +|+,+,-,-,+,-\rangle\big);$\\

6&5&$\big(-\frac{1}{{\sqrt{2}}}|-,+\rangle
+\frac{1}{{\sqrt{2}}}|+,-\rangle\big)^3.$\\\hline\hline
%& -\frac{1}{2
%{\sqrt{2}}}\big(|-1,1,-1,1,-1,1\rangle +
% |-1,1,1,-1,1,-1\rangle +
% |1,-1,-1,1,1,-1\rangle +$\\&&$+
% |1,-1,1,-1,-1,1\rangle\big)+
 %\frac{1}{2 {\sqrt{2}}}\big(|-1,1,-1,1,1,-1\rangle +
 %|-1,1,1,-1,-1,1\rangle +$\\&&$+
 %|1,-1,-1,1,-1,1\rangle +
 %|1,-1,1,-1,1,-1\rangle\big)

\end{tabular}
\caption{First singlet states of $N$ spin-${1\over 2}$ particles.
\label{2005-singlet-t12}}
\end{table}
\clearpage

\begin{longtable}{ccccc}
%\begin{tabular}{ccccc}
\hline\hline N & \# & \\\hline\hline
2&1&$\frac{1}{{\sqrt{3}}}\big(-|0,0\rangle+|-1,1\rangle+|1,-1\rangle\big);$\\\hline


3&1&$
-\frac{1}{{\sqrt{6}}}\big(|-1,0,1\rangle+|0,1,-1\rangle+|1,-1,0\rangle\big)+$\\&&$
+\frac{1}{{\sqrt{6}}}\big(|-1,1,0\rangle+|0,-1,1\rangle+|1,0,-1\rangle\big);$\\\hline

4&1&$ -\frac{1}{2 {\sqrt{5}}}\big(|-1,0,0,1\rangle+
|-1,0,1,0\rangle+|0,-1,0,1\rangle+
|0,-1,1,0\rangle+$\\&&$+|0,1,-1,0\rangle+
|0,1,0,-1\rangle+|1,0,-1,0\rangle+
|1,0,0,-1\rangle\big)+$\\&&$+\frac{1}{6
{\sqrt{5}}}\big(|-1,1,-1,1\rangle+
|-1,1,1,-1\rangle+|1,-1,-1,1\rangle+
|1,-1,1,-1\rangle\big)+$\\&&$+\frac{1}{3
{\sqrt{5}}}\big(|-1,1,0,0\rangle+
|0,0,-1,1\rangle+|0,0,1,-1\rangle+
|1,-1,0,0\rangle\big)+$\\&&$+\frac{2}{3
{\sqrt{5}}}|0,0,0,0\rangle+
\frac{1}{{\sqrt{5}}}\big(|-1,-1,1,1\rangle+|1,1,-1,-1\rangle\big);$\\

4&2&$-\frac{1}{2
{\sqrt{3}}}\big(|-1,0,1,0\rangle+|-1,1,-1,1\rangle+
|0,-1,0,1\rangle+|0,1,0,-1\rangle+$\\&&$+
|1,-1,1,-1\rangle+|1,0,-1,0\rangle\big)+\frac{1}{2
{\sqrt{3}}}\big(|-1,0,0,1\rangle+|-1,1,1,-1\rangle+$\\&&$+
|0,-1,1,0\rangle+|0,1,-1,0\rangle+
|1,-1,-1,1\rangle+|1,0,0,-1\rangle\big);$\\
4&3&$\big(\frac{1}{{\sqrt{3}}}\big(-|0,0\rangle+|-1,1\rangle+|1,-1\rangle\big)\big)^2
%&-\frac{1}{3}(|-1,1,0,0\rangle+|0,0,-1,1\rangle+|0,0,1,-1\rangle+
%|1,-1,0,0\rangle\big)+$\\&&$+\frac{1}{3}\big(|-1,1,-1,1\rangle+|-1,1,1,-1\rangle+
%|0,0,0,0\rangle+|1,-1,-1,1\rangle+|1,-1,1,-1\rangle\big)
;$\\\hline 5&1&$-{\sqrt{\frac{2}{15}}}|-1,-1,0,1,1\rangle+
-\frac{1}{{\sqrt{30}}}\big(|-1,0,1,0,0\rangle+|0,-1,1,0,0\rangle+$\\&&$+
|0,0,-1,0,1\rangle+|0,0,-1,1,0\rangle+
|0,1,1,-1,-1\rangle+$\\&&$+|1,0,1,-1,-1\rangle+
|1,1,-1,-1,0\rangle+|1,1,-1,0,-1\rangle\big)+$\\&&$+ -\frac{1}{2
{\sqrt{30}}}\big(|-1,0,1,-1,1\rangle+|-1,0,1,1,-1\rangle+
|-1,1,-1,0,1\rangle+$\\&&$+|-1,1,-1,1,0\rangle+
|0,-1,1,-1,1\rangle+|0,-1,1,1,-1\rangle+$\\&&$+
|0,1,0,-1,0\rangle+|0,1,0,0,-1\rangle+
|1,-1,-1,0,1\rangle+$\\&&$+|1,-1,-1,1,0\rangle+
|1,0,0,-1,0\rangle+|1,0,0,0,-1\rangle\big)+$\\&&$+ \frac{1}{2
{\sqrt{30}}}\big(|-1,0,0,0,1\rangle+|-1,0,0,1,0\rangle+
|-1,1,1,-1,0\rangle+$\\&&$+|-1,1,1,0,-1\rangle+
|0,-1,0,0,1\rangle+|0,-1,0,1,0\rangle+$\\&&$+
|0,1,-1,-1,1\rangle+|0,1,-1,1,-1\rangle+
|1,-1,1,-1,0\rangle+$\\&&$+|1,-1,1,0,-1\rangle+
|1,0,-1,-1,1\rangle+|1,0,-1,1,-1\rangle\big)+$\\&&$+
\frac{1}{{\sqrt{30}}}\big(|-1,-1,1,0,1\rangle+|-1,-1,1,1,0\rangle+
|-1,0,-1,1,1\rangle+$\\&&$+|0,-1,-1,1,1\rangle+
|0,0,1,-1,0\rangle+|0,0,1,0,-1\rangle+$\\&&$+
|0,1,-1,0,0\rangle+|1,0,-1,0,0\rangle\big)+{\sqrt{\frac{2}{15}}}|1,1,0,-1,-1\rangle;$\\\hline

6&7&$-\frac{1}{{\sqrt{15}}}\big(|-1,-1,0,1,1,0\rangle+|1,1,0,-1,-1,0\rangle)+
-\frac{1}{2 {\sqrt{15}}}(|-1,-1,1,0,0,1\rangle+$\\&&$+
|-1,-1,1,1,-1,1\rangle+ |-1,0,-1,1,0,1\rangle+
|-1,0,1,0,1,-1\rangle+$\\&&$+ |0,-1,-1,1,0,1\rangle+
|0,-1,1,0,1,-1\rangle+ |0,0,-1,0,1,0\rangle+$\\&&$+
|0,0,-1,1,1,-1\rangle+ |0,0,1,-1,-1,1\rangle+
|0,0,1,0,-1,0\rangle+$\\&&$+ |0,1,-1,0,-1,1\rangle+
|0,1,1,-1,0,-1\rangle+ |1,0,-1,0,-1,1\rangle+$\\&&$+
|1,0,1,-1,0,-1\rangle+ |1,1,-1,-1,1,-1\rangle+
|1,1,-1,0,0,-1\rangle\big)+$\\&&$+ -\frac{1}{4
{\sqrt{15}}}\big(|-1,0,0,0,0,1\rangle+ |-1,0,0,1,-1,1\rangle+
|-1,0,1,-1,1,0\rangle+$\\&&$+ |-1,0,1,1,0,-1\rangle+
|-1,1,-1,0,1,0\rangle+ |-1,1,-1,1,1,-1\rangle+$\\&&$+
|-1,1,1,-1,-1,1\rangle+ |-1,1,1,0,-1,0\rangle+
|0,-1,0,0,0,1\rangle+$\\&&$+ |0,-1,0,1,-1,1\rangle+
|0,-1,1,-1,1,0\rangle+ |0,-1,1,1,0,-1\rangle+$\\&&$+
|0,1,-1,-1,0,1\rangle+ |0,1,-1,1,-1,0\rangle+
|0,1,0,-1,1,-1\rangle+$\\&&$+ |0,1,0,0,0,-1\rangle+
|1,-1,-1,0,1,0\rangle+ |1,-1,-1,1,1,-1\rangle+$\\&&$+
|1,-1,1,-1,-1,1\rangle+ |1,-1,1,0,-1,0\rangle+
|1,0,-1,-1,0,1\rangle+$\\&&$+ |1,0,-1,1,-1,0\rangle+
|1,0,0,-1,1,-1\rangle+ |1,0,0,0,0,-1\rangle\big)+$\\&&$+
\frac{1}{4
{\sqrt{15}}}\big(|-1,0,0,0,1,0\rangle+|-1,0,0,1,1,-1\rangle+
|-1,0,1,-1,0,1\rangle+$\\&&$+ |-1,0,1,1,-1,0\rangle+
|-1,1,-1,0,0,1\rangle+ |-1,1,-1,1,-1,1\rangle+$\\&&$+
|-1,1,1,-1,1,-1\rangle+ |-1,1,1,0,0,-1\rangle+
|0,-1,0,0,1,0\rangle+$\\&&$+|0,-1,0,1,1,-1\rangle+
|0,-1,1,-1,0,1\rangle+ |0,-1,1,1,-1,0\rangle+$\\&&$+
|0,1,-1,-1,1,0\rangle+ |0,1,-1,1,0,-1\rangle+
|0,1,0,-1,-1,1\rangle+$\\&&$+
|0,1,0,0,-1,0\rangle+|1,-1,-1,0,0,1\rangle+
|1,-1,-1,1,-1,1\rangle+$\\&&$+ |1,-1,1,-1,1,-1\rangle+
|1,-1,1,0,0,-1\rangle+ |1,0,-1,-1,1,0\rangle+$\\&&$+
|1,0,-1,1,0,-1\rangle+ |1,0,0,-1,-1,1\rangle+
|1,0,0,0,-1,0\rangle\big)+$\\&&$+\frac{1}{2
{\sqrt{15}}}\big(|-1,-1,1,0,1,0\rangle+ |-1,-1,1,1,1,-1\rangle+
|-1,0,-1,1,1,0\rangle+$\\&&$+ |-1,0,1,0,-1,1\rangle+
|0,-1,-1,1,1,0\rangle+ |0,-1,1,0,-1,1\rangle+$\\&&$+
|0,0,-1,0,0,1\rangle+|0,0,-1,1,-1,1\rangle+
|0,0,1,-1,1,-1\rangle+$\\&&$+
|0,0,1,0,0,-1\rangle+|0,1,-1,0,1,-1\rangle+
|0,1,1,-1,-1,0\rangle+$\\&&$+ |1,0,-1,0,1,-1\rangle+
|1,0,1,-1,-1,0\rangle+ |1,1,-1,-1,-1,1\rangle+$\\&&$+
|1,1,-1,0,-1,0\rangle\big)+
\frac{1}{{\sqrt{15}}}\big(|-1,-1,0,1,0,1\rangle+|1,1,0,-1,0,-1\rangle\big).$\\
\hline\hline\\
%\end{tabular}
\caption{First singlet states of $N$ spin-${1}$ particles.
\label{2005-singlet-t1}}
\end{longtable}



%%%%%%%%%%%%%%%%%%%%%%%%%%%%%%%%%%%%%%%%%%%%%%%%%%%%%%%%%%%%%%%%%%%%%%%%%%


\section{Symmetries}
In what follows we shall discuss the symmetry behavior of singlet
states.
In our approach the singlet states are orthogonal to each other.
This can be demonstrated by considering the formula
\begin{align}
\langle(j_1'j_2')jm|(j_1j_2)jm\rangle =&\sum_{m_1'+m_2'=m,
m_1+m_2=m}\langle(j_1'j_2')jm|j_1'm_1'j_2'm_2'\rangle\times\nonumber\\
&\langle j_1'm_1'j_2'm_2'|j_1m_1j_2m_2\rangle\langle
j_1m_1j_2m_2|(j_1j_2)jm\rangle\\=&\delta_{j_1j_1'}\delta_{j_2j_2'}\delta_{m_1m_1'}\delta_{m_2m_2'}.\nonumber
%\sqrt{2j_3+1}(-1)^{j_1-j_2-m_3}\left(\begin{array}{ccc}
%j_1&j_2&j_3\\m_1&m_2&m_3\end{array}\right),
\end{align}
States stemming from different
$j_1$ values are orthogonal to each other. Hence, also the singlet
states derived from them are orthogonal. By iteration it follows
that even singlet states stemming from the same $j_1$ are
orthogonal.
The method allows us to construct the full basis for each singlet space
which has the appropriate dimension.


\subsection{Symmetries of the states with respect to changing
all magnetic quantum numbers into their negative}

%Further symmetries are revealed by considering the symmetry
%relations of Wigner's $3j$-symbol.
%\subsection{\bf Symmetry relations of Wigner's 3j-symbol}
%Instead of the Clebsch-Gordan coefficients, Wigner introduced a symbolic representation
%of the coupling of two spins which shows enhanced symmetry.
%These symbols are defined by
For the Clebsch-Gordan coefficients the following formula holds
\begin{align}\langle j_1,-m_1,j_2,-m_2|j,-m\rangle
=(-1)^{j_1+j_2-j}\langle j_1m_1j_2m_2|jm\rangle.\end{align} Let us
consider the $j=1$ case first. The symmetry described above
implies for the coupling of $j$ to $j+1$:
\begin{align}\langle j,-m-1,1,1|j+1,-m\rangle
= &(-1)^{0}\langle j,m+1,1,-1|j+1,m\rangle\nonumber\\
\langle j,-m,1,0|j+1,-m\rangle = &(-1)^{0}\langle
j,m,1,0|j+1,m\rangle,\end{align} i.e. the Clebsch-Gordan
coefficients are the same.
%\langle j,m+1,1,-1|j+1,m\rangle = &(-1)^{0}\langle
%j,-m-1,1,1|j+1,-m\rangle\nonumber.\end{align} As the overall sign
%is plus, we see that all these Clebsch-Gordan coefficients stay
%the same under a change of all the magnetic quantum numbers into
%their negatives.
For the coupling of $j$ to $j$,
\begin{align}\langle j,-m-1,1,1|j,-m\rangle
= &(-1)^{1}\langle j,m+1,1,-1|j,m\rangle\nonumber\\
\langle j,-m,1,0|j,-m\rangle = &(-1)^{1}\langle
j,m,1,0|j,m\rangle,\end{align} i.e. all Clebsch-Gordan
coefficients change their sign.
%\\\langle j,m+1,1,-1|j,m\rangle = &(-1)^{1}\langle
%j,-m-1,1,1|j,-m\rangle\nonumber.\end{align} Thus, all these
%Clebsch-Gordan coefficients change their signs if all the magnetic
%quantum numbers are changed into their negatives.
Similarly for the coupling of $j+1$ to $j$,
\begin{align}\langle j+1,m,1,1|j,m+1\rangle
= &(-1)^{2}\langle j+1,-m,1,-1|j,-m-1\rangle\nonumber\\
\langle j+1,m,1,0|j,m\rangle = &(-1)^{2}\langle
j+1,-m,1,0|j,-m\rangle,\end{align} i.e. they all stay the same.
%\\\langle j+1,-m,1,-1|j,-m-1\rangle = &(-1)^{2}\langle
%j+1,m,1,1|j,m+1\rangle\nonumber.\end{align} Thus, all these
%Clebsch-Gordan coefficients stay the same under a change of all
%the magnetic quantum numbers into their negatives.\\\\

%Thus we follow for coupling $j$ to $j+1$ and $j+1$ to $j$ that if
%we negate all magnetic quantum numbers all Clebsch-Gordan
%coefficients stay the same. For coupling $j$ to $j$ all the
%Clebsch-Gordan change their signs.\\\\
Using these symmetries, we conclude that the symmetry behavior
stays the same if one goes from the angular momentum subspace
$|N,j\rangle$ to the angular momentum subspace $|N+1,j+1\rangle$.
The symmetry behavior does not change for coupling $|N,j+1\rangle$
to $|N+1,j\rangle$.
%If one goes from the angular
%momentum subspace $|N,j\rangle$ to the angular momentum subspace
%$|N+1,j\rangle$ it changes from even to odd and from odd to even,
%respectively.
Coupling $|N,j\rangle$ to $|N+1,j\rangle$ changes the symmetry behaviour from even to
odd and from odd to even.
%respectively.
The situation is depicted in
Fig.~\ref{2005-singlet-f1-ta-takohalf}.
$N$-particle singlet states with $N$ even are even,
whereas $N$-particle singlet states with $N$ odd are odd.
\begin{figure}
\begin{center}
\unitlength 0.50mm \linethickness{0.4pt}
\begin{picture}(290.00,150.00)
\put(15.00,5.00){\makebox(0,0)[cc]{$0$}}
\put(45.00,5.00){\makebox(0,0)[cc]{$1$}}
\put(75.00,5.00){\makebox(0,0)[cc]{$2$}}
\put(105.0,5.00){\makebox(0,0)[cc]{$3$}}
\put(135.00,5.00){\makebox(0,0)[cc]{$4$}}
\put(165.00,5.00){\makebox(0,0)[cc]{$5$}}
\put(195.00,5.00){\makebox(0,0)[cc]{$6$}}
\put(225.00,5.00){\makebox(0,0)[cc]{$7$}}
\put(255.0,5.00){\makebox(0,0)[cc]{$8$}}
\put(285.00,5.00){\makebox(0,0)[cc]{$9$}}
\put(315.00,5.00){\makebox(0,0)[cc]{$10$}}
\put(330.00,5.00){\makebox(0,0)[cc]{$N$}}
\put(5.00,15.00){\makebox(0,0)[cc]{$0$}}
\put(5.00,45.00){\makebox(0,0)[cc]{$1$}}
\put(5.00,75.00){\makebox(0,0)[cc]{$2$}}
\put(5.00,105.00){\makebox(0,0)[cc]{$3$}}
\put(5.00,135.00){\makebox(0,0)[cc]{$4$}}
\put(5.00,165.00){\makebox(0,0)[cc]{$5$}}
\put(5.00,180.00){\makebox(0,0)[cc]{$j$}}
\put(10.00,10.00){\line(0,1){160.00}}
\put(10.00,10.00){\line(1,0){310.00}}
%\put(15.00,15.00){\line(1,1){125.00}}
%\put(15.00,15.00){\circle*{4.00}}
\put(75.00,15.00){\circle*{4.00}} \put(75.00,15.00){\circle{8.00}}
\put(105.00,15.00){\circle*{4.00}}
 \put(135.00,15.00){\circle*{4.00}}
\put(135.00,15.00){\circle{8.00}}
 \put(165.00,15.00){\circle*{4.00}}
\put(195.00,15.00){\circle*{4.00}}
 \put(195.00,15.00){\circle{8.00}}
 \put(225.00,15.00){\circle*{4.00}}
  \put(255.00,15.00){\circle*{4.00}}
\put(255.00,15.00){\circle{8.00}}
 \put(285.00,15.00){\circle*{4.00}}
\put(315.00,15.00){\circle*{4.00}}
 \put(315.00,15.00){\circle{8.00}}
\put(45.00,45.00){\circle*{4.00}}
 \put(45.00,45.00){\circle{8.00}}
 \put(75.00,45.00){\circle*{4.00}}
\put(105.00,45.00){\circle*{4.00}}
\put(105.00,45.00){\circle{8.00}}
 \put(135.00,45.00){\circle*{4.00}}
\put(165.00,45.00){\circle*{4.00}}
 \put(165.00,45.00){\circle{8.00}}
 \put(195.00,45.00){\circle*{4.00}}
\put(225.00,45.00){\circle{8.00}}
 \put(225.00,45.00){\circle*{4.00}}
\put(255.00,45.00){\circle*{4.00}}
 \put(285.00,45.00){\circle*{4.00}}
  \put(285.00,45.00){\circle{8.00}}
\put(75.00,75.00){\circle*{4.00}} \put(75.00,75.00){\circle{8.00}}
\put(105.00,75.00){\circle*{4.00}}
 \put(135.00,75.00){\circle*{4.00}}
\put(135.00,75.00){\circle{8.00}}
 \put(165.00,75.00){\circle*{4.00}}
\put(195.00,75.00){\circle*{4.00}}
 \put(195.00,75.00){\circle{8.00}}
 \put(225.00,75.00){\circle*{4.00}}
\put(255.00,75.00){\circle*{4.00}}
 \put(255.00,75.00){\circle{8.00}}
\put(105.00,105.00){\circle*{4.00}}
\put(105.00,105.00){\circle{8.00}}
 \put(135.00,105.00){\circle*{4.00}}
\put(165.00,105.00){\circle*{4.00}}
 \put(165.00,105.00){\circle{8.00}}
  \put(195.00,105.00){\circle*{4.00}}
\put(225.00,105.00){\circle{8.00}}
 \put(225.00,105.00){\circle*{4.00}}
  \put(135.00,135.00){\circle*{4.00}}
\put(135.00,135.00){\circle{8.00}}
 \put(165.00,135.00){\circle*{4.00}}
\put(195.00,135.00){\circle*{4.00}}
 \put(195.00,135.00){\circle{8.00}}
\put(165.00,165.00){\circle*{4.00}}
 \put(165.00,165.00){\circle{8.00}}


\put(75,15){\vector(1,1){28}}
\put(105,15){\vector(1,1){28}}\put(135,15){\vector(1,1){28}}
\put(165,15){\vector(1,1){28}}\put(195,15){\vector(1,1){28}}
\put(225,15){\vector(1,1){28}}\put(255,15){\vector(1,1){28}}
 \put(45,45){\vector(1,1){28}}
\put(45,45){\vector(1,0){28}} \put(45,45){\vector(1,-1){28}}
\put(75,45){\vector(1,1){28}} \put(75,45){\vector(1,0){28}}
\put(75,45){\vector(1,-1){28}} \put(105,45){\vector(1,1){28}}
\put(105,45){\vector(1,0){28}}
\put(105,45){\vector(1,-1){28}}\put(135,45){\vector(1,1){28}}
\put(135,45){\vector(1,0){28}}
\put(135,45){\vector(1,-1){28}}\put(165,45){\vector(1,1){28}}
\put(165,45){\vector(1,0){28}} \put(165,45){\vector(1,-1){28}}
\put(195,45){\vector(1,1){28}} \put(195,45){\vector(1,0){28}}
\put(195,45){\vector(1,-1){28}} \put(225,45){\vector(1,1){28}}
\put(225,45){\vector(1,0){28}} \put(225,45){\vector(1,-1){28}}
\put(255,45){\vector(1,0){28}} \put(255,45){\vector(1,-1){28}}
\put(285,45){\vector(1,-1){28}}
 \put(75,75){\vector(1,1){28}}
\put(75,75){\vector(1,0){28}} \put(75,75){\vector(1,-1){28}}
\put(105,75){\vector(1,1){28}} \put(105,75){\vector(1,0){28}}
\put(105,75){\vector(1,-1){28}}\put(135,75){\vector(1,1){28}}
\put(135,75){\vector(1,0){28}}
\put(135,75){\vector(1,-1){28}}\put(165,75){\vector(1,1){28}}
\put(165,75){\vector(1,0){28}} \put(165,75){\vector(1,-1){28}}
\put(195,75){\vector(1,1){28}} \put(195,75){\vector(1,0){28}}
\put(195,75){\vector(1,-1){28}} \put(225,75){\vector(1,0){28}}
\put(225,75){\vector(1,-1){28}} \put(225,75){\vector(1,-1){28}}
\put(255,75){\vector(1,-1){28}} \put(105,105){\vector(1,1){28}}
\put(105,105){\vector(1,0){28}}
\put(105,105){\vector(1,-1){28}}\put(135,105){\vector(1,1){28}}
\put(135,105){\vector(1,0){28}}
\put(135,105){\vector(1,-1){28}}\put(165,105){\vector(1,1){28}}
\put(165,105){\vector(1,0){28}} \put(165,105){\vector(1,-1){28}}
\put(195,105){\vector(1,0){28}} \put(195,105){\vector(1,-1){28}}
\put(195,105){\vector(1,-1){28}} \put(225,105){\vector(1,-1){28}}
\put(135,135){\vector(1,1){28}} \put(135,135){\vector(1,0){28}}
\put(135,135){\vector(1,-1){28}} \put(165,135){\vector(1,0){28}}
\put(165,135){\vector(1,-1){28}}
\put(195,135){\vector(1,-1){28}}\put(165,165){\vector(1,-1){28}}



\put(81,15)   {\makebox(10,5){\bf $1$}}
\put(111,15)  {\makebox(10,5){\it $1$}}
\put(141,15)  {\makebox(10,5){\bf $3$}}
\put(171,15)  {\makebox(10,5){\it $6$}}
\put(203,15)  {\makebox(10,5){\bf $15$}}
\put(233,15)  {\makebox(10,5){\it $36$}}
\put(263,15)  {\makebox(10,5){\bf $91$}}
\put(293,15)  {\makebox(10,5){\it $232$}}
\put(324,15)  {\makebox(10,5){\bf $603$}}
\put(53,45)   {\makebox(10,7){\bf $1.3$}}
\put(83,45)   {\makebox(10,7){\it $1.3$}}
\put(113,45)  {\makebox(10,7){\bf $3.3$}}
\put(143,45)  {\makebox(10,7){\it $6.3$}}
\put(173,45)  {\makebox(10,7){\bf $15.3$}}
\put(203,45)  {\makebox(10,7){\it $36.3$}}
\put(233,45)  {\makebox(10,7){\bf $91.3$}}
\put(263,45)  {\makebox(10,7){\it $232.3$}}
\put(296,45)  {\makebox(10,7){\bf $603.3$}}
\put(83,75)   {\makebox(10,7){\bf $1.5$}}
\put(113,75)  {\makebox(10,7){\it $2.5$}}
\put(143,75)  {\makebox(10,7){\bf $6.5$}}
\put(173,75)  {\makebox(10,7){\it $15.5$}}
\put(203,75)  {\makebox(10,7){\bf $40.5$}}
\put(233,75)  {\makebox(10,7){\it $105.5$}}
\put(262,75)  {\makebox(10,7){\bf $280.5$}}
\put(113,105){\makebox(10,7){\bf $1.7$}}
\put(143,105){\makebox(10,7){\it $3.7$}}
\put(173,105){\makebox(10,7){\bf $10.7$}}
\put(203,105){\makebox(10,7){\it $29.7$}}
\put(233,105){\makebox(10,7){\bf $84.7$}}
\put(143,135){\makebox(10,7){\bf $1.9$}}
\put(173,135){\makebox(10,7){\it $4.9$}}
\put(203,135){\makebox(10,7){\bf $15.9$}}
\put(173,165){\makebox(10,7){\bf $1.11$}}


%\put(225.00,145.00){\circle*{4.00}}
%\put(225.00,145.00){\circle{8.00}}
%\put(225.00,130.00){\circle*{4.00}}
%\put(230,145){\makebox(0,0)[l]{ Number of symmetric subspaces}}
%\put(230,130){\makebox(0,0)[l]{ Number of antisymmetric subspaces}}


%\put(15.00,15.00){\vector(1,1){28.00}}


\end{picture}
\end{center}
\caption{Symmetries of spin-$1$ particle states.
\label{2005-singlet-f1-ta-takohalf} Even subspaces are denoted by
concentric circles, odd subspaces are denoted by filled circles.
The numbers denote the dimension of the subspaces. The first
number stands for the number of states $|h,j\rangle$ and the
second stands for the $2j+1$ projections. Arrows represent the way
of coupling. }
\end{figure}


%\subsubsection{Application of the symmetries to the $j=\frac{1}{2}$ case}%
Next we turn to the $j=\frac{1}{2}$ case.
For the coupling of $j$ to $j+\frac{1}{2}$, the  Clebsch-Gordan
coefficients are
\begin{align}
\langle
j,-m-\frac{1}{2},\frac{1}{2},\frac{1}{2}|j+\frac{1}{2},-m\rangle =
&(-1)^{0}\langle
j,m+\frac{1}{2},\frac{1}{2},-\frac{1}{2}|j+\frac{1}{2},m\rangle\nonumber \\
\langle
j,m+\frac{1}{2},\frac{1}{2},-\frac{1}{2}|j+\frac{1}{2},m\rangle =
&(-1)^{0}\langle
j,-m-\frac{1}{2},\frac{1}{2},\frac{1}{2}|j+\frac{1}{2},-m\rangle.
\end{align}
%We conclude that all these Clebsch-Gordan coefficients stay the
%same under a change of all magnetic quantum numbers into their
%negatives.
If we negate all the magnetic quantum numbers, all these
Clebsch-Gordan coefficients stay the same. For the coupling of
$j+\frac{1}{2}$ to $j$,
\begin{align}
\langle
j+\frac{1}{2},m,\frac{1}{2},\frac{1}{2}|j,m+\frac{1}{2}\rangle =
&(-1)^{1}\langle
j+\frac{1}{2},-m,\frac{1}{2},-\frac{1}{2}|j,-m-\frac{1}{2}\rangle\nonumber\\
\langle
j+\frac{1}{2},-m,\frac{1}{2},-\frac{1}{2}|j,-m-\frac{1}{2}\rangle
= &(-1)^{1}\langle
j+\frac{1}{2},m,\frac{1}{2},\frac{1}{2}|j,m+\frac{1}{2}\rangle.
\end{align}
%Thus, all these Clebsch-Gordan coefficients change their signs if
%all the magnetic quantum numbers are changed into their
%negatives.\\\\
Here all the Clebsch-Gordan coefficients change their signs.


We conclude that the symmetry behavior stays the same if one goes
from the angular momentum subspace $|N,j\rangle$ to the angular
momentum subspace $|N+1,j+{1\over 2}\rangle$. Going from the
subspace $|N,j\rangle$ to the subspace $|N+1,j-{1\over2}\rangle$,
the symmetry behavior changes from even to odd and from odd to
even, respectively. The situation is depicted in
Fig.~\ref{2005-singlet-f1-ta-tako}. In particular, singlet states
where $N$ is $k\cdot 2\cdot 2$ (k is an integer) are even, and singlet states
where $N$ is $k\cdot 2\cdot (2+1)$ are odd.
\begin{figure}
\begin{center}
\unitlength 0.50mm \linethickness{0.4pt}
\begin{picture}(350.00,200.00)
\put(15.00,5.00){\makebox(0,0)[cc]{$0$}}
\put(45.00,5.00){\makebox(0,0)[cc]{$1$}}
\put(75.00,5.00){\makebox(0,0)[cc]{$2$}}
\put(105.0,5.00){\makebox(0,0)[cc]{$3$}}
\put(135.00,5.00){\makebox(0,0)[cc]{$4$}}
\put(165.00,5.00){\makebox(0,0)[cc]{$5$}}
\put(195.00,5.00){\makebox(0,0)[cc]{$6$}}
\put(225.0,5.00){\makebox(0,0)[cc]{$7$}}
\put(255.00,5.00){\makebox(0,0)[cc]{$8$}}
\put(285.0,5.00){\makebox(0,0)[cc]{$9$}}
\put(315.00,5.00){\makebox(0,0)[cc]{$10$}}
\put(330.00,5.00){\makebox(0,0)[cc]{$N$}}
\put(5.00,15.00){\makebox(0,0)[cc]{$0$}}
\put(5.00,45.00){\makebox(0,0)[cc]{${1\over 2}$}}
\put(5.00,75.00){\makebox(0,0)[cc]{$1$}}
\put(5.00,105.00){\makebox(0,0)[cc]{${3\over 2}$}}
\put(5.00,135.00){\makebox(0,0)[cc]{$2$}}
\put(5.00,165.00){\makebox(0,0)[cc]{${5\over 2}$}}
\put(5.00,180.00){\makebox(0,0)[cc]{$j$}}
\put(10.00,10.00){\line(0,1){160.00}}
\put(10.00,10.00){\line(1,0){310.00}}
%\put(15.00,15.00){\line(1,1){125.00}}
\put(75.00,15.00){\circle*{4.00}}
\put(135.00,15.00){\circle*{4.00}}
\put(135.00,15.00){\circle{8.00}}
\put(195.00,15.00){\circle*{4.00}}
\put(255.00,15.00){\circle*{4.00}}
\put(255.00,15.00){\circle{8.00}}
\put(315.00,15.00){\circle*{4.00}}
\put(45.00,45.00){\circle*{4.00}} \put(45.00,45.00){\circle{8.00}}
\put(105.00,45.00){\circle*{4.00}}
\put(165.00,45.00){\circle*{4.00}}
\put(165.00,45.00){\circle{8.00}}
\put(225.00,45.00){\circle*{4.00}}
\put(285.00,45.00){\circle*{4.00}}
\put(285.00,45.00){\circle{8.00}}
\put(75.00,75.00){\circle*{4.00}}
 \put(75.00,75.00){\circle{8.00}}
\put(135.00,75.00){\circle*{4.00}}
\put(195.00,75.00){\circle*{4.00}}
\put(195.00,75.00){\circle{8.00}}
\put(255.00,75.00){\circle*{4.00}}
\put(105.00,105.00){\circle*{4.00}}
\put(105.00,105.00){\circle{8.00}}
\put(165.00,105.00){\circle*{4.00}}
\put(225.00,105.00){\circle*{4.00}}
\put(225.00,105.00){\circle{8.00}}
\put(135.00,135.00){\circle*{4.00}}
\put(135.00,135.00){\circle{8.00}}
\put(195.00,135.00){\circle*{4.00}}
\put(165.00,165.00){\circle*{4.00}}
\put(165.00,165.00){\circle{8.00}}


\put(75.00,15.00){\vector(1,1){28.00}}
\put(135.00,15.00){\vector(1,1){28.00}}
\put(195.00,15.00){\vector(1,1){28.00}}
\put(255.00,15.00){\vector(1,1){28.00}}
\put(45.00,45.00){\vector(1,-1){28.00}}
\put(45.00,45.00){\vector(1,1){28.00}}
\put(105.00,45.00){\vector(1,-1){28.00}}
\put(105.00,45.00){\vector(1,1){28.00}}
\put(165.00,45.00){\vector(1,-1){28.00}}
\put(165.00,45.00){\vector(1,1){28.00}}
\put(225.00,45.00){\vector(1,-1){28.00}}
\put(225.00,45.00){\vector(1,1){28.00}}
\put(285.00,45.00){\vector(1,-1){28.00}}
\put(75.00,75.00){\vector(1,-1){28.00}}
\put(75.00,75.00){\vector(1,1){28.00}}
\put(135.00,75.00){\vector(1,-1){28.00}}
\put(135.00,75.00){\vector(1,1){28.00}}
\put(195.00,75.00){\vector(1,-1){28.00}}
\put(195.00,75.00){\vector(1,1){28.00}}
\put(255.00,75.00){\vector(1,-1){28.00}}
\put(105.00,105.00){\vector(1,-1){28.00}}
\put(105.00,105.00){\vector(1,1){28.00}}
\put(165.00,105.00){\vector(1,-1){28.00}}
\put(165.00,105.00){\vector(1,1){28.00}}
\put(225.00,105.00){\vector(1,-1){28.00}}
\put(135.00,135.00){\vector(1,-1){28.00}}
\put(135.00,135.00){\vector(1,1){28.00}}
\put(195.00,135.00){\vector(1,-1){28.00}}
\put(165.00,165.00){\vector(1,-1){28.00}}


\put(80,15){\makebox(10,0){\it $1$}}
 \put(140,15){\makebox(10,0){\bf $2$}}
 \put(200,14){\makebox(10,0){\it $5$}}
\put(262,15){\makebox(10,0){\bf $14$}}
 \put(322,15){\makebox(10,0){\it $42$}}
 \put(52,45){\makebox(10,0){\bf $1.2$}}
\put(112,45){\makebox(10,0){\it $2.2$}}
 \put(172,45){\makebox(10,0){\bf $5.2$}}
 \put(232,45){\makebox(10,0){\it $14.2$}}
\put(294,45){\makebox(10,0){\bf $42.2$}}
 \put(82,75){\makebox(10,0){\bf $1.3$}}
 \put(142,75){\makebox(10,0){\it $3.3$}}
\put(202,75){\makebox(10,0){\bf $9.3$}}
 \put(262,75){\makebox(10,0){\it $28.3$}}
 \put(112,105){\makebox(10,0){\bf $1.4$}}
\put(172,105){\makebox(10,0){\it $4.4$}}
 \put(232,105){\makebox(10,0){\bf $14.4$}}
 \put(142,135){\makebox(10,0){\bf $1.5$}}
\put(202,135){\makebox(10,0){\it $5.5$}}
\put(172,165){\makebox(10,0){\bf $1.6$}}
%\put(225.00,145.00){\circle*{4.00}}
%\put(225.00,145.00){\circle{8.00}}
%\put(225.00,130.00){\circle*{4.00}}
%\put(230,145){\makebox(0,0)[l]{Number of symmetric subspaces}}
%\put(230,130){\makebox(0,0)[l]{Number of antisymmetric subspaces}}



\end{picture}
\end{center}
\caption{Symmetries of spin-${1\over 2}$ particles.
\label{2005-singlet-f1-ta-tako} Even subspaces are denoted by
concentric circles, odd subspaces are denoted by filled circles.
The numbers denote the dimension of the subspaces. The first
number stands for the number of states $|h,j\rangle$ and the
second stands for the $2j+1$ projections. Arrows represent the way
of coupling.}
\end{figure}




\subsection{Symmetries of the states obtained from consideration of
the symmetric group}We want to assign the appropriate
representation to irreducible singlet spaces. Therefore we
consider the symmetric group. In every product state of every
$N$-particle state we permute the $N$ magnetic quantum numbers,
more explicitly, we apply $(N-1)$ transpositions, since every
permutation of $N$ particles can be written as the product of
(N-1) transpositions. We analyse $(N-1)$ transpositions of the
form $(j,j+1)$, the transposition of $j$ and $j+1$, which generate
the whole symmetric group and in particular all the $N(N-1)/2$ transpositions,
since $(j,k+1)=(k,k+1)(j,k)(k,k+1).$
%~\cite{fripertinger}.
Hence we consider the class $(21^{N-2})$. Each irreducible
representation can be labelled by an ordered partition of integers
which corresponds to a specific Young diagram.



\subsubsection{Application of the symmetries to the $j={1\over 2}$ case}
As stated in App.~D, Sec.~14 of Ref.~\cite{messiah-62}, the
space spanned by the vectors of total spins $(SM)$ formed by $N$
identical spins ${1\over 2}$ is associated with an irreducible
representation of $S_N$, the representation whose Young diagram
corresponds to the partition $[{1\over 2}N+S,{1\over 2}N-S]$ of
the integer $N$.  It is apparent that the Young tableaux for the
irreducible components of the representation of $S_N$ have at most
two lines. For $N>2$, any state contains at least two individual
spins in the same state.
Suppose the state contains the factor
$u_+^{(i)}u_+^{(j)}$, i.e. $m_i, m_j ={1\over 2}$.
Since
$A={1\over 2}(1-(i,j))$ is the antisymmetrizer and
${1\over 2}(1-(i,j))u_+^{(i)}u_+^{(j)}=0$,
it follows that $A|jm\rangle=0$.

Using the theorem mentioned above, the Young diagrams of the
irreducible spaces of the $N$-particle singlet states correspond to
the partitions $[{1\over 2}N,{1\over 2}N]$. Hence the two-particle
singlet state is an antisymmetric one dimensional space. The four-
and six-particle singlet spaces form a two and a five dimensional
irreducible space whose Young diagrams are of the form $[2,2]$ and
$[3,3]$.
Using the formula for the dimension of an irreducible
representation having the partition $[\lambda]$ (e.g., Ref.~\cite{wybourne})
\begin{align}
f^\lambda = n!
\,
\frac{\Pi_{i<j\leq
k}(\lambda_i-\lambda_j+j-i)}{\Pi^k_{i=1}(\lambda_i+k-i)!},\label{dim}\end{align}
 we can verify the dimension.


\subsubsection{Application of the symmetries to the $j=1$ case}
Here, the Young tableaux for the irreducible components of the
representation of $S_N$ have at most three lines. For $N>3$, any
state contains at least two individual spins in the same state,
hence we can argue as above. We calculate the matrix
representations of the
 (N-1) transpositions.

Next we use Schur's Lemma. If a matrix $S$ commutes
with all the matrices of an irreducible representation $G$ of a
group, then it is a multiple of the unit matrix (e.g., App.~D,
Sec.~8 of Ref.~\cite{messiah-62}).


Furthermore, we can calculate the characteristic of each
element. The %spur or %
trace of the matrix representing the element $S_i$ which belongs
to the irreducible representation $\Gamma^{(j)}$ is called the
characteristic of $S_i$ in $\Gamma^{(j)}$ and is denoted by
$\chi^{(j)}(S_i)$. The set of characteristics of all elements $S$
of the group as represented in $\Gamma^{(j)}$ is called a group
character $\chi^{(j)}$. All elements of the same class $\rho$ have
the same characteristic $\chi^{(j)}_\rho$ ~\cite{wybourne}.

The tables of characters can be used to find
the appropriate partition. The results can be verified by calculating the
characteristic of one element per class.
Moreover, consideration of the
outer product restricts the possible irreducible representations. % Considering tables
%(e.g., List of tables, B1 of Ref.~\cite{wybourne}).

The two-particle singlet state is a one dimensional symmetric
space. The three-particle singlet state is an antisymmetric one
dimensional space. Hence their Young diagrams are of the form
$[2]$ and $[1^3]$, respectively. Again we can check the dimension
using Eq.~(\ref{dim}). For four particles the three dimensional
space (see Table~\ref{2005-singlet-t1}) of singlets is reducible
and the class $(21^2)$ has in this representation characteristic
1.

The character of a reducible representation $\Gamma^{(j)}$ is
the sum of the characters of the irreducible parts of
$\Gamma^{(j)}$.
%(e.g., Satz 22 of Ref.~\cite{dolhaine}).
Hence we build the class sum, i.e., the sum of all the matrix
representations of the transpositions.
The calculation of the eigenvalues
and the eigenvectors yields the decomposition of  the space
into a one and a two dimensional subspace. The one dimensional
state is symmetric, denoted by $[4]$. The two dimensional irreducible
representation has characteristic 0 which, using the table
of characters, yields the partition [2,2]. We consider the outer product
of two irreducible representations whose Young diagrams correspond
to the partition [2],  the product of the representations of two two-particle singlet
states; i.e.,
\begin{align} [2]\times[2]=[4]+[3,1]+[2,2].\end{align}
This also yields the appropriate partitions.
There are further
restrictions, such as  $m_1=m_3$. Hence only some of the partitions
are realized.

For five particles, the six dimensional space is
irreducible and has the partition $[3,1^2]$. This can be derived from
the table of characters.
%(e.g.,Ref.~\cite{SYMMETRICA}).
Looking at the outer product of $[1,1,1]\times[2]$, the product of the partitions of the
three- and the two-particle singlet state yields
\begin{align}
[1,1,1]\times[2]=[3,1^2]+[2,1^3].
\end{align}
Here the first partition has the required dimension.

The 15
dimensional space of singlets for six particles is reducible. In
this representation the class $(21^4)$ has characteristic three.
The space decomposes into a one, a five and a nine dimensional
irreducible subspace.
Using the outer product yields for
$[2]\times[2]\times[2]$,  the product of the partitions of the two-particle singlet state to the cube
\begin{align}
[4]\times[2]=[4,2]+[5,1]+[6]\qquad\mbox{and}\qquad
[2,2]\times[2]=[4,2]+[3,2,1]+[2,2,2],\label{dim-6.1}
\end{align}
and for the product of the partitions of the two three-particle singlet states
\begin{align}
[1,1,1]\times[1,1,1]=[1,1,1,1,1,1]+[2,1,1,1,1]+[2,2,1,1]+[2,2,2]\label{dim-6.2}.
\end{align}
The nine dimensional representation has characteristic three.
By using the Eq.~(\ref{dim}) and the table of characters,
one finds
that the appropriate partition is [4,2].
For the five dimensional
irreducible representation we find characteristic $-1$, which leads
to the partition $[2^3]$. Crosschecking with Eq.~(\ref{dim-6.1}) and
Eq.~(\ref{dim-6.2}) we see that, as required, these partitions are
contained in the outer products. Finally, the one dimensional
state has characteristic one which leads to the partition [6];
hence it is a
symmetric state.
The one dimensional irreducible six-particle singlet
state is enumerated in Table~\ref{EV1}.  \clearpage
\begin{longtable}{llllll}
%\begin{tabular}{ccccc}
\hline\hline
&$-\frac{3\sqrt{3}}{7}|0,0,0,0,0,0\rangle +\frac{3
\sqrt{3}}{35}\big(|-1,0,0,0,0,1\rangle +|-1,0,0,0,1,0\rangle
+|-1,0,0,1,0,0\rangle +  $
\\&$ |-1,0,1,0,0,0\rangle
+|-1,1,0,0,0,0\rangle +|0,-1,0,0,0,1\rangle +|0,-1,0,0,1,0 \rangle
+ $\\&$ |0,-1,0,1,0,0\rangle +|0,-1,1,0,0,0\rangle
+|0,0,-1,0,0,1\rangle +|0,0,-1,0,1,0\rangle +  $\\&$
|0,0,-1,1,0,0\rangle +|0,0,0,-1,0,1\rangle +|0,0,0,-1,1,0\rangle
+|0,0,0,0,-1,1\rangle +  $\\&$ |0,0,0,0,1,-1\rangle
+|0,0,0,1,-1,0\rangle +|0,0,0,1,0,-1\rangle +|0,0,1,-1,0,0\rangle
+  $\\&$ |0,0,1,0,-1,0\rangle +|0,0,1,0,0,-1\rangle
+|0,1,-1,0,0,0\rangle +|0,1,0,-1,0,0\rangle +  $\\&$
|0,1,0,0,-1,0\rangle +|0,1,0,0,0,-1\rangle +|1,-1,0,0,0,0\rangle
+|1,0,-1,0,0,0\rangle +  $\\&$ |1,0,0,-1,0,0\rangle
+|1,0,0,0,-1,0\rangle +|1,0,0,0,0,-1\rangle\big) +
-\frac{2\sqrt{3}}{35}\big(|-1,-1,0,0,1,1\rangle +  $\\&$
|-1,-1,0,1,0,1\rangle +|-1,-1,0,1,1,0\rangle
+|-1,-1,1,0,0,1\rangle +|-1,-1,1,0,1,0\rangle +  $\\&$
|-1,-1,1,1,0,0\rangle +|-1,0,-1,0,1,1\rangle
+|-1,0,-1,1,0,1\rangle +|-1,0,-1,1,1,0\rangle +  $\\&$
|-1,0,0,-1,1,1\rangle +|-1,0,0,1,-1,1\rangle
+|-1,0,0,1,1,-1\rangle +|-1,0,1,-1,0,1\rangle +  $\\&$
|-1,0,1,-1,1,0\rangle +|-1,0,1,0,-1,1\rangle
+|-1,0,1,0,1,-1\rangle +|-1,0,1,1,-1,0\rangle +  $\\&$
|-1,0,1,1,0,-1\rangle +|-1,1,-1,0,0,1\rangle
+|-1,1,-1,0,1,0\rangle +|-1,1,-1,1,0,0\rangle +  $\\&$
|-1,1,0,-1,0,1\rangle +|-1,1,0,-1,1,0\rangle
+|-1,1,0,0,-1,1\rangle +|-1,1,0,0,1,-1\rangle +  $\\&$
|-1,1,0,1,-1,0\rangle +|-1,1,0,1,0,-1\rangle
+|-1,1,1,-1,0,0\rangle +|-1,1,1,0,-1,0\rangle +  $\\&$
|-1,1,1,0,0,-1\rangle +|0,-1,-1,0,1,1\rangle
+|0,-1,-1,1,0,1\rangle +|0,-1,-1,1,1,0\rangle +  $\\&$
|0,-1,0,-1,1,1\rangle +|0,-1,0,1,-1,1\rangle
+|0,-1,0,1,1,-1\rangle +|0,-1,1,-1,0,1\rangle +  $\\&$
|0,-1,1,-1,1,0\rangle +|0,-1,1,0,-1,1\rangle
+|0,-1,1,0,1,-1\rangle +|0,-1,1,1,-1,0\rangle +  $\\&$
|0,-1,1,1,0,-1\rangle +|0,0,-1,-1,1,1\rangle
+|0,0,-1,1,-1,1\rangle +|0,0,-1,1,1,-1\rangle +  $\\&$
|0,0,1,-1,-1,1\rangle +|0,0,1,-1,1,-1\rangle
+|0,0,1,1,-1,-1\rangle +|0,1,-1,-1,0,1\rangle +  $\\&$
|0,1,-1,-1,1,0\rangle +|0,1,-1,0,-1,1\rangle
+|0,1,-1,0,1,-1\rangle +|0,1,-1,1,-1,0\rangle +  $\\&$
|0,1,-1,1,0,-1\rangle +|0,1,0,-1,-1,1\rangle
+|0,1,0,-1,1,-1\rangle +|0,1,0,1,-1,-1\rangle +  $\\&$
|0,1,1,-1,-1,0\rangle +|0,1,1,-1,0,-1\rangle
+|0,1,1,0,-1,-1\rangle +|1,-1,-1,0,0,1\rangle +  $\\&$
|1,-1,-1,0,1,0\rangle +|1,-1,-1,1,0,0\rangle
+|1,-1,0,-1,0,1\rangle +|1,-1,0,-1,1,0\rangle +  $\\&$
|1,-1,0,0,-1,1\rangle +|1,-1,0,0,1,-1\rangle
+|1,-1,0,1,-1,0\rangle +|1,-1,0,1,0,-1\rangle +  $\\&$
|1,-1,1,-1,0,0\rangle +|1,-1,1,0,-1,0\rangle
+|1,-1,1,0,0,-1\rangle +|1,0,-1,-1,0,1\rangle +  $\\&$
|1,0,-1,-1,1,0\rangle +|1,0,-1,0,-1,1\rangle
+|1,0,-1,0,1,-1\rangle +|1,0,-1,1,-1,0\rangle +  $\\&$
|1,0,-1,1,0,-1\rangle +|1,0,0,-1,-1,1\rangle
+|1,0,0,-1,1,-1\rangle +|1,0,0,1,-1,-1\rangle +  $\\&$
|1,0,1,-1,-1,0\rangle +|1,0,1,-1,0,-1\rangle
+|1,0,1,0,-1,-1\rangle +|1,1,-1,-1,0,0\rangle +  $\\&$
|1,1,-1,0,-1,0\rangle +|1,1,-1,0,0,-1\rangle
+|1,1,0,-1,-1,0\rangle +|1,1,0,-1,0,-1\rangle +  $\\&$
|1,1,0,0,-1,-1\rangle\big)
+\frac{6\sqrt{3}}{35}\big(|-1,-1,-1,1,1,1\rangle
+|-1,-1,1,-1,1,1\rangle +|-1,-1,1,1,-1,1\rangle +  $\\&$
|-1,-1,1,1,1,-1\rangle +|-1,1,-1,-1,1,1\rangle
+|-1,1,-1,1,-1,1\rangle +|-1,1,-1,1,1,-1\rangle +  $\\&$
|-1,1,1,-1,-1,1\rangle +|-1,1,1,-1,1,-1\rangle
+|-1,1,1,1,-1,-1\rangle +|1,-1,-1,-1,1,1\rangle +  $\\&$
|1,-1,-1,1,-1,1\rangle +|1,-1,-1,1,1,-1\rangle
+|1,-1,1,-1,-1,1\rangle +|1,-1,1,-1,1,-1\rangle +  $\\&$
|1,-1,1,1,-1,-1\rangle +|1,1,-1,-1,-1,1\rangle
+|1,1,-1,-1,1,-1\rangle +|1,1,-1,1,-1,-1\rangle +  $\\&$
|1,1,1,-1,-1,-1\big)$\\
\hline\hline
%\end{tabular}
\caption{The one dimensional irreducible six-particle singlet
state. \label{EV1}}
\end{longtable}

The partitions for up to ten particles are obtained
by using the outer product and neglecting diagrams with more then three
lines. The are enumerated in Table~\ref{pfsttp}.
\begin{table}
\begin{tabular}{ll}
\hline\hline
N \qquad \qquad &partition \\
\hline
 7& [3,3,1],[5,1,1]\\
 8& [8],[6,2],[4,4],[4,2,2] \\
 9& [7,1,1],[5,3,1],[3,3,3]\\
10& [10],[8,2],[6,4],[4,4,2],[6,2,2] \\
\hline\hline
\end{tabular}
\caption{The partitions for seven to ten particles.\label{pfsttp}}
\end{table}

We conjecture that for $N$ even one
obtains all possible diagrams from the diagram [N] by splitting
the partition [2] as often
as possible and adding them again to all possible diagrams with at most three lines.
Furthermore, we conjecture that for odd $N$, all diagrams can be obtained by splitting [2] as often as
possible from the diagram [N-2,1,1] and adding them again to the
diagram to all possible regular diagrams with at most three lines.




\section{Summary}

In summary, we have present a detailed, algorithmic description of
how to obtain all singlet states of spin-${1\over 2}$ and spin-$1$ particles.
The method can applied analogously for the construction of $N$-particle singlet states from
particles with
higher dimensional spin.
We have also investigated the behaviour of these states under symmetry transformations.



%\bibliography{svozil}
%\bibliographystyle{osa}


\begin{thebibliography}{10}
\newcommand{\enquote}[1]{``#1''}
\expandafter\ifx\csname url\endcsname\relax
  \def\url#1{{#1}}\fi
\expandafter\ifx\csname urlprefix\endcsname\relax\def\urlprefix{}\fi

\bibitem{svozil-2006-uniquenessprinciple}
K.~Svozil, \enquote{Are simultaneous Bell measurements possible?} New Journal
  of Physics {\bf 8}, 39 (2006).
\newline http://dx.doi.org/10.1088/1367-2630/8/3/039

\bibitem{zeil-99}
A.~Zeilinger, \enquote{A Foundational Principle for Quantum Mechanics,}
  Foundations of Physics {\bf 29}, 631--643 (1999).
\newline http://dx.doi.org/10.1023/A:1018820410908

\bibitem{egbkzw}
M.~Eibl, S.~Gaertner, M.~Bourennane, C.~Kurtsiefer, M.~Zukowski, and
  H.~Weinfurter, \enquote{Experimental Observation of Four-Photon Entanglement
  from Parametric Down-Conversion,} Physical Review Letters {\bf 90}, 200\,403
  (2003).
\newline http://dx.doi.org/10.1103/PhysRevLett.90.200403

\bibitem{messiah-62}
A.~Messiah, {\em Quantum Mechanics\/}, Vol.~II (North-Holland, Amsterdam,
  1962).

\bibitem{wybourne}
B.~Wybourne, {\em Symmetry Principles and Atomic Spectroscopy\/} (Wiley
  Interscience, USA, 1970).

\bibitem{krenn1}
G.~Krenn and A.Zeilinger, \enquote{Entangled entanglement,} Physical Review A
  (Atomic, Molecular, and Optical Physics) {\bf 54}, 1793--1797 (1996).
\newline http://dx.doi.org/10.1103/PhysRevA.54.1793

\bibitem{ba-89}
L.~E. Ballentine, {\em Quantum Mechanics\/} (Prentice Hall, Englewood Cliffs,
  NJ, 1989).

\bibitem{epr}
A.~Einstein, B.~Podolsky, and N.~Rosen, \enquote{Can quantum-mechanical
  description of physical reality be considered complete?} Physical Review {\bf
  47}, 777--780 (1935).
\newline http://dx.doi.org/10.1103/PhysRev.47.777

\bibitem{svozil-2004-vax}
K.~Svozil, \enquote{On Counterfactuals and Contextuality,} in {\em AIP
  Conference Proceedings 750. {F}oundations of Probability and Physics-3\/},
  A.~Khrennikov, ed.,  pp. 351--360 (2005).
\newline http://dx.doi.org/10.1063/1.1874586

\bibitem{gill-03}
R.~D. Gill, \enquote{Time, Finite Statistics, and {B}ell's Fifth Position,} in
  {\em Proceedings of Foundations of Probability and Physics-2\/},
  A.~Khrennikov, ed.,  pp. 179--206 (2003).

\bibitem{rzbb}
M.~Reck, A.~Zeilinger, H.~J. Bernstein, and P.~Bertani, \enquote{Experimental
  realization of any discrete unitary operator,} Physical Review Letters {\bf
  73}, 58--61 (1994).
\newline http://dx.doi.org/10.1103/PhysRevLett.73.58

\bibitem{zukowski-97}
M.~Zukowski, A.~Zeilinger, and M.~A. Horne, \enquote{Realizable
  higher-dimensional two-particle entanglements via multiport beam splitters,}
  Physical Review A (Atomic, Molecular, and Optical Physics) {\bf 55},
  2564--2579 (1997).
\newline http://dx.doi.org/10.1103/PhysRevA.55.2564

\bibitem{svozil-2004-analog}
K.~Svozil, \enquote{Noncontextuality in multipartite entanglement,} J. Phys. A:
  Math. Gen. {\bf 38}, 5781--5798 (2005).
\newline http://dx.doi.org/10.1088/0305-4470/38/25/013

\end{thebibliography}

\end{document}
