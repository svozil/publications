%\documentclass[amsmath,table,sans,amsfonts, handout]{beamer}
\documentclass[amsmath,table,sans,amsfonts]{beamer}
\usepackage[T1]{fontenc}
%%\usepackage{beamerthemeshadow}
%%\usepackage[headheight=1pt,footheight=10pt]{beamerthemeboxes}
%%\addfootboxtemplate{\color{structure!80}}{\color{white}\tiny \hfill Karl Svozil (TU Vienna)\hfill}
%%\addfootboxtemplate{\color{structure!65}}{\color{white}\tiny \hfill mur.sat \hfill}
%%\addfootboxtemplate{\color{structure!50}}{\color{white}\tiny \hfill Graz, 2010-12-11\hfill}
%\usepackage[dark]{beamerthemesidebar}
%\usepackage[headheight=24pt,footheight=12pt]{beamerthemesplit}
%\usepackage{beamerthemesplit}
%\usepackage[bar]{beamerthemetree}
\usepackage{graphicx}
\usepackage{pgf}
%\usepackage{eepic}
%\usepackage[usenames]{color}
%\newcommand{\Red}{\color{Red}}  %(VERY-Approx.PANTONE-RED)
%\newcommand{\Green}{\color{Green}}  %(VERY-Approx.PANTONE-GREEN)


%\RequirePackage[german]{babel}
%\selectlanguage{german}
%\RequirePackage[isolatin]{inputenc}

%\pgfdeclareimage[height=0.5cm]{logo}{tu-logo}
%\logo{\pgfuseimage{logo}}
\beamertemplatetriangleitem
%\beamertemplateballitem

\beamerboxesdeclarecolorscheme{alert}{red}{red!15!averagebackgroundcolor}
%\begin{beamerboxesrounded}[scheme=alert,shadow=true]{}
%\end{beamerboxesrounded}

%\beamersetaveragebackground{yellow!10}

%\beamertemplatecircleminiframe

\newtheorem{question}{Question}
\newtheorem{conjecture}[question]{Principle}
\newtheorem{challenge}[question]{Challenge}
\usepackage{tikz}
\newcommand{\bra}[1]{\left< #1 \right|}
\newcommand{\ket}[1]{\left| #1 \right>}

\newcommand{\iprod}[2]{\langle #1 | #2 \rangle}
\newcommand{\mprod}[3]{\langle #1 | #2 | #3 \rangle}
\newcommand{\oprod}[2]{| #1 \rangle\langle #2 |}

\newcommand{\proj}[3]{\begin{smallmatrix} #1 & #2 & #3 \end{smallmatrix}}
\newcommand{\projbf}[3]{\begin{smallmatrix} \mathbf{#1} & \mathbf{#2} & \mathbf{#3} \end{smallmatrix}}

\sloppy
\parskip .7em %vskip between paragraphs

\newcommand{\seq}[1]{\mathbf{#1}}
\newcommand{\floor}[1]{\left\lfloor #1 \right\rfloor}
\newcommand{\ceil}[1]{\left\lceil #1 \right\rceil}
\newcommand{\m}[1]{\widetilde{#1}}
\newcommand{\p}[1]{\scriptsize\textcolor{black}{$[#1]$}}

\begin{document}

\title{\bf \textcolor{blue}{Status of Physical [In]Determinism}}
\subtitle{\textcolor{purple!60}{\small http://tph.tuwien.ac.at/$\sim$svozil/publ/2017-Svozil-Kalamazoo-pres.pdf
\\
Nature-Springer, in press 2017; drafts on demand
}}
\author{Karl Svozil}
\institute{ITP/Vienna University of Technology, Austria\\
\& CS/University of Auckland, NZ  \\
svozil@tuwien.ac.at
%{\tiny Disclaimer: Die hier vertretenen Meinungen des Autors verstehen sich als Diskussionsbeitr�ge und decken sich nicht notwendigerweise mit den Positionen der Technischen Universit�t Wien oder deren Vertreter.}
}
\date{Kalamazoo, MI, USA, May 11th, 2017}
\maketitle


% \frame{
% \frametitle{Contents}
% \tableofcontents
% }

\section{Contents}

 \frame{
 \frametitle{Contents}

{\Huge

\begin{itemize}

%\pause
\item provable [un]provables
%\pause
\item classical [in]determinism
%\pause
\item quantum [in]determinism

\end{itemize}
}
}

 \frame{
 \frametitle{Provable [un]provables -- what is entirely hopeless}


\begin{itemize}
%\pause
\item ``Haltung'' (``approach''): Freud, 1912: ``gleichschwebende Aufmerksamkeit''
(``evenly suspended attention'')

\item  epistemic method: intrinsic, embedded observers, interfacing, means relativity

\item  formal method: reduction to the halting or Rice's problems or other incompleteness and fixed point theorems
(eg, Yanofsky, doi 10.2178/bsl/1058448677)

\end{itemize}

}

 \frame{
 \frametitle{cntd. Provable [un]provables -- what is entirely hopeless}


\begin{itemize}
%\pause
\item general prediction/forecasting problem unsolvable by algorithmic means
%\pause
\item general induction problem unsolvable by algorithmic means
%\pause
\item provable impossibility to prove [in]determinism of (finite) phenomenology by algorithmic means

\item  nonalgorithmic ineffability may still be possible
(G\"odel seemed to have believed in this / Turing seemed to have denied it / cf. also recent book by Jonas, doi
10.1057/978-1-137-57955-3 / maybe idealistic self-delusional? Sagan's 1997 movie \& novel ``Contact'' /
Definition of love in Plato's Symposium / Stace's Refutation of Realism)
\end{itemize}
}

\section{Classical [in]determinism}

 \frame{
 \frametitle{Classical [in]determinism}


\begin{itemize}
%\pause
\item dependent on assumptions; eg. classical (nonconstructive) continua; means relativity

\item
deterministic chaos (strong dependence on initial values; ``unfolding'' of the algorithmic information content therein)

%\pause
\item instabilities and weak solutions of ordinary differential equations (not Lipschitz continuous):
discussion about gaps for free will by Poisson in 1806, Duhamel in 1845, Bertrand in 1878, Boussinesq in 1879,
and in 1873 by Maxwell; modern version ``Norton dome''

\item exotic constructions: Kreisel, Pour-El \& Richards, $\ldots$

%\pause
\end{itemize}
}

\section{Quantum [in]determinism}

 \frame{
 \frametitle{Quantum [in]determinism}


\begin{itemize}
%\pause
\item single events: creatio continua (spontaneous \& stimulated emissions)

%\pause
\item complementarity
%\pause
\item value-indefiniteness/contextuality a la Bell, Kochen-Specker

\item entanglement: individuality versus relationality in multipartite situations

\item unitarity (one-to-one-ness            permutation) of the quantum evolution versus irreversible (?) measurements:
quantum erasure experiments; nesting (von Neumann, Everett, Wigner)
\end{itemize}
}

\frame{
 \frametitle{Born, 1926}

``from the standpoint of our quantum mechanics, there is no quantity
which in any individual case causally fixes the consequence of the collision;
but also experimentally we have so far no reason to believe that there are some inner properties of the atom
which condition a definite outcome for the collision.
Ought we to hope later to discover such properties $\ldots$  and determine them in individual cases?
Or ought we to  believe that the agreement of theory and experiment  --  as to the impossibility
of prescribing conditions? I myself am inclined  to give up determinism in the world of atoms.''

}

\frame{
 \frametitle{Einstein in a letter to Born,
dated December~12, 1926}
``In any case I am convinced that he [the Old One] does not throw dice.''
}

\frame{
 \frametitle{Planck, 1932}
``the law of causality is neither true nor false, it rather is a heuristic principle,
a signpost and in my opinion the most valuable signpost
we possess, to guide us through the motley disorder of events and to indicate
the direction in which scientific inquiry should proceed in order to attain fruitful
results.''

}

\frame{
 \frametitle{Feynman, 1965}
$\ldots$ the ``perpetual torment that results
from [[the question]], `But how can it be like that?' which
is a reflection of uncontrolled but utterly vain desire to see
[[quantum mechanics]] in terms of an analogy with something familiar.''
Therefore, Feynman advises,
``do not keep saying to yourself, if you can possibly avoid it,
`But how can it be like that?'
because you will get `down the drain', into a blind alley from which nobody has yet
escaped.''

}

\frame{
 \frametitle{Zeilinger, 2005}
``The discovery that individual events are
irreducibly random is probably one of the
most significant findings of the twentieth
century. $\ldots$~For the individual event in quantum physics,
not only do we not know the cause, there is no cause.''

}


\frame{

\centerline{\Large {\color{magenta} Thank you for your attention!}}

\begin{center}\color{orange}
$\widetilde{\qquad \qquad }$
$\widetilde{\qquad \qquad}$
$\widetilde{\qquad \qquad }$
\end{center}
 }
 \end{document}
