%%tth:\begin{html}<LINK REL=STYLESHEET HREF="/~svozil/ssh.css">\end{html}
\documentclass[prl,showpacs,showkeys,amsfonts]{revtex4}
%\documentclass[prl,amsfonts,preprint]{revtex4}
%\documentclass[prl,preprint,amsfonts,twocolumn]{revtex4}
%\documentclass[pra,showpacs,showkeys,amsfonts]{revtex4}
\usepackage{graphicx}
%\documentstyle[amsfonts]{article}
\RequirePackage{times}
%\RequirePackage{courier}
%\RequirePackage{mathptm}
%\renewcommand{\baselinestretch}{1.3}
\begin{document}

%\def\frak{\cal }
%\sloppy



\title{Stochastic interference and auditory perception}

\author{Klaus Ehrenberger}
\email{hno@univie.ac.at}
\affiliation{Universit�tsklinik f�r Hals-, Nasen- und Ohrenkrankheiten,
Medizinische Universit\"at Wien,
W\"ahringer G\"urtel 18-20, A-1090 Vienna, Austria}

\author{Karl Svozil}
\email{svozil@tuwien.ac.at}
\homepage{http://tph.tuwien.ac.at/~svozil}
\affiliation{Institut f\"ur Theoretische Physik, University of Technology Vienna,
Wiedner Hauptstra\ss e 8-10/136, A-1040 Vienna, Austria}


\begin{abstract}

Introduction:

Due to physiologic memory effects, the formation of spike patterns of nerve activities
can be characterized by fractal geometry.
Acoustic stimuli such as music obey fractal geometry.
The converging and diverging neuronal pathways, in which such fractal geometries are processed
present additional challenge to an understanding of the processes contributing to auditory
perception; in particular to speech and to music.

Methods:

Any method which attempts a faithful representation and reconstruction of the electric activity pattern
has to implement fractal geometric stimuli of auditory nerves, and also has to cope with
the converging and diverging processing of those stimuli.

Results:

We present a study in which small changes in the fractal geometry of input signals yield a
potentially large change in the fractal geometry of the output signal by taking the joint
of the input signals equivalent to the logical AND operation.
This effect, in which the variation of the fractal dimension of the secondary spike pattern
is proportional to the number of converging input signals and to the variation of their
fractal dimension.
This effect remains unchanged if white noise is added.

Conclusion:

The increase in sensitivity to variations of the fractal dimension due to the convergent and divergent
signal processing contributes to the auditory perception.
We propose here to utilize this effect
for a better signal discrimination in a multielectrode cochlear implant configuration.

\end{abstract}

%\pacs{}
%\keywords{}

%References:

%http://www.unc.edu/~jimlee/JohnObrienFractalMusic.htm

\maketitle

\cite{voss-75}

\cite{voss-78}

\cite{svoz-ehr}

\cite{svoz-ehr2}

\bibliography{svozil}
%\bibliographystyle{apsrev}
\bibliographystyle{elsart-num}
 \end{document}
