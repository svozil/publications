\documentclass[runningheads]{llncs}

%\usepackage[breaklinks=true,colorlinks=true,anchorcolor=blue,citecolor=blue,filecolor=blue,menucolor=blue,pagecolor=blue,urlcolor=blue,linkcolor=blue]{hyperref}
\usepackage{graphicx}% Include figure files
\usepackage{xcolor}
\usepackage[numbers,sort&compress]{natbib}

\begin{document}

\title{Haunted Quantum Contextuality Versus Value Indefiniteness}


\author{Karl Svozil}
\institute{Institute of Theoretical Physics, Vienna
    University of Technology, Wiedner Hauptstra\ss e 8-10/136, A-1040
    Vienna, Austria\\
\email{svozil@tuwien.ac.at}
}

\maketitle

%\date{\today}

\begin{abstract}
Physical entities are ultimately (re)constructed from elementary yes/no events, in particular clicks in detectors or measurement devices recording quanta. Recently, the interpretation of certain such clicks has given rise to unfounded claims which are neither necessary nor sufficient, although they are presented in that way. In particular, clicks can neither inductively support nor ``(dis)prove'' the Kochen-Specker theorem, which is a formal result that has a deductive proof by contradiction. More importantly, the alleged empirical evidence of quantum contextuality, which is ``inferred'' from violations of bounds of classical probabilities by quantum correlations, is based on highly nontrivial assumptions, in particular on physical omniscience.
\end{abstract}

%\pacs{03.65.Ta, 03.65.Ud}
%\keywords{quantum  measurement theory, quantum contextuality, counterfactual observables}
%\preprint{CDMTCS preprint nr. 372/2009}


\section*{Discussion}

Time and again, in coffee houses and elsewhere,
members of the Viennese experimental physics community reminded me
always to keep in mind that all our physical ``facts''
are ultimately derived and constructed from detector clicks.
It is this basic wisdom that, when consequentially applied to recent experiments,
suggests to rethink certain claims of empirical proof.

Let us, for the sake of properly assessing the situation,
review some historical cornerstones.
Motivated by certain, as it turned out inapplicable,
no-go theorems by von Neumann regarding hidden parameters,
Bell came forward with criteria for classical probabilities and expectations,
resembling the {\em conditions of possible experience} that
had been contemplated by Boole a century earlier~\cite{Pit-94}.
Essentially, these criteria state that,
if one forces the (counterfactual) physical co-existence
upon certain finite sets of complementary, incompatible, potential observables
--
meaning that every single one could be measured,
although due to complementarity
it is impossible to simultaneously measure all of them
--
the associated potential measurement outcomes are
subject to certain algebraic bounds.

As these probabilistic bounds are not satisfied by quantum observables,
the respective measurements outcomes cannot consistently co-exist~\cite{peres222};
at least not under the classical
presumptions entering the calculations leading to these bounds.
These arguments have subsequently been strengthened by the Kochen-Specker
and the Greenberger-Horne-Zeilinger theorems, as for the latter ones
any violations of the conditions of possible experience must occur
on every single quantum and at least for a single observable~\cite{svozil_2010-pc09}
rather than occasionally.

Those results relate to situations in which {\em omniscience} is assumed; that is,
all observables which could potentially be observed can indeed
be associated with actual elements of physical reality of a single quantum.
For a realist this might appear self-evident~\cite{stace1}.
Also for experimentalists this seems to be obvious;
after all, any particular observation renders outcomes,
regardless of the mutual complementarity of some of the observables involved;
in this view, ``potentially operational'' means  ``existence.''
By this inkling, the situation suggests that the measurement ``reveals''
a pre-existing element of physical reality of the quantum observed.
Stated pointedly,
registration of some detector clicks is interpreted as a revelation
about what is taken as the quantized object.

If these pre-existing elements of physical reality are taken for granted,
it is not unreasonable to ``solve''
or ``explain'' the conundrum imposed by the various aforementioned theorems
by assuming that
any potential measurement outcome
may depend on whatever other maximal co-measurable collection of observables
(the context, interpretable as maximal operator~\cite[sect.~84]{halmos-vs})
are co-measured alongside.
This dependence of the outcome of a single quantum measurement on its context
-- that is, the influence of what is (sometimes implicitly) co-measured alongside this single quantum measurement --
is termed {\em quantum contextuality}.

Note that the Born rule, and also Gleason's theorem,
requires the quantum probabilities and expectations, and thus
all quantum statistical properties, to be {\em noncontextual.}
Notice also that
contextuality attempts to maintain a realistic, omniscient,
quasi-classical framework by abandoning context independence for single quantum observables.

Now, if one maintains realistic omniscience --
that is, the pre-existence of all outcomes of complementary potential observables
(as is implicitly assumed in Bell- and Kochen-Specker-type arguments)
--  then it is indeed true that,
as stated by Cabello~\cite{cabello:210401},
``the immense majority of the experimental violations of Bell inequalities [[proves]] quantum contextuality.''
Actually, the only difference between older evidence of violations of Bell-type inequalities
and more recent ones (
\cite{hasegawa:230401},
\cite{Bartosik-09},
\cite{kirch-09},
\cite{PhysRevLett.103.160405} and
\cite{Lapkiewicz-11}
)
seems to be based on the fact that the prior ones rely on spatially separated
quanta in Einstein-Podolsky-Rosen ``explosion'' type schemes,
whereas more recent ones are based on single quanta---a concept which appears to be more in the spirit of Kochen-Specker type theorems which
apply to the structure of observables of single quanta~\cite{cabello-nature-2011}.
But even these sorts of empirical findings referring to single quanta rely on the
non-instantaneous measurement of all but a few (mostly two or three in cases involving two- or three-particle
Einstein-Podolsky-Rosen and Greenberger-Horne-Zeilinger type) configurations,
and therefore cannot even counterfactually assure the operational existence of all elements of physical reality at once~\cite{svozil-2006-uniquenessprinciple}.

Alas, these assumptions are neither necessary (and sufficient),
as other, rather exotic options~\cite{pitowsky-82,meyer:99} demonstrate,
nor is there any more direct empirical evidence in their support.
Indeed, quantum predictions of Einstein-Podolsky-Rosen type setups involving
singlet states of qutrits suggest that contextuality
cannot be observed~\cite{svozil:040102}, although a direct experimental test is still lacking.

Thus with regards to quantum contextuality,
that is, the ``explanation'' of Bell- and Kochen-Specker-type arguments,
 the situation is rather discomforting:
insofar as contextuality seems to ``explain''
various findings related to quantum predictions and correlations,
it can only be indirectly inferred by assuming some extra assumptions, including classical omniscience;
otherwise it is not necessary.
And insofar it could be directly testable it is very unlikely to show up.
Because of this dilemma,
it is suggested to re-evaluate recent empirical findings in terms of a much broader picture
of {\em value indefiniteness;} including also the possibility
that there needs not exist a pre-existing element of physical reality
associated with certain observables.

Stated pointedly, value indefiniteness is the assumption that,
with regards to certain potential observables, a quantum system
cannot prepared to be in a specific, definite state,
because the quantum system has been prepared in a definite state of a different, complementary
observable.
Hence
there does not exist
any entity or property of a physical system under observation
which determines a measurement outcome of such a value indefinite observable completely.
If some observer chooses to measure
any such value indefinite observable
--
thus ``forcing'' an observation upon the {\em
combined} system of measurement apparatus and quantum
--
the actual measurement outcome or event
is also (if not entirely) determined by the
disposition of the measurement apparatus~\cite{bell-66}.
This should be contrasted to the definition of an element of physical reality
in the sense of Einstein, Podolsky, and Rosen~\cite{epr}:
in the latter case the measurement outcome is defined or linked to
a physical property of the quantum measured, rather than to the combination of both
measurement apparatus and the quantum measured.

Thus in situations involving counterfactual potential observables,
such as in Bell- and Kochen-Specker-type arguments,
 the experimental outcomes actually measured might not originate from such pre-existence,
but might depend on the interaction
between the quantum measured and the measurement apparatus.
Pointedly stated, the outcome might not reflect an intrinsic objective physical property
of the quantized object,
but rather originate in the way a measurement apparatus generates the outcome by interacting with the quantum.
Already Bell~\cite{bell-66} suggested that (cf. also Refs.~\cite{PhysRevLett.104.220401,PhysRevLett.106.190401}
for related experiments)
``the result of an observation may reasonably depend $\ldots$
on the complete disposition of the apparatus.''
Perhaps this was also what Bohr had in mind by mentioning~\cite{bohr-1949}
``the impossibility of any sharp separation between the behaviour of atomic
objects and the interaction with the measuring instruments which serve to define the conditions
under which the phenomena appear.''

So far no experiments have been performed
to quantify the different empirical consequences of the assumption of quantum
contextuality versus the assumption of quantum value indefiniteness.
One possibility would be to measure the varying capacities of the measurement apparatus
to translate between the context observed and a different context in which a quantum was prepared
\cite{svozil-2003-garda}.

These considerations are highly relevant for the computational capacities of quantized system exhibiting
incomputability~\cite{2008-cal-svo},
because, as it is commonly assumed~\cite{zeil-05_nature_ofQuantum},
quantum systems are irreducibly indeterministic.
How can we conceptualize and justify such computational capacities,
in particular in view of the uniform one-to-oneness of the quantum evolution at certain devices such
as fifty-fity beam splitters generating a coherent superposition of classical states~\cite{schroedinger-interpretation}?
One possibility would take into account the combined
action of a single quantum system, registered by a macroscopic measurement device with many degrees of freedom.

\section*{Acknowledgements}
This research was partly supported by Seventh Framework Program for research and technological development (FP7), PIRSES-2010-269151-RANPHYS.

%\bibliography{svozil}
%\bibliographystyle{plainnat}

\begin{thebibliography}{25}
\providecommand{\natexlab}[1]{#1}
\providecommand{\url}[1]{\texttt{#1}}
\expandafter\ifx\csname urlstyle\endcsname\relax
  \providecommand{\doi}[1]{doi: #1}\else
  \providecommand{\doi}{doi: \begingroup \urlstyle{rm}\Url}\fi

\bibitem[Amselem et~al.(2009)Amselem, R\aa{}dmark, Bourennane, and
  Cabello]{PhysRevLett.103.160405}
Elias Amselem, Magnus R\aa{}dmark, Mohamed Bourennane, and Ad\'an Cabello.
\newblock State-independent quantum contextuality with single photons.
\newblock \emph{Physical Review Letters}, 103\penalty0 (16):\penalty0 160405,
  Oct 2009.
\newblock \doi{10.1103/PhysRevLett.103.160405}.
\newblock URL \url{http://dx.doi.org/10.1103/PhysRevLett.103.160405}.

\bibitem[Bartosik et~al.(2009)Bartosik, Klepp, Schmitzer, Sponar, Cabello,
  Rauch, and Hasegawa]{Bartosik-09}
H.~Bartosik, J.~Klepp, C.~Schmitzer, S.~Sponar, A.~Cabello, H.~Rauch, and
  Y.~Hasegawa.
\newblock Experimental test of quantum contextuality in neutron interferometry.
\newblock \emph{Physical Review Letters}, 103\penalty0 (4):\penalty0 040403,
  Jul 2009.
\newblock \doi{10.1103/PhysRevLett.103.040403}.
\newblock URL \url{http://dx.doi.org/10.1103/PhysRevLett.103.040403}.

\bibitem[Bell(1966)]{bell-66}
John~S. Bell.
\newblock On the problem of hidden variables in quantum mechanics.
\newblock \emph{Reviews of Modern Physics}, 38:\penalty0 447--452, 1966.
\newblock \doi{10.1103/RevModPhys.38.447}.
\newblock URL \url{http://dx.doi.org/10.1103/RevModPhys.38.447}.

\bibitem[Bohr(1949)]{bohr-1949}
Niels Bohr.
\newblock Discussion with {E}instein on epistemological problems in atomic
  physics.
\newblock In P.~A. Schilpp, editor, \emph{{A}lbert {E}instein:
  Philosopher-Scientist}, pages 200--241. The Library of Living Philosophers,
  Evanston, Ill., 1949.
\newblock \doi{10.1016/S1876-0503(08)70379-7}.
\newblock URL \url{http://dx.doi.org/10.1016/S1876-0503(08)70379-7}.

\bibitem[Cabello(2008)]{cabello:210401}
Ad\'an Cabello.
\newblock Experimentally testable state-independent quantum contextuality.
\newblock \emph{Physical Review Letters}, 101\penalty0 (21):\penalty0 210401,
  2008.
\newblock \doi{10.1103/PhysRevLett.101.210401}.
\newblock URL \url{http://dx.doi.org/10.1103/PhysRevLett.101.210401}.

\bibitem[Cabello(2010)]{PhysRevLett.104.220401}
Ad\'an Cabello.
\newblock Proposal for revealing quantum nonlocality via local contextuality.
\newblock \emph{Physical Review Letters}, 104:\penalty0 220401, Jun 2010.
\newblock \doi{10.1103/PhysRevLett.104.220401}.
\newblock URL \url{http://dx.doi.org/10.1103/PhysRevLett.104.220401}.

\bibitem[Cabello(2011)]{cabello-nature-2011}
Ad\'an Cabello.
\newblock Quantum physics: Correlations without parts.
\newblock \emph{Nature}, 474\penalty0 (7352):\penalty0 456--458, 2011.
\newblock \doi{10.1038/474456a}.
\newblock URL \url{http://dx.doi.org/10.1038/474456a}.

\bibitem[Cabello and Cunha(2011)]{PhysRevLett.106.190401}
Ad\'an Cabello and Marcelo~Terra Cunha.
\newblock Proposal of a two-qutrit contextuality test free of the finite
  precision and compatibility loopholes.
\newblock \emph{Physical Review Letters}, 106:\penalty0 190401, May 2011.
\newblock \doi{10.1103/PhysRevLett.106.190401}.
\newblock URL \url{http://dx.doi.org/10.1103/PhysRevLett.106.190401}.

\bibitem[Calude and Svozil(2008)]{2008-cal-svo}
Cristian~S. Calude and Karl Svozil.
\newblock Quantum randomness and value indefiniteness.
\newblock \emph{Advanced Science Letters}, 1\penalty0 (2):\penalty0 165--168,
  December 2008.
\newblock \doi{10.1166/asl.2008.016}.
\newblock URL
  \url{http://www.ingentaconnect.com/content/asp/asl/2008/00000001/00000002/art00004}.

\bibitem[Einstein et~al.(1935)Einstein, Podolsky, and Rosen]{epr}
Albert Einstein, Boris Podolsky, and Nathan Rosen.
\newblock Can quantum-mechanical description of physical reality be considered
  complete?
\newblock \emph{Physical Review}, 47\penalty0 (10):\penalty0 777--780, May
  1935.
\newblock \doi{10.1103/PhysRev.47.777}.
\newblock URL \url{http://dx.doi.org/10.1103/PhysRev.47.777}.

\bibitem[Halmos(1974)]{halmos-vs}
Paul~R.. Halmos.
\newblock \emph{Finite-dimensional Vector Spaces}.
\newblock Springer, New York, Heidelberg, Berlin, 1974.

\bibitem[Hasegawa et~al.(2006)Hasegawa, Loidl, Badurek, Baron, and
  Rauch]{hasegawa:230401}
Yuji Hasegawa, Rudolf Loidl, Gerald Badurek, Matthias Baron, and Helmut Rauch.
\newblock Quantum contextuality in a single-neutron optical experiment.
\newblock \emph{Physical Review Letters}, 97\penalty0 (23):\penalty0 230401,
  2006.
\newblock \doi{10.1103/PhysRevLett.97.230401}.
\newblock URL \url{http://dx.doi.org/10.1103/PhysRevLett.97.230401}.

\bibitem[Kirchmair et~al.(2009)Kirchmair, Z{\"{a}}hringer, Gerritsma,
  Kleinmann, G{\"{u}}hne, Cabello, Blatt, and Roos]{kirch-09}
G.~Kirchmair, F.~Z{\"{a}}hringer, R.~Gerritsma, M.~Kleinmann, O.~G{\"{u}}hne,
  A.~Cabello, R.~Blatt, and C.~F. Roos.
\newblock State-independent experimental test of quantum contextuality.
\newblock \emph{Nature}, 460:\penalty0 494--497, 2009.
\newblock \doi{10.1038/nature08172}.
\newblock URL \url{http://dx.doi.org/10.1038/nature08172}.

\bibitem[Lapkiewicz et~al.(2011)Lapkiewicz, Li, Schaeff, Langford, Ramelow,
  Wie{\'{s}}niak, and Zeilinger]{Lapkiewicz-11}
Radek Lapkiewicz, Peizhe Li, Christoph Schaeff, Nathan~K. Langford, Sven
  Ramelow, Marcin Wie{\'{s}}niak, and Anton Zeilinger.
\newblock Experimental non-classicality of an indivisible quantum system.
\newblock \emph{Nature}, 474:\penalty0 490--493, June 2011.
\newblock \doi{10.1038/nature10119}.
\newblock URL \url{http://dx.doi.org/10.1038/nature10119}.

\bibitem[Meyer(1999)]{meyer:99}
David~A. Meyer.
\newblock Finite precision measurement nullifies the {K}ochen-{S}pecker
  theorem.
\newblock \emph{Physical Review Letters}, 83\penalty0 (19):\penalty0
  3751--3754, 1999.
\newblock \doi{10.1103/PhysRevLett.83.3751}.
\newblock URL \url{http://dx.doi.org/10.1103/PhysRevLett.83.3751}.

\bibitem[Peres(1978)]{peres222}
Asher Peres.
\newblock Unperformed experiments have no results.
\newblock \emph{American Journal of Physics}, 46:\penalty0 745--747, 1978.
\newblock \doi{10.1119/1.11393}.
\newblock URL \url{http://dx.doi.org/10.1119/1.11393}.

\bibitem[Pitowsky(1982)]{pitowsky-82}
Itamar Pitowsky.
\newblock Resolution of the {E}instein-{P}odolsky-{R}osen and {B}ell paradoxes.
\newblock \emph{Physical Review Letters}, 48:\penalty0 1299--1302, 1982.
\newblock \doi{10.1103/PhysRevLett.48.1299}.
\newblock URL \url{http://dx.doi.org/10.1103/PhysRevLett.48.1299}.

\bibitem[Pitowsky(1994)]{Pit-94}
Itamar Pitowsky.
\newblock {G}eorge {B}oole's `conditions of possible experience' and the
  quantum puzzle.
\newblock \emph{The British Journal for the Philosophy of Science},
  45:\penalty0 95--125, 1994.
\newblock \doi{10.1093/bjps/45.1.95}.
\newblock URL \url{http://dx.doi.org/10.1093/bjps/45.1.95}.

\bibitem[Schr{\"{o}}dinger(1995)]{schroedinger-interpretation}
Erwin Schr{\"{o}}dinger.
\newblock \emph{The Interpretation of Quantum Mechanics. {D}ublin Seminars
  (1949-1955) and Other Unpublished Essays}.
\newblock Ox Bow Press, Woodbridge, Connecticut, 1995.

\bibitem[Stace(1949)]{stace1}
Walter~Terence Stace.
\newblock The refutation of realism.
\newblock In Herbert Feigl and Wilfrid Sellars, editors, \emph{Readings in
  Philosophical Analysis}, pages 364--372. Appleton-Century-Crofts, New York,
  1949.
\newblock previously published in {\em Mind} {\bf 53}, 349-353 (1934).

\bibitem[Svozil(2004)]{svozil-2003-garda}
Karl Svozil.
\newblock Quantum information via state partitions and the context translation
  principle.
\newblock \emph{Journal of Modern Optics}, 51:\penalty0 811--819, 2004.
\newblock \doi{10.1080/09500340410001664179}.
\newblock URL \url{http://dx.doi.org/10.1080/09500340410001664179}.

\bibitem[Svozil(2006)]{svozil-2006-uniquenessprinciple}
Karl Svozil.
\newblock Are simultaneous {B}ell measurements possible?
\newblock \emph{New Journal of Physics}, 8:\penalty0 39, 1--8, 2006.
\newblock \doi{10.1088/1367-2630/8/3/039}.
\newblock URL \url{http://dx.doi.org/10.1088/1367-2630/8/3/039}.

\bibitem[Svozil(2009)]{svozil:040102}
Karl Svozil.
\newblock Proposed direct test of a certain type of noncontextuality in quantum
  mechanics.
\newblock \emph{Physical Review A}, 80\penalty0 (4):\penalty0 040102, 2009.
\newblock \doi{10.1103/PhysRevA.80.040102}.
\newblock URL \url{http://dx.doi.org/10.1103/PhysRevA.80.040102}.

\bibitem[Svozil(2010)]{svozil_2010-pc09}
Karl Svozil.
\newblock Quantum value indefiniteness.
\newblock \emph{Natural Computing}, online first:\penalty0 1--12, 2010.
\newblock ISSN 1567-7818.
\newblock \doi{10.1007/s11047-010-9241-x}.
\newblock URL \url{http://dx.doi.org/10.1007/s11047-010-9241-x}.

\bibitem[Zeilinger(2005)]{zeil-05_nature_ofQuantum}
Anton Zeilinger.
\newblock The message of the quantum.
\newblock \emph{Nature}, 438:\penalty0 743, 2005.
\newblock \doi{10.1038/438743a}.
\newblock URL \url{http://dx.doi.org/10.1038/438743a}.

\end{thebibliography}

\end{document}

~~~~~~~~~~~~~~~~~~~~~~~~~~~~~~~~~~


a)      explain the notion of quantum contextuality,

I have restated the definition by writing "This dependence of the outcome of a single quantum measurement on its context
-- that is, the influence of what is (sometimes implicitly) co-measured alongside this single quantum measurement --
is termed {\em quantum contextuality}."

b)      explain the notion of value indefiniteness,

I have added a paragraph "Stated pointedly, value indefiniteness is the assumption that,
with regards to certain potential observables, a quantum system
cannot prepared to be in a specific, definite state,
because the quantum system has been prepared in a definite state of a different, complementary
observable.
Hence
there does not exist
any entity or property of a physical system under observation
which determines a measurement outcome of such a value indefinite observable completely.
If some observer chooses to measure
any such value indefinite observable
--
thus ``forcing'' an observation upon the {\em
combined} system of measurement apparatus and quantum
--
the actual measurement outcome or event
is also (if not entirely) determined by the
disposition of the measurement apparatus~\cite{bell-66}.
This should be contrasted to the definition of an element of physical reality
in the sense of Einstein, Podolsky, and Rosen~\cite{epr}:
in the latter case the measurement outcome is defined or linked to
a physical property of the quantum measured, rather than to the combination of both
measurement apparatus and the quantum measured."

c) give appropriate examples to illustrate statements which in
the current form will are unintelligible to non-experts. Examples are
necessary in the second (from bottom) paragraph on page 2, starting with
"Now, if one ...",

I have added a note "(as is implicitly assumed in Bell- and Kochen-Specker-type arguments)"

in the second (from top) paragraph on page 3,
starting with "Thus with regards..."

I have added a note "that is, the ``explanation'' of Bell- and Kochen-Specker-type arguments,"

and in the last paragraph
of the paper starting with "In certain situations...".


I have changed the wording to "Thus in situations involving counterfactual potential observables,
such as in Bell- and Kochen-Specker-type arguments,"

d) Are there known different consequences of QM + the assumption of quantum
contextuality vs QM + the assumption of value indefiniteness? Such
examples will be useful to see the different roles played by these hypotheses.

I have added a paragraph "So far no experiments have been performed
to quantify the different empirical consequences of the assumption of quantum
contextuality versus the assumption of quantum value indefiniteness.
One possibility would be to measure the varying capacities of the measurement apparatus
to translate between the context observed and a different context in which a quantum was prepared
\cite{svozil-2003-garda}."

e) Given the importance of computability for the volume, a short
paragraph on this issue will be useful.

I have added a paragraph "These considerations are highly relevant for the computational capacities of quantized system exhibiting
incomputability~\cite{2008-cal-svo},
because, as it is commonly assumed~\cite{zeil-05_nature_ofQuantum},
quantum systems are irreducibly indeterministic.
How can we conceptualize and justify such computational capacities,
in particular in view of the uniform one-to-oneness of the quantum evolution at certain devices such
as fifty-fity beam splitters generating a coherent superposition of classical states~\cite{schroedinger-interpretation}?
One possibility would take into account the combined
action of a single quantum system, registered by a macroscopic measurement device with many degrees of freedom."


f) The references are funny: they include meta-words like  "textbf"
"bibinfo volume 46", which have to be deleted.

I have generated the references with a proper bibtex style (plainnat with numbering).



