%\documentclass[pra,showpacs,showkeys,amsfonts,amsmath,twocolumn,handou]{revtex4}
\documentclass[amsmath,red,table,sans,handout]{beamer}
%\documentclass[pra,showpacs,showkeys,amsfonts]{revtex4}
%\documentclass[pra,showpacs,showkeys,amsfonts]{revtex4}
\usepackage[T1]{fontenc}
%%\usepackage{beamerthemeshadow}
\usepackage[headheight=1pt,footheight=10pt]{beamerthemeboxes}
\addfootboxtemplate{\color{structure!80}}{\color{white}\tiny \hfill Volkmar Putz \& Karl Svozil (TU Vienna)\hfill}
\addfootboxtemplate{\color{structure!65}}{\color{white}\tiny \hfill Faster-than-light computation?\hfill}
\addfootboxtemplate{\color{structure!50}}{\color{white}\tiny \hfill HN10, June 22, 2010, Tokyo, Japan\hfill}
%\usepackage[dark]{beamerthemesidebar}
%\usepackage[headheight=24pt,footheight=12pt]{beamerthemesplit}
%\usepackage{beamerthemesplit}
%\usepackage[bar]{beamerthemetree}
\usepackage{graphicx}
\usepackage{pgf}
%\usepackage[usenames]{color}
%\newcommand{\Red}{\color{Red}}  %(VERY-Approx.PANTONE-RED)
%\newcommand{\Green}{\color{Green}}  %(VERY-Approx.PANTONE-GREEN)

%\RequirePackage[german]{babel}
%\selectlanguage{german}
%\RequirePackage[isolatin]{inputenc}

\pgfdeclareimage[height=0.5cm]{logo}{tu-logo}
\logo{\pgfuseimage{logo}}
\beamertemplatetriangleitem
%\beamertemplateballitem

\beamerboxesdeclarecolorscheme{alert}{red}{red!15!averagebackgroundcolor}
\beamerboxesdeclarecolorscheme{alert2}{orange}{orange!15!averagebackgroundcolor}
%\begin{beamerboxesrounded}[scheme=alert,shadow=true]{}
%\end{beamerboxesrounded}

%\beamersetaveragebackground{green!10}

%\beamertemplatecircleminiframe

\usepackage{feynmf}             %Package for feynman diagrams.

\begin{document}

\title{\bf \textcolor{red}{Can a computer be ``pushed'' to perform faster-than-light?}}
%\subtitle{Naturwissenschaftlich-Humanisticher Tag am BG 19\\Weltbild und Wissenschaft\\http://tph.tuwien.ac.at/\~{}svozil/publ/2005-BG18-pres.pdf}
\subtitle{\textcolor{orange!60}{\small http://tph.tuwien.ac.at/$\sim$svozil/publ/2010-Tokyo-pres.pdf}\\
\textcolor{gray!60}{\footnotesize http://arxiv.org/abs/1003.1238 \& http://arxiv.org/abs/physics/0210091}
}
\author{Volkmar Putz \& Karl Svozil}
\institute{Institut f\"ur Theoretische Physik, Vienna University of Technology, \\
Wiedner Hauptstra\ss e 8-10/136, A-1040 Vienna, Austria\\
svozil@tuwien.ac.at
%{\tiny Disclaimer: Die hier vertretenen Meinungen des Autors verstehen sich als Diskussionsbeitr�ge und decken sich nicht notwendigerweise mit den Positionen der Technischen Universit�t Wien oder deren Vertreter.}
}
\date{June 22, 2010, Tokyo, Japan}
\maketitle



%\frame{
%\frametitle{Contents}
%\tableofcontents
%}


\section{Quantum vacua}
\frame{
\frametitle{Some speculations about quantum vacua}
{
\begin{itemize}

\item<1->
Everybody has his or her ``toy model'' of the vacuum; eg, Prof. Urban (Graz) once stated
{\it ``The quantum vacuum feels like a supermarket (Kastner\&\"Ohler in Graz) on Saturdays!''}

\item<1->
Speculation 0: Different vacua result in different (renormalized) radiative corrections to masses, $g-2$, $n$.
\item<1->
Speculation 1: Changes of index of refraction bring about (inverse) changes of the velocity of light in vacua.
\item<1->
Super-Speculation 2: ``Supercavitation'' in the ``quantum ether'' [cf. Paul A. M. Dirac. Is there an aether? Nature {\bf 168}, 906--907 (1951)],
as a possibility for faster-than-light propagation?
\end{itemize}
}
}



\section{Mass, $g-2$ between parallel conducting plates}
\frame{
\frametitle{Mass, $g-2$ between parallel conducting plates}
K. Svozil, ``Mass and anomalous magnetic moment of an electron between two conducting parallel plates'', Physical Review Letters 54, 742-744 (1985) [DOI:10.1103/PhysRevLett.54.742].\\
M. Kreuzer and K. Svozil, ``QED between plates: mass and anomalous magnetic moment of an electron'', Physical Review D 34, 1429-1437 (1986) [DOI:10.1103/PhysRevD.34.1429].
$\;$\\
$\;$\\
Discretization of electromagnetic field modes between parallel plates

$${\Delta m} = -{\alpha \over 2a}\left[ \log (4am)+1\right]$$
$$\Delta (g-2) = -{\alpha \over am}\left[ \log (4am)-2\right]$$


}


\section{Scharnhorst effect: $n$ between parallel conducting plates}
\frame{
\frametitle{Scharnhorst effect: $n$ between parallel conducting plates}
K. Scharnhorst, On propagation of light in the vacuum between plates. Physics
Letters B 236 (1990) 354-359. \\
G. Barton and K. Scharnhorst, QED between parallel mirrors: light signals faster
than c, or amplifed by the vacuum. Journal of Physics A: Mathematical and
General 26 (1993) 2037-2046.\\
P. Milonni and K. Svozil, Impossibility of measuring faster-than-c signaling by the
Scharnhorst effect. Physics Letters B 248 (1990) 437-438

$$c(a)= {c\over n(a)} > c$$

}


\section{Refractive index of ``vacuum'' with fermions}
\frame{
\frametitle{Refractive index of ``vacuum'' with fermions}

$\;$\\   $\;$\\
Vacuum polarization term (lowest order pertubative contribution to refractive index)
$\;$\\   $\;$\\
\begin{center}
\begin{fmffile}{QED_vacuum_polarization}
\begin{fmfgraph*}(120,40)
\fmfleft{i}
\fmfright{o}
\fmf{photon,label=$k$}{i,v1}
\fmf{photon,label=$k$}{v2,o}
\fmf{fermion,left,tension=0.4}{v1,v2,v1}
%\fmf{photon}{v1,v2}
\fmfdot{v1,v2}
\put(30.00,-10.00){\framebox(60,60)[cc]{$\varepsilon_F$}}
\end{fmfgraph*}
\end{fmffile}
\end{center}
$\;$\\
$${\Delta }{\Pi}_{\mu \nu}(k^2)=-\left(g_{\mu \nu }k^2 - k_\mu k_\nu \right) \frac{2\alpha}{3\pi}  \log \frac{\varepsilon_F}{m},
$$
}



\frame{
\frametitle{Refractive index of ``vacuum'' with fermions cntd.}

Effective mass term:
$$
M(k)=\epsilon^\mu { \Pi}_{\mu \nu}(k)\epsilon^\nu
$$
such that the eigenvalue equation is
$$
{ k}^2+ M(k)=(k^0)^2,
$$
where $k^\mu=({\bf k},k^0=\omega)$; and
$$
\vert  {\bf k} \vert \approx \omega - \frac{1}{2 \omega} M(k).
$$
Thus the index of refraction can be defined by
$$
n(\omega )=\frac{\vert {\bf k} \vert}{\omega}\approx 1 - \frac{1}{2 \omega^2} M(k).
$$
Hence the change of the refractive index is given by
$$
{\Delta }n(\omega )\approx -\frac{\alpha}{3\pi \omega^2} (\epsilon^\mu k_\mu)^2  \log \frac{\varepsilon_F}{m}.
$$
 }



\frame[squeeze]{
\frametitle{Refractive index of ``vacuum'' with fermions cntd.}

The group velocity is given by
$$v_{gr}={c\over n_{gr}}$$ with
$$n_{gr}(\omega )= n (\omega )+ \omega \left[\partial n (\omega )/\partial \omega \right]$$
which, for transversal waves, turns out to be $n (\omega )$.
As a result, the speed of light $c/(1-{ \Delta }n)\approx c+{ \Delta }c$  is changed by
$${  \Delta }c = c {  \Delta }n.$$

}


\section{Physical realizations by generalized beam splitters}
\frame{
\frametitle{Physical realizations by generalized beam splitters}
M. Reck, A. Zeilinger, H. J. Bernstein, and P. Bertani, ``Experimental realization of any discrete unitary operator,'' Physical Review Letters 73, 58 (1994);
M. Zukowski, A. Zeilinger, and M. A. Horne, Physical Review A 55, 2564 (1997);
Karl Svozil,  J. Phys. A: Math. Gen. 38(25), 5781-5798 (2005)
\begin{center}
$\widetilde{\qquad \qquad }$
$\widetilde{\qquad \qquad}$
$\widetilde{\qquad \qquad }$
\\
``Every unitary operator, and thus universal quantum computation, can be realized by generalized beam splitters''
\\    $\,$\\
%TeXCAD Picture [1.pic]. Options:
%\grade{\on}
%\emlines{\off}
%\epic{\off}
%\beziermacro{\on}
%\reduce{\on}
%\snapping{\off}
%\quality{8.000}
%\graddiff{0.005}
%\snapasp{1}
%\zoom{4.0000}
\unitlength .45mm % = 1.423pt
\linethickness{0.4pt}
\ifx\plotpoint\undefined\newsavebox{\plotpoint}\fi % GNUPLOT compatibility
\begin{picture}(120,80)(0,0)
\put(20,0){\framebox(80,80)[cc]{}}
\put(57.67,40){\line(1,0){5}}
\put(64.33,40){\line(1,0){5}}
\put(50.67,40){\line(1,0){5}}
\put(78.67,50){\dashbox{1,1}(8,4.33)[cc]{}}
\put(82.67,58){\makebox(0,0)[cc]{$P_3,\varphi$}}
\put(73.33,40){\makebox(0,0)[lc]{$S(T(\omega ))$}}
\put(8.33,65.67){\makebox(0,0)[cc]{$\vert 0\rangle$}}
\put(110.67,65.67){\makebox(0,0)[cc]{${\vert 0\rangle}'$}}
\put(110.67,25.67){\makebox(0,0)[cc]{${\vert 1\rangle}'$}}
\put(8,25.67){\makebox(0,0)[cc]{$\vert 1\rangle$}}
\put(24.33,75.67){\makebox(0,0)[lc]
{${\textsf{\textbf{U}}}^{bs}(\omega ,\alpha ,\beta ,\varphi )$}}
\put(0,59.67){\vector(1,0){20}}
\put(0,20){\vector(1,0){20}}
\put(100,60){\vector(1,0){20}}
\put(100,20){\vector(1,0){20}}
\put(20,20){\line(2,1){80}}
\put(20,60){\line(2,-1){80}}
\put(32.67,50){\dashbox{1,1}(8,4.33)[cc]{}}
\put(36.67,62){\makebox(0,0)[cc]{$P_1,\alpha +\beta $}}
\put(32.67,27){\dashbox{1,1}(8,4.33)[cc]{}}
\put(36.67,35){\makebox(0,0)[cc]{$P_2,\beta$}}
\end{picture}
\end{center}

}

\frame{
\frametitle{Example of a physical realization of a particular unitary operator by generalized beam splitters}
{\tiny
\begin{center}
\begin{tabular}{c}
$
\left(
\begin{array}{ccccccccc}
0& 0& 0& 0& 0& 0& 0& 0& 1\\   %e3
0& 0& 0& 0& 0& 0& -{1\over \sqrt{2}}& {1\over \sqrt{2}}& 0\\  %e2
0& 0& 0& 0& 0& 0& {1\over \sqrt{2}}& {1\over \sqrt{2}}& 0\\  %e1
0& 0& 0& 0& 0& 1& 0& 0& 0\\   %e3
0& 0& 0& -{1\over \sqrt{2}}& {1\over \sqrt{2}}& 0& 0& 0& 0\\  %e2
0& 0& 0& {1\over \sqrt{2}}& {1\over \sqrt{2}}& 0& 0& 0& 0\\   %e1
0& 0& 1& 0& 0& 0& 0& 0& 0\\   %e3
-{1\over \sqrt{2}}& {1\over \sqrt{2}}& 0& 0& 0& 0& 0& 0& 0\\  %e2
{1\over \sqrt{2}}& {1\over \sqrt{2}}& 0& 0& 0& 0& 0& 0& 0\\   %e1  \begin{array}{ccccccccc}
\end{array}
\right)$
\\
$\;$\\
%TexCad Options
%\grade{\off}
%\emlines{\off}
%\beziermacro{\on}
%\reduce{\on}
%\snapping{\off}
%\quality{2.00}
%\graddiff{0.01}
%\snapasp{1}
%\zoom{15.00}
\unitlength 4.00mm
\linethickness{0.2pt}
\begin{picture}(9.44,9.50)
(0.0,-1.0)
%\put(0.90,7.50){\framebox(0.20,0.05)[cc]{}}
%\put(1.20,7.55){\makebox(0,0)[lc]{0}}
\put(0.50,7.00){\line(1,0){1.00}}
\put(1.00,6.50){\line(0,1){1.00}}
%\put(0.90,6.50){\framebox(0.20,0.05)[cc]{}}
%\put(1.20,6.55){\makebox(0,0)[lc]{0}}
\put(0.50,6.00){\line(1,0){1.00}}
\put(1.00,5.50){\line(0,1){1.00}}
%\put(1.90,6.50){\framebox(0.20,0.05)[cc]{}}
%\put(2.20,6.55){\makebox(0,0)[lc]{0}}
\put(1.50,6.00){\line(1,0){1.00}}
\put(2.00,5.50){\line(0,1){1.00}}
%\put(0.90,5.50){\framebox(0.20,0.05)[cc]{}}
%\put(1.20,5.55){\makebox(0,0)[lc]{0}}
\put(0.50,5.00){\line(1,0){1.00}}
\put(1.00,4.50){\line(0,1){1.00}}
%\put(1.90,5.50){\framebox(0.20,0.05)[cc]{}}
%\put(2.20,5.55){\makebox(0,0)[lc]{0}}
\put(1.50,5.00){\line(1,0){1.00}}
\put(2.00,4.50){\line(0,1){1.00}}
%\put(2.90,5.50){\framebox(0.20,0.05)[cc]{}}
%\put(3.20,5.55){\makebox(0,0)[lc]{0}}
\put(2.50,5.00){\line(1,0){1.00}}
\put(3.00,4.50){\line(0,1){1.00}}
%\put(0.90,4.50){\framebox(0.20,0.05)[cc]{}}
%\put(1.20,4.55){\makebox(0,0)[lc]{0}}
\put(0.50,4.00){\line(1,0){1.00}}
\put(1.00,3.50){\line(0,1){1.00}}
%\put(1.90,4.50){\framebox(0.20,0.05)[cc]{}}
%\put(2.20,4.55){\makebox(0,0)[lc]{0}}
\put(1.50,4.00){\line(1,0){1.00}}
\put(2.00,3.50){\line(0,1){1.00}}
%\put(2.90,4.50){\framebox(0.20,0.05)[cc]{}}
%\put(3.20,4.55){\makebox(0,0)[lc]{0}}
\put(2.50,4.00){\line(1,0){1.00}}
\put(3.00,3.50){\line(0,1){1.00}}
\put(4.00,4.00){\oval(0.14,0.14)[rt]}
%\put(4.14,4.12){\makebox(0,0)[cc]{p}}
\put(3.80,4.20){\line(1,-1){0.40}}
\put(3.90,4.50){\framebox(0.20,0.05)[cc]{}}
\put(4.20,4.55){\makebox(0,0)[lc]{$\pi$}}
\put(3.50,4.00){\line(1,0){1.00}}
\put(4.00,3.50){\line(0,1){1.00}}
%\put(0.90,3.50){\framebox(0.20,0.05)[cc]{}}
%\put(1.20,3.55){\makebox(0,0)[lc]{0}}
\put(0.50,3.00){\line(1,0){1.00}}
\put(1.00,2.50){\line(0,1){1.00}}
%\put(1.90,3.50){\framebox(0.20,0.05)[cc]{}}
%\put(2.20,3.55){\makebox(0,0)[lc]{0}}
\put(1.50,3.00){\line(1,0){1.00}}
\put(2.00,2.50){\line(0,1){1.00}}
%\put(2.90,3.50){\framebox(0.20,0.05)[cc]{}}
%\put(3.20,3.55){\makebox(0,0)[lc]{0}}
\put(2.50,3.00){\line(1,0){1.00}}
\put(3.00,2.50){\line(0,1){1.00}}
\put(4.00,3.00){\oval(0.14,0.14)[rt]}
%\put(4.14,3.12){\makebox(0,0)[cc]{p}}
\put(3.80,3.20){\line(1,-1){0.40}}
\put(3.80,2.80){\framebox(0.40,0.40)[cc]{}}
\put(3.90,3.50){\framebox(0.20,0.05)[cc]{}}
\put(4.20,3.55){\makebox(0,0)[lc]{$\pi$}}
\put(3.50,3.00){\line(1,0){1.00}}
\put(4.00,2.50){\line(0,1){1.00}}
\put(5.00,3.00){\oval(0.14,0.14)[rt]}
%\put(5.14,3.12){\makebox(0,0)[cc]{p}}
\put(4.80,3.20){\line(1,-1){0.40}}
\put(4.90,3.50){\framebox(0.20,0.05)[cc]{}}
\put(5.20,3.55){\makebox(0,0)[lc]{$\pi$}}
\put(4.50,3.00){\line(1,0){1.00}}
\put(5.00,2.50){\line(0,1){1.00}}
%\put(0.90,2.50){\framebox(0.20,0.05)[cc]{}}
%\put(1.20,2.55){\makebox(0,0)[lc]{0}}
\put(0.50,2.00){\line(1,0){1.00}}
\put(1.00,1.50){\line(0,1){1.00}}
%\put(1.90,2.50){\framebox(0.20,0.05)[cc]{}}
%\put(2.20,2.55){\makebox(0,0)[lc]{0}}
\put(1.50,2.00){\line(1,0){1.00}}
\put(2.00,1.50){\line(0,1){1.00}}
\put(3.00,2.00){\oval(0.14,0.14)[rt]}
%\put(3.14,2.12){\makebox(0,0)[cc]{p}}
\put(2.80,2.20){\line(1,-1){0.40}}
\put(2.90,2.50){\framebox(0.20,0.05)[cc]{}}
\put(3.20,2.55){\makebox(0,0)[lc]{$\pi$}}
\put(2.50,2.00){\line(1,0){1.00}}
\put(3.00,1.50){\line(0,1){1.00}}
%\put(3.90,2.50){\framebox(0.20,0.05)[cc]{}}
%\put(4.20,2.55){\makebox(0,0)[lc]{0}}
\put(3.50,2.00){\line(1,0){1.00}}
\put(4.00,1.50){\line(0,1){1.00}}
%\put(4.90,2.50){\framebox(0.20,0.05)[cc]{}}
%\put(5.20,2.55){\makebox(0,0)[lc]{0}}
\put(4.50,2.00){\line(1,0){1.00}}
\put(5.00,1.50){\line(0,1){1.00}}
%\put(5.90,2.50){\framebox(0.20,0.05)[cc]{}}
%\put(6.20,2.55){\makebox(0,0)[lc]{0}}
\put(5.50,2.00){\line(1,0){1.00}}
\put(6.00,1.50){\line(0,1){1.00}}
\put(1.00,1.00){\oval(0.14,0.14)[rt]}
%\put(1.14,1.12){\makebox(0,0)[cc]{p}}
\put(0.80,1.20){\line(1,-1){0.40}}
\put(0.90,1.50){\framebox(0.20,0.05)[cc]{}}
\put(1.20,1.55){\makebox(0,0)[lc]{$\pi$}}
\put(0.50,1.00){\line(1,0){1.00}}
\put(1.00,0.50){\line(0,1){1.00}}
%\put(1.90,1.50){\framebox(0.20,0.05)[cc]{}}
%\put(2.20,1.55){\makebox(0,0)[lc]{0}}
\put(1.50,1.00){\line(1,0){1.00}}
\put(2.00,0.50){\line(0,1){1.00}}
%\put(2.90,1.50){\framebox(0.20,0.05)[cc]{}}
%\put(3.20,1.55){\makebox(0,0)[lc]{0}}
\put(2.50,1.00){\line(1,0){1.00}}
\put(3.00,0.50){\line(0,1){1.00}}
%\put(3.90,1.50){\framebox(0.20,0.05)[cc]{}}
%\put(4.20,1.55){\makebox(0,0)[lc]{0}}
\put(3.50,1.00){\line(1,0){1.00}}
\put(4.00,0.50){\line(0,1){1.00}}
%\put(4.90,1.50){\framebox(0.20,0.05)[cc]{}}
%\put(5.20,1.55){\makebox(0,0)[lc]{0}}
\put(4.50,1.00){\line(1,0){1.00}}
\put(5.00,0.50){\line(0,1){1.00}}
%\put(5.90,1.50){\framebox(0.20,0.05)[cc]{}}
%\put(6.20,1.55){\makebox(0,0)[lc]{0}}
\put(5.50,1.00){\line(1,0){1.00}}
\put(6.00,0.50){\line(0,1){1.00}}
%\put(6.90,1.50){\framebox(0.20,0.05)[cc]{}}
%\put(7.20,1.55){\makebox(0,0)[lc]{0}}
\put(6.50,1.00){\line(1,0){1.00}}
\put(7.00,0.50){\line(0,1){1.00}}
\put(1.00,0.00){\oval(0.14,0.14)[rt]}
%\put(1.14,0.12){\makebox(0,0)[cc]{p}}
\put(0.80,0.20){\line(1,-1){0.40}}
\put(0.80,-0.20){\framebox(0.40,0.40)[cc]{}}
\put(0.90,0.50){\framebox(0.20,0.05)[cc]{}}
\put(1.20,0.55){\makebox(0,0)[lc]{$\pi$}}
\put(0.50,0.00){\line(1,0){1.00}}
\put(1.00,-0.50){\line(0,1){1.00}}
\put(2.00,0.00){\oval(0.14,0.14)[rt]}
%\put(2.14,0.12){\makebox(0,0)[cc]{p}}
\put(1.80,0.20){\line(1,-1){0.40}}
\put(1.90,0.50){\framebox(0.20,0.05)[cc]{}}
\put(2.20,0.55){\makebox(0,0)[lc]{$\pi$}}
\put(1.50,0.00){\line(1,0){1.00}}
\put(2.00,-0.50){\line(0,1){1.00}}
%\put(2.90,0.50){\framebox(0.20,0.05)[cc]{}}
%\put(3.20,0.55){\makebox(0,0)[lc]{0}}
\put(2.50,0.00){\line(1,0){1.00}}
\put(3.00,-0.50){\line(0,1){1.00}}
%\put(3.90,0.50){\framebox(0.20,0.05)[cc]{}}
%\put(4.20,0.55){\makebox(0,0)[lc]{0}}
\put(3.50,0.00){\line(1,0){1.00}}
\put(4.00,-0.50){\line(0,1){1.00}}
%\put(4.90,0.50){\framebox(0.20,0.05)[cc]{}}
%\put(5.20,0.55){\makebox(0,0)[lc]{0}}
\put(4.50,0.00){\line(1,0){1.00}}
\put(5.00,-0.50){\line(0,1){1.00}}
%\put(5.90,0.50){\framebox(0.20,0.05)[cc]{}}
%\put(6.20,0.55){\makebox(0,0)[lc]{0}}
\put(5.50,0.00){\line(1,0){1.00}}
\put(6.00,-0.50){\line(0,1){1.00}}
\put(7.00,0.00){\oval(0.14,0.14)[rt]}
%\put(7.14,0.12){\makebox(0,0)[cc]{p}}
\put(6.80,0.20){\line(1,-1){0.40}}
\put(6.80,-0.20){\framebox(0.40,0.40)[cc]{}}
\put(6.90,0.50){\framebox(0.20,0.05)[cc]{}}
\put(7.20,0.55){\makebox(0,0)[lc]{$\pi$}}
\put(6.50,0.00){\line(1,0){1.00}}
\put(7.00,-0.50){\line(0,1){1.00}}
\put(8.00,0.00){\oval(0.14,0.14)[rt]}
%\put(8.14,0.12){\makebox(0,0)[cc]{p}}
\put(7.80,0.20){\line(1,-1){0.40}}
\put(7.90,0.50){\framebox(0.20,0.05)[cc]{}}
\put(8.20,0.55){\makebox(0,0)[lc]{$\pi$}}
\put(7.50,0.00){\line(1,0){1.00}}
\put(8.00,-0.50){\line(0,1){1.00}}
\put(0.00,8.00){\line(1,0){0.50}}
\put(-0.10,8.00){\makebox(0,0)[cc]{9}}
\put(9.00,-0.50){\line(0,-1){0.50}}
\put(9.00,-1.00){\line(1,1){0.30}}
\put(9.00,-1.00){\line(-1,1){0.30}}
\put(9.00,-1.20){\makebox(0,0)[cc]{1}}
\put(0.50,8.00){\line(1,0){0.50}}
\put(1.00,8.00){\line(0,-1){0.50}}
\put(0.00,7.00){\line(1,0){0.50}}
\put(-0.10,7.00){\makebox(0,0)[cc]{8}}
\put(8.00,-0.50){\line(0,-1){0.50}}
\put(8.00,-1.00){\line(1,1){0.30}}
\put(8.00,-1.00){\line(-1,1){0.30}}
\put(8.00,-1.20){\makebox(0,0)[cc]{2}}
\put(1.50,7.00){\line(1,0){0.50}}
\put(2.00,7.00){\line(0,-1){0.50}}
\put(0.00,6.00){\line(1,0){0.50}}
\put(-0.10,6.00){\makebox(0,0)[cc]{7}}
\put(7.00,-0.50){\line(0,-1){0.50}}
\put(7.00,-1.00){\line(1,1){0.30}}
\put(7.00,-1.00){\line(-1,1){0.30}}
\put(7.00,-1.20){\makebox(0,0)[cc]{3}}
\put(2.50,6.00){\line(1,0){0.50}}
\put(3.00,6.00){\line(0,-1){0.50}}
\put(0.00,5.00){\line(1,0){0.50}}
\put(-0.10,5.00){\makebox(0,0)[cc]{6}}
\put(6.00,-0.50){\line(0,-1){0.50}}
\put(6.00,-1.00){\line(1,1){0.30}}
\put(6.00,-1.00){\line(-1,1){0.30}}
\put(6.00,-1.20){\makebox(0,0)[cc]{4}}
\put(3.50,5.00){\line(1,0){0.50}}
\put(4.00,5.00){\line(0,-1){0.50}}
\put(0.00,4.00){\line(1,0){0.50}}
\put(-0.10,4.00){\makebox(0,0)[cc]{5}}
\put(5.00,-0.50){\line(0,-1){0.50}}
\put(5.00,-1.00){\line(1,1){0.30}}
\put(5.00,-1.00){\line(-1,1){0.30}}
\put(5.00,-1.20){\makebox(0,0)[cc]{5}}
\put(4.50,4.00){\line(1,0){0.50}}
\put(5.00,4.00){\line(0,-1){0.50}}
\put(0.00,3.00){\line(1,0){0.50}}
\put(-0.10,3.00){\makebox(0,0)[cc]{4}}
\put(4.00,-0.50){\line(0,-1){0.50}}
\put(4.00,-1.00){\line(1,1){0.30}}
\put(4.00,-1.00){\line(-1,1){0.30}}
\put(4.00,-1.20){\makebox(0,0)[cc]{6}}
\put(5.50,3.00){\line(1,0){0.50}}
\put(6.00,3.00){\line(0,-1){0.50}}
\put(0.00,2.00){\line(1,0){0.50}}
\put(-0.10,2.00){\makebox(0,0)[cc]{3}}
\put(3.00,-0.50){\line(0,-1){0.50}}
\put(3.00,-1.00){\line(1,1){0.30}}
\put(3.00,-1.00){\line(-1,1){0.30}}
\put(3.00,-1.20){\makebox(0,0)[cc]{7}}
\put(6.50,2.00){\line(1,0){0.50}}
\put(7.00,2.00){\line(0,-1){0.50}}
\put(0.00,1.00){\line(1,0){0.50}}
\put(-0.10,1.00){\makebox(0,0)[cc]{2}}
\put(2.00,-0.50){\line(0,-1){0.50}}
\put(2.00,-1.00){\line(1,1){0.30}}
\put(2.00,-1.00){\line(-1,1){0.30}}
\put(2.00,-1.20){\makebox(0,0)[cc]{8}}
\put(7.50,1.00){\line(1,0){0.50}}
\put(8.00,1.00){\line(0,-1){0.50}}
\put(0.00,0.00){\line(1,0){0.50}}
\put(-0.10,0.00){\makebox(0,0)[cc]{1}}
\put(1.00,-0.50){\line(0,-1){0.50}}
\put(1.00,-1.00){\line(1,1){0.30}}
\put(1.00,-1.00){\line(-1,1){0.30}}
\put(1.00,-1.20){\makebox(0,0)[cc]{9}}
\put(8.50,0.00){\line(1,0){0.50}}
\put(9.00,0.00){\line(0,-1){0.50}}
%\put(0.90,-0.50){\framebox(0.20,0.05)[cc]{}}
%\put(1.20,-0.45){\makebox(0,0)[lc]{0}}
\put(1.90,-0.50){\framebox(0.20,0.05)[cc]{}}
\put(2.20,-0.45){\makebox(0,0)[lc]{$\pi$}}
%\put(2.90,-0.50){\framebox(0.20,0.05)[cc]{}}
%\put(3.20,-0.45){\makebox(0,0)[lc]{0}}
\put(3.90,-0.50){\framebox(0.20,0.05)[cc]{}}
\put(4.20,-0.45){\makebox(0,0)[lc]{$\pi$}}
%\put(4.90,-0.50){\framebox(0.20,0.05)[cc]{}}
%\put(5.20,-0.45){\makebox(0,0)[lc]{0}}
\put(5.90,-0.50){\framebox(0.20,0.05)[cc]{}}
\put(6.20,-0.45){\makebox(0,0)[lc]{$\pi$}}
\put(6.90,-0.50){\framebox(0.20,0.05)[cc]{}}
\put(7.20,-0.45){\makebox(0,0)[lc]{$\pi$}}
\put(7.90,-0.50){\framebox(0.20,0.05)[cc]{}}
\put(8.20,-0.45){\makebox(0,0)[lc]{$\pi$}}
%\put(8.90,-0.50){\framebox(0.20,0.05)[cc]{}}
%\put(9.20,-0.45){\makebox(0,0)[lc]{0}}
\put(0.60,8.41){\line(1,-1){8.84}}
\end{picture}
\end{tabular}
\end{center}
}
}


\section{Summary}
\frame{
\frametitle{Summary}
{
\begin{itemize}

\item<1->
Refractive indices smaller than unity, and thus ``faster-then-light'' propagation in exotic vacua may be feasible.
\item<1->
These phenomena need not necessarily lead to causality violations (cf. G\"odel's solution to the Hilbert-Einstein equations of general relativity).
\item<1->
They might be utilized for optical computers (possible based on generalized beam splitters) which perform at superluminal speed.
\end{itemize}
}
}

\frame{

\centerline{\Large Thank you for your attention!}

\begin{center}
$\widetilde{\qquad \qquad }$
$\widetilde{\qquad \qquad}$
$\widetilde{\qquad \qquad }$
\end{center}
 }

\end{document}
