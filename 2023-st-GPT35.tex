\documentclass[%
  reprint,
 %twocolumn,
 %onecolumn,
 %superscriptaddress,
 %groupedaddress,
 %unsortedaddress,
 %runinaddress,
 %frontmatterverbose,
 %preprint,
 showpacs,
 showkeys,
 preprintnumbers,
 nofootinbib,
 %nobibnotes,
 %bibnotes,
 amsmath,amssymb,
 aps,
 % prl,
 pra,
 % prb,
 % rmp,
 %prstab,
 %prstper,
  longbibliography,
 %floatfix,
 %lengthcheck,%
 ]{revtex4-2}

%\usepackage{cdmtcs-pdf}

\usepackage[dvipsnames]{xcolor}

\usepackage{mathptmx}% http://ctan.org/pkg/mathptmx

\usepackage{amssymb,amsthm,amsmath}

\usepackage{tikz}
\usetikzlibrary{calc,decorations.pathreplacing,decorations.markings,positioning,shapes}

\usepackage[breaklinks=true,colorlinks=true,anchorcolor=blue,citecolor=blue,filecolor=blue,menucolor=blue,pagecolor=blue,urlcolor=blue,linkcolor=blue]{hyperref}
\usepackage{graphicx}% Include figure files
\usepackage{url}


\begin{document}

\title{(Re)Construction of Quantum Space-Time: Transcribing Hilbert Into Configuration Space}


\author{Karl Svozil}
\email{svozil@tuwien.ac.at}
\homepage{http://tph.tuwien.ac.at/~svozil}

\affiliation{Institute for Theoretical Physics,
TU Wien,
\\
Wiedner Hauptstrasse 8-10/136,
1040 Vienna,  Austria}


\date{\today}

\begin{abstract}
Space-time in quantum mechanics is about bridging Hilbert and configuration space. Thereby, an entirely new perspective is obtained by replacing the Newtonian space-time theater with the image of a presumably high-dimensional Hilbert space, through which space-time becomes an epiphenomenon construed by internal observers.
\end{abstract}

\keywords{space-time frames, synchronization, induced relativity, quantum space-time}


\maketitle

%\begin{widetext}


\section{It-from-Click~Imaging}

This paper continues efforts to address the implications of quantum entanglement in the absence of gravitation for the construction of space-time coordinate frames.
Previous papers have focused on context communication costs for simulating uniform quantum correlations~\cite{svozil-2022-epr} and conducted a detailed analysis
of the violation of Boole's conditions of possible (classical) experience by quantum mechanics~\cite{svozil-2022-epr2}.

Physical categories and conceptualizations, such as time and space,
are formed in minds in accordance with the operational means available to observers.
They are, thus, idealistic~\cite{stace1} and epistemic and,~therefore, historic, preliminary, contextual, and~not~absolute.


Operationalists such as Bridgman~\cite{bridgman36}, Zeilinger~\cite{sv1,zeil-99}, or~Summhammer~\cite{Summhammer_1994}
have emphasized the empirical aspect of physical category formation~\cite{Hardy_2007}.
Hertz also highlighted the idealistic nature of physical `images' (or mental categories)
that internal observers construct to represent observations,
and how these formal structures should remain consistent with, and~connected to, empirical events or outcomes~\cite{hertz-94e}:
``We form for ourselves images or symbols of external objects; and the form which we give them is such that the necessary
consequences of the images in thought always mirror the images of the necessary consequences in nature of the things pictured''.
From these perspectives, physical theories may seem to reflect ontology.
However, their core `images' turn out to be epistemic constructions.

In the subsequent discussion, our focus will be on the construction of space-time frames, not in a Newtonian or Kantian sense,
portrayed as premeditated `as they are' and providing a sort of theater and arena in which (quantum) events take place,
but rather in a Leibnizian sense, constructing them as they can be by the available operational means~\cite{Ballard_1960}.
As stated by Leibniz~\cite{Leibniz2000Mar} (p.~14), ``space
[[is]]  something purely relative, as~time is---[[space is]] an order of coexistences, as~time is an order of~successions''.

Zooming in on the program of `it-from-click' (re)construction of space-time from elementary quantum events, the~roadmap is quite straightforward: as quanta are formalized by Hilbert space entities, such an endeavor must somehow `translate' arbitrary dimensional Hilbert spaces into four-dimensional configuration space equipped with space-time~frames.



\section{Conventions and the Necessity of Parameter Independence and, Thus, Choice}


We need to be particularly aware of the conventions involved in constructing space-time frames.
One such convention is the frame-independent determination of the velocity of light~\cite{pet-83,peres-84}
in the International System of Units (SI),
which means that light cones remain unchanged.
Alongside the assumption of bijective mappings of space-time point labels in distinct coordinate frames,
this convention, preserving the quadratic distance (Minkowski metric) of zero,
leads to affine Lorentzian transformations~\cite{alex1,lester}.

These conventions formally imply and define the Lorentz transformations of the theory of special relativity.
They are inspired by physics, but lack inherent physical content themselves.
Their physical significance arises from the preservation of the form invariance of equations of motion, such as Maxwell's equations,
under Lorentz transformations that include (the conventionally defined~\cite{pet-83,peres-84} constant and frame-independent)
velocity of~light.

With regard to synchronization within inertial frames,
it is essential to keep in mind that quantum measurements essentially amount to
`(ir)reversible'~\cite{PhysRevA.25.2208,greenberger2,Ma22012013} clicks in some detectors.
As long as those detections are statistically independent, we can synchronize time at different locations using
radar (`round-trip', `two-way') coordinates obtained by sending a (light-in-vacuum) signal back and forth
between the respective locations,
a procedure known as Poincar\'e--Einstein synchronization~\cite{Poincare1900,Poincare1904,ein-05,Einstein_1910,Jammer2006Nov,Minguzzi_2011}.
As pointed out by Poincar\'e in 1900~\cite{Poincare1900} (p.~272) (see also Poincar\'e's 1904 paper~\cite{Poincare1904} (p.~311)),
suppose that two embedded observers $A$ and $B$ are positioned at different points of a moving frame, and~are unaware of their shared motion,
and synchronize their clocks using light signals.
These observers believe, or~rather assume or define, that the signals travel at the same speed in both directions.
They conduct observations involving signals crossing from $A$ to $B$ and then, vice~versa, from~$B$ to $A$.
Their synchronized `local', intrinsic, time can be, according to Einstein~\cite[]{ein-05} (p.~894),
defined by (similar) clocks that have been adjusted such that, for~the light emission and return times $t_A$ and $t_A'$ at $A$,
and the reception and emission time $t_B$ at $B$, $t_B - t_A = t_A' - t_B$.
This type of synchronization, if~performed with light rays in vacuum, is consistent with the International System of Units (SI) standards.

A formal expression of the statistical independence of two events, outcomes, or~observables, $L$ and $R$,
is the fact that their joint state $\Psi_{LR}$ can be written as the product
of their individual states $\Psi_{L}$ and $\Psi_{R}$; that is, $\Psi_{LR} = \Psi_{L} \Psi_{R}$.
These states are then nonentangled and separable with respect to observables $L$ and $R$.

However, what about entangled states? In this case, independence cannot be assumed as,
by definition, the~joint state is not a product of the constituent states.
Quantum entangled states are encoded relationally~\cite{schrodinger-gwsidqm2,zeil-99,zeil-Zuk-bruk-01}.
Since the product rule does not hold for quantum entangled states, we cannot assume that the respective individual outcomes
are guaranteed to be mutually separate or mutually distinct in these~observables.

\section{Inseparability and the Lack of Mutual, Relational~Choice}

The forthcoming argument will contend that entangled quantum states do not appear to provide the means
for such spatial order of coexistences, nor for any order of successions.
Entangled states lack distinctness between their constituents.
A formal expression of such quantum relational encoding is the outcome dependence
of two respective events, outcomes, or~observations $L$ and $R$ belonging to the registrations of those entangled particle~pairs.

However, outcomes on either side $L$ or $R$ maintain their statistical parameter independence,
which means that any parameter measured at $L$ does not affect the outcome
or any other operationally accessible observable at $R$, and~vice~versa.
In Shimony's terminology~\cite{shimony2,shimony_1993}, ``an experimenter
at $R$, for~example, cannot affect the statistics of outcomes at $L$ by selective measurements''.
This can be ensured by the indefiniteness of the respective outcomes, which appear irreducibly random~\cite{zeil-05_nature_ofQuantum}
with respect to a range of physical operational means deployable by an intrinsic~observer.



State factorization guarantees a specific feature that is crucial for radar coordinates: choice.
Simultaneity conventions require the capacity to independently select space-time labels for both types of measurements
(parameter independence) and their outcomes,
regardless of what is being measured and recorded elsewhere.
Outcome independence, along with the resulting temporal and spatial distinctiveness,
is essential for establishing any internally operational space-time scale.
%However, without such temporal and spatial distinctiveness and the associated possibility of choice, no space-time scale, in particular, no clock, can be generated.

Without the freedom to make choices regarding spatiotemporal labeling,
the concept of clocks and the measurement of space and time they provide becomes unattainable.
Indeed, distinct labels require a distinction among entities to be labeled.
However, for~quantum entangled states that have traded individuality for relationality,
there is no distinction concerning the respective~observables.


Suppose, for~the sake of demonstration, an~isolated mini-universe composed
of entangled states, such as the singlet Bell state $\vert \Psi^-_{12} \rangle$ from the Bell basis
\begin{equation}
\begin{split}
\vert \Psi^\pm_{12} \rangle
=
\frac{1}{2}\left(
\vert 0_1 1_2 \rangle
\pm
\vert 1_1 0_2 \rangle
\right)
,\;
\vert \Phi^\pm_{12} \rangle
=
\frac{1}{2}\left(
\vert 0_1 0_2 \rangle
\pm
\vert 1_1 1_2 \rangle
\right)
.
\end{split}
\label{2023-st-EPRBstates}
\end{equation}

The first and second (from left to right) entries
refer to the first and second constituents, respectively.
Typically, these constituents are understood to be spatially separated, preferably under strict Einstein locality conditions~\cite{wjswz-98}.
For example, Einstein, Podolsky, and~Rosen (EPR) employed such spatially separated configurations to argue against the `completeness' of
quantum mechanics~\cite{epr,Howard1985171}.

However, we do not wish to confine ourselves to space-like entanglement.
We also aim to encompass time-like entanglement.  This type of entanglement can---in the customary space-time frames that
we assume to be ad~hoc creations of certain nonentangled elements, such as light rays of classical optics,
in the standard Poincar\'e--Einstein protocols mentioned earlier---be generated through processes such as delayed-choice entanglement swapping.
Formally, achieving this involves reordering the product $\vert \Psi^-_{12} \Psi^-_{34} \rangle$,
expressed in terms of the four individual product states
$\vert \Psi^+_{14}   \Psi^+_{23}  \rangle$,
$\vert \Psi^-_{14}   \Psi^-_{23}  \rangle$,
$\vert \Phi^+_{14}   \Phi^+_{23}  \rangle$, and
$\vert \Phi^-_{14}   \Phi^-_{23}  \rangle$
of the Bell bases of the `outer' (14) and `inner' (23)
particles~\cite{Zuk-1993-entanglementswapping,Megidish_2013,peres-DelayedChoiceEntanglementSwapping,svozil-2016-sampling}.
Bell state measurements of the latter, `inner' particles yield a rescrambling of the `outer'
correlations. Hence, postselecting the `inner' pair (23) results in the desired `outer'
Bell states (14), respectively.
In more detail, in~the  Bell basis~(\ref{2023-st-EPRBstates}),
\begin{equation}
\begin{split}
\vert \Psi_{12}^- \Psi_{34}^- \rangle &=   \frac{1}{2} \left(\vert \Psi_{14}^+ \Psi_{23}^+  \rangle - \vert \Psi_{14}^- \Psi_{23}^-  \rangle     - \vert \Phi_{14}^+ \Phi_{23}^+  \rangle + \vert \Phi_{14}^- \Phi_{23}^- \rangle \right), \\
\vert \Psi_{12}^+ \Psi_{34}^+ \rangle &=   \frac{1}{2} \left(\vert \Psi_{14}^+ \Psi_{23}^+  \rangle - \vert \Psi_{14}^- \Psi_{23}^-  \rangle     + \vert \Phi_{14}^+ \Phi_{23}^+  \rangle - \vert \Phi_{14}^- \Phi_{23}^- \rangle \right), \\
\vert \Phi_{12}^- \Phi_{34}^- \rangle &=   \frac{1}{2} \left(-\vert \Psi_{14}^+ \Psi_{23}^+  \rangle - \vert \Psi_{14}^- \Psi_{23}^-  \rangle   + \vert \Phi_{14}^+ \Phi_{23}^+  \rangle + \vert \Phi_{14}^- \Phi_{23}^- \rangle \right), \\
\vert \Phi_{12}^+ \Phi_{34}^+ \rangle &=   \frac{1}{2} \left(\vert \Psi_{14}^+ \Psi_{23}^+  \rangle + \vert \Psi_{14}^- \Psi_{23}^-  \rangle   + \vert \Phi_{14}^+ \Phi_{23}^+  \rangle + \vert \Phi_{14}^- \Phi_{23}^- \rangle \right).
\end{split}
\label{2023-st-MagicBellBasisstates1factoring}
\end{equation}

The first of these four equations undergoes careful analysis in References~\cite{Zuk-1993-entanglementswapping,Megidish_2013,peres-DelayedChoiceEntanglementSwapping}, while the remaining three represent generalizations of this~analysis.
In the `magic' Bell basis where $\vert \Psi^- \rangle $ and $\vert \Phi^+ \rangle $
are multiplied by the imaginary unit  $i$~\cite{Bennett_1996,peres-DelayedChoiceEntanglementSwapping}, the~relative phases change~accordingly.

Delay lines serve as essential components for temporal entanglement.
These delay lines could, in~principle, also lead to mixed temporal-spatial quantum correlations, where
for instance, pairs (12) are spatially entangled while pairs (34) are temporally entangled,
resulting in an `outer' pair (14) that is both spatially and temporally entangled.
As a consequence, we may consider the particle labels $1, \ldots, 4$, which have been written as subscripts, to~stand for generic spacetime coordinates; that is,
\begin{equation}
\begin{split}
1 \equiv \left(x^1_1,x^2_1,x^3_1,x^4_1=t_1\right),  \\
2 \equiv \left(x^1_2,x^2_2,x^3_2,x^4_2=t_2\right),  \\
3 \equiv \left(x^1_3,x^2_3,x^3_3,x^4_3=t_3\right),  \\
4 \equiv \left(x^1_4,x^2_4,x^3_4,x^4_4=t_4\right).
\end{split}
\label{2023-st-stlabels}
\end{equation}

Equation~(\ref{2023-st-stlabels}) is not an `equation' in the strict sense but represents equivalences, as~indicated by the equivalence signs.
The operationalization of the space-time coordinates referred to in Equation~(\ref{2023-st-stlabels}) by radar coordinates, using quasi-classical protocols for quantized systems, is a nontrivial task.
However, within~the constraints of preparation and measurement, it constitutes a standard procedure already mentioned by Poincar\'e and~Einstein.


We note that temporally entangled shares (as well as mixed temporal-spatial ones) could lead to standard
violations of Bell--Boole-type inequalities---for instance, at~a single point in space but at different times.
The derivation seems to be straightforward:
all that is required is a respective Hull computation
of the classical correlation polytope~\cite{froissart-81,pitowsky-86},
yielding inequalities that represent the edges of
the classical polytope,
followed by the evaluation of the (maximal) quantum violation thereof~\cite{cirelson:80,filipp-svo-04-qpoly-prl}.
One of the reasons for the seamless transfer of spatial and temporal variables
is their interoperability and their realization using delay lines, when~necessary.

While considering the question of whether and how such entangled shares could lead to space-time scales, and~ultimately frames,
or disallows their operational creation,
we make three observations:
First, the~two `constituents' of the relationally entangled share reveal themselves, if~compelled
into individual events, through two random outcomes that are mutually dependent due to quantum correlations
in the form of the quantum cosine expectation laws.
These single individual outcomes are expected to be independent of the experiments or parameters
applied on the respective `other side' or at the `other~time'.

Second, these correlations surpass the classical linear correlations~\cite{Peres222}
for almost all relative measurement directions (except for the collinear and orthogonal directions).
However, since these correlations are only dependent on (relative) outcomes and not on parameters, this does not lead to
inconsistencies with classical space-time scales generated by the conventional classical Poincar\'e--Einstein
synchronization convention. Indeed, even `stronger-than-quantum' correlations, such as a Heaviside
correlation function~\cite{svozil-krenn,svozil-2004-brainteaser} would, under~these conditions, not result in violations of causality through faster-than-light~\mbox{signaling}.

Third, since individual outcomes cannot be controlled, any synchronization convention and protocol
that depends on controlled outcomes cannot be carried out with entangled shares,
as there is no means of transmitting (arrival and departure) information
`across those shares'.
Due to  parameter independence, any space-time labeling using those outcomes is arbitrary.
For instance, `synchronizing' distant clocks (not with light ray exchange, but)
by the respective correlated outcomes of entangled particles, such as from spin state or polarization measurements,
results in correlated but random temporal scales.
These scales cannot be brought into any concordance with `local' time scales generated by the conventional
classical Poincar\'e--Einstein synchronization convention mentioned earlier.


Signaling from one space-time point to another assumes choice,
yet again, the~form of relational value definiteness that comes at the expense of individual value definiteness,
originating from the unitarity of quantum evolution,
between two or more constituents of a quantum entangled share
prevents signaling across its constituents.
Therefore, in~the hypothetical scenario of a universe composed of entangled particles,
Poincar\'e--Einstein synchronization may require classical means that are unavailable for entangled~particles.



%\section{Patchwork of space-time frames}



\section{Orthogonality of Configuration Space from Hilbert~Space}

Although entanglement does not provide a means for scale synchronization, it can be utilized for synchronizing directions, as well as orthogonality among different~frames.

Suppose that all observers agree to `measure the same type of observable', such as spin or linear polarization.
It is important to note that, at this stage, we have not yet established a spatial frame. Therefore, for~example, an~observable
like the `direction of spin' (or, for~photons, linear polarization) is initially undefined.
It must be defined in terms of quantum mechanical entities, such as the state~(\ref{2023-st-EPRBstates}), and~observables.
Ultimately, this process involves the interpretation of clicks in a~detector.

Directional synchronization of spatiotemporal frames can be established, for~instance, through the state~(\ref{2023-st-EPRBstates})
by employing successive measurements of particles in that state.
In this manner, the~directions can be synchronized by maximizing~correlations.

Three- and four-dimensionality can also be established by exploiting correlations:
(mutual) spatiotemporal orthogonality can be established by (mutually) minimizing the absolute value of these correlations.
In this manner, Hilbert space entities are indirectly translated into the orthogonality structure of the configuration~space.


\section{Controllable Nonlocality and Parameter Dependence of Outcomes Due to Nonlinearity of Quantum Field Theory?}

{We might hope that the addition of nonlinearity via interactions or statistical \mbox{effects---for}} example, higher-order perturbation expansions---might help overcome the parameter independence of outcomes in an EPR-type setup.
However, as~of now, there is no indication of any violation of Einstein locality in field theory~\cite{shirokov,Hegerfeldt_1998,Perez_PhysRevD.16.315,Svidzinsky-PhysRevResearch.3.013202}.


In my earlier publications~\cite{svozil-slash}, I have speculated that if one constituent of an EPR pair were to enter a region
of high or low density of a particular particle type---for instance, `boxes of particles in state $\vert 0 \rangle$'---then stimulated emission might
encourage the corresponding state of the constituent `to materialize' with a higher or lower probability.
This, in~turn, could be a scenario for the parameter dependence of outcomes, even under strict Einstein locality~conditions.


\section{Summary and~Afterthoughts}

As argued earlier, there is no independent choice among the individual outcomes of entangled particles:
an observer at the `one constituent end' of an entangled share has no ability to select or establish a specific time as a pointer~reading.

Nevertheless, it is important to note that not all observables of a collection of particles may be entangled; some could be factorizable. In~this case, the~latter type of observables may still be applicable for
the creation of relativistic space-time frames, unlike the entangled~ones.

These considerations are not directly related to the `problem of (lapse of) time' that has led to the notion of a
fictitious stationary `external' versus an `intrinsic' time~\cite{Page_1983,Wootters_1984,Moreva_2014}
by equating it with the measurement problem in quantum~mechanics.

The adage that ``If $\ldots$ two spacetime regions are spacelike separated, then the operators should commute''~\cite{Hardy_2007}
implicitly supposes two~assumptions:
\begin{itemize}
\item[(i)] First, Einstein's separation criterion (German `Trennungsprinzip'~\cite[]{Meyenn-2011} (pp. 537--539)),
which states that relativity theory, and~in particular its causal structure determined by light cones,
applies to observables formalized as~operators.

Recall that
Einstein, in a letter to Schr\"odinger~\cite{Meyenn-2011,Howard1985171},
emphasized (wrongly in my interpretation of the argument)
that following a collision that entangles the constituents $L$ and $R$,
the compound state could be thought of as comprising the actual state of $L$ and the actual state of $R$.
Einstein argues that those states should be considered unrelated---in particular, there is no relationality.
Therefore, the~real state of $L$ (due to possible spacelike separation)
cannot be influenced by the type of measurement conducted on $R$.

Our approach diverges from Einstein, insofar as we deny the existence
of a preexisting Newtonian space-time theater, even in the modified version proposed by Poincar\'e
and Einstein.
Therefore, we cannot depend on a preexisting space-time structure for operators to~commute.

\item[(ii)] Second, it assumes that states are distinct from operators, even though
pure states can be reinterpreted as the formalization of observables; specifically, as~the assertion that the system is in the respective state.
\end{itemize}



Since Poincar\'e--Einstein synchronization via radar coordinates requires a choice and thus parameter dependence,
the~utilization of entangled states becomes impossible.
Hence, we are restricted to separable states. The~separability and value definiteness of components within a physical system ultimately
reduces to the measurement problem in quantum mechanics.
This measurement problem, which involves understanding how an entangled system experiences `individualization' under strictly unitary transformations, with~associated value definite information on individual components of the system, remains notoriously~unresolved.

We must acknowledge that, at~least for now,
in the case of relationally encoded entangled quantum states, there is no spatiotemporal resolution. However, due to parameter independence, this type of `nonlocality' cannot be exploited for signaling or radar coordination.
Without individuation and measurement, there can be no operational significance assigned to space-time.
From this perspective, quantum coordinatization reduces to quantum measurements which,
at least in the author's view, remains unresolved, although~it is taken for granted for all practical purposes (FAPP)~\cite{bell-a}.

A final caveat seems to be in order: The matters and issues discussed in the article could not be fully resolved.
However, attempts towards their resolution in terms of entangled systems have been made.
One legitimate interpretation is that entangled states cannot be used to construct space-time
 frames via the Poincar\'e--Einstein synchronization procedure, resulting in radar coordinates.
This might be resolved by adding the particular context of coordinatization and acknowledging means relativity.
Thereby, a~framework for `relativizing relativity' has been~discussed.


\vspace{6pt}

\begin{acknowledgments}


This paper is intended as a contribution to a Symposium on the Foundations of Quantum Physics celebrating Danny Greenberger's 90th~birthday.

This research was funded in whole or in part by the Austrian Science Fund (FWF), Grant-DOI: 10.55776/I4579.
For open access purposes, the author has applied a CC BY public copyright license to any author accepted manuscript version arising from this submission.


No new data were created or analyzed in this study. Data sharing is not applicable to this article.


\end{acknowledgments}

\bibliography{svozil}

%\end{widetext}
\end{document}



%Thereby, quantum entanglement poses challenges by exhibiting outcome nonlocality while at the same time restricting the physical means to generate space-time frames.

Then, a time scale in $n$ different locations can be established by labeling points on those frames
according to the arrivals of the respective constituents of those successive states.
We may state that the time of the specific space-time point (among $n$)
when the constituent of the first share is detected (by a click in one of the detectors labeled ``0'' or ``1'',
regardless of which one) is temporally labeled as "0".

Sequentially, the specific space-time point when the constituent of the second (and subsequent) share
is detected is temporally labeled as "1" (and so on).
In fact, if all clocks are identical,
we only require one share to determine the time offset accurately.

Alternativey, we my chose the temporal labels, in a BB84~\cite{benn-84} type protocol,
not by the arrival time(s) but by the values observed, that is, by the value of the detector clicks.
Such synchronization conventions need not be restricted to quantum mechanical entangled shares,
but for quantum shares there is no apparent other way of how to proceed, as points appear not separated.
Indeed, if


\subsection{Construction of internal operational temporal scales}

In order to create time scales we need to assume that some form of temporal succession or order can be established.
This assumption presupposes temporal factorizability of ``successive'' states.
But any such order of the shares is not inherent in the quantum states.
Indeed, temporally entangled states
exhibit indefinite causal order~\cite{Oreshkov_2012,GoswamiPhysRevLett.121.090503}.

For the sake of a classical reconstruction of time frames we shall assume that there exists a temporal succession of states.
Suppose the $n$ observers have identical clocks.
We could, for instance, nominally fixate
two or more of entangled shares of the type of the ones in Equation~(\ref{2023-st-GHZstates})
in a pre-arranged order.

(* Mathematica code *)

psi12m = (1/Sqrt[2]) (H1*V2 - V1*H2);
psi12p = (1/Sqrt[2]) (H1*V2 + V1*H2);
phi12m = (1/Sqrt[2]) (H1*H2 - V1*V2);
phi12p = (1/Sqrt[2]) (H1*H2 + V1*V2);

psi34m = (1/Sqrt[2]) (H3*V4 - V3*H4);
psi34p = (1/Sqrt[2]) (H3*V4 + V3*H4);
phi34m = (1/Sqrt[2]) (H3*H4 - V3*V4);
phi34p = (1/Sqrt[2]) (H3*H4 + V3*V4);

psi14m = (1/Sqrt[2]) (H1*V4 - V1*H4);
psi14p = (1/Sqrt[2]) (H1*V4 + V1*H4);
phi14m = (1/Sqrt[2]) (H1*H4 - V1*V4);
phi14p = (1/Sqrt[2]) (H1*H4 + V1*V4);

psi23m = (1/Sqrt[2]) (H2*V3 - V2*H3);
psi23p = (1/Sqrt[2]) (H2*V3 + V2*H3);
phi23m = (1/Sqrt[2]) (H2*H3 - V2*V3);
phi23p = (1/Sqrt[2]) (H2*H3 + V2*V3);




FullSimplify[  psi12m*psi34m]

FullSimplify[  psi12m*psi34m == (1/2) (
psi14p*psi23p
- psi14m*psi23m
- phi14p*phi23p
+ phi14m*phi23m
) ]

(* magical base *)

psi12m = (I/Sqrt[2]) (H1*V2 - V1*H2);
psi12p = (1/Sqrt[2]) (H1*V2 + V1*H2);
phi12m = (1/Sqrt[2]) (H1*H2 - V1*V2);
phi12p = (I/Sqrt[2]) (H1*H2 + V1*V2);

psi34m = (I/Sqrt[2]) (H3*V4 - V3*H4);
psi34p = (1/Sqrt[2]) (H3*V4 + V3*H4);
phi34m = (1/Sqrt[2]) (H3*H4 - V3*V4);
phi34p = (I/Sqrt[2]) (H3*H4 + V3*V4);

psi14m = (I/Sqrt[2]) (H1*V4 - V1*H4);
psi14p = (1/Sqrt[2]) (H1*V4 + V1*H4);
phi14m = (1/Sqrt[2]) (H1*H4 - V1*V4);
phi14p = (I/Sqrt[2]) (H1*H4 + V1*V4);

psi23m = (I/Sqrt[2]) (H2*V3 - V2*H3);
psi23p = (1/Sqrt[2]) (H2*V3 + V2*H3);
phi23m = (1/Sqrt[2]) (H2*H3 - V2*V3);
phi23p = (I/Sqrt[2]) (H2*H3 + V2*V3);




FullSimplify[  psi12m*psi34m]

FullSimplify[  psi12m*psi34m == - (1/2) (psi14p*psi23p + psi14m*psi23m + phi14p*phi23p + phi14m*phi23m) ]

FullSimplify[  psi12p*psi34p ==   (1/2) (psi14p*psi23p + psi14m*psi23m - phi14p*phi23p - phi14m*phi23m) ]

FullSimplify[  phi12m*phi34m == - (1/2) (psi14p*psi23p - psi14m*psi23m + phi14p*phi23p - phi14m*phi23m) ]

FullSimplify[  phi12p*phi34p == - (1/2) (psi14p*psi23p - psi14m*psi23m - phi14p*phi23p + phi14m*phi23m) ]



(* standard Bell base *)

psi12m = (1/Sqrt[2]) (H1*V2 - V1*H2);
psi12p = (1/Sqrt[2]) (H1*V2 + V1*H2);
phi12m = (1/Sqrt[2]) (H1*H2 - V1*V2);
phi12p = (1/Sqrt[2]) (H1*H2 + V1*V2);

psi34m = (1/Sqrt[2]) (H3*V4 - V3*H4);
psi34p = (1/Sqrt[2]) (H3*V4 + V3*H4);
phi34m = (1/Sqrt[2]) (H3*H4 - V3*V4);
phi34p = (1/Sqrt[2]) (H3*H4 + V3*V4);

psi14m = (1/Sqrt[2]) (H1*V4 - V1*H4);
psi14p = (1/Sqrt[2]) (H1*V4 + V1*H4);
phi14m = (1/Sqrt[2]) (H1*H4 - V1*V4);
phi14p = (1/Sqrt[2]) (H1*H4 + V1*V4);

psi23m = (1/Sqrt[2]) (H2*V3 - V2*H3);
psi23p = (1/Sqrt[2]) (H2*V3 + V2*H3);
phi23m = (1/Sqrt[2]) (H2*H3 - V2*V3);
phi23p = (1/Sqrt[2]) (H2*H3 + V2*V3);




FullSimplify[  psi12m*psi34m]

FullSimplify[  psi12m*psi34m ==   (1/2) (psi14p*psi23p - psi14m*psi23m - phi14p*phi23p + phi14m*phi23m) ]

FullSimplify[  psi12p*psi34p ==   (1/2) (psi14p*psi23p - psi14m*psi23m + phi14p*phi23p - phi14m*phi23m) ]

FullSimplify[  phi12m*phi34m ==   (1/2)(-psi14p*psi23p - psi14m*psi23m + phi14p*phi23p + phi14m*phi23m) ]

FullSimplify[  phi12p*phi34p ==   (1/2) (psi14p*psi23p + psi14m*psi23m + phi14p*phi23p + phi14m*phi23m) ]




##################################################################################################################

RR1:


"This article discusses an interesting topic, namely, the possible emergence of an effectively classical space-time background from a more fundamental quantum description, expressed in terms of states in a Hilbert space."

I fully agree with the Referee that the topic of the article is perztinent and well worth studying.

"Unfortunately, it contains various flaws and shortcomings, which, I believe, make it unsuitable for publication in its present form. My main criticism of this paper is that it does not contain any new mathematical results, or, in fact, any non-textbook mathematical results at all. It contains only three equations. Equation (1) is simply the definition of the Bell basis for a two-fermion system and Eq. (2) gives its extension to four fermions. These are textbook results, while Equation (3) is not really an equation at all, as it has no well-defined mathematical meaning."

I agree with the Referee that "Equation (1) is simply the definition of the Bell basis for a two-fermion"; but I need this definition for further discussion and the numbering for reference to it.

I respectfully disagree with the Referee that "Eq. (2) gives its extension to four fermions system". Besides the fact that an extension to four fermions system would require 2^4=16 equations and not four, as displayed in Equation (2), those equations have very specific meanings: the first one of the four equations is subject to a careful analysis in References 31-33, and the remaining three are generalizations of this analysis.

Therefore, in order to avoid such misunderstandings, I have now added the following sentence after Equation (2): 'The first of these four equations undergoes careful analysis in References \cite{Zuk-1993-entanglementswapping Megidish_2013,peres-DelayedChoiceEntanglementSwapping}, while the remaining three represent generalizations of this analysis.'

I also respectfully disagree with the Referee on his evaluation "Equation (3) is not really an equation at all, as it has no well-defined mathematical meaning." Indeed, Equation (3) is no "equation" in the strict sense, but represent equivalences, as indicated by the equivalence signs.

Therefore, in order to avoid such misunderstandings, I have now added the following sentence after Equation (3): 'Equation~(\ref{2023-st-stlabels}) is not an `equation' in the strict sense but represents equivalences, as indicated by the equivalence signs.'


"The author seems to imply, in Eq. (3) and the surrounding text, that each particle in a four-fermion system can be identified with an abstract space-time point, (x0,x1,x2,x3). But what, exactly, doe this mean? There is, here, no concrete �transcription� from the states of the Hilbert space to the space-time submanifold of the classical configuration space, as implied by the title of the article. There is no well-defined map from one to the other."

I am referring to operational space-time coordinates displayed in Equation (3), not to particles. I agree that the operationalization of such points by radar coordinates, using quasi-classical protocols for quantized systems, is a nontrivial task. However, it is a standard procedure already mentioned by Poincar\'e and Einstein.

Therefore, in order to clarify the situation I have added the following sentence: 'The operationalization of the space-time coordinates referred to in Equation~(\ref{2023-st-stlabels}) by radar coordinates, using quasi-classical protocols for quantized systems, is a nontrivial task. However, within the constraints of measurement, it constitutes a standard procedure already mentioned by Poincar\'e and Einstein.'


"The remaining discussion in the text raises some interesting points and issues, but, without a concrete mathematical result at its core, adds little to the existing literature on the emergence of space-time from quantum states. The author�s main point is that entangled states cannot be used to construct space-time frames via the Poincare-Einstein synchronisation procedure. This is undoubtedly true, and worth pointing out, but what then? The paper�s title suggests that the author proposes an alternative construction, or some �quantum generalisation� of the notion of a classical space-time reference frame, but, in fact, he does not attempt this."

I fully agree with the Referee about the importance of the issues raised. I also agree that the discussed situation is not fully resolved. However, attempts towards its resolution in terms of entangled systems are being made.

Therefore, I have added the following sentence at the end of the article: 'A final caveat seems to be in order: The matters and issues discussed in the article could not be fully resolved. However, attempts towards their resolution in terms of entangled systems have been made. One legitimate interpretation is that entangled states cannot be used to construct space-time frames via the Poincar\'e-Einstein synchronization procedure, resulting in radar coordinates. This might be resolved by adding the particular context of coordinatization and acknowledging its means in relativity. Thereby, a framework for 'relativizing relativity' has been discussed."

"Though it is permissible for special issues to admit more speculative articles, and for the usual criteria for the publication of a research article to be relaxed to a certain degree, I do not believe that this article is, ultimately, substantive enough to warrant publication. Regretfully, I cannot recommend it for publication in Entropy."

I hope that, with the additions and explanations mentioned earlier, the manuscript can be published.

I also acknowledge that, with all due respect, I may disagree with the Referee about the novelty of the discussion and the findings.


~~~~~~~~~~~~~~~~~~~~~~~~~~~~~~~~~~~~~~~~~~~~~~~~~~~~~~~~~~~~~~~~~

RR2:


"This manuscript can be regarded as the serial efforts of Refs 1 and 2, which explore the implication of quantum entanglement in constructing space-time frame. Therein, Poincar\'e�Einstein synchronization is analyzed in terms of spatial and temporal entanglement. Indeed, it deserves its publication in Entropy."

I kindly thank the Referee for this evaluation.

"To make the manuscript more friendly to readers, figures can be useful to explain the connection between Poincar\'e�Einstein synchronization and spatial/temporal entanglement."

I respectfully and kindly note that, in this particular context of synchronization by radar coordinates, figures might not, for a considerable portion of the audience, contribute to a further understanding of the discussion and the argument.

~~~~~~~~~~~~~~~~~~~~~~~~~~~~~~~~~~~~~~~~~~~~~~~~~~~~~~~~~~~~~~~~~

RR3:

"This paper addresses the question of how spacetime structure can emerge out of the formalism of quantum mechanics. Quantum mechanics operates naturally within Hilbert spaces, of generally large dimension, and how observers of phenomena in such spaces apprehend the usual spacetime as an �epiphenomenon� is the main question studied in this paper. Among the measurements made at the microscopic level out of which spacetime is to be constructed are ones on entangled states, and these pose a special problem. The author illustrates these difficulties by looking at four-particle states that are direct products of Bell states between two pairs and reexpressing in terms of Bell states of particles that were initially uncoupled. His conclusion, after a detailed discussion, is captured in the statement �We must acknowledge that, at least for now, in the case of relationally encoded entangled quantum states, there is no spatiotemporal resolution.� "

The author�s discussion of the challenges posed by this approach is bound to be of interest to others who think about the problem, and for that reason I would recommend publication of this paper.


I kindly thank the Referee for this evaluation. I also refer to the caveat added at the end of the manuscript, which may align with some of the thoughts of the Referee.



~~~~~~~~~~~~~~~~~~~~~~~~~~~~~~~~~~~~~~~~~~~~~~~~~~~~~~~~~~~~~~~~~
~~~~~~~~~~~~~~~~~~~~~~~~~~~~~~~~~~~~~~~~~~~~~~~~~~~~~~~~~~~~~~~~~

Academic Editor Notes

"Based on the reviewers' reply, especially the concerns of Reviewer 1,the manuscript requires a final minor revision. The author should comply with the remaining criticism about a proper reference to papers by Einstein or Poincar\'e, which are mentioned but not explicitly cited in the manuscript. These works should be included in the reference list because they can be a useful reference to the readers."

As requested and suggested by the Academic Editor, I have added

* two references;

* a more detailed explanation of Poincare-Einstein synchronization, quoting those new references, and Einstein's 1905 paper:

"As pointed out by Poincar\'e in 1900~\cite[p.~272]{Poincare1900} (see also Poincar\'e's 1904 paper~\cite[p.~311]{Poincare1904}),
suppose that two embedded observers $A$ and $B$ are positioned at different points of a moving frame, and are unaware of their shared motion,
and synchronize their clocks using light signals.
These observers believe, or rather assume or define, that the signals travel at the same speed in both directions.
They conduct observations involving signals crossing from $A$ to $B$ and then, vice versa, from $B$ to $A$.
Their synchronized `local', intrinsic, time can be, according to Einstein~\cite[p.~894]{ein-05},
defined by (similar) clocks that have been adjusted such that, for the light emission and return times $t_A$ and $t_A'$ at $A$,
and the reception and emission time $t_B$ at $B$, $t_B - t_A = t_A' - t_B$. ";

* as well as eliminated the bracket "(in a zig-zag manner)".

I hope that, with these revisions, the paper can be published in its present form.

Thank you for all the attention and efforts dedicated to the paper.

With kind regards,

Karl Svozil



~~~~~~~~~~~~~~~~~~~~~~~~~~~~~~~~~~~~~~~~~~~~~~~~~~~~~~~~~~~~~~~~~~~~~~~

RR3-2:

"I have read the author's reply, and amendments to the text, carefully, but I still struggle to understand how abstract space-time coordinates can be made `equivalent' to particles, in any sense, and particularly when one restricts measurements on the latter only to spin. In his reply, the author refers to these as "operational space-time" coordinates and states that "'The operationalization of the space-time coordinates referred to in Equation (3) . . . constitutes a standard procedure already mentioned by Poincar� and Einstein". I am not aware of any work by either Einstein, or Poincar�, which associates abstract space-time points with the spin-states of material particles. If such a work exists, the author should clearly reference it at the point in the text where these claims are made.


For this reason, I still have quite strong concerns about scientific content of this article. Nonetheless, as a conference proceedings article, its main purpose is give an accurate record of what material the author did, in fact, present at the conference in question. Such presentations can, legitimately, be more speculative and less rigorous than a standard research article. Therefore, overall, I am able still to recommend publication, even with the reservations noted above. However, before final publication, I must insist that the work by Einstein, or Poincar�, referred to, but not clearly referenced, in the new draft, is explicitly included in the Bibliography. This will allow readers who are not familiar with it, like myself, to check it for themselves."



I agree with the Referee that there is no "work by either Einstein, or Poincar�, which associates abstract space-time points with the spin-states of material particles."

Therefore, I have enlarged the discussion on possible synchronization conventions using entangles shares a bit as follows:

"Due to outcome dependence yet parameter independence, any space-time labeling using those outcomes is arbitrary.
For instance, `synchronizing' distant clocks (not with light ray exchange but) by the respective correlated outcomes of entangled particles
results in correlated but random temporal scales, which cannot be brought into any concordance
with `local' time scales generated by the conventional classical Poincar\'e-Einstein synchronization convention.

Signaling from one space-time point to another assumes choice,
yet again, the form of relational value definiteness that comes at the expense of individual value definiteness,
originating from the unitarity of quantum evolution,
between two or more constituents of a quantum entangled share
prevents signaling across its constituents.
Therefore, in the hypothetical scenario of a universe composed of entangled particles,
Poincar\'e-Einstein synchronization may require classical means that are unavailable for entangled particles."

I have also added

* two references;

* a more detailed explanation of Poincare-Einstein synchronization, quoting those new references, and Einstein's 1905 paper:

"As pointed out by Poincar\'e in 1900~\cite[p.~272]{Poincare1900} (see also Poincar\'e's 1904 paper~\cite[p.~311]{Poincare1904}),
suppose that two embedded observers $A$ and $B$ are positioned at different points of a moving frame, and are unaware of their shared motion,
and synchronize their clocks using light signals.
These observers believe, or rather assume or define, that the signals travel at the same speed in both directions.
They conduct observations involving signals crossing from $A$ to $B$ and then, vice versa, from $B$ to $A$.
Their synchronized `local', intrinsic, time can be, according to Einstein~\cite[p.~894]{ein-05},
defined by (similar) clocks that have been adjusted such that, for the light emission and return times $t_A$ and $t_A'$ at $A$,
and the reception and emission time $t_B$ at $B$, $t_B - t_A = t_A' - t_B$. ";

* as well as eliminated the bracket "(in a zig-zag manner)".
