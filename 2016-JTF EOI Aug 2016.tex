\documentclass[11pt,  a4paper]{article}

\usepackage{color}
\PassOptionsToPackage{hyphens}{url}
\usepackage{hyperref}
\hypersetup{colorlinks=true,
	urlcolor=blue,citecolor=blue}
%%%%%%%%%%%%%%%%%%%%%%%%%%%%%%%%%%%%%%%%%%%%%%%%%%

\usepackage[table]{xcolor}

%%%%%%%%%%%%%%%%%%%%%%%%%%%%%%%%%%%%%%%%%%%%%%%%%%
\newcommand{\andre}[1]{\textcolor{blue}{#1  (Andre)}}
\newcommand{\cris}[1]{\textcolor{magenta}{#1} (Cris)}


\usepackage{amsmath}
\usepackage{amssymb}
\usepackage{amsthm}
%\usepackage{hyperref}
\usepackage[hmargin=2cm,vmargin=3cm]{geometry}
\usepackage{graphicx}

\usepackage[utf8]{inputenc}
\usepackage[T1]{fontenc}

\usepackage[colorinlistoftodos]{todonotes}


\begin{document}

\thispagestyle{empty}
\begin{center}
%{\Large \bf Free Will  and Quantum Randomness}\end{center}
{\Large \bf   Randomness, Quantum Physics, and Free Will}\end{center}

\section{Executive summary -- maximum 1,300 characters}
 {\small \color{blue}
 \small The Executive Summary is your opportunity to describe what you plan to do, why it is worth doing, and what impact the project will have, if it is successful. The Executive Summary should be written in such a manner that it would be comprehensible to a typical reader of the New York Times or London Times—someone who is intellectually engaged, but not necessarily a specialist in your area of expertise.
A good Executive Summary will answer—in brief—all of the following questions: (a) Why is this project needed? (b) What specific questions will the project help to answer? (c) What activities will you carry out to answer those questions? (d) What concrete deliverables will you produce by the end of the project? (e) What impact will your project have? You do not need to include information about the project team, organization, request amount, or project duration in your Executive Summary.}\\


 Can the idea of free will -- understood as the ability of an agent to choose between different possible courses of action -- be explained, or at least gain support, by quantum indeterminism?  Is free will compatible with randomness, in particular, with quantum randomness?  Can free will and randomness  arise  accidentally? Does human-equivalent artificial intelligence imply free will?



 Our  project, set at the confluence of mathematical logic, quantum physics and philosophy, will investigate both the promise of a philosophical accommodation
 between the concept of free will and the notions of  indeterminacy and randomness, in particular their quantum forms, and the deep problems that attend such an accommodation.  The impact will be primarily theoretical, but also practical:  progress in this area will answer  questions such as ``Does AI present a danger to humanity?'',  and lead to a better understanding of emergent interactions, such as the ones  produced by a network of ``talking'' self-driving  cars.
\\

{\color{red} 1020 characters}\\



\section{Project Description -- maximum 5000 characters}

%{\color{blue} \small A good description will answer all of the following questions:
%\begin{itemize}
%	
%\item
% Why is this project important?\\
%Please describe the current conditions in the field(s) relevant to the project, diagnose
%the problems that the project will attempt to address, and identify the specific opportunity that your project presents.
%\item  What specific activities will you be implementing?\\
%Please provide a specific, concise, and detailed description of the activities you plan to carry out with the funds that you are requesting.
%\item  Why is your team and/or organization positioned to be successful in this project? \\
% Describe why the project team and/or the organization(s) connected to your proposed project will be able to successfully complete—on time and on budget—the activities that you have proposed. You may cite any recent work that will demonstrate your capacity to implement the proposed project to a high standard.
% \item Why is your team and/or organization positioned to be successful in this project? \\
% Describe why the project team and/or the organization(s) connected to your proposed project will be able to successfully complete—on time and on budget—the activities that you have proposed. You may cite any recent work that will demonstrate your capacity to implement the proposed project to a high standard.
% \end{itemize}}
%



%%%Modern physics is a mix of deterministic and stochastic theories. Einstein  thought that any such final theory would have to be deterministic.  There is no sign that this is going to be the case, however.  In particular, quantum mechanics is probabilistic to the core, thus casting doubt on whether the universe is deterministic at all. Although there are interpretations of quantum mechanics that dispute such indeterminism, there is every sign that this standard interpretation of the theory will remain among the most widely accepted~\cite{41}.
%%%
%%%Assume, then, that an indeterministic interpretation of quantum mechanics is correct.  A frequent objection is that free will still does not gain support that way, on the grounds that such indeterminism is for all practical purposes confined to purely microscopic phenomena. This objection overlooks that  it is physically possible to chain indeterministic effects at the quantum level to macroscopic phenomena (consider Schr\"odinger's cat).  Also, many actual macroscopic phenomena are based on quantum effects; for instance,  there are hardware random number generators that use quantum %randomness
%%%effects to obtain practically usable signals.
%%%
%%%None of this settles the issue of how free will is possible.  The question remains whether the traditional idea of free will can be explained in this way, or whether it is even compatible with such indeterminacy~\cite{31}.  \andre{Connect these questions better- how are they related? What is the main question?}
%%%
%%%%For if a person's actions are the result of quantum randomness there seems to be no scope for any volition on the part of the agent -- the experience and phenomenology of freedom are simply left as a dangler.  In addition, there now appears to be no sense in which the agent is responsible for his action if the action is free.  While quantum indeterminacy may yield a sense in which the agent might have done otherwise, other essential characteristics of freedom as normally conceived seem to become inexplicable.
%%%
%%%
%%%%The project to be embarked on will investigate both the promise of a philosophical accommodation between the notion of free will and the notion of quantum indeterminacy and randomness, and the deep problems that attend such an accommodation.
%%%
%%%
%%%We will  address the following question: {\em Is free will compatible with randomness, in particular, quantum randomness?} We will  work towards a formal definition of free will, and then study it via  a  mathematical  formalisation of randomness  as ``unpredictability relative to an intended theory''. This theory has been far developed for classical bits of information~\cite{8,4,AN}; however a formalisation where the basic objects are quantum bits instead may be needed to make progress. First steps towards such a theory have recently been made in collaboration with Volkher Scholz at University Ghent, who works within the theoretical quantum physics group  of Frank Verstraete and will be associated with the project.
%%%
%%%Crucial for our approach is the fact that there are no  absolute random events in Nature.
%%%Randomness is only a theoretical concept  {\it sui generis} which is defined and produced in deterministic ways: random events appear to have no direct cause that precedes them.  %{\color{red}Randomness is not implied nor implicated by indeterminacy.}
%%%\andre{Better write about the thing that randomness only turns up when you measure. Or is this what you try to say?}
%%%
%%%
%%%
%%%
%%%This foundational project requires an interdisciplinary  approach. Our team will be composed by Cristian S. Calude and Andr\'e Nies (mathematicians and computer scientists),  both from the University of Auckland,
%%% Karl Svozil (quantum physics and its philosophy), Vienna University of Technology.   \\[-5ex]
%%
%%{\small
%%\begin{thebibliography}{99}
%%\bibitem{2}  A. A. Abbott,  C. S. Calude  and  K. Svozil.
%% Value-indefinite observables are almost everywhere, {\em Physical Review A}, 89, 3 (2014), 032109--032116; %DOI: 10.1103/PhysRevA.89.032109.
%%%
%%% \bibitem{3} A. Abbott, C. S. Calude, K. Svozil.
%% A variant of the Kochen-Specker theorem localising value indefiniteness, {\em Journal of Mathematical Physics} 56, 102201 (2015).
%% \bibitem{4} A. Abbott, C. S. Calude, K. Svozil. A non-probabilistic model of relativised predictability in physics, {\em Information}  6, (2015), 773--789.
%%
%%\bibitem{8}  C. S. Calude. {\em Information and Randomness},  Springer, Berlin, 2002. (2nd. ed.)
%%
%%\bibitem{31} A. R. Mele. {\em Free Will and Luck,} Oxford University Press, Oxford,  2006.
%%
%%\bibitem{AN} Andr\'e Nies. {\em Computability and Randomness},  Oxford University Press, 2009, 2011.
%%
%%\bibitem{41} A. Zeilinger. The message of the quantum,
%%{\em Nature} 438 (2005), 743.
%%\end{thebibliography}
%%}
%%
%\section{Relation to Sir John Templeton's Donor Intent -- maximum 1000 characters}
%
%{\color{blue}\small Please explain how your proposed project would advance the philanthropic vision of Sir John Templeton and aligns with his Donor Intent. (To learn more about our founder’s philanthropic vision, we invite you to read the summary available at the Foundation’s website, and to explore the fuller exposition of his ideas in his published writings.)
%\\}

{\small \color{blue}  A good description will answer all of the following questions.}

\bigskip


{\small \color{blue}	
\noindent  Why is this project important?\\
Please describe the current conditions in the field(s) relevant to the project, diagnose
the problems that the project will attempt to address, and identify the specific opportunity that your project presents.
}




Indeterminism plays an important role in any discussion about free will.  However,
it is frequently claimed that randomness conflicts with free will because ``if our actions are caused by chance we lack control'' and ``randomness, the operation of mere chance, clearly excludes control''.
Are these arguments tenable?
The standard philosophical framework is not precise enough for answering this question because the notion of randomness is often left un-explicated, or is identified with chance or even indeterminism. Like these notions,  the concept of free will  is  used only in various  intuitive ways.



{\color{blue}	
 \noindent  What specific activities will you be implementing?\\
Please provide a specific, concise, and detailed description of the activities you plan to carry out with the funds that you are requesting.
}


  Can the idea of free will -- understood as the ability of an agent to choose between different possible courses of action -- be explained, or at least gain support, in this way?  Is free will compatible with randomness, in particular, with quantum randomness?  Can free will and randomness  arise  accidentally? Does human-equivalent artificial intelligence imply free will?


%ii) quantum randomness is postulated to be ``true randomness'' -- a mathematically vacuous concept --


To be able to offer robust answers to these questions,  we need to study in a formal way the notion of  indeterminism and, in particular, its quantum form.
Assume that an indeterministic interpretation of quantum mechanics is correct~\cite{41}.  We will endeavour to answer the following  questions:
  i)  how randomness manifest itself in nature?,  ii) how does quantum randomness is produced, and how it relates to quantum indeterminism?,  iii) what is the ``quality'' of quantum randomness and how does it compare with other forms of randomness, in particular, with software-generated (pseudo-)randomness?,  iv)  how can free will arise?, v) is free will compatible with (some forms of) randomness, in particular with quantum randomness?, vi)   in which way the impetuous development of AI  is impacted by free will and randomness?


Much work is needed to shift the dominating ontic perspective (where quantum indeterminism and randomness are postulated) to
an epistemic, resource-relative, formal approach.

 The main steps will be to: a) develop a resource-relative theory of quantum indeterminism, b)
   study  quantum randomness   in the framework of algorithmic (forms of) randomness~\cite{8,AN}, as ``unpredictability relative to an intended theory''~\cite{2},
 c) propose and study a simple, formal definition of free will, d) investigate, in the framework developed by a) -- c), way free will can arise and the compatibility between  randomness (algorithmic and quantum) and free will, e) investigate the role played by free will and randomness in AI.


Crucial for b)  is the fact that there are no  absolute random events in Nature.
Randomness is only a theoretical concept  {\it sui generis} which is defined and produced {\color{green} please delete ``in deterministic ways''} via measurement, and evaluated with respect to a given theory. The  particular case of quantum randomness based on the measurement of a value indefinite observable~\cite{2} will be an important example.

{\color{green} Determinism does not necessarily imply causation~\cite{born-metaph-1950} and predictability~\cite{suppes-1993}.

There might also be some gaps~\cite{franke} in the physical laws.
Already in 1873, Maxwell identified a certain kind of {\em instability} at {\em singular points}
as rendering a gap in the natural laws~\cite[p.~211-212]{Campbell-1882}.
For the sake of an example, consider a particle or rock positioned on top of some sharp tip (potential).
Any slightest movement -- either through a microscopic asymmetry or imbalance of the particle,
or from fluctuations of any form, either in the particle's position due to quantum zero point fluctuations,
or by the surrounding environment of the particle --  might topple the particle over the tip;
thereby spoiling the original symmetry.

In principle, such gaps could
be exploited for dualistic interfaces.
The mere existence of gaps in the causal fabric are no sufficient condition for the existence of providence or free will,
because these gaps may be completely supplied by {\em creatio continua.}
As has already been observed by Frank~\cite[Chapter~{III}, Sects.~12--15]{frank},
in order for any {\em miracle} or free will to manifest itself
through any such gap in the natural laws, it needs to be {\em systematic,}
{\em according to a plan}
and
{\em intentional} (German {\it planm\"a\ss ig}).
Because if there were no possibilities to inject information or other matter or content
into the universe from beyond through such gaps, it might be difficult to interact with, and less so manipulate, the universe.

}

\begin{thebibliography}{99}
\bibitem{2}  A. A. Abbott,  C. S. Calude  and  K. Svozil.
 Value-indefinite observables are almost everywhere, {\em Physical Review A}, 89, 3 (2014), 032109--032116; %DOI: 10.1103/PhysRevA.89.032109.
%
% \bibitem{3} A. Abbott, C. S. Calude, K. Svozil.
 A variant of the Kochen-Specker theorem localising value indefiniteness, {\em Journal of Mathematical Physics} 56, 102201 (2015).
 \bibitem{4} A. Abbott, C. S. Calude, K. Svozil. A non-probabilistic model of relativised predictability in physics, {\em Information}  6, (2015), 773--789.

\bibitem{8}  C. S. Calude. {\em Information and Randomness},  Springer, Berlin, 2002. (2nd. ed.)

\bibitem{freewill} C. S. Calude, F. Kroon, N. Poznanovi\'{c}. Free will is compatible with randomness, {\em Philosophical Inquiries}, 4, 2 (2016), 37--52.

\bibitem{31} A. R. Mele. {\em Free Will and Luck,} Oxford University Press, Oxford,  2006.

\bibitem{AN} Andr\'e Nies. {\em Computability and Randomness},  Oxford University Press, 2009, 2011.

\bibitem{KS} Karl Svozil. {\em Randomness and Undecidability in Physics}, World Scientific, Singapore, 1993.

\bibitem{41} A. Zeilinger. The message of the quantum,
{\em Nature} 438 (2005), 743.

\bibitem{born-metaph-1950} M. Born. Physics and metaphysics,
{\em Science News}  17 (1949), 9-7.

\bibitem{suppes-1993} P. Suppes. The Transcendental Character of Determinism,
{\em Midwest Studies In Philosophy} 18 (1993), 242-257.

 \bibitem{franke} P. Frank and {R. S. Cohen (Editor)}. {\em The Law of Causality and its Limits (Vienna Circle Collection)},
Springer, Vienna, 1997.

 \bibitem{Campbell-1882} Lewis Campbell and William Garnett. {\em The life of {J}ames {C}lerk {M}axwell. {W}ith a selection from his correspondence and occasional writings and a sketch of his contributions to science},
MacMillan, London, 1882, 1999.


\end{thebibliography}

{\color{red} 3377 characters}\\

{\small \color{blue} \noindent  Strategic Promise: Why is the proposed project important relative to the current state of knowledge in your field or across fields? Please limit your response to 1,000 characters, including spaces and punctuation.}\\


The project is important from both theoretical and practical points of view.
Computability and complexity theories have matured enough to be useful in formally modelling  the main notions involved in the project, {\it indeterminism, quantum randomness},  and {\it free will}. Such models will be a premiere in the philosophical analysis of these notions and relations between them (currently done at an informal level) and will set up a new framework in which
more precise results will be possible to be obtained.

 {\color{red} 495 characters}\\


{\small \color{blue} \noindent
Capacity for Success. Please explain how you (the applicant, the project team, and/or the organization(s) connected to the proposed project) are positioned to carry out the proposed activities with distinction and a high standard of excellence. Please limit your response to 1,000 characters, including spaces and punctuation.
}\\

Our team will be composed by Cristian S. Calude and Andr\'e Nies (mathematicians and computer scientists),  both from the University of Auckland,
 Karl Svozil (quantum physics and its philosophy), Vienna University of Technology.
 Calude and Nies have done extensive work in computability and complexity theories, see~\cite{8,AN}; Calude and Svozil have jointly worked  in quantum physics for more than two decades, and developed a formal model  of a form of quantum randomness randomness using the Kochen-Specker theorem~\cite{2,4}; Svozil has deep knowledge on the foundations and philosophy of quantum physics~\cite{KS}. Together, they have the knowledge, experience and research capability to carry on this project with distinction and a high standard of excellence.

 {\color{red} 746 characters}\\



{\small \color{blue} \noindent Expected Outputs. Outputs are the specific, quantifiable work products that you will create during the project. Examples include but are not limited to: academic papers submitted for publication, book manuscripts, conference proceedings, training sessions, curricula, prize competitions, films, events, and publicity campaigns.}\\

 The expected outputs include six research articles, one article for the general audience
 (all posted first on open archives and then submitted to peer-reviewed journals), two  international academic interdisciplinary conferences on the topics of the project (reports on the key results presented at each conference), eight collaborative visits with researchers in computability and complexity theories and philosophy (reports of progress for each of them), training for  two undergraduates and three graduate students in techniques developed in project (summary of the training activities for each of the involved students), the development of a new curriculum for a course in ``Philosophy of Computation'', a   website detailing the  activities of the project.

{\color{red}  754 characters}\\

{\small \color{blue}\noindent  Expected Outcomes Outcomes are the realistic and measurable differences that you believe will result from your project’s outputs. Examples include but are not limited to: significant new lines of inquiry that might develop if your research hypotheses are confirmed, and measurable changes that your work might bring about in the actions or attitudes of your target audience.
Please specify the audiences that your project seeks to reach, as well as the specific outcomes that you expect will result. Please limit your response to 1,000 characters, including spaces and punctuation.}\\


The expected outcomes of the project are the following: (1) to provide  formal models for the notions of {\it indeterminism, quantum randomness} and {\it free will},
(2) a philosophical accommodation
 between the notion of free will and the notions of indeterminacy and randomness, in particular in their quantum forms, (3) a new framework to to study free will and  its impact  for the development of AI. The target audience of this project consists of mathematical logicians, quantum physicists, computer scientists and philosophers. An important part of the project will be to engage members of these communities as both contributors and beneficiaries of the proposed results. The impact will be primarily theoretical, but also practical:  progress in this area will allow   answering  questions such as ``Does AI present a danger to humanity?'',  and lead to a better understanding of emergent interactions, such as the ones  produced by a network of ``talking'' self-driving  cars.




{\color{red} 959  characters}\\

{\small \color{blue}\noindent  Enduring Impact. The Foundation is very interested to learn about your hopes for this project. Describe your vision of the realistic and beneficial long-term changes that could result from your work. Please limit your response to 1,000 characters, including spaces and punctuation.}\\

Formalising the key concepts of {\it indeterminism, quantum randomness},  and {\it free will} will  open a new way to study these concepts and their
inter-relations.  We hope that our results will stimulate philosophers and (quantum) physicists to move from the  dominating ontic perspective  to
an epistemic, resource-relative, formal approach.

{\color{red}  343 characters}\\

{\small \color{blue}	\noindent
Why is your team and/or organization positioned to be successful in this project? \\
 Describe why the project team and/or the organization(s) connected to your proposed project will be able to successfully complete—on time and on budget—the activities that you have proposed. You may cite any recent work that will demonstrate your capacity to implement the proposed project to a high standard.
}


All applicants have many publications in premier peer-reviewed journals and conferences in their areas of expertise; they all have been awarded many international and national grants.
\andre{Nies ... Marsden ...}
Calude and Svozil have successfully completed in late 2015 a five-year research project on on randomness in physics supported by the Marie Curie FP7-PEOPLE-2010-IRSES; the publications~\cite{2,4} come from this project.  Calude has been interested for a long time in the philosophy of computation and is teaching a course in this subject; he jointly published a paper relating randomness and free will~\cite{freewill}.
  Together the applicants can cover the main areas of mathematics, quantum physics and philosophy required by the project. The team will cooperate with at least one
 philosopher  with expertise in free will and an AI expert.


{\color{red}  820 characters}\\





\section{Relation to Sir John Templeton's Donor Intent -- maximum 1000 characters}

{\small \color{blue}\small Please explain how your proposed project would advance the philanthropic vision of Sir John Templeton and aligns with his Donor Intent. (To learn more about our founder’s philanthropic vision, we invite you to read the summary available at the Foundation’s website, and to explore the fuller exposition of his ideas in his published writings.)
\\}

The project approach is to acknowledge and work with  resource-relative notions, not with absolute, postulated ones.  We will constantly check the relevance of our models and results against the quantum physics reality and philosophical meaning,
keeping   both  eyes on  the limits of our findings.
 As ``true randomness'' does not mathematically exist, we will investigate various  forms of quantum randomness based on measuring specific observables under well-defined physical assumptions.
 Similarly, our models  will address specific forms of the intuitive, broad notion of  free will. Finally, we will carefully state all assumptions and hypotheses used in our results.
This aligns well with Sir John’s Donor Intent and vision:
``we must accept in all humility that our knowledge is still limited'' [Possibilities, p. 58].\\
%\cris{Please reformulate} \andre{slight changes made}

{\color{red} 810 characters}
%\begin{quote}
%	 It seems that centuries of human enterprise are now miraculously bursting into  flower. Is the development of human knowledge accelerating? Is the
% present generation reaping the fruits of generations of scientific thought? [Possibilities, p. 51]
%\end{quote}
%
%
%\begin{quote}
%	In spite of the enormous strides made by science and the incredible power
%of our new instruments to reveal the secrets of the universe, large and small, we must accept in all humility that our knowledge is still limited. We cannot even be sure that the vast universe unveiled to us by our telescopes is all that exists. There may be other regions of the universe far beyond the reach of our instruments having very different properties. It is even possible that entire other universes co-exist in parallel with our own. [Possibilities, p. 58]
%\end{quote}

\end{document}
