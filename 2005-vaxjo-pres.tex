%\documentclass[pra,showpacs,showkeys,amsfonts,amsmath,twocolumn]{revtex4}
\documentclass[amsmath,blue,handout]{beamer}
%\documentclass[pra,showpacs,showkeys,amsfonts]{revtex4}

\usepackage{beamerthemeshadow}
%\usepackage[dark]{beamerthemesidebar}
%\usepackage[headheight=24pt,footheight=12pt]{beamerthemesplit}
%\usepackage{beamerthemesplit}
%\usepackage[bar]{beamerthemetree}
\usepackage{graphicx}
\usepackage{pgf}

%\RequirePackage[german]{babel}
%\selectlanguage{german}
%\RequirePackage[isolatin]{inputenc}

\pgfdeclareimage[height=0.5cm]{logo}{tu-logo}
\logo{\pgfuseimage{logo}}
\beamertemplatetriangleitem
\begin{document}

\title{\bf Quantum computing as quantum state identification}
%\subtitle{Naturwissenschaftlich-Humanisticher Tag am BG 19\\Weltbild und Wissenschaft\\http://tph.tuwien.ac.at/\~{}svozil/publ/2005-BG18-pres.pdf}
\subtitle{arXiv quant-ph/0505129}
\author{Karl Svozil}
\institute{Institut f\"ur Theoretische Physik, University of Technology Vienna, \\
Wiedner Hauptstra\ss e 8-10/136, A-1040 Vienna, Austria\\
svozil@tuwien.ac.at
%{\tiny Disclaimer: Die hier vertretenen Meinungen des Autors verstehen sich als Diskussionsbeitr�ge und decken sich nicht notwendigerweise mit den Positionen der Technischen Universit�t Wien oder deren Vertreter.}
}
\date{June, 2005}
\maketitle

\frame{\tableofcontents}

\frame[shrink=2]{
\frametitle{Will not speak about ...}

\begin{itemize}
\item<+->
Daniel M. Greenberger and Karl Svozil, Quantum Theory Looks at Time Travel, arXiv quant-ph/0506027

\item<+->
Karl Svozil, ``Computational universes'' Chaos, Solitons \& Fractals 25(4), 845-859 (2005)

\item<+->
Karl Svozil, ``Eutactic quantum codes'', Physical Review A 69, 034303 (2004).

\item<+->
Karl Svozil, ``Communication cost of breaking the Bell barrier'', http://tph.tuwien.ac.at/~svozil/publ/2004-brainteaser.pdf \&  quant-ph/0411190

\item<+->
Karl Svozil, ``Aesthetics and scarcity. A physics perspective on ornament'', arXiv physics/0505088

\end{itemize}
 }







\section{Spread of information ``across'' single quanta}


\frame[shrink=2]{
\frametitle{Spread of information ``across'' single quanta in multipartite states}
\begin{itemize}
\item<+->
One advantage of quantum algorithms over classical computation
is the possibility to spread out, process and extract information
in multipartite configurations in coherent superpositions of classical states.



\item<+->
Quantum state identification problems
based on a proper partitioning of mutually orthogonal sets of states.

\item<+->
The question arises whether or not it is possible
to encode equibalanced decision problems
into quantum systems, so that a single invocation
of a filter used for state discrimination suffices to obtain the result.

\end{itemize}
 }


\section{Identifying states among contexts}
\subsection{Strategy}
\frame{
\frametitle{Identifying  orthogonal pure states among contexts}
Assume $k$ particles in $n=2$ or more dimensions per particle.

System of $k$ co-measurable  filters $\textsf{\textbf{F}}_i$, $i=1,\ldots, k$
with the following properties:
\begin{itemize}
\item<+->
Every filter $\textsf{\textbf{F}}_i$
corresponds to an operator  generating an
equi-$n$-partition of the $d$-dimensional state space into
$k$ slices (i.e., partition elements) containing $d/n=d^{1-1/k}$ states per slice.
A filter is said to separate two eigenstates if the eigenvalues are different.

\item<+->
From each one of these $k$ partitions, take an arbitrary element.
The intersection of the elements of all different partitions
results in a {\it single} one of the $d=n^k$ different states.

\item<+->
The union of all those single states generated by the intersections is the entire set of states.


\end{itemize}
 }


\subsection{Realizations}
\frame{
\frametitle{Realizations}
For $n=2$, an explicit construction of all the systems of filters and their associated propositions
can be given in terms of projectors and their orthogonal projectors;
every one of them projecting onto a $d/2$-dimensional subspace,
such that the serial composition of any complete set of (orthogonal) projectors (one per filter)
yields the finest resolution; i.e., some of the $d$ one-dimensional projectors $\textsf{\textbf{E}}_i$
spanning the context $\textsf{\textbf{C}}$.

{\scriptsize N. Donath and K.S., PRA 65, 044302 (2002). quant-ph/0105046}

 }

\frame{
\frametitle{Realizations cntd.}
The system of filters resolving $\textsf{\textbf{C}}$
is not unique;
all such systems of filters can be obtained by permutating
the columns of the matrix whose rows are
the diagonal elements of all the filters in diagonalized form.
Different contexts
$\textsf{\textbf{C}}'$
are resolved by different systems of filters which are obtained by transforming
$\textsf{\textbf{F}}_i$, $i=1,\ldots, k$ through
the same basis transformation which transforms
$\textsf{\textbf{C}}$
into
$\textsf{\textbf{C}}'$.
 }


\frame{
\frametitle{Realizations}
For $n>2$, an explicit construction of all the systems of filters and their associated propositions
can be given in terms of diagonal orthogonal projectors with $n$ different eigenvalues.

{\scriptsize K.S., PRA 66, 044306 (2002). quant-ph/0205031;
K.S., J. Mod. Opt 51, 811-819 (2004). quant-ph/0308110.}
 }

\subsection{Examples}
\frame[shrink=2]{
\frametitle{Example}

Take, as an example, three two-state quanta, i.e.,
the case $k=3$, $n=2$, and thus $d=2^3$.
The three projectors
$$
\begin{array}{ccc}
\textsf{\textbf{F}}_1&=&\textrm{diag}(1,1,1,1,0,0,0,0),\\
\textsf{\textbf{F}}_2&=&\textrm{diag}(1,1,0,0,1,1,0,0),\\
\textsf{\textbf{F}}_3&=&\textrm{diag}(1,0,1,0,1,0,1,0),\\
\end{array}
$$
 together with their orthogonal projectors
$$
\begin{array}{ccc}
\textsf{\textbf{F}}_1'&=&\textrm{diag}(0,0,0,0,1,1,1,1), \\
\textsf{\textbf{F}}_2'&=&\textrm{diag}(0,0,1,1,0,0,1,1), \\
\textsf{\textbf{F}}_3'&=&\textrm{diag}(0,1,0,1,0,1,0,1),   \\
\end{array}
$$
form the system of three filters
$
\{
\{ \textsf{\textbf{F}}_1,\textsf{\textbf{F}}_1' \},
\{ \textsf{\textbf{F}}_2,\textsf{\textbf{F}}_2' \},
\{ \textsf{\textbf{F}}_3,\textsf{\textbf{F}}_3' \}
\}
$
which have the desired filter properties.
$$
\left(
\begin{array}{cccccccc}
1&1&1&1&0&0&0&0\\
0&0&0&0&1&1&1&1\\
\hline
1&1&0&0&1&1&0&0\\
0&0&1&1&0&0&1&1\\
\hline
1&0&1&0&1&0&1&0\\
0&1&0&1&0&1&0&1\\
\end{array}
\right).
$$
}

\frame[shrink=2]{
\frametitle{Example cntd. }
Equivalent filters are obtained by permuting the columns of the diagonal rows
of the matrix on the previous slide; e.g.,
$$
\left(
\begin{array}{cccccccc}
1&1&1&0&0&0&0&1\\
0&0&0&1&1&1&1&0\\
\hline
1&0&0&1&1&0&0&1\\
0&1&1&0&0&1&1&0\\
\hline
0&1&0&1&0&1&0&1\\
1&0&1&0&1&0&1&0\\
\end{array}
\right),
\left(
\begin{array}{cccccccc}
1&1&0&0&0&0&1&1\\
0&0&1&1&1&1&0&0\\
\hline
0&0&1&1&0&0&1&1\\
1&1&0&0&1&1&0&0\\
\hline
1&0&1&0&1&0&1&0\\
0&1&0&1&0&1&0&1\\
\end{array}
\right),
\qquad
\ldots
$$
 }



\section{Deutsch's problem and Parity}
\subsection{Deutsch's problem}
\frame[shrink=2]{
\frametitle{Deutsch's problem: parity of a function of one bit}
As always, start with
$$
\textsf{\textbf{U}}_f(\vert x\rangle \vert y\rangle )
=\vert x\rangle \vert y \oplus f(x) \rangle
$$



{\scriptsize
\begin{table}
\centerline{
\begin{tabular}{cccccccccccccc}
\hline\hline
 & $\frac{1}{2}[\vert 0\rangle \vert 0 \oplus f(0)\rangle $ &-& $\vert 0\rangle \vert 1 \oplus f(0)\rangle $ &-& $\vert 1\rangle \vert 0 \oplus f(1)\rangle $ &+& $\vert 1\rangle \vert 1 \oplus f(1)\rangle ]$\\
\hline
$f_0$: $\psi_1$ & $\frac{1}{2}(\vert 0\rangle \vert 0\rangle $ &-& $\vert 0\rangle \vert 1\rangle $ &-& $\vert 1\rangle \vert 0\rangle $ &+& $\vert 1\rangle \vert 1\rangle )$\\
$f_1$: $\psi_2$ & $\frac{1}{2}(\vert 0\rangle \vert 0\rangle $ &-& $\vert 0\rangle \vert 1\rangle $ &-& $\vert 1\rangle \vert 1\rangle $ &+& $\vert 1\rangle \vert 0\rangle )$\\
$f_2$: -$\psi_2$ & $\frac{1}{2}(\vert 0\rangle \vert 1\rangle $ &-& $\vert 0\rangle \vert 0\rangle $ &-& $\vert 1\rangle \vert 0\rangle $ &+& $\vert 1\rangle \vert 1\rangle )$\\
$f_3$: -$\psi_1$ & $\frac{1}{2}(\vert 0\rangle \vert 1\rangle $ &-& $\vert 0\rangle \vert 0\rangle $ &-& $\vert 1\rangle \vert 1\rangle $ &+& $\vert 1\rangle \vert 0\rangle )$\\
\hline\hline
\end{tabular}
}
\caption{State evolution of $\textsf{\textbf{U}}_f(\textsf{\textbf{H}}\otimes\textsf{\textbf{H}})
(\textsf{\textbf{X}}\otimes\textsf{\textbf{X}})
(\vert 0\rangle \vert 0\rangle )$
for the four functions $f_0,f_1,f_2,f_3$.
$\textbf{X}$ and
$\textbf{H}$ stand for the not operator and the (normalized) Hadamard transformation.
 \label{2005-ko-t1}
}
\end{table}
}
 }

\subsection{State identification in Deutsch's case}
\frame{
\frametitle{Deutsch's problem by state identificatrion}
Start with the two operators
$\textsf{\textbf{F}}_1 =\textrm{diag}(1,1,0,0)$
and
$\textsf{\textbf{F}}_2 =\textrm{diag}(1,0,1,0)$
associated with a binary search type state separation in the  basis
$\textbf{B}=\{
(1,0,0,0)^T,
(0,1,0,0)^T,
(0,0,1,0)^T,
(0,0,0,1)^T\}$.
}

\frame{
\frametitle{Basis transformation}

Strategy: find transformation which takes the standard computational basis to the orthogonal basis
represented by the four states $\psi_1,\psi_2$ corresponding to the functions $f_0, f_1, f_2, f_3$.
(Completion of basis by Gram-Schmidt.)

$\textsf{\textbf{U}}\textsf{\textbf{F}}_1\textsf{\textbf{U}}^{-1} =\textsf{\textbf{F}}^D_1$ and
$\textsf{\textbf{U}}\textsf{\textbf{F}}_2\textsf{\textbf{U}}^{-1} =\textsf{\textbf{F}}^D_2$,
where
\begin{equation}
\textsf{\textbf{U}}=\frac{1}{2}
\left(
\begin{array}{cccc}
 1&  1& 1&  1\\
 1& -1& 1& -1\\
-1&  1& 1& -1\\
-1& -1& 1&  1
\end{array}
\right)
\end{equation}
is the unitary transformation which corresponds to a basis change
$\textbf{B} \rightarrow \textsf{\textbf{U}}\textbf{B}=\textbf{B}^D$.
}

\frame{
\frametitle{Solution}

By the eigenvalue spectrum,
$\textsf{\textbf{F}}^D_1$ separates between
$\psi_1$ and $\psi_3$ from
$\psi_2$ and $\psi_4$
(and at the same time,
$\textsf{\textbf{F}}^D_2$ separates between
$\psi_1$ and $\psi_2$ from
$\psi_3$ and $\psi_4$).
Hence, $\textsf{\textbf{F}}^D_1$ generates a partition
$\{\{\psi_1,\psi_3\},\{\psi_2,\psi_4\}\}$ of the set
$\{\psi_1,\psi_3,\psi_2,\psi_4\}$
of orthogonal states.
($\textsf{\textbf{F}}^D_2$ generates the partition
$\{\{\psi_1,\psi_2\},\{\psi_3,\psi_4\}\}$.)
}


\subsection{Parity of function of two bits}
\frame{
\frametitle{Parity of function of two bits}
\begin{table}
\centerline{
\begin{tabular}{cccccccccccccc}
\hline\hline
 $\pm$&f& \multicolumn{4}{c }{xy}\\
 && 00& 01& 10 &11\\
\hline
$+$&$f_0$ &0 &0& 0& 0\\
$-$&$f_1$ &0 &0& 0& 1\\
$-$&$f_2$ &0 &0& 1& 0\\
\multicolumn{6}{c}{$\cdots$}\\
$+$&$f_{15}$ &1 & 1& 1&1\\
\hline\hline
\end{tabular}
}
\caption{Listing of the 16 binary functions of two variables $x,y$ with their parity bits
``$\pm$''. \label{2005-ko-t2}
}
\end{table}

$$
F=
\left\{
\left\{
     f_{  0} ,
     f_{  5} ,
     f_{  6} ,
     f_{  7} ,
     f_{  8} ,
     f_{  9} ,
     f_{ 10} ,
     f_{ 15}
\right\},
\left\{ f_{  1} ,
     f_{  2} ,
     f_{  3} ,
     f_{  4} ,
     f_{ 11} ,
     f_{ 12} ,
     f_{ 13} ,
     f_{ 14}
\right\}
\right\}
.
$$
 }


\frame[shrink=2]{
\frametitle{Parity}
$$
\left(
\textsf{\textbf{1}}
\otimes
\textsf{\textbf{1}}
\otimes
\textsf{\textbf{H}}
\right)
\textsf{\textbf{U}}_f
\left(
\textsf{\textbf{1}}
\otimes
\textsf{\textbf{1}}
\otimes
\textsf{\textbf{H}}
\right)
\vert x,y \rangle \vert 1\rangle
=
(-1)^{f(x,y)}\vert x,y \rangle \vert 1\rangle
.
$$

Classically, parity checking grows exponentially $2^k$ with the number $k$ of bits of the
functional arguments, as there is no other was than to compute the functional
values on the entire set of $2^k$ arguments.
Quantum mechanically,
one may interpret this problem as a particular instance
of a generalized Grover algorithm with an unknown number of special states,
which can be solved by applying the quantum Fourier transform.


The parity of a function
has been proven quantum computationally hard
Farhi et al., 1998:
It is only possible to go from $2^k$ classical queries down to $2^k/2$
quantum queries, thereby gainig a factor of 2.
 }


%\section{Problem to map $f \mapsto U_f$}
\frame{
\frametitle{$f \mapsto U_f$}
The lack of efficient quantum algorithms
is due to the nonexistence of
mappings of functions $f$ and decision problems into suitable unitary
transformations
$
\textsf{\textbf{U}}_f
$
which could be used for
a system of states and of filter(s) resolving those states corresponding to that particular
algorithmic problem and no other one.

 }


\section{Recursion \& iteration}
\frame{
\frametitle{Recursion \& iteration}
In general, while all classical computable recursive functions
$f$ and decision problems can be coded quantum mechanically, there is no guarantee
that a problem can be coded efficiently by mapping it into the quantum domain.
By an efficient coding of a (binary or $n$-ary) decision
problem  we mean that some quantum circuit
$\textsf{\textbf{U}}_f$ exists
which outputs a state which is uniquely identifiable by a single filter
(or at least by a polynomial number of filters),
the outcome of which corresponds to the solution of this problem.

 }


%\section{Summary}
\frame{
\frametitle{Summary}
\begin{itemize}
\item<+->
Analysis of quantum computations in terms of state identification
whose complexity grows linearly with the number of bits.

\item<+->
Characterization this domain by partitions of state space,
as well as by unitary transformations of the associated filter systems.

\item<+->
Such systems are not bound to the individual classical values,
as information about the (parallelized)
result of a computation may be ``spread among'' the quanta
in a way which makes it impossible to reconstruct the result
by measuring the quanta separately.


\item<+->
The method does not yield a constructive, operational method
for deciding
whether or not (and if so, how) functions or decision problems of practical interest
can be efficiently coded into quantum algorithms.

\item<+->
From a foundational point of view it is interesting
to realize that, while every suitable equipartitioning of state space
is equivalent to some proposition which can be interpreted as an outcome of some
quantum computation,
not all decision problems or functional evaluations which can be rephrased as
state partitions can be translated into an efficient quantum computation.
\end{itemize}
 }



\end{document}
