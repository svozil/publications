%%%%%%%%%%%%%%%%%%%%%% pref.tex %%%%%%%%%%%%%%%%%%%%%%%%%%%%%%%%%%%%%
%
% sample preface
%
% Use this file as a template for your own input.
%
%%%%%%%%%%%%%%%%%%%%%%%% Springer-Verlag %%%%%%%%%%%%%%%%%%%%%%%%%%

\preface

%% Please write your preface here Here come the golden words

Our perception of what is knowable and what is unknown,
and, in particular,
our viewpoint on randomness,
lies at the metaphysical core of our worldview.
This view has been shaped by the narratives created and provided by the experts through various sources --
rational, effable and (at least subjectively) ineffable ones.

There are, and always have been, canonical narratives by the orthodox mainstream.
Often orthodoxy delights itself in personal narcissism, which is administered and mediated by the attention economy,
which in turn is nurtured by publicity and the desire of audiences ``to know'' --
to attain ``truth'' in a final rather than in a procedural, preliminary sense.
Alas, science is not in the position to provide final answers.

Alternatively, the narrative is revisionistic.
Already Emerson noted~\cite{Emerson-sr}, {\em ``whoso would be a man must be a nonconformist. $\ldots$
Nothing is at last sacred but the integrity of your  own mind.''}
But although iconoclasm, criticism and nonconformism seems to be indispensable for progress,
they bear the danger of diverting effort and attention to unworthy ``whacky'' attempts and degenerative research programs.

Both orthodoxy as well as iconoclasts are indispensable elements of progress, and different sides of the same coin.
They define themselves through the respective other, and their interplay and interchange facilitates the possibility to obtain knowledge about Nature.

And so it goes on and on; one is reminded of Nietzsche's {\it eternal recurrence}\footnote{German original: ewige Wiederkunft~\cite{Nietzsche-EcceHomo}}.
It might always be like that; at least there is not the slightest indication that our theories settle and become canonized even
for a human life span; let alone indefinitely.
Indeed, any canonization might indicate a dangerous situation and be detrimental to science.
Our universe seems to foster instability and change; indeed, volatility and compound interest is a universal feature of it.

Physical and other unknowns might be systemic and inevitable, and actually quite enjoyable, features of science and human cognition.
The sooner we learn how to perceive and handle them, the sooner we shall be able to exploit their innovative capacities.

But there is more practical, pragmatic utility to randomness and indeterminism than just this epistemological joy.
I shall try to explain this with two examples.
Suppose that you want to construct a bridge, or some building of sorts.
As you try to figure out the supporting framework, you might end up with the integral of some function
which has no analytic solution you can figure out.
Or even worse: the function is the result of some computation and has no closed analytic form which you know of.
So, all you can do is to try to compute this function numerically.

But this might be deceptive because the algorithm for numerical integration has to be ``atypical'' with respect to the function
in the sense that all parts of the function are treated  ``unbiased.''
Suppose, for instance, that the function shows some periodicity.
Then, if the integration would evaluate the function only at points which are in sync with that functional periodicity,
this would result in a strong bias toward those functional values which fall within a particular
sync period; and hence a bad approximation of the integral.

Of course, if, in the extreme case, the function is almost constant, any kind of sampling of points -- even very concentrated ones (even a single point),
or periodic ones yield reasonable approximations.
But ``random'' sampling alone guarantees that all kinds of functional scenarios are treated well, and thus yield good approximations.

Other examples for the utility of randomness are in politics.
Random selection plays a role in a {\it Gedankenexperiment} in which one is asked
to sketch a theory of justice and appropriation of wealth if a veil of ignorance is kept over one's own status and destiny;
or if one imagines being born into randomly selected families~\cite{Rawls-1971}.

And as far as the ancient Greeks are concerned,
those who practiced their form of democracy have been (unlike us) quite aware that sooner or later, democracies deteriorate into oligarchies.
This is almost inevitable:
because of mathematical mechanisms related to compound interest {\it et cetera},
an uninhibited growth tends to increase and accumulate wealth and political as well as economic power into fewer and fewer entities
and individuals.
We can see those aggregations of wealth and powers in action on all political scales,
local and global.
Two immediate consequences are misappropriations of all kinds of assets and means, as well as corruption.
%even in democratic republics:
%take for instance, the succession of presidency within a few families in the USA.
%This is a particular form of corruption which is also fostered by networks of people who change roles as time goes by
%-- thereby often causing conflicts of interest:
%government ministers  working as controllers and bank supervisors, and even  presidents of the European Commission, turn into bank employees;
%high ranking military personnel such as generals turn into defence contractors,
%and so on.

As the ancient Athenians watched similar tendencies in their times they came up with two solutions to neutralize the danger of
tyranny by compounded power:
one was ostracism, and the other one was sortition,
the widespread random selection of official ministry
as a remedy to curb corruption~\cite[p.~77]{Headlam-ebtliA}.
%{\em ``knowledge [[of public
%procedure or  acquaintance with official forms]]
%was spread over the whole city,
%because every one was at some time an official, and
%so the worst kind of legal cruelty, which results
%when the forms of law are used to give to oppression
%the appearance of justice, was prevented. Injustice
%there was of course; but it was usually unorganised;
%a few criminals were left unexecuted, a few innocent
%men were punished, a few paid too heavy taxes, a
%few made a little money out of their public duties;
%but systematic oppression and organised fraud were
%impossible''}.
As Aristotle noted,
{\em ``the appointment of
magistrates by lot is thought to be democratic, and the
election of them oligarchical''}~\cite[Politics~IV,~$1294^b8$, pp.~4408-4409]{barnes-aristotle}.
The ancient Greeks used fairly sophisticated random selection procedures, algorithms and machines
called \textgreek{klhrwt'hrion} {\it (kleroterion)}  for, say, the selection of lay judges~\cite{dow_aristotlekleroteria_1939,Demont-2003}.
\index{kleroterion}
Then and now accountable and certified ``randomized'' selection procedures have been of great importance for the public affairs.




This book has been greatly inspired by, and intends to be an ``update''
of, Philipp Frank's 1932 {\it The Law of Causality and its Limits}~\cite{frank,franke}.
It is written in the spirit of the Enlightenment and scientific rationality.
One of its objective is to give a {\em status quo} of the situation regarding physical indeterminism.
Another is the recognition that certain things are provable unknowable;
but that does not mean that they need to be ``irreducibly random.''

As a result the book is not in praise of what is often pronounced as ``discovery of indeterminism and chance in the natural sciences,''
but rather attempts two objectives: on the one hand, it locates and scrutinizes claims of absolute randomness and irreducible indeterminism.
On the other hand, it enumerates the means relative limits of expressing truth by finite formal systems.

It is amazing that, when it comes to the perception of chance {\it versus} determinism, people, in particular, scientists,
become very emotional~\cite{2002-cross} and seem to driven by ideologies and evangelical agendas
and furors which sometimes are hidden even to themselves.
Consequently there is an issue that we need to be aware of when discussing such matters at all times.
Already Freud advised analysts
to adopt a contemplative strategy of {\em evenly-suspended attention}~\cite{Freud-1912,Freud-1912-e};
\index{evenly-suspended attention}
and, in particular,  to be aware of the dangers
caused by {\em
``$\ldots$~the temptation of projecting outwards
some of the peculiarities of his own personality,
which he has dimly perceived, into the field of science,
as a theory having universal validity; he will bring the psycho-analytic method into discredit, and lead the inexperienced astray.''}~\cite{Freud-1912-e}\footnote{
German original~\cite{Freud-1912}: {\em ``Er wird leicht in die Versuchung geraten,
was er in dumpfer Selbstwahrnehmung von den Eigent\"umlichkeiten seiner eigenen Person erkennt,
als allgemeing\"ultige Theorie in die Wissenschaft hinauszuprojizieren,
er wird die psychoanalytische Methode in Misskredit bringen und Unerfahrene irreleiten.''}}
And the late Jaynes warns and disapproves of
the {\em Mind Projection Fallacy}~\cite{jaynes-89,jaynes-90,Powers596},
\index{Mind Projection Fallacy}
pointing out that
{\em ``we are all under an ego-driven temptation to project our private
thoughts out onto the real world, by supposing that the creations of one's own imagination are real
properties of Nature, or that one's own ignorance signifies some kind of indecision on the part of
Nature.''}

Let me finally acknowledge the help I got from friends and colleagues.

I have learned a lot from many colleagues, from their publications, from
their discussions and encouragements, from their co-operation. I warmly thank
Alastair Abbott,
Herbert Balasin, John Barrow, Douglas Bridges,
Ad\'an Cabello,  Cristian S. Calude,  Elena Calude, Kelly James Clark, John
Casti, Gregory Chaitin, Michael Dinneen,
Monica Dumitrescu, Jeffrey Koperski,
Andrei Khrennikov,
Frederick W. Kroon,
Jos\'{e} R. Portillo,
Jose Maria Isidro San Juan,
Ludwig Staiger,
Johann Summhammer,
Michiel van Lambalgen,
Udo Wid,
and  Noson Yanofsky.


This work was supported in part by the European Union, Research Executive Agency (REA),
Marie Curie FP7-PEOPLE-2010-IRSES-269151-RANPHYS grant.
In particular, I kindly thank  Pablo de Castro from the
Open Access Project of
LIBER - Ligue des Biblioth\`eques Europ\'eennes de Recherche
for his kind guidance and help with regards to the open access rendition of this book.

Last, but not least, I reserve a big thank you to Angela Lahee  from Springer-Verlag, Berlin, for a most pleasant and
efficient co-operation.


%% Please "sign" your preface
\vspace{1cm}
\begin{flushright}\noindent
Vienna, Zell am Moos \& Auckland,\hfill {\it Karl Svozil}\\
September 2017\hfill {\it }\\
\end{flushright}



