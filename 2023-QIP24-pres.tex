\PassOptionsToPackage{usenames,dvipsnames}{xcolor}
%\documentclass[amsmath,table,sans,amsfonts, handout]{beamer}
\documentclass[amsmath,table,sans,amsfonts,hyperref={colorlinks,citecolor=blue,linkcolor=blue,urlcolor=purple}]{beamer}
\usepackage[T1]{fontenc}
%%\usepackage{beamerthemeshadow}
%%\usepackage[headheight=1pt,footheight=10pt]{beamerthemeboxes}
%%\addfootboxtemplate{\color{structure!80}}{\color{white}\tiny \hfill Karl Svozil (TU Vienna)\hfill}
%%\addfootboxtemplate{\color{structure!65}}{\color{white}\tiny \hfill mur.sat \hfill}
%%\addfootboxtemplate{\color{structure!50}}{\color{white}\tiny \hfill Graz, 2010-12-11\hfill}
%\usepackage[dark]{beamerthemesidebar}
%\usepackage[headheight=24pt,footheight=12pt]{beamerthemesplit}
%\usepackage{beamerthemesplit}
%\usepackage[bar]{beamerthemetree}
\usepackage{graphicx}
\usepackage{pgf}
%\usepackage{eepic}
%\newcommand{\Red}{\color{Red}}  %(VERY-Approx.PANTONE-RED)
%\newcommand{\Green}{\color{Green}}  %(VERY-Approx.PANTONE-GREEN)

\definecolor{applegreen}{rgb}{0.55, 0.71, 0.0}

\usepackage{fourier-orns}  %fancy symbols https://mirror.easyname.at/ctan/fonts/fourier-GUT/doc/fourier-orns-doc.pdf

%\usepackage{musixtex}

\newcommand{\Abschnitt}[1]{{\section #1}}

%%%%%%%%%%%%%%%%%%%%%%%%%%%%%
\usepackage{iftex}
\ifxetex
\usepackage{fontspec}% Schriftumschaltung mit den nativen XeTeX-Anweisungen
                     % vornehmen. Voreinstellung: Latin Modern
%\usepackage[ngerman]{babel}% Sprachumschaltung: Deutsch nach neuer Rechtschreibung



%
% XeLaTeX
%
\XeTeXinputencoding cp1252
\usepackage{fontspec}
%%\setmainfont{Times New Roman}
\setmainfont{Garamond}
\setsansfont{Garamond}
%\setmainfont{EB Garamond}
%\setsansfont{EB Garamond}
%
\else
\usepackage[latin1]{inputenc}
\usepackage[T1]{fontenc}
\fi
%%%%%%%%%%%%%%%%%%%%%%%%%%%%%

%\RequirePackage[german]{babel}
%\selectlanguage{german}
%\RequirePackage[isolatin]{inputenc}

%\pgfdeclareimage[height=0.5cm]{logo}{tu-logo}
%\logo{\pgfuseimage{logo}}
\beamertemplatetriangleitem
%\beamertemplateballitem

\beamerboxesdeclarecolorscheme{alert}{red}{red!15!averagebackgroundcolor}
%\begin{beamerboxesrounded}[scheme=alert,shadow=true]{}
%\end{beamerboxesrounded}

%\beamersetaveragebackground{yellow!10}

%\beamertemplatecircleminiframe

\newtheorem{question}{Question}
\newtheorem{conjecture}[question]{Principle}
\newtheorem{challenge}[question]{Challenge}
\usepackage{tikz}
\newcommand{\bra}[1]{\left< #1 \right|}
\newcommand{\ket}[1]{\left| #1 \right>}

\newcommand{\iprod}[2]{\langle #1 | #2 \rangle}
\newcommand{\mprod}[3]{\langle #1 | #2 | #3 \rangle}
\newcommand{\oprod}[2]{| #1 \rangle\langle #2 |}

\newcommand{\proj}[3]{\begin{smallmatrix} #1 & #2 & #3 \end{smallmatrix}}
\newcommand{\projbf}[3]{\begin{smallmatrix} \mathbf{#1} & \mathbf{#2} & \mathbf{#3} \end{smallmatrix}}

\sloppy
\parskip .7em %vskip between paragraphs

\newcommand{\seq}[1]{\mathbf{#1}}
\newcommand{\floor}[1]{\left\lfloor #1 \right\rfloor}
\newcommand{\ceil}[1]{\left\lceil #1 \right\rceil}
\newcommand{\m}[1]{\widetilde{#1}}
%\newcommand{\p}[1]{\scriptsize\textcolor{black}{$[#1]$}}

\usepackage[most]{tcolorbox}
\begin{document}

\title{\textcolor{black}{\bf Converting nonlocality into contextuality (and back)}}
\subtitle{\footnotesize \url{http://tph.tuwien.ac.at/~svozil/publ/2023-QIP24-pres.pdf}
%%%\\
%%%\footnotesize based on \href{https://arxiv.org/abs/1903.10424}{arXiv:1903.10424}
}
\author{\textcolor{black}{Karl Svozil}}
\institute{\normalsize \textcolor{black}{Institute for Theoretical Physics, TU Wien}\\
\textcolor{black}{svozil@tuwien.ac.at}
%{\tiny Disclaimer: Die hier vertretenen Meinungen des Autors verstehen sich als Diskussionsbeitr�ge und decken sich nicht notwendigerweise mit den Positionen der Technischen Universit�t Wien oder deren Vertreter.}
}
\date{{\color{purple}Wednesday, 12 June 2024,
Quantum Information and Probability: from Foundations to Engineering (QIP24), V\"axj\"o, Sweden}}
\maketitle


% \frame{
% \frametitle{Contents}
% \tableofcontents
% }

\section{Five types of contextuality}

\begin{frame}%[shrink=4]
 \frametitle{Five types of contextuality: 1--3}

\begin{itemize}

\pause
\item[$\bullet$]   \color{orange}
Kochen-Specker all-out contextuality (1967, DOI 10.1512/iumj.1968.17.17004) --- complete absence of two-valued states interpretable as classical, binary  true-false valu(e)[ation];

\item[$\bullet$]   \color{blue}
Nonseparability (wrt two-valued states) of vertices --- cf KS ``demarcation criterion'' Theorem 0,  $\Gamma_3$? --- ``does anybody care''? I think not!

\item[$\bullet$]  \color{ForestGreen}
''Hardy-Type'' ones, such as TIFS and TITS, as exposed already by the KS ``bug'' (1965, DOI 10.1007/978-3-0348-9259-9\_19) two years before their ``major paper'', which is a TIFS; their $\Gamma_1$ is a TITS by an extension of the but TIFS;

\end{itemize}

\end{frame}

\begin{frame}%[shrink=4]
 \frametitle{Five types of contextuality:: 4,5}

\begin{itemize}

\item[$\bullet$]  \color{magenta}
Boole-Bell type violations of classical inequalities stemming from non-independent, non-separable quantum properties -- those violate classical predictions relative to the assumption of classical independent existence --- cf Froissart (1981, DOI 10.1007/BF02903286)
and Pitowsky (1986, DOI 10.1063/1.527066); eg, CHSH (4 disconnected contexts)
or intertwining contexts (aka orthonormal bases)  Svozil (2001, DOI 10.48550/arXiv.quant-ph/0012066) Specker bug, KLyashko (2008, DOI 10.1103/PhysRevLett.101.020403) pentagon/gram/house;

\item[$\bullet$]  \color{black}
GHZ Mermin type parity type proofs within a single context (more on this later).


\end{itemize}

\end{frame}


 \frame{
 \frametitle{Are there more?}

\centerline{\Large {\color{magenta} Are there more? Please let us know!}}

{\Large
\begin{center}\color{orange}
$\widetilde{\qquad \qquad }$
$\widetilde{\qquad \qquad}$
$\widetilde{\qquad \qquad }$
\end{center}
 }
 }


\frame{
 \frametitle{Operator valued arguments `mask' the respective contexts through spectral composition}

{\color{black}

\begin{itemize}

\item[$\bullet$]
Contexts $\equiv$ orthonormal basis.

\item[$\bullet$]
Normal (eg, hermitian, unitary) operators have a spectral (de)composition in terms of the sum of their their eigenvalues times orthogonal projection operators .

\item[$\bullet$]
Those (mutually orthogonal) orthogonal projection operators can be expressed in terms of the dyadic products of elements of
an orthonormal basis aka context.
\end{itemize}

}



\begin{frame}%[shrink=8]
 \frametitle{Challenges to space-time formation for ``non-localized'' contextuality cntd.}

 {\color{black}
In my opinion, as speculated earlier, in order to reasonably categorize space-time (frames) we need to take
quantum mechanics as primary, and develope and operationalize space-time frames entirely by quantum means.

This is different from Kantian conceptualizations of space-time as ``intuited a priori''.

As a consequence we need to observe what, in quantum terms, may be considered {\color{magenta} separate}, and what not.

I postulate that constituents in entangled states {\color{magenta} as well as measurement outcomes on such states cannot be considered separate}.

}

\end{frame}

\begin{frame}%[shrink=8]
 \frametitle{Challenges to space-time formation for ``non-localized'' contextuality cntd.}

 {\color{black}

Therefore, at least in the entangled observables, the constituents are not spatially separated at all: in other words,
for such ``affected'' observables, spatial distances shrink to zero.

This does not necessarily mean that with respect to other observables of these constituents, the separation is zero.

In this view {\color{magenta} spatial separation is means relative}, and thus {\color{magenta}  space-time frames are means relative} with respect to
the quantum shares involved.

}

\end{frame}


\section{Suggestions for a new protocol of clock synchronization}
\frame{
 \frametitle{Suggestions for a new protocol of clock synchronization}

 {\color{black}
For entangled shares I therefore suggest to abandon Einstein clock synchronization by exchanges of (light) signals.

I suggest to employ a Bennett-Brassard-Eckert-type protocol utilizing random outcomes of entangled multi-partite states
as a time standard.
Thereby, local entangled time is successively made precise and generated by the correlated outcomes of entangled states.

As a result, relativity theory is ``relativized'' further by into a multitude of means relative ``patches'' of space-time (frames).

}

}

\frame{

\centerline{\Large {\color{magenta} Thank you for your attention!}}

\begin{center}\color{orange}
$\widetilde{\qquad \qquad }$
$\widetilde{\qquad \qquad}$
$\widetilde{\qquad \qquad }$
\end{center}
 }
 \end{document}


















\section{ }

\frame{
 \frametitle{ }

\begin{itemize}
\item[$\bullet$] {
%\color{purple}
}
\pause
\item[$\bullet$] {
%\color{purple}
}
\end{itemize}
}

\section{ }

\frame{
 \frametitle{ }

\begin{itemize}
\item[$\bullet$] {
%\color{purple}
}
\pause
\item[$\bullet$] {
%\color{purple}
}
\end{itemize}
}

\section{ }

\frame{
 \frametitle{ }

\begin{itemize}
\item[$\bullet$] {
%\color{purple}
}
\pause
\item[$\bullet$] {
%\color{purple}
}
\end{itemize}
}

\section{ }

\frame{
 \frametitle{ }

\begin{itemize}
\item[$\bullet$] {
%\color{purple}
}
\pause
\item[$\bullet$] {
%\color{purple}
}
\end{itemize}
}

