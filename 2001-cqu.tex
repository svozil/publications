%%tth:\begin{html}<LINK REL=STYLESHEET HREF="/~svozil/ssh.css">\end{html}
\documentstyle[12pt]{article}
%\renewcommand{\baselinestretch}{2}
\renewcommand{\baselinestretch}{1.2}
\RequirePackage{times}
\RequirePackage{mathptm}
\begin{document}

\def\frak{\cal }
\def\Bbb{\bf }

\title{Forecast and event control\\On what is and what cannot be possible\\
Part II: Quantum mechanical case}
\author{Daniel Greenberger \\
 {\small Department of Physics,}
  {\small City College of the City University of New York}     \\
  {\small New York, NY 10031,}
  {\small USA   }            \\
{\small and}\\
Karl Svozil\\
 {\small Institut f\"ur Theoretische Physik,}
  {\small Technische Universit\"at Wien }     \\
  {\small Wiedner Hauptstra\ss e 8-10/136,}
  {\small A-1040 Vienna, Austria   }            \\
  {\small e-mail: svozil@tuwien.ac.at}}
\date{ }
\maketitle


\begin{abstract}
We present a quantum mechanical model of time travel and
discuss cronology protection schemes.
Such models impose restrictions upon certain capacities
of event control.
\end{abstract}


\subsubsection*{Quantum information}

By coherent superposition,
quantum theory manages to implement two classically inconsistent
bits of information by one quantum bit.
For example, consider the states $\vert +\rangle$ and $\vert -\rangle$
associated with the proposition that the spin of an electron
in a particular direction is ``up'' or ``down,'' respectively.
The coherent superposition  of these two states
$(\vert +\rangle + \vert -\rangle )/\sqrt{2}$
is a 50:50 mixture of these two classically distinct possibilities and at
the same time is a perfect quantum state.

Based upon this novel feature, we speculate that we may be able to solve
some tasks which are classically intractable or even inconsistent
by superposing quantum states in a self-consistent manner.
In particular, we could speculate that diagonalization tasks using
{\em not}--gates may become feasable,
although the capacities of agents within such semi-closed time loops may be limited
by requirements of (self-)consistency which translate into bounds by the
unitary quantum time evolution.
These quantum consistency requirements, however, may be less restrictive than in the classical
case \cite{maryland,svo-1995-paradox}.


\subsubsection*{Mach-Zehnder interferometer with feedback-loop}

In what follows we shall consider a Mach-Zehnder interferometer
as drawn in Fig. \ref{2001-cqu-f1} with two input and
two output ports \cite{green-horn-zei}.
The novel feature of this device is a feedback loop from the future
of one output port into the past of an input port.
Thereby we leave open the question of such a feedback loop into the past
and how it can (if ever) be realized.
Indeed, if one dislikes the idea of backwards-in-time communication,
one may think of this feedback loop as a channel which, by synchronizing the beams,
acts as if a beam  from the future enters the input port, while this
beam actually was emitted in the past from the output port.
\begin{figure}
\begin{center}
%TexCad Options
%\grade{\off}
%\emlines{\off}
%\beziermacro{\on}
%\reduce{\on}
%\snapping{\off}
%\quality{2.00}
%\graddiff{0.01}
%\snapasp{1}
%\zoom{1.00}
\unitlength 0.7mm
\linethickness{0.4pt}
\begin{picture}(90.00,100.00)
\bezier{452}(30.00,10.00)(-10.00,50.00)(30.00,90.00)
\bezier{112}(30.00,90.00)(40.00,100.00)(50.00,90.00)
\bezier{452}(50.00,90.00)(90.00,50.00)(50.00,10.00)
\bezier{112}(50.00,10.00)(40.00,0.00)(30.00,10.00)
\bezier{544}(75.00,95.00)(24.33,50.00)(75.00,5.00)
\put(10.33,50.00){\circle*{5.20}}
\put(50.00,50.00){\circle*{5.20}}
\put(70.00,50.00){\circle*{5.20}}
\put(20.00,50.00){\makebox(0,0)[cc]{$M$}}
\put(42.00,50.00){\makebox(0,0)[cc]{$G_1$}}
\put(80.00,50.00){\makebox(0,0)[cc]{$G_2$}}
\put(60.00,91.67){\line(0,-1){25.00}}
\put(60.00,35.00){\line(0,-1){25.00}}
\put(67.00,78.67){\makebox(0,0)[cc]{$t_2$}}
\put(67.00,21.00){\makebox(0,0)[cc]{$t_1$}}
\put(82.00,11.00){\makebox(0,0)[cc]{$\psi (t_1)$}}
\put(82.00,87.33){\makebox(0,0)[cc]{$\psi_3(t_2)$}}
\put(14.67,87.67){\makebox(0,0)[cc]{$\psi_4(t_2)$}}
\put(14.33,13.33){\makebox(0,0)[cc]{$\psi_4(t_1)$}}
\put(28.00,88.00){\vector(-1,-1){1.00}}
\put(23.67,16.67){\vector(2,-3){0.67}}
\put(50.67,41.33){\vector(-1,3){0.33}}
\put(69.00,41.33){\vector(1,2){0.33}}
\put(70.00,90.67){\vector(1,1){0.67}}
\put(70.67,8.67){\vector(-1,3){0.33}}
\put(75.00,32.67){\makebox(0,0)[cc]{$\psi_2$}}
\put(47.67,33.00){\makebox(0,0)[cc]{$\psi_1$}}
\end{picture}
\end{center}
    \caption{Mach-Zehnder device with backwards-in-time output $\psi_4(t_2)$
which passes $M$ and serves as input $\psi_4(t_1)$.
\label{2001-cqu-f1}}
\end{figure}


If one merely introduced feedback as in classical electrical engineering,
this would defy unitarity, as two input channels would be going into one forward channel,
which could not be uniquely reversed.
So one needs a feedback coupling that resembles a beam-splitter,
as in Fig. \ref{2001-cqu-f1}.
The operator $M$ generates the effects of the feedback in time.
These "beam-splitters" are figurative.
Their role is to couple the two incoming channels to two outgoing channels.
The operator $G_1$ represents the ordinary time development in the absence of time feedback.
The operator $G_2$ represents an alternate possible time evolution
that can take place and compete with $G_1$ because there is feedback.
We want to find   in the presence of the feedback in time that
is generated by the operator $M$.
At the beam splitters, the forward amplitude is $\alpha$,
while the reflected amplitude is $i \beta$.
The beam splitters are shown in Fig. \ref{2001-cqu-f2}.
\begin{figure}
\begin{center}
%TexCad Options
%\grade{\off}
%\emlines{\off}
%\beziermacro{\on}
%\reduce{\on}
%\snapping{\off}
%\quality{2.00}
%\graddiff{0.01}
%\snapasp{1}
%\zoom{1.00}
\unitlength 0.7mm
\linethickness{0.4pt}
\begin{picture}(40.00,40.00)
\put(0.00,0.00){\line(1,1){40.00}}
\put(40.00,0.00){\line(-1,1){40.00}}
\put(20.00,40.00){\line(0,-1){40.00}}
\put(10.00,30.00){\vector(-1,1){1.67}}
\put(30.00,30.00){\vector(3,4){1.25}}
\put(31.33,8.33){\vector(-3,4){1.00}}
\put(8.67,9.00){\vector(4,3){1.33}}
\put(10.00,3.00){\makebox(0,0)[cc]{$a$}}
\put(30.00,3.33){\makebox(0,0)[cc]{$b$}}
\put(30.00,37.67){\makebox(0,0)[cc]{$d$}}
\put(10.00,37.33){\makebox(0,0)[cc]{$c$}}
\end{picture}
\end{center}
    \caption{Reflection and transmission through a mirror with reflection coefficient
$\beta$ and transmission coefficient $\alpha$.
\label{2001-cqu-f2}}
\end{figure}
They perform the unitary transformation
\begin{equation}
\begin{array}{l}
\vert a\rangle =\alpha \vert d\rangle + i\beta\vert c\rangle \\
\vert b\rangle =\alpha \vert c\rangle + i\beta\vert d\rangle
\end{array}
\label{2001-cqu-e2}
\end{equation}
Here we assume for simplicity that $\alpha$  and $\beta$  are real.
We can invert this to obtain
\begin{equation}
\begin{array}{l}
\vert d\rangle =\alpha \vert a\rangle - i\beta\vert b\rangle \\
\vert c\rangle =\alpha \vert b\rangle - i\beta\vert a\rangle
\end{array}
\label{2001-cqu-e3}
\end{equation}
        The overall governing equations can be read from Fig. \ref{2001-cqu-f2}.
At time $t_2$ the second beam-splitter determines
$\psi_3(t_2)$  and  $\psi_4(t_2)$.
We have
\begin{equation}
\psi_3(t_2)  \equiv \psi_3' =\alpha \psi_1(t_2) -i\beta \psi_2(t_2) =\alpha \psi_1'-i\beta\psi_2',
\label{eq4}
\end{equation}
where the $'$ indicates the time $t_2$ in the argument,
and no prime indicates the time $t_1$.
The wave functions $\psi_1$  and   $\psi_2$ are determined at time $t_2$ by
\begin{eqnarray}
\psi_1(t_2) \equiv \psi_1' =G_1\psi_1(t_1)=G_1\psi_1, \\
\psi_2(t_2) \equiv \psi_2' =G_2\psi_2(t_1)=G_2\psi_2,
\end{eqnarray}
So that from eq. (\ref{eq4}),
\begin{equation}
\label{eq7}
\psi _{3} ^{\prime}  = \alpha G_{1} \psi _{1} - i\beta G_{2} \psi _{2}
,
\end{equation}
and equivalently
\begin{equation}
\label{eq8}
\psi _{4} ^{\prime}  = \alpha G_{2} \psi _{2} - i\beta G_{1} \psi _{1} .
\end{equation}
The propagator $M$ is what produces the feedback in time, propagating from
t$_{{\rm 2}}$ back to t$_{{\rm 1}}$, so that $\psi _{4} (t_{1} ) = M\psi
_{4} (t_{2} )$, or
\begin{equation}
\label{eq9}
\psi _{4} = M\psi _{4} ^{\prime} .
\end{equation}
At the t$_{{\rm 1}}$ beamsplitter,
\begin{eqnarray}
\label{eq10}
\psi _{1} &=& \alpha \psi - i\beta \psi _{4} ,             \\
\label{eq11}
\psi _{2} &=& \alpha \psi _{4} - i\beta \psi .
\end{eqnarray}





\subsubsection*{The Solution:}






First, we want to eliminate the $\psi _{4} $ in eqs. (\ref{eq10}) and (\ref{eq11}), to get
eqs. for $\psi _{1} $ and $\psi _{2} $. Then from eq. (\ref{eq7}) we can obtain
$\psi _{3} ^{\prime} $. From eqs. (\ref{eq8}) and (\ref{eq9}),
\begin{equation}
\label{eq12}
\psi _{4} = M\psi _{4} ^{\prime}  =  \alpha MG_{2} \psi _{2} - i\beta MG_{1}
\psi _{1}  .
\end{equation}
We plug this into eqs. (\ref{eq10}) and (\ref{eq11}),
\begin{eqnarray}
\label{eq13}
\psi _{1} &=& \alpha \psi - i\beta (\alpha MG_{2} \psi _{2} - i\beta MG_{1}
\psi _{1} ), \\
\label{eq14}
\psi _{2} &=& \alpha (\alpha MG_{2} \psi _{2} - i\beta MG_{1} \psi _{1} ) -
i\beta \psi .
\end{eqnarray}
We can rewrite these as
\begin{eqnarray}
\label{eq15}
\psi _{1} &=& (1 + \beta ^{2}MG_{1} )^{ - 1}( - i\alpha \beta MG_{2} )\psi
_{2} + \alpha (1 + \beta ^{2}MG_{1} )^{ - 1}\psi , \\
\label{eq16}
\psi _{2} &=& (1 - \alpha ^{2}MG_{2} )^{ - 1}( - i\alpha \beta MG_{1} )\psi
_{1} - i\beta (1 - \alpha ^{2}MG_{2} )^{ - 1}\psi .
\end{eqnarray}



These are two simultaneous equations that we must solve to find $\psi _{1} $
and $\psi _{2} $ as functions of $\psi $. To solve for $\psi _{1} $,
substitute eq. (\ref{eq16}) into (\ref{eq15}),
\begin{eqnarray}
\psi _{1} &=& (1 + \beta ^{2}MG_{1} )^{ - 1}( - i\alpha \beta MG_{2} )[(1 -
\alpha ^{2}MG_{2} )^{ - 1}( - i\alpha \beta MG_{1} )\psi _{1}
 \nonumber \\
&&\qquad  - i\beta (1-\alpha ^{2}MG_{2} )^{ - 1}\psi ]+ \alpha (1 + \beta ^{2}MG_{1} )^{ - 1}\psi
.
\label{eq17}
\end{eqnarray}
This can be rewritten as
\begin{equation}
\label{eq18}
\begin{array}{l}
 [1 + \alpha ^{2}\beta ^{2}(1 + \beta ^{2}MG_{1} )^{ - 1}(MG_{2} )(1 -
\alpha ^{2}MG_{2} )^{ - 1}(MG_{1} )]\psi _{1} \\
 = (1 + \beta ^{2}MG_{1} )^{ - 1}[ - \alpha \beta ^{2}MG_{2} (1 - \alpha
^{2}MG_{2} )^{ - 1} + \alpha ]\psi . \\
 \end{array}
\end{equation}
If we write this as
\begin{equation}
\label{eq19}
[X]\psi _{1} = (Y)^{ - 1}[Z]\psi ,
\end{equation}
then we can simplify the equation as follows:
\begin{eqnarray}
 XY&=& 1 + \beta ^2MG_1 + \alpha ^2\beta ^2MG_2 (1 - \alpha ^2MG_2 )^{- 1}MG_1 nonumber \\
  &=& 1 + \beta ^2[1 + (1 - \alpha ^2MG_2 )^{- 1}\alpha ^2MG_2 ]MG_1nonumber \\
  &=& 1 + \beta ^2(1 - \alpha ^2MG_2 )^ {- 1}MG_1 ,
\label{eq20}
\end{eqnarray}
and
\begin{eqnarray}
 Z&=& \alpha (1 - \alpha ^2MG_2 )^{- 1}(1 - \alpha ^2MG_2 - \beta ^2MG_2 )\nonumber \\
  &=& \alpha (1 - \alpha ^2MG_2 )^{- 1}(1 - MG_2 ).
\label{eq21}
\end{eqnarray}
Thus,
\begin{equation}
\label{eq22}
\psi _{1} = \alpha [1 + \beta ^{2}(1 - \alpha ^{2}MG_{2} )^{ - 1}MG_{1} ]^{
- 1}(1 - \alpha ^{2}MG_{2} )^{ - 1}(1 - MG_{2} )\psi .
\end{equation}


Then, using the identity $A^{ - 1}B^{ - 1} = (BA)^{ - 1}$, we finally obtain
\begin{equation}
\label{eq23}
\psi _{1} = \alpha (1 - \alpha ^{2}MG_{2} + \beta ^{2}MG_{1} )^{ - 1}(1 -
MG_{2} )\psi .
\end{equation}
We can solve for $\psi _{2} $ similarly, by substituting eq. (\ref{eq15}) into eq.
(\ref{eq16}),
\begin{equation}
\label{eq24}
\psi _{2} = - i\beta (1 - \alpha ^{2}MG_{2} + \beta ^{2}MG_{1} )^{ - 1}(1 +
MG_{1} )\psi .
\end{equation}



Notice that in the denominator term in both of eqs. (\ref{eq23}) and (\ref{eq24}),
$\alpha $ and $\beta $ have reversed the role of the operators they apply
to. We can finally use eq. (\ref{eq7}) to solve for $\psi _{3} ^{\prime}  = \psi
_{3} (t_{2} )$,
\begin{equation}
\label{eq25}
\begin{array}{l}
 \psi _{3} (t_{2} ) = [\alpha ^{2}G_{1} D(1 - MG_{2} ) - \beta ^{2}G_{2} D(1
+ MG_{1} )]\psi (t_{1} ), \\
 \textrm{where}\quad D = (1 + \beta ^{2}MG_{1} - \alpha ^{2}MG_{2} )^{ - 1}. \\
 \end{array}
\end{equation}





\subsubsection*{Important Special Cases}


\noindent
(i) For commuting $M$, $G_1$ and $G_2$,
$D = \beta ^{2}(1+MG_{1}) + \alpha ^{2}(1-MG_{2} )$, and
\begin{equation}
\label{eq25a}
\psi _{3} ^{\prime}  =
{\alpha^2G_1-\beta^2 G_2-MG_1G_2\over
1+  \beta^2 MG_1-\alpha^2 MG_2}
 \psi (t_{1} ).
\end{equation}



\noindent
(ii) For $\alpha = 1,\;\beta = 0$: This is the case where there is no
feedback. Here
\begin{equation}
\label{eq26}
\psi _{3} ^{\prime}  = G_{1} (1 - MG_{2} )^{ - 1}(1 - MG_{2} )\psi = G_{1}
\psi .
\end{equation}





\noindent
(iii) For $\beta = 1,\;\alpha = 0$: This is the case where there is only
feedback. Here
\begin{equation}
\label{eq27}
\psi _{3} ^{\prime}  = - G_{2} (1 + MG_{1} )^{ - 1}(1 + MG_{1} )\psi = -
G_{2} \psi .
\end{equation}





\noindent
(iv) $G_{1} = G_{2} \equiv G:$
\begin{equation}
\label{eq28}
\psi _{3} ^{\prime}  = G[1 + (\beta ^{2} - \alpha ^{2})MG]^{ - 1}(\alpha
^{2} - \beta ^{2} - MG)\psi .
\end{equation}





\noindent
(iv') If also, $\alpha ^{2} = \beta ^{2} = {\textstyle{{1} \over {2}}}:$ then
\begin{equation}
\label{eq29}
\psi _{3} ^{\prime}  = - GMG\psi .
\end{equation}





\noindent
(v) If $\beta < < 1,$ which is expected to be the usual case, then the answer
only depends on $\beta ^{2} = \gamma .$ Also, $\alpha ^{2} = 1 - \beta ^{2}
= 1 - \gamma .$ Then to lowest order in $\gamma $, the denominator D in eq.
(\ref{eq25}) becomes
\begin{equation}
\label{eq30}
\begin{array}{l}
 D = [1 + \gamma MG_{1} - (1 - \gamma )MG_{2} ]^{ - 1} \\
 = (1 - MG_{2} )^{ - 1} - \gamma (1 - MG_{2} )^{ - 1}(MG_{1} + MG_{2} )(1 -
MG_{2} )^{ - 1}, \\
 \end{array}
\end{equation}
so that
\begin{equation}
\label{eq31}
\begin{array}{l}
\psi_3'=\left\{
(1-\gamma)G_1[1-\gamma (1-MG_2)^{-1}(MG_1+MG_2)]-\gamma G_2(1-MG_2)^{-1}(1+MG_1) \right\} \psi
 \\
\qquad =\left[G_1  -\gamma (G_1+G_2)(1-MG_2)(1+MG_1)\right] \psi .
 \end{array}
\end{equation}





\noindent
(vi) The case that corresponds to the classical paradox where an agent shoots his
father before he has met the agent's mother, so that the agent can never be born, has an
interesting quantum-mechanical resolution. This is the case G$_{{\rm 1}}$=0,
where there is a perfect absorber in the beam so that the system would never
get to evolve to time t$_{{\rm 2}}$. But quantum mechanically, there is
another path along G$_{{\rm 2}}$, the one where the agent does not shoot his
father, that has a probability $\beta $ without feedback. The solution in
this case is
\begin{equation}
\label{eq32}
\psi _{3} ^{\prime}  = - \beta ^{2}G_{2} (1 - \alpha ^{2}MG_{2} )^{ - 1}\psi
.
\end{equation}
We assume for simplicity that G$_{{\rm 2}}$ is just the standard time
evolution operator
\begin{equation}
\label{eq33}
G_{2} = e^{ - iE(t_{2} - t_{1} ) / \hbar} ,
\end{equation}
and M is just the simplest backwards in time evolution operator
\begin{equation}
\label{eq34}
M = e^{ - iE(t_{1} - t_{2} ) / \hbar + i\varphi} ,
\end{equation}
where we have also allowed for an extra phase shift. Then
\begin{eqnarray}
\psi_3'&=&-\beta e^{ - iE(t_{2} - t_{1} ) / \hbar }  [1-\alpha^2e^{i\varphi}]^{-1}\psi ,
\nonumber \\
\vert \psi_3'  \vert^2&=&{\beta^4\over (1-\alpha^2 e^{i\varphi})(1-\alpha^2  e^{^-i\varphi})}
\vert \psi \vert^2 ,
\nonumber \\
&=&
{1\over 1+4(\alpha^2/\beta^2)\sin^2 (\varphi /2)}
\vert \psi \vert^2    .
\label{eq35}
\end{eqnarray}



Note that for $\varphi = 0$, $\psi _{3} ^{\prime}  = - e^{ - iE\Delta t /
\hbar} \psi $, for \textit{any} value of $\beta $. That means that no matter how small
the probability of the agent ever having reached here in the first place, the
fact that he \textit{is} here $(\alpha \ne 1)$ guarantees that even though he is
certain to have shot his father if he had met him (G$_{{\rm 1}}$=0),
nonetheless the agent will not have met him! The agent will have taken the other path,
with 100\% certainty.

How can we understand this result? In our model, with $\varphi = 0$, we have
G$_{{\rm 1}}$=0, and MG$_{{\rm 2}}$ = 1. Also, we will assume that $\beta <
< 1$, even though this is not necessary. The various amplitudes are
\begin{equation}
\label{eq36}
\begin{array}{l}
 {\left| {\psi _{1}}  \right|} = 0,\quad {\left| {\psi _{2} / \psi}
\right|} = 1 / \beta , \\
 {\left| {\psi _{4} / \psi}  \right|} = \alpha / \beta ,\quad {\left| {\psi
_{3} ^{\prime}  / \psi}  \right|} = 1. \\
 \end{array}
\end{equation}



So we see that the two paths of the beam-splitter at t$_{{\rm 1}}$ leading
to the path $\psi _{1} $ cancel out. But of the beam $\psi $, $\alpha $
passes through, while of the beam $\psi _{4} $, only $\beta $ leaks through.
So the beam $\psi _{4} $ must have a very large amplitude, which it does, as
we can see from (\ref{eq36}). In fact it has a much larger amplitude than the
original beam. Similarly, in order that ${\left| {\psi _{3} ^{\prime}}
\right|} = {\left| {\psi}  \right|}$, then $\psi _{2} $ must have a very
large amplitude. Thus we see that there is a large current flowing around
the system, between $\psi _{2} $ and $\psi _{4} $. But doesn't this violate
unitarity? The answer is that if they were both running forward in time, it
would. But one of these currents is running forward in time, while the other
runs backward in time, and so they do not in this case violate unitarity.
This is how our solution is possible.

So, according to our model, in quantum mechanics, if one could travel into
the past, one would only see those alternatives consistent with the world
one left. In other words, while one could see the past, one could not change
it. No matter how unlikely the events are that could have led to one's
present circumstances, once they have actually occurred, they cannot be
changed. One's trip would set up resonances that are consistent with the
future that has already unfolded.

This also has  consequences on the paradoxes of free will. It shows
that it is perfectly logical to assume that one has many choices and that
one is free to take any one of them. Until a choice is taken, the future is
not determined. However, once a choice is taken, it was inevitable. It could
not have been otherwise. So, looking backwards, the world is deterministic.
However, looking forwards, the future is probabilistic.

The model also has consequences concerning a many worlds interpretation of
quantum theory. The world may appear to keep splitting so far as the future
is concerned, however once a measurement is made, only those histories
consistent with that measurement are possible. In other words, with time
travel, other alternative worlds do not exist, as once a measurement has
been made, they would be impossible to reach from the original one.

Another interesting point comes from examining eq. (\ref{eq35}). For small angles
$\varphi $ we see that
\begin{equation}
\label{eq37}
{\left| {\psi _{3} ^{\prime}}  \right|}^{2} = {\frac{{1}}{{1 +
4{\textstyle{{\alpha ^{2}} \over {\beta ^{4}}}}\sin ^{2}(\varphi /
2)}}}{\left| {\psi}  \right|}^{2} \to {\frac{{1}}{{1 + {\textstyle{{\alpha
^{2}\varphi ^{2}} \over {\beta ^{4}}}}}}}{\left| {\psi}  \right|}^{2},
\end{equation}
so that the above result is strongly resonant, with a Lorentzian shape, and
a width $\Delta \varphi \sim \beta ^{2}$, since $\alpha \sim 1$. Thus less
``deterministic'' and fuzzier time-travelling might be possible.


(vi) Sustaint case:
if we require the input and output state to be identical; i.e.,
$\psi_3(t_2)=\psi (t_1)$,
then we obtain a sustainment condition (for commuting $M,G_1,G_2$) of
\begin{equation}
\label{eq38}
1=
G_1(\alpha^2-\beta^2 M)+
G_2(\alpha^2M-\beta^2 )- MG_1G_2.
\end{equation}

\noindent
Another  case is $G_1=G_2=1$, a phase shift in $M=e^{i\varphi}$, and
$\alpha=\beta=1/\sqrt{2}$, for which we obtain
$\vert \psi_3'\vert=\vert \psi \vert$. For $\beta=\sqrt{1-\alpha^2}=1/4$,
\begin{equation}
\label{eq39}
\vert \psi_3'\vert=
{112-113\cos(\varphi)-15i\sin{\varphi}\over
54\vert 1-7e^{i\varphi}/8\vert^2}
\vert \psi \vert
\end{equation}


We summarize by stating that the structure of a quantum time travel through a
Mach-Zehnder device is rich and unexpectedly elaborate.
This suggests totally new szenarios for the possibility of free will
and the capacities available to an agent acting in such a time loop.


\bibliography{svozil}
\bibliographystyle{unsrt}
%\bibliographystyle{plain}

\end{document}
