\documentclass[prl,preprint,amsfonts]{revtex4}
%\documentclass[pra,showpacs,showkeys,amsfonts]{revtex4}
\RequirePackage{times}
\RequirePackage{courier}
\RequirePackage{mathptm}
%\RequirePackage{bookman}
%\RequirePackage{helvetic}
%\RequirePackage{times}
\renewcommand{\baselinestretch}{1.3}

\begin{document}
The paper consists of three main parts, which are not directly related:
(i) In the first part, the author critically reviews attempts to represent a halting state in quantum computation;
(ii) The second part is about a criticism of Kieu's proposal for hyper-Turing computability;
(iii) the third part is about the ``essence'' of quantum computation.

I am afraid that although all parts contain interesting thoughts,
not one of them justifies publication.
The first part is mostly a review,
in which little efforts are made to find solutions for the posed problems.

The second part which carries the rather misleading
title ``second halting problem'' is the most technical one,
but I find the arguments not convincing.
Without going into details my opinion is that
most of them could be circumvented with a little maneuvering from Kieu's side,
which does not mean that I find Kieu's arguments are convincing in the first place.
It is just that an unconvincing argument cannot be disproved with another unconvincing argument.

The third part for me is the most interesting and the most speculative.
There exist many articles attempting to condense the gist of quantum computation;
for instance

One complexity theorist's view of quantum computing
by Lance Fortnow
http://dx.doi.org/10.1016/S0304-3975(01)00377-2

which the author does not mention. This is one of them.

I am afraid that, as it stands, I cannot recommend publication of the manuscript.



\end{document}


