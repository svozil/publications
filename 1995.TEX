\documentstyle[12pt]{article}
\begin{document}

\section{Publications}

1.\\
 K. Svozil\\
``On the computational power of physical systems,
undecidability, the consistency of phenomena and
 the practical uses of paradoxa''\\
in {\sl Fundamental Problems in
Quantum Theory: A Conference Held in Honor of Professor John A.
Wheeler}, ed. by D. M. Greenberger and A. Zeilinger,  {\sl Annals of the
New York Academy of Sciences} {\bf 755}, 834-841 (1995)\\
J\\
((maryland))


2.\\
Cristian Calude,
Douglas I. Campbell,
 Karl Svozil and
Doru \c{S}tef\u anescu \\
``Strong Determinism vs.  Computability''\\
in {\it The Foundational Debate, Complexity and Constructivity in
Mathematics and Physics}, Werner DePauli Schimanovich, Eckehart K\"ohler
and Friedrich Stadler, eds. (Kluwer, Dordrecht, Boston, London, 1995),
p. 115-131\\
J\\
((calude))


3.\\
 K. Svozil\\
``A constructivist manifesto for the physical  sciences''\\
 in
{\it The Foundational Debate, Complexity and Constructivity in
Mathematics and Physics}, Werner DePauli Schimanovich, Eckehart K\"ohler
and Friedrich Stadler, eds. (Kluwer, Dordrecht, Boston, London, 1995),
p. 65-88\\
J         \\
((complexity))

4.\\
K. Svozil\\
``Time paradoxa reviewed''\\
 Phys. Lett. {\bf A 199}, 323-326 (1995)\\
J       \\
((afraid))


5.\\
 K. Svozil, \\
``Quantum computation and complexity theory I''\\
 Bulletin of the European Association of Theoretical Computer
Sciences {\bf 55}, 170-207 (1995)\\
J\\
((qct1))

6.\\
K. Svozil,\\
``Quantum computation and complexity theory II''\\
{\sl Bulletin of the European Association of Theoretical Computer
Sciences} {\bf 56}, 116-145 (1995)\\
J\\
((qct2))

7.\\
K. Svozil\\
``Halting probability amplitude of quantum
computers''\\
 Journal of Universal Computer Science
{\bf 1}, nr. 3, 1-4 (March 1995)\\
J\\
((omega))




\section{Lectures \& posters}


1.\\
K. Svozil\\
``Description and Prediction'' \\
1st International UNESCO Conference  on  Transdisciplinarity\\
Ariba (Portugal)\\
7. 11. 1994\\
((1-svoz1))


2.\\
K. Ehrenberger und K. Svozil\\
``Fraktale Kodierung von Nervenimpulsen'' \\
Medizinertagung\\
Salzburg (\"Osterreich)\\
1. 12. 1994\\
((2-svoz2))


3.\\
K. Svozil\\
``Varieties of Physical Undecidability''\\
Limits to Scientific Knowledge\\
Abisko (Schweden)\\
15. 5.95\\
((1-svoz3))

4.\\
K. Svozil\\
``Extrinsic versus intrinsic representability''
Limits to Scientific Knowledge\\
Abisko (Schweden)\\
17. 5.95  \\
((2-svoz4))

5.\\
K. Svozil\\
``Bridgman's operationalism and other set theoretical issues related to
physics''\\
Kurt G\"odel Seminar, Technische Universit\"at Wien\\
Wien (\"Osterreich)\\
19. 6.95    \\
((3-svoz5))

6.\\
K. Svozil\\
``Quantum algorithmic information theory''\\
Chaitin Complexity and its Applications\\
Mangalia (Rum\"anien) \\
2. 7.95       \\
((1-svoz6))

7.\\
K. Svozil\\
``Set theory and physics''     \\
Chaitin Complexity and its Applications\\
Mangalia (Rum\"anien) \\
2. 7.95         \\
((2-svoz7))


8.\\
K. Svozil\\
``Quantum recursion theory''     \\
10th International Congress of Logic, Methodology and Philosophy of
Science\\
Florenz (Italien) \\
23. 8.95          \\
((2-svoz8))


9.\\
K. Krenn, J. Sumhammer and K. Svozil\\
``Interaction-free preparation''\\
\"OPG-Tagung\\
Leoben (\"Osterreich)\\
20. 9. 1995         \\
((2-svoz9))


10.\\
K. Svozil\\
Consistent use of paradoxes in deriving constraints on the
dynamics of physical systems and of no-go-theorems\\
\"OPG-Tagung\\
Leoben (\"Osterreich)\\
21. 9. 1995           \\
((2-svoz10))

11.\\
K. Svozil\\
``The diagonalization method in quantum recursion theory''\\
\"OPG-Tagung\\
Leoben (\"Osterreich)\\
21. 9. 1995             \\
((2-svoz11))

12.\\
K. Svozil, D. Felix and K. Ehrenberger\\
``Amplification by Stochastic Interference''\\
\"OPG-Tagung\\
Leoben (\"Osterreich)\\
21. 9. 1995               \\
((2-svoz12))

13.\\
Cristian Calude,
Douglas I. Campbell,
 Karl Svozil and
Doru \c{S}tef\u anescu \\
``Strong Determinism vs.  Computability''\\
\"OPG-Tagung\\
Leoben (\"Osterreich)\\
21. 9. 1995                 \\
((2-svoz13))

14.\\
K. Svozil\\
``Halting probability amplitude of quantum
computers''\\
\"OPG-Tagung\\
Leoben (\"Osterreich)\\
21. 9. 1995                   \\
((2-svoz14))



\section{G\"aste}

1.\\
Professor Dr. Anatolij Dvure\v{c}enskij (Svozil)\\
Mathematical Institute of the Slovak Academy of Sciences, Bratislava
(Slowakei)\\
1.2.95-11.2.95


2.\\
Professor Dr. Silvia Pulmannova (Svozil)\\
Mathematical Institute of the Slovak Academy of Sciences, Bratislava
(Slowakei)\\
1.2.95-11.2.95

3.\\
Professor Dr. Christian Calude (Svozil)\\
Department of Mathematics and Computer Sciences, University of
Auckland (Neuseeland)\\
20.06.95-30.06.95


4.\\
Professor Dr. Douglas Bridges (Svozil)\\
Department of Mathematics and Computer Sciences, University of
Waikato (Neuseeland)\\
20.06.95-30.06.95

5.\\
Professor Dr.
Roman R. Zapatrin (Svozil)\\
Department of Mathematics, SPb UEF,
St. Petersburg (Ru\ss land)\\
14.05.95-25.05.95

6.\\
Professor Dr.
Josef Tkadlec (Svozil)\\
Department of Mathematics, Technical University
Prag (Tschechien)\\
September 1995



\section{Forschungst\"atigkeit}

\subsection{Quantencomputer}
(Svozil)\\
Die Ma\ss einheit der quantenmechanischen Information, das
``Quanten-bit'', ist die koh\"arente \"Uberlagerung der beiden
klassischen Bitzust\"ande.
Dies erlaubt die formale Darstellung von inkonsistenter klassischer
Information, z. B. in Datenbanken oder Expertensystemen, in
Quantencomputer.
Paper:
\ref{maryland},
\ref{qct1},
\ref{qct2},
\ref{omega}
\\
Vortrag:
\ref{1-svoz6},
\ref{2-svoz8},
\ref{2-svoz11},
\ref{2-svoz14},
\\

\subsection{Automatenlogik}
(Svozil)\\
Durch Ein- und Ausgabeexperimente wird der Anfangszustand eines
endlichen deterministischen Automaten gemessen. Die Propositionsstruktur
kann man algebraisch mit Hilfe der in diesem Zusammenhang entwickelten
sogennanten
``Partitionslogik'' beschreiben.
Es wird untersucht, inwieweit die so gefundene Automaten- oder
Partitionslogik mit der Quanenlogik \"ubereinstimmt.
Paper:
\ref{maryland},
\ref{complexity}
\\
Vortrag:
\ref{1-svoz3},
\ref{2-svoz4},
\ref{2-svoz13}
\\



\subsection{Fraktale Signalcodierung}
(Svozil)\\
Die prim\"aren nervlichen Erregungsmuster, welche die Ohrnerven
erzeugen, haben eine stochastische fraktale geometrische Struktur. Es
kann gezeigt werden, da\ss $\,$ schon kleine \"Anderungen der Dimension
dieser prim\"aren Erregungsmuster durch geeignete Verarbeitung
(Stichwort ``Konvergenz und Divergenz'' der Nervenbahnen) zu gro\ss en
\"Anderungen in der Dimension von s\"akunderen Erregungsmustern
f\"uhren.
Paper:
\\
Vortrag:
\ref{2-svoz12},
\ref{2-svoz2}
\\

\end{document}
