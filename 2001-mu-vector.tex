%%tth:\begin{html}<LINK REL=STYLESHEET HREF="/~svozil/ssh.css">\end{html}
\documentclass[prl,preprint,showpacs,showkeys,amsfonts]{revtex4}
\usepackage{graphicx}
%\documentstyle[amsfonts]{article}
 \RequirePackage{times}
%\RequirePackage{courier}
\RequirePackage{mathptm}
%\renewcommand{\baselinestretch}{1.3}
\begin{document}

%\def\mathfrak{\cal }
%\def\Bbb{\bf }
%\sloppy




\title{Finite-dimensional  vector and Hilbert spaces\\
Handout ``Methoden der Theoretischen Physik-\"Ubungen''}
\author{Karl Svozil}
 \email{svozil@tuwien.ac.at}
\homepage{http://tph.tuwien.ac.at/~svozil}
\affiliation{Institut f\"ur Theoretische Physik, University of Technology Vienna,
Wiedner Hauptstra\ss e 8-10/136, A-1040 Vienna, Austria}

\begin{abstract}
Linear vector spaces and some of their basic principles are introduced.
\end{abstract}


\pacs{45.10.Na,02.70}
\keywords{Linear vector spaces}

\maketitle

%\section{Definitions}

\section{Definition}


A quantum mechanical {\em (linear) vector space} is a linear
\index{vector space}
vector space ${\frak V}$ over the field ${\Bbb C}$ of complex numbers
(with vector addition
and scalar multiplication),
such that
(i)
$(x,x)=0$ if and only if $x=0$;
(ii)
$(x,x)\ge 0$ for all $x \in{\frak V}$;
(iii)
$(x+y,z)=(x,z)+(y,z)$ for all $x,y,z \in {\frak V}$;
(iv)
$(\alpha x,y)=\alpha (x,y)$ for all $x,y \in {\frak V}, \alpha \in {\Bbb C}$;
(v)
$(x,y)={(y,x)}^\ast $ for all $x,y \in {\frak V}$
(${\alpha }^\ast $ stands for the complex conjugate of $\alpha$);



(vi)
If $x_n\in {\frak V}$, $n=1,2,\ldots$, and if $\lim_{n,m\rightarrow
\infty} (x_n-x_m,x_n-x_m)=0$, then there exists an $x\in {\frak V}$ with
$\lim_{n\rightarrow \infty} (x_n-x,x_n-x)=0$.




Unless stated differently, only
finite-dimensional vector spaces are considered.\footnote{
Infinite dimensional cases and continuous spectra are nontrivial
extensions of the finite
dimensional Hilbert space treatment. As a heuristic rule, which is not
always correct, it might be
stated that the sums become integrals, and the Kronecker delta function
$\delta_{ij}$
becomes the Dirac delta function $\delta (i-j)$, which is a
generalized function in the continuous variables $i,j$.
In the Dirac bra-ket notation, unity is given by
${\bf 1}=\int_{-\infty}^{+\infty} \vert i)( i\vert \, di$.
For a careful treatment, see, for instance,
the books by
Reed and Simon \cite{reed-sim1,reed-sim2}.}

\section{Scalar product}




To avoid a shock from a too early exposure  to ``exotic''
nomenclature prevalent in physics --- the Dirac bra-ket notation --- the
notation of Dunford-Schwartz
\cite{dunford-schwartz} is adopted.\footnote{
The bra-ket
notation introduced by Dirac is widely used in physics. To
translate expressions into the bra-ket notation, the following
identifications work for most practical purposes: for the scalar
product,
``$\langle  \equiv \;($'',
``$\rangle  \equiv \; )$'',
``$, \equiv \; \mid $''.
States are written as
$\mid  \psi \rangle  \equiv \psi$, operators as
$\langle  i\mid  A\mid  j \rangle  \equiv A_{ij}$.}


The {\em scalar} or {\em inner product} is a complex function
$(\cdot ,\cdot )$ defined on ${\frak V}\times{\frak V}$
such that
(i)
$(x,x)=0$ if and only if $x=0$;
(ii)
$(x,x)\ge 0$ for all $x \in{\frak V}$;
(iii)
$(x+y,z)=(x,z)+(y,z)$ for all $x,y,z \in {\frak V}$;
(iv)
$(\alpha x,y)=\alpha (x,y)$ for all $x,y \in {\frak V}, \alpha \in {\Bbb C}$;
(v)
$(x,y)={(y,x)}^\ast $ for all $x,y \in {\frak V}$
(${\alpha }^\ast $ stands for the complex conjugate of $\alpha$);



 \bibliography{/mytex/svozil}
 \bibliographystyle{aalpha}
\end{document}
\begin{eqnarray}

