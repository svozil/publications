%\documentclass[pra,showpacs,showkeys,amsfonts,amsmath,twocolumn]{revtex4}
\documentclass[amsmath,blue]{beamer}
%\documentclass[pra,showpacs,showkeys,amsfonts]{revtex4}

\usepackage{beamerthemeshadow}
%\usepackage[dark]{beamerthemesidebar}
%\usepackage[headheight=24pt,footheight=12pt]{beamerthemesplit}
%\usepackage{beamerthemesplit}
%\usepackage[bar]{beamerthemetree}
\usepackage{graphicx}
\usepackage{pgf}

\RequirePackage[german]{babel}
\selectlanguage{german}
\RequirePackage[isolatin]{inputenc}

\pgfdeclareimage[height=0.5cm]{logo}{tu-logo}
\logo{\pgfuseimage{logo}}
\beamertemplatetriangleitem
\begin{document}
\title{\bf Statistical Mechanics I}
\subtitle{(132.058 VO Statistische Physik I SS 2,0) \\http://tph.tuwien.ac.at/\~{}svozil/publ/2006-statphys1}
%\subtitle{http://www.arxiv.org/abs/quant-ph/0406014}
\author{Karl Svozil}
\institute{Institut f\"ur Theoretische Physik, University of Technology Vienna, \\
Wiedner Hauptstra\ss e 8-10/136, A-1040 Vienna, Austria\\
svozil@tuwien.ac.at}
\date{SS 06 ;-)}
\maketitle

\frame[shrink=2]{\tableofcontents}


\section{Wiederholung Thermodynamik}
\section{Postulate der (Quanten) Statistischen Mechanik}
\section{Dichtematrix}
\section{Ensembletheorie (mikrokanonisches, kanonisches, gro�kanonisches Ensemble)}
\section{klassische Ensembles als Grenzwerte}
\section{Ableitung der Thermodynamik aus der Statistischen Mechanik}
\section{Legendre-Transformationen}
\section{Gleichverteilungssatz}
\section{ideales Gas}
\section{Phasenregel  von Gibbs}
\section{Response-Funktionen;}
\section{spezifische W�rme (Festk�rper, Gas)}
\section{Fermi-Systeme}
\section{Bose-Systeme}



\frame{
\frametitle{References}

Huang, Kerson (1990). Statistical Mechanics, 2nd Edition, Wiley, John \& Sons, Inc. ISBN 0471815187.
}
\end{document}

