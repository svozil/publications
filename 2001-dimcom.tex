%%tth:\begin{html}<LINK REL=STYLESHEET HREF="/~svozil/ssh.css">\end{html}
\documentclass[prl,preprint,showpacs,showkeys]{revtex4}
\usepackage{graphicx}
%\documentstyle[amsfonts]{article}
%\RequirePackage{times}
%\RequirePackage{courier}
%\RequirePackage{mathptm}
%\renewcommand{\baselinestretch}{1.3}
\begin{document}

%\def\frak{\cal }
%\def\Bbb{\bf }
%\sloppy



\title{Comment on ``Precise Measurement of the Positive Muon Anomalous Magnetic Moment''}
\author{Karl Svozil}
\email{svozil@tuwien.ac.at}
\homepage{http://tph.tuwien.ac.at/~svozil}
\affiliation{Institut for Theoretical Physics, University of Technology Vienna,
Wiedner Hauptstra\ss e 8-10/136, A-1040 Vienna, Austria}
\author{Anton Zeilinger}
\email{anton.zeilinger@univie.ac.at}
\affiliation{Institute for Experimantal Physics, University of Vienna,
Boltzmanngasse 5, A-1090 Vienna, Austria}

\pacs{???}
\keywords{Muon Anomalous Magnetic Moment, Quantum Electrodynamics}

\maketitle


In the most precise measurement of the muon anomalous magnetic moment ever performed, the
Muon g-2 Collaboration reported a significant deviation of the measurement result
from the theoretical value \cite{muong-2col2001} by
$a_\mu(\textrm{exp})-a_\mu (\textrm{SM})= 43(16)\times 10^{-10}$.
They mentioned several speculative theories, in particular supersymmetry,
muon substructure and anomalous $W$ couplings.
Here we would like to briefly refer to another speculative
explanation of the discrepancy between theory and experiment:
a noninteger dimension of spacetime. This possibility has been originally proposed
by the authors \cite{sv1,sv2,sv3,sv4}
(for a  review see Jammer \cite{jammer-s3})
in the context of the electron anomalous magnetic moment.
In the approximation given these results can be readily applied to the muon case
and yield a fractal dimension $D$ of space-time of
\begin{eqnarray}
 D&=& 4+{8\pi \over \alpha_f (C+\log \pi)}\left[a_\mu(\textrm{exp})-a_\mu (\textrm{SM})\right]
\nonumber \\
  &=& 4+ 56(32)\times 10^{7}.
\nonumber
\end{eqnarray}
$C\approx 0.57722$ and $\alpha_f$ stand for Euler's and the fine structure constant.
Since $D>4$, the fractality of space-stime could not give rise to the elimination
of ultraviolet divergencies in the standard model of electroweak interactions.

We would also like to point to the possibility
that the dimensional parameter may vary with the scale $\delta$ involved.
For macroscopic distances and large times, $D(\delta )$ might have the value four ``for all practical purposes,''
whereas in the microphysical domain
the dimension might be different from four due to quantum and
other, possibly nonlinear effects.
Such nonlinear effects may for instance occur if the space-time structure and the quantized fields
are tied together in a nonlinear feedback loop, such that space-time is coupled to quantized fields
and conversely these fields determine the space-time structure.
Evidently, the exact form of any such scale behavior depends on the particular model.
One such concrete model giving rise to scale dependent fractal dimension parameter has for instance
been reviewed by Crane and Smolin  \cite{crane-smolin}.



\bibliography{svozil}
\bibliographystyle{apsrev}
%\bibliographystyle{unsrt}
%\bibliographystyle{plain}




\end{document}
