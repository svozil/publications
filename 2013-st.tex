\documentclass[%
 %reprint,
  twocolumn,
 %superscriptaddress,
 %groupedaddress,
 %unsortedaddress,
 %runinaddress,
 %frontmatterverbose,
 % preprint,
 showpacs,
 showkeys,
 preprintnumbers,
 %nofootinbib,
 %nobibnotes,
 %bibnotes,
 amsmath,amssymb,
 aps,
 prl,
 %  pra,
 % prb,
 % rmp,
 %prstab,
 %prstper,
  longbibliography,
 %floatfix,
 %lengthcheck,%
 ]{revtex4-1}

%\usepackage{cdmtcs-pdf}

\usepackage{amssymb,amsthm,amsmath}


\theoremstyle{definition}
\newtheorem{definition}{Definition}
\newtheorem{theorem}{Theorem}
\newtheorem{conjecture}{Conjecture}

\theoremstyle{remark}
\newtheorem*{motivation}{Motivation}
\newtheorem*{note}{Note}

\usepackage{tikz}
\usepackage[breaklinks=true,colorlinks=true,anchorcolor=blue,citecolor=blue,filecolor=blue,menucolor=blue,pagecolor=blue,urlcolor=blue,linkcolor=blue]{hyperref}
\usepackage{graphicx}% Include figure files
\usepackage{url}

\usepackage{xcolor}

\begin{document}


\title{Space and time in a quantized world}

%\cdmtcsauthor{Karl Svozil}
%\cdmtcsaffiliation{Vienna University of Technology}
%\cdmtcstrnumber{407}
%\cdmtcsdate{September 2011}
%\coverpage

\author{Karl Svozil}
\affiliation{Institute for Theoretical Physics, Vienna
    University of Technology, Wiedner Hauptstra\ss e 8-10/136, A-1040
    Vienna, Austria}
\email{svozil@tuwien.ac.at} \homepage[]{http://tph.tuwien.ac.at/~svozil}


\pacs{03.65.Ta, 03.65.Ud}
\keywords{construction of space-time, relativity theory, quantum mechanics, conventions}
%\preprint{CDMTCS preprint nr. 407/2011}

\begin{abstract}
Rather than consider space-time as an {\em a priori} arena in which events take place, it is a construction of our mind making possible a particular kind of ordering of events. As quantum entanglement is a property of states independent of classical distances, one has to revise the notion of space and time to represent the holistic interconnectedness of quanta. Various forms of reprogramming, or reconfiguring, propagation of information are discussed in quantum field theory.
\end{abstract}

\maketitle


\section{Intrinsic construction of space-time frames}

Einstein's centennial paper on space-time \cite{ein-05,naber}
(and to a certain extent Poincare's thoughts \cite{poincare02}) has introduced
{\em conventions and operational algorithmic procedures} that allow
the generation of space-time frames by relying on intrinsically feasible methods and techniques alone \cite{toffoli:79,svozil-94}.
This revolutionary step renders a space-time (in terms of frames and their transformation) which
is {\em means relative} \cite{Myrvold2011237} with respect to conventions (such as for {\em defining} simultaneity employing round-trip time, which is
nowadays even used by {\em Cristian's Algorithm} for computer networks)
and unanimously executable measurements and ``algorithmic'' physical procedures
that need not rely upon any kind of absolute metaphysical knowledge (such as ``absolute space or time'').

This leads to the idea of constructing operational, intrinsic space-time frames based on physical events,
rather then staging physical events in a metaphysical ``space-time arena.''
One step in this direction is, for instance, the determination of the {\em dimensionality} of space and of space-time from
empirical evidence \cite{sv2}.

It is evident that, in this line of thought, space and time emerge as concepts that are not independent of the physical phenomena
(as well as on assumptions or conventions) by which they are constructed.
Therefore, it is quite legitimate to ask whether space-time frames of classical physics can be
carried over to quantum space-time \cite{Kreinovich-94,Myrvold2002435}.

From this perspective, any attempt to unify the standard quantum field theories with gravity (as a ``geometrodynamic theory of distortet space-time'') must inevitably fail; just as the Baron M\"unchhausen cannot tear himself out of a swamp by his pigtail.


\section{Encoding information on single quanta}

So far, there is evidence that ``will- and useful''
information in the sense of non-random bit(stream)s  can be transferred from some space-time point $A$
to another space-time point $B$ only {\em via} individual quanta:
single quanta are emitted at some space-time point   $A$, and absorbed at another space-time point $B$.



\subsection{Time scales}

If indeed one takes seriously the idea that ``quantum events can be utilized to create space-time frames''
then we need to base space and time scales used in such frames on quantum mechanical entities alone.
One such approach might be to consider general distances and metrics on permutations, in particular,
on the symmetric groups
(by Cayley's representation theorem the unitary quantum evolution essentially is some subgroup of the symmetric group),
thereby relating changes in quantum states to time.

Indeed, the current definition of the second in the International System (SI) of units
is {\it via} 9 192 631 770 transitions beween two orthogonal quantum states of a caesium 133 atom.
That is, if we encode the two ground states by the subspaces spanned by the two orthogonal vectors
$\vert \psi_0 \rangle \equiv (0,1)$ and $\vert \psi_1 \rangle \equiv (1,0)$,
[or, equivalently, by the projectors $\text{diag}(0,1)$    and  $\text{diag}(1,0)$]
in two-dimensional Hilbert space,
then the 9 192 631 770'th fraction of a second is delivered by the unitary operator that is known as the
{\em not gate} \cite{mermin-07} $\textsf{\textbf{X}}
=   \begin{pmatrix}0& 1\\1&0
\end{pmatrix}
$,
representing a single permutation-transition $\textsf{\textbf{X}}\vert \psi_i \rangle $  between $\vert \psi_i \rangle  \leftrightarrow \vert \psi_{i\oplus 1} \rangle $,
$i\in \{ 0, 1\}$,
of states of a caesium 133 atom.

\subsection{Space scales}

The current definition of spatial distances in the International System of units
is in terms of the propagation of light quanta in vacuum.
More specifically, the metre is the length of the path travelled by light in vacuum during a time interval of 1/299 792 458 of a second
--
or, equivalently, during
9 192 631 770/299 792 458 $\approx 31$ transitions of states of a caesium 133 atom.

What, exactly, is a {\em ``spatial distance?''}
More precisely, what quantum meaning can be ascribed to a ``path travelled by light in vacuum?''
First and foremost, any spatial distance seems to depend on two criteria:
(i) separateness, or disconnectedness; as well as
(ii) the capacity to (inter-)connect.
The latter connection must, by quantum rules,
be mediated {\it via} {\em permutations.}

In the simplest sense, one could algorithmically model such a contact transmission by a {\em cellular automaton}
\cite{fredkin,svozil-1996-time,thooft-2013}.


\subsection{Alexandrov's theorem}

In order to make operational sense without regress to absolute space-time frames,
the SI definition of length implicitly assumes that the velocity of light in vacuum for all space-time frames is constant,
regardless of the state of motion of that frame
\cite{peres-84}.
By these assumptions and other conventions, such as Einstein's definition of simultaneity \cite{ein-05} and bijectivity of coordinate transformations,
the Lorentz transformations are essentially (up to shift-translations and
dilations with positive scalar constants) a consequence
of the Alexandrov-Zeemann theorem of incidence geometry \cite{alex3,zeeman,lester,naber}.
Pointedly stated, if two observers ``presiding over their reference frames agree'' \cite{naber} that points connected by
light rays (representing permutations) can be interconnected, then the Lorentz transformation follow.

{\it A priori,} the use of (the velocity of) light is purely conventional, as other velocities, both sub- as well as superluminal
-- even no-signalling ones such as from phased arrays (see below) --
would also be possible.
The convenience with the vacuum velocity of light lies in the {\em form invariance}
of the equations of motion, such as Maxwell's equation, in vacuum.



\section{Encoding information across quanta}

\subsection{Entanglement characteristics}

At the time of conceptualizing special relativity theory, quantum mechanics was in its infancy,
and quantum effects were therefore not considered for the definition of space-time scales.
Alas, this has changed since Schr\"odinger pointed out the possibility of entangled quantum states of multipartite quantized systems;
states that do not have any classical local counterpart.
Entanglement is characterized by an encoding of (classical) information ``across quanta'' \cite{zeil-99,svozil-2002-statepart-prl}
that defy any kind of spatial apartness or locality,
and that yield experimental violations \cite{wjswz-98} of classical probabilities \cite{pitowsky}.
These features alone suggest to consider quantum mechanical processes for the definition of space-time frames.

One of the characteristics of quantum entanglement is that information is not encoded in the single particles that constitute
an entangled system. Therefore, through context translation, any enquiry about the state of a single quantum
must inevitable fail, because no such information is available prior to this ``forced measurement.''
The archetypical example of this situation is the  Bell state
$\vert \Psi_- \rangle = \left( 1/\sqrt{2} \right) \left(\vert +-\rangle - \vert -+\rangle \right)$
which is totally and irreducible indeterminate about the states $\vert -\rangle$ or $\vert +\rangle$
of its individual two constituents,
but totally determined by the two propositions  {\em ``the spin states of the two particles along two orthogonal spatial directions
are different''}   \cite{Zeilinger-97,zeil-99,svozil-2002-statepart-prl}.

In this view,
for the Bell state as well as for other non-localized states,
in which the constituents can be thought of as ``torn apart'' arbitrary spatial distances,
there is no ``spooky action at a distance'' \cite{Nikolic} whatsoever,
because the multiple constituents, if they become separated and ``drift away'' from their joint space-time preparation regions,
do so at speeds not exceeding the velocity of light.

Thereby, any greater-than-classical correlations and expectations these constituents carry are due to the particular type
of quantum probabilities.
Recall that the quantum probabilities are generalizations of classical probabilities:
Due to Gleason's theorem the Born rule can be derived from
the noncontextual pasting of blocks of subalgebras (that is, maximal, co-measurable observables);
whereas all classical probability distributions result from convex sums of two-valued states on the Boolean algebra of classical propositions.

Pointedly stated, the so-called ``quantum non-locality'' is not non-local at all, because
these correlations reside in the (entangled) quantum states which must be perceived holistically (as being one compound state)
rather than as being constructed from separate single quantum states; regardless of the spatial separation
of the constituent quanta forming such states.
The measurements in spatially different regions (regardless of whether they are space-like separated or not)
just recover this property encoded in the quantum states; thereby nothing needs to be exchanged at all.
Claims that this expresses some kind of ``spooky action at a distance'' mistake correlation for causality.

There exist even  quasi-classical models (which are nonlocal as they require the exchange of one bit per particle pair)
capable of realizing stronger-than-quantum correlations \cite{svozil-2004-brainteaser}.
The terminology ``peaceful coexistence'' \cite{shimony-78}
between quantum theory and special relativity, implying some perceivable kind of inconsistency between them,
is thus misleading at this stage.

\subsection{Quantum statistics}

The situation becomes more sophisticated if the constituent quanta of entangled particles are subjected to multi-partite quantum statistics;
in particular to stimulated emission or absorprtion.
For the sake of an attack \cite{svozil-slash} on ``peaceful coexistence,'' the delayed choice of, say, either scattering a photon into a ``box of identical photons''
(or directing an electron into a region filled with other electrons occupying certain states attainable by the original electron), or
leaving it pass this region without any other identical quanta,
might be used to communicate a message across the particle pair.
It can be speculated that, as some agent has free will to ``induce''
some state of one photon of a photon pair in an entangled singlet state,
the other photon has no (random) choice any longer to scatter into the corresponding state.

Another possibility would be to transmit information across spatially extended quantum states of a large number of particles by affecting
the statistical constraints on one side and observing the effects on the other end.
For the sake of a concrete example consider a superconduvcting rod which is heated into the non-superconducting state (or otherwise ``destroying it'') on one end of the rod,
and observing the gap energy on the other end.

In such many-partite cases invoking quantum statistics, the doctrin of ``peaceful coexistence'' is less obvious.
We will turn our attention now to ``second quantization'' quantum effects on single (nonentangled) quanta.
They are due to the presence of (spontanuous or controlled) many-partite excitations of the quantized fields involved.


\section{Field theoretic models of signal propagation}

When considering the propagation of light and other potential signal carriers in vacuum
there appear to exist at least two alternative conceptions.
First, we could assume that light is ``attenuated'' by polarization
and other (e.g., quantum statistical) effects.
Without any such interactions such signals might travel arbitrarily fast.
Thus, in order to increase signalling speeds,
we must attempt to disentangle the signal carrier from interacting with the vacuum.

A second viewpoint may be that light needs a substratum for propagation;
very much like a phonon needs, or rather subsumes,
collective excitations of some carrier medium.
In such scenarios, stronger couplings might result in higher signalling speeds.

If any such speculation will eventually yield superluminal communication and space travel is highly unceretain,
but should not be outrightly excluded for the mere sake of orthodoxy.
In what follows we briefly mention some possible directions of looking into this issue.



\subsection{Multiple side hopping}

The capacity to transfer information can be modelled by the interconnection between different spatial regions.
One such microphysical model is the vibrating (linear) chain \cite[Sec.~1.2]{Henley-Thirring-EQFT}
which requires some coupled (linearized) oscillators.
The spatial substratum  \cite{einstein-aether,dirac-aether} is modelled by an interconnected array of coupled oscillators.
Thereby, (the energy of) an excitation is transferred from one oscillator to the next by the coupling between the two.

One possibility to change the resulting signal velocity would be to assume that any oscillator is coupled not only to its next neighbor,
but to other oscillators which are spatially farther apart. In this way, by increasing the ``hopping distance,''
say, in a periodic medium, as depicted in Fig.~\ref{2013-st1-msh},
faster modes of propagation (as compared to single side hopping) seem possible.

We suggest to employ {\em  phased array} (radar) with faster-than-light synchronization, such as the
one enumerated in Table~\ref{2013-tablest1-msh}, of electrical signals
for the exploration of multiple side hopping and the resulting higher order harmonics $2c, 3c,\ldots $ of the velocity of light $c$.
Thereby, the signals generated by the phased array of electrical charges
might resonate with the propagation modes of the substratum carring those collective excitations.
For random hopping distances, any such discretization cannot be expected.

\begin{figure*}
\begin{center}
% This is a LaTeX picture output by TeXCAD.
% File name: [1.pic].
% Version of TeXCAD: 4.3
% Reference / build: 30-Jun-2012 (rev. 105)
% For new versions, check: http://texcad.sf.net/
% Options on the following lines.
%\grade{\on}
%\emlines{\off}
%\epic{\off}
%\beziermacro{\on}
%\reduce{\on}
%\snapping{\off}
%\pvinsert{% Your \input, \def, etc. here}
%\quality{8.000}
%\graddiff{0.005}
%\snapasp{1}
%\zoom{6.7271}
\unitlength 0.5mm
\linethickness{0.4pt}
\ifx\plotpoint\undefined\newsavebox{\plotpoint}\fi % GNUPLOT compatibility
\begin{picture}(301,127.817)(0,0)
{\thicklines
\put(0,10){\color{gray}\line(1,0){50}}
\put(50,10){\color{gray}\line(1,0){50}}
\put(99.962,10){\color{gray}\line(1,0){50}}
\put(150,10){\color{gray}\line(1,0){50}}
\put(200,10){\color{gray}\line(1,0){50}}
\put(250,10){\color{gray}\line(1,0){50}}
}
\put(0,10){\circle*{4}}
\put(50,10){\circle*{4}}
\put(100,10){\circle*{4}}
\put(150,10){\circle*{4}}
\put(200,10){\circle*{4}}
\put(250,10){\circle*{4}}
\put(300,10){\circle*{4}}
\thinlines
{%\qbezvec[both](.42,15.136)(25.437,30.272)(49.613,15.136)
{\put(49.613,15.136){\color{red}\vector(3,-2){.07}}\put(.42,15.136){\color{red}\vector(-3,-2){.07}}\color{red}\qbezier(.42,15.136)(25.437,30.272)(49.613,15.136)}
%\end
%\qbezvec[both](50.454,15.136)(75.471,30.272)(99.647,15.136)
\put(99.647,15.136){\color{red}\vector(3,-2){.07}}\put(50.454,15.136){\color{red}\vector(-3,-2){.07}}\color{red}\qbezier(50.454,15.136)(75.471,30.272)(99.647,15.136)
%\end
%\qbezvec[both](100.488,15.136)(125.505,30.272)(149.68,15.136)
\put(149.68,15.136){\color{red}\vector(3,-2){.07}}\put(100.488,15.136){\color{red}\vector(-3,-2){.07}}\color{red}\qbezier(100.488,15.136)(125.505,30.272)(149.68,15.136)
%\end
%\qbezvec[both](150.521,15.136)(175.538,30.272)(199.714,15.136)
\put(199.714,15.136){\color{red}\vector(3,-2){.07}}\put(150.521,15.136){\color{red}\vector(-3,-2){.07}}\color{red}\qbezier(150.521,15.136)(175.538,30.272)(199.714,15.136)
%\end
%\qbezvec[both](200.555,15.136)(225.572,30.272)(249.748,15.136)
\put(249.748,15.136){\color{red}\vector(3,-2){.07}}\put(200.555,15.136){\color{red}\vector(-3,-2){.07}}\color{red}\qbezier(200.555,15.136)(225.572,30.272)(249.748,15.136)
%\end
%\qbezvec[both](250.589,15.136)(275.605,30.272)(299.781,15.136)
\put(299.781,15.136){\color{red}\vector(3,-2){.07}}\put(250.589,15.136){\color{red}\vector(-3,-2){.07}}\color{red}\qbezier(250.589,15.136)(275.605,30.272)(299.781,15.136)}
%\end
{%\qbezvec[both](0,15.977)(50.875,49.613)(100.067,15.977)
\put(100.067,15.977){\color{blue}\vector(3,-2){.07}}\put(0,15.977){\color{blue}\vector(-3,-2){.07}}\color{blue}\qbezier(0.42,15.977)(50.875,49.613)(100.067,15.977)
%\end
%\qbezvec[both](100.067,15.977)(150.942,49.613)(200.135,15.977)
\put(200.135,15.977){\color{blue}\vector(3,-2){.07}}\put(100.067,15.977){\color{blue}\vector(-3,-2){.07}}\color{blue}\qbezier(100.067,15.977)(150.942,49.613)(200.135,15.977)
%\end
%\qbezvec[both](200.135,15.977)(251.009,49.613)(300.202,15.977)
\put(300.202,15.977){\color{blue}\vector(3,-2){.07}}\put(200.135,15.977){\color{blue}\vector(-3,-2){.07}}\color{blue}\qbezier(200.135,15.977)(251.009,49.613)(300.202,15.977)}
%\end
{
%\qbezvec[both](.42,17.238)(74.42,74.42)(150.101,17.238)
\put(150.101,17.238){\color{green}\vector(4,-3){.07}}\put(.42,17.238){\color{green}\vector(-4,-3){.07}}\color{green}\qbezier(.42,17.238)(74.42,74.42)(150.101,17.238)
%\end
%\qbezvec[both](150.521,17.659)(224.521,74.84)(300.202,17.659)
\put(300.202,17.659){\color{green}\vector(4,-3){.07}}\put(150.521,17.659){\color{green}\vector(-4,-3){.07}}\color{green}\qbezier(150.521,17.659)(224.521,74.84)(300.202,17.659)}
%\end
{
%\qbezvec[both](.42,19.341)(99.857,91.658)(200.135,19.341)
\put(200.135,19.341){\color{olive}\vector(4,-3){.07}}\put(.42,19.341){\color{olive}\vector(-4,-3){.07}}\color{olive}\qbezier(.42,19.341)(99.857,91.658)(200.135,19.341)}
%\end
{
%\qbezvec[both](.42,20.602)(137.908,117.306)(250.168,20.602)
\put(250.168,20.602){\color{orange}\vector(4,-3){.07}}\put(.42,20.602){\color{orange}\vector(-3,-2){.07}}\color{orange}\qbezier(.42,20.602)(137.908,117.306)(250.168,20.602)}
%\end
{
%\qbezvec[both](.42,22.704)(152.203,127.817)(298.94,22.704)
\put(298.94,22.704){\color{violet}\vector(4,-3){.07}}\put(.42,22.704){\color{violet}\vector(-3,-2){.07}}\color{violet}\qbezier(.42,22.704)(152.203,127.817)(298.94,22.704)}
%\end
\put(0,0){\makebox(0,0)[cc]{$1$}}
\put(50,0){\makebox(0,0)[cc]{$2$}}
\put(100,0){\makebox(0,0)[cc]{$3$}}
\put(150,0){\makebox(0,0)[cc]{$4$}}
\put(200,0){\makebox(0,0)[cc]{$5$}}
\put(250,0){\makebox(0,0)[cc]{$6$}}
\put(300,0){\makebox(0,0)[cc]{$7$}}
\put(25,16){\makebox(0,0)[cc]{\color{red}$a=2-1$}}
\put(50,26){\makebox(0,0)[cc]{\color{blue}$2a$}}
\put(75,40){\makebox(0,0)[cc]{\color{green}$3a$}}
\put(100,50.5){\makebox(0,0)[cc]{\color{olive}$4a$}}
\put(135,63){\makebox(0,0)[cc]{\color{orange}$5a$}}
\put(150,81){\makebox(0,0)[cc]{\color{violet}$6a$}}
\end{picture}
\end{center}
\caption{(Color online) Multiple side hopping might give rise to higher harmonics of the speed of light.}
\label{2013-st1-msh}
\end{figure*}

\begin{table}
\begin{center}
\begin{tabular}{c|ccccccccccc}
\hline\hline
&1&2&3&4&5&6&7&$\cdots$\\
\hline
\color{red}
a=1
&\color{red}1&0&0&0&0&0&0&$\cdots$\\
&0&\color{red}1&0&0&0&0&0&$\cdots$\\
&0&0&\color{red}1&0&0&0&0&$\cdots$\\
&0&0&0&\color{red}1&0&0&0&$\cdots$\\
&0&0&0&0&\color{red}1&0&0&$\cdots$\\
&0&0&0&0&0&\color{red}1&0&$\cdots$\\
&0&0&0&0&0&0&\color{red}1&$\cdots$\\
&\multicolumn{8}{l}{$\cdots$}\\
\hline
\color{blue}
a=2
&\color{blue}1&0&0&0&0&0&0&$\cdots$\\
&0&0&\color{blue}1&0&0&0&0&$\cdots$\\
&0&0&0&0&\color{blue}1&0&0&$\cdots$\\
&0&0&0&0&0&0&\color{blue}1&$\cdots$\\
&\multicolumn{8}{l}{$\cdots$}\\
\hline
\color{green}
a=3
&\color{green}1&0&0&0&0&0&0&$\cdots$\\
&0&0&0&\color{green}1&0&0&0&$\cdots$\\
&0&0&0&0&0&0&\color{green}1&$\cdots$\\
&\multicolumn{8}{l}{$\cdots$}\\
\hline
\color{olive}
a=4
&\color{olive}1&0&0&0&0&0&0&$\cdots$\\
&0&0&0&0&\color{olive}1&0&0&$\cdots$\\
&\multicolumn{8}{l}{$\cdots$}\\
\hline
\color{orange}
a=5
&\color{orange}1&0&0&0&0&0&0&$\cdots$\\
&0&0&0&0&0&\color{orange}1&0&$\cdots$\\
&\multicolumn{8}{l}{$\cdots$}\\
\hline
\color{violet}
a=6
&\color{violet}1&0&0&0&0&0&0&$\cdots$\\
&0&0&0&0&0&0&\color{violet}1&$\cdots$\\
&\multicolumn{8}{l}{$\cdots$}\\
\hline\hline
\end{tabular}
\end{center}
\caption{(Color online) Array synchronization for multiple side hopping.}
\label{2013-tablest1-msh}
\end{table}

\subsection{Change of vacuum}

Another possibility to change the propagation velocity of the substratum would be to alter its ability
to carry a signal through attenuation and amplification of the processes responsible for sinalling.
The most direct form would be to change the coupling between oscillators in the vibrating chain scheme mentioned earlier.

Another possibility would be to again use quantum statistical effects to reduce or increase the polarizability of the vacuum
by placing bosons or fermions along the signalling path.
A photon, for instance, seems to become accelerated if polarizability is reduced \cite{Scharnhorst-1998,svozil-putz-sol}.

\section{Dimensionality}

One could speculate that the apparent three-dimensionality of physical configuration space is
a reflection of the {\em three-dimensional interconnectedness} of the substratum of this universe
on a very fundamental level.
In this way, information is ``permuted by point contact from one node to the other.''
A discrete version of this would be a three-dimensional cellular automaton.

In another scenario the intrinsic, operational three-dimensionality
is a (fractal) ``shadow'' on a higher dimensional substratum \cite{sv4}.
In this view, if there is no ``bending (yielding nontrivial topologies), folding or compactification'' of the extra dimensions involved,
information transfer might become even ``slower'' than in the lower dimensional case, since every extra dimension is nothing but
an extra degree of freedom the bit can pursue, thereby even ``getting lost'' if, say, it travels a direction orthogonal to,
or in other ways inaccessible for, three-space.
On the other hand, if this fractal shadow constituting our accessible configuration space
can be bent or even intersected by itself,
then information transfer, and thus signalling and space travel, from any point $A$ to any other point $B$
could in principle be obtained with arbitrary velocity.




\section{Other considerations}

One also needs to bear in mind that, beyond electromagnetic and gravitational interactions,
other ``fundamental'' (strong, weak) interactions have been discovered, which, according to the standard model,
propagate at the same speed as light, although no direct empirical evidence is available.
In any case, {\it a priori}, different interactions need not always propagate with the same velocity,
making necessary a sort of ``relativized relativity'' \cite{svozil-relrel} that has to cope with
consistency issues, such as the ``grandfather paradox.''
The latter one is also resolved in ``quantum time travelling'' scenarios \cite{svozil-greenberger-2005}.




\begin{acknowledgments}
This research has been partly supported by FP7-PEOPLE-2010-IRSES-269151-RANPHYS.
\end{acknowledgments}

 \bibliography{svozil}


\end{document}
