

\documentclass[a4paper,11pt]{article}


\newcommand{\rka}  {\left(}
\newcommand{\rkz}  {\right)}
\newcommand{\eka}  {\left[}
\newcommand{\ekz}  {\right]}
\newcommand{\gka}  {\left\{}
\newcommand{\gkz}  {\right\}}

\newcommand{\w}  {\vec{\omega}}
\newcommand{\we} {\omega_E}
\newcommand{\vb} {\vec{v}}

\newcommand{\al} {\!\alpha}

\newcommand{\er} {\,\vec{e}_r}
\newcommand{\eph}{\,\vec{e}_\varphi}
\newcommand{\eth}{\,\vec{e}_\vartheta}

\setlength{\parindent}{0pt}
%\renewcommand{\baselinestretch}{1.2}



\begin{document}

\title{Coriolis in Pisa}
\author{{\small Jedinger Martina}}
\date{} 
\maketitle

How big is the influence on a stone falling from the Leaning Tower of Pisa,
caused by the Coriolis force?\\
\\
Answering this question is the aim of the following calculations.\\
To achieve this some values about Pisa and earth
\begin{quote}
\begin{tabular}{l@{\hspace{0,5cm}}l}
 latitude of  Pisa:                    &  $ \al = 43^\circ43' = 0,7629 $ rad\\
 height of the Leaning Tower of Pisa:  &  $ h = 55 $ m\\
 earth's radius:                       &  $ R = 6,37 \cdot 10^6 $ m\\
 angular frequency:                    &  $ \we = 7,29 \cdot 10^{-5} $ rad/s\\
 acceleration due to earth gravity:    &  $ g= 9,832 $ m/s$^2$\\
\end{tabular}
\end{quote}

and that formula

\begin{equation}
 \vec{a}  = \vec{a}\,^\prime + 2\w \times \vb\,^\prime
             + \w\times \rka  \w\times\vec{r}  \rkz 
 \footnotemark[1]
\end{equation}
\footnotetext[1]{{\bfseries H. J. Paus},
                 {\itshape Physik in Experimenten und Beispielen}.\\
                  Hanser, M\"unchen, Wien, 1995.  p. 119}

(which can also be written in the following form)

\begin{equation}\label{accel}
 \vec{a}\,^\prime = \vec{a}
                    \underbrace{-2 \rka  \w\times\vb\,^\prime  \rkz}
                               _{\vec{a}_{\scriptscriptstyle C}}
                    \underbrace{ -\w\times \rka  \w\times\vec{r}
                                 \rkz}_{\vec{a}_{\scriptscriptstyle ZF}}
\end{equation}

will be needed.\\

In the preceeding equation $\vec{a}_{\scriptscriptstyle ZF}$ is  the term
caused by the centrifugal force and $\vec{a}_{\scriptscriptstyle C}$ is the acceleration caused by
the Coriolis force, the term of interest.\\
Quantities with primes are measured in the rotating system (earth),
quantities without in an inertial system.\\
(All the following calculations are done without considering the
changing of coordinates!)\\

The acceleration in the inertial system is caused by the gravitational pull,
which means
\begin{equation}\label{a}
 \vec{a} = -g\er
\end{equation}\\

By considering the picture below\\

\begin{figure}[h]

\setlength{\unitlength}{1mm}

\begin{center}
\begin{picture}(60,60)

\qbezier(30,55)(27,55)(29,57)
\qbezier(30,55)(33,55)(31,57)
\put(31,56.8){\vector(-3,2){0.7}}

%Kreis:
\qbezier(30,50)(38.2843,50)(44.1421,44.1421)
\qbezier(44.1421,44.1421)(50,38.2843)(50,30)
\qbezier(50,30)(50,21.7157)(44.1421,15.8579)
\qbezier(44.1421,15.8579)(38.2843,10)(30,10)
\qbezier(30,10)(21.7157,10)(15.8579,15.8579)
\qbezier(15.8579,15.8579)(10,21.7157)(10,30)
\qbezier(10,30)(10,38.2843)(15.8579,44,1421)
\qbezier(15.8579,44,1421)(21.7157,50)(30,50)

%Drehachse:
\put(30,0){\line(0,1){60}}

%�quator:
\put(10,30){\line(1,0){40}}

%Radius:
\put(30,30){\line(1,1){14.2}}
%Beschriftung:
\put(34,38){$R$}

%Einheitsvektor e_r:
\put(44.1421,44.1421){\vector(1,1){8}}
\put(53,51){$\vec e_r$}

%Einheitsvektor e_th:
\put(44.1421,44.1421){\vector(1,-1){8}}
\put(53,35){$\vec e_\vartheta$}

%Vektor w:
\put(44.1421,44.1421){\vector(0,1){17}}
\put(46,58){$\vec\omega$}

%Breitengrad:
\qbezier(37.0711,37.0711)(40,34.1422)(40,30)
\put(35,32){$\alpha$}

\end{picture}
\end{center}
\end{figure}

one can easily conclude that
\begin{equation}\label{omega}
 \w = \we\sin\al \er - \we\cos\al \eth \; \rka  +\,0\eph  \rkz 
\end{equation}
\\
With the help of (\ref{omega}) it is now possible to calculate
$\vec{a}_{\scriptscriptstyle ZF}$:
\begin{eqnarray}
 \w\times\vec{r}             & = & \rka  \we\sin\al\er - \we\cos\al\eth  \rkz
                                   \times R\er =\nonumber\\
		             & = & \we R\cos\al\eph  \\
 \w\times
 \rka  \w\times\vec{r}  \rkz & = & \rka  \we\sin\al\er + \we\cos\al\eth  \rkz 
                                    \times \we R\cos\al\eph =\nonumber\\
	                     & = &  - \we^2R\cos^2\al\er
			            - \we^2R\sin\al\cos\al \eth
\end{eqnarray}
it follows
\begin{equation}\label{azf}
 \vec{a}_{\scriptscriptstyle ZF}\quad = \quad \we^2R\cos^2\al\er
                                   + \we^2R\sin\al\cos\al\eth
\end{equation}
\\
Combining now the equations (\ref{accel}) (\ref{a}) and (\ref{azf}) leads to
a differential equation
\begin{equation}\label{dgl}
 \vec{a}\,^\prime  =  \dot{\vb}\,^\prime 
                   = \rka  -g + \we^2R\cos^2\al  \rkz \er
                    + \we^2R\sin\al\cos\al \eth
                     - 2 \rka  \w\times\vb\,^\prime  \rkz 
\end{equation}
\\
\\
A possible ansatz for the solution of (\ref{dgl}) is to calculate it
iteratively.\\
Zero order pertubation theory: the velocity only due to earth
gravity and centrifugal force
\begin{eqnarray}
 \vb\,^\prime   & = &  \rka  -g + \we^2R\cos^2\al  \rkz t\er
                       + \we^2R\sin\al\cos\al t\eth
                       + \vb_0\!^\prime\\
 \vb_0\!^\prime & = &  0
\end{eqnarray}
First order pertubation theory
\begin{eqnarray}
 \dot{\vb}\,^\prime & = &  \rka  -g + \we^2R\cos^2\al  \rkz\er
                           + \we^2R\sin\al\cos\al \eth - \nonumber\\
                    &   &  {} - 2 \gka   \w\times  \eka  
                              \rka  -g + \we^2R\cos^2\al  \rkz
                              t\er  + \we^2R\sin\al\cos\al t\eth  \ekz  \gkz\nonumber\\
                    & = &  \rka  -g + \we^2R\cos^2\al  \rkz \er
		           + \we^2R\sin\al\cos\al \eth -\nonumber\\
                    &   &  {} - 2\eka  \rka  -g\we\cos\al
                                            + \we^3R\cos^3\al  \rkz t\eph
                                 + \we^3\sin^2\al\cos\al Rt \eph  \ekz\\
 \vb\,^\prime       & = &  \rka  -g + \we^2R\cos^2\al  \rkz t\er
                          + \we^2R\sin\al\cos\al t \eth - \nonumber\\
                    &   &  {} - \rka  -g\we\cos\al + \we^3R\cos^3\al
                                 + \we^3R\sin^2\al\cos\al  \rkz t^2 \eph
                            + \vb_0\!^\prime\\
 \vb_0\!^\prime     & = &  0
\end{eqnarray}
\\
integrating $ \dot{\vec{s}}\,^\prime= {\vb}\,^\prime$ leads to
\begin{eqnarray}
 \vec{s}\,^\prime   & = &  \rka  - g + \we^2R\cos^2\al  \rkz \frac{t^2}{2} \er
                           + \we^2R\sin\al\cos\al \frac{t^2}{2} \eth \nonumber\\
	            &   &  {} -\rka  -g\we\cos\al + \we^3R\cos^3\al
  		           +\we^3R\sin^2\cos\al  \rkz \frac{t^3}{3}\eph
                           + \vec{s}_0\!^\prime\\
 \vec{s}_0\!^\prime & = &  h\er \; \rka  +\,0\eph + 0\eth  \rkz 
\end{eqnarray}

it follows
\begin{eqnarray}
 \vec{s}\,^\prime & = &  \eka\rka  -g + \we^2R\cos^2\al  \rkz \frac{t^2}{2}
                             + h  \ekz \er
                         +\we^2R\sin\al\cos\al \frac{t^2}{2} \eth - \nonumber\\
                  &   &  \quad{} - \rka  -g\we\cos\al + \we^3R\cos^3\al
                             + \we^3R\sin^2\al\cos\al  \rkz \frac{t^3}{3} \eph
\end{eqnarray}
\\
The next step is to calculate the time until the stone hits the surface
of the earth. To do that one must solve this equation 
\begin{eqnarray}
 s_{\scriptscriptstyle R}\!\!\!^\prime & = & \vec{s}\,^\prime\cdot\er =\nonumber\\
                & = & \rka  -g + \we^2R\cos^2\al  \rkz \frac{t_i{}^2}{2} + h =
                0
\end{eqnarray}

By doing so, for the time of impact ($t_i$) follows
\begin{equation}
 t_i = \sqrt{\frac{2h}{g-\we^2R\cos^2\al}}
\end{equation}

and in numerical values
\begin{equation}
 t_i = 3,35 \mbox{ s}
\end{equation}
\\
Now all data needed to calculate $s_{\scriptscriptstyle C}\!\!\!^\prime$\, (and $s_{\scriptscriptstyle
  ZF}\!\!\!\!\!\!^\prime $ \,\,) is provided
\begin{eqnarray}
 s_{\scriptscriptstyle C}\!\!\!^\prime & = &  \vec{s}\,^\prime\cdot\eph =\nonumber\\
                                       & = &  \rka  g\we\cos\al - \we^3R\cos^3\al
                                            - \we^3R\sin^2\al\cos\al  \rkz 
                                               \frac{t_i{}^3}{3} \eph\nonumber\\
 s_{\scriptscriptstyle C}\!\!\!^\prime & = &  6,46\mbox{ mm}\\
 (s_{\scriptscriptstyle ZF}\!\!\!\!\!\!^\prime & = & 94,7\mbox{ mm})\nonumber
\end{eqnarray}


So the answer to the question from the beginning is, that a stone is
$6,46\mbox{ mm}$ east of the point he would be without the Coriolis force (and
$94,6\mbox{ mm}$ south of the point he would be without the centrifugal
force).  \\ \\ This result is only an approximation. So we should look on the
exact result of equation (\ref{dgl}) and on the integral of this result, which
can be done by a computer program.\\
{\slshape mathematica 5.0} gives\\
\begin{eqnarray}
\vb\,^\prime &=&\quad\; \eka
-gt\sin^2\al+\frac{\rka-g+\we^2R\rkz\cos^2\al\sin(2\we t)}{2\we}\ekz \er +\nonumber\\
 & & {}+\eka  \frac
            {\cos\al\sin\al\rka2g\we t+\rka-g+\we^2R\rkz\sin(2\we t)\rkz} 
            {2\we}\ekz \eth +\nonumber\\
 & & {}+ \eka  \frac{\rka g-\we^2R\rkz\cos\al\sin^2\rka \we t\rkz}{\we}\ekz
	    \eph
\end{eqnarray}
\begin{eqnarray}
 s_{\scriptscriptstyle r}\!\!^\prime & = & h+\frac{1}{4}R\cos^2\al -
 \frac{g\cos^2\al}{4\we^2} -\nonumber\\
     & & {} -\frac{1}{4}R\cos^2\al\cos\rka2\we t\rkz +
 \frac{g\cos^2\al\cos\rka2\we t\rkz}{4\we^2} - \frac{1}{2}gt^2\sin^2\al \\
 s_{\scriptscriptstyle \vartheta}\!\!^\prime & = &
 \frac{1}{4}R\cos\al\sin\al +\frac{1}{2}gt^2\cos\al\sin\al -
 \frac{g\cos\al\sin\al}{4\we^2} -\nonumber\\
 & &{}-\frac{1}{4}R\cos\al\cos\rka2\we
 t\rkz\sin\al + \frac{g\cos\al\cos\rka2\we t\rkz\sin\al}{4\we^2}\\
 s_{\scriptscriptstyle \varphi}\!\!^\prime & = & \frac{\rka-g +
   \we^2R\rkz\cos\al\rka -2\we t+ \sin\rka2\we t\rkz\rkz}{4\we^2}
\end{eqnarray}
\\
which leads to the numeric values:
\begin{eqnarray}
 s_{\scriptscriptstyle \vartheta}\!\!^\prime\rka t_i\rkz& =& 94,8\mbox{ mm}\\
 s_{\scriptscriptstyle \varphi}\!\!^\prime \rka t_i\rkz& =& 6,46\mbox{ mm}
\end{eqnarray}
\\
This proves that $s_{\scriptscriptstyle ZF}\!\!\!^\prime = 
\vec{s}\,^\prime\cdot\eth$ was a good approximination, and that first order
perturbation theory was sufficient.
\\
\\
\\
{\bfseries Considering friction:}\\
If one wants to consider the friction between the stone and the air one has to
add the following term to (\ref{dgl})
\begin{equation}
 {}-\frac{1}{2}c_w\rho \frac{A}{m}v^{\prime2}\,\vec{e}_v
\end{equation}
\\
The used symbols mean:
\begin{quote}
\begin{tabular}{l@{\hspace{0,5cm}}l}
 friction constant for spherical shape &  $ c_w = 0,4$\\
 cross section of the stone            &  $ A = (2,5$ cm$)^2\pi$\\
 mass of the stone                     &  $ m = 327$ g\\
 air density                           &  $ \rho = 1,3$ kg/m$^3$\\
\end{tabular}
\end{quote}

Numerical calculation with {\slshape mathematica 5.0} gives the following
results:
\begin{eqnarray}
 t_i & = & 3,40 \mbox{ s}\\
 s_{\scriptscriptstyle \vartheta}\!\!^\prime\rka t_i\rkz& =& 94,8\mbox{ mm}\\
 s_{\scriptscriptstyle \varphi}\!\!^\prime \rka t_i\rkz& =& 6,46\mbox{ mm}
\end{eqnarray}
\\
One sees that friction slows down the process, but doesn't alter the shape of
the movement in the first few seconds.\\
\\

For more details see the {\slshape mathematica 5.0} session in the appendix.
\end{document}