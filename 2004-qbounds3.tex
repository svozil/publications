\documentclass[prl,showpacs,showkeys,amsfonts,amsmath,twocolumn]{revtex4}
%\documentclass[pra,showpacs,showkeys,amsfonts]{revtex4}
\usepackage{graphicx}
\RequirePackage{times}
\RequirePackage{courier}
\RequirePackage{mathptm}

\newcommand{\ket}[1]{|#1\rangle}
\newcommand{\bra}[1]{\langle #1|}
\newcommand{\Tr}{\text{Tr}\,}
\newcommand{\vect}[1]{\boldsymbol{#1}}


\begin{document}



\title{The min-max principle generalizes Cirel'son's bound}
%\title{Tracing the quantum bounds of Bell-type inequalities}
\author{Stefan Filipp}
\email{sfilipp@ati.ac.at}
\affiliation{Atominstitut der {\"{O}}sterreichischen Universit{\"{a}}ten,
Stadionallee 2, A-1020 Vienna, Austria}
\author{Karl Svozil}
\email{svozil@tuwien.ac.at}
\homepage{http://tph.tuwien.ac.at/~svozil}
\affiliation{Institut f\"ur Theoretische Physik, University of Technology Vienna,
Wiedner Hauptstra\ss e 8-10/136, A-1040 Vienna, Austria}


\begin{abstract}
Bounds on the norm of quantum operators associated with classical Bell-type inequalities
can be derived from their maximal eigenvalues.
This quantative method enables detailed
predictions of the maximal violations of Bell-type inequalities.
\end{abstract}

\pacs{03.67.-a,03.65.Ta}
\keywords{tests of quantum mechanics, correlation polytopes, probability theory}

\maketitle

The violations of Bell-type inequalities represent
a cornerstone of our present understanding of quantum probability theory
\cite{peres}.
Thereby, the usual procedure is as follows:
(i)
First, the (in)equalities bounding the classical probabilities and expectations are
derived systematically; e.g., by enumerating all conceivable classical possibilities
and their associated two-valued measures.
These form the extreme points which span
the classical correlation polytopes
\cite{cirelson:80,cirelson,froissart-81,pitowsky-86,pitowsky,pitowsky-89a,Pit-91,Pit-94,2000-poly,collins-gisin-2003,sliwa-2003};
the faces of which are expressed by Bell-type inequalities
which characterize the bounds of the classical probabilities and expectations;
in Boole's term \cite{Boole,Boole-62}, the ``conditions of possible experience.''
(Generating functions are another method to find bounds on classical expectations \cite{werner-wolf-2001,schachner-2003}.)
The Bell-type inequalities contain sums of (joint) probabilities and expectations.
(ii)
In a second step, the classical probabilities and expectations in
the Bell-type inequalities are substituted by quantum probabilities and expectations.
The resulting operators violate the classical bounds.
Until recently, little was known about the fine structure of the violations.
Cirel'son published an absolute bound for the violation for a particular Bell-type inequality,
the Clauser-Horne-Shimony-Holt (CHSH) inequality \cite{cirelson:80,cirelson:87,cirelson,khalfin-97}.
Cabello has published a violation of the CHSH inequality beyond the quantum mechanical
bound by applying selection schemes to particles in a GHZ-state
\cite{cabello-02a,cabello-02b}.
Recently, detailed numerical  \cite{filipp-svo-04-qpoly}
and analytical studies \cite{cabello-2003a} stimulated
experiments \cite{bovino-2003} to test the quantum bounds of certain Bell-type inequalities.

In what follows,
a general method to compute quantum bounds on Bell-type inequalities
will be reviewed systematically.
It makes use of the {\em min-max principle} for self-adjoint transformations
(Ref.~\cite{halmos-vs}, Sec.~90) and (Ref.~\cite{reed-sim4}, Sec.~75)
stating that the operator norm is bounded by the minimal and maximal eigenvalues.
These ideas are not entirely new and have been mentioned previously
\cite{werner-wolf-2001,filipp-svo-04-qpoly,cabello-2003a},
yet to our knowledge no systematic investigation has been undertaken yet.
It should also be kept in mind that this method {\it a priori}
cannot produce quantum polytopes \cite{pit:range-2001,filipp-svo-04-qpoly},
only the quantum correspondents of classical polytopes.
Indeed, as will be demonstrated explicitly, the resulting geometric forms will not be convex.
This, however,
does not diminish the relevance of these quantum predictions
to experiments testing the quantum violations
of classical Bell-type inequalities.

As a starting point note that,
since $(A+B)^\dagger =A^\dagger +B^\dagger = (A+B)$ for arbitrary self-adjoint transformations $A,B$,
the sum of self-adjoint transformations is again self-adjoint.
That is, all self-adjoint transformations entering the quantum correspondent of any Bell-type inequality
is again a self-adjoint transformation.
Note that the sum does not preserve eigenvectors and eigenvalues;
i.e., $A+B$ can have different eigenvectors and eigenvalues than $A$ and $B$ taken separately.
This property is to be expected, since $A$ and $B$ need not necessarily commute;
i.e., $[A,B]\neq 0$.
The norm of the self-adjoint transformation resulting from summing the quantum counterparts
of all the classical terms contributing to a particular Bell inequality obeys the min-max principle.
Thus determining the maximal violations of classical Bell inequalities amounts to
solving an eigenvalue problem.
The associated eigenstates are the multi-partite states which yield a maximum violation
of the classical bounds under the given experimental (parameter) setup.


Let us demonstrate the method with a few examples.
The simplest nontrivial case is two particles measured along a {\em single}
(but not necessarily identical) direction on either side.
The vertices are $(p_1,p_2,p_{12}=p_1p_2)$ for $p_1, p_2 \in \{0,1\}$ and thus
$(0,  0,  0)$,
$(0,  1,  0)$,
$(1,  0,  0)$,
$(1,  1,  1)$;
the corresponding face (Bell-type) inequalities of the polytope spanned by the four vertices
are given by
$p_{12} \le p_2$,
$0\le p_{12}\le 1$, and
\begin{equation}
p_1+p_2-p_{2}\le 1.
\label{2004-qbounds-e1}
\end{equation}
The classical probabilities have to be substituted by the quantum ones;
i.e.,
\begin{equation}
\begin{array}{lll}
p_1 &\rightarrow& q_1 (\theta ) =
{\frac{1}{2}}\left[{\mathbb I}_2 + {\bf \sigma}( \theta )\right] \otimes  {\mathbb I}_2,
\\
p_2 &\rightarrow& q_2 (\theta ) =
{\mathbb I}_2 \otimes {\frac{1}{2}}\left[{\mathbb I}_2 + {\bf \sigma}( \theta )\right],
\\
p_{12}&\rightarrow& q_{12} (\theta ,\theta ') =
{\frac{1}{2}}\left[{\mathbb I}_2 + {\bf \sigma}( \theta )\right]
\otimes
{\frac{1}{2}}\left[{\mathbb I}_2 + {\bf \sigma}( \theta ')\right],
\end{array}
\label{2004-qbounds-e2}
\end{equation}
with
$
{\bf \sigma}( \theta )=
\begin{pmatrix} \cos \theta & \sin \theta  \\
  \sin\theta & -\cos \theta
  \end{pmatrix}
$,
where $\theta $ is the relative measurement angle in the $x$--$z$-plane, and the two particles propagate along the $y$-axis.
Then, the self-adjoint transformation corresponding to the classical Bell-type inequality
(\ref{2004-qbounds-e1}) can be defined by
\begin{equation}
O_{11}(0,\theta) = q_1(0)+q_2(\theta)-q_{12} (0, \theta )
=
\begin{pmatrix}
   1 & 0 & 0 & 0 \cr 0 & 1 & 0 & 0 \cr 0 & 0 & {\cos^2
     \frac{\theta }{2}} & \frac{\sin \theta }
   {2} \cr 0 & 0 & \frac{\sin \theta }{2} & {\sin^2  \frac{\theta
     }{2}} \cr
 \end{pmatrix}.
\end{equation}

The eigenvalues of $O_{11}$ are $0$ and $1$, irrespective of $\theta$.
The min-max principle thus predicts a maximal bound of $O_{11}$
which does not exceed the classical bound 1.
In what follows, we shall enumerate analytical quantum bounds
for the more interesting cases comprising two and more $(m)$ distinct measurement directions on either
side yielding the quantum equivalents of the
Clauser-Horne (CH) inequality, as well as of more general inequalities
 for $m > 2$ \cite{2000-poly,collins-gisin-2003,sliwa-2003}.

For $m=2$, a complete
set of classical inequalities restricting possible probability values includes terms
like in Eq. (\ref{2004-qbounds-e1}),
and additionally the CH-inequality
$-1 \leq p_{13} + p_{14} + p_{23} - p_{24}- p_{1} -p_{3} \leq 0$, as well as permutations thereof.
Substituting
the classical probabilities by quantum probabilities according to the
rules in Eq. (\ref{2004-qbounds-e2}) provides the quantum
transformation
\begin{eqnarray}
O_{22}(0,2\theta,\theta,3\theta)&=&  q_{13}(0,\theta) +
q_{14}(0,3\theta) + q_{23}(2\theta,\theta) \nonumber\\
&&  - q_{24}(2\theta,3\theta)- q_{1}(0) - q_{3}(\theta)
\label{2004-qbounds-e4}
\end{eqnarray}
which again can be parameterized using the relative measurement angle $\theta$ lying in
the $x$--$z$-plane. The argument list of
$O_{22}(0,2\theta,\theta,3\theta)$ denotes  a measurement
along the angle $0$ and $2\theta$ on one particle, and $\theta$ and
$3\theta$ on the other particle, respectively.
The eigenvalues of the self-adjoint transformation in
(\ref{2004-qbounds-e4})
are $\lambda_{1,2} = \frac{1}{2}\big(\pm \cos 2\theta -
1\big)$ and $\lambda_{3,4} = \frac{1}{2}\left\{\pm\left[ \left(3-\cos 4\theta \right)/2\right]^{1/2} -1\right\}$,
yielding the maximum bound $\|
O_{22} \|= \max_{i=1,2,3,4} \lambda_i =\lambda_3 $ \cite{cabello-2003a,filipp-svo-04-qpoly}.


Generalizations for $m$ measurements per particle are
straightforward;
%\footnote{The dimension of the
%matrices representing $O_{mm}$ does not change, as only when
%investigating correlations of more than two particles the
%dimensionality of the eigenvalue-problem increases.}.
for example, the extension to \emph{three} measurement operators for each particle
yields only one additional nonequivalent (with respect to permutations)
inequality \cite{2000-poly,collins-gisin-2003,sliwa-2003}
%$  - p_{14} + p_{15} + p_{16} +
%  p_{24} + p_{26} + p_{34} + p_{35} - p_{36} \leq +p_{1}+ p_{2} +
%  p_{4} + p_{5}$
$I_{33}=p_{14} + p_{15} + p_{16} + p_{24} + p_{25} - p_{26} + p_{34} - p_{35}
- p_{1} - 2 p_{4} - p_{5} \leq$ among the 684 inequalities representing the
 faces of the associated classical correlation polytope.
The associated operator for symmetric
measurement directions is given by
\begin{widetext}
\begin{equation}
\begin{array}{lll}
&O_{33}(0,\theta,2\theta,0,\theta,2\theta)= q_{14}(0,0) + q_{15}(0,\theta) + q_{16}(0,2\theta) + q_{24}(\theta,0) +
q_{25}(\theta,\theta) - q_{26}(\theta,2\theta) +\\
&\qquad  + q_{34}(2\theta,\theta)- q_{35}(2\theta,\theta)-q_{1}(0) - 2 q_{4}(0) - q_{5}(\theta) \\
&\qquad \qquad =\frac{1}{4}\left(
\begin{smallmatrix}
-4\sin^2\theta & 0 & 0 & 0\\
0 & -5-2\cos\theta - 3\cos 2\theta + 2\cos 3\theta &
4\cos^2\frac{\theta}{2} & 2\sin\theta + 3 \sin 2\theta - 2 \sin
3\theta\\
0 & 4\cos^2\frac{\theta}{2} & -2(3+\cos 2\theta) & - 2\sin\theta \\
0 &  2\sin\theta + 3 \sin 2\theta - 2 \sin 3\theta & - 2\sin\theta &
2\sin^2\theta(4\cos\theta -3)
%  2 (-5+4 \cos \theta) {{\sin \theta}^2}&
%  3 \sin 2 \theta-2 \sin 3 \theta&
%  4 \sin \theta+3 \sin 2 \theta-2 \sin 3 \theta&
%  -2 (-1+4 \cos \theta) {{\sin \theta}^2} \\
%   3 \sin 2 \theta-2 \sin 3 \theta &
%   -7+2 \cos \theta-5 \cos 2 \theta+2 \cos 3 \theta &
%  -2 (-1+4 \cos \theta) {{\sin \theta}^2}&
%   -3 \sin 2 \theta+2 \sin 3 \theta  \\
%   4 \sin \theta+3 \sin 2 \theta-2 \sin 3 \theta &
%  -2 (-1+4 \cos \theta) {{\sin\theta}^2}&
%   -6 \cos \theta-5 (3+\cos 2 \theta)+2 \cos 3 \theta &
%   -4 \sin \theta-3 \sin 2 \theta+2 \sin 3 \theta  \\
%  -2 (-1+4 \cos \theta) {{\sin \theta}^2}&
%   -3 \sin 2 \theta+2 \sin 3 \theta &
%   -4 \sin \theta-3 \sin 2 \theta+2 \sin 3 \theta &
%  2 (-5+4 \cos \theta) {{\sin \theta}^2}
\end{smallmatrix}\right),
\end{array}
\label{2004-qbounds-e5}
\end{equation}
\end{widetext}
in the Bell-basis
$\{\ket{\phi^+},\ket{\psi^+},\ket{\psi^-},\ket{\phi^-}\}$ with
$\ket{\psi^\pm} = 1/\sqrt{2}(\ket{01} \pm \ket{10})$ and
$\ket{\phi^\pm} = 1/\sqrt{2}(\ket{00} \pm \ket{11})$.
(The states in
$\{\ket{00},\ket{01},\ket{10},\ket{11}\}$ are represented by spans of the vectors in
$\{(1,0,0,0)^T,(0,1,0,0)^T,(0,0,1,0)^T,(0,0,0,1)^T\}$.)
In this basis
the operator $O_{33}(0,\theta,2\theta,0,\theta,2\theta)$ splits into a direct sum of a one-dimensional
and a three-dimensional part simplifying the calculation of the
eigenvalues and eigenstates. Using the Cardano method
(see \cite{cocolicchio00} and references therein) one can
solve the characteristic equation of the submatrix $o$ with $\dim o
=3$ of $O_{33}$
\begin{equation}
  \lambda^3 + b(\theta) \lambda^2 + c(\theta) \lambda + d(\theta) = 0,
\label{2004-qbounds-characteristic}
\end{equation}
with the coefficients $b = -\Tr o,\ c = 1/2\Big(\Tr^2 o -
\Tr o^2 \Big),\ d = -\det o$. (For convenience we omit here the
dependence on $\theta$.) The (real) eigenvalues can then be written as \cite{cocolicchio00}
\begin{eqnarray}
\lambda_2 = -2 \sqrt{|u|}\cos\frac{\xi}{3}-\frac{b}{3}\nonumber\\
\lambda_{3,4} = \sqrt{|u(x)|}\Big[\cos\frac{\xi}{3} \pm
\sin\frac{\xi}{3}\Big]-\frac{b}{3},
\label{2004-qbounds-o33ev}
\end{eqnarray}
with $u=1/9(-b^2+3 c)$ and $\cos\xi = - \frac{1}{54}\big(2
b^3-9 b c + 27 d\big)/\big(u\sqrt{|u|}\big)$.
%The maximum eigenvalue $\lambda_{\rm max}(\theta)$ can be obtained straightforwardly but is of a rather lengthy form.
Together with the eigenvalue $\lambda_1 = -\sin^2\theta$ from the
one-dimensional part of $O_{33}$ we can visualize the eigenvalues plotted as functions of
the parameter $\theta$ in Fig. \ref{fig:2004-qbounds-f1}.
\begin{figure}[htbp]
  \centering
  \includegraphics[width=90mm]{2004-qbounds-f1}
  \caption{Eigenvalues of $O_{33}$ in dependence of the relative angle $\theta$.}
  \label{fig:2004-qbounds-f1}
\end{figure}
The maximum violation of $1/4$ is obtained for $\theta=\pi/3$ with the associated
eigenvector
\begin{equation}
  \ket{\Psi_{\rm max}}=\frac{\sqrt{3}}{2}\ket{\phi^-}+\frac{1}{2}\ket{\psi^+}.
\label{2004-qbounds-pmax33}
\end{equation}


As indicated in Ref.~\cite{collins-gisin-2003}, this scheme can be extended to $m$
measurements on each particle, by considering inequalities $I_{mm}
\leq 0$ and
corresponding operators $O_{mm}$ of the form
\begin{eqnarray}
  I_{mm}&=& \sum_{j=1}^{m}\sum_{i=1}^{m-j+1}P({A_i B_j})-\sum_{i=1}^{m-1}
  P({A_{i+1}B_{m-i+1}}) \nonumber\\
  &&-\sum_{i=1}^{m}(m-i)P(B_{i}) - P(A_1) \leq 0,
\end{eqnarray}
where $P(A_i B_j)$ denotes the joint probability of obtaining the value one of the
projection operators $A_i$ and $B_j$ operators on the left and on the
right hand side, and $P(A_i), P(B_j)$ the marginal probabilities on
one side, respectively.
For a choice of measurement directions
$\{0,\theta,2\theta,\ldots,m\theta\}$ on both sides, we can plot the
maximizing eigenvalues
(cf. Fig. \ref{fig:2004-qbounds-f2}). Furthermore, the matrices belonging to the operators $O_{mm},\ m \leq
6$ [... I suspect that this is valid also for higher $m$, but I
couldn't find a prove ...] are of the same form
as is $O_{33}$, i.~e. they
split up into a direct sum of two matrices in the
Bell-basis; the maximal eigenvalues can therefore be  calculated
explicitly using Eqs. (\ref{2004-qbounds-characteristic}) and (\ref{2004-qbounds-o33ev}).
\begin{figure}[htbp]
  \centering
  \includegraphics[width=90mm]{2004-qbounds-f2}
  \caption{Maximum violation of the operator $O_{mm}$ for
    $m=2,\ldots,6$ for a symmetric measurement setup; longer dashes indicate larger $m$.}
  \label{fig:2004-qbounds-f2}
\end{figure}


As the example in Eq.~(\ref{2004-qbounds-pmax33}) shows,
the experimental realization of the bounds of $O_{33}$ have to utilize nonmaximally entangled pure states.
Unlike the
Cabello \cite{cabello-2003a} and Bovino \emph{et al.} \cite{bovino-2003} ansatz for $m=2$,
such states cannot be
created from maximal entangled states
or by any generalized local unitary
operation $U_{2\times 2} \in SU(2) \otimes SU(2)$.
Experimental tests of the bounds of more general inequalities require
more general pure states
\footnote{
Nondegenerate eigenstates are always representable by one-dimensional subspaces
and thus are pure, the exception being the possibility of a mixing between
degenerate eigenstates \cite{braunstein92}.}.



Nevertheless,
for the $O_{33}$ case,
the {\it ansatz} of the Cabello \cite{cabello-2003a} and Bovino \emph{et al.} \cite{bovino-2003}
can be generalized to arbitrary {\em local}
unitary transformations applied to each one of the two particles separately.
One can start from the Bell state
$\ket{\psi^+} = 1/\sqrt{2}(\ket{01} + \ket{10}) \in S_{2\times 2}$ and
apply the local unitary operation $U(\omega_1,\theta_1,\phi_1)\otimes
U(\omega_2,\theta_2,\phi_2)$ with $\omega_1 =2\pi/3$,
$\theta_1=\phi_1=\pi/2$ and $\omega_2=\theta_2=\phi_2=0$. The single
qubit operators are taken as $U(\omega,\theta,\phi) =
e^{i\frac{\omega}{2} \vec{n}\cdot\vec{\sigma}} \in SU(2)$ with
$\omega$ as the rotation angle about the axis $\vec{n}=
(\sin\theta\cos\phi,\sin\theta\sin\phi,\cos\theta)^T$.
This yields the maximal violating eigenvector
$\ket{\Psi_{\rm max}}$ from
Eq. (\ref{2004-qbounds-pmax33}) which is maximally entangled.
Yet, it should be emphasized that for
general $\theta$, the experimental realization requires a two-qubit transformation in
$SU(4)/(SU(2)\otimes SU(2))$, followed by a local
unitary operation.


Alternatively, multiport interferometry \cite{rzbb,zukowski-97,svozil-2004-analog}
offers a direct proof-of-principle
implementation:
By choosing the appropriate transmission coefficients and phases in a generalized
beam splitter setup, one can prepare any pure state
from an input state $\ket{11} \equiv \{0,0,0,1\}^T$ corresponding to a photon
in a single input port. Take, for example, the maximal eigenstate of
the $O_{33}$ operator at $\theta=\pi/2$, $\ket{\Psi_{\pi/2}}= 0.86
  \ket{\psi^+} + 0.17\ket{\psi^-} + 0.47\ket{\phi^-} \equiv
  \{0.34,0.73,0.49,0.34\}^T$. The appropriate transmittance parameters
  can be calculated via the identification \cite{rzbb}
  \begin{equation}
    \begin{pmatrix}0\\0\\0\\1\end{pmatrix}^T R(N)^{-1} =  \begin{pmatrix}0.34\\0.73\\0.49\\0.34\end{pmatrix}^T =
    \begin{pmatrix}e^{-i\phi_1}\cos\omega_1 \\-e^{-i\phi_2}\cos\omega_2\sin\omega_1\\e^{-i\phi_3}\cos\omega_3\sin\omega_2\sin\omega_1\\-\sin\omega_3\sin\omega_2\sin\omega_1\end{pmatrix}^T
  \end{equation}
to $\omega_1 = 1.23, \omega_2=2.46,\omega_3=0.60$ and $\phi_1=\phi_2=\phi_3=0$,
where $R(N)$ is a
$SU(4)$ rotation serially composed by two-dimensional beamsplitter
matrices.


In summary, we have shown how to construct the exact quantum bounds of
Bell-type inequalities
by solving the eigenvalue problem of the associated self-adjoint
transformation.
Some problems which we now briefly mention remain unsolved. First, we may conjecture that the exact quantum correlation hull is directly derivable by extending the
classical Bell-type
inequalities in the same way presented above;
i.e., by substituting the quantum probabilities for the classical ones. This is by no means trivial, as the sections of the quantum hull need not necessarily be derivable by mere classical extensions. A second open question is related to the geometric structures arising from quantum expectation values. These need not necessarily be convex. Again, the question of direct extensibility remains open for the hull of
quantum expectations
from the classical ones.

This research has been supported by the Austrian Science Foundation (FWF),
Project Nr. F1513.


\bibliography{svozil}
\bibliographystyle{apsrev}


\end{document}

%TextStyle -> {FontFamily -> "Roman", FontSize -> 16}, Axes -> False, Frame -> True

########### 1-1
---------------------- 1-1.ext
* 1-1
*  $a1  $b1   $a1*$b1
*
V-representation
begin
   4  4  real
1  0  0  0
1  0  1  0
1  1  0  0
1  1  1  1
end
hull

---------------------- 1-1.ine
* cdd+: Double Description Method in C++:Version 0.76a1 (June 8, 1999)
* Copyright (C) 1999, Komei Fukuda, fukuda@ifor.math.ethz.ch
* Compiled for Floating-Point Arithmetic
*Input File:1-1.ext(4x4)
*HyperplaneOrder: LexMin
*Degeneracy preknowledge for computation: None (possible degeneracy)
*Hull computation is chosen.
*Zero tolerance = 1e-06
*Computation starts     at Fri Feb 13 09:44:17 2004
*            terminates at Fri Feb 13 09:44:17 2004
*Total processor time = 0 seconds
*                     = 0h 0m 0s
*Since hull computation is chosen, the output is a minimal inequality system
*FINAL RESULT:
*Number of Facets = 4
H-representation
begin
4  4  real
 1 -1 -1 1
 0 1 0 -1
 0 0 1 -1
 0 0 0 1
end

---------------------- 1-1.m

<< Algebra`ReIm`

TensorProduct[a_, b_] :=    Table[(*a, b are nxn and mxm - matrices*) a[[Ceiling[s/Length[b]], Ceiling[t/Length[b]]]]*b[[s - Floor[(s - 1)/Length[b]]*Length[b],t - Floor[(t - 1)/Length[b]]*Length[b]]], {s, 1,Length[a]*Length[b]}, {t, 1, Length[a]*Length[b]}];

vecsig[r_, tt_, p_ ]:= r * { {Cos[tt], Sin[tt] Exp[-I p]}, {Sin[tt] Exp[I p], -Cos[tt]}}

SingleProbH1[x_] :=   TensorProduct[1/2(IdentityMatrix[2] + vecsig[1, x, 0]), IdentityMatrix[2]]

SingleProbH2[x_] :=   TensorProduct[IdentityMatrix[2], 1/2(IdentityMatrix[2] + vecsig[1, x, 0])]

JointProb[x_, y_] :=  TensorProduct[1/2(IdentityMatrix[2] + vecsig[1, x, 0]), 1/2(IdentityMatrix[2] + vecsig[1, y, 0])]

O11[a_, b_] :=   -JointProb[a, b]  + SingleProbH1[a] + SingleProbH2[b]

O11[0, \[Theta]] // FullSimplify // MatrixForm

EVO11[\[Theta]_] := Eigenvalues[O11[0, \[Theta]]]

EVO11[1] // FullSimplify

Plot[{EVO11[\[Theta]][[1]],EVO11[\[Theta]][[2]],EVO11[\[Theta]][[3]],EVO11[\[Theta]][[4]]}, {\[Theta], 0, \[Pi]}, PlotStyle -> {Hue[0.1], Hue[0.3], Hue[0.5], Hue[0.7]}]

########### CH

<< Algebra`ReIm`

TensorProduct[a_, b_] :=    Table[(*a, b are nxn and mxm - matrices*)
a[[Ceiling[s/Length[b]], Ceiling[t/Length[b]]]]*b[[s - Floor[(s -
1)/Length[b]]*Length[b],t - Floor[(t - 1)/Length[b]]*Length[b]]], {s,
  1,Length[a]*Length[b]}, {t, 1, Length[a]*Length[b]}];

PartialTrace[
    A_List] := {{A[[1, 1]] + A[[3, 3]],
      A[[1, 2]] + A[[3, 4]]}, {A[[2, 1]] + A[[4, 3]], A[[2, 2]] + A[[4, 4]]}}

vecsig[r_, tt_, p_ ]:= r * { {Cos[tt], Sin[tt] Exp[-I p]}, {Sin[tt] Exp[I p], -Cos[tt]}}

SingleProbH1[x_] :=   TensorProduct[1/2(IdentityMatrix[2] + vecsig[1, x, 0]), IdentityMatrix[2]]

SingleProbH2[x_] :=   TensorProduct[IdentityMatrix[2], 1/2(IdentityMatrix[2] + vecsig[1, x, 0])]

JointProb[x_, y_] :=  TensorProduct[1/2(IdentityMatrix[2] + vecsig[1, x, 0]), 1/2(IdentityMatrix[2] + vecsig[1, y, 0])]

CHOp[a_, b_, c_, d_] :=
  JointProb[a, c] + JointProb[a, d] + JointProb[b, d] - JointProb[b, c] -
    SingleProbH1[a] - SingleProbH2[d]

CHOp[0, 2\[Theta], \[Theta], 3\[Theta]] // FullSimplify // MatrixForm

EVCH[a_, b_, c_, d_] := Eigenvalues[CHOp[a, b, c, d]]
EVCH[0, 2\[Theta], \[Theta], 3\[Theta]] // FullSimplify

\!\(Plot[{1\/4\ \((\(-2\) - \@\(6 - 2\ Cos[4\ \[Theta]]\))\),
      1\/4\ \((\(-2\) + \@\(6 - 2\ Cos[4\ \[Theta]]\))\), \((\@2 - 1)\)/
        2, \(-\((\@2 + 1)\)\)/2}, {\[Theta], 0, \[Pi]}]\)

EVectorsCH =
  Eigenvectors[CHOp[0, 2\[Theta], \[Theta], 3\[Theta]]] //
  FullSimplify

NormedEVectorsCH = Map[#/(Sqrt[#.#]) &, EVectorsCH];

PartialTrace[
    Outer[Times, NormedEVectorsCH[[4]],
      NormedEVectorsCH[[4]]]] // FullSimplify


########### 3-3 - case (Collins-Gisin)

<< Algebra`ReIm`

TensorProduct[a_, b_] :=    Table[(*a, b are nxn and mxm - matrices*) a[[Ceiling[s/Length[b]], Ceiling[t/Length[b]]]]*b[[s - Floor[(s - 1)/Length[b]]*Length[b],t - Floor[(t - 1)/Length[b]]*Length[b]]], {s, 1,Length[a]*Length[b]}, {t, 1, Length[a]*Length[b]}];

vecsig[r_, tt_, p_ ]:= r * { {Cos[tt], Sin[tt] Exp[-I p]}, {Sin[tt] Exp[I p], -Cos[tt]}}

SingleProbH1[x_] :=   TensorProduct[1/2(IdentityMatrix[2] + vecsig[1, x, 0]), IdentityMatrix[2]]

SingleProbH2[x_] :=   TensorProduct[IdentityMatrix[2], 1/2(IdentityMatrix[2] + vecsig[1, x, 0])]

JointProb[x_, y_] :=  TensorProduct[1/2(IdentityMatrix[2] + vecsig[1, x, 0]), 1/2(IdentityMatrix[2] + vecsig[1, y, 0])]

PartialTrace[vec_List] :=
  Module[{A = Outer[Times, vec, Conjugate[vec]]}, {{A[[1, 1]] + A[[2, 2]],
        A[[1, 3]] + A[[2, 4]]}, {A[[3, 1]] + A[[4, 2]],
        A[[3, 3]] + A[[4, 4]]}}]
PartialTraceB[vec_List] :=
  Module[{A = Outer[Times, vec, Conjugate[vec]]}, {{A[[1, 1]] + A[[3, 3]],
        A[[1, 2]] + A[[3, 4]]}, {A[[2, 1]] + A[[4, 3]],
        A[[2, 2]] + A[[4, 4]]}}]
Diagonal[A_] :=
  Module[{y = (#/Sqrt[#.Conjugate[#]]) & /@ Transpose[Eigenvectors[A]]},
    Conjugate[Transpose[y]].A.y]

\!\(\(BellBasis =
      1/\@2 {{1, 0, 0, 1}, {0, 1, 1, 0}, {0, 1, \(-1\), 0}, {1, 0,
            0, \(-1\)}};\)\[IndentingNewLine]
  BellTrans[X_List] :=
    BellBasis . X . Transpose[BellBasis]\[IndentingNewLine]
  InverseBellTrans[X_List] := Transpose[BellBasis] . X . BellBasis\)


CoGi[a_, b_, c_, d_, e_, f_] :=   JointProb[a, d] + JointProb[a, e] + JointProb[a, f] + JointProb[b, d] + JointProb[b, e] - JointProb[b, f] + JointProb[c, d] - JointProb[c, e] - SingleProbH1[a] - 2*SingleProbH2[d] - SingleProbH2[e]

CoGi[0, \[Theta], 2*\[Theta], 0,\[Theta], 2*\[Theta]] // FullSimplify // MatrixForm

EVCoGi[\[Theta]_] := Eigenvalues[CoGi[0, \[Theta], 2*\[Theta], 0, \[Theta], 2*\[Theta]]]

EVCoGi[x] // FullSimplify

cogiadapted[x_] =
    BellTrans[CoGi[0, x, 2x, 0, x, 2x]] // FullSimplify // TrigReduce;
% // MatrixForm

MyEigenvalues[x_] = Eigenvalues[cogiadapted[x]];

\!\(Plot[Evaluate[MyEigenvalues[x]], {x, 0, 2  \[Pi]},
    PlotStyle -> {Dashing[{2/200, 2/300}], Dashing[{4/200, 2/300}],
        Dashing[{4/200, 4/300}], Dashing[{1/200, 4/300}]},
    AxesLabel -> {"\<\[Theta][rad]\>", \*"\"\<\!\(I\_33\)\>\""}]\)

EVectorsCoGi[x_] := Eigenvectors[CoGi[0, x, 2x, 0, x, 2x]]
NormedEVectorsCoGi[x_] := Map[#/(Sqrt[#.#]) &, EVectorsCoGi[x]];
EVectorsCoGiBell[x_] :=
  Eigenvectors[FullSimplify[BellTrans[CoGi[0, x, 2x, 0, x, 2x]]]]
NormedEVectorsCoGiBell[x_] := Map[#/(Sqrt[#.#]) &, EVectorsCoGiBell[x]];

EVMax = NormedEVectorsCoGiBell[\[Pi]/3][[4]] // FullSimplify

Diagonal[PartialTrace[Transpose[BellBasis].EVMax]]

########### 4-4 - case (Collins-Gisin)

CoGi4[a_, b_, c_, d_, e_, f_, g_, h_] :=
  JointProb[a, e] + JointProb[a, f] + JointProb[a, g] + JointProb[a, h] +
    JointProb[b, e] + JointProb[b, f] + JointProb[b, g] - JointProb[b, h] +
    JointProb[c, e] + JointProb[c, f] - JointProb[c, g] + JointProb[d, e] -
    JointProb[d, f] - SingleProbH1[a] - 3*SingleProbH2[e] -
    2*SingleProbH2[f] - SingleProbH2[g]

CoGi4[0, x, 2x, 3x, 0, x, 2x, 3x] // FullSimplify // MatrixForm

EVCoGi4[\[Theta]_] := Eigenvalues[CoGi4[0, \[Theta], 2*\[Theta],
   3*\[Theta], 0, \[Theta], 2*\[Theta], 3*\[Theta]]]


Plot[{EVCoGi4[\[Theta]][[1]], EVCoGi[\[Theta]][[2]], EVCoGi[\[Theta]][[3]],
    EVCoGi[\[Theta]][[4]]}, {\[Theta], 0, \[Pi]},
  PlotStyle -> {Hue[0.1], Hue[0.3], Hue[0.5], Hue[0.7]}]

########### m-m - case (Collins-Gisin)

CoGiM[left_List, right_List] :=
  Sum[Sum[JointProb[left[[i]], right[[j]]], {i, 1, Length[left] - j + 1}], {j,
         1, Length[right]}] -
    Sum[JointProb[left[[i + 1]], right[[Length[right] - i + 1]]], {i, 1,
        Length[left] - 1}] -
    Sum[(Length[right] - i)*SingleProbH2[right[[i]]], {i, 1, Length[right]}] -
     SingleProbH1[left[[1]]]

EVCoGiM[x_, m_] :=
  Eigenvalues[CoGiM[Table[i*x, {i, 0, m - 1}], Table[i*x, {i, 0, m -
    1}]]]

CoGiPlots =
  Table[Table[{x, Max[Re[N[EVCoGiM[x, i]]]]}, {x, 0, 2\[Pi], N[\[Pi]/60]}] //
      ListPlot[#, PlotStyle -> {Hue[i/10]}, PlotJoined -> True,
          DisplayFunction -> Identity] &, {i, 2, 5}]

\!\(Show[CoGiPlots, DisplayFunction -> $DisplayFunction,
    AxesLabel -> {"\<\[Theta][rad]\>", \*"\"\<\!\(I\_mm\)\>\""}]\)

##########

a=
{
{a11,a12,a13},
{a21,a22,a23},
{a31,a32,a33}
};

b={b1,b2,b3};

Solve[{
a.{0,0,0}=={-1,-1,+1},
a.{0,1,0}=={-1,+1,-1},
a.{1,0,0}=={+1,-1,-1},
a.{1,1,1}=={1,1,1}},{a11,a12,a13,a21,a22,a23,a31,a32,a33,b1,b2,b3}]

a=
{
{a11,a12,a13},
{a21,a22,a23},
{a31,a32,a33}
};

b={b1,b2,b3};

Solve[{
a.{0,0,0}=={0,0,1},
a.{0,1,0}=={0,1,0},
a.{1,0,0}=={1,0,0},
a.{1,1,1}=={1,1,1}},{a11,a12,a13,a21,a22,a23,a31,a32,a33,b1,b2,b3}]



#####################################################################


Sliwa hull
Mime-Version: 1.0
Content-Type: text/plain; charset="us-ascii"; format=flowed
X-Virus-Scanned: by amavisd-milter (http://amavis.org/)
X-Spam-Status: LOW ; -30
X-Spam-Level: -
X-Spam-TU-Processing-Host: mri1
X-UIDL: a&i!!S"/!!UX"#!HO%#!
Status: RO

<< Algebra`ReIm`

TensorProduct[a_, b_] :=    Table[(*a, b are nxn and mxm - matrices*) a[[Ceiling[s/Length[b]], Ceiling[t/Length[b]]]]*b[[s - Floor[(s - 1)/Length[b]]*Length[b],t - Floor[(t - 1)/Length[b]]*Length[b]]], {s, 1,Length[a]*Length[b]}, {t, 1, Length[a]*Length[b]}];

vecsig[r_, t_, p_ ]:= r * { {Cos[t], Sin[t] Exp[-I p]}, {Sin[t] Exp[I p], -Cos[t]}}

CorrellOp[x_, y_] := TensorProduct[vecsig[1, x, 0], vecsig[1, y, 0]];
SingleOpH1[x_] := TensorProduct[vecsig[1, x, 0], IdentityMatrix[2]];
SingleOpH2[x_] := TensorProduct[IdentityMatrix[2], vecsig[1, x, 0]];

PITOpSingle[a_, b_, c_, d_, e_, f_] := CorrellOp[a, d] + CorrellOp[a, e] + CorrellOp[b, d] + CorrellOp[b, e] +      CorrellOp[c, d] - CorrellOp[c, e] + CorrellOp[a, f] - CorrellOp[b, f] +      SingleOpH1[a] + SingleOpH1[b] + SingleOpH2[d] + SingleOpH2[e];

PITOpSingle[0, x, 2*x, 0, x, 2*x] // FullSimplify // MatrixForm

Eigenvalues[PITOpSingle[0, x, 2*x, 0, x, 2*x]] // FullSimplify

eeeSingle[x_] := Eigenvalues[PITOpSingle[0, x, 2*x, 0, x, 2*x]]

Plot[{eeeSingle[x][[1]], eeeSingle[x][[2]], eeeSingle[x][[3]], eeeSingle[x][[4]]}, {x, 0, Pi}, PlotStyle -> {Hue[0.1], Hue[0.3], Hue[0.5], Hue[0.7]}]




