\input easy \input art12
\doublespacing
\article{Dimension of space--time}
\vskip 1cm
\bf
\centerline{Karl Svozil}
\sl
\centerline{Institut f\"ur Theoretische Physik}
\centerline{Technische Universit\"at Wien}
\centerline{Karlsplatz 13/136, A-1040 Vienna, Austria}
\vskip 0.5cm
\centerline{and}
\vskip 0.5cm
\bf
\centerline{Anton Zeilinger}
\sl
\centerline{Atominstitut der \"Osterreichischen Universit\"aten}
\centerline{Sch\"uttelstra\3e 115, A-1020 Vienna, Austria}
\centerline{and}
\centerline{Department of Physics}
\centerline{Massachusetts Institute of Technology}
\centerline{Cambridge, MA 02139, U.S.A.}
\rm
\vskip 1cm
\chapter{Abstract}
In order to make it operationally accessible, it
is proposed to base the notion of the dimension
of space--time on measure--theoretic concepts,
thus admitting the possibility of noninteger dimensions.
It is found then, that the Hausdorff covering procedure
is operationally unrealizeable because of the inherent
finite space--time resolution of any real experiment.
We therefore propose to define an operational dimension
which, due to the quantum nature of the coverings,
is smaller than the idealized Hausdorff dimension.
As a consequence of the dimension of space--time less
than four relativistic quantum field theory becomes finite.
Also, the radiative corrections of perturbation theory
are sensitive on the actual value of the
dimension $4-\epsilon $.
Present experimental results and standard theoretical
predictions for the electromagnetic moment of the electron
seem to suggest a non--vanishing value for $\epsilon $.
\endchapter
\chapter{I. Introduction}
The perception of a seemingly threedimensional
space is as old as occidental civilisation
itself, possibly much older.
Theaitetos, a contemporary of Plato$^{1}$ (around
400 B.C.), pursued a geometric approach by looking
for regular convex bodies covering all
space$^{2}$, a method very similar to modern
techniques.
Among others, also the Alexandrian mathematician
Ptolemy (2nd century A.D.) reportedly$^{3}$
finished a treatise on the threedimensionality of space.
Many modern philosophers such as Kant$^{4}$ and
also physicists have considered the dimension of space
and space--time as something {\it a priori}
given.
Such an approach implies that dimension is a proposition which, though
it may be elicited by experience,
is seen to have a basis other than experience.



The objective of this article is to show the
existence of a basis of experience which, contrary to
{\it a priori}
notions, leads to a {\it measureable} dimension
of space or space--time.
It turns out that such an operationalistically
defined dimension will not necessarily be an integer;
rather a real number, and lower than the associated
``ideal'' dimensions of three and four.



Before concentrating on the physics, an overlook
of mathematical concepts and reasoning concerning
dimensionality seems appropriate.
One of the most intuitive dimensional concepts has been
introduced by Brower in 1922 and worked out by
Menger and Uryson$^{5}$.
It is called the
{\it topological dimension}
$\alpha _T$ and defined via a recursion:
\itemize{1cm}
\litem{(i)} $\alpha_T(\emptyset )=-1,$ and
\litem{(ii)} $\alpha_T(E)$ is the least integer $n$ for which every
point of an arbitrary set $E$ has small neighborhoods whose
boundaries have dimension less than $n$.
\enditemize
This definition yields only integer dimensions and is too
rude a criterion to characterize many sets developed in the late
nineteenth century.
At that time a debate took place after Cantor had proposed
a set, often referred to as Cantor ternary set, with zero Lebesque measure which, in the sense of length,
seems a trivial subset of a line.
On the other hand, a bijective mapping
between the points on the line
and the points of the Cantor ternary set can easily be found with a suitable parametrisation$^{6}$.



With the works of Caratheodory and Hausdorff$^{7}$
these problems could be eased, however for the price
of introducing noninteger dimensions. The new
notion of measure was based on a covering $\cup_iB_i$ of a
given set $E$ and a limit in which all individual
constituents $B_i$ of this covering become infinitesimal
in diameter.
Hausdorff showed that there exists a measure $\mu $, called the {\it Hausdorff measure}, and a unique number
$\alpha_H$, called the
{\it Hausdorff dimension},
such that for any set $E$,
$$\mu (E,\alpha )=\lim_{\epsilon \rightarrow 0+} \inf_{\{ B_i\} }
\left\{ \sum_i(diam\,B_i)^\alpha \, :\; \alpha \in R,\, \alpha >0,\,
\cup_iB_i\supset E,\, (diam\, B_i)\le \epsilon \right\} ,\eqno(1.1a)$$
$$\mu (E,\alpha )=
\cases{0,&if $\alpha >\alpha_H(E)$;\cr
       \infty ,&if $\alpha <\alpha_H(E)$.\cr}\eqno(1.1b)$$
since the diameter presupposes the notion of a distance, we remark
that with respect to variation of the metric,
$\alpha_H$ need not be an invariant.



A couple of other characteristic measures and their associated
dimensions have been introduced since Hausdorff's article$^{8}$.
One of the most important is the
{\it capacity dimension}
$\alpha_C$,
which for self--similar sets, equals $\alpha_H $
and is defined as
$$\alpha_C=\lim_{\epsilon \rightarrow 0+}\;\log \left[ n(\epsilon )\right] /\log \left(
\epsilon^{-1}\right),\eqno(1.2)$$
where $n(\epsilon )$ is the number of segments
of reduced length $\epsilon $.



The Hausdorff measure has a second, rather important
application for the definition of integral measures,
although this analytic aspect is rarely appreciated.
It gives some crude and heuristic hints on the
{\it packing density} of space--time points and thus
the support of [quantized] fields.
Whereas in section II an operational definition of a
physical measure is given, section III envisages
analytical consequences of such a measure.
The importance of an upper bound on $\alpha_H$
of four lies in an improvement of convergence of formerly
weak divergencies in continuous quantum
field theory, which becomes defined and finite.
At the same time it is possible to preserve symmetries such as
Lorentz covariance.
Since the measure changes all transition matrix elements,
a value for $\alpha_H$ can be obtained by
comparing sensitive theoretical predictions with experiment.



In this context, {\it extrinsic} and {\it operational}
(or {\it intrinsic}) concepts$^{10,11,12}$ are
extremely important for an understanding of the
meaning of the physical dimension.
A quantity is called {\it extrinsic} if it refers to
a system, although it is not obtained by measurements
that are feasible within that system. Rather it refers
to some sort of knowledge coming from the
``outside environment''.
It is quite obvious that it will never be possible to
measure the extrinsic dimension of the whole universe.


On the other hand, an operationally obtained quantity
is derived from measurements and procedures
within a given system.
From this point of view, a ``surrounding
environment'' need not be assumed and the knowledge
of an ``outside world'' may be considered as
complicating and superfluous.
When we speak of an operational measurement of the
dimension of space--time, this is all we can do.
Even if we would concede the reality of
a space--time arena and an associated external dimension,
we may never be able to know it, since it could very well be, that the operational dimension is
only an approximation to some presumably ``true'' value.
However, a criterion will be given to indicate
if the extrinsic dimension of a local region of space--time is four.



Since the introduction of so--called {\it fractals
}$^{13}$ and even before$^{14,15}$,
there have been proposals to utilize Hausdorff's
dimensional concepts. However, to our knowledge, no
research has been pursued to clarify the dimension
of space--time (compare references 4 and 16-20).
\endchapter
\chapter{II. Operational definition of dimension}
We propose that dimensional concepts in physics are only meaningful
if they have an operational base.
This means that it has to be at least in principle possible
to define procedures and construct devices for a
measurement of dimension.
Conceptual difficulties are encountered by a
straightforward adoption of mathematical notions
of dimension.
In particular, two limiting conditions have to be recognized for
the implementation of definitions:
\itemize{1cm}
\litem{(i)}There is no physical meaning to an infinitesimal covering with
the diameters of all constituents of this covering (balls etc.)
going to zero, as implied by Eqs.(1.1) and (1.2). Since
the physical systems available to us have only
finite energy content, it is impossible
to realize configurations of infinitesimal spacial
or time resolution;
\litem{(ii)}There are always uncertainties associated with
a measureable quantity.
Therefore, the physical dimension, as all
parameters derived from such quantities, will be determined with some
degree of uncertainty.
\enditemize

In what follows we suggest a modification of the Hausdorff
measure which takes these restrictions into account and will
thus be applicable to physical systems.
\section{A. Operational measure}
In analogy to the Hausdorff measure $\mu $,
a physically meaningful measure $\nu $ can be defined via
a limit. The coverings however, must be restricted to those of
finite diameter $\delta_{exp} $.
This diameter can be identified with the space--time resolution
in a  specific system.
We assume that space--time is a set $E$, and
 arbitrary coverings $\lbrace B_i\rbrace $ of $E$
such that $E\subset \bigcup_i\; B_i$.
Then the operational measure $\nu (\alpha ,\delta_{exp} )$
can be defined as a function of an arbitrary dimension $\alpha $
and the maximal experimental resolution $\delta_{exp} $
associated with a specific experiment:
$$\nu (\alpha ,\delta_{exp} )=\lim_{\epsilon \rightarrow \delta_{exp} +}
\inf_{\{ B_i\} } \left\{ \sum_i(diam\, B_i)^\alpha \, :\, \alpha >0,\,
\cup_iB_i\supset E,\,
\delta_{exp} \le (diam\,B_i)\le \epsilon \right\}.\eqno(2.1)$$
This limit exists, since the infimum guarantees$^7$ that the
value of $\nu $ increases for decreasing $\epsilon $.
In the limit the coverings become smaller in diameter
$\epsilon $
until they reach the resolution $\delta_{exp}$. For infinitesimal
resolution, $\nu (\alpha ,\delta_{exp} )$ tends
to the Hausdorff measure of $E$ with an associated dimensional parameter $\alpha $:
$$\lim_{\delta_{exp} \to 0+}\nu (\alpha ,\delta_{exp} )=
\mu (\alpha ).\eqno(2.2)$$
Before defining an operational dimension associated with $\nu $,
it is necessary to work out in greater detail the classical
and quantum meaning of a covering.
\endsection
\section{B. Classical and quantum meaning of a covering}
In mathematics a covering $\lbrace B_i\rbrace $ of $E$ is understood
as a set of sets $\lbrace B_i\rbrace $ covering all of $E$, i.e. $E\subset \cup_iB_i$,
no matter if there are multiple overlaps, such that
$\bigcup_{i\not= j} B_i\cap B_j\not= \emptyset $
[see Fig. 1].
It is not necessary to know the dimension of the coverings,
since this would result in a recursion and would considerably
weaken the power and the elegance of Hausdorff's definition.
Only in the limit $(diam\; B_i)\longrightarrow 0+$
ambiguities from multiple countings are resolved and the
measure is defined uniquely.
In physics, we do not have this limit at our disposal.
The resulting ambiguities will have far--reaching consequences.



The next question is what meaning can be given to a
covering in a microscopic world governed by
quantization of action ?
And just what can serve as a covering ?
To define coverings in these domains, a further move towards
abstraction seems necessary. A form of stochastic covering is introduced
by the following requirement:
Assume a quantum state $\mid \psi \rangle $
is localized in the sense that it is possible to define its
momenta
$$M_n=\sum_i\int x_i^n\mid \langle x_i\mid \psi \rangle
\mid ^2dx_i < \infty.$$
Then a covering can be defined by the condition that
it includes all  greater than
or equal to an arbitrary, fixed value $p$ (see Fig. 2):
$$B_i=\left\{ x\in R^4\; : \; \mid \langle x\mid
\psi_i \rangle \mid ^2\ge p\right\} .\eqno(2.3a)$$
Alternatively, $B_i$ can be represented by a fuzzy set with
its characteristic function$^{41}$ identified with
$$\chi_{B_i}(x)=\mid \langle x\mid \psi_i\rangle \mid^2.\eqno(2.3b)$$
For simplicity we consider only states yielding convex
coverings. In varying the width of the state, the
resolution is changed.
In principle the resolution of these coverings
could go to zero by changing the definition
and taking a value for the probability density
$p_s$ such that $p_s\le \mid \langle x_s\mid \psi \rangle \mid ^2$
is fulfilled only for a singular point $x_s$. Then the limit
$\delta_{exp} \longrightarrow 0+$ could be performed and
Hausdorff's definition adopted without changes.
However, the problem then arises just how to cover all
of $E$ with states available, which would result in
infinitely many states with infinite energy and thus
would again encounter unresolvable conceptual difficulties in the physical realization.
\endsection
\section{C. Operational dimension}
There is no unique or most evident definition of physical
dimension, hence several forms will be given.
It depends on the particular problem which convention
is more suitable for a physical application.



Although the concept of
{\it topological dimension}
seems quite straightforward, it is difficult to realize operationally.
Both prerequisites, the notion of a neighborhood
as well as a point to start the recursion [having
as surrounding the
empty set with $\alpha_T(\emptyset )=-1$]
cause problems in their implementation.
Furthermore, this notion of dimension is not
suitable for analytic applications, since it is not integrated
into some concept of measure.



We have defined $\delta_{exp}$ as the maximal resolution
associated with a specific experiment,
and $\epsilon \ge 1$, measured in units of $\delta_{exp}$,
as the diameter of coverings used in the limit of (2.1).
Our major concern will therefore be dimensional concepts
originating in measure theory.
The {\it capacity dimension} $\alpha_C$ has been
mentioned already in the introduction.
Its definition can be maintained
if $E$ is assumed to be self--similar$^{9}$: for $\delta_{exp}$ fixed,
$$\alpha_C=-\log \left[ n(\epsilon )\right]/\log
(\epsilon ).\eqno(2.4a)$$
Here, $n(\epsilon )$ is the number of segments or constituents
of equal diameter $\epsilon $, covering all of $E$, where
$E$ is normalized to unity. An equivalent definition for $\delta_{exp}$ fixed is
$$\alpha =-{\Delta \log [n(\epsilon )]\over \Delta \log (\epsilon )},
\eqno(2.4b)$$
which as its limit has,
$$\alpha_C=-{d\; \log [n(\epsilon )]\over d\; \log (\epsilon )}.\eqno(2.4c)$$
For our purposes, $\alpha_C$ can very well be a
function of the experimental resolution $\delta_{exp} $
$$r_0\delta_{exp} = \left[ (\Delta x)^2+
(c\Delta t)^2\right] ^{1/2},\eqno(2.5)$$
where $r_0$ is some reference length measured in the
same units as $\Delta x$.
As can be argued using uncertainty relation considerations, the
maximal resolution in a measurement involving photons of total
energy $E_{tot}$ within a time span $\Delta t$ is given by
$$r_0\delta_{exp} \ge {ch^2\over 4\pi E^2_{tot}\Delta t}.\eqno(2.6)$$
From now on, we drop the index ``exp'' whenever we refer to the maximal experimental resolution.
Taking an estimated energy content of the universe
and the age of the universe
yields a maximal resolution of$^{21}$
$$God knows what.$$
Equation (2.4a) can be derived from the definition of a
modified Hausdorff dimension (2.1) in the
following way:
with the assumption of a unit ``volume'' or measure covered
with identical objects of diameter $\epsilon $,
Eq. (2.1) reduces to
$$n(\epsilon )\epsilon ^{\alpha_C}=\nu (\alpha_C,\epsilon )
=1.\eqno(2.7)$$
The capacity dimension is widely used in mathematics
as well as in physics because of its applicability.
However, it has to be assumed that the sets conine space--time to be {\it self--similar} if its
capacity dimension is a constant with respect
to the covering diameter $\epsilon $ at a fixed  resolution $\delta $.


Furthermore, we propose it to be reasonable, that  the operational measure (2.1) should {\it not} depend
on the resolution $\delta $. This implies that for two
different resolutions
$\delta $ and $\delta '$, the dimension parameter $\alpha_{op}(\delta )$
[the index ``op'' indicates that $\alpha_{op}$ is an operator
$$(\delta ),\delta )=\nu (\alpha_{op}(\delta ' ),\delta ').\eqno(2.8a)$$
In differential form, this reads
$${d\nu (\alpha (\delta ),\delta )\over d\delta }\biggm\vert_{\alpha =
\alpha_{op}(\delta )}=0.\eqno(2.8b)$$
A better understanding of the behavior of
$\nu (\alpha ,\delta )$ for self--similar sets may be obtained by ``smearing out''
the Hausdorff measure.
As an example, we discuss the case, where a modified
Heavyside function smeared out in $\epsilon $,
$$\theta_\epsilon (\alpha )=
\left\{ {1\over 2}-{1\over \pi }
\arctan [(\alpha -\alpha_H )/\epsilon ]
\right\}$$
could serve as a model for the measure.
In Fig. 3, $\theta_\epsilon (\alpha )$ is plotted as a function
of covering diameter $\epsilon $ and dimension $\alpha $.
For this case we find:
\itemize{1cm}
\litem{(i)}For constant diameter $\epsilon $,the measure decreases monotonously in $\alpha $:
$${\partial \nu (\alpha ,\epsilon )\over \partial
\alpha}< 0\eqno(2.9)$$
for all $\alpha $ and $\epsilon \not= 0$, and
\litem{(ii)}the Hausdorff dimension is an umklapp
point in the sense that
$${\partial \nu (\alpha ,\epsilon )\over \partial \epsilon }=
\cases{>0,&if $\alpha > \alpha_H,$\cr
=0,&if $\alpha =\alpha_H,$\cr
<0,&if $\alpha <\alpha_H.$\cr}\eqno(2.10)$$
\enditemize


We propose to generalize
equation (2.10) as a criterion on $\nu $ such that it
may serve as a definition of an operationally
defined dimension $\alpha_{op}$ for all self--similar coverings.
For constant resolution $\delta $,
$${\partial \nu (\alpha ,\epsilon )\over \partial \epsilon }
\biggm\vert_{\alpha =\alpha_{op}} = 0.\eqno(2.11)$$
Notice however, that even for self--similar sets,
this criterion might not apply,
since the associated physical coverings need not
be self--similar.
For general purposes, the nondifferential form
(2,8a) will be most useful,
since it is not restricted to self--similar
sets or coverings.



Another differential criterion may be obtained in
a similar way as a generalization of the umklapp
property (1.1) of the Hausdorff measure.
Here again, the jump of the measure at $\alpha_H$
will be replaced by a smooth transition as a result
of the finite resolution.
It is therefore a natural generalization of Hausdorff's
original approach to define as the new operational dimension
the point of maximal slope:
for constant resolution $\delta $,
$${\partial ^2\nu (\alpha ,\epsilon )\over \partial \alpha^2}
\biggm\vert_{\alpha =\alpha_{op}} = 0.\eqno(2.12)$$
This definition does not employ variations
of resolution and is not restricted to
self--similar sets.
Rather, the operational dimension may generally be a function of
the resolution and thus scale--dependent:
$\alpha_{op}=\alpha_{op}(\delta )$
[This would imply that space-time is not self--similar.
It should be noted however, that if self--similarity is assumed,
$\alpha_{op}=\alpha_C$].
However, definition (2.12) cannot be applied to all coverings,
as can be seen from the discussion of the Koch curve below.
In these cases, some other generalization of the original umklapp property (1.1b) has to be
utilized to obtain $\alpha_{op}$.
\endsection
\section{D. Bounds on the operational dimension}
In this section it is argued that the double or
multiple counting of some space--time points which are then contained
in two or more constituents of a covering $\lbrace B_i\rbrace $
has decisive impact on the operational dimension as compared
to the ``real'' or Hausdorff dimension.
Such a multiple counting is inevitable in the experimental
realization of a covering:
the boundaries of the constituents $B_i$ are never known
with certainty.
Thus to be sure that all of space or space--time is covered,
more $B_i's$ with a larger ``volume'' than necessary have
to be assumed.


The consequences are straightforward: assume
$\mu_H(\alpha_H)$ is the [extrinsic] Hausdorff measure of
space--time with an associated Hausdorff dimension $\alpha_H$
[of four ?].
Because of multiple counting one obtains
$$\nu (\alpha_H,\epsilon )>\mu_H(\alpha_H).\eqno(2.13)$$
Eq. (2.9)  can only be satisfied by an adjustment
of the operational dimension $\alpha_{op}$ such that
$$\nu (\alpha_{op},\epsilon )=\mu_H(\alpha_H).\eqno(2.14)$$
Since the number of constituents
$card(\lbrace B_i\rbrace )=n(\epsilon )$ as well as the
resolution $\delta $ is fixed, and when $\epsilon $ is measured in units of $\delta $, (2.14) can only be
satisfied for
$$\alpha_{op}<\alpha_H.\eqno(2.15)$$
This condition is a direct consequence of the impossibility
to perform the limit $\delta \longrightarrow 0+$
for physically realizeable coverings.
Only in this limit there is no double counting.



The experimental uncertainty intrinsic in the determination of $\alpha_{op}$
can be obtained immediately if a homogenuous covering can be
applied such that
$$\nu (\alpha_{op},\epsilon )=n(\epsilon )\epsilon^{\alpha_{op}}
=const.$$
Then,
$$\Delta \alpha_{op}={1\over \log \epsilon }\left[
{\Delta n\over n(\epsilon )}+
\alpha_{op}{\Delta \epsilon \over \epsilon }\right],\eqno(2.16)$$
where $\Delta n$ and $\Delta \epsilon $ are uncertainties
in the number of constituents and the covering diameter respectively.
\endsection
\section{E. Examples of coverings and dimensionality
of physical units}
In what follows two examples for physical coverings are given.
First, we consider a cavity filled with longitudinal modes.
We study a configuration with waves propagating in a
onedimensional waveguide, as shown in Fig. 4.



By defining the wavelength $\lambda $ as the fundamental
constituency of the covering, the measure is just the number of wavelengths
$n(\lambda )$ filling the cavity, times $\lambda^{\alpha }$, plus
an extra term $t(\lambda )$ from double counting and boundary effects.
On the Gedankenexperiment level,
$n(\lambda )$ is directly obtained by measurement of the induction
current in a loop perpendicular to the field, and the wavelength
$\lambda $ is varied by tuning the frequency.
$t(\lambda )$ was introduced just to make sure that the modes
really cover all of the cavity. It represents corrections due
to systematic errors steming from uncertainties in the determination
of $\lambda $ and $n(\lambda )$ and becomes important if the fine
structure of the wall affects the resonance frequency. For all these
reasons, $t(\lambda )$ will never vanish as for the case of
absolute precision.
From (2.1), the measure is then given by
$$\nu (\alpha ,\lambda )=n(\lambda )\lambda^{\alpha }+
t(\lambda ).\eqno(2.17)$$
Applying condition (2.8) for two different
wavelengths $\lambda $ and $\lambda '$ and
assuming $t(\lambda )\sim t(\lambda ')$, an
explicit expression for $\alpha_{op}$ is obtained:
$$\alpha_{op}\sim {\log \left[n(\lambda ')/n(\lambda )\right]\over
\log (\lambda /\lambda ')}.\eqno(2.18)$$
If the cavity is onedimensional and of length $L$,
then $n(\lambda )=L/\lambda $ and thus $\alpha_{op}=1$.




Another example is the covering of space or space--time
with holographic images of balls or objects of arbitrary shape.
Since all considerations of the last paragraphs
also apply to this sort of covering,
it will not be treated in detail.



The following study of the Koch curve $\cal K $ [see Fig. 5 and
ref. 13] is not directly connected to space--time measurements.
However, it yields some insight for the basic applications of
(2.8) and (2.11) to define $\alpha_{op}$.


Let $\mu =1$ be the Hausdorff measure (``volume'') of $\cal K$, normalized
to unity.
Ideally, with increasing resolution $\delta =3^{-N}$, which can be thought of going
in discrete steps labelled by $N$, more and more structure appears.
At the $N$th step, $n(\delta )=4^N$ identical segments
[all of length $3^{-N}$] can be seen.
Identifying the covering diameter $\epsilon $ with the resolution $\delta $, and applying (2.11), yields
$${\partial \mu \over \partial \delta }=
{\partial [4^{-\log \delta /\log 3}\delta^{\alpha_H}] \over
\partial \delta }=0\eqno(2.19)$$
This renders $\alpha_H ({\cal K})=\log 4/\log 3 $.


A more physical implementation of a covering of $\cal K$ has to
take into account a finite and fixed uncertainty $\rho $
independent of
the diameter $\epsilon $ for a fixed resolution $\delta $ of the coverings.
To make sure that all of $\cal K $ is covered,
for a calculation of the number $n(\epsilon )$ of covering
constituents, the diameter has to be substituted by
a reduced covering diameter $\Delta =\epsilon -\rho ,\,
0\le \Delta \le \epsilon $.
[From now on, we consider coverings of diameter $\epsilon $,
measured in units of the resolution $\delta =3^{-M}$.
Hence, $\epsilon \ge \rho \ge  1$].
A decrease in the effective ball size in turn increases $n(\epsilon )$ by
$$n(\rho ,\epsilon )=n(\rho =0,\epsilon )[\epsilon /\Delta ]^{\alpha_H}.\eqno(2.20)$$
Taking this into account, yields an operational measure of the form
$$\nu (\epsilon ,\rho ,\alpha )= n(\epsilon )[1-\rho /\epsilon]^{-\alpha_H}
\epsilon^\alpha .\eqno(2.21)$$
Utilizing (2.8) for  a definition of $\alpha_{op}$, and
inserting $n(\epsilon )=4^{M-\log \epsilon /\log 3}$ and $\epsilon =3^{-N}$, one obtains
for $\epsilon /\rho \ge 1$
$$\alpha_{op}({\cal K})=\alpha_H({\cal K})
{\log (\epsilon -\rho )\over \log \epsilon }.\eqno(2.22)$$
Note, that (2.11) cannot be applied straightforwardly, since
the covering is not self--similar [although the Koch curve
is a self--similar set].
This dimensional parameter has the following features:
\itemize{1.0 true cm}
\litem{(i)}in the limit $\rho \longrightarrow 0,\;
\alpha_{op}({\cal K})\longrightarrow \alpha_H({\cal K})$;
\litem{(ii)}$\alpha_{op}({\cal K})$
is strictly monotonous decreasing
in $\rho $ [see  (2.15)]:
the higher $\rho $ is, the more constituents $n(\rho ,\epsilon )$
have to be taken into account to guarantee that all of
${\cal K}$ is covered.
Since in this scale, $\epsilon >1$, $\alpha_{op}$ has to decrease
in order to compensate for these additional coverings.
\litem{(iii)}When the uncertainty approaches the resolution,
$\rho \longrightarrow (\epsilon -1) -$, $\alpha_{op}({\cal K})
\longrightarrow 0 $.
Physically, the $\rho \sim \epsilon $--limit corresponds to the
perception of each of the finite number of segments of the Koch curve [seen with finite resolution]  as a point set
with zero diameter.
As for all countable point sets, the dimension of the Koch curve
in this limit is zero.
For even greater uncertainties
(for $\rho \in ]\epsilon -1,\epsilon ]$)
one can hardly speak of a covering anymore,
since the uncertainty is of the same size as the resolution.
The argument then yields negative values of the operational
dimension.
It is certainly an interestin question,
whether these negative values can be given a conceptually
significant meaning.
\enditemize


We only note, that in this particular example, definition (2.12)
cannot be employed to define a dimension.
This shows that the operationalization of standard metrological
concepts on fractals is a subtle problem worth of careful
analysis in every specific case.


With a non--integer dimension of space--time the question as to
the dimensionality of physical units naturally arises.
Yet, it turns out that
 the
{\it dimension of physical units$^{4}$ }
[or parameters and constants] such as length, time, energy
and so on turns out to be a matter of definition.
All measurements are either digital in nature, such as
a click in an apparatus, or a comparison with a
standard already existing. The experimental outcome is always a relative
number, such as a fraction of some scale.
We therefore propose to {\it define}
a set of scale dimensions consistently [as has been
done for the SI]
and use these standards irrespective
of the operational dimension of the associated physical quantity.
\endsection
\section{F. Packing versus covering}

In many instances it is impossible to produce a covering of the
fractal structure,
when rigid bodies have to be used.
There, no overlaps are conceiveable.
In these cases, only a {\it packing}$^{22}$ would be possible,
leaving parts of space--time uncovered.
A packing $\{ P_i\} $ is defined as a set of sets, such that
there are only isolated points which are common to two or
more sets of $\{ P_i\}$ [see Fig. 6].


An experiment has already be performed$^{23}$, in which
thousands of ball bearings were being poured into spherical
flasks of various sizes; thereby gently shaking each flask
as it was being filled.
The densities $\sigma $ obtained are
$$\sigma \approx \eta -\epsilon N^{-1/3},\eqno(2.23)$$
where the packing fraction $\eta =$(filled volume) $/$
(all volume) and the parameter $\epsilon $ are constants
depending on the type of packing.
The right term of (2.23) is a surface term, which can be
significantly reduced and is therefore often neglected
in computer simulations with periodic boundary
conditions$^{24,25}$.

In three dimensions$^{22,23}$, the closest random packing turns
out to be a configuration with $\eta =0.6366$ and
$\epsilon =0.33$.
The loosest incompressible random packing is found with
$\eta =0.6000$ and $\epsilon =0.37$;
and for the cubic close packing one calculates
$\eta =0.7405$.


We propose here to (i) generate covering configurations from
packing configurations $\{ P_i\} $ $\longrightarrow \{ B_i\} $
by virtually extending the diameter $2r_i$ of [spherical] packing
constituents
$$P_i=\{x\in R^4\, :\;\vert x-x_{i,0}\vert \le r_i\} \eqno(2.24)$$
to the greatest diameter $2R_c$ of the circumcercle between
any neighboring balls [see Fig.7]:
$$B_i=\{ x\in R^4\, :\;\vert x-x_{i,0}\vert \le R_c\} ;\eqno(2.25)$$


(ii) to generalize these considerations concerning packings
of rigid bodies to noninteger dimensions.
In this way a ``hard--sphere'' covering of space and space--time
would make the definition of a dimensional parameter possible.
Hence, $\eta (\alpha )$ would depend on the dimension
of the geometric space.
This would provide an alternate operationalization of dimension, not
restricted to coverings.
\endsection
\endchapter
\chapter{III. Analytic applications of the operational dimension}
Measures are of importance in mathematics in two different ways.
They can be used to estimate the size of sets in number theory,
and they can be used to define integrals$^{16,26,27}$.
Although Cauchy's original quest was initiated by analytic
aspects of measures in connection with Fourier transforms,
little has been published on this second and equally important
application$^{28}$.
One reason is certainly the difficulties encountered in
the actual evaluation of integrals as compared to more
attractive applications in number theory.
\section{A. Upper dimensional bounds from quantum theory}
We consider perturbative calculations in continuous quantum
field theory, such as Quantum Electrodynamics (QED).
By evaluating transition matrix elements, integrals of
the following type are encountered$^{29}$:
$$J=\int K d\mu .\eqno(3.1)$$
Here $K$ stands for the integral kernel and $d\mu $ is some
integral measure, usually identified with the Hausdorff
measure $d^4x=dtdxdydz$ of $R^4$.
The type of kernel depends on the quantum theory.
For example, nonrelativistic static electrodynamics
yields kernels for which the associated integral $J$
diverges linearly.
Introduction of covariant QED improves the situation:
there the divergence of $J$ is of logarithmic type
and thus much weaker$^{30}$.
Several approaches have been proposed to overcome these
remaining infinities,
most of them trying to alter the structure of the
theory and also the kernels by some physical
cutoff such as the Planck length or by formal
arguments such as renormalization.



The following approach is very different.
In its center stands the question:
Given a particular model, for instance QED,
Which space--time structure renders a defined,
finite field theory ?
In other words:
Which measure and which associated dimension
has to be taken in order for the integrals and thus
the theory to be finite ?



As the infinities of QED are logarithmic in nature,
it turns out that these changes in measure may be
extremely small.
In particular, an identification of the integral
measure with the operationally defined measures
of section II yields a finite theory.



Since $K$ as well as $d\mu $ may be very complex in
their space--time representation and we shall be
only interested in the dimension [and not in their
explicit form, since this would require more information
on the space--time structure], it is of some advantage to consider
the Fourier transformation of the integral $J$.
By means of the convolution theorem, the product
in $J$ factorizes:
$$J= Kd\mu .\eqno(3.2)$$
The problematic ultraviolet (UV) structure of
conventional QED stems from kernels proportional to
$$K\propto k^{-4}.\eqno(3.3)$$
Thus in order for $J$ to be UV-finite,
$d\mu $ has to behave like
$k^{\alpha }$, with
$$\alpha <4.\eqno(3.4)$$
Since the dimension of the Fourier transform$^{28}$
$\mu (k)$ is equal to the dimension of the measure
in space--time $\mu (x)$, this requirement is satisfied
by all operationalistically defined measures provided
the Hausdorff dimension is less than four.
\endsection
\section{B. Lower dimensional bounds from experiment}
A modification of the integral measure changes all
predictions of perturbative quantum field theory.
On the other hand, the standard Hausdorff measure
$d^4k$ agrees quite well with experiment.
From this qualitative argument it can be inferred
that the change of measure has to be ``very small''.
Thus the dimension of the measure will not differ ``too much''
from four.
For the following quantitative analysis we shall calculate
corrections to the best known value of
quantum field theory, the anomalous magnetic moment
of the electron (g-2).
From the difference between the theoretic and experimental
value of (g-2), a value for the Hausdorff dimension of
space--time can be derived.



Since the mathematics of fractional integration and
differentiation
can be found in the literature [see for instance Refs. 16,31--33],
we shall just enumerate the results necessary for further calculations.
In what follows, then the following way:
assume a symmetric test function $f(k^2)$.
Then $d^\alpha k$ is defined as
$$\int f(k^2)d^\alpha k
=\int d^{\alpha -1}\Omega \int_0^\infty f(k^2)
k^{\alpha -1}dk
={2\pi^{\alpha /2}\over \Gamma (\alpha /2)} \int_0^\infty
f(k^2)k^{\alpha -1}dk.\eqno(3.5)$$
In particular, if $f(k^2)=[k^2+l^2]^{-n}$,
$$\int {d^\alpha k\over [k^2+l^2]^n }=
{\pi^{\alpha /2}l^{\alpha -2n}\Gamma (n-\alpha /2)\over
\Gamma (n)}.\eqno(3.6)$$
All these integrals are used for dimensional regularization
of continuous field theory [see for instance reference 32].
Their evaluation as well as their application is standard.
Since perturbative calculations are standard as well,
we shall not explicate the detailed calculation of the lowest
order contribution to the
anomalous magnetic moment of the electron, derived from
a graph shown in Fig. 8.
With $\alpha_f=e^2/4\pi $ standing for the fine structure
constant, the result is
$$(g-2)(\alpha )={\alpha_f\over 2\pi }\pi^{{\alpha \over 2}-2}
\Gamma (3-{\alpha \over 2}).\eqno(3.7)$$
For $(g-2)(\alpha =4)$ the expression reduces to the
well known standard value of $\alpha_f/2\pi $.
A theoretical deviation of $(g-2)$ from the
experimentally observed value can be defined as
$$\Delta g=(g-2)_{theor}\Bigm\vert_{\alpha =4} -
(g-2)_{exp}.\eqno(3.8)$$
We propose that such a deviation, if it exists, could also be explained
by changes of the dimension of the measure and
thus the Hausdorff dimension of space--time.
The present best values for $a_e=(g-2)/2$ are$^{34-38}$:
$$a_e^{exp}=1\,159\,652\,193(4)\times 10^{-12}$$
$$a_e^{theor}=1\,159\,652\,460(128)(43)\times 10^{-12}$$
For the theoretical value, corrections up to fourth
order in $\alpha_f$, as well as strong and weak contributions
have been taken into account.
It is interesting to note, that the difference
between  experimental and standard
theoretical value $a^{exp}_e-a^{theor}_e=
-267 (128) (43)\times 10^{-12}$
is larger than two standard deviations.
In fact, if this difference in the values
is assumed not merely statistical in nature, and
if they are not attributed to other factors
[such as apparatus dependencies$^{36,37}$ or
asymptotic behavior of the perturbation series],
one obtains to first order in $\Delta \alpha =4-\alpha_H$
$$\Delta \alpha ={2\pi \over \alpha_f}
{2\over C+\log (\pi )}\Delta g.\eqno(3.9)$$
Here, $C\sim 0.57722$ is Euler's constant. Insertion of
$\Delta g$ yields an estimate of the dimension
of space--time
$$\alpha_H =4-5.3(2.5)(0.8)\times 10^{-7}.\eqno(3.10)$$
\endsection
\section{D. Relativistic invariance of the measure}
As in nonrelativistic physics, covariant theories
assume Lorentz or Poincare invariance of the dimension
{\it a priori}.
Since the main objective of an operational definition
of the measure and the dimension is their determination
by experiment, the assumption of invariance under
coordinate transformation cannot be taken for granted any
longer.
The question arises if $\nu $ and $\alpha $ are invariants
and if it is possible to formulate covariant theories including
operational dimensions different from four.
This is by no means trivial, since other regulators
such as a spacial lattice spoils the covariance of relativistic
field theory and yields a preferred frame of reference
relative to which the lattice is at rest.



We shall consider an arbitrary covering $\lbrace B_i\rbrace $
realized in some frame of reference $I$.
For the evaluation of the diameters $(diam\;B_i)$,
the metric plays a decisive role.
For space--like coverings, the Minkowski metric
$g_{\mu \nu }=diag(+,+,+,-)$ yields a positive definite
metric.
If instead the covering is time--like, the
metric $-g_{\mu \nu }=diag(-,-,-,+)$ must be used.
Coverings on the light--cone have to be excluded, since they
render zero measure.
With the Minkowski metric, the diameter $(diam\; B_i)$
is an invariant under the proper Lorentz group.
Since the dimension is [for space--like and time--like
regions separately] invariant with respect to
the variation to positive definite equivalent metrices$^{39}$,
it is also an invariant under proper Lorentz transformations,
leaving out reflections from space--like to time--like surfaces.
However, the  resolution $\delta $ depends on the experimental
setup and is {\it not} relativistically invariant.
This leaves us with the situation that, although formally
the dimension of space--time is invariant, the particular
experiment is not.
\endsection
\section{E. Hausdorff versus operational dimension}
As has been already pointed out, one could take the viewpoint, that an extrinsic
quantity and thus the Hausdorff dimension is ``
the real thing'', if such a thing has a meaning whatsoever.
Since its value will probably never be known, we are relegated
to what we can measure.
However, throughout this investigation we have
encountered two different approaches to measure
the dimension of space--time:
\itemize{1cm}
\litem{(i)}the algebraic approach, utilizing
the umklapp property (2.12) of the modified Hausdorff
measure (2.1),
yielding a dimension $\alpha_{op}$; and
\litem{(ii)}the analytic approach, yielding an
approximation to the Hausdorff dimension of
space--time via the calculation of sensitive
radiative corrections.
The dimensional values obtained in that way
bear uncertainties similar to the algebraically obtained
values, and are operational as well.
\enditemize



It is possible to establish a criterion to answer
the question whether the Hausdorff dimension of space--time
is four:
Suppose $\alpha_H$ is the Hausdorff dimension of space--time,
and $\Delta \alpha_{op}$ is the uncertainty in the
determination of the operational
dimension [this should not be confused with the expression in
(3.10)].
Then a deviation of the external dimension from its
ideal value of four can be experimentally
observed, if the following condition is satisfied:
$$\mid 4-\alpha_H\mid >\Delta \alpha_{op}.\eqno(3.11)$$
\endsection
\endchapter
\chapter{IV. Conclusion}
Throughout this paper it has been avoided on purpose to speculate
on reasons why the Hausdorff dimension of space--time should differ
from four [for an interesting suggestion, see for instance Ref. 40].
In particular, no specific scaling of
$\alpha (\delta )$ has been proposed, since this would
require a dynamical model.
The point rather is:  once the dimension is
measureable, then why should it be exactly an integer and four?



Several criteria have been introduced for
operational definitions of the dimension of space--time.
The existing mathematical concepts of measure had to
be adopted mainly to account for the finite resolution
available in experiments.
As could have been expected, there will always
be some uncertainty in the determination of the
dimension.
Due to the nature of physically realizeable coverings,
the operational dimension will be smaller than the
Hausdorff dimension of space--time .



A smaller Hausdorff dimension of space--time
would also result in the resolution of ultraviolet
divergencies of continuous field theory.
Furthermore, it would modify all field theoretic
calculations. Although most transition matrix elements are
insensitive with respect to dimensional variations,
comparison between the best experimental values for the
electron anomalous magnetic moment with
theoretical predictions gives the value
$\alpha_H=4-5.3(2.5)(0.8)\times 10^{-7}$.




We pass the question for further confirmation of noninteger
dimensionality of space--time to experiment.
Although this is not everyday laboratory work,
it certainly poses new and interesting challenges.


It is certainly clear to us,
that parts of this paper are not presentations of results
of research but rather should be valued as
outlining a research programme.
We think, that it very well fulfills the
definitions of a progressive scientific
research programme in the sense of Lakatos$^{42}$.
$\bigcirc $


This project was supported by the Austrian Federal Ministry
for Science and Research [BMWF], project number 19.153/3--26/85.
One of us (K.S.) gratefully acknowledges an invitation
to BiBoS [Bielefeld-Bochum-Stochastic] while working on parts of
this paper.


\endchapter
\chapter{References}
\itemize{1cm}
\litem{[1]}B.Russell,``A History of Western Philosophy''
(Allen and Unwin, London 1946)
\litem{[2]}G.Sarton,``History of Science'', vol.1
(Norton, New York 1959)
\litem{[3]}O.Neugebauer,``A History of Ancient Mathematical
Astronomy'' (Springer, New York 1975)
\litem{[4]}see for instance J.D.Barrow,
Phil.Trans.R.Soc. London A310, 337 (1983),
who mentiones Kant's early efforts on the dimensionality
problem
\litem{[5]}W.Hurewicz and H.Wallmann,
``Dimension Theory'' (Princeton University press 1948), p.4
\litem{[6]}R.J.Adler, ``The Geometry of Random Fields''
(Wiley and Sons 1981), p.188
\litem{[7]}F.Hausdorff, Math. Ann. 79, 157 (1918),
see also H. Federer, ``Geometric Measure Theory''
(Springer, Berlin 1969)
\litem{[8]}J.D.Farmer, E.Ott and J.A.Yorke,
Physica 7D, 153 (1983)
\litem{[9]}J.E.Hutchinson, Indiana Univ. Math. J. 30, 713 (1981)
\litem{[10]}B.Misra, Proc.Natl.Acad.Sci. (USA) 75, 1627 (1978)
\litem{[11]}C.M.Lockhart,
``Time Operators in Classical and Quantum Systems''
(Ph.D. thesis, University of Texas at Austin 1981)
\litem{[12]}K. Svozil,
``Operationalistic perception of space--time
in a quantum medium'' (TUW- preprint, 1985)
\litem{[13]}B.B.Mandelbrot,
``Fractals: Form, Chance and Dimension''
(Freeman, San Francisco 1977)
\litem{[14]}H.B.Nielsen, NORDITA preprint 1971
(unpublished)
\litem{[15]}A.B.Kraemmer, H.B.Nielsen and H.C.Tze,
Nucl. Phys. B81, 145 (1974)
\litem{[16]}F.H.Stillinger, J.Math.Phys. 18, 1224 (1977)
\litem{[17]}L.F.Abbot and M.B.Wise, Am. J. Phys. 49, 37 (1981)
, and E.Campesino-Romeo, J.C.D'Olivo and
M.Socolovsky, Phys. Lett. 89A, 321 (1982)
\litem{[18]}P.C.W.Davies, in
``Quantum Gravity 2'', ed. by C.J.Isham et al.
(Claderon Press, Oxford 1981), p.207
\litem{[19]}C.J.Isham, in
``Quantum Theory of Gravity'', ed. by St. Christensen
(Adam Hilger Ltd., Bristol    ), p.313
\litem{[20]}G.N.Ord, J. Phys. A16, 1869 (1983)
\litem{[21]}In view of the present uncertainty
of the total energy of the universe it
seems to be premature to put a number on that lower
limit beyond which the concept of a dimension
of space--time certainly looses its meaning.
\litem{[22]}C.A.Rogers, ``Packing and Covering'' (Cambridge University
Press, Cambridge 1964)
\litem{[23]}H.S.M.Coxeter, ``Introduction to Geometry'' (Wiley and Sons,
New York 1961, 1969)
\litem{[24]}J.G.Berrymann, Phys. Rev. A27,1053 (1983)
\litem{[25]}W.S.Jodrey and E.M. Torey, Phys. Rev. D32, 2347 (1985)
\litem{[26]}C.A.Rogers,
``Hausdorff Measures''
(Cambridge University Press 1970), p.147 - 168
\litem{[27]}K.Svozil, ``Quantum field theory on fractal space--time'',
Technical University Vienna preprint, September 1985
\litem{[28]}see for instance J.--P. Kahane and
R.Salem,
``Ensembles Parfaits et S\'eries Trigonom\'e\-triques''
(Hermann, Paris 1963)
\litem{[29]}E.B.Manoukian,
``Renormalization''
(Academic Press, New York 1983)
\litem{[30]}V.F.Weisskopf, Phys.Rev. 56, 72 (1939)
\litem{[31]}P.L.Butzer and R.L.Nessel,
``Fourier Analysis and Approximation''
(Birkhauser, Stuttgart 1971), p. 391
and K. B. Oldham and J. Spanier,
``The Fractional Calculus'' (Academic Press, New York 1974)
\litem{[32]}G.'tHooft and M. Veltman, Nucl. Phys. B44, 189 (1972)
\litem{[33]}G.Leibbrand, Rev. Mod. Phys. 47, 849 (1975)
\litem{[34]}the experimental value of $a_e^{exp}$
was announced by R.S. Van Dyck at the Ninth
International Conference on Atomic Physics
(ICAP - IX, Seattle 1984)
publ. in ``Atomic Physics 9'', ed. by R.S. Van Dyck and
E. Norval Fortson (World Scientific, Singapore 1984)
\litem{[35]}T. Kinoshita and W.B. Lindquist,
Phys. Rev. Lett. 47, 1573 (1981); Phys.
Rev. D27, 867 (1983)
and T. Kinoshita and J. Sapirstein, in ``Atomic Physics 9'',
ed. by R. S. Van Dyck and E. Norval Fortson (World
Scientific, Singapore 1984)
\litem{[36]}K.Svozil, Phys. Rev. Lett. 54, 742 (1985)
\litem{[37]}L. S. Brown et al., Phys. Rev. Lett. 55, 44 (1985)
\litem{[38]}A.Zeilinger and K.Svozil,
Phys. Rev. Lett. 54, 2553 (1985)
\litem{[39]}E.Hewitt and K.Stromberg,
``Real and Abstract Analysis'' (Springer, New York 1965)
; a sketch of the proof goes as follows: define two metrices
$d_1$ and $d_2$ to be equivalent, if there exists two positive
real numbers $c_1$ and $c_2$ such that for two arbitrary
elements $x$ and $y$ of a compact set $E$ the following
relation holds:
$$c_1d_1(x,y)\leq d_2(x,y)\leq c_2d_1(x,y).$$
Now, from the definition of the Hausdorff measure (1.1) follows, that
$$c_1^\alpha \mu_{H,d_1}(\alpha )\leq \mu_{H,d_2}(\alpha )
\leq c_2^\alpha \mu_{H,d_1}(\alpha );$$
assume $\alpha $ greater or smaller than $\alpha_H$, then if $\mu_{H,d_1}(\alpha )=0$ or
$\infty $, so will be $\mu_{H,d_2}(\alpha )$, from
which property follows that the Hausdorff dimension
$\alpha_H$ will be invariant with respect to variation
of the metric $d(\cdot )=(diam\cdot )$.
For a more general study see E. Ott, W. D. Withers and J. A.
Yorke, J. Stat. Phys. 36, 687 (1984)
\litem{[40]}L.Crane and L.Smolin, Yale University preprints
YTP 85-08 and YTP 85-09
\litem{[41]}D.Dubois and H. Prade,
``Fuzzy Sets and Systems'' (Academic Press, New York 1980)
\litem{[42]}I. Lakatos,
``The Methology of Scientific Research Programmes''
(Cambridge University Press, Cambridge 1978)
\enditemize
\endchapter
\chapter{Figure captions}
\itemize{1.5cm}
\litem{Fig. 1:}One of the many possible coverings
$\lbrace B_i\rbrace $ of a string $E$.
The sets $B_i$ may overlap.
\litem{Fig. 2:}Definition of a stochastic covering.
The state is assumed gaussian and the area covered
depends on the state width as well as on the parameter
$p$ in Eq. (2.3): the smaller $p$ is, the more area
is covered.
\litem{Fig. 3:}A smeared out Heavyside function may serve
as a model for the functional behavior of the
operational measure $\nu $.
\litem{Fig. 4:}Cavity with resonant mode and HF-source
\litem{Fig. 5:}The Koch curve is drawn with increasing
resolution $\delta $: more and more structure appears.
\litem{Fig. 6:}Packing of the set $E$ from Fig. 1
\litem{Fig. 7:}Covering generated from the packing
of Fig. 6
\litem{Fig. 8:}Lowest order vertex correction diagram
contributing to the electron anomalous magnetic moment.
\enditemize
\endchapter
\endarticle
\end
