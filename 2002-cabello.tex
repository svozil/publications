%%tth:\begin{html}<LINK REL=STYLESHEET HREF="/~svozil/ssh.css">\end{html}
\documentclass[prl,preprint]{revtex4}
%\documentclass[pra,twocolumn,showpacs,showkeys]{revtex4}
%\usepackage{graphicx}
%\documentstyle[amsfonts]{article}
%\RequirePackage{times}
%\RequirePackage{courier}
%\RequirePackage{mathptm}
%\renewcommand{\baselinestretch}{1.3}
\begin{document}

%\def\frak{\cal }
%\def\Bbb{\bf }
%\sloppy



\title{Comment on violating {B}ell's inequality beyond {C}irel'son's bound}
\author{G\"unther Krenn}
\email{gunther.krenn@philips.com}
\affiliation{Philips Austria, CDS Development,
Gutheil Schoder-Gasse 17, A-1230 Vienna, Austria}
\author{Karl Svozil}
\email{svozil@tuwien.ac.at}
\homepage{http://tph.tuwien.ac.at/~svozil}
\affiliation{Institut f\"ur Theoretische Physik, University of Technology Vienna,
Wiedner Hauptstra\ss e 8-10/136, A-1040 Vienna, Austria}


%\begin{abstract}
%The proposal of violating {B}ell's inequality beyond {C}irel'son's bound is reviewed from three perspectives: (i) historical, (ii) using post-selection in classical systems, and (iii) a suggestion to look into the "higher harmonics" of the correlation function.
%\end{abstract}

%\pacs{03.65.Tad,03.65.U}
%\keywords{correlation polytopes, probability theory}

\maketitle

In a recent Letter \cite{cabello-02a}, Ad\'{a}n Cabello pointed out the
possibility to violate the Clauser, Horne, Shimony, and Holt (CHSH)
inequality beyond {C}irel'son's bound.
The correlation of a {\em postselected} subsystem of a three-qubit system
prepared in a Greenberger, Horne, and Zeilinger (GHZ) state
allows the maximal violation of the CHSH inequality.

However, we do not see any way to use the author's scheme to test
quantum nonlocality, since, by using postselection, even classical systems can
violate the CHSH inequality up to the maximal value of 4.
It would therefore be desirable to explicate the advantages of quantum
systems over classical ones.

Moreover, for the GHZ-state, the nonlocality can be tested experimentally
by a {\em single} experiment on all three particles
leading to a complete contradiction with classical locality.
Analogously, in the two-particle case, a violation of the CHSH inequality by 4
would result in a complete contradiction with classical locality which
could also be tested by a single experiment.
However, postselection (as considered in \cite{cabello-02a}) requiring measurement of the qubit in the $z$-direction excludes knowledge of the qubit in the $x$-direction,
thereby making impossible direct evaluation of Eq. (10) in \cite{cabello-02a}.
The proposed experimental testing procedures presented later
are of mere statistical nature and therefore do not utilize the whole phenomenology
of a maximal violation of the CHSH inequality.


%\bibliography{svozil}
%\bibliographystyle{apsrev}
%\bibliographystyle{unsrt}
%\bibliographystyle{plain}

\begin{thebibliography}{1}
\expandafter\ifx\csname natexlab\endcsname\relax\def\natexlab#1{#1}\fi
\expandafter\ifx\csname bibnamefont\endcsname\relax
  \def\bibnamefont#1{#1}\fi
\expandafter\ifx\csname bibfnamefont\endcsname\relax
  \def\bibfnamefont#1{#1}\fi
\expandafter\ifx\csname citenamefont\endcsname\relax
  \def\citenamefont#1{#1}\fi
\expandafter\ifx\csname url\endcsname\relax
  \def\url#1{\texttt{#1}}\fi
\expandafter\ifx\csname urlprefix\endcsname\relax\def\urlprefix{URL }\fi
\providecommand{\bibinfo}[2]{#2}
\providecommand{\eprint}[2][]{\url{#2}}

\bibitem[{\citenamefont{Cabello}(2002)}]{cabello-02a}
\bibinfo{author}{\bibfnamefont{A.}~\bibnamefont{Cabello}},
  \bibinfo{journal}{Physical Review Letters} \textbf{\bibinfo{volume}{88}},
  \bibinfo{pages}{060403} (\bibinfo{year}{2002}), \eprint{quant-ph/0108084}.

\end{thebibliography}

\end{document}
