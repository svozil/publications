%\documentclass[pra,showpacs,showkeys,amsfonts,amsmath,twocolumn]{revtex4}
\documentclass[amsmath,blue,handout]{beamer}
%\documentclass[pra,showpacs,showkeys,amsfonts]{revtex4}

\usepackage{beamerthemeshadow}
%\usepackage[dark]{beamerthemesidebar}
%\usepackage[headheight=24pt,footheight=12pt]{beamerthemesplit}
%\usepackage{beamerthemesplit}
%\usepackage[bar]{beamerthemetree}
\usepackage{graphicx}
\usepackage{pgf}

\RequirePackage[german]{babel}
\selectlanguage{german}
\RequirePackage[isolatin]{inputenc}

\pgfdeclareimage[height=0.5cm]{logo}{tu-logo}
\logo{\pgfuseimage{logo}}
\beamertemplatetriangleitem
\begin{document}
\title{\bf Quantum Cryptography}
\subtitle{http://tph.tuwien.ac.at/\~{}svozil/publ/2005-qcrypt-pres.pdf}
%\subtitle{http://www.arxiv.org/abs/quant-ph/0406014}
\author{Karl Svozil}
\institute{Institut f\"ur Theoretische Physik, University of Technology Vienna, \\
Wiedner Hauptstra\ss e 8-10/136, A-1040 Vienna, Austria\\
svozil@tuwien.ac.at\\
% {\tiny Disclaimer: Die hier vertretenen Meinungen des Autors verstehen sich als Diskussionsbeitr�ge und decken sich nicht notwendigerweise mit den Positionen der Technischen Universit�t Wien oder deren Vertreter.}
}
\date{16. 3. 2005}
\maketitle

%\frame{\tableofcontents}



\section{History \& References}
\subsection{References}

\frame{
\frametitle{References}



\begin{itemize}

\item<+-> [WIE83]
Stephen Wiesner, {\em ``Conjugate coding,''} Sigact News, 15, 78-88 (1983)
[manuscript written {\it circa} 1970]

\item<+-> [BBBSS92]
Charles H. Bennett and F. Bessette and G. Brassard and L. Salvail and J. Smolin,
{\em ``Experimental Quantum Cryptography,''}
Journal of Cryptology, 5, 3-28 (1992)

\item<+->
Charles H. Bennett and Gilles Brassard and Artur K. Ekert,
{\em ``Quantum Cryptography,'' }
Scientific American, 267, 50-57 (1992)

\item<+-> [GRTZ02]
Nicolas Gisin, Gr{\'e}goire Ribordy, Wolfgang Tittel, and Hugo Zbinden,
{\em ``Quantum cryptography,''} Rev. Mod. Phys. 74, 145-195 (2002)
http://link.aps.org/abstract/RMP/v74/p145

\item<+-> David Mermin,
{\em ``Lecture Notes on Quantum Computation,''}
[Cornell University, Physics 481-681, CS 483; Spring, 2005]\\
%http://people.ccmr.cornell.edu/\~{}mermin/qcomp/CS483.html \\
http://people.ccmr.cornell.edu/\~{}mermin/qcomp/chap6.pdf

\end{itemize}
 }


\subsection{History}
\frame{
\frametitle{History}

\begin{itemize}
\item<+-> [1970]
Stephen Wiesner, {\em ``Conjugate coding:''}
noisy transmission of two or more ``complementary messages'' by using single photons in
two or more complementary polarization directions/bases.

\item<+-> [1984]
BB84 Protocol: key growing via quantum channel \& additional classical bidirectional communication channel

\item<+-> [1991]
EPR-Ekert protocol: maximally entangled state, three complementary polarization directions;
additional security confirmation by violation of Bell-type inequality
through data which cannot be directly used for coding


\end{itemize}
 }


\section{Protocols}

\subsection{Wiesner's conjugate coding scheme}
\frame{
\frametitle{Wiesner's conjugate coding scheme}
\begin{center}
\includegraphics[height=7cm]{2005-qcrypt-pres-wiesner.pdf}
\\
from [WIE83](ca. 1970)
\end{center}
}

\subsection{BB84 Protocol}
\frame{
\frametitle{BB84 Protocol}
%$\longrightarrow$ time\\
\includegraphics[height=8cm]{2005-qcrypt-pres-BBBSS92.pdf}
from [BBBSS92]
}

\subsection{EPR-Ekert protocol}
\frame{
\frametitle{EPR-Ekert protocol}
Parametrization of
$
\vert \psi \rangle =
x\vert + \rangle +
y\vert - \rangle
$
by two angles
$0\le \theta \le \pi$ (azimutal)
and
$0\le \varphi \le 2\pi$.

Let the expectation value measured by a pair of particles along the directions $a_i$ and $b_j$ be
$
E(a_i,b_j) =
P_{++}(a_i,b_j)+
P_{--}(a_i,b_j)-
P_{+-}(a_i,b_j)-
P_{-+}(a_i,b_j)$.
Consider the Clauser-Horne-Shimony-Halt (CHSH) term
$
S= E(a_1,b_1) - E(a_1,b_3) + E(a_3,b_1)+ E(a_3,b_3)
$.

With
the six measurement directions corresponding to $\varphi =0$ (for all six),
and
$\theta_1^a =0$,
$\theta_2^a =\pi /4$,
$\theta_3^a =\pi /2$,
$\theta_1^b =\pi /4$,
$\theta_2^b =\pi /2$, and
$\theta_3^b =3\pi /4$ (three per side),
$S=-2\sqrt{2}$ is maximally violated by the Tsirelson bound.

{\bf Constant monitoring of $S$ certifies the absence of an eavesdropper.}

}

\subsection{Interferometric protocols}
\frame{
\frametitle{Interferometric protocols}
%$\longrightarrow$ time\\
\includegraphics[height=7.5cm]{2005-qcrypt-pres-GRTZ022.pdf}
from [GRTZ02]
}

\section{Secrecy protection principles}
\subsection{Single particle events}
\frame{
\frametitle{Single particle production, manipulation \& detection}
It is essential to use single particle states, otherwise ``Eve'' could
eavesdrop on the extra particles.
}

\subsection{Complementarity}
\frame{
\frametitle{Complementarity}
Eavesdropping randomizes the state transmitted from Alice to Bob.
}

\subsection{No-cloning (no-copy) theorem}
\frame{
\frametitle{No-cloning (no-copy) theorem}

\begin{itemize}
\item<+->
Ideally, a perfect Qcopy device $A$, acting upon an arbitrary state $\psi$
and some arbitrary blank state $b$, would do this:\\
$$
\psi
\otimes
\vert b\rangle
\otimes
\vert A_i\rangle
\longrightarrow
\psi
\otimes
\psi
\otimes
\vert A_f\rangle .
$$

\item<+->
Suppose it would copy the two ``quasi-classical'' state
``$+$''
and
``$-$'' accordingly:
$$
\vert +,b,  A_i\rangle
\longrightarrow
\vert +,+,  A_f\rangle
,\qquad
\vert -,b,  A_i\rangle
\longrightarrow
\vert -,-, A_f\rangle.
$$

\item<+->
By the linearity of quantum mechanics, the state
$
{1\over \sqrt{2}}(
\vert + \rangle +
\vert - \rangle  )
$
is copied according to
$$
{1\over \sqrt{2}}(
\vert + \rangle  +
\vert - \rangle  )
\otimes
\vert
b,
 A_i\rangle
\longrightarrow
{1\over \sqrt{2}}(
\vert + ,+, A_f\rangle +
\vert - ,-,A_f \rangle  )
$$
$$
\qquad
\qquad
\neq
{1\over \sqrt{2}}(
\vert + \rangle  +
\vert - \rangle  )
\otimes
{1\over \sqrt{2}}(
\vert + \rangle  +
\vert - \rangle  )
\otimes
 \vert
 A_i\rangle
.
$$


\end{itemize}
}

\subsection{Man-in-the-middle attack}


\frame{
\frametitle{Man-in-the-middle attack using both the classical \& quantum channels}
%TexCad Options
%\grade{\off}
%\emlines{\off}
%\beziermacro{\on}
%\reduce{\on}
%\snapping{\off}
%\quality{2.00}
%\graddiff{0.01}
%\snapasp{1}
%\zoom{1.00}
\unitlength 0.70mm
\linethickness{0.4pt}
\begin{picture}(155.00,45.00)
\put(24.98,9.98){\line(1,0){20.06}}
\put(34.96,12.48){\circle{4.99}}
\put(35.46,12.48){\circle{4.99}}
\put(24.98,9.68){\line(1,0){20.06}}
\put(24.98,19.96){\line(1,0){20.06}}
\put(34.96,22.46){\circle{4.99}}
\put(34.96,22.46){\makebox(0,0)[cc]{c}}
\put(35.06,12.48){\makebox(0,0)[cc]{q}}
\put(45.00,5.00){\dashbox{1.33}(20.00,20.00)[cc]{}}
\put(135.02,9.98){\line(-1,0){20.06}}
\put(125.04,12.48){\circle{4.99}}
\put(124.54,12.48){\circle{4.99}}
\put(135.02,9.68){\line(-1,0){20.06}}
\put(135.02,19.96){\line(-1,0){20.06}}
\put(125.04,22.46){\circle{4.99}}
\put(125.04,22.46){\makebox(0,0)[cc]{c}}
\put(124.94,12.48){\makebox(0,0)[cc]{q}}
\put(95.00,5.00){\dashbox{1.33}(20.00,20.00)[cc]{}}
\put(40.00,0.00){\framebox(80.00,30.00)[cc]{}}
\put(42.33,35.00){\makebox(0,0)[lc]{box-in-the-middle}}
\put(45.00,27.33){\makebox(0,0)[lc]{fake ``Bob''}}
\put(95.33,27.33){\makebox(0,0)[lc]{fake ``Alice''}}
\put(80.00,45.00){\makebox(0,0)[cc]{Eve}}
\put(135.00,5.00){\dashbox{1.33}(20.00,20.00)[cc]{}}
\put(135.33,27.33){\makebox(0,0)[lc]{Bob}}
\put(5.00,5.00){\dashbox{1.33}(20.00,20.00)[cc]{}}
\put(5.00,27.33){\makebox(0,0)[lc]{Alice}}
\put(65.00,20.00){\line(1,0){30.00}}
\put(80.00,20.00){\line(0,1){20.00}}
\put(80.00,15.00){\makebox(0,0)[ct]{copy or}}
\put(80.00,10.00){\makebox(0,0)[ct]{misinform}}
\end{picture}
\begin{center}
from http://arxiv.org/abs/quant-ph/0501062
\end{center}
}

\frame[shrink=1]{   {\small
\frametitle{Man-in-the-middle attack using both the classical \& quantum channels}

\begin{itemize}
\item<+->
Compare: ``Standard quantum key distribution protocols are provably secure against eavesdropping attacks, if quantum theory is correct.''
(from http://arxiv.org/abs/quant-ph/0405101).

\item<+->
To: ``The need for the public (non-quantum) channel in this scheme to be immune to active eavesdropping can be
relaxed if the Alice and Bob have agreed beforehand on a small secret [[classical cryptographic]]  key,.."
(from BB84: C. H. Bennett and G. Brassard, in Proceedings of the IEEE International Conference on Computers, Systems, and Signal Processing, Bangalore, India (IEEE Computer Society Press, 1984), pp. 175-179.)

\item<+->
``In accordance with our general philosophy that QKD forms a part of an overall cryptographic architecture, and not an
entirely novel architecture of its own, the DARPA Quantum Network currently employs the standardized authentication
mechanisms built into the Internet security architecture (IPsec), and in particular those provided by the Internet Key
Exchange (IKE) protocol.''
(from http://arxiv.org/abs/quant-ph/0503058)
\end{itemize}
}}

\section{Realizations}

\subsection{Techniques \& gadgets}
\frame{
\frametitle{Techniques \& gadgets}
\begin{itemize}


\item<+->
Photon sources: faint laser pulses,
photon pairs generated by parametric downconversion, photon guns,
 $\ldots$

\item<+->
Quantum channels: single-mode fibers, free-space links, $\ldots$

\item<+->
Single-photon detection: photon counters,  $\ldots$

\item<+->
(Quantum) Random number generators: calcite prism,
 $\ldots$
\end{itemize}
}

\subsection{1989 IBM Yorktown Heights}
\frame{
\frametitle{1989 IBM Yorktown Heights}
\includegraphics[height=7.5cm]{2005-qcrypt-pres-ibm.pdf}
}

\subsection{1993 Lake Geneva \& 2004 Vienna}
\frame{
\frametitle{1993 Lake Geneva}
\includegraphics[height=7.5cm]{2005-qcrypt-pres-genevaex.pdf}
}

\frame{
\frametitle{2004 Vienna}
\includegraphics[height=7.5cm]{2005-qcrypt-pres-vienna.pdf}
}

\subsection{2003-present DARPA Network Boston}
\frame{
\frametitle{2003-present DARPA Network Boston}
\includegraphics[height=7.5cm]{2005-qcrypt-pres-DARPA.pdf}
}

\frame{
\frametitle{ }
\begin{center}
{\bf Thank you for your attention!}
\end{center}
}

\end{document}


