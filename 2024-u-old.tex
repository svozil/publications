\newif\ifws
%\wstrue
\ifws

\documentclass{article}

\usepackage{graphicx}        % standard LaTeX graphics tool
\usepackage[dvipsnames]{xcolor}

\usepackage{hyperref}
\hypersetup{
    colorlinks,
    linkcolor={blue!80!black},
    citecolor={red!75!black},
    urlcolor={blue!80!black}
}

% Damit die Verwendung der deutschen Sprache nicht ganz so umst\"andlich wird,
% sollte man die folgenden Pakete einbinden:


%German
%\usepackage[latin1]{inputenc}% erm\"oglich die direkte Eingabe der Umlaute
%\usepackage[T1]{fontenc} % das Trennen der Umlaute
%\usepackage{ngerman} % hiermit werden deutsche Bezeichnungen genutzt und
                     % die W\"orter werden anhand der neue Rechtschreibung
                     % automatisch getrennt.
&\title{Ridding Unitarity Through Nested Wigner's Friends}
\title{From Unitarity to Irreversibility: Infinite Tensor Products and Nested Wigner's Friends}
%Nesting of Wigner's friends gets rid of unitarity  at the end
%"Breaking Unitarity Through Nested Wigner's Friend Scenarios"
%"Unitarity Violation in Extended Wigner's Friend Experiments"
%"Nested Wigner's Friend Paradigms and the Demise of Unitarity"
%"Exploring Unitarity Breakdown via Nested Wigner's Friend Models"
%"Challenging Unitarity with Nested Wigner's Friend Scenarios"

\author{Karl Svozil \\
        Institute for Theoretical Physics,
TU Wien,  \\
Wiedner Hauptstrasse 8-10/136,
1040 Vienna,  Austria
        }

\date{\today}
% Hinweis: \title{um was auch immer es geht}, \author{wer es auch immer
% geschrieben hat} und  \date{wann auch immer das war} k\"onnen vor
% oder nach dem  Kommando \begin{document} stehen
% Aber der \maketitle Befehl mu\ss{} nach dem \begin{document} Kommando stehen!
\begin{document}

\maketitle


\begin{abstract}
\end{abstract}


\else
\PassOptionsToPackage{dvipsnames}{xcolor}
\documentclass[
reprint,
%   preprint,
 % twocolumn,
 %superscriptaddress,
 %groupedaddress,
 %unsortedaddress,
 %runinaddress,
 %frontmatterverbose,
 showpacs,
 showkeys,
 preprintnumbers,
 %nofootinbib,
 %nobibnotes,
 %bibnotes,
 amsmath,amssymb,
 aps,
 % prl,
  pra,
 % prb,
 % rmp,
 %prstab,
 %prstper,
  longbibliography,
 floatfix,
 %lengthcheck,
 ]{revtex4-2}

%\usepackage{cdmtcs-pdf}

\usepackage{mathptmx}% http://ctan.org/pkg/mathptmx



\usepackage{amssymb,amsthm,amsmath}

\usepackage{mathbbol}

\usepackage{tikz}
\usetikzlibrary{calc,math}
\usepackage {pgfplots}
\pgfplotsset {compat=1.8}
\usepackage{graphicx}% Include figure files
%\usepackage{url}

\usepackage{xcolor}

\usepackage{hyperref}
\hypersetup{
    colorlinks,
    linkcolor={blue},
    citecolor={red!75!black},
    urlcolor={blue}
}


\begin{document}


\title{From Unitarity to Irreversibility: Infinite Tensor Products and Nested Wigner's Friends}

\author{Karl Svozil}
\email{svozil@tuwien.ac.at}
\homepage{http://tph.tuwien.ac.at/~svozil}

\affiliation{Institute for Theoretical Physics,
TU Wien,
Wiedner Hauptstrasse 8-10/136,
1040 Vienna,  Austria}



\date{\today}

\begin{abstract}
This paper explores the transition from unitary, reversible quantum processes to non-unitary, irreversible measurements through the lens of infinite tensor products and nested Wigner's friend scenarios. We demonstrate that infinite tensor products of Hilbert spaces can violate unitarity due to issues with normalization, inner product definition, and the behavior of bounded operators. These mathematical constructs are then applied to a conceptualization of nested Wigner's friend experiments, where repeated entanglement leads to large and potentially infinite tensor products. We argue that this framework provides a formal mechanism for the emergence of non-unitary, irreversible measurements from underlying unitary evolution. The paper draws parallels between this quantum mechanical phenomenon and concepts from real analysis, recursive mathematics, and statistical physics. We conclude by discussing the practical implications of this approach for understanding quantum measurement and decoherence, emphasizing both the formal mathematical argument and its finite, pragmatic interpretations in experimental contexts.
\end{abstract}


%\pacs{03.65.Aa, 03.65.Ta, 03.65.Ud, 03.67.-a}
\keywords{Quantum measurement,
Unitarity,
Infinite tensor products,
Nested Wigner's friend,
Quantum decoherence,
Von Neumann algebras,
Irreversibility,
Quantum entanglement}
%\preprint{CDMTCS preprint nr. x}

\maketitle

\newpage
\fi



\section{Introduction}

%\section{Interactions from measurement create entanglement}

In the von Neumann scheme for ideal quantum measurement~\cite{landau-1927}
the `object' is (prepared) in an initial state $\vert \Psi \rangle$.
With respect to a `mismatching' (relative to that preparation) context---or, equivalenntly,
orthonormal basis or maximal operator~\cite[Satz~8]{v-neumann-31} (\cite[Theorem~1, \S~84]{halmos-vs})---$\vert \psi \rangle$ is in
a coherent superposition (linear combination)
$\vert \psi \rangle =  \sum_{i=1}^n a_i \vert \psi_i \rangle$
of (basis) elements $\vert \psi_i\rangle$
of that context.
The `measurement apparatus' can be described by another state $\vert \varphi \rangle$ which,
relative to a suitable basis
$\{ \vert \varphi_1\rangle, \ldots ,\vert \varphi_n\rangle \}$,
can also be written as a coherent superposition
$\vert \varphi \rangle = \sum_{j=1}^n b_j \vert \varphi_j \rangle$.
If there is an interaction between the `object' and the `measurement apparatus' the combined state $\vert \Gamma \rangle$
is a non-factorizable ($c_{ij}\neq a_ib_j$) tensor product
\begin{equation}
\vert \Psi \rangle
= \sum_{i,j=1}^n c_{ij} \vert \psi_i \rangle \otimes \vert \varphi_j \rangle
= \sum_{i,j=1}^n c_{ij} \vert \psi_i  \varphi_j \rangle
.
\label{2024-u-vNsiqm}
\end{equation}
From now on, when referring to `object' and the `measurement apparatus', apostrophes will be omitted.
Since in entangled systems individuality is traded for relationality between `individual components',
any conceptualization of a Heisenberg cut  between these entangled constituents is a classical notion that may be maintained
for all practical purposes (FAPP) but strictly speaking, is applicable~\cite{bell-a}.

\section{Infinite tensor products}

We can, in principle, recursively repeat the von Neumann scheme for ideal quantum measurement~(\ref{2024-u-vNsiqm})
and, as we add more factors, arrive at  ever (exponential) larger  product spaces.
To obtain a Hilbert space, we take the closure of this space with respect to a suitable norm,
derived from the inner product.
This can also be conceptualized by taking the `double dual', or,  more precisely, the dual of the vector space of
all bilinear forms on the vector spaces entering the product~\cite{halmos-vs}.

%\begin{equation}
%\vert \Gamma \rangle  =
%\lim_{k \rightarrow \infty}
%\vert \Gamma_k \rangle
%=
%\lim_{k \rightarrow \infty}
%\sum_{i_1,\ldots ,i_k=1}^n c_{i_1\cdots  i_k} \vert \psi_{i_1} \cdots  \psi_{i_k} \rangle
%.
%\label{2024-u-vNsiqmlimit}
%\end{equation}




\subsection{Elementary Tensors as Products of Basis Vectors}

Given a countable collection of Hilbert spaces \(\{ \mathcal{H}_n \}_{n=1}^\infty\),
let \(\{e_{n,k}\}_{k=1}^{d_n}\) be an orthonormal basis for each \(\mathcal{H}_n\),
where \(d_n\) could be finite or countably infinite.

An elementary tensor \(\bigotimes_{n=1}^\infty e_{n,k_n}\) is then given by:
\[ \bigotimes_{n=1}^\infty e_{n,k_n} = e_{1,k_1} \otimes e_{2,k_2} \otimes e_{3,k_3} \otimes \cdots \]
where \(e_{n,k_n}\) is a basis vector from \(\mathcal{H}_n\).


\subsection{Tensor Product Space}

To form the tensor product space \(\bigotimes_{n=1}^\infty \mathcal{H}_n\)
\begin{itemize}

\item
we start with elementary tensors \(\bigotimes_{n=1}^\infty e_{n,k_n}\).

\item
We then consider finite linear combinations of these elementary tensors:
   \[ \sum_{i} c_i \left( \bigotimes_{n=1}^\infty e_{n,k_n^{(i)}} \right), \]
   where \(c_i\) are complex coefficients and \(k_n^{(i)}\) are sequences of indices.

\item
We define the inner product on these elementary tensors:
   \[ \left\langle \bigotimes_{n=1}^\infty e_{n,k_n}, \bigotimes_{n=1}^\infty e_{n,j_n} \right\rangle = \prod_{n=1}^\infty \langle e_{n,k_n}, e_{n,j_n} \rangle_{\mathcal{H}_n}. \]
   Since \(\{e_{n,k}\}\) are orthonormal bases, \(\langle e_{n,k_n}, e_{n,j_n} \rangle = \delta_{k_n, j_n}\), so:
   \[ \left\langle \bigotimes_{n=1}^\infty e_{n,k_n}, \bigotimes_{n=1}^\infty e_{n,j_n} \right\rangle = \prod_{n=1}^\infty \delta_{k_n, j_n}, \]
   which equals 1 if \(k_n = j_n\) for all
which equals 1 if \(k_n = j_n\) for all \(n\), and 0 otherwise.

\item
We finally obtain the Hilbert space \(\bigotimes_{n=1}^\infty \mathcal{H}_n\) by taking the completion of the space of finite linear combinations of such elementary tensors with respect to the norm induced by the inner product.

\end{itemize}

Thus by defining elementary tensors as products of basis vectors from the bases of the factors,
one obtains a concrete and manageable set of elementary tensors that span the tensor product space.
This approach simplifies both the definition and the computation of the inner product,
ensuring the space \(\bigotimes_{n=1}^\infty \mathcal{H}_n\) has a well-defined Hilbert space structure.


\subsection{Violation of unitarity}

Infinite tensor products violate unitarity primarily due to issues with normalization, defining a consistent inner product,
maintaining the necessary topological structure of the Hilbert space, and handling unbounded operators. These problems make it
challenging to preserve the fundamental principles of quantum mechanics, including the conservation of probability (unitarity).

\subsubsection{Normalization Issues}

In quantum mechanics, the state vectors in a Hilbert space need to be normalized, meaning their inner product with themselves equals 1. For finite tensor products, the normalization can be managed, but for infinite tensor products, ensuring that the infinite product of norms converges to 1 becomes problematic. In many cases, this product diverges or converges to zero, which means the state vector can't be normalized.

Consider the Hilbert space $\mathcal{H} = \mathbb{C}^2$ (2-dimensional complex space).
For the sake of intuition consider the finite product case first. Let \(|\psi\rangle\) and \(|\phi\rangle\) be two normalized states
\(
|\psi\rangle = \begin{pmatrix} \alpha , \beta \end{pmatrix}^\intercal
\),
\(
|\phi\rangle = \begin{pmatrix} \gamma , \delta \end{pmatrix}^\intercal
\)
($\intercal$ stands for hermitian conjugation)
where \(\alpha, \beta, \gamma, \delta \in \mathbb{C}\) and \(|\alpha|^2 + |\beta|^2 = 1\) and \(|\gamma|^2 + |\delta|^2 = 1\).

The tensor product state \(|\psi\rangle \otimes |\phi\rangle\) is:
\(
|\psi\rangle \otimes |\phi\rangle = \begin{pmatrix} \alpha \gamma , \alpha \delta , \beta \gamma , \beta \delta \end{pmatrix}^\intercal
\).
The norm of this tensor product state is:
\(
\langle \psi \otimes \phi | \psi \otimes \phi \rangle =
|\alpha \gamma|^2 + |\alpha \delta|^2 + |\beta \gamma|^2 + |\beta \delta|^2 = (|\alpha|^2 + |\beta|^2)(|\gamma|^2 + |\delta|^2)   = 1
\).
Therefore,  the state is normalized.

Now,  consider an infinite sequence of normalized states
\(
|\psi_i\rangle = \begin{pmatrix} \alpha_i , \beta_i \end{pmatrix}^\intercal
\),
\(
\quad |\alpha_i|^2 + |\beta_i|^2 = 1
\),
for \(i = 1, 2, 3, \ldots\).
The infinite tensor product state thereof is
\(
|\Psi\rangle = |\psi_1\rangle \otimes |\psi_2\rangle \otimes |\psi_3\rangle \otimes \cdots
\).

The norm of this infinite tensor product state is given by
\(
\langle \Psi | \Psi \rangle = \prod_{i=1}^{\infty} \langle \psi_i | \psi_i \rangle = \prod_{i=1}^{\infty} (|\alpha_i|^2 + |\beta_i|^2)
\).
Since each \(|\psi_i\rangle\) is normalized, \(|\alpha_i|^2 + |\beta_i|^2 = 1\) for each \(i\).
However, consider the case where \(|\alpha_i|^2 + |\beta_i|^2\)
is not exactly 1 but slightly less or greater than 1 for each \(i\), such as \(|\alpha_i|^2 + |\beta_i|^2 = 1 - \epsilon_i\) with \(\epsilon_i > 0\)
or with \(\epsilon_i < 0\).
%For small \(\epsilon_i\),
This leads to vanishing or diverging
\(
\langle \Psi | \Psi \rangle = \prod_{i=1}^{\infty} (1 - \epsilon_i)
\).

For instance, even if \(\epsilon_i\) are very small and positive, the infinite product of terms slightly less than 1 will converge to 0,
assuming \(\sum_{i=1}^{\infty} \epsilon_i\) diverges.
For instance,   if \(\epsilon_i = \frac{1}{i}\) except $\epsilon_1=0$, then
\(
\prod_{i=1}^{\infty} \left(1 - \frac{1}{i}\right) = \prod_{i=2}^{\infty} \left(1 - \frac{1}{i}\right) = 0
\).

Therefore, in many cases, the infinite product of norms can converge to zero or diverge, indicating that the state is not normalizable.
This loss of normalization demonstrates why infinite tensor products can violate unitarity.



\subsubsection{Inner Product Definition and orthogonality}

The inner product is a fundamental concept in quantum mechanics, ensuring that the probability amplitudes are well-defined and that the evolution is unitary. In an infinite tensor product space, defining an inner product that respects the properties of a Hilbert space is challenging. There are usually issues with convergence of series or integrals involved in the inner product.

Consider again the Hilbert space $\mathcal{H} = \mathbb{C}^2$ (2-dimensional complex space).
Consider the finite tensor product first. Let
\(
|\psi\rangle = \begin{pmatrix} \alpha , \beta \end{pmatrix}^\intercal
\)
and
\(
|\phi\rangle = \begin{pmatrix} \gamma , \delta \end{pmatrix}^\intercal
\)
be two normalized states,
where \(\alpha, \beta, \gamma, \delta \in \mathbb{C}\) and \(|\alpha|^2 + |\beta|^2 = 1\) and \(|\gamma|^2 + |\delta|^2 = 1\).
The inner product of the tensor product states \(|\psi\rangle \otimes |\phi\rangle\) and \(|\psi'\rangle \otimes |\phi'\rangle\) is
\(
\left( \langle \psi| \otimes \langle \phi| \right) \left( |\psi'\rangle \otimes |\phi'\rangle \right) =
\langle \psi | \psi' \rangle \cdot \langle \phi | \phi' \rangle
\).
If
\(|\psi'\rangle = \begin{pmatrix} \alpha' , \beta' \end{pmatrix}^\intercal
\)
and
\(|\phi'\rangle = \begin{pmatrix} \gamma' , \delta' \end{pmatrix}^\intercal
\), then:
\(
\langle \psi | \psi' \rangle = \alpha^* \alpha' + \beta^* \beta', \quad \langle \phi | \phi' \rangle = \gamma^* \gamma' + \delta^* \delta'
\), where $*$ stands for complex conjugation.
Thus,
\(
\left( \langle \psi | \otimes \langle \phi | \right) \left( |\psi'\rangle \otimes |\phi'\rangle \right)
= (\alpha^* \alpha' + \beta^* \beta') (\gamma^* \gamma' + \delta^* \delta')
\).

Now,  consider an infinite sequence of normalized states
\(
|\psi_i\rangle = \begin{pmatrix} \alpha_i , \beta_i \end{pmatrix}^\intercal
\)
and
\(
|\psi_i'\rangle = \begin{pmatrix} \alpha_i' , \beta_i' \end{pmatrix}^\intercal
\),
with
\(
|\alpha_i|^2 + |\beta_i|^2 = 1
\)
and
\(
|\alpha_i'|^2 + |\beta_i'|^2 = 1,
\)
for \(i = 1, 2, 3, \ldots\).

The infinite tensor product states are
\(
|\Psi\rangle = |\psi_1\rangle \otimes |\psi_2\rangle \otimes |\psi_3\rangle \otimes \cdots
\)
and
\(
|\Psi'\rangle = |\psi_1'\rangle \otimes |\psi_2'\rangle \otimes |\psi_3'\rangle \otimes \cdots
\).
The inner product of these infinite tensor product states is given by
\(
\langle \Psi | \Psi' \rangle = \prod_{i=1}^{\infty} \langle \psi_i | \psi_i' \rangle
\).
For each \(i\),
\(
\langle \psi_i | \psi_i' \rangle = \alpha_i^* \alpha_i' + \beta_i^* \beta_i'
\).

For the sake of demonstration, suppose again that, except for \(\langle \psi_1 | \psi_1' \rangle =1 \)
 each \(\langle \psi_i | \psi_i' \rangle\) is slightly less than 1, such as
\(
\langle \psi_i | \psi_i' \rangle = 1 - \epsilon_i
\), with \( 0 < \epsilon_i < 1
\).
Then the inner product of the infinite tensor product states becomes
\(
\langle \Psi | \Psi' \rangle = \prod_{i=2}^{\infty} (1 - \epsilon_i)
\).
If again \(\epsilon_i = \frac{1}{i}\), then
\(
\prod_{i=2}^{\infty} \left( 1 - \frac{1}{i} \right) = \prod_{i=2}^{\infty} \left( \frac{i-1}{i} \right) = 0
\).

% Plot[(Product[1 - 1/j, {j, 2, i}]), {i, 2, 100}]

In this case, the inner product \(\langle \Psi | \Psi' \rangle\) converges to zero, indicating that \(|\Psi\rangle\) and \(|\Psi'\rangle\)
are orthogonal even if each \(\langle \psi_i | \psi_i' \rangle\) is very close to 1.

This loss of the meaningful inner product demonstrates why infinite tensor products can have issues with maintaining a consistent inner product structure.
In many cases, the inner product can become undefined or lead to unintuitive results,
violating the expected properties of a Hilbert space and thereby affecting unitarity.




%Topological Considerations:

%The Hilbert space structure relies on the completeness and the norm topology. For infinite tensor products, the resulting space might not be complete or might not respect the norm topology. This affects the ability to define unitary operators that preserve the inner product structure, leading to potential violations of unitarity.

\subsubsection{Bounded Operators}


Let us consider an example involving an infinite tensor product of projection operators to illustrate issues with bounded operators on infinite tensor products.
Consider again the Hilbert space $\mathcal{H} = \mathbb{C}^2$ (2-dimensional complex space).
Let $E$ be the rank-one projection operator onto the subspace spanned by the vector $\begin{pmatrix}1,0\end{pmatrix}^\intercal$,
with
\(
E = \text{diag}\begin{pmatrix}
1 , 0
\end{pmatrix}
\).
Consider the operator $F = \bigotimes_{n=1}^{\infty} E$, which is the infinite tensor product of $E$ with itself.

At first glance, since $E$ is a projection operator with $\|E\| = 1$, we might expect $F$ to be a well-defined bounded operator
with $\|F\| = 1$. However, this is not the case.
To see why,  consider the action of $F$ on particular vectors.

For the vector $|\psi \rangle = \begin{pmatrix}1,0\end{pmatrix}^\intercal \otimes \begin{pmatrix}1,0\end{pmatrix}^\intercal  \otimes \cdots$,
we obtain $F |\psi \rangle = |\psi \rangle$ and $\| F |\psi \rangle \| = \||\psi \rangle\| = 1$.
However, for any vector that has a component orthogonal to $\begin{pmatrix}1,0\end{pmatrix}^\intercal$ in any factor,
$F$ will map it to the zero vector. For example,
$| \varphi \rangle = \begin{pmatrix}1,0\end{pmatrix}^\intercal \otimes \begin{pmatrix}1,0\end{pmatrix}^\intercal
 \otimes \cdots \otimes \begin{pmatrix}0,1\end{pmatrix}^\intercal \otimes \begin{pmatrix}1,0\end{pmatrix}^\intercal \otimes \cdots$
we obtain     $F | \varphi \rangle = 0$.

This behavior leads to some counterintuitive properties because
%(i) $F$ is a projection operator, but its range is only one-dimensional (spanned by $|\psi \rangle$), despite being an infinite tensor product of rank-one projections.
 $F$ is extremely sensitive to changes in its input: changing even one factor from $\begin{pmatrix}1,0\end{pmatrix}^\intercal$
to any other vector results a scalar multiple of $|\psi \rangle$.
Moreover, the behavior of $F$ is consistent with finite tensor products of $E$.
In both finite and infinite cases, the result is a rank-one projection.
However, the key difference is that in the infinite case,
this leads to a projection onto a one-dimensional subspace of an infinite-dimensional space, which has some unique properties.

The issue here is that the infinite tensor product of projection operators does not preserve the properties we expect from finite-dimensional cases.
This example demonstrates how the transition to infinite tensor products can dramatically change the nature of operators,
even when starting with simple, well-behaved operators like projections.
This could be seen as a Hilbert space analogue of Cantor's diagonal argument which demonstrates the uncountability of the continuum.

\subsection{Some hints on factors}

Von Neumann Algebras  are specific subalgebras of the algebra of bounded linear operators on a Hilbert space which are closed under adjoint transformations.
A von Neumann algebra is a factor if its center consists only of scalar multiples of the identity operator.
Factors are the `atomic' building blocks of von Neumann algebras.
Type I  factors are essentially matrix algebras of infinite size.

an explicit construction of factors from infinite tensor products involves several steps:

\begin{enumerate}
\item \textit{Starting Point:} Begin with a sequence of Hilbert spaces, denoted as $\{\mathcal{H}_i\}$. Each $\mathcal{H}_i$ is a Hilbert space.
\item \textit{Reference Vectors:} Choose a unit vector $\psi_i$ in each $\mathcal{H}_i$. These vectors are crucial for defining the infinite tensor product.
\item \textit{Elementary Tensors:} Construct elementary tensors of the form
\(
\psi_1 \otimes \psi_2 \otimes \cdots \otimes \psi_n \otimes \mathbb{1} \otimes \mathbb{1} \otimes \cdots
\)
where the first $n$ components are the chosen vectors, and the rest are identity operators $\mathbb{1}$ or unit vectors.
\item \textit{Hilbert Space Completion:} Take the linear span of all such elementary tensors and complete it with respect to the Hilbert space norm to obtain the infinite tensor product Hilbert space, denoted as $\mathcal{H}$.
\item \textit{Algebraic Tensor Product:} Construct the algebraic tensor product of the corresponding von Neumann algebras acting on each $\mathcal{H}_i$. This gives a preliminary algebra acting on $\mathcal{H}$.
\item \textit{Weak Closure:} Take the weak closure of this algebraic tensor product in the space of bounded operators on $\mathcal{H}$. This weak closure is a von Neumann algebra.
\item \textit{Factor Condition:} Under certain conditions on the reference vectors $\psi_i$, the von Neumann algebra constructed in step 6 will be a factor. The specific conditions depend on the desired type of factor (type I, II, or III).
\end{enumerate}

Type I factors often arise from product states that are highly symmetric or have specific algebraic properties~\cite{sorce-2023}.
For instance, if the reference vectors are chosen to be eigenvectors of commuting operators, the resulting factor is likely to be type I.

Type II factors typically emerge from product states that exhibit some degree of randomness or irregularity. They often require more sophisticated analysis to determine their exact type.

Type III factors are usually associated with product states that have strong correlations or exhibit asymptotic independence properties. Their construction and analysis are particularly challenging.


For the sake of an example of a type I factor from Symmetric Reference Vectors
consider an infinite tensor product of Hilbert spaces $\mathcal{H}_i = \mathbb{C}^2$ for all $i \in \mathbb{N}$.
As reference vectors choose the Cartesian basis for $\mathbb{C}^2$, that is,
$\psi_i^1 = \begin{pmatrix}1,0\end{pmatrix}^\intercal$ and
$\psi_i^2 = \begin{pmatrix}0,1\end{pmatrix}^\intercal$.
 Note that these basis vectors are eigenvectors of the Pauli-$z$ operator, which is a diagonal matrix with eigenvalues $\pm 1$.
Thus, they are eigenvectors of a commuting set of operators (in this case, just the Pauli-$z$ operator for each Hilbert space.


The construction of the factor proceeds in three steps:
(i) Form the infinite tensor product Hilbert space $\mathcal{H}$ as described earlier.
(ii) Construct the algebraic tensor product of the von Neumann algebras associated with each $\mathcal{H}_i$.
(iii) Take the weak closure to obtain a von Neumann algebra on $\mathcal{H}$.

The key to understanding why this is a type I factor lies in the structure of the von Neumann algebra.
Because the reference vectors are eigenvectors of commuting operators,
the resulting von Neumann algebra will be generated by a family of commuting projections.
This structure is characteristic of type I factors.

More formally, the von Neumann algebra in this case can be shown to be isomorphic to the infinite tensor product of the matrix algebra
$M_2(\mathbb{C})$,the algebra of $2 \times 2$ complex matrices. This is a canonical example of a type I factor.


\section{Nested Wigner's friends as infinite tensor products}

Nesting is the repeated and iterated application of the Landau-von Neumann type measurement-by-entanglement
formalized by Equation~(\ref{2024-u-vNsiqm}).
As a consequence we end up with large and potentially infinite tensor products.
It is tempting to ascribe this  measurement conceptualization to von Neumann~\cite{Taub:1961:JNCc,vonNeumannCompendium}.
Recently it has been proposed and discussed by
Philippe Grangier and Mathias Van Den Bossche~\cite{Grangier-2020,van-den-bossche-2023-a,van-den-bossche-2023-b}.

In view of what has been mentioned earlier any such construction is extremely sensible to
changes in the focus of observations of Wigner's friends---that is, with respect to changes in the chain of entangled basis vectors.
Because not only may the state change inside a (Type I) factor,
but any correlation between (successive) measurement mismatches might cause it to fall into a type II or type III factor.
Therefore, the slightest change and mismatch of nested observables results in a complete loss of information about the initial state (preparation).

More explicity, as has been pointed out earlier in the context of difficulties in defining the inner product, any slightest mismatch between (successive) friends measurements `builts up' in a total loss of coherence.
This results in a vanishing inner product \(\langle \Psi | \Psi' \rangle\) which converges to zero, indicating that the product states
\(|\Psi\rangle\) and \(|\Psi'\rangle\)
are orthogonal even if each single mismatch characterized by \(\langle \psi_i | \psi_i' \rangle\) is very close to 1.
This type of `decoherence' is gradual and smooth in the sense that there is no abrupt discontinuous
transition---indicating a well defined, localizable Heisenberg cut at some scale---but a gradual, continuous loss
of information about the initial state: Let $0 \ll | \langle \psi_i | \psi_i' \rangle | = \delta_i \le 1$ be this match per friends $i$ and $i'$, then
\begin{equation}
| \langle \Psi | \Psi' \rangle |
= \lim_{N\rightarrow \infty}\prod_{i=1}^{N} | \langle \psi_i | \psi_i' \rangle |
= \lim_{N\rightarrow \infty}\prod_{i=1}^{N} \delta_i
=0.
\end{equation}




\section{Historical parallels and analogues}

We have answered the formal conundrum of how an irreversible non-unitary measurements---von Neumann's~\cite[Chapter~5]{vonNeumann2018Feb}
(and Everett's~\cite{everett}) process 1 type---would
originate in uniform, even one-to-one, reversible unitary evolution---von Neumann's process 2 type---essentially
by an infinite recursion of the latter.

\subsection{Infinity means induce capacities}

This has parallels in the real numbers, where in the limit, sequences of rational numbers converge towards an irrational number:
Take, for instance, the continued fraction or the binomial series expansion
$
\sqrt{2} = (1 + 1)^{1/2} = \sum_{n=0}^{\infty} \binom{1/2}{n} \cdot 1^n =
1 + \frac{1}{2} \cdot 1 - \frac{1}{8} \cdot 1^2 + \frac{1}{16} \cdot 1^3 - \cdots
$
of $\sqrt{2}$, truncated at various points.

Another analogue is from recursive analysis: Specker sequences of computable numbers
converge to an uncomputable limit~\cite{specker49,specker-ges,kreisel,simonsen-2005}.
One example is Chaitin's constant, the halting probability of prefix-free  program codes on a universal computer~\cite{rtx100200236p,calude-dinneen06}
whose rate of convergence is tied to the halting time, and therefore, `grows faster' than any computable function.

Many of these metamathematical results are based on Cantor's diagonal argument~\cite{book:486992} which proves that,
`in the limit, enumerable sets become non-enumerable continua.'

\subsection{Statistical physics}

Similarly,
{L}oschmidt's {\it `Umkehreinwand'}~\cite{darrigol-2021} challenges the notion of irreversible processes at the macrolevel,
given the time-reversibility of microphysical laws. Loschmidt pointed out that if the microscopic laws are reversible,
then any macroscopic process should also be reversible if we could precisely reverse the velocities of all particles in a system.
This seems to contradict our everyday experience of irreversible processes, and the postulate of the increas of entropy.

The canonical answer to the Umkehreinwand appears to be evasive:
Although formally correct,
because of statistical-probabilistic considerations, the Umkehreinwand is means relative~\cite{Myrvold2011237},
and thus FAPP~\cite{bell-a} invalid,
thereby~\cite{Maxwell-1879,garber}
{\it ``avoiding all personal enquiries [[about individual molecules]] which would only get me into trouble.''}





\section{Summary}

We have presented both a formal as well as a pragmatic argument
for the  conversion of
unitary and thus one-to-one reversible von Neumann--Everett type 2 processes
into
non-unitary and many-to-one irreversible von Neumann--Everett type 1 processes.

Thereby, we have made use of infinite tensor products that are no longer bounded by unitarity as we know them from finite dimensional Hilbert spaces.
Objections to this formal argument may be that infinite mathematical `processes' have no operational correspondence~\cite{bridgman}
and are sophistry of sorts.

A more practical approach is to consider finite subsequences of this construction. We can argue that these can be
interpreted as nested Wigner friends that finnd it increasingly difficult to extract the original information from a
quantum state. This holds in particular, if there is a mismatch between preparation of that state and measurement.

One might find such finite means relative arguments not dissimilar
to environmental monitoring effectively resulting in quantum decoherence quantum decoherence,
sometimes also called dynamical decoherence or
environment-induced decoherence~\cite{schlosshauer-2005,schlosshauer-2019}.
There are also parallels to the introduction of noise in the amplification of micro-states~\cite{Glauber-cat-86},
which in turn can be seen as a form of the quantum no-cloning theorem.

In this context should also mention quantum erasure arguments~\cite{PhysRevA.25.2208,greenberger2,kim-2000,Ma22012013}
and the {H}umpty-{D}umpty-problem~\cite{engrt-sg-I,engrt-sg-II,Englert2013},
that push the Heisenberg cut boundaries to what is feasible towards means relativity.
The classical statistical argument against reversibility and the Umkehreinwand---that,
with a macrostate, one always holds a huge collection of microstates which, if individually reversed,
`almost always' go into disorder and thus entropy increase---is mirrorred in the observation that,
for quantum measurements, we `almost always' hold, as macrophysical mixed states, ensembles or buquets of microstates so diverse
that attempts to reconstruct the original state by reversing their unitary evolution is futile---just
as in the classical {H}umpty-{D}umpty metaphor a crashed egg will `never' become an hedgeable egg again.

How can we approach the late Schr\"odinger' jellyfication argument~\cite{schroedinger-interpretation}---that the unobserved world tends to become quantum jelly
in the form of coherent superpositions of states without `fixation' by irreversible measurement---which is a modification of
his cat problem?
We could say that the nesting of Wigner friends makes it increasingly difficult to maintain superposition: the scalar product between the
original ideal state and the slightest mismatches in the friends basis frames (of observation) render the orthogonalization
between the prepared and the measured state, amplified through multiple nestings.
This occurs even within a type I factor. The orthogonalization occurs smoothly but swiftly as the nesting increases.




\begin{acknowledgments}
NN

The authors declare no conflict of interest.

The AI assistants
ChatGPT~4o from OpenAI,
Gemini of Google,
as well as Claude from Anthropic were used for a formulation of parts of the argument and symbolic transformations into \LaTeX, as well as for grammar and syntax checks.
\end{acknowledgments}


\bibliography{svozil}
\ifws

\bibliographystyle{spmpsci}

\else
 %\bibliographystyle{apsrev}

\fi

\end{document}





%%%%%%%%%%%%%%%%%%%%%%%%%%%%%%%%%%%%%%%%%%%%%%%%%%%%%%%%%%%%%%%%%%%%%%%%%%%%%%%%%%%%%%%%%%%%%%%%%%%%%%%%%%%%
%%%%%%%%%%%%%%%%%%%%%%%%%%%%%%%%%%%%%%%%%%%%%%%%%%%%%%%%%%%%%%%%%%%%%%%%%%%%%%%%%%%%%%%%%%%%%%%%%%%%%%%%%%%%
%%%%%%%%%%%%%%%%%%%%%%%%%%%%%%%%%%%%%%%%%%%%%%%%%%%%%%%%%%%%%%%%%%%%%%%%%%%%%%%%%%%%%%%%%%%%%%%%%%%%%%%%%%%%
%%%%%%%%%%%%%%%%%%%%%%%%%%%%%%%%%%%%%%%%%%%%%%%%%%%%%%%%%%%%%%%%%%%%%%%%%%%%%%%%%%%%%%%%%%%%%%%%%%%%%%%%%%%%
%%%%%%%%%%%%%%%%%%%%%%%%%%%%%%%%%%%%%%%%%%%%%%%%%%%%%%%%%%%%%%%%%%%%%%%%%%%%%%%%%%%%%%%%%%%%%%%%%%%%%%%%%%%%
%%%%%%%%%%%%%%%%%%%%%%%%%%%%%%%%%%%%%%%%%%%%%%%%%%%%%%%%%%%%%%%%%%%%%%%%%%%%%%%%%%%%%%%%%%%%%%%%%%%%%%%%%%%%
%%%%%%%%%%%%%%%%%%%%%%%%%%%%%%%%%%%%%%%%%%%%%%%%%%%%%%%%%%%%%%%%%%%%%%%%%%%%%%%%%%%%%%%%%%%%%%%%%%%%%%%%%%%%
%%%%%%%%%%%%%%%%%%%%%%%%%%%%%%%%%%%%%%%%%%%%%%%%%%%%%%%%%%%%%%%%%%%%%%%%%%%%%%%%%%%%%%%%%%%%%%%%%%%%%%%%%%%%
%%%%%%%%%%%%%%%%%%%%%%%%%%%%%%%%%%%%%%%%%%%%%%%%%%%%%%%%%%%%%%%%%%%%%%%%%%%%%%%%%%%%%%%%%%%%%%%%%%%%%%%%%%%%
%%%%%%%%%%%%%%%%%%%%%%%%%%%%%%%%%%%%%%%%%%%%%%%%%%%%%%%%%%%%%%%%%%%%%%%%%%%%%%%%%%%%%%%%%%%%%%%%%%%%%%%%%%%%
%%%%%%%%%%%%%%%%%%%%%%%%%%%%%%%%%%%%%%%%%%%%%%%%%%%%%%%%%%%%%%%%%%%%%%%%%%%%%%%%%%%%%%%%%%%%%%%%%%%%%%%%%%%%
%%%%%%%%%%%%%%%%%%%%%%%%%%%%%%%%%%%%%%%%%%%%%%%%%%%%%%%%%%%%%%%%%%%%%%%%%%%%%%%%%%%%%%%%%%%%%%%%%%%%%%%%%%%%
%%%%%%%%%%%%%%%%%%%%%%%%%%%%%%%%%%%%%%%%%%%%%%%%%%%%%%%%%%%%%%%%%%%%%%%%%%%%%%%%%%%%%%%%%%%%%%%%%%%%%%%%%%%%
%%%%%%%%%%%%%%%%%%%%%%%%%%%%%%%%%%%%%%%%%%%%%%%%%%%%%%%%%%%%%%%%%%%%%%%%%%%%%%%%%%%%%%%%%%%%%%%%%%%%%%%%%%%%
%%%%%%%%%%%%%%%%%%%%%%%%%%%%%%%%%%%%%%%%%%%%%%%%%%%%%%%%%%%%%%%%%%%%%%%%%%%%%%%%%%%%%%%%%%%%%%%%%%%%%%%%%%%%
%%%%%%%%%%%%%%%%%%%%%%%%%%%%%%%%%%%%%%%%%%%%%%%%%%%%%%%%%%%%%%%%%%%%%%%%%%%%%%%%%%%%%%%%%%%%%%%%%%%%%%%%%%%%
%%%%%%%%%%%%%%%%%%%%%%%%%%%%%%%%%%%%%%%%%%%%%%%%%%%%%%%%%%%%%%%%%%%%%%%%%%%%%%%%%%%%%%%%%%%%%%%%%%%%%%%%%%%%
%%%%%%%%%%%%%%%%%%%%%%%%%%%%%%%%%%%%%%%%%%%%%%%%%%%%%%%%%%%%%%%%%%%%%%%%%%%%%%%%%%%%%%%%%%%%%%%%%%%%%%%%%%%%
%%%%%%%%%%%%%%%%%%%%%%%%%%%%%%%%%%%%%%%%%%%%%%%%%%%%%%%%%%%%%%%%%%%%%%%%%%%%%%%%%%%%%%%%%%%%%%%%%%%%%%%%%%%%
%%%%%%%%%%%%%%%%%%%%%%%%%%%%%%%%%%%%%%%%%%%%%%%%%%%%%%%%%%%%%%%%%%%%%%%%%%%%%%%%%%%%%%%%%%%%%%%%%%%%%%%%%%%%
%%%%%%%%%%%%%%%%%%%%%%%%%%%%%%%%%%%%%%%%%%%%%%%%%%%%%%%%%%%%%%%%%%%%%%%%%%%%%%%%%%%%%%%%%%%%%%%%%%%%%%%%%%%%
%%%%%%%%%%%%%%%%%%%%%%%%%%%%%%%%%%%%%%%%%%%%%%%%%%%%%%%%%%%%%%%%%%%%%%%%%%%%%%%%%%%%%%%%%%%%%%%%%%%%%%%%%%%%
%%%%%%%%%%%%%%%%%%%%%%%%%%%%%%%%%%%%%%%%%%%%%%%%%%%%%%%%%%%%%%%%%%%%%%%%%%%%%%%%%%%%%%%%%%%%%%%%%%%%%%%%%%%%
%%%%%%%%%%%%%%%%%%%%%%%%%%%%%%%%%%%%%%%%%%%%%%%%%%%%%%%%%%%%%%%%%%%%%%%%%%%%%%%%%%%%%%%%%%%%%%%%%%%%%%%%%%%%
%%%%%%%%%%%%%%%%%%%%%%%%%%%%%%%%%%%%%%%%%%%%%%%%%%%%%%%%%%%%%%%%%%%%%%%%%%%%%%%%%%%%%%%%%%%%%%%%%%%%%%%%%%%%
%%%%%%%%%%%%%%%%%%%%%%%%%%%%%%%%%%%%%%%%%%%%%%%%%%%%%%%%%%%%%%%%%%%%%%%%%%%%%%%%%%%%%%%%%%%%%%%%%%%%%%%%%%%%
%%%%%%%%%%%%%%%%%%%%%%%%%%%%%%%%%%%%%%%%%%%%%%%%%%%%%%%%%%%%%%%%%%%%%%%%%%%%%%%%%%%%%%%%%%%%%%%%%%%%%%%%%%%%
%%%%%%%%%%%%%%%%%%%%%%%%%%%%%%%%%%%%%%%%%%%%%%%%%%%%%%%%%%%%%%%%%%%%%%%%%%%%%%%%%%%%%%%%%%%%%%%%%%%%%%%%%%%%
%%%%%%%%%%%%%%%%%%%%%%%%%%%%%%%%%%%%%%%%%%%%%%%%%%%%%%%%%%%%%%%%%%%%%%%%%%%%%%%%%%%%%%%%%%%%%%%%%%%%%%%%%%%%
%%%%%%%%%%%%%%%%%%%%%%%%%%%%%%%%%%%%%%%%%%%%%%%%%%%%%%%%%%%%%%%%%%%%%%%%%%%%%%%%%%%%%%%%%%%%%%%%%%%%%%%%%%%%
%%%%%%%%%%%%%%%%%%%%%%%%%%%%%%%%%%%%%%%%%%%%%%%%%%%%%%%%%%%%%%%%%%%%%%%%%%%%%%%%%%%%%%%%%%%%%%%%%%%%%%%%%%%%
%%%%%%%%%%%%%%%%%%%%%%%%%%%%%%%%%%%%%%%%%%%%%%%%%%%%%%%%%%%%%%%%%%%%%%%%%%%%%%%%%%%%%%%%%%%%%%%%%%%%%%%%%%%%
%%%%%%%%%%%%%%%%%%%%%%%%%%%%%%%%%%%%%%%%%%%%%%%%%%%%%%%%%%%%%%%%%%%%%%%%%%%%%%%%%%%%%%%%%%%%%%%%%%%%%%%%%%%%
%%%%%%%%%%%%%%%%%%%%%%%%%%%%%%%%%%%%%%%%%%%%%%%%%%%%%%%%%%%%%%%%%%%%%%%%%%%%%%%%%%%%%%%%%%%%%%%%%%%%%%%%%%%%
%%%%%%%%%%%%%%%%%%%%%%%%%%%%%%%%%%%%%%%%%%%%%%%%%%%%%%%%%%%%%%%%%%%%%%%%%%%%%%%%%%%%%%%%%%%%%%%%%%%%%%%%%%%%
%%%%%%%%%%%%%%%%%%%%%%%%%%%%%%%%%%%%%%%%%%%%%%%%%%%%%%%%%%%%%%%%%%%%%%%%%%%%%%%%%%%%%%%%%%%%%%%%%%%%%%%%%%%%

