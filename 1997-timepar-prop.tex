\documentstyle[12pt,pslatex]{book}
%  \makeindex
% \makeglossary
 \begin{document}

\title{Time paradoxes}
\author{K. Svozil\\
 {\small Institut f\"ur Theoretische Physik}  \\
  {\small University of Technology Vienna }     \\
  {\small Wiedner Hauptstra\ss e 8-10/136}    \\
  {\small A-1040 Vienna, Austria   }            \\
  {\small e-mail: svozil@tph.tuwien.ac.at}\\
  {\small www: http://tph.tuwien.ac.at/$\widetilde{\;\;}\,$svozil}}
\date{ }
\maketitle

 \tableofcontents

\chapter*{Foreword}
 \addcontentsline{toc}{chapter}{Foreword}



   This book is about two main issues:  the consequences of faster-than-light
phenomena,  and  about  time  paradoxes.    Both  issues are strongly linked.
Because, as will be discussed in greater detail, if one assumes the  validity
of the special theory of relativity and the existence of free will, then time
paradoxes  result  from  faster-than-light  phenomena.   But since observables
should be consistently organized, time  paradoxes must not exist.   Therefore,
it is  often concluded,  faster-than-light phenomena  and free  will must not
co-exist.


   The book has been  written by a theoretical  physicist but it is  intended
for the much wider audience of scientifically interested citizens.  In  fact,
it  has  been  deliberately  not  been  written  for  the  narrow  market  of
physicists.  One reason has been the author's suspicion that physicists  will
mostly ignore it.   Many of  them will pretend  that any effort  dedicated to
consequences of faster-than-light communication  and time paradoxes is  a mere
waste of time, that  the author has read  too many science fiction  novels or
listened to  too many  fantasy movies.   Worse,  some physicists  will deeply
resent the topics and will consider them sinful heresy; neglecting the  kanon
of presently consolidated physical  fact.  In summary:   even if the  science
introduced in  this book  is basically  correct, compared  to the  mainstream
physical endavour of our time, the entire subject appears exotic.



   But the stakes are high.  The  main issue here is not a mere  academic one
of a consistent description of phenomena, or of levels of decription.  It  is
ultimately the technology of superluminal  signalling and space travel.
These technologies would change the way we live dramatically. A new frontier
would open up, new challenges would result. At present, new
virtual worlds are created like metaphoric ``bubbles'', thereby virtually
extending the limiting physical space. Space-travel could extend
our primary reality in a very concrete form. Therefore, this book should be
interesting to a broader audience.

   At the moment,  there are very  few if any  indications that anytime  soon
there  will  be  technologically  useful  faster-than-light  phenomena.  Some
speculations exist that nonlocal quantum correlations and quantum  tunnelling
can  be  utilized  for  these  purposes.    Thereby,  quantum phenomena would
invalidate  certain  assumptions  which  are  usually  linked  with   special
relativity; in particular the assumption that the speed of light is an  upper
bound for all  practical purposes.   But this possibility  would seems to  be
blocked by what is sometimes called the hypothesis of ``peaceful  coexistence
between relativity theory and quantum mechanics''.  Other possible  phenomena
which have been discussed in the literature will be mentioned below.


   The author confesses that faster-than-light space travel and  superluminal
phenomena is one of his obsessions.  As seems to be often with
obsessions,
this one  broke out  unexpectedly.   The story  of the  writing of  this book
starts on a fairly  sunny May night in  Riksgr\"anser, well  above
the arctic  circle. It  was the  final day  of a  meeting on  the
workshop on
intrinsic limits to  scientific knowledge, sponsored  by the Swedish  Council
for  Planning  and  Coordination  of  Research.    John  Casti,  one  of  the
organizers,  confronted  me  with  a  scenario,  which  truly  shattered  me.

\begin{quote}
{\em  [[Imagine  some  creatures---]]they  may be carbon-based
creatures just like you and me.  The difference is that they live in a  world
in which the primary sensory inputs are not from the electromagnetic spectrum
like light, but rather come from sound waves.  Note that this is not simply a
world of  the blind;  rather, it  is a  world in  which there  are no sensory
organs for  perceiving any  part of  the electromagnetic  spectrum.   In this
case,  then,  $\ldots$  such  creatures  would  see  the  speed of sound as a
fundamental  barrier  to  the  velocity  of  any  material object.  Yet we as
creatures  that  {\em   do}  possess  sensory   organs  for  perceiving   the
electromagnetic spectrum see the sound barrier as no fundamental barrier,  at
all.  So by  analogical extension, there may  be creatures ``out there''  who
regard the speed of light as no more of a barrier than we regard the speed of
sound.  }
\end{quote}


This scenario had such an impact upon me because
very similar thoughts have been in my mind for a very long time ago.
Indeed, almost ten years ago, I had published a closely related
scenario.

\begin{quote}
{\em
   In what follows I shall give an example of such a configuration:  assume a
pool filled  with water  $\ldots$ suppose  there is  a water-flea [Cladocera]
living on the surface  of the water.   Suppose further that this  creature is
blind, i.e. it  is not able  to employ electromagnetic  radiation.  Then  its
operational parameter description will  almost certainly purely have  to rely
on water wave  dynamics as operational  device.  On  the other hand,  another
creature [such as a bird], seeing the surface of the water by light  $\ldots$
Whereas the water-flea, if it  is not imaginative enough, will  always wonder
about  the  form  of  its  world  and  its embedding and limitations, the air
creature will immediately  overlook the situation,  thereby rather using  its
eyes than produce  wave scattering.   That of course  does not mean  that the
water-flea in principle can never detect light; its just that electromagnetic
effects  yield  such  small  contributions  to  phenomenology  that  they are
difficult to detect.
}
\end{quote}

   The  most  important  stimulus   to  these  development  must   come  from
experiment.    It  is  the  author's  conviction  that  the  precise  form of
faster-than-light phenomena will not be forecasted by theory, but will  first
be detected by experiments.   Nevertheless, theoretical physics has the  task
to  elaborate  a  framework  to  cope  with  them.    And  it  should provide
encouragement to seek ahead of the present limitations rather than erect stop
signs and punishment for speculative thought in this direction.






\chapter{A short history of time paradoxes, superluminal phenomena and
time travel}
\section{Anectodal claims of velocity barriers in history}
\section{The rise of time paradoxes}
\section{Faster-than-light phenomena?}


\chapter{Time paradoxes and their relation to the classical paradoxes of
mathematics and formal logic}
\section{The diagonalization method}
\section{From Cantor to G\"odel to Turing to Chaitin}

\chapter{Strategies to block paradoxes I: non-existence of free will}
\chapter{Strategies to block paradoxes II: quantum information theory}


\chapter{Connection to superluminal signalling and space travel}
\section{The construction of physical space and time}
Light clocks, scales, Einstein synchronization and Alexandrov's theorem
\section{A pedestrian approach to relativity theory}
\section{The flow of time cannot be properly defined for superluminal
observers}

\chapter{Sound creatures learn about light}
\section{A tale of two velocity constants}
\section{Who is afraid of paradoxes?}

\chapter{Strategies to block paradoxes III: Relativity theory
relativized}
\section{Intrinsic observers}
\section{Description levels}

\chapter{Outlook}

 \end{document}
