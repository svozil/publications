%%%%%%%%%%%%%%%%%%%%% chapter.tex %%%%%%%%%%%%%%%%%%%%%%%%%%%%%%%%%
%
% sample chapter
%
% Use this file as a template for your own input.
%
%%%%%%%%%%%%%%%%%%%%%%%% Springer-Verlag %%%%%%%%%%%%%%%%%%%%%%%%%%
%\motto{Use the template \emph{chapter.tex} to style the various elements of your chapter content.}
\chapter{Possible consequences of disclosure or contact}
\label{2023-UFO-part-Speculation-possible-consequences} % Always give a unique label
% use \chaptermark{}
% to alter or adjust the chapter heading in the running head


\abstract*{The idea of disclosing information about extraterrestrial life may need to be reevaluated, as their motivations may not align with ours and they may have advanced technology that could lead to our subjugation. This can be compared to the historical conversation known as the Melian Dialog, which explores power and justice in international relations. The Brookings Report of 1961 also raises concerns about the impact of such a discovery on primitive societies and the potential dangers posed to extraterrestrial visitors. Governments may even view their territories as off-limits to extraterrestrial visitors for safety reasons.}


\abstract{The idea of disclosing information about extraterrestrial life may need to be reevaluated, as their motivations may not align with ours and they may have advanced technology that could lead to our subjugation. This can be compared to the historical conversation known as the Melian Dialog, which explores power and justice in international relations. The Brookings Report of 1961 also raises concerns about the impact of such a discovery on primitive societies and the potential dangers posed to extraterrestrial visitors. Governments may even view their territories as off-limits to extraterrestrial visitors for safety reasons.}

\section{Loss of autonomy}

The potential loss of autonomy for individuals, organizations, and states in relation to UFO knowledge could have significant implications for the way governments handle and respond to this information. In particular, if the existence of UFOs or extraterrestrial life is confirmed, it could challenge long-held beliefs and structures within society, particularly those relating to religion~\cite{Weintraub2014Jul}, science, and politics.

For individuals, the possibility of extraterrestrial life may lead to a crisis of identity and purpose, as it would challenge our understanding of humanity's place in the universe. This could also cause individuals to question the legitimacy of their governments and other established institutions, leading to a loss of trust and confidence in these entities.

For organizations and states, the potential implications of UFO knowledge could be even more profound. The discovery of extraterrestrial life could have significant economic and geopolitical consequences, including the possibility of new technologies and resources, as well as potential threats to national security. This could lead to a race among nations to acquire and control this knowledge, potentially leading to international conflict and even war.

In response to these potential challenges, governments may be inclined to suppress information relating to UFOs and extraterrestrial life~\cite{Wendt_2008}. This could take many forms, including censorship, disinformation campaigns, and even outright denial of the existence of UFOs. By suppressing this information, governments may be able to maintain their power and control over their citizens but at the cost of scientific progress, societal openness, and individual freedom.

Overall, the potential loss of autonomy related to UFO knowledge highlights the complex and far-reaching implications of any potential discoveries relating to extraterrestrial life. While such discoveries could be transformative and exciting, they could also challenge our deepest beliefs and structures and require careful consideration and management to ensure that the benefits are realized, while minimizing the potential negative consequences.

\section{Imbalance of Power}
\label{2023-UFO-part-Speculation-possible-consequences-pt} % Always give a unique label

Suppose extraterrestrial visits were a reality.
Initially, our response would be to exclaim, ``We want to know!'' However, upon further reflection, we may rethink our eagerness for the Others to make themselves known
and break their nonfraternization and silence.

This is because, despite optimistic attitudes~\cite{Mazzola2020Apr} toward ``disclosure''
and fervent denials of potential ``badness,'' the motivations of these visitors may not be favorable for humankind,
and may not align with our own goals.
On the contrary, such ``disclosure'' or contact might be devastating.

Any such disclosure would mean that we would become aware that we have lost at least some autonomy and control.
Undoubtedly, we would find ourselves at a disadvantage in terms of science and technology, and ultimately,
we would be subjugated by the Others' superior abilities. After all, they possess the capability to reach us,
but we do not possess the means to travel to them.
In short, we are dealing with an asymmetric situation that is to our disadvantage.

Based on our limited understanding, their technological prowess in operating their vessels appears to be superior and intimidating.
In the event of a conflict, we would not be able to stand up to them.
This creates a significant imbalance of power, leaving us vulnerable to their mercy.



\section{The Melian Dialog}

At this point, I believe that a little ancient history is pertinent. The Melian Dialog is a conversation between the representatives of Athens and the city-state of Melos, recorded by the historian Thucydides in his book \textit{The Peloponnesian War}~\cite{ThucydidesHoPW}. It took place during the Peloponnesian War between the Peloponnesian League led by Sparta and the Delian League led by Athens. At the time, the Melians had ancestral ties to Sparta. In the summer of 416 BC, Athens launched an invasion against Melos, insisting that the city surrender and become a tributary state, paying tribute to Athens. Failing to comply would result in their destruction. Despite the dire consequences, the Melians refused to comply because they wished to remain neutral.

In this most remarkable dialog laid out by Thucydides, Athens tries to convince Melos to surrender and become a tributary state to Athens, while Melos offers neutrality in the conflict between Sparta and Athens and thereby resists the demands of Athens. Throughout the conversation, Athens asserts its military superiority and argues that ``might makes right,'' while Melos appeals to universal principles of justice and fair treatment. In my view, the core argument presented by Athens is based on the unequal distribution of power between themselves and the Melians, from which they claim the following privileges~\cite{ThucydidesHoPW}:
\begin{svgraybox}
``$\ldots$~you know as well as we do that right [[i.e., fairness]], as the world goes, is only in question between equals in power,
 while the strong do what they can and the weak suffer what they must.''
\end{svgraybox}

The dialog is a classic example of the principles of power and justice in international relations. The outcome of the dialog demonstrates the brutal realities of power politics in ancient Greece. Melos' subsequent refusal to comply resulted in a siege by Athenian forces. Following the defeat of Melos, the Athenians ordered the execution of all adult male citizens. The women and children were sold into slavery. Athenian colonists then relocated to the island to establish a new settlement.

I believe that all past empires or colonization attempts have built upon these principles: fairness can only be expected among equals, and those in power take what they demand and need. Even institutions and organizations that ostensibly convey very different tactics and protocols effectively act that way. Therefore, as humans are inferior in such an exchange with the Others, we need to be aware of our possible fate.

\section{The Brookings Report of 1961}

In a section entitled ``The implications of a discovery of extraterrestrial life''
in a report prepared for NASA by the Brookings Institution
the following issues are raised~\cite[p.~215 and Footnote~37, p.~226]{MichaelDonald1961}:
\begin{svgraybox}
``Anthropological files contain many examples of societies, sure of their place in the universe,
which have disintegrated when they had to associate with previously unfamiliar societies
espousing different ideas and different life ways; others that survived such an experience usually did so
by paying the price of changes in values and attitudes and behavior.''

$\ldots$

``Fruitful understanding might be gained from a comparative
study of factors affecting the responses of primitive societies to exposure to
technologically advanced societies. Some thrived, some endured, and some died.''
\end{svgraybox}

We might speculate that this has already happened to us.

\section{We may pose dangers to visitors}

It is possible that the concept of a cargo cult could work in reverse. Governments on Earth might effectively view their territories as
``no-fly zones''
for extraterrestrial visitors (without proper prior identification and credentials),
and attack them with the intention of destroying them.

By analogy, the Sentinelese living on the small North Sentinel Island in the Andaman Sea, thought to be the last tribe on Earth still living in a pre-Neolithic way,
fiercely resist attempts by outsiders to approach their island and frequently attack intruders~\cite{SentinelIsland2018Nov}.
Access to their territory is strictly limited by the government of India.

Illegal fishing operations, run by poachers, are prevalent in the waters surrounding the island. These poachers target turtles, lobsters, and sea cucumbers.

In 2006, the Sentinelese tribespeople took the lives of two Indian fishermen after their boat strayed onto the shore.
An attempt to retrieve the bodies by an Indian Coast Guard helicopter was met with a barrage of arrows from the tribe, causing the helicopter to retreat.

In 2018, an American preacher who ventured to the remote Indian island---considered by the tribe to be
``theirs''---and got killed, apparently shot and killed by arrows,
although it has not been officially confirmed.
According to his family, the man was a missionary and held nothing but love for the Sentinelese tribe.

\section{Possible scenarios}

There are several possible scenarios that could result from disclosure, ranging from benign to catastrophic. Unfortunately, due to the scientific and technological imbalance between ``us'' and ``them,'' it will not be within our control to initiate or oversee such a situation.

Richard Dolan and Bryce Zabel conducted a thorough examination of lifelike scenarios in their work~\cite{DolanZabel2012May}. Steven J. Dick's collection of articles deals with ``the impact of discovering life beyond Earth''~\cite{Dick2015}. In particular, Dick's contribution in Chapter~3 contains some encouraging analogies, such as the great voyages of the fifteenth-century Ming China treasure fleets 50 years before Columbus. Dick notes that ``there are other models of culture contact than the destructive ones usually cited, including Jesuit models of culture contact that were not disastrous.'' Let us hope for the best!
