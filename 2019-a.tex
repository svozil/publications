\documentclass[%
 %reprint,
  twocolumn,
 %superscriptaddress,
 %groupedaddress,
 %unsortedaddress,
 %runinaddress,
 %frontmatterverbose,
 % preprint,
 showpacs,
 showkeys,
 preprintnumbers,
 %nofootinbib,
 %nobibnotes,
 %bibnotes,
 amsmath,amssymb,
 aps,
 % prl,
  pra,
 % prb,
 % rmp,
 %prstab,
 %prstper,
  longbibliography,
 %floatfix,
 %lengthcheck,%
 ]{revtex4-1}

%\usepackage{cdmtcs-pdf}

\usepackage[dvipsnames]{xcolor}

\usepackage{mathptmx}% http://ctan.org/pkg/mathptmx

\usepackage{amssymb,amsthm,amsmath}

\usepackage{tikz}
\usetikzlibrary{calc}

%\usepackage[breaklinks=true,colorlinks=true,anchorcolor=blue,citecolor=blue,filecolor=blue,menucolor=blue,pagecolor=blue,urlcolor=blue,linkcolor=blue]{hyperref}
\usepackage{graphicx}% Include figure files
\usepackage{url}

\usepackage{hyperref}
\hypersetup{
    colorlinks=true, %set true if you want colored links
    linktoc=all,     %set to all if you want both sections and subsections linked
    linkcolor=blue,  %choose some color if you want links to stand out
    citecolor=blue,
    filecolor=red,
    urlcolor=blue
}

\usepackage{xcolor}



\begin{document}

\title{Tau regularization of divergent series}


\author{Tom Summers and Karl Svozil}
\email{svozil@tuwien.ac.at}
\homepage{http://tph.tuwien.ac.at/~svozil}

\affiliation{Institute for Theoretical Physics,
Vienna  University of Technology,
Wiedner Hauptstrasse 8-10/136,
1040 Vienna,  Austria}



\date{\today}

\begin{abstract}
A toy model calculation in zero space-time dimensions compares the convergence of the asymptotic divergent perturbation series obtaind as ``penalty' for illegally exchanging limits with the convergent series obtained by using Tao-type test functions which legally allow an exchange of those limits.
\end{abstract}


\keywords{perturbation series, asymptodic divergence, Tao regularization}

\maketitle

\section{Different encodings of mathematical entities}

A formal entity such as the solution of an ordinary differential equation may have very
different representations and encodings; some of them with problematic issues.
This is often not a matter of choice but means relative, and therefore one of pragmatism or even desperation.
In particular, theoretical physicists are often criticised for their ``relaxed'' stance on formal rigor.
Dirac's introduction of the needle-shaped delta function is often quoted as an example.
Heaviside, in another instance, responded to critism for his use of the ``highly nonsmooth''
unit step function\cite[p.~9, \S~225]{heaviside-EMT}:
{\em ``But then the
rigorous logic of the matter is not plain! Well, what of that?
Shall I refuse my dinner because I do not fully understand the
process of digestion? No, not if I am satisfied with the result.''}

This, in a nutshell, seems to be the attitude of field theorists
with regards to use of perturbation series:
It is well documented~\cite{PhysRev.85.631,LeGuillou-Zinn-Justin,Vainshtein1964-2002}
that the commonly used power series expansion
$F(\alpha )=a_0+a_1\alpha + a_2\alpha^2+ \cdots $
in terms of the fine structure constant $\alpha$---that is, the square of the) coupling constant---is divergent.

In order for this power series to converge there has to be a finite radius of convergence
centered at the origin at (fictitious) value $\alpha=0$,
thereby including (fictitious) nonvanishing negative values $\alpha < 0$
within which $F$ has to be analytic.
However, because if the (ficticious) coupling between like charges becomes negative, and because by tunnelling this cannot be ``contained'',
the vacuum becomes unstable due to pair creation, and (ficticiously) disintegraties explosively.
Hence, Dyson concludes,
the power series $F(-\alpha )=a_0-a_1\alpha + a_2\alpha^2+ \cdots $ cannot be converge and $F(-\alpha )$
cannot be analytic---a complete contradiction to the assumption.

One critical step in the derivation of $F$
amounts to interchanging a sum with an integral
in the case of nonuniform convergence of the former~\cite[Sect.~II.A]{PhysRevD.57.1144}.
One may perceive asymptotic divergence as a ``penalty'' for such manipulations.
It may come as a surprise that those calculations performed well with respect to empirical predictions.

An immediate reaction would be to perceive these coincidences as ``bordering on the mysterious''~\cite{wigner}.
This spirit is corroborated by statements like Carrier's Rule, pointing out that
``divergent series converge faster than convergent series because they don't have to converge.''
However, as already surmised by Dyson, quantitative considerations from partial sum show~\cite[p.~4]{Bleistein-Handelsman}
the convergent series ``initially'' may diverge from
the true value of the function it encodes, whereas the divergent series ``initially'' converges toward
this value: a ``reasonable'' approximation can be obtained by taking
relatively few terms of the divergent series; whereas many more terms of the
convergent series are needed to achieve that same degree of accuracy.

Dyson already mentioned a possible remedy, his
{\em ``Alternative A: There may be discovered a new method
of carrying through the renormalization program, not making use of power series expansions.''}



Suppose it is unknown whether some mathematical entity has a representation
in terms of common analytic functions.
Nevertheless, in such cases often (power) series representations can be found.
The partial sums of those series may converge -- hopefully with a ``good'' rate of convergence -- or diverge.
In the latter case one can still hope for asymptotic~\cite{Erdelyi-1956,Dingle-1973,Bender-Orszag}
divergence which is heuristically
characterised by reasonable, increasingly better estimates of the  solution up to some ``optimal'' order of the (power) series,
at which point the quality of the approximation deteriorates.

One way of coping with the apparent asymtotic divergence are resummation techniques,
in particular, Borel (re)summations~\cite{Boyd99thedevil,rousseau-2004,Helling-2012,Costin-2009,ZINNJUSTIN20101454,Costin_Dunne_2017},
which are in some instances capable to reconstruct an analytic function
from its asymptotic expansion~\cite{Bruning-1996}.

In what follows we shall adopt another strategy inspired by Ritt's theorem~\cite{Pittnauer-73,Remmert-1991-tocf}
stating that any (not necessarily asymptotic) divergent power series with arbitrary coefficients can
be converted into nonunique analytic functions. Thereby, every summand is
multiplied with a suitable nonunique {\em ``convergence factor.''}
(Conversely, every analytic function can be approximated by a unique asymptotic series.)

A general regularisation
of divergent series  utilizing such convergence factors,
also called {\em cutoff functions}, has been recently introduced by
Tao~\cite[Section~3.7]{Tao-2013}.
The resulting smoothed sums may become uniformly convergent, thereby
allowing interchanging a sum with an integral
and avoiding the aforementioned issues while at the same time preserving
inherent properties of the original divergent series.
This is not dissimilar to the use of {\em test functions} in the theory of distributions.





A ``canonical'' example~\cite{Bleistein-Handelsman} is the Stieltjes function
\begin{equation}
\begin{split}
S(x) =  \int_0^\infty    \frac{e^{-t}}{1+tx} dt
\end{split}
\label{2019-m-ch-ds-Stieltjes-function}
\end{equation}
which is a solution of the ordinary differential equation
\begin{equation}
\left(\frac{d}{dx} x+\frac{1}{x}\right) S(x) = \frac{1}{x}
.
\end{equation}
It can be  represented by
power series in two different ways:
\begin{itemize}
\item[(i)]
by the asymptotic Stieltjes series
\index{Stieltjes series}
\begin{equation}
\begin{split}
S(x)=  \sum_{j=0}^n (-x)^j j!  +  (-x)^{n+1}(n+1)!  \int_0^\infty   \frac{e^{-t}}{(1+tx)^{n+2}} dt
\end{split}
\label{2019-m-ch-ds-asymptotic-Stieltjes-series}
\end{equation}
as well as
\item[(ii)]
by classical convergent Maclaurin series such as (Ramanujan found a series which converges even more rapidly)
\begin{equation}
\begin{split}
S(x)
=  \frac{e^\frac{1}{x}}{x}  \Gamma \left( 0, \frac{1}{x} \right)
=  -\frac{e^\frac{1}{x}}{x}  \left[  \gamma - \log x +\sum_{j=1}^\infty \frac{(-1)^j}{j!j x^j} \right]
,
\end{split}
\label{2019-m-ch-ds-Stieltjes-series}
\end{equation}
\end{itemize}
where
\begin{equation}
\gamma
=\lim_{n\rightarrow \infty}\left(
\sum_{j=1}^n \frac{1}{j}- \log n
\right) \approx 0.5772 \end{equation}
is the Euler-Mascheroni constant~\cite{Sloane_oeis.org/A001620}.
$\Gamma (z,x)$ represents the upper incomplete gamma function).

The Stieltjes function $S(x)$ for real positive $x>0$ can be rewritten in terms of
the exponential integral~\cite[formul\ae~5.1.1, 5.1.2, 5.1.4, page~227]{abramowitz:1964:hmf}
\begin{equation}
\begin{split}
E_1(y) =   -\text{Ei}(-y)
= \Gamma \left( 0, y \right)     \\
=
\int_{1}^\infty    \frac{e^{-uy}}{u} du
=
\int_{y}^\infty    \frac{e^{-u}}{u} du
\end{split}
\label{2019-m-ch-ds-Stieltjes-Ei}
\end{equation}
% http://mathworld.wolfram.com/En-Function.html
% https://en.wikipedia.org/wiki/Exponential_integral
by first substituting $x=\frac{1}{y}$ in $S(x)$ as defined in~(\ref{2019-m-ch-ds-Stieltjes-function}), followed by the transformation of integration variable
$t = y(u-1)$, so that, for $y>0$,
\begin{equation}
\begin{split}
S\left(\frac{1}{y}\right)
=  \int_0^\infty    \frac{e^{-t}}{1+\frac{t}{y}} dt\\
=ye^{y}\int_1^\infty    \frac{e^{-yu}}{u}  du
=  ye^{y}E_1(y)  =  -ye^{y}\text{Ei}(-y),  \\
\text{or }
S(x)=  \frac{e^\frac{1}{x}}{x} E_1\left(\frac{1}{x}\right)
%=  -\frac{e^\frac{1}{x}}{x}\text{Ei}\left(-\frac{1}{x}\right)
= \frac{e^\frac{1}{x}}{x} \Gamma \left( 0, \frac{1}{x} \right)
.
\end{split}
\label{2019-m-ch-ds-Stieltjes-function-Ei}
\end{equation}
The asymptotic Stieltjes
series~(\ref{2019-m-ch-ds-asymptotic-Stieltjes-series})
quoted in (i)
as well as the convergent
series~(\ref{2019-m-ch-ds-Stieltjes-series})
quoted in (ii)
can, for positive (real) arguments, be obtained by substituting the respective series for the
exponential integral~\cite[formul\ae~5.1.51, page~231 and~5.1.10,5.1.11, page~229]{abramowitz:1964:hmf}:
\begin{equation}
\begin{split}
E_1(y)
\sim
\frac{e^{-y}}{y}\sum_{j=0}^\infty (-1)^j j!  \frac{1}{y^j} \\
 = \Gamma \left( 0, y \right)=- \gamma - \log y - \sum_{j=1}^\infty \frac{ (-y)^j }{j (j!)}
,
\end{split}
\label{2019-m-ch-ds-Stieltjes-Ei-se}
\end{equation}
where again $\gamma$ stands for the Euler-Mascheroni constant
and $\Gamma (z,x)$ represents the upper incomplete gamma function~\cite[6.5.1, p~260]{abramowitz:1964:hmf}.

There exists a vast literature~\cite{berry-92,Boyd99thedevil,Costin-2009,Dorigoni-2014} on how to recover the exact entity from its asymptotic solution; one method employing Borel (re)summation.
Thereby the (re)summation of the divergent perturbation theory in quantum electrodynamics~\cite{PhysRev.85.631,Thirring-1953}
may become justifiable and useful.

In what follows we shall pursue a different approach:
as the divergence of the perturbation theory could be traced to an illegal exchange of two limits~\cite{PhysRevD.57.1144}
we shall employ methods to formally justify and ``legalize'' this exchange of limits by suitable cutoff or test functions
introduced by Terence Tao~\cite[\S~3.7]{Tao-2013}.
Like in the theorems of generalized functions the results do not depend on
the explicit form of those test functions.
Already Ritt's theorem -- stating that a power series with {\em arbitrary} coefficients sequence $a_i$ corresponds to a nonunique
holomorphic function~\cite{Pittnauer-73,Remmert-1991-tocf,Costin-2009}.


\bibliography{svozil}

\end{document}
