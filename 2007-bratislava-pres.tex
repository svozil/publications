%\documentclass[pra,showpacs,showkeys,amsfonts,amsmath,twocolumn]{revtex4}
\documentclass[amsmath,blue,table,sans]{beamer}
%\documentclass[pra,showpacs,showkeys,amsfonts]{revtex4}
\usepackage[T1]{fontenc}
%%\usepackage{beamerthemeshadow}
\usepackage[headheight=1pt,footheight=10pt]{beamerthemeboxes}
\addfootboxtemplate{\color{structure!80}}{\color{white}\tiny \hfill Karl Svozil (TU Vienna)\hfill}
\addfootboxtemplate{\color{structure!65}}{\color{white}\tiny \hfill Quantum Scholasticism $\ldots$\hfill}
\addfootboxtemplate{\color{structure!50}}{\color{white}\tiny \hfill QL\&P, Bratislava, Nov. 24, 2007\hfill}
%\usepackage[dark]{beamerthemesidebar}
%\usepackage[headheight=24pt,footheight=12pt]{beamerthemesplit}
%\usepackage{beamerthemesplit}
%\usepackage[bar]{beamerthemetree}
\usepackage{graphicx}
\usepackage{pgf}
%\usepackage[usenames]{color}
%\newcommand{\Red}{\color{Red}}  %(VERY-Approx.PANTONE-RED)
%\newcommand{\Green}{\color{Green}}  %(VERY-Approx.PANTONE-GREEN)

%\RequirePackage[german]{babel}
%\selectlanguage{german}
%\RequirePackage[isolatin]{inputenc}

\renewcommand{\baselinestretch}{1}

\pgfdeclareimage[height=0.5cm]{logo}{tu-logo}
\logo{\pgfuseimage{logo}}
\beamertemplatetriangleitem
%\beamertemplateballitem

\beamerboxesdeclarecolorscheme{alert}{red}{red!15!averagebackgroundcolor}
%\begin{beamerboxesrounded}[scheme=alert,shadow=true]{}
%\end{beamerboxesrounded}

%\beamersetaveragebackground{green!10}

%\beamertemplatecircleminiframe

\begin{document}

\title{\bf \textcolor{blue}{Quantum Scholasticism: On Quantum Contexts,  Counterfactuals, and the Absurdities of Quantum Omniscience}}
%\subtitle{Naturwissenschaftlich-Humanisticher Tag am BG 19\\Weltbild und Wissenschaft\\http://tph.tuwien.ac.at/\~{}svozil/publ/2005-BG18-pres.pdf}
\subtitle{\textcolor{orange!60}{\small http://tph.tuwien.ac.at/$\sim$svozil/publ/2007-bratislava-pres.pdf}\\
          \textcolor{orange!60}{\small http://arxiv.org/abs/0711.1473}}
\author{Karl Svozil}
\institute{Institut f\"ur Theoretische Physik, University of Technology Vienna, \\
Wiedner Hauptstra\ss e 8-10/136, A-1040 Vienna, Austria\\
svozil@tuwien.ac.at
%{\tiny Disclaimer: Die hier vertretenen Meinungen des Autors verstehen sich als Diskussionsbeitr�ge und decken sich nicht notwendigerweise mit den Positionen der Technischen Universit�t Wien oder deren Vertreter.}
}
\date{Quantum Logic and Probability 2007, \\Bratislava, November 24, 2007}
\maketitle


\renewcommand{\baselinestretch}{1.2}

\frame{
\frametitle{Contents}
\tableofcontents
}

 \renewcommand{\baselinestretch}{1.8}

\section{Quantum contexts}

\frame{
\frametitle{Main issues}

{\LARGE
\begin{itemize}
\item<1->
Counterfactuals and context--dependence

\item<1->
Alternatives to contextuality
\end{itemize}

}
}

\renewcommand{\baselinestretch}{1}
\subsection{Quasi-experimental status}

\frame{
\frametitle{Quasi-experimental status}

\begin{itemize}
\item<1->
Einstein--Podolsky--Rosen (EPR): ``Explosion view'' of two contexts in a singlet state

\item<1->
Boole's ``conditions of possible experience'' \& the associated Bell-type inequalities,
such as Clauser--Horne--Shimony--Holt (CHSH): bounds on correlation functions from convexity conditions of classical probabilities.

\item<1->
Kochen--Specker (KS) theorem: finite proof by contradiction that on quantum logics (dim $>3$) there does not exist any two-valued state associated truth assignments on propositions about a quantized system.

\item<1->
Greenberger--Horne--Zeilinger (GHZ) theorem: multipartite argument yielding a complete contradiction between classical and quantum predictions.

\item<1->
Now what?

\end{itemize}

}

\subsection{Quantum context}
\frame{
\frametitle{Quantum context}

\begin{itemize}
\item<1->
A context is a maximal collection of co-measurable observables associated with commuting operators.

\item<1->
Every context can also be characterized by a single (but nonunique) maximal operator.
All operators within a context are functions thereof.

\item<1->
In quantum logic,
contexts are represented by Boolean subalgebras or blocks
pasted together to form the Hilbert lattice.
(For the sake of nontriviality, Hilbert spaces of dimension higher than two are considered.)

\item<1->
Heuristically, a context represents a ``classical mini--universe,''
which is distributive and allows for as many two--valued states --- interpretable as classical truth assignments ---
as there are atoms.
\end{itemize}
}


\subsection{Features of quasi--experimental setup}
\frame{
\frametitle{Features of quasi--experimental setup}

\begin{itemize}
\item<1->
the proofs require the assumption of counterfactuals;
i.e., of ``potential'' observables which, due to quantum complementarity,
are incompatible with the ``actual'' measurement context;
yet could have been
measured if the measurement apparatus were different.
%
%
These counterfactuals are  organized into groups of interconnected contexts which,
due to quantum complementarity, are incompatible
and therefore cannot be measured simultaneously; not even in Einstein-Podolsky-Rosen (EPR) type setups.

\item<1->
The proofs by contradiction have no direct experimental realizations.
As has already been pointedly stated by Robert Clifton, ``how can you measure a contradiction?''

\item<1->
So--called ``experimental tests'' inspired by Bell-type inequalities,  KS as well as GHZ
measure the incompatible contexts which are considered  in the proofs one after another;
i.e., temporally sequentially, and not simultaneously.
Hence, different contexts can only be measured on different particles.

\end{itemize}
}


\renewcommand{\baselinestretch}{1.2}

\section{``Scarcity'' of two--valued states on quantum structures}

\subsection{Principle of explosion}
\frame{
\frametitle{Principle of explosion}
{\Large
\begin{itemize}
\item<1->
The  {\em principle of explosion}:  {\it ``ex falso quodlibet,''} or {\it ``contradictione sequitur quodlibet''}
amounts to ``anything follows from a contradiction.''
\item<1->
Due to the pasting construction of Hilbert lattices, the principle of explosion holds also in quantum logic.
\end{itemize}
}
}

\renewcommand{\baselinestretch}{1}

\subsection{Star configurations}
\frame{
\frametitle{Star configurations}

For the $n$-dimensional Hilbert space, an $n$-star configuration represents
$n$ different contexts joined in $n$ different atoms of the center context.
\begin{center}
\begin{tabular}{ccc}
%TeXCAD Picture [1.pic]. Options:
%\grade{\on}
%\emlines{\off}
%\epic{\off}
%\beziermacro{\on}
%\reduce{\on}
%\snapping{\off}
%\pvinsert{% Your \input, \def, etc. here}
%\quality{8.000}
%\graddiff{0.005}
%\snapasp{1}
%\zoom{8.0000}
\unitlength 0.2mm % = 2.845pt
\linethickness{0.8pt}
\ifx\plotpoint\undefined\newsavebox{\plotpoint}\fi % GNUPLOT compatibility
\begin{picture}(151.75,61.75)(0,0)
\put(150,0){\circle{7}}
\put(100,0){\circle{7}}
\put(50,0){\circle{7}}
\put(0,0){\circle{7}}
\put(150,20){\circle{7}}
\put(100,20){\circle{7}}
\put(50,20){\circle{7}}
\put(0,20){\circle{7}}
\put(150,40){\circle{7}}
\put(100,40){\circle{7}}
\put(50,40){\circle{7}}
\put(0,40){\circle{7}}
\put(150,60){\circle{7}}
\put(100,60){\circle{7}}
\put(50,60){\circle{7}}
\put(0,60){\circle{7}}
\put(150,0){\line(0,1){60}}
\put(100,0){\line(0,1){60}}
\put(50,0){\line(0,1){60}}
\put(0,0){\line(0,1){60}}
\put(0,0){\line(1,0){150}}
%
\put(75,5){\makebox(0,0)[cc]{$a$}}
\put(10,50){\makebox(0,0)[cc]{$b$}}
\put(60,50){\makebox(0,0)[cc]{$c$}}
\put(110,50){\makebox(0,0)[cc]{$d$}}
\put(160,50){\makebox(0,0)[cc]{$e$}}
\end{picture}
&
$\qquad$
$\qquad$
&
%TeXCAD Picture [1.pic]. Options:
%\grade{\on}
%\emlines{\off}
%\epic{\off}
%\beziermacro{\on}
%\reduce{\on}
%\snapping{\off}
%\pvinsert{% Your \input, \def, etc. here}
%\quality{8.000}
%\graddiff{0.005}
%\snapasp{1}
%\zoom{8.0000}
\unitlength 0.125mm % = 2.845pt
\linethickness{0.8pt}
\ifx\plotpoint\undefined\newsavebox{\plotpoint}\fi % GNUPLOT compatibility
\begin{picture}(103.5,103.75)(0,0)
\put(100,100){\line(-1,-1){100}}
\put(0,100){\line(1,-1){100}}
\put(100,100){\circle*{15}}
\put(100,0)  {\circle*{15}}
\put(50,50)  {\circle*{15}}
\put(0,0)    {\circle*{15}}
\put(0,100)  {\circle*{15}}
%
\put(70,50){\makebox(0,0)[cc]{$a$}}
\put(120,100){\makebox(0,0)[cc]{$b$}}
\put(120,0){\makebox(0,0)[cc]{$c$}}
\put(20,0){\makebox(0,0)[cc]{$d$}}
\put(20,100){\makebox(0,0)[cc]{$e$}}
\end{picture}
\\
a)&&b)
\end{tabular}
\end{center}
Four-star configuration in four-dimensional Hilbert space
%
a) Greechie diagram representing atoms by points, and  contexts by maximal smooth, unbroken curves.
%
b) Dual Tkadlec diagram representing contexts by filled points, and interconnected contexts by lines.


}

\subsection{The one--zero rule}
\frame{
\frametitle{The one--zero rule}

\begin{center}
\begin{tabular}{ccc}
%TeXCAD Picture [1.pic]. Options:
%\grade{\on}
%\emlines{\off}
%\epic{\off}
%\beziermacro{\on}
%\reduce{\on}
%\snapping{\off}
%\quality{8.00}
%\graddiff{0.01}
%\snapasp{1}
%\zoom{5.6569}
\unitlength .3mm % = 1.42pt
\linethickness{0.8pt}
\ifx\plotpoint\undefined\newsavebox{\plotpoint}\fi % GNUPLOT compatibility
\begin{picture}(120.92,114.73)(0,0)
%\emline(86.57,102.14)(111.57,58.64)
\multiput(86.57,102.14)(.11961722,-.20813397){209}{\line(0,-1){.20813397}}
%\end
%\emline(86.57,15.14)(111.57,58.64)
\multiput(86.57,15.14)(.11961722,.20813397){209}{\line(0,1){.20813397}}
%\end
%\emline(36.65,102.14)(11.65,58.64)
\multiput(36.65,102.14)(-.11961722,-.20813397){209}{\line(0,-1){.20813397}}
%\end
%\emline(36.65,15.14)(11.65,58.64)
\multiput(36.65,15.14)(-.11961722,.20813397){209}{\line(0,1){.20813397}}
%\end
\put(86.57,101.89){\line(-1,0){50}}
\put(86.57,15.39){\line(-1,0){50}}
\put(86.46,101.94){\circle{4}}
\put(86.46,15.34){\circle{4}}
\put(111.39,58.63){\circle{4}}
\put(11.74,58.63){\circle{4}}
\put(61.77,58.63){\circle{4}}
\put(36.52,101.94){\circle{4}}
\put(61.62,101.94){\circle{4}}
\put(61.62,15.44){\circle{4}}
\put(97.68,82.85){\circle{4}}
\put(25.71,82.85){\circle{4}}
\put(98.74,36.35){\circle{4}}
\put(24.65,36.35){\circle{4}}
\put(36.52,15.34){\circle{4}}
\put(61.69,101.82){\line(0,-1){86.27}}
\put(30.41,2.65){\makebox(0,0)[cc]{$A$}}
\put(61.87,2.3){\makebox(0,0)[cc]{$B$}}
\put(91.93,2.48){\makebox(0,0)[cc]{$C$}}
\put(110.84,30.94){\makebox(0,0)[cc]{$D$}}
\put(120.92,57.98){\makebox(0,0)[lc]{$E$}}
\put(108.41,88.92){\makebox(0,0)[cc]{$F$}}
\put(91.93,114.2){\makebox(0,0)[cc]{$G$}}
\put(61.7,114.73){\makebox(0,0)[cc]{$H$}}
\put(30.41,114.02){\makebox(0,0)[cc]{$I$}}
\put(13.56,87.86){\makebox(0,0)[cc]{$J$}}
\put(1.77,57.98){\makebox(0,0)[rc]{$ K$}}
\put(14.67,30.05){\makebox(0,0)[rc]{$L$}}
\put(67.88,55.51){\makebox(0,0)[cc]{$M$}}
\put(71.34,9.19){\makebox(0,0)[cc]{$a$}}
\put(107.91,40.35){\makebox(0,0)[cc]{$b$}}
\put(98.53,95.32){\makebox(0,0)[cc]{$c$}}
\put(54.46,108.01){\makebox(0,0)[cc]{$d$}}
\put(15.03,78.14){\makebox(0,0)[cc]{$e$}}
\put(21.56,27.06){\makebox(0,0)[cc]{$f$}}
\put(67.88,75.51){\makebox(0,0)[cc]{$g$}}
\end{picture}
&
$\qquad$
&

%TeXCAD Picture [1.pic]. Options:
%\grade{\on}
%\emlines{\off}
%\epic{\off}
%\beziermacro{\on}
%\reduce{\on}
%\snapping{\off}
%\quality{8.00}
%\graddiff{0.01}
%\snapasp{1}
%\zoom{5.6569}
\unitlength .3mm % = 1.42pt
\linethickness{0.8pt}
\ifx\plotpoint\undefined\newsavebox{\plotpoint}\fi % GNUPLOT compatibility
\begin{picture}(120.92,114.2)(0,0)
%\emline(86.57,102.14)(111.57,58.64)
\multiput(86.57,102.14)(.11961722,-.20813397){209}{\line(0,-1){.20813397}}
%\end
%\emline(86.57,15.14)(111.57,58.64)
\multiput(86.57,15.14)(.11961722,.20813397){209}{\line(0,1){.20813397}}
%\end
%\emline(36.65,102.14)(11.65,58.64)
\multiput(36.65,102.14)(-.11961722,-.20813397){209}{\line(0,-1){.20813397}}
%\end
%\emline(36.65,15.14)(11.65,58.64)
\multiput(36.65,15.14)(-.11961722,.20813397){209}{\line(0,1){.20813397}}
%\end
\put(86.57,101.89){\line(-1,0){50}}
\put(86.57,15.39){\line(-1,0){50}}
\put(86.46,101.94){\circle*{4}}
\put(86.46,15.34){\circle*{4}}
\put(111.39,58.63){\circle*{4}}
\put(11.74,58.63){\circle*{4}}
\put(36.52,101.94){\circle*{4}}
\put(36.52,15.34){\circle*{4}}
\put(61.77,58.63){\circle*{4}}
%\emline(86.44,102)(36.59,15.56)
\multiput(86.44,102)(-.119831731,-.207788462){416}{\line(0,-1){.207788462}}
%\end
%-
\put(30.41,2.65){\makebox(0,0)[cc]{$a$}}
\put(91.93,2.48){\makebox(0,0)[cc]{$b$}}
\put(120.92,57.98){\makebox(0,0)[lc]{$c$}}
\put(91.93,114.2){\makebox(0,0)[cc]{$d$}}
\put(30.41,114.02){\makebox(0,0)[cc]{$e$}}
\put(1.77,57.98){\makebox(0,0)[rc]{$f$}}
\put(67.88,55.51){\makebox(0,0)[cc]{$g$}}
\end{picture}
\\
a)&&b)
\end{tabular}
\end{center}
{\footnotesize
Configuration of observables in three-dimensional Hilbert space implying
that whenever $E$ is true, $E$ must be false.
The seven interconnected contexts
$a=\{A,B,C\}$,
$b=\{C,D,E\}$,
$c=\{E,F,G\}$,
$d=\{G,H,I\}$,
$e=\{I,J,K\}$,
$f=\{K,L,A\}$,
$g=\{B,H,M\}$,
consist of the 13 projectors associated with the one dimensional subspaces spanned by
$ A= ( 1,\sqrt{2},-1)      $,
$ B= ( 1,0,1)   $,
$ C= ( -1,\sqrt{2},1)    $,
$ D= ( -1,\sqrt{2},-3)    $,
$  E=( \sqrt{2},1,0) $,
$  F=( 1,-\sqrt{2},-3)            $,
$  G=( -1,\sqrt{2},-1)           $,
$  H=( 1,0,-1)    $,
$  I=( 1,\sqrt{2},1)   $,
$ J= ( 1,\sqrt{2},-3)     $,
$ K=( \sqrt{2},-1,0)    $,
$ L=( 1,\sqrt{2},3)     $,
$ M=(0,1,0)    $.
%
%   kp[a1_, a2_, a3_, b1_, b2_, b3_] =   {a2  b3 - a3 b2, a3 b1 - a1 b3, a1 b2 - a2 b1}
%
}
}

\subsection{The one--one rule}
\frame{
\frametitle{The one--one rule}

\begin{center}
\begin{tabular}{c}

%TeXCAD Picture [1.pic]. Options:
%\grade{\on}
%\emlines{\off}
%\epic{\off}
%\beziermacro{\on}
%\reduce{\on}
%\snapping{\off}
%\quality{8.00}
%\graddiff{0.01}
%\snapasp{1}
%\zoom{4.0000}
\unitlength .25mm % = 1.42pt
\linethickness{0.8pt}
\ifx\plotpoint\undefined\newsavebox{\plotpoint}\fi % GNUPLOT compatibility
\begin{picture}(320.85,118.44)(0,0)
%\emline(105.32,33.64)(61.82,8.64)
\multiput(105.32,33.64)(-.20813397,-.11961722){209}{\line(-1,0){.20813397}}
%\end
%\emline(308.26,33.64)(264.76,8.64)
\multiput(308.26,33.64)(-.20813397,-.11961722){209}{\line(-1,0){.20813397}}
%\end
%\emline(18.32,33.64)(61.82,8.64)
\multiput(18.32,33.64)(.20813397,-.11961722){209}{\line(1,0){.20813397}}
%\end
%\emline(221.26,33.64)(264.76,8.64)
\multiput(221.26,33.64)(.20813397,-.11961722){209}{\line(1,0){.20813397}}
%\end
%\emline(105.32,83.56)(61.82,108.56)
\multiput(105.32,83.56)(-.20813397,.11961722){209}{\line(-1,0){.20813397}}
%\end
%\emline(308.26,83.56)(264.76,108.56)
\multiput(308.26,83.56)(-.20813397,.11961722){209}{\line(-1,0){.20813397}}
%\end
%\emline(18.32,83.56)(61.82,108.56)
\multiput(18.32,83.56)(.20813397,.11961722){209}{\line(1,0){.20813397}}
%\end
%\emline(221.26,83.56)(264.76,108.56)
\multiput(221.26,83.56)(.20813397,.11961722){209}{\line(1,0){.20813397}}
%\end
\put(105.07,33.64){\line(0,1){50}}
\put(308.01,33.64){\line(0,1){50}}
\put(18.57,33.64){\line(0,1){50}}
\put(221.51,33.64){\line(0,1){50}}
\put(105.12,33.75){\circle{4}}
\put(308.06,33.75){\circle{4}}
\put(18.52,33.75){\circle{4}}
\put(221.46,33.75){\circle{4}}
\put(61.81,8.82){\circle{4}}
\put(264.75,8.82){\circle{4}}
\put(61.81,108.47){\circle{4}}
\put(264.75,108.47){\circle{4}}
\put(61.81,58.44){\circle{4}}
\put(264.75,58.44){\circle{4}}
\put(105.12,83.69){\circle{4}}
\put(308.06,83.69){\circle{4}}
\put(105.12,58.59){\circle{4}}
\put(308.06,58.59){\circle{4}}
\put(18.62,58.59){\circle{4}}
\put(221.56,58.59){\circle{4}}
\put(86.03,22.53){\circle{4}}
\put(288.97,22.53){\circle{4}}
\put(86.03,94.5){\circle{4}}
\put(288.97,94.5){\circle{4}}
\put(39.53,21.47){\circle{4}}
\put(242.47,21.47){\circle{4}}
\put(39.53,95.56){\circle{4}}
\put(242.47,95.56){\circle{4}}
\put(18.52,83.69){\circle{4}}
\put(221.46,83.69){\circle{4}}
\put(163.21,58.69){\circle{4}}
\put(105,58.52){\line(-1,0){86.27}}
\put(307.94,58.52){\line(-1,0){86.27}}
\put(5.83,89.8){\makebox(0,0)[]{$A$}}
\put(208.77,89.8){\makebox(0,0)[]{$A'$}}
\put(5.48,58.34){\makebox(0,0)[]{$B$}}
\put(208.42,58.34){\makebox(0,0)[]{$B'$}}
\put(5.66,28.28){\makebox(0,0)[]{$C$}}
\put(208.6,28.28){\makebox(0,0)[]{$C'$}}
\put(34.12,9.37){\makebox(0,0)[]{$D$}}
\put(237.06,9.37){\makebox(0,0)[]{$D'$}}
\put(61.16,-.71){\makebox(0,0)[t]{$E$}}
\put(264.1,-.71){\makebox(0,0)[t]{$E'$}}
\put(92.1,11.8){\makebox(0,0)[]{$F$}}
\put(295.04,11.8){\makebox(0,0)[]{$F'$}}
\put(117.38,28.28){\makebox(0,0)[]{$G$}}
\put(320.32,28.28){\makebox(0,0)[]{$G'$}}
\put(117.91,58.51){\makebox(0,0)[]{$H$}}
\put(320.85,58.51){\makebox(0,0)[]{$H'$}}
\put(117.2,89.8){\makebox(0,0)[]{$I$}}
\put(320.14,89.8){\makebox(0,0)[]{$I'$}}
\put(91.04,106.65){\makebox(0,0)[]{$J$}}
\put(293.98,106.65){\makebox(0,0)[]{$J'$}}
\put(61.16,118.44){\makebox(0,0)[b]{$ K$}}
\put(264.1,118.44){\makebox(0,0)[b]{$ K'$}}
\put(33.23,105.54){\makebox(0,0)[b]{$L$}}
\put(236.17,105.54){\makebox(0,0)[b]{$L'$}}
\put(58.69,52.33){\makebox(0,0)[]{$M$}}
\put(261.63,52.33){\makebox(0,0)[]{$M'$}}
\put(12.37,48.87){\makebox(0,0)[]{$a$}}
\put(215.31,48.87){\makebox(0,0)[]{$a'$}}
\put(43.53,12.3){\makebox(0,0)[]{$b$}}
\put(246.47,12.3){\makebox(0,0)[]{$b'$}}
\put(98.5,21.68){\makebox(0,0)[]{$c$}}
\put(301.44,21.68){\makebox(0,0)[]{$c'$}}
\put(111.19,65.75){\makebox(0,0)[]{$d$}}
\put(314.13,65.75){\makebox(0,0)[]{$d'$}}
\put(81.32,105.18){\makebox(0,0)[]{$e$}}
\put(284.26,105.18){\makebox(0,0)[]{$e'$}}
\put(30.24,98.65){\makebox(0,0)[]{$f$}}
\put(233.18,98.65){\makebox(0,0)[]{$f'$}}
\put(38.69,52.33){\makebox(0,0)[]{$g$}}
\put(281.63,52.33){\makebox(0,0)[]{$g'$}}
%\emline(61.75,8.75)(264.5,108.5)
\multiput(61.75,8.75)(.243687933,.1198908573){832}{\line(1,0){.243687933}}
%\end
%\emline(61.75,108.25)(264.75,8.75)
\multiput(61.75,108.25)(.2445763352,-.1198785485){830}{\line(1,0){.2445763352}}
%\end
\put(163,46.25){\makebox(0,0)[cc]{$N$}}
\put(182.75,78.25){\makebox(0,0)[cc]{$h$}}
\put(182.5,41.25){\makebox(0,0)[cc]{$i$}}
\qbezier(61.75,58.5)(104.62,30)(163,58.5)
\qbezier(264.25,58.5)(221.37,87)(163,58.5)
\put(241.75,76.75){\makebox(0,0)[cc]{$j$}}
\end{picture}
\\

a)
\\

%TeXCAD Picture [1.pic]. Options:
%\grade{\on}
%\emlines{\off}
%\epic{\off}
%\beziermacro{\on}
%\reduce{\on}
%\snapping{\off}
%\quality{8.000}
%\graddiff{0.010}
%\snapasp{1}
%\zoom{4.0000}
\unitlength .25mm % = 1.423pt
\linethickness{0.8pt}
\ifx\plotpoint\undefined\newsavebox{\plotpoint}\fi % GNUPLOT compatibility
\begin{picture}(324.48,149.57)(0,0)
%\emline(86.57,102.14)(111.57,58.64)
\multiput(86.57,102.14)(.119617225,-.208133971){209}{\line(0,-1){.208133971}}
%\end
%\emline(239.68,102.14)(214.68,58.64)
\multiput(239.68,102.14)(-.119617225,-.208133971){209}{\line(0,-1){.208133971}}
%\end
%\emline(86.57,15.14)(111.57,58.64)
\multiput(86.57,15.14)(.119617225,.208133971){209}{\line(0,1){.208133971}}
%\end
%\emline(239.68,15.14)(214.68,58.64)
\multiput(239.68,15.14)(-.119617225,.208133971){209}{\line(0,1){.208133971}}
%\end
%\emline(36.65,102.14)(11.65,58.64)
\multiput(36.65,102.14)(-.119617225,-.208133971){209}{\line(0,-1){.208133971}}
%\end
%\emline(289.6,102.14)(314.6,58.64)
\multiput(289.6,102.14)(.119617225,-.208133971){209}{\line(0,-1){.208133971}}
%\end
%\emline(36.65,15.14)(11.65,58.64)
\multiput(36.65,15.14)(-.119617225,.208133971){209}{\line(0,1){.208133971}}
%\end
%\emline(289.6,15.14)(314.6,58.64)
\multiput(289.6,15.14)(.119617225,.208133971){209}{\line(0,1){.208133971}}
%\end
\put(86.57,101.89){\line(-1,0){50}}
\put(239.68,101.89){\line(1,0){50}}
\put(86.57,15.39){\line(-1,0){50}}
\put(239.68,15.39){\line(1,0){50}}
\put(86.46,101.94){\circle*{4}}
\put(163.71,79.44){\circle*{4}}
\put(178.46,58.94){\circle*{4}}
\put(163.71,37.56){\circle*{4}}
\put(239.79,101.94){\circle*{4}}
\put(86.46,15.34){\circle*{4}}
\put(239.79,15.34){\circle*{4}}
\put(111.39,58.63){\circle*{4}}
\put(214.86,58.63){\circle*{4}}
\put(11.74,58.63){\circle*{4}}
\put(314.51,58.63){\circle*{4}}
\put(36.52,101.94){\circle*{4}}
\put(289.73,101.94){\circle*{4}}
\put(36.52,15.34){\circle*{4}}
\put(289.73,15.34){\circle*{4}}
\put(61.77,58.63){\circle*{4}}
\put(264.48,58.63){\circle*{4}}
%-
%-
\put(30.41,2.65){\makebox(0,0)[cc]{$b$}}
\put(295.84,2.65){\makebox(0,0)[]{$f'$}}
\put(91.93,2.48){\makebox(0,0)[cc]{$c$}}
\put(234.32,2.48){\makebox(0,0)[]{$e'$}}
\put(120.92,57.98){\makebox(0,0)[lc]{$d$}}
\put(205.33,57.98){\makebox(0,0)[r]{$d'$}}
\put(91.93,114.2){\makebox(0,0)[cc]{$e$}}
\put(234.32,114.2){\makebox(0,0)[]{$c'$}}
\put(30.41,114.02){\makebox(0,0)[cc]{$f$}}
\put(295.84,114.02){\makebox(0,0)[]{$b'$}}
\put(1.77,57.98){\makebox(0,0)[rc]{$a$}}
\put(324.48,57.98){\makebox(0,0)[l]{$a'$}}
\put(61.77,47.51){\makebox(0,0)[cc]{$g$}}
\put(264.48,47.51){\makebox(0,0)[]{$g'$}}
\put(163.71,27.56){\makebox(0,0)[]{$h$}}
\put(163.71,89.44){\makebox(0,0)[]{$i$}}
\put(163.75,79.25){\line(0,-1){41.5}}
%\emline(86.8,15.38)(163.87,37.65)
\multiput(86.8,15.38)(.414354839,.119731183){186}{\line(1,0){.414354839}}
%\end
%\emline(86.8,101.62)(163.87,79.35)
\multiput(86.8,101.62)(.414354839,-.119731183){186}{\line(1,0){.414354839}}
%\end
%\emline(163.76,37.84)(239.87,15.35)
\multiput(163.76,37.84)(.404840426,-.11962766){188}{\line(1,0){.404840426}}
%\end
%\emline(163.76,79.16)(239.87,101.65)
\multiput(163.76,79.16)(.404840426,.11962766){188}{\line(1,0){.404840426}}
%\end
\qbezier(36.37,101.96)(114.36,149.57)(163.76,79.04)
\qbezier(36.37,15.04)(114.36,-32.57)(163.76,37.96)
\qbezier(291.16,101.96)(213.17,149.57)(163.76,79.04)
\qbezier(291.16,15.04)(213.17,-32.57)(163.76,37.96)
\put(12,58.25){\line(1,0){99.25}}
%\emline(215,58.25)(314.5,58.5)
\multiput(215,58.25)(33.16667,.08333){3}{\line(1,0){33.16667}}
%\end
%\emline(163.75,79.25)(178.5,58.5)
\multiput(163.75,79.25)(.119918699,-.168699187){123}{\line(0,-1){.168699187}}
%\end
%\emline(178.5,58.5)(164,37.25)
\multiput(178.5,58.5)(-.119834711,-.175619835){121}{\line(0,-1){.175619835}}
%\end
\qbezier(61.75,58.25)(118.375,32.375)(178.5,59)
\qbezier(178.5,58.75)(225.5,79.375)(264.5,58.5)
\put(179.25,69.75){\makebox(0,0)[cc]{$j$}}
\end{picture}
\\
b)
\end{tabular}
\end{center}
{\footnotesize
Configuration of observables implying that the occurrences of $K$ and $K'$ coincide.
%
a) Greechie diagram representing atoms by points, and  contexts by maximal smooth, unbroken curves.
The coordinates of the ``primed'' points $A'$--$M'$ are obtained by interchanging the first and the
second components of the unprimed coordinates $A$--$M$ ;
and $N=(0,0,1)$.
The two contexts $h$ and $i$ linking the primed with the unprimed observables allow the following argument:
Whenever $K$ occurs, then by the one-zero rule $E$ cannot occur;
moreover $N$ cannot occur, hence $K'$ must occur.
Conversely, by symmetry whenever $K'$ occurs, $K$ must occur.
}

}

\renewcommand{\baselinestretch}{0.75}

\subsection{Kochen-Specker Constructions}
\frame{
\frametitle{Kochen-Specker Constructions}

\begin{center}
\begin{tabular}{cc}
%TeXCAD Picture [1.pic]. Options:
%\grade{\on}
%\emlines{\off}
%\epic{\off}
%\beziermacro{\on}
%\reduce{\on}
%\snapping{\off}
%\quality{8.000}
%\graddiff{0.010}
%\snapasp{1}
%\zoom{5.6569}
\unitlength .3mm % = 1.423pt
\linethickness{0.8pt}
\ifx\plotpoint\undefined\newsavebox{\plotpoint}\fi % GNUPLOT compatibility
\begin{picture}(134.09,125.99)(0,0)
%\emline(86.39,101.96)(111.39,58.46)
\multiput(86.39,101.96)(.119617225,-.208133971){209}{\line(0,-1){.208133971}}
%\end
%\emline(86.39,14.96)(111.39,58.46)
\multiput(86.39,14.96)(.119617225,.208133971){209}{\line(0,1){.208133971}}
%\end
%\emline(36.47,101.96)(11.47,58.46)
\multiput(36.47,101.96)(-.119617225,-.208133971){209}{\line(0,-1){.208133971}}
%\end
%\emline(36.47,14.96)(11.47,58.46)
\multiput(36.47,14.96)(-.119617225,.208133971){209}{\line(0,1){.208133971}}
%\end
\put(86.39,101.71){\line(-1,0){50}}
\put(86.39,15.21){\line(-1,0){50}}
\put(86.28,101.76){\circle{4}}
\put(86.28,15.16){\circle{4}}
\put(93.53,89.21){\circle{4}}
\put(93.53,27.71){\circle{4}}
\put(29.24,89.21){\circle{4}}
\put(29.24,27.71){\circle{4}}
\put(102.37,73.47){\circle{4}}
\put(102.37,43.44){\circle{4}}
\put(20.4,73.47){\circle{4}}
\put(20.4,43.44){\circle{4}}
\put(111.21,58.45){\circle{4}}
\put(11.56,58.45){\circle{4}}
\put(36.34,101.76){\circle{4}}
\put(36.34,15.16){\circle{4}}
\put(52.99,101.76){\circle{4}}
\put(52.99,15.16){\circle{4}}
\put(69.68,101.76){\circle{4}}
\put(69.68,15.16){\circle{4}}
\qbezier(29.2,27.73)(23.55,-5.86)(52.99,15.24)
\qbezier(93.57,27.73)(99.22,-5.86)(69.78,15.24)
\qbezier(29.2,27.88)(36.93,75)(69.63,101.91)
\qbezier(93.57,27.88)(85.84,75)(53.13,101.91)
\qbezier(52.69,15.24)(87.47,40.96)(93.72,89.27)
\qbezier(70.08,15.24)(35.3,40.96)(29.05,89.27)
\qbezier(93.72,89.27)(98.4,125.99)(69.49,102.06)
\qbezier(29.05,89.27)(24.37,125.99)(53.28,102.06)
\qbezier(20.15,73.72)(-11.67,58.52)(20.15,43.31)
\qbezier(20.33,73.72)(61.34,93.16)(102.36,73.72)
\qbezier(102.36,73.72)(134.09,58.52)(102.53,43.31)
\qbezier(102.53,43.31)(60.99,23.43)(20.15,43.49)
\put(30.41,114.02){\makebox(0,0)[cc]{$M$}}
\put(30.41,2.65){\makebox(0,0)[cc]{$A$}}
\put(52.68,114.38){\makebox(0,0)[cc]{$L$}}
\put(52.68,2.3){\makebox(0,0)[cc]{$B$}}
\put(91.93,114.2){\makebox(0,0)[cc]{$J$}}
\put(91.93,2.48){\makebox(0,0)[cc]{$D$}}
\put(69.65,114.38){\makebox(0,0)[cc]{$K$}}
\put(73.65,2.3){\makebox(0,0)[cc]{$C$}}
\put(103.24,94.22){\makebox(0,0)[cc]{$I$}}
\put(17.45,94.22){\makebox(0,0)[cc]{$ N$}}
\put(106.24,22.45){\makebox(0,0)[cc]{$E$}}
\put(17.45,22.45){\makebox(0,0)[cc]{$ R$}}
\put(115.13,77.96){\makebox(0,0)[cc]{$H$}}
\put(8.55,77.96){\makebox(0,0)[cc]{$ O$}}
\put(115.13,38.72){\makebox(0,0)[cc]{$F$}}
\put(10.55,38.72){\makebox(0,0)[cc]{$ Q$}}
\put(120.92,57.98){\makebox(0,0)[l]{$ G$}}
\put(1.77,57.98){\makebox(0,0)[rc]{$  P$}}
\put(61.341,9.192){\makebox(0,0)[cc]{$a$}}
\put(102.883,35.355){\makebox(0,0)[cc]{$b$}}
\put(102.53,84.322){\makebox(0,0)[cc]{$c$}}
\put(60.457,108.01){\makebox(0,0)[cc]{$d$}}
\put(18.031,84.145){\makebox(0,0)[cc]{$e$}}
\put(18.561,33.057){\makebox(0,0)[cc]{$f$}}
\put(61.341,39.774){\makebox(0,0)[cc]{$g$}}
\put(72.124,67.882){\makebox(0,0)[cc]{$h$}}
\put(48.79,67.705){\makebox(0,0)[cc]{$i$}}
\end{picture}
&
%TeXCAD Picture [1.pic]. Options:
%\grade{\on}
%\emlines{\off}
%\epic{\off}
%\beziermacro{\on}
%\reduce{\on}
%\snapping{\off}
%\quality{8.000}
%\graddiff{0.010}
%\snapasp{1}
%\zoom{5.6569}
\unitlength .3mm % = 1.423pt
\linethickness{0.8pt}
\ifx\plotpoint\undefined\newsavebox{\plotpoint}\fi % GNUPLOT compatibility
\begin{picture}(119.854,112.606)(0,0)
%\emline(86.567,102.137)(111.567,58.637)
\multiput(86.567,102.137)(.119617225,-.208133971){209}{\line(0,-1){.208133971}}
%\end
%\emline(86.567,15.137)(111.567,58.637)
\multiput(86.567,15.137)(.119617225,.208133971){209}{\line(0,1){.208133971}}
%\end
%\emline(36.647,102.137)(11.647,58.637)
\multiput(36.647,102.137)(-.119617225,-.208133971){209}{\line(0,-1){.208133971}}
%\end
%\emline(36.647,15.137)(11.647,58.637)
\multiput(36.647,15.137)(-.119617225,.208133971){209}{\line(0,1){.208133971}}
%\end
\put(86.567,101.887){\line(-1,0){50}}
\put(86.567,15.387){\line(-1,0){50}}
\put(86.457,101.937){\circle*{4}}
\put(86.457,15.337){\circle*{4}}
\put(85.927,73.127){\circle*{4}}
\put(34.486,73.849){\circle*{4}}
\put(111.387,58.627){\circle*{4}}
\put(11.737,58.627){\circle*{4}}
\put(36.517,101.937){\circle*{4}}
\put(36.517,15.337){\circle*{4}}
\put(60.941,29.586){\circle*{4}}
%\emline(86.087,73.357)(86.447,101.997)
\multiput(86.087,73.357)(.12,9.54667){3}{\line(0,1){9.54667}}
%\end
%\emline(86.267,73.537)(111.187,58.687)
\multiput(86.267,73.537)(.200967742,-.119758065){124}{\line(1,0){.200967742}}
%\end
%\emline(86.087,73.357)(11.667,58.517)
\multiput(86.087,73.357)(-.60016129,-.119677419){124}{\line(-1,0){.60016129}}
%\end
%\emline(86.087,73.187)(36.237,15.207)
\multiput(86.087,73.187)(-.1198317308,-.139375){416}{\line(0,-1){.139375}}
%\end
%\emline(36.951,15.376)(61.341,29.696)
\multiput(36.951,15.376)(.20325,.119333333){120}{\line(1,0){.20325}}
%\end
%\emline(61.341,29.696)(86.801,15.376)
\multiput(61.341,29.696)(.212166667,-.119333333){120}{\line(1,0){.212166667}}
%\end
%\emline(60.991,29.696)(36.591,101.296)
\multiput(60.991,29.696)(-.119607843,.350980392){204}{\line(0,1){.350980392}}
%\end
%\emline(61.161,29.526)(86.801,101.646)
\multiput(61.161,29.526)(.119813084,.337009346){214}{\line(0,1){.337009346}}
%\end
\put(11.844,58.69){\line(3,2){22.804}}
%\emline(34.648,73.892)(36.593,101.823)
\multiput(34.648,73.892)(.1144118,1.643){17}{\line(0,1){1.643}}
%\end
%\emline(86.62,15.38)(34.472,73.716)
\multiput(86.62,15.38)(-.1198804598,.1341057471){435}{\line(0,1){.1341057471}}
%\end
%\emline(34.472,73.716)(111.369,58.513)
\multiput(34.472,73.716)(.605488189,-.119708661){127}{\line(1,0){.605488189}}
%\end
\put(35.885,5.834){\makebox(0,0)[cc]{$e$}}
\put(86.266,6.364){\makebox(0,0)[cc]{$d$}}
\put(119.854,55.861){\makebox(0,0)[cc]{$c$}}
\put(86.09,111.722){\makebox(0,0)[cc]{$b$}}
\put(35.885,112.606){\makebox(0,0)[cc]{$a$}}
\put(3.359,58.689){\makebox(0,0)[cc]{$f$}}
\put(60.634,22.45){\makebox(0,0)[cc]{$h$}}
\put(93.161,77.074){\makebox(0,0)[cc]{$g$}}
\put(28.814,76.544){\makebox(0,0)[cc]{$i$}}
\end{picture}
\\
a)&b)\\
\end{tabular}
\end{center}
{ \footnotesize
Proof of the Kochen-Specker theorem by {\it Cabello et al.} in four-dimensional real vector space.
The nine tightly interconnected contexts
$a$--$i$
consist of the 18 projectors associated with the one dimensional subspaces spanned by
 $ A=(0,0,1,-1)    $,
$  \ldots  $,
$  R=(0,0,1,1)    $.
%
b)  Tkadlec diagram of the nine contexts  interlinked in a four-star form;
hence every observable proposition occurs in exactly two contexts.
Enumeration of the four observable propositions of each of the nine contexts:
there appears to be an {\em even} number of true propositions; yet the number of contexts --- with a single true proposition each --- is {\em odd}.
}

}


\renewcommand{\baselinestretch}{0.9}

\section{Contextuality and its alternatives}

\frame{
\frametitle{Contextuality and its alternatives}


\begin{itemize}

\item[(i)] abandonment of classical omniscience: it is wrong to assume that
all observables which could in principle (``potentially'') have been measured also co--exist,
irrespective of whether or not they have or even could have been actually measured.
Realism might still be assumed for a {\em single} context, in particular the one in which the system was prepared;

\item[(ii)]   abandonment of realism: it is wrong to assume that physical entities exist
even without being experienced by any finite mind.
Quite literary, with this assumption, the proofs of KS and similar decay into thin air because
there are no counterfactuals or unobserved physical observables
or inferred (rather than measured) elements of physical reality.


\item[(iii)] contextuality; i.e., the abandonment of context independence of measurement outcomes:
it is wrong to assume that the
result of an observation is independent
not only of the state of the system
but also of the complete disposition  of the apparatus ---
possible test in the $\{A,B,C\}-\{C,D,E\}$ system of observables with an
explosion type EPR setup \& a singlet state of two spin--one particles.

\end{itemize}

 }

%%%%%%%%%%%%%%%%%%%%%%%%%%

\frame{
\centerline{\Large Thank you for your attention!}
 }


\end{document}
















